%
% Copyright � 2015 Peeter Joot.  All Rights Reserved.
% Licenced as described in the file LICENSE under the root directory of this GIT repository.
%
%\newcommand{\authorname}{Peeter Joot}
\newcommand{\email}{peeterjoot@protonmail.com}
\newcommand{\basename}{FIXMEbasenameUndefined}
\newcommand{\dirname}{notes/FIXMEdirnameUndefined/}

%\renewcommand{\basename}{quadraticZeeman}
%\renewcommand{\dirname}{notes/phy1520/}
%%\newcommand{\dateintitle}{}
%%\newcommand{\keywords}{}
%
%\newcommand{\authorname}{Peeter Joot}
\newcommand{\onlineurl}{http://sites.google.com/site/peeterjoot2/math2013/\basename.pdf}
\newcommand{\sourcepath}{\dirname\basename.tex}
\newcommand{\generatetitle}[1]{\chapter{#1}}

\newcommand{\vcsinfo}{%
\section*{}
\noindent{\color{DarkOliveGreen}{\rule{\linewidth}{0.1mm}}}
\paragraph{Document version}
%\paragraph{\color{Maroon}{Document version}}
{
\small
\begin{itemize}
\item Available online at:\\ 
\href{\onlineurl}{\onlineurl}
\item Git Repository: \input{./.revinfo/gitRepo.tex}
\item Source: \sourcepath
\item last commit: \input{./.revinfo/gitCommitString.tex}
\item commit date: \input{./.revinfo/gitCommitDate.tex}
\end{itemize}
}
}

%\PassOptionsToPackage{dvipsnames,svgnames}{xcolor}
\PassOptionsToPackage{square,numbers}{natbib}
\documentclass{scrreprt}

\usepackage[left=2cm,right=2cm]{geometry}
\usepackage[svgnames]{xcolor}
\usepackage{peeters_layout}

\usepackage{natbib}

\usepackage[
colorlinks=true,
bookmarks=false,
pdfauthor={\authorname, \email},
backref 
]{hyperref}

% http://tex.stackexchange.com/questions/75773/how-to-reference-problems-by-the-text-label-in-an-exercise-envioronment
\usepackage[english]{cleveref}
\crefname{Exercise}{exercise}{exercises}
\Crefname{Exercise}{Exercise}{Exercises}

\RequirePackage{titlesec}
\RequirePackage{ifthen}

% http://stackoverflow.com/questions/4932910/date-in-the-tabular-environment
\makeatletter
\let\insertdate\@date
\makeatother

\titleformat{\chapter}[display]
{\bfseries\Large}
{\color{DarkSlateGrey}\filleft \authorname
\ifthenelse{\isundefined{\studentnumber}}{}{\\ \studentnumber}
\ifthenelse{\isundefined{\email}}{}{\\ \email}
\ifthenelse{\isundefined{\dateintitle}}{}{\\ \insertdate}
%\ifthenelse{\isundefined{\coursename}}{}{\\ \coursename} % put in title instead.
}
{4ex}
{\color{DarkOliveGreen}{\titlerule}\color{Maroon}
\vspace{2ex}%
\filright}
[\vspace{2ex}%
\color{DarkOliveGreen}\titlerule
]

\newcommand{\beginArtWithToc}[0]{\begin{document}\tableofcontents}
\newcommand{\beginArtNoToc}[0]{\begin{document}}
\newcommand{\EndNoBibArticle}[0]{\end{document}}
\newcommand{\EndArticle}[0]{\bibliography{Bibliography}\bibliographystyle{plainnat}\end{document}}

% 
%\newcommand{\citep}[1]{\cite{#1}}

\colorSectionsForArticle


%
%\usepackage{peeters_layout_exercise}
%\usepackage{peeters_braket}
%\usepackage{peeters_figures}
%
%\beginArtNoToc
%
%\generatetitle{Quadratic Zeeman effect}
%%\chapter{Quadradic Zeeman effect}
%%\label{chap:quadraticZeeman}

\makeoproblem{Quadratic Zeeman effect.}{problem:quadraticZeeman:1}{\citep{sakurai2014modern} pr. 5.18}{
\index{Zeeman effect!quadratic}

Work out the quadratic Zeeman effect for the ground state hydrogen atom due to the usually neglected \( e^2 \BA^2/2 m_e c^2 \) term in the Hamiltonian.

} % problem

\makeanswer{problem:quadraticZeeman:1}{

The first order energy shift is

For a z-oriented magnetic field we can use

\begin{dmath}\label{eqn:quadraticZeeman:20}
\BA = \frac{B}{2} \lr{ -y, x, 0 },
\end{dmath}

so the perturbation potential is

\begin{dmath}\label{eqn:quadraticZeeman:40}
V
= \frac{e^2 \BA^2}{2 m_e c^2}
= \frac{e^2 \BB^2 (x^2 + y^2)}{8 m_e c^2}
= \frac{ e^2 \BB^2 r^2 \sin^2\theta }{8 m_e c^2}
\end{dmath}

The ground state wave function is

\begin{dmath}\label{eqn:quadraticZeeman:60}
\psi_0
= \braket{\Bx}{0}
= \inv{\sqrt{\pi a_0^3}} e^{-r/a_0},
\end{dmath}

so the energy shift is

\begin{dmath}\label{eqn:quadraticZeeman:80}
\Delta
= \bra{0} V \ket{0}
= \inv{ \pi a_0^3 } 2 \pi \frac{ e^2 \BB^2 }{8 m_e c^2} \int_0^\infty r^2 \sin\theta e^{-2r/a_0} r^2 \sin^2\theta dr d\theta
=
\frac{ e^2 \BB^2 }{4 a_0^3 m_e c^2}
\int_0^\infty r^4 e^{-2r/a_0} dr \int_0^\pi \sin^3\theta d\theta
= -
\frac{ e^2 \BB^2 }{4 a_0^3 m_e c^2}
\frac{4!}{(2/a_0)^{4+1} } \evalrange{\lr{u - \frac{u^3}{3}}}{1}{-1}
=
\frac{ e^2 a_0^2 \BB^2 }{4 m_e c^2}.
\end{dmath}

If this energy shift is written in terms of a diamagnetic susceptibility \( \chi \) defined by

\begin{dmath}\label{eqn:quadraticZeeman:100}
\Delta = -\inv{2} \chi \BB^2,
\end{dmath}

the diamagnetic susceptibility is

\begin{dmath}\label{eqn:quadraticZeeman:120}
\chi = -\frac{ e^2 a_0^2 }{2 m_e c^2}.
\end{dmath}
} % answer

%\EndArticle
