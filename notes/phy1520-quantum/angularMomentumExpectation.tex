%
% Copyright � 2015 Peeter Joot.  All Rights Reserved.
% Licenced as described in the file LICENSE under the root directory of this GIT repository.
%
%{
%\newcommand{\authorname}{Peeter Joot}
\newcommand{\email}{peeterjoot@protonmail.com}
\newcommand{\basename}{FIXMEbasenameUndefined}
\newcommand{\dirname}{notes/FIXMEdirnameUndefined/}

%\renewcommand{\basename}{angularMomentumExpectation}
%\renewcommand{\dirname}{notes/phy1520/}
%%\newcommand{\dateintitle}{}
%%\newcommand{\keywords}{}
%
%\newcommand{\authorname}{Peeter Joot}
\newcommand{\onlineurl}{http://sites.google.com/site/peeterjoot2/math2013/\basename.pdf}
\newcommand{\sourcepath}{\dirname\basename.tex}
\newcommand{\generatetitle}[1]{\chapter{#1}}

\newcommand{\vcsinfo}{%
\section*{}
\noindent{\color{DarkOliveGreen}{\rule{\linewidth}{0.1mm}}}
\paragraph{Document version}
%\paragraph{\color{Maroon}{Document version}}
{
\small
\begin{itemize}
\item Available online at:\\ 
\href{\onlineurl}{\onlineurl}
\item Git Repository: \input{./.revinfo/gitRepo.tex}
\item Source: \sourcepath
\item last commit: \input{./.revinfo/gitCommitString.tex}
\item commit date: \input{./.revinfo/gitCommitDate.tex}
\end{itemize}
}
}

%\PassOptionsToPackage{dvipsnames,svgnames}{xcolor}
\PassOptionsToPackage{square,numbers}{natbib}
\documentclass{scrreprt}

\usepackage[left=2cm,right=2cm]{geometry}
\usepackage[svgnames]{xcolor}
\usepackage{peeters_layout}

\usepackage{natbib}

\usepackage[
colorlinks=true,
bookmarks=false,
pdfauthor={\authorname, \email},
backref 
]{hyperref}

% http://tex.stackexchange.com/questions/75773/how-to-reference-problems-by-the-text-label-in-an-exercise-envioronment
\usepackage[english]{cleveref}
\crefname{Exercise}{exercise}{exercises}
\Crefname{Exercise}{Exercise}{Exercises}

\RequirePackage{titlesec}
\RequirePackage{ifthen}

% http://stackoverflow.com/questions/4932910/date-in-the-tabular-environment
\makeatletter
\let\insertdate\@date
\makeatother

\titleformat{\chapter}[display]
{\bfseries\Large}
{\color{DarkSlateGrey}\filleft \authorname
\ifthenelse{\isundefined{\studentnumber}}{}{\\ \studentnumber}
\ifthenelse{\isundefined{\email}}{}{\\ \email}
\ifthenelse{\isundefined{\dateintitle}}{}{\\ \insertdate}
%\ifthenelse{\isundefined{\coursename}}{}{\\ \coursename} % put in title instead.
}
{4ex}
{\color{DarkOliveGreen}{\titlerule}\color{Maroon}
\vspace{2ex}%
\filright}
[\vspace{2ex}%
\color{DarkOliveGreen}\titlerule
]

\newcommand{\beginArtWithToc}[0]{\begin{document}\tableofcontents}
\newcommand{\beginArtNoToc}[0]{\begin{document}}
\newcommand{\EndNoBibArticle}[0]{\end{document}}
\newcommand{\EndArticle}[0]{\bibliography{Bibliography}\bibliographystyle{plainnat}\end{document}}

% 
%\newcommand{\citep}[1]{\cite{#1}}

\colorSectionsForArticle


%
%\usepackage{peeters_layout_exercise}
%\usepackage{peeters_braket}
%\usepackage{peeters_figures}
%
%\beginArtNoToc
%
%\generatetitle{Angular momentum expectation}
%%\chapter{Angular momentum expectation}
%%\label{chap:angularMomentumExpectation}

\makeoproblem{Angular momentum expectation values.}{problem:angularMomentumExpectation:n}{\citep{sakurai2014modern} pr. 3.18}{
\index{angular momentum!expectation}

Compute the expectation values for the first and second powers of the angular momentum operators with respect to states \( \ket{lm} \).

} % problem

\makeanswer{problem:angularMomentumExpectation:n}{
We can write the expectation values for the \( L_z \) powers immediately

\begin{dmath}\label{eqn:angularMomentumExpectation:20}
\expectation{L_z}
= m \Hbar,
\end{dmath}

and

\begin{dmath}\label{eqn:angularMomentumExpectation:40}
\expectation{L_z^2} = (m \Hbar)^2.
\end{dmath}

For the x and y components first express the operators in terms of the ladder operators.

\begin{equation}\label{eqn:angularMomentumExpectation:60}
\begin{aligned}
L_{+} &= L_x + i L_y \\
L_{-} &= L_x - i L_y.
\end{aligned}
\end{equation}

Rearranging gives

\begin{equation}\label{eqn:angularMomentumExpectation:80}
\begin{aligned}
L_x &= \inv{2} \lr{ L_{+} + L_{-} } \\
L_y &= \inv{2i} \lr{ L_{+} - L_{-} }.
\end{aligned}
\end{equation}

The first order expectations \( \expectation{L_x}, \expectation{L_y} \) are both zero since \( \expectation{L_{+}} = \expectation{L_{-}} \).  For the second order expectation values we have

\begin{dmath}\label{eqn:angularMomentumExpectation:100}
L_x^2
= \inv{4} \lr{ L_{+} + L_{-} } \lr{ L_{+} + L_{-} }
= \inv{4} \lr{ L_{+} L_{+} + L_{-} L_{-} + L_{+} L_{-} + L_{-} L_{+} }
= \inv{4} \lr{ L_{+} L_{+} + L_{-} L_{-} + 2 (L_x^2 + L_y^2) }
= \inv{4} \lr{ L_{+} L_{+} + L_{-} L_{-} + 2 (\BL^2 - L_z^2) },
\end{dmath}

and
\begin{dmath}\label{eqn:angularMomentumExpectation:120}
L_y^2
= -\inv{4} \lr{ L_{+} - L_{-} } \lr{ L_{+} - L_{-} }
= -\inv{4} \lr{ L_{+} L_{+} + L_{-} L_{-} - L_{+} L_{-} - L_{-} L_{+} }
= -\inv{4} \lr{ L_{+} L_{+} + L_{-} L_{-} - 2 (L_x^2 + L_y^2) }
= -\inv{4} \lr{ L_{+} L_{+} + L_{-} L_{-} - 2 (\BL^2 - L_z^2) }.
\end{dmath}

Any expectation value \( \bra{lm} L_{+} L_{+} \ket{lm} \) or \( \bra{lm} L_{-} L_{-} \ket{lm} \) will be zero, leaving

\begin{dmath}\label{eqn:angularMomentumExpectation:140}
\expectation{L_x^2}
=
\expectation{L_y^2}
=
\inv{4} \expectation{2 (\BL^2 - L_z^2) }
=
\inv{2} \lr{ \Hbar^2 l(l+1) - (\Hbar m)^2 }.
\end{dmath}

Observe that we have
\begin{equation}\label{eqn:angularMomentumExpectation:160}
\expectation{L_x^2}
+
\expectation{L_y^2}
+
\expectation{L_z^2}
=
\Hbar^2 l(l+1)
=
\expectation{\BL^2},
\end{equation}

which is the quantum mechanical analogue of the classical scalar equation \( \BL^2 = L_x^2 + L_y^2 + L_z^2 \).
} % answer

%\EndArticle
