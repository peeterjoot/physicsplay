%
% Copyright � 2015 Peeter Joot.  All Rights Reserved.
% Licenced as described in the file LICENSE under the root directory of this GIT repository.
%
\newcommand{\authorname}{Peeter Joot}
\newcommand{\email}{peeter.joot@utoronto.ca}
\newcommand{\studentnumber}{920798560}
\newcommand{\basename}{FIXMEbasenameUndefined}
\newcommand{\dirname}{notes/FIXMEdirnameUndefined/}

\renewcommand{\basename}{gradQuantumProblemSet8}
\renewcommand{\dirname}{notes/phy1520-quantum/}
\newcommand{\keywords}{Graduate Quantum Mechanics, PHY1520H}
\newcommand{\dateintitle}{}
\newcommand{\authorname}{Peeter Joot}
\newcommand{\onlineurl}{http://sites.google.com/site/peeterjoot2/math2013/\basename.pdf}
\newcommand{\sourcepath}{\dirname\basename.tex}
\newcommand{\generatetitle}[1]{\chapter{#1}}

\newcommand{\vcsinfo}{%
\section*{}
\noindent{\color{DarkOliveGreen}{\rule{\linewidth}{0.1mm}}}
\paragraph{Document version}
%\paragraph{\color{Maroon}{Document version}}
{
\small
\begin{itemize}
\item Available online at:\\ 
\href{\onlineurl}{\onlineurl}
\item Git Repository: \input{./.revinfo/gitRepo.tex}
\item Source: \sourcepath
\item last commit: \input{./.revinfo/gitCommitString.tex}
\item commit date: \input{./.revinfo/gitCommitDate.tex}
\end{itemize}
}
}

%\PassOptionsToPackage{dvipsnames,svgnames}{xcolor}
\PassOptionsToPackage{square,numbers}{natbib}
\documentclass{scrreprt}

\usepackage[left=2cm,right=2cm]{geometry}
\usepackage[svgnames]{xcolor}
\usepackage{peeters_layout}

\usepackage{natbib}

\usepackage[
colorlinks=true,
bookmarks=false,
pdfauthor={\authorname, \email},
backref 
]{hyperref}

% http://tex.stackexchange.com/questions/75773/how-to-reference-problems-by-the-text-label-in-an-exercise-envioronment
\usepackage[english]{cleveref}
\crefname{Exercise}{exercise}{exercises}
\Crefname{Exercise}{Exercise}{Exercises}

\RequirePackage{titlesec}
\RequirePackage{ifthen}

% http://stackoverflow.com/questions/4932910/date-in-the-tabular-environment
\makeatletter
\let\insertdate\@date
\makeatother

\titleformat{\chapter}[display]
{\bfseries\Large}
{\color{DarkSlateGrey}\filleft \authorname
\ifthenelse{\isundefined{\studentnumber}}{}{\\ \studentnumber}
\ifthenelse{\isundefined{\email}}{}{\\ \email}
\ifthenelse{\isundefined{\dateintitle}}{}{\\ \insertdate}
%\ifthenelse{\isundefined{\coursename}}{}{\\ \coursename} % put in title instead.
}
{4ex}
{\color{DarkOliveGreen}{\titlerule}\color{Maroon}
\vspace{2ex}%
\filright}
[\vspace{2ex}%
\color{DarkOliveGreen}\titlerule
]

\newcommand{\beginArtWithToc}[0]{\begin{document}\tableofcontents}
\newcommand{\beginArtNoToc}[0]{\begin{document}}
\newcommand{\EndNoBibArticle}[0]{\end{document}}
\newcommand{\EndArticle}[0]{\bibliography{Bibliography}\bibliographystyle{plainnat}\end{document}}

% 
%\newcommand{\citep}[1]{\cite{#1}}

\colorSectionsForArticle



\usepackage{peeters_layout_exercise}
\usepackage{peeters_braket}
%\usepackage{phy1520}
%\usepackage{siunitx}
%\usepackage{esint} % \oiint

\renewcommand{\QuestionNB}{\alph{Question}.\ }
\renewcommand{\theQuestion}{\alph{Question}}

\newcommand{\nbref}[1]{%
\itemRef{phy1520}{#1}%
}

\beginArtNoToc
\generatetitle{PHY1520H Graduate Quantum Mechanics.  Problem Set 8: Perturbation theory}
%\chapter{Pertubation theory}
\label{chap:gradQuantumProblemSet8}

%\section{Disclaimer}
%
%This is an ungraded set of answers to the problems posed.

%
% Copyright � 2015 Peeter Joot.  All Rights Reserved.
% Licenced as described in the file LICENSE under the root directory of this GIT repository.
%
\makeoproblem{Anharmonic oscillator.}{gradQuantum:problemSet8:1}{2015 ps8 p1}{
\index{anharmonic oscillator}

Consider a quantum particle in the ground state of a 1D anharmonic oscillator potential

\begin{dmath}\label{eqn:gradQuantumProblemSet8Problem1:20}
V(x) = \inv{2} m \omega^2 x^2 + \lambda x^4 = V_0 + \lambda V'.
\end{dmath}

Compute the first and second order energy shift of this oscillator perturbatively in \( \lambda \).

%\makesubproblem{}{gradQuantum:problemSet8:1a}
} % makeproblem

\makeanswer{gradQuantum:problemSet8:1}{
\withproblemsetsParagraph{

Using \nbref{ps8:harmonicOscillatorRaiseAndLoweringOperators.nb} the action of the potential on the ground state is

\begin{dmath}\label{eqn:gradQuantumProblemSet8Problem1:40}
V' \ket{0}
= x^4 \ket{0}
=
x_0^4 \lr{ \frac{3}{4} \ket{0}
+ \frac{3}{\sqrt{2}} \ket{2}
+ \sqrt{\frac{3}{2}} \ket{4}
}.
\end{dmath}

That allows us to compute the first order energy shift

\begin{dmath}\label{eqn:gradQuantumProblemSet8Problem1:60}
\Delta^{(1)}
= \bra{0} V' \ket{0}
= \frac{3}{4} x_0^4.
\end{dmath}

Writing the perturbed state as

\begin{dmath}\label{eqn:gradQuantumProblemSet8Problem1:80}
\ket{n} = \ket{0} + \lambda \ket{0}' + \lambda^2 \ket{0}'' + \cdots,
\end{dmath}

the first order perturbation \( \ket{0}' \) of the ground state is

\begin{dmath}\label{eqn:gradQuantumProblemSet8Problem1:100}
\ket{0}'
= \sum_{m \ne 0} \frac{\ket{m}\bra{m} x^4 \ket{0} }{\Hbar \omega/2 - \Hbar \omega( m + 1/2 ) }
=
- \frac{ x_0^4}{\Hbar \omega} \sum_{m \ne 0} \frac{\ket{m}\bra{m} }{m}
\lr{ \frac{3}{4} \ket{0}
+ \frac{3}{\sqrt{2}} \ket{2}
+ \sqrt{\frac{3}{2}} \ket{4}
}
=
- \frac{ x_0^4}{\Hbar \omega}
\lr{
\inv{2}
  \frac{3}{\sqrt{2}} \ket{2}
+
\inv{4}\sqrt{\frac{3}{2}} \ket{4}
}.
\end{dmath}

The second order energy shift can now be calculated, and is

\begin{dmath}\label{eqn:gradQuantumProblemSet8Problem1:120}
\Delta^{(2)}
=
\bra{0} V' \ket{0}'
=
- \frac{ x_0^8}{\Hbar \omega}
\lr{ \frac{3}{4} \bra{0}
+ \frac{3}{\sqrt{2}} \bra{2}
+ \sqrt{\frac{3}{2}} \bra{4}
}
\lr{
  \inv{2}\frac{3}{\sqrt{2}} \ket{2}
+ \inv{4}\sqrt{\frac{3}{2}} \ket{4}
}
=
- \frac{ x_0^8}{\Hbar \omega} \frac{21}{8}.
\end{dmath}

To second order the total energy shift is
%\begin{dmath}\label{eqn:gradQuantumProblemSet8Problem1:140}
\boxedEquation{eqn:gradQuantumProblemSet8Problem1:160}{
\Delta
= \frac{3}{4} \lambda x_0^4
- \frac{21 x_0^8 \lambda^2}{8 \Hbar \omega}.
}
%\end{dmath}

%\makeSubAnswer{}{gradQuantum:problemSet8:1a}
}
}

%
% Copyright � 2015 Peeter Joot.  All Rights Reserved.
% Licenced as described in the file LICENSE under the root directory of this GIT repository.
%
\makeproblem{Quadrupolar potential}{gradQuantum:problemSet8:2}{ 

Consider a p-orbital electron of hydrogen with \( \ket{ n,l = 1, m } \), with \( m = 0, \pm 1 \), subject to an external potential

\begin{dmath}\label{eqn:gradQuantumProblemSet8Problem2:20}
V(x, y, z) = \lambda(x^2 - y^2),
\end{dmath}

with \( \lambda \) being a constant. For fixed \( n \), obtain the correct eigenstates which diagonalize
the perturbation, without worrying about doing radial integrals explicitly. Show that the three-fold degeneracy of the
p-orbital is completely broken by the perturbation to linear order in \( \lambda \).

%\makesubproblem{}{gradQuantum:problemSet8:2a}
} % makeproblem

\makeanswer{gradQuantum:problemSet8:2}{ 
%\makeSubAnswer{}{gradQuantum:problemSet8:2a}

The potential in spherical coordinates is

\begin{dmath}\label{eqn:gradQuantumProblemSet8Problem2:n}
V = \lambda r^2 \sin^2\theta \lr{ \cos^2\phi - \sin^2\phi } = \lambda r^2 \sin^2\theta \cos(2 \phi).
\end{dmath}

The p-orbital wave functions are

\begin{dmath}\label{eqn:gradQuantumProblemSet8Problem2:n}
\psi_{n1m} = R_n(r) Y_{1,m}(\theta, \phi),
\end{dmath}

where
\begin{equation}\label{eqn:gradQuantumProblemSet8Problem2:n}
\begin{aligned}
Y_{1,1}
\end{aligned}
\end{equation}
so the matrix element 

}

%
% Copyright � 2015 Peeter Joot.  All Rights Reserved.
% Licenced as described in the file LICENSE under the root directory of this GIT repository.
%
\makeproblem{Harmonic oscillator}{gradQuantum:problemSet8:3}{ 
%\makesubproblem{}{gradQuantum:problemSet8:3a}

Consider a 2D harmonic oscillator with

\begin{dmath}\label{eqn:gradQuantumProblemSet8Problem3:20}
H =
\frac{p_x^2}{2m}
+\frac{p_y^2}{2m}
+ \inv{2} m \omega^2 \lr{ x^2 + y^2 }
\end{dmath}

Turn on an anharmonic perturbation 

\begin{dmath}\label{eqn:gradQuantumProblemSet8Problem3:40}
V = \lambda \lr{ x^4 + y^4 } + \lambda^2 x y.
\end{dmath}

Find the unperturbed eigenstates and the corresponding energy shifts upto \( O(\lambda^2) \)
for the ground state and the two-fold degenerate excited states. Ignore terms of \( O(\lambda^3) \).

} % makeproblem

\makeanswer{gradQuantum:problemSet8:3}{ 
%\makeSubAnswer{}{gradQuantum:problemSet8:3a}

TODO.
}

%
% Copyright � 2015 Peeter Joot.  All Rights Reserved.
% Licenced as described in the file LICENSE under the root directory of this GIT repository.
%
\makeproblem{Hyperfine levels}{gradQuantum:problemSet8:4}{ 

We can schematically model the hyperfine interaction between the electron and proton spins as \( A \BS_e \cdot \BS_p \) where \( A \) is the hyperfine interaction energy. 

\makesubproblem{}{gradQuantum:problemSet8:4a}
Consider the spin-1/2 proton interacting with a spin-1/2 electron. 
What are the spin eigenstates and eigenvalues? 

\makesubproblem{}{gradQuantum:problemSet8:4b}
Now consider applying a magnetic field which leads to an extra term  

\begin{dmath}\label{eqn:gradQuantumProblemSet8Problem4:20}
-B \lr{ g_e \mu_e S_e^z + g_p \mu_p S_N^z }
\end{dmath}

with gyromagnetic ratios \( g_e \approx -2 \) and \( g_p \approx 5.5 \), with magnetic moments \( \mu_e = e/2m_e \) and
\( \mu_p = e/2m_p \). The large nuclear mass ensures \( \mu_e/\mu_p \sim 2000 \), so let us simply set \( \mu_p = 0\). For convenience, you can club \( B g_e \mu_e \rightarrow B_{\textrm{eff}} \) so the Hamiltonian becomes

\begin{dmath}\label{eqn:gradQuantumProblemSet8Problem4:40}
H = A \BS_e \cdot \BS_p - B_{\textrm{eff}} S_e^z,
\end{dmath}

so the only dimensionless parameter is \( B_{\textrm{eff}}/A \).

Using perturbation theory (degenerate or nondegenerate as appropriate) find how the coupled hyperfine levels split
for weak field \( B_{\textrm{eff}}/A \ll 1 \).
Also consider the strong field limit \( B_{\textrm{eff}}/A \gg 1 \). 
Compute the full field evolution of the levels and compare with the perturbative low field regime result and the high field regime result.

} % makeproblem

\makeanswer{gradQuantum:problemSet8:4}{ 
\makeSubAnswer{}{gradQuantum:problemSet8:4a}
%What are the spin eigenstates and eigenvalues? 

With respect to the basis \( \beta = \ket{++}, \ket{-+}, \ket{+-}, \ket{--} \), the interaction Hamiltonian is

\begin{dmath}\label{eqn:gradQuantumProblemSet8Problem4:60}
A \BS_e \cdot \BS_p
=
A
\begin{bmatrix}
\bra{++} \BS_e \cdot \BS_p \ket{++} & \bra{++} \BS_e \cdot \BS_p \ket{-+} & \bra{++} \BS_e \cdot \BS_p \ket{+-} & \bra{++} \BS_e \cdot \BS_p \ket{--} \\
\bra{-+} \BS_e \cdot \BS_p \ket{++} & \bra{-+} \BS_e \cdot \BS_p \ket{-+} & \bra{-+} \BS_e \cdot \BS_p \ket{+-} & \bra{-+} \BS_e \cdot \BS_p \ket{--} \\
\bra{+-} \BS_e \cdot \BS_p \ket{++} & \bra{+-} \BS_e \cdot \BS_p \ket{-+} & \bra{+-} \BS_e \cdot \BS_p \ket{+-} & \bra{+-} \BS_e \cdot \BS_p \ket{--} \\
\bra{--} \BS_e \cdot \BS_p \ket{++} & \bra{--} \BS_e \cdot \BS_p \ket{-+} & \bra{--} \BS_e \cdot \BS_p \ket{+-} & \bra{--} \BS_e \cdot \BS_p \ket{--} \\
\end{bmatrix}
=
\frac{A \Hbar^2}{4}
\begin{bmatrix}
\bra{+} \sigma_\txte \ket{+} \bra{+} \sigma_\txtp \ket{+} & \bra{+} \sigma_\txte \ket{-} \bra{+} \sigma_\txtp \ket{+} & \bra{+} \sigma_\txte \ket{+} \bra{+} \sigma_\txtp \ket{-} & \bra{+} \sigma_\txte \ket{-} \bra{+} \sigma_\txtp \ket{-} \\
\bra{-} \sigma_\txte \ket{+} \bra{+} \sigma_\txtp \ket{+} & \bra{-} \sigma_\txte \ket{-} \bra{+} \sigma_\txtp \ket{+} & \bra{-} \sigma_\txte \ket{+} \bra{+} \sigma_\txtp \ket{-} & \bra{-} \sigma_\txte \ket{-} \bra{+} \sigma_\txtp \ket{-} \\
\bra{+} \sigma_\txte \ket{+} \bra{-} \sigma_\txtp \ket{+} & \bra{+} \sigma_\txte \ket{-} \bra{-} \sigma_\txtp \ket{+} & \bra{+} \sigma_\txte \ket{+} \bra{-} \sigma_\txtp \ket{-} & \bra{+} \sigma_\txte \ket{-} \bra{-} \sigma_\txtp \ket{-} \\
\bra{-} \sigma_\txte \ket{+} \bra{-} \sigma_\txtp \ket{+} & \bra{-} \sigma_\txte \ket{-} \bra{-} \sigma_\txtp \ket{+} & \bra{-} \sigma_\txte \ket{+} \bra{-} \sigma_\txtp \ket{-} & \bra{-} \sigma_\txte \ket{-} \bra{-} \sigma_\txtp \ket{-} \\
\end{bmatrix}
=
\frac{A \Hbar^2}{4}
\begin{bmatrix}
(1) (1) & (0) (1) & (1) (0) & (0) (0) \\
(0) (1) & (-1) (1) & (0) (0) & (-1) (0) \\
(1) (0) & (0) (0) & (1) (-1) & (0) (-1) \\
(0) (0) & (-1) (0) & (0) (-1) & (-1) (-1) \\
\end{bmatrix}
=
\frac{A \Hbar^2}{4}
\begin{bmatrix}
\sigma_3 & 0 \\
0 & -\sigma_3
\end{bmatrix}.
\end{dmath}

The spin eigenstates are the basis elements of \( \beta \) above, with respective eigenvalues 

\begin{dmath}\label{eqn:gradQuantumProblemSet8Problem4:80}
\setlr{ A \Hbar^2/4, -A \Hbar^2/4, -A \Hbar^2/4, A \Hbar^2/4}
\end{dmath}

The matrix representation of the pertubation potential is

\begin{dmath}\label{eqn:gradQuantumProblemSet8Problem4:100}
-B_{\textrm{eff}} S^z_e
=
-\frac{B_{\textrm{eff}} \Hbar}{2}
\begin{bmatrix}
\bra{+} \sigma^z_e \ket{+} \braket{+}{+} & \bra{+} \sigma^z_e \ket{-} \braket{+}{+} & \bra{+} \sigma^z_e \ket{+} \braket{+}{-} & \bra{+} \sigma^z_e \ket{-} \braket{+}{-} \\
\bra{-} \sigma^z_e \ket{+} \braket{+}{+} & \bra{-} \sigma^z_e \ket{-} \braket{+}{+} & \bra{-} \sigma^z_e \ket{+} \braket{+}{-} & \bra{-} \sigma^z_e \ket{-} \braket{+}{-} \\
\bra{+} \sigma^z_e \ket{+} \braket{-}{+} & \bra{+} \sigma^z_e \ket{-} \braket{-}{+} & \bra{+} \sigma^z_e \ket{+} \braket{-}{-} & \bra{+} \sigma^z_e \ket{-} \braket{-}{-} \\
\bra{-} \sigma^z_e \ket{+} \braket{-}{+} & \bra{-} \sigma^z_e \ket{-} \braket{-}{+} & \bra{-} \sigma^z_e \ket{+} \braket{-}{-} & \bra{-} \sigma^z_e \ket{-} \braket{-}{-} \\
\end{bmatrix}
=
-\frac{B_{\textrm{eff}} \Hbar}{2}
\begin{bmatrix}
\sigma^z_e & 0 \\
0 & \sigma^z_e
\end{bmatrix},
\end{dmath}

Assuming the \( \BS_e \) operator is directed along \( \ncap = (\sin\theta \cos\phi, \sin\theta \sin\phi, \cos\theta) \) with eigenkets
\begin{dmath}\label{eqn:gradQuantumProblemSet8Problem4:120}
\ket{+} =
\begin{bmatrix}
e^{-i\phi} \cos(\theta/2) \\
\sin(\theta/2) \\
\end{bmatrix}
\end{dmath}
\begin{dmath}\label{eqn:gradQuantumProblemSet8Problem4:140}
\ket{-} =
\begin{bmatrix}
-e^{-i\phi} \sin(\theta/2) \\
\cos(\theta/2) \\
\end{bmatrix},
\end{dmath}

the representation of the \( \sigma^z_\txte \) operator is

\begin{dmath}\label{eqn:gradQuantumProblemSet8Problem4:160}
\sigma^z_\txte
=
\begin{bmatrix}
\cos\theta & -\sin\theta \\
-\sin\theta & -\cos\theta \\
\end{bmatrix}
= 
U \PauliZ U^{-1},
\end{dmath}

where
\begin{equation}\label{eqn:gradQuantumProblemSet8Problem4:180}
U = 
\begin{bmatrix}
-\cos(\theta/2) & \sin(\theta/2) \\
\sin(\theta/2) & \cos(\theta/2)
\end{bmatrix}.
\end{equation}

The full Hamiltonian can now be written in block matrix form

\begin{dmath}\label{eqn:gradQuantumProblemSet8Problem4:200}
H
= 
\frac{A \Hbar^2}{4}
\begin{bmatrix}
\sigma_z & 0 \\
0 & -\sigma_z
\end{bmatrix}
-\frac{B_{\textrm{eff}} \Hbar}{2}
\begin{bmatrix}
U \sigma_z U^{-1} & 0 \\
0 & U \sigma_z U^{-1}
\end{bmatrix}
\end{dmath}

Transforming the Hamiltonian to the \( S^z_\txte \) basis we have

\begin{dmath}\label{eqn:gradQuantumProblemSet8Problem4:220}
H' = 
\frac{A \Hbar^2}{4}
\begin{bmatrix}
U^{-1} \sigma_z U & 0 \\
0 & -U^{-1} \sigma_z U
\end{bmatrix}
-\frac{B_{\textrm{eff}} \Hbar}{2}
\begin{bmatrix}
\sigma_z & 0 \\
0 & \sigma_z
\end{bmatrix}
\end{dmath}

With \( C = \cos(\theta/2), S = \sin(\theta/2) \) these \( U^{-1} \sigma_z U \) block matrices are

\begin{dmath}\label{eqn:gradQuantumProblemSet8Problem4:240}
U^{-1} \sigma_z U 
=
\inv{-C^2 - S^2}
\begin{bmatrix}
C & -S \\
-S & -C
\end{bmatrix}
\begin{bmatrix}
1 & 0 \\
0 & -1
\end{bmatrix}
\begin{bmatrix}
-C & S \\
S & C
\end{bmatrix}
=
\begin{bmatrix}
-C & S \\
S & C
\end{bmatrix}
\begin{bmatrix}
-C & S \\
-S & -C
\end{bmatrix}
=
\begin{bmatrix}
C^2 - S^2 & -2 S C \\
-2 C S & S^2 - C^2
\end{bmatrix}
=
\begin{bmatrix}
\cos\theta & - \sin\theta \\
-\sin\theta & -\cos\theta
\end{bmatrix}.
\end{dmath}

\makeSubAnswer{}{gradQuantum:problemSet8:4b}

TODO.
}


\clearpage
\paragraph{Mathematica Sources}

Mathematica code associated with these notes is available under
\href{https://github.com/peeterjoot/mathematica/tree/master/phy1520-quantum/ps8}{phy1520/ps8/}
within the github repository:

\begin{itemize}
\item git@github.com:peeterjoot/mathematica.git
\end{itemize}

Notebooks created for this problem set

\input{ps8mathematica.tex}

The notebooks referenced in these notes were generated with versions not greater than:

\begin{itemize}
\item commit 891d35b80ba67a96fe5c34619c0018e01b1d8f09
\end{itemize}

%\EndArticle
\EndNoBibArticle
