%
% Copyright � 2015 Peeter Joot.  All Rights Reserved.
% Licenced as described in the file LICENSE under the root directory of this GIT repository.
%
\makeoproblem{Boosts.}{gradQuantum:problemSet5:2}{2015 ps5 p2}{
\index{translation!generator}
\index{boost}

The momentum operator \( \hatp \) was shown, in class, to act as the generator of space translations. Show by following the exact same steps that the position operator \( \hatx \) acts as the generator of momentum boosts (i.e., \( \hatx \) is a generator of `momentum translation').

%\makesubproblem{}{gradQuantum:problemSet5:2a}
} % makeproblem

\makeanswer{gradQuantum:problemSet5:2}{
\withproblemsetsParagraph{
%\makeSubAnswer{}{gradQuantum:problemSet5:2a}

Borrowing the same notation as in class, for an infinitesimal change in momentum \( \delta \tilde{p} \), define a momentum translation operator with the action on a momentum space state of

\begin{dmath}\label{eqn:gradQuantumProblemSet5Problem2:20}
\hatT_{\delta \tilde{p}} \ket{p}
=
\ket{p + \delta \tilde{p}}.
\end{dmath}

A wave function matrix element for this momentum translation is
\begin{dmath}\label{eqn:gradQuantumProblemSet5Problem2:40}
\bra{p} \hatT_{\delta \tilde{p}} \ket{\psi}
=
\braket{p - \delta \tilde{p}}{\psi}
=
\psi_p(p - \delta \tilde{p})
\approx
\psi_p(p) - \delta \tilde{p} \PD{p}{\psi_p(p)}.
\end{dmath}

Since the momentum space representation of the position operator is
\index{position operator!momentum space representation}

\begin{dmath}\label{eqn:gradQuantumProblemSet5Problem2:60}
x \dotEquals i \Hbar \PD{p}{},
\end{dmath}

we have

\begin{dmath}\label{eqn:gradQuantumProblemSet5Problem2:80}
\bra{p} \hatT_{\delta \tilde{p}} \ket{\psi}
\approx
\psi_p(p) - \delta \tilde{p} \frac{x}{i \Hbar} \psi_p(p)
=
\lr{ 1 + \frac{\delta \tilde{p} i x}{\Hbar} } \psi_p(p)
\end{dmath}

Given a finite change of momentum \( \tilde{p} = N \delta \tilde{p} \), the matrix element has the limiting form

\begin{dmath}\label{eqn:gradQuantumProblemSet5Problem2:100}
\bra{p} \hatT_{\tilde{p}} \ket{\psi}
= \lim_{N \rightarrow \infty, \delta \tilde{p} \rightarrow 0 }
\lr{ 1 + \frac{\tilde{p} i x}{N \Hbar} }^N \psi_p(p)
=
e^{ i \tilde{p} x/\Hbar } \psi_p(p),
\end{dmath}

showing that \( x \) is the generator of the momentum translation operator

%\begin{dmath}\label{eqn:gradQuantumProblemSet5Problem2:120}
\boxedEquation{eqn:gradQuantumProblemSet5Problem2:140}{
\hatT_{\tilde{p}} = e^{ i \tilde{p} x/\Hbar }.
}
%\end{dmath}
}
}
