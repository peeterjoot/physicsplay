%
% Copyright � 2015 Peeter Joot.  All Rights Reserved.
% Licenced as described in the file LICENSE under the root directory of this GIT repository.
%
%\newcommand{\authorname}{Peeter Joot}
\newcommand{\email}{peeterjoot@protonmail.com}
\newcommand{\basename}{FIXMEbasenameUndefined}
\newcommand{\dirname}{notes/FIXMEdirnameUndefined/}

%\renewcommand{\basename}{positionCommutator}
%\renewcommand{\dirname}{notes/phy1520/}
%%\newcommand{\dateintitle}{}
%%\newcommand{\keywords}{}
%
%\newcommand{\authorname}{Peeter Joot}
\newcommand{\onlineurl}{http://sites.google.com/site/peeterjoot2/math2013/\basename.pdf}
\newcommand{\sourcepath}{\dirname\basename.tex}
\newcommand{\generatetitle}[1]{\chapter{#1}}

\newcommand{\vcsinfo}{%
\section*{}
\noindent{\color{DarkOliveGreen}{\rule{\linewidth}{0.1mm}}}
\paragraph{Document version}
%\paragraph{\color{Maroon}{Document version}}
{
\small
\begin{itemize}
\item Available online at:\\ 
\href{\onlineurl}{\onlineurl}
\item Git Repository: \input{./.revinfo/gitRepo.tex}
\item Source: \sourcepath
\item last commit: \input{./.revinfo/gitCommitString.tex}
\item commit date: \input{./.revinfo/gitCommitDate.tex}
\end{itemize}
}
}

%\PassOptionsToPackage{dvipsnames,svgnames}{xcolor}
\PassOptionsToPackage{square,numbers}{natbib}
\documentclass{scrreprt}

\usepackage[left=2cm,right=2cm]{geometry}
\usepackage[svgnames]{xcolor}
\usepackage{peeters_layout}

\usepackage{natbib}

\usepackage[
colorlinks=true,
bookmarks=false,
pdfauthor={\authorname, \email},
backref 
]{hyperref}

% http://tex.stackexchange.com/questions/75773/how-to-reference-problems-by-the-text-label-in-an-exercise-envioronment
\usepackage[english]{cleveref}
\crefname{Exercise}{exercise}{exercises}
\Crefname{Exercise}{Exercise}{Exercises}

\RequirePackage{titlesec}
\RequirePackage{ifthen}

% http://stackoverflow.com/questions/4932910/date-in-the-tabular-environment
\makeatletter
\let\insertdate\@date
\makeatother

\titleformat{\chapter}[display]
{\bfseries\Large}
{\color{DarkSlateGrey}\filleft \authorname
\ifthenelse{\isundefined{\studentnumber}}{}{\\ \studentnumber}
\ifthenelse{\isundefined{\email}}{}{\\ \email}
\ifthenelse{\isundefined{\dateintitle}}{}{\\ \insertdate}
%\ifthenelse{\isundefined{\coursename}}{}{\\ \coursename} % put in title instead.
}
{4ex}
{\color{DarkOliveGreen}{\titlerule}\color{Maroon}
\vspace{2ex}%
\filright}
[\vspace{2ex}%
\color{DarkOliveGreen}\titlerule
]

\newcommand{\beginArtWithToc}[0]{\begin{document}\tableofcontents}
\newcommand{\beginArtNoToc}[0]{\begin{document}}
\newcommand{\EndNoBibArticle}[0]{\end{document}}
\newcommand{\EndArticle}[0]{\bibliography{Bibliography}\bibliographystyle{plainnat}\end{document}}

% 
%\newcommand{\citep}[1]{\cite{#1}}

\colorSectionsForArticle


%
%\usepackage{peeters_layout_exercise}
%\usepackage{peeters_braket}
%\usepackage{peeters_figures}
%
%\beginArtNoToc
%
%\generatetitle{Heisenberg picture position commutator}
%\chapter{Heisenberg picture position commutator}
%\label{chap:positionCommutator}


\makeoproblem{Heisenberg picture position commutator.}{problem:positionCommutator:2.5}{\citep{sakurai2014modern} pr. 2.5}{
\index{Heisenberg picture}
Evaluate

\begin{dmath}\label{eqn:positionCommutator:20}
\antisymmetric{x(t)}{x(0)},
\end{dmath}

for a Heisenberg picture operator \( x(t) \) for a free particle.
} % problem

\makeanswer{problem:positionCommutator:2.5}{

The free particle Hamiltonian is

\begin{dmath}\label{eqn:positionCommutator:40}
H = \frac{p^2}{2m},
\end{dmath}

so the time evolution operator is

\begin{dmath}\label{eqn:positionCommutator:60}
U(t) = e^{-i p^2 t/(2 m \Hbar)}.
\end{dmath}

The Heisenberg picture position operator is

\begin{dmath}\label{eqn:positionCommutator:80}
x^\txtH
= U^\dagger x U
= e^{i p^2 t/(2 m \Hbar)} x e^{-i p^2 t/(2 m \Hbar)}
= \sum_{k = 0}^\infty \inv{k!} \lr{ \frac{i p^2 t}{2 m \Hbar} }^k
x
e^{-i p^2 t/(2 m \Hbar)}
= \sum_{k = 0}^\infty \inv{k!} \lr{ \frac{i t}{2 m \Hbar} }^k p^{2k} x
e^{-i p^2 t/(2 m \Hbar)}
=
\sum_{k = 0}^\infty \inv{k!} \lr{ \frac{i t}{2 m \Hbar} }^k \lr{ \antisymmetric{p^{2k}}{x} + x p^{2k} }
e^{-i p^2 t/(2 m \Hbar)}
= x +
\sum_{k = 0}^\infty \inv{k!} \lr{ \frac{i t}{2 m \Hbar} }^k \antisymmetric{p^{2k}}{x}
e^{-i p^2 t/(2 m \Hbar)}
= x +
\sum_{k = 0}^\infty \inv{k!} \lr{ \frac{i t}{2 m \Hbar} }^k \lr{ -i \Hbar \PD{p}{p^{2k}} }
e^{-i p^2 t/(2 m \Hbar)}
= x +
\sum_{k = 0}^\infty \inv{k!} \lr{ \frac{i t}{2 m \Hbar} }^k \lr{ -i \Hbar 2 k p^{2 k -1} }
e^{-i p^2 t/(2 m \Hbar)}
= x +
-2 i \Hbar p \frac{i t}{2 m \Hbar} \sum_{k = 1}^\infty \inv{(k-1)!} \lr{ \frac{i t}{2 m \Hbar} }^{k-1} p^{2(k - 1)}
e^{-i p^2 t/(2 m \Hbar)}
= x + t \frac{p}{m}.
\end{dmath}

This has the structure of a classical free particle \( x(t) = x + v t \), but in this case \( x,p \) are operators.

The evolved position commutator is
\begin{dmath}\label{eqn:positionCommutator:100}
\antisymmetric{x(t)}{x(0)}
=
\antisymmetric{x + t p/m}{x}
=
\frac{t}{m} \antisymmetric{p}{x}
=
-i \Hbar \frac{t}{m}.
\end{dmath}

Compare this to the classical Poisson bracket
\begin{dmath}\label{eqn:positionCommutator:120}
\antisymmetric{x(t)}{x(0)}_{\textrm{classical}}
=
\PD{x}{}\lr{x + p t/m} \PD{p}{x} - \PD{p}{}\lr{x + p t/m} \PD{x}{x}
=
- \frac{t}{m}.
\end{dmath}

This has the expected relation \( \antisymmetric{x(t)}{x(0)} = i \Hbar \antisymmetric{x(t)}{x(0)}_{\textrm{classical}} \).

} % answer

%\EndArticle
