%
% Copyright � 2012 Peeter Joot.  All Rights Reserved.
% Licenced as described in the file LICENSE under the root directory of this GIT repository.
%

%
%
%%
% Copyright � 2015 Peeter Joot.  All Rights Reserved.
% Licenced as described in the file LICENSE under the root directory of this GIT repository.
%
\documentclass[]{eliblog}

\usepackage{amsmath}
\usepackage{mathpazo}

%
% shorthand for bold symbols, convenient for vectors and matrices
%
\newcommand{\Ba}[0]{\mathbf{a}}
\newcommand{\Bb}[0]{\mathbf{b}}
\newcommand{\Bc}[0]{\mathbf{c}}
\newcommand{\Bd}[0]{\mathbf{d}}
\newcommand{\Be}[0]{\mathbf{e}}
\newcommand{\Bf}[0]{\mathbf{f}}
\newcommand{\Bg}[0]{\mathbf{g}}
\newcommand{\Bh}[0]{\mathbf{h}}
\newcommand{\Bi}[0]{\mathbf{i}}
\newcommand{\Bj}[0]{\mathbf{j}}
\newcommand{\Bk}[0]{\mathbf{k}}
\newcommand{\Bl}[0]{\mathbf{l}}
\newcommand{\Bm}[0]{\mathbf{m}}
\newcommand{\Bn}[0]{\mathbf{n}}
\newcommand{\Bo}[0]{\mathbf{o}}
\newcommand{\Bp}[0]{\mathbf{p}}
\newcommand{\Bq}[0]{\mathbf{q}}
\newcommand{\Br}[0]{\mathbf{r}}
\newcommand{\Bs}[0]{\mathbf{s}}
\newcommand{\Bt}[0]{\mathbf{t}}
\newcommand{\Bu}[0]{\mathbf{u}}
\newcommand{\Bv}[0]{\mathbf{v}}
\newcommand{\Bw}[0]{\mathbf{w}}
\newcommand{\Bx}[0]{\mathbf{x}}
\newcommand{\By}[0]{\mathbf{y}}
\newcommand{\Bz}[0]{\mathbf{z}}
\newcommand{\BA}[0]{\mathbf{A}}
\newcommand{\BB}[0]{\mathbf{B}}
\newcommand{\BC}[0]{\mathbf{C}}
\newcommand{\BD}[0]{\mathbf{D}}
\newcommand{\BE}[0]{\mathbf{E}}
\newcommand{\BF}[0]{\mathbf{F}}
\newcommand{\BG}[0]{\mathbf{G}}
\newcommand{\BH}[0]{\mathbf{H}}
\newcommand{\BI}[0]{\mathbf{I}}
\newcommand{\BJ}[0]{\mathbf{J}}
\newcommand{\BK}[0]{\mathbf{K}}
\newcommand{\BL}[0]{\mathbf{L}}
\newcommand{\BM}[0]{\mathbf{M}}
\newcommand{\BN}[0]{\mathbf{N}}
\newcommand{\BO}[0]{\mathbf{O}}
\newcommand{\BP}[0]{\mathbf{P}}
\newcommand{\BQ}[0]{\mathbf{Q}}
\newcommand{\BR}[0]{\mathbf{R}}
\newcommand{\BS}[0]{\mathbf{S}}
\newcommand{\BT}[0]{\mathbf{T}}
\newcommand{\BU}[0]{\mathbf{U}}
\newcommand{\BV}[0]{\mathbf{V}}
\newcommand{\BW}[0]{\mathbf{W}}
\newcommand{\BX}[0]{\mathbf{X}}
\newcommand{\BY}[0]{\mathbf{Y}}
\newcommand{\BZ}[0]{\mathbf{Z}}

\newcommand{\Bzero}[0]{\mathbf{0}}
\newcommand{\Btheta}[0]{\boldsymbol{\theta}}
\newcommand{\Btau}[0]{\boldsymbol{\tau}}
\newcommand{\Bomega}[0]{\boldsymbol{\omega}}

%
% shorthand for unit vectors
%
\newcommand{\acap}[0]{\hat{\Ba}}
\newcommand{\bcap}[0]{\hat{\Bb}}
\newcommand{\ccap}[0]{\hat{\Bc}}
\newcommand{\dcap}[0]{\hat{\Bd}}
\newcommand{\ecap}[0]{\hat{\Be}}
\newcommand{\fcap}[0]{\hat{\Bf}}
\newcommand{\gcap}[0]{\hat{\Bg}}
\newcommand{\hcap}[0]{\hat{\Bh}}
\newcommand{\icap}[0]{\hat{\Bi}}
\newcommand{\jcap}[0]{\hat{\Bj}}
\newcommand{\kcap}[0]{\hat{\Bk}}
\newcommand{\lcap}[0]{\hat{\Bl}}
\newcommand{\mcap}[0]{\hat{\Bm}}
\newcommand{\ncap}[0]{\hat{\Bn}}
\newcommand{\ocap}[0]{\hat{\Bo}}
\newcommand{\pcap}[0]{\hat{\Bp}}
\newcommand{\qcap}[0]{\hat{\Bq}}
\newcommand{\rcap}[0]{\hat{\Br}}
\newcommand{\scap}[0]{\hat{\Bs}}
\newcommand{\tcap}[0]{\hat{\Bt}}
\newcommand{\ucap}[0]{\hat{\Bu}}
\newcommand{\vcap}[0]{\hat{\Bv}}
\newcommand{\wcap}[0]{\hat{\Bw}}
\newcommand{\xcap}[0]{\hat{\Bx}}
\newcommand{\ycap}[0]{\hat{\By}}
\newcommand{\zcap}[0]{\hat{\Bz}}
\newcommand{\thetacap}[0]{\hat{\Btheta}}

%
% to write R^n and C^n in a distinguishable fashion.  Perhaps change this
% to the double lined characters upon figuring out how to do so.
%
\newcommand{\C}[1]{$\mathbb{C}^{#1}$}
\newcommand{\R}[1]{$\mathbb{R}^{#1}$}

%
% various generally useful helpers
%

% derivative of #1 wrt. #2:
\newcommand{\D}[2] {\frac {d#2} {d#1}}

\newcommand{\inv}[1]{\frac{1}{#1}}
\newcommand{\cross}[0]{\times}

\newcommand{\abs}[1]{\lvert{#1}\rvert}
\newcommand{\norm}[1]{\lVert{#1}\rVert}
\newcommand{\innerprod}[2]{\langle{#1}, {#2}\rangle}
\newcommand{\dotprod}[2]{{#1} \cdot {#2}}
\newcommand{\bdotprod}[2]{\left({#1} \cdot {#2}\right)}
\newcommand{\crossprod}[2]{{#1} \cross {#2}}
\newcommand{\tripleprod}[3]{\dotprod{\left(\crossprod{#1}{#2}\right)}{#3}}

\DeclareMathOperator{\Proj}{Proj}
\DeclareMathOperator{\Span}{span}
\DeclareMathOperator{\Sgn}{sgn}
\DeclareMathOperator{\Area}{Area}
\DeclareMathOperator{\Volume}{Volume}

%
% A few miscellaneous things specific to this document
%
\newcommand{\crossop}[1]{\crossprod{#1}{}}

% R2 vector.
\newcommand{\VectorTwo}[2]{
\begin{bmatrix}
 {#1} \\
 {#2}
\end{bmatrix}
}

\newcommand{\VectorN}[1]{
\begin{bmatrix}
{#1}_1 \\
{#1}_2 \\
\vdots \\
{#1}_N \\
\end{bmatrix}
}

\newcommand{\DETuvij}[4]{
\begin{vmatrix}
 {#1}_{#3} & {#1}_{#4} \\
 {#2}_{#3} & {#2}_{#4}
\end{vmatrix}
}

\newcommand{\DETuvwijk}[6]{
\begin{vmatrix}
 {#1}_{#4} & {#1}_{#5} & {#1}_{#6} \\
 {#2}_{#4} & {#2}_{#5} & {#2}_{#6} \\
 {#3}_{#4} & {#3}_{#5} & {#3}_{#6}
\end{vmatrix}
}

\newcommand{\DETuvwxijkl}[8]{
\begin{vmatrix}
 {#1}_{#5} & {#1}_{#6} & {#1}_{#7} & {#1}_{#8} \\
 {#2}_{#5} & {#2}_{#6} & {#2}_{#7} & {#2}_{#8} \\
 {#3}_{#5} & {#3}_{#6} & {#3}_{#7} & {#3}_{#8} \\
 {#4}_{#5} & {#4}_{#6} & {#4}_{#7} & {#4}_{#8} \\
\end{vmatrix}
}

%\newcommand{\DETuvwxyijklm}[10]{
%\begin{vmatrix}
% {#1}_{#6} & {#1}_{#7} & {#1}_{#8} & {#1}_{#9} & {#1}_{#10} \\
% {#2}_{#6} & {#2}_{#7} & {#2}_{#8} & {#2}_{#9} & {#2}_{#10} \\
% {#3}_{#6} & {#3}_{#7} & {#3}_{#8} & {#3}_{#9} & {#3}_{#10} \\
% {#4}_{#6} & {#4}_{#7} & {#4}_{#8} & {#4}_{#9} & {#4}_{#10} \\
% {#5}_{#6} & {#5}_{#7} & {#5}_{#8} & {#5}_{#9} & {#5}_{#10}
%\end{vmatrix}
%}

% R3 vector.
\newcommand{\VectorThree}[3]{
\begin{bmatrix}
 {#1} \\
 {#2} \\
 {#3}
\end{bmatrix}
}



\author{Peeter Joot}
\email{peeter.joot@gmail.com}

%\documentclass[]{eliblogwidescreen}

\usepackage{amsmath}
\usepackage{mathpazo}

%
% shorthand for bold symbols, convenient for vectors and matrices
%
\newcommand{\Ba}[0]{\mathbf{a}}
\newcommand{\Bb}[0]{\mathbf{b}}
\newcommand{\Bc}[0]{\mathbf{c}}
\newcommand{\Bd}[0]{\mathbf{d}}
\newcommand{\Be}[0]{\mathbf{e}}
\newcommand{\Bf}[0]{\mathbf{f}}
\newcommand{\Bg}[0]{\mathbf{g}}
\newcommand{\Bh}[0]{\mathbf{h}}
\newcommand{\Bi}[0]{\mathbf{i}}
\newcommand{\Bj}[0]{\mathbf{j}}
\newcommand{\Bk}[0]{\mathbf{k}}
\newcommand{\Bl}[0]{\mathbf{l}}
\newcommand{\Bm}[0]{\mathbf{m}}
\newcommand{\Bn}[0]{\mathbf{n}}
\newcommand{\Bo}[0]{\mathbf{o}}
\newcommand{\Bp}[0]{\mathbf{p}}
\newcommand{\Bq}[0]{\mathbf{q}}
\newcommand{\Br}[0]{\mathbf{r}}
\newcommand{\Bs}[0]{\mathbf{s}}
\newcommand{\Bt}[0]{\mathbf{t}}
\newcommand{\Bu}[0]{\mathbf{u}}
\newcommand{\Bv}[0]{\mathbf{v}}
\newcommand{\Bw}[0]{\mathbf{w}}
\newcommand{\Bx}[0]{\mathbf{x}}
\newcommand{\By}[0]{\mathbf{y}}
\newcommand{\Bz}[0]{\mathbf{z}}
\newcommand{\BA}[0]{\mathbf{A}}
\newcommand{\BB}[0]{\mathbf{B}}
\newcommand{\BC}[0]{\mathbf{C}}
\newcommand{\BD}[0]{\mathbf{D}}
\newcommand{\BE}[0]{\mathbf{E}}
\newcommand{\BF}[0]{\mathbf{F}}
\newcommand{\BG}[0]{\mathbf{G}}
\newcommand{\BH}[0]{\mathbf{H}}
\newcommand{\BI}[0]{\mathbf{I}}
\newcommand{\BJ}[0]{\mathbf{J}}
\newcommand{\BK}[0]{\mathbf{K}}
\newcommand{\BL}[0]{\mathbf{L}}
\newcommand{\BM}[0]{\mathbf{M}}
\newcommand{\BN}[0]{\mathbf{N}}
\newcommand{\BO}[0]{\mathbf{O}}
\newcommand{\BP}[0]{\mathbf{P}}
\newcommand{\BQ}[0]{\mathbf{Q}}
\newcommand{\BR}[0]{\mathbf{R}}
\newcommand{\BS}[0]{\mathbf{S}}
\newcommand{\BT}[0]{\mathbf{T}}
\newcommand{\BU}[0]{\mathbf{U}}
\newcommand{\BV}[0]{\mathbf{V}}
\newcommand{\BW}[0]{\mathbf{W}}
\newcommand{\BX}[0]{\mathbf{X}}
\newcommand{\BY}[0]{\mathbf{Y}}
\newcommand{\BZ}[0]{\mathbf{Z}}

\newcommand{\Bzero}[0]{\mathbf{0}}
\newcommand{\Btheta}[0]{\boldsymbol{\theta}}
\newcommand{\Btau}[0]{\boldsymbol{\tau}}
\newcommand{\Bomega}[0]{\boldsymbol{\omega}}

%
% shorthand for unit vectors
%
\newcommand{\acap}[0]{\hat{\Ba}}
\newcommand{\bcap}[0]{\hat{\Bb}}
\newcommand{\ccap}[0]{\hat{\Bc}}
\newcommand{\dcap}[0]{\hat{\Bd}}
\newcommand{\ecap}[0]{\hat{\Be}}
\newcommand{\fcap}[0]{\hat{\Bf}}
\newcommand{\gcap}[0]{\hat{\Bg}}
\newcommand{\hcap}[0]{\hat{\Bh}}
\newcommand{\icap}[0]{\hat{\Bi}}
\newcommand{\jcap}[0]{\hat{\Bj}}
\newcommand{\kcap}[0]{\hat{\Bk}}
\newcommand{\lcap}[0]{\hat{\Bl}}
\newcommand{\mcap}[0]{\hat{\Bm}}
\newcommand{\ncap}[0]{\hat{\Bn}}
\newcommand{\ocap}[0]{\hat{\Bo}}
\newcommand{\pcap}[0]{\hat{\Bp}}
\newcommand{\qcap}[0]{\hat{\Bq}}
\newcommand{\rcap}[0]{\hat{\Br}}
\newcommand{\scap}[0]{\hat{\Bs}}
\newcommand{\tcap}[0]{\hat{\Bt}}
\newcommand{\ucap}[0]{\hat{\Bu}}
\newcommand{\vcap}[0]{\hat{\Bv}}
\newcommand{\wcap}[0]{\hat{\Bw}}
\newcommand{\xcap}[0]{\hat{\Bx}}
\newcommand{\ycap}[0]{\hat{\By}}
\newcommand{\zcap}[0]{\hat{\Bz}}
\newcommand{\thetacap}[0]{\hat{\Btheta}}

%
% to write R^n and C^n in a distinguishable fashion.  Perhaps change this
% to the double lined characters upon figuring out how to do so.
%
\newcommand{\C}[1]{$\mathbb{C}^{#1}$}
\newcommand{\R}[1]{$\mathbb{R}^{#1}$}

%
% various generally useful helpers
%

% derivative of #1 wrt. #2:
\newcommand{\D}[2] {\frac {d#2} {d#1}}

\newcommand{\inv}[1]{\frac{1}{#1}}
\newcommand{\cross}[0]{\times}

\newcommand{\abs}[1]{\lvert{#1}\rvert}
\newcommand{\norm}[1]{\lVert{#1}\rVert}
\newcommand{\innerprod}[2]{\langle{#1}, {#2}\rangle}
\newcommand{\dotprod}[2]{{#1} \cdot {#2}}
\newcommand{\bdotprod}[2]{\left({#1} \cdot {#2}\right)}
\newcommand{\crossprod}[2]{{#1} \cross {#2}}
\newcommand{\tripleprod}[3]{\dotprod{\left(\crossprod{#1}{#2}\right)}{#3}}

\DeclareMathOperator{\Proj}{Proj}
\DeclareMathOperator{\Span}{span}
\DeclareMathOperator{\Sgn}{sgn}
\DeclareMathOperator{\Area}{Area}
\DeclareMathOperator{\Volume}{Volume}

%
% A few miscellaneous things specific to this document
%
\newcommand{\crossop}[1]{\crossprod{#1}{}}

% R2 vector.
\newcommand{\VectorTwo}[2]{
\begin{bmatrix}
 {#1} \\
 {#2}
\end{bmatrix}
}

\newcommand{\VectorN}[1]{
\begin{bmatrix}
{#1}_1 \\
{#1}_2 \\
\vdots \\
{#1}_N \\
\end{bmatrix}
}

\newcommand{\DETuvij}[4]{
\begin{vmatrix}
 {#1}_{#3} & {#1}_{#4} \\
 {#2}_{#3} & {#2}_{#4}
\end{vmatrix}
}

\newcommand{\DETuvwijk}[6]{
\begin{vmatrix}
 {#1}_{#4} & {#1}_{#5} & {#1}_{#6} \\
 {#2}_{#4} & {#2}_{#5} & {#2}_{#6} \\
 {#3}_{#4} & {#3}_{#5} & {#3}_{#6}
\end{vmatrix}
}

\newcommand{\DETuvwxijkl}[8]{
\begin{vmatrix}
 {#1}_{#5} & {#1}_{#6} & {#1}_{#7} & {#1}_{#8} \\
 {#2}_{#5} & {#2}_{#6} & {#2}_{#7} & {#2}_{#8} \\
 {#3}_{#5} & {#3}_{#6} & {#3}_{#7} & {#3}_{#8} \\
 {#4}_{#5} & {#4}_{#6} & {#4}_{#7} & {#4}_{#8} \\
\end{vmatrix}
}

%\newcommand{\DETuvwxyijklm}[10]{
%\begin{vmatrix}
% {#1}_{#6} & {#1}_{#7} & {#1}_{#8} & {#1}_{#9} & {#1}_{#10} \\
% {#2}_{#6} & {#2}_{#7} & {#2}_{#8} & {#2}_{#9} & {#2}_{#10} \\
% {#3}_{#6} & {#3}_{#7} & {#3}_{#8} & {#3}_{#9} & {#3}_{#10} \\
% {#4}_{#6} & {#4}_{#7} & {#4}_{#8} & {#4}_{#9} & {#4}_{#10} \\
% {#5}_{#6} & {#5}_{#7} & {#5}_{#8} & {#5}_{#9} & {#5}_{#10}
%\end{vmatrix}
%}

% R3 vector.
\newcommand{\VectorThree}[3]{
\begin{bmatrix}
 {#1} \\
 {#2} \\
 {#3}
\end{bmatrix}
}



\author{Peeter Joot}
\email{peeter.joot@gmail.com}


\chapter{Errata for Feynman's Quantum Electrodynamics (Addison-Wesley)?}
\label{chap:feynmanQEDerrata}
%\useCCL
\blogpage{http://sites.google.com/site/peeterjoot/math2010/feynmanQEDerrata.pdf}
\date{May 28, 2010}
\revisionInfo{feynmanQEDerrata.tex}

\beginArtWithToc
%\beginArtNoToc

\section{Motivation}

I got a nice present today which included one of \href{http://www.amazon.com/Quantum-Electrodynamics-Advanced-Book-Classics/dp/0201360756/ref=sr_1_1?ie=UTF8&s=books&qid=1275092228&sr=8-1}{Feynman's QED books} (Addison-Wesley Feb 98 first printing).  I noticed some early mistakes, and since I can not find an errata page anywhere, I will collect them here, along with some other notes.

Eventually, if I get through the book, I will see about sending this into the publisher.
% The original editor (a prof emeritus): David Pines <david.pines@gmail.com>
% knows of no official errata, and suggested a facebook site to allow people to
% collaboratively collect notes on errors.

\section{On what I believe should be in the errata if it existed}
\subsection{Third Lecture}
\subsubsection{Page 6}

The electric field is given in terms of only the scalar potential
\begin{equation}\label{eqn:feynmanQEDerrata:20}
\begin{aligned}
\BE = -\spacegrad \phi + \partial \phi/ \partial t,
\end{aligned}
\end{equation}

and should be
\begin{equation}\label{eqn:feynmanQEDerrata:40}
\begin{aligned}
\BE = -\spacegrad \phi - \inv{c} \partial \BA/ \partial t.
\end{aligned}
\end{equation}

The invariant gauge transformation for the vector and scalar potentials are given as

\begin{equation}\label{eqn:feynmanQEDerrata:60}
\begin{aligned}
\BA' &= \BA + \spacegrad \chi \\
\phi' &= \phi + \partial \chi / \partial t
\end{aligned}
\end{equation}

But these should be
\begin{equation}\label{eqn:feynmanQEDerrata:80}
\begin{aligned}
\BA' &= \BA + \spacegrad \chi \\
\phi' &= \phi - \inv{c} \partial \chi / \partial t
\end{aligned}
\end{equation}

The sign was crossed on the scalar potential transformation.  Perhaps Feynman used \(c=1\) in his lectures, and whoever made the notes was not consistent about including these in all the right places (but did so in some).

\subsubsection{Page 7}

With the signs and constant terms of the gauge transformation for the potentials being off, so is the end result for the final set of transformations that leave the Pauli equation invariant.  That should be:

\begin{equation}\label{eqn:feynmanQEDerrata:100}
\begin{aligned}
\BA' &= \BA + \spacegrad \chi \\
\phi' &= \phi - \inv{c} \PD{t}{ \chi } \\
\Psi' &= \exp\left( i \frac{e}{ \Hbar c } \chi \right) \Psi,
\end{aligned}
\end{equation}

(with the intermediate steps corrected accordingly).

\subsubsection{Page 8}

It is written

\begin{equation}\label{eqn:feynmanQEDerrata:120}
\begin{aligned}
H = \inv{2m} \left( \Bp - \frac{e}{c}\BA \right)^2 - \frac{e \Hbar}{2 m c} (\Bsigma \cdot \spacegrad \cross \BA) + e V
\end{aligned}
\end{equation}

It appears that the minus should be a positive here.

\begin{equation}\label{eqn:feynmanQEDerrata:140}
\begin{aligned}
H = \inv{2m} \left( \Bp - \frac{e}{c}\BA \right)^2 + \frac{e \Hbar}{2 m c} (\Bsigma \cdot \spacegrad \cross \BA) + e V
\end{aligned}
\end{equation}

It also appears that \(\Bx^2 \equiv (\Bx \cdot \Bx) I\), where the identity matrix \(I\) is implied.

Then, equation (1) which reads

\begin{equation}\label{eqn:feynmanQEDerrata:160}
\begin{aligned}
\spacegrad \cross \BA = \BK \cross \Be e^{i \BK \cdot \Bx} e^{i \omega t}
\end{aligned}
\end{equation}

should be
\begin{equation}\label{eqn:feynmanQEDerrata:180}
\begin{aligned}
\spacegrad \cross \BA
&= a \BK \cross \Be e^{i \BK \cdot \Bx} e^{-i \omega t}
- a e^{i \BK \cdot \Bx} e^{-i \omega t} \Be \cross \spacegrad \\
&= \BA \cross (i \BK - \spacegrad )
\end{aligned}
\end{equation}

And in equation two the sign is wrong.  It reads

\begin{equation}\label{eqn:feynmanQEDerrata:200}
\begin{aligned}
\Bp e^{i \BK \cdot \Bx} = e^{i \BK \cdot \Bx} (\Bp - \Hbar \BK)
\end{aligned}
\end{equation}

but should be
\begin{equation}\label{eqn:feynmanQEDerrata:220}
\begin{aligned}
\Bp e^{i \BK \cdot \Bx} = e^{i \BK \cdot \Bx} (\Bp + \Hbar \BK)
\end{aligned}
\end{equation}

similarly the following \(-\Hbar \BK \cdot \Be\) should be positive, \(\Hbar \BK \cdot \Be\).  (this last has no effect since \(\BK \cdot \Be\) is assumed zero since \(\Be\) was picked as the transverse propagation direction for the electrodynamic wave).

\subsection{Seventh Lecture}
\subsubsection{Page 25}

Last equation reads

\begin{equation}\label{eqn:feynmanQEDerrata:240}
\begin{aligned}
E t - p_x x - p_y y - p_z z = p_\mu p_\mu
\end{aligned}
\end{equation}

should be

\begin{equation}\label{eqn:feynmanQEDerrata:260}
\begin{aligned}
E t - p_x x - p_y y - p_z z = p_\mu x_\mu
\end{aligned}
\end{equation}

\subsubsection{Page 26}

After ``but'' we have
\begin{equation}\label{eqn:feynmanQEDerrata:280}
\begin{aligned}
p_0^2 = E^2 - m
\end{aligned}
\end{equation}

which should be

\begin{equation}\label{eqn:feynmanQEDerrata:300}
\begin{aligned}
p_0^2 = E^2 - m^2
\end{aligned}
\end{equation}

\subsubsection{Page 29}

The gauge transformation once again has the sign messed up.  It was written (from \({A_\mu}' = A_\mu + \nabla_\mu \chi\))

\begin{equation}\label{eqn:feynmanQEDerrata:320}
\begin{aligned}
\BA' &= \BA + \spacegrad \chi \\
\phi' &= \phi + {\partial \chi}/{\partial t}
\end{aligned}
\end{equation}

but it should be

\begin{equation}\label{eqn:feynmanQEDerrata:340}
\begin{aligned}
\BA' &= \BA - \spacegrad \chi \\
\phi' &= \phi + {\partial \chi}/{\partial t}
\end{aligned}
\end{equation}

(ie: \(\nabla_m = -\partial_m\))

Then a bit later

\begin{equation}\label{eqn:feynmanQEDerrata:360}
\begin{aligned}
\grad \cdot A' = \grad \cdot A + \grad \cdot \chi
\end{aligned}
\end{equation}

should be

\begin{equation}\label{eqn:feynmanQEDerrata:380}
\begin{aligned}
\grad \cdot A' = \grad \cdot A + \grad \cdot \grad \chi
\end{aligned}
\end{equation}

\subsubsection{Page 29}

\begin{equation}\label{eqn:feynmanQEDerrata:400}
\begin{aligned}
dx/ds = (dx/dt)(dt/ds) = v_x/(1-y^2)^{1/2}
\end{aligned}
\end{equation}

should be
\begin{equation}\label{eqn:feynmanQEDerrata:420}
\begin{aligned}
dx/ds = (dx/dt)(dt/ds) = v_x/(1-v^2)^{1/2}
\end{aligned}
\end{equation}

\section{Extended notes}

\subsection{Second Lecture}

This is not errata, but I found the following required slight exploration.  He gives (implicitly)

\begin{equation}\label{eqn:feynmanQEDerrata:440}
\begin{aligned}
\overline{\sin^2(\omega t - \BK \cdot \Bx)} = \inv{2}
\end{aligned}
\end{equation}

Is this an average over space and time?  How would one do that?  What do we get just integrating this over the volume?  That dot product is \(\BK \cdot \Bx = 2 \pi \left(\frac{m}{\lambda_1} x + \frac{n}{\lambda_2} y + \frac{o}{\lambda_3} z \right)\).  Our average over the volume, for \(m \ne 0\), using \href{http://www.wolframalpha.com/input/?i=\int+sin^2(a+x+%2B+b)+dx}{wolfram alpha to do the dirty work}, is then

\begin{equation}\label{eqn:feynmanQEDerrata:460}
\begin{aligned}
&\inv{\lambda_1 \lambda_2 \lambda_3}
\int_{z=0}^{\lambda_3} dz
\int_{y=0}^{\lambda_2} dy
\int_{x=0}^{\lambda_1}
dx \sin^2 \left(
-\frac{2 \pi m x}{\lambda_1}
-\frac{2 \pi n y}{\lambda_2}
-\frac{2 \pi o z}{\lambda_3}
+ \omega t \right) \\
&=
\inv{\lambda_1 \lambda_2 \lambda_3}
\int_{z=0}^{\lambda_3} dz
\int_{y=0}^{\lambda_2} dy
{\left.
\frac{-\lambda_1}{4 \pi m} \left(
-\frac{2 \pi m }{\lambda_1} x
-\frac{2 \pi n y}{\lambda_2}
-\frac{2 \pi o z}{\lambda_3}
+ \omega t \right)
\right\vert}_{x=0}^{\lambda_1} \\
&-
\inv{\lambda_1 \lambda_2 \lambda_3}
\int_{z=0}^{\lambda_3} dz
\int_{y=0}^{\lambda_2} dy
{\left.
\frac{-\lambda_1}{8 \pi m}
\sin \left( 2 \left(
-\frac{2 \pi m }{\lambda_1} x
-\frac{2 \pi n y}{\lambda_2}
-\frac{2 \pi o z}{\lambda_3}
+ \omega t \right) \right)
\right\vert}_{x=0}^{\lambda_1}
\end{aligned}
\end{equation}

Since the sine integral vanishes, we have just \(1/2\) as expected regardless of the angular frequency \(\omega\).  Okay, that makes sense now.  Looks like \(\omega\) is only relevant for the single \(\BK = 0\) Fourier component, but that likely does not matter since I seem to recall that the \(\BK = 0\) Fourier component of this oscillators in a box problem was entirely constant (and perhaps zero?).

\subsection{Third Lecture.  Page 7 notes}

The units in the transformation for the wave function do not look right.  We want to transform the Pauli equation

\begin{equation}\label{eqn:feynmanQEDerrata:480}
\begin{aligned}
i \Hbar \PD{t}{\Psi} = \inv{2 m} \left( \Bp - \frac{e}{c} \BA \right)^2 \Psi + e \phi \Psi,
\end{aligned}
\end{equation}

with a transformation of the form
\begin{equation}\label{eqn:feynmanQEDerrata:500}
\begin{aligned}
\BA' &= \BA + \spacegrad \chi \\
\phi' &= \phi - \inv{c} \PD{t}{\chi} \\
\Psi' &= e^{-i \mu} \Psi,
\end{aligned}
\end{equation}

Where \(\mu \propto \chi\) is presumed, and we want to find the proportionality constant required for invariance.  With \(\Bp = - i \Hbar \spacegrad\) we have

\begin{equation}\label{eqn:feynmanQEDerrata:520}
\begin{aligned}
\Bp \Psi'
&=
-i \Hbar \spacegrad e^{-i \mu} \Psi \\
&=
-i \Hbar \left(
-i (\spacegrad \mu) e^{-i \mu} \Psi
+ e^{-i \mu} \spacegrad \Psi
\right) \\
&=
+ e^{-i \mu} \left( \Bp + \Hbar \spacegrad \mu \right) \Psi,
\end{aligned}
\end{equation}

so
\begin{equation}\label{eqn:feynmanQEDerrata:540}
\begin{aligned}
(\Bp -\frac{e}{c} \BA' )\Psi'
&=
e^{-i \mu} \left( \Bp - \frac{e}{c} \BA - \spacegrad (\Hbar \mu + \frac{e}{c} \chi) \right) \Psi.
\end{aligned}
\end{equation}

For the time partial we have

\begin{equation}\label{eqn:feynmanQEDerrata:560}
\begin{aligned}
\PD{t}{\Psi'} &= e^{-i \mu} \PD{t}{\Psi} -i \PD{t}{\mu} e^{-i \mu} \Psi,
\end{aligned}
\end{equation}

and the scalar potential term transforms as
\begin{equation}\label{eqn:feynmanQEDerrata:580}
\begin{aligned}
e \phi' \Psi'
&=
e \left( \phi - \inv{c} \PD{t}{\chi} \right) e^{-i \mu } \Psi
\end{aligned}
\end{equation}

Putting the pieces together we have

\begin{equation}\label{eqn:feynmanQEDerrata:600}
\begin{aligned}
i \Hbar e^{-i \mu}
\left( \PD{t}{} -i \PD{t}{\mu} \right) \Psi
&=
\inv{2m}
\left(\Bp -\frac{e}{c} \BA -\frac{e}{c} \spacegrad \chi \right)
e^{-i \mu} \left( \Bp - \frac{e}{c} \BA - \spacegrad (\Hbar \mu + \frac{e}{c} \chi) \right) \Psi
+ e \left( \phi - \inv{c} \PD{t}{\chi} \right) e^{-i \mu } \Psi
\end{aligned}
\end{equation}

We need one more intermediate result, that of

\begin{equation}\label{eqn:feynmanQEDerrata:620}
\begin{aligned}
\Bp e^{-i \mu } \BD
&=
- i \Hbar e^{-i \mu} \left( -i (\spacegrad \mu) + \spacegrad \right) \BD \\
&=
e^{-i\mu} (\Bp - \Hbar \spacegrad \mu) \BD.
\end{aligned}
\end{equation}

So we have
\begin{equation}\label{eqn:feynmanQEDerrata:640}
\begin{aligned}
i \Hbar \PD{t}{\Psi}
+\Hbar \PD{t}{\mu} \Psi
&=
\inv{2m}
\left(\Bp - \Hbar \spacegrad \mu -\frac{e}{c} \BA -\frac{e}{c} \spacegrad \chi \right)
\left( \Bp - \frac{e}{c} \BA - \spacegrad (\Hbar \mu + \frac{e}{c} \chi) \right) \Psi
+ e \left( \phi - \inv{c} \PD{t}{\chi} \right) \Psi.
\end{aligned}
\end{equation}

To get rid of the \(\mu\), and \(\chi\) time partials we need

\begin{equation}\label{eqn:feynmanQEDerrata:660}
\begin{aligned}
\Hbar \PD{t}{\mu} = - \frac{e}{c} \PD{t}{\chi}
\end{aligned}
\end{equation}

Or
\begin{equation}\label{eqn:feynmanQEDerrata:680}
\begin{aligned}
\mu = - \frac{e}{c\Hbar} \chi
\end{aligned}
\end{equation}

This also kills off all the additional undesirable terms in the transformed \(\BP^2\) operator (with \(\BP = \Bp - e \BA/c\)), leaving the invariant transformation completely specified

\begin{equation}\label{eqn:feynmanQEDerrata:700}
\begin{aligned}
\BA' &= \BA + \spacegrad \chi \\
\phi' &= \phi - \inv{c} \PD{t}{ \chi } \\
\Psi' &= \exp\left( i \frac{e}{ \Hbar c } \chi \right) \Psi,
\end{aligned}
\end{equation}

This is a fair bit different than the final result as noted in the text, but since that starts with the wrong electrodynamic gauge transformation, this is not too unexpected.

\subsection{Third Lecture.  Page 8 notes}

Here we have

\begin{equation}\label{eqn:feynmanQEDerrata:720}
\begin{aligned}
H = \inv{2m} \left( \Bp - \frac{e}{c}\BA \right)^2 - \frac{e \Hbar}{2 m c} (\Bsigma \cdot \spacegrad \cross \BA) + e V
\end{aligned}
\end{equation}

whereas previously it was

\begin{equation}\label{eqn:feynmanQEDerrata:740}
\begin{aligned}
i \Hbar \PD{t}{\Psi} = \inv{2m}
\left[\sigma \cdot \left( \Bp - \frac{e}{c} \BA \right)\right]
\left[\sigma \cdot \left( \Bp - \frac{e}{c} \BA \right)\right]
 \Psi + e \phi \Psi.
\end{aligned}
\end{equation}

What is this \([\sigma \cdot \Bx]\) notation?  In \citep{wiki:pauli} we have

\begin{equation}\label{eqn:feynmanQEDerrata:760}
\begin{aligned}
\Ba \cdot \Bsigma &= a_i \sigma_i,
\end{aligned}
\end{equation}

Within these square braces it appears that this product is intended to be a tensor product, like so

\begin{equation}\label{eqn:feynmanQEDerrata:780}
\begin{aligned}
\left[\sigma \cdot \Ba\right]
\left[\sigma \cdot \Bb\right]
&\questionEquals \sum_{i,j} a_i \sigma_i b_j \sigma_j \\
&= (\Ba \cdot \Bb) I + i \sigma \cdot (\Ba \cross \Bb).
\end{aligned}
\end{equation}

For \(H\) this would be

\begin{equation}\label{eqn:feynmanQEDerrata:800}
\begin{aligned}
i \Hbar \PD{t}{\Psi}
&= \inv{2m}
\left( \Bp - \frac{e}{c} \BA \right) \cdot \left( \Bp - \frac{e}{c} \BA \right)
- \frac{i^2 \Hbar e}{2m c} \Bsigma \cdot (\spacegrad \cross \BA) \Psi + e \phi \Psi \\
&= \inv{2m}
\left( \Bp - \frac{e}{c} \BA \right) \cdot \left( \Bp - \frac{e}{c} \BA \right)
+ \frac{\Hbar e}{2m c} \Bsigma \cdot (\spacegrad \cross \BA) \Psi + e \phi \Psi.
\end{aligned}
\end{equation}

Ah.  The \(i\) in \(\Bp = -i \Hbar \spacegrad\) is what does away with the \(i\) in the Pauli matrix product.  However, there does appear to be a sign error.

Instead of guessing what Feynman means when he writes Pauli's equation, it would be better to just check what Pauli says.  In

Now, how does one reconcile this with Pauli's text \citep{pauli2000wm} he writes

\begin{equation}\label{eqn:feynmanQEDerrata:820}
\begin{aligned}
H = \inv{2m} \sum_{k=1}^3 \left( p_k - \frac{e}{c}A_k \right)^2 + e \phi V.
\end{aligned}
\end{equation}

There is no \(\spacegrad \cross \BA\) operator term in Pauli's own text, just the scalar operator?

%\section{Followup}
%
%On \href{http://www.feynmanlectures.info/}{In the errata section}, they say ``We invite you to contact us with contributions of errata, including: ... links to
%other lists of errata for works by Richard Feynman''.
%
%Above are what I believe to be typo notes for Feynman's "Quantum Electrodynamics" (Addison-Wesley), the Feb 98 first printing.
%
%However, some of these notes were elaboration for myself and are not entirely restricted to the observed typos.
%
%Since I was unable to find an errata text for this book, I sent a link to these notes as indicated.  Was given the following advice:
%
%``Unfortunately I do not own a copy of Quantum Electrodynamics, so I am unable to check your errata. However, I do know that the book has about 200 pages, from which I can guess that your list is far from complete. I therefore suggest that you continue to keep notes as you read. When you have a (more or less) complete set of errata for the entire book (with errors listed in lexical order, and all "elaborations" removed) you should submit it to your peers for review, and then to the publisher \href{http://www.perseusbooksgroup.com/westview/book_detail.jsp?isbn=0201360756}{Perseus Westview Press}) and/or to the editor of the book. After your list of errata has been completed, reviewed and edited as needed, we would be glad to post a link to it on The Feynman Lectures website.''
%
%I will continue to update these latex typo notes, and eventually, presuming I finish working my way through this text resubmit them.

\EndArticle
%\EndNoBibArticle
