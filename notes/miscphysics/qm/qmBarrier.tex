%
% Copyright � 2012 Peeter Joot.  All Rights Reserved.
% Licenced as described in the file LICENSE under the root directory of this GIT repository.
%

% 
% 
%\documentclass{article}

%\usepackage{amsmath}
\usepackage{mathpazo}

%
% shorthand for bold symbols, convenient for vectors and matrices
%
\newcommand{\Ba}[0]{\mathbf{a}}
\newcommand{\Bb}[0]{\mathbf{b}}
\newcommand{\Bc}[0]{\mathbf{c}}
\newcommand{\Bd}[0]{\mathbf{d}}
\newcommand{\Be}[0]{\mathbf{e}}
\newcommand{\Bf}[0]{\mathbf{f}}
\newcommand{\Bg}[0]{\mathbf{g}}
\newcommand{\Bh}[0]{\mathbf{h}}
\newcommand{\Bi}[0]{\mathbf{i}}
\newcommand{\Bj}[0]{\mathbf{j}}
\newcommand{\Bk}[0]{\mathbf{k}}
\newcommand{\Bl}[0]{\mathbf{l}}
\newcommand{\Bm}[0]{\mathbf{m}}
\newcommand{\Bn}[0]{\mathbf{n}}
\newcommand{\Bo}[0]{\mathbf{o}}
\newcommand{\Bp}[0]{\mathbf{p}}
\newcommand{\Bq}[0]{\mathbf{q}}
\newcommand{\Br}[0]{\mathbf{r}}
\newcommand{\Bs}[0]{\mathbf{s}}
\newcommand{\Bt}[0]{\mathbf{t}}
\newcommand{\Bu}[0]{\mathbf{u}}
\newcommand{\Bv}[0]{\mathbf{v}}
\newcommand{\Bw}[0]{\mathbf{w}}
\newcommand{\Bx}[0]{\mathbf{x}}
\newcommand{\By}[0]{\mathbf{y}}
\newcommand{\Bz}[0]{\mathbf{z}}
\newcommand{\BA}[0]{\mathbf{A}}
\newcommand{\BB}[0]{\mathbf{B}}
\newcommand{\BC}[0]{\mathbf{C}}
\newcommand{\BD}[0]{\mathbf{D}}
\newcommand{\BE}[0]{\mathbf{E}}
\newcommand{\BF}[0]{\mathbf{F}}
\newcommand{\BG}[0]{\mathbf{G}}
\newcommand{\BH}[0]{\mathbf{H}}
\newcommand{\BI}[0]{\mathbf{I}}
\newcommand{\BJ}[0]{\mathbf{J}}
\newcommand{\BK}[0]{\mathbf{K}}
\newcommand{\BL}[0]{\mathbf{L}}
\newcommand{\BM}[0]{\mathbf{M}}
\newcommand{\BN}[0]{\mathbf{N}}
\newcommand{\BO}[0]{\mathbf{O}}
\newcommand{\BP}[0]{\mathbf{P}}
\newcommand{\BQ}[0]{\mathbf{Q}}
\newcommand{\BR}[0]{\mathbf{R}}
\newcommand{\BS}[0]{\mathbf{S}}
\newcommand{\BT}[0]{\mathbf{T}}
\newcommand{\BU}[0]{\mathbf{U}}
\newcommand{\BV}[0]{\mathbf{V}}
\newcommand{\BW}[0]{\mathbf{W}}
\newcommand{\BX}[0]{\mathbf{X}}
\newcommand{\BY}[0]{\mathbf{Y}}
\newcommand{\BZ}[0]{\mathbf{Z}}

\newcommand{\Bzero}[0]{\mathbf{0}}
\newcommand{\Btheta}[0]{\boldsymbol{\theta}}
\newcommand{\Btau}[0]{\boldsymbol{\tau}}
\newcommand{\Bomega}[0]{\boldsymbol{\omega}}

%
% shorthand for unit vectors
%
\newcommand{\acap}[0]{\hat{\Ba}}
\newcommand{\bcap}[0]{\hat{\Bb}}
\newcommand{\ccap}[0]{\hat{\Bc}}
\newcommand{\dcap}[0]{\hat{\Bd}}
\newcommand{\ecap}[0]{\hat{\Be}}
\newcommand{\fcap}[0]{\hat{\Bf}}
\newcommand{\gcap}[0]{\hat{\Bg}}
\newcommand{\hcap}[0]{\hat{\Bh}}
\newcommand{\icap}[0]{\hat{\Bi}}
\newcommand{\jcap}[0]{\hat{\Bj}}
\newcommand{\kcap}[0]{\hat{\Bk}}
\newcommand{\lcap}[0]{\hat{\Bl}}
\newcommand{\mcap}[0]{\hat{\Bm}}
\newcommand{\ncap}[0]{\hat{\Bn}}
\newcommand{\ocap}[0]{\hat{\Bo}}
\newcommand{\pcap}[0]{\hat{\Bp}}
\newcommand{\qcap}[0]{\hat{\Bq}}
\newcommand{\rcap}[0]{\hat{\Br}}
\newcommand{\scap}[0]{\hat{\Bs}}
\newcommand{\tcap}[0]{\hat{\Bt}}
\newcommand{\ucap}[0]{\hat{\Bu}}
\newcommand{\vcap}[0]{\hat{\Bv}}
\newcommand{\wcap}[0]{\hat{\Bw}}
\newcommand{\xcap}[0]{\hat{\Bx}}
\newcommand{\ycap}[0]{\hat{\By}}
\newcommand{\zcap}[0]{\hat{\Bz}}
\newcommand{\thetacap}[0]{\hat{\Btheta}}

%
% to write R^n and C^n in a distinguishable fashion.  Perhaps change this
% to the double lined characters upon figuring out how to do so.
%
\newcommand{\C}[1]{$\mathbb{C}^{#1}$}
\newcommand{\R}[1]{$\mathbb{R}^{#1}$}

%
% various generally useful helpers
%

% derivative of #1 wrt. #2:
\newcommand{\D}[2] {\frac {d#2} {d#1}}

\newcommand{\inv}[1]{\frac{1}{#1}}
\newcommand{\cross}[0]{\times}

\newcommand{\abs}[1]{\lvert{#1}\rvert}
\newcommand{\norm}[1]{\lVert{#1}\rVert}
\newcommand{\innerprod}[2]{\langle{#1}, {#2}\rangle}
\newcommand{\dotprod}[2]{{#1} \cdot {#2}}
\newcommand{\bdotprod}[2]{\left({#1} \cdot {#2}\right)}
\newcommand{\crossprod}[2]{{#1} \cross {#2}}
\newcommand{\tripleprod}[3]{\dotprod{\left(\crossprod{#1}{#2}\right)}{#3}}

\DeclareMathOperator{\Proj}{Proj}
\DeclareMathOperator{\Span}{span}
\DeclareMathOperator{\Sgn}{sgn}
\DeclareMathOperator{\Area}{Area}
\DeclareMathOperator{\Volume}{Volume}

%
% A few miscellaneous things specific to this document
%
\newcommand{\crossop}[1]{\crossprod{#1}{}}

% R2 vector.
\newcommand{\VectorTwo}[2]{
\begin{bmatrix}
 {#1} \\
 {#2}
\end{bmatrix}
}

\newcommand{\VectorN}[1]{
\begin{bmatrix}
{#1}_1 \\
{#1}_2 \\
\vdots \\
{#1}_N \\
\end{bmatrix}
}

\newcommand{\DETuvij}[4]{
\begin{vmatrix}
 {#1}_{#3} & {#1}_{#4} \\
 {#2}_{#3} & {#2}_{#4}
\end{vmatrix}
}

\newcommand{\DETuvwijk}[6]{
\begin{vmatrix}
 {#1}_{#4} & {#1}_{#5} & {#1}_{#6} \\
 {#2}_{#4} & {#2}_{#5} & {#2}_{#6} \\
 {#3}_{#4} & {#3}_{#5} & {#3}_{#6}
\end{vmatrix}
}

\newcommand{\DETuvwxijkl}[8]{
\begin{vmatrix}
 {#1}_{#5} & {#1}_{#6} & {#1}_{#7} & {#1}_{#8} \\
 {#2}_{#5} & {#2}_{#6} & {#2}_{#7} & {#2}_{#8} \\
 {#3}_{#5} & {#3}_{#6} & {#3}_{#7} & {#3}_{#8} \\
 {#4}_{#5} & {#4}_{#6} & {#4}_{#7} & {#4}_{#8} \\
\end{vmatrix}
}

%\newcommand{\DETuvwxyijklm}[10]{
%\begin{vmatrix}
% {#1}_{#6} & {#1}_{#7} & {#1}_{#8} & {#1}_{#9} & {#1}_{#10} \\
% {#2}_{#6} & {#2}_{#7} & {#2}_{#8} & {#2}_{#9} & {#2}_{#10} \\
% {#3}_{#6} & {#3}_{#7} & {#3}_{#8} & {#3}_{#9} & {#3}_{#10} \\
% {#4}_{#6} & {#4}_{#7} & {#4}_{#8} & {#4}_{#9} & {#4}_{#10} \\
% {#5}_{#6} & {#5}_{#7} & {#5}_{#8} & {#5}_{#9} & {#5}_{#10}
%\end{vmatrix}
%}

% R3 vector.
\newcommand{\VectorThree}[3]{
\begin{bmatrix}
 {#1} \\
 {#2} \\
 {#3}
\end{bmatrix}
}


%%<misc>
%
\newcommand{\Abs}[1]{{\left\lvert{#1}\right\rvert}}
\newcommand{\spacegrad}[0]{\boldsymbol{\nabla}}
\newcommand{\grad}[0]{\nabla}
\newcommand{\LL}[0]{\mathcal{L}}

% == \partial_{#1} {#2}
\newcommand{\PD}[2]{\frac{\partial {#2}}{\partial {#1}}}
% inline variant
\newcommand{\PDi}[2]{{\partial {#2}}/{\partial {#1}}}

\newcommand{\PDD}[3]{\frac{\partial^2 {#3}}{\partial {#1}\partial {#2}}}
%\newcommand{\PDd}[2]{\frac{\partial^2 {#2}}{{\partial{#1}}^2}}
\newcommand{\PDsq}[2]{\frac{\partial^2 {#2}}{(\partial {#1})^2}}

\newcommand{\Partial}[2]{\frac{\partial {#1}}{\partial {#2}}}
\DeclareMathOperator{\RejName}{Rej}
\newcommand{\Rej}[2]{\RejName_{#1}\left( {#2} \right)}
\newcommand{\Rm}[1]{\mathbb{R}^{#1}}
\newcommand{\Cm}[1]{\mathbb{C}^{#1}}
\newcommand{\conj}[0]{{*}}

%</misc>

% <grade selection>
%
\newcommand{\gpgrade}[2] {{\left\langle{{#1}}\right\rangle}_{#2}}

\newcommand{\gpgradezero}[1] {\gpgrade{#1}{}}
%\newcommand{\gpscalargrade}[1] {{\left\langle{{#1}}\right\rangle}}
%\newcommand{\gpgradezero}[1] {\gpgrade{#1}{0}}

%\newcommand{\gpgradeone}[1] {{\left\langle{{#1}}\right\rangle}_{1}}
\newcommand{\gpgradeone}[1] {\gpgrade{#1}{1}}

\newcommand{\gpgradetwo}[1] {\gpgrade{#1}{2}}
\newcommand{\gpgradethree}[1] {\gpgrade{#1}{3}}
\newcommand{\gpgradefour}[1] {\gpgrade{#1}{4}}
%
% </grade selection>



\newcommand{\adot}[0]{{\dot{a}}}
\newcommand{\bdot}[0]{{\dot{b}}}
% taken for centered dot:
%\newcommand{\cdot}[0]{{\dot{c}}}
%\newcommand{\ddot}[0]{{\dot{d}}}
\newcommand{\edot}[0]{{\dot{e}}}
\newcommand{\fdot}[0]{{\dot{f}}}
\newcommand{\gdot}[0]{{\dot{g}}}
\newcommand{\hdot}[0]{{\dot{h}}}
\newcommand{\idot}[0]{{\dot{i}}}
\newcommand{\jdot}[0]{{\dot{j}}}
\newcommand{\kdot}[0]{{\dot{k}}}
\newcommand{\ldot}[0]{{\dot{l}}}
\newcommand{\mdot}[0]{{\dot{m}}}
\newcommand{\ndot}[0]{{\dot{n}}}
%\newcommand{\odot}[0]{{\dot{o}}}
\newcommand{\pdot}[0]{{\dot{p}}}
\newcommand{\qdot}[0]{{\dot{q}}}
\newcommand{\rdot}[0]{{\dot{r}}}
\newcommand{\sdot}[0]{{\dot{s}}}
\newcommand{\tdot}[0]{{\dot{t}}}
\newcommand{\udot}[0]{{\dot{u}}}
\newcommand{\vdot}[0]{{\dot{v}}}
\newcommand{\wdot}[0]{{\dot{w}}}
\newcommand{\xdot}[0]{{\dot{x}}}
\newcommand{\ydot}[0]{{\dot{y}}}
\newcommand{\zdot}[0]{{\dot{z}}}
\newcommand{\addot}[0]{{\ddot{a}}}
\newcommand{\bddot}[0]{{\ddot{b}}}
\newcommand{\cddot}[0]{{\ddot{c}}}
%\newcommand{\dddot}[0]{{\ddot{d}}}
\newcommand{\eddot}[0]{{\ddot{e}}}
\newcommand{\fddot}[0]{{\ddot{f}}}
\newcommand{\gddot}[0]{{\ddot{g}}}
\newcommand{\hddot}[0]{{\ddot{h}}}
\newcommand{\iddot}[0]{{\ddot{i}}}
\newcommand{\jddot}[0]{{\ddot{j}}}
\newcommand{\kddot}[0]{{\ddot{k}}}
\newcommand{\lddot}[0]{{\ddot{l}}}
\newcommand{\mddot}[0]{{\ddot{m}}}
\newcommand{\nddot}[0]{{\ddot{n}}}
\newcommand{\oddot}[0]{{\ddot{o}}}
\newcommand{\pddot}[0]{{\ddot{p}}}
\newcommand{\qddot}[0]{{\ddot{q}}}
\newcommand{\rddot}[0]{{\ddot{r}}}
\newcommand{\sddot}[0]{{\ddot{s}}}
\newcommand{\tddot}[0]{{\ddot{t}}}
\newcommand{\uddot}[0]{{\ddot{u}}}
\newcommand{\vddot}[0]{{\ddot{v}}}
\newcommand{\wddot}[0]{{\ddot{w}}}
\newcommand{\xddot}[0]{{\ddot{x}}}
\newcommand{\yddot}[0]{{\ddot{y}}}
\newcommand{\zddot}[0]{{\ddot{z}}}

%<bold and dot greek symbols>
%

\newcommand{\Deltadot}[0]{{\dot{\Delta}}}
\newcommand{\Gammadot}[0]{{\dot{\Gamma}}}
\newcommand{\Lambdadot}[0]{{\dot{\Lambda}}}
\newcommand{\Omegadot}[0]{{\dot{\Omega}}}
\newcommand{\Phidot}[0]{{\dot{\Phi}}}
\newcommand{\Pidot}[0]{{\dot{\Pi}}}
\newcommand{\Psidot}[0]{{\dot{\Psi}}}
\newcommand{\Sigmadot}[0]{{\dot{\Sigma}}}
\newcommand{\Thetadot}[0]{{\dot{\Theta}}}
\newcommand{\Upsilondot}[0]{{\dot{\Upsilon}}}
\newcommand{\Xidot}[0]{{\dot{\Xi}}}
\newcommand{\alphadot}[0]{{\dot{\alpha}}}
\newcommand{\betadot}[0]{{\dot{\beta}}}
\newcommand{\chidot}[0]{{\dot{\chi}}}
\newcommand{\deltadot}[0]{{\dot{\delta}}}
\newcommand{\epsilondot}[0]{{\dot{\epsilon}}}
\newcommand{\etadot}[0]{{\dot{\eta}}}
\newcommand{\gammadot}[0]{{\dot{\gamma}}}
\newcommand{\kappadot}[0]{{\dot{\kappa}}}
\newcommand{\lambdadot}[0]{{\dot{\lambda}}}
\newcommand{\mudot}[0]{{\dot{\mu}}}
\newcommand{\nudot}[0]{{\dot{\nu}}}
\newcommand{\omegadot}[0]{{\dot{\omega}}}
\newcommand{\phidot}[0]{{\dot{\phi}}}
\newcommand{\pidot}[0]{{\dot{\pi}}}
\newcommand{\psidot}[0]{{\dot{\psi}}}
\newcommand{\rhodot}[0]{{\dot{\rho}}}
\newcommand{\sigmadot}[0]{{\dot{\sigma}}}
\newcommand{\taudot}[0]{{\dot{\tau}}}
\newcommand{\thetadot}[0]{{\dot{\theta}}}
\newcommand{\upsilondot}[0]{{\dot{\upsilon}}}
\newcommand{\varepsilondot}[0]{{\dot{\varepsilon}}}
\newcommand{\varphidot}[0]{{\dot{\varphi}}}
\newcommand{\varpidot}[0]{{\dot{\varpi}}}
\newcommand{\varrhodot}[0]{{\dot{\varrho}}}
\newcommand{\varsigmadot}[0]{{\dot{\varsigma}}}
\newcommand{\varthetadot}[0]{{\dot{\vartheta}}}
\newcommand{\xidot}[0]{{\dot{\xi}}}
\newcommand{\zetadot}[0]{{\dot{\zeta}}}

\newcommand{\Deltaddot}[0]{{\ddot{\Delta}}}
\newcommand{\Gammaddot}[0]{{\ddot{\Gamma}}}
\newcommand{\Lambdaddot}[0]{{\ddot{\Lambda}}}
\newcommand{\Omegaddot}[0]{{\ddot{\Omega}}}
\newcommand{\Phiddot}[0]{{\ddot{\Phi}}}
\newcommand{\Piddot}[0]{{\ddot{\Pi}}}
\newcommand{\Psiddot}[0]{{\ddot{\Psi}}}
\newcommand{\Sigmaddot}[0]{{\ddot{\Sigma}}}
\newcommand{\Thetaddot}[0]{{\ddot{\Theta}}}
\newcommand{\Upsilonddot}[0]{{\ddot{\Upsilon}}}
\newcommand{\Xiddot}[0]{{\ddot{\Xi}}}
\newcommand{\alphaddot}[0]{{\ddot{\alpha}}}
\newcommand{\betaddot}[0]{{\ddot{\beta}}}
\newcommand{\chiddot}[0]{{\ddot{\chi}}}
\newcommand{\deltaddot}[0]{{\ddot{\delta}}}
\newcommand{\epsilonddot}[0]{{\ddot{\epsilon}}}
\newcommand{\etaddot}[0]{{\ddot{\eta}}}
\newcommand{\gammaddot}[0]{{\ddot{\gamma}}}
\newcommand{\kappaddot}[0]{{\ddot{\kappa}}}
\newcommand{\lambdaddot}[0]{{\ddot{\lambda}}}
\newcommand{\muddot}[0]{{\ddot{\mu}}}
\newcommand{\nuddot}[0]{{\ddot{\nu}}}
\newcommand{\omegaddot}[0]{{\ddot{\omega}}}
\newcommand{\phiddot}[0]{{\ddot{\phi}}}
\newcommand{\piddot}[0]{{\ddot{\pi}}}
\newcommand{\psiddot}[0]{{\ddot{\psi}}}
\newcommand{\rhoddot}[0]{{\ddot{\rho}}}
\newcommand{\sigmaddot}[0]{{\ddot{\sigma}}}
\newcommand{\tauddot}[0]{{\ddot{\tau}}}
\newcommand{\thetaddot}[0]{{\ddot{\theta}}}
\newcommand{\upsilonddot}[0]{{\ddot{\upsilon}}}
\newcommand{\varepsilonddot}[0]{{\ddot{\varepsilon}}}
\newcommand{\varphiddot}[0]{{\ddot{\varphi}}}
\newcommand{\varpiddot}[0]{{\ddot{\varpi}}}
\newcommand{\varrhoddot}[0]{{\ddot{\varrho}}}
\newcommand{\varsigmaddot}[0]{{\ddot{\varsigma}}}
\newcommand{\varthetaddot}[0]{{\ddot{\vartheta}}}
\newcommand{\xiddot}[0]{{\ddot{\xi}}}
\newcommand{\zetaddot}[0]{{\ddot{\zeta}}}

\newcommand{\BDelta}[0]{\boldsymbol{\Delta}}
\newcommand{\BGamma}[0]{\boldsymbol{\Gamma}}
\newcommand{\BLambda}[0]{\boldsymbol{\Lambda}}
\newcommand{\BOmega}[0]{\boldsymbol{\Omega}}
\newcommand{\BPhi}[0]{\boldsymbol{\Phi}}
\newcommand{\BPi}[0]{\boldsymbol{\Pi}}
\newcommand{\BPsi}[0]{\boldsymbol{\Psi}}
\newcommand{\BSigma}[0]{\boldsymbol{\Sigma}}
\newcommand{\BTheta}[0]{\boldsymbol{\Theta}}
\newcommand{\BUpsilon}[0]{\boldsymbol{\Upsilon}}
\newcommand{\BXi}[0]{\boldsymbol{\Xi}}
\newcommand{\Balpha}[0]{\boldsymbol{\alpha}}
\newcommand{\Bbeta}[0]{\boldsymbol{\beta}}
\newcommand{\Bchi}[0]{\boldsymbol{\chi}}
\newcommand{\Bdelta}[0]{\boldsymbol{\delta}}
\newcommand{\Bepsilon}[0]{\boldsymbol{\epsilon}}
\newcommand{\Beta}[0]{\boldsymbol{\eta}}
\newcommand{\Bgamma}[0]{\boldsymbol{\gamma}}
\newcommand{\Bkappa}[0]{\boldsymbol{\kappa}}
\newcommand{\Blambda}[0]{\boldsymbol{\lambda}}
\newcommand{\Bmu}[0]{\boldsymbol{\mu}}
\newcommand{\Bnu}[0]{\boldsymbol{\nu}}
%\newcommand{\Bomega}[0]{\boldsymbol{\omega}}
\newcommand{\Bphi}[0]{\boldsymbol{\phi}}
\newcommand{\Bpi}[0]{\boldsymbol{\pi}}
\newcommand{\Bpsi}[0]{\boldsymbol{\psi}}
\newcommand{\Brho}[0]{\boldsymbol{\rho}}
\newcommand{\Bsigma}[0]{\boldsymbol{\sigma}}
%\newcommand{\Btau}[0]{\boldsymbol{\tau}}
%\newcommand{\Btheta}[0]{\boldsymbol{\theta}}
\newcommand{\Bupsilon}[0]{\boldsymbol{\upsilon}}
\newcommand{\Bvarepsilon}[0]{\boldsymbol{\varepsilon}}
\newcommand{\Bvarphi}[0]{\boldsymbol{\varphi}}
\newcommand{\Bvarpi}[0]{\boldsymbol{\varpi}}
\newcommand{\Bvarrho}[0]{\boldsymbol{\varrho}}
\newcommand{\Bvarsigma}[0]{\boldsymbol{\varsigma}}
\newcommand{\Bvartheta}[0]{\boldsymbol{\vartheta}}
\newcommand{\Bxi}[0]{\boldsymbol{\xi}}
\newcommand{\Bzeta}[0]{\boldsymbol{\zeta}}
%
%</bold and dot greek symbols>
%<infrequent>
%
%\newcommand{\AreaOp}[1]{\AName_{#1}}
%\newcommand{\Babs}[0]{\abs{\BB}}
%\newcommand{\Bcap}[0]{\hat{\BB}}
%\newcommand{\BrPrimeRej}[0]{\rcap(\rcap \wedge \Br')}
%\newcommand{\CA}[0]{\mathcal{A}}
%\newcommand{\Cos}[1]{\cos{\left({#1}\right)}}
%\newcommand{\Det}[1] {\abs{#1}}
%\newcommand{\Dsq}[2] {\frac {\partial^2 {#1}} {\partial {#2}^2}}
%\newcommand{\Exp}[1]{\exp{\left({#1}\right)}}
%\newcommand{\Norm}[1]{\left\lVert{#1}\right\rVert}
%\newcommand{\Sin}[1]{\sin{\left({#1}\right)}}
%\newcommand{\T}[0]{\text{T}}
%\newcommand{\VolumeOp}[1]{\VName_{#1}}
%\newcommand{\agrad}[0]{\Ba \cdot \nabla}
%\newcommand{\alphacap}[0]{\hat{\boldsymbol{\alpha}}}
%\newcommand{\Fcap}[0]{\hat{\BF}}
%\newcommand{\bithree}[0]{{\Bi}_3}
%\newcommand{\bxa}[0]{\Bx\Ba}
%\newcommand{\coordvec}[2]{
%\newcommand{\costheta}[0]{\acap \cdot \xcap}
%\newcommand{\ddt}[1]{\ddot{#1}}
%\newcommand{\ddu}[1] {\frac {d{#1}} {du}}
%\newcommand{\dsqxj}[2] {\frac {\partial^2 {#1}} {\partial {x_{#2}}^2}}
%\newcommand{\dtheta}[1]{\frac{d {#1}}{d \theta}}
%\newcommand{\dt}[1]{\dot{#1}}
%\newcommand{\dt}[1]{\frac{d {#1}}{dt}}
%\newcommand{\dxj}[2] {\frac {\partial {#1}} {\partial {x_{#2}}}}
%\newcommand{\halfPhi}[0]{\frac{\phi}{2}}
%\newcommand{\half}[0]{\inv{2}}
%\newcommand{\inv}[1]{\frac{1}{#1}}
%\newcommand{\laplacian}[0]{\nabla^2}
%\newcommand{\matrixoftx}[3]{
%\newcommand{\nrrp}[0]{\norm{\rcap \wedge \Br'}}
%\newcommand{\oiint}{\bigcirc \hspace{-1.4em} \int \hspace{-.8em} \int}
%\newcommand{\transpose}[1]{{#1}^{\text{T}}}
%\newcommand{\transpose}[1]{{{#1}^{\TextTranspose}}}
%\newcommand{\transpose}[1]{{{#1}^{\text{T}}}}
%\newcommand{\barA}[0]{\bar{A}}
%\newcommand{\qbar}[0]{\bar{q}}
%\newcommand{\qdotbar}[0]{\dot{\bar{q}}}
%
%</infrequent>





%\usepackage{listings}
%\usepackage{txfonts} % for ointctr... (also appears to make "prettier" \int and \sum's)
% makes \grad look funny though (almost like spacegrad, but narrower)
%\usepackage[bookmarks=true]{hyperref}

%\usepackage{color,cite,graphicx}
   % use colour in the document, put your citations as [1-4]
   % rather than [1,2,3,4] (it looks nicer, and the extended LaTeX2e
   % graphics package. 
%\usepackage{latexsym,amssymb,epsf} % do not remember if these are
   % needed, but their inclusion can not do any damage


\chapter{One dimensional rectangular Quantum barrier penetration problem}
\label{chap:qmBarrier}
%\author{Peeter Joot \quad peeter.joot@gmail.com }
\date{ May 11, 2009.  qmBarrier.tex }

%\begin{document}

%\maketitle{}
%\tableofcontents
\section{Motivation}

My first attempt at the probability current calculation for this
while doing the chapter 11 problems of \citep{bohm1989qt} led to (algebraic) trouble.

Here is a new try from scratch.

There is very little physics here, just a lot of algebra, but let us try
to get all this algebra correct this time.

\section{Wave functions.  Find the coefficients}

The potential is taken to be zero everywhere except $x \in [0,a]$, where it is
$V$.  This divides the problem into three regions, I, II, and III, for 
before ($x<0$), in and after the barrier, and the wave functions for the 
$E <V$ case are respectively

\begin{equation}
\psi =
\left\{
\begin{array}{l l}
A e^{i k x} + B e^{-i k x} & \quad \mbox{if $x <0$} \\
C e^{ \beta(x-a)} + D e^{ -\beta(x-a)} & \quad \mbox{if $x \in [0,a]$} \\
E_0 e^{i k(x-a)} & \quad \mbox{if $x >0$} \\
\end{array}
\right.
\end{equation}

Following \citep{mcmahon2005qmd} the wave functions have been written in terms of wave number 
$k = \sqrt{2 m E_0}/\Hbar$, and $\beta = \sqrt{2 m (V-E_0)}/\Hbar$.  Unfortunately the 
\citep{mcmahon2005qmd} has too many typos to attempt to follow directly.

Unlike Bohm or QMD wave functions expressed using functions of $x-a$ are used here since a first
attempt at solution indicated this would be natural.

\subsection{Equality at \texorpdfstring{$x=a$}{x = a}}

Equality of the wave functions and derivatives at $x=a$ gives

\begin{align*}
C + D &= E \\
C - D &= \frac{i k}{\beta}E
\end{align*}

which has solutions

\begin{align*}
C &= \frac{E}{2}( 1 + i k/\beta ) \\
D &= \frac{E}{2}( 1 - i k/\beta )
\end{align*}

Two of the free variables of the wave equation are now eliminated, and the wave function in the barrier region can now be written as
\begin{align}\label{eqn:qm_barrier:psiInBarrier}
\psi =
\frac{E}{2}\left( \left( 1 + \frac{i k}{\beta} \right) e^{ \beta(x-a)} + \left( 1 - \frac{i k}{\beta} \right) e^{ -\beta(x-a)} \right)
\end{align}

\subsection{Equality at \texorpdfstring{$x=0$}{x equal zero}}

Equating values and first derivatives at $x=0$ we have

\begin{align*}
A + B &=
\frac{E}{2}\left( \left( 1 + \frac{i k}{\beta} \right) e^{ -\beta a} + \left( 1 - \frac{i k}{\beta} \right) e^{ \beta a } \right) \\
A - B &=
\frac{\beta E}{2 i k}\left( \left( 1 + \frac{i k}{\beta} \right) e^{ -\beta a} - \left( 1 - \frac{i k}{\beta} \right) e^{ \beta a } \right)
\end{align*}

Taking sums and differences we have

\begin{align*}
A &= \frac{E}{4}\left( \left(1 + \frac{\beta}{ik}\right)\left( 1 + \frac{i k}{\beta} \right) e^{ -\beta a} + \left(1 - \frac{\beta}{ik}\right)\left( 1 - \frac{i k}{\beta} \right) e^{ \beta a } \right) \\
B &= \frac{E}{4}\left( \left(1 - \frac{\beta}{ik}\right)\left( 1 + \frac{i k}{\beta} \right) e^{ -\beta a} + \left(1 + \frac{\beta}{ik}\right)\left( 1 - \frac{i k}{\beta} \right) e^{ \beta a } \right) \\
\end{align*}

Expanding the products first for $B$

\begin{align*}
B 
&= \frac{E}{4}\left( \left(\frac{-\beta}{ik} + \frac{ik}{\beta} \right) e^{ -\beta a} + \left(\frac{\beta}{ik} - \frac{ik}{\beta} \right) e^{ \beta a } \right) \\
&= \frac{iE}{4}\left( \left(\frac{\beta}{k} + \frac{k}{\beta} \right) e^{ -\beta a} - \left(\frac{\beta}{k} + \frac{k}{\beta} \right) e^{ \beta a } \right) \\
&= \frac{iE}{4}\left(\frac{\beta}{k} + \frac{k}{\beta} \right) \left( e^{ -\beta a} - e^{ \beta a } \right) \\
&= \frac{-iE}{2}\left(\frac{\beta}{k} + \frac{k}{\beta} \right) \sinh\left( \beta a \right) \\
\end{align*}

Now for $A$ we have
\begin{align*}
A &= \frac{E}{4}\left( \left(2 + \frac{\beta}{ik} + \frac{ik}{\beta} \right) e^{ -\beta a} + \left(2 - \frac{\beta}{ik} - \frac{ik}{\beta} \right) e^{ \beta a } \right) \\
\end{align*}

The factor of exponentials are complex conjugates and can be put into polar form to simplify.  Writing

\begin{align*}
\gamma 
&= 2 + \frac{\beta}{ik} + \frac{ik}{\beta} \\
&= 2 + i \left( \frac{k}{\beta} -\frac{\beta}{k} \right) \\
&= \mu e^{i \theta} 
\end{align*}

Where 

\begin{align}\label{eqn:qm_barrier:muAndTheta}
\mu^2 &= 4 + \left( \frac{k}{\beta} -\frac{\beta}{k} \right)^2 \\
\theta &= \Atan\left( \inv{2} \left(\frac{k}{\beta} - \frac{\beta}{k} \right) \right)
\end{align}

We have

\begin{align*}
A 
&= \frac{\mu E}{4}\left( e^{ i\theta -\beta a} + e^{ -(i \theta - \beta a) } \right) \\
&= \frac{\mu E}{2}\cosh\left( i\theta -\beta a \right) \\
\end{align*}

For notational consistency it looks desirable to write something like

\begin{align}\label{eqn:qm_barrier:nuDefined}
\nu &= -i \left(\frac{\beta}{k} + \frac{k}{\beta} \right) 
\end{align}

Which leaves us with the wave function in the $x<0$ region as
\begin{align}\label{eqn:qm_barrier:incidentAndReflected}
\psi &=
\frac{\mu E}{2}\cosh\left( i\theta -\beta a \right) e^{ i k x }
+\frac{\nu E}{2}\sinh\left( \beta a \right) e^{ -i k x }
\end{align}

Observed later (after trying to sum the transmission and reflection coefficients) is the fact that we can actually write

\begin{align*}
\mu^2 = \Abs{\nu}^2
\end{align*}

So an additional simplification of this wave function is possible.

\begin{align}\label{eqn:qm_barrier:incidentAndReflectedSimplified}
\psi &=
\frac{E}{2} \left(\frac{\beta}{k} + \frac{k}{\beta} \right) 
\left(
\cosh\left( i\theta -\beta a \right) e^{ i k x }
-i \sinh\left( \beta a \right) e^{ -i k x }
\right)
\end{align}

\subsection{(Aside) Barrier wave function in polar form}

Having seen how the polar form simplifies the final expression of the first region wave function, it can be seen that 
something similar can be done in the barrier region.  

In \eqnref{eqn:qm_barrier:psiInBarrier}, we can utilize polar form for the $1+ik/\beta$ constant and its conjugate.  Writing

\begin{align}
\tan\phi &= \frac{k}{\beta} = \sqrt{\frac{E_0}{V-E_0}}
\end{align}

we then have
\begin{align*}
\psi 
&=
\frac{E}{2}\sqrt{1 + \frac{k^2}{\beta^2}} \left( e^{i\phi} e^{ \beta(x-a)} + e^{-i\phi} e^{ -\beta(x-a)} \right) \\
&=
\frac{E}{2}\sqrt{\frac{V}{V-E_0}} \left( e^{i\phi} e^{ \beta(x-a)} + e^{-i\phi} e^{ -\beta(x-a)} \right) \\
\end{align*}

So we have in the $x \in [0,a]$ region
\begin{align}\label{eqn:qm_barrier:inBarrierCosh}
\psi &= E \sqrt{1 + \frac{k^2}{\beta^2}} \cosh\left( i\phi + \beta(x-a) \right) 
\end{align}

Or in terms of energies
\begin{align}
\psi &= E 
\sqrt{\frac{V}{V-E_0}}
\cosh\left( i\phi + \beta(x-a) \right) 
\end{align}

Utilizing 
\eqnref{eqn:qm_barrier:inBarrierCosh}
should produce the same result for $x<0$.  Equality at $x=0$ then gives

\begin{align*}
A + B &= E \sqrt{1 + \frac{k^2}{\beta^2}} \cosh\left( i\phi -\beta a \right) \\
A - B &= E \frac{i\beta}{k} \sqrt{1 + \frac{k^2}{\beta^2}} \sinh\left( i\phi -\beta a \right) \\
\end{align*}

Or

\begin{align*}
A &= \frac{E}{2} \sqrt{1 + \frac{k^2}{\beta^2}} \left(
\cosh\left( i\phi -\beta a \right) + \frac{i\beta}{k} \sinh\left( i\phi -\beta a \right) 
\right) \\
B &= \frac{E}{2} \sqrt{1 + \frac{k^2}{\beta^2}} \left(
\cosh\left( i\phi -\beta a \right) - \frac{i\beta}{k} \sinh\left( i\phi -\beta a \right) 
\right) \\
\end{align*}

This produces values for both $A$, $B$ very directly, but this 
form of solution in the $x<0$ region begs for some additional reduction.  Let us
defer trying that until after the $T$, and $R$ computations.

\section{Probability currents and densities}

\subsection{Probability densities}

\subsubsection{Barrier region probability density}

Using \eqnref{eqn:qm_barrier:inBarrierCosh} we have for the barrier region

\begin{align*}
\rho 
&= \psi\psi^\conj \\
&= \Abs{E}^2 
\left(1 + \frac{k^2}{\beta^2}\right) \Abs{\cosh\left( i\phi + \beta(x-a) \right) }^2
\end{align*}

Utilizing the identities \eqnref{eqn:qm_barrier:coshSquared} \eqnref{eqn:qm_barrier:sinhSquared} \eqnref{eqn:qm_barrier:cosineTwiceArcTan}, we have

\begin{align*}
\rho 
&= \inv{2} \Abs{E}^2 
\left(1 + \frac{k^2}{\beta^2}\right) \left( \cosh( 2 \beta(x-a)) + \cos(2\phi) \right) \\
&= \inv{2} \Abs{E}^2 
\left(1 + \frac{k^2}{\beta^2}\right) \left( (2 \sinh^2( \beta(x-a)) + 1) + \frac{1 - k^2/\beta^2}{1 + k^2/\beta^2}\right) \\
&= \inv{2} \Abs{E}^2 \left(
\left(1 + \frac{k^2}{\beta^2}\right) (2 \sinh^2( \beta(x-a)) + 1) + 1 - \frac{k^2}{\beta^2} \right) \\
\end{align*}

This is
\begin{align}\label{eqn:qm_barrier:pDensityInBarrier}
\rho
&= \Abs{E}^2 
\left( 1 + \left(1 + \frac{k^2}{\beta^2}\right) \sinh^2( \beta(x-a)) \right) 
\end{align}

Or

\begin{align}
\rho
&= \Abs{E}^2 
\left( 1 + \frac{V}{V-E_0} \sinh^2( \beta(x-a)) \right) 
\end{align}

In particular observe that at $x=a$ we have $\rho = \Abs{E^2}$ the expected probability density for the entirety 
of the $x>a$ transmission region.

\subsubsection{Total probability density in the incident region}

To calculate the current in the potential free region $x<0$, write for short

\begin{align}\label{eqn:qm_barrier:forShort}
\psi 
&= u e^{ik x} + v e^{-ik x} = u \epsilon + v \epsilon^\conj
\end{align}

The probability density is

\begin{align*}
\rho 
&=
(u^\conj \epsilon^\conj + v^\conj \epsilon)
(u \epsilon + v \epsilon^\conj) \\
&= 
\Abs{u}^2 + \Abs{v}^2 + 
u^\conj v (\epsilon^\conj)^2 
+
v^\conj u \epsilon^2 \\
&=
\Abs{u}^2 + \Abs{v}^2 + 
+ 2 \Re(v^\conj u \epsilon^2) \\
\end{align*}

Putting back in the factors from 
\eqnref{eqn:qm_barrier:incidentAndReflectedSimplified}
we have

\begin{align*}
\rho
&=
\frac{\Abs{E}^2}{4}
\left(
\frac{\beta}{k}
+\frac{k}{\beta}
\right)^2 
\left(
\Abs{\cosh\left( i\theta -\beta a \right)}^2
+\sinh^2\left( \beta a \right) 
+ 2 \Re\left( i \sinh(\beta a) \cosh(i\theta - \beta a) e^{2 i k x} \right)
\right)
\end{align*}

Now this can likely be reduced further without too much trouble but doing so does not seem terribly interesting.

\subsection{Probability current densities}

\subsubsection{Current densities outside of the barrier region}

We want to calculate the probability currents in each of the regions, taking the $x$ components of the vector 
probability current

\begin{align}
\BJ
&=
\frac{\Hbar}{2 m i}
\left(
\psi^\conj \grad \psi
-\psi \grad \psi^\conj
\right)
\end{align}

In the $x<0$ region we can calculate the incident, reflected, and the total currents.  For the incident current we have

\begin{align*}
J_i
&= \frac{\Hbar}{2mi}\left( A^\conj e^{-i k x} (ik)A e^{i k x} -Ae^{i k x} (-ik)A^\conj e^{-i k x} \right) \\
\end{align*}

which reduces to just
\begin{align}
J_i &= \frac{\Hbar k \Abs{A}^2 }{m}
\end{align}

by comparison we then also have for the reflected and transmitted currents, respectively

\begin{align}
J_r &= \frac{\Hbar k \Abs{B}^2 }{m} \\
J_t &= \frac{\Hbar k \Abs{E}^2 }{m}
\end{align}

That leaves only the current in the barrier region.

\subsubsection{Current densities in the barrier region}

Starting from \eqnref{eqn:qm_barrier:psiInBarrier}, writing $a = 1 + ik/\beta$, and $b=e^{\beta(x-a)}$ we have

\begin{align*}
\psi &= \frac{E}{2} (a b + a^\conj/b)
\end{align*}

For the current we then have

\begin{align*}
\inv{\beta}&\left(\psi^\conj \psi' - \psi (\psi^\conj)' \right) \\
&=
\frac{\Abs{E}^2}{4}
\left(
(a^\conj b + a/b) (a b - a^\conj/b)
-(a b + a^\conj/b) (a^\conj b - a/b)
\right) \\
&=
\frac{\Abs{E}^2}{2} \left(a^2 - (a^\conj)^2 \right) \\
&=
\frac{\Abs{E}^2}{2} \left( 
(1 + ik/\beta)^2
-(1 - ik/\beta)^2
 \right) \\
&=
\Abs{E}^2 \frac{2ik}{\beta}
\end{align*}

So we have in the barrier

\begin{align*}
J_b 
&=
\frac{\Hbar k \Abs{E}^2 }{m}
\end{align*}

So despite the probability density itself varying with position once into and past the barrier, the current density itself
retains this
constant value.

\subsubsection{Total current density in the incident region}

As above for the density using the shorthand of \eqnref{eqn:qm_barrier:forShort}, the total current is

\begin{align*}
J &=
\frac{\Hbar}{2mi}\left(
(u^\conj \epsilon^\conj + v^\conj \epsilon)
(i k )
(u \epsilon - v \epsilon^\conj)
-
(u \epsilon + v \epsilon^\conj)
(i k )
(-u^\conj \epsilon^\conj + v^\conj \epsilon) 
\right)
\\
&=
\frac{\Hbar k}{m}\left( \Abs{u}^2 - \Abs{v}^2 \right)
\end{align*}

Putting back in the factors from 
\eqnref{eqn:qm_barrier:incidentAndReflectedSimplified}
we have

\begin{align*}
J &=
\frac{\Hbar k}{m}\left( \Abs{u}^2 - \Abs{v}^2 \right)
\end{align*}

\begin{align*}
J
&=
\frac{\Hbar k}{m}
\frac{\Abs{E}^2}{4} \left(\frac{\beta}{k} + \frac{k}{\beta} \right)^2
\left(
\Abs{\cosh\left( i\theta -\beta a \right) }^2
- \sinh^2\left( \beta a \right) 
\right) \\
&=
\frac{\Hbar k}{m}
\frac{\Abs{E}^2}{4} \left(\frac{\beta}{k} + \frac{k}{\beta} \right)^2
%(cosh 2u)/2 = sinh^2 u + 1/2
\left(
\inv{2}\cosh(2\beta a) + \inv{2}\cos(2\theta) 
- \sinh^2( \beta a ) 
\right) \\
&=
\frac{\Hbar k}{m}
\frac{\Abs{E}^2}{8} \left(\frac{\beta}{k} + \frac{k}{\beta} \right)^2
\left(
1 + \cos(2\theta) 
\right) \\
&=
\frac{\Hbar k}{m}
\frac{\Abs{E}^2}{8} \left(\frac{\beta}{k} + \frac{k}{\beta} \right)^2
\left(
1 + \frac{1 - \inv{4} \left(\frac{k}{\beta} - \frac{\beta}{k} \right)^2}{1 + \inv{4} \left(\frac{k}{\beta} - \frac{\beta}{k} \right)^2}
\right) \\
&=
\frac{\Hbar k}{m}
\frac{\Abs{E}^2}{8} 
\left(
\left(\frac{\beta}{k} + \frac{k}{\beta} \right)^2
+ 
4 - \left(\frac{k}{\beta} - \frac{\beta}{k} \right)^2
\right) \\
&=
\frac{\Hbar k}{m} \Abs{E}^2
\end{align*}

So we have 

\begin{align}
J = \frac{\Hbar k}{m} \Abs{E}^2 = J_b = J_t
\end{align}

The total probability current density equals the barrier probability current density and also the transmitted probability current density.  For
this problem, with steady state flow we have conservation of total probability current density in all regions despite the fact that the probability 
density itself varies both in the region and before it.

\subsection{Transmission and reflection coefficients}

\subsubsection{Transmission coefficient}

From \eqnref{eqn:qm_barrier:incidentAndReflected} we can calculate the transmission
coefficient

\begin{align*}
T 
&= \Abs{\frac{J_t}{J_i}} \\
&= \frac{\Abs{E}^2}{\Abs{A}^2} \\
&= \Abs{\frac{2  } { \mu \cosh\left( i\theta -\beta a \right) }}^2 \\
&= \frac{4} { 
\left(4 + \left( \frac{k}{\beta} -\frac{\beta}{k} \right)^2 \right)
\Abs{\cosh\left( i\theta -\beta a \right)}^2 }
\end{align*}

Utilizing \eqnref{eqn:qm_barrier:coshSquared}
puts us a bit closer to a fully reduced expression for $T$

\begin{align*}
T 
&= \frac{8} { 
\left(4 + \left( \frac{k}{\beta} -\frac{\beta}{k} \right)^2 \right)
\left( \cosh(2\beta a) + \cos(2\theta) \right)
}
\end{align*}

To eliminate the $\theta$ we use \eqnref{eqn:qm_barrier:cosineTwiceArcTan}
with $\alpha = ({k}/{\beta} - {\beta}/{k} )/2$, for

\begin{align*}
\cos(2\theta) 
&= 
\frac{ 1 - \left(\inv{2} \left(\frac{k}{\beta} - \frac{\beta}{k} \right)\right)^2 }
{1 + \left(\inv{2} \left(\frac{k}{\beta} - \frac{\beta}{k} \right)\right)^2}
\end{align*}

So, finally we have

\begin{align}\label{eqn:qm_barrier:Tcoeff1}
T 
&= \frac{8} { 
\left(4 + \left( \frac{k}{\beta} -\frac{\beta}{k} \right)^2 \right)
\cosh(2\beta a)
+ 
4 - \left(\frac{k}{\beta} - \frac{\beta}{k} \right)^2 
}
\end{align}

\subsubsection{Reflection coefficient}

The reflection coefficient is
\begin{align*}
R 
&= \Abs{\frac{J_r}{J_i}} \\
&= \frac{\Abs{B}^2}{\Abs{A}^2} \\
&= 
\Abs{\frac{\nu \sinh\left( \beta a \right) }
{\mu \cosh\left( i\theta -\beta a \right) }}^2 \\
&= \frac{2 \Abs{\nu}^2 \sinh^2( \beta a)} { 
\left(4 + \left( \frac{k}{\beta} -\frac{\beta}{k} \right)^2 \right)
\cosh(2\beta a)
+ 
4 - \left(\frac{k}{\beta} - \frac{\beta}{k} \right)^2 
}
\end{align*}

So we have
\begin{align}\label{eqn:qm_barrier:Rcoeff1}
R 
&= \frac{2 
\left(\frac{\beta}{k} + \frac{k}{\beta} \right)^2 
\sinh^2( \beta a)} { 
\left(4 + \left( \frac{k}{\beta} -\frac{\beta}{k} \right)^2 \right)
\cosh(2\beta a)
+ 
4 - \left(\frac{k}{\beta} - \frac{\beta}{k} \right)^2 
}
\end{align}

\subsubsection{Summing the Transmission and Reflection coefficients}

The expectation is for the 
sum of the Transmission and Reflection coefficients to be unity.

Employing \eqnref{eqn:qm_barrier:sinhSquared} we have

\begin{align*}
T + R 
&= \frac{8 + 2 
\left(\frac{\beta}{k} + \frac{k}{\beta} \right)^2 
%\sinh^2( \beta a)
\inv{2} \left( \cosh(2 \beta a) -1 \right) 
} { 
\left(4 + \left( \frac{k}{\beta} -\frac{\beta}{k} \right)^2 \right)
\cosh(2\beta a)
+ 
4 - \left(\frac{k}{\beta} - \frac{\beta}{k} \right)^2 
} \\
&= \frac{ 
\left(\frac{\beta}{k} + \frac{k}{\beta} \right)^2 
\cosh(2 \beta a) + 8 - \left(\frac{\beta}{k} + \frac{k}{\beta} \right)^2 
} { 
\left(4 + \left( \frac{k}{\beta} -\frac{\beta}{k} \right)^2 \right)
\cosh(2\beta a)
+ 
4 - \left(\frac{k}{\beta} - \frac{\beta}{k} \right)^2 
} \\
\end{align*}

Expanding out each of these four $\beta$, and $k$ terms and comparing shows that
we indeed have $T + R = 1$.  This also shows that we can write the denominator for both of
these 
in a 
slightly tidier form

\begin{align*}
&\left(4 + \left( \frac{k}{\beta} -\frac{\beta}{k} \right)^2 \right)
\cosh(2\beta a)
+ 
4 - \left(\frac{k}{\beta} - \frac{\beta}{k} \right)^2  \\
&=
\left( \frac{k}{\beta} +\frac{\beta}{k} \right)^2 
\cosh(2\beta a)
+ 
4 - \left(\frac{k}{\beta} - \frac{\beta}{k} \right)^2  \\
&=
\left( \frac{k}{\beta} +\frac{\beta}{k} \right)^2 
\cosh(2\beta a)
+ 
8 - \left(\frac{k}{\beta} + \frac{\beta}{k} \right)^2  \\
&=
\left( \frac{k}{\beta} +\frac{\beta}{k} \right)^2 
%\cosh(2\beta a)
\left( 2 \sinh^2(a \beta) + 1\right)
+ 
8 - \left(\frac{k}{\beta} + \frac{\beta}{k} \right)^2  \\
&=
2 \left( \frac{k}{\beta} +\frac{\beta}{k} \right)^2 
\sinh^2(a \beta) + 8 \\
\end{align*}

So finally from \eqnref{eqn:qm_barrier:Tcoeff1} and \eqnref{eqn:qm_barrier:Rcoeff1} we have
\begin{align}
T 
&= \frac{4} { 
\left( \frac{k}{\beta} +\frac{\beta}{k} \right)^2 
\sinh^2(a \beta) + 4
}
\end{align}

\begin{align}
R 
&= \frac{ 
\left(\frac{\beta}{k} + \frac{k}{\beta} \right)^2 
\sinh^2( \beta a)} { 
\left( \frac{k}{\beta} +\frac{\beta}{k} \right)^2 
\sinh^2(a \beta) + 4 
}
\end{align}

In this form demonstrating that $T + R=1$ becomes trivial!

The last desirable substitution to make is to express the $k$, and $\beta$ quotients in terms of $E_0$ and $V$.  This is

\begin{align*}
\left(\frac{k}{\beta} + \frac{\beta}{k}\right)^2
&=
\left(\sqrt{\frac{2mE}{2m(V-E_0)}} + \sqrt{\frac{2m(V-E_0)}{2mE}}\right)^2 \\
&=
\frac{E_0}{V-E_0} + \frac{V-E_0}{E_0} + 2 \\
&=
\frac{E_0^2 + (V-E_0)^2}{E_0(V-E_0)} + 2 \\
&=
\frac{2 E_0^2 + -2 E_0 V + V^2 }{E_0(V-E_0)} + 2 \\
&=
\frac{2 E_0 V -2 E_0^2 + 2 E_0^2 + -2 E_0 V + V^2 }{E_0(V-E_0)} \\
&=
\frac{V^2 }{E_0(V-E_0)} \\
\end{align*}

So, after masses and masses of algebra we have a result that is consistent with that of
\href{http://en.wikipedia.org/wiki/Potential_barrier}{the wiki tunneling article}

\begin{align}
T 
&= \frac{1} { \frac{V^2}{4E(V-E_0)} \sinh^2(a \beta) + 1 }
\end{align}

\begin{align}
R 
&= \frac{1} { \frac{4V(V-E_0)}{V^2 \sinh^2(a \beta)} + 1 }
\end{align}

\section{Appendix.  Messy trig stuff, and rough notes}

\subsection{Complex \texorpdfstring{$\cosh$}{cosh} absolute square expansion}

Expanding the $\cosh$ term we have

\begin{align*}
\Abs{\cosh\left( i\theta -\beta a \right)}^2
&=
\inv{4}
\left(
e^{i\theta - \beta a}
+ e^{-i\theta + \beta a}
\right)
\left(
e^{-i\theta - \beta a}
+ e^{i\theta + \beta a}
\right) \\
&=
\inv{4}\left(
e^{2\beta a}
+ e^{-2\beta a}
+e^{2i\theta }
+e^{-2i\theta }
\right) \\
\end{align*}

So we have
\begin{align}\label{eqn:qm_barrier:coshSquared}
\Abs{\cosh\left( i\theta -\beta a \right)}^2
&=\inv{2}
\left( \cosh(2\beta a) + \cos(2\theta) \right)
\end{align}

\subsection{Cosine of twice arctan}

\begin{align*}
\cos(2\Atan(\alpha)) 
&= 
2 \cos^2\left( \Atan(\alpha) \right) -1 \\
\end{align*}

Now let $\Atan(\alpha) = u$.  So we have

\begin{align*}
\alpha^2 
&=
\tan^2(u)  \\
&=
\frac{\sin^2(u)}{\cos^2(u)} \\
&=
\frac{1 -\cos^2(u)}{\cos^2(u)} \\
\end{align*}

Or
\begin{align*}
\cos^2(u) \alpha^2 &= 1 - \cos^2(u) \\
\implies
\cos^2(u) (\alpha^2 +1 ) &= 1 \\
\end{align*}

For 
\begin{align*}
\cos(2\Atan(\alpha)) 
&= 
2 \cos^2( u ) -1 \\
&= 
2 \inv{1 + \alpha^2} -1 \\
\end{align*}

\begin{align}\label{eqn:qm_barrier:cosineTwiceArcTan}
\cos(2\Atan(\alpha)) 
&= 
\frac{1 - \alpha^2}{1 + \alpha^2}
\end{align}

\subsection{Hyperbolic sine squared}

\begin{align*}
\sinh^2(u) 
&=
\inv{4} (e^u - e^{-u}) (e^u - e^{-u}) \\
&=
\inv{4} \left(e^{2u} + e^{-2u} -2 \right) \\
\end{align*}

\begin{align}\label{eqn:qm_barrier:sinhSquared}
\sinh^2(u) 
&=
\inv{2} \left( \cosh(2u) -1 \right) 
\end{align}

\subsection{Verify correctness of barrier solution}

If we want to verify that \eqnref{eqn:qm_barrier:inBarrierCosh} 
does still match both value and derivative to the $x>a$ wave function
showing both of the following is required

\begin{align*}
1 &= \sqrt{1 + \frac{k^2}{\beta^2}} \cos\left( \Atan(k/\beta) \right) \\
ik &=
\sqrt{1 + \frac{k^2}{\beta^2}} \beta \sinh\left( i \Atan(k/\beta) \right)
\end{align*}

This last is
\begin{align*}
k/\beta &=
\sqrt{1 + \frac{k^2}{\beta^2}} \sin\left( \Atan(k/\beta) \right)
\end{align*}

But these are just the identities

\begin{align*}
\sin\left( \Atan(u) \right) &= \frac{u}{\sqrt{1 + u^2}} \\
\cos\left( \Atan(u) \right) &= \frac{1}{\sqrt{1 + u^2}}
\end{align*}

\subsection{Consistency check.  Current densities in the barrier region}

Initially attempted to calculate the barrier current density using
using \eqnref{eqn:qm_barrier:inBarrierCosh}, but got it
wrong.  Found the mistake only later, and this calculation should provide
a consistency check.

\begin{align*}
J_b
= \frac{\Hbar \beta \Abs{E}^2}{2mi} \left(1 + \frac{k^2}{\beta^2} \right)
( 
&\cosh\left( -i\phi + \beta(x-a) \right) 
\sinh\left( i\phi + \beta(x-a) \right) \\
&-
\cosh\left( i\phi + \beta(x-a) \right) 
\sinh\left( -i\phi + \beta(x-a) \right) 
) \\
\end{align*}

Writing $z = \beta(x-a) + i\phi$, the hyperbolic products above are

\begin{align*}
\cosh z^\conj \sinh z - \cosh z \sinh z^\conj
&=
\inv{4} \left(
( e^{z^\conj} + e^{-z^\conj}) ( e^{z} - e^{-z}) 
- ( e^{z} + e^{-z}) ( e^{z^\conj} - e^{-z^\conj}) 
\right)
\end{align*}

Write $a = e^{z}$ we have

\begin{align*}
( e^{z^\conj} + e^{-z^\conj}) ( e^{z} - e^{-z}) 
- ( e^{z} + e^{-z}) ( e^{z^\conj} - e^{-z^\conj}) 
&=
( a^\conj + 1/a^\conj ) ( a - 1/a )
- 
( a + 1/a ) ( a^\conj - 1/a^\conj )  \\
&=
2\left( \frac{a}{a^\conj} - \frac{a^\conj}{a} \right) \\
&=
\frac{2}{\Abs{a}^2}\left( a^2 - {a^\conj}^2 \right) \\
&=
\frac{ 4\Im(a^2) }{\Abs{a}^2}  \\
&=
4 i \sin( 2\phi ) \\
&=
8 i \sin( \phi ) \cos(\phi) \\
&=
8 i \frac{k/\beta}{1 + (k/\beta)^2} \\
\end{align*}
% tan \phi = k/\beta

which gives 

\begin{align*}
J_b
= \frac{\Hbar k \Abs{E}^2}{m} 
\end{align*}

In the initial incorrect calculation I got zero, which would imply that the result of \eqnref{eqn:qm_barrier:pDensityInBarrier} is also suspect.
However, repeating the calculation for $\rho$ in the barrier region directly from \eqnref{eqn:qm_barrier:psiInBarrier} produces the same result.

Writing, 
\begin{align*}
\psi &=
\frac{E}{2}\left( \left( 1 + \frac{i k}{\beta} \right) e^{ \beta(x-a)} + \left( 1 - \frac{i k}{\beta} \right) e^{ -\beta(x-a)} \right) \\
&=
\frac{E}{2}\left( \mu \alpha + \mu^\conj/\alpha \right)
\end{align*}

the density is

\begin{align*}
\rho 
&=
\psi\psi^\conj \\
&=
\frac{\Abs{E}^2}{4}
\left( \mu \alpha + \mu^\conj/\alpha \right) \left( \mu^\conj \alpha + \mu/\alpha \right) \\
&=
\frac{\Abs{E}^2}{4}
\left( 
\Abs{\mu}^2\left( \alpha^2 + \inv{\alpha^2}\right) 
+ \mu^2
+ {\mu^\conj}^2
\right) \\
&=
\frac{\Abs{E}^2}{2}
\left( 
\left( 1 + \frac{k^2}{\beta^2} \right)\frac{ 
e^{2\beta(x-a)}
+e^{-2\beta(x-a)}
}{2}
+ \inv{2}\left( 
1 + 2 i k/\beta - k^2/\beta^2
1 - 2 i k/\beta - k^2/\beta^2
\right)
\right) \\
&=
\frac{\Abs{E}^2}{2}
\left( 
\left( 1 + \frac{k^2}{\beta^2} \right) \cosh(2\beta(x-a))
+ \left( 
1 - k^2/\beta^2
\right)
\right) \\
&=
\frac{\Abs{E}^2}{2}
\left( 
\left( 1 + \frac{k^2}{\beta^2} \right) \left( 2 \sinh^2(\beta(x-a)) + 1 \right)
+ \left( 
1 - k^2/\beta^2
\right)
\right) \\
&=
\Abs{E}^2
\left( 
\left( 1 + \frac{k^2}{\beta^2} \right) \sinh^2(\beta(x-a)) 
+1 
\right) \\
\end{align*}

which is exactly \eqnref{eqn:qm_barrier:pDensityInBarrier}.

\subsection{Verify continuity of simplified region I wave function}

The wave function \eqnref{eqn:qm_barrier:incidentAndReflectedSimplified}
should match the barrier wave function in value and derivative at
$x=0$.  Let us verify this and ensure all the algebra worked out.

At $x=0$ we have for the $x<0$ wave function value
\begin{align*}
\psi &=
\frac{E}{2} \left(\frac{\beta}{k} + \frac{k}{\beta} \right) 
\left(
\cosh\left( i\theta -\beta a \right) -i \sinh\left( \beta a \right) 
\right) \\
\end{align*}

Let us expand the trig part of this

\begin{align*}
\cosh\left( i\theta -\beta a \right) -i \sinh\left( \beta a \right) 
&=
\cos( \theta ) \cosh(\beta a ) 
-i\sin( \theta ) \sinh(\beta a ) 
-i \sinh( \beta a )  \\
&=
\cos( \theta ) \cosh(\beta a ) 
-i \sinh( \beta a ) (1 + \sin(\theta))
\end{align*}

Since we have

\begin{align*}
\theta &= \Atan\left( \inv{2} \left(\frac{k}{\beta} - \frac{\beta}{k} \right) \right) \\
\cos(\Atan(u)) &= \inv{\sqrt{1 + u^2}} \\
\sin(\Atan(u)) &= \frac{u}{\sqrt{1 + u^2}} \\
\end{align*}

So we have
\begin{align*}
\cosh\left( i\theta -\beta a \right) -i \sinh\left( \beta a \right) 
&=
\inv{\sqrt{1 + \inv{4} \left(\frac{k}{\beta} - \frac{\beta}{k} \right)^2}}
\left(\cosh(\beta a ) 
-i
\inv{2} \left(\frac{k}{\beta} - \frac{\beta}{k} \right) \sinh(\beta a ) \right)
-i \sinh( \beta a ) \\
&=
\frac{2}{\frac{k}{\beta} + \frac{\beta}{k} }
\left(\cosh(\beta a ) 
-i
\inv{2} \left(\frac{k}{\beta} - \frac{\beta}{k} \right) \sinh(\beta a ) \right)
-i \sinh( \beta a )  \\
&=
\frac{ 2 \cosh(\beta a ) }{\frac{k}{\beta} + \frac{\beta}{k} }
-i \sinh( \beta a ) \left( 1 + \frac{\frac{k}{\beta} - \frac{\beta}{k}}{ \frac{k}{\beta} + \frac{\beta}{k}} \right) \\
&=
\frac{ 2 }{\frac{k}{\beta} + \frac{\beta}{k} }
\left(
\cosh(\beta a ) - i \frac{k}{\beta}\sinh( \beta a ) 
\right)
\end{align*}

and finally, after too much messy algebra, from the region I wave function at $x=0$ we have

\begin{align*}
\psi(0) &= E \left( \cosh(\beta a ) - i \frac{k}{\beta}\sinh( \beta a )  \right)
\end{align*}

Now compare to the barrier wave function \eqnref{eqn:qm_barrier:inBarrierCosh} at $x=0$

\begin{align*}
\psi(0) 
&= E \sqrt{1 + \frac{k^2}{\beta^2}} \cosh\left( i\Atan(k/\beta) - \beta a \right)  \\
&= E \sqrt{1 + \frac{k^2}{\beta^2}} 
\left(
\cos(\Atan(k/\beta))\cosh(\beta a)
-i\sin(\Atan(k/\beta))\sinh(\beta a)
\right) \\
&= E 
\left(
\cosh(\beta a)
-i\frac{k}{\beta} \sinh(\beta a)
\right) \\
\end{align*}

Good.  First consistency check is done.  Now for equality of derivatives.

For \eqnref{eqn:qm_barrier:incidentAndReflectedSimplified}, the derivative at $x=0$ is

\begin{align*}
\psi' 
&=
\frac{E}{2} \left(\frac{\beta}{k} + \frac{k}{\beta} \right) (ik)
\left(
\cosh\left( i\theta -\beta a \right) 
+i \sinh\left( \beta a \right) 
\right) \\
&=
\frac{E}{2} \left(\frac{\beta}{k} + \frac{k}{\beta} \right) (ik)
\left(
\frac{ 2 \cosh(\beta a ) }{\frac{k}{\beta} + \frac{\beta}{k} }
-i \sinh( \beta a ) \left( -1 + \frac{\frac{k}{\beta} - \frac{\beta}{k}}{ \frac{k}{\beta} + \frac{\beta}{k}} \right) 
\right) \\
&=
i k E \left( \cosh(\beta a ) + i \frac{\beta}{k}\sinh(\beta a) \right) \\
&=
E \left( i k \cosh(\beta a ) - \beta \sinh(\beta a) \right) \\
\end{align*}

Comparing to the derivative of the barrier wave function \eqnref{eqn:qm_barrier:inBarrierCosh} at $x=0$

\begin{align*}
\psi'(0)
&= E \sqrt{1 + \frac{k^2}{\beta^2}} \beta \sinh\left( i\phi - \beta a \right)  \\
&= E \beta \sqrt{1 + \frac{k^2}{\beta^2}} 
\left(
i\sin(\phi)\cosh(\beta a)
-\cos(\phi)\sinh(\beta a)
\right) \\
&= E \beta 
\left(
i\frac{k}{\beta}\cosh(\beta a)
-\sinh(\beta a)
\right) \\
&= E 
\left(
ik \cosh(\beta a)
- \beta \sinh(\beta a)
\right) \\
\end{align*}
% tan \phi = k/\beta

%\bibliographystyle{plainnat}
%\bibliography{myrefs}

%\end{document}
