%
% Copyright � 2012 Peeter Joot.  All Rights Reserved.
% Licenced as described in the file LICENSE under the root directory of this GIT repository.
%

% 
% 
%%
% Copyright � 2015 Peeter Joot.  All Rights Reserved.
% Licenced as described in the file LICENSE under the root directory of this GIT repository.
%
\documentclass[]{eliblog}

\usepackage{amsmath}
\usepackage{mathpazo}

%
% shorthand for bold symbols, convenient for vectors and matrices
%
\newcommand{\Ba}[0]{\mathbf{a}}
\newcommand{\Bb}[0]{\mathbf{b}}
\newcommand{\Bc}[0]{\mathbf{c}}
\newcommand{\Bd}[0]{\mathbf{d}}
\newcommand{\Be}[0]{\mathbf{e}}
\newcommand{\Bf}[0]{\mathbf{f}}
\newcommand{\Bg}[0]{\mathbf{g}}
\newcommand{\Bh}[0]{\mathbf{h}}
\newcommand{\Bi}[0]{\mathbf{i}}
\newcommand{\Bj}[0]{\mathbf{j}}
\newcommand{\Bk}[0]{\mathbf{k}}
\newcommand{\Bl}[0]{\mathbf{l}}
\newcommand{\Bm}[0]{\mathbf{m}}
\newcommand{\Bn}[0]{\mathbf{n}}
\newcommand{\Bo}[0]{\mathbf{o}}
\newcommand{\Bp}[0]{\mathbf{p}}
\newcommand{\Bq}[0]{\mathbf{q}}
\newcommand{\Br}[0]{\mathbf{r}}
\newcommand{\Bs}[0]{\mathbf{s}}
\newcommand{\Bt}[0]{\mathbf{t}}
\newcommand{\Bu}[0]{\mathbf{u}}
\newcommand{\Bv}[0]{\mathbf{v}}
\newcommand{\Bw}[0]{\mathbf{w}}
\newcommand{\Bx}[0]{\mathbf{x}}
\newcommand{\By}[0]{\mathbf{y}}
\newcommand{\Bz}[0]{\mathbf{z}}
\newcommand{\BA}[0]{\mathbf{A}}
\newcommand{\BB}[0]{\mathbf{B}}
\newcommand{\BC}[0]{\mathbf{C}}
\newcommand{\BD}[0]{\mathbf{D}}
\newcommand{\BE}[0]{\mathbf{E}}
\newcommand{\BF}[0]{\mathbf{F}}
\newcommand{\BG}[0]{\mathbf{G}}
\newcommand{\BH}[0]{\mathbf{H}}
\newcommand{\BI}[0]{\mathbf{I}}
\newcommand{\BJ}[0]{\mathbf{J}}
\newcommand{\BK}[0]{\mathbf{K}}
\newcommand{\BL}[0]{\mathbf{L}}
\newcommand{\BM}[0]{\mathbf{M}}
\newcommand{\BN}[0]{\mathbf{N}}
\newcommand{\BO}[0]{\mathbf{O}}
\newcommand{\BP}[0]{\mathbf{P}}
\newcommand{\BQ}[0]{\mathbf{Q}}
\newcommand{\BR}[0]{\mathbf{R}}
\newcommand{\BS}[0]{\mathbf{S}}
\newcommand{\BT}[0]{\mathbf{T}}
\newcommand{\BU}[0]{\mathbf{U}}
\newcommand{\BV}[0]{\mathbf{V}}
\newcommand{\BW}[0]{\mathbf{W}}
\newcommand{\BX}[0]{\mathbf{X}}
\newcommand{\BY}[0]{\mathbf{Y}}
\newcommand{\BZ}[0]{\mathbf{Z}}

\newcommand{\Bzero}[0]{\mathbf{0}}
\newcommand{\Btheta}[0]{\boldsymbol{\theta}}
\newcommand{\Btau}[0]{\boldsymbol{\tau}}
\newcommand{\Bomega}[0]{\boldsymbol{\omega}}

%
% shorthand for unit vectors
%
\newcommand{\acap}[0]{\hat{\Ba}}
\newcommand{\bcap}[0]{\hat{\Bb}}
\newcommand{\ccap}[0]{\hat{\Bc}}
\newcommand{\dcap}[0]{\hat{\Bd}}
\newcommand{\ecap}[0]{\hat{\Be}}
\newcommand{\fcap}[0]{\hat{\Bf}}
\newcommand{\gcap}[0]{\hat{\Bg}}
\newcommand{\hcap}[0]{\hat{\Bh}}
\newcommand{\icap}[0]{\hat{\Bi}}
\newcommand{\jcap}[0]{\hat{\Bj}}
\newcommand{\kcap}[0]{\hat{\Bk}}
\newcommand{\lcap}[0]{\hat{\Bl}}
\newcommand{\mcap}[0]{\hat{\Bm}}
\newcommand{\ncap}[0]{\hat{\Bn}}
\newcommand{\ocap}[0]{\hat{\Bo}}
\newcommand{\pcap}[0]{\hat{\Bp}}
\newcommand{\qcap}[0]{\hat{\Bq}}
\newcommand{\rcap}[0]{\hat{\Br}}
\newcommand{\scap}[0]{\hat{\Bs}}
\newcommand{\tcap}[0]{\hat{\Bt}}
\newcommand{\ucap}[0]{\hat{\Bu}}
\newcommand{\vcap}[0]{\hat{\Bv}}
\newcommand{\wcap}[0]{\hat{\Bw}}
\newcommand{\xcap}[0]{\hat{\Bx}}
\newcommand{\ycap}[0]{\hat{\By}}
\newcommand{\zcap}[0]{\hat{\Bz}}
\newcommand{\thetacap}[0]{\hat{\Btheta}}

%
% to write R^n and C^n in a distinguishable fashion.  Perhaps change this
% to the double lined characters upon figuring out how to do so.
%
\newcommand{\C}[1]{$\mathbb{C}^{#1}$}
\newcommand{\R}[1]{$\mathbb{R}^{#1}$}

%
% various generally useful helpers
%

% derivative of #1 wrt. #2:
\newcommand{\D}[2] {\frac {d#2} {d#1}}

\newcommand{\inv}[1]{\frac{1}{#1}}
\newcommand{\cross}[0]{\times}

\newcommand{\abs}[1]{\lvert{#1}\rvert}
\newcommand{\norm}[1]{\lVert{#1}\rVert}
\newcommand{\innerprod}[2]{\langle{#1}, {#2}\rangle}
\newcommand{\dotprod}[2]{{#1} \cdot {#2}}
\newcommand{\bdotprod}[2]{\left({#1} \cdot {#2}\right)}
\newcommand{\crossprod}[2]{{#1} \cross {#2}}
\newcommand{\tripleprod}[3]{\dotprod{\left(\crossprod{#1}{#2}\right)}{#3}}

\DeclareMathOperator{\Proj}{Proj}
\DeclareMathOperator{\Span}{span}
\DeclareMathOperator{\Sgn}{sgn}
\DeclareMathOperator{\Area}{Area}
\DeclareMathOperator{\Volume}{Volume}

%
% A few miscellaneous things specific to this document
%
\newcommand{\crossop}[1]{\crossprod{#1}{}}

% R2 vector.
\newcommand{\VectorTwo}[2]{
\begin{bmatrix}
 {#1} \\
 {#2}
\end{bmatrix}
}

\newcommand{\VectorN}[1]{
\begin{bmatrix}
{#1}_1 \\
{#1}_2 \\
\vdots \\
{#1}_N \\
\end{bmatrix}
}

\newcommand{\DETuvij}[4]{
\begin{vmatrix}
 {#1}_{#3} & {#1}_{#4} \\
 {#2}_{#3} & {#2}_{#4}
\end{vmatrix}
}

\newcommand{\DETuvwijk}[6]{
\begin{vmatrix}
 {#1}_{#4} & {#1}_{#5} & {#1}_{#6} \\
 {#2}_{#4} & {#2}_{#5} & {#2}_{#6} \\
 {#3}_{#4} & {#3}_{#5} & {#3}_{#6}
\end{vmatrix}
}

\newcommand{\DETuvwxijkl}[8]{
\begin{vmatrix}
 {#1}_{#5} & {#1}_{#6} & {#1}_{#7} & {#1}_{#8} \\
 {#2}_{#5} & {#2}_{#6} & {#2}_{#7} & {#2}_{#8} \\
 {#3}_{#5} & {#3}_{#6} & {#3}_{#7} & {#3}_{#8} \\
 {#4}_{#5} & {#4}_{#6} & {#4}_{#7} & {#4}_{#8} \\
\end{vmatrix}
}

%\newcommand{\DETuvwxyijklm}[10]{
%\begin{vmatrix}
% {#1}_{#6} & {#1}_{#7} & {#1}_{#8} & {#1}_{#9} & {#1}_{#10} \\
% {#2}_{#6} & {#2}_{#7} & {#2}_{#8} & {#2}_{#9} & {#2}_{#10} \\
% {#3}_{#6} & {#3}_{#7} & {#3}_{#8} & {#3}_{#9} & {#3}_{#10} \\
% {#4}_{#6} & {#4}_{#7} & {#4}_{#8} & {#4}_{#9} & {#4}_{#10} \\
% {#5}_{#6} & {#5}_{#7} & {#5}_{#8} & {#5}_{#9} & {#5}_{#10}
%\end{vmatrix}
%}

% R3 vector.
\newcommand{\VectorThree}[3]{
\begin{bmatrix}
 {#1} \\
 {#2} \\
 {#3}
\end{bmatrix}
}



\author{Peeter Joot}
\email{peeter.joot@gmail.com}

%\documentclass[]{eliblogwidescreen}

\usepackage{amsmath}
\usepackage{mathpazo}

%
% shorthand for bold symbols, convenient for vectors and matrices
%
\newcommand{\Ba}[0]{\mathbf{a}}
\newcommand{\Bb}[0]{\mathbf{b}}
\newcommand{\Bc}[0]{\mathbf{c}}
\newcommand{\Bd}[0]{\mathbf{d}}
\newcommand{\Be}[0]{\mathbf{e}}
\newcommand{\Bf}[0]{\mathbf{f}}
\newcommand{\Bg}[0]{\mathbf{g}}
\newcommand{\Bh}[0]{\mathbf{h}}
\newcommand{\Bi}[0]{\mathbf{i}}
\newcommand{\Bj}[0]{\mathbf{j}}
\newcommand{\Bk}[0]{\mathbf{k}}
\newcommand{\Bl}[0]{\mathbf{l}}
\newcommand{\Bm}[0]{\mathbf{m}}
\newcommand{\Bn}[0]{\mathbf{n}}
\newcommand{\Bo}[0]{\mathbf{o}}
\newcommand{\Bp}[0]{\mathbf{p}}
\newcommand{\Bq}[0]{\mathbf{q}}
\newcommand{\Br}[0]{\mathbf{r}}
\newcommand{\Bs}[0]{\mathbf{s}}
\newcommand{\Bt}[0]{\mathbf{t}}
\newcommand{\Bu}[0]{\mathbf{u}}
\newcommand{\Bv}[0]{\mathbf{v}}
\newcommand{\Bw}[0]{\mathbf{w}}
\newcommand{\Bx}[0]{\mathbf{x}}
\newcommand{\By}[0]{\mathbf{y}}
\newcommand{\Bz}[0]{\mathbf{z}}
\newcommand{\BA}[0]{\mathbf{A}}
\newcommand{\BB}[0]{\mathbf{B}}
\newcommand{\BC}[0]{\mathbf{C}}
\newcommand{\BD}[0]{\mathbf{D}}
\newcommand{\BE}[0]{\mathbf{E}}
\newcommand{\BF}[0]{\mathbf{F}}
\newcommand{\BG}[0]{\mathbf{G}}
\newcommand{\BH}[0]{\mathbf{H}}
\newcommand{\BI}[0]{\mathbf{I}}
\newcommand{\BJ}[0]{\mathbf{J}}
\newcommand{\BK}[0]{\mathbf{K}}
\newcommand{\BL}[0]{\mathbf{L}}
\newcommand{\BM}[0]{\mathbf{M}}
\newcommand{\BN}[0]{\mathbf{N}}
\newcommand{\BO}[0]{\mathbf{O}}
\newcommand{\BP}[0]{\mathbf{P}}
\newcommand{\BQ}[0]{\mathbf{Q}}
\newcommand{\BR}[0]{\mathbf{R}}
\newcommand{\BS}[0]{\mathbf{S}}
\newcommand{\BT}[0]{\mathbf{T}}
\newcommand{\BU}[0]{\mathbf{U}}
\newcommand{\BV}[0]{\mathbf{V}}
\newcommand{\BW}[0]{\mathbf{W}}
\newcommand{\BX}[0]{\mathbf{X}}
\newcommand{\BY}[0]{\mathbf{Y}}
\newcommand{\BZ}[0]{\mathbf{Z}}

\newcommand{\Bzero}[0]{\mathbf{0}}
\newcommand{\Btheta}[0]{\boldsymbol{\theta}}
\newcommand{\Btau}[0]{\boldsymbol{\tau}}
\newcommand{\Bomega}[0]{\boldsymbol{\omega}}

%
% shorthand for unit vectors
%
\newcommand{\acap}[0]{\hat{\Ba}}
\newcommand{\bcap}[0]{\hat{\Bb}}
\newcommand{\ccap}[0]{\hat{\Bc}}
\newcommand{\dcap}[0]{\hat{\Bd}}
\newcommand{\ecap}[0]{\hat{\Be}}
\newcommand{\fcap}[0]{\hat{\Bf}}
\newcommand{\gcap}[0]{\hat{\Bg}}
\newcommand{\hcap}[0]{\hat{\Bh}}
\newcommand{\icap}[0]{\hat{\Bi}}
\newcommand{\jcap}[0]{\hat{\Bj}}
\newcommand{\kcap}[0]{\hat{\Bk}}
\newcommand{\lcap}[0]{\hat{\Bl}}
\newcommand{\mcap}[0]{\hat{\Bm}}
\newcommand{\ncap}[0]{\hat{\Bn}}
\newcommand{\ocap}[0]{\hat{\Bo}}
\newcommand{\pcap}[0]{\hat{\Bp}}
\newcommand{\qcap}[0]{\hat{\Bq}}
\newcommand{\rcap}[0]{\hat{\Br}}
\newcommand{\scap}[0]{\hat{\Bs}}
\newcommand{\tcap}[0]{\hat{\Bt}}
\newcommand{\ucap}[0]{\hat{\Bu}}
\newcommand{\vcap}[0]{\hat{\Bv}}
\newcommand{\wcap}[0]{\hat{\Bw}}
\newcommand{\xcap}[0]{\hat{\Bx}}
\newcommand{\ycap}[0]{\hat{\By}}
\newcommand{\zcap}[0]{\hat{\Bz}}
\newcommand{\thetacap}[0]{\hat{\Btheta}}

%
% to write R^n and C^n in a distinguishable fashion.  Perhaps change this
% to the double lined characters upon figuring out how to do so.
%
\newcommand{\C}[1]{$\mathbb{C}^{#1}$}
\newcommand{\R}[1]{$\mathbb{R}^{#1}$}

%
% various generally useful helpers
%

% derivative of #1 wrt. #2:
\newcommand{\D}[2] {\frac {d#2} {d#1}}

\newcommand{\inv}[1]{\frac{1}{#1}}
\newcommand{\cross}[0]{\times}

\newcommand{\abs}[1]{\lvert{#1}\rvert}
\newcommand{\norm}[1]{\lVert{#1}\rVert}
\newcommand{\innerprod}[2]{\langle{#1}, {#2}\rangle}
\newcommand{\dotprod}[2]{{#1} \cdot {#2}}
\newcommand{\bdotprod}[2]{\left({#1} \cdot {#2}\right)}
\newcommand{\crossprod}[2]{{#1} \cross {#2}}
\newcommand{\tripleprod}[3]{\dotprod{\left(\crossprod{#1}{#2}\right)}{#3}}

\DeclareMathOperator{\Proj}{Proj}
\DeclareMathOperator{\Span}{span}
\DeclareMathOperator{\Sgn}{sgn}
\DeclareMathOperator{\Area}{Area}
\DeclareMathOperator{\Volume}{Volume}

%
% A few miscellaneous things specific to this document
%
\newcommand{\crossop}[1]{\crossprod{#1}{}}

% R2 vector.
\newcommand{\VectorTwo}[2]{
\begin{bmatrix}
 {#1} \\
 {#2}
\end{bmatrix}
}

\newcommand{\VectorN}[1]{
\begin{bmatrix}
{#1}_1 \\
{#1}_2 \\
\vdots \\
{#1}_N \\
\end{bmatrix}
}

\newcommand{\DETuvij}[4]{
\begin{vmatrix}
 {#1}_{#3} & {#1}_{#4} \\
 {#2}_{#3} & {#2}_{#4}
\end{vmatrix}
}

\newcommand{\DETuvwijk}[6]{
\begin{vmatrix}
 {#1}_{#4} & {#1}_{#5} & {#1}_{#6} \\
 {#2}_{#4} & {#2}_{#5} & {#2}_{#6} \\
 {#3}_{#4} & {#3}_{#5} & {#3}_{#6}
\end{vmatrix}
}

\newcommand{\DETuvwxijkl}[8]{
\begin{vmatrix}
 {#1}_{#5} & {#1}_{#6} & {#1}_{#7} & {#1}_{#8} \\
 {#2}_{#5} & {#2}_{#6} & {#2}_{#7} & {#2}_{#8} \\
 {#3}_{#5} & {#3}_{#6} & {#3}_{#7} & {#3}_{#8} \\
 {#4}_{#5} & {#4}_{#6} & {#4}_{#7} & {#4}_{#8} \\
\end{vmatrix}
}

%\newcommand{\DETuvwxyijklm}[10]{
%\begin{vmatrix}
% {#1}_{#6} & {#1}_{#7} & {#1}_{#8} & {#1}_{#9} & {#1}_{#10} \\
% {#2}_{#6} & {#2}_{#7} & {#2}_{#8} & {#2}_{#9} & {#2}_{#10} \\
% {#3}_{#6} & {#3}_{#7} & {#3}_{#8} & {#3}_{#9} & {#3}_{#10} \\
% {#4}_{#6} & {#4}_{#7} & {#4}_{#8} & {#4}_{#9} & {#4}_{#10} \\
% {#5}_{#6} & {#5}_{#7} & {#5}_{#8} & {#5}_{#9} & {#5}_{#10}
%\end{vmatrix}
%}

% R3 vector.
\newcommand{\VectorThree}[3]{
\begin{bmatrix}
 {#1} \\
 {#2} \\
 {#3}
\end{bmatrix}
}



\author{Peeter Joot}
\email{peeter.joot@gmail.com}


\chapter{Dirac spinor notes}
\label{chap:desaiDiracAdjoint}
%\useCCL
\blogpage{http://sites.google.com/site/peeterjoot/math2011/desaiDiracAdjoint.pdf}
\date{May 25, 2011}
\revisionInfo{desaiDiracAdjoint.tex}

\beginArtWithToc
%\beginArtNoToc

\section{Motivation}

I was having algebraic trouble verifying orthonormality relationships for spinor solutions to the Dirac free particle equation, and initially started preparing these notes to post a question to physicsforums.  However, in the process of doing so, I spotted my error.  A side effect of making these notes is that I got a nice summary of some of the relationships, and it was a good starting point for some personal notes expanding on the content of these chapters.

\section{Context for the original question}

In Desai's QM book \citep{desai2009quantum}, the non-covariant form of the free particle equation is developed as

\begin{equation}\label{eqn:desaiDiracAdjoint:10}
\begin{bmatrix}
E - m & - \Bsigma \cdot \Bp \\
- \Bsigma \cdot \Bp & E + m
\end{bmatrix}
u
= 0,
\end{equation}

where each block in the matrix above is two by two.  Recall that

\begin{subequations}
\begin{equation}\label{eqn:desaiDiracAdjoint:200}
\begin{aligned}
\sigma_1 &= \PauliX \\
\sigma_2 &= \PauliY \\
\sigma_3 &= \PauliZ,
\end{aligned}
\end{equation}
\end{subequations}

so 

\begin{equation}\label{eqn:desaiDiracAdjoint:220}
\Bsigma \cdot \Bp =
\begin{bmatrix}
p_z &  p_x - i p_y \\
p_x + i p_y & - p_z
\end{bmatrix}.
\end{equation}

For spin up \(\ket{+}\) and spin down \(\ket{-}\) states, the positive energy solutions \(E = \Abs{E} = \sqrt{\Bp^2 + m^2}\) are found to be

\begin{equation}\label{eqn:desaiDiracAdjoint:30}
u^{\pm}(\Bp) =
\sqrt{\frac{\Abs{E} + m}{2m}}
\begin{bmatrix}
\ket{\pm} \\
\frac{\Bsigma \cdot \Bp}{\Abs{E} + m} \ket{\pm}
\end{bmatrix},
\end{equation}

and the negative energy states associated with \(E = -\Abs{E} = -\sqrt{\Bp^2 + m^2}\) are found to be

\begin{equation}\label{eqn:desaiDiracAdjoint:50}
v^{\pm}(\Bp) =
\sqrt{\frac{\Abs{E} + m}{2m}}
\begin{bmatrix}
-\frac{\Bsigma \cdot \Bp}{\Abs{E} + m} \ket{\pm} \\
\ket{\pm} \\
\end{bmatrix}.
\end{equation}

The z-axis spin up state \(\ket{+} = (1, 0)\) and spin down state \(\ket{-} = (0, 1)\) are also used to find one specific set of states for the positive energy solutions

\begin{subequations}
\begin{equation}\label{eqn:desaiDiracAdjoint:70}
\begin{aligned}
u^{+}(\Bp) &=
\sqrt{\frac{\Abs{E} + m}{2m}}
\begin{bmatrix}
1 \\
0 \\
\frac{p_z}{\Abs{E} + m} \\
\frac{p_x + i p_y}{\Abs{E} + m} \\
\end{bmatrix} \\
u^{-}(\Bp) &=
\sqrt{\frac{\Abs{E} + m}{2m}}
\begin{bmatrix}
0 \\
1 \\
\frac{p_x - i p_y}{\Abs{E} + m} \\
-\frac{p_z}{\Abs{E} + m} \\
\end{bmatrix},
\end{aligned}
\end{equation}
\end{subequations}

and negative energy solutions
\begin{subequations}
\begin{equation}\label{eqn:desaiDiracAdjoint:90}
\begin{aligned}
v^{+}(\Bp) &=
\sqrt{\frac{\Abs{E} + m}{2m}}
\begin{bmatrix}
-\frac{p_z}{\Abs{E} + m} \\
-\frac{p_x + i p_y}{\Abs{E} + m} \\
1 \\
0 \\
\end{bmatrix} \\
v^{-}(\Bp) &=
\sqrt{\frac{\Abs{E} + m}{2m}}
\begin{bmatrix}
-\frac{p_x - i p_y}{\Abs{E} + m} \\
\frac{p_z}{\Abs{E} + m} \\
0 \\
1 \\
\end{bmatrix}.
\end{aligned}
\end{equation}
\end{subequations}

(the book uses \(u^{\pm}\) for both the negative energy states, but I have used \(v^{\pm}\) here for the negative states for consistency with the covariant equation solutions). 

Later a complete set of states \(u_r(\Bp), v_r(\Bp)\) are identified as solutions to the covariant Dirac equations \((\gamma \cdot p -m)u = 0\), \((\gamma \cdot p + m) v = 0\), where \(p^\mu = (\Bp, \Abs{E})\) as follows

\begin{subequations}
\begin{equation}\label{eqn:desaiDiracAdjoint:110}
\begin{aligned}
u_1(\Bp) &= u^{+}(\Bp) \\
u_2(\Bp) &= u^{-}(\Bp) \\
v_1(\Bp) &= v^{+}(-\Bp) \\
v_2(\Bp) &= v^{-}(-\Bp),
\end{aligned}
\end{equation}
\end{subequations}

Note very carefully the sign change above.  This is important, since without that we do not have a zero inner product between all \(u_r\) and \(v_s\) states.  Spelled out explicitly, these states for the z-axis spin up case are

\begin{subequations}
\begin{equation}\label{eqn:desaiDiracAdjoint:130}
\begin{aligned}
u_1(\Bp) &=
\sqrt{\frac{\Abs{E} + m}{2m}}
\begin{bmatrix}
1 \\
0 \\
\frac{p_z}{\Abs{E} + m} \\
\frac{p_x + i p_y}{\Abs{E} + m} \\
\end{bmatrix} \\
u_2(\Bp) &=
\sqrt{\frac{\Abs{E} + m}{2m}}
\begin{bmatrix}
0 \\
1 \\
\frac{p_x - i p_y}{\Abs{E} + m} \\
-\frac{p_z}{\Abs{E} + m} \\
\end{bmatrix} \\
v_1(\Bp) &=
\sqrt{\frac{\Abs{E} + m}{2m}}
\begin{bmatrix}
\frac{p_z}{\Abs{E} + m} \\
\frac{p_x + i p_y}{\Abs{E} + m} \\
1 \\
0 \\
\end{bmatrix} \\
v_2(\Bp) &=
\sqrt{\frac{\Abs{E} + m}{2m}}
\begin{bmatrix}
\frac{p_x - i p_y}{\Abs{E} + m} \\
-\frac{p_z}{\Abs{E} + m} \\
0 \\
1 \\
\end{bmatrix}.
\end{aligned}
\end{equation}
\end{subequations}

In order to construct a covariant current conservation relationship a quantity, the Dirac adjoint, was defined as

\begin{equation}\label{eqn:desaiDiracAdjoint:150}
\overbar{\psi} = \psi^\dagger \gamma^4,
\end{equation}

where

\begin{equation}\label{eqn:desaiDiracAdjoint:170}
\gamma^4 = 
\begin{bmatrix}
1 & 0 & 0 & 0 \\
0 & 1 & 0 & 0 \\
0 & 0 & -1 & 0 \\
0 & 0 & 0 & -1 \\
\end{bmatrix}
\end{equation}

This Dirac adjoint can be used to form an inner product of the form

\begin{equation}\label{eqn:desaiDiracAdjoint:190}
\overbar{\psi}\psi
\end{equation}

It is claimed in the text that we have \(\overbar{u_r} u_s = \delta_{rs}\), \(\overbar{v_r} v_s = \delta_{rs}\), and \(\overbar{u_r} v_s = 0\).  Let us verify all these relationships.

\section{Some checks}

\subsection{Verify the non-covariant solutions}

A non-relativistic approximation argument was used to determine the solutions \eqnref{eqn:desaiDiracAdjoint:70}, but we can verify that these hold generally by substitution.  For example, for the positive energy z-axis spin up state we have

\begin{equation}\label{eqn:desaiDiracAdjoint:880}
\begin{aligned}
&\begin{bmatrix}
E - m & - \Bsigma \cdot \Bp \\
- \Bsigma \cdot \Bp & E + m
\end{bmatrix}
u^{+}(\Bp) \\
&=
\sqrt{\frac{\Abs{E} + m}{2m}}
\begin{bmatrix}
E - m & 0 & -p_z &  -p_x + i p_y \\
0 & E - m & -p_x - i p_y & p_z \\
-p_z &  -p_x + i p_y & E + m & 0 \\
-p_x - i p_y & p_z & 0 & E + m 
\end{bmatrix}
\begin{bmatrix}
1 \\
0 \\
\frac{p_z}{\Abs{E} + m} \\
\frac{p_x + i p_y}{\Abs{E} + m} \\
\end{bmatrix} \\
&\sim 
\begin{bmatrix}
E^2 - m^2 - p_x^2 - p_y^2 - p_z^2 \\
-(p_x + i p_y) p_z + p_z (p_x + i p_y) \\
- p_z( E + m ) + p_z( E + m ) \\
-(p_x + i p_y) (E + m) + (E + m)(p_x + i p_y)
\end{bmatrix} \\
&= 0.
\end{aligned}
\end{equation}

Here the relationship between the free particle's energy and momentum \(E^2 - m^2 - \Bp^2 = 0\) has been used, so we have a zero as desired, and no non-relativistic approximations are required.  We can show this generally too, without requiring the specifics of the z-axis spin up or down solutions.  This is actually even easier.  For the positive energy solutions \eqnref{eqn:desaiDiracAdjoint:30} we have

\begin{equation}\label{eqn:desaiDiracAdjoint:900}
\begin{aligned}
\begin{bmatrix}
E - m & - \Bsigma \cdot \Bp \\
- \Bsigma \cdot \Bp & E + m
\end{bmatrix}
u
&\sim
\begin{bmatrix}
E - m & - \Bsigma \cdot \Bp \\
- \Bsigma \cdot \Bp & E + m
\end{bmatrix}
\begin{bmatrix}
(E + m) \ket{\pm} \\
(\Bsigma \cdot \Bp) \ket{\pm}
\end{bmatrix} \\
&=
\begin{bmatrix}
(E^2 - m^2 - (\Bsigma \cdot \Bp)^2) \ket{\pm} \\
0 \ket{\pm}
\end{bmatrix} \\
&=
\begin{bmatrix}
(E^2 - m^2 - \Bp^2) \ket{\pm} \\
0 \ket{\pm}
\end{bmatrix} \\
&=
0,
\end{aligned}
\end{equation}

where the identity \((\Bsigma \cdot \Bp)^2 = \Bp^2\) has been used.  For the negative energy solutions \eqnref{eqn:desaiDiracAdjoint:50} we have

\begin{equation}\label{eqn:desaiDiracAdjoint:920}
\begin{aligned}
\begin{bmatrix}
E - m & - \Bsigma \cdot \Bp \\
- \Bsigma \cdot \Bp & E + m
\end{bmatrix}
u
&\sim
\begin{bmatrix}
E - m & - \Bsigma \cdot \Bp \\
- \Bsigma \cdot \Bp & E + m
\end{bmatrix}
\begin{bmatrix}
-(\Bsigma \cdot \Bp) \ket{\pm} \\
(-E + m) \ket{\pm} \\
\end{bmatrix} \\
&=
\begin{bmatrix}
0 \ket{\pm} \\
(-E^2 + m^2 + (\Bsigma \cdot \Bp)^2) \ket{\pm} \\
\end{bmatrix} \\
&=
0.
\end{aligned}
\end{equation}

\subsection{Is there something special about the z-axis orientation?}

Why was the z-axis spin orientation picked?  It does not seem to me that there would be any reason for this.   For y-axis spin, recall that our eigenstates are

\begin{equation}\label{eqn:desaiDiracAdjoint:240}
\ket{\pm}
=
\inv{\sqrt{2}}
\begin{bmatrix}
1 \\
\pm i
\end{bmatrix}
\end{equation}

Our positive energy states should therefore be
\begin{equation}\label{eqn:desaiDiracAdjoint:940}
\begin{aligned}
u^{\pm}(\Bp) &\sim
\begin{bmatrix}
\begin{bmatrix}
1 \\
\pm i
\end{bmatrix} \\
\frac{\Bsigma \cdot \Bp}{\Abs{E} + m} 
\begin{bmatrix}
1 \\
\pm i
\end{bmatrix} 
\end{bmatrix} \\
&=
\begin{bmatrix}
1 \\
\pm i \\
\frac{1}{\Abs{E} + m} 
\begin{bmatrix}
p_z &  p_x - i p_y \\
p_x + i p_y & - p_z
\end{bmatrix}
\begin{bmatrix}
1 \\
\pm i
\end{bmatrix} 
\end{bmatrix} \\
&\sim
\begin{bmatrix}
E + m \\
\pm i (E + m) \\
p_z \pm i p_x \pm p_y \\
p_x + i p_y \mp p_z 
\end{bmatrix}
\end{aligned}
\end{equation}

It is straightforward to verify that these are solutions.  We find for example that

\begin{equation}\label{eqn:desaiDiracAdjoint:260}
\begin{bmatrix}
E - m & - \Bsigma \cdot \Bp \\
- \Bsigma \cdot \Bp & E + m
\end{bmatrix}
u^{+}
\sim 
\begin{bmatrix}
E^2 - m^2 - \Bp^2 \\
i (E^2 - m^2 - \Bp^2 ) \\
0 \\
0
\end{bmatrix}
= 0,
\end{equation}

as expected.  What is the general solution?  For 

\begin{equation}\label{eqn:desaiDiracAdjoint:280}
\Bn = 
\begin{bmatrix}
\sin\theta \cos\phi \\
\sin\theta \sin\phi \\
\cos\theta 
\end{bmatrix},
\end{equation}

we find 
\begin{equation}\label{eqn:desaiDiracAdjoint:300}
\Bsigma \cdot \Bn =
\begin{bmatrix}
\cos\theta & \sin\theta e^{-i\phi} \\
\sin\theta e^{i\phi} & -\cos\theta
\end{bmatrix},
\end{equation}

with eigenstates
\begin{subequations}
\begin{equation}\label{eqn:desaiDiracAdjoint:320}
\begin{aligned}
\ket{+} 
&=
\begin{bmatrix}
\cos(\theta/2) e^{-i\phi/2} \\
\sin(\theta/2) e^{i\phi/2} \\
\end{bmatrix} \\
\ket{-} 
&=
\begin{bmatrix}
-\sin(\theta/2) e^{-i\phi/2} \\
\cos(\theta/2) e^{i\phi/2} \\
\end{bmatrix} 
\end{aligned}
\end{equation}
\end{subequations}

Should we wish to consider an arbitrarily oriented spin, expressing \(\Bp\) in spherical coordinates also makes sense

\begin{equation}\label{eqn:desaiDiracAdjoint:340}
\Bp = 
\Abs{\Bp}
\begin{bmatrix}
\sin\alpha \cos\beta \\
\sin\alpha \sin\beta \\
\cos\alpha 
\end{bmatrix}
\end{equation}

and we find (with \(S\) and \(C\) for \(\sin\) and \(\cos\) respectively)

\begin{subequations}
\begin{equation}\label{eqn:desaiDiracAdjoint:360}
\begin{aligned}
\Bsigma \cdot \Bp \ket{+}
&=
\Abs{\Bp}
\begin{bmatrix}
 C_\alpha C_{\theta/2} e^{-i \phi} + S_\alpha S_{\theta/2} e^{-i \beta} \\
 S_\alpha C_{\theta/2} e^{i (\beta - \phi)} - C_\alpha S_{\theta/2} 
\end{bmatrix} \\
\Bsigma \cdot \Bp \ket{-}
&=
\Abs{\Bp}
\begin{bmatrix}
- C_\alpha S_{\theta/2} e^{-i \phi} + S_\alpha C_{\theta/2} e^{-i \beta} \\
- S_\alpha S_{\theta/2} e^{i (\beta - \phi)} - C_\alpha C_{\theta/2} 
\end{bmatrix}
\end{aligned}
\end{equation}
\end{subequations}

Substitution back into \eqnref{eqn:desaiDiracAdjoint:30}, and \eqnref{eqn:desaiDiracAdjoint:50} is then easy.  Expressing these with the angles expressed as sums and differences is strongly suggested.  With \(\Delta = (\beta - \phi)/2\), and \(\delta = (\beta + \phi)/2\) this gives

\begin{subequations}
\begin{equation}\label{eqn:desaiDiracAdjoint:380}
\begin{aligned}
\Bsigma \cdot \Bp \ket{+}
&=
\Abs{\Bp}
\begin{bmatrix}
e^{-i\delta}
\left(
C_{\alpha - \theta/2} C_\Delta + i C_{\alpha + \theta/2} S_\Delta 
\right) \\
e^{i \Delta}
\left(
S_{\alpha - \theta/2} C_\Delta + i S_{\alpha + \theta/2} S_\Delta 
\right) \\
\end{bmatrix} \\
\Bsigma \cdot \Bp \ket{-}
&=
\Abs{\Bp}
\begin{bmatrix}
e^{-i\delta}
\left(
S_{\alpha - \theta/2} C_\Delta - i S_{\alpha + \theta/2} S_\Delta 
\right) \\
e^{i \Delta}
\left(
-C_{\alpha - \theta/2} C_\Delta + i C_{\alpha + \theta/2} S_\Delta 
\right) \\
\end{bmatrix} 
\end{aligned}
\end{equation}
\end{subequations}

This is probably about as tidy as things can be made for the general case.

\subsection{Expanding the current equation}

With 

\begin{equation}\label{eqn:desaiDiracAdjoint:720}
\Bj = \psi^\dagger \Balpha \psi = 
\begin{bmatrix}
u_1^\dagger & u_2^\dagger
\end{bmatrix}
\begin{bmatrix}
0 & \Bsigma \\
\Bsigma & 0 
\end{bmatrix}
\begin{bmatrix}
u_1 \\
u_2
\end{bmatrix}
= u_1^\dagger \Bsigma u_2 + u_2^\dagger \Bsigma u_1
\end{equation}

We can expand the current for a general spin up or spin down state \(\ket{r}\) with respect to either the positive energy or negative energy solutions.

Those (normalized) solutions are respectively

\begin{equation}\label{eqn:desaiDiracAdjoint:740}
\begin{aligned}
\psi_{+} 
&=
\sqrt{\frac{\Abs{E} + m}{2m}}
\begin{bmatrix}
\ket{r} \\
\frac{\Bsigma \cdot \Bp}{\Abs{E}+ m} \ket{r} \\
\end{bmatrix} \\
\psi_{-} 
&=
\sqrt{\frac{\Abs{E} + m}{2m}}
\begin{bmatrix}
-\frac{\Bsigma \cdot \Bp}{\Abs{E}+ m} \ket{r} \\
\ket{r} \\
\end{bmatrix}
\end{aligned}
\end{equation}

For the \(i\)th component of the positive energy solution current we have

\begin{equation}\label{eqn:desaiDiracAdjoint:960}
\begin{aligned}
\psi_{+}^\dagger \Balpha \psi_{+}
&=
\frac{\Abs{E} + m}{2m}
\begin{bmatrix}
\bra{r} & \bra{r} \frac{\Bsigma \cdot \Bp}{\Abs{E} + m}
\end{bmatrix}
\begin{bmatrix}
\sigma_i \frac{\Bsigma \cdot \Bp}{\Abs{E} + m} \ket{r} \\
\sigma_i \ket{r}
\end{bmatrix} \\
&=
\frac{1}{2m}
\bra{r} \left(
\sigma_i (\Bsigma \cdot \Bp)
+
(\Bsigma \cdot \Bp) \sigma_i 
\right) \ket{r}
\end{aligned}
\end{equation}

Similarly for a negative energy solution we have

\begin{equation}\label{eqn:desaiDiracAdjoint:980}
\begin{aligned}
\psi_{-}^\dagger \Balpha \psi_{-}
&=
\frac{\Abs{E} + m}{2m}
\begin{bmatrix}
-\bra{r} \frac{\Bsigma \cdot \Bp}{\Abs{E} + m} & \bra{r}
\end{bmatrix}
\begin{bmatrix}
\sigma_i \ket{r} \\
-\sigma_i \frac{\Bsigma \cdot \Bp}{\Abs{E} + m} \ket{r}
\end{bmatrix} \\
&=
\frac{1}{2m}
\bra{r} \left(
-\sigma_i (\Bsigma \cdot \Bp)
-
(\Bsigma \cdot \Bp) \sigma_i 
\right) \ket{r}
\end{aligned}
\end{equation}

We can expand the inner term of both easily

\begin{equation}\label{eqn:desaiDiracAdjoint:760}
\sigma_i (\Bsigma \cdot \Bp) + (\Bsigma \cdot \Bp) \sigma_i 
=
2 \sigma_i^2 p_i + \sum_{i \ne j} (\cancel{\sigma_i \sigma_j + \sigma_j \sigma_i}) p^j
\end{equation}

so that we have for the positive and negative energy solutions currents of
\begin{equation}\label{eqn:desaiDiracAdjoint:780}
\begin{aligned}
j_i &= \bra{r} \frac{p_i}{m} \ket{r} \\
j_i &= -\bra{r} \frac{p_i}{m} \ket{r}.
\end{aligned}
\end{equation}

This finds the velocity dependence noted in \S 33.4, but does not require taking any specific spin orientation, nor any specific momentum direction.

\subsection{Unpacking the covariant equation}

Pre-multiplication of the covariant Dirac equation by \(\gamma^4\) should provide a space-time split of the Dirac equation.  Let us verify this

\begin{equation}\label{eqn:desaiDiracAdjoint:1000}
\begin{aligned}
\gamma^4 (\gamma \cdot p - m)
&=
\gamma^4 (\gamma_\mu p^\mu - m) \\
&=
E
\begin{bmatrix}
1 & 0 \\
0 & 1 
\end{bmatrix}
 + \gamma^4 \gamma_a p^a - m 
\begin{bmatrix}
1 & 0 \\
0 & -1 
\end{bmatrix},
\end{aligned}
\end{equation}

but
\begin{equation}\label{eqn:desaiDiracAdjoint:1020}
\begin{aligned}
\gamma^4 \gamma_a 
=
\begin{bmatrix}
1 & 0 \\
0 & -1 
\end{bmatrix}
\begin{bmatrix}
0 & -\sigma_a \\
\sigma_a & 0 
\end{bmatrix}
=
\begin{bmatrix}
0 & -\sigma_a \\
-\sigma_a & 0 
\end{bmatrix}
\end{aligned}
\end{equation}

\begin{equation}\label{eqn:desaiDiracAdjoint:400}
\gamma^4 (\gamma \cdot p - m)
=
E
\begin{bmatrix}
1 & 0 \\
0 & 1 
\end{bmatrix}
 - 
\begin{bmatrix}
0 & \sigma_a \\
\sigma_a & 0 
\end{bmatrix}
p^a - m 
\begin{bmatrix}
1 & 0 \\
0 & -1 
\end{bmatrix} =
\begin{bmatrix}
E - m & - \sigma \cdot \Bp \\
- \sigma \cdot \Bp & E + m
\end{bmatrix}.
\end{equation}

This recovers \eqnref{eqn:desaiDiracAdjoint:10} as expected.

\subsection{Two by two form for the covariant equations}

If we put the covariant Dirac equations in two by two matrix form we get

\begin{equation}\label{eqn:desaiDiracAdjoint:1040}
\begin{aligned}
0
&= 
(\gamma \cdot p - m ) u \\
&= 
\left(
\Abs{E} 
\begin{bmatrix}
1 & 0 \\
0 & -1 
\end{bmatrix}
+ 
\begin{bmatrix}
0 & - \sigma_a \\
\sigma_a & 0
\end{bmatrix}
p^a
- m
\begin{bmatrix}
1 & 0 \\
0 & 1 
\end{bmatrix}
\right) u \\
&=
\begin{bmatrix}
\Abs{E} - m & - \Bsigma \cdot \Bp \\
\Bsigma \cdot \Bp & -\Abs{E} - m
\end{bmatrix} u
\end{aligned}
\end{equation}

and

\begin{equation}\label{eqn:desaiDiracAdjoint:1060}
\begin{aligned}
0 &= (\gamma \cdot p + m ) v \\
&= 
\left(
\Abs{E} 
\begin{bmatrix}
1 & 0 \\
0 & -1 
\end{bmatrix}
+ 
\begin{bmatrix}
0 & - \sigma_a \\
\sigma_a & 0
\end{bmatrix}
p^a
+ m
\begin{bmatrix}
1 & 0 \\
0 & 1 
\end{bmatrix}
\right) v \\
&=
\begin{bmatrix}
\Abs{E} + m & - \Bsigma \cdot \Bp \\
\Bsigma \cdot \Bp & -\Abs{E} + m
\end{bmatrix} v
\end{aligned}
\end{equation}

This form makes it easy to verify that our solutions are

\begin{equation}\label{eqn:desaiDiracAdjoint:30a}
u_r(\Bp) =
\sqrt{\frac{\Abs{E} + m}{2m}}
\begin{bmatrix}
\ket{r} \\
\frac{\Bsigma \cdot \Bp}{\Abs{E} + m} \ket{r}
\end{bmatrix},
\end{equation}

and

\begin{equation}\label{eqn:desaiDiracAdjoint:50a}
v_r(\Bp) =
\sqrt{\frac{\Abs{E} + m}{2m}}
\begin{bmatrix}
\frac{\Bsigma \cdot \Bp}{\Abs{E} + m} \ket{r} \\
\ket{r} \\
\end{bmatrix}.
\end{equation}

It is curious to consider these part of a basis for a single equation.  I suppose that all together they are actually eigenstates of the equation

\begin{equation}\label{eqn:desaiDiracAdjoint:500}
(\gamma \cdot p + m) (\gamma \cdot p - m) u = ((\gamma \cdot p)^2 - m^2) u = 0,
\end{equation}

or
\begin{equation}\label{eqn:desaiDiracAdjoint:520}
(\gamma \cdot p - m) (\gamma \cdot p + m) v = ((\gamma \cdot p)^2 - m^2) v = 0,
\end{equation}

which have the form of the Klein-Gordan equation.

\subsection{Orthonormality}

Orthonormality for the \(u\) vectors is easy to show, and we can do so without requiring any specific spin orientation

\begin{equation}\label{eqn:desaiDiracAdjoint:1080}
\begin{aligned}
\overbar{u}_r u_s 
&= 
\frac{\Abs{E} + m}{2m}
\begin{bmatrix}
\bra{r} &
\bra{r} \frac{\Bsigma \cdot \Bp}{\Abs{E} + m} 
\end{bmatrix}
\gamma^4
\begin{bmatrix}
\ket{s} \\
\frac{\Bsigma \cdot \Bp}{\Abs{E} + m} \ket{s}
\end{bmatrix} \\
&=
\frac{\Abs{E} + m}{2m}
\begin{bmatrix}
\bra{r} &
-\bra{r} \frac{\Bsigma \cdot \Bp}{\Abs{E} + m} 
\end{bmatrix}
\begin{bmatrix}
\ket{s} \\
\frac{\Bsigma \cdot \Bp}{\Abs{E} + m} \ket{s}
\end{bmatrix} \\
&=
\frac{1}{2m(\Abs{E} + m)}
\braket{r}{s} \left( E^2 + m^2 + 2 \Abs{E} m - \Bp^2 \right) \\
&=
\braket{r}{s}.
\end{aligned}
\end{equation}

It is also easy for \(v\) vectors

\begin{equation}\label{eqn:desaiDiracAdjoint:1100}
\begin{aligned}
\overbar{v}_r v_s 
&= 
\frac{\Abs{E} + m}{2m}
\begin{bmatrix}
\bra{r} \frac{\Bsigma \cdot \Bp}{\Abs{E} + m} &
\bra{r} 
\end{bmatrix}
\gamma^4
\begin{bmatrix}
\frac{\Bsigma \cdot \Bp}{\Abs{E} + m} \ket{s} \\
\ket{s} 
\end{bmatrix} \\
&=
\frac{\Abs{E} + m}{2m}
\begin{bmatrix}
\bra{r} \frac{\Bsigma \cdot \Bp}{\Abs{E} + m} &
-\bra{r} 
\end{bmatrix}
\begin{bmatrix}
\frac{\Bsigma \cdot \Bp}{\Abs{E} + m} \ket{s} \\
\ket{s} 
\end{bmatrix} \\
&=
-\frac{1}{2m(\Abs{E} + m)}
\braket{r}{s} \left( E^2 + m^2 + 2 \Abs{E} m - \Bp^2 \right) \\
&=
-\braket{r}{s}.
\end{aligned}
\end{equation}

For the cross terms we have

\begin{equation}\label{eqn:desaiDiracAdjoint:1120}
\begin{aligned}
\overbar{u}_r v_s 
&= 
\frac{\Abs{E} + m}{2m}
\begin{bmatrix}
\bra{r} &
\bra{r} \frac{\Bsigma \cdot \Bp}{\Abs{E} + m} 
\end{bmatrix}
\gamma^4
\begin{bmatrix}
\frac{\Bsigma \cdot \Bp}{\Abs{E} + m} \ket{s} \\
\ket{s} 
\end{bmatrix} \\
&= 
\frac{\Abs{E} + m}{2m}
\begin{bmatrix}
\bra{r} &
-\bra{r} \frac{\Bsigma \cdot \Bp}{\Abs{E} + m} 
\end{bmatrix}
\begin{bmatrix}
\frac{\Bsigma \cdot \Bp}{\Abs{E} + m} \ket{s} \\
\ket{s} 
\end{bmatrix} \\
&=
\frac{1}{2m}
\bra{r} ( \Bsigma \cdot \Bp - \Bsigma \cdot \Bp ) \ket{s} \\
&= 0
\end{aligned}
\end{equation}

\subsection{Resolution of identity}

It is claimed that an identity representation is

\begin{equation}\label{eqn:desaiDiracAdjoint:540}
\BOne = \sum_r u_r \overbar{u}_r - v_r \overbar{v}_r
\end{equation}

This makes some sense, but we can see systematically why we have this negative sign.  Suppose that we have a basis \(\ket{a_i}\) for which we have \(\braket{a_i}{a_j} = \pm \delta_{ij}\) (rather than the strict orthonormality condition \(\braket{a_i}{a_j} = \delta_{ij}\)).  Consider the calculation of the Fourier coefficients of a state

\begin{equation}\label{eqn:desaiDiracAdjoint:560}
\ket{a} = \alpha_i \ket{a_i}.
\end{equation}

We have 
\begin{equation}\label{eqn:desaiDiracAdjoint:580}
\braket{a_j}{a} = \alpha_i \braket{a_j}{a_i}.
\end{equation}

For \(i \ne j\) \(\braket{a_i}{a_j} = 0\), so that the coefficient is
\begin{equation}\label{eqn:desaiDiracAdjoint:600}
\alpha_j =
\frac{\braket{a_j}{a}}{\braket{a_j}{a_j}}.
\end{equation}

The coordinate representation of this state vector with respect to this basis is thus

\begin{equation}\label{eqn:desaiDiracAdjoint:620}
\ket{a} 
= 
\sum_i \left( \frac{\braket{a_i}{a}}{\braket{a_i}{a_i}} \right)
\ket{a_i}.
\end{equation}

Shuffling things around, employing the somewhat abusive seeming Dirac ket-bra operator notation, we find the general identity operation takes the form

\begin{equation}\label{eqn:desaiDiracAdjoint:640}
\ket{a} = \left( \frac{\ket{a_i} \bra{a_i} }{\braket{a_i}{a_i}} \right) \ket{a},
\end{equation}

so that the identity itself has the form
\begin{equation}\label{eqn:desaiDiracAdjoint:660}
\BOne = \frac{\ket{a_i} \bra{a_i} }{\braket{a_i}{a_i}}.
\end{equation}

This is the sum of all the ket-bras for which the braket is one, minus the sum of all the ket-bras for which the braket is negative, showing that the form of the claimed identity is justified.

We can also verify this directly by computation, and find

\begin{equation}\label{eqn:desaiDiracAdjoint:1140}
\begin{aligned}
\sum_r u_r \overbar{u}_r 
&=
\frac{\Abs{E} + m}{2m}
\sum_r 
\begin{bmatrix}
\ket{r} \\
\frac{\Bsigma \cdot \Bp \ket{r}}{\Abs{E} + m}
\end{bmatrix}
\begin{bmatrix}
\bra{r} &
-\frac{\bra{r} \Bsigma \cdot \Bp}{\Abs{E} + m}
\end{bmatrix} \\
&=
\frac{\Abs{E} + m}{2m}
\sum_r 
\begin{bmatrix}
\ket{r}\bra{r} & -\ket{r}\bra{r} \frac{\Bsigma \cdot \Bp}{\Abs{E} + m} \\
\frac{\Bsigma \cdot \Bp}{\Abs{E} + m} \ket{r}\bra{r} &
-\frac{\Bsigma \cdot \Bp}{\Abs{E} + m} \ket{r}\bra{r} \frac{\Bsigma \cdot \Bp}{\Abs{E} + m}  \\
\end{bmatrix}
\end{aligned}
\end{equation}

We can pull the summation into the matrices and note that \(\sum_r \ket{r}\bra{r} = \BOne\) (the two by two identity), so that we are left with

\begin{equation}\label{eqn:desaiDiracAdjoint:680}
\sum_r u_r \overbar{u}_r 
=
\inv{2m}
\begin{bmatrix}
\Abs{E} + m & -\Bsigma \cdot \Bp \\
\Bsigma \cdot \Bp &
-\frac{\Bp^2}{\Abs{E} + m} 
\end{bmatrix}.
\end{equation}

Similarly, we find
\begin{equation}\label{eqn:desaiDiracAdjoint:700}
-\sum_r v_r \overbar{v}_r 
=
\inv{2m}
\begin{bmatrix}
-\frac{\Bp^2}{\Abs{E} + m}  & \Bsigma \cdot \Bp \\
-\Bsigma \cdot \Bp  & \Abs{E} + m 
\end{bmatrix},
\end{equation}

summing the two (noting that \(E^2 - \Bp^2 - m^2 = 0\)) we get the block identity matrix as desired.

We have also just calculated the projection operators.  Let us verify that expanding the covariant form in the text produces the same result

\begin{equation}\label{eqn:desaiDiracAdjoint:1160}
\begin{aligned}
\inv{2m}(m \pm \gamma \cdot p) 
&=
\inv{2m}(m \pm \gamma^4 \Abs{E} \pm \gamma_a p^a ) \\
&=
\inv{2m}
\left(
m
\begin{bmatrix}
1 & 0 \\
0 & 1
\end{bmatrix}
\pm \Abs{E}
\begin{bmatrix}
1 & 0 \\
0 & -1
\end{bmatrix}
\pm p^a
\begin{bmatrix}
0 & -\sigma_a \\
\sigma_a & 0 
\end{bmatrix}
\right) \\
&=
\inv{2m}
\begin{bmatrix}
m \pm \Abs{E} & \mp \Bsigma \cdot \Bp \\
\pm \Bsigma \cdot \Bp & m \mp \Abs{E} 
\end{bmatrix}
\end{aligned}
\end{equation}

Now compare to \eqnref{eqn:desaiDiracAdjoint:680}, and \eqnref{eqn:desaiDiracAdjoint:700}, which we rewrite using \(-\Bp^2/(m + \Abs{E}) = m - \Abs{E}\) as 

\begin{equation}\label{eqn:desaiDiracAdjoint:680b}
\begin{aligned}
\sum_r u_r \overbar{u}_r 
&=
\inv{2m}
\begin{bmatrix}
\Abs{E} + m & -\Bsigma \cdot \Bp \\
\Bsigma \cdot \Bp &
m - \Abs{E}
\end{bmatrix} \\
-\sum_r v_r \overbar{v}_r 
&=
\inv{2m}
\begin{bmatrix}
m - \Abs{E} & \Bsigma \cdot \Bp \\
-\Bsigma \cdot \Bp  & \Abs{E} + m 
\end{bmatrix}
\end{aligned}
\end{equation}

\subsection{Lorentz transformation of Dirac equation}

Equation (35.107) in the text is missing the positional notation to show the placement of the indices, and should be

\begin{equation}\label{eqn:desaiDiracAdjoint:800}
\antisymmetric{\Sigma}{\gamma^\nu} = e_\mu^{.\nu} \gamma^\mu,
\end{equation}

where the solution is

\begin{equation}\label{eqn:desaiDiracAdjoint:820}
\Sigma = \inv{4} \gamma^\alpha \gamma^\beta e_{\alpha \beta}
\end{equation}

This does have the form I had expect, a bivector, but we can show explicitly that this is the solution without too much trouble.  Consider the commutator

\begin{equation}\label{eqn:desaiDiracAdjoint:1180}
\begin{aligned}
\antisymmetric{ \gamma^\alpha \gamma^\beta e_{\alpha \beta} }{\gamma^\nu}
&=
e_{\alpha \beta} \antisymmetric{ \gamma^\alpha \gamma^\beta }{\gamma^\nu} \\
&=
e_{\alpha \beta} \left( 
\gamma^\alpha \gamma^\beta \gamma^\nu
-\gamma^\nu \gamma^\alpha \gamma^\beta 
\right) \\
&=
e_{\alpha \beta} \left( 
\left( 
\cancel{\gamma^\alpha \cdot \gamma^\beta }
+\gamma^\alpha \wedge \gamma^\beta 
\right) 
\gamma^\nu
-\gamma^\nu 
\left( 
\cancel{\gamma^\alpha \cdot \gamma^\beta }
+\gamma^\alpha \wedge \gamma^\beta 
\right) 
\right) \\
&=
e_{\alpha \beta} 
\left(
\left(\gamma^\alpha \wedge \gamma^\beta \right)
\gamma^\nu
-\gamma^\nu 
\left(\gamma^\alpha \wedge \gamma^\beta \right)
\right)
\\
&=
e_{\alpha \beta} 
\left(
\left(\gamma^\alpha \wedge \gamma^\beta \right) \wedge \gamma^\nu
-\gamma^\nu \wedge \left(\gamma^\alpha \wedge \gamma^\beta \right)
\right)
+
e_{\alpha \beta} 
\left(
\left(\gamma^\alpha \wedge \gamma^\beta \right) \cdot \gamma^\nu
-\gamma^\nu \cdot \left(\gamma^\alpha \wedge \gamma^\beta \right)
\right)
\\
&=
2 e_{\alpha \beta} 
\left(\gamma^\alpha \wedge \gamma^\beta \right) \cdot \gamma^\nu
\\
&=
2 e^{\alpha \beta} 
\left(\gamma_\alpha \wedge \gamma_\beta \right) \cdot \gamma^\nu
\\
&=
2 e^{\alpha \beta} 
\left(
\gamma_\alpha \delta_\beta^{.\nu}
-
\gamma_\beta  \delta_\alpha^{.\nu}
\right)
\\
&=
4 e^{\alpha \nu} 
\gamma_\alpha 
\\
&=
4 e_\alpha^{.\nu} 
\gamma^\alpha 
\\
\end{aligned}
\end{equation}

Would this be any easier to prove without utilizing the dot and wedge product identities?  I used a few of them, starting with

\begin{equation}\label{eqn:desaiDiracAdjoint:840}
\begin{aligned}
a \cdot b &= \inv{2} (a b + b a) = \inv{2} \symmetric{a}{b} \\
a \wedge b &= \inv{2} (a b - b a) = \inv{2} \antisymmetric{a}{b} \\
a b &= a \cdot b + a \wedge b = \inv{2} ( \symmetric{a}{b} + \antisymmetric{a}{b} )
\end{aligned}
\end{equation}

In matrix notation we would have to show that the anticommutator \(\symmetric{\gamma^\alpha}{\gamma^\beta}\) commutes with any \(\gamma^\nu\) to make the first cancellation.  We can do so by noting

\begin{equation}\label{eqn:desaiDiracAdjoint:1200}
\begin{aligned}
\antisymmetric{\gamma^\alpha \gamma^\beta + \gamma^\beta \gamma^\alpha}{\gamma^\nu} 
&= \antisymmetric{ 2 g^{\alpha \beta} \BOne}{\gamma^\nu} \\
&= 2 g^{\alpha \beta} \antisymmetric{\BOne}{\gamma^\nu} \\
&= 0
\end{aligned}
\end{equation}

That is enough to get us on the path to how to prove this in matrix form

\begin{equation}\label{eqn:desaiDiracAdjoint:1220}
\begin{aligned}
\antisymmetric{ \gamma^\alpha \gamma^\beta e_{\alpha \beta} }{\gamma^\nu}
&=
e_{\alpha \beta} \antisymmetric{ \gamma^\alpha \gamma^\beta }{\gamma^\nu} \\
&=
e_{\alpha \beta} \left( 
\gamma^\alpha \gamma^\beta \gamma^\nu
-\gamma^\nu \gamma^\alpha \gamma^\beta 
\right) \\
&=
\inv{2} e_{\alpha \beta} \left( 
\left( 
\symmetric{\gamma^\alpha}{\gamma^\beta}
+\antisymmetric{\gamma^\alpha }{ \gamma^\beta }
\right) 
\gamma^\nu
-\gamma^\nu 
\left( 
\symmetric{\gamma^\alpha }{\gamma^\beta }
+\antisymmetric{\gamma^\alpha }{\gamma^\beta }
\right) 
\right) \\
&=
\inv{2} e_{\alpha \beta} \left( 
\antisymmetric{\gamma^\alpha }{ \gamma^\beta } \gamma^\nu
-\gamma^\nu \antisymmetric{\gamma^\alpha }{ \gamma^\beta } 
\right) \\
&=
\inv{2} e_{\alpha \beta} \antisymmetric{
\antisymmetric{\gamma^\alpha }{ \gamma^\beta } }{\gamma^\nu} \\
&=
\inv{2} e_{\alpha \beta} \antisymmetric{
\antisymmetric{\gamma^\alpha }{ \gamma^\beta } }{\gamma^\nu} \\
&=
\inv{2} e_{\alpha \beta} \antisymmetric{
\gamma^\alpha \gamma^\beta 
-\gamma^\beta \gamma^\alpha 
}
{\gamma^\nu} \\
&=
e_{\alpha \beta} \antisymmetric{
\gamma^\alpha \gamma^\beta 
}
{\gamma^\nu} \\
&=
e_{\alpha \beta} 
\left(
\gamma^\alpha \gamma^\beta \gamma^\nu
-\gamma^\nu \gamma^\alpha \gamma^\beta 
\right) \\
&=
e_{\alpha \beta} 
\left(
\gamma^\alpha 
( 2 g^{\beta \nu} - \gamma^\nu \gamma^\beta )
-\gamma^\nu \gamma^\alpha \gamma^\beta 
\right) \\
&=
e_{\alpha \beta} 
\left(
2 \gamma^\alpha g^{\beta \nu} 
- \gamma^\alpha \gamma^\nu \gamma^\beta 
-\gamma^\nu \gamma^\alpha \gamma^\beta 
\right) \\
&=
2 e_{\alpha \beta} 
\left(
\gamma^\alpha g^{\beta \nu} 
- g^{\alpha \nu} \gamma^\beta 
\right) \\
&=
2 e_{\alpha \beta} 
\gamma^\alpha g^{\beta \nu} 
+ 
2 e_{\beta \alpha} 
g^{\alpha \nu} \gamma^\beta  \\
&=
2 e_{\alpha}^{. \nu} 
\gamma^\alpha 
+ 
2 e_{\beta}^{.\nu} 
\gamma^\beta  \\
&=
4 e_{\alpha}^{. \nu} 
\gamma^\alpha 
\end{aligned}
\end{equation}

A last relation in this section was that we have for the inverse of the transformation

\begin{equation}\label{eqn:desaiDiracAdjoint:860}
S^{-1} = \gamma_4 S^\dagger \gamma_4.
\end{equation}

The incremental version \(S^{-1} \sim 1 - \epsilon \Sigma\) of this transformation was already used, so to verify this we can compute directly

\begin{equation}\label{eqn:desaiDiracAdjoint:1240}
\begin{aligned}
\gamma_4 S^\dagger \gamma_4
&=
\gamma_4 \left(1 + \epsilon \Sigma\right)^\dagger \gamma_4 \\
&=
\gamma_4 \left(1 + \epsilon \Sigma^\dagger\right) \gamma_4 \\
&=
\gamma_4 \left(1 + \epsilon \inv{4} \left(\gamma^\alpha \gamma^\beta\right)^\dagger e_{\alpha \beta} \right) \gamma_4 \\
&=
\gamma_4 \left(1 + \epsilon \inv{4} \left(\gamma^\beta\right)^\dagger \left(\gamma^\alpha\right)^\dagger e_{\alpha \beta} \right) \gamma_4 \\
&=
\gamma_4 \left(1 + \epsilon \inv{4} \left(\gamma_4 \gamma^\beta \gamma_4\right) \left(\gamma_4 \gamma^\alpha \gamma_4\right) e_{\alpha \beta} \right) \gamma_4 \\
&=
1 + \epsilon \inv{4} \gamma^\beta \gamma^\alpha e_{\alpha \beta} \\
&=
1 - \epsilon \Sigma
\end{aligned}
\end{equation}

\EndArticle
