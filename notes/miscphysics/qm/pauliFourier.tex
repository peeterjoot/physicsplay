%
% Copyright � 2012 Peeter Joot.  All Rights Reserved.
% Licenced as described in the file LICENSE under the root directory of this GIT repository.
%

%
%
%%
% Copyright � 2015 Peeter Joot.  All Rights Reserved.
% Licenced as described in the file LICENSE under the root directory of this GIT repository.
%
\documentclass[]{eliblog}

\usepackage{amsmath}
\usepackage{mathpazo}

%
% shorthand for bold symbols, convenient for vectors and matrices
%
\newcommand{\Ba}[0]{\mathbf{a}}
\newcommand{\Bb}[0]{\mathbf{b}}
\newcommand{\Bc}[0]{\mathbf{c}}
\newcommand{\Bd}[0]{\mathbf{d}}
\newcommand{\Be}[0]{\mathbf{e}}
\newcommand{\Bf}[0]{\mathbf{f}}
\newcommand{\Bg}[0]{\mathbf{g}}
\newcommand{\Bh}[0]{\mathbf{h}}
\newcommand{\Bi}[0]{\mathbf{i}}
\newcommand{\Bj}[0]{\mathbf{j}}
\newcommand{\Bk}[0]{\mathbf{k}}
\newcommand{\Bl}[0]{\mathbf{l}}
\newcommand{\Bm}[0]{\mathbf{m}}
\newcommand{\Bn}[0]{\mathbf{n}}
\newcommand{\Bo}[0]{\mathbf{o}}
\newcommand{\Bp}[0]{\mathbf{p}}
\newcommand{\Bq}[0]{\mathbf{q}}
\newcommand{\Br}[0]{\mathbf{r}}
\newcommand{\Bs}[0]{\mathbf{s}}
\newcommand{\Bt}[0]{\mathbf{t}}
\newcommand{\Bu}[0]{\mathbf{u}}
\newcommand{\Bv}[0]{\mathbf{v}}
\newcommand{\Bw}[0]{\mathbf{w}}
\newcommand{\Bx}[0]{\mathbf{x}}
\newcommand{\By}[0]{\mathbf{y}}
\newcommand{\Bz}[0]{\mathbf{z}}
\newcommand{\BA}[0]{\mathbf{A}}
\newcommand{\BB}[0]{\mathbf{B}}
\newcommand{\BC}[0]{\mathbf{C}}
\newcommand{\BD}[0]{\mathbf{D}}
\newcommand{\BE}[0]{\mathbf{E}}
\newcommand{\BF}[0]{\mathbf{F}}
\newcommand{\BG}[0]{\mathbf{G}}
\newcommand{\BH}[0]{\mathbf{H}}
\newcommand{\BI}[0]{\mathbf{I}}
\newcommand{\BJ}[0]{\mathbf{J}}
\newcommand{\BK}[0]{\mathbf{K}}
\newcommand{\BL}[0]{\mathbf{L}}
\newcommand{\BM}[0]{\mathbf{M}}
\newcommand{\BN}[0]{\mathbf{N}}
\newcommand{\BO}[0]{\mathbf{O}}
\newcommand{\BP}[0]{\mathbf{P}}
\newcommand{\BQ}[0]{\mathbf{Q}}
\newcommand{\BR}[0]{\mathbf{R}}
\newcommand{\BS}[0]{\mathbf{S}}
\newcommand{\BT}[0]{\mathbf{T}}
\newcommand{\BU}[0]{\mathbf{U}}
\newcommand{\BV}[0]{\mathbf{V}}
\newcommand{\BW}[0]{\mathbf{W}}
\newcommand{\BX}[0]{\mathbf{X}}
\newcommand{\BY}[0]{\mathbf{Y}}
\newcommand{\BZ}[0]{\mathbf{Z}}

\newcommand{\Bzero}[0]{\mathbf{0}}
\newcommand{\Btheta}[0]{\boldsymbol{\theta}}
\newcommand{\Btau}[0]{\boldsymbol{\tau}}
\newcommand{\Bomega}[0]{\boldsymbol{\omega}}

%
% shorthand for unit vectors
%
\newcommand{\acap}[0]{\hat{\Ba}}
\newcommand{\bcap}[0]{\hat{\Bb}}
\newcommand{\ccap}[0]{\hat{\Bc}}
\newcommand{\dcap}[0]{\hat{\Bd}}
\newcommand{\ecap}[0]{\hat{\Be}}
\newcommand{\fcap}[0]{\hat{\Bf}}
\newcommand{\gcap}[0]{\hat{\Bg}}
\newcommand{\hcap}[0]{\hat{\Bh}}
\newcommand{\icap}[0]{\hat{\Bi}}
\newcommand{\jcap}[0]{\hat{\Bj}}
\newcommand{\kcap}[0]{\hat{\Bk}}
\newcommand{\lcap}[0]{\hat{\Bl}}
\newcommand{\mcap}[0]{\hat{\Bm}}
\newcommand{\ncap}[0]{\hat{\Bn}}
\newcommand{\ocap}[0]{\hat{\Bo}}
\newcommand{\pcap}[0]{\hat{\Bp}}
\newcommand{\qcap}[0]{\hat{\Bq}}
\newcommand{\rcap}[0]{\hat{\Br}}
\newcommand{\scap}[0]{\hat{\Bs}}
\newcommand{\tcap}[0]{\hat{\Bt}}
\newcommand{\ucap}[0]{\hat{\Bu}}
\newcommand{\vcap}[0]{\hat{\Bv}}
\newcommand{\wcap}[0]{\hat{\Bw}}
\newcommand{\xcap}[0]{\hat{\Bx}}
\newcommand{\ycap}[0]{\hat{\By}}
\newcommand{\zcap}[0]{\hat{\Bz}}
\newcommand{\thetacap}[0]{\hat{\Btheta}}

%
% to write R^n and C^n in a distinguishable fashion.  Perhaps change this
% to the double lined characters upon figuring out how to do so.
%
\newcommand{\C}[1]{$\mathbb{C}^{#1}$}
\newcommand{\R}[1]{$\mathbb{R}^{#1}$}

%
% various generally useful helpers
%

% derivative of #1 wrt. #2:
\newcommand{\D}[2] {\frac {d#2} {d#1}}

\newcommand{\inv}[1]{\frac{1}{#1}}
\newcommand{\cross}[0]{\times}

\newcommand{\abs}[1]{\lvert{#1}\rvert}
\newcommand{\norm}[1]{\lVert{#1}\rVert}
\newcommand{\innerprod}[2]{\langle{#1}, {#2}\rangle}
\newcommand{\dotprod}[2]{{#1} \cdot {#2}}
\newcommand{\bdotprod}[2]{\left({#1} \cdot {#2}\right)}
\newcommand{\crossprod}[2]{{#1} \cross {#2}}
\newcommand{\tripleprod}[3]{\dotprod{\left(\crossprod{#1}{#2}\right)}{#3}}

\DeclareMathOperator{\Proj}{Proj}
\DeclareMathOperator{\Span}{span}
\DeclareMathOperator{\Sgn}{sgn}
\DeclareMathOperator{\Area}{Area}
\DeclareMathOperator{\Volume}{Volume}

%
% A few miscellaneous things specific to this document
%
\newcommand{\crossop}[1]{\crossprod{#1}{}}

% R2 vector.
\newcommand{\VectorTwo}[2]{
\begin{bmatrix}
 {#1} \\
 {#2}
\end{bmatrix}
}

\newcommand{\VectorN}[1]{
\begin{bmatrix}
{#1}_1 \\
{#1}_2 \\
\vdots \\
{#1}_N \\
\end{bmatrix}
}

\newcommand{\DETuvij}[4]{
\begin{vmatrix}
 {#1}_{#3} & {#1}_{#4} \\
 {#2}_{#3} & {#2}_{#4}
\end{vmatrix}
}

\newcommand{\DETuvwijk}[6]{
\begin{vmatrix}
 {#1}_{#4} & {#1}_{#5} & {#1}_{#6} \\
 {#2}_{#4} & {#2}_{#5} & {#2}_{#6} \\
 {#3}_{#4} & {#3}_{#5} & {#3}_{#6}
\end{vmatrix}
}

\newcommand{\DETuvwxijkl}[8]{
\begin{vmatrix}
 {#1}_{#5} & {#1}_{#6} & {#1}_{#7} & {#1}_{#8} \\
 {#2}_{#5} & {#2}_{#6} & {#2}_{#7} & {#2}_{#8} \\
 {#3}_{#5} & {#3}_{#6} & {#3}_{#7} & {#3}_{#8} \\
 {#4}_{#5} & {#4}_{#6} & {#4}_{#7} & {#4}_{#8} \\
\end{vmatrix}
}

%\newcommand{\DETuvwxyijklm}[10]{
%\begin{vmatrix}
% {#1}_{#6} & {#1}_{#7} & {#1}_{#8} & {#1}_{#9} & {#1}_{#10} \\
% {#2}_{#6} & {#2}_{#7} & {#2}_{#8} & {#2}_{#9} & {#2}_{#10} \\
% {#3}_{#6} & {#3}_{#7} & {#3}_{#8} & {#3}_{#9} & {#3}_{#10} \\
% {#4}_{#6} & {#4}_{#7} & {#4}_{#8} & {#4}_{#9} & {#4}_{#10} \\
% {#5}_{#6} & {#5}_{#7} & {#5}_{#8} & {#5}_{#9} & {#5}_{#10}
%\end{vmatrix}
%}

% R3 vector.
\newcommand{\VectorThree}[3]{
\begin{bmatrix}
 {#1} \\
 {#2} \\
 {#3}
\end{bmatrix}
}



\author{Peeter Joot}
\email{peeter.joot@gmail.com}

%\documentclass[]{eliblogwidescreen}

\usepackage{amsmath}
\usepackage{mathpazo}

%
% shorthand for bold symbols, convenient for vectors and matrices
%
\newcommand{\Ba}[0]{\mathbf{a}}
\newcommand{\Bb}[0]{\mathbf{b}}
\newcommand{\Bc}[0]{\mathbf{c}}
\newcommand{\Bd}[0]{\mathbf{d}}
\newcommand{\Be}[0]{\mathbf{e}}
\newcommand{\Bf}[0]{\mathbf{f}}
\newcommand{\Bg}[0]{\mathbf{g}}
\newcommand{\Bh}[0]{\mathbf{h}}
\newcommand{\Bi}[0]{\mathbf{i}}
\newcommand{\Bj}[0]{\mathbf{j}}
\newcommand{\Bk}[0]{\mathbf{k}}
\newcommand{\Bl}[0]{\mathbf{l}}
\newcommand{\Bm}[0]{\mathbf{m}}
\newcommand{\Bn}[0]{\mathbf{n}}
\newcommand{\Bo}[0]{\mathbf{o}}
\newcommand{\Bp}[0]{\mathbf{p}}
\newcommand{\Bq}[0]{\mathbf{q}}
\newcommand{\Br}[0]{\mathbf{r}}
\newcommand{\Bs}[0]{\mathbf{s}}
\newcommand{\Bt}[0]{\mathbf{t}}
\newcommand{\Bu}[0]{\mathbf{u}}
\newcommand{\Bv}[0]{\mathbf{v}}
\newcommand{\Bw}[0]{\mathbf{w}}
\newcommand{\Bx}[0]{\mathbf{x}}
\newcommand{\By}[0]{\mathbf{y}}
\newcommand{\Bz}[0]{\mathbf{z}}
\newcommand{\BA}[0]{\mathbf{A}}
\newcommand{\BB}[0]{\mathbf{B}}
\newcommand{\BC}[0]{\mathbf{C}}
\newcommand{\BD}[0]{\mathbf{D}}
\newcommand{\BE}[0]{\mathbf{E}}
\newcommand{\BF}[0]{\mathbf{F}}
\newcommand{\BG}[0]{\mathbf{G}}
\newcommand{\BH}[0]{\mathbf{H}}
\newcommand{\BI}[0]{\mathbf{I}}
\newcommand{\BJ}[0]{\mathbf{J}}
\newcommand{\BK}[0]{\mathbf{K}}
\newcommand{\BL}[0]{\mathbf{L}}
\newcommand{\BM}[0]{\mathbf{M}}
\newcommand{\BN}[0]{\mathbf{N}}
\newcommand{\BO}[0]{\mathbf{O}}
\newcommand{\BP}[0]{\mathbf{P}}
\newcommand{\BQ}[0]{\mathbf{Q}}
\newcommand{\BR}[0]{\mathbf{R}}
\newcommand{\BS}[0]{\mathbf{S}}
\newcommand{\BT}[0]{\mathbf{T}}
\newcommand{\BU}[0]{\mathbf{U}}
\newcommand{\BV}[0]{\mathbf{V}}
\newcommand{\BW}[0]{\mathbf{W}}
\newcommand{\BX}[0]{\mathbf{X}}
\newcommand{\BY}[0]{\mathbf{Y}}
\newcommand{\BZ}[0]{\mathbf{Z}}

\newcommand{\Bzero}[0]{\mathbf{0}}
\newcommand{\Btheta}[0]{\boldsymbol{\theta}}
\newcommand{\Btau}[0]{\boldsymbol{\tau}}
\newcommand{\Bomega}[0]{\boldsymbol{\omega}}

%
% shorthand for unit vectors
%
\newcommand{\acap}[0]{\hat{\Ba}}
\newcommand{\bcap}[0]{\hat{\Bb}}
\newcommand{\ccap}[0]{\hat{\Bc}}
\newcommand{\dcap}[0]{\hat{\Bd}}
\newcommand{\ecap}[0]{\hat{\Be}}
\newcommand{\fcap}[0]{\hat{\Bf}}
\newcommand{\gcap}[0]{\hat{\Bg}}
\newcommand{\hcap}[0]{\hat{\Bh}}
\newcommand{\icap}[0]{\hat{\Bi}}
\newcommand{\jcap}[0]{\hat{\Bj}}
\newcommand{\kcap}[0]{\hat{\Bk}}
\newcommand{\lcap}[0]{\hat{\Bl}}
\newcommand{\mcap}[0]{\hat{\Bm}}
\newcommand{\ncap}[0]{\hat{\Bn}}
\newcommand{\ocap}[0]{\hat{\Bo}}
\newcommand{\pcap}[0]{\hat{\Bp}}
\newcommand{\qcap}[0]{\hat{\Bq}}
\newcommand{\rcap}[0]{\hat{\Br}}
\newcommand{\scap}[0]{\hat{\Bs}}
\newcommand{\tcap}[0]{\hat{\Bt}}
\newcommand{\ucap}[0]{\hat{\Bu}}
\newcommand{\vcap}[0]{\hat{\Bv}}
\newcommand{\wcap}[0]{\hat{\Bw}}
\newcommand{\xcap}[0]{\hat{\Bx}}
\newcommand{\ycap}[0]{\hat{\By}}
\newcommand{\zcap}[0]{\hat{\Bz}}
\newcommand{\thetacap}[0]{\hat{\Btheta}}

%
% to write R^n and C^n in a distinguishable fashion.  Perhaps change this
% to the double lined characters upon figuring out how to do so.
%
\newcommand{\C}[1]{$\mathbb{C}^{#1}$}
\newcommand{\R}[1]{$\mathbb{R}^{#1}$}

%
% various generally useful helpers
%

% derivative of #1 wrt. #2:
\newcommand{\D}[2] {\frac {d#2} {d#1}}

\newcommand{\inv}[1]{\frac{1}{#1}}
\newcommand{\cross}[0]{\times}

\newcommand{\abs}[1]{\lvert{#1}\rvert}
\newcommand{\norm}[1]{\lVert{#1}\rVert}
\newcommand{\innerprod}[2]{\langle{#1}, {#2}\rangle}
\newcommand{\dotprod}[2]{{#1} \cdot {#2}}
\newcommand{\bdotprod}[2]{\left({#1} \cdot {#2}\right)}
\newcommand{\crossprod}[2]{{#1} \cross {#2}}
\newcommand{\tripleprod}[3]{\dotprod{\left(\crossprod{#1}{#2}\right)}{#3}}

\DeclareMathOperator{\Proj}{Proj}
\DeclareMathOperator{\Span}{span}
\DeclareMathOperator{\Sgn}{sgn}
\DeclareMathOperator{\Area}{Area}
\DeclareMathOperator{\Volume}{Volume}

%
% A few miscellaneous things specific to this document
%
\newcommand{\crossop}[1]{\crossprod{#1}{}}

% R2 vector.
\newcommand{\VectorTwo}[2]{
\begin{bmatrix}
 {#1} \\
 {#2}
\end{bmatrix}
}

\newcommand{\VectorN}[1]{
\begin{bmatrix}
{#1}_1 \\
{#1}_2 \\
\vdots \\
{#1}_N \\
\end{bmatrix}
}

\newcommand{\DETuvij}[4]{
\begin{vmatrix}
 {#1}_{#3} & {#1}_{#4} \\
 {#2}_{#3} & {#2}_{#4}
\end{vmatrix}
}

\newcommand{\DETuvwijk}[6]{
\begin{vmatrix}
 {#1}_{#4} & {#1}_{#5} & {#1}_{#6} \\
 {#2}_{#4} & {#2}_{#5} & {#2}_{#6} \\
 {#3}_{#4} & {#3}_{#5} & {#3}_{#6}
\end{vmatrix}
}

\newcommand{\DETuvwxijkl}[8]{
\begin{vmatrix}
 {#1}_{#5} & {#1}_{#6} & {#1}_{#7} & {#1}_{#8} \\
 {#2}_{#5} & {#2}_{#6} & {#2}_{#7} & {#2}_{#8} \\
 {#3}_{#5} & {#3}_{#6} & {#3}_{#7} & {#3}_{#8} \\
 {#4}_{#5} & {#4}_{#6} & {#4}_{#7} & {#4}_{#8} \\
\end{vmatrix}
}

%\newcommand{\DETuvwxyijklm}[10]{
%\begin{vmatrix}
% {#1}_{#6} & {#1}_{#7} & {#1}_{#8} & {#1}_{#9} & {#1}_{#10} \\
% {#2}_{#6} & {#2}_{#7} & {#2}_{#8} & {#2}_{#9} & {#2}_{#10} \\
% {#3}_{#6} & {#3}_{#7} & {#3}_{#8} & {#3}_{#9} & {#3}_{#10} \\
% {#4}_{#6} & {#4}_{#7} & {#4}_{#8} & {#4}_{#9} & {#4}_{#10} \\
% {#5}_{#6} & {#5}_{#7} & {#5}_{#8} & {#5}_{#9} & {#5}_{#10}
%\end{vmatrix}
%}

% R3 vector.
\newcommand{\VectorThree}[3]{
\begin{bmatrix}
 {#1} \\
 {#2} \\
 {#3}
\end{bmatrix}
}



\author{Peeter Joot}
\email{peeter.joot@gmail.com}


\chapter{Fourier transformation of the Pauli QED wave equation (Take I)}
\label{chap:pauliFourier}
%\useCCL
\blogpage{http://sites.google.com/site/peeterjoot/math2010/pauliFourier.pdf}
\date{May 29, 2010}
\revisionInfo{pauliFourier.tex}

%\beginArtWithToc
\beginArtNoToc

\section{Motivation}

In \citep{feynman1961qed}, Feynman writes the Pauli wave equation for a non-relativistic treatment of a mass in a scalar and vector potential electrodynamic field.  That is

\begin{equation}\label{eqn:pauliFourier:1}
i \Hbar \PD{t}{\Psi} = \inv{2m} \left( \Bp - \frac{e}{c} \BA \right)^2 \Psi + e \phi \Psi
\end{equation}

Is this amenable to Fourier transform solution like so many other PDEs?  Let us give it a try.  It would also be interesting to attempt to apply such a computation to see if it is possible to calculate \(\gpgradezero{\Bx}\), and the first two derivatives of this expectation value.  I would guess that this would produce the Lorentz force equation.

\section{Prep}
\subsection{Fourier Notation}

Our transform pair will be written

\begin{subequations}
\begin{align}
\Psi(\Bx, t) &= \inv{(\sqrt{2 \pi})^3} \int \hat{\Psi}(\Bk, t) e^{i \Bk \cdot \Bx} d^3 \Bk \label{eqn:pauliFourier:2a} \\
\hat{\Psi}(\Bk, t) &= \inv{(\sqrt{2 \pi})^3} \int \Psi(\Bx, t) e^{-i \Bk \cdot \Bx} d^3 \Bx \label{eqn:pauliFourier:2b}
\end{align}
\end{subequations}

\subsection{Interpretation of the squared momentum operator}

Feynman actually wrote

\begin{equation}\label{eqn:pauliFourier:1f}
i \Hbar \PD{t}{\Psi} = \inv{2m}
\left[\sigma \cdot \left( \Bp - \frac{e}{c} \BA \right)\right]
\left[\sigma \cdot \left( \Bp - \frac{e}{c} \BA \right)\right]
 \Psi + e \phi \Psi
\end{equation}

That \(\sigma \cdot\) notation I am not familiar with, and I have written this as a plain old vector square.  If \(\Bp\) were not an operator, then this would be a scalar, but as written this actually also includes a bivector term proportional to \(\spacegrad \wedge \BA = I \BB\).  To see that, lets expand this operator explicitly.

\begin{equation}\label{eqn:pauliFourier:71}
\begin{aligned}
\left( \Bp - \frac{e}{c} \BA \right) \left( \Bp - \frac{e}{c} \BA \right) \Psi
&=
\left( \Bp^2 - \frac{e}{c} ( \Bp \BA + \BA \Bp ) + \frac{e^2}{c^2} \BA^2 \right) \Psi \\
&=
\left( - \Hbar^2 \spacegrad^2 + \frac{i e \Hbar }{c} ( \spacegrad \BA + \BA \spacegrad ) + \frac{e^2}{c^2} \BA^2 \right) \Psi \\
\end{aligned}
\end{equation}

This anticommutator of the vector potential and the gradient is only a scalar \(\BA \) has zero divergence.  More generally, expanding by chain rules, and using braces to indicate the scope of the differential operations, we have

\begin{equation}\label{eqn:pauliFourier:91}
\begin{aligned}
( \spacegrad \BA + \BA \spacegrad ) \Psi
&=
(\spacegrad \Psi) \BA + \BA (\spacegrad \Psi) + (\spacegrad \BA) \Psi \\
&=
2 \BA \cdot (\spacegrad \Psi) + (\spacegrad \cdot \BA) \Psi + I (\spacegrad \cross \BA) \Psi \\
&=
2 \BA \cdot (\spacegrad \Psi) + (\spacegrad \cdot \BA) \Psi + I \BB \Psi - I \BA \cross (\spacegrad \Psi) \\
\end{aligned}
\end{equation}

where \(I = \Be_1 \Be_2 \Be_3\) is the spatial unit trivector, and \(\BB = \spacegrad \cross \BA\).

This is assuming \(\Psi\) should be treated as a complex valued scalar, and not a complex-like geometric object of any sort.  Does this bivector term have physical meaning?  Should it be discarded or retained?  If we assume discarded, then we really want to write the Pauli equation utilizing an explicit scalar selection, as in

\begin{equation}\label{eqn:pauliFourier:1c}
i \Hbar \PD{t}{\Psi} = \inv{2m} \gpgradezero{ \left( \Bp - \frac{e}{c} \BA \right)^2 } \Psi + e \phi \Psi.
\end{equation}

Assuming that to be the case, our squared momentum operator takes the form

\begin{equation}\label{eqn:pauliFourier:21}
\begin{aligned}
\gpgradezero{ \left( \Bp - \frac{e}{c} \BA \right)^2 } \Psi
&=
\left( - \Hbar^2 \spacegrad^2 + 2
\frac{i e \Hbar }{c}
\BA \cdot \spacegrad
+
\frac{i e \Hbar }{c}
(\spacegrad \cdot \BA)
+
\frac{e^2}{c^2} \BA^2 \right) \Psi.
\end{aligned}
\end{equation}

The Pauli equation, written out explicitly in terms of the gradient is then

\begin{equation}\label{eqn:pauliFourier:22}
\begin{aligned}
i \Hbar \PD{t}{\Psi}
&=
\inv{2m}
\left( - \Hbar^2 \spacegrad^2 + 2
\frac{i e \Hbar }{c}
\BA \cdot \spacegrad
+
\frac{e^2}{c^2} \BA^2 \right) \Psi
+
e \left( \frac{i \Hbar }{ 2 m c} (\spacegrad \cdot \BA) + \phi \right) \Psi.
\end{aligned}
\end{equation}

\subsubsection{Confirmation}

Instead of guessing what Feynman means when he writes Pauli's equation, it would be better to just check what Pauli says.  In \citep{pauli2000wm} he uses the more straightforward notation

\begin{equation}\label{eqn:pauliFourier:25}
\inv{2m} \sum_{k=1}^3 \left( p_k - \frac{e}{c}A_k \right)^2
\end{equation}

for the vector potential dependent part of the Hamiltonian operator.  This is just the scalar part as was guessed.

\section{Guts}

Using the expansion \eqnref{eqn:pauliFourier:22} of the Pauli equation, and writing \(V = \phi + i \Hbar (\spacegrad \cdot \BA)/ (2 m c)\) for the effective complex potential we have

\begin{equation}\label{eqn:pauliFourier:3}
i \Hbar \PD{t}{\Psi} = \inv{2m} \left( - \Hbar^2 \spacegrad^2 + 2 i \Hbar \frac{e}{c} \BA \cdot \spacegrad + \frac{e^2}{c^2} \BA^2 \right) \Psi + e V \Psi.
\end{equation}

Let us now apply each of these derivative operations to our assumed Fourier solution \(\Psi(\Bx, t)\) from \eqnref{eqn:pauliFourier:2a}.  Starting with the Laplacian we have

\begin{equation}\label{eqn:pauliFourier:xx1}
\spacegrad^2 \Psi(\Bx, t) =
\inv{(\sqrt{2 \pi})^3} \int \hat{\Psi}(\Bk, t) (i\Bk)^2 e^{i \Bk \cdot \Bx} d^3 \Bk.
\end{equation}

For the \(\BA \cdot \spacegrad\) operator application we have

\begin{equation}\label{eqn:pauliFourier:xx2}
\BA \cdot \spacegrad \Psi(\Bx, t) =
\inv{(\sqrt{2 \pi})^3} \int \hat{\Psi}(\Bk, t) (i \BA \cdot \Bk) e^{i \Bk \cdot \Bx} d^3 \Bk.
\end{equation}

Putting both together we have

\begin{equation}\label{eqn:pauliFourier:5}
0 =
\inv{(\sqrt{2 \pi})^3} \int
\left(
-i \Hbar \PD{t}{\hat{\Psi}} + \inv{2m} \left( \Hbar^2 \Bk^2 - 2 \Hbar \frac{e}{c} \BA \cdot \Bk + \frac{e^2}{c^2} \BA^2 \right) \hat{\Psi} + e V \hat{\Psi} \right)
e^{i \Bk \cdot \Bx} d^3 \Bk.
\end{equation}

We can tidy this up slightly by completing the square, yielding

\begin{equation}\label{eqn:pauliFourier:5b}
0 =
\inv{(\sqrt{2 \pi})^3} \int
\left(
-i \Hbar \PD{t}{\hat{\Psi}} + \left( \inv{2m} \left( \Hbar \Bk - \frac{e}{c} \BA(\Bx, t) \right)^2 + e V(\Bx, t) \right) \hat{\Psi} \right)
e^{i \Bk \cdot \Bx} d^3 \Bk.
\end{equation}

If this is to be zero for all \((\Bx, t)\), it seems clear that we need \(\hat{\Psi}(\Bk, t)\) to be the solution of the first order non-linear PDE

\begin{equation}\label{eqn:pauliFourier:6}
\PD{t}{\hat{\Psi}}(\Bk, t) = \inv{i \Hbar} \left( \inv{2m} \left( \Hbar \Bk - \frac{e}{c} \BA(\Bx, t) \right)^2 + e V(\Bx, t) \right) \hat{\Psi}(\Bk, t)
\end{equation}

Somewhere along the way this got a bit confused.  Our Fourier transform function is somehow a function of not just wave number, but position, since \(\hat{\Psi} = \hat{\Psi}(\Bx, \Bk, t)\) by virtue of being a solution to a differential equation involving \(\BA(\Bx,t)\), and \(V(\Bx, t)\)?  Can we pretend to not to have noticed this and continue on anyways?  Let us try the further simplification of the system by imposing a constraint of constant time potentials (\(\PDi{t}{\BA} = \PDi{t}{V} = 0\)).  That allows for direct integration of the wave function's Fourier transform

\begin{equation}\label{eqn:pauliFourier:7}
\hat{\Psi}(\Bk, t) = \hat{\Psi}(\Bk, 0) \exp\left(
\inv{i \Hbar} \left( \inv{2m} \left( \Hbar \Bk - \frac{e}{c} \BA \right)^2 + e V \right) t
\right).
\end{equation}

And inverse transforming this

\begin{equation}\label{eqn:pauliFourier:8}
\Psi(\Bx, t) = \inv{(\sqrt{2 \pi})^3} \int
\hat{\Psi}(\Bk, 0) \exp\left(
\inv{i \Hbar} \left( \inv{2m} \left( \Hbar \Bk - \frac{e}{c} \BA(\Bx) \right)^2 + e V(\Bx) \right) t
+ i \Bk \cdot \Bx
\right)
d^3 \Bk.
\end{equation}

%This is still a bit of a fudge since the spatial dependence of \(\BA\) and \(V\) is not clear.  A reasonable guess of what is required to fix this up can be had
By inserting the inverse Fourier transform of \(\hat{\Psi}(\Bk, 0)\), we have the time evolution of the wave function as a convolution integral

\begin{equation}\label{eqn:pauliFourier:9}
\Psi(\Bx, t) = \inv{(2 \pi)^3} \int
\Psi(\Bx', 0) \exp\left(
\inv{i \Hbar} \left( \inv{2m} \left( \Hbar \Bk - \frac{e}{c} \BA(\Bx) \right)^2 + e V(\Bx) \right) t
+ i \Bk \cdot (\Bx - \Bx')
\right)
d^3 \Bk d^3 \Bx'.
\end{equation}

Splitting out the convolution kernel, this takes a slightly tidier form

\begin{subequations}
\begin{align}
\Psi(\Bx, t) &=
\int \hat{U}(\Bx, \Bx', t) \Psi(\Bx', 0) d^3 \Bx' \\
\hat{U}(\Bx, \Bx', t) &=
\inv{(2 \pi)^3}
\int
\exp\left(
\inv{i \Hbar} \left( \inv{2m} \left( \Hbar \Bk - \frac{e}{c} \BA(\Bx) \right)^2 + e V(\Bx) \right) t
+ i \Bk \cdot (\Bx - \Bx')
\right)
d^3 \Bk.
\end{align}
\end{subequations}

\subsection{Verification attempt}

If we apply the Pauli equation \eqnref{eqn:pauliFourier:1} to \eqnref{eqn:pauliFourier:10} does it produce the correct answer?

For the LHS we have

\begin{equation}\label{eqn:pauliFourier:30}
i \Hbar \PD{t}{\Psi}
=
\int \left(
\inv{2m} \left( \Hbar \Bk - \frac{e}{c} \BA(\Bx) \right)^2 + e V(\Bx)
\right) \hat{U}(\Bx, \Bx', t) \Psi(\Bx', 0) d^3 \Bx',
\end{equation}

but for the RHS we have

\begin{equation}\label{eqn:pauliFourier:31}
\begin{aligned}
&\left( \inv{2m} \gpgradezero{(\Bp - \frac{e}{c}\BA)^2} + e \phi \right) \Psi
=
\int
d^3 \Bx'
\hat{U}(\Bx, \Bx', t) \Psi(\Bx', 0)  \\
&\qquad\left(
\inv{2m} \left( \Hbar \Bk - \frac{e}{c} \BA(\Bx) \right)^2 + e V(\Bx)
+\frac{t}{2m i \Hbar}
\left( - \Hbar^2 \spacegrad^2 + 2 \frac{i e \Hbar }{c} \BA \cdot \spacegrad \right)
\left(
\inv{2m} \left( \Hbar \Bk - \frac{e}{c} \BA(\Bx) \right)^2 + e V(\Bx)
\right)
\right)
\end{aligned}
\end{equation}

So if it were not for the spatial dependence of \(\BA\) and \(\phi\), we would have LHS equal to the RHS.  It appears that ignoring the odd \(\Bx\) dependence in the \(\hat{\Psi}\) differential equation definitely leads to trouble, and only works for constant potential distributions, a rather boring special case.

\EndArticle
