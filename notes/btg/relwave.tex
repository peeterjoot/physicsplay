\documentclass[]{eliblog}

\title{Schr\"{o}dinger Equation as a Relativistic wave equation approximation.}
\author{Peeter Joot}
\date{June 18, 2009}
\blogpage{http://behindtheguesses.blogspot.com/2009/06/blah.html}
\email{peeter.joot@gmail.com}

\newcommand{\Abs}[1]{{\left\lvert{#1}\right\rvert}}
\newcommand{\BB}[0]{\mathbf{B}}
\newcommand{\BE}[0]{\mathbf{E}}
\newcommand{\BS}[0]{\mathbf{S}}
\newcommand{\Bk}[0]{\mathbf{k}}
\newcommand{\Bp}[0]{\mathbf{p}}
\newcommand{\Bx}[0]{\mathbf{x}}
\newcommand{\LL}[0]{\mathcal{L}}
\newcommand{\cross}[0]{\times}
\newcommand{\spacegrad}[0]{\boldsymbol{\nabla}}
\newcommand{\delambertian}[0]{\square}

\begin{document}
\maketitle

\section{Alternative approaches to motivate Schr\"{o}dinger's equation.}

It does not take fancy mathematics to motivate the Schr\"{o}dinger's equation.  In
\cite{french1998iqp} the specialization of Schr\"{o}dinger's equation for a 
particle in a one dimensional potential

\begin{align}\label{eqn:oneDimSch}
-\frac{\hbar^2}{2m} \frac{\partial^2 \psi}{\partial x^2} + V\psi = i \hbar \frac{\partial \psi}{\partial t}
\end{align}

is presented from start to finish in only seven pages.  However this does include a disclaimer upfront that
``we ... simply try to make the form of the equation plausible''.  The plausibility arguments
found in this text are not uncommon, and can be found for example (without the surrounding discussion of the text) 
in the ``Short heuristic derivation'' of the Wikipedia's Schr\"{o}dinger's equation article (\cite{wikiSchH}).

Calling these derivations is justifably debatable and is perhaps the origin of statements like
``Schr\"{o}dinger's equation cannot be derived'' (\cite{hyperphysicsSch})
Considerable care 
is required to construct logically consistent arguments that motivate the quantum wave
equation in a fashion that is not simply playing the equation in reverse.  Bohm's Quantum Theory
(\cite{bohm1989qt}) contains such a carefully crafted treatment.  The cost 
of this presentation
is that it takes nine chapters and two hundred pages to get to
the starting point of many other introductory Quantum texts.

Perhaps agreeing with a it cannot be derived opinion, the text of Liboff (\cite{liboff2003iqm}) appears to take a 
``let's calculate approach''.  There is little attempt to motivate the equation.  Instead one is left to learn
its characteristics by using it.

The form of Schr\"{o}dinger's equation given in (\ref{eqn:oneDimSch}) or its three dimensional generalization is
explicitly non-relativisitic.  Time and space are not treated uniformly.  The order of the derivatives are not
even the same!

Given this apparent non-relativistic nature it is perhaps somewhat suprising that Pauli's Wave Mechanics (\cite{pauli2000wm}),
starts not with the Bohr model, nor black bodies, nor or any of the usual starting points for a Quantum text,
but instead with a lighting fast review of special relativity.
He then outlines the procedure for a small momentum approximation of a relativistic wave equation (i.e. the wave equation for light).
This utilizes the DeBroglie hypothesis that an energy momentum four vector for matter may also be quantized.  
It is assumed here that
the solution of the matter wave equation will take the same functional form as the solution for the classical wave equations
for light.

Comments in
the wikipedia Klien Gordon article (\cite{wikiKG}), suggest that Pauli's approach is probably closer historically to
the reasoning used initially
by Schr\"{o}dinger.  It is also interesting to note that the thesis of DeBroglie (\cite{AFkracklauerDeBroglie}) which
originally hypothosized extension of the notion of quantized freqencies from light to particles 
also heavily relied on special relativity arguments.

The following discussion follows the brief treatment in Pauli text, and attempts to fill in some of the in between steps.
Required to fill in these gaps are familiarity with Maxwell's equations in four vector form,
Fourier transforms and the Lorentz invariance of an energy-momentum four vector.

\section{Solution of the Vacuum Maxwell equation}
%
%Maxwell's equation 
%
%One does not have to go far to find 

\section{first try.}
Continuing the thread of Schr\"{o}dinger equation motivation, a peek at the relativistic origins for the quantum wave equation will be made.

The work of DeBroglie (\cite{AFkracklauerDeBroglie}) used by Schr\"{o}dinger to construct the quantum wave equation as we now know it relied heavily on special relativity arguments.  
Schr\"{o}dinger's initial attempt at the wave equation was in fact relativistic, but he ended up publishing a non-relativistic 
approximation.
According to the wikipedia Klien Gordon page (\cite{wikiKG}), 



%%In particular, one can derive the potential free Schr\"{o}dinger equation by employing a small momentum approximation to a solution of the relativistic wave equation.
%
% blah
%.  The following notes follow Pauli's treatment closely.
%Schr\"{o}dinger equation as an relativistic approximation in his Wave Mechanics text 
%for the quantum wave equation will be made.
%motivation, a peek at the relativistic 
%Here the relativistic origins for the quantum wave equation how do we end up with something so 
%
%It is interesting to note that Pauli, a master of relativity, having
%written the definitive encyclopedic article
%\cite{pauli1981tr}
%on special and general relativity
%as a teenager, introduces the
%Schr\"{o}dinger equation as an relativistic approximation in his Wave Mechanics text \cite{pauli2000wm}.  The following notes follow Pauli's treatment closely.
%%however
%%Pauli's preference for using
%%$h$ instead of $\hbar$ is avoided.
%%.  The second is an attempt to
%%avoid some of the terseness of Pauli's presentation.

\subsection{Invariant length for an energy momentum four vector}

The Lorentz invariant length of an energy momentum four vector

\begin{align}
P = (E/c, \Bp)
\end{align}

is given by

\begin{align}\label{eqn:Psquared}
P^2 \equiv \frac{E^2}{c^2} - \Abs{\Bp}^2 = \frac{(m c^2)^2}{c^2}
\end{align}

This invariance has meaning for particles, such as electrons, where $m \ne 0$, but also for photons where $m = 0$.
Einstein's explanation of the photoelectric effect introduced the concept that light comes in
quantized packets of energy (photons), with photon energy proportional to frequency

\begin{align}
E = h \nu = \hbar \omega
\end{align}

The relativistic invariance relation (\ref{eqn:Psquared}) implies that there is a corresponding photon momentum, also
quantized in magnitude, and this is typically written

\begin{align}
\Bp = \hbar \Bk
\end{align}

The magnitudes of the vector $\Bk$ and the frequency $\omega$ are not independent, since by (\ref{eqn:Psquared}) we have

\begin{align}\label{eqn:omegaKsquared}
c^2 \Abs{\Bk}^2 = \omega^2
\end{align}

%It is also worth noting that we have the concepts of electromagnetic energy and momentum in the continuous theory of
%electromagnetic radiation (i.e.: Maxwell's equations).  For an electromagnetic wave in vacuum the
%field energy density and momentum density also form a four vector
%
%\begin{align}
%%P = \epsilon_0 \left(\frac{\Abs{\BE}^2}{2} + \frac{c^2 \Abs{\BB}^2}{2}, c \BE \cross \BB \right)
%\end{align}
%
%Because $\Abs{\BE}^2 = c^2 \Abs{\BB}^2$, and $\BE \cdot \BB = 0$ for an electromagnetic wave one can show with some vector manipulation that in this continuous representation of light we also have $P^2 = 0$.
%
%\cite{bohm1989qt}
%\cite{feynman}
\subsection{Wave equation solutions to Maxwell's equations}

The coupling between wave number and angular frequency can 
%\begin{align}\label{eqn:omegaKsquared}
The 
In classical electrodynamics there are the field energy and momentum densities can also be are also coupled as in 
The 
%\begin{align}\label{eqn:omegaKsquared}

The equation for a wave $\psi(t,\Bx)$ propagating at the speed of light is

\begin{align}
\delambertian \psi = \left( \frac{1}{c^2}\frac{\partial^2}{{\partial t}^2} - \spacegrad^2 \right) \psi = 0
\end{align}

This is in fact the form that Maxwell's equations for both the electric and magnetic field take in the absence of charge or currents.
For example we have for the electric field

\begin{align}\label{eqn:eField}
\delambertian \BE = \left( \frac{1}{c^2}\frac{\partial^2}{{\partial t}^2} - \spacegrad^2 \right) \BE = 0
\end{align}

If we form the vector valued function

\begin{align}\label{eqn:testSolutionE}
\BE = \BE_0 e^{ i( \Bk \cdot \Bx - \omega t ) }
\end{align}

then application of the wave equation operator produces

\begin{align}\label{eqn:boxApplied}
\delambertian \BE = -\left( \frac{\omega^2}{c^2} - \Abs{\Bk}^2 \right) \BE
\end{align}

Therefore we have a solution of (\ref{eqn:eField}) provided

\begin{align}
\omega^2 = c^2 \Abs{\Bk}^2
\end{align}

This is exactly the relativistic invariance condition (\ref{eqn:omegaKsquared}) for a photon of energy $\hbar \omega$.

\subsection{Matter Wave equation}

Suppose that one has a function

\begin{align}\label{eqn:waveFunction}
\psi = e^{ i( \Bk \cdot \Bx - \omega t) }
\end{align}

that in some way represents a matter wave for a particle with energy $E = \hbar \omega$, and momentum
 $\Bp = \hbar \Bk$.  The multiplicative factor has been omitted for clarity.
From (\ref{eqn:boxApplied}), and
(\ref{eqn:Psquared})
we see that such a function will satisfy the relativistic wave equation

\begin{align}
\delambertian \psi = \left( \frac{1}{c^2}\frac{\partial^2}{{\partial t}^2} - \spacegrad^2 \right) \psi = - \frac{m^2 c^2}{\hbar^2} \psi
\end{align}

We have now reached the point that a non relativistic approximation can be made.
Again from (\ref{eqn:Psquared}) we have

\begin{align}
\omega = \frac{m c^2}{\hbar} \sqrt{ 1 + \frac{\hbar^2 \Abs{\Bk}^2}{m^2 c^2}}
\end{align}

If $(\hbar^2 \Abs{\Bk}^2)/(m^2 c^2)$ is small enough, then a first order Taylor gives

\begin{align}
\omega \approx \frac{m c^2}{\hbar} + \frac{\hbar \Abs{\Bk}^2}{2 m}
\end{align}

This allows for a factorization of the wave function (\ref{eqn:waveFunction})

\begin{align}\label{eqn:intermed}
\psi =
\exp\left( i \frac{m c^2}{\hbar} t \right)
\exp\left( i \left(
\Bk \cdot \Bx
-\frac{\hbar \Abs{\Bk}^2}{2 m}t
\right) \right)
\end{align}

The infinite plane wave solution of (\ref{eqn:waveFunction}) was only one possible solution to the original wave equation.
A general solution could be any superposition of such solutions, such as a weighted integral over wave numbers (i.e.: an
inverse Fourier transform from a wave number representation).  For this reason it is not unreasonable to drop the specific plane wave nature that generated (\ref{eqn:intermed}) instead writing $\psi(t,\Bx)$ in terms of an unknown function $\psi'$ as follows

\begin{align}
\psi = \exp\left( i \frac{m c^2}{\hbar} t \right) \psi'(t, \Bx)
\end{align}

We are now set to apply the relativistic wave equation operator to the wave function that we have now first order Taylor expanded
around the rest frequency $m c^2/\hbar$

\begin{align}
\delambertian \psi
=
\left(
\frac{1}{c^2}
\left(
\left( \frac{i m c^2}{\hbar} \right)^2 \psi' - 2 i \frac{m c^2}{\hbar} \frac{\partial \psi'}{\partial t} +
\frac{\partial^2 \psi'}{\partial t^2}
\right)
%\exp\left( i \frac{m c^2}{\hbar} t \right)
-
\spacegrad^2 \psi'
\right)
\exp\left( i \frac{m c^2}{\hbar} t \right)
= -
\exp\left( i \frac{m c^2}{\hbar} t \right)
\frac{m^2 c^2}{\hbar^2} \psi'
\end{align}

Rearranging we have
%\begin{align}
%\frac{1}{c^2}
%\left(
%-2 i \frac{m c^2}{\hbar} \frac{\partial \psi'}{\partial t} +
%\frac{\partial^2 \psi'}{\partial t^2}
%\right)
%=
%\spacegrad^2 \psi'
%\end{align}

\begin{align}
-\frac{\hbar^2}{2m} \spacegrad^2 \psi' = i \hbar \frac{\partial \psi'}{\partial t} -\frac{\hbar^2}{2m c^2} \frac{\partial^2 \psi'}{\partial t^2}
\end{align}

Now for the mass of an electron $\hbar^2/(m_e c^2) \approx 10^{-55} m^2 kg$, which is pretty small seeming.

Assuming this last term can be dropped we have the potential free Schr\"{o}dinger's equation.

\begin{align}
-\frac{\hbar^2}{2m} \spacegrad^2 \psi' = i \hbar \frac{\partial \psi'}{\partial t}
\end{align}

Observe that
this isn't the only wave equation we could construct in this fashion.  For example, we could also have started with $\psi \propto e^{i(\omega t - \Bx \cdot \Bk)}$.  Provided $\omega^2 = c^2 \Abs{\Bk}^2$ as before, this is still a solution to the wave equation, and this procedure would produce a wave equation with a negated sign on the time derivative.

\bibliography{blog}
\bibliographystyle{unsrturl}

\end{document}
