
F1

\begin{dmath}\label{eqn:uwavesDeck8ResonatorQfactorCore:20}
P_{\textrm{loss} = \inv{2} \Abs{I}^2 R
\end{dmath}

Average stored magnetic energy 

\begin{dmath}\label{eqn:uwavesDeck8ResonatorQfactorCore:40}
W_m = \inv{4} \Abs{I}^2 L
\end{dmath}

Average stored electric energy 

\begin{dmath}\label{eqn:uwavesDeck8ResonatorQfactorCore:60}
W_e 
= \inv{4} \Abs{V_c}^2 C
= \inv{4} \Abs{I}^2 \inv{\omega^2 C}
\end{dmath}

At resonance \( W_m = W_e \) or

\begin{dmath}\label{eqn:uwavesDeck8ResonatorQfactorCore:80}
L = \inv{\omega_0^2 C}
\end{dmath}

and 

\begin{dmath}\label{eqn:uwavesDeck8ResonatorQfactorCore:100}
\omega_0 = \inv{\sqrt{L C}}
\end{dmath}

\begin{dmath}\label{eqn:uwavesDeck8ResonatorQfactorCore:120}
Q = \omega_0 \frac{\textrm{average stored energy}}{\textrm{energy loss/second}}
= \omega_0 \frac{W_e + W_m}{P_\loss}
\end{dmath}

At resonance,  \( W_m = W_e = \inv{4} \Abs{I}^2 L\), so

\begin{dmath}\label{eqn:uwavesDeck8ResonatorQfactorCore:140}
Q 
= \omega_0 \frac{\inv{2} \Abs{I}^2 L}{\inv{2} \Abs{I}^2 R} 
= \omega_0 \frac{L}{R}
= \frac{X}{R}
\end{dmath}

High Q, means that we have low losses, but is also a measure of the bandwidth.

\paragraph{Bandwith of resonators and Q-factor}

To see why this measures the bandwidth, lets find the impedance

\begin{dmath}\label{eqn:uwavesDeck8ResonatorQfactorCore:160}
Z_{\textrm{in}} 
= R + j \omega L - \frac{j}{\omega C}
= R + j \omega L \lr{ 1 - \inv{\omega^2 L C} }
= R + j \omega L \frac{\omega^2 - \omega_0^2}{\omega^2}, 
\end{dmath}

where the resonant frequency is

\begin{dmath}\label{eqn:uwavesDeck8ResonatorQfactorCore:180}
\omega_0 = \inv{\sqrt{LC}}
\end{dmath}

Also, 

\begin{dmath}\label{eqn:uwavesDeck8ResonatorQfactorCore:200}
\omega^2 - \omega_0^2 = \lr{ \omega - \omega_0 }\lr{ \omega + \omega_0 }
= 
...
\end{dmath}

F2

The half-power bandwidth is defined when \( \Abs{Z_{\textrm{in}}} = \sqrt{2} R \).  In this case the fractional bandwidth 
\begin{dmath}\label{eqn:uwavesDeck8ResonatorQfactorCore:220}
BW = \frac{2 \Delta \omega}{\omega_0}
\end{dmath}

\begin{dmath}\label{eqn:uwavesDeck8ResonatorQfactorCore:240}
Z_{\textrm{in}} = R + j R Q BW, 
\end{dmath}

gives

\begin{dmath}\label{eqn:uwavesDeck8ResonatorQfactorCore:260}
\Abs{Z_{\textrm{in}}}^2 = R^2\lr{ 1 + Q^2 BW^2 } = 2 R^2,
\end{dmath}

or

\begin{dmath}\label{eqn:uwavesDeck8ResonatorQfactorCore:280}
Q^2 BW^2 = 1,
\end{dmath}

or

\begin{dmath}\label{eqn:uwavesDeck8ResonatorQfactorCore:300}
BW = \inv{Q}
\end{dmath}

...

F3

If you want a resonator (such as an antenna) you want a high-Q, but for a filter, low-Q.

For microstrip circuits the Q's are small, perhaps in the 10-500 range.  For example a ring on a substrate

F4

you may have a high-Q ( \( Q \sim 500 \) ).

whereas printing on a CMOS grade silicon substrate, you may get a \(Q \sim 10\).

F5

In optics you may find really high Q's (say 10000).  The reason is that \( X \) value is huge because the structure can be made many wavelengths in size.

\paragraph{Q-Factor for a capacitor}

series case:

F6

\begin{dmath}\label{eqn:uwavesDeck8ResonatorQfactorCore:320}
V_c = \frac{I}{j \omega C}
\end{dmath}

...

Parallel (shunt) case:

\end{dmath}
Q 
= \omega \frac{ W_e + W_m }{P_L} 
= \omega \frac{\textrm{...}}{\textrm{...}}
\begin{dmath}\label{eqn:uwavesDeck8ResonatorQfactorCore:340}

F7

\begin{dmath}\label{eqn:uwavesDeck8ResonatorQfactorCore:360}
Q = \frac{B_c}{G} = \omega R C
\end{dmath}

Large resistance (low loss), means high Q.

\paragraph{Q-factor for an inductor}

...



