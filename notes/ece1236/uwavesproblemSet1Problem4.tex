%
% Copyright � 2016 Peeter Joot.  All Rights Reserved.
% Licenced as described in the file LICENSE under the root directory of this GIT repository.
%
\makeproblem{Parallel-plate transmission line.}{uwaves:problemSet1:4}{ 

Consider a parallel-plate transmission line of height \( d \) and width \( w \) (\(w \gg d\)). The plates
are filled with a lossless dielectric \( \epsilon \).

\makesubproblem{}{uwaves:problemSet1:4a}
Calculate the capacitance per unit length \( C' \).

\makesubproblem{}{uwaves:problemSet1:4b}
Calculate the inductance per unit length \( L' \).

\makesubproblem{}{uwaves:problemSet1:4c}
Determine an expression for the characteristic impedance \( Z_0 \).

\makesubproblem{}{uwaves:problemSet1:4d}
Calculate the resistance per unit length \( R' \). Assume that the conductivity of the plates is \( \sigma \) 
and the skin depth is \( \delta_s \).

\makesubproblem{}{uwaves:problemSet1:4e}
Calculate an expression for the attenuation constant \( \alpha \).

Now consider that \( d/w = 0.1 \) and \( \epsilon_r = 1 \).
The plates have a conductivity of \( \sigma = 3.538 \times 10^7 \textrm{siemens}/m \).
and the frequency of operation is \( f = 30 \si{GHz} \).

\makesubproblem{}{uwaves:problemSet1:4f}
Calculate the characteristic impedance \( Z_0 \).

\makesubproblem{}{uwaves:problemSet1:4g}
Calculate the attenuation constant \( \alpha \) in terms of \( d \).

\makesubproblem{}{uwaves:problemSet1:4h}
Calculate the attenuation constant \( \alpha \) in \( \si{dB/m} \) when \( d = 1 \si{cm}, 1 \si{mm} \) and \( 1 \mu \si{m} \).

} % makeproblem

\makeanswer{uwaves:problemSet1:4}{ 
\makeSubAnswer{}{uwaves:problemSet1:4a}

To get some quick results, the transmission line can be treated as a large parallel plate capacitor.  First recall that for a single plate with linear charge density \( +(Q/l) \) on it, the electric field is normal and outwards from the plate as in sketched in

F1

Using a Gaussian volume the width of the plate we have 

\begin{dmath}\label{eqn:uwavesproblemSet1Problem4:20}
\int \BD \cdot \ncap dA
= 2 \epsilon E w l 
= Q,
\end{dmath}

so the magnitude of the field for one plate is

\begin{dmath}\label{eqn:uwavesproblemSet1Problem4:40}
E = \frac{Q}{2 \epsilon w l}
\end{dmath}

Introducing a second plate with equal but opposite charge density on it, we have cancelation of the electric field outside of the plates, but a doubling within.  The magnitude of the total electric field between the plates is therefore

\begin{dmath}\label{eqn:uwavesproblemSet1Problem4:60}
\BE = -\frac{Q}{\epsilon w l} \ycap.
\end{dmath}

The voltage difference between the plates is

\begin{dmath}\label{eqn:uwavesproblemSet1Problem4:80}
V 
= \int_0^d E dl
= frac{Q d}{\epsilon w l} .
\end{dmath}

The capacitance per unit length is

\begin{dmath}\label{eqn:uwavesproblemSet1Problem4:100}
C' 
= \frac{Q/l}{V} 
= \frac{(Q/l) (\epsilon w l)}{Q d}
\end{dmath}

\boxedEquation{eqn:uwavesproblemSet1Problem4:120}{
C' = \frac{\epsilon w )}{d}.
}

Here edge effects and charge distribution on the plates has been completely ignored.  This is also only one of the possible field modes in the cavity (TEM mode).  This mode and the others are covered in greater detail in \citep{pozar2009microwave} ch. 3.

\makeSubAnswer{}{uwaves:problemSet1:4b}

For the magnetic field the situation is similar.  Suppose the magnetic field is oriented as in 

F2

Integrating over the loop \( C \) that surrounds the top plate we have

\begin{dmath}\label{eqn:uwavesproblemSet1Problem4:140}
\oint \BD \cdot d\Bl 
= 
2 H w = Ifrac{Q}{\epsilon w l} = I,
\end{dmath}

or
\begin{dmath}\label{eqn:uwavesproblemSet1Problem4:160}
H = \frac{I}{2w}.
\end{dmath}

Like the electric field, an opposite charge on the other plate leads to a doubled magnetic field in the cavity, and no magnetic field outside the plates, so the total magnetic field magnitude in the cavity is

\begin{dmath}\label{eqn:uwavesproblemSet1Problem4:180}
H = \frac{I}{w}.
\end{dmath}

Calculating the flux through the side we have

\begin{dmath}\label{eqn:uwavesproblemSet1Problem4:200}
\Phi 
= \int \BB \cdot d\BA 
= \mu H l d
= \frac{\mu I l d}{w}
\end{dmath}

The inductance per unit length is 
\begin{dmath}\label{eqn:uwavesproblemSet1Problem4:220}
L'
= \frac{\Phi/l}{I},
\end{dmath}

or

\boxedEquation{eqn:uwavesproblemSet1Problem4:240}{
L'
= \frac{\mu d}{w}.
}
 
\makeSubAnswer{}{uwaves:problemSet1:4c}

The impedance 

\begin{dmath}\label{eqn:uwavesproblemSet1Problem4:260}
Z_0
= \sqrt{\frac{L}{C}}
= \sqrt{ \frac{\mu d}{w}
\frac{d}{\epsilon w }
}
= \frac{d}{w} \sqrt{ \frac{\mu}{\epsilon} },
\end{dmath}

or
\begin{dmath}\label{eqn:uwavesproblemSet1Problem4:280}
Z_0
= \frac{d}{w} \eta.
\end{dmath}

\makeSubAnswer{}{uwaves:problemSet1:4d}
\makeSubAnswer{}{uwaves:problemSet1:4e}
\makeSubAnswer{}{uwaves:problemSet1:4f}
\makeSubAnswer{}{uwaves:problemSet1:4g}
\makeSubAnswer{}{uwaves:problemSet1:4h}

%TODO.
}
