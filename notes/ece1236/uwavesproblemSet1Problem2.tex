%
% Copyright � 2016 Peeter Joot.  All Rights Reserved.
% Licenced as described in the file LICENSE under the root directory of this GIT repository.
%
\makeproblem{Transmission line with non-magnetic dielectric.}{uwaves:problemSet1:2}{ 
%\makesubproblem{}{uwaves:problemSet1:2a}
A transmission line is filled with a non-magnetic dielectric of \( \epsilon_r = 2.5\). The line has a capacitance per unit length of \( C = 200 \,\si{pF/m}\) and a resistance per unit length of \( R = 2 \Omega/\si{m} \).
Calculate the corresponding phase velocity, characteristic impedance and attenuation constant \( \alpha \) (assume \(G = 0\)).
} % makeproblem

\makeanswer{uwaves:problemSet1:2}{ 

%\makeSubAnswer{}{uwaves:problemSet1:2a}

The impedance of the line is

\begin{dmath}\label{eqn:uwavesproblemSet1Problem2:20}
Z_0 = \sqrt{\frac{ R + j \omega L }{ G + j \omega C }},
\end{dmath}

we know \( R, C, G \), but not \( L \).  We'd found that

\begin{dmath}\label{eqn:uwavesproblemSet1Problem2:40}
v_\phi = \inv{\sqrt{L C}},
\end{dmath}

for lossless, low-loss, and distortionless lines.  If we assume that is the case here too, and 

\begin{dmath}\label{eqn:uwavesproblemSet1Problem2:60}
v_\phi 
= \frac{c}{\sqrt{\epsilon_r}}
= \frac{3 \times 10^9}{\sqrt{2.5}}
= 1.9 \times 10^9 \,\si{m/s},
\end{dmath}

we have

\begin{dmath}\label{eqn:uwavesproblemSet1Problem2:80}
L 
= \inv{v_\phi^2 C}
= \frac{1}{(1.9 \times 10^9)^2 (200 \times 10^{-12})} \,\si{H/m}
= 1.4 \,\si{ n H/m }.
\end{dmath}

The characteristic impedance is
\begin{dmath}\label{eqn:uwavesproblemSet1Problem2:100}
Z_0 
= \sqrt{\frac{ R + j \omega L }{ G + j \omega C }}
= \sqrt{\frac{ 2 + j \omega 1.4 \times 10^{-9} }{ j \omega 200 \times 10^{-12} }} \Omega.
\end{dmath}

This is a rather ugly looking result, but it turns out that this system is close to lossless at \si{GHz} frequencies.  Numerically, the impedance is very close to real, meaning that the resistance isn't high enough to make much difference.  At frequencies starting at 1 \,\si{GHz}, we have

\begin{equation}\label{eqn:uwavesproblemSet1Problem2:140}
\begin{aligned}
Z_0(1 \,\si{GHz}) &= 2.669 \phase{ \ang{-6.454} } \\
Z_0(2 \,\si{GHz}) &= 2.644 \phase{ \ang{-3.269} } \\
Z_0(3 \,\si{GHz}) &= 2.639 \phase{ \ang{-2.184} } \\
Z_0(10 \,\si{GHz}) &= 2.636 \phase{ \ang{-0.6564} } \\
\end{aligned}
\end{equation}

whereas with \( R = 0 \) we have almost the same value as the 10 \si{GHz} impedance

\begin{equation}\label{eqn:uwavesproblemSet1Problem2:160}
Z_0 = 2.635.
\end{equation}

For the attenuation factor first note that 
\begin{dmath}\label{eqn:uwavesproblemSet1Problem2:120}
\gamma 
= \lr{ \lr{ 2 + j \omega 1.4 \times 10^{-9} } \lr{ j \omega 200 \times 10^{-12} } }^{1/2}
= \lr{ \lr{ 2 + j 2 \pi f 1.4 } \lr{ j 0.4 \pi f } }^{1/2}
= f \sqrt{ 0.4 \pi } \lr{ (2/f) j - 2.8 \pi }^{1/2},
\end{dmath}

where \( f \) is in \si{GHz}.  We see that for \( f \in [1, 10] \), this \( \gamma \) is close to purely imaginary, as is also the case for \( R = 0 \).  Specifically

\begin{equation}\label{eqn:uwavesproblemSet1Problem2:180}
\begin{aligned}
\gamma(1 \,\si{GHz})/f &= 3.354 \phase{ \ang{83.55} } \\
\gamma(2 \,\si{GHz})/f &= 3.322 \phase{ \ang{86.73} } \\
\gamma(3 \,\si{GHz})/f &= 3.316 \phase{ \ang{87.82} } \\
\gamma(10 \,\si{GHz})/f &= 3.312 \phase{ \ang{89.34} } \\
\end{aligned}
\end{equation}

Whereas for \( R = 0 \) we have

\begin{equation}\label{eqn:uwavesproblemSet1Problem2:200}
\gamma/f = 3.312 j.
\end{equation}

It appears that \( \alpha \sim 0 \) for this system in the \si{GHz} range.  For the same frequencies, some specific values are

\begin{equation}\label{eqn:uwavesproblemSet1Problem2:220}
\begin{aligned}
\alpha(1 \,\si{GHz}) &= 0.377 \\
\alpha(2 \,\si{GHz}) &= 0.3789 \\
\alpha(3 \,\si{GHz}) &= 0.3792 \\
\alpha(10 \,\si{GHz}) &= 0.3794.
\end{aligned}
\end{equation}

Note however, that for lower frequencies, where the wavelength is larger, this is actually a fairly significant attenuation, not very zeroish.  For the same respective frequencies we have

\begin{equation}\label{eqn:uwavesproblemSet1Problem2:240}
\begin{aligned}
e^{-\lambda \alpha} &= 0.489 \\
e^{-\lambda \alpha} &= 0.6981 \\
e^{-\lambda \alpha} &= 0.7868 \\
e^{-\lambda \alpha} &= 0.9305.
\end{aligned}
\end{equation}

The numerical calculations made for this problem can be found in \nbref{ps1:ps1_2.jl}. 
}
