%
% Copyright � 2015 Peeter Joot.  All Rights Reserved.
% Licenced as described in the file LICENSE under the root directory of this GIT repository.
%

% 
%\chapter{Preface}
% this suppresses an explicit chapter number for the preface.
\chapter*{Preface}%\normalsize
  \addcontentsline{toc}{chapter}{Preface}

This document was produced while taking the Spring 2016, University of Toronto Microwave Circuits course (ECE1236H), taught by Prof.\ G. V. Eleftheriades.

\paragraph{Course Syllabus}

This course outlines the principles of designing modern microwave and RF circuits.  Signal-integrity issues in high-speed digital circuits are also examined.

\begin{itemize}
\item The wave equation.
\item Ideal transmission lines.
\item Transients on transmission-lines.
\item Planar transmission lines and introduction to MMIC's.
\item Designing with scattering parameters.
\item Planar power dividers.
\item Directional couplers.
\item Microwave filters.
\item Solid-state microwave amplifiers.
\item Noise.
\item Diode-mixers.
\item RF receiver chains.
\item Oscillators.
\end{itemize}

\withproblemsetsMessage{
\textcolor{Maroon}{
\textit{THIS DOCUMENT IS REDACTED.  THE PROBLEM SET SOLUTIONS AND ASSOCIATED MATHEMATICA CODE IS NOT VISIBLE.  PLEASE EMAIL ME FOR THE FULL VERSION IF YOU ARE NOT TAKING ECE1236.}
}
}

\paragraph{This document contains:}

\begin{itemize}
\item Lecture notes.
\item Personal notes exploring auxiliary details.
\item Worked practice problems.

\ifthenelse{\boolean{redacted}}%
{%
\item Links to Mathematica notebooks associated with the course material and problems (but not problem sets).
}%
{
\item Assigned problems.%
\item Links to Mathematica notebooks associated with problems and course material.%
}
\end{itemize}

%This set of notes is significantly different from my notes for many other classes.  With the class taught on slides (and some of those slides mirroring the text closely), I did not take live notes in class.
%These notes fill in details that I felt deserved clarification, contain problem sets solutions, as well as a number of loosely related musings on Geometric Algebra equivalents to some of the generalized concepts of electromagnetic theory encountered in this class (i.e. magnetic sources).
%
My thanks go to Professor Eleftheriades for teaching this course.

Peeter Joot  \quad peeterjoot@protonmail.com 
