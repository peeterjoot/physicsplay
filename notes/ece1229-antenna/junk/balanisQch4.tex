%
% Copyright � 2015 Peeter Joot.  All Rights Reserved.
% Licenced as described in the file LICENSE under the root directory of this GIT repository.
%
\newcommand{\authorname}{Peeter Joot}
\newcommand{\email}{peeterjoot@protonmail.com}
\newcommand{\basename}{FIXMEbasenameUndefined}
\newcommand{\dirname}{notes/FIXMEdirnameUndefined/}

\renewcommand{\basename}{reciprocityTheorem}
\renewcommand{\dirname}{notes/ece1229/}
%\newcommand{\dateintitle}{}
%\newcommand{\keywords}{}

\newcommand{\authorname}{Peeter Joot}
\newcommand{\onlineurl}{http://sites.google.com/site/peeterjoot2/math2013/\basename.pdf}
\newcommand{\sourcepath}{\dirname\basename.tex}
\newcommand{\generatetitle}[1]{\chapter{#1}}

\newcommand{\vcsinfo}{%
\section*{}
\noindent{\color{DarkOliveGreen}{\rule{\linewidth}{0.1mm}}}
\paragraph{Document version}
%\paragraph{\color{Maroon}{Document version}}
{
\small
\begin{itemize}
\item Available online at:\\ 
\href{\onlineurl}{\onlineurl}
\item Git Repository: \input{./.revinfo/gitRepo.tex}
\item Source: \sourcepath
\item last commit: \input{./.revinfo/gitCommitString.tex}
\item commit date: \input{./.revinfo/gitCommitDate.tex}
\end{itemize}
}
}

%\PassOptionsToPackage{dvipsnames,svgnames}{xcolor}
\PassOptionsToPackage{square,numbers}{natbib}
\documentclass{scrreprt}

\usepackage[left=2cm,right=2cm]{geometry}
\usepackage[svgnames]{xcolor}
\usepackage{peeters_layout}

\usepackage{natbib}

\usepackage[
colorlinks=true,
bookmarks=false,
pdfauthor={\authorname, \email},
backref 
]{hyperref}

% http://tex.stackexchange.com/questions/75773/how-to-reference-problems-by-the-text-label-in-an-exercise-envioronment
\usepackage[english]{cleveref}
\crefname{Exercise}{exercise}{exercises}
\Crefname{Exercise}{Exercise}{Exercises}

\RequirePackage{titlesec}
\RequirePackage{ifthen}

% http://stackoverflow.com/questions/4932910/date-in-the-tabular-environment
\makeatletter
\let\insertdate\@date
\makeatother

\titleformat{\chapter}[display]
{\bfseries\Large}
{\color{DarkSlateGrey}\filleft \authorname
\ifthenelse{\isundefined{\studentnumber}}{}{\\ \studentnumber}
\ifthenelse{\isundefined{\email}}{}{\\ \email}
\ifthenelse{\isundefined{\dateintitle}}{}{\\ \insertdate}
%\ifthenelse{\isundefined{\coursename}}{}{\\ \coursename} % put in title instead.
}
{4ex}
{\color{DarkOliveGreen}{\titlerule}\color{Maroon}
\vspace{2ex}%
\filright}
[\vspace{2ex}%
\color{DarkOliveGreen}\titlerule
]

\newcommand{\beginArtWithToc}[0]{\begin{document}\tableofcontents}
\newcommand{\beginArtNoToc}[0]{\begin{document}}
\newcommand{\EndNoBibArticle}[0]{\end{document}}
\newcommand{\EndArticle}[0]{\bibliography{Bibliography}\bibliographystyle{plainnat}\end{document}}

% 
%\newcommand{\citep}[1]{\cite{#1}}

\colorSectionsForArticle


\usepackage{peeters_layout_exercise}
\usepackage{macros_bm}

\beginArtNoToc

\generatetitle{Far field electric field for two horizontal dipole configurations}

\section{Horizontal dipole reflection coefficient}
\index{horizontal dipole}

Suppose an infinitesimal horizontal dipole is in a ``coming out of the page'' configuration.  With the page representing the z-y plane, this is a magnetic vector potential directed along the x-axis direction, as follows

\begin{equation}\label{eqn:t:640}
\BA = \xcap \frac{\mu_0 I_0 l}{4 \pi r} e^{-j k r}.
%= \frac{A_r}{4 \pi r} e^{-j k r}
\end{equation}

We've seen in class (UofT ece1229, taught by Prof. Eleftheriades) that the far-field electric field given by

\begin{dmath}\label{eqn:t:n}
\BE = j \omega \Proj_\T \BA,
\end{dmath}

where \( \Proj_\T \BA \) represents the transverse projection of \( \BA \).  So, for a wave vector directed in the z-y plane, \( \kcap = \zcap \cos\theta + \ycap \sin\theta \), the electric far field is directed along

\begin{dmath}\label{eqn:t:660}
%\lr{ \xcap \wedge \kcap } \cdot \kcap
%= 
\xcap - \lr{ \xcap \cdot \kcap } \kcap
=
\xcap - \lr{ \cancel{\xcap \cdot 
\lr{ \zcap \cos\theta + \ycap \sin\theta }
} } \kcap
= \xcap.
\end{dmath}

For such a ray, the electric far field lies completely in the plane of reflection.  From \citep{hecht1998hecht} (\eqntext 4.34), the Fresnel reflection coefficients is
\index{Fresnel equations}

\begin{dmath}\label{eqn:t:680}
R =
\frac{
n_i \cos\theta_i - n_t \cos\theta_t
}
{
n_i \cos\theta_i + n_t \cos\theta_t
},
\end{dmath}

which approaches \( -1 \) in the no-transmission limit where \( v_t \rightarrow 0 \), and \( n_t = c/v_t \rightarrow \infty \).

\paragraph{Azimuthal angle dependency of the reflection coefficient}

Now consider a horizontal dipole directed along the y-axis.  For the same wave vector direction as above, the electric far field is now directed along

\begin{dmath}\label{eqn:t:700}
\ycap - \lr{ \ycap \cdot \kcap } \kcap
=
\ycap - \lr{ \ycap \cdot \lr{
\zcap \cos\theta + \ycap \sin\theta
} } \kcap
=
\ycap - \kcap \sin\theta
= 
\ycap - \sin\theta \lr{
\zcap \cos\theta + \ycap \sin\theta
}
= 
\ycap \cos^2 \theta - \sin\theta \cos\theta \zcap
= \cos\theta \lr{ \ycap \cos\theta - \sin\theta \zcap }.
\end{dmath}

That is

\begin{dmath}\label{eqn:t:720}
\BE = 
-j \omega \frac{\mu_0 I_0 l}{4 \pi r} e^{-j k r}
\cos\theta \lr{ \ycap \cos\theta - \sin\theta \zcap }.
\end{dmath}

This far field electric field lies in the plane of incidence (a direction of \( \thetacap \) rotated by \( \pi/2 \)), not in the plane of reflection.  The corresponding magnetic field should be directed along the plane of reflection, which is easily confirmed by calculation

\begin{dmath}\label{eqn:t:740}
\kcap \cross
\lr{ \ycap \cos\theta - \sin\theta \zcap }
=
\lr{ \zcap \cos\theta + \ycap \sin\theta } \cross
\lr{ \ycap \cos\theta - \sin\theta \zcap }
=
-\xcap \cos^2 \theta - \xcap \sin^2\theta
= -\xcap.
\end{dmath}

The far field magnetic field is seen to be

\begin{dmath}\label{eqn:t:721}
\BH = 
j \omega \frac{I_0 l}{4 \pi r} e^{-j k r}
\cos\theta \xcap.
\end{dmath}

Referring again to \citep{hecht1998hecht} (\eqntext 4.40) for the coefficient of reflection coefficient for this polarization

\begin{dmath}\label{eqn:t:620}
R
=
\frac{
n_t \cos\theta_i - n_i \cos\theta_t
}
{
n_i \cos\theta_i + n_t \cos\theta_t
},
\end{dmath}

In the no-transmission limit, this tends to \( 1 \), and would be the value required to calculation the superposition associated with the ground reflection.

For a more general azimuthal orientation of the horizontal dipole, I'd guess (but have not calculated), there would be components of the both the electric and magnetic far-field rays that lie in the reflection plane, so both reflection coefficients would be required in proportion to the components in question.

\EndArticle
%\EndNoBibArticle
