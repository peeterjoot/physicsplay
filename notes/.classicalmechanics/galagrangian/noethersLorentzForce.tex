%
% Copyright � 2012 Peeter Joot.  All Rights Reserved.
% Licenced as described in the file LICENSE under the root directory of this GIT repository.
%

%
%
\chapter{Lorentz transform Noether current (interaction Lagrangian)}
\index{Noether current!Lorentz transform}
\label{chap:PJLorentzTxInteraction}
\label{chap:noethersLorentzForce}
%\date{October 22, 2008.  noethersLorentzForce.tex}

\section{Motivation}

Here we consider Noether's theorem applied to the covariant form of the Lorentz force Lagrangian.  Boost under rotation or boost or a combination of the two will be considered.

\section{Covariant result}

For proper velocity \(v\), four potential \(A\), and positive time metric signature \((\gamma_0)^2 = 1\), the Lorentz for
Lagrangian is

\begin{equation}\label{eqn:noethers_lorentz_force:lorentzforce}
\begin{aligned}
\LL = \inv{2}m v \cdot v + q A \cdot v/c
\end{aligned}
\end{equation}

Let us see if Noether's can be used to extract an invariant from
the Lorentz force Lagrangian \eqnref{eqn:noethers_lorentz_force:lorentzforce} under a
Lorentz boost or a spatial rotational transformation.

Four vector dot products are Lorentz invariants.  This can be thought of as the definition of a Lorentz transform (ie: the transformations
that leave the four vector dot products unchanged).  Alternatively, this can be shown using the exponential form of the boost

\begin{equation}\label{eqn:noethersLorentzForce:20}
\begin{aligned}
L(x) = \exp(-\alpha \acap/2) x \exp(\alpha \acap/2)
\end{aligned}
\end{equation}

\begin{equation}\label{eqn:noethersLorentzForce:40}
\begin{aligned}
L(x) \cdot L(y)
&= \gpgradezero{ \exp(-\alpha \acap/2) x \exp(\alpha \acap/2) \exp(-\alpha \acap/2) y \exp(\alpha \acap/2) } \\
&= \gpgradezero{ \exp(-\alpha \acap/2) x y \exp(\alpha \acap/2) } \\
&= x \cdot y \gpgradezero{ \exp(-\alpha \acap/2) \exp(\alpha \acap/2) } \\
&= x \cdot y \\
\end{aligned}
\end{equation}

Using the exponential form of the boost operation, boosting \(v\), \(A\) leaves the Lagrangian unchanged.
Therefore there is a conserved quantity according to Noether's, but what is it?

Also observe that the spacetime nature of the bivector \(\acap\) has not actually been specified, which means that all the
subsequent results apply to spatial rotation as well.  Due to the negative spatial signature (\((\gamma_i)^2=-1\)) used here, for a spatial rotation \(\alpha\) will represent a rotation in the negative sense in the oriented plane specified by the unit bivector \(\acap\).

Consider change with respect to the rapidity factor (or rotational angle) \(\alpha\)

\begin{equation}\label{eqn:noethers_lorentz_force:lorentzboosted}
\begin{aligned}
\PD{\alpha}{\LL'} = \frac{d}{d\tau} \left( \PD{\alpha}{x'} \cdot \grad_{v'} \LL \right)
\end{aligned}
\end{equation}

The boost spacetime plane (or rotational plane) \(\acap\) could also be considered a parameter in the transformation, but to use that or the combination of the
two we need the multivector form of Noether's.  These notes were in fact originally part of an attempt \chapcite{PJEulerLagrange}
to get a feeling for the scalar case as lead up to that so this is an exercise for later.

As for the derivatives in \eqnref{eqn:noethers_lorentz_force:lorentzboosted} we have

\begin{equation}\label{eqn:noethersLorentzForce:60}
\begin{aligned}
\PD{\alpha}{x'}
&= \PD{\alpha}{} \exp(-\alpha \acap/2) x \exp(\alpha \acap/2) \\
&= -\inv{2} \left(\acap x' - x'\acap\right) \\
&= - \acap \cdot x' \\
\end{aligned}
\end{equation}

\begin{equation}\label{eqn:noethersLorentzForce:80}
\begin{aligned}
\grad_{v'} \LL &= p' + qA'/c
\end{aligned}
\end{equation}

So the conserved quantity is
\begin{equation}\label{eqn:noethersLorentzForce:100}
\begin{aligned}
- (\acap \cdot x') \cdot \left( p' + qA'/c \right)
&= - \acap \cdot (x' \wedge (p' + qA'/c ) ) \\
&= -\acap \cdot \kappa
\end{aligned}
\end{equation}

So we have a conserved quantity

\begin{equation}\label{eqn:noethersLorentzForce:120}
\begin{aligned}
x \wedge (p + qA/c ) = \kappa
\end{aligned}
\end{equation}

This has the looks of the three dimensional angular momentum conservation expression (with an added term due to non-radial potential),
but does not look like any quantity from relativistic texts that I have seen (not that I have really seen too much).

As an example to get a feeling for this take \(x\) to be a rest frame worldline.  Then we have

\begin{equation}\label{eqn:noethersLorentzForce:140}
\begin{aligned}
c t \gamma_0 \wedge ( m \tdot \gamma_0 + qA/c ) = -q t \BA = \kappa
\end{aligned}
\end{equation}

Which indicates that the product of observer time and the observers' three vector potential is a constant of motion.  Curious.  Not a familiar result.

Assuming these calculations are correct, then if this holds for all time for then \(\kappa =0\) due to the origin time of \(x\).
I would interpret this to mean that for the charged mass to be at rest, the vector potential must also be zero.  So while \(x = ct\gamma_0\)
is simple for calculations, it does not appear to be a terribly interesting case.

FIXME: try plugging in specific solutions to the Lorentz force equation here to validate or invalidate this calculation.

One further thing that can be observed about this is that if we take derivatives of

\begin{equation}\label{eqn:noethersLorentzForce:160}
\begin{aligned}
x \wedge (p + qA/c ) = \kappa
\end{aligned}
\end{equation}

we have
\begin{equation}\label{eqn:noethersLorentzForce:180}
\begin{aligned}
v \wedge (p + qA/c ) + x \wedge (\pdot + q\dot{A}/c ) = 0
\end{aligned}
\end{equation}

Or
\begin{dmath}\label{eqn:noethers_lorentz_force:noethersLxTx}
x \wedge \pdot
= \frac{d}{d\tau} \left( q/c A \wedge x \right)
= qA \wedge v/c + q/c \dot{A} \wedge x
\end{dmath}

So we have a relativistic torque expressed in terms of the potential, proper velocity and the variation of the potential.

\section{Expansion in observer frame}

This still is not familiar looking, but lets expand this in terms of a particular observable, and see what falls out.  First the LHS, with \(dt/d\tau = \gamma\)

\begin{equation}\label{eqn:noethersLorentzForce:200}
\begin{aligned}
x \wedge \pdot &= (ct \gamma_0 + x^i \gamma_i) \wedge
\left( \gamma \frac{d}{dt}\left( m \gamma (c \gamma_0 + \frac{dx^j}{dt} \gamma_j ) \right) \right) \\
\end{aligned}
\end{equation}

So
\begin{equation}\label{eqn:noethersLorentzForce:220}
\begin{aligned}
\inv{\gamma} (x \wedge \pdot)
&=
- ct \frac{d (\gamma \Bp)}{dt}
+ \Bx \frac{d (m c \gamma)}{dt}
+ x^i \gamma_i \wedge
\frac{d}{dt}\left( m \gamma \frac{dx^j}{dt} \gamma_j \right) \\
\end{aligned}
\end{equation}

But
\begin{equation}\label{eqn:noethersLorentzForce:240}
\begin{aligned}
\sigma_i \wedge \sigma_j
&= \inv{2}( \gamma_i \gamma_0 \gamma_j \gamma_0 - \gamma_j \gamma_0 \gamma_i \gamma_0) \\
&= -\frac{(\gamma_0)^2}{2}( \gamma_i \gamma_j - \gamma_j \gamma_i ) \\
&= -\gamma_i \wedge \gamma_j \\
\end{aligned}
\end{equation}

for
\begin{equation}\label{eqn:noethersLorentzForce:260}
\begin{aligned}
\inv{\gamma} (x \wedge \pdot)
&=
- ct \frac{d (\gamma \Bp)}{dt}
+ \Bx \frac{d (m c \gamma)}{dt}
- \Bx \wedge \frac{d( \gamma \Bp )}{dt} \\
\end{aligned}
\end{equation}

Now, for the RHS of \eqnref{eqn:noethers_lorentz_force:noethersLxTx}, with \(A^0 = \phi\)

\begin{equation}\label{eqn:noethersLorentzForce:280}
\begin{aligned}
\frac{q}{c} \gamma \frac{d(x \wedge A)}{dt}
&= \frac{q}{c} \gamma \frac{d}{dt} (ct \gamma_0 + x^i \gamma_i) \wedge (\phi \gamma_0 + A^j \gamma_j) \\
&= \frac{q}{c} \gamma \frac{d}{dt} \left( -ct \BA + \phi \Bx - \Bx \wedge \BA \right)
\end{aligned}
\end{equation}

Equating the vector and bivector parts, and employing a duality transformation for the bivector parts leaves two vector relationships
\begin{equation}\label{eqn:noethers_lorentz_force:boostresultspacetime}
\begin{aligned}
ct \frac{d (\gamma \Bp)}{dt} - \Bx \frac{d (m c \gamma)}{dt} &= \frac{q}{c} \frac{d \left( ct \BA - \phi \Bx \right) }{dt}
\end{aligned}
\end{equation}
\begin{equation}\label{eqn:noethers_lorentz_force:rotation}
\begin{aligned}
\Bx \cross \frac{d( \gamma \Bp )}{dt} &= \frac{q}{c} \frac{d}{dt} \left( \Bx \cross \BA \right)
\end{aligned}
\end{equation}

FIXME: the first equation looks like it could also be expressed in some sort more symmetric form.  Perhaps a grade two (commutator) product between the multivectors \((mc\gamma, \Bp) = p \gamma_0\), and \((\phi, \BA) = A \gamma_0\)?

\section{In tensor form}

As can be seen above, the four vector form of \eqnref{eqn:noethers_lorentz_force:noethersLxTx} is much more symmetric.  What does it look like in
tensor form?  After first re-consolidating the proper time derivatives we can read the coordinate form off by inspection

\begin{equation}\label{eqn:noethersLorentzForce:300}
\begin{aligned}
x \wedge \pdot &= \frac{d}{d\tau} \left( q/c A \wedge x \right) \\
\end{aligned}
\end{equation}

\begin{equation}\label{eqn:noethersLorentzForce:320}
\begin{aligned}
\gamma_\mu \wedge \gamma_\nu x^\mu m v^\nu &= \frac{d}{d\tau} \left( q/c A^\alpha x^\beta \right) \gamma_\alpha \wedge \gamma_\beta \\
\end{aligned}
\end{equation}

Which gives the tensor expression

\begin{equation}\label{eqn:noethersLorentzForce:340}
\begin{aligned}
\epsilon_{\mu\nu} \left(x^\mu v^\nu - \frac{d}{d\tau} \left( \frac{q}{mc} A^\mu x^\nu \right) \right) = 0
\end{aligned}
\end{equation}

This in turn implies the following six equations in \(\mu\), and \(\nu\)

\begin{equation}\label{eqn:noethers_lorentz_force:coords}
\begin{aligned}
x^\mu v^\nu - x^\nu v^\mu = \frac{q}{mc} \frac{d}{d\tau} \left( A^\mu x^\nu - A^\nu x^\mu \right)
\end{aligned}
\end{equation}

Looking to see if I got the right result, I asked on PF, and was pointed to
\citep{BaezBoosts}.
That ASCII thread is hard to read but at least my result is similar.  I will have to massage things to match them up more closely.

What I did not realize until I read that is that my rotation was not fixed as either hyperbolic or euclidean since I did not actually specify the specific nature of the bivector for the rotational plane.  So I ended up with results for both the spatial invariance and the boost invariance at the same time.  Have adjusted things above, but that is why the spatial rotation references all appear as afterthoughts.

Of the six equations in \eqnref{eqn:noethers_lorentz_force:coords}, taking space time indices
yields the vector \eqnref{eqn:noethers_lorentz_force:boostresultspacetime} as the conserved quantity for a boost.  Similarly
the second vector result in \eqnref{eqn:noethers_lorentz_force:rotation} for purely spatial indices is the conserved quantity for spatial rotation.
That makes my result seem more reasonable since I did not expect to get so much only considering boost.
