%
% Copyright � 2016 Peeter Joot.  All Rights Reserved.
% Licenced as described in the file LICENSE under the root directory of this GIT repository.
%
%{
%\newcommand{\authorname}{Peeter Joot}
\newcommand{\email}{peeterjoot@protonmail.com}
\newcommand{\basename}{FIXMEbasenameUndefined}
\newcommand{\dirname}{notes/FIXMEdirnameUndefined/}

%\renewcommand{\basename}{hestenesElipseParameterization}
%\renewcommand{\dirname}{notes/phy1520/}
%%\newcommand{\dateintitle}{}
%%\newcommand{\keywords}{}
%
%\newcommand{\authorname}{Peeter Joot}
\newcommand{\onlineurl}{http://sites.google.com/site/peeterjoot2/math2013/\basename.pdf}
\newcommand{\sourcepath}{\dirname\basename.tex}
\newcommand{\generatetitle}[1]{\chapter{#1}}

\newcommand{\vcsinfo}{%
\section*{}
\noindent{\color{DarkOliveGreen}{\rule{\linewidth}{0.1mm}}}
\paragraph{Document version}
%\paragraph{\color{Maroon}{Document version}}
{
\small
\begin{itemize}
\item Available online at:\\ 
\href{\onlineurl}{\onlineurl}
\item Git Repository: \input{./.revinfo/gitRepo.tex}
\item Source: \sourcepath
\item last commit: \input{./.revinfo/gitCommitString.tex}
\item commit date: \input{./.revinfo/gitCommitDate.tex}
\end{itemize}
}
}

%\PassOptionsToPackage{dvipsnames,svgnames}{xcolor}
\PassOptionsToPackage{square,numbers}{natbib}
\documentclass{scrreprt}

\usepackage[left=2cm,right=2cm]{geometry}
\usepackage[svgnames]{xcolor}
\usepackage{peeters_layout}

\usepackage{natbib}

\usepackage[
colorlinks=true,
bookmarks=false,
pdfauthor={\authorname, \email},
backref 
]{hyperref}

% http://tex.stackexchange.com/questions/75773/how-to-reference-problems-by-the-text-label-in-an-exercise-envioronment
\usepackage[english]{cleveref}
\crefname{Exercise}{exercise}{exercises}
\Crefname{Exercise}{Exercise}{Exercises}

\RequirePackage{titlesec}
\RequirePackage{ifthen}

% http://stackoverflow.com/questions/4932910/date-in-the-tabular-environment
\makeatletter
\let\insertdate\@date
\makeatother

\titleformat{\chapter}[display]
{\bfseries\Large}
{\color{DarkSlateGrey}\filleft \authorname
\ifthenelse{\isundefined{\studentnumber}}{}{\\ \studentnumber}
\ifthenelse{\isundefined{\email}}{}{\\ \email}
\ifthenelse{\isundefined{\dateintitle}}{}{\\ \insertdate}
%\ifthenelse{\isundefined{\coursename}}{}{\\ \coursename} % put in title instead.
}
{4ex}
{\color{DarkOliveGreen}{\titlerule}\color{Maroon}
\vspace{2ex}%
\filright}
[\vspace{2ex}%
\color{DarkOliveGreen}\titlerule
]

\newcommand{\beginArtWithToc}[0]{\begin{document}\tableofcontents}
\newcommand{\beginArtNoToc}[0]{\begin{document}}
\newcommand{\EndNoBibArticle}[0]{\end{document}}
\newcommand{\EndArticle}[0]{\bibliography{Bibliography}\bibliographystyle{plainnat}\end{document}}

% 
%\newcommand{\citep}[1]{\cite{#1}}

\colorSectionsForArticle


%
%\usepackage{peeters_layout_exercise}
%\usepackage{peeters_braket}
%\usepackage{peeters_figures}
%\usepackage{siunitx}
%
%\beginArtNoToc
%
%\generatetitle{Elliptic parameterization}
%%\chapter{Elliptic parameterization}
%%\label{chap:hestenesElipseParameterization}
%% \citep{sakurai2014modern} pr X.Y
%% \citep{pozar2009microwave}
%% \citep{qftLectureNotes}
%% \citep{griffiths1999introduction}
%
\makeoproblem{Elliptic parameterization}{problem:hestenesElipseParameterization:1}{\citep{hestenes1999nfc} ch. 3, pr. 8.6}{
Show that an ellipse can be parameterized by
\index{ellipse}

\begin{dmath}\label{eqn:hestenesElipseParameterization:20}
\Br(t) = \Bc \cosh( \mu + i t ).
\end{dmath}

Here \( i \) is a unit bivector, and \( i \wedge \Bc \) is zero (i.e. \( \Bc \) must be in the plane of the bivector \( i \)).
} % problem

\makeanswer{problem:hestenesElipseParameterization:1}{

Note that \( \mu, t, i \) all commute since \( \mu, t \) are both scalars.
That allows a complex-like expansion of the hyperbolic cosine to be used

\begin{dmath}\label{eqn:hestenesElipseParameterization:40}
\cosh( \mu + i t )
=
\inv{2} \lr{ e^{\mu + i t} + e^{-\mu -i t} }
=
\inv{2} \lr{ e^{\mu} (\cos t + i \sin t) + e^{-\mu} (\cos t -i \sin t) }
=
\cosh \mu \cos t + i \sinh \mu \sin t.
\end{dmath}

Since an ellipse can be parameterized as

\begin{dmath}\label{eqn:hestenesElipseParameterization:60}
\Br(t) = \Ba \cos t + \Bb \sin t,
\end{dmath}

where the vector directions \( \Ba \) and \( \Bb \) are perpendicular, the multivector hyperbolic cosine representation parameterizes the ellipse provided

\begin{dmath}\label{eqn:hestenesElipseParameterization:80}
\begin{aligned}
\Ba &= \Bc \cosh \mu \\
\Bb &= \Bc i \sinh \mu.
\end{aligned}
\end{dmath}

It is desirable to relate the parameters \( \mu, i \) to the vectors \( \Ba, \Bb \).  Because \( \Bc \wedge i = 0 \), the vector \( \Bc \) anticommutes with \( i \), and therefore \( (\Bc i)^2 = -\Bc i i \Bc = \Bc^2 \), which means

\begin{dmath}\label{eqn:hestenesElipseParameterization:100}
\begin{aligned}
\Ba^2 &= \Bc^2 \cosh^2 \mu \\
\Bb^2 &= \Bc^2 \sinh^2 \mu,
\end{aligned}
\end{dmath}

or
\begin{dmath}\label{eqn:hestenesElipseParameterization:120}
\mu = \tanh^{-1} \frac{\Abs{\Bb}}{\Abs{\Ba}}.
\end{dmath}

The bivector \( i \) is just the unit bivector for the plane containing \( \Ba \) and \( \Bb \)

\begin{dmath}\label{eqn:hestenesElipseParameterization:140}
\Ba \wedge \Bb
= \cosh \mu \sinh \mu \Bc \wedge (\Bc i)
= \cosh \mu \sinh \mu \gpgradetwo{ \Bc \Bc i }
= \cosh \mu \sinh \mu i \Bc^2
= \cosh \mu \sinh \mu i \frac{ \Ba^2 }{ \cosh^2 \mu }
= \Ba^2 \tanh \mu i
= \Ba^2 i \frac{\Abs{\Bb}}{\Abs{\Ba}},
\end{dmath}

so
\begin{dmath}\label{eqn:hestenesElipseParameterization:160}
i = \frac{ \Ba \wedge \Bb }{\Abs{\Ba}\Abs{\Bb}}.
\end{dmath}

Observe that \( i \) is a unit bivector provided the vectors \( \Ba, \Bb \) are perpendicular, as required

\begin{dmath}\label{eqn:hestenesElipseParameterization:180}
(\Ba \wedge \Bb)^2
=
(\Ba \wedge \Bb) \cdot (\Ba \wedge \Bb)
=
( (\Ba \wedge \Bb) \cdot \Ba ) \cdot \Bb
=
( \Ba (\Bb \cdot \Ba) - \Bb \Ba^2 ) \cdot \Bb
=
(\Ba \cdot \Bb)^2 - \Bb^2 \Ba^2
=
- \Bb^2 \Ba^2.
\end{dmath}
} % answer

%%}
%\EndArticle
%%\EndNoBibArticle
