%
% Copyright � 2012 Peeter Joot.  All Rights Reserved.
% Licenced as described in the file LICENSE under the root directory of this GIT repository.
%

%
%
\chapter{Trivector geometry}
\index{trivector}
\label{chap:trivector}
%\date{Mar 9, 2008.  trivector.tex}

\section{Motivation}

The direction vector for two intersecting planes can be found to have the
form:

\begin{equation}\label{eqn:trivector:dirvecintersection}
a\left( (\Bv \wedge \BB)^2 \Bu
- (\Bv \wedge \BB)(\Bu \wedge \BB)\Bv \right)
+ b\left((\Bu \wedge \BB)^2 \Bv
- (\Bu \wedge \BB)(\Bv \wedge \BB)\Bu\right)
\end{equation}

While trying to put \eqnref{eqn:trivector:dirvecintersection} into a form
that eliminated \(\Bu\), and \(\Bv\) in favor of \(\BA = \Bu \wedge \Bv\)
symmetric and antisymmetric formulations for the various grade terms
of a trivector product looked like they could be handy.  Here is a summary
of those results.

\section{Grade components of a trivector product}

\subsection{Grade 6 term}

Writing two trivectors in terms
of mutually orthogonal components

\begin{equation}\label{eqn:trivector:26}
\BA = \Bx \wedge \By \wedge \Bz = \Bx\By\Bz
\end{equation}

and

\begin{equation}\label{eqn:trivector:46}
\BB = \Bu \wedge \Bv \wedge \Bw =\Bu\Bv\Bw
\end{equation}

Assuming that there is no common vector between the two, the
wedge of these is

\begin{equation}\label{eqn:trivector:186}
\begin{aligned}
\BA \wedge \BB
&= \gpgrade{\BA\BB}{6} \\
&= \gpgrade{\Bx\By\Bz\Bu\Bv\Bw}{6} \\
&= \gpgrade{\By\Bz(\Bx\Bu)\Bv\Bw}{6} \\
&= \gpgrade{\By\Bz(-\Bu\Bx + 2\Bu \cdot \Bx)\Bv\Bw}{6} \\
&= -\gpgrade{\By\Bz\Bu(\Bx\Bv)\Bw}{6} \\
&= -\gpgrade{\By\Bz\Bu(-\Bv\Bx + 2\Bv \cdot \Bx)\Bw}{6} \\
&= \gpgrade{\By\Bz\Bu\Bv(\Bx\Bw)}{6} \\
&= \cdots \\
&= -\gpgrade{\Bu\Bv\Bw\Bx\By\Bz}{6} \\
&= -\gpgrade{\BB\BA}{6} \\
&= -\BB \wedge \BA
\end{aligned}
\end{equation}

Note above that any interchange of terms inverts the sign (demonstrated
explicitly for all the \(\Bx\) interchanges).

As an aside, this
sign change on interchange is taken as the defining property of the
wedge product in differential forms.  That property also
implies also that the wedge product is
zero when a vector is wedged with itself since zero is the only
value that is the negation of itself.  Thus we see explicitly
how the notation of using the wedge for the highest grade term
of two blades is consistent with the traditional
wedge product definition.

The end result here is that the grade 6 term of a trivector trivector product
changes sign on interchange of the trivectors:

\begin{equation}\label{eqn:trivector:trivecgpgrade6}
\gpgrade{\BA\BB}{6} = -\gpgrade{\BB\BA}{6}
\end{equation}

\subsection{Grade 4 term}

For a trivector product to have a grade 4 term there must be a common
vector between the two

\begin{equation}\label{eqn:trivector:66}
\BA = \Bx \wedge \By \wedge \Bz = \Bx\By\Bz
\end{equation}

and

\begin{equation}\label{eqn:trivector:86}
\BB = \Bu \wedge \Bv \wedge \Bz =\Bu\Bv\Bz
\end{equation}

The grade four term of the product is

\begin{equation}\label{eqn:trivector:206}
\begin{aligned}
\gpgrade{\BB \BA}{4}
&= \gpgrade{ \Bu\Bv\Bz \Bx\By\Bz }{4} \\
&= \gpgrade{ \Bu\Bv\Bz \Bz\Bx\By }{4} \\
&= \Bz^2\gpgrade{ \Bu\Bv\Bx\By }{4} \\
&= \Bz^2\gpgrade{ \Bu(\Bv\Bx)\By }{4} \\
&= \Bz^2\gpgrade{ \Bu(-\Bx\Bv + 2 \Bx \cdot \Bv)\By }{4} \\
&= -\Bz^2\gpgrade{ \Bu\Bx\Bv\By }{4} \\
&= \cdots \\
&= \Bz^2\gpgrade{ \Bx\By\Bu\Bv }{4} \\
&= \gpgrade{ \Bx\By\Bz\Bz\Bu\Bv }{4} \\
&= \gpgrade{ \Bx\By\Bz\Bu\Bv\Bz }{4} \\
&= \gpgrade{ \Bx\By\Bz\Bu\Bv\Bz }{4} \\
&= \gpgrade{\BA \BB}{4}
\end{aligned}
\end{equation}

Thus the grade 4 term commutes on interchange:

\begin{equation}\label{eqn:trivector:trivecgpgrade4}
\gpgrade{\BA\BB}{4} = \gpgrade{\BB\BA}{4}
\end{equation}

\subsection{Grade 2 term}

Similar to above,
for a trivector product to have a grade 2 term there must be two common
vectors between the two

\begin{equation}\label{eqn:trivector:106}
\BA = \Bx \wedge \By \wedge \Bz = \Bx\By\Bz
\end{equation}

and

\begin{equation}\label{eqn:trivector:126}
\BB = \Bu \wedge \By \wedge \Bz =\Bu\By\Bz
\end{equation}

The grade two term of the product is

\begin{equation}\label{eqn:trivector:226}
\begin{aligned}
\gpgrade{\BA \BB}{2}
&= \gpgrade{ \Bx\By\Bz \Bu\By\Bz }{2} \\
&= \gpgrade{ \Bx\By\Bz \By\Bz \Bu}{2} \\
&= (\By\Bz)^2\gpgrade{ \Bx \Bu}{2} \\
&= -(\By\Bz)^2\gpgrade{ \Bu \Bx}{2} \\
&= -\gpgrade{ \BB \BA }{2} \\
\end{aligned}
\end{equation}

The grade 2 term anticommutes on interchange:

\begin{equation}\label{eqn:trivector:trivecgpgrade2}
\gpgrade{\BA\BB}{2} = -\gpgrade{\BB\BA}{2}
\end{equation}

\subsection{Grade 0 term}

Any grade 0 terms are due to products of the form \(\BA = k\BB\)

\begin{equation}\label{eqn:trivector:246}
\begin{aligned}
\gpgrade{\BA \BB}{0}
&= \gpgrade{k\BB \BB}{0} \\
&= \gpgrade{\BB k\BB}{0} \\
&= \gpgrade{\BB \BA}{0} \\
\end{aligned}
\end{equation}

The grade 2 term commutes on interchange:

\begin{equation}\label{eqn:trivector:trivecgpgrade0}
\gpgrade{\BA\BB}{0} = \gpgrade{\BB\BA}{0}
\end{equation}

\subsection{combining results}

\begin{equation*}
\BA \BB
=\gpgrade{\BA\BB}{0}
+\gpgrade{\BA\BB}{2}
+\gpgrade{\BA\BB}{4}
+\gpgrade{\BA\BB}{6}
\end{equation*}

\begin{equation}\label{eqn:trivector:266}
\begin{aligned}
\BB\BA
&=\gpgrade{\BB\BA}{0}
+\gpgrade{\BB\BA}{2}
+\gpgrade{\BB\BA}{4}
+\gpgrade{\BB\BA}{6} \\
&=\gpgrade{\BA\BB}{0}
-\gpgrade{\BA\BB}{2}
+\gpgrade{\BA\BB}{4}
-\gpgrade{\BA\BB}{6} \\
\end{aligned}
\end{equation}

These can be combined to express each of the grade terms as subsets
of the symmetric and antisymmetric parts:

\begin{equation}\label{eqn:trivector:286}
\begin{aligned}
\BA \cdot \BB = \gpgrade{\BA\BB}{0} &= \gpgrade{\frac{\BA\BB + \BB\BA}{2}}{0} \\
\gpgrade{\BA\BB}{2} &= \gpgrade{\frac{\BA\BB - \BB\BA}{2}}{2} \\
\gpgrade{\BA\BB}{4} &= \gpgrade{\frac{\BA\BB + \BB\BA}{2}}{4} \\
\BA \wedge \BB = \gpgrade{\BA\BB}{6} &= \gpgrade{\frac{\BA\BB - \BB\BA}{2}}{6} \\
\end{aligned}
\end{equation}

Note that above I have been somewhat loose with the argument above.  A grade three vector
will have the following form:

\begin{equation}\label{eqn:trivector:146}
\sum_{i<j<k} D_{ijk} \Be_{ijk}
\end{equation}

Where \(D_{ijk}\) is the determinant of \(ijk\) components of the vectors being wedged.  Thus the product
of two trivectors will be of the following form:

\begin{equation}\label{eqn:trivector:166}
\sum_{i<j<k} \sum_{i'<j'<k'} D_{ijk} D'_{i'j'k'} (\Be_{ijk} \Be_{i'j'k'})
\end{equation}

It is really each of these \(\Be_{ijk} \Be_{i'j'k'}\) products that have to be considered in the grade
and sign arguments above.  The end result will be the same though... one would just have to present
it a bit more carefully for a true proof.

\subsection{Intersecting trivector cases}

As with the intersecting bivector case, when there is a line of intersection between the two volumes one can
write:

\begin{equation}\label{eqn:trivector:306}
\begin{aligned}
\BA \cdot \BB = \gpgrade{\BA\BB}{0} &= \gpgrade{\frac{\BA\BB + \BB\BA}{2}}{0} \\
\gpgrade{\BA\BB}{2} &= \frac{\BA\BB - \BB\BA}{2} \\
\gpgrade{\BA\BB}{4} &= \gpgrade{\frac{\BA\BB + \BB\BA}{2}}{4} \\
\BA \wedge \BB = \gpgrade{\BA\BB}{6} &= 0 \\
\end{aligned}
\end{equation}

And if these volumes intersect in a plane a further simplification is possible:
\begin{equation}\label{eqn:trivector:326}
\begin{aligned}
\BA \cdot \BB = \gpgrade{\BA\BB}{0} &= \frac{\BA\BB + \BB\BA}{2} \\
\gpgrade{\BA\BB}{2} &= \frac{\BA\BB - \BB\BA}{2} \\
\gpgrade{\BA\BB}{4} &= 0 \\
\BA \wedge \BB = \gpgrade{\BA\BB}{6} &= 0 \\
\end{aligned}
\end{equation}

