%
% Copyright � 2012 Peeter Joot.  All Rights Reserved.
% Licenced as described in the file LICENSE under the root directory of this GIT repository.
%

%
%
\chapter{Orthogonal decomposition take II}
\label{chap:orthodecomp}
%\date{April 1, 2008.  orthodecomp.tex}
\section{Lemma.  Orthogonal decomposition}
To do so we first need to be able to express a vector \(\Bx\) in terms
of components parallel and perpendicular to the blade \(\BA \in \wedge^k\).

\begin{equation}\label{eqn:orthodecomp:20}
\begin{aligned}
\Bx
&= \Bx \BA \inv{\BA} \\
&= (\Bx \cdot \BA + \Bx \wedge \BA) \inv{\BA} \\
&=
(\Bx \cdot \BA) \cdot \inv{\BA}
+ \sum_{i=3,5,\cdots,2k-1} \gpgrade{(\Bx \cdot \BA) \inv{\BA}}{i} \\
&+
(\Bx \wedge \BA) \cdot \inv{\BA}
+ \sum_{i=3,5,\cdots,2k-1} \gpgrade{(\Bx \wedge \BA) \inv{\BA}}{i}
+ \mathLabelBox{(\Bx \wedge \BA) \wedge \inv{\BA}}{\(=0\)}
\end{aligned}
\end{equation}

Since the LHS and RHS must both be vectors all the non-grade one terms
are either zero or cancel out.  This can be observed directly since:

\begin{equation}\label{eqn:orthodecomp:40}
\begin{aligned}
\gpgrade{\Bx \cdot \BA \inv{\BA}}{i}
&= \gpgrade{ \frac{\Bx \BA - (-1)^{k}\BA\Bx}{2}\inv{\BA} }{i}  \\
&= -\frac{(-1)^{k}}{2} \gpgrade{ \BA\Bx \inv{\BA} }{i}  \\
\end{aligned}
\end{equation}

and

\begin{equation}\label{eqn:orthodecomp:60}
\begin{aligned}
\gpgrade{\Bx \wedge \BA \inv{\BA}}{i}
&= \gpgrade{ \frac{\Bx \BA + (-1)^{k}\BA\Bx}{2}\inv{\BA} }{i}  \\
&= +\frac{(-1)^{k}}{2} \gpgrade{ \BA\Bx \inv{\BA} }{i}  \\
\end{aligned}
\end{equation}

Thus all of the grade \(3, \cdots ,2k-1\) terms cancel each other out.  Some terms
like \((\Bx \cdot \BA) \wedge \inv{\BA}\) are also independently zero.

(This is a result I have got in other places, but I thought it is worth
 writing down since I thought the direct cancellation is elegant).
