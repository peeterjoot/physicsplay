%
% Copyright � 2012 Peeter Joot.  All Rights Reserved.
% Licenced as described in the file LICENSE under the root directory of this GIT repository.
%

%
%
%%
% Copyright � 2015 Peeter Joot.  All Rights Reserved.
% Licenced as described in the file LICENSE under the root directory of this GIT repository.
%
\documentclass[]{eliblog}

\usepackage{amsmath}
\usepackage{mathpazo}

%
% shorthand for bold symbols, convenient for vectors and matrices
%
\newcommand{\Ba}[0]{\mathbf{a}}
\newcommand{\Bb}[0]{\mathbf{b}}
\newcommand{\Bc}[0]{\mathbf{c}}
\newcommand{\Bd}[0]{\mathbf{d}}
\newcommand{\Be}[0]{\mathbf{e}}
\newcommand{\Bf}[0]{\mathbf{f}}
\newcommand{\Bg}[0]{\mathbf{g}}
\newcommand{\Bh}[0]{\mathbf{h}}
\newcommand{\Bi}[0]{\mathbf{i}}
\newcommand{\Bj}[0]{\mathbf{j}}
\newcommand{\Bk}[0]{\mathbf{k}}
\newcommand{\Bl}[0]{\mathbf{l}}
\newcommand{\Bm}[0]{\mathbf{m}}
\newcommand{\Bn}[0]{\mathbf{n}}
\newcommand{\Bo}[0]{\mathbf{o}}
\newcommand{\Bp}[0]{\mathbf{p}}
\newcommand{\Bq}[0]{\mathbf{q}}
\newcommand{\Br}[0]{\mathbf{r}}
\newcommand{\Bs}[0]{\mathbf{s}}
\newcommand{\Bt}[0]{\mathbf{t}}
\newcommand{\Bu}[0]{\mathbf{u}}
\newcommand{\Bv}[0]{\mathbf{v}}
\newcommand{\Bw}[0]{\mathbf{w}}
\newcommand{\Bx}[0]{\mathbf{x}}
\newcommand{\By}[0]{\mathbf{y}}
\newcommand{\Bz}[0]{\mathbf{z}}
\newcommand{\BA}[0]{\mathbf{A}}
\newcommand{\BB}[0]{\mathbf{B}}
\newcommand{\BC}[0]{\mathbf{C}}
\newcommand{\BD}[0]{\mathbf{D}}
\newcommand{\BE}[0]{\mathbf{E}}
\newcommand{\BF}[0]{\mathbf{F}}
\newcommand{\BG}[0]{\mathbf{G}}
\newcommand{\BH}[0]{\mathbf{H}}
\newcommand{\BI}[0]{\mathbf{I}}
\newcommand{\BJ}[0]{\mathbf{J}}
\newcommand{\BK}[0]{\mathbf{K}}
\newcommand{\BL}[0]{\mathbf{L}}
\newcommand{\BM}[0]{\mathbf{M}}
\newcommand{\BN}[0]{\mathbf{N}}
\newcommand{\BO}[0]{\mathbf{O}}
\newcommand{\BP}[0]{\mathbf{P}}
\newcommand{\BQ}[0]{\mathbf{Q}}
\newcommand{\BR}[0]{\mathbf{R}}
\newcommand{\BS}[0]{\mathbf{S}}
\newcommand{\BT}[0]{\mathbf{T}}
\newcommand{\BU}[0]{\mathbf{U}}
\newcommand{\BV}[0]{\mathbf{V}}
\newcommand{\BW}[0]{\mathbf{W}}
\newcommand{\BX}[0]{\mathbf{X}}
\newcommand{\BY}[0]{\mathbf{Y}}
\newcommand{\BZ}[0]{\mathbf{Z}}

\newcommand{\Bzero}[0]{\mathbf{0}}
\newcommand{\Btheta}[0]{\boldsymbol{\theta}}
\newcommand{\Btau}[0]{\boldsymbol{\tau}}
\newcommand{\Bomega}[0]{\boldsymbol{\omega}}

%
% shorthand for unit vectors
%
\newcommand{\acap}[0]{\hat{\Ba}}
\newcommand{\bcap}[0]{\hat{\Bb}}
\newcommand{\ccap}[0]{\hat{\Bc}}
\newcommand{\dcap}[0]{\hat{\Bd}}
\newcommand{\ecap}[0]{\hat{\Be}}
\newcommand{\fcap}[0]{\hat{\Bf}}
\newcommand{\gcap}[0]{\hat{\Bg}}
\newcommand{\hcap}[0]{\hat{\Bh}}
\newcommand{\icap}[0]{\hat{\Bi}}
\newcommand{\jcap}[0]{\hat{\Bj}}
\newcommand{\kcap}[0]{\hat{\Bk}}
\newcommand{\lcap}[0]{\hat{\Bl}}
\newcommand{\mcap}[0]{\hat{\Bm}}
\newcommand{\ncap}[0]{\hat{\Bn}}
\newcommand{\ocap}[0]{\hat{\Bo}}
\newcommand{\pcap}[0]{\hat{\Bp}}
\newcommand{\qcap}[0]{\hat{\Bq}}
\newcommand{\rcap}[0]{\hat{\Br}}
\newcommand{\scap}[0]{\hat{\Bs}}
\newcommand{\tcap}[0]{\hat{\Bt}}
\newcommand{\ucap}[0]{\hat{\Bu}}
\newcommand{\vcap}[0]{\hat{\Bv}}
\newcommand{\wcap}[0]{\hat{\Bw}}
\newcommand{\xcap}[0]{\hat{\Bx}}
\newcommand{\ycap}[0]{\hat{\By}}
\newcommand{\zcap}[0]{\hat{\Bz}}
\newcommand{\thetacap}[0]{\hat{\Btheta}}

%
% to write R^n and C^n in a distinguishable fashion.  Perhaps change this
% to the double lined characters upon figuring out how to do so.
%
\newcommand{\C}[1]{$\mathbb{C}^{#1}$}
\newcommand{\R}[1]{$\mathbb{R}^{#1}$}

%
% various generally useful helpers
%

% derivative of #1 wrt. #2:
\newcommand{\D}[2] {\frac {d#2} {d#1}}

\newcommand{\inv}[1]{\frac{1}{#1}}
\newcommand{\cross}[0]{\times}

\newcommand{\abs}[1]{\lvert{#1}\rvert}
\newcommand{\norm}[1]{\lVert{#1}\rVert}
\newcommand{\innerprod}[2]{\langle{#1}, {#2}\rangle}
\newcommand{\dotprod}[2]{{#1} \cdot {#2}}
\newcommand{\bdotprod}[2]{\left({#1} \cdot {#2}\right)}
\newcommand{\crossprod}[2]{{#1} \cross {#2}}
\newcommand{\tripleprod}[3]{\dotprod{\left(\crossprod{#1}{#2}\right)}{#3}}

\DeclareMathOperator{\Proj}{Proj}
\DeclareMathOperator{\Span}{span}
\DeclareMathOperator{\Sgn}{sgn}
\DeclareMathOperator{\Area}{Area}
\DeclareMathOperator{\Volume}{Volume}

%
% A few miscellaneous things specific to this document
%
\newcommand{\crossop}[1]{\crossprod{#1}{}}

% R2 vector.
\newcommand{\VectorTwo}[2]{
\begin{bmatrix}
 {#1} \\
 {#2}
\end{bmatrix}
}

\newcommand{\VectorN}[1]{
\begin{bmatrix}
{#1}_1 \\
{#1}_2 \\
\vdots \\
{#1}_N \\
\end{bmatrix}
}

\newcommand{\DETuvij}[4]{
\begin{vmatrix}
 {#1}_{#3} & {#1}_{#4} \\
 {#2}_{#3} & {#2}_{#4}
\end{vmatrix}
}

\newcommand{\DETuvwijk}[6]{
\begin{vmatrix}
 {#1}_{#4} & {#1}_{#5} & {#1}_{#6} \\
 {#2}_{#4} & {#2}_{#5} & {#2}_{#6} \\
 {#3}_{#4} & {#3}_{#5} & {#3}_{#6}
\end{vmatrix}
}

\newcommand{\DETuvwxijkl}[8]{
\begin{vmatrix}
 {#1}_{#5} & {#1}_{#6} & {#1}_{#7} & {#1}_{#8} \\
 {#2}_{#5} & {#2}_{#6} & {#2}_{#7} & {#2}_{#8} \\
 {#3}_{#5} & {#3}_{#6} & {#3}_{#7} & {#3}_{#8} \\
 {#4}_{#5} & {#4}_{#6} & {#4}_{#7} & {#4}_{#8} \\
\end{vmatrix}
}

%\newcommand{\DETuvwxyijklm}[10]{
%\begin{vmatrix}
% {#1}_{#6} & {#1}_{#7} & {#1}_{#8} & {#1}_{#9} & {#1}_{#10} \\
% {#2}_{#6} & {#2}_{#7} & {#2}_{#8} & {#2}_{#9} & {#2}_{#10} \\
% {#3}_{#6} & {#3}_{#7} & {#3}_{#8} & {#3}_{#9} & {#3}_{#10} \\
% {#4}_{#6} & {#4}_{#7} & {#4}_{#8} & {#4}_{#9} & {#4}_{#10} \\
% {#5}_{#6} & {#5}_{#7} & {#5}_{#8} & {#5}_{#9} & {#5}_{#10}
%\end{vmatrix}
%}

% R3 vector.
\newcommand{\VectorThree}[3]{
\begin{bmatrix}
 {#1} \\
 {#2} \\
 {#3}
\end{bmatrix}
}



\author{Peeter Joot}
\email{peeter.joot@gmail.com}

%\documentclass[]{eliblogwidescreen}

\usepackage{amsmath}
\usepackage{mathpazo}

%
% shorthand for bold symbols, convenient for vectors and matrices
%
\newcommand{\Ba}[0]{\mathbf{a}}
\newcommand{\Bb}[0]{\mathbf{b}}
\newcommand{\Bc}[0]{\mathbf{c}}
\newcommand{\Bd}[0]{\mathbf{d}}
\newcommand{\Be}[0]{\mathbf{e}}
\newcommand{\Bf}[0]{\mathbf{f}}
\newcommand{\Bg}[0]{\mathbf{g}}
\newcommand{\Bh}[0]{\mathbf{h}}
\newcommand{\Bi}[0]{\mathbf{i}}
\newcommand{\Bj}[0]{\mathbf{j}}
\newcommand{\Bk}[0]{\mathbf{k}}
\newcommand{\Bl}[0]{\mathbf{l}}
\newcommand{\Bm}[0]{\mathbf{m}}
\newcommand{\Bn}[0]{\mathbf{n}}
\newcommand{\Bo}[0]{\mathbf{o}}
\newcommand{\Bp}[0]{\mathbf{p}}
\newcommand{\Bq}[0]{\mathbf{q}}
\newcommand{\Br}[0]{\mathbf{r}}
\newcommand{\Bs}[0]{\mathbf{s}}
\newcommand{\Bt}[0]{\mathbf{t}}
\newcommand{\Bu}[0]{\mathbf{u}}
\newcommand{\Bv}[0]{\mathbf{v}}
\newcommand{\Bw}[0]{\mathbf{w}}
\newcommand{\Bx}[0]{\mathbf{x}}
\newcommand{\By}[0]{\mathbf{y}}
\newcommand{\Bz}[0]{\mathbf{z}}
\newcommand{\BA}[0]{\mathbf{A}}
\newcommand{\BB}[0]{\mathbf{B}}
\newcommand{\BC}[0]{\mathbf{C}}
\newcommand{\BD}[0]{\mathbf{D}}
\newcommand{\BE}[0]{\mathbf{E}}
\newcommand{\BF}[0]{\mathbf{F}}
\newcommand{\BG}[0]{\mathbf{G}}
\newcommand{\BH}[0]{\mathbf{H}}
\newcommand{\BI}[0]{\mathbf{I}}
\newcommand{\BJ}[0]{\mathbf{J}}
\newcommand{\BK}[0]{\mathbf{K}}
\newcommand{\BL}[0]{\mathbf{L}}
\newcommand{\BM}[0]{\mathbf{M}}
\newcommand{\BN}[0]{\mathbf{N}}
\newcommand{\BO}[0]{\mathbf{O}}
\newcommand{\BP}[0]{\mathbf{P}}
\newcommand{\BQ}[0]{\mathbf{Q}}
\newcommand{\BR}[0]{\mathbf{R}}
\newcommand{\BS}[0]{\mathbf{S}}
\newcommand{\BT}[0]{\mathbf{T}}
\newcommand{\BU}[0]{\mathbf{U}}
\newcommand{\BV}[0]{\mathbf{V}}
\newcommand{\BW}[0]{\mathbf{W}}
\newcommand{\BX}[0]{\mathbf{X}}
\newcommand{\BY}[0]{\mathbf{Y}}
\newcommand{\BZ}[0]{\mathbf{Z}}

\newcommand{\Bzero}[0]{\mathbf{0}}
\newcommand{\Btheta}[0]{\boldsymbol{\theta}}
\newcommand{\Btau}[0]{\boldsymbol{\tau}}
\newcommand{\Bomega}[0]{\boldsymbol{\omega}}

%
% shorthand for unit vectors
%
\newcommand{\acap}[0]{\hat{\Ba}}
\newcommand{\bcap}[0]{\hat{\Bb}}
\newcommand{\ccap}[0]{\hat{\Bc}}
\newcommand{\dcap}[0]{\hat{\Bd}}
\newcommand{\ecap}[0]{\hat{\Be}}
\newcommand{\fcap}[0]{\hat{\Bf}}
\newcommand{\gcap}[0]{\hat{\Bg}}
\newcommand{\hcap}[0]{\hat{\Bh}}
\newcommand{\icap}[0]{\hat{\Bi}}
\newcommand{\jcap}[0]{\hat{\Bj}}
\newcommand{\kcap}[0]{\hat{\Bk}}
\newcommand{\lcap}[0]{\hat{\Bl}}
\newcommand{\mcap}[0]{\hat{\Bm}}
\newcommand{\ncap}[0]{\hat{\Bn}}
\newcommand{\ocap}[0]{\hat{\Bo}}
\newcommand{\pcap}[0]{\hat{\Bp}}
\newcommand{\qcap}[0]{\hat{\Bq}}
\newcommand{\rcap}[0]{\hat{\Br}}
\newcommand{\scap}[0]{\hat{\Bs}}
\newcommand{\tcap}[0]{\hat{\Bt}}
\newcommand{\ucap}[0]{\hat{\Bu}}
\newcommand{\vcap}[0]{\hat{\Bv}}
\newcommand{\wcap}[0]{\hat{\Bw}}
\newcommand{\xcap}[0]{\hat{\Bx}}
\newcommand{\ycap}[0]{\hat{\By}}
\newcommand{\zcap}[0]{\hat{\Bz}}
\newcommand{\thetacap}[0]{\hat{\Btheta}}

%
% to write R^n and C^n in a distinguishable fashion.  Perhaps change this
% to the double lined characters upon figuring out how to do so.
%
\newcommand{\C}[1]{$\mathbb{C}^{#1}$}
\newcommand{\R}[1]{$\mathbb{R}^{#1}$}

%
% various generally useful helpers
%

% derivative of #1 wrt. #2:
\newcommand{\D}[2] {\frac {d#2} {d#1}}

\newcommand{\inv}[1]{\frac{1}{#1}}
\newcommand{\cross}[0]{\times}

\newcommand{\abs}[1]{\lvert{#1}\rvert}
\newcommand{\norm}[1]{\lVert{#1}\rVert}
\newcommand{\innerprod}[2]{\langle{#1}, {#2}\rangle}
\newcommand{\dotprod}[2]{{#1} \cdot {#2}}
\newcommand{\bdotprod}[2]{\left({#1} \cdot {#2}\right)}
\newcommand{\crossprod}[2]{{#1} \cross {#2}}
\newcommand{\tripleprod}[3]{\dotprod{\left(\crossprod{#1}{#2}\right)}{#3}}

\DeclareMathOperator{\Proj}{Proj}
\DeclareMathOperator{\Span}{span}
\DeclareMathOperator{\Sgn}{sgn}
\DeclareMathOperator{\Area}{Area}
\DeclareMathOperator{\Volume}{Volume}

%
% A few miscellaneous things specific to this document
%
\newcommand{\crossop}[1]{\crossprod{#1}{}}

% R2 vector.
\newcommand{\VectorTwo}[2]{
\begin{bmatrix}
 {#1} \\
 {#2}
\end{bmatrix}
}

\newcommand{\VectorN}[1]{
\begin{bmatrix}
{#1}_1 \\
{#1}_2 \\
\vdots \\
{#1}_N \\
\end{bmatrix}
}

\newcommand{\DETuvij}[4]{
\begin{vmatrix}
 {#1}_{#3} & {#1}_{#4} \\
 {#2}_{#3} & {#2}_{#4}
\end{vmatrix}
}

\newcommand{\DETuvwijk}[6]{
\begin{vmatrix}
 {#1}_{#4} & {#1}_{#5} & {#1}_{#6} \\
 {#2}_{#4} & {#2}_{#5} & {#2}_{#6} \\
 {#3}_{#4} & {#3}_{#5} & {#3}_{#6}
\end{vmatrix}
}

\newcommand{\DETuvwxijkl}[8]{
\begin{vmatrix}
 {#1}_{#5} & {#1}_{#6} & {#1}_{#7} & {#1}_{#8} \\
 {#2}_{#5} & {#2}_{#6} & {#2}_{#7} & {#2}_{#8} \\
 {#3}_{#5} & {#3}_{#6} & {#3}_{#7} & {#3}_{#8} \\
 {#4}_{#5} & {#4}_{#6} & {#4}_{#7} & {#4}_{#8} \\
\end{vmatrix}
}

%\newcommand{\DETuvwxyijklm}[10]{
%\begin{vmatrix}
% {#1}_{#6} & {#1}_{#7} & {#1}_{#8} & {#1}_{#9} & {#1}_{#10} \\
% {#2}_{#6} & {#2}_{#7} & {#2}_{#8} & {#2}_{#9} & {#2}_{#10} \\
% {#3}_{#6} & {#3}_{#7} & {#3}_{#8} & {#3}_{#9} & {#3}_{#10} \\
% {#4}_{#6} & {#4}_{#7} & {#4}_{#8} & {#4}_{#9} & {#4}_{#10} \\
% {#5}_{#6} & {#5}_{#7} & {#5}_{#8} & {#5}_{#9} & {#5}_{#10}
%\end{vmatrix}
%}

% R3 vector.
\newcommand{\VectorThree}[3]{
\begin{bmatrix}
 {#1} \\
 {#2} \\
 {#3}
\end{bmatrix}
}



\author{Peeter Joot}
\email{peeter.joot@gmail.com}


\chapter{Newton's method for intersection of curves in a plane}
\index{Newton's method}
\index{intersection}
\label{chap:intersectionNewton}

%\blogpage{http://sites.google.com/site/peeterjoot/math2010/intersectionNewton.pdf}
%\date{Mar 7, 2010}
%\revisionInfo{intersectionNewton.tex}

%\beginArtWithToc
\beginArtNoToc

\section{Motivation}

Reading the blog post \href{http://www.cafelinear.com/2010/02/problem-solving-artificial-intelligence-and-linear-algebra/}{Problem solving, artificial intelligence and computational linear algebra} some variations of Newton's method for finding local minimums and maximums are given.

While I had seen the Hessian matrix eons ago in the context of back propagation feedback methods, Newton's method itself I remember as a first order root finding method.  Here I refresh my memory what that simpler Newton's method was about, and build on that slightly to find the form of the solution for the intersection of an arbitrarily oriented line with a curve, and finally the problem of refining an approximation for the intersection of two curves using the same technique.

\section{Root finding as the intersection with a horizontal}

\imageFigure{../../figures/gabook/newtonsIntersectionHorizontal}{Refining an approximate horizontal intersection}{fig:newtonsIntersectionHorizontal}{0.5}

The essence of Newton's method for finding roots is following the tangent from the point of first guess down to the line that one wants to intersect with the curve.  This is illustrated in \cref{fig:newtonsIntersectionHorizontal}.

Algebraically, the problem is that of finding the point \(x_1\), which is given by the tangent

\begin{align}\label{eqn:intersectionNewton:10}
\frac{f(x_0) - b}{x_0 - x_1} = f'(x_0).
\end{align}

Rearranging and solving for \(x_1\), we have

\begin{align}\label{eqn:intersectionNewton:11}
x_1 = x_0 - \frac{f(x_0) - b}{f'(x_0)}
\end{align}

If one presumes convergence, something not guaranteed, then a first guess, if good enough, will get closer and closer to the target with each iteration.  If this first guess is far from the target, following the tangent line could ping pong you to some other part of the curve, and it is possible not to find the root, or to find some other one.

\section{Intersection with a line}

\imageFigure{../../figures/gabook/newtonsIntersectionAnyOrientation}{Refining an approximation for the intersection with an arbitrarily oriented line}{fig:newtonsIntersectionAnyOrientation}{0.5}

The above pictorial treatment works nicely for the intersection of a horizontal line with a curve.  Now consider the intersection of an arbitrarily oriented line with a curve, as illustrated in \cref{fig:newtonsIntersectionAnyOrientation}.  Here it is useful to setup the problem algebraically from the beginning.  Our problem is really still just that of finding the intersection of two lines.  The curve itself can be considered the set of end points of the vector

\begin{align}\label{eqn:intersectionNewton:20}
\Br(x) = x \Be_1 + f(x) \Be_2,
\end{align}

for which the tangent direction vector is

\begin{align}\label{eqn:intersectionNewton:21}
\Bt(x) = \frac{d\Br}{dx} = \Be_1 + f'(x) \Be_2.
\end{align}

The set of points on this tangent, taken at the point \(x_0\), can also be written as a vector, namely

\begin{align}\label{eqn:intersectionNewton:22}
(x_0, f(x)) + \alpha \Bt(x_0).
\end{align}

For the line to intersect this, suppose we have one point on the line \(\Bp_0\), and a direction vector for that line \(\ucap\).  The points on this line are therefore all the endpoints of

\begin{align}\label{eqn:intersectionNewton:23}
\Bp_0 + \beta \ucap.
\end{align}

Provided that the tangent and the line of intersection do in fact intersect then our problem becomes finding \(\alpha\) or \(\beta\) after equating \eqnref{eqn:intersectionNewton:22} and \eqnref{eqn:intersectionNewton:23}.  This is the solution of

\begin{align}\label{eqn:intersectionNewton:24}
(x_0, f(x_0)) + \alpha \Bt(x_0) = \Bp_0 + \beta \ucap.
\end{align}

Since we do not care which of \(\alpha\) or \(\beta\) we solve for, setting this up as a matrix equation in two variables is not the best approach.  Instead we wedge both sides with \(\Bt(x_0)\) (or \(\ucap\)), essentially using Cramer's method.  This gives

\begin{align}\label{eqn:intersectionNewton:25}
\left((x_0, f(x_0)) -\Bp_0 \right) \wedge \Bt(x_0) = \beta \ucap \wedge \Bt(x_0).
\end{align}

If the lines are not parallel, then both sides are scalar multiples of \(\Be_1 \wedge \Be_2\), and dividing out one gets

\begin{align}\label{eqn:intersectionNewton:26}
\beta = \frac{\left((x_0, f(x_0)) -\Bp_0 \right) \wedge \Bt(x_0)}{\ucap \wedge \Bt(x_0)}.
\end{align}

Writing out \(\Bt(x_0) = \Be_1 + f'(x_0) \Be_2\), explicitly, this is

\begin{align}\label{eqn:intersectionNewton:26b}
\beta = \frac{\left((x_0, f(x_0)) -\Bp_0 \right) \wedge \left(\Be_1 + f'(x_0) \Be_2\right)}{\ucap \wedge \left(\Be_1 + f'(x_0) \Be_2\right)}.
\end{align}

Further, dividing out the common \(\Be_1 \wedge \Be_2\) bivector, we have a ratio of determinants

\begin{align}\label{eqn:intersectionNewton:26c}
\beta =
\frac{
\begin{vmatrix}
x_0 -\Bp_0 \cdot \Be_1 & f(x_0) - \Bp_0 \cdot \Be_2 \\
1 & f'(x_0) \\
\end{vmatrix}
}
{
\begin{vmatrix}
\ucap \cdot \Be_1 & \ucap \cdot \Be_2 \\
1 & f'(x_0) \\
\end{vmatrix}
}.
\end{align}

The final step in the solution is noting that the point of intersection is just

\begin{align}\label{eqn:intersectionNewton:27}
\Bp_0 + \beta \ucap,
\end{align}

and in particular, the \(x\) coordinate of this is the desired result of one step of iteration

\begin{align}\label{eqn:intersectionNewton:28}
x_1 = \Bp_0 \cdot \Be_1 + (\ucap \cdot \Be_1)
\frac{
\begin{vmatrix}
x_0 -\Bp_0 \cdot \Be_1 & f(x_0) - \Bp_0 \cdot \Be_2 \\
1 & f'(x_0) \\
\end{vmatrix}
}
{
\begin{vmatrix}
\ucap \cdot \Be_1 & \ucap \cdot \Be_2 \\
1 & f'(x_0) \\
\end{vmatrix}
}.
\end{align}

This looks a whole lot different than the original \(x_1\) for the horizontal from back at \eqnref{eqn:intersectionNewton:11}, but substitution of \(\ucap = \Be_1\), and \(\Bp_0 = b \Be_2\), shows that these are identical.

\section{Intersection of two curves}

Can we generalize this any further?  It seems reasonable that we would be able to use this Newton's method technique of following the tangent to refine an approximation for the intersection point of two general curves.  This is not expected to be much harder, and the geometric idea is illustrated in \cref{fig:newtonsIntersectionTwoCurves}

\imageFigure{../../figures/gabook/newtonsIntersectionTwoCurves}{Refining an approximation for the intersection of two curves in a plane}{fig:newtonsIntersectionTwoCurves}{0.5}

The task at hand is to setup this problem algebraically.  Suppose the two curves \(s(x)\), and \(r(x)\) are parametrized as vectors

\begin{align}\label{eqn:intersectionNewton:40}
\Bs(x) &= x \Be_1 + s(x) \Be_2 \\
\Br(x) &= x \Be_1 + r(x) \Be_2.
\end{align}

Tangent direction vectors at the point \(x_0\) are then

\begin{align}\label{eqn:intersectionNewton:41}
\Bs'(x_0) &= \Be_1 + s'(x_0) \Be_2 \\
\Br'(x_0) &= \Be_1 + r'(x_0) \Be_2.
\end{align}

The intersection of interest is therefore the solution of

\begin{align}\label{eqn:intersectionNewton:42}
(x_0, s(x_0)) + \alpha \Bs' = (x_0, r(x_0)) + \beta \Br'.
\end{align}

Wedging with one of tangent vectors \(\Bs'\) or \(\Br'\) provides our solution.  Solving for \(\alpha\) this is

\begin{align}\label{eqn:intersectionNewton:43}
\alpha = \frac{(0, r(x_0) - s(x_0)) \wedge \Br'}{\Bs' \wedge \Br'}
=
\frac
{
\begin{vmatrix}
0 & r(x_0) - s(x_0) \\
\Br' \cdot \Be_1 & \Br' \cdot \Be_2
\end{vmatrix}
}
{
\begin{vmatrix}
\Bs' \cdot \Be_1 & \Bs' \cdot \Be_2 \\
\Br' \cdot \Be_1 & \Br' \cdot \Be_2
\end{vmatrix}
}
=
-
\frac
{
r(x_0) - s(x_0)
}
{
r'(x_0) - s'(x_0)
}.
\end{align}

To finish things off, we just have to calculate the new \(x\) coordinate on the line for this value of \(\alpha\), which gives us

\begin{align}\label{eqn:intersectionNewton:44}
x_1 = x_0 -
\frac
{
r(x_0) - s(x_0)
}
{
r'(x_0) - s'(x_0)
}.
\end{align}

It is ironic that generalizing things to two curves leads to a tidier result than the more specific line and curve result from \eqnref{eqn:intersectionNewton:28}.  With a substitution of \(r(x) = f(x)\), and \(s(x) = b\), we once again recover the result \eqnref{eqn:intersectionNewton:11}, for the horizontal line intersecting a curve.

\section{Followup}

Having completed the play that I set out to do, the next logical step would be to try the min/max problem that leads to the Hessian.  That can be for another day.

%%\EndArticle
%\EndNoBibArticle
