%
% Copyright � 2012 Peeter Joot.  All Rights Reserved.
% Licenced as described in the file LICENSE under the root directory of this GIT repository.
%
% pick one:
%\newcommand{\authorname}{Peeter Joot}
\newcommand{\email}{peeter.joot@utoronto.ca}
\newcommand{\studentnumber}{920798560}
\newcommand{\basename}{FIXMEbasenameUndefined}
\newcommand{\dirname}{notes/FIXMEdirnameUndefined/}

%\newcommand{\authorname}{Peeter Joot}
\newcommand{\email}{peeterjoot@protonmail.com}
\newcommand{\basename}{FIXMEbasenameUndefined}
\newcommand{\dirname}{notes/FIXMEdirnameUndefined/}

%\renewcommand{\basename}{snellsFermat}
%\renewcommand{\dirname}{notes/phy485/}
%\newcommand{\dateintitle}{}
%\newcommand{\keywords}{}
%
%\newcommand{\authorname}{Peeter Joot}
\newcommand{\onlineurl}{http://sites.google.com/site/peeterjoot2/math2013/\basename.pdf}
\newcommand{\sourcepath}{\dirname\basename.tex}
\newcommand{\generatetitle}[1]{\chapter{#1}}

\newcommand{\vcsinfo}{%
\section*{}
\noindent{\color{DarkOliveGreen}{\rule{\linewidth}{0.1mm}}}
\paragraph{Document version}
%\paragraph{\color{Maroon}{Document version}}
{
\small
\begin{itemize}
\item Available online at:\\ 
\href{\onlineurl}{\onlineurl}
\item Git Repository: \input{./.revinfo/gitRepo.tex}
\item Source: \sourcepath
\item last commit: \input{./.revinfo/gitCommitString.tex}
\item commit date: \input{./.revinfo/gitCommitDate.tex}
\end{itemize}
}
}

%\PassOptionsToPackage{dvipsnames,svgnames}{xcolor}
\PassOptionsToPackage{square,numbers}{natbib}
\documentclass{scrreprt}

\usepackage[left=2cm,right=2cm]{geometry}
\usepackage[svgnames]{xcolor}
\usepackage{peeters_layout}

\usepackage{natbib}

\usepackage[
colorlinks=true,
bookmarks=false,
pdfauthor={\authorname, \email},
backref 
]{hyperref}

% http://tex.stackexchange.com/questions/75773/how-to-reference-problems-by-the-text-label-in-an-exercise-envioronment
\usepackage[english]{cleveref}
\crefname{Exercise}{exercise}{exercises}
\Crefname{Exercise}{Exercise}{Exercises}

\RequirePackage{titlesec}
\RequirePackage{ifthen}

% http://stackoverflow.com/questions/4932910/date-in-the-tabular-environment
\makeatletter
\let\insertdate\@date
\makeatother

\titleformat{\chapter}[display]
{\bfseries\Large}
{\color{DarkSlateGrey}\filleft \authorname
\ifthenelse{\isundefined{\studentnumber}}{}{\\ \studentnumber}
\ifthenelse{\isundefined{\email}}{}{\\ \email}
\ifthenelse{\isundefined{\dateintitle}}{}{\\ \insertdate}
%\ifthenelse{\isundefined{\coursename}}{}{\\ \coursename} % put in title instead.
}
{4ex}
{\color{DarkOliveGreen}{\titlerule}\color{Maroon}
\vspace{2ex}%
\filright}
[\vspace{2ex}%
\color{DarkOliveGreen}\titlerule
]

\newcommand{\beginArtWithToc}[0]{\begin{document}\tableofcontents}
\newcommand{\beginArtNoToc}[0]{\begin{document}}
\newcommand{\EndNoBibArticle}[0]{\end{document}}
\newcommand{\EndArticle}[0]{\bibliography{Bibliography}\bibliographystyle{plainnat}\end{document}}

% 
%\newcommand{\citep}[1]{\cite{#1}}

\colorSectionsForArticle


%
%\beginArtNoToc
%
%\generatetitle{Derivation of Snell's law using Fermat's theorem}
\label{chap:snellsFermat}

\makeproblem{Derive Snell's law}{snellsFermat:pr:1}{
Fermat's theorem, that light takes the path of least time, can be used to derive Snell's law without resorting to Maxwell's equations.

Note that a proof of Fermat's theorem using the Ray equation can be found in \S 3.3.2 \citep{born1980principles}.
}
\makeanswer{snellsFermat:pr:1}{

We refer to \cref{fig:snellsFermat:snellsFermatFig1}, and seek to express the optical path length.

\imageFigure{../../figures/phy485/snellsFermatFig1}{Snell's law light paths}{fig:snellsFermat:snellsFermatFig1}{0.4}

Since \(n(s) = c/v(s)\), the time spent along any portion of the path is proportional to \(n(s) ds\).  For the two leg linear route that is

\begin{dmath}\label{eqn:snellsFermat:10}
OPL = n r + n' r'.
\end{dmath}

Since

\begin{subequations}
\begin{dmath}\label{eqn:snellsFermat:30}
r = \sqrt{ h^2 + (L - x)^2 }
\end{dmath}
\begin{dmath}\label{eqn:snellsFermat:50}
r' = \sqrt{ {h'}^2 + x^2 }
\end{dmath}
\end{subequations}

We want to find \(x\) such that

\begin{dmath}\label{eqn:snellsFermat:70}
0
= \frac{d(OPL)}{dx}
=
\ddx{} \left(
n \sqrt{ h^2 + (L - x)^2 }
+ n' \sqrt{ {h'}^2 + x^2 }
\right)
=
n \inv{2 r} 2 (L - x)(-1)
+ n' \inv{2 r'} 2 x
=
- n \sin\theta + n' \sin\theta'
\end{dmath}

This gives us

\boxedEquation{eqn:snellsFermat:90}{
n \sin\theta = n' \sin\theta',
}

as desired.
} % makeanswer
\shipoutAnswer

%\vcsinfo
%\EndNoBibArticle
