%
% Copyright � 2012 Peeter Joot.  All Rights Reserved.
% Licenced as described in the file LICENSE under the root directory of this GIT repository.
%
% pick one:
%\newcommand{\authorname}{Peeter Joot}
\newcommand{\email}{peeter.joot@utoronto.ca}
\newcommand{\studentnumber}{920798560}
\newcommand{\basename}{FIXMEbasenameUndefined}
\newcommand{\dirname}{notes/FIXMEdirnameUndefined/}

%\newcommand{\authorname}{Peeter Joot}
\newcommand{\email}{peeterjoot@protonmail.com}
\newcommand{\basename}{FIXMEbasenameUndefined}
\newcommand{\dirname}{notes/FIXMEdirnameUndefined/}

%\renewcommand{\basename}{optics2010ExamFraunhofer}
%\renewcommand{\dirname}{notes/phy485/}
%%\newcommand{\dateintitle}{}
%%\newcommand{\keywords}{}
%
%\newcommand{\authorname}{Peeter Joot}
\newcommand{\onlineurl}{http://sites.google.com/site/peeterjoot2/math2013/\basename.pdf}
\newcommand{\sourcepath}{\dirname\basename.tex}
\newcommand{\generatetitle}[1]{\chapter{#1}}

\newcommand{\vcsinfo}{%
\section*{}
\noindent{\color{DarkOliveGreen}{\rule{\linewidth}{0.1mm}}}
\paragraph{Document version}
%\paragraph{\color{Maroon}{Document version}}
{
\small
\begin{itemize}
\item Available online at:\\ 
\href{\onlineurl}{\onlineurl}
\item Git Repository: \input{./.revinfo/gitRepo.tex}
\item Source: \sourcepath
\item last commit: \input{./.revinfo/gitCommitString.tex}
\item commit date: \input{./.revinfo/gitCommitDate.tex}
\end{itemize}
}
}

%\PassOptionsToPackage{dvipsnames,svgnames}{xcolor}
\PassOptionsToPackage{square,numbers}{natbib}
\documentclass{scrreprt}

\usepackage[left=2cm,right=2cm]{geometry}
\usepackage[svgnames]{xcolor}
\usepackage{peeters_layout}

\usepackage{natbib}

\usepackage[
colorlinks=true,
bookmarks=false,
pdfauthor={\authorname, \email},
backref 
]{hyperref}

% http://tex.stackexchange.com/questions/75773/how-to-reference-problems-by-the-text-label-in-an-exercise-envioronment
\usepackage[english]{cleveref}
\crefname{Exercise}{exercise}{exercises}
\Crefname{Exercise}{Exercise}{Exercises}

\RequirePackage{titlesec}
\RequirePackage{ifthen}

% http://stackoverflow.com/questions/4932910/date-in-the-tabular-environment
\makeatletter
\let\insertdate\@date
\makeatother

\titleformat{\chapter}[display]
{\bfseries\Large}
{\color{DarkSlateGrey}\filleft \authorname
\ifthenelse{\isundefined{\studentnumber}}{}{\\ \studentnumber}
\ifthenelse{\isundefined{\email}}{}{\\ \email}
\ifthenelse{\isundefined{\dateintitle}}{}{\\ \insertdate}
%\ifthenelse{\isundefined{\coursename}}{}{\\ \coursename} % put in title instead.
}
{4ex}
{\color{DarkOliveGreen}{\titlerule}\color{Maroon}
\vspace{2ex}%
\filright}
[\vspace{2ex}%
\color{DarkOliveGreen}\titlerule
]

\newcommand{\beginArtWithToc}[0]{\begin{document}\tableofcontents}
\newcommand{\beginArtNoToc}[0]{\begin{document}}
\newcommand{\EndNoBibArticle}[0]{\end{document}}
\newcommand{\EndArticle}[0]{\bibliography{Bibliography}\bibliographystyle{plainnat}\end{document}}

% 
%\newcommand{\citep}[1]{\cite{#1}}

\colorSectionsForArticle


%
%\beginArtNoToc
%
%\generatetitle{Fraunhofer diffraction pattern for four circular apertures}
%\chapter{Fraunhofer diffraction pattern for four circular apertures}
\index{Fraunhofer diffraction}
\index{circular aperture}
%\label{chap:\basename}
%\section{Motivation}
%\section{Guts}

\makeoproblem{Fraunhofer diffraction through four circular apertures}
{pr:optics2010ExamFraunhofer:1}
{2010 final exam, question 3}
{ 

Calculate the diffraction pattern for the geometry of \cref{fig:optics2010ExamFraunhofer:optics2010ExamFraunhoferFig2}.

\imageFigure{../../figures/phy485/optics2010ExamFraunhoferFig2}{Four circular apertures}{fig:optics2010ExamFraunhofer:optics2010ExamFraunhoferFig2}{0.3}

} % makeoproblem

\makeanswer{pr:optics2010ExamFraunhofer:1}{ 

We are working with distances illustrated in \cref{fig:optics2010ExamFraunhofer:optics2010ExamFraunhoferFig1}.

\imageFigure{../../figures/phy485/optics2010ExamFraunhoferFig1}{Four apertures with observation point and distances}{fig:optics2010ExamFraunhofer:optics2010ExamFraunhoferFig1}{0.3}

As usual we write

\begin{subequations}
\begin{dmath}\label{eqn:optics2010ExamFraunhofer:20}
\BR = \Br' - \Br_s
\end{dmath}
\begin{dmath}\label{eqn:optics2010ExamFraunhofer:40}
R 
= 
r' \left( 1 + \frac{r_s^2}{{r'}^2} - 2 \frac{\Br_s \cdot \Br'}{{r'}^2} \right)^{1/2}
\approx
r' + \frac{r_s^2}{2 {r'}} - \Br_s \cdot \rcap' 
\end{dmath}
\end{subequations}

so that

\begin{dmath}\label{eqn:optics2010ExamFraunhofer:60}
\Psi(\Br') = \frac{\Psi_0}{e^{i k r'}}{i \lambda r'} \int_A e^{-i k \Br_s \cdot \rcap'}
\end{dmath}

Let's write

\begin{subequations}
\begin{dmath}\label{eqn:optics2010ExamFraunhofer:80}
\rcap \cdot \xcap = \alpha
\end{dmath}
\begin{dmath}\label{eqn:optics2010ExamFraunhofer:100}
\rcap \cdot \ycap = \beta,
\end{dmath}
\end{subequations}

and introduce an aperture function

\begin{dmath}\label{eqn:optics2010ExamFraunhofer:120}
g(x, y) = 
\left\{
\begin{array}{l l}
1 & \quad \mbox{if \(x^2 + y^2 \le R^2\)} \\
0 & \quad \mbox{otherwise} 
\end{array}
\right.
\end{dmath}

This allows us to write our diffraction integral as

\begin{dmath}\label{eqn:optics2010ExamFraunhofer:140}
\Psi(\Br') 
= 
\frac{\Psi_0}{e^{i k r'}}{i \lambda r'} 
\int du dv 
\left(
e^{-i k ((u + b/2) \alpha + (v + b/2) \beta) }
+
e^{-i k ((u - b/2) \alpha + (v - b/2) \beta) }
+
e^{-i k ((u + b/2) \alpha + (v - b/2) \beta) }
+
e^{-i k ((u - b/2) \alpha + (v + b/2) \beta) }
\right)
=
\frac{\Psi_0}{e^{i k r'}}{i \lambda r'} 
\left(
e^{-i k (\alpha b/ 2 + \beta b/2) }
+
e^{-i k (-\alpha b/ 2 - \beta b/2) }
+
e^{-i k (\alpha b/ 2 - \beta b/2) }
+
e^{-i k (-\alpha b/ 2 + \beta b/2) }
\right)
\int du dv 
e^{-i k (u \alpha + v \beta) }
=
2 \frac{\Psi_0}{e^{i k r'}}{i \lambda r'} 
\left(
e^{-i k \alpha b/ 2 } \cos( k \beta b/2 )
+
e^{i k \alpha b/ 2 } \cos( k \beta b/2 )
\right)
\int du dv 
e^{-i k (u \alpha + v \beta) }
=
4 \frac{\Psi_0}{e^{i k r'}}{i \lambda r'} 
\cos( k \alpha b/ 2 ) \cos( k \beta b/2 )
\int_{\rho=0}^R \int_{\theta = 0}^{2 \pi} 
\rho d\rho d\theta
e^{-i k \rho (\cos\theta \alpha + \sin \theta \beta) }
\end{dmath}

This last integral isn't something that we can evaluate in just Bessel functions unless one of \(\alpha\) or \(\beta\) is zero.  For example, if \(\beta = 0\), so that the observation axis lies in the one of the perpendicular planes, then we have

\begin{dmath}\label{eqn:optics2010ExamFraunhofer:160}
\Psi \sim
\frac{e^{i k r'}}{r'} 
\cos( k \alpha b/ 2 ) 
\int_{\rho=0}^R \int_{\theta = 0}^{2 \pi}
\rho d\rho d\theta
e^{-i k \rho \cos\theta \alpha }
=
\frac{e^{i k r'}}{r'} 
\cos( k \alpha b/ 2 ) 
2 \pi R \frac{J_1( -k \alpha R)}{ -k \alpha }
=
\frac{e^{i k r'}}{r'} 
\cos( k \alpha b/ 2 ) 
2 \pi R \frac{J_1( k \alpha R)}{ k \alpha }
\end{dmath}

This looks fairly \(\sinc\) like \cref{fig:optics2010ExamFraunhofer:optics2010ExamFraunhoferFig3}.

\imageFigure{../../figures/phy485/optics2010ExamFraunhoferFig3}{Plot of \(J_1\)}{fig:optics2010ExamFraunhofer:optics2010ExamFraunhoferFig3}{0.3}

We can also solve for the case when \(\alpha = \beta\), because we can write

\begin{dmath}\label{eqn:optics2010ExamFraunhofer:180}
\cos\theta \pm \sin\theta = \sqrt{2} \cos(\theta \mp \pi/4).
\end{dmath}

The phase shift doesn't make a difference when we are integrating over \([0, 2 \pi]\), so we are left with

\begin{dmath}\label{eqn:optics2010ExamFraunhofer:170}
\Psi \sim
\frac{e^{i k r'}}{r'} 
\cos( k \alpha b/ 2 ) 
\int_{\rho=0}^R \int_{\theta = 0}^{2 \pi}
=
\frac{e^{i k r'}}{r'} 
\cos( k \alpha b/ 2 ) 
2 \pi R \frac{J_1( \sqrt{2} k \alpha R)}{ \sqrt{2} k \alpha }
\end{dmath}

For arbitrary \(\alpha\) and \(\beta\) there's no such obvious change of variables.  Mathematica calls the result a regularized hypergeometric function

\begin{dmath}\label{eqn:optics2010ExamFraunhofer:200}
\Psi \sim
\frac{e^{i k r'}}{r'} 
\cos( k \alpha b/ 2 ) 
\pi R^2 \, _0\tilde{F}_1\left(;2;-\frac{1}{4} k^2 R^2 \left(\alpha^2+\beta^2\right)\right),
\end{dmath}

The hypergeometric function itself looks fairly sinc like, but not with the \(R^2\) multiplicative factor (plotted as a function of \(R\)).  This is plotted in \cref{fig:optics2010ExamFraunhofer:optics2010ExamFraunhoferFig4}, but curiously, this appears to be a divergent function?  On the exam, I expect that the expectation was just to look on axis, but it would probably also be useful to plug in some actually representitive numbers.

\imageFigure{../../figures/phy485/optics2010ExamFraunhoferFig4}{Plot of hypergeometric function}{fig:optics2010ExamFraunhofer:optics2010ExamFraunhoferFig4}{0.3}
} % makeanswer

% this is to produce the sites.google url and version info and so forth (for blog posts)
%\vcsinfo
%\EndArticle
%\EndNoBibArticle
