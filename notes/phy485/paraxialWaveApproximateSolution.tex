%
% Copyright � 2012 Peeter Joot.  All Rights Reserved.
% Licenced as described in the file LICENSE under the root directory of this GIT repository.
%
% pick one:
%\newcommand{\authorname}{Peeter Joot}
\newcommand{\email}{peeter.joot@utoronto.ca}
\newcommand{\studentnumber}{920798560}
\newcommand{\basename}{FIXMEbasenameUndefined}
\newcommand{\dirname}{notes/FIXMEdirnameUndefined/}

%\newcommand{\authorname}{Peeter Joot}
\newcommand{\email}{peeterjoot@protonmail.com}
\newcommand{\basename}{FIXMEbasenameUndefined}
\newcommand{\dirname}{notes/FIXMEdirnameUndefined/}

%\renewcommand{\basename}{paraxialWaveApproximateSolution}
%\renewcommand{\dirname}{notes/phy485/}
%%\newcommand{\dateintitle}{}
%%\newcommand{\keywords}{}
%
%\newcommand{\authorname}{Peeter Joot}
\newcommand{\onlineurl}{http://sites.google.com/site/peeterjoot2/math2013/\basename.pdf}
\newcommand{\sourcepath}{\dirname\basename.tex}
\newcommand{\generatetitle}[1]{\chapter{#1}}

\newcommand{\vcsinfo}{%
\section*{}
\noindent{\color{DarkOliveGreen}{\rule{\linewidth}{0.1mm}}}
\paragraph{Document version}
%\paragraph{\color{Maroon}{Document version}}
{
\small
\begin{itemize}
\item Available online at:\\ 
\href{\onlineurl}{\onlineurl}
\item Git Repository: \input{./.revinfo/gitRepo.tex}
\item Source: \sourcepath
\item last commit: \input{./.revinfo/gitCommitString.tex}
\item commit date: \input{./.revinfo/gitCommitDate.tex}
\end{itemize}
}
}

%\PassOptionsToPackage{dvipsnames,svgnames}{xcolor}
\PassOptionsToPackage{square,numbers}{natbib}
\documentclass{scrreprt}

\usepackage[left=2cm,right=2cm]{geometry}
\usepackage[svgnames]{xcolor}
\usepackage{peeters_layout}

\usepackage{natbib}

\usepackage[
colorlinks=true,
bookmarks=false,
pdfauthor={\authorname, \email},
backref 
]{hyperref}

% http://tex.stackexchange.com/questions/75773/how-to-reference-problems-by-the-text-label-in-an-exercise-envioronment
\usepackage[english]{cleveref}
\crefname{Exercise}{exercise}{exercises}
\Crefname{Exercise}{Exercise}{Exercises}

\RequirePackage{titlesec}
\RequirePackage{ifthen}

% http://stackoverflow.com/questions/4932910/date-in-the-tabular-environment
\makeatletter
\let\insertdate\@date
\makeatother

\titleformat{\chapter}[display]
{\bfseries\Large}
{\color{DarkSlateGrey}\filleft \authorname
\ifthenelse{\isundefined{\studentnumber}}{}{\\ \studentnumber}
\ifthenelse{\isundefined{\email}}{}{\\ \email}
\ifthenelse{\isundefined{\dateintitle}}{}{\\ \insertdate}
%\ifthenelse{\isundefined{\coursename}}{}{\\ \coursename} % put in title instead.
}
{4ex}
{\color{DarkOliveGreen}{\titlerule}\color{Maroon}
\vspace{2ex}%
\filright}
[\vspace{2ex}%
\color{DarkOliveGreen}\titlerule
]

\newcommand{\beginArtWithToc}[0]{\begin{document}\tableofcontents}
\newcommand{\beginArtNoToc}[0]{\begin{document}}
\newcommand{\EndNoBibArticle}[0]{\end{document}}
\newcommand{\EndArticle}[0]{\bibliography{Bibliography}\bibliographystyle{plainnat}\end{document}}

% 
%\newcommand{\citep}[1]{\cite{#1}}

\colorSectionsForArticle


%
%\beginArtNoToc
%
%\generatetitle{Fresnel form approximate solution to paraxial wave equation}
%%\chapter{Fresnel form approximate solution to paraxial wave equation}

\makeproblem{Apply the paraxial wave equation operator to a Fresnel approximate form}{pr:paraxialWaveApproximateSolution:1}{

In some supplementary class notes, it is stated that

\begin{dmath}\label{eqn:paraxialWaveApproximateSolution:1}
h(x, y, z) = \inv{z} e^{i k z} e^{i k (x^2 + y^2)/2z }
\end{dmath}

is an exact solution to the \underlineAndIndex{paraxial wave equation}

\begin{dmath}\label{eqn:paraxialWaveApproximateSolution:10}
\spacegrad_{\mathrm{T}}^2 u + 2 i k \PD{z}{u} = 0.
\end{dmath}

From our lectures, this doesn't seem possible, since we found that this Fresnel like function was an approximation to the \(u_{00}\) function for large \(z\).  Calculate this directly and verify this suspicion.

} % makeproblem

\makeanswer{pr:paraxialWaveApproximateSolution:1}{

Let's first apply the \(\partial_{xx}\) portion of the transverse Laplacian.  We find

\begin{dmath}\label{eqn:paraxialWaveApproximateSolution:30}
\PDSq{x}{h}
=
\PD{x}{} \PD{x}{} \left(
\inv{z} e^{i k z} e^{ i k (x^2 + y^2)/2z }
\right)
=
\inv{z} e^{i k z}
e^{ i k y^2 /2z }
\PD{x}{} \PD{x}{} \left(
e^{ i k x^2 /2z }
\right)
=
\inv{z} e^{i k z}
e^{ i k y^2 /2z }
\PD{x}{} \left(
\frac{i k x}{z} e^{ i k x^2 /2z }
\right)
=
\inv{z} e^{i k z}
e^{ i k y^2 /2z }
\left(
\frac{i k }{z}
-\frac{k^2 x^2}{z^2}
\right)
e^{ i k x^2 /2z }
=
\left(
\frac{i k }{z}
-\frac{k^2 x^2}{z^2}
\right)
h
\end{dmath}

This gives us, for \(r^2 = x^2 + y^2\)

\begin{dmath}\label{eqn:paraxialWaveApproximateSolution:50}
\spacegrad_{\mathrm{T}}^2 h =
\left(
\frac{2 i k }{z}
-\frac{k^2 r^2}{z^2}
\right)
h.
\end{dmath}

For the first partial with respect to \(z\) we find

\begin{dmath}\label{eqn:paraxialWaveApproximateSolution:70}
\PD{z}{h} =
-\inv{z^2} e^{i k z} e^{ i k r^2 /2z }
+
\inv{z}
\left(
i k - \frac{i k r^2}{2 z^2}
\right)
e^{i k z} e^{ i k r^2 /2z }
=
\left(
- \inv{z} +
i k - \frac{i k r^2}{2 z^2}
\right) h
\end{dmath}

Putting things together we have

\begin{dmath}\label{eqn:paraxialWaveApproximateSolution:90}
\left( \spacegrad_{\mathrm{T}}^2 + 2 i k \PD{z}{} \right) h
=
\left(
\cancel{
\frac{2 i k }{z}
}
-
\cancel{\frac{k^2 r^2}{z^2} }
+
2 i k \left(
\cancel{
- \inv{z}
} +
i k -
\cancel{
\frac{i k r^2}{2 z^2}
}
\right)
\right)
h
=
- 2 k^2 h \ne 0
\end{dmath}

However, since \(h \rightarrow 0\) as \(z \rightarrow \infty\), this does at least give zero in the far \(z\) limit.
} % makeanswer

%\EndArticle
%\EndNoBibArticle
