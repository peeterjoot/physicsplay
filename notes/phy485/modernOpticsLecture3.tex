%
% Copyright � 2012 Peeter Joot.  All Rights Reserved.
% Licenced as described in the file LICENSE under the root directory of this GIT repository.
%
%\newcommand{\authorname}{Peeter Joot}
\newcommand{\email}{peeterjoot@protonmail.com}
\newcommand{\basename}{FIXMEbasenameUndefined}
\newcommand{\dirname}{notes/FIXMEdirnameUndefined/}

%\renewcommand{\basename}{modernOpticsLecture3}
%\renewcommand{\dirname}{notes/phy485/}
%\newcommand{\keywords}{Eikonal equation, ray equation, Optics, PHY485H1F}
%\newcommand{\authorname}{Peeter Joot}
\newcommand{\onlineurl}{http://sites.google.com/site/peeterjoot2/math2013/\basename.pdf}
\newcommand{\sourcepath}{\dirname\basename.tex}
\newcommand{\generatetitle}[1]{\chapter{#1}}

\newcommand{\vcsinfo}{%
\section*{}
\noindent{\color{DarkOliveGreen}{\rule{\linewidth}{0.1mm}}}
\paragraph{Document version}
%\paragraph{\color{Maroon}{Document version}}
{
\small
\begin{itemize}
\item Available online at:\\ 
\href{\onlineurl}{\onlineurl}
\item Git Repository: \input{./.revinfo/gitRepo.tex}
\item Source: \sourcepath
\item last commit: \input{./.revinfo/gitCommitString.tex}
\item commit date: \input{./.revinfo/gitCommitDate.tex}
\end{itemize}
}
}

%\PassOptionsToPackage{dvipsnames,svgnames}{xcolor}
\PassOptionsToPackage{square,numbers}{natbib}
\documentclass{scrreprt}

\usepackage[left=2cm,right=2cm]{geometry}
\usepackage[svgnames]{xcolor}
\usepackage{peeters_layout}

\usepackage{natbib}

\usepackage[
colorlinks=true,
bookmarks=false,
pdfauthor={\authorname, \email},
backref 
]{hyperref}

% http://tex.stackexchange.com/questions/75773/how-to-reference-problems-by-the-text-label-in-an-exercise-envioronment
\usepackage[english]{cleveref}
\crefname{Exercise}{exercise}{exercises}
\Crefname{Exercise}{Exercise}{Exercises}

\RequirePackage{titlesec}
\RequirePackage{ifthen}

% http://stackoverflow.com/questions/4932910/date-in-the-tabular-environment
\makeatletter
\let\insertdate\@date
\makeatother

\titleformat{\chapter}[display]
{\bfseries\Large}
{\color{DarkSlateGrey}\filleft \authorname
\ifthenelse{\isundefined{\studentnumber}}{}{\\ \studentnumber}
\ifthenelse{\isundefined{\email}}{}{\\ \email}
\ifthenelse{\isundefined{\dateintitle}}{}{\\ \insertdate}
%\ifthenelse{\isundefined{\coursename}}{}{\\ \coursename} % put in title instead.
}
{4ex}
{\color{DarkOliveGreen}{\titlerule}\color{Maroon}
\vspace{2ex}%
\filright}
[\vspace{2ex}%
\color{DarkOliveGreen}\titlerule
]

\newcommand{\beginArtWithToc}[0]{\begin{document}\tableofcontents}
\newcommand{\beginArtNoToc}[0]{\begin{document}}
\newcommand{\EndNoBibArticle}[0]{\end{document}}
\newcommand{\EndArticle}[0]{\bibliography{Bibliography}\bibliographystyle{plainnat}\end{document}}

% 
%\newcommand{\citep}[1]{\cite{#1}}

\colorSectionsForArticle


%\beginArtNoToc
%\generatetitle{PHY485H1F Modern Optics.  Lecture 3: GRIN (Graded Refractive INdex).  Taught by Prof.\ Joseph Thywissen}
\label{chap:modernOpticsLecture3}

%\section{Disclaimer}
%
%Peeter's lecture notes from class.  May not be entirely coherent.
%
\section{Gradium Lens}

We'll use Fermat's theorem

\makedefinition{Fermat's theorem}{dfn:modernOpticsLecture3:10}{
The pathlength is the same for all rays.
}

Looking to \cref{fig:modernOpticsLecture3:modernOpticsLecture3Fig1}.
\imageFigure{figures/modernOpticsLecture3Fig1}{Gradium lens}{fig:modernOpticsLecture3:modernOpticsLecture3Fig1}{0.2}

we write

\begin{dmath}\label{eqn:modernOpticsLecture3:310}
\text{edge} = n(\Br) d + \sqrt{r^2 + f^2}
\end{dmath}
\begin{dmath}\label{eqn:modernOpticsLecture3:330}
\text{center} = n_0 d + f.
\end{dmath}

Employing the paraxial approximation $r << f$ we have

\begin{dmath}\label{eqn:modernOpticsLecture3:350}
\sqrt{r^2 + f^2} \approx f \left( 1 + \inv{2} \frac{r^2}{f^2} \right)
\end{dmath}

so that the edge is

\begin{dmath}\label{eqn:modernOpticsLecture3:370}
n(\Br) d + \sqrt{r^2 + f^2} \approx n(\Br) d + f,
\end{dmath}

and

\begin{dmath}\label{eqn:modernOpticsLecture3:390}
n(\Br) = n_0 - \inv{2} \frac{r^2}{f d}
\end{dmath}

\section{GRIN (Graded Refractive Index) optics}

We will now consider a cylindrical system as illustrated in \cref{fig:modernOpticsLecture3:modernOpticsLecture3Fig2}.
\imageFigure{figures/modernOpticsLecture3Fig2}{Cylindrical coordinates for the problem}{fig:modernOpticsLecture3:modernOpticsLecture3Fig2}{0.2}

and look at ray path for 

\begin{dmath}\label{eqn:modernOpticsLecture3:10}
n(\rho) = n_0 - \alpha \rho^2/2.
\end{dmath}

We seek a relationship as potentially illustrated in \cref{fig:modernOpticsLecture3:modernOpticsLecture3Fig3}.
\imageFigure{figures/modernOpticsLecture3Fig3}{Arc length}{fig:modernOpticsLecture3:modernOpticsLecture3Fig3}{0.2}

where 

\begin{dmath}\label{eqn:modernOpticsLecture3:30}
ds = dz \sqrt{ 1 + \Abs{ \ddz{\rho} }^2} \sim dz,
\end{dmath}

for the paraxial approximation.

\paragraph{Approach} Eikonal Equation (Ray equation).

\begin{dmath}\label{eqn:modernOpticsLecture3:50}
\dds{} \left(  n(\Br) \dds{\Br} \right) = \spacegrad n
\end{dmath}

paraxial

\begin{dmath}\label{eqn:modernOpticsLecture3:70}
\ddz{} ( (n_0 - \cancel{\alpha \rho^2/2} ) \ddz{\Brho} = \spacegrad ( n_0 - \alpha/2 \rho^2 ) = - \alpha \Brho
\end{dmath}

We have a SHO

\begin{dmath}\label{eqn:modernOpticsLecture3:90}
\boxed{
n_0 \frac{d^2}{dz^2} \Brho = - \alpha \Brho
}
\end{dmath}

Let's write this out

\begin{dmath}\label{eqn:modernOpticsLecture3:110}
\Brho(z) = 
\begin{bmatrix}
x(z) \\
y(z) \\
\end{bmatrix}
=
\begin{bmatrix}
A \cos( \sqrt{\frac{\alpha}{n_0}} z ) + B \sin( \sqrt{\frac{\alpha}{n_0}} z) \\
C \cos( \sqrt{\frac{\alpha}{n_0}} z ) + D \sin( \sqrt{\frac{\alpha}{n_0}} z) \\
\end{bmatrix}
\end{dmath}

We have 4 constants determined by the initial conditions 

\begin{dmath}\label{eqn:modernOpticsLecture3:130}
\Brho(0) = 
\begin{bmatrix}
A \\
C
\end{bmatrix}
\end{dmath}

\begin{dmath}\label{eqn:modernOpticsLecture3:150}
\evalbar{\ddz{\Br}}{0} = 
\sqrt{\frac{\alpha}{n_0}}
\begin{bmatrix}
B \\
D
\end{bmatrix},
\end{dmath}

as illustrated in \cref{fig:modernOpticsLecture3:modernOpticsLecture3Fig4}.
\imageFigure{figures/modernOpticsLecture3Fig4}{Nodal distribution}{fig:modernOpticsLecture3:modernOpticsLecture3Fig4}{0.2}

If the length is an integer $L$ as in \cref{fig:modernOpticsLecture3:modernOpticsLecture3Fig5}.
\imageFigure{figures/modernOpticsLecture3Fig5}{first order solution}{fig:modernOpticsLecture3:modernOpticsLecture3Fig5}{0.2}

If the length is a half integer $L$ as in \cref{fig:modernOpticsLecture3:modernOpticsLecture3Fig6}.
\imageFigure{figures/modernOpticsLecture3Fig6}{second order solution}{fig:modernOpticsLecture3:modernOpticsLecture3Fig6}{0.2}

If $\text{length} = \left(n + \inv{4} \right)L$ as in \cref{fig:modernOpticsLecture3:modernOpticsLecture3Fig7}.
\imageFigure{figures/modernOpticsLecture3Fig7}{third order solution}{fig:modernOpticsLecture3:modernOpticsLecture3Fig7}{0.2}

It was mentioned that this solved a problem with regular fiber optic cables illustrated in \cref{fig:modernOpticsLecture3:modernOpticsLecture3Fig8}

\imageFigure{figures/modernOpticsLecture3Fig8}{regular fiber effects}{fig:modernOpticsLecture3:modernOpticsLecture3Fig8}{0.2}

so that for the GRIN configuration we have something more like \cref{fig:modernOpticsLecture3:modernOpticsLecture3Fig9}.
\imageFigure{figures/modernOpticsLecture3Fig9}{step index fiber}{fig:modernOpticsLecture3:modernOpticsLecture3Fig9}{0.2}

FIXME: It wasn't clear to me exactly what was being described (except that it was some sort of end phenomena).

\section{Applications of GRIN fibers and Fermat's theorem}

Are path lengths equal?  Instead of dropping all but the $dz$ term in our $ds$ approximation above (FIXME: ref), how about we retain the first order Taylor expansion

\begin{dmath}\label{eqn:modernOpticsLecture3:170}
T 
= \int_0^L \frac{ds}{c/n(\Br)} 
= \int_0^L dz \left( 1 + \inv{2} \Abs{ \ddz{\Brho}}^2 \right) \left(n_0 - \frac{\alpha}{2} \rho^2 \right)
\approx \frac{n_0}{c} \int_0^L dz \left( 1 + \inv{2} \Abs{ \ddz{\Brho} }^2 - \frac{\alpha}{2 n_0} \rho^2 \right) + O(\text{higher order corrections})
\end{dmath}

\begin{dmath}\label{eqn:modernOpticsLecture3:190}
\ddz{} \left( \Brho \cdot \ddz{\Brho} \right) 
=
\Abs{ \ddz{\Brho} }^2 + \rho \cdot \frac{d^2 \Brho }{d^2 z}
\end{dmath}

so 

\begin{dmath}\label{eqn:modernOpticsLecture3:210}
T 
= \frac{n_0}{c} \int_0^L dz 
\left( 1 
       + \inv{2} \ddz{} \left( \Brho \cdot \ddz{\Brho} 
\right) 
       - \rho \cdot \frac{d^2 \Brho }{d^2 z}
       -\inv{2} \frac{\alpha}{n_0} \rho^2
\right) 
\end{dmath}

But
\begin{dmath}\label{eqn:modernOpticsLecture3:230}
- \rho \cdot \frac{d^2 \Brho }{d^2 z}
-\inv{2} \frac{\alpha}{n_0} \rho^2
=
-\inv{2} \underbrace{\left( \frac{d^2}{dz^2} \Brho - \alpha/n_0 \Brho \right)}_{ = 0 \text{by Eikonal}} \cdot \Brho
\end{dmath}

so 
\begin{dmath}\label{eqn:modernOpticsLecture3:250}
T 
= \frac{n_0 L}{c} 
+ \underbrace{\evalrange{ \Brho \cdot \ddz{\Brho} }{0}{L}}_{\mbox{$= 0$ if refocused}}
\end{dmath}

Here if refocused means $\Brho = 0$ at both sides.

\paragraph{Reading}: \S 3 of \citep{born1980principles} for details on this topic.

Ray equation gives paths of stationary action 

In general our action is 

\begin{dmath}\label{eqn:modernOpticsLecture3:270}
S = \int_{t_1}^{t_2} dt \LL
\end{dmath}

where $\LL$ is the Lagrangian.  We recall Hamilton's principle which states that if 

\begin{dmath}\label{eqn:modernOpticsLecture3:290}
\delta S = 0,
\end{dmath}

(a path variation as illustrated in \cref{fig:modernOpticsLecture3:modernOpticsLecture3Fig10}

\imageFigure{figures/modernOpticsLecture3Fig10}{Action minimization}{fig:modernOpticsLecture3:modernOpticsLecture3Fig10}{0.2}

), then the statement \cref{eqn:modernOpticsLecture3:290} gives us Hamiltonian dynamics (Hamilton's equations).

Why does this work?  One explanation is that we have in quantum mechanics the most general action

\begin{dmath}\label{eqn:modernOpticsLecture3:410}
amplitude (path) = \exp( i S[path]/\hbar ),
\end{dmath}

This is called Feynman's Path Integral.

We have a Lagrangian for electromagnetism

\begin{dmath}\label{eqn:modernOpticsLecture3:430}
\LL_{\text{EM}} = \inv{c} \rho \phi + \inv{c} \Bj \cdot \BA + \inv{8 \pi} \BE^2 - \inv{8 \pi} \BB^2.
\end{dmath}

Using this we can derive Maxwell's equations.

As a problem we are going to calculate the amplitude for the Cornu spiral \cref{fig:modernOpticsLecture3:modernOpticsLecture3Fig11}.

\imageFigure{figures/modernOpticsLecture3Fig11}{Cornu spiral path of interest}{fig:modernOpticsLecture3:modernOpticsLecture3Fig11}{0.2}

Propose some experiments

1) Block the primary path \cref{fig:modernOpticsLecture3:modernOpticsLecture3Fig12}.

\imageFigure{figures/modernOpticsLecture3Fig12}{Single slit}{fig:modernOpticsLecture3:modernOpticsLecture3Fig12}{0.2}

2) Block 2 paths \cref{fig:modernOpticsLecture3:modernOpticsLecture3Fig13}.

\imageFigure{figures/modernOpticsLecture3Fig13}{Double slit}{fig:modernOpticsLecture3:modernOpticsLecture3Fig13}{0.2}

3) Block half paths \cref{fig:modernOpticsLecture3:modernOpticsLecture3Fig14}.

\imageFigure{figures/modernOpticsLecture3Fig14}{Wall blocking half path}{fig:modernOpticsLecture3:modernOpticsLecture3Fig14}{0.2}

4) Allow paths with the same phase.

\Cref{fig:modernOpticsLecture3:modernOpticsLecture3Fig15}.
\imageFigure{figures/modernOpticsLecture3Fig15}{Many slits}{fig:modernOpticsLecture3:modernOpticsLecture3Fig15}{0.2}

We'll see that the Action principle does in fact provide us all the real physical effects (Fresnel diffraction, multi-slit diffraction, ...)
%\vcsinfo
%\EndArticle
