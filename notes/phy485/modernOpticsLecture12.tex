%
% Copyright � 2012 Peeter Joot.  All Rights Reserved.
% Licenced as described in the file LICENSE under the root directory of this GIT repository.
%
%\newcommand{\authorname}{Peeter Joot}
\newcommand{\email}{peeterjoot@protonmail.com}
\newcommand{\basename}{FIXMEbasenameUndefined}
\newcommand{\dirname}{notes/FIXMEdirnameUndefined/}

%\renewcommand{\basename}{modernOpticsLecture12}
%\renewcommand{\dirname}{notes/phy485/}
%\newcommand{\keywords}{Optics, PHY485H1F}
%\newcommand{\authorname}{Peeter Joot}
\newcommand{\onlineurl}{http://sites.google.com/site/peeterjoot2/math2013/\basename.pdf}
\newcommand{\sourcepath}{\dirname\basename.tex}
\newcommand{\generatetitle}[1]{\chapter{#1}}

\newcommand{\vcsinfo}{%
\section*{}
\noindent{\color{DarkOliveGreen}{\rule{\linewidth}{0.1mm}}}
\paragraph{Document version}
%\paragraph{\color{Maroon}{Document version}}
{
\small
\begin{itemize}
\item Available online at:\\ 
\href{\onlineurl}{\onlineurl}
\item Git Repository: \input{./.revinfo/gitRepo.tex}
\item Source: \sourcepath
\item last commit: \input{./.revinfo/gitCommitString.tex}
\item commit date: \input{./.revinfo/gitCommitDate.tex}
\end{itemize}
}
}

%\PassOptionsToPackage{dvipsnames,svgnames}{xcolor}
\PassOptionsToPackage{square,numbers}{natbib}
\documentclass{scrreprt}

\usepackage[left=2cm,right=2cm]{geometry}
\usepackage[svgnames]{xcolor}
\usepackage{peeters_layout}

\usepackage{natbib}

\usepackage[
colorlinks=true,
bookmarks=false,
pdfauthor={\authorname, \email},
backref 
]{hyperref}

% http://tex.stackexchange.com/questions/75773/how-to-reference-problems-by-the-text-label-in-an-exercise-envioronment
\usepackage[english]{cleveref}
\crefname{Exercise}{exercise}{exercises}
\Crefname{Exercise}{Exercise}{Exercises}

\RequirePackage{titlesec}
\RequirePackage{ifthen}

% http://stackoverflow.com/questions/4932910/date-in-the-tabular-environment
\makeatletter
\let\insertdate\@date
\makeatother

\titleformat{\chapter}[display]
{\bfseries\Large}
{\color{DarkSlateGrey}\filleft \authorname
\ifthenelse{\isundefined{\studentnumber}}{}{\\ \studentnumber}
\ifthenelse{\isundefined{\email}}{}{\\ \email}
\ifthenelse{\isundefined{\dateintitle}}{}{\\ \insertdate}
%\ifthenelse{\isundefined{\coursename}}{}{\\ \coursename} % put in title instead.
}
{4ex}
{\color{DarkOliveGreen}{\titlerule}\color{Maroon}
\vspace{2ex}%
\filright}
[\vspace{2ex}%
\color{DarkOliveGreen}\titlerule
]

\newcommand{\beginArtWithToc}[0]{\begin{document}\tableofcontents}
\newcommand{\beginArtNoToc}[0]{\begin{document}}
\newcommand{\EndNoBibArticle}[0]{\end{document}}
\newcommand{\EndArticle}[0]{\bibliography{Bibliography}\bibliographystyle{plainnat}\end{document}}

% 
%\newcommand{\citep}[1]{\cite{#1}}

\colorSectionsForArticle


%\beginArtNoToc
%\generatetitle{PHY485H1F Modern Optics.  Lecture 12: Multiple interference.  Taught by Prof.\ Joseph Thywissen}
%%\chapter{Multiple interference}
\index{multiple interference}
%\label{chap:modernOpticsLecture12}

What if

\begin{dmath}\label{eqn:modernOpticsLecture12:10}
\Psi = \Psi_1 + \Psi_2 + \Psi_3 + \Psi_4 + \cdots
\end{dmath}

then

\begin{dmath}\label{eqn:modernOpticsLecture12:30}
I
= \expectation{\Abs{\Psi}^2}
=
\expectation{
\left( \sum_i \Psi_i^\conj \right)
\left( \sum_j \Psi_j \right)
}
=
\sum_{i,j} \expectation{ \Psi_i^\conj \Psi_j }
=
\sum_{i} \expectation{ \Psi_i^\conj \Psi_i }
+\sum_{i> j} \expectation{
\Psi_i^\conj \Psi_j
+\Psi_j^\conj \Psi_i
}
=
\text{incoherent sum} + \text{ interference term}
\end{dmath}

\begin{subequations}
\begin{dmath}\label{eqn:modernOpticsLecture12:50}
\text{incoherent sum}
=
\sum_i I_i
\end{dmath}
\begin{dmath}\label{eqn:modernOpticsLecture12:70}
\text{interference sum}
=
2 \Real
\left(  \sum_{i > j} \expectation{ \Psi_i^\conj \Psi_j }
\right)
=
2 \Real
\left(  \sum_{i > j} \Gamma_{ij}
\right)
\end{dmath}
\end{subequations}

We recognize our mutual coherence in the interference term.

Now consider a partially silvered mirror configuration with two mirrors as in \cref{fig:modernOpticsLecture12:modernOpticsLecture12Fig1}.

\imageFigure{../../figures/phy485/modernOpticsLecture12Fig1}{Two partially silvered mirror configuration with thickness}{fig:modernOpticsLecture12:modernOpticsLecture12Fig1}{0.3}

We are going to ignore the thickness and assume the reflection and transmission coefficients are the same for both surfaces and assume that we have no absorption and only treat the \(n = 1\) everywhere case for now.  We'll look at the interference of all the internal reflections on eventual exit from the pair of mirrors as shown in \cref{fig:modernOpticsLecture12:modernOpticsLecture12Fig2}.

\imageFigure{../../figures/phy485/modernOpticsLecture12Fig2}{Ignoring thickness}{fig:modernOpticsLecture12:modernOpticsLecture12Fig2}{0.3}

We'll want to remember the phase.  What is the phase delay between each interfering wave?

We'll find that we get an extra path length of

\begin{dmath}\label{eqn:modernOpticsLecture12:90}
2 k L \cos\theta
\end{dmath}

The geometry to consider is \cref{fig:modernOpticsLecture12:modernOpticsLecture12Fig3}.

\imageFigure{../../figures/phy485/modernOpticsLecture12Fig3}{Just the geometry of the problem}{fig:modernOpticsLecture12:modernOpticsLecture12Fig3}{0.3}

We see that we have

\begin{subequations}
\begin{dmath}\label{eqn:modernOpticsLecture12:110}
s_1 \cos\theta = L
\end{dmath}
\begin{dmath}\label{eqn:modernOpticsLecture12:130}
\frac{\Delta y}{2} = s_1 \sin\theta
\end{dmath}
\begin{dmath}\label{eqn:modernOpticsLecture12:150}
\Bk = k (\cos\theta, \sin\theta, 0),
\end{dmath}
\end{subequations}

So that our phase change to the point \(A\) where we have the second internal reflection

\begin{dmath}\label{eqn:modernOpticsLecture12:170}
\Bk \cdot \Bx - \omega t
=
k (\cos\theta, \sin\theta, 0) \cdot (0, \Delta y, 0) - \omega \frac{2 s_1}{c}
=
k \sin\theta \Delta y - \cancel{c} k \frac{2 s_1}{\cancel{c}}
=
k (\sin\theta \Delta y - 2 s_1)
=
k (2 s_1 \sin^2\theta - 2 s_1)
=
2 k s_1 (\sin^2\theta - 1)
=
-2 k s_1 \cos^2\theta
=
-2 k L \cos\theta
\end{dmath}

Observe that \(\Bk \cdot \Delta \Bx - \omega \Delta t\) must be negative for any non-straight line path (in which case it will be zero) between endpoints, provided the media through which the rays travel is of constant index of refraction.  In class (and the class notes) there was no such negative sign, and we just considered the absolute difference in phase.  This can also be calculated by considering just the contributing portions of the path that lead to interference.  Fowles' fig 4.2 marks those as \(AB\), \(BC\), or \(s_1 \cos(2\theta) + s_1\) in the figure above.  The idea is that we consider the second reflection as a generator of plane waves, and those will only start interfering with plane waves at the first transmission (in steady state), after those first transmission waves have traversed that little leg of the path \(s_1 - s_1 \cos(2 \theta)\).  This was illustrated in office hours as in \cref{fig:modernOpticsLecture12:modernOpticsLecture12Fig4}.

\imageFigure{../../figures/phy485/modernOpticsLecture12Fig4}{Internal interference regions of the path}{fig:modernOpticsLecture12:modernOpticsLecture12Fig4}{0.3}

Of this Prof Thywissen also says:  We normally discuss paths as a ``time delay''. Longer paths have longer delays. Since in the convention that you and I are using, time enters as \(e^{-i \omega t}\), this means that multiplying by \(e^{i \text{phase}}\) gives you an effective time delay of \(t = - \text{phase}/\omega\).  So, this brings us back to the conclusion that positive path-length giving a negative phase in the exponent is self-consistent.

Anyways, moving on, we get to the point where the wavefunction for transmission is

\begin{dmath}\label{eqn:modernOpticsLecture12:190}
\Psi_{\mathrm{transmission}} =
\Psi_0 t^2
+
\Psi_0 t^2 \left( r^2 e^{i \delta} \right)
+
\Psi_0 t^2 \left( r^2 e^{i \delta} \right)^2
\end{dmath}

We lookup (in a spec sheet) the transmission and reflection coefficients and the (associated phase shifts after reflection)

\begin{subequations}
\begin{dmath}\label{eqn:modernOpticsLecture12:210}
r = e^{i \delta_r} \sqrt{R}
\end{dmath}
\begin{dmath}\label{eqn:modernOpticsLecture12:230}
t = e^{i \delta_t} \sqrt{T}
\end{dmath}
\end{subequations}

and get

\begin{dmath}\label{eqn:modernOpticsLecture12:250}
\Psi_{\mathrm{transmission}} =
\Psi_0 T e^{2 i \delta_t}
+
\Psi_0 T e^{2 i \delta_t} R e^{2 i \delta_r + i \delta}
+
\Psi_0 T e^{2 i \delta_t} \left( R e^{2 i \delta_r + i \delta} \right)^2
+ \cdots
=
\Psi_0 t^2 \sum_{n = 0}^\infty \left( R e^{ 2 i \delta_r + i \delta} \right)^n
=
\Psi_0 t^2 \inv{ 1 - R e^{i \Delta}}
\end{dmath}

where we've used (for \(\Abs{a} < 1\))

\begin{dmath}\label{eqn:modernOpticsLecture12:270}
\sum_{n = 0}^\infty a^n = \inv{1 - a}
\end{dmath}

and written

\begin{dmath}\label{eqn:modernOpticsLecture12:290}
\Delta = 2 \delta_r + \delta
\end{dmath}

Our measured intensity is

\begin{dmath}\label{eqn:modernOpticsLecture12:350}
I_{\mathrm{trans}}
=
\expectation{
\Abs{ \psi_{\mathrm{trans}} }^2
}
=
I_0 \frac{T^2}{\Abs{ 1 - R e^{i \Delta} }^2}
=
I_0 \frac{T^2}{
(1 - R e^{i \Delta})
(1 - R e^{-i \Delta})
}
=
I_0 \frac{T^2}{
1 + R^2 - 2 R \cos \Delta
}
=
I_0 \frac{T^2}{
1 + R^2 - 2 R (1 - 2 \sin^2(\Delta/2))
}
=
I_0 \frac{T^2}{
(1 - R)^2 + 4 R \sin^2(\Delta/2)
}
=
I_0 \frac{T^2/(1 - R)^2}{
1 + (4 R/(1 - R)^2) \sin^2(\Delta/2)
}
\end{dmath}

or

\begin{subequations}
\begin{dmath}\label{eqn:modernOpticsLecture12:370}
I_{\mathrm{trans}}
=
\frac{I_{\mathrm{max}}}{1 + F \sin^2 (\Delta/2)}
\end{dmath}
\begin{dmath}\label{eqn:modernOpticsLecture12:310}
I_{\mathrm{max}} = \frac{I_0 T^2}{(1 -R)^2}
\end{dmath}
\begin{dmath}\label{eqn:modernOpticsLecture12:330}
F = \frac{4 R}{(1 - R)^2}.
\end{dmath}
\end{subequations}

This is called \underlineAndIndex{Etalon transmission}.  Plots of \(I/I_{\mathrm{max}}\) vs. phase shifts for \(R \in \{0.1, 0.3, 0.6, 0.97\}\) can be found in \cref{fig:etalon:etalonFig1}.

\imageFigure{../../figures/phy485/etalonFig1}{Etalon transmission}{fig:etalon:etalonFig1}{0.3}

%\EndArticle
%\EndNoBibArticle
