%
% Copyright � 2013 Peeter Joot.  All Rights Reserved.
% Licenced as described in the file LICENSE under the root directory of this GIT repository.
%
\newcommand{\authorname}{Peeter Joot}
\newcommand{\email}{peeterjoot@protonmail.com}
\newcommand{\basename}{FIXMEbasenameUndefined}
\newcommand{\dirname}{notes/FIXMEdirnameUndefined/}

\renewcommand{\basename}{numberGame}
\renewcommand{\dirname}{notes/incoherentramblings/}
%\newcommand{\dateintitle}{}
%\newcommand{\keywords}{}

\newcommand{\authorname}{Peeter Joot}
\newcommand{\onlineurl}{http://sites.google.com/site/peeterjoot2/math2013/\basename.pdf}
\newcommand{\sourcepath}{\dirname\basename.tex}
\newcommand{\generatetitle}[1]{\chapter{#1}}

\newcommand{\vcsinfo}{%
\section*{}
\noindent{\color{DarkOliveGreen}{\rule{\linewidth}{0.1mm}}}
\paragraph{Document version}
%\paragraph{\color{Maroon}{Document version}}
{
\small
\begin{itemize}
\item Available online at:\\ 
\href{\onlineurl}{\onlineurl}
\item Git Repository: \input{./.revinfo/gitRepo.tex}
\item Source: \sourcepath
\item last commit: \input{./.revinfo/gitCommitString.tex}
\item commit date: \input{./.revinfo/gitCommitDate.tex}
\end{itemize}
}
}

%\PassOptionsToPackage{dvipsnames,svgnames}{xcolor}
\PassOptionsToPackage{square,numbers}{natbib}
\documentclass{scrreprt}

\usepackage[left=2cm,right=2cm]{geometry}
\usepackage[svgnames]{xcolor}
\usepackage{peeters_layout}

\usepackage{natbib}

\usepackage[
colorlinks=true,
bookmarks=false,
pdfauthor={\authorname, \email},
backref 
]{hyperref}

% http://tex.stackexchange.com/questions/75773/how-to-reference-problems-by-the-text-label-in-an-exercise-envioronment
\usepackage[english]{cleveref}
\crefname{Exercise}{exercise}{exercises}
\Crefname{Exercise}{Exercise}{Exercises}

\RequirePackage{titlesec}
\RequirePackage{ifthen}

% http://stackoverflow.com/questions/4932910/date-in-the-tabular-environment
\makeatletter
\let\insertdate\@date
\makeatother

\titleformat{\chapter}[display]
{\bfseries\Large}
{\color{DarkSlateGrey}\filleft \authorname
\ifthenelse{\isundefined{\studentnumber}}{}{\\ \studentnumber}
\ifthenelse{\isundefined{\email}}{}{\\ \email}
\ifthenelse{\isundefined{\dateintitle}}{}{\\ \insertdate}
%\ifthenelse{\isundefined{\coursename}}{}{\\ \coursename} % put in title instead.
}
{4ex}
{\color{DarkOliveGreen}{\titlerule}\color{Maroon}
\vspace{2ex}%
\filright}
[\vspace{2ex}%
\color{DarkOliveGreen}\titlerule
]

\newcommand{\beginArtWithToc}[0]{\begin{document}\tableofcontents}
\newcommand{\beginArtNoToc}[0]{\begin{document}}
\newcommand{\EndNoBibArticle}[0]{\end{document}}
\newcommand{\EndArticle}[0]{\bibliography{Bibliography}\bibliographystyle{plainnat}\end{document}}

% 
%\newcommand{\citep}[1]{\cite{#1}}

\colorSectionsForArticle



%%%\usepackage{listings}
%%%\lstset{ %
%%%language=sh,                % choose the language of the code
%%%basicstyle=\footnotesize,       % the size of the fonts that are used for the code
%%%numbers=left,                   % where to put the line-numbers
%%%numberstyle=\footnotesize,      % the size of the fonts that are used for the line-numbers
%%%stepnumber=1,                   % the step between two line-numbers. If it is 1 each line will be numbered
%%%numbersep=5pt,                  % how far the line-numbers are from the code
%%%backgroundcolor=\color{white},  % choose the background color. You must add \usepackage{color}
%%%showspaces=false,               % show spaces adding particular underscores
%%%showstringspaces=false,         % underline spaces within strings
%%%showtabs=false,                 % show tabs within strings adding particular underscores
%%%frame=single,                   % adds a frame around the code
%%%tabsize=2,              % sets default tabsize to 2 spaces
%%%captionpos=b,                   % sets the caption-position to bottom
%%%breaklines=true,        % sets automatic line breaking
%%%breakatwhitespace=false,    % sets if automatic breaks should only happen at whitespace
%%%%escapeinside={(*@}{@*)},
%%%%alsodigit=-,
%%%%escapeinside={\%}{)}          % if you want to add a comment within your code
%%%}



% http://tex.stackexchange.com/questions/6827/minus-signs-vanish-with-listings-and-breqn-any-solutions
%\let\origlstlisting=\lstlisting
%\let\endoriglstlisting=\endlstlisting
%\renewenvironment{lstlisting}
%    {\mathcode`\-=\hyphenmathcode
%     \everymath{}\mathsurround=0pt\origlstlisting}
%    {\endoriglstlisting}


%\makeatletter
%\lst@AddToHook{Init}{\mathligsoff}
%\makeatother

%\lstnewenvironment{perl}[1][]
%  {\lstset{language=Perl}\lstset{%
%   #1
%}}
%{}

\beginArtNoToc

%\normalfont
%
%\makeatletter
%\texttt{\meaning\lst@InputCatcodes\relax}
%\makeatother

%\mathligsoff % http://tex.stackexchange.com/questions/184648/minus-sign-disappears-in-lstlisting-if-i-import-the-semantic-package

\generatetitle{Sum of digits of small powers of nine}
%\chapter{Sum of digits of small powers of nine}
%\label{chap:numberGame}

In a previous post I wondered how to prove that for integer \(d \in [1,N]\)

\begin{equation}\label{eqn:numberGame:20}
((N-1) d) \Mod N + ((N-1) d) \Div N = N-1.
\end{equation}

Here's a proof in two steps.  First for \(N = 10\), and then by search and replace for arbitrary \(N\).

\paragraph{\(N = 10\)}

Let

\begin{equation}\label{eqn:numberGame:40}
x = 9 d = 10 a + b,
\end{equation}

where \(1 \le a, b < 9\), and let

\begin{equation}\label{eqn:numberGame:180}
y = a + b,
\end{equation}

the sum of the digits in a base \(10\) numeral system.

We wish to solve the following integer system of equations

\begin{equation}\label{eqn:numberGame:60}
\begin{aligned}
9 d &= 10 a + b \\
y &= a + b \\
\end{aligned}.
\end{equation}

Scaling and subtracting we have

\begin{equation}\label{eqn:numberGame:80}
10 y - 9 d = 9 b,
\end{equation}

or

\begin{equation}\label{eqn:numberGame:100}
y = \frac{9}{10} \lr{ b + d }.
\end{equation}

Because \(y\) is an integer, we have to conclude that \(b + d\) is a power of \(10\), and \(b + d \ge 10\).  Because we have a constraint on the maximum value of this sum

\begin{equation}\label{eqn:numberGame:120}
b + d \le 2 ( 9 ),
\end{equation}

we can only conclude that

\begin{equation}\label{eqn:numberGame:140}
b + d = 10.
\end{equation}

or
\begin{equation}\label{eqn:numberGame:160}
\myBoxed{
b = 10 - d.
}
\end{equation}

Back substitution into \eqnref{eqn:numberGame:40} we have

\begin{equation}\label{eqn:numberGame:200}
\begin{aligned}
10 a
&= 9 d - b \\
&= 9 d - 10 + d \\
&= 10 d - 10 \\
&= 10 \lr{ d - 1 },
\end{aligned}
\end{equation}

or
\begin{equation}\label{eqn:numberGame:220}
\myBoxed{
a = d - 1.
}
\end{equation}

Summing \eqnref{eqn:numberGame:220} and \eqnref{eqn:numberGame:160}, the sum of digits is

\begin{equation}\label{eqn:numberGame:240}
a + b = d - 1 + 10 - d = 9.
\end{equation}

\paragraph{For arbitrary \(N\)}

There was really nothing special about \(9, 10\) in the above proof, so generalizing requires nothing more than some search and replace.  I used the following vim commands for this ``proof generalization''

\begin{verbatim}
:,/For arb/-1 y
:+/For arb/+1
:p
:,$ s/\<9\>/(N-1)/cg
:,$ s/\<10\>/N/cg
:,$ s/numberGame:/&2:/g
\end{verbatim}

Let

\begin{equation}\label{eqn:numberGame:2:40}
x = (N-1) d = N a + b,
\end{equation}

where \(1 \le a, b < N-1\), and let

\begin{equation}\label{eqn:numberGame:2:180}
y = a + b,
\end{equation}

the sum of the digits in a base \(N\) numeral system.

We wish to solve the following integer system of equations

\begin{equation}\label{eqn:numberGame:2:60}
\begin{aligned}
(N-1) d &= N a + b \\
y &= a + b \\
\end{aligned}.
\end{equation}

Scaling and subtracting we have

\begin{equation}\label{eqn:numberGame:2:80}
N y - (N-1) d = (N-1) b,
\end{equation}

or

\begin{equation}\label{eqn:numberGame:2:100}
y = \frac{N-1}{N} \lr{ b + d }.
\end{equation}

Because \(y\) is an integer, we have to conclude that \(b + d\) is a power of \(N\), and \(b + d \ge N\).  Because we have a constraint on the maximum value of this sum

\begin{equation}\label{eqn:numberGame:2:120}
b + d \le 2 ( N-1 ),
\end{equation}

we can only conclude that

\begin{equation}\label{eqn:numberGame:2:140}
b + d = N.
\end{equation}

or
\begin{equation}\label{eqn:numberGame:2:160}
\myBoxed{
b = N - d.
}
\end{equation}

Back substitution into \eqnref{eqn:numberGame:2:40} we have

\begin{equation}\label{eqn:numberGame:2:200}
\begin{aligned}
N a
&= (N-1) d - b \\
&= (N-1) d - N + d \\
&= N d - N \\
&= N \lr{ d - 1 },
\end{aligned}
\end{equation}

or
\begin{equation}\label{eqn:numberGame:2:220}
\myBoxed{
a = d - 1.
}
\end{equation}

Summing \eqnref{eqn:numberGame:2:220} and \eqnref{eqn:numberGame:2:160}, the sum of digits is

\begin{equation}\label{eqn:numberGame:2:260}
a + b = d - 1 + N - d = N-1.
\end{equation}

This completes the proof of \eqnref{eqn:numberGame:20}.

%\EndArticle
\EndNoBibArticle
