%
% Copyright � 2013 Peeter Joot.  All Rights Reserved.
% Licenced as described in the file LICENSE under the root directory of this GIT repository.
%
\newcommand{\authorname}{Peeter Joot}
\newcommand{\email}{peeterjoot@protonmail.com}
\newcommand{\basename}{FIXMEbasenameUndefined}
\newcommand{\dirname}{notes/FIXMEdirnameUndefined/}

\renewcommand{\basename}{calculusMadeEasy}
\renewcommand{\dirname}{notes/FIXMEwheretodirname/}
%\newcommand{\dateintitle}{}
%\newcommand{\keywords}{}

\newcommand{\authorname}{Peeter Joot}
\newcommand{\onlineurl}{http://sites.google.com/site/peeterjoot2/math2013/\basename.pdf}
\newcommand{\sourcepath}{\dirname\basename.tex}
\newcommand{\generatetitle}[1]{\chapter{#1}}

\newcommand{\vcsinfo}{%
\section*{}
\noindent{\color{DarkOliveGreen}{\rule{\linewidth}{0.1mm}}}
\paragraph{Document version}
%\paragraph{\color{Maroon}{Document version}}
{
\small
\begin{itemize}
\item Available online at:\\ 
\href{\onlineurl}{\onlineurl}
\item Git Repository: \input{./.revinfo/gitRepo.tex}
\item Source: \sourcepath
\item last commit: \input{./.revinfo/gitCommitString.tex}
\item commit date: \input{./.revinfo/gitCommitDate.tex}
\end{itemize}
}
}

%\PassOptionsToPackage{dvipsnames,svgnames}{xcolor}
\PassOptionsToPackage{square,numbers}{natbib}
\documentclass{scrreprt}

\usepackage[left=2cm,right=2cm]{geometry}
\usepackage[svgnames]{xcolor}
\usepackage{peeters_layout}

\usepackage{natbib}

\usepackage[
colorlinks=true,
bookmarks=false,
pdfauthor={\authorname, \email},
backref 
]{hyperref}

% http://tex.stackexchange.com/questions/75773/how-to-reference-problems-by-the-text-label-in-an-exercise-envioronment
\usepackage[english]{cleveref}
\crefname{Exercise}{exercise}{exercises}
\Crefname{Exercise}{Exercise}{Exercises}

\RequirePackage{titlesec}
\RequirePackage{ifthen}

% http://stackoverflow.com/questions/4932910/date-in-the-tabular-environment
\makeatletter
\let\insertdate\@date
\makeatother

\titleformat{\chapter}[display]
{\bfseries\Large}
{\color{DarkSlateGrey}\filleft \authorname
\ifthenelse{\isundefined{\studentnumber}}{}{\\ \studentnumber}
\ifthenelse{\isundefined{\email}}{}{\\ \email}
\ifthenelse{\isundefined{\dateintitle}}{}{\\ \insertdate}
%\ifthenelse{\isundefined{\coursename}}{}{\\ \coursename} % put in title instead.
}
{4ex}
{\color{DarkOliveGreen}{\titlerule}\color{Maroon}
\vspace{2ex}%
\filright}
[\vspace{2ex}%
\color{DarkOliveGreen}\titlerule
]

\newcommand{\beginArtWithToc}[0]{\begin{document}\tableofcontents}
\newcommand{\beginArtNoToc}[0]{\begin{document}}
\newcommand{\EndNoBibArticle}[0]{\end{document}}
\newcommand{\EndArticle}[0]{\bibliography{Bibliography}\bibliographystyle{plainnat}\end{document}}

% 
%\newcommand{\citep}[1]{\cite{#1}}

\colorSectionsForArticle



\beginArtNoToc

\generatetitle{Project gutenberg has ``Calculus Made Easy'' by Silvanus P. Thompson}
%\chapter{Project gutenberg has ``Calculus Made Easy'' by Silvanus P. Thompson}
\label{chap:calculusMadeEasy}

One of my favorite books \citep{thompson1914calculus}, a great little book that my grandfather gave me, is now available on project gutenburg (free ebooks transcribed from old out of print material).  Check out their \href{http://www.gutenberg.org/wiki/Mathematics_\%28Bookshelf\%29}{mathematics bookshelf}.

I'd seen this book recently in the Markham public library.  It's been republished with additions, but I didn't feel they really added anything.

It's interesting to see that this project also makes the tex sources available.  Because of that I can include the awesome prologue and first chapter from this text in this post.  Check it out.  Doesn't it whet your appetite for more calculus?

\section{Prologue}

Considering how many fools can calculate, it is
surprising that it should be thought either a difficult
or a tedious task for any other fool to learn how to
master the same tricks.

Some calculus-tricks are quite easy. Some are
enormously difficult. The fools who write the textbooks
of advanced mathematics---and they are mostly
clever fools---seldom take the trouble to show you how
easy the easy calculations are. On the contrary, they
seem to desire to impress you with their tremendous
cleverness by going about it in the most difficult way.

Being myself a remarkably stupid fellow, I have
had to unteach myself the difficulties, and now beg
to present to my fellow fools the parts that are not
hard. Master these thoroughly, and the rest will
follow. What one fool can do, another can.

\section{To deliver you from the Preliminary Terrors}

The preliminary terror, which chokes off most fifth-form
boys from even attempting to learn how to
calculate, can be abolished once for all by simply stating
what is the meaning---in common-sense terms---of the
two principal symbols that are used in calculating.

These dreadful symbols are:

(1) $d$ which merely means ``a little bit of.''

Thus $dx$ means a little bit of~$x$; or $du$ means a
little bit of~$u$. Ordinary mathematicians think it
more polite to say ``an element of,'' instead of ``a little
bit of.'' Just as you please. But you will find that
these little bits (or elements) may be considered to be
indefinitely small.

(2) $\int$ which is merely a long $S$, and may be called
(if you like) ``the sum of.''

Thus $\int dx$ means the sum of all the little bits
of~$x$; or $\int dt$ means the sum of all the little bits
of~$t$. Ordinary mathematicians call this symbol ``the
%\DPPageSep{014.png}{2}%
integral of.'' Now any fool can see that if $x$~is
considered as made up of a lot of little bits, each of
which is called~$dx$, if you add them all up together
you get the sum of all the~$dx$'s, (which is the same
thing as the whole of~$x$). The word ``integral'' simply
means ``the whole.'' If you think of the duration
of time for one hour, you may (if you like) think of
it as cut up into $3600$ little bits called seconds. The
whole of the $3600$ little bits added up together make
one hour.

When you see an expression that begins with this
terrifying symbol, you will henceforth know that it
is put there merely to give you instructions that you
are now to perform the operation (if you can) of
totalling up all the little bits that are indicated by
the symbols that follow.

That's all.

\EndArticle
%\EndNoBibArticle
