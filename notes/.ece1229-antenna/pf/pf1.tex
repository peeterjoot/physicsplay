%
% Copyright � 2015 Peeter Joot.  All Rights Reserved.
% Licenced as described in the file LICENSE under the root directory of this GIT repository.
%
\newcommand{\authorname}{Peeter Joot}
\newcommand{\email}{peeter.joot@utoronto.ca}
\newcommand{\studentnumber}{920798560}
\newcommand{\basename}{FIXMEbasenameUndefined}
\newcommand{\dirname}{notes/FIXMEdirnameUndefined/}

\renewcommand{\basename}{advancedantennaProblemSet1}
\renewcommand{\dirname}{notes/ece1229/}
\newcommand{\keywords}{Advanced Antenna Theory, ECE1229H}
\newcommand{\dateintitle}{}
\newcommand{\authorname}{Peeter Joot}
\newcommand{\onlineurl}{http://sites.google.com/site/peeterjoot2/math2013/\basename.pdf}
\newcommand{\sourcepath}{\dirname\basename.tex}
\newcommand{\generatetitle}[1]{\chapter{#1}}

\newcommand{\vcsinfo}{%
\section*{}
\noindent{\color{DarkOliveGreen}{\rule{\linewidth}{0.1mm}}}
\paragraph{Document version}
%\paragraph{\color{Maroon}{Document version}}
{
\small
\begin{itemize}
\item Available online at:\\ 
\href{\onlineurl}{\onlineurl}
\item Git Repository: \input{./.revinfo/gitRepo.tex}
\item Source: \sourcepath
\item last commit: \input{./.revinfo/gitCommitString.tex}
\item commit date: \input{./.revinfo/gitCommitDate.tex}
\end{itemize}
}
}

%\PassOptionsToPackage{dvipsnames,svgnames}{xcolor}
\PassOptionsToPackage{square,numbers}{natbib}
\documentclass{scrreprt}

\usepackage[left=2cm,right=2cm]{geometry}
\usepackage[svgnames]{xcolor}
\usepackage{peeters_layout}

\usepackage{natbib}

\usepackage[
colorlinks=true,
bookmarks=false,
pdfauthor={\authorname, \email},
backref 
]{hyperref}

% http://tex.stackexchange.com/questions/75773/how-to-reference-problems-by-the-text-label-in-an-exercise-envioronment
\usepackage[english]{cleveref}
\crefname{Exercise}{exercise}{exercises}
\Crefname{Exercise}{Exercise}{Exercises}

\RequirePackage{titlesec}
\RequirePackage{ifthen}

% http://stackoverflow.com/questions/4932910/date-in-the-tabular-environment
\makeatletter
\let\insertdate\@date
\makeatother

\titleformat{\chapter}[display]
{\bfseries\Large}
{\color{DarkSlateGrey}\filleft \authorname
\ifthenelse{\isundefined{\studentnumber}}{}{\\ \studentnumber}
\ifthenelse{\isundefined{\email}}{}{\\ \email}
\ifthenelse{\isundefined{\dateintitle}}{}{\\ \insertdate}
%\ifthenelse{\isundefined{\coursename}}{}{\\ \coursename} % put in title instead.
}
{4ex}
{\color{DarkOliveGreen}{\titlerule}\color{Maroon}
\vspace{2ex}%
\filright}
[\vspace{2ex}%
\color{DarkOliveGreen}\titlerule
]

\newcommand{\beginArtWithToc}[0]{\begin{document}\tableofcontents}
\newcommand{\beginArtNoToc}[0]{\begin{document}}
\newcommand{\EndNoBibArticle}[0]{\end{document}}
\newcommand{\EndArticle}[0]{\bibliography{Bibliography}\bibliographystyle{plainnat}\end{document}}

% 
%\newcommand{\citep}[1]{\cite{#1}}

\colorSectionsForArticle



\usepackage{peeters_layout_exercise}
\usepackage{ece1229}
\usepackage{siunitx}
\usepackage{esint} % \oiint

\renewcommand{\QuestionNB}{\alph{Question}.\ }
\renewcommand{\theQuestion}{\alph{Question}}

\beginArtNoToc
\generatetitle{ECE1229H Advanced Antenna Theory.  Problem Set 2: Fundamental parameters and Field radiation}
%\chapter{Fundamental parameters and Field radiation}
\label{chap:advancedantennaProblemSet1}

In Jackson's 'classical electrodynamics' he re-expresses a volume integral of a vector in terms of a moment like divergence:

\begin{equation*}\int \mathbf{J} d^3 x = - \int \mathbf{x} ( \boldsymbol{\nabla} \cdot \mathbf{J} ) d^3 x\end{equation*}

He calls this change "integration by parts".  If this is integration by parts, there must be some form of chain rule (where one of the terms is zero on the boundry), but I can't figure out what that chain rule would be.  I initially thought that the expansion of

\begin{equation*}\boldsymbol{\nabla} (\mathbf{x} \cdot \mathbf{J})\end{equation*}

might have the structure I was looking for (i.e. something like $\mathbf{x} \boldsymbol{\nabla} \cdot \mathbf{J}+\mathbf{J} \boldsymbol{\nabla} \cdot \mathbf{x}$), however

\begin{equation*}\boldsymbol{\nabla} (\mathbf{x} \cdot \mathbf{J})
=
\mathbf{x} \cdot \boldsymbol{\nabla} \mathbf{J}
+\mathbf{J} \cdot \boldsymbol{\nabla} \mathbf{x}
+
\mathbf{x} \times ( \boldsymbol{\nabla} \times \mathbf{J} )
= \mathbf{J} + \sum_a x_a \boldsymbol{\nabla} J_a.
\end{equation*}

I tried a few other gradients of various vector products (including $\boldsymbol{\nabla}  \times ( \mathbf{x}  \times \mathbf{J} )$), but wasn't able to figure out one that justifies what the author did with this integral.


\paragraph{reply}

I found a hint in Griffiths (problem 5.7) which poses the problem of relating the volume integral of $\BJ$ to the dipole moment using by expanding

\begin{equation}\label{eqn:t:20}
\int \spacegrad \cdot ( x \BJ ) d^3 x.
\end{equation}

That expansion is

\begin{equation}\label{eqn:t:40}
\int \spacegrad \cdot ( x \BJ ) d^3 x
= \int (\spacegrad x \cdot \BJ) d^3 x
+ \int x (\spacegrad \cdot \BJ) d^3 x.
\end{equation}

Doing the same for the other coordinates and summing gives

\begin{equation}\label{eqn:t:60}
\sum_{i = 1}^3 \Be_i \int \spacegrad \cdot ( x_i \BJ ) d^3 x
=
\int \BJ d^3 x
+ \int \Bx (\spacegrad \cdot \BJ) d^3 x.
\end{equation}

I think the boundary condition argument would be to transform the left hand side using the divergence theorem

\begin{equation}\label{eqn:t:80}
\int \BJ d^3 x
+ \int \Bx (\spacegrad \cdot \BJ) d^3 x
=
\sum_{i = 1}^3 \Be_i \int_{S} ( x_i \BJ ) \cdot \ncap dS
\end{equation}

and then argue that this is zero for localized-enough currents by taking this surface to infinity, where $\BJ$ is zero.

\EndNoBibArticle
