%
% Copyright � 2012 Peeter Joot.  All Rights Reserved.
% Licenced as described in the file LICENSE under the root directory of this GIT repository.
%

%
%
%%
% Copyright � 2015 Peeter Joot.  All Rights Reserved.
% Licenced as described in the file LICENSE under the root directory of this GIT repository.
%
\documentclass[]{eliblog}

\usepackage{amsmath}
\usepackage{mathpazo}

%
% shorthand for bold symbols, convenient for vectors and matrices
%
\newcommand{\Ba}[0]{\mathbf{a}}
\newcommand{\Bb}[0]{\mathbf{b}}
\newcommand{\Bc}[0]{\mathbf{c}}
\newcommand{\Bd}[0]{\mathbf{d}}
\newcommand{\Be}[0]{\mathbf{e}}
\newcommand{\Bf}[0]{\mathbf{f}}
\newcommand{\Bg}[0]{\mathbf{g}}
\newcommand{\Bh}[0]{\mathbf{h}}
\newcommand{\Bi}[0]{\mathbf{i}}
\newcommand{\Bj}[0]{\mathbf{j}}
\newcommand{\Bk}[0]{\mathbf{k}}
\newcommand{\Bl}[0]{\mathbf{l}}
\newcommand{\Bm}[0]{\mathbf{m}}
\newcommand{\Bn}[0]{\mathbf{n}}
\newcommand{\Bo}[0]{\mathbf{o}}
\newcommand{\Bp}[0]{\mathbf{p}}
\newcommand{\Bq}[0]{\mathbf{q}}
\newcommand{\Br}[0]{\mathbf{r}}
\newcommand{\Bs}[0]{\mathbf{s}}
\newcommand{\Bt}[0]{\mathbf{t}}
\newcommand{\Bu}[0]{\mathbf{u}}
\newcommand{\Bv}[0]{\mathbf{v}}
\newcommand{\Bw}[0]{\mathbf{w}}
\newcommand{\Bx}[0]{\mathbf{x}}
\newcommand{\By}[0]{\mathbf{y}}
\newcommand{\Bz}[0]{\mathbf{z}}
\newcommand{\BA}[0]{\mathbf{A}}
\newcommand{\BB}[0]{\mathbf{B}}
\newcommand{\BC}[0]{\mathbf{C}}
\newcommand{\BD}[0]{\mathbf{D}}
\newcommand{\BE}[0]{\mathbf{E}}
\newcommand{\BF}[0]{\mathbf{F}}
\newcommand{\BG}[0]{\mathbf{G}}
\newcommand{\BH}[0]{\mathbf{H}}
\newcommand{\BI}[0]{\mathbf{I}}
\newcommand{\BJ}[0]{\mathbf{J}}
\newcommand{\BK}[0]{\mathbf{K}}
\newcommand{\BL}[0]{\mathbf{L}}
\newcommand{\BM}[0]{\mathbf{M}}
\newcommand{\BN}[0]{\mathbf{N}}
\newcommand{\BO}[0]{\mathbf{O}}
\newcommand{\BP}[0]{\mathbf{P}}
\newcommand{\BQ}[0]{\mathbf{Q}}
\newcommand{\BR}[0]{\mathbf{R}}
\newcommand{\BS}[0]{\mathbf{S}}
\newcommand{\BT}[0]{\mathbf{T}}
\newcommand{\BU}[0]{\mathbf{U}}
\newcommand{\BV}[0]{\mathbf{V}}
\newcommand{\BW}[0]{\mathbf{W}}
\newcommand{\BX}[0]{\mathbf{X}}
\newcommand{\BY}[0]{\mathbf{Y}}
\newcommand{\BZ}[0]{\mathbf{Z}}

\newcommand{\Bzero}[0]{\mathbf{0}}
\newcommand{\Btheta}[0]{\boldsymbol{\theta}}
\newcommand{\Btau}[0]{\boldsymbol{\tau}}
\newcommand{\Bomega}[0]{\boldsymbol{\omega}}

%
% shorthand for unit vectors
%
\newcommand{\acap}[0]{\hat{\Ba}}
\newcommand{\bcap}[0]{\hat{\Bb}}
\newcommand{\ccap}[0]{\hat{\Bc}}
\newcommand{\dcap}[0]{\hat{\Bd}}
\newcommand{\ecap}[0]{\hat{\Be}}
\newcommand{\fcap}[0]{\hat{\Bf}}
\newcommand{\gcap}[0]{\hat{\Bg}}
\newcommand{\hcap}[0]{\hat{\Bh}}
\newcommand{\icap}[0]{\hat{\Bi}}
\newcommand{\jcap}[0]{\hat{\Bj}}
\newcommand{\kcap}[0]{\hat{\Bk}}
\newcommand{\lcap}[0]{\hat{\Bl}}
\newcommand{\mcap}[0]{\hat{\Bm}}
\newcommand{\ncap}[0]{\hat{\Bn}}
\newcommand{\ocap}[0]{\hat{\Bo}}
\newcommand{\pcap}[0]{\hat{\Bp}}
\newcommand{\qcap}[0]{\hat{\Bq}}
\newcommand{\rcap}[0]{\hat{\Br}}
\newcommand{\scap}[0]{\hat{\Bs}}
\newcommand{\tcap}[0]{\hat{\Bt}}
\newcommand{\ucap}[0]{\hat{\Bu}}
\newcommand{\vcap}[0]{\hat{\Bv}}
\newcommand{\wcap}[0]{\hat{\Bw}}
\newcommand{\xcap}[0]{\hat{\Bx}}
\newcommand{\ycap}[0]{\hat{\By}}
\newcommand{\zcap}[0]{\hat{\Bz}}
\newcommand{\thetacap}[0]{\hat{\Btheta}}

%
% to write R^n and C^n in a distinguishable fashion.  Perhaps change this
% to the double lined characters upon figuring out how to do so.
%
\newcommand{\C}[1]{$\mathbb{C}^{#1}$}
\newcommand{\R}[1]{$\mathbb{R}^{#1}$}

%
% various generally useful helpers
%

% derivative of #1 wrt. #2:
\newcommand{\D}[2] {\frac {d#2} {d#1}}

\newcommand{\inv}[1]{\frac{1}{#1}}
\newcommand{\cross}[0]{\times}

\newcommand{\abs}[1]{\lvert{#1}\rvert}
\newcommand{\norm}[1]{\lVert{#1}\rVert}
\newcommand{\innerprod}[2]{\langle{#1}, {#2}\rangle}
\newcommand{\dotprod}[2]{{#1} \cdot {#2}}
\newcommand{\bdotprod}[2]{\left({#1} \cdot {#2}\right)}
\newcommand{\crossprod}[2]{{#1} \cross {#2}}
\newcommand{\tripleprod}[3]{\dotprod{\left(\crossprod{#1}{#2}\right)}{#3}}

\DeclareMathOperator{\Proj}{Proj}
\DeclareMathOperator{\Span}{span}
\DeclareMathOperator{\Sgn}{sgn}
\DeclareMathOperator{\Area}{Area}
\DeclareMathOperator{\Volume}{Volume}

%
% A few miscellaneous things specific to this document
%
\newcommand{\crossop}[1]{\crossprod{#1}{}}

% R2 vector.
\newcommand{\VectorTwo}[2]{
\begin{bmatrix}
 {#1} \\
 {#2}
\end{bmatrix}
}

\newcommand{\VectorN}[1]{
\begin{bmatrix}
{#1}_1 \\
{#1}_2 \\
\vdots \\
{#1}_N \\
\end{bmatrix}
}

\newcommand{\DETuvij}[4]{
\begin{vmatrix}
 {#1}_{#3} & {#1}_{#4} \\
 {#2}_{#3} & {#2}_{#4}
\end{vmatrix}
}

\newcommand{\DETuvwijk}[6]{
\begin{vmatrix}
 {#1}_{#4} & {#1}_{#5} & {#1}_{#6} \\
 {#2}_{#4} & {#2}_{#5} & {#2}_{#6} \\
 {#3}_{#4} & {#3}_{#5} & {#3}_{#6}
\end{vmatrix}
}

\newcommand{\DETuvwxijkl}[8]{
\begin{vmatrix}
 {#1}_{#5} & {#1}_{#6} & {#1}_{#7} & {#1}_{#8} \\
 {#2}_{#5} & {#2}_{#6} & {#2}_{#7} & {#2}_{#8} \\
 {#3}_{#5} & {#3}_{#6} & {#3}_{#7} & {#3}_{#8} \\
 {#4}_{#5} & {#4}_{#6} & {#4}_{#7} & {#4}_{#8} \\
\end{vmatrix}
}

%\newcommand{\DETuvwxyijklm}[10]{
%\begin{vmatrix}
% {#1}_{#6} & {#1}_{#7} & {#1}_{#8} & {#1}_{#9} & {#1}_{#10} \\
% {#2}_{#6} & {#2}_{#7} & {#2}_{#8} & {#2}_{#9} & {#2}_{#10} \\
% {#3}_{#6} & {#3}_{#7} & {#3}_{#8} & {#3}_{#9} & {#3}_{#10} \\
% {#4}_{#6} & {#4}_{#7} & {#4}_{#8} & {#4}_{#9} & {#4}_{#10} \\
% {#5}_{#6} & {#5}_{#7} & {#5}_{#8} & {#5}_{#9} & {#5}_{#10}
%\end{vmatrix}
%}

% R3 vector.
\newcommand{\VectorThree}[3]{
\begin{bmatrix}
 {#1} \\
 {#2} \\
 {#3}
\end{bmatrix}
}



\author{Peeter Joot}
\email{peeter.joot@gmail.com}

%\documentclass[]{eliblogwidescreen}

\usepackage{amsmath}
\usepackage{mathpazo}

%
% shorthand for bold symbols, convenient for vectors and matrices
%
\newcommand{\Ba}[0]{\mathbf{a}}
\newcommand{\Bb}[0]{\mathbf{b}}
\newcommand{\Bc}[0]{\mathbf{c}}
\newcommand{\Bd}[0]{\mathbf{d}}
\newcommand{\Be}[0]{\mathbf{e}}
\newcommand{\Bf}[0]{\mathbf{f}}
\newcommand{\Bg}[0]{\mathbf{g}}
\newcommand{\Bh}[0]{\mathbf{h}}
\newcommand{\Bi}[0]{\mathbf{i}}
\newcommand{\Bj}[0]{\mathbf{j}}
\newcommand{\Bk}[0]{\mathbf{k}}
\newcommand{\Bl}[0]{\mathbf{l}}
\newcommand{\Bm}[0]{\mathbf{m}}
\newcommand{\Bn}[0]{\mathbf{n}}
\newcommand{\Bo}[0]{\mathbf{o}}
\newcommand{\Bp}[0]{\mathbf{p}}
\newcommand{\Bq}[0]{\mathbf{q}}
\newcommand{\Br}[0]{\mathbf{r}}
\newcommand{\Bs}[0]{\mathbf{s}}
\newcommand{\Bt}[0]{\mathbf{t}}
\newcommand{\Bu}[0]{\mathbf{u}}
\newcommand{\Bv}[0]{\mathbf{v}}
\newcommand{\Bw}[0]{\mathbf{w}}
\newcommand{\Bx}[0]{\mathbf{x}}
\newcommand{\By}[0]{\mathbf{y}}
\newcommand{\Bz}[0]{\mathbf{z}}
\newcommand{\BA}[0]{\mathbf{A}}
\newcommand{\BB}[0]{\mathbf{B}}
\newcommand{\BC}[0]{\mathbf{C}}
\newcommand{\BD}[0]{\mathbf{D}}
\newcommand{\BE}[0]{\mathbf{E}}
\newcommand{\BF}[0]{\mathbf{F}}
\newcommand{\BG}[0]{\mathbf{G}}
\newcommand{\BH}[0]{\mathbf{H}}
\newcommand{\BI}[0]{\mathbf{I}}
\newcommand{\BJ}[0]{\mathbf{J}}
\newcommand{\BK}[0]{\mathbf{K}}
\newcommand{\BL}[0]{\mathbf{L}}
\newcommand{\BM}[0]{\mathbf{M}}
\newcommand{\BN}[0]{\mathbf{N}}
\newcommand{\BO}[0]{\mathbf{O}}
\newcommand{\BP}[0]{\mathbf{P}}
\newcommand{\BQ}[0]{\mathbf{Q}}
\newcommand{\BR}[0]{\mathbf{R}}
\newcommand{\BS}[0]{\mathbf{S}}
\newcommand{\BT}[0]{\mathbf{T}}
\newcommand{\BU}[0]{\mathbf{U}}
\newcommand{\BV}[0]{\mathbf{V}}
\newcommand{\BW}[0]{\mathbf{W}}
\newcommand{\BX}[0]{\mathbf{X}}
\newcommand{\BY}[0]{\mathbf{Y}}
\newcommand{\BZ}[0]{\mathbf{Z}}

\newcommand{\Bzero}[0]{\mathbf{0}}
\newcommand{\Btheta}[0]{\boldsymbol{\theta}}
\newcommand{\Btau}[0]{\boldsymbol{\tau}}
\newcommand{\Bomega}[0]{\boldsymbol{\omega}}

%
% shorthand for unit vectors
%
\newcommand{\acap}[0]{\hat{\Ba}}
\newcommand{\bcap}[0]{\hat{\Bb}}
\newcommand{\ccap}[0]{\hat{\Bc}}
\newcommand{\dcap}[0]{\hat{\Bd}}
\newcommand{\ecap}[0]{\hat{\Be}}
\newcommand{\fcap}[0]{\hat{\Bf}}
\newcommand{\gcap}[0]{\hat{\Bg}}
\newcommand{\hcap}[0]{\hat{\Bh}}
\newcommand{\icap}[0]{\hat{\Bi}}
\newcommand{\jcap}[0]{\hat{\Bj}}
\newcommand{\kcap}[0]{\hat{\Bk}}
\newcommand{\lcap}[0]{\hat{\Bl}}
\newcommand{\mcap}[0]{\hat{\Bm}}
\newcommand{\ncap}[0]{\hat{\Bn}}
\newcommand{\ocap}[0]{\hat{\Bo}}
\newcommand{\pcap}[0]{\hat{\Bp}}
\newcommand{\qcap}[0]{\hat{\Bq}}
\newcommand{\rcap}[0]{\hat{\Br}}
\newcommand{\scap}[0]{\hat{\Bs}}
\newcommand{\tcap}[0]{\hat{\Bt}}
\newcommand{\ucap}[0]{\hat{\Bu}}
\newcommand{\vcap}[0]{\hat{\Bv}}
\newcommand{\wcap}[0]{\hat{\Bw}}
\newcommand{\xcap}[0]{\hat{\Bx}}
\newcommand{\ycap}[0]{\hat{\By}}
\newcommand{\zcap}[0]{\hat{\Bz}}
\newcommand{\thetacap}[0]{\hat{\Btheta}}

%
% to write R^n and C^n in a distinguishable fashion.  Perhaps change this
% to the double lined characters upon figuring out how to do so.
%
\newcommand{\C}[1]{$\mathbb{C}^{#1}$}
\newcommand{\R}[1]{$\mathbb{R}^{#1}$}

%
% various generally useful helpers
%

% derivative of #1 wrt. #2:
\newcommand{\D}[2] {\frac {d#2} {d#1}}

\newcommand{\inv}[1]{\frac{1}{#1}}
\newcommand{\cross}[0]{\times}

\newcommand{\abs}[1]{\lvert{#1}\rvert}
\newcommand{\norm}[1]{\lVert{#1}\rVert}
\newcommand{\innerprod}[2]{\langle{#1}, {#2}\rangle}
\newcommand{\dotprod}[2]{{#1} \cdot {#2}}
\newcommand{\bdotprod}[2]{\left({#1} \cdot {#2}\right)}
\newcommand{\crossprod}[2]{{#1} \cross {#2}}
\newcommand{\tripleprod}[3]{\dotprod{\left(\crossprod{#1}{#2}\right)}{#3}}

\DeclareMathOperator{\Proj}{Proj}
\DeclareMathOperator{\Span}{span}
\DeclareMathOperator{\Sgn}{sgn}
\DeclareMathOperator{\Area}{Area}
\DeclareMathOperator{\Volume}{Volume}

%
% A few miscellaneous things specific to this document
%
\newcommand{\crossop}[1]{\crossprod{#1}{}}

% R2 vector.
\newcommand{\VectorTwo}[2]{
\begin{bmatrix}
 {#1} \\
 {#2}
\end{bmatrix}
}

\newcommand{\VectorN}[1]{
\begin{bmatrix}
{#1}_1 \\
{#1}_2 \\
\vdots \\
{#1}_N \\
\end{bmatrix}
}

\newcommand{\DETuvij}[4]{
\begin{vmatrix}
 {#1}_{#3} & {#1}_{#4} \\
 {#2}_{#3} & {#2}_{#4}
\end{vmatrix}
}

\newcommand{\DETuvwijk}[6]{
\begin{vmatrix}
 {#1}_{#4} & {#1}_{#5} & {#1}_{#6} \\
 {#2}_{#4} & {#2}_{#5} & {#2}_{#6} \\
 {#3}_{#4} & {#3}_{#5} & {#3}_{#6}
\end{vmatrix}
}

\newcommand{\DETuvwxijkl}[8]{
\begin{vmatrix}
 {#1}_{#5} & {#1}_{#6} & {#1}_{#7} & {#1}_{#8} \\
 {#2}_{#5} & {#2}_{#6} & {#2}_{#7} & {#2}_{#8} \\
 {#3}_{#5} & {#3}_{#6} & {#3}_{#7} & {#3}_{#8} \\
 {#4}_{#5} & {#4}_{#6} & {#4}_{#7} & {#4}_{#8} \\
\end{vmatrix}
}

%\newcommand{\DETuvwxyijklm}[10]{
%\begin{vmatrix}
% {#1}_{#6} & {#1}_{#7} & {#1}_{#8} & {#1}_{#9} & {#1}_{#10} \\
% {#2}_{#6} & {#2}_{#7} & {#2}_{#8} & {#2}_{#9} & {#2}_{#10} \\
% {#3}_{#6} & {#3}_{#7} & {#3}_{#8} & {#3}_{#9} & {#3}_{#10} \\
% {#4}_{#6} & {#4}_{#7} & {#4}_{#8} & {#4}_{#9} & {#4}_{#10} \\
% {#5}_{#6} & {#5}_{#7} & {#5}_{#8} & {#5}_{#9} & {#5}_{#10}
%\end{vmatrix}
%}

% R3 vector.
\newcommand{\VectorThree}[3]{
\begin{bmatrix}
 {#1} \\
 {#2} \\
 {#3}
\end{bmatrix}
}



\author{Peeter Joot}
\email{peeter.joot@gmail.com}


%\chapter{PHY456H1F: Quantum Mechanics II.  Lecture 2 (Taught by Prof J.E. Sipe).  Approximate methods}
\index{approximate methods}
\label{chap:qmTwoL2}
\blogpage{http://sites.google.com/site/peeterjoot/math2011/qmTwoL2.pdf}
%\date{Sept 14, 2011}

\section{Approximate methods for finding energy eigenvalues and eigenkets}
\index{energy eigenvalue!approximate}
\index{energy eigenket!approximate}

In many situations one has a Hamiltonian \(H\)

\begin{equation}\label{eqn:qmTwoL2:10}
H \ket{\Psi_{n \alpha}} = E_n \ket{\Psi_{n \alpha}}
\end{equation}

Here \(\alpha\) is a ``degeneracy index'' (example: as in Hydrogen atom).

\paragraph{Why?}

\begin{itemize}
\item Simplifies dynamics

take

\begin{equation}\label{eqn:qmTwoL2:730}
\begin{aligned}
\ket{\Psi(0)}
= \sum_{n\alpha}
\ket{\Psi_{n \alpha}}
\braket{\Psi_{n \alpha}}{\Psi(0)}
&
= \sum_{n\alpha} c_{n \alpha} \ket{\Psi_{n \alpha}}
\end{aligned}
\end{equation}

Then
\begin{equation}\label{eqn:qmTwoL2:750}
\begin{aligned}
\ket{\Psi(t)}
&=
e^{-i H t/\Hbar}
\ket{\Psi(0)} \\
&=
\sum_{n\alpha} c_{n \alpha}
e^{-i H t/\Hbar}
\ket{\Psi_{n \alpha}}  \\
&=
\sum_{n\alpha} c_{n \alpha}
e^{-i E_n t/\Hbar}
\ket{\Psi_{n \alpha}}
\end{aligned}
\end{equation}

\item ``Applied  field"' can often be thought of a driving the system from one eigenstate to another.

\imageFigure{../../figures/phy456/qmTwoL2fig1}{qmTwoL2fig1}{fig:qmTwoL2:1}{0.4}

\item Stat mech.

In thermal equilibrium

\begin{equation}\label{eqn:qmTwoL2:30}
\expectation{\calO} =
\frac{\sum_{n \alpha} \bra{\Psi_{n\alpha}} \calO \ket{\Psi_{n \alpha}}  e^{-\beta E_n}}{
Z
}
\end{equation}

where

\begin{equation}\label{eqn:qmTwoL2:50}
\beta = \inv{\kB T},
\end{equation}

and

\begin{equation}\label{eqn:qmTwoL2:70}
Z = \sum_{n \alpha} e^{-\beta E_n}
\end{equation}
\end{itemize}

\section{Variational principle}
\index{variational principle}

Consider any ket

\begin{equation}\label{eqn:qmTwoL2:90}
\ket{\Psi} = \sum_{n \alpha} c_{n \alpha} \ket{\Psi_{n \alpha}}
\end{equation}

(perhaps not even normalized), and where

\begin{equation}\label{eqn:qmTwoL2:110}
c_{n \alpha} = \braket{\Psi_{n \alpha}}{\Psi}
\end{equation}

but we do not know these.

\begin{equation}\label{eqn:qmTwoL2:130}
\braket{\Psi}{\Psi} = \sum_{n \alpha} \Abs{c_{n \alpha}}^2
\end{equation}

\begin{equation}\label{eqn:qmTwoL2:770}
\begin{aligned}
\frac{
\bra{\Psi} H \ket{\Psi}
}{
\braket{\Psi}{\Psi}
}
&=
\frac{
\sum_{n \alpha} \Abs{c_{n \alpha}}^2 E_n
}{
\sum_{m \beta} \Abs{c_{m \beta}}^2
} \\
&\ge
\frac{
\sum_{n \alpha} \Abs{c_{n \alpha}}^2 E_0
}{
\sum_{m \beta} \Abs{c_{m \beta}}^2
}  \\
&=
E_0
\end{aligned}
\end{equation}

So for any ket we can form the upper bound for the ground state energy

\begin{equation}\label{eqn:qmTwoL2:150}
\frac{
\bra{\Psi} H \ket{\Psi}
}{
\braket{\Psi}{\Psi}
}
\ge E_0
\end{equation}

There is a whole set of strategies based on estimating the ground state energy.  This is called the Variational principle for ground state.  See \S 24.2 in the text \citep{desai2009quantum}.

We define the functional

\begin{equation}\label{eqn:qmTwoL2:170}
E[\Psi] =
\frac{
\bra{\Psi} H \ket{\Psi}
}{
\braket{\Psi}{\Psi}
}
\ge E_0
\end{equation}

If \(\ket{\Psi} = c \ket{\Psi_0}\) where \(\ket{\Psi_0}\) is the normalized ground state, then

\begin{equation}\label{eqn:qmTwoL2:190}
E[ c \Psi_0 ] = E_0
\end{equation}

\makeexample{Hydrogen atom}{approx:ex1}{

\begin{equation}\label{eqn:qmTwoL2:210}
\bra{\Br} H \ket{\Br'} = \calH \delta^3(\Br - \Br')
\end{equation}

where

\begin{equation}\label{eqn:qmTwoL2:230}
\calH = -\frac{\Hbar^2}{2 \mu} \spacegrad^2 - \frac{e^2}{r}
\end{equation}

Here \(\mu\) is the reduced mass.

We know the exact solution:

\begin{equation}\label{eqn:qmTwoL2:250}
H \ket{\Psi_0}
\end{equation}

\begin{equation}\label{eqn:qmTwoL2:270}
E_0 = -R_y
\end{equation}

\begin{equation}\label{eqn:qmTwoL2:290}
R_y = \frac{\mu e^4}{2 \Hbar^2} \approx 13.6 \text{eV}
\end{equation}

\begin{equation}\label{eqn:qmTwoL2:310}
\braket{\Br}{\Psi_0} = \Phi_{100}(\Br) = \left( \inv{\pi a_0^3}\right)^{1/2} e^{-r/a_0}
\end{equation}

\begin{equation}\label{eqn:qmTwoL2:330}
a_0 = \frac{\Hbar^2}{\mu e^2} \approx 0.53 \angstrom
\end{equation}

\imageFigure{../../figures/phy456/qmTwoL2fig2}{qmTwoL2fig2}{fig:qmTwoL2:2}{0.4}

\imageFigure{../../figures/phy456/qmTwoL2fig3}{qmTwoL2fig3}{fig:qmTwoL2:3}{0.4}

estimate

\begin{equation}\label{eqn:qmTwoL2:350}
\begin{aligned}
\bra{\Psi} H \ket{\Psi} &= \int d^3 \Br \Psi^\conj(\Br) \left( -\frac{\Hbar^2}{2 \mu} \spacegrad^2 - \frac{e^2}{r} \right) \Psi(\Br) \\
\braket{\Psi}{\Psi} &= \int d^3 \Br \Abs{\Psi(\Br)}^2
\end{aligned}
\end{equation}

Or guess shape

\imageFigure{../../figures/phy456/qmTwoL2fig4}{qmTwoL2fig4}{fig:qmTwoL2:4}{0.4}

Using the trial wave function \(e^{-\alpha r^2}\)

\begin{equation}\label{eqn:qmTwoL2:370}
E[\Psi] \rightarrow E(\alpha)
\end{equation}

\begin{equation}\label{eqn:qmTwoL2:390}
E(\alpha) =
\frac{\int d^3 \Br e^{-\alpha r^2} \left( -\frac{\Hbar^2}{2 \mu} \spacegrad^2 - \frac{e^2}{r} \right) e^{-\alpha r^2}}{
\int d^3\Br e^{-2 \alpha r^2}
}
\end{equation}

find
\begin{equation}\label{eqn:qmTwoL2:410}
E(\alpha) = A \alpha - B \alpha^{1/2}
\end{equation}

\begin{equation}\label{eqn:qmTwoL2:430}
\begin{aligned}
A &= \frac{3 \Hbar^2}{2\mu} \\
B &= 2 e^2 \left( \frac{2}{\pi} \right)^{1/2}
\end{aligned}
\end{equation}

\imageFigure{../../figures/phy456/qmTwoL2fig5}{qmTwoL2fig5}{fig:qmTwoL2:5}{0.4}

Minimum at

\begin{equation}\label{eqn:qmTwoL2:450}
\alpha_0 =
\left( \frac{\mu e^2}{ \Hbar^2 } \right) \frac{8 }{9 \pi}
\end{equation}

So

\begin{equation}\label{eqn:qmTwoL2:470}
E(\alpha_0) =
- \frac{\mu e^4 }{2 \Hbar^2} \frac{8 }{3 \pi} = -0.85 R_y
\end{equation}

maybe not too bad...
}

\makeexample{Helium atom}{approx:ex2}{

Assume an infinite nuclear mass with nucleus charge \(2 e\)

\imageFigure{../../figures/phy456/qmTwoL2fig6}{qmTwoL2fig6}{fig:qmTwoL2:6}{0.4}

ground state wavefunction

\begin{equation}\label{eqn:qmTwoL2:490}
\Psi_0(\Br_1, \Br_2)
\end{equation}

The problem that we want to solve is

\begin{equation}\label{eqn:qmTwoL2:510}
\left(
-\frac{\Hbar^2}{2 m} \spacegrad_1^2
-\frac{\Hbar^2}{2 m} \spacegrad_2^2
- \frac{2 e}{r}
+
\frac{e^2}{\Abs{\Br_1 - \Br_2}}
\right)
\Psi_0(\Br_1, \Br_2) = E_0 \Psi_0(\Br_1, \Br_2)
\end{equation}

Nobody can solve this problem.  It is one of the simplest real problems in QM that cannot be solved exactly.

Suppose that we neglected the electron, electron repulsion.  Then

\begin{equation}\label{eqn:qmTwoL2:530}
\Psi_0(\Br_1, \Br_2)
=
\overbar{\Phi}_{100}(\Br_1)
\overbar{\Phi}_{100}(\Br_2)
\end{equation}

where

\begin{equation}\label{eqn:qmTwoL2:550}
\left( -\frac{\Hbar^2}{2 m} \spacegrad^2
- \frac{2 e}{r} \right)
\overbar{\Phi}_{100}(\Br) = \epsilon \overbar{\Phi}_{100}(\Br)
\end{equation}

with

\begin{equation}\label{eqn:qmTwoL2:570}
\epsilon = - 4 R_y
\end{equation}

\begin{equation}\label{eqn:qmTwoL2:590}
R_y = \frac{m e^4}{2 \Hbar^2}
\end{equation}

This is the solution to

\begin{equation}\label{eqn:qmTwoL2:610}
\left(
-\frac{\Hbar^2}{2 m} \spacegrad_1^2
-\frac{\Hbar^2}{2 m} \spacegrad_2^2
- \frac{2 e}{r}
\right)
\Psi_0^{(0)}(\Br_1, \Br_2) = E_0 \Psi_0(\Br_1, \Br_2)
=
E_0^{(0)} \Psi_0^{(0)}(\Br_1, \Br_2)
\end{equation}

\begin{equation}\label{eqn:qmTwoL2:630}
E_0^{(0)} = - 8 R_y.
\end{equation}

Now we want to put back in the electron electron repulsion, and make an estimate.

Trial wavefunction

\begin{equation}\label{eqn:qmTwoL2:650}
\Psi(\Br_1, \Br_2, Z) =
\left(
\left(\frac{Z^3}{ \pi a_0^3 }\right)^{1/2} e^{-Z r_1/a_0}
\right)
\left(
\left(\frac{Z^3}{ \pi a_0^3 }\right)^{1/2} e^{-Z r_2/a_0}
\right)
\end{equation}

expect that the best estimate is for \(Z \in [1,2]\).

This can be calculated numerically, and we find

\begin{equation}\label{eqn:qmTwoL2:670}
E(Z) = 2 R_Y \left( Z^2 - 4 Z + \frac{5}{8} Z \right)
\end{equation}

The \(Z^2\) comes from the kinetic energy.  The \(-4 Z\) is the electron nuclear attraction, and the final term is from the electron-electron repulsion.

The actual minimum is

\begin{equation}\label{eqn:qmTwoL2:690}
Z = 2 - \frac{5}{16}
\end{equation}

\begin{equation}\label{eqn:qmTwoL2:710}
E(2 - 5/16) = -77.5 \text{eV}
\end{equation}

Whereas the measured value is \(-78.6 \text{eV}\).
}

\shipoutAnswer
