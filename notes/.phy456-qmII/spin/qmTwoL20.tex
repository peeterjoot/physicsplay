%
% Copyright � 2012 Peeter Joot.  All Rights Reserved.
% Licenced as described in the file LICENSE under the root directory of this GIT repository.
%

%
%
%%
% Copyright � 2015 Peeter Joot.  All Rights Reserved.
% Licenced as described in the file LICENSE under the root directory of this GIT repository.
%
\documentclass[]{eliblog}

\usepackage{amsmath}
\usepackage{mathpazo}

%
% shorthand for bold symbols, convenient for vectors and matrices
%
\newcommand{\Ba}[0]{\mathbf{a}}
\newcommand{\Bb}[0]{\mathbf{b}}
\newcommand{\Bc}[0]{\mathbf{c}}
\newcommand{\Bd}[0]{\mathbf{d}}
\newcommand{\Be}[0]{\mathbf{e}}
\newcommand{\Bf}[0]{\mathbf{f}}
\newcommand{\Bg}[0]{\mathbf{g}}
\newcommand{\Bh}[0]{\mathbf{h}}
\newcommand{\Bi}[0]{\mathbf{i}}
\newcommand{\Bj}[0]{\mathbf{j}}
\newcommand{\Bk}[0]{\mathbf{k}}
\newcommand{\Bl}[0]{\mathbf{l}}
\newcommand{\Bm}[0]{\mathbf{m}}
\newcommand{\Bn}[0]{\mathbf{n}}
\newcommand{\Bo}[0]{\mathbf{o}}
\newcommand{\Bp}[0]{\mathbf{p}}
\newcommand{\Bq}[0]{\mathbf{q}}
\newcommand{\Br}[0]{\mathbf{r}}
\newcommand{\Bs}[0]{\mathbf{s}}
\newcommand{\Bt}[0]{\mathbf{t}}
\newcommand{\Bu}[0]{\mathbf{u}}
\newcommand{\Bv}[0]{\mathbf{v}}
\newcommand{\Bw}[0]{\mathbf{w}}
\newcommand{\Bx}[0]{\mathbf{x}}
\newcommand{\By}[0]{\mathbf{y}}
\newcommand{\Bz}[0]{\mathbf{z}}
\newcommand{\BA}[0]{\mathbf{A}}
\newcommand{\BB}[0]{\mathbf{B}}
\newcommand{\BC}[0]{\mathbf{C}}
\newcommand{\BD}[0]{\mathbf{D}}
\newcommand{\BE}[0]{\mathbf{E}}
\newcommand{\BF}[0]{\mathbf{F}}
\newcommand{\BG}[0]{\mathbf{G}}
\newcommand{\BH}[0]{\mathbf{H}}
\newcommand{\BI}[0]{\mathbf{I}}
\newcommand{\BJ}[0]{\mathbf{J}}
\newcommand{\BK}[0]{\mathbf{K}}
\newcommand{\BL}[0]{\mathbf{L}}
\newcommand{\BM}[0]{\mathbf{M}}
\newcommand{\BN}[0]{\mathbf{N}}
\newcommand{\BO}[0]{\mathbf{O}}
\newcommand{\BP}[0]{\mathbf{P}}
\newcommand{\BQ}[0]{\mathbf{Q}}
\newcommand{\BR}[0]{\mathbf{R}}
\newcommand{\BS}[0]{\mathbf{S}}
\newcommand{\BT}[0]{\mathbf{T}}
\newcommand{\BU}[0]{\mathbf{U}}
\newcommand{\BV}[0]{\mathbf{V}}
\newcommand{\BW}[0]{\mathbf{W}}
\newcommand{\BX}[0]{\mathbf{X}}
\newcommand{\BY}[0]{\mathbf{Y}}
\newcommand{\BZ}[0]{\mathbf{Z}}

\newcommand{\Bzero}[0]{\mathbf{0}}
\newcommand{\Btheta}[0]{\boldsymbol{\theta}}
\newcommand{\Btau}[0]{\boldsymbol{\tau}}
\newcommand{\Bomega}[0]{\boldsymbol{\omega}}

%
% shorthand for unit vectors
%
\newcommand{\acap}[0]{\hat{\Ba}}
\newcommand{\bcap}[0]{\hat{\Bb}}
\newcommand{\ccap}[0]{\hat{\Bc}}
\newcommand{\dcap}[0]{\hat{\Bd}}
\newcommand{\ecap}[0]{\hat{\Be}}
\newcommand{\fcap}[0]{\hat{\Bf}}
\newcommand{\gcap}[0]{\hat{\Bg}}
\newcommand{\hcap}[0]{\hat{\Bh}}
\newcommand{\icap}[0]{\hat{\Bi}}
\newcommand{\jcap}[0]{\hat{\Bj}}
\newcommand{\kcap}[0]{\hat{\Bk}}
\newcommand{\lcap}[0]{\hat{\Bl}}
\newcommand{\mcap}[0]{\hat{\Bm}}
\newcommand{\ncap}[0]{\hat{\Bn}}
\newcommand{\ocap}[0]{\hat{\Bo}}
\newcommand{\pcap}[0]{\hat{\Bp}}
\newcommand{\qcap}[0]{\hat{\Bq}}
\newcommand{\rcap}[0]{\hat{\Br}}
\newcommand{\scap}[0]{\hat{\Bs}}
\newcommand{\tcap}[0]{\hat{\Bt}}
\newcommand{\ucap}[0]{\hat{\Bu}}
\newcommand{\vcap}[0]{\hat{\Bv}}
\newcommand{\wcap}[0]{\hat{\Bw}}
\newcommand{\xcap}[0]{\hat{\Bx}}
\newcommand{\ycap}[0]{\hat{\By}}
\newcommand{\zcap}[0]{\hat{\Bz}}
\newcommand{\thetacap}[0]{\hat{\Btheta}}

%
% to write R^n and C^n in a distinguishable fashion.  Perhaps change this
% to the double lined characters upon figuring out how to do so.
%
\newcommand{\C}[1]{$\mathbb{C}^{#1}$}
\newcommand{\R}[1]{$\mathbb{R}^{#1}$}

%
% various generally useful helpers
%

% derivative of #1 wrt. #2:
\newcommand{\D}[2] {\frac {d#2} {d#1}}

\newcommand{\inv}[1]{\frac{1}{#1}}
\newcommand{\cross}[0]{\times}

\newcommand{\abs}[1]{\lvert{#1}\rvert}
\newcommand{\norm}[1]{\lVert{#1}\rVert}
\newcommand{\innerprod}[2]{\langle{#1}, {#2}\rangle}
\newcommand{\dotprod}[2]{{#1} \cdot {#2}}
\newcommand{\bdotprod}[2]{\left({#1} \cdot {#2}\right)}
\newcommand{\crossprod}[2]{{#1} \cross {#2}}
\newcommand{\tripleprod}[3]{\dotprod{\left(\crossprod{#1}{#2}\right)}{#3}}

\DeclareMathOperator{\Proj}{Proj}
\DeclareMathOperator{\Span}{span}
\DeclareMathOperator{\Sgn}{sgn}
\DeclareMathOperator{\Area}{Area}
\DeclareMathOperator{\Volume}{Volume}

%
% A few miscellaneous things specific to this document
%
\newcommand{\crossop}[1]{\crossprod{#1}{}}

% R2 vector.
\newcommand{\VectorTwo}[2]{
\begin{bmatrix}
 {#1} \\
 {#2}
\end{bmatrix}
}

\newcommand{\VectorN}[1]{
\begin{bmatrix}
{#1}_1 \\
{#1}_2 \\
\vdots \\
{#1}_N \\
\end{bmatrix}
}

\newcommand{\DETuvij}[4]{
\begin{vmatrix}
 {#1}_{#3} & {#1}_{#4} \\
 {#2}_{#3} & {#2}_{#4}
\end{vmatrix}
}

\newcommand{\DETuvwijk}[6]{
\begin{vmatrix}
 {#1}_{#4} & {#1}_{#5} & {#1}_{#6} \\
 {#2}_{#4} & {#2}_{#5} & {#2}_{#6} \\
 {#3}_{#4} & {#3}_{#5} & {#3}_{#6}
\end{vmatrix}
}

\newcommand{\DETuvwxijkl}[8]{
\begin{vmatrix}
 {#1}_{#5} & {#1}_{#6} & {#1}_{#7} & {#1}_{#8} \\
 {#2}_{#5} & {#2}_{#6} & {#2}_{#7} & {#2}_{#8} \\
 {#3}_{#5} & {#3}_{#6} & {#3}_{#7} & {#3}_{#8} \\
 {#4}_{#5} & {#4}_{#6} & {#4}_{#7} & {#4}_{#8} \\
\end{vmatrix}
}

%\newcommand{\DETuvwxyijklm}[10]{
%\begin{vmatrix}
% {#1}_{#6} & {#1}_{#7} & {#1}_{#8} & {#1}_{#9} & {#1}_{#10} \\
% {#2}_{#6} & {#2}_{#7} & {#2}_{#8} & {#2}_{#9} & {#2}_{#10} \\
% {#3}_{#6} & {#3}_{#7} & {#3}_{#8} & {#3}_{#9} & {#3}_{#10} \\
% {#4}_{#6} & {#4}_{#7} & {#4}_{#8} & {#4}_{#9} & {#4}_{#10} \\
% {#5}_{#6} & {#5}_{#7} & {#5}_{#8} & {#5}_{#9} & {#5}_{#10}
%\end{vmatrix}
%}

% R3 vector.
\newcommand{\VectorThree}[3]{
\begin{bmatrix}
 {#1} \\
 {#2} \\
 {#3}
\end{bmatrix}
}



\author{Peeter Joot}
\email{peeter.joot@gmail.com}

%\documentclass[]{eliblogwidescreen}

\usepackage{amsmath}
\usepackage{mathpazo}

%
% shorthand for bold symbols, convenient for vectors and matrices
%
\newcommand{\Ba}[0]{\mathbf{a}}
\newcommand{\Bb}[0]{\mathbf{b}}
\newcommand{\Bc}[0]{\mathbf{c}}
\newcommand{\Bd}[0]{\mathbf{d}}
\newcommand{\Be}[0]{\mathbf{e}}
\newcommand{\Bf}[0]{\mathbf{f}}
\newcommand{\Bg}[0]{\mathbf{g}}
\newcommand{\Bh}[0]{\mathbf{h}}
\newcommand{\Bi}[0]{\mathbf{i}}
\newcommand{\Bj}[0]{\mathbf{j}}
\newcommand{\Bk}[0]{\mathbf{k}}
\newcommand{\Bl}[0]{\mathbf{l}}
\newcommand{\Bm}[0]{\mathbf{m}}
\newcommand{\Bn}[0]{\mathbf{n}}
\newcommand{\Bo}[0]{\mathbf{o}}
\newcommand{\Bp}[0]{\mathbf{p}}
\newcommand{\Bq}[0]{\mathbf{q}}
\newcommand{\Br}[0]{\mathbf{r}}
\newcommand{\Bs}[0]{\mathbf{s}}
\newcommand{\Bt}[0]{\mathbf{t}}
\newcommand{\Bu}[0]{\mathbf{u}}
\newcommand{\Bv}[0]{\mathbf{v}}
\newcommand{\Bw}[0]{\mathbf{w}}
\newcommand{\Bx}[0]{\mathbf{x}}
\newcommand{\By}[0]{\mathbf{y}}
\newcommand{\Bz}[0]{\mathbf{z}}
\newcommand{\BA}[0]{\mathbf{A}}
\newcommand{\BB}[0]{\mathbf{B}}
\newcommand{\BC}[0]{\mathbf{C}}
\newcommand{\BD}[0]{\mathbf{D}}
\newcommand{\BE}[0]{\mathbf{E}}
\newcommand{\BF}[0]{\mathbf{F}}
\newcommand{\BG}[0]{\mathbf{G}}
\newcommand{\BH}[0]{\mathbf{H}}
\newcommand{\BI}[0]{\mathbf{I}}
\newcommand{\BJ}[0]{\mathbf{J}}
\newcommand{\BK}[0]{\mathbf{K}}
\newcommand{\BL}[0]{\mathbf{L}}
\newcommand{\BM}[0]{\mathbf{M}}
\newcommand{\BN}[0]{\mathbf{N}}
\newcommand{\BO}[0]{\mathbf{O}}
\newcommand{\BP}[0]{\mathbf{P}}
\newcommand{\BQ}[0]{\mathbf{Q}}
\newcommand{\BR}[0]{\mathbf{R}}
\newcommand{\BS}[0]{\mathbf{S}}
\newcommand{\BT}[0]{\mathbf{T}}
\newcommand{\BU}[0]{\mathbf{U}}
\newcommand{\BV}[0]{\mathbf{V}}
\newcommand{\BW}[0]{\mathbf{W}}
\newcommand{\BX}[0]{\mathbf{X}}
\newcommand{\BY}[0]{\mathbf{Y}}
\newcommand{\BZ}[0]{\mathbf{Z}}

\newcommand{\Bzero}[0]{\mathbf{0}}
\newcommand{\Btheta}[0]{\boldsymbol{\theta}}
\newcommand{\Btau}[0]{\boldsymbol{\tau}}
\newcommand{\Bomega}[0]{\boldsymbol{\omega}}

%
% shorthand for unit vectors
%
\newcommand{\acap}[0]{\hat{\Ba}}
\newcommand{\bcap}[0]{\hat{\Bb}}
\newcommand{\ccap}[0]{\hat{\Bc}}
\newcommand{\dcap}[0]{\hat{\Bd}}
\newcommand{\ecap}[0]{\hat{\Be}}
\newcommand{\fcap}[0]{\hat{\Bf}}
\newcommand{\gcap}[0]{\hat{\Bg}}
\newcommand{\hcap}[0]{\hat{\Bh}}
\newcommand{\icap}[0]{\hat{\Bi}}
\newcommand{\jcap}[0]{\hat{\Bj}}
\newcommand{\kcap}[0]{\hat{\Bk}}
\newcommand{\lcap}[0]{\hat{\Bl}}
\newcommand{\mcap}[0]{\hat{\Bm}}
\newcommand{\ncap}[0]{\hat{\Bn}}
\newcommand{\ocap}[0]{\hat{\Bo}}
\newcommand{\pcap}[0]{\hat{\Bp}}
\newcommand{\qcap}[0]{\hat{\Bq}}
\newcommand{\rcap}[0]{\hat{\Br}}
\newcommand{\scap}[0]{\hat{\Bs}}
\newcommand{\tcap}[0]{\hat{\Bt}}
\newcommand{\ucap}[0]{\hat{\Bu}}
\newcommand{\vcap}[0]{\hat{\Bv}}
\newcommand{\wcap}[0]{\hat{\Bw}}
\newcommand{\xcap}[0]{\hat{\Bx}}
\newcommand{\ycap}[0]{\hat{\By}}
\newcommand{\zcap}[0]{\hat{\Bz}}
\newcommand{\thetacap}[0]{\hat{\Btheta}}

%
% to write R^n and C^n in a distinguishable fashion.  Perhaps change this
% to the double lined characters upon figuring out how to do so.
%
\newcommand{\C}[1]{$\mathbb{C}^{#1}$}
\newcommand{\R}[1]{$\mathbb{R}^{#1}$}

%
% various generally useful helpers
%

% derivative of #1 wrt. #2:
\newcommand{\D}[2] {\frac {d#2} {d#1}}

\newcommand{\inv}[1]{\frac{1}{#1}}
\newcommand{\cross}[0]{\times}

\newcommand{\abs}[1]{\lvert{#1}\rvert}
\newcommand{\norm}[1]{\lVert{#1}\rVert}
\newcommand{\innerprod}[2]{\langle{#1}, {#2}\rangle}
\newcommand{\dotprod}[2]{{#1} \cdot {#2}}
\newcommand{\bdotprod}[2]{\left({#1} \cdot {#2}\right)}
\newcommand{\crossprod}[2]{{#1} \cross {#2}}
\newcommand{\tripleprod}[3]{\dotprod{\left(\crossprod{#1}{#2}\right)}{#3}}

\DeclareMathOperator{\Proj}{Proj}
\DeclareMathOperator{\Span}{span}
\DeclareMathOperator{\Sgn}{sgn}
\DeclareMathOperator{\Area}{Area}
\DeclareMathOperator{\Volume}{Volume}

%
% A few miscellaneous things specific to this document
%
\newcommand{\crossop}[1]{\crossprod{#1}{}}

% R2 vector.
\newcommand{\VectorTwo}[2]{
\begin{bmatrix}
 {#1} \\
 {#2}
\end{bmatrix}
}

\newcommand{\VectorN}[1]{
\begin{bmatrix}
{#1}_1 \\
{#1}_2 \\
\vdots \\
{#1}_N \\
\end{bmatrix}
}

\newcommand{\DETuvij}[4]{
\begin{vmatrix}
 {#1}_{#3} & {#1}_{#4} \\
 {#2}_{#3} & {#2}_{#4}
\end{vmatrix}
}

\newcommand{\DETuvwijk}[6]{
\begin{vmatrix}
 {#1}_{#4} & {#1}_{#5} & {#1}_{#6} \\
 {#2}_{#4} & {#2}_{#5} & {#2}_{#6} \\
 {#3}_{#4} & {#3}_{#5} & {#3}_{#6}
\end{vmatrix}
}

\newcommand{\DETuvwxijkl}[8]{
\begin{vmatrix}
 {#1}_{#5} & {#1}_{#6} & {#1}_{#7} & {#1}_{#8} \\
 {#2}_{#5} & {#2}_{#6} & {#2}_{#7} & {#2}_{#8} \\
 {#3}_{#5} & {#3}_{#6} & {#3}_{#7} & {#3}_{#8} \\
 {#4}_{#5} & {#4}_{#6} & {#4}_{#7} & {#4}_{#8} \\
\end{vmatrix}
}

%\newcommand{\DETuvwxyijklm}[10]{
%\begin{vmatrix}
% {#1}_{#6} & {#1}_{#7} & {#1}_{#8} & {#1}_{#9} & {#1}_{#10} \\
% {#2}_{#6} & {#2}_{#7} & {#2}_{#8} & {#2}_{#9} & {#2}_{#10} \\
% {#3}_{#6} & {#3}_{#7} & {#3}_{#8} & {#3}_{#9} & {#3}_{#10} \\
% {#4}_{#6} & {#4}_{#7} & {#4}_{#8} & {#4}_{#9} & {#4}_{#10} \\
% {#5}_{#6} & {#5}_{#7} & {#5}_{#8} & {#5}_{#9} & {#5}_{#10}
%\end{vmatrix}
%}

% R3 vector.
\newcommand{\VectorThree}[3]{
\begin{bmatrix}
 {#1} \\
 {#2} \\
 {#3}
\end{bmatrix}
}



\author{Peeter Joot}
\email{peeter.joot@gmail.com}


%\chapter{PHY456H1F: Quantum Mechanics II.  Lecture 20 (Taught by Prof J.E. Sipe).  Spherical tensors}
%\chapter{Spherical tensors}
\index{spherical tensor}
\label{chap:qmTwoL20}

\blogpage{http://sites.google.com/site/peeterjoot2/math2011/qmTwoL20.pdf}
%\date{Nov 21, 2011}





\section{Spherical tensors (cont)}

READING: \S 29 of \citep{desai2009quantum}.

\paragraph{definition}.  Any \((2k + 1)\) operator \(T(k, q)\), \(q = -k, \cdots, k\) are the elements of a spherical tensor of rank \(k\) if

\begin{equation}\label{eqn:qmTwoL20:10}
U[M] T(k, q) U^{-1}[M]
= \sum_{q'} T(k, q') D^{(k)}_{q q'}
\end{equation}

where \(D^{(k)}_{q q'}\) was the matrix element of the rotation operator

\begin{equation}\label{eqn:qmTwoL20:20}
D^{(k)}_{q q'} = \bra{k q'} U[M] \ket{k q''}.
\end{equation}

So, if we have a Cartesian vector operator with components \(V_x, V_y, V_z\) then we can construct a corresponding spherical vector operator

\begin{equation}\label{eqn:qmTwoL20:30}
\begin{array}{l l l}
T(1, 1) &= - \frac{V_x + i V_y}{\sqrt{2}} &\equiv V_{+1} \\
T(1, 0) &= V_z &\equiv V_0 \\
T(1, -1) &= - \frac{V_x - i V_y}{\sqrt{2}} &\equiv V_{-1}
\end{array}.
\end{equation}

By considering infinitesimal rotations we can come up with the commutation relations between the angular momentum operators

\begin{equation}\label{eqn:qmTwoL20:50}
\begin{aligned}
\antisymmetric{J_{\pm}}{T(k, q)} &= \Hbar \sqrt{(k \mp q)(k \pm q + 1)} T(k, q \pm 1) \\
\antisymmetric{J_{z}}{T(k, q)} &= \Hbar q T(k, q)
\end{aligned}
\end{equation}

Note that the text in (29.15) defines these, whereas in class these were considered consequences of \eqnref{eqn:qmTwoL20:10}, once infinitesimal rotations were used.

Recall that these match our angular momentum raising and lowering identities

\begin{equation}\label{eqn:qmTwoL20:50b}
\begin{aligned}
J_{\pm} \ket{k q} &= \Hbar \sqrt{(k \mp q)(k \pm q + 1)} \ket{k, q \pm 1} \\
J_{z} \ket{k q} &= \Hbar q \ket{k, q}.
\end{aligned}
\end{equation}

Consider two problems

\begin{equation}\label{eqn:qmTwoL20:70}
\begin{array}{l l l}
T(k, q)						& & \ket{k q} \\
\antisymmetric{J_{\pm}}{T(k, q)} 		&\leftrightarrow &J_{\pm} \ket{k q} \\
\antisymmetric{J_{z}}{T(k, q)} 			&\leftrightarrow &J_{z} \ket{k q}
\end{array}
\end{equation}

We have a correspondence between the spherical tensors and angular momentum kets

\begin{equation}\label{eqn:qmTwoL20:330}
\begin{array}{l l l l}
T_1(k_1, q_1)&\qquad q_1 = -k_1, \cdots, k_1 		& \qquad \ket{k_1 q_1} 		& \ket{k_2 q_2} \\
T_2(k_2, q_2)&\qquad q_2 = -k_2, \cdots, k_2		& \qquad q_1 = -k_1, \cdots k_1 	& q_2 = -k_2, \cdots k_2 \\
\end{array}
\end{equation}

So, as we can write for angular momentum
\begin{equation}\label{eqn:qmTwoL20:410}
\begin{aligned}
\ket{kq} &= \sum_{q_1, q_2}
\ket{k_1, q_1}
\ket{k_2, q_2}
\mathLabelBox{\braket{ k_1 q_1 k_2 q_2 }{ k q}}{These are the C.G coefficients}  \\
\ket{k_1 q_1 ; k_2 q_2}
&=
\sum_{k, q'}
\ket{k q'} \braket{ k q'}{ k_1 q_1 k_2 q_2 }
\end{aligned}
\end{equation}

We also have for spherical tensors

\begin{equation}\label{eqn:qmTwoL20:430}
\begin{aligned}
T(k, q) &= \sum_{q_1, q_2}
T_1(k_1, q_1)
T_2(k_2, q_2)
\braket{ k_1 q_1 k_2 q_2 }{ k q}
	\\
T_1(k_1, q_1)
T_2(k_2, q_2)
&=
\sum_{k, q'}
T(k, q') \braket{ k q'}{ k_1 q_1 k_2 q_2 } &
\end{aligned}
\end{equation}

Can form eigenstates \(\ket{kq}\) of \((\text{total angular momentum})^2\) and (z-comp of the total angular momentum).
FIXME: this will not be proven, but we are strongly suggested to try this ourselves.

\begin{equation}\label{eqn:qmTwoL20:350}
\begin{array}{l l l}
\text{spherical tensor (3)} 				&\leftrightarrow &\text{Cartesian vector (3)} \\
(\text{spherical vector})(\text{spherical vector})	&		 &\text{Cartesian tensor}
\end{array}
\end{equation}

We can check the dimensions for a spherical tensor decomposition into rank 0, rank 1 and rank 2 tensors.

\begin{equation}\label{eqn:qmTwoL20:370}
\begin{array}{l l l}
\text{spherical tensor rank \(0\)} 	&	(1)	&	(\text{Cartesian vector})(\text{Cartesian vector}) \\
\text{spherical tensor rank \(1\)} 	&	(3)	&	(3)(3) \\
\text{spherical tensor rank \(2\)} 	&	(5)	&       9 \\
\hline
\text{dimension check sum}       	&	 9	&         \\
\end{array}
\end{equation}

Or in the direct product and sum shorthand

\begin{equation}\label{eqn:qmTwoL20:90}
1 \otimes 1 = 0 \oplus 1 \oplus 2
\end{equation}

Note that this is just like problem 4 in problem set 10 where we calculated the CG kets for the \(1 \otimes 1 = 0 \oplus 1 \oplus 2\) decomposition starting from kets \(\ket{1 m}\ket{1 m'}\).

\begin{equation}\label{eqn:qmTwoL20:390}
\begin{array}{l l l}
\ket{22}		&				& 		\\
\ket{21}		& \ket{11} 			& 		\\
\ket{20}		& \ket{10} 			& \ket{00} 	\\
\ket{2\overline{1}}	& \ket{1\overline{1}} 		& 		\\
\ket{2\overline{2}}	&				&
\end{array}
\end{equation}

\paragraph{Example}.

How about a Cartesian tensor of rank 3?

\begin{equation}\label{eqn:qmTwoL20:110}
A_{ijk}
\end{equation}

\begin{equation}\label{eqn:qmTwoL20:450}
\begin{aligned}
1 \otimes 1 \otimes 1
&=
1 \otimes ( 0 \oplus 1 \oplus 2) \\
&=
(1 \otimes 0) \oplus (1 \otimes 1) \oplus (1 \otimes 2) \\
&=
\begin{array}{l l l l l l l l l l l l l l}
1 &\oplus   &(0 &\oplus &1 &\oplus &2) &\oplus &(3  &\oplus & 2 &\oplus &1) \\
3 &+        &1 &+      &3 &+      &5  &+       &7  &+      & 5 &+      &3 = 27
\end{array}
\end{aligned}
\end{equation}

\paragraph{Why bother?}

Consider a tensor operator \(T(k, q)\) and an eigenket of angular momentum \(\ket{\alpha j m}\), where \(\alpha\) is a degeneracy index.

Look at

\begin{equation}\label{eqn:qmTwoL20:470}
\begin{aligned}
T(k, q) \ket{\alpha j m}
U[M] T(k, q) \ket{\alpha j m}
&=
U[M] T(k, q) U^\dagger[M] U[M] \ket{\alpha j m} \\
&=
\sum_{q' m'}
D^{(k)}_{q q'}
D^{(j)}_{m m'}
T(k, q') \ket{\alpha j m'}
\end{aligned}
\end{equation}

This transforms like \(\ket{k q} \otimes \ket{j m}\).  We can say immediately

\begin{equation}\label{eqn:qmTwoL20:150}
\bra{\alpha' j' m'} T(k, q) \ket{\alpha j m} = 0
\end{equation}

unless
\begin{equation}\label{eqn:qmTwoL20:170}
\begin{aligned}
\Abs{k - j} &\le j' \le k + j \\
m' &= m + q
\end{aligned}
\end{equation}

This is the ``selection rule''.

Examples.

\begin{itemize}
\item Scalar \(T(0, 0)\)

\begin{equation}\label{eqn:qmTwoL20:190}
\bra{\alpha' j' m'} T(0, 0) \ket{\alpha j m} = 0 ,
\end{equation}

unless \(j = j'\) and \(m = m'\).

\item \(V_x, V_y, V_z\).  What are the non-vanishing matrix elements?

\begin{equation}\label{eqn:qmTwoL20:210}
V_x = \frac{ V_{-1} - V_{+1}}{\sqrt{2}}, \cdots
\end{equation}

\begin{equation}\label{eqn:qmTwoL20:230}
\bra{\alpha' j' m'} V_{x, y} \ket{\alpha j m} = 0 ,
\end{equation}

unless
\begin{equation}\label{eqn:qmTwoL20:250}
\begin{aligned}
\Abs{j - 1} &\le j' \le j + 1  \\
m' &= m \pm 1
\end{aligned}
\end{equation}

\begin{equation}\label{eqn:qmTwoL20:270}
\bra{\alpha' j' m'} V_{z} \ket{\alpha j m} = 0 ,
\end{equation}

unless
\begin{equation}\label{eqn:qmTwoL20:290}
\begin{aligned}
\Abs{j - 1} &\le j' \le j + 1  \\
m' &= m
\end{aligned}
\end{equation}
\end{itemize}

Very generally one can prove (the Wigner-Eckart theory in the text \S 29.3)

\begin{equation}\label{eqn:qmTwoL20:310}
\bra{\alpha_2 j_2 m_2} T(k, q) \ket{\alpha_1 j_1 m_1}
=
\bra{\alpha_2 j_2 } T(k) \ket{\alpha_1 j_1} \cdot
\braket{j_2 m_2}{k q_1 ; j_1 m_1}
\end{equation}

where we split into a ``reduced matrix element'' describing the ``physics'', and the CG coefficient for ``geometry'' respectively.


