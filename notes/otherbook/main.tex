\documentclass[12pt,leqno]{book}

\usepackage{amsmath,amssymb,amsfonts} % Typical maths resource packages
%\usepackage{graphics}                 % Packages to allow inclusion of graphics
\usepackage{graphicx}
\usepackage{color}                   % For creating coloured text and background

% for ointctr... (also appears to make "prettier" \int and \sum's)
% ... but messes up other stuff (grad vs \spacegrad).
\usepackage{txfonts} 

% levi.tex:
\usepackage{listings}

%\usepackage{latexsym,epsf}

\usepackage[bookmarks=true]{hyperref}

\parindent 1cm
\parskip 0.2cm
\topmargin 0.2cm
\oddsidemargin 1cm
\evensidemargin 0.5cm
\textwidth 15cm
\textheight 21cm

% how do these ones work?
\newtheorem{theorem}{Theorem}[section]
\newtheorem{proposition}[theorem]{Proposition}
\newtheorem{corollary}[theorem]{Corollary}
\newtheorem{lemma}[theorem]{Lemma}
\newtheorem{remark}[theorem]{Remark}
\newtheorem{definition}[theorem]{Definition}

\usepackage{amsmath}
\usepackage{mathpazo}

%
% shorthand for bold symbols, convenient for vectors and matrices
%
\newcommand{\Ba}[0]{\mathbf{a}}
\newcommand{\Bb}[0]{\mathbf{b}}
\newcommand{\Bc}[0]{\mathbf{c}}
\newcommand{\Bd}[0]{\mathbf{d}}
\newcommand{\Be}[0]{\mathbf{e}}
\newcommand{\Bf}[0]{\mathbf{f}}
\newcommand{\Bg}[0]{\mathbf{g}}
\newcommand{\Bh}[0]{\mathbf{h}}
\newcommand{\Bi}[0]{\mathbf{i}}
\newcommand{\Bj}[0]{\mathbf{j}}
\newcommand{\Bk}[0]{\mathbf{k}}
\newcommand{\Bl}[0]{\mathbf{l}}
\newcommand{\Bm}[0]{\mathbf{m}}
\newcommand{\Bn}[0]{\mathbf{n}}
\newcommand{\Bo}[0]{\mathbf{o}}
\newcommand{\Bp}[0]{\mathbf{p}}
\newcommand{\Bq}[0]{\mathbf{q}}
\newcommand{\Br}[0]{\mathbf{r}}
\newcommand{\Bs}[0]{\mathbf{s}}
\newcommand{\Bt}[0]{\mathbf{t}}
\newcommand{\Bu}[0]{\mathbf{u}}
\newcommand{\Bv}[0]{\mathbf{v}}
\newcommand{\Bw}[0]{\mathbf{w}}
\newcommand{\Bx}[0]{\mathbf{x}}
\newcommand{\By}[0]{\mathbf{y}}
\newcommand{\Bz}[0]{\mathbf{z}}
\newcommand{\BA}[0]{\mathbf{A}}
\newcommand{\BB}[0]{\mathbf{B}}
\newcommand{\BC}[0]{\mathbf{C}}
\newcommand{\BD}[0]{\mathbf{D}}
\newcommand{\BE}[0]{\mathbf{E}}
\newcommand{\BF}[0]{\mathbf{F}}
\newcommand{\BG}[0]{\mathbf{G}}
\newcommand{\BH}[0]{\mathbf{H}}
\newcommand{\BI}[0]{\mathbf{I}}
\newcommand{\BJ}[0]{\mathbf{J}}
\newcommand{\BK}[0]{\mathbf{K}}
\newcommand{\BL}[0]{\mathbf{L}}
\newcommand{\BM}[0]{\mathbf{M}}
\newcommand{\BN}[0]{\mathbf{N}}
\newcommand{\BO}[0]{\mathbf{O}}
\newcommand{\BP}[0]{\mathbf{P}}
\newcommand{\BQ}[0]{\mathbf{Q}}
\newcommand{\BR}[0]{\mathbf{R}}
\newcommand{\BS}[0]{\mathbf{S}}
\newcommand{\BT}[0]{\mathbf{T}}
\newcommand{\BU}[0]{\mathbf{U}}
\newcommand{\BV}[0]{\mathbf{V}}
\newcommand{\BW}[0]{\mathbf{W}}
\newcommand{\BX}[0]{\mathbf{X}}
\newcommand{\BY}[0]{\mathbf{Y}}
\newcommand{\BZ}[0]{\mathbf{Z}}

\newcommand{\Bzero}[0]{\mathbf{0}}
\newcommand{\Btheta}[0]{\boldsymbol{\theta}}
\newcommand{\Btau}[0]{\boldsymbol{\tau}}
\newcommand{\Bomega}[0]{\boldsymbol{\omega}}

%
% shorthand for unit vectors
%
\newcommand{\acap}[0]{\hat{\Ba}}
\newcommand{\bcap}[0]{\hat{\Bb}}
\newcommand{\ccap}[0]{\hat{\Bc}}
\newcommand{\dcap}[0]{\hat{\Bd}}
\newcommand{\ecap}[0]{\hat{\Be}}
\newcommand{\fcap}[0]{\hat{\Bf}}
\newcommand{\gcap}[0]{\hat{\Bg}}
\newcommand{\hcap}[0]{\hat{\Bh}}
\newcommand{\icap}[0]{\hat{\Bi}}
\newcommand{\jcap}[0]{\hat{\Bj}}
\newcommand{\kcap}[0]{\hat{\Bk}}
\newcommand{\lcap}[0]{\hat{\Bl}}
\newcommand{\mcap}[0]{\hat{\Bm}}
\newcommand{\ncap}[0]{\hat{\Bn}}
\newcommand{\ocap}[0]{\hat{\Bo}}
\newcommand{\pcap}[0]{\hat{\Bp}}
\newcommand{\qcap}[0]{\hat{\Bq}}
\newcommand{\rcap}[0]{\hat{\Br}}
\newcommand{\scap}[0]{\hat{\Bs}}
\newcommand{\tcap}[0]{\hat{\Bt}}
\newcommand{\ucap}[0]{\hat{\Bu}}
\newcommand{\vcap}[0]{\hat{\Bv}}
\newcommand{\wcap}[0]{\hat{\Bw}}
\newcommand{\xcap}[0]{\hat{\Bx}}
\newcommand{\ycap}[0]{\hat{\By}}
\newcommand{\zcap}[0]{\hat{\Bz}}
\newcommand{\thetacap}[0]{\hat{\Btheta}}

%
% to write R^n and C^n in a distinguishable fashion.  Perhaps change this
% to the double lined characters upon figuring out how to do so.
%
\newcommand{\C}[1]{$\mathbb{C}^{#1}$}
\newcommand{\R}[1]{$\mathbb{R}^{#1}$}

%
% various generally useful helpers
%

% derivative of #1 wrt. #2:
\newcommand{\D}[2] {\frac {d#2} {d#1}}

\newcommand{\inv}[1]{\frac{1}{#1}}
\newcommand{\cross}[0]{\times}

\newcommand{\abs}[1]{\lvert{#1}\rvert}
\newcommand{\norm}[1]{\lVert{#1}\rVert}
\newcommand{\innerprod}[2]{\langle{#1}, {#2}\rangle}
\newcommand{\dotprod}[2]{{#1} \cdot {#2}}
\newcommand{\bdotprod}[2]{\left({#1} \cdot {#2}\right)}
\newcommand{\crossprod}[2]{{#1} \cross {#2}}
\newcommand{\tripleprod}[3]{\dotprod{\left(\crossprod{#1}{#2}\right)}{#3}}

\DeclareMathOperator{\Proj}{Proj}
\DeclareMathOperator{\Span}{span}
\DeclareMathOperator{\Sgn}{sgn}
\DeclareMathOperator{\Area}{Area}
\DeclareMathOperator{\Volume}{Volume}

%
% A few miscellaneous things specific to this document
%
\newcommand{\crossop}[1]{\crossprod{#1}{}}

% R2 vector.
\newcommand{\VectorTwo}[2]{
\begin{bmatrix}
 {#1} \\
 {#2}
\end{bmatrix}
}

\newcommand{\VectorN}[1]{
\begin{bmatrix}
{#1}_1 \\
{#1}_2 \\
\vdots \\
{#1}_N \\
\end{bmatrix}
}

\newcommand{\DETuvij}[4]{
\begin{vmatrix}
 {#1}_{#3} & {#1}_{#4} \\
 {#2}_{#3} & {#2}_{#4}
\end{vmatrix}
}

\newcommand{\DETuvwijk}[6]{
\begin{vmatrix}
 {#1}_{#4} & {#1}_{#5} & {#1}_{#6} \\
 {#2}_{#4} & {#2}_{#5} & {#2}_{#6} \\
 {#3}_{#4} & {#3}_{#5} & {#3}_{#6}
\end{vmatrix}
}

\newcommand{\DETuvwxijkl}[8]{
\begin{vmatrix}
 {#1}_{#5} & {#1}_{#6} & {#1}_{#7} & {#1}_{#8} \\
 {#2}_{#5} & {#2}_{#6} & {#2}_{#7} & {#2}_{#8} \\
 {#3}_{#5} & {#3}_{#6} & {#3}_{#7} & {#3}_{#8} \\
 {#4}_{#5} & {#4}_{#6} & {#4}_{#7} & {#4}_{#8} \\
\end{vmatrix}
}

%\newcommand{\DETuvwxyijklm}[10]{
%\begin{vmatrix}
% {#1}_{#6} & {#1}_{#7} & {#1}_{#8} & {#1}_{#9} & {#1}_{#10} \\
% {#2}_{#6} & {#2}_{#7} & {#2}_{#8} & {#2}_{#9} & {#2}_{#10} \\
% {#3}_{#6} & {#3}_{#7} & {#3}_{#8} & {#3}_{#9} & {#3}_{#10} \\
% {#4}_{#6} & {#4}_{#7} & {#4}_{#8} & {#4}_{#9} & {#4}_{#10} \\
% {#5}_{#6} & {#5}_{#7} & {#5}_{#8} & {#5}_{#9} & {#5}_{#10}
%\end{vmatrix}
%}

% R3 vector.
\newcommand{\VectorThree}[3]{
\begin{bmatrix}
 {#1} \\
 {#2} \\
 {#3}
\end{bmatrix}
}


%<misc>
%
\newcommand{\Abs}[1]{{\left\lvert{#1}\right\rvert}}
\newcommand{\spacegrad}[0]{\boldsymbol{\nabla}}
\newcommand{\grad}[0]{\nabla}
\newcommand{\LL}[0]{\mathcal{L}}

% == \partial_{#1} {#2}
\newcommand{\PD}[2]{\frac{\partial {#2}}{\partial {#1}}}
% inline variant
\newcommand{\PDi}[2]{{\partial {#2}}/{\partial {#1}}}

\newcommand{\PDD}[3]{\frac{\partial^2 {#3}}{\partial {#1}\partial {#2}}}
%\newcommand{\PDd}[2]{\frac{\partial^2 {#2}}{{\partial{#1}}^2}}
\newcommand{\PDsq}[2]{\frac{\partial^2 {#2}}{(\partial {#1})^2}}

\newcommand{\Partial}[2]{\frac{\partial {#1}}{\partial {#2}}}
\DeclareMathOperator{\RejName}{Rej}
\newcommand{\Rej}[2]{\RejName_{#1}\left( {#2} \right)}
\newcommand{\Rm}[1]{\mathbb{R}^{#1}}
\newcommand{\Cm}[1]{\mathbb{C}^{#1}}
\newcommand{\conj}[0]{{*}}

%</misc>

% <grade selection>
%
\newcommand{\gpgrade}[2] {{\left\langle{{#1}}\right\rangle}_{#2}}

\newcommand{\gpgradezero}[1] {\gpgrade{#1}{}}
%\newcommand{\gpscalargrade}[1] {{\left\langle{{#1}}\right\rangle}}
%\newcommand{\gpgradezero}[1] {\gpgrade{#1}{0}}

%\newcommand{\gpgradeone}[1] {{\left\langle{{#1}}\right\rangle}_{1}}
\newcommand{\gpgradeone}[1] {\gpgrade{#1}{1}}

\newcommand{\gpgradetwo}[1] {\gpgrade{#1}{2}}
\newcommand{\gpgradethree}[1] {\gpgrade{#1}{3}}
\newcommand{\gpgradefour}[1] {\gpgrade{#1}{4}}
%
% </grade selection>



\newcommand{\adot}[0]{{\dot{a}}}
\newcommand{\bdot}[0]{{\dot{b}}}
% taken for centered dot:
%\newcommand{\cdot}[0]{{\dot{c}}}
%\newcommand{\ddot}[0]{{\dot{d}}}
\newcommand{\edot}[0]{{\dot{e}}}
\newcommand{\fdot}[0]{{\dot{f}}}
\newcommand{\gdot}[0]{{\dot{g}}}
\newcommand{\hdot}[0]{{\dot{h}}}
\newcommand{\idot}[0]{{\dot{i}}}
\newcommand{\jdot}[0]{{\dot{j}}}
\newcommand{\kdot}[0]{{\dot{k}}}
\newcommand{\ldot}[0]{{\dot{l}}}
\newcommand{\mdot}[0]{{\dot{m}}}
\newcommand{\ndot}[0]{{\dot{n}}}
%\newcommand{\odot}[0]{{\dot{o}}}
\newcommand{\pdot}[0]{{\dot{p}}}
\newcommand{\qdot}[0]{{\dot{q}}}
\newcommand{\rdot}[0]{{\dot{r}}}
\newcommand{\sdot}[0]{{\dot{s}}}
\newcommand{\tdot}[0]{{\dot{t}}}
\newcommand{\udot}[0]{{\dot{u}}}
\newcommand{\vdot}[0]{{\dot{v}}}
\newcommand{\wdot}[0]{{\dot{w}}}
\newcommand{\xdot}[0]{{\dot{x}}}
\newcommand{\ydot}[0]{{\dot{y}}}
\newcommand{\zdot}[0]{{\dot{z}}}
\newcommand{\addot}[0]{{\ddot{a}}}
\newcommand{\bddot}[0]{{\ddot{b}}}
\newcommand{\cddot}[0]{{\ddot{c}}}
%\newcommand{\dddot}[0]{{\ddot{d}}}
\newcommand{\eddot}[0]{{\ddot{e}}}
\newcommand{\fddot}[0]{{\ddot{f}}}
\newcommand{\gddot}[0]{{\ddot{g}}}
\newcommand{\hddot}[0]{{\ddot{h}}}
\newcommand{\iddot}[0]{{\ddot{i}}}
\newcommand{\jddot}[0]{{\ddot{j}}}
\newcommand{\kddot}[0]{{\ddot{k}}}
\newcommand{\lddot}[0]{{\ddot{l}}}
\newcommand{\mddot}[0]{{\ddot{m}}}
\newcommand{\nddot}[0]{{\ddot{n}}}
\newcommand{\oddot}[0]{{\ddot{o}}}
\newcommand{\pddot}[0]{{\ddot{p}}}
\newcommand{\qddot}[0]{{\ddot{q}}}
\newcommand{\rddot}[0]{{\ddot{r}}}
\newcommand{\sddot}[0]{{\ddot{s}}}
\newcommand{\tddot}[0]{{\ddot{t}}}
\newcommand{\uddot}[0]{{\ddot{u}}}
\newcommand{\vddot}[0]{{\ddot{v}}}
\newcommand{\wddot}[0]{{\ddot{w}}}
\newcommand{\xddot}[0]{{\ddot{x}}}
\newcommand{\yddot}[0]{{\ddot{y}}}
\newcommand{\zddot}[0]{{\ddot{z}}}

%<bold and dot greek symbols>
%

\newcommand{\Deltadot}[0]{{\dot{\Delta}}}
\newcommand{\Gammadot}[0]{{\dot{\Gamma}}}
\newcommand{\Lambdadot}[0]{{\dot{\Lambda}}}
\newcommand{\Omegadot}[0]{{\dot{\Omega}}}
\newcommand{\Phidot}[0]{{\dot{\Phi}}}
\newcommand{\Pidot}[0]{{\dot{\Pi}}}
\newcommand{\Psidot}[0]{{\dot{\Psi}}}
\newcommand{\Sigmadot}[0]{{\dot{\Sigma}}}
\newcommand{\Thetadot}[0]{{\dot{\Theta}}}
\newcommand{\Upsilondot}[0]{{\dot{\Upsilon}}}
\newcommand{\Xidot}[0]{{\dot{\Xi}}}
\newcommand{\alphadot}[0]{{\dot{\alpha}}}
\newcommand{\betadot}[0]{{\dot{\beta}}}
\newcommand{\chidot}[0]{{\dot{\chi}}}
\newcommand{\deltadot}[0]{{\dot{\delta}}}
\newcommand{\epsilondot}[0]{{\dot{\epsilon}}}
\newcommand{\etadot}[0]{{\dot{\eta}}}
\newcommand{\gammadot}[0]{{\dot{\gamma}}}
\newcommand{\kappadot}[0]{{\dot{\kappa}}}
\newcommand{\lambdadot}[0]{{\dot{\lambda}}}
\newcommand{\mudot}[0]{{\dot{\mu}}}
\newcommand{\nudot}[0]{{\dot{\nu}}}
\newcommand{\omegadot}[0]{{\dot{\omega}}}
\newcommand{\phidot}[0]{{\dot{\phi}}}
\newcommand{\pidot}[0]{{\dot{\pi}}}
\newcommand{\psidot}[0]{{\dot{\psi}}}
\newcommand{\rhodot}[0]{{\dot{\rho}}}
\newcommand{\sigmadot}[0]{{\dot{\sigma}}}
\newcommand{\taudot}[0]{{\dot{\tau}}}
\newcommand{\thetadot}[0]{{\dot{\theta}}}
\newcommand{\upsilondot}[0]{{\dot{\upsilon}}}
\newcommand{\varepsilondot}[0]{{\dot{\varepsilon}}}
\newcommand{\varphidot}[0]{{\dot{\varphi}}}
\newcommand{\varpidot}[0]{{\dot{\varpi}}}
\newcommand{\varrhodot}[0]{{\dot{\varrho}}}
\newcommand{\varsigmadot}[0]{{\dot{\varsigma}}}
\newcommand{\varthetadot}[0]{{\dot{\vartheta}}}
\newcommand{\xidot}[0]{{\dot{\xi}}}
\newcommand{\zetadot}[0]{{\dot{\zeta}}}

\newcommand{\Deltaddot}[0]{{\ddot{\Delta}}}
\newcommand{\Gammaddot}[0]{{\ddot{\Gamma}}}
\newcommand{\Lambdaddot}[0]{{\ddot{\Lambda}}}
\newcommand{\Omegaddot}[0]{{\ddot{\Omega}}}
\newcommand{\Phiddot}[0]{{\ddot{\Phi}}}
\newcommand{\Piddot}[0]{{\ddot{\Pi}}}
\newcommand{\Psiddot}[0]{{\ddot{\Psi}}}
\newcommand{\Sigmaddot}[0]{{\ddot{\Sigma}}}
\newcommand{\Thetaddot}[0]{{\ddot{\Theta}}}
\newcommand{\Upsilonddot}[0]{{\ddot{\Upsilon}}}
\newcommand{\Xiddot}[0]{{\ddot{\Xi}}}
\newcommand{\alphaddot}[0]{{\ddot{\alpha}}}
\newcommand{\betaddot}[0]{{\ddot{\beta}}}
\newcommand{\chiddot}[0]{{\ddot{\chi}}}
\newcommand{\deltaddot}[0]{{\ddot{\delta}}}
\newcommand{\epsilonddot}[0]{{\ddot{\epsilon}}}
\newcommand{\etaddot}[0]{{\ddot{\eta}}}
\newcommand{\gammaddot}[0]{{\ddot{\gamma}}}
\newcommand{\kappaddot}[0]{{\ddot{\kappa}}}
\newcommand{\lambdaddot}[0]{{\ddot{\lambda}}}
\newcommand{\muddot}[0]{{\ddot{\mu}}}
\newcommand{\nuddot}[0]{{\ddot{\nu}}}
\newcommand{\omegaddot}[0]{{\ddot{\omega}}}
\newcommand{\phiddot}[0]{{\ddot{\phi}}}
\newcommand{\piddot}[0]{{\ddot{\pi}}}
\newcommand{\psiddot}[0]{{\ddot{\psi}}}
\newcommand{\rhoddot}[0]{{\ddot{\rho}}}
\newcommand{\sigmaddot}[0]{{\ddot{\sigma}}}
\newcommand{\tauddot}[0]{{\ddot{\tau}}}
\newcommand{\thetaddot}[0]{{\ddot{\theta}}}
\newcommand{\upsilonddot}[0]{{\ddot{\upsilon}}}
\newcommand{\varepsilonddot}[0]{{\ddot{\varepsilon}}}
\newcommand{\varphiddot}[0]{{\ddot{\varphi}}}
\newcommand{\varpiddot}[0]{{\ddot{\varpi}}}
\newcommand{\varrhoddot}[0]{{\ddot{\varrho}}}
\newcommand{\varsigmaddot}[0]{{\ddot{\varsigma}}}
\newcommand{\varthetaddot}[0]{{\ddot{\vartheta}}}
\newcommand{\xiddot}[0]{{\ddot{\xi}}}
\newcommand{\zetaddot}[0]{{\ddot{\zeta}}}

\newcommand{\BDelta}[0]{\boldsymbol{\Delta}}
\newcommand{\BGamma}[0]{\boldsymbol{\Gamma}}
\newcommand{\BLambda}[0]{\boldsymbol{\Lambda}}
\newcommand{\BOmega}[0]{\boldsymbol{\Omega}}
\newcommand{\BPhi}[0]{\boldsymbol{\Phi}}
\newcommand{\BPi}[0]{\boldsymbol{\Pi}}
\newcommand{\BPsi}[0]{\boldsymbol{\Psi}}
\newcommand{\BSigma}[0]{\boldsymbol{\Sigma}}
\newcommand{\BTheta}[0]{\boldsymbol{\Theta}}
\newcommand{\BUpsilon}[0]{\boldsymbol{\Upsilon}}
\newcommand{\BXi}[0]{\boldsymbol{\Xi}}
\newcommand{\Balpha}[0]{\boldsymbol{\alpha}}
\newcommand{\Bbeta}[0]{\boldsymbol{\beta}}
\newcommand{\Bchi}[0]{\boldsymbol{\chi}}
\newcommand{\Bdelta}[0]{\boldsymbol{\delta}}
\newcommand{\Bepsilon}[0]{\boldsymbol{\epsilon}}
\newcommand{\Beta}[0]{\boldsymbol{\eta}}
\newcommand{\Bgamma}[0]{\boldsymbol{\gamma}}
\newcommand{\Bkappa}[0]{\boldsymbol{\kappa}}
\newcommand{\Blambda}[0]{\boldsymbol{\lambda}}
\newcommand{\Bmu}[0]{\boldsymbol{\mu}}
\newcommand{\Bnu}[0]{\boldsymbol{\nu}}
%\newcommand{\Bomega}[0]{\boldsymbol{\omega}}
\newcommand{\Bphi}[0]{\boldsymbol{\phi}}
\newcommand{\Bpi}[0]{\boldsymbol{\pi}}
\newcommand{\Bpsi}[0]{\boldsymbol{\psi}}
\newcommand{\Brho}[0]{\boldsymbol{\rho}}
\newcommand{\Bsigma}[0]{\boldsymbol{\sigma}}
%\newcommand{\Btau}[0]{\boldsymbol{\tau}}
%\newcommand{\Btheta}[0]{\boldsymbol{\theta}}
\newcommand{\Bupsilon}[0]{\boldsymbol{\upsilon}}
\newcommand{\Bvarepsilon}[0]{\boldsymbol{\varepsilon}}
\newcommand{\Bvarphi}[0]{\boldsymbol{\varphi}}
\newcommand{\Bvarpi}[0]{\boldsymbol{\varpi}}
\newcommand{\Bvarrho}[0]{\boldsymbol{\varrho}}
\newcommand{\Bvarsigma}[0]{\boldsymbol{\varsigma}}
\newcommand{\Bvartheta}[0]{\boldsymbol{\vartheta}}
\newcommand{\Bxi}[0]{\boldsymbol{\xi}}
\newcommand{\Bzeta}[0]{\boldsymbol{\zeta}}
%
%</bold and dot greek symbols>
%<infrequent>
%
%\newcommand{\AreaOp}[1]{\AName_{#1}}
%\newcommand{\Babs}[0]{\abs{\BB}}
%\newcommand{\Bcap}[0]{\hat{\BB}}
%\newcommand{\BrPrimeRej}[0]{\rcap(\rcap \wedge \Br')}
%\newcommand{\CA}[0]{\mathcal{A}}
%\newcommand{\Cos}[1]{\cos{\left({#1}\right)}}
%\newcommand{\Det}[1] {\abs{#1}}
%\newcommand{\Dsq}[2] {\frac {\partial^2 {#1}} {\partial {#2}^2}}
%\newcommand{\Exp}[1]{\exp{\left({#1}\right)}}
%\newcommand{\Norm}[1]{\left\lVert{#1}\right\rVert}
%\newcommand{\Sin}[1]{\sin{\left({#1}\right)}}
%\newcommand{\T}[0]{\text{T}}
%\newcommand{\VolumeOp}[1]{\VName_{#1}}
%\newcommand{\agrad}[0]{\Ba \cdot \nabla}
%\newcommand{\alphacap}[0]{\hat{\boldsymbol{\alpha}}}
%\newcommand{\Fcap}[0]{\hat{\BF}}
%\newcommand{\bithree}[0]{{\Bi}_3}
%\newcommand{\bxa}[0]{\Bx\Ba}
%\newcommand{\coordvec}[2]{
%\newcommand{\costheta}[0]{\acap \cdot \xcap}
%\newcommand{\ddt}[1]{\ddot{#1}}
%\newcommand{\ddu}[1] {\frac {d{#1}} {du}}
%\newcommand{\dsqxj}[2] {\frac {\partial^2 {#1}} {\partial {x_{#2}}^2}}
%\newcommand{\dtheta}[1]{\frac{d {#1}}{d \theta}}
%\newcommand{\dt}[1]{\dot{#1}}
%\newcommand{\dt}[1]{\frac{d {#1}}{dt}}
%\newcommand{\dxj}[2] {\frac {\partial {#1}} {\partial {x_{#2}}}}
%\newcommand{\halfPhi}[0]{\frac{\phi}{2}}
%\newcommand{\half}[0]{\inv{2}}
%\newcommand{\inv}[1]{\frac{1}{#1}}
%\newcommand{\laplacian}[0]{\nabla^2}
%\newcommand{\matrixoftx}[3]{
%\newcommand{\nrrp}[0]{\norm{\rcap \wedge \Br'}}
%\newcommand{\oiint}{\bigcirc \hspace{-1.4em} \int \hspace{-.8em} \int}
%\newcommand{\transpose}[1]{{#1}^{\text{T}}}
%\newcommand{\transpose}[1]{{{#1}^{\TextTranspose}}}
%\newcommand{\transpose}[1]{{{#1}^{\text{T}}}}
%\newcommand{\barA}[0]{\bar{A}}
%\newcommand{\qbar}[0]{\bar{q}}
%\newcommand{\qdotbar}[0]{\dot{\bar{q}}}
%
%</infrequent>





%
% Copyright � 2012 Peeter Joot.  All Rights Reserved.
% Licenced as described in the file LICENSE under the root directory of this GIT repository.
%

% 
% 
\DeclareMathOperator{\Div}{div}
\DeclareMathOperator{\Mod}{mod}
\DeclareMathOperator{\PV}{PV}
\DeclareMathOperator{\Prob}{Prob}
\DeclareMathOperator{\rank}{rank}
\DeclareMathOperator{\sgn}{sgn}
\DeclareMathOperator{\sinc}{sinc}
%\DeclareMathOperator{\Atan2}{atan2}
\DeclareMathOperator{\atan}{atan}


\newcommand{\expectation}[1]{\langle{#1}\rangle}
%\newcommand{\gpgradefour}[1] {\gpgrade{#1}{4}}
%\newcommand{\gpgradeone}[1] {\gpgrade{#1}{1}}
%\newcommand{\gpgradethree}[1] {\gpgrade{#1}{3}}
%\newcommand{\gpgradetwo}[1] {\gpgrade{#1}{2}}
%\newcommand{\gpgradezero}[1] {\gpgrade{#1}{}}
%\newcommand{\gpgrade}[2] {{\left\langle{{#1}}\right\rangle}_{#2}}
%\newcommand{\grad}[0]{\boldsymbol{\nabla}}
%\newcommand{\grad}[0]{\nabla}


\newcommand{\ketbra}[2]{\ket{#1}\bra{#2}}
\newcommand{\ket}[1]{\lvert {#1} \rangle}
%\newcommand{\norm}[1]{\lVert#1\rVert}
\newcommand{\questionEquals}[0]{\stackrel{?}{=}}
\newcommand{\rightshift}[0]{\gg}
%\newcommand{\spacegrad}[0]{\boldsymbol{\nabla}}
\newcommand{\symmetric}[2]{{\left\{{#1},{#2}\right\}}}
\newcommand{\antisymmetric}[2]{\left[{#1},{#2}\right]}

%\newcommand{\Abs}[1]{\left\lvert{#1}\right\rvert}

%\newcommand{\BB}[0]{\mathbf{B}}
%\newcommand{\BE}[0]{\mathbf{E}}
%\newcommand{\BF}[0]{\mathbf{F}}
%\newcommand{\BS}[0]{\mathbf{S}}
%\newcommand{\BV}[0]{\mathbf{V}}
%\newcommand{\Bj}[0]{\mathbf{j}}

\newcommand{\BraOpKet}[3]{\bra{#1} \hat{#2} \ket{#3} }
%\newcommand{\Brho}[0]{\boldsymbol{\rho}}
\newcommand{\CC}[0]{c^2}
\newcommand{\Cos}[1]{\cos{\left({#1}\right)}}

% not working anymore.  think it's a conflicting macro for \not.
% compared to original usage in klien_gordon.ltx
%
%\newcommand{\Dslash}[0]{{\not}D}
%\newcommand{\Dslash}[0]{{\not{}}D}
% switched to cancel in macros.tex
%\newcommand{\Dslash}[0]{D\!\!\!/}

\newcommand{\Expectation}[1]{\left\langle {#1} \right\rangle}
\newcommand{\Exp}[1]{\exp{\left({#1}\right)}}
\newcommand{\FF}[0]{\mathcal{F}}
\newcommand{\FM}[0]{\inv{\sqrt{2\pi\hbar}}}
\newcommand{\IIinf}[0]{ \int_{-\infty}^\infty }
\newcommand{\Innerprod}[2]{\left\langle{#1}, {#2}\right\rangle}
%\newcommand{\LL}[0]{\mathcal{L}}

%\newcommand{\PD}[2] {\frac {\partial #2} {\partial #1}}

% backwards from ../peeterj_macros2:
\newcommand{\PDb}[2]{ \frac{\partial{#1}}{\partial {#2}} }

%\newcommand{\PDD}[3]{\frac{\partial^2 {#3}}{\partial {#1}\partial {#2}}}
\newcommand{\PDN}[3]{\frac{\partial^{#3} {#2}}{\partial {#1}^{#3}}}

\newcommand{\PDSq}[2]{\frac{\partial^2 {#2}}{\partial {#1}^2}}
\newcommand{\PDsQ}[2]{\frac{\partial^2 {#2}}{\partial^2 {#1}}}

\newcommand{\Sch}[0]{{Schr\"{o}dinger} }
\newcommand{\Sin}[1]{\sin{\left({#1}\right)}}
\newcommand{\Sw}[0]{\mathcal{S}}
%\newcommand{\T}[0]{\text{T}}
\newcommand{\T}[0]{{\text{T}}}

\newcommand{\braket}[2]{\langle{#1} \vert {#2}\rangle}
\newcommand{\bra}[1]{\langle {#1} \rvert}
\newcommand{\curl}[0]{\grad \times}
\newcommand{\delambert}[0]{\sum_{\alpha = 1}^4{\PDSq{x_\alpha}{}}}
\newcommand{\delsquared}[0]{\nabla^2}
\newcommand{\diverg}[0]{\grad \cdot}

\newcommand{\halfPhi}[0]{\frac{\phi}{2}}
\newcommand{\hatH}[0]{\hat{H}}
\newcommand{\hatS}[0]{\hat{S}}
\newcommand{\hatk}[0]{\hat{k}}
\newcommand{\hatp}[0]{\hat{p}}
\newcommand{\hatx}[0]{\hat{x}}


\newcommand{\Rdot}[0]{\dot{R}}
%\newcommand{\addot}[0]{\ddot{a}}
%\newcommand{\adot}[0]{\dot{a}}
%\newcommand{\fddot}[0]{\ddot{f}}
%\newcommand{\fdot}[0]{\dot{f}}
%\newcommand{\bddot}[0]{\ddot{b}}
%\newcommand{\bdot}[0]{\dot{b}}
\newcommand{\ddotOmega}[0]{\ddot{\Omega}}
\newcommand{\ddotalpha}[0]{\ddot{\alpha}}
\newcommand{\ddotomega}[0]{\ddot{\omega}}
\newcommand{\ddotphi}[0]{\ddot{\phi}}
\newcommand{\ddotpsi}[0]{\ddot{\psi}}
\newcommand{\ddottheta}[0]{\ddot{\theta}}
\newcommand{\dotOmega}[0]{\dot{\Omega}}
\newcommand{\dotalpha}[0]{\dot{\alpha}}
\newcommand{\dotomega}[0]{\dot{\omega}}
\newcommand{\dotphi}[0]{\dot{\phi}}
\newcommand{\dotpsi}[0]{\dot{\psi}}
\newcommand{\dottheta}[0]{\dot{\theta}}
%\newcommand{\pddot}[0]{\ddot{p}}
%\newcommand{\pdot}[0]{\dot{p}}
%\newcommand{\qddot}[0]{\ddot{q}}
%\newcommand{\qdot}[0]{\dot{q}}
%\newcommand{\rddot}[0]{\ddot{r}}
%\newcommand{\rdot}[0]{\dot{r}}
%\newcommand{\tddot}[0]{\ddot{t}}
%\newcommand{\tdot}[0]{\dot{t}}
%\newcommand{\uddot}[0]{\ddot{u}}
%\newcommand{\udot}[0]{\dot{u}}
%\newcommand{\xddot}[0]{\ddot{x}}
%\newcommand{\xdot}[0]{\dot{x}}
%\newcommand{\yddot}[0]{\ddot{y}}
%\newcommand{\ydot}[0]{\dot{y}}
%\newcommand{\zddot}[0]{\ddot{z}}
%\newcommand{\zdot}[0]{\dot{z}}








%-------------------------------------------------------------------
% ORIGINS:
%
% bohm11.tex

%\DeclareMathOperator{\sgn}{sgn}
%\newcommand{\PDSq}[2]{\frac{\partial^2 {#2}}{\partial {#1}^2}}
%\newcommand{\PDN}[3]{\frac{\partial^{#3} {#2}}{\partial {#1}^{#3}}}
%\DeclareMathOperator{\sinc}{sinc}
%\DeclareMathOperator{\PV}{PV}
%\newcommand{\FF}[0]{\mathcal{F}}
%\newcommand{\Sw}[0]{\mathcal{S}}
%\newcommand{\IIinf}[0]{ \int_{-\infty}^\infty }
%\newcommand{\FM}[0]{\inv{\sqrt{2\pi\hbar}}}
%\newcommand{\expectation}[1]{\langle{#1}\rangle}
%
%

% bohm_ch10.tex

%\DeclareMathOperator{\sgn}{sgn}
%\newcommand{\expectation}[1]{\langle{#1}\rangle}
%\newcommand{\IIinf}[0]{ \int_{-\infty}^\infty }
%\DeclareMathOperator{\PV}{PV}
%
%

% bohm_ch9.tex

%\newcommand{\PDSq}[2]{\frac{\partial^2 {#2}}{\partial {#1}^2}}
%\newcommand{\PDN}[3]{\frac{\partial^{#3} {#2}}{\partial {#1}^{#3}}}
%\DeclareMathOperator{\sinc}{sinc}
%\DeclareMathOperator{\PV}{PV}
%\newcommand{\FF}[0]{\mathcal{F}}
%\newcommand{\Sw}[0]{\mathcal{S}}
%\newcommand{\IIinf}[0]{ \int_{-\infty}^\infty }
%\newcommand{\FM}[0]{\inv{\sqrt{2\pi\hbar}}}
%\newcommand{\expectation}[1]{\langle{#1}\rangle}
%
%

% commutator_herm.tex

%\newcommand{\symmetric}[2]{{\left\{{#1},{#2}\right\}}}
%\newcommand{\antisymmetric}[2]{\left[{#1},{#2}\right]}
%
%%\newcommand{\ket}[1]{\lvert {#1} \rangle}
%%\newcommand{\bra}[1]{\langle {#1} \rvert}
%%\newcommand{\braket}[2]{\langle{#1} \vert {#2}\rangle}
%%\newcommand{\ketbra}[2]{\ket{#1}\bra{#2}}
%%\newcommand{\BraOpKet}[3]{\bra{#1} \hat{#2} \ket{#3} }
%%\newcommand{\Innerprod}[2]{\left\langle{#1}, {#2}\right\rangle}
%\newcommand{\Expectation}[1]{\left\langle {#1} \right\rangle}
%
%

% delta_ortho_series.tex

%\newcommand{\IIinf}[0]{ \int_{-\infty}^\infty }
%\newcommand{\ket}[1]{\lvert {#1} \rangle}
%\newcommand{\bra}[1]{\langle {#1} \rvert}
%\newcommand{\braket}[2]{\langle{#1} \vert {#2}\rangle}
%\newcommand{\ketbra}[2]{\ket{#1}\bra{#2}}
%\newcommand{\BraOpKet}[3]{\bra{#1} \hat{#2} \ket{#3} }
%\newcommand{\Innerprod}[2]{\left\langle{#1}, {#2}\right\rangle}
%
%

% distributions.tex

%\newcommand{\PDSq}[2]{\frac{\partial^2 {#2}}{\partial {#1}^2}}
%\DeclareMathOperator{\sinc}{sinc}
%\DeclareMathOperator{\PV}{PV}
%\newcommand{\FF}[0]{\mathcal{F}}
%\newcommand{\Sw}[0]{\mathcal{S}}
%\newcommand{\IIinf}[0]{ \int_{-\infty}^\infty }
%
%

% ehrenfest.tex

%\newcommand{\PDSq}[2]{\frac{\partial^2 {#2}}{\partial {#1}^2}}
%
%

% fletcher.tex

%\DeclareMathOperator{\Div}{div}
%\DeclareMathOperator{\Mod}{mod}
%\newcommand{\rightshift}[0]{\gg}
%\newcommand{\questionEquals}[0]{\stackrel{?}{=}}
%
%

% fvec.tex

%\newcommand{\grad}[0]{\nabla}
%\newcommand{\PD}[2]{ \frac{\partial{#1}}{\partial {#2}} }
%
%

% gacs_q8_8.tex

%\newcommand{\halfPhi}[0]{\frac{\phi}{2}}
%\newcommand{\Sin}[1]{\sin{\left({#1}\right)}}
%\newcommand{\Cos}[1]{\cos{\left({#1}\right)}}
%\newcommand{\Exp}[1]{\exp{\left({#1}\right)}}
%
%

% goldstein_ch1_2.tex

%\newcommand{\spacegrad}[0]{\boldsymbol{\nabla}}
%\newcommand{\Brho}[0]{\boldsymbol{\rho}}
%\newcommand{\LL}[0]{\mathcal{L}}
%\newcommand{\Abs}[1]{\left\lvert{#1}\right\rvert}
%\newcommand{\qdot}[0]{\dot{q}}
%\newcommand{\qddot}[0]{\ddot{q}}
%\newcommand{\xdot}[0]{\dot{x}}
%\newcommand{\xddot}[0]{\ddot{x}}
%\newcommand{\ydot}[0]{\dot{y}}
%\newcommand{\yddot}[0]{\ddot{y}}
%\newcommand{\dotalpha}[0]{\dot{\alpha}}
%\newcommand{\ddotalpha}[0]{\ddot{\alpha}}
%\newcommand{\dottheta}[0]{\dot{\theta}}
%\newcommand{\ddottheta}[0]{\ddot{\theta}}
%\newcommand{\dotphi}[0]{\dot{\phi}}
%\newcommand{\ddotphi}[0]{\ddot{\phi}}
%% == \partial_{#1} {#2}
%\newcommand{\PD}[2]{\frac{\partial {#2}}{\partial {#1}}}
%\newcommand{\PDD}[3]{\frac{\partial^2 {#3}}{\partial {#1}\partial {#2}}}
%
%% <grade selection>
%%
%\newcommand{\gpgrade}[2] {{\left\langle{{#1}}\right\rangle}_{#2}}
%
%\newcommand{\gpgradezero}[1] {\gpgrade{#1}{}}
%%\newcommand{\gpscalargrade}[1] {{\left\langle{{#1}}\right\rangle}}
%%\newcommand{\gpgradezero}[1] {\gpgrade{#1}{0}}
%
%%\newcommand{\gpgradeone}[1] {{\left\langle{{#1}}\right\rangle}_{1}}
%\newcommand{\gpgradeone}[1] {\gpgrade{#1}{1}}
%
%\newcommand{\gpgradetwo}[1] {\gpgrade{#1}{2}}
%\newcommand{\gpgradethree}[1] {\gpgrade{#1}{3}}
%\newcommand{\gpgradefour}[1] {\gpgrade{#1}{4}}
%%
%% </grade selection>
%
%
%

% harmonic_osc.tex

%\newcommand{\IIinf}[0]{ \int_{-\infty}^\infty }
%
%

% klein_gordon.tex

%\newcommand{\PDSq}[2]{\frac{\partial^2 {#2}}{\partial {#1}^2}}
%%\newcommand{\Dslash}[0]{D\!\!\!/}
%\newcommand{\Dslash}[0]{{\not}D}
%
%

% matrix_to_operator.tex

%\newcommand{\T}[0]{{\text{T}}}
%
%

% maxwell.tex

%\newcommand{\norm}[1]{\lVert#1\rVert}
%\newcommand{\grad}[0]{\boldsymbol{\nabla}}
%\newcommand{\curl}[0]{\grad \times}
%\newcommand{\diverg}[0]{\grad \cdot}
%\newcommand{\delsquared}[0]{\nabla^2}
%\newcommand{\delambert}[0]{\sum_{\alpha = 1}^4{\PDSq{x_\alpha}{}}}
%
%% partial derivative of #1 wrt. #2:
%\newcommand{\PD}[2] {\frac {\partial #2} {\partial #1}}
%% second partial derivative of #1 wrt. #2:
%\newcommand{\PDSq}[2] {\frac {\partial^2 #2} {\partial {#1}^2}}
%
%%
%% shorthand for bold symbols:
%%
%\newcommand{\Bj}[0]{\mathbf{j}}
%\newcommand{\BB}[0]{\mathbf{B}}
%\newcommand{\BE}[0]{\mathbf{E}}
%\newcommand{\BF}[0]{\mathbf{F}}
%\newcommand{\BS}[0]{\mathbf{S}}
%\newcommand{\BV}[0]{\mathbf{V}}
%
%

% mp_inverse_svd_rough_notes.tex

%\newcommand{\T}[0]{\text{T}}
%\DeclareMathOperator{\rank}{rank}
%
%

% outermorphism_det.tex

%\newcommand{\gpgrade}[2] {{\left\langle{{#1}}\right\rangle}_{#2}}
%\newcommand{\gpgradeone}[1] {\gpgrade{#1}{1}}
%\newcommand{\gpgradetwo}[1] {\gpgrade{#1}{2}}
%
%

% pauli_qm_relativity_intro.tex

%\newcommand{\Sch}[0]{{Schr\"{o}dinger} }
%
%

% pe.tex

%
%\newcommand{\grad}[0]{\nabla}

% qm_susskind.tex

%\newcommand{\ket}[1]{\lvert {#1} \rangle}
%\newcommand{\bra}[1]{\langle {#1} \rvert}
%\newcommand{\braket}[2]{\langle{#1} \vert {#2}\rangle}
%\newcommand{\ketbra}[2]{\ket{#1}\bra{#2}}
%\newcommand{\BraOpKet}[3]{\bra{#1} \hat{#2} \ket{#3} }
%\newcommand{\hatH}[0]{\hat{H}}
%\newcommand{\hatS}[0]{\hat{S}}
%\newcommand{\hatk}[0]{\hat{k}}
%\newcommand{\hatx}[0]{\hat{x}}
%\newcommand{\hatp}[0]{\hat{p}}
%\DeclareMathOperator{\Prob}{Prob}
%
%

% schwartzchild_metric.tex

%\newcommand{\grad}[0]{\nabla}
%\newcommand{\Abs}[1]{\left\lvert{#1}\right\rvert}
%\newcommand{\spacegrad}[0]{\boldsymbol{\nabla}}
%\newcommand{\LL}[0]{\mathcal{L}}
%\newcommand{\PD}[2]{\frac{\partial {#2}}{\partial {#1}}}
%\newcommand{\PDsQ}[2]{\frac{\partial^2 {#2}}{\partial^2 {#1}}}
%\newcommand{\dotalpha}[0]{\dot{\alpha}}
%\newcommand{\ddotalpha}[0]{\ddot{\alpha}}
%
%\newcommand{\dotomega}[0]{\dot{\omega}}
%\newcommand{\ddotomega}[0]{\ddot{\omega}}
%
%\newcommand{\dotOmega}[0]{\dot{\Omega}}
%\newcommand{\ddotOmega}[0]{\ddot{\Omega}}
%
%\newcommand{\CC}[0]{c^2}
%
%\newcommand{\dottheta}[0]{\dot{\theta}}
%\newcommand{\ddottheta}[0]{\ddot{\theta}}
%
%\newcommand{\dotpsi}[0]{\dot{\psi}}
%\newcommand{\ddotpsi}[0]{\ddot{\psi}}
%
%\newcommand{\adot}[0]{\dot{a}}
%\newcommand{\addot}[0]{\ddot{a}}
%\newcommand{\udot}[0]{\dot{u}}
%\newcommand{\uddot}[0]{\ddot{u}}
%\newcommand{\fdot}[0]{\dot{f}}
%\newcommand{\fddot}[0]{\ddot{f}}
%\newcommand{\bdot}[0]{\dot{b}}
%\newcommand{\bddot}[0]{\ddot{b}}
%\newcommand{\qdot}[0]{\dot{q}}
%\newcommand{\qddot}[0]{\ddot{q}}
%\newcommand{\tdot}[0]{\dot{t}}
%\newcommand{\tddot}[0]{\ddot{t}}
%
%\newcommand{\Rdot}[0]{\dot{R}}
%
%\newcommand{\pdot}[0]{\dot{p}}
%\newcommand{\pddot}[0]{\ddot{p}}
%
%\newcommand{\xdot}[0]{\dot{x}}
%\newcommand{\xddot}[0]{\ddot{x}}
%
%\newcommand{\zdot}[0]{\dot{z}}
%\newcommand{\zddot}[0]{\ddot{z}}
%
%\newcommand{\rdot}[0]{\dot{r}}
%\newcommand{\rddot}[0]{\ddot{r}}
%
%

% shear.tex

%\newcommand{\gpgrade}[2] {{\left\langle{{#1}}\right\rangle}_{#2}}
%
%

% tong_mf1.tex

%\newcommand{\Abs}[1]{\left\lvert{#1}\right\rvert}
%\newcommand{\grad}[0]{\nabla}
%\newcommand{\LL}[0]{\mathcal{L}}
%
%\newcommand{\dotalpha}[0]{\dot{\alpha}}
%\newcommand{\ddotalpha}[0]{\ddot{\alpha}}
%
%\newcommand{\dotomega}[0]{\dot{\omega}}
%\newcommand{\ddotomega}[0]{\ddot{\omega}}
%
%\newcommand{\dottheta}[0]{\dot{\theta}}
%\newcommand{\ddottheta}[0]{\ddot{\theta}}
%
%\newcommand{\dotpsi}[0]{\dot{\psi}}
%\newcommand{\ddotpsi}[0]{\ddot{\psi}}
%
%\newcommand{\qdot}[0]{\dot{q}}
%\newcommand{\qddot}[0]{\ddot{q}}
%
%\newcommand{\Rdot}[0]{\dot{R}}
%
%\newcommand{\pdot}[0]{\dot{p}}
%\newcommand{\pddot}[0]{\ddot{p}}
%
%\newcommand{\xdot}[0]{\dot{x}}
%\newcommand{\xddot}[0]{\ddot{x}}
%
%\newcommand{\zdot}[0]{\dot{z}}
%\newcommand{\zddot}[0]{\ddot{z}}
%
%\newcommand{\rdot}[0]{\dot{r}}
%\newcommand{\rddot}[0]{\ddot{r}}
%
%% == \partial_{#1} {#2}
%\newcommand{\PD}[2]{\frac{\partial {#2}}{\partial {#1}}}
%\newcommand{\PDD}[3]{\frac{\partial^2 {#3}}{\partial {#1}\partial {#2}}}
%
%

% wavepacket.tex

%\newcommand{\PDSq}[2]{\frac{\partial^2 {#2}}{\partial {#1}^2}}
%\newcommand{\IIinf}[0]{ \int_{-\infty}^\infty }
%
%

% wavevariation.tex

%\newcommand{\PDSq}[2]{\frac{\partial^2 {#2}}{\partial {#1}^2}}
%
%

% qm_barrier
%\DeclareMathOperator{\Atan2}{atan2}
%\DeclareMathOperator{\atan}{atan}

% twobodies.tex
%\DeclareMathOperator{\sgn}{sgn}


% sr_lagrangian_q.tex

%\newcommand{\PD}[2]{\frac{\partial {#2}}{\partial {#1}}}
%\newcommand{\xdot}[0]{\dot{x}}
%\newcommand{\xddot}[0]{\ddot{x}}

% stub_em_fields.tex

%\newcommand{\EE}[0]{\boldsymbol{\mathcal{E}}}
%\newcommand{\HH}[0]{\boldsymbol{\mathcal{H}}}
%\newcommand{\PDSq}[2]{\frac{\partial^2 {#2}}{\partial {#1}^2}}

% long_wire_q.tex

%\newcommand{\grad}[0]{\nabla}

% lorentz_tx_em_potential.tex
%\newcommand{\LL}[0]{\mathcal{L}}
%\newcommand{\grad}[0]{\nabla}
%\newcommand{\pdot}[0]{\dot{p}}
%\newcommand{\pddot}[0]{\ddot{p}}

%------------------------------------------------------
% cross_old.tex

%%
%% shorthand for bold symbols, convenient for vectors and matrices
%%
%\newcommand{\Ba}[0]{\mathbf{a}}
%\newcommand{\Bb}[0]{\mathbf{b}}
%\newcommand{\Bc}[0]{\mathbf{c}}
%\newcommand{\Bd}[0]{\mathbf{d}}
%\newcommand{\Be}[0]{\mathbf{e}}
%\newcommand{\Bf}[0]{\mathbf{f}}
%\newcommand{\Bg}[0]{\mathbf{g}}
%\newcommand{\Bh}[0]{\mathbf{h}}
%\newcommand{\Bi}[0]{\mathbf{i}}
%\newcommand{\Bj}[0]{\mathbf{j}}
%\newcommand{\Bk}[0]{\mathbf{k}}
%\newcommand{\Bl}[0]{\mathbf{l}}
%\newcommand{\Bm}[0]{\mathbf{m}}
%\newcommand{\Bn}[0]{\mathbf{n}}
%\newcommand{\Bo}[0]{\mathbf{o}}
%\newcommand{\Bp}[0]{\mathbf{p}}
%\newcommand{\Bq}[0]{\mathbf{q}}
%\newcommand{\Br}[0]{\mathbf{r}}
%\newcommand{\Bs}[0]{\mathbf{s}}
%\newcommand{\Bt}[0]{\mathbf{t}}
%\newcommand{\Bu}[0]{\mathbf{u}}
%\newcommand{\Bv}[0]{\mathbf{v}}
%\newcommand{\Bw}[0]{\mathbf{w}}
%\newcommand{\Bx}[0]{\mathbf{x}}
%\newcommand{\By}[0]{\mathbf{y}}
%\newcommand{\Bz}[0]{\mathbf{z}}
%\newcommand{\BA}[0]{\mathbf{A}}
%\newcommand{\BB}[0]{\mathbf{B}}
%\newcommand{\BC}[0]{\mathbf{C}}
%\newcommand{\BD}[0]{\mathbf{D}}
%\newcommand{\BE}[0]{\mathbf{E}}
%\newcommand{\BF}[0]{\mathbf{F}}
%\newcommand{\BG}[0]{\mathbf{G}}
%\newcommand{\BH}[0]{\mathbf{H}}
%\newcommand{\BI}[0]{\mathbf{I}}
%\newcommand{\BJ}[0]{\mathbf{J}}
%\newcommand{\BK}[0]{\mathbf{K}}
%\newcommand{\BL}[0]{\mathbf{L}}
%\newcommand{\BM}[0]{\mathbf{M}}
%\newcommand{\BN}[0]{\mathbf{N}}
%\newcommand{\BO}[0]{\mathbf{O}}
%\newcommand{\BP}[0]{\mathbf{P}}
%\newcommand{\BQ}[0]{\mathbf{Q}}
%\newcommand{\BR}[0]{\mathbf{R}}
%\newcommand{\BS}[0]{\mathbf{S}}
%\newcommand{\BT}[0]{\mathbf{T}}
%\newcommand{\BU}[0]{\mathbf{U}}
%\newcommand{\BV}[0]{\mathbf{V}}
%\newcommand{\BW}[0]{\mathbf{W}}
%\newcommand{\BX}[0]{\mathbf{X}}
%\newcommand{\BY}[0]{\mathbf{Y}}
%\newcommand{\BZ}[0]{\mathbf{Z}}
%
%\newcommand{\Bzero}[0]{\mathbf{0}}
%\newcommand{\Btheta}[0]{\boldsymbol{\theta}}
%\newcommand{\Btau}[0]{\boldsymbol{\tau}}
%\newcommand{\Bomega}[0]{\boldsymbol{\omega}}
%
%%
%% shorthand for unit vectors
%%
%\newcommand{\acap}[0]{\hat{\Ba}}
%\newcommand{\bcap}[0]{\hat{\Bb}}
%\newcommand{\ccap}[0]{\hat{\Bc}}
%\newcommand{\dcap}[0]{\hat{\Bd}}
%\newcommand{\ecap}[0]{\hat{\Be}}
%\newcommand{\fcap}[0]{\hat{\Bf}}
%\newcommand{\gcap}[0]{\hat{\Bg}}
%\newcommand{\hcap}[0]{\hat{\Bh}}
%\newcommand{\icap}[0]{\hat{\Bi}}
%\newcommand{\jCap}[0]{\hat{\Bj}}
%\newcommand{\kcap}[0]{\hat{\Bk}}
%\newcommand{\lcap}[0]{\hat{\Bl}}
%\newcommand{\mcap}[0]{\hat{\Bm}}
%\newcommand{\ncap}[0]{\hat{\Bn}}
%\newcommand{\ocap}[0]{\hat{\Bo}}
%\newcommand{\pcap}[0]{\hat{\Bp}}
%\newcommand{\qcap}[0]{\hat{\Bq}}
%\newcommand{\rcap}[0]{\hat{\Br}}
%\newcommand{\scap}[0]{\hat{\Bs}}
%\newcommand{\tcap}[0]{\hat{\Bt}}
%\newcommand{\ucap}[0]{\hat{\Bu}}
%\newcommand{\vcap}[0]{\hat{\Bv}}
%\newcommand{\wcap}[0]{\hat{\Bw}}
%\newcommand{\xcap}[0]{\hat{\Bx}}
%\newcommand{\ycap}[0]{\hat{\By}}
%\newcommand{\zcap}[0]{\hat{\Bz}}
%\newcommand{\thetacap}[0]{\hat{\Btheta}}
%
%%
%% to write R^n and C^n in a distinguishable fashion.  Perhaps change this
%% to the double lined characters upon figuring out how to do so.
%%
%\newcommand{\C}[1]{${\BC}^{#1}$}
%\newcommand{\R}[1]{${\BR}^{#1}$}
%
%%
%% various generally useful helpers
%%
%
%% derivative of #1 wrt. #2:
%\newcommand{\D}[2] {\frac {d#2} {d#1}}

%\newcommand{\inv}[1]{\frac{1}{#1}}
%\newcommand{\cross}[0]{\times}

%\newcommand{\abs}[1]{\lvert#1\rvert}
%\newcommand{\norm}[1]{\lVert#1\rVert}
%\newcommand{\innerprod}[2]{\langle{#1}, {#2}\rangle}
%\newcommand{\dotprod}[2]{#1 \cdot #2}
%\newcommand{\crossprod}[2]{#1 \cross #2}
%\newcommand{\tripleprod}[3]{\dotprod{\crossprod{#1}{#2}}{#3}}

%
% A few miscellaneous things specific to this document
%
%\newcommand{\crossop}[1]{\crossprod{#1}{}}

\newcommand{\PDP}[2]{\BP^{#1}\BD{\BP^{#2}}}
\newcommand{\PDPDP}[3]{\Bv^T\BP^{#1}\BD\BP^{#2}\BD\BP^{#3}\Bv}

\newcommand{\Mp}[0]{
\begin{bmatrix}
0 & 1 & 0 & 0 \\
0 & 0 & 1 & 0 \\
0 & 0 & 0 & 1 \\
1 & 0 & 0 & 0
\end{bmatrix}
}
\newcommand{\Mpp}[0]{
\begin{bmatrix}
0 & 0 & 1 & 0 \\
0 & 0 & 0 & 1 \\
1 & 0 & 0 & 0 \\
0 & 1 & 0 & 0
\end{bmatrix}
}
\newcommand{\Mppp}[0]{
\begin{bmatrix}
0 & 0 & 0 & 1 \\
1 & 0 & 0 & 0 \\
0 & 1 & 0 & 0 \\
0 & 0 & 1 & 0
\end{bmatrix}
}
\newcommand{\Mpu}[0]{
\begin{bmatrix}
u_1 & 0 & 0 & 0 \\
0 & u_2 & 0 & 0 \\
0 & 0 & u_3 & 0 \\
0 & 0 & 0 & u_4
\end{bmatrix}
}

%------------------------------------------------------



%\makeindex

\title{Misc Physics and Math Play.}

\author{Peeter Joot  \quad peeter.joot@gmail.com \\
{\small\em \copyright \  Draft date \today }}

\date{ May 31, 2009.  Last Revision: $Date: 2009/06/03 20:01:38 $ }
\begin{document}
\maketitle
 \addcontentsline{toc}{chapter}{Contents}
\pagenumbering{roman}
\tableofcontents
\listoffigures
\listoftables
\chapter*{Preface}\normalsize
  \addcontentsline{toc}{chapter}{Preface}
\pagestyle{plain}

%%
% Copyright � 2015 Peeter Joot.  All Rights Reserved.
% Licenced as described in the file LICENSE under the root directory of this GIT repository.
%

% 
%\chapter{Preface}
% this suppresses an explicit chapter number for the preface.
\chapter*{Preface}%\normalsize
  \addcontentsline{toc}{chapter}{Preface}

This document was produced while taking the Spring 2016, University of Toronto Microwave Circuits course (ECE1236H), taught by Prof.\ G. V. Eleftheriades.

\paragraph{Course Syllabus}

This course outlines the principles of designing modern microwave and RF circuits.  Signal-integrity issues in high-speed digital circuits are also examined.

\begin{itemize}
\item The wave equation.
\item Ideal transmission lines.
\item Transients on transmission-lines.
\item Planar transmission lines and introduction to MMIC's.
\item Designing with scattering parameters.
\item Planar power dividers.
\item Directional couplers.
\item Microwave filters.
\item Solid-state microwave amplifiers.
\item Noise.
\item Diode-mixers.
\item RF receiver chains.
\item Oscillators.
\end{itemize}

\withproblemsetsMessage{
\textcolor{Maroon}{
\textit{THIS DOCUMENT IS REDACTED.  THE PROBLEM SET SOLUTIONS AND ASSOCIATED MATHEMATICA CODE IS NOT VISIBLE.  PLEASE EMAIL ME FOR THE FULL VERSION IF YOU ARE NOT TAKING ECE1236.}
}
}

\paragraph{This document contains:}

\begin{itemize}
\item Lecture notes.
\item Personal notes exploring auxiliary details.
\item Worked practice problems.

\ifthenelse{\boolean{redacted}}%
{%
\item Links to Mathematica notebooks associated with the course material and problems (but not problem sets).
}%
{
\item Assigned problems.%
\item Links to Mathematica notebooks associated with problems and course material.%
}
\end{itemize}

%This set of notes is significantly different from my notes for many other classes.  With the class taught on slides (and some of those slides mirroring the text closely), I did not take live notes in class.
%These notes fill in details that I felt deserved clarification, contain problem sets solutions, as well as a number of loosely related musings on Geometric Algebra equivalents to some of the generalized concepts of electromagnetic theory encountered in this class (i.e. magnetic sources).
%
My thanks go to Professor Eleftheriades for teaching this course.

Peeter Joot  \quad peeterjoot@protonmail.com 

\pagestyle{headings}
\pagenumbering{arabic}

%-------------------------------------------------------

\part{algebra}
\documentclass{article}

\usepackage{amsmath}
\usepackage{mathpazo}

%
% shorthand for bold symbols, convenient for vectors and matrices
%
\newcommand{\Ba}[0]{\mathbf{a}}
\newcommand{\Bb}[0]{\mathbf{b}}
\newcommand{\Bc}[0]{\mathbf{c}}
\newcommand{\Bd}[0]{\mathbf{d}}
\newcommand{\Be}[0]{\mathbf{e}}
\newcommand{\Bf}[0]{\mathbf{f}}
\newcommand{\Bg}[0]{\mathbf{g}}
\newcommand{\Bh}[0]{\mathbf{h}}
\newcommand{\Bi}[0]{\mathbf{i}}
\newcommand{\Bj}[0]{\mathbf{j}}
\newcommand{\Bk}[0]{\mathbf{k}}
\newcommand{\Bl}[0]{\mathbf{l}}
\newcommand{\Bm}[0]{\mathbf{m}}
\newcommand{\Bn}[0]{\mathbf{n}}
\newcommand{\Bo}[0]{\mathbf{o}}
\newcommand{\Bp}[0]{\mathbf{p}}
\newcommand{\Bq}[0]{\mathbf{q}}
\newcommand{\Br}[0]{\mathbf{r}}
\newcommand{\Bs}[0]{\mathbf{s}}
\newcommand{\Bt}[0]{\mathbf{t}}
\newcommand{\Bu}[0]{\mathbf{u}}
\newcommand{\Bv}[0]{\mathbf{v}}
\newcommand{\Bw}[0]{\mathbf{w}}
\newcommand{\Bx}[0]{\mathbf{x}}
\newcommand{\By}[0]{\mathbf{y}}
\newcommand{\Bz}[0]{\mathbf{z}}
\newcommand{\BA}[0]{\mathbf{A}}
\newcommand{\BB}[0]{\mathbf{B}}
\newcommand{\BC}[0]{\mathbf{C}}
\newcommand{\BD}[0]{\mathbf{D}}
\newcommand{\BE}[0]{\mathbf{E}}
\newcommand{\BF}[0]{\mathbf{F}}
\newcommand{\BG}[0]{\mathbf{G}}
\newcommand{\BH}[0]{\mathbf{H}}
\newcommand{\BI}[0]{\mathbf{I}}
\newcommand{\BJ}[0]{\mathbf{J}}
\newcommand{\BK}[0]{\mathbf{K}}
\newcommand{\BL}[0]{\mathbf{L}}
\newcommand{\BM}[0]{\mathbf{M}}
\newcommand{\BN}[0]{\mathbf{N}}
\newcommand{\BO}[0]{\mathbf{O}}
\newcommand{\BP}[0]{\mathbf{P}}
\newcommand{\BQ}[0]{\mathbf{Q}}
\newcommand{\BR}[0]{\mathbf{R}}
\newcommand{\BS}[0]{\mathbf{S}}
\newcommand{\BT}[0]{\mathbf{T}}
\newcommand{\BU}[0]{\mathbf{U}}
\newcommand{\BV}[0]{\mathbf{V}}
\newcommand{\BW}[0]{\mathbf{W}}
\newcommand{\BX}[0]{\mathbf{X}}
\newcommand{\BY}[0]{\mathbf{Y}}
\newcommand{\BZ}[0]{\mathbf{Z}}

\newcommand{\Bzero}[0]{\mathbf{0}}
\newcommand{\Btheta}[0]{\boldsymbol{\theta}}
\newcommand{\Btau}[0]{\boldsymbol{\tau}}
\newcommand{\Bomega}[0]{\boldsymbol{\omega}}

%
% shorthand for unit vectors
%
\newcommand{\acap}[0]{\hat{\Ba}}
\newcommand{\bcap}[0]{\hat{\Bb}}
\newcommand{\ccap}[0]{\hat{\Bc}}
\newcommand{\dcap}[0]{\hat{\Bd}}
\newcommand{\ecap}[0]{\hat{\Be}}
\newcommand{\fcap}[0]{\hat{\Bf}}
\newcommand{\gcap}[0]{\hat{\Bg}}
\newcommand{\hcap}[0]{\hat{\Bh}}
\newcommand{\icap}[0]{\hat{\Bi}}
\newcommand{\jcap}[0]{\hat{\Bj}}
\newcommand{\kcap}[0]{\hat{\Bk}}
\newcommand{\lcap}[0]{\hat{\Bl}}
\newcommand{\mcap}[0]{\hat{\Bm}}
\newcommand{\ncap}[0]{\hat{\Bn}}
\newcommand{\ocap}[0]{\hat{\Bo}}
\newcommand{\pcap}[0]{\hat{\Bp}}
\newcommand{\qcap}[0]{\hat{\Bq}}
\newcommand{\rcap}[0]{\hat{\Br}}
\newcommand{\scap}[0]{\hat{\Bs}}
\newcommand{\tcap}[0]{\hat{\Bt}}
\newcommand{\ucap}[0]{\hat{\Bu}}
\newcommand{\vcap}[0]{\hat{\Bv}}
\newcommand{\wcap}[0]{\hat{\Bw}}
\newcommand{\xcap}[0]{\hat{\Bx}}
\newcommand{\ycap}[0]{\hat{\By}}
\newcommand{\zcap}[0]{\hat{\Bz}}
\newcommand{\thetacap}[0]{\hat{\Btheta}}

%
% to write R^n and C^n in a distinguishable fashion.  Perhaps change this
% to the double lined characters upon figuring out how to do so.
%
\newcommand{\C}[1]{$\mathbb{C}^{#1}$}
\newcommand{\R}[1]{$\mathbb{R}^{#1}$}

%
% various generally useful helpers
%

% derivative of #1 wrt. #2:
\newcommand{\D}[2] {\frac {d#2} {d#1}}

\newcommand{\inv}[1]{\frac{1}{#1}}
\newcommand{\cross}[0]{\times}

\newcommand{\abs}[1]{\lvert{#1}\rvert}
\newcommand{\norm}[1]{\lVert{#1}\rVert}
\newcommand{\innerprod}[2]{\langle{#1}, {#2}\rangle}
\newcommand{\dotprod}[2]{{#1} \cdot {#2}}
\newcommand{\bdotprod}[2]{\left({#1} \cdot {#2}\right)}
\newcommand{\crossprod}[2]{{#1} \cross {#2}}
\newcommand{\tripleprod}[3]{\dotprod{\left(\crossprod{#1}{#2}\right)}{#3}}

\DeclareMathOperator{\Proj}{Proj}
\DeclareMathOperator{\Span}{span}
\DeclareMathOperator{\Sgn}{sgn}
\DeclareMathOperator{\Area}{Area}
\DeclareMathOperator{\Volume}{Volume}

%
% A few miscellaneous things specific to this document
%
\newcommand{\crossop}[1]{\crossprod{#1}{}}

% R2 vector.
\newcommand{\VectorTwo}[2]{
\begin{bmatrix}
 {#1} \\
 {#2}
\end{bmatrix}
}

\newcommand{\VectorN}[1]{
\begin{bmatrix}
{#1}_1 \\
{#1}_2 \\
\vdots \\
{#1}_N \\
\end{bmatrix}
}

\newcommand{\DETuvij}[4]{
\begin{vmatrix}
 {#1}_{#3} & {#1}_{#4} \\
 {#2}_{#3} & {#2}_{#4}
\end{vmatrix}
}

\newcommand{\DETuvwijk}[6]{
\begin{vmatrix}
 {#1}_{#4} & {#1}_{#5} & {#1}_{#6} \\
 {#2}_{#4} & {#2}_{#5} & {#2}_{#6} \\
 {#3}_{#4} & {#3}_{#5} & {#3}_{#6}
\end{vmatrix}
}

\newcommand{\DETuvwxijkl}[8]{
\begin{vmatrix}
 {#1}_{#5} & {#1}_{#6} & {#1}_{#7} & {#1}_{#8} \\
 {#2}_{#5} & {#2}_{#6} & {#2}_{#7} & {#2}_{#8} \\
 {#3}_{#5} & {#3}_{#6} & {#3}_{#7} & {#3}_{#8} \\
 {#4}_{#5} & {#4}_{#6} & {#4}_{#7} & {#4}_{#8} \\
\end{vmatrix}
}

%\newcommand{\DETuvwxyijklm}[10]{
%\begin{vmatrix}
% {#1}_{#6} & {#1}_{#7} & {#1}_{#8} & {#1}_{#9} & {#1}_{#10} \\
% {#2}_{#6} & {#2}_{#7} & {#2}_{#8} & {#2}_{#9} & {#2}_{#10} \\
% {#3}_{#6} & {#3}_{#7} & {#3}_{#8} & {#3}_{#9} & {#3}_{#10} \\
% {#4}_{#6} & {#4}_{#7} & {#4}_{#8} & {#4}_{#9} & {#4}_{#10} \\
% {#5}_{#6} & {#5}_{#7} & {#5}_{#8} & {#5}_{#9} & {#5}_{#10}
%\end{vmatrix}
%}

% R3 vector.
\newcommand{\VectorThree}[3]{
\begin{bmatrix}
 {#1} \\
 {#2} \\
 {#3}
\end{bmatrix}
}


%<misc>
%
\newcommand{\Abs}[1]{{\left\lvert{#1}\right\rvert}}
\newcommand{\spacegrad}[0]{\boldsymbol{\nabla}}
\newcommand{\grad}[0]{\nabla}
\newcommand{\LL}[0]{\mathcal{L}}

% == \partial_{#1} {#2}
\newcommand{\PD}[2]{\frac{\partial {#2}}{\partial {#1}}}
% inline variant
\newcommand{\PDi}[2]{{\partial {#2}}/{\partial {#1}}}

\newcommand{\PDD}[3]{\frac{\partial^2 {#3}}{\partial {#1}\partial {#2}}}
%\newcommand{\PDd}[2]{\frac{\partial^2 {#2}}{{\partial{#1}}^2}}
\newcommand{\PDsq}[2]{\frac{\partial^2 {#2}}{(\partial {#1})^2}}

\newcommand{\Partial}[2]{\frac{\partial {#1}}{\partial {#2}}}
\DeclareMathOperator{\RejName}{Rej}
\newcommand{\Rej}[2]{\RejName_{#1}\left( {#2} \right)}
\newcommand{\Rm}[1]{\mathbb{R}^{#1}}
\newcommand{\Cm}[1]{\mathbb{C}^{#1}}
\newcommand{\conj}[0]{{*}}

%</misc>

% <grade selection>
%
\newcommand{\gpgrade}[2] {{\left\langle{{#1}}\right\rangle}_{#2}}

\newcommand{\gpgradezero}[1] {\gpgrade{#1}{}}
%\newcommand{\gpscalargrade}[1] {{\left\langle{{#1}}\right\rangle}}
%\newcommand{\gpgradezero}[1] {\gpgrade{#1}{0}}

%\newcommand{\gpgradeone}[1] {{\left\langle{{#1}}\right\rangle}_{1}}
\newcommand{\gpgradeone}[1] {\gpgrade{#1}{1}}

\newcommand{\gpgradetwo}[1] {\gpgrade{#1}{2}}
\newcommand{\gpgradethree}[1] {\gpgrade{#1}{3}}
\newcommand{\gpgradefour}[1] {\gpgrade{#1}{4}}
%
% </grade selection>



\newcommand{\adot}[0]{{\dot{a}}}
\newcommand{\bdot}[0]{{\dot{b}}}
% taken for centered dot:
%\newcommand{\cdot}[0]{{\dot{c}}}
%\newcommand{\ddot}[0]{{\dot{d}}}
\newcommand{\edot}[0]{{\dot{e}}}
\newcommand{\fdot}[0]{{\dot{f}}}
\newcommand{\gdot}[0]{{\dot{g}}}
\newcommand{\hdot}[0]{{\dot{h}}}
\newcommand{\idot}[0]{{\dot{i}}}
\newcommand{\jdot}[0]{{\dot{j}}}
\newcommand{\kdot}[0]{{\dot{k}}}
\newcommand{\ldot}[0]{{\dot{l}}}
\newcommand{\mdot}[0]{{\dot{m}}}
\newcommand{\ndot}[0]{{\dot{n}}}
%\newcommand{\odot}[0]{{\dot{o}}}
\newcommand{\pdot}[0]{{\dot{p}}}
\newcommand{\qdot}[0]{{\dot{q}}}
\newcommand{\rdot}[0]{{\dot{r}}}
\newcommand{\sdot}[0]{{\dot{s}}}
\newcommand{\tdot}[0]{{\dot{t}}}
\newcommand{\udot}[0]{{\dot{u}}}
\newcommand{\vdot}[0]{{\dot{v}}}
\newcommand{\wdot}[0]{{\dot{w}}}
\newcommand{\xdot}[0]{{\dot{x}}}
\newcommand{\ydot}[0]{{\dot{y}}}
\newcommand{\zdot}[0]{{\dot{z}}}
\newcommand{\addot}[0]{{\ddot{a}}}
\newcommand{\bddot}[0]{{\ddot{b}}}
\newcommand{\cddot}[0]{{\ddot{c}}}
%\newcommand{\dddot}[0]{{\ddot{d}}}
\newcommand{\eddot}[0]{{\ddot{e}}}
\newcommand{\fddot}[0]{{\ddot{f}}}
\newcommand{\gddot}[0]{{\ddot{g}}}
\newcommand{\hddot}[0]{{\ddot{h}}}
\newcommand{\iddot}[0]{{\ddot{i}}}
\newcommand{\jddot}[0]{{\ddot{j}}}
\newcommand{\kddot}[0]{{\ddot{k}}}
\newcommand{\lddot}[0]{{\ddot{l}}}
\newcommand{\mddot}[0]{{\ddot{m}}}
\newcommand{\nddot}[0]{{\ddot{n}}}
\newcommand{\oddot}[0]{{\ddot{o}}}
\newcommand{\pddot}[0]{{\ddot{p}}}
\newcommand{\qddot}[0]{{\ddot{q}}}
\newcommand{\rddot}[0]{{\ddot{r}}}
\newcommand{\sddot}[0]{{\ddot{s}}}
\newcommand{\tddot}[0]{{\ddot{t}}}
\newcommand{\uddot}[0]{{\ddot{u}}}
\newcommand{\vddot}[0]{{\ddot{v}}}
\newcommand{\wddot}[0]{{\ddot{w}}}
\newcommand{\xddot}[0]{{\ddot{x}}}
\newcommand{\yddot}[0]{{\ddot{y}}}
\newcommand{\zddot}[0]{{\ddot{z}}}

%<bold and dot greek symbols>
%

\newcommand{\Deltadot}[0]{{\dot{\Delta}}}
\newcommand{\Gammadot}[0]{{\dot{\Gamma}}}
\newcommand{\Lambdadot}[0]{{\dot{\Lambda}}}
\newcommand{\Omegadot}[0]{{\dot{\Omega}}}
\newcommand{\Phidot}[0]{{\dot{\Phi}}}
\newcommand{\Pidot}[0]{{\dot{\Pi}}}
\newcommand{\Psidot}[0]{{\dot{\Psi}}}
\newcommand{\Sigmadot}[0]{{\dot{\Sigma}}}
\newcommand{\Thetadot}[0]{{\dot{\Theta}}}
\newcommand{\Upsilondot}[0]{{\dot{\Upsilon}}}
\newcommand{\Xidot}[0]{{\dot{\Xi}}}
\newcommand{\alphadot}[0]{{\dot{\alpha}}}
\newcommand{\betadot}[0]{{\dot{\beta}}}
\newcommand{\chidot}[0]{{\dot{\chi}}}
\newcommand{\deltadot}[0]{{\dot{\delta}}}
\newcommand{\epsilondot}[0]{{\dot{\epsilon}}}
\newcommand{\etadot}[0]{{\dot{\eta}}}
\newcommand{\gammadot}[0]{{\dot{\gamma}}}
\newcommand{\kappadot}[0]{{\dot{\kappa}}}
\newcommand{\lambdadot}[0]{{\dot{\lambda}}}
\newcommand{\mudot}[0]{{\dot{\mu}}}
\newcommand{\nudot}[0]{{\dot{\nu}}}
\newcommand{\omegadot}[0]{{\dot{\omega}}}
\newcommand{\phidot}[0]{{\dot{\phi}}}
\newcommand{\pidot}[0]{{\dot{\pi}}}
\newcommand{\psidot}[0]{{\dot{\psi}}}
\newcommand{\rhodot}[0]{{\dot{\rho}}}
\newcommand{\sigmadot}[0]{{\dot{\sigma}}}
\newcommand{\taudot}[0]{{\dot{\tau}}}
\newcommand{\thetadot}[0]{{\dot{\theta}}}
\newcommand{\upsilondot}[0]{{\dot{\upsilon}}}
\newcommand{\varepsilondot}[0]{{\dot{\varepsilon}}}
\newcommand{\varphidot}[0]{{\dot{\varphi}}}
\newcommand{\varpidot}[0]{{\dot{\varpi}}}
\newcommand{\varrhodot}[0]{{\dot{\varrho}}}
\newcommand{\varsigmadot}[0]{{\dot{\varsigma}}}
\newcommand{\varthetadot}[0]{{\dot{\vartheta}}}
\newcommand{\xidot}[0]{{\dot{\xi}}}
\newcommand{\zetadot}[0]{{\dot{\zeta}}}

\newcommand{\Deltaddot}[0]{{\ddot{\Delta}}}
\newcommand{\Gammaddot}[0]{{\ddot{\Gamma}}}
\newcommand{\Lambdaddot}[0]{{\ddot{\Lambda}}}
\newcommand{\Omegaddot}[0]{{\ddot{\Omega}}}
\newcommand{\Phiddot}[0]{{\ddot{\Phi}}}
\newcommand{\Piddot}[0]{{\ddot{\Pi}}}
\newcommand{\Psiddot}[0]{{\ddot{\Psi}}}
\newcommand{\Sigmaddot}[0]{{\ddot{\Sigma}}}
\newcommand{\Thetaddot}[0]{{\ddot{\Theta}}}
\newcommand{\Upsilonddot}[0]{{\ddot{\Upsilon}}}
\newcommand{\Xiddot}[0]{{\ddot{\Xi}}}
\newcommand{\alphaddot}[0]{{\ddot{\alpha}}}
\newcommand{\betaddot}[0]{{\ddot{\beta}}}
\newcommand{\chiddot}[0]{{\ddot{\chi}}}
\newcommand{\deltaddot}[0]{{\ddot{\delta}}}
\newcommand{\epsilonddot}[0]{{\ddot{\epsilon}}}
\newcommand{\etaddot}[0]{{\ddot{\eta}}}
\newcommand{\gammaddot}[0]{{\ddot{\gamma}}}
\newcommand{\kappaddot}[0]{{\ddot{\kappa}}}
\newcommand{\lambdaddot}[0]{{\ddot{\lambda}}}
\newcommand{\muddot}[0]{{\ddot{\mu}}}
\newcommand{\nuddot}[0]{{\ddot{\nu}}}
\newcommand{\omegaddot}[0]{{\ddot{\omega}}}
\newcommand{\phiddot}[0]{{\ddot{\phi}}}
\newcommand{\piddot}[0]{{\ddot{\pi}}}
\newcommand{\psiddot}[0]{{\ddot{\psi}}}
\newcommand{\rhoddot}[0]{{\ddot{\rho}}}
\newcommand{\sigmaddot}[0]{{\ddot{\sigma}}}
\newcommand{\tauddot}[0]{{\ddot{\tau}}}
\newcommand{\thetaddot}[0]{{\ddot{\theta}}}
\newcommand{\upsilonddot}[0]{{\ddot{\upsilon}}}
\newcommand{\varepsilonddot}[0]{{\ddot{\varepsilon}}}
\newcommand{\varphiddot}[0]{{\ddot{\varphi}}}
\newcommand{\varpiddot}[0]{{\ddot{\varpi}}}
\newcommand{\varrhoddot}[0]{{\ddot{\varrho}}}
\newcommand{\varsigmaddot}[0]{{\ddot{\varsigma}}}
\newcommand{\varthetaddot}[0]{{\ddot{\vartheta}}}
\newcommand{\xiddot}[0]{{\ddot{\xi}}}
\newcommand{\zetaddot}[0]{{\ddot{\zeta}}}

\newcommand{\BDelta}[0]{\boldsymbol{\Delta}}
\newcommand{\BGamma}[0]{\boldsymbol{\Gamma}}
\newcommand{\BLambda}[0]{\boldsymbol{\Lambda}}
\newcommand{\BOmega}[0]{\boldsymbol{\Omega}}
\newcommand{\BPhi}[0]{\boldsymbol{\Phi}}
\newcommand{\BPi}[0]{\boldsymbol{\Pi}}
\newcommand{\BPsi}[0]{\boldsymbol{\Psi}}
\newcommand{\BSigma}[0]{\boldsymbol{\Sigma}}
\newcommand{\BTheta}[0]{\boldsymbol{\Theta}}
\newcommand{\BUpsilon}[0]{\boldsymbol{\Upsilon}}
\newcommand{\BXi}[0]{\boldsymbol{\Xi}}
\newcommand{\Balpha}[0]{\boldsymbol{\alpha}}
\newcommand{\Bbeta}[0]{\boldsymbol{\beta}}
\newcommand{\Bchi}[0]{\boldsymbol{\chi}}
\newcommand{\Bdelta}[0]{\boldsymbol{\delta}}
\newcommand{\Bepsilon}[0]{\boldsymbol{\epsilon}}
\newcommand{\Beta}[0]{\boldsymbol{\eta}}
\newcommand{\Bgamma}[0]{\boldsymbol{\gamma}}
\newcommand{\Bkappa}[0]{\boldsymbol{\kappa}}
\newcommand{\Blambda}[0]{\boldsymbol{\lambda}}
\newcommand{\Bmu}[0]{\boldsymbol{\mu}}
\newcommand{\Bnu}[0]{\boldsymbol{\nu}}
%\newcommand{\Bomega}[0]{\boldsymbol{\omega}}
\newcommand{\Bphi}[0]{\boldsymbol{\phi}}
\newcommand{\Bpi}[0]{\boldsymbol{\pi}}
\newcommand{\Bpsi}[0]{\boldsymbol{\psi}}
\newcommand{\Brho}[0]{\boldsymbol{\rho}}
\newcommand{\Bsigma}[0]{\boldsymbol{\sigma}}
%\newcommand{\Btau}[0]{\boldsymbol{\tau}}
%\newcommand{\Btheta}[0]{\boldsymbol{\theta}}
\newcommand{\Bupsilon}[0]{\boldsymbol{\upsilon}}
\newcommand{\Bvarepsilon}[0]{\boldsymbol{\varepsilon}}
\newcommand{\Bvarphi}[0]{\boldsymbol{\varphi}}
\newcommand{\Bvarpi}[0]{\boldsymbol{\varpi}}
\newcommand{\Bvarrho}[0]{\boldsymbol{\varrho}}
\newcommand{\Bvarsigma}[0]{\boldsymbol{\varsigma}}
\newcommand{\Bvartheta}[0]{\boldsymbol{\vartheta}}
\newcommand{\Bxi}[0]{\boldsymbol{\xi}}
\newcommand{\Bzeta}[0]{\boldsymbol{\zeta}}
%
%</bold and dot greek symbols>
%<infrequent>
%
%\newcommand{\AreaOp}[1]{\AName_{#1}}
%\newcommand{\Babs}[0]{\abs{\BB}}
%\newcommand{\Bcap}[0]{\hat{\BB}}
%\newcommand{\BrPrimeRej}[0]{\rcap(\rcap \wedge \Br')}
%\newcommand{\CA}[0]{\mathcal{A}}
%\newcommand{\Cos}[1]{\cos{\left({#1}\right)}}
%\newcommand{\Det}[1] {\abs{#1}}
%\newcommand{\Dsq}[2] {\frac {\partial^2 {#1}} {\partial {#2}^2}}
%\newcommand{\Exp}[1]{\exp{\left({#1}\right)}}
%\newcommand{\Norm}[1]{\left\lVert{#1}\right\rVert}
%\newcommand{\Sin}[1]{\sin{\left({#1}\right)}}
%\newcommand{\T}[0]{\text{T}}
%\newcommand{\VolumeOp}[1]{\VName_{#1}}
%\newcommand{\agrad}[0]{\Ba \cdot \nabla}
%\newcommand{\alphacap}[0]{\hat{\boldsymbol{\alpha}}}
%\newcommand{\Fcap}[0]{\hat{\BF}}
%\newcommand{\bithree}[0]{{\Bi}_3}
%\newcommand{\bxa}[0]{\Bx\Ba}
%\newcommand{\coordvec}[2]{
%\newcommand{\costheta}[0]{\acap \cdot \xcap}
%\newcommand{\ddt}[1]{\ddot{#1}}
%\newcommand{\ddu}[1] {\frac {d{#1}} {du}}
%\newcommand{\dsqxj}[2] {\frac {\partial^2 {#1}} {\partial {x_{#2}}^2}}
%\newcommand{\dtheta}[1]{\frac{d {#1}}{d \theta}}
%\newcommand{\dt}[1]{\dot{#1}}
%\newcommand{\dt}[1]{\frac{d {#1}}{dt}}
%\newcommand{\dxj}[2] {\frac {\partial {#1}} {\partial {x_{#2}}}}
%\newcommand{\halfPhi}[0]{\frac{\phi}{2}}
%\newcommand{\half}[0]{\inv{2}}
%\newcommand{\inv}[1]{\frac{1}{#1}}
%\newcommand{\laplacian}[0]{\nabla^2}
%\newcommand{\matrixoftx}[3]{
%\newcommand{\nrrp}[0]{\norm{\rcap \wedge \Br'}}
%\newcommand{\oiint}{\bigcirc \hspace{-1.4em} \int \hspace{-.8em} \int}
%\newcommand{\transpose}[1]{{#1}^{\text{T}}}
%\newcommand{\transpose}[1]{{{#1}^{\TextTranspose}}}
%\newcommand{\transpose}[1]{{{#1}^{\text{T}}}}
%\newcommand{\barA}[0]{\bar{A}}
%\newcommand{\qbar}[0]{\bar{q}}
%\newcommand{\qdotbar}[0]{\dot{\bar{q}}}
%
%</infrequent>





%\usepackage{listings}
\usepackage{txfonts} % for ointctr... (also appears to make "prettier" \int and \sum's)
\usepackage[bookmarks=true]{hyperref}

\usepackage{color,cite,graphicx}
   % use colour in the document, put your citations as [1-4]
   % rather than [1,2,3,4] (it looks nicer, and the extended LaTeX2e
   % graphics package. 
\usepackage{latexsym,amssymb,epsf} % don't remember if these are
   % needed, but their inclusion can't do any damage


\title{ Integer binomial theorem induction, the easy dumb way. }
\author{Peeter Joot \quad peeter.joot@gmail.com }
\date{ March 26, 2009.  Last Revision: $Date: 2009/03/27 00:59:32 $ }

\begin{document}
\maketitle{}

%\tableofcontents

\section{ Motivation. }

While working a problem with an induction requirement similar to but more 
complicated than the binomial theorem, I steped back and thought I'd try
this as an easier first step.  Had some trouble doing it, until I tried it
explicitly with for the power of three case.  Ironically, working it
out for an explicit index takes the abstraction out of the problem, and
generalizing further really only requires a search and replace.

\section{ Do it. }

Want to prove

\begin{align}
(t + x)^k = \sum_{m=0}^k \binom{k}{m} t^m x^{k-m}
\end{align}

where

\begin{align}
\binom{k}{m} = \frac{k!}{(k-m)!m!}
\end{align}

In particular want to prove this for the $k+1$ case, given $k$.

\subsection{ step with k+1 = 3 }

Three isn't actually the best way to start since it is almost too trivial, but
one gets the idea easily by doing it.

\begin{align*}
(t + x)^3 
&= (t + x)(t + x)^2  \\
&= (t + x)\sum_{m=0}^2 \binom{2}{m} t^m x^{2-m} \\
&= 
\sum_{m=0}^2 \binom{2}{m} t^{m+1} x^{2-m} 
+ \sum_{m=0}^2 \binom{2}{m} t^{m} x^{2-m + 1} \\
&= 
\sum_{m=1}^{2 + 1} \binom{2}{m-1} t^{m} x^{2 - m + 1} 
+ \sum_{m=0}^2 \binom{2}{m} t^{m} x^{2-m + 1} \\
\end{align*}

Now pull the lowest and highest order terms out of the sums, and group the
remaining bits.

\begin{align*}
(t + x)^3 
&= 
 \binom{2}{0} t^{0} x^{2 - 0 + 1} 
+ \sum_{m=1}^{2} \left( \binom{2}{m} + \binom{2}{m-1} \right) t^{m} x^{2 - m + 1} 
+ \binom{2}{2 + 1 -1} t^{2 + 1} x^{2 - (2 + 1) + 1} 
\\
\end{align*}

Now, observe that in all the steps above everywhere if all places that $2$, a nice easy to think with and concrete number, we could have used some
abstract index.

\subsection{ step with k+1 }

A straight text search and replace on $2$ with $k$ gives

\begin{align*}
(t + x)^{k+1}
&= (t + x)(t + x)^k  \\
&= (t + x)\sum_{m=0}^k \binom{k}{m} t^m x^{k-m} \\
&= 
\sum_{m=0}^k \binom{k}{m} t^{m+1} x^{k-m} 
+ \sum_{m=0}^k \binom{k}{m} t^{m} x^{k-m + 1} \\
&= 
\sum_{m=1}^{k + 1} \binom{k}{m-1} t^{m} x^{k - m + 1} 
+ \sum_{m=0}^k \binom{k}{m} t^{m} x^{k-m + 1} \\
&= 
 \binom{k}{0} t^{0} x^{k - 0 + 1} 
+ \sum_{m=1}^{k} \left( \binom{k}{m} + \binom{k}{m-1} \right) t^{m} x^{k - m + 1} 
+ \binom{k}{k + 1 -1} t^{k + 1} x^{k - (k + 1) + 1} \\
\end{align*}

This is EXACTLY the same as above with $k=2$ but it sure looks more complicated with an an abstract index.

To finish off we need a couple observations, that 
$\binom{k}{0} = 1 = \binom{k+1}{0}$, and $\binom{k}{k} = 1 = \binom{k+1}{k+1}$.  This leaves us with

\begin{align*}
(t + x)^{k+1}
= \binom{k+1}{0} t^{0} x^{k + 1} 
+ \sum_{m=1}^{k} \left( \binom{k}{m} + \binom{k}{m-1} \right) t^{m} x^{k - m + 1} 
+ \binom{k + 1}{k + 1} t^{k + 1} x^{0} \\
\end{align*}

So if we can show
\begin{align*}
\binom{k}{m} + \binom{k}{m-1} = \binom{k+1}{m}
\end{align*}

then we would have

\begin{align*}
(t + x)^{k+1}
&= \sum_{m=0}^{k+1} \binom{k+1}{m} t^{m} x^{k + 1 - m} 
\end{align*}

which is what was desired.

\subsection{ That last little piece. }

To prove that last little piece, let's do it again the dumb way, and let a regular expression $s/2/k/g$ in vim do the hard work.

\begin{align*}
\binom{2}{m} + \binom{2}{m-1} 
&=
\frac{2!}{(2-m)!m!}
+ \frac{2!}{(2-m + 1)!(m-1)!} \\
&=
\frac{2!}{(2-m)!m(m-1)!}
+ \frac{2!}{(3-m)!(m-1)!} \\
&=
\frac{2!}{(2-m)!m(m-1)!}
+ \frac{2!}{(3-m)(2-m)!(m-1)!} \\
&=
\frac{2!}{(2-m)!(m-1)!} \left( \inv{m} + \inv{3-m}\right) \\
&=
\frac{2!}{(2-m)!(m-1)!} \frac{3 -m + m }{m(3-m)} \\
&=
\frac{3!}{(3-m)!(m)!} \\
\end{align*}

So, generalizing the easy way with $s/3/k+1/g$, we have 

\begin{align*}
\binom{k}{m} + \binom{k}{m-1} 
&=
\frac{k!}{(k-m)!m!}
+ \frac{k!}{(k-m + 1)!(m-1)!} \\
&=
\frac{k!}{(k-m)!m(m-1)!}
+ \frac{k!}{((k+1)-m)!(m-1)!} \\
&=
\frac{k!}{(k-m)!m(m-1)!}
+ \frac{k!}{((k+1)-m)(k-m)!(m-1)!} \\
&=
\frac{k!}{(k-m)!(m-1)!} \left( \inv{m} + \inv{(k+1)-m}\right) \\
&=
\frac{k!}{(k-m)!(m-1)!} \frac{(k+1) -m + m }{m((k+1)-m)} \\
&=
\frac{(k+1)!}{((k+1)-m)!(m)!} \\
\end{align*}

Every step is EXACTLY the same as with $k=2$, the only differences were straight text substitution.  That leaves us with

\begin{align}
\binom{k}{m} + \binom{k}{m-1} 
&=
\binom{k+1}{m}
\end{align}

That's all we needed to complete the proof.

I think this is a superior way to do inductive proofs.  Just do the absolute easiest case and do it with a number that is easy to think with.  Search and replace in an editor does all the bits that would make you look clever if you were to leave off the fact that you were really only doing the easy version!

%\bibliographystyle{plainnat}
%\bibliography{myrefs}

\end{document}

\include{dot_linearity}

\part{calculus}
\include{byron_fuller_calc_var}
%
% Copyright � 2012 Peeter Joot.  All Rights Reserved.
% Licenced as described in the file LICENSE under the root directory of this GIT repository.
%

% 
% 
%\documentclass{article}

%\usepackage{amsmath}
\usepackage{mathpazo}

%
% shorthand for bold symbols, convenient for vectors and matrices
%
\newcommand{\Ba}[0]{\mathbf{a}}
\newcommand{\Bb}[0]{\mathbf{b}}
\newcommand{\Bc}[0]{\mathbf{c}}
\newcommand{\Bd}[0]{\mathbf{d}}
\newcommand{\Be}[0]{\mathbf{e}}
\newcommand{\Bf}[0]{\mathbf{f}}
\newcommand{\Bg}[0]{\mathbf{g}}
\newcommand{\Bh}[0]{\mathbf{h}}
\newcommand{\Bi}[0]{\mathbf{i}}
\newcommand{\Bj}[0]{\mathbf{j}}
\newcommand{\Bk}[0]{\mathbf{k}}
\newcommand{\Bl}[0]{\mathbf{l}}
\newcommand{\Bm}[0]{\mathbf{m}}
\newcommand{\Bn}[0]{\mathbf{n}}
\newcommand{\Bo}[0]{\mathbf{o}}
\newcommand{\Bp}[0]{\mathbf{p}}
\newcommand{\Bq}[0]{\mathbf{q}}
\newcommand{\Br}[0]{\mathbf{r}}
\newcommand{\Bs}[0]{\mathbf{s}}
\newcommand{\Bt}[0]{\mathbf{t}}
\newcommand{\Bu}[0]{\mathbf{u}}
\newcommand{\Bv}[0]{\mathbf{v}}
\newcommand{\Bw}[0]{\mathbf{w}}
\newcommand{\Bx}[0]{\mathbf{x}}
\newcommand{\By}[0]{\mathbf{y}}
\newcommand{\Bz}[0]{\mathbf{z}}
\newcommand{\BA}[0]{\mathbf{A}}
\newcommand{\BB}[0]{\mathbf{B}}
\newcommand{\BC}[0]{\mathbf{C}}
\newcommand{\BD}[0]{\mathbf{D}}
\newcommand{\BE}[0]{\mathbf{E}}
\newcommand{\BF}[0]{\mathbf{F}}
\newcommand{\BG}[0]{\mathbf{G}}
\newcommand{\BH}[0]{\mathbf{H}}
\newcommand{\BI}[0]{\mathbf{I}}
\newcommand{\BJ}[0]{\mathbf{J}}
\newcommand{\BK}[0]{\mathbf{K}}
\newcommand{\BL}[0]{\mathbf{L}}
\newcommand{\BM}[0]{\mathbf{M}}
\newcommand{\BN}[0]{\mathbf{N}}
\newcommand{\BO}[0]{\mathbf{O}}
\newcommand{\BP}[0]{\mathbf{P}}
\newcommand{\BQ}[0]{\mathbf{Q}}
\newcommand{\BR}[0]{\mathbf{R}}
\newcommand{\BS}[0]{\mathbf{S}}
\newcommand{\BT}[0]{\mathbf{T}}
\newcommand{\BU}[0]{\mathbf{U}}
\newcommand{\BV}[0]{\mathbf{V}}
\newcommand{\BW}[0]{\mathbf{W}}
\newcommand{\BX}[0]{\mathbf{X}}
\newcommand{\BY}[0]{\mathbf{Y}}
\newcommand{\BZ}[0]{\mathbf{Z}}

\newcommand{\Bzero}[0]{\mathbf{0}}
\newcommand{\Btheta}[0]{\boldsymbol{\theta}}
\newcommand{\Btau}[0]{\boldsymbol{\tau}}
\newcommand{\Bomega}[0]{\boldsymbol{\omega}}

%
% shorthand for unit vectors
%
\newcommand{\acap}[0]{\hat{\Ba}}
\newcommand{\bcap}[0]{\hat{\Bb}}
\newcommand{\ccap}[0]{\hat{\Bc}}
\newcommand{\dcap}[0]{\hat{\Bd}}
\newcommand{\ecap}[0]{\hat{\Be}}
\newcommand{\fcap}[0]{\hat{\Bf}}
\newcommand{\gcap}[0]{\hat{\Bg}}
\newcommand{\hcap}[0]{\hat{\Bh}}
\newcommand{\icap}[0]{\hat{\Bi}}
\newcommand{\jcap}[0]{\hat{\Bj}}
\newcommand{\kcap}[0]{\hat{\Bk}}
\newcommand{\lcap}[0]{\hat{\Bl}}
\newcommand{\mcap}[0]{\hat{\Bm}}
\newcommand{\ncap}[0]{\hat{\Bn}}
\newcommand{\ocap}[0]{\hat{\Bo}}
\newcommand{\pcap}[0]{\hat{\Bp}}
\newcommand{\qcap}[0]{\hat{\Bq}}
\newcommand{\rcap}[0]{\hat{\Br}}
\newcommand{\scap}[0]{\hat{\Bs}}
\newcommand{\tcap}[0]{\hat{\Bt}}
\newcommand{\ucap}[0]{\hat{\Bu}}
\newcommand{\vcap}[0]{\hat{\Bv}}
\newcommand{\wcap}[0]{\hat{\Bw}}
\newcommand{\xcap}[0]{\hat{\Bx}}
\newcommand{\ycap}[0]{\hat{\By}}
\newcommand{\zcap}[0]{\hat{\Bz}}
\newcommand{\thetacap}[0]{\hat{\Btheta}}

%
% to write R^n and C^n in a distinguishable fashion.  Perhaps change this
% to the double lined characters upon figuring out how to do so.
%
\newcommand{\C}[1]{$\mathbb{C}^{#1}$}
\newcommand{\R}[1]{$\mathbb{R}^{#1}$}

%
% various generally useful helpers
%

% derivative of #1 wrt. #2:
\newcommand{\D}[2] {\frac {d#2} {d#1}}

\newcommand{\inv}[1]{\frac{1}{#1}}
\newcommand{\cross}[0]{\times}

\newcommand{\abs}[1]{\lvert{#1}\rvert}
\newcommand{\norm}[1]{\lVert{#1}\rVert}
\newcommand{\innerprod}[2]{\langle{#1}, {#2}\rangle}
\newcommand{\dotprod}[2]{{#1} \cdot {#2}}
\newcommand{\bdotprod}[2]{\left({#1} \cdot {#2}\right)}
\newcommand{\crossprod}[2]{{#1} \cross {#2}}
\newcommand{\tripleprod}[3]{\dotprod{\left(\crossprod{#1}{#2}\right)}{#3}}

\DeclareMathOperator{\Proj}{Proj}
\DeclareMathOperator{\Span}{span}
\DeclareMathOperator{\Sgn}{sgn}
\DeclareMathOperator{\Area}{Area}
\DeclareMathOperator{\Volume}{Volume}

%
% A few miscellaneous things specific to this document
%
\newcommand{\crossop}[1]{\crossprod{#1}{}}

% R2 vector.
\newcommand{\VectorTwo}[2]{
\begin{bmatrix}
 {#1} \\
 {#2}
\end{bmatrix}
}

\newcommand{\VectorN}[1]{
\begin{bmatrix}
{#1}_1 \\
{#1}_2 \\
\vdots \\
{#1}_N \\
\end{bmatrix}
}

\newcommand{\DETuvij}[4]{
\begin{vmatrix}
 {#1}_{#3} & {#1}_{#4} \\
 {#2}_{#3} & {#2}_{#4}
\end{vmatrix}
}

\newcommand{\DETuvwijk}[6]{
\begin{vmatrix}
 {#1}_{#4} & {#1}_{#5} & {#1}_{#6} \\
 {#2}_{#4} & {#2}_{#5} & {#2}_{#6} \\
 {#3}_{#4} & {#3}_{#5} & {#3}_{#6}
\end{vmatrix}
}

\newcommand{\DETuvwxijkl}[8]{
\begin{vmatrix}
 {#1}_{#5} & {#1}_{#6} & {#1}_{#7} & {#1}_{#8} \\
 {#2}_{#5} & {#2}_{#6} & {#2}_{#7} & {#2}_{#8} \\
 {#3}_{#5} & {#3}_{#6} & {#3}_{#7} & {#3}_{#8} \\
 {#4}_{#5} & {#4}_{#6} & {#4}_{#7} & {#4}_{#8} \\
\end{vmatrix}
}

%\newcommand{\DETuvwxyijklm}[10]{
%\begin{vmatrix}
% {#1}_{#6} & {#1}_{#7} & {#1}_{#8} & {#1}_{#9} & {#1}_{#10} \\
% {#2}_{#6} & {#2}_{#7} & {#2}_{#8} & {#2}_{#9} & {#2}_{#10} \\
% {#3}_{#6} & {#3}_{#7} & {#3}_{#8} & {#3}_{#9} & {#3}_{#10} \\
% {#4}_{#6} & {#4}_{#7} & {#4}_{#8} & {#4}_{#9} & {#4}_{#10} \\
% {#5}_{#6} & {#5}_{#7} & {#5}_{#8} & {#5}_{#9} & {#5}_{#10}
%\end{vmatrix}
%}

% R3 vector.
\newcommand{\VectorThree}[3]{
\begin{bmatrix}
 {#1} \\
 {#2} \\
 {#3}
\end{bmatrix}
}


%%<misc>
%
\newcommand{\Abs}[1]{{\left\lvert{#1}\right\rvert}}
\newcommand{\spacegrad}[0]{\boldsymbol{\nabla}}
\newcommand{\grad}[0]{\nabla}
\newcommand{\LL}[0]{\mathcal{L}}

% == \partial_{#1} {#2}
\newcommand{\PD}[2]{\frac{\partial {#2}}{\partial {#1}}}
% inline variant
\newcommand{\PDi}[2]{{\partial {#2}}/{\partial {#1}}}

\newcommand{\PDD}[3]{\frac{\partial^2 {#3}}{\partial {#1}\partial {#2}}}
%\newcommand{\PDd}[2]{\frac{\partial^2 {#2}}{{\partial{#1}}^2}}
\newcommand{\PDsq}[2]{\frac{\partial^2 {#2}}{(\partial {#1})^2}}

\newcommand{\Partial}[2]{\frac{\partial {#1}}{\partial {#2}}}
\DeclareMathOperator{\RejName}{Rej}
\newcommand{\Rej}[2]{\RejName_{#1}\left( {#2} \right)}
\newcommand{\Rm}[1]{\mathbb{R}^{#1}}
\newcommand{\Cm}[1]{\mathbb{C}^{#1}}
\newcommand{\conj}[0]{{*}}

%</misc>

% <grade selection>
%
\newcommand{\gpgrade}[2] {{\left\langle{{#1}}\right\rangle}_{#2}}

\newcommand{\gpgradezero}[1] {\gpgrade{#1}{}}
%\newcommand{\gpscalargrade}[1] {{\left\langle{{#1}}\right\rangle}}
%\newcommand{\gpgradezero}[1] {\gpgrade{#1}{0}}

%\newcommand{\gpgradeone}[1] {{\left\langle{{#1}}\right\rangle}_{1}}
\newcommand{\gpgradeone}[1] {\gpgrade{#1}{1}}

\newcommand{\gpgradetwo}[1] {\gpgrade{#1}{2}}
\newcommand{\gpgradethree}[1] {\gpgrade{#1}{3}}
\newcommand{\gpgradefour}[1] {\gpgrade{#1}{4}}
%
% </grade selection>



\newcommand{\adot}[0]{{\dot{a}}}
\newcommand{\bdot}[0]{{\dot{b}}}
% taken for centered dot:
%\newcommand{\cdot}[0]{{\dot{c}}}
%\newcommand{\ddot}[0]{{\dot{d}}}
\newcommand{\edot}[0]{{\dot{e}}}
\newcommand{\fdot}[0]{{\dot{f}}}
\newcommand{\gdot}[0]{{\dot{g}}}
\newcommand{\hdot}[0]{{\dot{h}}}
\newcommand{\idot}[0]{{\dot{i}}}
\newcommand{\jdot}[0]{{\dot{j}}}
\newcommand{\kdot}[0]{{\dot{k}}}
\newcommand{\ldot}[0]{{\dot{l}}}
\newcommand{\mdot}[0]{{\dot{m}}}
\newcommand{\ndot}[0]{{\dot{n}}}
%\newcommand{\odot}[0]{{\dot{o}}}
\newcommand{\pdot}[0]{{\dot{p}}}
\newcommand{\qdot}[0]{{\dot{q}}}
\newcommand{\rdot}[0]{{\dot{r}}}
\newcommand{\sdot}[0]{{\dot{s}}}
\newcommand{\tdot}[0]{{\dot{t}}}
\newcommand{\udot}[0]{{\dot{u}}}
\newcommand{\vdot}[0]{{\dot{v}}}
\newcommand{\wdot}[0]{{\dot{w}}}
\newcommand{\xdot}[0]{{\dot{x}}}
\newcommand{\ydot}[0]{{\dot{y}}}
\newcommand{\zdot}[0]{{\dot{z}}}
\newcommand{\addot}[0]{{\ddot{a}}}
\newcommand{\bddot}[0]{{\ddot{b}}}
\newcommand{\cddot}[0]{{\ddot{c}}}
%\newcommand{\dddot}[0]{{\ddot{d}}}
\newcommand{\eddot}[0]{{\ddot{e}}}
\newcommand{\fddot}[0]{{\ddot{f}}}
\newcommand{\gddot}[0]{{\ddot{g}}}
\newcommand{\hddot}[0]{{\ddot{h}}}
\newcommand{\iddot}[0]{{\ddot{i}}}
\newcommand{\jddot}[0]{{\ddot{j}}}
\newcommand{\kddot}[0]{{\ddot{k}}}
\newcommand{\lddot}[0]{{\ddot{l}}}
\newcommand{\mddot}[0]{{\ddot{m}}}
\newcommand{\nddot}[0]{{\ddot{n}}}
\newcommand{\oddot}[0]{{\ddot{o}}}
\newcommand{\pddot}[0]{{\ddot{p}}}
\newcommand{\qddot}[0]{{\ddot{q}}}
\newcommand{\rddot}[0]{{\ddot{r}}}
\newcommand{\sddot}[0]{{\ddot{s}}}
\newcommand{\tddot}[0]{{\ddot{t}}}
\newcommand{\uddot}[0]{{\ddot{u}}}
\newcommand{\vddot}[0]{{\ddot{v}}}
\newcommand{\wddot}[0]{{\ddot{w}}}
\newcommand{\xddot}[0]{{\ddot{x}}}
\newcommand{\yddot}[0]{{\ddot{y}}}
\newcommand{\zddot}[0]{{\ddot{z}}}

%<bold and dot greek symbols>
%

\newcommand{\Deltadot}[0]{{\dot{\Delta}}}
\newcommand{\Gammadot}[0]{{\dot{\Gamma}}}
\newcommand{\Lambdadot}[0]{{\dot{\Lambda}}}
\newcommand{\Omegadot}[0]{{\dot{\Omega}}}
\newcommand{\Phidot}[0]{{\dot{\Phi}}}
\newcommand{\Pidot}[0]{{\dot{\Pi}}}
\newcommand{\Psidot}[0]{{\dot{\Psi}}}
\newcommand{\Sigmadot}[0]{{\dot{\Sigma}}}
\newcommand{\Thetadot}[0]{{\dot{\Theta}}}
\newcommand{\Upsilondot}[0]{{\dot{\Upsilon}}}
\newcommand{\Xidot}[0]{{\dot{\Xi}}}
\newcommand{\alphadot}[0]{{\dot{\alpha}}}
\newcommand{\betadot}[0]{{\dot{\beta}}}
\newcommand{\chidot}[0]{{\dot{\chi}}}
\newcommand{\deltadot}[0]{{\dot{\delta}}}
\newcommand{\epsilondot}[0]{{\dot{\epsilon}}}
\newcommand{\etadot}[0]{{\dot{\eta}}}
\newcommand{\gammadot}[0]{{\dot{\gamma}}}
\newcommand{\kappadot}[0]{{\dot{\kappa}}}
\newcommand{\lambdadot}[0]{{\dot{\lambda}}}
\newcommand{\mudot}[0]{{\dot{\mu}}}
\newcommand{\nudot}[0]{{\dot{\nu}}}
\newcommand{\omegadot}[0]{{\dot{\omega}}}
\newcommand{\phidot}[0]{{\dot{\phi}}}
\newcommand{\pidot}[0]{{\dot{\pi}}}
\newcommand{\psidot}[0]{{\dot{\psi}}}
\newcommand{\rhodot}[0]{{\dot{\rho}}}
\newcommand{\sigmadot}[0]{{\dot{\sigma}}}
\newcommand{\taudot}[0]{{\dot{\tau}}}
\newcommand{\thetadot}[0]{{\dot{\theta}}}
\newcommand{\upsilondot}[0]{{\dot{\upsilon}}}
\newcommand{\varepsilondot}[0]{{\dot{\varepsilon}}}
\newcommand{\varphidot}[0]{{\dot{\varphi}}}
\newcommand{\varpidot}[0]{{\dot{\varpi}}}
\newcommand{\varrhodot}[0]{{\dot{\varrho}}}
\newcommand{\varsigmadot}[0]{{\dot{\varsigma}}}
\newcommand{\varthetadot}[0]{{\dot{\vartheta}}}
\newcommand{\xidot}[0]{{\dot{\xi}}}
\newcommand{\zetadot}[0]{{\dot{\zeta}}}

\newcommand{\Deltaddot}[0]{{\ddot{\Delta}}}
\newcommand{\Gammaddot}[0]{{\ddot{\Gamma}}}
\newcommand{\Lambdaddot}[0]{{\ddot{\Lambda}}}
\newcommand{\Omegaddot}[0]{{\ddot{\Omega}}}
\newcommand{\Phiddot}[0]{{\ddot{\Phi}}}
\newcommand{\Piddot}[0]{{\ddot{\Pi}}}
\newcommand{\Psiddot}[0]{{\ddot{\Psi}}}
\newcommand{\Sigmaddot}[0]{{\ddot{\Sigma}}}
\newcommand{\Thetaddot}[0]{{\ddot{\Theta}}}
\newcommand{\Upsilonddot}[0]{{\ddot{\Upsilon}}}
\newcommand{\Xiddot}[0]{{\ddot{\Xi}}}
\newcommand{\alphaddot}[0]{{\ddot{\alpha}}}
\newcommand{\betaddot}[0]{{\ddot{\beta}}}
\newcommand{\chiddot}[0]{{\ddot{\chi}}}
\newcommand{\deltaddot}[0]{{\ddot{\delta}}}
\newcommand{\epsilonddot}[0]{{\ddot{\epsilon}}}
\newcommand{\etaddot}[0]{{\ddot{\eta}}}
\newcommand{\gammaddot}[0]{{\ddot{\gamma}}}
\newcommand{\kappaddot}[0]{{\ddot{\kappa}}}
\newcommand{\lambdaddot}[0]{{\ddot{\lambda}}}
\newcommand{\muddot}[0]{{\ddot{\mu}}}
\newcommand{\nuddot}[0]{{\ddot{\nu}}}
\newcommand{\omegaddot}[0]{{\ddot{\omega}}}
\newcommand{\phiddot}[0]{{\ddot{\phi}}}
\newcommand{\piddot}[0]{{\ddot{\pi}}}
\newcommand{\psiddot}[0]{{\ddot{\psi}}}
\newcommand{\rhoddot}[0]{{\ddot{\rho}}}
\newcommand{\sigmaddot}[0]{{\ddot{\sigma}}}
\newcommand{\tauddot}[0]{{\ddot{\tau}}}
\newcommand{\thetaddot}[0]{{\ddot{\theta}}}
\newcommand{\upsilonddot}[0]{{\ddot{\upsilon}}}
\newcommand{\varepsilonddot}[0]{{\ddot{\varepsilon}}}
\newcommand{\varphiddot}[0]{{\ddot{\varphi}}}
\newcommand{\varpiddot}[0]{{\ddot{\varpi}}}
\newcommand{\varrhoddot}[0]{{\ddot{\varrho}}}
\newcommand{\varsigmaddot}[0]{{\ddot{\varsigma}}}
\newcommand{\varthetaddot}[0]{{\ddot{\vartheta}}}
\newcommand{\xiddot}[0]{{\ddot{\xi}}}
\newcommand{\zetaddot}[0]{{\ddot{\zeta}}}

\newcommand{\BDelta}[0]{\boldsymbol{\Delta}}
\newcommand{\BGamma}[0]{\boldsymbol{\Gamma}}
\newcommand{\BLambda}[0]{\boldsymbol{\Lambda}}
\newcommand{\BOmega}[0]{\boldsymbol{\Omega}}
\newcommand{\BPhi}[0]{\boldsymbol{\Phi}}
\newcommand{\BPi}[0]{\boldsymbol{\Pi}}
\newcommand{\BPsi}[0]{\boldsymbol{\Psi}}
\newcommand{\BSigma}[0]{\boldsymbol{\Sigma}}
\newcommand{\BTheta}[0]{\boldsymbol{\Theta}}
\newcommand{\BUpsilon}[0]{\boldsymbol{\Upsilon}}
\newcommand{\BXi}[0]{\boldsymbol{\Xi}}
\newcommand{\Balpha}[0]{\boldsymbol{\alpha}}
\newcommand{\Bbeta}[0]{\boldsymbol{\beta}}
\newcommand{\Bchi}[0]{\boldsymbol{\chi}}
\newcommand{\Bdelta}[0]{\boldsymbol{\delta}}
\newcommand{\Bepsilon}[0]{\boldsymbol{\epsilon}}
\newcommand{\Beta}[0]{\boldsymbol{\eta}}
\newcommand{\Bgamma}[0]{\boldsymbol{\gamma}}
\newcommand{\Bkappa}[0]{\boldsymbol{\kappa}}
\newcommand{\Blambda}[0]{\boldsymbol{\lambda}}
\newcommand{\Bmu}[0]{\boldsymbol{\mu}}
\newcommand{\Bnu}[0]{\boldsymbol{\nu}}
%\newcommand{\Bomega}[0]{\boldsymbol{\omega}}
\newcommand{\Bphi}[0]{\boldsymbol{\phi}}
\newcommand{\Bpi}[0]{\boldsymbol{\pi}}
\newcommand{\Bpsi}[0]{\boldsymbol{\psi}}
\newcommand{\Brho}[0]{\boldsymbol{\rho}}
\newcommand{\Bsigma}[0]{\boldsymbol{\sigma}}
%\newcommand{\Btau}[0]{\boldsymbol{\tau}}
%\newcommand{\Btheta}[0]{\boldsymbol{\theta}}
\newcommand{\Bupsilon}[0]{\boldsymbol{\upsilon}}
\newcommand{\Bvarepsilon}[0]{\boldsymbol{\varepsilon}}
\newcommand{\Bvarphi}[0]{\boldsymbol{\varphi}}
\newcommand{\Bvarpi}[0]{\boldsymbol{\varpi}}
\newcommand{\Bvarrho}[0]{\boldsymbol{\varrho}}
\newcommand{\Bvarsigma}[0]{\boldsymbol{\varsigma}}
\newcommand{\Bvartheta}[0]{\boldsymbol{\vartheta}}
\newcommand{\Bxi}[0]{\boldsymbol{\xi}}
\newcommand{\Bzeta}[0]{\boldsymbol{\zeta}}
%
%</bold and dot greek symbols>
%<infrequent>
%
%\newcommand{\AreaOp}[1]{\AName_{#1}}
%\newcommand{\Babs}[0]{\abs{\BB}}
%\newcommand{\Bcap}[0]{\hat{\BB}}
%\newcommand{\BrPrimeRej}[0]{\rcap(\rcap \wedge \Br')}
%\newcommand{\CA}[0]{\mathcal{A}}
%\newcommand{\Cos}[1]{\cos{\left({#1}\right)}}
%\newcommand{\Det}[1] {\abs{#1}}
%\newcommand{\Dsq}[2] {\frac {\partial^2 {#1}} {\partial {#2}^2}}
%\newcommand{\Exp}[1]{\exp{\left({#1}\right)}}
%\newcommand{\Norm}[1]{\left\lVert{#1}\right\rVert}
%\newcommand{\Sin}[1]{\sin{\left({#1}\right)}}
%\newcommand{\T}[0]{\text{T}}
%\newcommand{\VolumeOp}[1]{\VName_{#1}}
%\newcommand{\agrad}[0]{\Ba \cdot \nabla}
%\newcommand{\alphacap}[0]{\hat{\boldsymbol{\alpha}}}
%\newcommand{\Fcap}[0]{\hat{\BF}}
%\newcommand{\bithree}[0]{{\Bi}_3}
%\newcommand{\bxa}[0]{\Bx\Ba}
%\newcommand{\coordvec}[2]{
%\newcommand{\costheta}[0]{\acap \cdot \xcap}
%\newcommand{\ddt}[1]{\ddot{#1}}
%\newcommand{\ddu}[1] {\frac {d{#1}} {du}}
%\newcommand{\dsqxj}[2] {\frac {\partial^2 {#1}} {\partial {x_{#2}}^2}}
%\newcommand{\dtheta}[1]{\frac{d {#1}}{d \theta}}
%\newcommand{\dt}[1]{\dot{#1}}
%\newcommand{\dt}[1]{\frac{d {#1}}{dt}}
%\newcommand{\dxj}[2] {\frac {\partial {#1}} {\partial {x_{#2}}}}
%\newcommand{\halfPhi}[0]{\frac{\phi}{2}}
%\newcommand{\half}[0]{\inv{2}}
%\newcommand{\inv}[1]{\frac{1}{#1}}
%\newcommand{\laplacian}[0]{\nabla^2}
%\newcommand{\matrixoftx}[3]{
%\newcommand{\nrrp}[0]{\norm{\rcap \wedge \Br'}}
%\newcommand{\oiint}{\bigcirc \hspace{-1.4em} \int \hspace{-.8em} \int}
%\newcommand{\transpose}[1]{{#1}^{\text{T}}}
%\newcommand{\transpose}[1]{{{#1}^{\TextTranspose}}}
%\newcommand{\transpose}[1]{{{#1}^{\text{T}}}}
%\newcommand{\barA}[0]{\bar{A}}
%\newcommand{\qbar}[0]{\bar{q}}
%\newcommand{\qdotbar}[0]{\dot{\bar{q}}}
%
%</infrequent>




%\usepackage{listings}
%\usepackage{txfonts} % for ointctr... (also appears to make "prettier" \int and \sum's)
%\usepackage[bookmarks=true]{hyperref}

%\usepackage{color,cite,graphicx}
   % use colour in the document, put your citations as [1-4]
   % rather than [1,2,3,4] (it looks nicer, and the extended LaTeX2e
   % graphics package. 
%\usepackage{latexsym,amssymb,epsf} % do not remember if these are
   % needed, but their inclusion can not do any damage


\chapter{Simple minded variation of one dimensional wave equation Lagrangian}
\label{chap:wavevariation}
%\author{Peeter Joot \quad peeterjoot@protonmail.com }
\date{ April 27, 2009.  wavevariation.tex }

%\begin{document}

%\maketitle{}
%\tableofcontents
%\section{}

From the action

\begin{equation}\label{eqn:wavevariation:20}
\begin{aligned}
S_\eta = \int dx dt \inv{2} \left( \left(\PD{x}{\eta}\right)^2 - \inv{v^2} \left(\PD{t}{\eta}\right)^2 \right)
\end{aligned}
\end{equation}

We can vary the field \(\eta = \psi + \epsilon\), where \(\psi\) is the field variable to be determined, and \(\epsilon\) is the field variable allowed to vary within the volume of integration.

Forming the difference, subtracts off the ``constant'' parts of the action due only to the optimal field variable \(\psi\)

\begin{equation}\label{eqn:wavevariation:40}
\begin{aligned}
S_{\psi + \epsilon} - S_\psi
&=
\int dx dt \inv{2} \left( \left(\PD{x}{(\psi + \epsilon)}\right)^2 - \inv{v^2} \left(\PD{t}{(\psi + \epsilon)}\right)^2 \right)
-\int dx dt \inv{2} \left( \left(\PD{x}{\psi}\right)^2 - \inv{v^2} \left(\PD{t}{\psi}\right)^2 \right) \\
&=
\int dx dt \left( \PD{x}{\psi}\PD{x}{\epsilon} - \PD{t}{\psi}\PD{t}{\epsilon} \right)
+ \int dx dt \inv{2} \left( \left(\PD{x}{\epsilon }\right)^2 - \inv{v^2} \left(\PD{t}{\epsilon }\right)^2 \right) \\
\end{aligned}
\end{equation}

Now integrating by parts

\begin{equation}\label{eqn:wavevariation:60}
\begin{aligned}
S_{\psi + \epsilon} - S_\psi
&=
\int dx dt \left( \inv{v^2} \PDSq{t}{\psi} -\PDSq{x}{\psi} \right) {\epsilon} 
+ \int dx dt \inv{2} \left( -\PDSq{x}{\epsilon } + \inv{v^2} \PDSq{t}{\epsilon } \right) \epsilon \\
\end{aligned}
\end{equation}

Roughly speaking the terms that are quadratic in \(\epsilon\) can be discarded as small, and if the remaining differential
is to be zero for all \(\epsilon\), we are left with the wave equation

\begin{equation}\label{eqn:wavevariation:80}
\begin{aligned}
\inv{v^2} \PDSq{t}{\psi} -\PDSq{x}{\psi} = 0 
\end{aligned}
\end{equation}

with solutions 
%(like your exponential) 
of the form 

\begin{equation}\label{eqn:wavevariation:100}
\begin{aligned}
\psi = f(x \pm vt)
\end{aligned}
\end{equation}

%\bibliographystyle{plainnat}
%\bibliography{myrefs}

%\end{document}


\part{debroglie}
% 
% 
% 
% Copyright � 2012 Peeter Joot
% All Rights Reserved
% 
% This file may be reproduced and distributed in whole or in part, without fee, subject to the following conditions:
% 
% o The copyright notice above and this permission notice must be preserved complete on all complete or partial copies.
% 
% o Any translation or derived work must be approved by the author in writing before distribution.
% 
% o If you distribute this work in part, instructions for obtaining the complete version of this file must be included, and a means for obtaining a complete version provided.
% 
% 
% Exceptions to these rules may be granted for academic purposes: Write to the author and ask.
% 
% 
% 
%\documentclass{article}

%\usepackage{amsmath}
\usepackage{mathpazo}

%
% shorthand for bold symbols, convenient for vectors and matrices
%
\newcommand{\Ba}[0]{\mathbf{a}}
\newcommand{\Bb}[0]{\mathbf{b}}
\newcommand{\Bc}[0]{\mathbf{c}}
\newcommand{\Bd}[0]{\mathbf{d}}
\newcommand{\Be}[0]{\mathbf{e}}
\newcommand{\Bf}[0]{\mathbf{f}}
\newcommand{\Bg}[0]{\mathbf{g}}
\newcommand{\Bh}[0]{\mathbf{h}}
\newcommand{\Bi}[0]{\mathbf{i}}
\newcommand{\Bj}[0]{\mathbf{j}}
\newcommand{\Bk}[0]{\mathbf{k}}
\newcommand{\Bl}[0]{\mathbf{l}}
\newcommand{\Bm}[0]{\mathbf{m}}
\newcommand{\Bn}[0]{\mathbf{n}}
\newcommand{\Bo}[0]{\mathbf{o}}
\newcommand{\Bp}[0]{\mathbf{p}}
\newcommand{\Bq}[0]{\mathbf{q}}
\newcommand{\Br}[0]{\mathbf{r}}
\newcommand{\Bs}[0]{\mathbf{s}}
\newcommand{\Bt}[0]{\mathbf{t}}
\newcommand{\Bu}[0]{\mathbf{u}}
\newcommand{\Bv}[0]{\mathbf{v}}
\newcommand{\Bw}[0]{\mathbf{w}}
\newcommand{\Bx}[0]{\mathbf{x}}
\newcommand{\By}[0]{\mathbf{y}}
\newcommand{\Bz}[0]{\mathbf{z}}
\newcommand{\BA}[0]{\mathbf{A}}
\newcommand{\BB}[0]{\mathbf{B}}
\newcommand{\BC}[0]{\mathbf{C}}
\newcommand{\BD}[0]{\mathbf{D}}
\newcommand{\BE}[0]{\mathbf{E}}
\newcommand{\BF}[0]{\mathbf{F}}
\newcommand{\BG}[0]{\mathbf{G}}
\newcommand{\BH}[0]{\mathbf{H}}
\newcommand{\BI}[0]{\mathbf{I}}
\newcommand{\BJ}[0]{\mathbf{J}}
\newcommand{\BK}[0]{\mathbf{K}}
\newcommand{\BL}[0]{\mathbf{L}}
\newcommand{\BM}[0]{\mathbf{M}}
\newcommand{\BN}[0]{\mathbf{N}}
\newcommand{\BO}[0]{\mathbf{O}}
\newcommand{\BP}[0]{\mathbf{P}}
\newcommand{\BQ}[0]{\mathbf{Q}}
\newcommand{\BR}[0]{\mathbf{R}}
\newcommand{\BS}[0]{\mathbf{S}}
\newcommand{\BT}[0]{\mathbf{T}}
\newcommand{\BU}[0]{\mathbf{U}}
\newcommand{\BV}[0]{\mathbf{V}}
\newcommand{\BW}[0]{\mathbf{W}}
\newcommand{\BX}[0]{\mathbf{X}}
\newcommand{\BY}[0]{\mathbf{Y}}
\newcommand{\BZ}[0]{\mathbf{Z}}

\newcommand{\Bzero}[0]{\mathbf{0}}
\newcommand{\Btheta}[0]{\boldsymbol{\theta}}
\newcommand{\Btau}[0]{\boldsymbol{\tau}}
\newcommand{\Bomega}[0]{\boldsymbol{\omega}}

%
% shorthand for unit vectors
%
\newcommand{\acap}[0]{\hat{\Ba}}
\newcommand{\bcap}[0]{\hat{\Bb}}
\newcommand{\ccap}[0]{\hat{\Bc}}
\newcommand{\dcap}[0]{\hat{\Bd}}
\newcommand{\ecap}[0]{\hat{\Be}}
\newcommand{\fcap}[0]{\hat{\Bf}}
\newcommand{\gcap}[0]{\hat{\Bg}}
\newcommand{\hcap}[0]{\hat{\Bh}}
\newcommand{\icap}[0]{\hat{\Bi}}
\newcommand{\jcap}[0]{\hat{\Bj}}
\newcommand{\kcap}[0]{\hat{\Bk}}
\newcommand{\lcap}[0]{\hat{\Bl}}
\newcommand{\mcap}[0]{\hat{\Bm}}
\newcommand{\ncap}[0]{\hat{\Bn}}
\newcommand{\ocap}[0]{\hat{\Bo}}
\newcommand{\pcap}[0]{\hat{\Bp}}
\newcommand{\qcap}[0]{\hat{\Bq}}
\newcommand{\rcap}[0]{\hat{\Br}}
\newcommand{\scap}[0]{\hat{\Bs}}
\newcommand{\tcap}[0]{\hat{\Bt}}
\newcommand{\ucap}[0]{\hat{\Bu}}
\newcommand{\vcap}[0]{\hat{\Bv}}
\newcommand{\wcap}[0]{\hat{\Bw}}
\newcommand{\xcap}[0]{\hat{\Bx}}
\newcommand{\ycap}[0]{\hat{\By}}
\newcommand{\zcap}[0]{\hat{\Bz}}
\newcommand{\thetacap}[0]{\hat{\Btheta}}

%
% to write R^n and C^n in a distinguishable fashion.  Perhaps change this
% to the double lined characters upon figuring out how to do so.
%
\newcommand{\C}[1]{$\mathbb{C}^{#1}$}
\newcommand{\R}[1]{$\mathbb{R}^{#1}$}

%
% various generally useful helpers
%

% derivative of #1 wrt. #2:
\newcommand{\D}[2] {\frac {d#2} {d#1}}

\newcommand{\inv}[1]{\frac{1}{#1}}
\newcommand{\cross}[0]{\times}

\newcommand{\abs}[1]{\lvert{#1}\rvert}
\newcommand{\norm}[1]{\lVert{#1}\rVert}
\newcommand{\innerprod}[2]{\langle{#1}, {#2}\rangle}
\newcommand{\dotprod}[2]{{#1} \cdot {#2}}
\newcommand{\bdotprod}[2]{\left({#1} \cdot {#2}\right)}
\newcommand{\crossprod}[2]{{#1} \cross {#2}}
\newcommand{\tripleprod}[3]{\dotprod{\left(\crossprod{#1}{#2}\right)}{#3}}

\DeclareMathOperator{\Proj}{Proj}
\DeclareMathOperator{\Span}{span}
\DeclareMathOperator{\Sgn}{sgn}
\DeclareMathOperator{\Area}{Area}
\DeclareMathOperator{\Volume}{Volume}

%
% A few miscellaneous things specific to this document
%
\newcommand{\crossop}[1]{\crossprod{#1}{}}

% R2 vector.
\newcommand{\VectorTwo}[2]{
\begin{bmatrix}
 {#1} \\
 {#2}
\end{bmatrix}
}

\newcommand{\VectorN}[1]{
\begin{bmatrix}
{#1}_1 \\
{#1}_2 \\
\vdots \\
{#1}_N \\
\end{bmatrix}
}

\newcommand{\DETuvij}[4]{
\begin{vmatrix}
 {#1}_{#3} & {#1}_{#4} \\
 {#2}_{#3} & {#2}_{#4}
\end{vmatrix}
}

\newcommand{\DETuvwijk}[6]{
\begin{vmatrix}
 {#1}_{#4} & {#1}_{#5} & {#1}_{#6} \\
 {#2}_{#4} & {#2}_{#5} & {#2}_{#6} \\
 {#3}_{#4} & {#3}_{#5} & {#3}_{#6}
\end{vmatrix}
}

\newcommand{\DETuvwxijkl}[8]{
\begin{vmatrix}
 {#1}_{#5} & {#1}_{#6} & {#1}_{#7} & {#1}_{#8} \\
 {#2}_{#5} & {#2}_{#6} & {#2}_{#7} & {#2}_{#8} \\
 {#3}_{#5} & {#3}_{#6} & {#3}_{#7} & {#3}_{#8} \\
 {#4}_{#5} & {#4}_{#6} & {#4}_{#7} & {#4}_{#8} \\
\end{vmatrix}
}

%\newcommand{\DETuvwxyijklm}[10]{
%\begin{vmatrix}
% {#1}_{#6} & {#1}_{#7} & {#1}_{#8} & {#1}_{#9} & {#1}_{#10} \\
% {#2}_{#6} & {#2}_{#7} & {#2}_{#8} & {#2}_{#9} & {#2}_{#10} \\
% {#3}_{#6} & {#3}_{#7} & {#3}_{#8} & {#3}_{#9} & {#3}_{#10} \\
% {#4}_{#6} & {#4}_{#7} & {#4}_{#8} & {#4}_{#9} & {#4}_{#10} \\
% {#5}_{#6} & {#5}_{#7} & {#5}_{#8} & {#5}_{#9} & {#5}_{#10}
%\end{vmatrix}
%}

% R3 vector.
\newcommand{\VectorThree}[3]{
\begin{bmatrix}
 {#1} \\
 {#2} \\
 {#3}
\end{bmatrix}
}


%%<misc>
%
\newcommand{\Abs}[1]{{\left\lvert{#1}\right\rvert}}
\newcommand{\spacegrad}[0]{\boldsymbol{\nabla}}
\newcommand{\grad}[0]{\nabla}
\newcommand{\LL}[0]{\mathcal{L}}

% == \partial_{#1} {#2}
\newcommand{\PD}[2]{\frac{\partial {#2}}{\partial {#1}}}
% inline variant
\newcommand{\PDi}[2]{{\partial {#2}}/{\partial {#1}}}

\newcommand{\PDD}[3]{\frac{\partial^2 {#3}}{\partial {#1}\partial {#2}}}
%\newcommand{\PDd}[2]{\frac{\partial^2 {#2}}{{\partial{#1}}^2}}
\newcommand{\PDsq}[2]{\frac{\partial^2 {#2}}{(\partial {#1})^2}}

\newcommand{\Partial}[2]{\frac{\partial {#1}}{\partial {#2}}}
\DeclareMathOperator{\RejName}{Rej}
\newcommand{\Rej}[2]{\RejName_{#1}\left( {#2} \right)}
\newcommand{\Rm}[1]{\mathbb{R}^{#1}}
\newcommand{\Cm}[1]{\mathbb{C}^{#1}}
\newcommand{\conj}[0]{{*}}

%</misc>

% <grade selection>
%
\newcommand{\gpgrade}[2] {{\left\langle{{#1}}\right\rangle}_{#2}}

\newcommand{\gpgradezero}[1] {\gpgrade{#1}{}}
%\newcommand{\gpscalargrade}[1] {{\left\langle{{#1}}\right\rangle}}
%\newcommand{\gpgradezero}[1] {\gpgrade{#1}{0}}

%\newcommand{\gpgradeone}[1] {{\left\langle{{#1}}\right\rangle}_{1}}
\newcommand{\gpgradeone}[1] {\gpgrade{#1}{1}}

\newcommand{\gpgradetwo}[1] {\gpgrade{#1}{2}}
\newcommand{\gpgradethree}[1] {\gpgrade{#1}{3}}
\newcommand{\gpgradefour}[1] {\gpgrade{#1}{4}}
%
% </grade selection>



\newcommand{\adot}[0]{{\dot{a}}}
\newcommand{\bdot}[0]{{\dot{b}}}
% taken for centered dot:
%\newcommand{\cdot}[0]{{\dot{c}}}
%\newcommand{\ddot}[0]{{\dot{d}}}
\newcommand{\edot}[0]{{\dot{e}}}
\newcommand{\fdot}[0]{{\dot{f}}}
\newcommand{\gdot}[0]{{\dot{g}}}
\newcommand{\hdot}[0]{{\dot{h}}}
\newcommand{\idot}[0]{{\dot{i}}}
\newcommand{\jdot}[0]{{\dot{j}}}
\newcommand{\kdot}[0]{{\dot{k}}}
\newcommand{\ldot}[0]{{\dot{l}}}
\newcommand{\mdot}[0]{{\dot{m}}}
\newcommand{\ndot}[0]{{\dot{n}}}
%\newcommand{\odot}[0]{{\dot{o}}}
\newcommand{\pdot}[0]{{\dot{p}}}
\newcommand{\qdot}[0]{{\dot{q}}}
\newcommand{\rdot}[0]{{\dot{r}}}
\newcommand{\sdot}[0]{{\dot{s}}}
\newcommand{\tdot}[0]{{\dot{t}}}
\newcommand{\udot}[0]{{\dot{u}}}
\newcommand{\vdot}[0]{{\dot{v}}}
\newcommand{\wdot}[0]{{\dot{w}}}
\newcommand{\xdot}[0]{{\dot{x}}}
\newcommand{\ydot}[0]{{\dot{y}}}
\newcommand{\zdot}[0]{{\dot{z}}}
\newcommand{\addot}[0]{{\ddot{a}}}
\newcommand{\bddot}[0]{{\ddot{b}}}
\newcommand{\cddot}[0]{{\ddot{c}}}
%\newcommand{\dddot}[0]{{\ddot{d}}}
\newcommand{\eddot}[0]{{\ddot{e}}}
\newcommand{\fddot}[0]{{\ddot{f}}}
\newcommand{\gddot}[0]{{\ddot{g}}}
\newcommand{\hddot}[0]{{\ddot{h}}}
\newcommand{\iddot}[0]{{\ddot{i}}}
\newcommand{\jddot}[0]{{\ddot{j}}}
\newcommand{\kddot}[0]{{\ddot{k}}}
\newcommand{\lddot}[0]{{\ddot{l}}}
\newcommand{\mddot}[0]{{\ddot{m}}}
\newcommand{\nddot}[0]{{\ddot{n}}}
\newcommand{\oddot}[0]{{\ddot{o}}}
\newcommand{\pddot}[0]{{\ddot{p}}}
\newcommand{\qddot}[0]{{\ddot{q}}}
\newcommand{\rddot}[0]{{\ddot{r}}}
\newcommand{\sddot}[0]{{\ddot{s}}}
\newcommand{\tddot}[0]{{\ddot{t}}}
\newcommand{\uddot}[0]{{\ddot{u}}}
\newcommand{\vddot}[0]{{\ddot{v}}}
\newcommand{\wddot}[0]{{\ddot{w}}}
\newcommand{\xddot}[0]{{\ddot{x}}}
\newcommand{\yddot}[0]{{\ddot{y}}}
\newcommand{\zddot}[0]{{\ddot{z}}}

%<bold and dot greek symbols>
%

\newcommand{\Deltadot}[0]{{\dot{\Delta}}}
\newcommand{\Gammadot}[0]{{\dot{\Gamma}}}
\newcommand{\Lambdadot}[0]{{\dot{\Lambda}}}
\newcommand{\Omegadot}[0]{{\dot{\Omega}}}
\newcommand{\Phidot}[0]{{\dot{\Phi}}}
\newcommand{\Pidot}[0]{{\dot{\Pi}}}
\newcommand{\Psidot}[0]{{\dot{\Psi}}}
\newcommand{\Sigmadot}[0]{{\dot{\Sigma}}}
\newcommand{\Thetadot}[0]{{\dot{\Theta}}}
\newcommand{\Upsilondot}[0]{{\dot{\Upsilon}}}
\newcommand{\Xidot}[0]{{\dot{\Xi}}}
\newcommand{\alphadot}[0]{{\dot{\alpha}}}
\newcommand{\betadot}[0]{{\dot{\beta}}}
\newcommand{\chidot}[0]{{\dot{\chi}}}
\newcommand{\deltadot}[0]{{\dot{\delta}}}
\newcommand{\epsilondot}[0]{{\dot{\epsilon}}}
\newcommand{\etadot}[0]{{\dot{\eta}}}
\newcommand{\gammadot}[0]{{\dot{\gamma}}}
\newcommand{\kappadot}[0]{{\dot{\kappa}}}
\newcommand{\lambdadot}[0]{{\dot{\lambda}}}
\newcommand{\mudot}[0]{{\dot{\mu}}}
\newcommand{\nudot}[0]{{\dot{\nu}}}
\newcommand{\omegadot}[0]{{\dot{\omega}}}
\newcommand{\phidot}[0]{{\dot{\phi}}}
\newcommand{\pidot}[0]{{\dot{\pi}}}
\newcommand{\psidot}[0]{{\dot{\psi}}}
\newcommand{\rhodot}[0]{{\dot{\rho}}}
\newcommand{\sigmadot}[0]{{\dot{\sigma}}}
\newcommand{\taudot}[0]{{\dot{\tau}}}
\newcommand{\thetadot}[0]{{\dot{\theta}}}
\newcommand{\upsilondot}[0]{{\dot{\upsilon}}}
\newcommand{\varepsilondot}[0]{{\dot{\varepsilon}}}
\newcommand{\varphidot}[0]{{\dot{\varphi}}}
\newcommand{\varpidot}[0]{{\dot{\varpi}}}
\newcommand{\varrhodot}[0]{{\dot{\varrho}}}
\newcommand{\varsigmadot}[0]{{\dot{\varsigma}}}
\newcommand{\varthetadot}[0]{{\dot{\vartheta}}}
\newcommand{\xidot}[0]{{\dot{\xi}}}
\newcommand{\zetadot}[0]{{\dot{\zeta}}}

\newcommand{\Deltaddot}[0]{{\ddot{\Delta}}}
\newcommand{\Gammaddot}[0]{{\ddot{\Gamma}}}
\newcommand{\Lambdaddot}[0]{{\ddot{\Lambda}}}
\newcommand{\Omegaddot}[0]{{\ddot{\Omega}}}
\newcommand{\Phiddot}[0]{{\ddot{\Phi}}}
\newcommand{\Piddot}[0]{{\ddot{\Pi}}}
\newcommand{\Psiddot}[0]{{\ddot{\Psi}}}
\newcommand{\Sigmaddot}[0]{{\ddot{\Sigma}}}
\newcommand{\Thetaddot}[0]{{\ddot{\Theta}}}
\newcommand{\Upsilonddot}[0]{{\ddot{\Upsilon}}}
\newcommand{\Xiddot}[0]{{\ddot{\Xi}}}
\newcommand{\alphaddot}[0]{{\ddot{\alpha}}}
\newcommand{\betaddot}[0]{{\ddot{\beta}}}
\newcommand{\chiddot}[0]{{\ddot{\chi}}}
\newcommand{\deltaddot}[0]{{\ddot{\delta}}}
\newcommand{\epsilonddot}[0]{{\ddot{\epsilon}}}
\newcommand{\etaddot}[0]{{\ddot{\eta}}}
\newcommand{\gammaddot}[0]{{\ddot{\gamma}}}
\newcommand{\kappaddot}[0]{{\ddot{\kappa}}}
\newcommand{\lambdaddot}[0]{{\ddot{\lambda}}}
\newcommand{\muddot}[0]{{\ddot{\mu}}}
\newcommand{\nuddot}[0]{{\ddot{\nu}}}
\newcommand{\omegaddot}[0]{{\ddot{\omega}}}
\newcommand{\phiddot}[0]{{\ddot{\phi}}}
\newcommand{\piddot}[0]{{\ddot{\pi}}}
\newcommand{\psiddot}[0]{{\ddot{\psi}}}
\newcommand{\rhoddot}[0]{{\ddot{\rho}}}
\newcommand{\sigmaddot}[0]{{\ddot{\sigma}}}
\newcommand{\tauddot}[0]{{\ddot{\tau}}}
\newcommand{\thetaddot}[0]{{\ddot{\theta}}}
\newcommand{\upsilonddot}[0]{{\ddot{\upsilon}}}
\newcommand{\varepsilonddot}[0]{{\ddot{\varepsilon}}}
\newcommand{\varphiddot}[0]{{\ddot{\varphi}}}
\newcommand{\varpiddot}[0]{{\ddot{\varpi}}}
\newcommand{\varrhoddot}[0]{{\ddot{\varrho}}}
\newcommand{\varsigmaddot}[0]{{\ddot{\varsigma}}}
\newcommand{\varthetaddot}[0]{{\ddot{\vartheta}}}
\newcommand{\xiddot}[0]{{\ddot{\xi}}}
\newcommand{\zetaddot}[0]{{\ddot{\zeta}}}

\newcommand{\BDelta}[0]{\boldsymbol{\Delta}}
\newcommand{\BGamma}[0]{\boldsymbol{\Gamma}}
\newcommand{\BLambda}[0]{\boldsymbol{\Lambda}}
\newcommand{\BOmega}[0]{\boldsymbol{\Omega}}
\newcommand{\BPhi}[0]{\boldsymbol{\Phi}}
\newcommand{\BPi}[0]{\boldsymbol{\Pi}}
\newcommand{\BPsi}[0]{\boldsymbol{\Psi}}
\newcommand{\BSigma}[0]{\boldsymbol{\Sigma}}
\newcommand{\BTheta}[0]{\boldsymbol{\Theta}}
\newcommand{\BUpsilon}[0]{\boldsymbol{\Upsilon}}
\newcommand{\BXi}[0]{\boldsymbol{\Xi}}
\newcommand{\Balpha}[0]{\boldsymbol{\alpha}}
\newcommand{\Bbeta}[0]{\boldsymbol{\beta}}
\newcommand{\Bchi}[0]{\boldsymbol{\chi}}
\newcommand{\Bdelta}[0]{\boldsymbol{\delta}}
\newcommand{\Bepsilon}[0]{\boldsymbol{\epsilon}}
\newcommand{\Beta}[0]{\boldsymbol{\eta}}
\newcommand{\Bgamma}[0]{\boldsymbol{\gamma}}
\newcommand{\Bkappa}[0]{\boldsymbol{\kappa}}
\newcommand{\Blambda}[0]{\boldsymbol{\lambda}}
\newcommand{\Bmu}[0]{\boldsymbol{\mu}}
\newcommand{\Bnu}[0]{\boldsymbol{\nu}}
%\newcommand{\Bomega}[0]{\boldsymbol{\omega}}
\newcommand{\Bphi}[0]{\boldsymbol{\phi}}
\newcommand{\Bpi}[0]{\boldsymbol{\pi}}
\newcommand{\Bpsi}[0]{\boldsymbol{\psi}}
\newcommand{\Brho}[0]{\boldsymbol{\rho}}
\newcommand{\Bsigma}[0]{\boldsymbol{\sigma}}
%\newcommand{\Btau}[0]{\boldsymbol{\tau}}
%\newcommand{\Btheta}[0]{\boldsymbol{\theta}}
\newcommand{\Bupsilon}[0]{\boldsymbol{\upsilon}}
\newcommand{\Bvarepsilon}[0]{\boldsymbol{\varepsilon}}
\newcommand{\Bvarphi}[0]{\boldsymbol{\varphi}}
\newcommand{\Bvarpi}[0]{\boldsymbol{\varpi}}
\newcommand{\Bvarrho}[0]{\boldsymbol{\varrho}}
\newcommand{\Bvarsigma}[0]{\boldsymbol{\varsigma}}
\newcommand{\Bvartheta}[0]{\boldsymbol{\vartheta}}
\newcommand{\Bxi}[0]{\boldsymbol{\xi}}
\newcommand{\Bzeta}[0]{\boldsymbol{\zeta}}
%
%</bold and dot greek symbols>
%<infrequent>
%
%\newcommand{\AreaOp}[1]{\AName_{#1}}
%\newcommand{\Babs}[0]{\abs{\BB}}
%\newcommand{\Bcap}[0]{\hat{\BB}}
%\newcommand{\BrPrimeRej}[0]{\rcap(\rcap \wedge \Br')}
%\newcommand{\CA}[0]{\mathcal{A}}
%\newcommand{\Cos}[1]{\cos{\left({#1}\right)}}
%\newcommand{\Det}[1] {\abs{#1}}
%\newcommand{\Dsq}[2] {\frac {\partial^2 {#1}} {\partial {#2}^2}}
%\newcommand{\Exp}[1]{\exp{\left({#1}\right)}}
%\newcommand{\Norm}[1]{\left\lVert{#1}\right\rVert}
%\newcommand{\Sin}[1]{\sin{\left({#1}\right)}}
%\newcommand{\T}[0]{\text{T}}
%\newcommand{\VolumeOp}[1]{\VName_{#1}}
%\newcommand{\agrad}[0]{\Ba \cdot \nabla}
%\newcommand{\alphacap}[0]{\hat{\boldsymbol{\alpha}}}
%\newcommand{\Fcap}[0]{\hat{\BF}}
%\newcommand{\bithree}[0]{{\Bi}_3}
%\newcommand{\bxa}[0]{\Bx\Ba}
%\newcommand{\coordvec}[2]{
%\newcommand{\costheta}[0]{\acap \cdot \xcap}
%\newcommand{\ddt}[1]{\ddot{#1}}
%\newcommand{\ddu}[1] {\frac {d{#1}} {du}}
%\newcommand{\dsqxj}[2] {\frac {\partial^2 {#1}} {\partial {x_{#2}}^2}}
%\newcommand{\dtheta}[1]{\frac{d {#1}}{d \theta}}
%\newcommand{\dt}[1]{\dot{#1}}
%\newcommand{\dt}[1]{\frac{d {#1}}{dt}}
%\newcommand{\dxj}[2] {\frac {\partial {#1}} {\partial {x_{#2}}}}
%\newcommand{\halfPhi}[0]{\frac{\phi}{2}}
%\newcommand{\half}[0]{\inv{2}}
%\newcommand{\inv}[1]{\frac{1}{#1}}
%\newcommand{\laplacian}[0]{\nabla^2}
%\newcommand{\matrixoftx}[3]{
%\newcommand{\nrrp}[0]{\norm{\rcap \wedge \Br'}}
%\newcommand{\oiint}{\bigcirc \hspace{-1.4em} \int \hspace{-.8em} \int}
%\newcommand{\transpose}[1]{{#1}^{\text{T}}}
%\newcommand{\transpose}[1]{{{#1}^{\TextTranspose}}}
%\newcommand{\transpose}[1]{{{#1}^{\text{T}}}}
%\newcommand{\barA}[0]{\bar{A}}
%\newcommand{\qbar}[0]{\bar{q}}
%\newcommand{\qdotbar}[0]{\dot{\bar{q}}}
%
%</infrequent>





%\usepackage[bookmarks=true]{hyperref}

\chapter{Some notes on DeBroglie paper.}
\label{chap:debroglie}
%\author{Peeter Joot \quad peeter.joot@gmail.com}
\date{ Oct. 25, 2008.  debroglie.tex }

%\begin{document}

%\maketitle{}
%\tableofcontents

\section{Motivation. }

The translation of the DeBroglie thesis \citep{AFkracklauerDeBroglie}
appears to have a quite readable introduction to many relativity and 
quantum phenomena.  Here I collect additional notes on things that were
not clear to me.

\section{Chapter 2. }

\subsection{Equation 2.2.10}

Let 

\begin{align*}
\LL = k = g_{ij} \qdot^i \qdot^j
\end{align*}
\begin{align*}
\PD{\qdot^k}{\LL} = 2 g_{ik} \qdot^i \\
\PD{\qdot^k}{\LL} \qdot^k = 2 g_{ik} \qdot^i \qdot^k = 2 K
\end{align*}


\subsection{Equation 2.3.3}

A worldline velocity with respect to some parametrization is

\begin{align*}
\left(\frac{ds}{d\lambda}\right)^2 = \left( \frac{dx}{d\lambda} \cdot \frac{dx}{d\lambda} \right)^2
&= \left(\frac{dx^4}{d\lambda}\right)^2 - \sum_i \left(\frac{dx^i}{d\lambda}\right)^2
\end{align*}

For $\lambda = s$, we can therefore calculate $u^4$:

\begin{align*}
\left(\frac{ds}{ds}\right)^2 = 1 
&= \left(\frac{dx^4}{ds}\right)^2 - \sum_i \left(\frac{dx^i}{ds}\right)^2 \\
&= \left(\frac{dx^4}{ds}\right)^2 - \sum_i \left( \frac{dx^i}{dx^4} \frac{dx^4}{ds} \right)^2 \\
&= \left(\frac{dx^4}{ds}\right)^2 \left( 1 - \sum_i \left( \frac{dx^i}{dx^4} \right)^2 \right) \\
\end{align*}

Or
\begin{align}\label{eqn:debroglie:timearc}
\left(\frac{dx^4}{ds}\right)^2 = \inv{1 - \Bv^2/c^2 } \\
\end{align}

There is a freedom to pick either plus or minus here.  Returning to that later, first 
calculate the remainder of this table of derivatives.  Picking $x^1$ as representative

\begin{align*}
1 &= \inv{1 - \Bv^2/c^2 } 
- \left(\frac{dx^1}{ds}\right)^2
- \left( \frac{dy}{dx^4} \frac{dx^4}{ds} \right)^2
- \left( \frac{dz}{dx^4} \frac{dx^4}{ds} \right)^2 \\
\left(\frac{dx^1}{ds}\right)^2
&= \frac{\Bv^2/c^2 }{1 - \Bv^2/c^2 } 
-\inv{1 - \Bv^2/c^2 } \left( \left( \frac{dy}{dx^4} \right)^2 + \left( \frac{dz}{dx^4} \right)^2 \right) \\
&= \inv{c^2 (1 - \Bv^2/c^2) } \left( \Bv^2 - \left( \frac{dy}{dt} \right)^2 - \left( \frac{dz}{dt} \right)^2 \right) \\
&= \frac{v_x^2/c^2}{1 - \Bv^2/c^2}
\end{align*}

Now, express the coordinate vector for the worldline differential in its entirety:

\begin{align*}
\frac{dx}{ds} =
\frac{d}{ds}(x^1, x^2, x^3, x^4)
&= \frac{dx^4}{ds} \left( \frac{dx^1}{dx^4}, \frac{dx^2}{dx^4}, \frac{dx^3}{dx^4}, 1 \right) \\
&= \frac{dx^4}{ds} ( v_x/c, v_y/c, v_z/c, 1) \\
\end{align*}

This shows that the flexibility to choose a sign for the square roots to obtain $dx^\mu/ds$ must all match the sign for the $dx^4/ds$ term.  Considering a particle at rest in the implied frame associated with these coordinates, one has, by \ref{eqn:debroglie:timearc}

\begin{align*}
\frac{dx}{ds} 
&= \pm \left(0, 0, 0, \inv{\sqrt{1 - \Bv^2/c^2}} \right) \\
&= \pm \left(0, 0, 0, 1 \right) \\
\end{align*}

If we take positive $ds$ to measure increase of time in the rest frame, then there is some sense to picking the positive root.  One
wouldn't have to, since there is also a corresponding freedom to bury a sign adjustment in the $dx_\mu/ds$ derivatives.

%\bibliographystyle{plainnat}
%\bibliography{myrefs}

%\end{document}


\part{dynamics}
\documentclass{article}

\usepackage{amsmath}
\usepackage{mathpazo}

%
% shorthand for bold symbols, convenient for vectors and matrices
%
\newcommand{\Ba}[0]{\mathbf{a}}
\newcommand{\Bb}[0]{\mathbf{b}}
\newcommand{\Bc}[0]{\mathbf{c}}
\newcommand{\Bd}[0]{\mathbf{d}}
\newcommand{\Be}[0]{\mathbf{e}}
\newcommand{\Bf}[0]{\mathbf{f}}
\newcommand{\Bg}[0]{\mathbf{g}}
\newcommand{\Bh}[0]{\mathbf{h}}
\newcommand{\Bi}[0]{\mathbf{i}}
\newcommand{\Bj}[0]{\mathbf{j}}
\newcommand{\Bk}[0]{\mathbf{k}}
\newcommand{\Bl}[0]{\mathbf{l}}
\newcommand{\Bm}[0]{\mathbf{m}}
\newcommand{\Bn}[0]{\mathbf{n}}
\newcommand{\Bo}[0]{\mathbf{o}}
\newcommand{\Bp}[0]{\mathbf{p}}
\newcommand{\Bq}[0]{\mathbf{q}}
\newcommand{\Br}[0]{\mathbf{r}}
\newcommand{\Bs}[0]{\mathbf{s}}
\newcommand{\Bt}[0]{\mathbf{t}}
\newcommand{\Bu}[0]{\mathbf{u}}
\newcommand{\Bv}[0]{\mathbf{v}}
\newcommand{\Bw}[0]{\mathbf{w}}
\newcommand{\Bx}[0]{\mathbf{x}}
\newcommand{\By}[0]{\mathbf{y}}
\newcommand{\Bz}[0]{\mathbf{z}}
\newcommand{\BA}[0]{\mathbf{A}}
\newcommand{\BB}[0]{\mathbf{B}}
\newcommand{\BC}[0]{\mathbf{C}}
\newcommand{\BD}[0]{\mathbf{D}}
\newcommand{\BE}[0]{\mathbf{E}}
\newcommand{\BF}[0]{\mathbf{F}}
\newcommand{\BG}[0]{\mathbf{G}}
\newcommand{\BH}[0]{\mathbf{H}}
\newcommand{\BI}[0]{\mathbf{I}}
\newcommand{\BJ}[0]{\mathbf{J}}
\newcommand{\BK}[0]{\mathbf{K}}
\newcommand{\BL}[0]{\mathbf{L}}
\newcommand{\BM}[0]{\mathbf{M}}
\newcommand{\BN}[0]{\mathbf{N}}
\newcommand{\BO}[0]{\mathbf{O}}
\newcommand{\BP}[0]{\mathbf{P}}
\newcommand{\BQ}[0]{\mathbf{Q}}
\newcommand{\BR}[0]{\mathbf{R}}
\newcommand{\BS}[0]{\mathbf{S}}
\newcommand{\BT}[0]{\mathbf{T}}
\newcommand{\BU}[0]{\mathbf{U}}
\newcommand{\BV}[0]{\mathbf{V}}
\newcommand{\BW}[0]{\mathbf{W}}
\newcommand{\BX}[0]{\mathbf{X}}
\newcommand{\BY}[0]{\mathbf{Y}}
\newcommand{\BZ}[0]{\mathbf{Z}}

\newcommand{\Bzero}[0]{\mathbf{0}}
\newcommand{\Btheta}[0]{\boldsymbol{\theta}}
\newcommand{\Btau}[0]{\boldsymbol{\tau}}
\newcommand{\Bomega}[0]{\boldsymbol{\omega}}

%
% shorthand for unit vectors
%
\newcommand{\acap}[0]{\hat{\Ba}}
\newcommand{\bcap}[0]{\hat{\Bb}}
\newcommand{\ccap}[0]{\hat{\Bc}}
\newcommand{\dcap}[0]{\hat{\Bd}}
\newcommand{\ecap}[0]{\hat{\Be}}
\newcommand{\fcap}[0]{\hat{\Bf}}
\newcommand{\gcap}[0]{\hat{\Bg}}
\newcommand{\hcap}[0]{\hat{\Bh}}
\newcommand{\icap}[0]{\hat{\Bi}}
\newcommand{\jcap}[0]{\hat{\Bj}}
\newcommand{\kcap}[0]{\hat{\Bk}}
\newcommand{\lcap}[0]{\hat{\Bl}}
\newcommand{\mcap}[0]{\hat{\Bm}}
\newcommand{\ncap}[0]{\hat{\Bn}}
\newcommand{\ocap}[0]{\hat{\Bo}}
\newcommand{\pcap}[0]{\hat{\Bp}}
\newcommand{\qcap}[0]{\hat{\Bq}}
\newcommand{\rcap}[0]{\hat{\Br}}
\newcommand{\scap}[0]{\hat{\Bs}}
\newcommand{\tcap}[0]{\hat{\Bt}}
\newcommand{\ucap}[0]{\hat{\Bu}}
\newcommand{\vcap}[0]{\hat{\Bv}}
\newcommand{\wcap}[0]{\hat{\Bw}}
\newcommand{\xcap}[0]{\hat{\Bx}}
\newcommand{\ycap}[0]{\hat{\By}}
\newcommand{\zcap}[0]{\hat{\Bz}}
\newcommand{\thetacap}[0]{\hat{\Btheta}}

%
% to write R^n and C^n in a distinguishable fashion.  Perhaps change this
% to the double lined characters upon figuring out how to do so.
%
\newcommand{\C}[1]{$\mathbb{C}^{#1}$}
\newcommand{\R}[1]{$\mathbb{R}^{#1}$}

%
% various generally useful helpers
%

% derivative of #1 wrt. #2:
\newcommand{\D}[2] {\frac {d#2} {d#1}}

\newcommand{\inv}[1]{\frac{1}{#1}}
\newcommand{\cross}[0]{\times}

\newcommand{\abs}[1]{\lvert{#1}\rvert}
\newcommand{\norm}[1]{\lVert{#1}\rVert}
\newcommand{\innerprod}[2]{\langle{#1}, {#2}\rangle}
\newcommand{\dotprod}[2]{{#1} \cdot {#2}}
\newcommand{\bdotprod}[2]{\left({#1} \cdot {#2}\right)}
\newcommand{\crossprod}[2]{{#1} \cross {#2}}
\newcommand{\tripleprod}[3]{\dotprod{\left(\crossprod{#1}{#2}\right)}{#3}}

\DeclareMathOperator{\Proj}{Proj}
\DeclareMathOperator{\Span}{span}
\DeclareMathOperator{\Sgn}{sgn}
\DeclareMathOperator{\Area}{Area}
\DeclareMathOperator{\Volume}{Volume}

%
% A few miscellaneous things specific to this document
%
\newcommand{\crossop}[1]{\crossprod{#1}{}}

% R2 vector.
\newcommand{\VectorTwo}[2]{
\begin{bmatrix}
 {#1} \\
 {#2}
\end{bmatrix}
}

\newcommand{\VectorN}[1]{
\begin{bmatrix}
{#1}_1 \\
{#1}_2 \\
\vdots \\
{#1}_N \\
\end{bmatrix}
}

\newcommand{\DETuvij}[4]{
\begin{vmatrix}
 {#1}_{#3} & {#1}_{#4} \\
 {#2}_{#3} & {#2}_{#4}
\end{vmatrix}
}

\newcommand{\DETuvwijk}[6]{
\begin{vmatrix}
 {#1}_{#4} & {#1}_{#5} & {#1}_{#6} \\
 {#2}_{#4} & {#2}_{#5} & {#2}_{#6} \\
 {#3}_{#4} & {#3}_{#5} & {#3}_{#6}
\end{vmatrix}
}

\newcommand{\DETuvwxijkl}[8]{
\begin{vmatrix}
 {#1}_{#5} & {#1}_{#6} & {#1}_{#7} & {#1}_{#8} \\
 {#2}_{#5} & {#2}_{#6} & {#2}_{#7} & {#2}_{#8} \\
 {#3}_{#5} & {#3}_{#6} & {#3}_{#7} & {#3}_{#8} \\
 {#4}_{#5} & {#4}_{#6} & {#4}_{#7} & {#4}_{#8} \\
\end{vmatrix}
}

%\newcommand{\DETuvwxyijklm}[10]{
%\begin{vmatrix}
% {#1}_{#6} & {#1}_{#7} & {#1}_{#8} & {#1}_{#9} & {#1}_{#10} \\
% {#2}_{#6} & {#2}_{#7} & {#2}_{#8} & {#2}_{#9} & {#2}_{#10} \\
% {#3}_{#6} & {#3}_{#7} & {#3}_{#8} & {#3}_{#9} & {#3}_{#10} \\
% {#4}_{#6} & {#4}_{#7} & {#4}_{#8} & {#4}_{#9} & {#4}_{#10} \\
% {#5}_{#6} & {#5}_{#7} & {#5}_{#8} & {#5}_{#9} & {#5}_{#10}
%\end{vmatrix}
%}

% R3 vector.
\newcommand{\VectorThree}[3]{
\begin{bmatrix}
 {#1} \\
 {#2} \\
 {#3}
\end{bmatrix}
}


%<misc>
%
\newcommand{\Abs}[1]{{\left\lvert{#1}\right\rvert}}
\newcommand{\spacegrad}[0]{\boldsymbol{\nabla}}
\newcommand{\grad}[0]{\nabla}
\newcommand{\LL}[0]{\mathcal{L}}

% == \partial_{#1} {#2}
\newcommand{\PD}[2]{\frac{\partial {#2}}{\partial {#1}}}
% inline variant
\newcommand{\PDi}[2]{{\partial {#2}}/{\partial {#1}}}

\newcommand{\PDD}[3]{\frac{\partial^2 {#3}}{\partial {#1}\partial {#2}}}
%\newcommand{\PDd}[2]{\frac{\partial^2 {#2}}{{\partial{#1}}^2}}
\newcommand{\PDsq}[2]{\frac{\partial^2 {#2}}{(\partial {#1})^2}}

\newcommand{\Partial}[2]{\frac{\partial {#1}}{\partial {#2}}}
\DeclareMathOperator{\RejName}{Rej}
\newcommand{\Rej}[2]{\RejName_{#1}\left( {#2} \right)}
\newcommand{\Rm}[1]{\mathbb{R}^{#1}}
\newcommand{\Cm}[1]{\mathbb{C}^{#1}}
\newcommand{\conj}[0]{{*}}

%</misc>

% <grade selection>
%
\newcommand{\gpgrade}[2] {{\left\langle{{#1}}\right\rangle}_{#2}}

\newcommand{\gpgradezero}[1] {\gpgrade{#1}{}}
%\newcommand{\gpscalargrade}[1] {{\left\langle{{#1}}\right\rangle}}
%\newcommand{\gpgradezero}[1] {\gpgrade{#1}{0}}

%\newcommand{\gpgradeone}[1] {{\left\langle{{#1}}\right\rangle}_{1}}
\newcommand{\gpgradeone}[1] {\gpgrade{#1}{1}}

\newcommand{\gpgradetwo}[1] {\gpgrade{#1}{2}}
\newcommand{\gpgradethree}[1] {\gpgrade{#1}{3}}
\newcommand{\gpgradefour}[1] {\gpgrade{#1}{4}}
%
% </grade selection>



\newcommand{\adot}[0]{{\dot{a}}}
\newcommand{\bdot}[0]{{\dot{b}}}
% taken for centered dot:
%\newcommand{\cdot}[0]{{\dot{c}}}
%\newcommand{\ddot}[0]{{\dot{d}}}
\newcommand{\edot}[0]{{\dot{e}}}
\newcommand{\fdot}[0]{{\dot{f}}}
\newcommand{\gdot}[0]{{\dot{g}}}
\newcommand{\hdot}[0]{{\dot{h}}}
\newcommand{\idot}[0]{{\dot{i}}}
\newcommand{\jdot}[0]{{\dot{j}}}
\newcommand{\kdot}[0]{{\dot{k}}}
\newcommand{\ldot}[0]{{\dot{l}}}
\newcommand{\mdot}[0]{{\dot{m}}}
\newcommand{\ndot}[0]{{\dot{n}}}
%\newcommand{\odot}[0]{{\dot{o}}}
\newcommand{\pdot}[0]{{\dot{p}}}
\newcommand{\qdot}[0]{{\dot{q}}}
\newcommand{\rdot}[0]{{\dot{r}}}
\newcommand{\sdot}[0]{{\dot{s}}}
\newcommand{\tdot}[0]{{\dot{t}}}
\newcommand{\udot}[0]{{\dot{u}}}
\newcommand{\vdot}[0]{{\dot{v}}}
\newcommand{\wdot}[0]{{\dot{w}}}
\newcommand{\xdot}[0]{{\dot{x}}}
\newcommand{\ydot}[0]{{\dot{y}}}
\newcommand{\zdot}[0]{{\dot{z}}}
\newcommand{\addot}[0]{{\ddot{a}}}
\newcommand{\bddot}[0]{{\ddot{b}}}
\newcommand{\cddot}[0]{{\ddot{c}}}
%\newcommand{\dddot}[0]{{\ddot{d}}}
\newcommand{\eddot}[0]{{\ddot{e}}}
\newcommand{\fddot}[0]{{\ddot{f}}}
\newcommand{\gddot}[0]{{\ddot{g}}}
\newcommand{\hddot}[0]{{\ddot{h}}}
\newcommand{\iddot}[0]{{\ddot{i}}}
\newcommand{\jddot}[0]{{\ddot{j}}}
\newcommand{\kddot}[0]{{\ddot{k}}}
\newcommand{\lddot}[0]{{\ddot{l}}}
\newcommand{\mddot}[0]{{\ddot{m}}}
\newcommand{\nddot}[0]{{\ddot{n}}}
\newcommand{\oddot}[0]{{\ddot{o}}}
\newcommand{\pddot}[0]{{\ddot{p}}}
\newcommand{\qddot}[0]{{\ddot{q}}}
\newcommand{\rddot}[0]{{\ddot{r}}}
\newcommand{\sddot}[0]{{\ddot{s}}}
\newcommand{\tddot}[0]{{\ddot{t}}}
\newcommand{\uddot}[0]{{\ddot{u}}}
\newcommand{\vddot}[0]{{\ddot{v}}}
\newcommand{\wddot}[0]{{\ddot{w}}}
\newcommand{\xddot}[0]{{\ddot{x}}}
\newcommand{\yddot}[0]{{\ddot{y}}}
\newcommand{\zddot}[0]{{\ddot{z}}}

%<bold and dot greek symbols>
%

\newcommand{\Deltadot}[0]{{\dot{\Delta}}}
\newcommand{\Gammadot}[0]{{\dot{\Gamma}}}
\newcommand{\Lambdadot}[0]{{\dot{\Lambda}}}
\newcommand{\Omegadot}[0]{{\dot{\Omega}}}
\newcommand{\Phidot}[0]{{\dot{\Phi}}}
\newcommand{\Pidot}[0]{{\dot{\Pi}}}
\newcommand{\Psidot}[0]{{\dot{\Psi}}}
\newcommand{\Sigmadot}[0]{{\dot{\Sigma}}}
\newcommand{\Thetadot}[0]{{\dot{\Theta}}}
\newcommand{\Upsilondot}[0]{{\dot{\Upsilon}}}
\newcommand{\Xidot}[0]{{\dot{\Xi}}}
\newcommand{\alphadot}[0]{{\dot{\alpha}}}
\newcommand{\betadot}[0]{{\dot{\beta}}}
\newcommand{\chidot}[0]{{\dot{\chi}}}
\newcommand{\deltadot}[0]{{\dot{\delta}}}
\newcommand{\epsilondot}[0]{{\dot{\epsilon}}}
\newcommand{\etadot}[0]{{\dot{\eta}}}
\newcommand{\gammadot}[0]{{\dot{\gamma}}}
\newcommand{\kappadot}[0]{{\dot{\kappa}}}
\newcommand{\lambdadot}[0]{{\dot{\lambda}}}
\newcommand{\mudot}[0]{{\dot{\mu}}}
\newcommand{\nudot}[0]{{\dot{\nu}}}
\newcommand{\omegadot}[0]{{\dot{\omega}}}
\newcommand{\phidot}[0]{{\dot{\phi}}}
\newcommand{\pidot}[0]{{\dot{\pi}}}
\newcommand{\psidot}[0]{{\dot{\psi}}}
\newcommand{\rhodot}[0]{{\dot{\rho}}}
\newcommand{\sigmadot}[0]{{\dot{\sigma}}}
\newcommand{\taudot}[0]{{\dot{\tau}}}
\newcommand{\thetadot}[0]{{\dot{\theta}}}
\newcommand{\upsilondot}[0]{{\dot{\upsilon}}}
\newcommand{\varepsilondot}[0]{{\dot{\varepsilon}}}
\newcommand{\varphidot}[0]{{\dot{\varphi}}}
\newcommand{\varpidot}[0]{{\dot{\varpi}}}
\newcommand{\varrhodot}[0]{{\dot{\varrho}}}
\newcommand{\varsigmadot}[0]{{\dot{\varsigma}}}
\newcommand{\varthetadot}[0]{{\dot{\vartheta}}}
\newcommand{\xidot}[0]{{\dot{\xi}}}
\newcommand{\zetadot}[0]{{\dot{\zeta}}}

\newcommand{\Deltaddot}[0]{{\ddot{\Delta}}}
\newcommand{\Gammaddot}[0]{{\ddot{\Gamma}}}
\newcommand{\Lambdaddot}[0]{{\ddot{\Lambda}}}
\newcommand{\Omegaddot}[0]{{\ddot{\Omega}}}
\newcommand{\Phiddot}[0]{{\ddot{\Phi}}}
\newcommand{\Piddot}[0]{{\ddot{\Pi}}}
\newcommand{\Psiddot}[0]{{\ddot{\Psi}}}
\newcommand{\Sigmaddot}[0]{{\ddot{\Sigma}}}
\newcommand{\Thetaddot}[0]{{\ddot{\Theta}}}
\newcommand{\Upsilonddot}[0]{{\ddot{\Upsilon}}}
\newcommand{\Xiddot}[0]{{\ddot{\Xi}}}
\newcommand{\alphaddot}[0]{{\ddot{\alpha}}}
\newcommand{\betaddot}[0]{{\ddot{\beta}}}
\newcommand{\chiddot}[0]{{\ddot{\chi}}}
\newcommand{\deltaddot}[0]{{\ddot{\delta}}}
\newcommand{\epsilonddot}[0]{{\ddot{\epsilon}}}
\newcommand{\etaddot}[0]{{\ddot{\eta}}}
\newcommand{\gammaddot}[0]{{\ddot{\gamma}}}
\newcommand{\kappaddot}[0]{{\ddot{\kappa}}}
\newcommand{\lambdaddot}[0]{{\ddot{\lambda}}}
\newcommand{\muddot}[0]{{\ddot{\mu}}}
\newcommand{\nuddot}[0]{{\ddot{\nu}}}
\newcommand{\omegaddot}[0]{{\ddot{\omega}}}
\newcommand{\phiddot}[0]{{\ddot{\phi}}}
\newcommand{\piddot}[0]{{\ddot{\pi}}}
\newcommand{\psiddot}[0]{{\ddot{\psi}}}
\newcommand{\rhoddot}[0]{{\ddot{\rho}}}
\newcommand{\sigmaddot}[0]{{\ddot{\sigma}}}
\newcommand{\tauddot}[0]{{\ddot{\tau}}}
\newcommand{\thetaddot}[0]{{\ddot{\theta}}}
\newcommand{\upsilonddot}[0]{{\ddot{\upsilon}}}
\newcommand{\varepsilonddot}[0]{{\ddot{\varepsilon}}}
\newcommand{\varphiddot}[0]{{\ddot{\varphi}}}
\newcommand{\varpiddot}[0]{{\ddot{\varpi}}}
\newcommand{\varrhoddot}[0]{{\ddot{\varrho}}}
\newcommand{\varsigmaddot}[0]{{\ddot{\varsigma}}}
\newcommand{\varthetaddot}[0]{{\ddot{\vartheta}}}
\newcommand{\xiddot}[0]{{\ddot{\xi}}}
\newcommand{\zetaddot}[0]{{\ddot{\zeta}}}

\newcommand{\BDelta}[0]{\boldsymbol{\Delta}}
\newcommand{\BGamma}[0]{\boldsymbol{\Gamma}}
\newcommand{\BLambda}[0]{\boldsymbol{\Lambda}}
\newcommand{\BOmega}[0]{\boldsymbol{\Omega}}
\newcommand{\BPhi}[0]{\boldsymbol{\Phi}}
\newcommand{\BPi}[0]{\boldsymbol{\Pi}}
\newcommand{\BPsi}[0]{\boldsymbol{\Psi}}
\newcommand{\BSigma}[0]{\boldsymbol{\Sigma}}
\newcommand{\BTheta}[0]{\boldsymbol{\Theta}}
\newcommand{\BUpsilon}[0]{\boldsymbol{\Upsilon}}
\newcommand{\BXi}[0]{\boldsymbol{\Xi}}
\newcommand{\Balpha}[0]{\boldsymbol{\alpha}}
\newcommand{\Bbeta}[0]{\boldsymbol{\beta}}
\newcommand{\Bchi}[0]{\boldsymbol{\chi}}
\newcommand{\Bdelta}[0]{\boldsymbol{\delta}}
\newcommand{\Bepsilon}[0]{\boldsymbol{\epsilon}}
\newcommand{\Beta}[0]{\boldsymbol{\eta}}
\newcommand{\Bgamma}[0]{\boldsymbol{\gamma}}
\newcommand{\Bkappa}[0]{\boldsymbol{\kappa}}
\newcommand{\Blambda}[0]{\boldsymbol{\lambda}}
\newcommand{\Bmu}[0]{\boldsymbol{\mu}}
\newcommand{\Bnu}[0]{\boldsymbol{\nu}}
%\newcommand{\Bomega}[0]{\boldsymbol{\omega}}
\newcommand{\Bphi}[0]{\boldsymbol{\phi}}
\newcommand{\Bpi}[0]{\boldsymbol{\pi}}
\newcommand{\Bpsi}[0]{\boldsymbol{\psi}}
\newcommand{\Brho}[0]{\boldsymbol{\rho}}
\newcommand{\Bsigma}[0]{\boldsymbol{\sigma}}
%\newcommand{\Btau}[0]{\boldsymbol{\tau}}
%\newcommand{\Btheta}[0]{\boldsymbol{\theta}}
\newcommand{\Bupsilon}[0]{\boldsymbol{\upsilon}}
\newcommand{\Bvarepsilon}[0]{\boldsymbol{\varepsilon}}
\newcommand{\Bvarphi}[0]{\boldsymbol{\varphi}}
\newcommand{\Bvarpi}[0]{\boldsymbol{\varpi}}
\newcommand{\Bvarrho}[0]{\boldsymbol{\varrho}}
\newcommand{\Bvarsigma}[0]{\boldsymbol{\varsigma}}
\newcommand{\Bvartheta}[0]{\boldsymbol{\vartheta}}
\newcommand{\Bxi}[0]{\boldsymbol{\xi}}
\newcommand{\Bzeta}[0]{\boldsymbol{\zeta}}
%
%</bold and dot greek symbols>
%<infrequent>
%
%\newcommand{\AreaOp}[1]{\AName_{#1}}
%\newcommand{\Babs}[0]{\abs{\BB}}
%\newcommand{\Bcap}[0]{\hat{\BB}}
%\newcommand{\BrPrimeRej}[0]{\rcap(\rcap \wedge \Br')}
%\newcommand{\CA}[0]{\mathcal{A}}
%\newcommand{\Cos}[1]{\cos{\left({#1}\right)}}
%\newcommand{\Det}[1] {\abs{#1}}
%\newcommand{\Dsq}[2] {\frac {\partial^2 {#1}} {\partial {#2}^2}}
%\newcommand{\Exp}[1]{\exp{\left({#1}\right)}}
%\newcommand{\Norm}[1]{\left\lVert{#1}\right\rVert}
%\newcommand{\Sin}[1]{\sin{\left({#1}\right)}}
%\newcommand{\T}[0]{\text{T}}
%\newcommand{\VolumeOp}[1]{\VName_{#1}}
%\newcommand{\agrad}[0]{\Ba \cdot \nabla}
%\newcommand{\alphacap}[0]{\hat{\boldsymbol{\alpha}}}
%\newcommand{\Fcap}[0]{\hat{\BF}}
%\newcommand{\bithree}[0]{{\Bi}_3}
%\newcommand{\bxa}[0]{\Bx\Ba}
%\newcommand{\coordvec}[2]{
%\newcommand{\costheta}[0]{\acap \cdot \xcap}
%\newcommand{\ddt}[1]{\ddot{#1}}
%\newcommand{\ddu}[1] {\frac {d{#1}} {du}}
%\newcommand{\dsqxj}[2] {\frac {\partial^2 {#1}} {\partial {x_{#2}}^2}}
%\newcommand{\dtheta}[1]{\frac{d {#1}}{d \theta}}
%\newcommand{\dt}[1]{\dot{#1}}
%\newcommand{\dt}[1]{\frac{d {#1}}{dt}}
%\newcommand{\dxj}[2] {\frac {\partial {#1}} {\partial {x_{#2}}}}
%\newcommand{\halfPhi}[0]{\frac{\phi}{2}}
%\newcommand{\half}[0]{\inv{2}}
%\newcommand{\inv}[1]{\frac{1}{#1}}
%\newcommand{\laplacian}[0]{\nabla^2}
%\newcommand{\matrixoftx}[3]{
%\newcommand{\nrrp}[0]{\norm{\rcap \wedge \Br'}}
%\newcommand{\oiint}{\bigcirc \hspace{-1.4em} \int \hspace{-.8em} \int}
%\newcommand{\transpose}[1]{{#1}^{\text{T}}}
%\newcommand{\transpose}[1]{{{#1}^{\TextTranspose}}}
%\newcommand{\transpose}[1]{{{#1}^{\text{T}}}}
%\newcommand{\barA}[0]{\bar{A}}
%\newcommand{\qbar}[0]{\bar{q}}
%\newcommand{\qdotbar}[0]{\dot{\bar{q}}}
%
%</infrequent>





\usepackage[bookmarks=true]{hyperref}

\usepackage{color,cite,graphicx}
   % use colour in the document, put your citations as [1-4]
   % rather than [1,2,3,4] (it looks nicer, and the extended LaTeX2e
   % graphics package. 
\usepackage{latexsym,amssymb,epsf} % don't remember if these are
   % needed, but their inclusion can't do any damage


\title{ Time for two bodies to converge in space. }
\author{Peeter Joot}
\date{ Dec 15, 2008.  Last Revision: $Date: 2008/12/16 04:45:14 $ }

\begin{document}

\maketitle{}

%\tableofcontents
%
%\section{}

Assuming that the initial relative velocities of the two bodies lies along
the line between them this can be treated as a one dimensional problem.

We have two force equations

\begin{align*}
\text{Force on} \quad m_2 &= - G m_1 m_2 (x_2 - x_1)^{-2} = m_2 {x_2}'' \\
\text{Force on} \quad m_1 &=  G m_1 m_2 (x_2 - x_1)^{-2} = m_1 {x_1}''
\end{align*}

Subtraction (scaled) of these two equations leaves an integration problem for a difference of position variable $u = x_2 - x_1$

\begin{align*}
- G (x_2 - x_1)^{-2} (m_1 + m_2)= (x_2 - x_1)'' \\
\implies
- G (m_1 + m_2) \inv{u^2} = u'' \\
\end{align*}

The chain rule can be used for a first elimination of a time derivative

\begin{align*}
\frac{d}{dt} = \frac{du}{dt} \frac{d}{du}
\end{align*}

With $k = G(m_1 + m_2)$, this is

\begin{align*}
\frac{du}{dt} \frac{d}{du} \left(\frac{du}{dt}\right) &= -k u^{-2}
\end{align*}

Introducing a difference of position "velocity" the LHS of this is

\begin{align*}
\frac{du}{dt} \frac{d}{du} \left(\frac{du}{dt}\right)
&= v \frac{dv}{du} \\
&= \frac{d}{du} \left( \inv{2}v^2 \right) \\
\end{align*}

So now we have something that can be directly integrated:

\begin{align*}
\inv{2}v^2 &= k \inv{u} + \inv{2}C
\end{align*}

Fixing the constant in terms of the initial position and speed this is

\begin{align}\label{eqn:velocity}
\frac{du}{dt} &= \pm \sqrt{v_0^2 + 2 G M \left(\inv{u} - \inv{u_0}\right)}
\end{align}

and we can integrate for the time

\begin{align*}
\delta t &= \pm \int_{u_0}^{u_1} du \left(v_0^2 + 2 G M \left(\inv{u} - \inv{u_0}\right)\right)^{-1/2}
\end{align*}

The integral here is of the form

\begin{align*}
\int \frac{\sqrt{u}}{\sqrt{u + \alpha}} du
\end{align*}

and now it's time to pull out the table of integrals, and put in some numbers.

\href{http://calc101.com/webMathematica/integrals.jsp}{webmathematica} gives, in figure \ref{fig:msp}

\begin{figure}[htp]
\centering
\includegraphics[totalheight=0.4\textheight]{msp}
\caption{the integral}\label{fig:msp}
\end{figure}

More interesting than the numbers themselves is to think through the physical meaning for the $\pm$.  It makes sense in the velocity equation \ref{eqn:velocity} since the particles can be initially moving towards or away from each other (or stationary).  However, do negative time "solutions" indicate the particles can never converge to the desired separation?

%\bibliographystyle{plainnat}
%\bibliography{myrefs}

\end{document}


\part{feynman}
\include{star_distance}

\part{fletcher}
%
% Copyright � 2012 Peeter Joot.  All Rights Reserved.
% Licenced as described in the file LICENSE under the root directory of this GIT repository.
%

% 
% 
%\documentclass{article}

%\usepackage{amsmath}
\usepackage{mathpazo}

%
% shorthand for bold symbols, convenient for vectors and matrices
%
\newcommand{\Ba}[0]{\mathbf{a}}
\newcommand{\Bb}[0]{\mathbf{b}}
\newcommand{\Bc}[0]{\mathbf{c}}
\newcommand{\Bd}[0]{\mathbf{d}}
\newcommand{\Be}[0]{\mathbf{e}}
\newcommand{\Bf}[0]{\mathbf{f}}
\newcommand{\Bg}[0]{\mathbf{g}}
\newcommand{\Bh}[0]{\mathbf{h}}
\newcommand{\Bi}[0]{\mathbf{i}}
\newcommand{\Bj}[0]{\mathbf{j}}
\newcommand{\Bk}[0]{\mathbf{k}}
\newcommand{\Bl}[0]{\mathbf{l}}
\newcommand{\Bm}[0]{\mathbf{m}}
\newcommand{\Bn}[0]{\mathbf{n}}
\newcommand{\Bo}[0]{\mathbf{o}}
\newcommand{\Bp}[0]{\mathbf{p}}
\newcommand{\Bq}[0]{\mathbf{q}}
\newcommand{\Br}[0]{\mathbf{r}}
\newcommand{\Bs}[0]{\mathbf{s}}
\newcommand{\Bt}[0]{\mathbf{t}}
\newcommand{\Bu}[0]{\mathbf{u}}
\newcommand{\Bv}[0]{\mathbf{v}}
\newcommand{\Bw}[0]{\mathbf{w}}
\newcommand{\Bx}[0]{\mathbf{x}}
\newcommand{\By}[0]{\mathbf{y}}
\newcommand{\Bz}[0]{\mathbf{z}}
\newcommand{\BA}[0]{\mathbf{A}}
\newcommand{\BB}[0]{\mathbf{B}}
\newcommand{\BC}[0]{\mathbf{C}}
\newcommand{\BD}[0]{\mathbf{D}}
\newcommand{\BE}[0]{\mathbf{E}}
\newcommand{\BF}[0]{\mathbf{F}}
\newcommand{\BG}[0]{\mathbf{G}}
\newcommand{\BH}[0]{\mathbf{H}}
\newcommand{\BI}[0]{\mathbf{I}}
\newcommand{\BJ}[0]{\mathbf{J}}
\newcommand{\BK}[0]{\mathbf{K}}
\newcommand{\BL}[0]{\mathbf{L}}
\newcommand{\BM}[0]{\mathbf{M}}
\newcommand{\BN}[0]{\mathbf{N}}
\newcommand{\BO}[0]{\mathbf{O}}
\newcommand{\BP}[0]{\mathbf{P}}
\newcommand{\BQ}[0]{\mathbf{Q}}
\newcommand{\BR}[0]{\mathbf{R}}
\newcommand{\BS}[0]{\mathbf{S}}
\newcommand{\BT}[0]{\mathbf{T}}
\newcommand{\BU}[0]{\mathbf{U}}
\newcommand{\BV}[0]{\mathbf{V}}
\newcommand{\BW}[0]{\mathbf{W}}
\newcommand{\BX}[0]{\mathbf{X}}
\newcommand{\BY}[0]{\mathbf{Y}}
\newcommand{\BZ}[0]{\mathbf{Z}}

\newcommand{\Bzero}[0]{\mathbf{0}}
\newcommand{\Btheta}[0]{\boldsymbol{\theta}}
\newcommand{\Btau}[0]{\boldsymbol{\tau}}
\newcommand{\Bomega}[0]{\boldsymbol{\omega}}

%
% shorthand for unit vectors
%
\newcommand{\acap}[0]{\hat{\Ba}}
\newcommand{\bcap}[0]{\hat{\Bb}}
\newcommand{\ccap}[0]{\hat{\Bc}}
\newcommand{\dcap}[0]{\hat{\Bd}}
\newcommand{\ecap}[0]{\hat{\Be}}
\newcommand{\fcap}[0]{\hat{\Bf}}
\newcommand{\gcap}[0]{\hat{\Bg}}
\newcommand{\hcap}[0]{\hat{\Bh}}
\newcommand{\icap}[0]{\hat{\Bi}}
\newcommand{\jcap}[0]{\hat{\Bj}}
\newcommand{\kcap}[0]{\hat{\Bk}}
\newcommand{\lcap}[0]{\hat{\Bl}}
\newcommand{\mcap}[0]{\hat{\Bm}}
\newcommand{\ncap}[0]{\hat{\Bn}}
\newcommand{\ocap}[0]{\hat{\Bo}}
\newcommand{\pcap}[0]{\hat{\Bp}}
\newcommand{\qcap}[0]{\hat{\Bq}}
\newcommand{\rcap}[0]{\hat{\Br}}
\newcommand{\scap}[0]{\hat{\Bs}}
\newcommand{\tcap}[0]{\hat{\Bt}}
\newcommand{\ucap}[0]{\hat{\Bu}}
\newcommand{\vcap}[0]{\hat{\Bv}}
\newcommand{\wcap}[0]{\hat{\Bw}}
\newcommand{\xcap}[0]{\hat{\Bx}}
\newcommand{\ycap}[0]{\hat{\By}}
\newcommand{\zcap}[0]{\hat{\Bz}}
\newcommand{\thetacap}[0]{\hat{\Btheta}}

%
% to write R^n and C^n in a distinguishable fashion.  Perhaps change this
% to the double lined characters upon figuring out how to do so.
%
\newcommand{\C}[1]{$\mathbb{C}^{#1}$}
\newcommand{\R}[1]{$\mathbb{R}^{#1}$}

%
% various generally useful helpers
%

% derivative of #1 wrt. #2:
\newcommand{\D}[2] {\frac {d#2} {d#1}}

\newcommand{\inv}[1]{\frac{1}{#1}}
\newcommand{\cross}[0]{\times}

\newcommand{\abs}[1]{\lvert{#1}\rvert}
\newcommand{\norm}[1]{\lVert{#1}\rVert}
\newcommand{\innerprod}[2]{\langle{#1}, {#2}\rangle}
\newcommand{\dotprod}[2]{{#1} \cdot {#2}}
\newcommand{\bdotprod}[2]{\left({#1} \cdot {#2}\right)}
\newcommand{\crossprod}[2]{{#1} \cross {#2}}
\newcommand{\tripleprod}[3]{\dotprod{\left(\crossprod{#1}{#2}\right)}{#3}}

\DeclareMathOperator{\Proj}{Proj}
\DeclareMathOperator{\Span}{span}
\DeclareMathOperator{\Sgn}{sgn}
\DeclareMathOperator{\Area}{Area}
\DeclareMathOperator{\Volume}{Volume}

%
% A few miscellaneous things specific to this document
%
\newcommand{\crossop}[1]{\crossprod{#1}{}}

% R2 vector.
\newcommand{\VectorTwo}[2]{
\begin{bmatrix}
 {#1} \\
 {#2}
\end{bmatrix}
}

\newcommand{\VectorN}[1]{
\begin{bmatrix}
{#1}_1 \\
{#1}_2 \\
\vdots \\
{#1}_N \\
\end{bmatrix}
}

\newcommand{\DETuvij}[4]{
\begin{vmatrix}
 {#1}_{#3} & {#1}_{#4} \\
 {#2}_{#3} & {#2}_{#4}
\end{vmatrix}
}

\newcommand{\DETuvwijk}[6]{
\begin{vmatrix}
 {#1}_{#4} & {#1}_{#5} & {#1}_{#6} \\
 {#2}_{#4} & {#2}_{#5} & {#2}_{#6} \\
 {#3}_{#4} & {#3}_{#5} & {#3}_{#6}
\end{vmatrix}
}

\newcommand{\DETuvwxijkl}[8]{
\begin{vmatrix}
 {#1}_{#5} & {#1}_{#6} & {#1}_{#7} & {#1}_{#8} \\
 {#2}_{#5} & {#2}_{#6} & {#2}_{#7} & {#2}_{#8} \\
 {#3}_{#5} & {#3}_{#6} & {#3}_{#7} & {#3}_{#8} \\
 {#4}_{#5} & {#4}_{#6} & {#4}_{#7} & {#4}_{#8} \\
\end{vmatrix}
}

%\newcommand{\DETuvwxyijklm}[10]{
%\begin{vmatrix}
% {#1}_{#6} & {#1}_{#7} & {#1}_{#8} & {#1}_{#9} & {#1}_{#10} \\
% {#2}_{#6} & {#2}_{#7} & {#2}_{#8} & {#2}_{#9} & {#2}_{#10} \\
% {#3}_{#6} & {#3}_{#7} & {#3}_{#8} & {#3}_{#9} & {#3}_{#10} \\
% {#4}_{#6} & {#4}_{#7} & {#4}_{#8} & {#4}_{#9} & {#4}_{#10} \\
% {#5}_{#6} & {#5}_{#7} & {#5}_{#8} & {#5}_{#9} & {#5}_{#10}
%\end{vmatrix}
%}

% R3 vector.
\newcommand{\VectorThree}[3]{
\begin{bmatrix}
 {#1} \\
 {#2} \\
 {#3}
\end{bmatrix}
}


%%<misc>
%
\newcommand{\Abs}[1]{{\left\lvert{#1}\right\rvert}}
\newcommand{\spacegrad}[0]{\boldsymbol{\nabla}}
\newcommand{\grad}[0]{\nabla}
\newcommand{\LL}[0]{\mathcal{L}}

% == \partial_{#1} {#2}
\newcommand{\PD}[2]{\frac{\partial {#2}}{\partial {#1}}}
% inline variant
\newcommand{\PDi}[2]{{\partial {#2}}/{\partial {#1}}}

\newcommand{\PDD}[3]{\frac{\partial^2 {#3}}{\partial {#1}\partial {#2}}}
%\newcommand{\PDd}[2]{\frac{\partial^2 {#2}}{{\partial{#1}}^2}}
\newcommand{\PDsq}[2]{\frac{\partial^2 {#2}}{(\partial {#1})^2}}

\newcommand{\Partial}[2]{\frac{\partial {#1}}{\partial {#2}}}
\DeclareMathOperator{\RejName}{Rej}
\newcommand{\Rej}[2]{\RejName_{#1}\left( {#2} \right)}
\newcommand{\Rm}[1]{\mathbb{R}^{#1}}
\newcommand{\Cm}[1]{\mathbb{C}^{#1}}
\newcommand{\conj}[0]{{*}}

%</misc>

% <grade selection>
%
\newcommand{\gpgrade}[2] {{\left\langle{{#1}}\right\rangle}_{#2}}

\newcommand{\gpgradezero}[1] {\gpgrade{#1}{}}
%\newcommand{\gpscalargrade}[1] {{\left\langle{{#1}}\right\rangle}}
%\newcommand{\gpgradezero}[1] {\gpgrade{#1}{0}}

%\newcommand{\gpgradeone}[1] {{\left\langle{{#1}}\right\rangle}_{1}}
\newcommand{\gpgradeone}[1] {\gpgrade{#1}{1}}

\newcommand{\gpgradetwo}[1] {\gpgrade{#1}{2}}
\newcommand{\gpgradethree}[1] {\gpgrade{#1}{3}}
\newcommand{\gpgradefour}[1] {\gpgrade{#1}{4}}
%
% </grade selection>



\newcommand{\adot}[0]{{\dot{a}}}
\newcommand{\bdot}[0]{{\dot{b}}}
% taken for centered dot:
%\newcommand{\cdot}[0]{{\dot{c}}}
%\newcommand{\ddot}[0]{{\dot{d}}}
\newcommand{\edot}[0]{{\dot{e}}}
\newcommand{\fdot}[0]{{\dot{f}}}
\newcommand{\gdot}[0]{{\dot{g}}}
\newcommand{\hdot}[0]{{\dot{h}}}
\newcommand{\idot}[0]{{\dot{i}}}
\newcommand{\jdot}[0]{{\dot{j}}}
\newcommand{\kdot}[0]{{\dot{k}}}
\newcommand{\ldot}[0]{{\dot{l}}}
\newcommand{\mdot}[0]{{\dot{m}}}
\newcommand{\ndot}[0]{{\dot{n}}}
%\newcommand{\odot}[0]{{\dot{o}}}
\newcommand{\pdot}[0]{{\dot{p}}}
\newcommand{\qdot}[0]{{\dot{q}}}
\newcommand{\rdot}[0]{{\dot{r}}}
\newcommand{\sdot}[0]{{\dot{s}}}
\newcommand{\tdot}[0]{{\dot{t}}}
\newcommand{\udot}[0]{{\dot{u}}}
\newcommand{\vdot}[0]{{\dot{v}}}
\newcommand{\wdot}[0]{{\dot{w}}}
\newcommand{\xdot}[0]{{\dot{x}}}
\newcommand{\ydot}[0]{{\dot{y}}}
\newcommand{\zdot}[0]{{\dot{z}}}
\newcommand{\addot}[0]{{\ddot{a}}}
\newcommand{\bddot}[0]{{\ddot{b}}}
\newcommand{\cddot}[0]{{\ddot{c}}}
%\newcommand{\dddot}[0]{{\ddot{d}}}
\newcommand{\eddot}[0]{{\ddot{e}}}
\newcommand{\fddot}[0]{{\ddot{f}}}
\newcommand{\gddot}[0]{{\ddot{g}}}
\newcommand{\hddot}[0]{{\ddot{h}}}
\newcommand{\iddot}[0]{{\ddot{i}}}
\newcommand{\jddot}[0]{{\ddot{j}}}
\newcommand{\kddot}[0]{{\ddot{k}}}
\newcommand{\lddot}[0]{{\ddot{l}}}
\newcommand{\mddot}[0]{{\ddot{m}}}
\newcommand{\nddot}[0]{{\ddot{n}}}
\newcommand{\oddot}[0]{{\ddot{o}}}
\newcommand{\pddot}[0]{{\ddot{p}}}
\newcommand{\qddot}[0]{{\ddot{q}}}
\newcommand{\rddot}[0]{{\ddot{r}}}
\newcommand{\sddot}[0]{{\ddot{s}}}
\newcommand{\tddot}[0]{{\ddot{t}}}
\newcommand{\uddot}[0]{{\ddot{u}}}
\newcommand{\vddot}[0]{{\ddot{v}}}
\newcommand{\wddot}[0]{{\ddot{w}}}
\newcommand{\xddot}[0]{{\ddot{x}}}
\newcommand{\yddot}[0]{{\ddot{y}}}
\newcommand{\zddot}[0]{{\ddot{z}}}

%<bold and dot greek symbols>
%

\newcommand{\Deltadot}[0]{{\dot{\Delta}}}
\newcommand{\Gammadot}[0]{{\dot{\Gamma}}}
\newcommand{\Lambdadot}[0]{{\dot{\Lambda}}}
\newcommand{\Omegadot}[0]{{\dot{\Omega}}}
\newcommand{\Phidot}[0]{{\dot{\Phi}}}
\newcommand{\Pidot}[0]{{\dot{\Pi}}}
\newcommand{\Psidot}[0]{{\dot{\Psi}}}
\newcommand{\Sigmadot}[0]{{\dot{\Sigma}}}
\newcommand{\Thetadot}[0]{{\dot{\Theta}}}
\newcommand{\Upsilondot}[0]{{\dot{\Upsilon}}}
\newcommand{\Xidot}[0]{{\dot{\Xi}}}
\newcommand{\alphadot}[0]{{\dot{\alpha}}}
\newcommand{\betadot}[0]{{\dot{\beta}}}
\newcommand{\chidot}[0]{{\dot{\chi}}}
\newcommand{\deltadot}[0]{{\dot{\delta}}}
\newcommand{\epsilondot}[0]{{\dot{\epsilon}}}
\newcommand{\etadot}[0]{{\dot{\eta}}}
\newcommand{\gammadot}[0]{{\dot{\gamma}}}
\newcommand{\kappadot}[0]{{\dot{\kappa}}}
\newcommand{\lambdadot}[0]{{\dot{\lambda}}}
\newcommand{\mudot}[0]{{\dot{\mu}}}
\newcommand{\nudot}[0]{{\dot{\nu}}}
\newcommand{\omegadot}[0]{{\dot{\omega}}}
\newcommand{\phidot}[0]{{\dot{\phi}}}
\newcommand{\pidot}[0]{{\dot{\pi}}}
\newcommand{\psidot}[0]{{\dot{\psi}}}
\newcommand{\rhodot}[0]{{\dot{\rho}}}
\newcommand{\sigmadot}[0]{{\dot{\sigma}}}
\newcommand{\taudot}[0]{{\dot{\tau}}}
\newcommand{\thetadot}[0]{{\dot{\theta}}}
\newcommand{\upsilondot}[0]{{\dot{\upsilon}}}
\newcommand{\varepsilondot}[0]{{\dot{\varepsilon}}}
\newcommand{\varphidot}[0]{{\dot{\varphi}}}
\newcommand{\varpidot}[0]{{\dot{\varpi}}}
\newcommand{\varrhodot}[0]{{\dot{\varrho}}}
\newcommand{\varsigmadot}[0]{{\dot{\varsigma}}}
\newcommand{\varthetadot}[0]{{\dot{\vartheta}}}
\newcommand{\xidot}[0]{{\dot{\xi}}}
\newcommand{\zetadot}[0]{{\dot{\zeta}}}

\newcommand{\Deltaddot}[0]{{\ddot{\Delta}}}
\newcommand{\Gammaddot}[0]{{\ddot{\Gamma}}}
\newcommand{\Lambdaddot}[0]{{\ddot{\Lambda}}}
\newcommand{\Omegaddot}[0]{{\ddot{\Omega}}}
\newcommand{\Phiddot}[0]{{\ddot{\Phi}}}
\newcommand{\Piddot}[0]{{\ddot{\Pi}}}
\newcommand{\Psiddot}[0]{{\ddot{\Psi}}}
\newcommand{\Sigmaddot}[0]{{\ddot{\Sigma}}}
\newcommand{\Thetaddot}[0]{{\ddot{\Theta}}}
\newcommand{\Upsilonddot}[0]{{\ddot{\Upsilon}}}
\newcommand{\Xiddot}[0]{{\ddot{\Xi}}}
\newcommand{\alphaddot}[0]{{\ddot{\alpha}}}
\newcommand{\betaddot}[0]{{\ddot{\beta}}}
\newcommand{\chiddot}[0]{{\ddot{\chi}}}
\newcommand{\deltaddot}[0]{{\ddot{\delta}}}
\newcommand{\epsilonddot}[0]{{\ddot{\epsilon}}}
\newcommand{\etaddot}[0]{{\ddot{\eta}}}
\newcommand{\gammaddot}[0]{{\ddot{\gamma}}}
\newcommand{\kappaddot}[0]{{\ddot{\kappa}}}
\newcommand{\lambdaddot}[0]{{\ddot{\lambda}}}
\newcommand{\muddot}[0]{{\ddot{\mu}}}
\newcommand{\nuddot}[0]{{\ddot{\nu}}}
\newcommand{\omegaddot}[0]{{\ddot{\omega}}}
\newcommand{\phiddot}[0]{{\ddot{\phi}}}
\newcommand{\piddot}[0]{{\ddot{\pi}}}
\newcommand{\psiddot}[0]{{\ddot{\psi}}}
\newcommand{\rhoddot}[0]{{\ddot{\rho}}}
\newcommand{\sigmaddot}[0]{{\ddot{\sigma}}}
\newcommand{\tauddot}[0]{{\ddot{\tau}}}
\newcommand{\thetaddot}[0]{{\ddot{\theta}}}
\newcommand{\upsilonddot}[0]{{\ddot{\upsilon}}}
\newcommand{\varepsilonddot}[0]{{\ddot{\varepsilon}}}
\newcommand{\varphiddot}[0]{{\ddot{\varphi}}}
\newcommand{\varpiddot}[0]{{\ddot{\varpi}}}
\newcommand{\varrhoddot}[0]{{\ddot{\varrho}}}
\newcommand{\varsigmaddot}[0]{{\ddot{\varsigma}}}
\newcommand{\varthetaddot}[0]{{\ddot{\vartheta}}}
\newcommand{\xiddot}[0]{{\ddot{\xi}}}
\newcommand{\zetaddot}[0]{{\ddot{\zeta}}}

\newcommand{\BDelta}[0]{\boldsymbol{\Delta}}
\newcommand{\BGamma}[0]{\boldsymbol{\Gamma}}
\newcommand{\BLambda}[0]{\boldsymbol{\Lambda}}
\newcommand{\BOmega}[0]{\boldsymbol{\Omega}}
\newcommand{\BPhi}[0]{\boldsymbol{\Phi}}
\newcommand{\BPi}[0]{\boldsymbol{\Pi}}
\newcommand{\BPsi}[0]{\boldsymbol{\Psi}}
\newcommand{\BSigma}[0]{\boldsymbol{\Sigma}}
\newcommand{\BTheta}[0]{\boldsymbol{\Theta}}
\newcommand{\BUpsilon}[0]{\boldsymbol{\Upsilon}}
\newcommand{\BXi}[0]{\boldsymbol{\Xi}}
\newcommand{\Balpha}[0]{\boldsymbol{\alpha}}
\newcommand{\Bbeta}[0]{\boldsymbol{\beta}}
\newcommand{\Bchi}[0]{\boldsymbol{\chi}}
\newcommand{\Bdelta}[0]{\boldsymbol{\delta}}
\newcommand{\Bepsilon}[0]{\boldsymbol{\epsilon}}
\newcommand{\Beta}[0]{\boldsymbol{\eta}}
\newcommand{\Bgamma}[0]{\boldsymbol{\gamma}}
\newcommand{\Bkappa}[0]{\boldsymbol{\kappa}}
\newcommand{\Blambda}[0]{\boldsymbol{\lambda}}
\newcommand{\Bmu}[0]{\boldsymbol{\mu}}
\newcommand{\Bnu}[0]{\boldsymbol{\nu}}
%\newcommand{\Bomega}[0]{\boldsymbol{\omega}}
\newcommand{\Bphi}[0]{\boldsymbol{\phi}}
\newcommand{\Bpi}[0]{\boldsymbol{\pi}}
\newcommand{\Bpsi}[0]{\boldsymbol{\psi}}
\newcommand{\Brho}[0]{\boldsymbol{\rho}}
\newcommand{\Bsigma}[0]{\boldsymbol{\sigma}}
%\newcommand{\Btau}[0]{\boldsymbol{\tau}}
%\newcommand{\Btheta}[0]{\boldsymbol{\theta}}
\newcommand{\Bupsilon}[0]{\boldsymbol{\upsilon}}
\newcommand{\Bvarepsilon}[0]{\boldsymbol{\varepsilon}}
\newcommand{\Bvarphi}[0]{\boldsymbol{\varphi}}
\newcommand{\Bvarpi}[0]{\boldsymbol{\varpi}}
\newcommand{\Bvarrho}[0]{\boldsymbol{\varrho}}
\newcommand{\Bvarsigma}[0]{\boldsymbol{\varsigma}}
\newcommand{\Bvartheta}[0]{\boldsymbol{\vartheta}}
\newcommand{\Bxi}[0]{\boldsymbol{\xi}}
\newcommand{\Bzeta}[0]{\boldsymbol{\zeta}}
%
%</bold and dot greek symbols>
%<infrequent>
%
%\newcommand{\AreaOp}[1]{\AName_{#1}}
%\newcommand{\Babs}[0]{\abs{\BB}}
%\newcommand{\Bcap}[0]{\hat{\BB}}
%\newcommand{\BrPrimeRej}[0]{\rcap(\rcap \wedge \Br')}
%\newcommand{\CA}[0]{\mathcal{A}}
%\newcommand{\Cos}[1]{\cos{\left({#1}\right)}}
%\newcommand{\Det}[1] {\abs{#1}}
%\newcommand{\Dsq}[2] {\frac {\partial^2 {#1}} {\partial {#2}^2}}
%\newcommand{\Exp}[1]{\exp{\left({#1}\right)}}
%\newcommand{\Norm}[1]{\left\lVert{#1}\right\rVert}
%\newcommand{\Sin}[1]{\sin{\left({#1}\right)}}
%\newcommand{\T}[0]{\text{T}}
%\newcommand{\VolumeOp}[1]{\VName_{#1}}
%\newcommand{\agrad}[0]{\Ba \cdot \nabla}
%\newcommand{\alphacap}[0]{\hat{\boldsymbol{\alpha}}}
%\newcommand{\Fcap}[0]{\hat{\BF}}
%\newcommand{\bithree}[0]{{\Bi}_3}
%\newcommand{\bxa}[0]{\Bx\Ba}
%\newcommand{\coordvec}[2]{
%\newcommand{\costheta}[0]{\acap \cdot \xcap}
%\newcommand{\ddt}[1]{\ddot{#1}}
%\newcommand{\ddu}[1] {\frac {d{#1}} {du}}
%\newcommand{\dsqxj}[2] {\frac {\partial^2 {#1}} {\partial {x_{#2}}^2}}
%\newcommand{\dtheta}[1]{\frac{d {#1}}{d \theta}}
%\newcommand{\dt}[1]{\dot{#1}}
%\newcommand{\dt}[1]{\frac{d {#1}}{dt}}
%\newcommand{\dxj}[2] {\frac {\partial {#1}} {\partial {x_{#2}}}}
%\newcommand{\halfPhi}[0]{\frac{\phi}{2}}
%\newcommand{\half}[0]{\inv{2}}
%\newcommand{\inv}[1]{\frac{1}{#1}}
%\newcommand{\laplacian}[0]{\nabla^2}
%\newcommand{\matrixoftx}[3]{
%\newcommand{\nrrp}[0]{\norm{\rcap \wedge \Br'}}
%\newcommand{\oiint}{\bigcirc \hspace{-1.4em} \int \hspace{-.8em} \int}
%\newcommand{\transpose}[1]{{#1}^{\text{T}}}
%\newcommand{\transpose}[1]{{{#1}^{\TextTranspose}}}
%\newcommand{\transpose}[1]{{{#1}^{\text{T}}}}
%\newcommand{\barA}[0]{\bar{A}}
%\newcommand{\qbar}[0]{\bar{q}}
%\newcommand{\qdotbar}[0]{\dot{\bar{q}}}
%
%</infrequent>





%\usepackage[bookmarks=true]{hyperref}

%\usepackage{color,cite,graphicx}
   % use colour in the document, put your citations as [1-4]
   % rather than [1,2,3,4] (it looks nicer, and the extended LaTeX2e
   % graphics package. 
%\usepackage{latexsym,amssymb,epsf} % do not remember if these are
   % needed, but their inclusion can not do any damage


\chapter{fletcher64}
\label{chap:fletcher}
%\author{Peeter Joot \quad peeter.joot@gmail.com }
\date{ March dd, 2009.  fletcher.tex }

%\begin{document}

%\maketitle{}
%\tableofcontents
\section{Fast modulus by one less than power of 2?}

Statement in some code is

``\((x \& 0xFFFFFFFF) + (x \rightshift 32)\) is a faster way to do \(x \Mod 0xFFFFFFFF\), except the max result may be
\(0x1FFFFFFFE\), so need another round of this.''

How can this be justified?

\subsection{Translating the bit ops to math ops}

Suppose we have 

\begin{equation}\label{eqn:fletcher:modPower2}
\begin{aligned}
x = y \times 2^B + r \quad \mbox{where \(r \le 2^{B-1}\)}
\end{aligned}
\end{equation}

then we have

\begin{equation}\label{eqn:fletcher:22}
\begin{aligned}
y &= x \Div 2^{B} = x \rightshift B \\
r &= x \Mod 2^B = x \& 2^{B-1}
\end{aligned}
\end{equation}

So we have
\begin{equation}\label{eqn:fletcher:42}
\begin{aligned}
x 
&= (x \& 2^{B-1}) \times 2^B + x \rightshift B \\
&= (x \Mod 2^B) \times 2^B + x \Div 2^B \\
\end{aligned}
\end{equation}

\subsection{The shift expression}

From \eqnref{eqn:fletcher:modPower2} we have

\begin{equation}\label{eqn:fletcher:62}
\begin{aligned}
x 
&= y \times 2^B + r \\
&= y \times (2^B -1) + y + r \\
\end{aligned}
\end{equation}

Therefore we have

\begin{equation}\label{eqn:fletcher:82}
\begin{aligned}
x \Mod (2^B -1) &= (y + r) \Mod (2^B -1) \\
\end{aligned}
\end{equation}

The net effect is that the original desired modulus calculation has been reduced to one of a lesser numerical order.  In particular, for the
code in question we have \(B=32\), and an unsigned 64-bit value \(x\).  This means the value \(y \le 0xFFFFFFFF\), whereas \(r \le 0xFFFFFFFF\), so the new value
\(y + r \le 0x1FFFFFFFE\) with up to 31 bits knocked off.  A second round of reduction can give a value no more than \(0xFFFFFFFE + 1 = 0xFFFFFFFF\).

The conclusion is that this ``fast modulus'' is not exactly a modulus replacement.  Instead it is either the true modulus (ie: \(x \Mod 0xFFFFFFFF < 0xFFFFFFFF\)), but may for some \(x\) actually be equal to \(0xFFFFFFFF\).  This almost modulus operation was however sufficient in the code in question, since the idea was only to reduce the magnitude to a 32-bit quantity.

\section{Fletcher64}

A 32-bit word extension of the \href{http://en.wikipedia.org/wiki/Fletcher%27s_checksum}{wikipedia fletcher algorithm} is

\begin{equation}\label{eqn:fletcher:102}
\begin{aligned}
&\text{while} ( i < n )  \\
&\quad   S_1 += A_i \\
&\quad   S_2 += S_1 \\
\\
&C = \Mod_{32}(S_2) | \Mod_{32}(S_1)
\end{aligned}
\end{equation}

Where \(\Mod_{32}(x)\) is the fast modulus operation described above.

The summation equivalent to the loop above is
\begin{equation}\label{eqn:fletcher:122}
\begin{aligned}
S_1 &= \sum_{i=0}^{n-1} A_i \\
S_2 &= \sum_{i=0}^{n-1} (n-i)A_i \\
\end{aligned}
\end{equation}

Note that the above loop assumes that the data is constrained enough that no arithmetic overflows occur within the loop iteration.
Strictly speaking a more accurate transcription of the wiki fletcher32 algorithm using 64-bit accumulators is

\begin{equation}\label{eqn:fletcher:142}
\begin{aligned}
&\text{while} ( i < n )  \\
&\quad   S_1 += A_i \\
&\quad   S_2 += S_1 \\
&\quad   S_1 = \Mod_{32}(S_1) \\
&\quad   S_2 = \Mod_{32}(S_2) \\
\\
&C = S_2 | S_1
\end{aligned}
\end{equation}

Optimizations (like the use of the magic number 360 in the wikipedia article) are possible to avoid the modulus reduction in each iteration of the loop.

In particular, the modulus accumulation of residues is equivalent to the modulus of the sums.  Say

\begin{equation}\label{eqn:fletcher:162}
\begin{aligned}
a_1 &= (a_1 \Div m ) m + r_1 \\
a_2 &= (a_2 \Div m ) m + r_2
\end{aligned}
\end{equation}

So we have
\begin{equation}\label{eqn:fletcher:182}
\begin{aligned}
a_1 + a_2 &= (a_1 \Div m + a_2 \Div m ) m + r_1 + r_2 \\
\implies \\
(a_1 + a_2) \Mod m &= (r_1 + r_2) \Mod m \\
\end{aligned}
\end{equation}

Or more generally

\begin{equation}\label{eqn:fletcher:202}
\begin{aligned}
\left(\sum_i a_i \right) \Mod m &= \sum_i (a_i \Mod m) \Mod m \\
\end{aligned}
\end{equation}

\subsection{Max loops before truncation}

With 32K page sizes and 4 byte words we have a max number of iterations for the loop of

\begin{equation}\label{eqn:fletcher:222}
\begin{aligned}
\frac{32 \times 1024 }{4} = 8192 
\end{aligned}
\end{equation}

Assuming the biggest size value of \(M = A_i = 0xFFFFFFFF\) when do our summation variables wrap?

\begin{equation}\label{eqn:fletcher:242}
\begin{aligned}
S_1 &= \sum_{i=0}^{n-1} A_i \le n M \\
S_2 &= \sum_{i=0}^{n-1} (n-i)A_i <= M \sum_{i=0}^{n-1} (n-i) = M n(n-1)/2 < M n^2/2
\end{aligned}
\end{equation}

The overflow point is dominated by the \(S_2\) sum, so if our max accumulator value is \(M_a\) we want

\begin{equation}\label{eqn:fletcher:262}
\begin{aligned}
M_a \ge M n^2/2 \\
\implies \\
\sqrt{\frac{ 2 M_a }{ M}} \ge n \\
\end{aligned}
\end{equation}

For 64-bit max accumulator and 32-bit words we have
\begin{equation}\label{eqn:fletcher:282}
\begin{aligned}
\sqrt{2 \frac{2^{64} - 1}{2^{32} - 1}} = 92 681.9
\end{aligned}
\end{equation}

which is far bigger than \(8192\).

% google calc:
%log( (2^64-1)/8192 + 1)/log( 2 )

The biggest number of bits in the word that we can use without having to worry about carries are as follows

\begin{tabular}{|l|l|l|}
\hline
Page Size & N & bits \\
\hline
4K & 712 & 46 \\
8K & 1524 & 43 \\
16K & 3196 & 41 \\
32K & 6721 & 39 \\
\hline
\end{tabular}


%\section{Scratch notes}
%
%\begin{align*}
%f(x)
%&= (x \& 2^{B-1}) + x \rightshift B \\
%&= (x \Mod 2^B) + x \Div 2^B \\
%&\questionEquals x \Mod (2^B-1)
%\end{align*}

%\bibliographystyle{plainnat}
%\bibliography{myrefs}

%\end{document}


\part{Fourier treatments}
\documentclass{article}

\usepackage{amsmath}
\usepackage{mathpazo}

%
% shorthand for bold symbols, convenient for vectors and matrices
%
\newcommand{\Ba}[0]{\mathbf{a}}
\newcommand{\Bb}[0]{\mathbf{b}}
\newcommand{\Bc}[0]{\mathbf{c}}
\newcommand{\Bd}[0]{\mathbf{d}}
\newcommand{\Be}[0]{\mathbf{e}}
\newcommand{\Bf}[0]{\mathbf{f}}
\newcommand{\Bg}[0]{\mathbf{g}}
\newcommand{\Bh}[0]{\mathbf{h}}
\newcommand{\Bi}[0]{\mathbf{i}}
\newcommand{\Bj}[0]{\mathbf{j}}
\newcommand{\Bk}[0]{\mathbf{k}}
\newcommand{\Bl}[0]{\mathbf{l}}
\newcommand{\Bm}[0]{\mathbf{m}}
\newcommand{\Bn}[0]{\mathbf{n}}
\newcommand{\Bo}[0]{\mathbf{o}}
\newcommand{\Bp}[0]{\mathbf{p}}
\newcommand{\Bq}[0]{\mathbf{q}}
\newcommand{\Br}[0]{\mathbf{r}}
\newcommand{\Bs}[0]{\mathbf{s}}
\newcommand{\Bt}[0]{\mathbf{t}}
\newcommand{\Bu}[0]{\mathbf{u}}
\newcommand{\Bv}[0]{\mathbf{v}}
\newcommand{\Bw}[0]{\mathbf{w}}
\newcommand{\Bx}[0]{\mathbf{x}}
\newcommand{\By}[0]{\mathbf{y}}
\newcommand{\Bz}[0]{\mathbf{z}}
\newcommand{\BA}[0]{\mathbf{A}}
\newcommand{\BB}[0]{\mathbf{B}}
\newcommand{\BC}[0]{\mathbf{C}}
\newcommand{\BD}[0]{\mathbf{D}}
\newcommand{\BE}[0]{\mathbf{E}}
\newcommand{\BF}[0]{\mathbf{F}}
\newcommand{\BG}[0]{\mathbf{G}}
\newcommand{\BH}[0]{\mathbf{H}}
\newcommand{\BI}[0]{\mathbf{I}}
\newcommand{\BJ}[0]{\mathbf{J}}
\newcommand{\BK}[0]{\mathbf{K}}
\newcommand{\BL}[0]{\mathbf{L}}
\newcommand{\BM}[0]{\mathbf{M}}
\newcommand{\BN}[0]{\mathbf{N}}
\newcommand{\BO}[0]{\mathbf{O}}
\newcommand{\BP}[0]{\mathbf{P}}
\newcommand{\BQ}[0]{\mathbf{Q}}
\newcommand{\BR}[0]{\mathbf{R}}
\newcommand{\BS}[0]{\mathbf{S}}
\newcommand{\BT}[0]{\mathbf{T}}
\newcommand{\BU}[0]{\mathbf{U}}
\newcommand{\BV}[0]{\mathbf{V}}
\newcommand{\BW}[0]{\mathbf{W}}
\newcommand{\BX}[0]{\mathbf{X}}
\newcommand{\BY}[0]{\mathbf{Y}}
\newcommand{\BZ}[0]{\mathbf{Z}}

\newcommand{\Bzero}[0]{\mathbf{0}}
\newcommand{\Btheta}[0]{\boldsymbol{\theta}}
\newcommand{\Btau}[0]{\boldsymbol{\tau}}
\newcommand{\Bomega}[0]{\boldsymbol{\omega}}

%
% shorthand for unit vectors
%
\newcommand{\acap}[0]{\hat{\Ba}}
\newcommand{\bcap}[0]{\hat{\Bb}}
\newcommand{\ccap}[0]{\hat{\Bc}}
\newcommand{\dcap}[0]{\hat{\Bd}}
\newcommand{\ecap}[0]{\hat{\Be}}
\newcommand{\fcap}[0]{\hat{\Bf}}
\newcommand{\gcap}[0]{\hat{\Bg}}
\newcommand{\hcap}[0]{\hat{\Bh}}
\newcommand{\icap}[0]{\hat{\Bi}}
\newcommand{\jcap}[0]{\hat{\Bj}}
\newcommand{\kcap}[0]{\hat{\Bk}}
\newcommand{\lcap}[0]{\hat{\Bl}}
\newcommand{\mcap}[0]{\hat{\Bm}}
\newcommand{\ncap}[0]{\hat{\Bn}}
\newcommand{\ocap}[0]{\hat{\Bo}}
\newcommand{\pcap}[0]{\hat{\Bp}}
\newcommand{\qcap}[0]{\hat{\Bq}}
\newcommand{\rcap}[0]{\hat{\Br}}
\newcommand{\scap}[0]{\hat{\Bs}}
\newcommand{\tcap}[0]{\hat{\Bt}}
\newcommand{\ucap}[0]{\hat{\Bu}}
\newcommand{\vcap}[0]{\hat{\Bv}}
\newcommand{\wcap}[0]{\hat{\Bw}}
\newcommand{\xcap}[0]{\hat{\Bx}}
\newcommand{\ycap}[0]{\hat{\By}}
\newcommand{\zcap}[0]{\hat{\Bz}}
\newcommand{\thetacap}[0]{\hat{\Btheta}}

%
% to write R^n and C^n in a distinguishable fashion.  Perhaps change this
% to the double lined characters upon figuring out how to do so.
%
\newcommand{\C}[1]{$\mathbb{C}^{#1}$}
\newcommand{\R}[1]{$\mathbb{R}^{#1}$}

%
% various generally useful helpers
%

% derivative of #1 wrt. #2:
\newcommand{\D}[2] {\frac {d#2} {d#1}}

\newcommand{\inv}[1]{\frac{1}{#1}}
\newcommand{\cross}[0]{\times}

\newcommand{\abs}[1]{\lvert{#1}\rvert}
\newcommand{\norm}[1]{\lVert{#1}\rVert}
\newcommand{\innerprod}[2]{\langle{#1}, {#2}\rangle}
\newcommand{\dotprod}[2]{{#1} \cdot {#2}}
\newcommand{\bdotprod}[2]{\left({#1} \cdot {#2}\right)}
\newcommand{\crossprod}[2]{{#1} \cross {#2}}
\newcommand{\tripleprod}[3]{\dotprod{\left(\crossprod{#1}{#2}\right)}{#3}}

\DeclareMathOperator{\Proj}{Proj}
\DeclareMathOperator{\Span}{span}
\DeclareMathOperator{\Sgn}{sgn}
\DeclareMathOperator{\Area}{Area}
\DeclareMathOperator{\Volume}{Volume}

%
% A few miscellaneous things specific to this document
%
\newcommand{\crossop}[1]{\crossprod{#1}{}}

% R2 vector.
\newcommand{\VectorTwo}[2]{
\begin{bmatrix}
 {#1} \\
 {#2}
\end{bmatrix}
}

\newcommand{\VectorN}[1]{
\begin{bmatrix}
{#1}_1 \\
{#1}_2 \\
\vdots \\
{#1}_N \\
\end{bmatrix}
}

\newcommand{\DETuvij}[4]{
\begin{vmatrix}
 {#1}_{#3} & {#1}_{#4} \\
 {#2}_{#3} & {#2}_{#4}
\end{vmatrix}
}

\newcommand{\DETuvwijk}[6]{
\begin{vmatrix}
 {#1}_{#4} & {#1}_{#5} & {#1}_{#6} \\
 {#2}_{#4} & {#2}_{#5} & {#2}_{#6} \\
 {#3}_{#4} & {#3}_{#5} & {#3}_{#6}
\end{vmatrix}
}

\newcommand{\DETuvwxijkl}[8]{
\begin{vmatrix}
 {#1}_{#5} & {#1}_{#6} & {#1}_{#7} & {#1}_{#8} \\
 {#2}_{#5} & {#2}_{#6} & {#2}_{#7} & {#2}_{#8} \\
 {#3}_{#5} & {#3}_{#6} & {#3}_{#7} & {#3}_{#8} \\
 {#4}_{#5} & {#4}_{#6} & {#4}_{#7} & {#4}_{#8} \\
\end{vmatrix}
}

%\newcommand{\DETuvwxyijklm}[10]{
%\begin{vmatrix}
% {#1}_{#6} & {#1}_{#7} & {#1}_{#8} & {#1}_{#9} & {#1}_{#10} \\
% {#2}_{#6} & {#2}_{#7} & {#2}_{#8} & {#2}_{#9} & {#2}_{#10} \\
% {#3}_{#6} & {#3}_{#7} & {#3}_{#8} & {#3}_{#9} & {#3}_{#10} \\
% {#4}_{#6} & {#4}_{#7} & {#4}_{#8} & {#4}_{#9} & {#4}_{#10} \\
% {#5}_{#6} & {#5}_{#7} & {#5}_{#8} & {#5}_{#9} & {#5}_{#10}
%\end{vmatrix}
%}

% R3 vector.
\newcommand{\VectorThree}[3]{
\begin{bmatrix}
 {#1} \\
 {#2} \\
 {#3}
\end{bmatrix}
}


%<misc>
%
\newcommand{\Abs}[1]{{\left\lvert{#1}\right\rvert}}
\newcommand{\spacegrad}[0]{\boldsymbol{\nabla}}
\newcommand{\grad}[0]{\nabla}
\newcommand{\LL}[0]{\mathcal{L}}

% == \partial_{#1} {#2}
\newcommand{\PD}[2]{\frac{\partial {#2}}{\partial {#1}}}
% inline variant
\newcommand{\PDi}[2]{{\partial {#2}}/{\partial {#1}}}

\newcommand{\PDD}[3]{\frac{\partial^2 {#3}}{\partial {#1}\partial {#2}}}
%\newcommand{\PDd}[2]{\frac{\partial^2 {#2}}{{\partial{#1}}^2}}
\newcommand{\PDsq}[2]{\frac{\partial^2 {#2}}{(\partial {#1})^2}}

\newcommand{\Partial}[2]{\frac{\partial {#1}}{\partial {#2}}}
\DeclareMathOperator{\RejName}{Rej}
\newcommand{\Rej}[2]{\RejName_{#1}\left( {#2} \right)}
\newcommand{\Rm}[1]{\mathbb{R}^{#1}}
\newcommand{\Cm}[1]{\mathbb{C}^{#1}}
\newcommand{\conj}[0]{{*}}

%</misc>

% <grade selection>
%
\newcommand{\gpgrade}[2] {{\left\langle{{#1}}\right\rangle}_{#2}}

\newcommand{\gpgradezero}[1] {\gpgrade{#1}{}}
%\newcommand{\gpscalargrade}[1] {{\left\langle{{#1}}\right\rangle}}
%\newcommand{\gpgradezero}[1] {\gpgrade{#1}{0}}

%\newcommand{\gpgradeone}[1] {{\left\langle{{#1}}\right\rangle}_{1}}
\newcommand{\gpgradeone}[1] {\gpgrade{#1}{1}}

\newcommand{\gpgradetwo}[1] {\gpgrade{#1}{2}}
\newcommand{\gpgradethree}[1] {\gpgrade{#1}{3}}
\newcommand{\gpgradefour}[1] {\gpgrade{#1}{4}}
%
% </grade selection>



\newcommand{\adot}[0]{{\dot{a}}}
\newcommand{\bdot}[0]{{\dot{b}}}
% taken for centered dot:
%\newcommand{\cdot}[0]{{\dot{c}}}
%\newcommand{\ddot}[0]{{\dot{d}}}
\newcommand{\edot}[0]{{\dot{e}}}
\newcommand{\fdot}[0]{{\dot{f}}}
\newcommand{\gdot}[0]{{\dot{g}}}
\newcommand{\hdot}[0]{{\dot{h}}}
\newcommand{\idot}[0]{{\dot{i}}}
\newcommand{\jdot}[0]{{\dot{j}}}
\newcommand{\kdot}[0]{{\dot{k}}}
\newcommand{\ldot}[0]{{\dot{l}}}
\newcommand{\mdot}[0]{{\dot{m}}}
\newcommand{\ndot}[0]{{\dot{n}}}
%\newcommand{\odot}[0]{{\dot{o}}}
\newcommand{\pdot}[0]{{\dot{p}}}
\newcommand{\qdot}[0]{{\dot{q}}}
\newcommand{\rdot}[0]{{\dot{r}}}
\newcommand{\sdot}[0]{{\dot{s}}}
\newcommand{\tdot}[0]{{\dot{t}}}
\newcommand{\udot}[0]{{\dot{u}}}
\newcommand{\vdot}[0]{{\dot{v}}}
\newcommand{\wdot}[0]{{\dot{w}}}
\newcommand{\xdot}[0]{{\dot{x}}}
\newcommand{\ydot}[0]{{\dot{y}}}
\newcommand{\zdot}[0]{{\dot{z}}}
\newcommand{\addot}[0]{{\ddot{a}}}
\newcommand{\bddot}[0]{{\ddot{b}}}
\newcommand{\cddot}[0]{{\ddot{c}}}
%\newcommand{\dddot}[0]{{\ddot{d}}}
\newcommand{\eddot}[0]{{\ddot{e}}}
\newcommand{\fddot}[0]{{\ddot{f}}}
\newcommand{\gddot}[0]{{\ddot{g}}}
\newcommand{\hddot}[0]{{\ddot{h}}}
\newcommand{\iddot}[0]{{\ddot{i}}}
\newcommand{\jddot}[0]{{\ddot{j}}}
\newcommand{\kddot}[0]{{\ddot{k}}}
\newcommand{\lddot}[0]{{\ddot{l}}}
\newcommand{\mddot}[0]{{\ddot{m}}}
\newcommand{\nddot}[0]{{\ddot{n}}}
\newcommand{\oddot}[0]{{\ddot{o}}}
\newcommand{\pddot}[0]{{\ddot{p}}}
\newcommand{\qddot}[0]{{\ddot{q}}}
\newcommand{\rddot}[0]{{\ddot{r}}}
\newcommand{\sddot}[0]{{\ddot{s}}}
\newcommand{\tddot}[0]{{\ddot{t}}}
\newcommand{\uddot}[0]{{\ddot{u}}}
\newcommand{\vddot}[0]{{\ddot{v}}}
\newcommand{\wddot}[0]{{\ddot{w}}}
\newcommand{\xddot}[0]{{\ddot{x}}}
\newcommand{\yddot}[0]{{\ddot{y}}}
\newcommand{\zddot}[0]{{\ddot{z}}}

%<bold and dot greek symbols>
%

\newcommand{\Deltadot}[0]{{\dot{\Delta}}}
\newcommand{\Gammadot}[0]{{\dot{\Gamma}}}
\newcommand{\Lambdadot}[0]{{\dot{\Lambda}}}
\newcommand{\Omegadot}[0]{{\dot{\Omega}}}
\newcommand{\Phidot}[0]{{\dot{\Phi}}}
\newcommand{\Pidot}[0]{{\dot{\Pi}}}
\newcommand{\Psidot}[0]{{\dot{\Psi}}}
\newcommand{\Sigmadot}[0]{{\dot{\Sigma}}}
\newcommand{\Thetadot}[0]{{\dot{\Theta}}}
\newcommand{\Upsilondot}[0]{{\dot{\Upsilon}}}
\newcommand{\Xidot}[0]{{\dot{\Xi}}}
\newcommand{\alphadot}[0]{{\dot{\alpha}}}
\newcommand{\betadot}[0]{{\dot{\beta}}}
\newcommand{\chidot}[0]{{\dot{\chi}}}
\newcommand{\deltadot}[0]{{\dot{\delta}}}
\newcommand{\epsilondot}[0]{{\dot{\epsilon}}}
\newcommand{\etadot}[0]{{\dot{\eta}}}
\newcommand{\gammadot}[0]{{\dot{\gamma}}}
\newcommand{\kappadot}[0]{{\dot{\kappa}}}
\newcommand{\lambdadot}[0]{{\dot{\lambda}}}
\newcommand{\mudot}[0]{{\dot{\mu}}}
\newcommand{\nudot}[0]{{\dot{\nu}}}
\newcommand{\omegadot}[0]{{\dot{\omega}}}
\newcommand{\phidot}[0]{{\dot{\phi}}}
\newcommand{\pidot}[0]{{\dot{\pi}}}
\newcommand{\psidot}[0]{{\dot{\psi}}}
\newcommand{\rhodot}[0]{{\dot{\rho}}}
\newcommand{\sigmadot}[0]{{\dot{\sigma}}}
\newcommand{\taudot}[0]{{\dot{\tau}}}
\newcommand{\thetadot}[0]{{\dot{\theta}}}
\newcommand{\upsilondot}[0]{{\dot{\upsilon}}}
\newcommand{\varepsilondot}[0]{{\dot{\varepsilon}}}
\newcommand{\varphidot}[0]{{\dot{\varphi}}}
\newcommand{\varpidot}[0]{{\dot{\varpi}}}
\newcommand{\varrhodot}[0]{{\dot{\varrho}}}
\newcommand{\varsigmadot}[0]{{\dot{\varsigma}}}
\newcommand{\varthetadot}[0]{{\dot{\vartheta}}}
\newcommand{\xidot}[0]{{\dot{\xi}}}
\newcommand{\zetadot}[0]{{\dot{\zeta}}}

\newcommand{\Deltaddot}[0]{{\ddot{\Delta}}}
\newcommand{\Gammaddot}[0]{{\ddot{\Gamma}}}
\newcommand{\Lambdaddot}[0]{{\ddot{\Lambda}}}
\newcommand{\Omegaddot}[0]{{\ddot{\Omega}}}
\newcommand{\Phiddot}[0]{{\ddot{\Phi}}}
\newcommand{\Piddot}[0]{{\ddot{\Pi}}}
\newcommand{\Psiddot}[0]{{\ddot{\Psi}}}
\newcommand{\Sigmaddot}[0]{{\ddot{\Sigma}}}
\newcommand{\Thetaddot}[0]{{\ddot{\Theta}}}
\newcommand{\Upsilonddot}[0]{{\ddot{\Upsilon}}}
\newcommand{\Xiddot}[0]{{\ddot{\Xi}}}
\newcommand{\alphaddot}[0]{{\ddot{\alpha}}}
\newcommand{\betaddot}[0]{{\ddot{\beta}}}
\newcommand{\chiddot}[0]{{\ddot{\chi}}}
\newcommand{\deltaddot}[0]{{\ddot{\delta}}}
\newcommand{\epsilonddot}[0]{{\ddot{\epsilon}}}
\newcommand{\etaddot}[0]{{\ddot{\eta}}}
\newcommand{\gammaddot}[0]{{\ddot{\gamma}}}
\newcommand{\kappaddot}[0]{{\ddot{\kappa}}}
\newcommand{\lambdaddot}[0]{{\ddot{\lambda}}}
\newcommand{\muddot}[0]{{\ddot{\mu}}}
\newcommand{\nuddot}[0]{{\ddot{\nu}}}
\newcommand{\omegaddot}[0]{{\ddot{\omega}}}
\newcommand{\phiddot}[0]{{\ddot{\phi}}}
\newcommand{\piddot}[0]{{\ddot{\pi}}}
\newcommand{\psiddot}[0]{{\ddot{\psi}}}
\newcommand{\rhoddot}[0]{{\ddot{\rho}}}
\newcommand{\sigmaddot}[0]{{\ddot{\sigma}}}
\newcommand{\tauddot}[0]{{\ddot{\tau}}}
\newcommand{\thetaddot}[0]{{\ddot{\theta}}}
\newcommand{\upsilonddot}[0]{{\ddot{\upsilon}}}
\newcommand{\varepsilonddot}[0]{{\ddot{\varepsilon}}}
\newcommand{\varphiddot}[0]{{\ddot{\varphi}}}
\newcommand{\varpiddot}[0]{{\ddot{\varpi}}}
\newcommand{\varrhoddot}[0]{{\ddot{\varrho}}}
\newcommand{\varsigmaddot}[0]{{\ddot{\varsigma}}}
\newcommand{\varthetaddot}[0]{{\ddot{\vartheta}}}
\newcommand{\xiddot}[0]{{\ddot{\xi}}}
\newcommand{\zetaddot}[0]{{\ddot{\zeta}}}

\newcommand{\BDelta}[0]{\boldsymbol{\Delta}}
\newcommand{\BGamma}[0]{\boldsymbol{\Gamma}}
\newcommand{\BLambda}[0]{\boldsymbol{\Lambda}}
\newcommand{\BOmega}[0]{\boldsymbol{\Omega}}
\newcommand{\BPhi}[0]{\boldsymbol{\Phi}}
\newcommand{\BPi}[0]{\boldsymbol{\Pi}}
\newcommand{\BPsi}[0]{\boldsymbol{\Psi}}
\newcommand{\BSigma}[0]{\boldsymbol{\Sigma}}
\newcommand{\BTheta}[0]{\boldsymbol{\Theta}}
\newcommand{\BUpsilon}[0]{\boldsymbol{\Upsilon}}
\newcommand{\BXi}[0]{\boldsymbol{\Xi}}
\newcommand{\Balpha}[0]{\boldsymbol{\alpha}}
\newcommand{\Bbeta}[0]{\boldsymbol{\beta}}
\newcommand{\Bchi}[0]{\boldsymbol{\chi}}
\newcommand{\Bdelta}[0]{\boldsymbol{\delta}}
\newcommand{\Bepsilon}[0]{\boldsymbol{\epsilon}}
\newcommand{\Beta}[0]{\boldsymbol{\eta}}
\newcommand{\Bgamma}[0]{\boldsymbol{\gamma}}
\newcommand{\Bkappa}[0]{\boldsymbol{\kappa}}
\newcommand{\Blambda}[0]{\boldsymbol{\lambda}}
\newcommand{\Bmu}[0]{\boldsymbol{\mu}}
\newcommand{\Bnu}[0]{\boldsymbol{\nu}}
%\newcommand{\Bomega}[0]{\boldsymbol{\omega}}
\newcommand{\Bphi}[0]{\boldsymbol{\phi}}
\newcommand{\Bpi}[0]{\boldsymbol{\pi}}
\newcommand{\Bpsi}[0]{\boldsymbol{\psi}}
\newcommand{\Brho}[0]{\boldsymbol{\rho}}
\newcommand{\Bsigma}[0]{\boldsymbol{\sigma}}
%\newcommand{\Btau}[0]{\boldsymbol{\tau}}
%\newcommand{\Btheta}[0]{\boldsymbol{\theta}}
\newcommand{\Bupsilon}[0]{\boldsymbol{\upsilon}}
\newcommand{\Bvarepsilon}[0]{\boldsymbol{\varepsilon}}
\newcommand{\Bvarphi}[0]{\boldsymbol{\varphi}}
\newcommand{\Bvarpi}[0]{\boldsymbol{\varpi}}
\newcommand{\Bvarrho}[0]{\boldsymbol{\varrho}}
\newcommand{\Bvarsigma}[0]{\boldsymbol{\varsigma}}
\newcommand{\Bvartheta}[0]{\boldsymbol{\vartheta}}
\newcommand{\Bxi}[0]{\boldsymbol{\xi}}
\newcommand{\Bzeta}[0]{\boldsymbol{\zeta}}
%
%</bold and dot greek symbols>
%<infrequent>
%
%\newcommand{\AreaOp}[1]{\AName_{#1}}
%\newcommand{\Babs}[0]{\abs{\BB}}
%\newcommand{\Bcap}[0]{\hat{\BB}}
%\newcommand{\BrPrimeRej}[0]{\rcap(\rcap \wedge \Br')}
%\newcommand{\CA}[0]{\mathcal{A}}
%\newcommand{\Cos}[1]{\cos{\left({#1}\right)}}
%\newcommand{\Det}[1] {\abs{#1}}
%\newcommand{\Dsq}[2] {\frac {\partial^2 {#1}} {\partial {#2}^2}}
%\newcommand{\Exp}[1]{\exp{\left({#1}\right)}}
%\newcommand{\Norm}[1]{\left\lVert{#1}\right\rVert}
%\newcommand{\Sin}[1]{\sin{\left({#1}\right)}}
%\newcommand{\T}[0]{\text{T}}
%\newcommand{\VolumeOp}[1]{\VName_{#1}}
%\newcommand{\agrad}[0]{\Ba \cdot \nabla}
%\newcommand{\alphacap}[0]{\hat{\boldsymbol{\alpha}}}
%\newcommand{\Fcap}[0]{\hat{\BF}}
%\newcommand{\bithree}[0]{{\Bi}_3}
%\newcommand{\bxa}[0]{\Bx\Ba}
%\newcommand{\coordvec}[2]{
%\newcommand{\costheta}[0]{\acap \cdot \xcap}
%\newcommand{\ddt}[1]{\ddot{#1}}
%\newcommand{\ddu}[1] {\frac {d{#1}} {du}}
%\newcommand{\dsqxj}[2] {\frac {\partial^2 {#1}} {\partial {x_{#2}}^2}}
%\newcommand{\dtheta}[1]{\frac{d {#1}}{d \theta}}
%\newcommand{\dt}[1]{\dot{#1}}
%\newcommand{\dt}[1]{\frac{d {#1}}{dt}}
%\newcommand{\dxj}[2] {\frac {\partial {#1}} {\partial {x_{#2}}}}
%\newcommand{\halfPhi}[0]{\frac{\phi}{2}}
%\newcommand{\half}[0]{\inv{2}}
%\newcommand{\inv}[1]{\frac{1}{#1}}
%\newcommand{\laplacian}[0]{\nabla^2}
%\newcommand{\matrixoftx}[3]{
%\newcommand{\nrrp}[0]{\norm{\rcap \wedge \Br'}}
%\newcommand{\oiint}{\bigcirc \hspace{-1.4em} \int \hspace{-.8em} \int}
%\newcommand{\transpose}[1]{{#1}^{\text{T}}}
%\newcommand{\transpose}[1]{{{#1}^{\TextTranspose}}}
%\newcommand{\transpose}[1]{{{#1}^{\text{T}}}}
%\newcommand{\barA}[0]{\bar{A}}
%\newcommand{\qbar}[0]{\bar{q}}
%\newcommand{\qdotbar}[0]{\dot{\bar{q}}}
%
%</infrequent>





\newcommand{\PDSq}[2]{\frac{\partial^2 {#2}}{\partial {#1}^2}}
\DeclareMathOperator{\sinc}{sinc}
\DeclareMathOperator{\PV}{PV}
\newcommand{\FF}[0]{\mathcal{F}}
\newcommand{\IIinf}[0]{ \int_{-\infty}^\infty }

\usepackage[bookmarks=true]{hyperref}

\usepackage{color,cite,graphicx}
   % use colour in the document, put your citations as [1-4]
   % rather than [1,2,3,4] (it looks nicer, and the extended LaTeX2e
   % graphics package. 
\usepackage{latexsym,amssymb,epsf} % don't remember if these are
   % needed, but their inclusion can't do any damage


\title{ Applications of Fourier distribution theory to some PDEs. }
\author{Peeter Joot \quad peeter.joot@gmail.com }
\date{ March 4, 2009.  Last Revision: $Date: 2009/03/04 22:35:55 $ }

\begin{document}
\maketitle{}
\tableofcontents

\section{ Motivation. }

I recently listened to Prof Brad Osgood's lectures on distributions in Fourier
transform theory and read the associated lecture notes
\cite{osgoodFourier}.  Here is an attempt to apply these ideas to solution
of some of the common PDEs of physics (wave and Poisson equations).

Some of these were tackled recently using ``classical'' Fourier methods
in \cite{PJpoisson}. 
This requires ad-hoc PV associations of integrals with delta and step
functions.  Such solutions do not inspire confidence.  Without
validation of the solutions by substituion back into the generating PDE
or comparison to a known solution one is left wondering
if all the right fudges were actually performed to get the answer.

\section{ Conventions and definitions. }

\subsection{ Transform pairs. }

The form of the Fourier transform pairs used here are
\begin{align*}
\hat{f}(\mathbf{k}) &= \frac{1}{(\sqrt{2\pi})^n} \iiint f(\mathbf{x}) e^{-i \mathbf{k} \cdot \mathbf{x} } d^n x \\
{f}(\mathbf{x}) &= \frac{1}{(\sqrt{2\pi})^n} \iiint \hat{f}(\mathbf{k}) e^{i \mathbf{k} \cdot \mathbf{x} } d^n k \\
\end{align*}

\subsection{ Space of Schwarz functions. }

\section{ 1D first order homogeneous wave equation. }

This is the simplest PDE that I can think of that one should be able 
to apply Fourier techniques to.  We seek solutions $f(x,t)$ of

\begin{align}
\inv{v} \PD{t}{f} - \PD{x}{f} = 0
\end{align}

\subsection{ Classical way. }

\subsection{ Using distributions. }

\bibliographystyle{plainnat}
\bibliography{myrefs}

\end{document}


\part{geometric-algebra}
\chapter{Hyper complex numbers and symplectic structure. }
\label{chap:complex}
\date{November 8, 2008.  complex.tex}

\section{On 4.2 Hermitian Norms and Unitary Groups. }

These are some rather rough notes filling in some details 
on the treatment of \citep{DoranHamiltonian}.

Expanding equation 4.17

\begin{align*}
J &= e_i \wedge f_i \\
a &= u_i e_i + v_i f_i \\
b &= x_i e_i + y_i f_i \\
B &= a \wedge b + (a \cdot J) \wedge (b \cdot J)  \\
\end{align*}

\begin{align*}
a \wedge b
&= (u_i e_i + v_i f_i) \wedge (x_j e_j + y_j f_j) \\
&= 
u_i x_j e_i \wedge e_j 
+ u_i y_j e_i \wedge f_j
+ v_i x_j f_i \wedge e_j
+ v_i y_j f_i \wedge f_j \\
\end{align*}

\begin{align*}
a \cdot J
&=
u_i e_i \cdot ( e_j \wedge f_j )
+ v_i f_i \cdot ( e_j \wedge f_j ) \\
&= u_j f_j - v_j e_j
\end{align*}

Search and replace for $b \cdot J$ gives

\begin{align*}
b \cdot J
&=
x_i e_i \cdot ( e_j \wedge f_j )
+ y_i f_i \cdot ( e_j \wedge f_j ) \\
&= x_j f_j - y_j e_j
\end{align*}

So we have

\begin{align*}
(a \cdot J) \wedge (b \cdot J) 
&= (u_i f_i - v_i e_i) \wedge (x_j f_j - y_j e_j) \\
&=
 u_i x_j f_i \wedge f_j 
-u_i y_j f_i \wedge e_j
- v_i x_j e_i \wedge f_j
+ v_i y_j e_i \wedge e_j
\end{align*}

For
\begin{align*}
a \wedge b + (a \cdot J) \wedge (b \cdot J) 
&= 
 ( u_i y_j - v_i x_j ) (e_i \wedge f_j - f_i \wedge e_j)
+ ( u_i x_j + v_i y_j ) (e_i \wedge e_j + f_i \wedge f_j)
\end{align*}

This shows why the elements were picked as a basis
\begin{align*}
e_i \wedge f_j - f_i \wedge e_j
\end{align*}
\begin{align*}
e_i \wedge e_j + f_i \wedge f_j
\end{align*}

The first of which is a multiple of $J_i = e_i \wedge f_i$ when $i=j$, and the second of which is zero if $i=j$.

\section{5.1 Conservation Theorems and Flows. } 

equation 5.10 is

\begin{align*}
\fdot = \xdot \cdot \grad f = (\grad f \wedge \grad H) \cdot J
\end{align*}

This one isn't obvious to me.  For $\fdot$ we have

\begin{align*}
\fdot = \PD{p_i}{f} \pdot_i +\PD{q_i}{f} \qdot_i + \underbrace{\PD{t}{f}}_{=0}
\end{align*}

compare to 

\begin{align*}
\xdot \cdot \grad f 
&= (\pdot_i e_i + \qdot_i f_i) \cdot (e_j \PD{p_j}{f} + f_j \PD{q_j}{f}) \\
&= \pdot_i \PD{p_i}{f} + \qdot_i \PD{q_i}{f}
\end{align*}

Okay, this part matches the first part of (5.10).  Writing this in terms of the Hamiltonian relation (5.9) $\xdot = \grad H \cdot J$ we have

\begin{align*}
\fdot
&= (\grad H \cdot J) \cdot \grad f \\
&= \grad f \cdot (\grad H \cdot J) \\
\end{align*}

The relation $a \cdot (b \cdot (c \wedge d)) = (a \wedge b) \cdot (c \wedge d)$, 
can be used here to factor out the $J$, we have
\begin{align*}
\fdot
&= \grad f \cdot (\grad H \cdot J) \\
&= (\grad f \wedge \grad H) \cdot J \\
\end{align*}

which completes (5.10).

Also with $f=H$ since H was also specified as having no explicit time dependence, one has

\begin{align*}
\dot{H} &= (\grad H \wedge \grad H) \cdot J = 0 \cdot J = 0
\end{align*}

%\bibliographystyle{plainnat}
%\bibliography{myrefs}

%\end{document}

\include{faraday_lagrangian}
\documentclass{article}      % Specifies the document class

\usepackage{amsmath}
\usepackage{mathpazo}

%
% shorthand for bold symbols, convenient for vectors and matrices
%
\newcommand{\Ba}[0]{\mathbf{a}}
\newcommand{\Bb}[0]{\mathbf{b}}
\newcommand{\Bc}[0]{\mathbf{c}}
\newcommand{\Bd}[0]{\mathbf{d}}
\newcommand{\Be}[0]{\mathbf{e}}
\newcommand{\Bf}[0]{\mathbf{f}}
\newcommand{\Bg}[0]{\mathbf{g}}
\newcommand{\Bh}[0]{\mathbf{h}}
\newcommand{\Bi}[0]{\mathbf{i}}
\newcommand{\Bj}[0]{\mathbf{j}}
\newcommand{\Bk}[0]{\mathbf{k}}
\newcommand{\Bl}[0]{\mathbf{l}}
\newcommand{\Bm}[0]{\mathbf{m}}
\newcommand{\Bn}[0]{\mathbf{n}}
\newcommand{\Bo}[0]{\mathbf{o}}
\newcommand{\Bp}[0]{\mathbf{p}}
\newcommand{\Bq}[0]{\mathbf{q}}
\newcommand{\Br}[0]{\mathbf{r}}
\newcommand{\Bs}[0]{\mathbf{s}}
\newcommand{\Bt}[0]{\mathbf{t}}
\newcommand{\Bu}[0]{\mathbf{u}}
\newcommand{\Bv}[0]{\mathbf{v}}
\newcommand{\Bw}[0]{\mathbf{w}}
\newcommand{\Bx}[0]{\mathbf{x}}
\newcommand{\By}[0]{\mathbf{y}}
\newcommand{\Bz}[0]{\mathbf{z}}
\newcommand{\BA}[0]{\mathbf{A}}
\newcommand{\BB}[0]{\mathbf{B}}
\newcommand{\BC}[0]{\mathbf{C}}
\newcommand{\BD}[0]{\mathbf{D}}
\newcommand{\BE}[0]{\mathbf{E}}
\newcommand{\BF}[0]{\mathbf{F}}
\newcommand{\BG}[0]{\mathbf{G}}
\newcommand{\BH}[0]{\mathbf{H}}
\newcommand{\BI}[0]{\mathbf{I}}
\newcommand{\BJ}[0]{\mathbf{J}}
\newcommand{\BK}[0]{\mathbf{K}}
\newcommand{\BL}[0]{\mathbf{L}}
\newcommand{\BM}[0]{\mathbf{M}}
\newcommand{\BN}[0]{\mathbf{N}}
\newcommand{\BO}[0]{\mathbf{O}}
\newcommand{\BP}[0]{\mathbf{P}}
\newcommand{\BQ}[0]{\mathbf{Q}}
\newcommand{\BR}[0]{\mathbf{R}}
\newcommand{\BS}[0]{\mathbf{S}}
\newcommand{\BT}[0]{\mathbf{T}}
\newcommand{\BU}[0]{\mathbf{U}}
\newcommand{\BV}[0]{\mathbf{V}}
\newcommand{\BW}[0]{\mathbf{W}}
\newcommand{\BX}[0]{\mathbf{X}}
\newcommand{\BY}[0]{\mathbf{Y}}
\newcommand{\BZ}[0]{\mathbf{Z}}

\newcommand{\Bzero}[0]{\mathbf{0}}
\newcommand{\Btheta}[0]{\boldsymbol{\theta}}
\newcommand{\Btau}[0]{\boldsymbol{\tau}}
\newcommand{\Bomega}[0]{\boldsymbol{\omega}}

%
% shorthand for unit vectors
%
\newcommand{\acap}[0]{\hat{\Ba}}
\newcommand{\bcap}[0]{\hat{\Bb}}
\newcommand{\ccap}[0]{\hat{\Bc}}
\newcommand{\dcap}[0]{\hat{\Bd}}
\newcommand{\ecap}[0]{\hat{\Be}}
\newcommand{\fcap}[0]{\hat{\Bf}}
\newcommand{\gcap}[0]{\hat{\Bg}}
\newcommand{\hcap}[0]{\hat{\Bh}}
\newcommand{\icap}[0]{\hat{\Bi}}
\newcommand{\jcap}[0]{\hat{\Bj}}
\newcommand{\kcap}[0]{\hat{\Bk}}
\newcommand{\lcap}[0]{\hat{\Bl}}
\newcommand{\mcap}[0]{\hat{\Bm}}
\newcommand{\ncap}[0]{\hat{\Bn}}
\newcommand{\ocap}[0]{\hat{\Bo}}
\newcommand{\pcap}[0]{\hat{\Bp}}
\newcommand{\qcap}[0]{\hat{\Bq}}
\newcommand{\rcap}[0]{\hat{\Br}}
\newcommand{\scap}[0]{\hat{\Bs}}
\newcommand{\tcap}[0]{\hat{\Bt}}
\newcommand{\ucap}[0]{\hat{\Bu}}
\newcommand{\vcap}[0]{\hat{\Bv}}
\newcommand{\wcap}[0]{\hat{\Bw}}
\newcommand{\xcap}[0]{\hat{\Bx}}
\newcommand{\ycap}[0]{\hat{\By}}
\newcommand{\zcap}[0]{\hat{\Bz}}
\newcommand{\thetacap}[0]{\hat{\Btheta}}

%
% to write R^n and C^n in a distinguishable fashion.  Perhaps change this
% to the double lined characters upon figuring out how to do so.
%
\newcommand{\C}[1]{$\mathbb{C}^{#1}$}
\newcommand{\R}[1]{$\mathbb{R}^{#1}$}

%
% various generally useful helpers
%

% derivative of #1 wrt. #2:
\newcommand{\D}[2] {\frac {d#2} {d#1}}

\newcommand{\inv}[1]{\frac{1}{#1}}
\newcommand{\cross}[0]{\times}

\newcommand{\abs}[1]{\lvert{#1}\rvert}
\newcommand{\norm}[1]{\lVert{#1}\rVert}
\newcommand{\innerprod}[2]{\langle{#1}, {#2}\rangle}
\newcommand{\dotprod}[2]{{#1} \cdot {#2}}
\newcommand{\bdotprod}[2]{\left({#1} \cdot {#2}\right)}
\newcommand{\crossprod}[2]{{#1} \cross {#2}}
\newcommand{\tripleprod}[3]{\dotprod{\left(\crossprod{#1}{#2}\right)}{#3}}

\DeclareMathOperator{\Proj}{Proj}
\DeclareMathOperator{\Span}{span}
\DeclareMathOperator{\Sgn}{sgn}
\DeclareMathOperator{\Area}{Area}
\DeclareMathOperator{\Volume}{Volume}

%
% A few miscellaneous things specific to this document
%
\newcommand{\crossop}[1]{\crossprod{#1}{}}

% R2 vector.
\newcommand{\VectorTwo}[2]{
\begin{bmatrix}
 {#1} \\
 {#2}
\end{bmatrix}
}

\newcommand{\VectorN}[1]{
\begin{bmatrix}
{#1}_1 \\
{#1}_2 \\
\vdots \\
{#1}_N \\
\end{bmatrix}
}

\newcommand{\DETuvij}[4]{
\begin{vmatrix}
 {#1}_{#3} & {#1}_{#4} \\
 {#2}_{#3} & {#2}_{#4}
\end{vmatrix}
}

\newcommand{\DETuvwijk}[6]{
\begin{vmatrix}
 {#1}_{#4} & {#1}_{#5} & {#1}_{#6} \\
 {#2}_{#4} & {#2}_{#5} & {#2}_{#6} \\
 {#3}_{#4} & {#3}_{#5} & {#3}_{#6}
\end{vmatrix}
}

\newcommand{\DETuvwxijkl}[8]{
\begin{vmatrix}
 {#1}_{#5} & {#1}_{#6} & {#1}_{#7} & {#1}_{#8} \\
 {#2}_{#5} & {#2}_{#6} & {#2}_{#7} & {#2}_{#8} \\
 {#3}_{#5} & {#3}_{#6} & {#3}_{#7} & {#3}_{#8} \\
 {#4}_{#5} & {#4}_{#6} & {#4}_{#7} & {#4}_{#8} \\
\end{vmatrix}
}

%\newcommand{\DETuvwxyijklm}[10]{
%\begin{vmatrix}
% {#1}_{#6} & {#1}_{#7} & {#1}_{#8} & {#1}_{#9} & {#1}_{#10} \\
% {#2}_{#6} & {#2}_{#7} & {#2}_{#8} & {#2}_{#9} & {#2}_{#10} \\
% {#3}_{#6} & {#3}_{#7} & {#3}_{#8} & {#3}_{#9} & {#3}_{#10} \\
% {#4}_{#6} & {#4}_{#7} & {#4}_{#8} & {#4}_{#9} & {#4}_{#10} \\
% {#5}_{#6} & {#5}_{#7} & {#5}_{#8} & {#5}_{#9} & {#5}_{#10}
%\end{vmatrix}
%}

% R3 vector.
\newcommand{\VectorThree}[3]{
\begin{bmatrix}
 {#1} \\
 {#2} \\
 {#3}
\end{bmatrix}
}



%
% The real thing:
%

                             % The preamble begins here.
\title{} % Declares the document's title.
\author{Peeter Joot}         % Declares the author's name.
%\date{}        % Deleting this command produces today's date.

\begin{document}             % End of preamble and beginning of text.

\maketitle{}

\section{}

I was summarizing for myself the various four-vectors of mechanics:

\begin{align*}
x &= ct + \mathbf{x} \\
V &= \frac{dx}{d\tau} = \gamma(c + \mathbf{v}) \\
P &= m V = E/c + \gamma\mathbf{p} \\
f &= m\frac{d^2 x}{d\tau^2} = m\frac{d V}{d\tau} \\
\end{align*}

where:

\begin{align*}
\gamma^{-2} &= 1 - {\lvert \mathbf{v}/c \rvert}^2 \\
d\tau &= {\left(\frac{dx}{d\lambda} \cdot \frac{dx}{d\lambda}\right)}^{1/2} d\lambda \\
x \cdot x = {\lvert x \rvert}^2 &= c^2t^2 - {\lvert \mathbf{x} \rvert}^2 \\
E &= \int f \cdot (c d\tau) \\
\mathbf{v} &= \frac{d\mathbf{x}}{dt} \\
\mathbf{p} &= m\mathbf{v} \\
\end{align*}

Invarients for the first three four vectors are:

\begin{align*}
{\lvert x \rvert}^2 &= c^2 t^2 - {\lvert \mathbf{x} \rvert}^2 = c^2 \tau^2 \\
{\lvert V \rvert}^2 &= \gamma^2 (c^2 - {\lvert \mathbf{v} \rvert}^2) = c^2 \\
{\lvert P \rvert}^2 &= m^2 {\lvert V \rvert}^2 = m^2 c^2 \\
\end{align*}

Is the minkowski norm of the four vector force:

\[
f = m\frac{d^2 x}{d\tau^2} 
\]

also an invarient?  I think it has to be.  Assuming that is the case, what would the value (and significance if any) of this be?

\end{document}               % End of document.

\include{gacs_q8_8}
%
% Copyright � 2012 Peeter Joot.  All Rights Reserved.
% Licenced as described in the file LICENSE under the root directory of this GIT repository.
%

% 
% 
%\documentclass{article}

%\usepackage{amsmath}
\usepackage{mathpazo}

%
% shorthand for bold symbols, convenient for vectors and matrices
%
\newcommand{\Ba}[0]{\mathbf{a}}
\newcommand{\Bb}[0]{\mathbf{b}}
\newcommand{\Bc}[0]{\mathbf{c}}
\newcommand{\Bd}[0]{\mathbf{d}}
\newcommand{\Be}[0]{\mathbf{e}}
\newcommand{\Bf}[0]{\mathbf{f}}
\newcommand{\Bg}[0]{\mathbf{g}}
\newcommand{\Bh}[0]{\mathbf{h}}
\newcommand{\Bi}[0]{\mathbf{i}}
\newcommand{\Bj}[0]{\mathbf{j}}
\newcommand{\Bk}[0]{\mathbf{k}}
\newcommand{\Bl}[0]{\mathbf{l}}
\newcommand{\Bm}[0]{\mathbf{m}}
\newcommand{\Bn}[0]{\mathbf{n}}
\newcommand{\Bo}[0]{\mathbf{o}}
\newcommand{\Bp}[0]{\mathbf{p}}
\newcommand{\Bq}[0]{\mathbf{q}}
\newcommand{\Br}[0]{\mathbf{r}}
\newcommand{\Bs}[0]{\mathbf{s}}
\newcommand{\Bt}[0]{\mathbf{t}}
\newcommand{\Bu}[0]{\mathbf{u}}
\newcommand{\Bv}[0]{\mathbf{v}}
\newcommand{\Bw}[0]{\mathbf{w}}
\newcommand{\Bx}[0]{\mathbf{x}}
\newcommand{\By}[0]{\mathbf{y}}
\newcommand{\Bz}[0]{\mathbf{z}}
\newcommand{\BA}[0]{\mathbf{A}}
\newcommand{\BB}[0]{\mathbf{B}}
\newcommand{\BC}[0]{\mathbf{C}}
\newcommand{\BD}[0]{\mathbf{D}}
\newcommand{\BE}[0]{\mathbf{E}}
\newcommand{\BF}[0]{\mathbf{F}}
\newcommand{\BG}[0]{\mathbf{G}}
\newcommand{\BH}[0]{\mathbf{H}}
\newcommand{\BI}[0]{\mathbf{I}}
\newcommand{\BJ}[0]{\mathbf{J}}
\newcommand{\BK}[0]{\mathbf{K}}
\newcommand{\BL}[0]{\mathbf{L}}
\newcommand{\BM}[0]{\mathbf{M}}
\newcommand{\BN}[0]{\mathbf{N}}
\newcommand{\BO}[0]{\mathbf{O}}
\newcommand{\BP}[0]{\mathbf{P}}
\newcommand{\BQ}[0]{\mathbf{Q}}
\newcommand{\BR}[0]{\mathbf{R}}
\newcommand{\BS}[0]{\mathbf{S}}
\newcommand{\BT}[0]{\mathbf{T}}
\newcommand{\BU}[0]{\mathbf{U}}
\newcommand{\BV}[0]{\mathbf{V}}
\newcommand{\BW}[0]{\mathbf{W}}
\newcommand{\BX}[0]{\mathbf{X}}
\newcommand{\BY}[0]{\mathbf{Y}}
\newcommand{\BZ}[0]{\mathbf{Z}}

\newcommand{\Bzero}[0]{\mathbf{0}}
\newcommand{\Btheta}[0]{\boldsymbol{\theta}}
\newcommand{\Btau}[0]{\boldsymbol{\tau}}
\newcommand{\Bomega}[0]{\boldsymbol{\omega}}

%
% shorthand for unit vectors
%
\newcommand{\acap}[0]{\hat{\Ba}}
\newcommand{\bcap}[0]{\hat{\Bb}}
\newcommand{\ccap}[0]{\hat{\Bc}}
\newcommand{\dcap}[0]{\hat{\Bd}}
\newcommand{\ecap}[0]{\hat{\Be}}
\newcommand{\fcap}[0]{\hat{\Bf}}
\newcommand{\gcap}[0]{\hat{\Bg}}
\newcommand{\hcap}[0]{\hat{\Bh}}
\newcommand{\icap}[0]{\hat{\Bi}}
\newcommand{\jcap}[0]{\hat{\Bj}}
\newcommand{\kcap}[0]{\hat{\Bk}}
\newcommand{\lcap}[0]{\hat{\Bl}}
\newcommand{\mcap}[0]{\hat{\Bm}}
\newcommand{\ncap}[0]{\hat{\Bn}}
\newcommand{\ocap}[0]{\hat{\Bo}}
\newcommand{\pcap}[0]{\hat{\Bp}}
\newcommand{\qcap}[0]{\hat{\Bq}}
\newcommand{\rcap}[0]{\hat{\Br}}
\newcommand{\scap}[0]{\hat{\Bs}}
\newcommand{\tcap}[0]{\hat{\Bt}}
\newcommand{\ucap}[0]{\hat{\Bu}}
\newcommand{\vcap}[0]{\hat{\Bv}}
\newcommand{\wcap}[0]{\hat{\Bw}}
\newcommand{\xcap}[0]{\hat{\Bx}}
\newcommand{\ycap}[0]{\hat{\By}}
\newcommand{\zcap}[0]{\hat{\Bz}}
\newcommand{\thetacap}[0]{\hat{\Btheta}}

%
% to write R^n and C^n in a distinguishable fashion.  Perhaps change this
% to the double lined characters upon figuring out how to do so.
%
\newcommand{\C}[1]{$\mathbb{C}^{#1}$}
\newcommand{\R}[1]{$\mathbb{R}^{#1}$}

%
% various generally useful helpers
%

% derivative of #1 wrt. #2:
\newcommand{\D}[2] {\frac {d#2} {d#1}}

\newcommand{\inv}[1]{\frac{1}{#1}}
\newcommand{\cross}[0]{\times}

\newcommand{\abs}[1]{\lvert{#1}\rvert}
\newcommand{\norm}[1]{\lVert{#1}\rVert}
\newcommand{\innerprod}[2]{\langle{#1}, {#2}\rangle}
\newcommand{\dotprod}[2]{{#1} \cdot {#2}}
\newcommand{\bdotprod}[2]{\left({#1} \cdot {#2}\right)}
\newcommand{\crossprod}[2]{{#1} \cross {#2}}
\newcommand{\tripleprod}[3]{\dotprod{\left(\crossprod{#1}{#2}\right)}{#3}}

\DeclareMathOperator{\Proj}{Proj}
\DeclareMathOperator{\Span}{span}
\DeclareMathOperator{\Sgn}{sgn}
\DeclareMathOperator{\Area}{Area}
\DeclareMathOperator{\Volume}{Volume}

%
% A few miscellaneous things specific to this document
%
\newcommand{\crossop}[1]{\crossprod{#1}{}}

% R2 vector.
\newcommand{\VectorTwo}[2]{
\begin{bmatrix}
 {#1} \\
 {#2}
\end{bmatrix}
}

\newcommand{\VectorN}[1]{
\begin{bmatrix}
{#1}_1 \\
{#1}_2 \\
\vdots \\
{#1}_N \\
\end{bmatrix}
}

\newcommand{\DETuvij}[4]{
\begin{vmatrix}
 {#1}_{#3} & {#1}_{#4} \\
 {#2}_{#3} & {#2}_{#4}
\end{vmatrix}
}

\newcommand{\DETuvwijk}[6]{
\begin{vmatrix}
 {#1}_{#4} & {#1}_{#5} & {#1}_{#6} \\
 {#2}_{#4} & {#2}_{#5} & {#2}_{#6} \\
 {#3}_{#4} & {#3}_{#5} & {#3}_{#6}
\end{vmatrix}
}

\newcommand{\DETuvwxijkl}[8]{
\begin{vmatrix}
 {#1}_{#5} & {#1}_{#6} & {#1}_{#7} & {#1}_{#8} \\
 {#2}_{#5} & {#2}_{#6} & {#2}_{#7} & {#2}_{#8} \\
 {#3}_{#5} & {#3}_{#6} & {#3}_{#7} & {#3}_{#8} \\
 {#4}_{#5} & {#4}_{#6} & {#4}_{#7} & {#4}_{#8} \\
\end{vmatrix}
}

%\newcommand{\DETuvwxyijklm}[10]{
%\begin{vmatrix}
% {#1}_{#6} & {#1}_{#7} & {#1}_{#8} & {#1}_{#9} & {#1}_{#10} \\
% {#2}_{#6} & {#2}_{#7} & {#2}_{#8} & {#2}_{#9} & {#2}_{#10} \\
% {#3}_{#6} & {#3}_{#7} & {#3}_{#8} & {#3}_{#9} & {#3}_{#10} \\
% {#4}_{#6} & {#4}_{#7} & {#4}_{#8} & {#4}_{#9} & {#4}_{#10} \\
% {#5}_{#6} & {#5}_{#7} & {#5}_{#8} & {#5}_{#9} & {#5}_{#10}
%\end{vmatrix}
%}

% R3 vector.
\newcommand{\VectorThree}[3]{
\begin{bmatrix}
 {#1} \\
 {#2} \\
 {#3}
\end{bmatrix}
}


%%<misc>
%
\newcommand{\Abs}[1]{{\left\lvert{#1}\right\rvert}}
\newcommand{\spacegrad}[0]{\boldsymbol{\nabla}}
\newcommand{\grad}[0]{\nabla}
\newcommand{\LL}[0]{\mathcal{L}}

% == \partial_{#1} {#2}
\newcommand{\PD}[2]{\frac{\partial {#2}}{\partial {#1}}}
% inline variant
\newcommand{\PDi}[2]{{\partial {#2}}/{\partial {#1}}}

\newcommand{\PDD}[3]{\frac{\partial^2 {#3}}{\partial {#1}\partial {#2}}}
%\newcommand{\PDd}[2]{\frac{\partial^2 {#2}}{{\partial{#1}}^2}}
\newcommand{\PDsq}[2]{\frac{\partial^2 {#2}}{(\partial {#1})^2}}

\newcommand{\Partial}[2]{\frac{\partial {#1}}{\partial {#2}}}
\DeclareMathOperator{\RejName}{Rej}
\newcommand{\Rej}[2]{\RejName_{#1}\left( {#2} \right)}
\newcommand{\Rm}[1]{\mathbb{R}^{#1}}
\newcommand{\Cm}[1]{\mathbb{C}^{#1}}
\newcommand{\conj}[0]{{*}}

%</misc>

% <grade selection>
%
\newcommand{\gpgrade}[2] {{\left\langle{{#1}}\right\rangle}_{#2}}

\newcommand{\gpgradezero}[1] {\gpgrade{#1}{}}
%\newcommand{\gpscalargrade}[1] {{\left\langle{{#1}}\right\rangle}}
%\newcommand{\gpgradezero}[1] {\gpgrade{#1}{0}}

%\newcommand{\gpgradeone}[1] {{\left\langle{{#1}}\right\rangle}_{1}}
\newcommand{\gpgradeone}[1] {\gpgrade{#1}{1}}

\newcommand{\gpgradetwo}[1] {\gpgrade{#1}{2}}
\newcommand{\gpgradethree}[1] {\gpgrade{#1}{3}}
\newcommand{\gpgradefour}[1] {\gpgrade{#1}{4}}
%
% </grade selection>



\newcommand{\adot}[0]{{\dot{a}}}
\newcommand{\bdot}[0]{{\dot{b}}}
% taken for centered dot:
%\newcommand{\cdot}[0]{{\dot{c}}}
%\newcommand{\ddot}[0]{{\dot{d}}}
\newcommand{\edot}[0]{{\dot{e}}}
\newcommand{\fdot}[0]{{\dot{f}}}
\newcommand{\gdot}[0]{{\dot{g}}}
\newcommand{\hdot}[0]{{\dot{h}}}
\newcommand{\idot}[0]{{\dot{i}}}
\newcommand{\jdot}[0]{{\dot{j}}}
\newcommand{\kdot}[0]{{\dot{k}}}
\newcommand{\ldot}[0]{{\dot{l}}}
\newcommand{\mdot}[0]{{\dot{m}}}
\newcommand{\ndot}[0]{{\dot{n}}}
%\newcommand{\odot}[0]{{\dot{o}}}
\newcommand{\pdot}[0]{{\dot{p}}}
\newcommand{\qdot}[0]{{\dot{q}}}
\newcommand{\rdot}[0]{{\dot{r}}}
\newcommand{\sdot}[0]{{\dot{s}}}
\newcommand{\tdot}[0]{{\dot{t}}}
\newcommand{\udot}[0]{{\dot{u}}}
\newcommand{\vdot}[0]{{\dot{v}}}
\newcommand{\wdot}[0]{{\dot{w}}}
\newcommand{\xdot}[0]{{\dot{x}}}
\newcommand{\ydot}[0]{{\dot{y}}}
\newcommand{\zdot}[0]{{\dot{z}}}
\newcommand{\addot}[0]{{\ddot{a}}}
\newcommand{\bddot}[0]{{\ddot{b}}}
\newcommand{\cddot}[0]{{\ddot{c}}}
%\newcommand{\dddot}[0]{{\ddot{d}}}
\newcommand{\eddot}[0]{{\ddot{e}}}
\newcommand{\fddot}[0]{{\ddot{f}}}
\newcommand{\gddot}[0]{{\ddot{g}}}
\newcommand{\hddot}[0]{{\ddot{h}}}
\newcommand{\iddot}[0]{{\ddot{i}}}
\newcommand{\jddot}[0]{{\ddot{j}}}
\newcommand{\kddot}[0]{{\ddot{k}}}
\newcommand{\lddot}[0]{{\ddot{l}}}
\newcommand{\mddot}[0]{{\ddot{m}}}
\newcommand{\nddot}[0]{{\ddot{n}}}
\newcommand{\oddot}[0]{{\ddot{o}}}
\newcommand{\pddot}[0]{{\ddot{p}}}
\newcommand{\qddot}[0]{{\ddot{q}}}
\newcommand{\rddot}[0]{{\ddot{r}}}
\newcommand{\sddot}[0]{{\ddot{s}}}
\newcommand{\tddot}[0]{{\ddot{t}}}
\newcommand{\uddot}[0]{{\ddot{u}}}
\newcommand{\vddot}[0]{{\ddot{v}}}
\newcommand{\wddot}[0]{{\ddot{w}}}
\newcommand{\xddot}[0]{{\ddot{x}}}
\newcommand{\yddot}[0]{{\ddot{y}}}
\newcommand{\zddot}[0]{{\ddot{z}}}

%<bold and dot greek symbols>
%

\newcommand{\Deltadot}[0]{{\dot{\Delta}}}
\newcommand{\Gammadot}[0]{{\dot{\Gamma}}}
\newcommand{\Lambdadot}[0]{{\dot{\Lambda}}}
\newcommand{\Omegadot}[0]{{\dot{\Omega}}}
\newcommand{\Phidot}[0]{{\dot{\Phi}}}
\newcommand{\Pidot}[0]{{\dot{\Pi}}}
\newcommand{\Psidot}[0]{{\dot{\Psi}}}
\newcommand{\Sigmadot}[0]{{\dot{\Sigma}}}
\newcommand{\Thetadot}[0]{{\dot{\Theta}}}
\newcommand{\Upsilondot}[0]{{\dot{\Upsilon}}}
\newcommand{\Xidot}[0]{{\dot{\Xi}}}
\newcommand{\alphadot}[0]{{\dot{\alpha}}}
\newcommand{\betadot}[0]{{\dot{\beta}}}
\newcommand{\chidot}[0]{{\dot{\chi}}}
\newcommand{\deltadot}[0]{{\dot{\delta}}}
\newcommand{\epsilondot}[0]{{\dot{\epsilon}}}
\newcommand{\etadot}[0]{{\dot{\eta}}}
\newcommand{\gammadot}[0]{{\dot{\gamma}}}
\newcommand{\kappadot}[0]{{\dot{\kappa}}}
\newcommand{\lambdadot}[0]{{\dot{\lambda}}}
\newcommand{\mudot}[0]{{\dot{\mu}}}
\newcommand{\nudot}[0]{{\dot{\nu}}}
\newcommand{\omegadot}[0]{{\dot{\omega}}}
\newcommand{\phidot}[0]{{\dot{\phi}}}
\newcommand{\pidot}[0]{{\dot{\pi}}}
\newcommand{\psidot}[0]{{\dot{\psi}}}
\newcommand{\rhodot}[0]{{\dot{\rho}}}
\newcommand{\sigmadot}[0]{{\dot{\sigma}}}
\newcommand{\taudot}[0]{{\dot{\tau}}}
\newcommand{\thetadot}[0]{{\dot{\theta}}}
\newcommand{\upsilondot}[0]{{\dot{\upsilon}}}
\newcommand{\varepsilondot}[0]{{\dot{\varepsilon}}}
\newcommand{\varphidot}[0]{{\dot{\varphi}}}
\newcommand{\varpidot}[0]{{\dot{\varpi}}}
\newcommand{\varrhodot}[0]{{\dot{\varrho}}}
\newcommand{\varsigmadot}[0]{{\dot{\varsigma}}}
\newcommand{\varthetadot}[0]{{\dot{\vartheta}}}
\newcommand{\xidot}[0]{{\dot{\xi}}}
\newcommand{\zetadot}[0]{{\dot{\zeta}}}

\newcommand{\Deltaddot}[0]{{\ddot{\Delta}}}
\newcommand{\Gammaddot}[0]{{\ddot{\Gamma}}}
\newcommand{\Lambdaddot}[0]{{\ddot{\Lambda}}}
\newcommand{\Omegaddot}[0]{{\ddot{\Omega}}}
\newcommand{\Phiddot}[0]{{\ddot{\Phi}}}
\newcommand{\Piddot}[0]{{\ddot{\Pi}}}
\newcommand{\Psiddot}[0]{{\ddot{\Psi}}}
\newcommand{\Sigmaddot}[0]{{\ddot{\Sigma}}}
\newcommand{\Thetaddot}[0]{{\ddot{\Theta}}}
\newcommand{\Upsilonddot}[0]{{\ddot{\Upsilon}}}
\newcommand{\Xiddot}[0]{{\ddot{\Xi}}}
\newcommand{\alphaddot}[0]{{\ddot{\alpha}}}
\newcommand{\betaddot}[0]{{\ddot{\beta}}}
\newcommand{\chiddot}[0]{{\ddot{\chi}}}
\newcommand{\deltaddot}[0]{{\ddot{\delta}}}
\newcommand{\epsilonddot}[0]{{\ddot{\epsilon}}}
\newcommand{\etaddot}[0]{{\ddot{\eta}}}
\newcommand{\gammaddot}[0]{{\ddot{\gamma}}}
\newcommand{\kappaddot}[0]{{\ddot{\kappa}}}
\newcommand{\lambdaddot}[0]{{\ddot{\lambda}}}
\newcommand{\muddot}[0]{{\ddot{\mu}}}
\newcommand{\nuddot}[0]{{\ddot{\nu}}}
\newcommand{\omegaddot}[0]{{\ddot{\omega}}}
\newcommand{\phiddot}[0]{{\ddot{\phi}}}
\newcommand{\piddot}[0]{{\ddot{\pi}}}
\newcommand{\psiddot}[0]{{\ddot{\psi}}}
\newcommand{\rhoddot}[0]{{\ddot{\rho}}}
\newcommand{\sigmaddot}[0]{{\ddot{\sigma}}}
\newcommand{\tauddot}[0]{{\ddot{\tau}}}
\newcommand{\thetaddot}[0]{{\ddot{\theta}}}
\newcommand{\upsilonddot}[0]{{\ddot{\upsilon}}}
\newcommand{\varepsilonddot}[0]{{\ddot{\varepsilon}}}
\newcommand{\varphiddot}[0]{{\ddot{\varphi}}}
\newcommand{\varpiddot}[0]{{\ddot{\varpi}}}
\newcommand{\varrhoddot}[0]{{\ddot{\varrho}}}
\newcommand{\varsigmaddot}[0]{{\ddot{\varsigma}}}
\newcommand{\varthetaddot}[0]{{\ddot{\vartheta}}}
\newcommand{\xiddot}[0]{{\ddot{\xi}}}
\newcommand{\zetaddot}[0]{{\ddot{\zeta}}}

\newcommand{\BDelta}[0]{\boldsymbol{\Delta}}
\newcommand{\BGamma}[0]{\boldsymbol{\Gamma}}
\newcommand{\BLambda}[0]{\boldsymbol{\Lambda}}
\newcommand{\BOmega}[0]{\boldsymbol{\Omega}}
\newcommand{\BPhi}[0]{\boldsymbol{\Phi}}
\newcommand{\BPi}[0]{\boldsymbol{\Pi}}
\newcommand{\BPsi}[0]{\boldsymbol{\Psi}}
\newcommand{\BSigma}[0]{\boldsymbol{\Sigma}}
\newcommand{\BTheta}[0]{\boldsymbol{\Theta}}
\newcommand{\BUpsilon}[0]{\boldsymbol{\Upsilon}}
\newcommand{\BXi}[0]{\boldsymbol{\Xi}}
\newcommand{\Balpha}[0]{\boldsymbol{\alpha}}
\newcommand{\Bbeta}[0]{\boldsymbol{\beta}}
\newcommand{\Bchi}[0]{\boldsymbol{\chi}}
\newcommand{\Bdelta}[0]{\boldsymbol{\delta}}
\newcommand{\Bepsilon}[0]{\boldsymbol{\epsilon}}
\newcommand{\Beta}[0]{\boldsymbol{\eta}}
\newcommand{\Bgamma}[0]{\boldsymbol{\gamma}}
\newcommand{\Bkappa}[0]{\boldsymbol{\kappa}}
\newcommand{\Blambda}[0]{\boldsymbol{\lambda}}
\newcommand{\Bmu}[0]{\boldsymbol{\mu}}
\newcommand{\Bnu}[0]{\boldsymbol{\nu}}
%\newcommand{\Bomega}[0]{\boldsymbol{\omega}}
\newcommand{\Bphi}[0]{\boldsymbol{\phi}}
\newcommand{\Bpi}[0]{\boldsymbol{\pi}}
\newcommand{\Bpsi}[0]{\boldsymbol{\psi}}
\newcommand{\Brho}[0]{\boldsymbol{\rho}}
\newcommand{\Bsigma}[0]{\boldsymbol{\sigma}}
%\newcommand{\Btau}[0]{\boldsymbol{\tau}}
%\newcommand{\Btheta}[0]{\boldsymbol{\theta}}
\newcommand{\Bupsilon}[0]{\boldsymbol{\upsilon}}
\newcommand{\Bvarepsilon}[0]{\boldsymbol{\varepsilon}}
\newcommand{\Bvarphi}[0]{\boldsymbol{\varphi}}
\newcommand{\Bvarpi}[0]{\boldsymbol{\varpi}}
\newcommand{\Bvarrho}[0]{\boldsymbol{\varrho}}
\newcommand{\Bvarsigma}[0]{\boldsymbol{\varsigma}}
\newcommand{\Bvartheta}[0]{\boldsymbol{\vartheta}}
\newcommand{\Bxi}[0]{\boldsymbol{\xi}}
\newcommand{\Bzeta}[0]{\boldsymbol{\zeta}}
%
%</bold and dot greek symbols>
%<infrequent>
%
%\newcommand{\AreaOp}[1]{\AName_{#1}}
%\newcommand{\Babs}[0]{\abs{\BB}}
%\newcommand{\Bcap}[0]{\hat{\BB}}
%\newcommand{\BrPrimeRej}[0]{\rcap(\rcap \wedge \Br')}
%\newcommand{\CA}[0]{\mathcal{A}}
%\newcommand{\Cos}[1]{\cos{\left({#1}\right)}}
%\newcommand{\Det}[1] {\abs{#1}}
%\newcommand{\Dsq}[2] {\frac {\partial^2 {#1}} {\partial {#2}^2}}
%\newcommand{\Exp}[1]{\exp{\left({#1}\right)}}
%\newcommand{\Norm}[1]{\left\lVert{#1}\right\rVert}
%\newcommand{\Sin}[1]{\sin{\left({#1}\right)}}
%\newcommand{\T}[0]{\text{T}}
%\newcommand{\VolumeOp}[1]{\VName_{#1}}
%\newcommand{\agrad}[0]{\Ba \cdot \nabla}
%\newcommand{\alphacap}[0]{\hat{\boldsymbol{\alpha}}}
%\newcommand{\Fcap}[0]{\hat{\BF}}
%\newcommand{\bithree}[0]{{\Bi}_3}
%\newcommand{\bxa}[0]{\Bx\Ba}
%\newcommand{\coordvec}[2]{
%\newcommand{\costheta}[0]{\acap \cdot \xcap}
%\newcommand{\ddt}[1]{\ddot{#1}}
%\newcommand{\ddu}[1] {\frac {d{#1}} {du}}
%\newcommand{\dsqxj}[2] {\frac {\partial^2 {#1}} {\partial {x_{#2}}^2}}
%\newcommand{\dtheta}[1]{\frac{d {#1}}{d \theta}}
%\newcommand{\dt}[1]{\dot{#1}}
%\newcommand{\dt}[1]{\frac{d {#1}}{dt}}
%\newcommand{\dxj}[2] {\frac {\partial {#1}} {\partial {x_{#2}}}}
%\newcommand{\halfPhi}[0]{\frac{\phi}{2}}
%\newcommand{\half}[0]{\inv{2}}
%\newcommand{\inv}[1]{\frac{1}{#1}}
%\newcommand{\laplacian}[0]{\nabla^2}
%\newcommand{\matrixoftx}[3]{
%\newcommand{\nrrp}[0]{\norm{\rcap \wedge \Br'}}
%\newcommand{\oiint}{\bigcirc \hspace{-1.4em} \int \hspace{-.8em} \int}
%\newcommand{\transpose}[1]{{#1}^{\text{T}}}
%\newcommand{\transpose}[1]{{{#1}^{\TextTranspose}}}
%\newcommand{\transpose}[1]{{{#1}^{\text{T}}}}
%\newcommand{\barA}[0]{\bar{A}}
%\newcommand{\qbar}[0]{\bar{q}}
%\newcommand{\qdotbar}[0]{\dot{\bar{q}}}
%
%</infrequent>





%\usepackage[bookmarks=true]{hyperref}

%\usepackage{color,cite,graphicx}
   % use colour in the document, put your citations as [1-4]
   % rather than [1,2,3,4] (it looks nicer, and the extended LaTeX2e
   % graphics package. 
%\usepackage{latexsym,amssymb,epsf} % don't remember if these are
   % needed, but their inclusion can't do any damage


\chapter{Green's functions}
\label{chap:greens}
%\author{Peeter Joot \quad peeter.joot@gmail.com}
\date{ Dec 19, 2008.  $RCSfile: greens.tex,v $ Last $Revision: 1.10 $ $Date: 2009/08/20 02:24:45 $ }

%\begin{document}
%\maketitle{}
%\tableofcontents

\section{Motivation}

For electromagnetism the vector potential equation to solve is

\begin{align*}
\grad (\grad \wedge A) = \frac{J}{c \epsilon_0}
\end{align*}

Imposing a gauge condition $\grad \cdot A$ on the solutions $A$, we have

\begin{align*}
\grad^2 A = \frac{J}{c \epsilon_0}
\end{align*}

Which gives us our four scalar potential equations

\begin{align*}
\left(\inv{c^2}\PDsq{t}{} - \sum_k \PDsq{x^k}{} \right) A^0 &= \frac{\rho}{\epsilon_0} \\
\left(\inv{c^2}\PDsq{t}{} - \sum_k \PDsq{x^k}{} \right) A^i &= \frac{J^i}{c \epsilon_0} \\
\end{align*}

One form of solution, found for example in \citep{feynman1963flp} for these equations is the retarded time potentials.  For example, with $\phi = A^0$, the retarded solution for $\phi$ at some time $t$, is expressed as an integral over all space

\begin{align*}
\phi(\Bx, t) &= \inv{4 \pi \epsilon_0} \int \frac{\rho(\Bx', t - \Abs{\Bx-\Bx'}/c)}{\Abs{\Bx - \Bx'}} dV'
\end{align*}

The first time I saw this equation, I was struck by the relativistic nature of the solution.  Only the charges with speed of light separation from the point of interest has an effect on the field at that point.

Unfortunately understanding the origin of those solutions, as covered in \citep{FitzRelEandM}, appears to require Green's functions, Fourier transforms and a whole mess of complex seeming stuff.

Intuition tells me that relativistic treatment of the electrostatics potential solution (ie: Lorentz boosting the Coulomb statics potential) can be used to develop the retarded time solutions, and that will be worth a try.  However, it also appears that direct study of the Green's function tools will be worthwhile.

These notes will examine topics related to Green's functions, with the eventual goal of building towards these retarded time potential solutions.

\section{Green's functions}

The basic idea is given a linear differential operator $\LL$, and the Dirac delta function $\delta$, a Green's function solution to the operator equation is

\begin{align*}
\LL G(x,s) = \delta(x-s)
\end{align*}

Using convolution the impulse (delta) function can be used to form arbitrary driving functions, and the general solution, by superposition then only requires an equivalent convolution integration.

Examples of possible differential operators are

\begin{align*}
\LL &= \frac{d}{dx} + a \\
\LL &= a \PDsq{x}{} + b \PDsq{y}{} \\
\LL &= \spacegrad^2 - \inv{v^2}\PDsq{t}{}\\
\LL &= \grad^2 \\
\end{align*}

The last of which is the desired operator for the electromagnetism case.

\section{Solving linear differential equations}

We don't necessarily need Green's functions and transform theory explicitly to solve linear equations.  Lets review some simple cases to see how to directly solve some of these.

\subsection{Simplest case}

The simplest case is a homogeneous linear first order equation of one variable.

\begin{align*}
f' + a f = 0
\end{align*}

The usual exponential characteristic solution method solves this one

\begin{align*}
f &= e^{rx} \\
\implies \\
(r + a) e^{rx} &= 0
\end{align*}

So a homogeneous solution is

\begin{align*}
f = e^{-ax}
\end{align*}

The next level of complexity is the inhomogeneous case, with a constant forcing function

\begin{align*}
f' + a f = b
\end{align*}

This is separable and the solution follows from direct integration

\begin{align*}
\int \frac{df}{b - a f} &= \int dx \\
\inv{-a}\ln(b - af) &= x - \inv{a} ln(-B) \\
\end{align*}

So our complete solution is

\begin{align*}
f &= A e^{-ax} + \inv{a}\left(b + B e^{-a x}\right)
\end{align*}

\subsubsection{With a sampling impulse function}

Now consider a unit area rectangular spike driving function of width $\epsilon$, and height $1/\epsilon$.  Will this be a reasonable approximation of a delta function when $\epsilon$ is let approach zero?

\begin{align*}
b(x) = 
\left\{
\begin{array}{l l}
1/{\epsilon} & \quad \mbox{if $x \in \left[-\epsilon/2, \epsilon/2 \right]$} \\
0              & \quad \mbox{otherwise}
\end{array} \right.
\end{align*}

With a degree of freedom for the homogeneous solution in each of the three ranges, and one degree of freedom for the inhomogeneous part in the impulse interval we have the following general solution

\begin{align*}
f(x) = 
\left\{
\begin{array}{l l}
A_{-} e^{-a(x + \epsilon/2)} & \quad x < -\epsilon/2 \\
%\inv{a \epsilon}
B e^{-ax} + \inv{a}\left(\inv{\epsilon} + C e^{-a x}\right)& \quad x \in \left[-\epsilon/2, \epsilon/2 \right] \\
A_{+} e^{-a(x - \epsilon/2)} & \quad x > \epsilon/2 \\
\end{array} \right.
\end{align*}

%There's actually one degree of freedom too many in the middle term as can be seen by regrouping slightly
%\begin{align*}
%\inv{a \epsilon} B e^{-ax} + \inv{a}\left(\inv{\epsilon} + C e^{-a x}\right) 
%&= \inv{a \epsilon} (B+C) e^{-ax} + \inv{a \epsilon} \\
%&= \inv{a \epsilon} \left( (B + C)e^{-ax} + 1 \right)
%\end{align*}

This isn't necessarily a continuous function, and there is in fact a degree of freedom too many in the middle term above.  Suppose that one wanted this term
to take values $B_{-}$, and $B_{+}$ at the boundaries of the impulse interval, one is then left with the following set of simultaneous equations

\begin{align*}
B_{-} &= \left(B + \inv{a \epsilon } C \right) e^{a \epsilon/2} + \inv{a \epsilon} \\
B_{+} &= \left(B + \inv{a \epsilon } C \right) e^{-a \epsilon/2} + \inv{a \epsilon} \\
\end{align*}

As a vector equation this is
\begin{align*}
a \epsilon
\begin{bmatrix}
B_{-} \\
B_{+} 
\end{bmatrix}
=
\left(a \epsilon B + C \right) 
\begin{bmatrix}
e^{a \epsilon/2} \\
e^{-a \epsilon/2} 
\end{bmatrix}
+
\begin{bmatrix}
1 \\
1
\end{bmatrix}
\end{align*}

and we can wedge with $(1,1)$ to eliminate it, leaving

\begin{align*}
a \epsilon
(B_{-} - B_{+})
=
\left(a \epsilon B + C \right) 
(e^{a \epsilon/2} - e^{-a \epsilon/2})
\end{align*}

So we have

\begin{align*}
B + \inv{a \epsilon}C = \frac{B_{-} - B_{+}}{2 \sinh(a \epsilon/2)}
\end{align*}

and the solution to our equation is therefore

\begin{align*}
f(x) = 
\left\{
\begin{array}{l l}
A_{-} e^{-a(x + \epsilon/2)} & \quad x < -\epsilon/2 \\
\frac{B_{-} - B_{+}}{2 \sinh(a \epsilon/2)} e^{-ax} + \inv{a \epsilon} & \quad x \in \left[-\epsilon/2, \epsilon/2 \right] \\
A_{+} e^{-a(x - \epsilon/2)} & \quad x > \epsilon/2 \\
\end{array} \right.
\end{align*}

However, with the middle term being a function of only the difference between the endpoints, means that we must have an additional constraint on the endpoints.  In general one could make this match the a desired value at only one of the endpoints, and then the other is then determined.  How about picking the middle term so that its value at $x=0$ is the midpoint between $A_{-}$ and $A_{+}$.  If one writes $f = \mu e^{-ax} + 1/a \epsilon$ we have

\begin{align*}
\mu + \inv{a\epsilon} &= \inv{2}\left(A_{-} + A_{+}\right) \\
\implies \\
f &= \inv{2}\left(A_{-} + A_{+}\right) e^{-ax} + \inv{a \epsilon} \left(1 - e^{-a x}\right)
\end{align*}

Now, this was for only the impulse interval.  If however, we have $A = A_{-} = A_{+}$, then writing $u(x)$ for the unit step function, we have for the complete interval the following general solution

\begin{align*}
f &= A e^{-ax} + \inv{2 a \epsilon} \left(1 - e^{-a x}\right)\left( u(\epsilon/2 + x) + u(\epsilon/2 -x)\right)
\end{align*}

This finally has a desirable form, with the sum of a regular old homogeneous solution, plus a specific solution.  The specific solution here also has the role of the Green's function for our delta like impulse function.  Doing a convolution sum with this impulse function has a sampling like action, but it isn't clear how to take the limit of the Green's like function as $\epsilon \rightarrow 0$.

Seeing the delta like function here suggests that the proper Green's function for this operator may in fact be a weighted delta function.  Specifically if we had

\begin{align*}
f' + a f &= \delta(x) \\
\end{align*}

Then is the corresponding general solution in terms of homogeneous solution and a Green's function like so:
\begin{align*}
f &= A e^{-ax} + \inv{a} \left(1 - e^{-a x}\right) \delta(x) \\
\end{align*}

How would one even see if this makes sense (ie: how does one take the derivative of a delta function)?  Probably need the convolution before this would make sense.  Let's try this.

%\bibliographystyle{plainnat}
%\bibliography{myrefs}

%\end{document}

\include{matrix_to_operator}
\include{mp_inverse_svd_rough_notes}
\include{nfcm_ch2}
\include{outermorphism_det}
\documentclass{article}      % Specifies the document class

\usepackage{amsmath}
\usepackage{mathpazo}

%
% shorthand for bold symbols, convenient for vectors and matrices
%
\newcommand{\Ba}[0]{\mathbf{a}}
\newcommand{\Bb}[0]{\mathbf{b}}
\newcommand{\Bc}[0]{\mathbf{c}}
\newcommand{\Bd}[0]{\mathbf{d}}
\newcommand{\Be}[0]{\mathbf{e}}
\newcommand{\Bf}[0]{\mathbf{f}}
\newcommand{\Bg}[0]{\mathbf{g}}
\newcommand{\Bh}[0]{\mathbf{h}}
\newcommand{\Bi}[0]{\mathbf{i}}
\newcommand{\Bj}[0]{\mathbf{j}}
\newcommand{\Bk}[0]{\mathbf{k}}
\newcommand{\Bl}[0]{\mathbf{l}}
\newcommand{\Bm}[0]{\mathbf{m}}
\newcommand{\Bn}[0]{\mathbf{n}}
\newcommand{\Bo}[0]{\mathbf{o}}
\newcommand{\Bp}[0]{\mathbf{p}}
\newcommand{\Bq}[0]{\mathbf{q}}
\newcommand{\Br}[0]{\mathbf{r}}
\newcommand{\Bs}[0]{\mathbf{s}}
\newcommand{\Bt}[0]{\mathbf{t}}
\newcommand{\Bu}[0]{\mathbf{u}}
\newcommand{\Bv}[0]{\mathbf{v}}
\newcommand{\Bw}[0]{\mathbf{w}}
\newcommand{\Bx}[0]{\mathbf{x}}
\newcommand{\By}[0]{\mathbf{y}}
\newcommand{\Bz}[0]{\mathbf{z}}
\newcommand{\BA}[0]{\mathbf{A}}
\newcommand{\BB}[0]{\mathbf{B}}
\newcommand{\BC}[0]{\mathbf{C}}
\newcommand{\BD}[0]{\mathbf{D}}
\newcommand{\BE}[0]{\mathbf{E}}
\newcommand{\BF}[0]{\mathbf{F}}
\newcommand{\BG}[0]{\mathbf{G}}
\newcommand{\BH}[0]{\mathbf{H}}
\newcommand{\BI}[0]{\mathbf{I}}
\newcommand{\BJ}[0]{\mathbf{J}}
\newcommand{\BK}[0]{\mathbf{K}}
\newcommand{\BL}[0]{\mathbf{L}}
\newcommand{\BM}[0]{\mathbf{M}}
\newcommand{\BN}[0]{\mathbf{N}}
\newcommand{\BO}[0]{\mathbf{O}}
\newcommand{\BP}[0]{\mathbf{P}}
\newcommand{\BQ}[0]{\mathbf{Q}}
\newcommand{\BR}[0]{\mathbf{R}}
\newcommand{\BS}[0]{\mathbf{S}}
\newcommand{\BT}[0]{\mathbf{T}}
\newcommand{\BU}[0]{\mathbf{U}}
\newcommand{\BV}[0]{\mathbf{V}}
\newcommand{\BW}[0]{\mathbf{W}}
\newcommand{\BX}[0]{\mathbf{X}}
\newcommand{\BY}[0]{\mathbf{Y}}
\newcommand{\BZ}[0]{\mathbf{Z}}

\newcommand{\Bzero}[0]{\mathbf{0}}
\newcommand{\Btheta}[0]{\boldsymbol{\theta}}
\newcommand{\Btau}[0]{\boldsymbol{\tau}}
\newcommand{\Bomega}[0]{\boldsymbol{\omega}}

%
% shorthand for unit vectors
%
\newcommand{\acap}[0]{\hat{\Ba}}
\newcommand{\bcap}[0]{\hat{\Bb}}
\newcommand{\ccap}[0]{\hat{\Bc}}
\newcommand{\dcap}[0]{\hat{\Bd}}
\newcommand{\ecap}[0]{\hat{\Be}}
\newcommand{\fcap}[0]{\hat{\Bf}}
\newcommand{\gcap}[0]{\hat{\Bg}}
\newcommand{\hcap}[0]{\hat{\Bh}}
\newcommand{\icap}[0]{\hat{\Bi}}
\newcommand{\jcap}[0]{\hat{\Bj}}
\newcommand{\kcap}[0]{\hat{\Bk}}
\newcommand{\lcap}[0]{\hat{\Bl}}
\newcommand{\mcap}[0]{\hat{\Bm}}
\newcommand{\ncap}[0]{\hat{\Bn}}
\newcommand{\ocap}[0]{\hat{\Bo}}
\newcommand{\pcap}[0]{\hat{\Bp}}
\newcommand{\qcap}[0]{\hat{\Bq}}
\newcommand{\rcap}[0]{\hat{\Br}}
\newcommand{\scap}[0]{\hat{\Bs}}
\newcommand{\tcap}[0]{\hat{\Bt}}
\newcommand{\ucap}[0]{\hat{\Bu}}
\newcommand{\vcap}[0]{\hat{\Bv}}
\newcommand{\wcap}[0]{\hat{\Bw}}
\newcommand{\xcap}[0]{\hat{\Bx}}
\newcommand{\ycap}[0]{\hat{\By}}
\newcommand{\zcap}[0]{\hat{\Bz}}
\newcommand{\thetacap}[0]{\hat{\Btheta}}

%
% to write R^n and C^n in a distinguishable fashion.  Perhaps change this
% to the double lined characters upon figuring out how to do so.
%
\newcommand{\C}[1]{$\mathbb{C}^{#1}$}
\newcommand{\R}[1]{$\mathbb{R}^{#1}$}

%
% various generally useful helpers
%

% derivative of #1 wrt. #2:
\newcommand{\D}[2] {\frac {d#2} {d#1}}

\newcommand{\inv}[1]{\frac{1}{#1}}
\newcommand{\cross}[0]{\times}

\newcommand{\abs}[1]{\lvert{#1}\rvert}
\newcommand{\norm}[1]{\lVert{#1}\rVert}
\newcommand{\innerprod}[2]{\langle{#1}, {#2}\rangle}
\newcommand{\dotprod}[2]{{#1} \cdot {#2}}
\newcommand{\bdotprod}[2]{\left({#1} \cdot {#2}\right)}
\newcommand{\crossprod}[2]{{#1} \cross {#2}}
\newcommand{\tripleprod}[3]{\dotprod{\left(\crossprod{#1}{#2}\right)}{#3}}

\DeclareMathOperator{\Proj}{Proj}
\DeclareMathOperator{\Span}{span}
\DeclareMathOperator{\Sgn}{sgn}
\DeclareMathOperator{\Area}{Area}
\DeclareMathOperator{\Volume}{Volume}

%
% A few miscellaneous things specific to this document
%
\newcommand{\crossop}[1]{\crossprod{#1}{}}

% R2 vector.
\newcommand{\VectorTwo}[2]{
\begin{bmatrix}
 {#1} \\
 {#2}
\end{bmatrix}
}

\newcommand{\VectorN}[1]{
\begin{bmatrix}
{#1}_1 \\
{#1}_2 \\
\vdots \\
{#1}_N \\
\end{bmatrix}
}

\newcommand{\DETuvij}[4]{
\begin{vmatrix}
 {#1}_{#3} & {#1}_{#4} \\
 {#2}_{#3} & {#2}_{#4}
\end{vmatrix}
}

\newcommand{\DETuvwijk}[6]{
\begin{vmatrix}
 {#1}_{#4} & {#1}_{#5} & {#1}_{#6} \\
 {#2}_{#4} & {#2}_{#5} & {#2}_{#6} \\
 {#3}_{#4} & {#3}_{#5} & {#3}_{#6}
\end{vmatrix}
}

\newcommand{\DETuvwxijkl}[8]{
\begin{vmatrix}
 {#1}_{#5} & {#1}_{#6} & {#1}_{#7} & {#1}_{#8} \\
 {#2}_{#5} & {#2}_{#6} & {#2}_{#7} & {#2}_{#8} \\
 {#3}_{#5} & {#3}_{#6} & {#3}_{#7} & {#3}_{#8} \\
 {#4}_{#5} & {#4}_{#6} & {#4}_{#7} & {#4}_{#8} \\
\end{vmatrix}
}

%\newcommand{\DETuvwxyijklm}[10]{
%\begin{vmatrix}
% {#1}_{#6} & {#1}_{#7} & {#1}_{#8} & {#1}_{#9} & {#1}_{#10} \\
% {#2}_{#6} & {#2}_{#7} & {#2}_{#8} & {#2}_{#9} & {#2}_{#10} \\
% {#3}_{#6} & {#3}_{#7} & {#3}_{#8} & {#3}_{#9} & {#3}_{#10} \\
% {#4}_{#6} & {#4}_{#7} & {#4}_{#8} & {#4}_{#9} & {#4}_{#10} \\
% {#5}_{#6} & {#5}_{#7} & {#5}_{#8} & {#5}_{#9} & {#5}_{#10}
%\end{vmatrix}
%}

% R3 vector.
\newcommand{\VectorThree}[3]{
\begin{bmatrix}
 {#1} \\
 {#2} \\
 {#3}
\end{bmatrix}
}


\newcommand{\grad}[0]{\nabla}

%
% The real thing:
%

\title{ Potential and Kinetic Energy.} 
\author{Peeter Joot}         
\date{}

\begin{document}             

\maketitle{}

\section{}

Attempting some Lagrangian calculation problems I found I got all the signs of my potential energy terms wrong.  Here's a quick step back to basics to clarify for myself what the definition of potential energy is, and thus implicitly determine the correct signs.

Starting with kinetic energy, expressed in vector form:

\begin{equation*}
K 
= \inv{2} m \Br' \cdot \Br' 
= \inv{2} \Bp \cdot \Br',
\end{equation*}

one can calculate the rate of change of that energy:

\begin{align*}
\frac{dK }{dt}
&= \inv{2} \left(\Bp' \cdot \Br' + \Bp \cdot \Br''\right) \\
&= \inv{2} \left(\Bp' \cdot \Br' + \Br' \cdot \Bp'\right) \\
&= \Bp' \cdot \Br'.
\end{align*}

Note that the mass has been assumed constant above.

Integrating this time rate of change of kinetic energy produces a force
line 
integral:

\begin{align*}
K_2 - K_1 
&= \int_{t1}^{t2} \frac{dK}{dt} dt \\
&= \int_{t1}^{t2} \Bp' \cdot \Br' dt \\
&= \int_{t1}^{t2} \Bp' \cdot \frac{d\Br'}{dt} dt \\
&= \int_{\Br_1}^{\Br_2} \BF \cdot d\Br
\end{align*}

For the path integral to depend on only the end points or the corresponding end times requires a conservative force that can be expressed as a gradient.
Let's say that $\BF = \grad f$, then integrating:

\begin{align*}
K_2 - K_1 
&= \int_{\Br_1}^{\Br_2} \BF \cdot d\Br \\
&= \int_{\Br_1}^{\Br_2} \grad f \cdot d\Br \\
&= {\text{limit}}_{\epsilon \rightarrow 0} \int_{\Br_1}^{\Br_1 + \epsilon\rcap}
   \left(\rcap \frac{f(\Br + \epsilon \rcap)}{\epsilon}\right)
      \cdot d\Br \\
&= \text{handwaving} \\
&= f(\Br_2) - f(\Br_1).
\end{align*}

Assembling the quantities for times $1$, and $2$, we have

\begin{equation}
K_2 -f(\Br_2) = K_1 - f(\Br_1) = \text{constant}.
\end{equation}

This constant is what we give the name Energy.  The quantities $-f(\Br_i)$ we label potential energy $V_i$, and finally write the total energy as the sum of the kinetic and potential energies for a particle at a point in time and space:

\begin{equation}
K_2 + V_2 = K_1 + V_1 = E
\end{equation}
\begin{equation}
\BF = -\grad V
\end{equation}

\subsection{ Work with a specific example.  Newtonian gravitational force.}

Take the gravitional force:

\begin{equation}
F = -\frac{GmM}{r^2} \rcap
\end{equation}

The rate of change of kinetic energy with respect to such a force (FIXME: think though signs ... with or against?), is:

\begin{align*}
\frac{dK}{dt} 
&= \Bp' \cdot \Br' \\
&= -\frac{GmM}{r^2} \rcap \cdot \frac{d\Br}{dt} \\
&= -\frac{GmM}{r^3} \Br \cdot \frac{d\Br}{dt}.
\end{align*}

The vector dot products above can be eliminated with the standard trick:

\begin{align*}
\frac{dr^2}{dt} 
&= \frac{\Br \cdot \Br}{dt} \\
&= 2 \frac{d\Br}{dt} \cdot \Br.
\end{align*}

Thus,
\begin{align*}
\frac{dK}{dt} 
&= -\frac{GmM }{2r^3} \frac{dr^2}{dt} \\
&= -\frac{GmM }{r^2} \frac{dr}{dt} \\
&= \frac{d}{dt} \left( \frac{GmM }{r} \right).
\end{align*}

This can be integrated to find the kinetic energy difference associated with a change of position in a gravitational field:

\begin{align*}
K_2 - K_1 
&= \int_{t_1}^{t_2} \frac{d}{dt} \left( \frac{GmM }{r} \right) dt \\
&= GmM \left( \inv{r_2} - \inv{r_1} \right).
\end{align*}

Or, 

\begin{align*}
K_2 - \frac{GmM}{r_2} = K_1 - \frac{GmM}{r_1} = E.
\end{align*}

Taking gradients of this negative term:

\begin{align*}
\grad \left( - \frac{GmM}{r} \right)
&= \rcap \frac{\partial}{\partial r} \left( - \frac{GmM}{r} \right) \\
&= \rcap \frac{GmM}{r^2},
\end{align*}

returns the negation of the original force, so if we write $V = -Gmm/r^2$, it implies the force is:

\begin{equation}
\BF = -\grad V.
\end{equation}

By this example we see how one arrives at the negative sign convention for the potential energy, and also how we end up with strictly positive terms in the Lagrangian associated with a gravitational force:

\begin{equation}
L = \inv{2} m \Bv^2 + \frac{GmM}{r}.
\end{equation}

\end{document}               

%
% Copyright � 2012 Peeter Joot.  All Rights Reserved.
% Licenced as described in the file LICENSE under the root directory of this GIT repository.
%

% 
% 
%\documentclass{article}      % Specifies the document class

%\usepackage{amsmath}
\usepackage{mathpazo}

%
% shorthand for bold symbols, convenient for vectors and matrices
%
\newcommand{\Ba}[0]{\mathbf{a}}
\newcommand{\Bb}[0]{\mathbf{b}}
\newcommand{\Bc}[0]{\mathbf{c}}
\newcommand{\Bd}[0]{\mathbf{d}}
\newcommand{\Be}[0]{\mathbf{e}}
\newcommand{\Bf}[0]{\mathbf{f}}
\newcommand{\Bg}[0]{\mathbf{g}}
\newcommand{\Bh}[0]{\mathbf{h}}
\newcommand{\Bi}[0]{\mathbf{i}}
\newcommand{\Bj}[0]{\mathbf{j}}
\newcommand{\Bk}[0]{\mathbf{k}}
\newcommand{\Bl}[0]{\mathbf{l}}
\newcommand{\Bm}[0]{\mathbf{m}}
\newcommand{\Bn}[0]{\mathbf{n}}
\newcommand{\Bo}[0]{\mathbf{o}}
\newcommand{\Bp}[0]{\mathbf{p}}
\newcommand{\Bq}[0]{\mathbf{q}}
\newcommand{\Br}[0]{\mathbf{r}}
\newcommand{\Bs}[0]{\mathbf{s}}
\newcommand{\Bt}[0]{\mathbf{t}}
\newcommand{\Bu}[0]{\mathbf{u}}
\newcommand{\Bv}[0]{\mathbf{v}}
\newcommand{\Bw}[0]{\mathbf{w}}
\newcommand{\Bx}[0]{\mathbf{x}}
\newcommand{\By}[0]{\mathbf{y}}
\newcommand{\Bz}[0]{\mathbf{z}}
\newcommand{\BA}[0]{\mathbf{A}}
\newcommand{\BB}[0]{\mathbf{B}}
\newcommand{\BC}[0]{\mathbf{C}}
\newcommand{\BD}[0]{\mathbf{D}}
\newcommand{\BE}[0]{\mathbf{E}}
\newcommand{\BF}[0]{\mathbf{F}}
\newcommand{\BG}[0]{\mathbf{G}}
\newcommand{\BH}[0]{\mathbf{H}}
\newcommand{\BI}[0]{\mathbf{I}}
\newcommand{\BJ}[0]{\mathbf{J}}
\newcommand{\BK}[0]{\mathbf{K}}
\newcommand{\BL}[0]{\mathbf{L}}
\newcommand{\BM}[0]{\mathbf{M}}
\newcommand{\BN}[0]{\mathbf{N}}
\newcommand{\BO}[0]{\mathbf{O}}
\newcommand{\BP}[0]{\mathbf{P}}
\newcommand{\BQ}[0]{\mathbf{Q}}
\newcommand{\BR}[0]{\mathbf{R}}
\newcommand{\BS}[0]{\mathbf{S}}
\newcommand{\BT}[0]{\mathbf{T}}
\newcommand{\BU}[0]{\mathbf{U}}
\newcommand{\BV}[0]{\mathbf{V}}
\newcommand{\BW}[0]{\mathbf{W}}
\newcommand{\BX}[0]{\mathbf{X}}
\newcommand{\BY}[0]{\mathbf{Y}}
\newcommand{\BZ}[0]{\mathbf{Z}}

\newcommand{\Bzero}[0]{\mathbf{0}}
\newcommand{\Btheta}[0]{\boldsymbol{\theta}}
\newcommand{\Btau}[0]{\boldsymbol{\tau}}
\newcommand{\Bomega}[0]{\boldsymbol{\omega}}

%
% shorthand for unit vectors
%
\newcommand{\acap}[0]{\hat{\Ba}}
\newcommand{\bcap}[0]{\hat{\Bb}}
\newcommand{\ccap}[0]{\hat{\Bc}}
\newcommand{\dcap}[0]{\hat{\Bd}}
\newcommand{\ecap}[0]{\hat{\Be}}
\newcommand{\fcap}[0]{\hat{\Bf}}
\newcommand{\gcap}[0]{\hat{\Bg}}
\newcommand{\hcap}[0]{\hat{\Bh}}
\newcommand{\icap}[0]{\hat{\Bi}}
\newcommand{\jcap}[0]{\hat{\Bj}}
\newcommand{\kcap}[0]{\hat{\Bk}}
\newcommand{\lcap}[0]{\hat{\Bl}}
\newcommand{\mcap}[0]{\hat{\Bm}}
\newcommand{\ncap}[0]{\hat{\Bn}}
\newcommand{\ocap}[0]{\hat{\Bo}}
\newcommand{\pcap}[0]{\hat{\Bp}}
\newcommand{\qcap}[0]{\hat{\Bq}}
\newcommand{\rcap}[0]{\hat{\Br}}
\newcommand{\scap}[0]{\hat{\Bs}}
\newcommand{\tcap}[0]{\hat{\Bt}}
\newcommand{\ucap}[0]{\hat{\Bu}}
\newcommand{\vcap}[0]{\hat{\Bv}}
\newcommand{\wcap}[0]{\hat{\Bw}}
\newcommand{\xcap}[0]{\hat{\Bx}}
\newcommand{\ycap}[0]{\hat{\By}}
\newcommand{\zcap}[0]{\hat{\Bz}}
\newcommand{\thetacap}[0]{\hat{\Btheta}}

%
% to write R^n and C^n in a distinguishable fashion.  Perhaps change this
% to the double lined characters upon figuring out how to do so.
%
\newcommand{\C}[1]{$\mathbb{C}^{#1}$}
\newcommand{\R}[1]{$\mathbb{R}^{#1}$}

%
% various generally useful helpers
%

% derivative of #1 wrt. #2:
\newcommand{\D}[2] {\frac {d#2} {d#1}}

\newcommand{\inv}[1]{\frac{1}{#1}}
\newcommand{\cross}[0]{\times}

\newcommand{\abs}[1]{\lvert{#1}\rvert}
\newcommand{\norm}[1]{\lVert{#1}\rVert}
\newcommand{\innerprod}[2]{\langle{#1}, {#2}\rangle}
\newcommand{\dotprod}[2]{{#1} \cdot {#2}}
\newcommand{\bdotprod}[2]{\left({#1} \cdot {#2}\right)}
\newcommand{\crossprod}[2]{{#1} \cross {#2}}
\newcommand{\tripleprod}[3]{\dotprod{\left(\crossprod{#1}{#2}\right)}{#3}}

\DeclareMathOperator{\Proj}{Proj}
\DeclareMathOperator{\Span}{span}
\DeclareMathOperator{\Sgn}{sgn}
\DeclareMathOperator{\Area}{Area}
\DeclareMathOperator{\Volume}{Volume}

%
% A few miscellaneous things specific to this document
%
\newcommand{\crossop}[1]{\crossprod{#1}{}}

% R2 vector.
\newcommand{\VectorTwo}[2]{
\begin{bmatrix}
 {#1} \\
 {#2}
\end{bmatrix}
}

\newcommand{\VectorN}[1]{
\begin{bmatrix}
{#1}_1 \\
{#1}_2 \\
\vdots \\
{#1}_N \\
\end{bmatrix}
}

\newcommand{\DETuvij}[4]{
\begin{vmatrix}
 {#1}_{#3} & {#1}_{#4} \\
 {#2}_{#3} & {#2}_{#4}
\end{vmatrix}
}

\newcommand{\DETuvwijk}[6]{
\begin{vmatrix}
 {#1}_{#4} & {#1}_{#5} & {#1}_{#6} \\
 {#2}_{#4} & {#2}_{#5} & {#2}_{#6} \\
 {#3}_{#4} & {#3}_{#5} & {#3}_{#6}
\end{vmatrix}
}

\newcommand{\DETuvwxijkl}[8]{
\begin{vmatrix}
 {#1}_{#5} & {#1}_{#6} & {#1}_{#7} & {#1}_{#8} \\
 {#2}_{#5} & {#2}_{#6} & {#2}_{#7} & {#2}_{#8} \\
 {#3}_{#5} & {#3}_{#6} & {#3}_{#7} & {#3}_{#8} \\
 {#4}_{#5} & {#4}_{#6} & {#4}_{#7} & {#4}_{#8} \\
\end{vmatrix}
}

%\newcommand{\DETuvwxyijklm}[10]{
%\begin{vmatrix}
% {#1}_{#6} & {#1}_{#7} & {#1}_{#8} & {#1}_{#9} & {#1}_{#10} \\
% {#2}_{#6} & {#2}_{#7} & {#2}_{#8} & {#2}_{#9} & {#2}_{#10} \\
% {#3}_{#6} & {#3}_{#7} & {#3}_{#8} & {#3}_{#9} & {#3}_{#10} \\
% {#4}_{#6} & {#4}_{#7} & {#4}_{#8} & {#4}_{#9} & {#4}_{#10} \\
% {#5}_{#6} & {#5}_{#7} & {#5}_{#8} & {#5}_{#9} & {#5}_{#10}
%\end{vmatrix}
%}

% R3 vector.
\newcommand{\VectorThree}[3]{
\begin{bmatrix}
 {#1} \\
 {#2} \\
 {#3}
\end{bmatrix}
}



%\usepackage{color,cite,graphicx}
   % use colour in the document, put your citations as [1-4]
   % rather than [1,2,3,4] (it looks nicer, and the extended LaTeX2e
   % graphics package. 
%\usepackage{latexsym,amssymb,epsf} % do not remember if these are
   % needed, but their inclusion can not do any damage


%
% The real thing:
%

                             % The preamble begins here.
\chapter{Pythagoras law}
\label{chap:pythagoras}
%\author{Peeter Joot}         % Declares the author's name.
\date{ March 17, 2008.  pythagoras.tex }

%\begin{document}             % End of preamble and beginning of text.

%\maketitle{}

\section{Length}

To base vector multiplication on length, and examine all the consequences of having done so, it is first necessary to 
The geometrical definition of length for vectors generalizes Pythagoras theorem to higher dimensions.

In two dimensions this theorem can be proved with the aid of the following diagram

\imageFigure{../../figures/miscphysics/square_in_square}{Geometrical Proof of Pythagoras Theorem for Right Triangle}{fig:phthagoras}{0.3}

The area of the interior and exterior squares is \(c^2\), and \((a+b)^2\) respectively.  The interior area can also be calculated by subtracting the area of the triangles from the exterior area:

\begin{equation}\label{eqn:pythagoras:20}
(a+b)^2 - 4(ab/2) = a^2 + b^2 + 2ab - 2ab
\end{equation}

Thus proving Pythagoras theorem for the length of the diagonal in a right angle triangle

\begin{equation}\label{eqn:pythagoras:40}
c^2 = a^2 + b^2
\end{equation}

%for a geometrical proof like this one should perhaps show that the inscribed shape is a square ; all the lengths being equal is sufficient IMO.

The length of a vector in three dimensions can be found by repeated application of Pythagoras theorem, as in the following figure 

\imageFigure{../../figures/miscphysics/3d_vector_len}{Length of vector in three dimensions}{fig:3dveclen}{0.3}

The vector \(\Be = \Ba + \Bb + \Bc\), where each of the vectors \(\Ba\), \(\Bb\), and \(\Bc\) are mutually perpendicular can be found by first calculating

\begin{equation}\label{eqn:pythagoras:60}
d^2 = a^2 + b^2
\end{equation}

Then

\begin{equation}\label{eqn:pythagoras:80}
e^2 = d^2 + c^2 = a^2 + b^2 + c^2
\end{equation}

This process can be repeated for any number of higher dimensions.  Having calculated the length of a \(N-1\) dimensional vector

\begin{equation}\label{eqn:pythagoras:100}
{L(\Bv)}^2 = \sum_{i=1}^{N-1} {{l_i}^2}
\end{equation}

Once an additional component of length \(l_N\) is added to that vector in a direction mutually perpendicular to all previous components the new length of this vector becomes

\begin{equation}\label{eqn:pythagoras:120}
\sum_{i=1}^{N-1} {{l_i}^2} + l_N^2 = \sum_{i=0}^{N} {{l_i}^2}
\end{equation}

This is what we mean by the geometrical length of a vector.

%\subsection{Pythagoras Law, and the vector product}
%
%We have a rule for vector multiplication when two vectors are collinear.  Comparison to Pythagoras law will provide an additional rule for vector multiplication when the vectors are completely perpendicular.
%

%\end{document}               % End of document.

%\documentclass{article}

%\usepackage{amsmath}
\usepackage{mathpazo}

%
% shorthand for bold symbols, convenient for vectors and matrices
%
\newcommand{\Ba}[0]{\mathbf{a}}
\newcommand{\Bb}[0]{\mathbf{b}}
\newcommand{\Bc}[0]{\mathbf{c}}
\newcommand{\Bd}[0]{\mathbf{d}}
\newcommand{\Be}[0]{\mathbf{e}}
\newcommand{\Bf}[0]{\mathbf{f}}
\newcommand{\Bg}[0]{\mathbf{g}}
\newcommand{\Bh}[0]{\mathbf{h}}
\newcommand{\Bi}[0]{\mathbf{i}}
\newcommand{\Bj}[0]{\mathbf{j}}
\newcommand{\Bk}[0]{\mathbf{k}}
\newcommand{\Bl}[0]{\mathbf{l}}
\newcommand{\Bm}[0]{\mathbf{m}}
\newcommand{\Bn}[0]{\mathbf{n}}
\newcommand{\Bo}[0]{\mathbf{o}}
\newcommand{\Bp}[0]{\mathbf{p}}
\newcommand{\Bq}[0]{\mathbf{q}}
\newcommand{\Br}[0]{\mathbf{r}}
\newcommand{\Bs}[0]{\mathbf{s}}
\newcommand{\Bt}[0]{\mathbf{t}}
\newcommand{\Bu}[0]{\mathbf{u}}
\newcommand{\Bv}[0]{\mathbf{v}}
\newcommand{\Bw}[0]{\mathbf{w}}
\newcommand{\Bx}[0]{\mathbf{x}}
\newcommand{\By}[0]{\mathbf{y}}
\newcommand{\Bz}[0]{\mathbf{z}}
\newcommand{\BA}[0]{\mathbf{A}}
\newcommand{\BB}[0]{\mathbf{B}}
\newcommand{\BC}[0]{\mathbf{C}}
\newcommand{\BD}[0]{\mathbf{D}}
\newcommand{\BE}[0]{\mathbf{E}}
\newcommand{\BF}[0]{\mathbf{F}}
\newcommand{\BG}[0]{\mathbf{G}}
\newcommand{\BH}[0]{\mathbf{H}}
\newcommand{\BI}[0]{\mathbf{I}}
\newcommand{\BJ}[0]{\mathbf{J}}
\newcommand{\BK}[0]{\mathbf{K}}
\newcommand{\BL}[0]{\mathbf{L}}
\newcommand{\BM}[0]{\mathbf{M}}
\newcommand{\BN}[0]{\mathbf{N}}
\newcommand{\BO}[0]{\mathbf{O}}
\newcommand{\BP}[0]{\mathbf{P}}
\newcommand{\BQ}[0]{\mathbf{Q}}
\newcommand{\BR}[0]{\mathbf{R}}
\newcommand{\BS}[0]{\mathbf{S}}
\newcommand{\BT}[0]{\mathbf{T}}
\newcommand{\BU}[0]{\mathbf{U}}
\newcommand{\BV}[0]{\mathbf{V}}
\newcommand{\BW}[0]{\mathbf{W}}
\newcommand{\BX}[0]{\mathbf{X}}
\newcommand{\BY}[0]{\mathbf{Y}}
\newcommand{\BZ}[0]{\mathbf{Z}}

\newcommand{\Bzero}[0]{\mathbf{0}}
\newcommand{\Btheta}[0]{\boldsymbol{\theta}}
\newcommand{\Btau}[0]{\boldsymbol{\tau}}
\newcommand{\Bomega}[0]{\boldsymbol{\omega}}

%
% shorthand for unit vectors
%
\newcommand{\acap}[0]{\hat{\Ba}}
\newcommand{\bcap}[0]{\hat{\Bb}}
\newcommand{\ccap}[0]{\hat{\Bc}}
\newcommand{\dcap}[0]{\hat{\Bd}}
\newcommand{\ecap}[0]{\hat{\Be}}
\newcommand{\fcap}[0]{\hat{\Bf}}
\newcommand{\gcap}[0]{\hat{\Bg}}
\newcommand{\hcap}[0]{\hat{\Bh}}
\newcommand{\icap}[0]{\hat{\Bi}}
\newcommand{\jcap}[0]{\hat{\Bj}}
\newcommand{\kcap}[0]{\hat{\Bk}}
\newcommand{\lcap}[0]{\hat{\Bl}}
\newcommand{\mcap}[0]{\hat{\Bm}}
\newcommand{\ncap}[0]{\hat{\Bn}}
\newcommand{\ocap}[0]{\hat{\Bo}}
\newcommand{\pcap}[0]{\hat{\Bp}}
\newcommand{\qcap}[0]{\hat{\Bq}}
\newcommand{\rcap}[0]{\hat{\Br}}
\newcommand{\scap}[0]{\hat{\Bs}}
\newcommand{\tcap}[0]{\hat{\Bt}}
\newcommand{\ucap}[0]{\hat{\Bu}}
\newcommand{\vcap}[0]{\hat{\Bv}}
\newcommand{\wcap}[0]{\hat{\Bw}}
\newcommand{\xcap}[0]{\hat{\Bx}}
\newcommand{\ycap}[0]{\hat{\By}}
\newcommand{\zcap}[0]{\hat{\Bz}}
\newcommand{\thetacap}[0]{\hat{\Btheta}}

%
% to write R^n and C^n in a distinguishable fashion.  Perhaps change this
% to the double lined characters upon figuring out how to do so.
%
\newcommand{\C}[1]{$\mathbb{C}^{#1}$}
\newcommand{\R}[1]{$\mathbb{R}^{#1}$}

%
% various generally useful helpers
%

% derivative of #1 wrt. #2:
\newcommand{\D}[2] {\frac {d#2} {d#1}}

\newcommand{\inv}[1]{\frac{1}{#1}}
\newcommand{\cross}[0]{\times}

\newcommand{\abs}[1]{\lvert{#1}\rvert}
\newcommand{\norm}[1]{\lVert{#1}\rVert}
\newcommand{\innerprod}[2]{\langle{#1}, {#2}\rangle}
\newcommand{\dotprod}[2]{{#1} \cdot {#2}}
\newcommand{\bdotprod}[2]{\left({#1} \cdot {#2}\right)}
\newcommand{\crossprod}[2]{{#1} \cross {#2}}
\newcommand{\tripleprod}[3]{\dotprod{\left(\crossprod{#1}{#2}\right)}{#3}}

\DeclareMathOperator{\Proj}{Proj}
\DeclareMathOperator{\Span}{span}
\DeclareMathOperator{\Sgn}{sgn}
\DeclareMathOperator{\Area}{Area}
\DeclareMathOperator{\Volume}{Volume}

%
% A few miscellaneous things specific to this document
%
\newcommand{\crossop}[1]{\crossprod{#1}{}}

% R2 vector.
\newcommand{\VectorTwo}[2]{
\begin{bmatrix}
 {#1} \\
 {#2}
\end{bmatrix}
}

\newcommand{\VectorN}[1]{
\begin{bmatrix}
{#1}_1 \\
{#1}_2 \\
\vdots \\
{#1}_N \\
\end{bmatrix}
}

\newcommand{\DETuvij}[4]{
\begin{vmatrix}
 {#1}_{#3} & {#1}_{#4} \\
 {#2}_{#3} & {#2}_{#4}
\end{vmatrix}
}

\newcommand{\DETuvwijk}[6]{
\begin{vmatrix}
 {#1}_{#4} & {#1}_{#5} & {#1}_{#6} \\
 {#2}_{#4} & {#2}_{#5} & {#2}_{#6} \\
 {#3}_{#4} & {#3}_{#5} & {#3}_{#6}
\end{vmatrix}
}

\newcommand{\DETuvwxijkl}[8]{
\begin{vmatrix}
 {#1}_{#5} & {#1}_{#6} & {#1}_{#7} & {#1}_{#8} \\
 {#2}_{#5} & {#2}_{#6} & {#2}_{#7} & {#2}_{#8} \\
 {#3}_{#5} & {#3}_{#6} & {#3}_{#7} & {#3}_{#8} \\
 {#4}_{#5} & {#4}_{#6} & {#4}_{#7} & {#4}_{#8} \\
\end{vmatrix}
}

%\newcommand{\DETuvwxyijklm}[10]{
%\begin{vmatrix}
% {#1}_{#6} & {#1}_{#7} & {#1}_{#8} & {#1}_{#9} & {#1}_{#10} \\
% {#2}_{#6} & {#2}_{#7} & {#2}_{#8} & {#2}_{#9} & {#2}_{#10} \\
% {#3}_{#6} & {#3}_{#7} & {#3}_{#8} & {#3}_{#9} & {#3}_{#10} \\
% {#4}_{#6} & {#4}_{#7} & {#4}_{#8} & {#4}_{#9} & {#4}_{#10} \\
% {#5}_{#6} & {#5}_{#7} & {#5}_{#8} & {#5}_{#9} & {#5}_{#10}
%\end{vmatrix}
%}

% R3 vector.
\newcommand{\VectorThree}[3]{
\begin{bmatrix}
 {#1} \\
 {#2} \\
 {#3}
\end{bmatrix}
}


%%<misc>
%
\newcommand{\Abs}[1]{{\left\lvert{#1}\right\rvert}}
\newcommand{\spacegrad}[0]{\boldsymbol{\nabla}}
\newcommand{\grad}[0]{\nabla}
\newcommand{\LL}[0]{\mathcal{L}}

% == \partial_{#1} {#2}
\newcommand{\PD}[2]{\frac{\partial {#2}}{\partial {#1}}}
% inline variant
\newcommand{\PDi}[2]{{\partial {#2}}/{\partial {#1}}}

\newcommand{\PDD}[3]{\frac{\partial^2 {#3}}{\partial {#1}\partial {#2}}}
%\newcommand{\PDd}[2]{\frac{\partial^2 {#2}}{{\partial{#1}}^2}}
\newcommand{\PDsq}[2]{\frac{\partial^2 {#2}}{(\partial {#1})^2}}

\newcommand{\Partial}[2]{\frac{\partial {#1}}{\partial {#2}}}
\DeclareMathOperator{\RejName}{Rej}
\newcommand{\Rej}[2]{\RejName_{#1}\left( {#2} \right)}
\newcommand{\Rm}[1]{\mathbb{R}^{#1}}
\newcommand{\Cm}[1]{\mathbb{C}^{#1}}
\newcommand{\conj}[0]{{*}}

%</misc>

% <grade selection>
%
\newcommand{\gpgrade}[2] {{\left\langle{{#1}}\right\rangle}_{#2}}

\newcommand{\gpgradezero}[1] {\gpgrade{#1}{}}
%\newcommand{\gpscalargrade}[1] {{\left\langle{{#1}}\right\rangle}}
%\newcommand{\gpgradezero}[1] {\gpgrade{#1}{0}}

%\newcommand{\gpgradeone}[1] {{\left\langle{{#1}}\right\rangle}_{1}}
\newcommand{\gpgradeone}[1] {\gpgrade{#1}{1}}

\newcommand{\gpgradetwo}[1] {\gpgrade{#1}{2}}
\newcommand{\gpgradethree}[1] {\gpgrade{#1}{3}}
\newcommand{\gpgradefour}[1] {\gpgrade{#1}{4}}
%
% </grade selection>



\newcommand{\adot}[0]{{\dot{a}}}
\newcommand{\bdot}[0]{{\dot{b}}}
% taken for centered dot:
%\newcommand{\cdot}[0]{{\dot{c}}}
%\newcommand{\ddot}[0]{{\dot{d}}}
\newcommand{\edot}[0]{{\dot{e}}}
\newcommand{\fdot}[0]{{\dot{f}}}
\newcommand{\gdot}[0]{{\dot{g}}}
\newcommand{\hdot}[0]{{\dot{h}}}
\newcommand{\idot}[0]{{\dot{i}}}
\newcommand{\jdot}[0]{{\dot{j}}}
\newcommand{\kdot}[0]{{\dot{k}}}
\newcommand{\ldot}[0]{{\dot{l}}}
\newcommand{\mdot}[0]{{\dot{m}}}
\newcommand{\ndot}[0]{{\dot{n}}}
%\newcommand{\odot}[0]{{\dot{o}}}
\newcommand{\pdot}[0]{{\dot{p}}}
\newcommand{\qdot}[0]{{\dot{q}}}
\newcommand{\rdot}[0]{{\dot{r}}}
\newcommand{\sdot}[0]{{\dot{s}}}
\newcommand{\tdot}[0]{{\dot{t}}}
\newcommand{\udot}[0]{{\dot{u}}}
\newcommand{\vdot}[0]{{\dot{v}}}
\newcommand{\wdot}[0]{{\dot{w}}}
\newcommand{\xdot}[0]{{\dot{x}}}
\newcommand{\ydot}[0]{{\dot{y}}}
\newcommand{\zdot}[0]{{\dot{z}}}
\newcommand{\addot}[0]{{\ddot{a}}}
\newcommand{\bddot}[0]{{\ddot{b}}}
\newcommand{\cddot}[0]{{\ddot{c}}}
%\newcommand{\dddot}[0]{{\ddot{d}}}
\newcommand{\eddot}[0]{{\ddot{e}}}
\newcommand{\fddot}[0]{{\ddot{f}}}
\newcommand{\gddot}[0]{{\ddot{g}}}
\newcommand{\hddot}[0]{{\ddot{h}}}
\newcommand{\iddot}[0]{{\ddot{i}}}
\newcommand{\jddot}[0]{{\ddot{j}}}
\newcommand{\kddot}[0]{{\ddot{k}}}
\newcommand{\lddot}[0]{{\ddot{l}}}
\newcommand{\mddot}[0]{{\ddot{m}}}
\newcommand{\nddot}[0]{{\ddot{n}}}
\newcommand{\oddot}[0]{{\ddot{o}}}
\newcommand{\pddot}[0]{{\ddot{p}}}
\newcommand{\qddot}[0]{{\ddot{q}}}
\newcommand{\rddot}[0]{{\ddot{r}}}
\newcommand{\sddot}[0]{{\ddot{s}}}
\newcommand{\tddot}[0]{{\ddot{t}}}
\newcommand{\uddot}[0]{{\ddot{u}}}
\newcommand{\vddot}[0]{{\ddot{v}}}
\newcommand{\wddot}[0]{{\ddot{w}}}
\newcommand{\xddot}[0]{{\ddot{x}}}
\newcommand{\yddot}[0]{{\ddot{y}}}
\newcommand{\zddot}[0]{{\ddot{z}}}

%<bold and dot greek symbols>
%

\newcommand{\Deltadot}[0]{{\dot{\Delta}}}
\newcommand{\Gammadot}[0]{{\dot{\Gamma}}}
\newcommand{\Lambdadot}[0]{{\dot{\Lambda}}}
\newcommand{\Omegadot}[0]{{\dot{\Omega}}}
\newcommand{\Phidot}[0]{{\dot{\Phi}}}
\newcommand{\Pidot}[0]{{\dot{\Pi}}}
\newcommand{\Psidot}[0]{{\dot{\Psi}}}
\newcommand{\Sigmadot}[0]{{\dot{\Sigma}}}
\newcommand{\Thetadot}[0]{{\dot{\Theta}}}
\newcommand{\Upsilondot}[0]{{\dot{\Upsilon}}}
\newcommand{\Xidot}[0]{{\dot{\Xi}}}
\newcommand{\alphadot}[0]{{\dot{\alpha}}}
\newcommand{\betadot}[0]{{\dot{\beta}}}
\newcommand{\chidot}[0]{{\dot{\chi}}}
\newcommand{\deltadot}[0]{{\dot{\delta}}}
\newcommand{\epsilondot}[0]{{\dot{\epsilon}}}
\newcommand{\etadot}[0]{{\dot{\eta}}}
\newcommand{\gammadot}[0]{{\dot{\gamma}}}
\newcommand{\kappadot}[0]{{\dot{\kappa}}}
\newcommand{\lambdadot}[0]{{\dot{\lambda}}}
\newcommand{\mudot}[0]{{\dot{\mu}}}
\newcommand{\nudot}[0]{{\dot{\nu}}}
\newcommand{\omegadot}[0]{{\dot{\omega}}}
\newcommand{\phidot}[0]{{\dot{\phi}}}
\newcommand{\pidot}[0]{{\dot{\pi}}}
\newcommand{\psidot}[0]{{\dot{\psi}}}
\newcommand{\rhodot}[0]{{\dot{\rho}}}
\newcommand{\sigmadot}[0]{{\dot{\sigma}}}
\newcommand{\taudot}[0]{{\dot{\tau}}}
\newcommand{\thetadot}[0]{{\dot{\theta}}}
\newcommand{\upsilondot}[0]{{\dot{\upsilon}}}
\newcommand{\varepsilondot}[0]{{\dot{\varepsilon}}}
\newcommand{\varphidot}[0]{{\dot{\varphi}}}
\newcommand{\varpidot}[0]{{\dot{\varpi}}}
\newcommand{\varrhodot}[0]{{\dot{\varrho}}}
\newcommand{\varsigmadot}[0]{{\dot{\varsigma}}}
\newcommand{\varthetadot}[0]{{\dot{\vartheta}}}
\newcommand{\xidot}[0]{{\dot{\xi}}}
\newcommand{\zetadot}[0]{{\dot{\zeta}}}

\newcommand{\Deltaddot}[0]{{\ddot{\Delta}}}
\newcommand{\Gammaddot}[0]{{\ddot{\Gamma}}}
\newcommand{\Lambdaddot}[0]{{\ddot{\Lambda}}}
\newcommand{\Omegaddot}[0]{{\ddot{\Omega}}}
\newcommand{\Phiddot}[0]{{\ddot{\Phi}}}
\newcommand{\Piddot}[0]{{\ddot{\Pi}}}
\newcommand{\Psiddot}[0]{{\ddot{\Psi}}}
\newcommand{\Sigmaddot}[0]{{\ddot{\Sigma}}}
\newcommand{\Thetaddot}[0]{{\ddot{\Theta}}}
\newcommand{\Upsilonddot}[0]{{\ddot{\Upsilon}}}
\newcommand{\Xiddot}[0]{{\ddot{\Xi}}}
\newcommand{\alphaddot}[0]{{\ddot{\alpha}}}
\newcommand{\betaddot}[0]{{\ddot{\beta}}}
\newcommand{\chiddot}[0]{{\ddot{\chi}}}
\newcommand{\deltaddot}[0]{{\ddot{\delta}}}
\newcommand{\epsilonddot}[0]{{\ddot{\epsilon}}}
\newcommand{\etaddot}[0]{{\ddot{\eta}}}
\newcommand{\gammaddot}[0]{{\ddot{\gamma}}}
\newcommand{\kappaddot}[0]{{\ddot{\kappa}}}
\newcommand{\lambdaddot}[0]{{\ddot{\lambda}}}
\newcommand{\muddot}[0]{{\ddot{\mu}}}
\newcommand{\nuddot}[0]{{\ddot{\nu}}}
\newcommand{\omegaddot}[0]{{\ddot{\omega}}}
\newcommand{\phiddot}[0]{{\ddot{\phi}}}
\newcommand{\piddot}[0]{{\ddot{\pi}}}
\newcommand{\psiddot}[0]{{\ddot{\psi}}}
\newcommand{\rhoddot}[0]{{\ddot{\rho}}}
\newcommand{\sigmaddot}[0]{{\ddot{\sigma}}}
\newcommand{\tauddot}[0]{{\ddot{\tau}}}
\newcommand{\thetaddot}[0]{{\ddot{\theta}}}
\newcommand{\upsilonddot}[0]{{\ddot{\upsilon}}}
\newcommand{\varepsilonddot}[0]{{\ddot{\varepsilon}}}
\newcommand{\varphiddot}[0]{{\ddot{\varphi}}}
\newcommand{\varpiddot}[0]{{\ddot{\varpi}}}
\newcommand{\varrhoddot}[0]{{\ddot{\varrho}}}
\newcommand{\varsigmaddot}[0]{{\ddot{\varsigma}}}
\newcommand{\varthetaddot}[0]{{\ddot{\vartheta}}}
\newcommand{\xiddot}[0]{{\ddot{\xi}}}
\newcommand{\zetaddot}[0]{{\ddot{\zeta}}}

\newcommand{\BDelta}[0]{\boldsymbol{\Delta}}
\newcommand{\BGamma}[0]{\boldsymbol{\Gamma}}
\newcommand{\BLambda}[0]{\boldsymbol{\Lambda}}
\newcommand{\BOmega}[0]{\boldsymbol{\Omega}}
\newcommand{\BPhi}[0]{\boldsymbol{\Phi}}
\newcommand{\BPi}[0]{\boldsymbol{\Pi}}
\newcommand{\BPsi}[0]{\boldsymbol{\Psi}}
\newcommand{\BSigma}[0]{\boldsymbol{\Sigma}}
\newcommand{\BTheta}[0]{\boldsymbol{\Theta}}
\newcommand{\BUpsilon}[0]{\boldsymbol{\Upsilon}}
\newcommand{\BXi}[0]{\boldsymbol{\Xi}}
\newcommand{\Balpha}[0]{\boldsymbol{\alpha}}
\newcommand{\Bbeta}[0]{\boldsymbol{\beta}}
\newcommand{\Bchi}[0]{\boldsymbol{\chi}}
\newcommand{\Bdelta}[0]{\boldsymbol{\delta}}
\newcommand{\Bepsilon}[0]{\boldsymbol{\epsilon}}
\newcommand{\Beta}[0]{\boldsymbol{\eta}}
\newcommand{\Bgamma}[0]{\boldsymbol{\gamma}}
\newcommand{\Bkappa}[0]{\boldsymbol{\kappa}}
\newcommand{\Blambda}[0]{\boldsymbol{\lambda}}
\newcommand{\Bmu}[0]{\boldsymbol{\mu}}
\newcommand{\Bnu}[0]{\boldsymbol{\nu}}
%\newcommand{\Bomega}[0]{\boldsymbol{\omega}}
\newcommand{\Bphi}[0]{\boldsymbol{\phi}}
\newcommand{\Bpi}[0]{\boldsymbol{\pi}}
\newcommand{\Bpsi}[0]{\boldsymbol{\psi}}
\newcommand{\Brho}[0]{\boldsymbol{\rho}}
\newcommand{\Bsigma}[0]{\boldsymbol{\sigma}}
%\newcommand{\Btau}[0]{\boldsymbol{\tau}}
%\newcommand{\Btheta}[0]{\boldsymbol{\theta}}
\newcommand{\Bupsilon}[0]{\boldsymbol{\upsilon}}
\newcommand{\Bvarepsilon}[0]{\boldsymbol{\varepsilon}}
\newcommand{\Bvarphi}[0]{\boldsymbol{\varphi}}
\newcommand{\Bvarpi}[0]{\boldsymbol{\varpi}}
\newcommand{\Bvarrho}[0]{\boldsymbol{\varrho}}
\newcommand{\Bvarsigma}[0]{\boldsymbol{\varsigma}}
\newcommand{\Bvartheta}[0]{\boldsymbol{\vartheta}}
\newcommand{\Bxi}[0]{\boldsymbol{\xi}}
\newcommand{\Bzeta}[0]{\boldsymbol{\zeta}}
%
%</bold and dot greek symbols>
%<infrequent>
%
%\newcommand{\AreaOp}[1]{\AName_{#1}}
%\newcommand{\Babs}[0]{\abs{\BB}}
%\newcommand{\Bcap}[0]{\hat{\BB}}
%\newcommand{\BrPrimeRej}[0]{\rcap(\rcap \wedge \Br')}
%\newcommand{\CA}[0]{\mathcal{A}}
%\newcommand{\Cos}[1]{\cos{\left({#1}\right)}}
%\newcommand{\Det}[1] {\abs{#1}}
%\newcommand{\Dsq}[2] {\frac {\partial^2 {#1}} {\partial {#2}^2}}
%\newcommand{\Exp}[1]{\exp{\left({#1}\right)}}
%\newcommand{\Norm}[1]{\left\lVert{#1}\right\rVert}
%\newcommand{\Sin}[1]{\sin{\left({#1}\right)}}
%\newcommand{\T}[0]{\text{T}}
%\newcommand{\VolumeOp}[1]{\VName_{#1}}
%\newcommand{\agrad}[0]{\Ba \cdot \nabla}
%\newcommand{\alphacap}[0]{\hat{\boldsymbol{\alpha}}}
%\newcommand{\Fcap}[0]{\hat{\BF}}
%\newcommand{\bithree}[0]{{\Bi}_3}
%\newcommand{\bxa}[0]{\Bx\Ba}
%\newcommand{\coordvec}[2]{
%\newcommand{\costheta}[0]{\acap \cdot \xcap}
%\newcommand{\ddt}[1]{\ddot{#1}}
%\newcommand{\ddu}[1] {\frac {d{#1}} {du}}
%\newcommand{\dsqxj}[2] {\frac {\partial^2 {#1}} {\partial {x_{#2}}^2}}
%\newcommand{\dtheta}[1]{\frac{d {#1}}{d \theta}}
%\newcommand{\dt}[1]{\dot{#1}}
%\newcommand{\dt}[1]{\frac{d {#1}}{dt}}
%\newcommand{\dxj}[2] {\frac {\partial {#1}} {\partial {x_{#2}}}}
%\newcommand{\halfPhi}[0]{\frac{\phi}{2}}
%\newcommand{\half}[0]{\inv{2}}
%\newcommand{\inv}[1]{\frac{1}{#1}}
%\newcommand{\laplacian}[0]{\nabla^2}
%\newcommand{\matrixoftx}[3]{
%\newcommand{\nrrp}[0]{\norm{\rcap \wedge \Br'}}
%\newcommand{\oiint}{\bigcirc \hspace{-1.4em} \int \hspace{-.8em} \int}
%\newcommand{\transpose}[1]{{#1}^{\text{T}}}
%\newcommand{\transpose}[1]{{{#1}^{\TextTranspose}}}
%\newcommand{\transpose}[1]{{{#1}^{\text{T}}}}
%\newcommand{\barA}[0]{\bar{A}}
%\newcommand{\qbar}[0]{\bar{q}}
%\newcommand{\qdotbar}[0]{\dot{\bar{q}}}
%
%</infrequent>





%\usepackage[bookmarks=true]{hyperref}

%\usepackage{color,cite,graphicx}
   % use colour in the document, put your citations as [1-4]
   % rather than [1,2,3,4] (it looks nicer, and the extended LaTeX2e
   % graphics package. 
%\usepackage{latexsym,amssymb,epsf} % don't remember if these are
   % needed, but their inclusion can't do any damage


\chapter{Some rapidity angle notes. }
\label{chap:rapidity}
%\author{Peeter Joot \quad peeter.joot@gmail.com}
\date{ Dec 18, 2008.  $RCSfile: rapidity.tex,v $ Last $Revision: 1.8 $ $Date: 2009/06/14 23:51:45 $ }

%\begin{document}

%\maketitle{}
%\tableofcontents

\section{Motivation. }

Lut writes, "setting up a little calculation, I'm writing a 4-velocity as"
 
\begin{align*}
( \gamma, \gamma \beta_x, \gamma \beta_y, \gamma \beta_z ),
\end{align*}
 
which has length $-1$ if $\gamma^{-2} = 1 - (\beta_x)^2+(\beta_y)^2+(\beta_z)^2$.
 
Can you write this in terms of the 3 rapidities $a_1, a_2, a_3$ ?

I wasn't able to answer this right away so it is worth an examination
of rapidity angles to ensure that I understand the ideas.

\section{Stuff. }

Putting back in the $c$ factors, and switching to the $+---$ signature I'm used to, the position
vector is

\begin{align*}
x &= x^\mu \gamma_\mu = ct \gamma_0 + x^i \gamma_i,
\end{align*}

for which the corresponding proper velocity is
\begin{align*}
v &= \frac{dx}{d\tau} = c \frac{dt}{d\tau} \gamma_0 + \frac{dx^i}{dt} \frac{dt}{d\tau} \gamma_i
\end{align*}

Writing $\gamma = dt/d\tau$, and squaring the proper velocity we have

\begin{align*}
\frac{v^2}{c^2}
&= 1 \\
&= \gamma^2 \left(\gamma_0 + \inv{c}\frac{dx^i}{dt} \gamma_i\right)^2 \\
&= \gamma^2 \left(1 - \sum_i \inv{c^2} \left(\frac{dx^i}{dt}\right)^2 \right) \\
\end{align*}

So we have 

\begin{align*}
\gamma 
&= \inv{\sqrt{1 - \sum_i \inv{c^2} \left(\frac{dx^i}{dt}\right)^2 }} \\
\end{align*}

Observe that $\gamma$ ranges from $1$ to infinity, and can thus be described by the $[0,\infty]$ range of the hyperbolic cosine function.  With
the relative velocity $\Bv = \sum_i (dx^i/dt) \sigma_i$, this is

\begin{align*}
\frac{dt}{d\tau} &= \gamma  \\
&= \cosh\alpha \\
&= \inv{\sqrt{1 - (\Bv/c)^2}}
\end{align*}

In terms of the hyperbolic cosine for $\gamma$ our proper velocity then becomes

\begin{align*}
v/c &= \cosh\alpha \left(1 + \frac{\Bv}{c} \right) \gamma_0
\end{align*}

Taking the hint from the Lorentz transform where we have both $\sinh$ and $\cosh$ factors can one write

\begin{align*}
\gamma \frac{\Bv}{c} = \sinh\alpha
\end{align*}

This gives

\begin{align*}
\frac{\Bv}{c} = \tanh\alpha
\end{align*}

so we need $\alpha$ to be a spacetime relative vector.  With $\cosh$ being an even function $\cosh{\alpha} = \cosh{\Abs{\alpha}}$, so this
is still a 
scalar as desired.  Inverting the relationship for $\alpha$ we have

\begin{align*}
\alpha = \tanh^{-1} (\Bv/c) = \vcap \tanh^{-1} (\Abs{\Bv/c})
\end{align*}

The unit vector $\vcap$ can be factored out of the inverse hyperbolic tangent function since it is odd (consider the Taylor series expansion of $\tanh^{-1}$ to see why one can do this).

Finally, we have by 
dotting with the spatial basis vectors $\sigma_i$ three quantities in terms of spacetime vector rapidity angle

\begin{align*}
\alpha_i = (\vcap \cdot \sigma_i) \tanh^{-1} (\Abs{\Bv/c}).
\end{align*}

The $\vcap \cdot \sigma_i$ parts are direction cosines, so the three rapidities Lut was asking about all appear to be weighted direction cosines.

%\bibliographystyle{plainnat}
%\bibliography{myrefs}

%\end{document}

%
% Copyright � 2012 Peeter Joot.  All Rights Reserved.
% Licenced as described in the file LICENSE under the root directory of this GIT repository.
%

% 
% 
%\documentclass{article}

%\usepackage{amsmath}
\usepackage{mathpazo}

%
% shorthand for bold symbols, convenient for vectors and matrices
%
\newcommand{\Ba}[0]{\mathbf{a}}
\newcommand{\Bb}[0]{\mathbf{b}}
\newcommand{\Bc}[0]{\mathbf{c}}
\newcommand{\Bd}[0]{\mathbf{d}}
\newcommand{\Be}[0]{\mathbf{e}}
\newcommand{\Bf}[0]{\mathbf{f}}
\newcommand{\Bg}[0]{\mathbf{g}}
\newcommand{\Bh}[0]{\mathbf{h}}
\newcommand{\Bi}[0]{\mathbf{i}}
\newcommand{\Bj}[0]{\mathbf{j}}
\newcommand{\Bk}[0]{\mathbf{k}}
\newcommand{\Bl}[0]{\mathbf{l}}
\newcommand{\Bm}[0]{\mathbf{m}}
\newcommand{\Bn}[0]{\mathbf{n}}
\newcommand{\Bo}[0]{\mathbf{o}}
\newcommand{\Bp}[0]{\mathbf{p}}
\newcommand{\Bq}[0]{\mathbf{q}}
\newcommand{\Br}[0]{\mathbf{r}}
\newcommand{\Bs}[0]{\mathbf{s}}
\newcommand{\Bt}[0]{\mathbf{t}}
\newcommand{\Bu}[0]{\mathbf{u}}
\newcommand{\Bv}[0]{\mathbf{v}}
\newcommand{\Bw}[0]{\mathbf{w}}
\newcommand{\Bx}[0]{\mathbf{x}}
\newcommand{\By}[0]{\mathbf{y}}
\newcommand{\Bz}[0]{\mathbf{z}}
\newcommand{\BA}[0]{\mathbf{A}}
\newcommand{\BB}[0]{\mathbf{B}}
\newcommand{\BC}[0]{\mathbf{C}}
\newcommand{\BD}[0]{\mathbf{D}}
\newcommand{\BE}[0]{\mathbf{E}}
\newcommand{\BF}[0]{\mathbf{F}}
\newcommand{\BG}[0]{\mathbf{G}}
\newcommand{\BH}[0]{\mathbf{H}}
\newcommand{\BI}[0]{\mathbf{I}}
\newcommand{\BJ}[0]{\mathbf{J}}
\newcommand{\BK}[0]{\mathbf{K}}
\newcommand{\BL}[0]{\mathbf{L}}
\newcommand{\BM}[0]{\mathbf{M}}
\newcommand{\BN}[0]{\mathbf{N}}
\newcommand{\BO}[0]{\mathbf{O}}
\newcommand{\BP}[0]{\mathbf{P}}
\newcommand{\BQ}[0]{\mathbf{Q}}
\newcommand{\BR}[0]{\mathbf{R}}
\newcommand{\BS}[0]{\mathbf{S}}
\newcommand{\BT}[0]{\mathbf{T}}
\newcommand{\BU}[0]{\mathbf{U}}
\newcommand{\BV}[0]{\mathbf{V}}
\newcommand{\BW}[0]{\mathbf{W}}
\newcommand{\BX}[0]{\mathbf{X}}
\newcommand{\BY}[0]{\mathbf{Y}}
\newcommand{\BZ}[0]{\mathbf{Z}}

\newcommand{\Bzero}[0]{\mathbf{0}}
\newcommand{\Btheta}[0]{\boldsymbol{\theta}}
\newcommand{\Btau}[0]{\boldsymbol{\tau}}
\newcommand{\Bomega}[0]{\boldsymbol{\omega}}

%
% shorthand for unit vectors
%
\newcommand{\acap}[0]{\hat{\Ba}}
\newcommand{\bcap}[0]{\hat{\Bb}}
\newcommand{\ccap}[0]{\hat{\Bc}}
\newcommand{\dcap}[0]{\hat{\Bd}}
\newcommand{\ecap}[0]{\hat{\Be}}
\newcommand{\fcap}[0]{\hat{\Bf}}
\newcommand{\gcap}[0]{\hat{\Bg}}
\newcommand{\hcap}[0]{\hat{\Bh}}
\newcommand{\icap}[0]{\hat{\Bi}}
\newcommand{\jcap}[0]{\hat{\Bj}}
\newcommand{\kcap}[0]{\hat{\Bk}}
\newcommand{\lcap}[0]{\hat{\Bl}}
\newcommand{\mcap}[0]{\hat{\Bm}}
\newcommand{\ncap}[0]{\hat{\Bn}}
\newcommand{\ocap}[0]{\hat{\Bo}}
\newcommand{\pcap}[0]{\hat{\Bp}}
\newcommand{\qcap}[0]{\hat{\Bq}}
\newcommand{\rcap}[0]{\hat{\Br}}
\newcommand{\scap}[0]{\hat{\Bs}}
\newcommand{\tcap}[0]{\hat{\Bt}}
\newcommand{\ucap}[0]{\hat{\Bu}}
\newcommand{\vcap}[0]{\hat{\Bv}}
\newcommand{\wcap}[0]{\hat{\Bw}}
\newcommand{\xcap}[0]{\hat{\Bx}}
\newcommand{\ycap}[0]{\hat{\By}}
\newcommand{\zcap}[0]{\hat{\Bz}}
\newcommand{\thetacap}[0]{\hat{\Btheta}}

%
% to write R^n and C^n in a distinguishable fashion.  Perhaps change this
% to the double lined characters upon figuring out how to do so.
%
\newcommand{\C}[1]{$\mathbb{C}^{#1}$}
\newcommand{\R}[1]{$\mathbb{R}^{#1}$}

%
% various generally useful helpers
%

% derivative of #1 wrt. #2:
\newcommand{\D}[2] {\frac {d#2} {d#1}}

\newcommand{\inv}[1]{\frac{1}{#1}}
\newcommand{\cross}[0]{\times}

\newcommand{\abs}[1]{\lvert{#1}\rvert}
\newcommand{\norm}[1]{\lVert{#1}\rVert}
\newcommand{\innerprod}[2]{\langle{#1}, {#2}\rangle}
\newcommand{\dotprod}[2]{{#1} \cdot {#2}}
\newcommand{\bdotprod}[2]{\left({#1} \cdot {#2}\right)}
\newcommand{\crossprod}[2]{{#1} \cross {#2}}
\newcommand{\tripleprod}[3]{\dotprod{\left(\crossprod{#1}{#2}\right)}{#3}}

\DeclareMathOperator{\Proj}{Proj}
\DeclareMathOperator{\Span}{span}
\DeclareMathOperator{\Sgn}{sgn}
\DeclareMathOperator{\Area}{Area}
\DeclareMathOperator{\Volume}{Volume}

%
% A few miscellaneous things specific to this document
%
\newcommand{\crossop}[1]{\crossprod{#1}{}}

% R2 vector.
\newcommand{\VectorTwo}[2]{
\begin{bmatrix}
 {#1} \\
 {#2}
\end{bmatrix}
}

\newcommand{\VectorN}[1]{
\begin{bmatrix}
{#1}_1 \\
{#1}_2 \\
\vdots \\
{#1}_N \\
\end{bmatrix}
}

\newcommand{\DETuvij}[4]{
\begin{vmatrix}
 {#1}_{#3} & {#1}_{#4} \\
 {#2}_{#3} & {#2}_{#4}
\end{vmatrix}
}

\newcommand{\DETuvwijk}[6]{
\begin{vmatrix}
 {#1}_{#4} & {#1}_{#5} & {#1}_{#6} \\
 {#2}_{#4} & {#2}_{#5} & {#2}_{#6} \\
 {#3}_{#4} & {#3}_{#5} & {#3}_{#6}
\end{vmatrix}
}

\newcommand{\DETuvwxijkl}[8]{
\begin{vmatrix}
 {#1}_{#5} & {#1}_{#6} & {#1}_{#7} & {#1}_{#8} \\
 {#2}_{#5} & {#2}_{#6} & {#2}_{#7} & {#2}_{#8} \\
 {#3}_{#5} & {#3}_{#6} & {#3}_{#7} & {#3}_{#8} \\
 {#4}_{#5} & {#4}_{#6} & {#4}_{#7} & {#4}_{#8} \\
\end{vmatrix}
}

%\newcommand{\DETuvwxyijklm}[10]{
%\begin{vmatrix}
% {#1}_{#6} & {#1}_{#7} & {#1}_{#8} & {#1}_{#9} & {#1}_{#10} \\
% {#2}_{#6} & {#2}_{#7} & {#2}_{#8} & {#2}_{#9} & {#2}_{#10} \\
% {#3}_{#6} & {#3}_{#7} & {#3}_{#8} & {#3}_{#9} & {#3}_{#10} \\
% {#4}_{#6} & {#4}_{#7} & {#4}_{#8} & {#4}_{#9} & {#4}_{#10} \\
% {#5}_{#6} & {#5}_{#7} & {#5}_{#8} & {#5}_{#9} & {#5}_{#10}
%\end{vmatrix}
%}

% R3 vector.
\newcommand{\VectorThree}[3]{
\begin{bmatrix}
 {#1} \\
 {#2} \\
 {#3}
\end{bmatrix}
}


%%<misc>
%
\newcommand{\Abs}[1]{{\left\lvert{#1}\right\rvert}}
\newcommand{\spacegrad}[0]{\boldsymbol{\nabla}}
\newcommand{\grad}[0]{\nabla}
\newcommand{\LL}[0]{\mathcal{L}}

% == \partial_{#1} {#2}
\newcommand{\PD}[2]{\frac{\partial {#2}}{\partial {#1}}}
% inline variant
\newcommand{\PDi}[2]{{\partial {#2}}/{\partial {#1}}}

\newcommand{\PDD}[3]{\frac{\partial^2 {#3}}{\partial {#1}\partial {#2}}}
%\newcommand{\PDd}[2]{\frac{\partial^2 {#2}}{{\partial{#1}}^2}}
\newcommand{\PDsq}[2]{\frac{\partial^2 {#2}}{(\partial {#1})^2}}

\newcommand{\Partial}[2]{\frac{\partial {#1}}{\partial {#2}}}
\DeclareMathOperator{\RejName}{Rej}
\newcommand{\Rej}[2]{\RejName_{#1}\left( {#2} \right)}
\newcommand{\Rm}[1]{\mathbb{R}^{#1}}
\newcommand{\Cm}[1]{\mathbb{C}^{#1}}
\newcommand{\conj}[0]{{*}}

%</misc>

% <grade selection>
%
\newcommand{\gpgrade}[2] {{\left\langle{{#1}}\right\rangle}_{#2}}

\newcommand{\gpgradezero}[1] {\gpgrade{#1}{}}
%\newcommand{\gpscalargrade}[1] {{\left\langle{{#1}}\right\rangle}}
%\newcommand{\gpgradezero}[1] {\gpgrade{#1}{0}}

%\newcommand{\gpgradeone}[1] {{\left\langle{{#1}}\right\rangle}_{1}}
\newcommand{\gpgradeone}[1] {\gpgrade{#1}{1}}

\newcommand{\gpgradetwo}[1] {\gpgrade{#1}{2}}
\newcommand{\gpgradethree}[1] {\gpgrade{#1}{3}}
\newcommand{\gpgradefour}[1] {\gpgrade{#1}{4}}
%
% </grade selection>



\newcommand{\adot}[0]{{\dot{a}}}
\newcommand{\bdot}[0]{{\dot{b}}}
% taken for centered dot:
%\newcommand{\cdot}[0]{{\dot{c}}}
%\newcommand{\ddot}[0]{{\dot{d}}}
\newcommand{\edot}[0]{{\dot{e}}}
\newcommand{\fdot}[0]{{\dot{f}}}
\newcommand{\gdot}[0]{{\dot{g}}}
\newcommand{\hdot}[0]{{\dot{h}}}
\newcommand{\idot}[0]{{\dot{i}}}
\newcommand{\jdot}[0]{{\dot{j}}}
\newcommand{\kdot}[0]{{\dot{k}}}
\newcommand{\ldot}[0]{{\dot{l}}}
\newcommand{\mdot}[0]{{\dot{m}}}
\newcommand{\ndot}[0]{{\dot{n}}}
%\newcommand{\odot}[0]{{\dot{o}}}
\newcommand{\pdot}[0]{{\dot{p}}}
\newcommand{\qdot}[0]{{\dot{q}}}
\newcommand{\rdot}[0]{{\dot{r}}}
\newcommand{\sdot}[0]{{\dot{s}}}
\newcommand{\tdot}[0]{{\dot{t}}}
\newcommand{\udot}[0]{{\dot{u}}}
\newcommand{\vdot}[0]{{\dot{v}}}
\newcommand{\wdot}[0]{{\dot{w}}}
\newcommand{\xdot}[0]{{\dot{x}}}
\newcommand{\ydot}[0]{{\dot{y}}}
\newcommand{\zdot}[0]{{\dot{z}}}
\newcommand{\addot}[0]{{\ddot{a}}}
\newcommand{\bddot}[0]{{\ddot{b}}}
\newcommand{\cddot}[0]{{\ddot{c}}}
%\newcommand{\dddot}[0]{{\ddot{d}}}
\newcommand{\eddot}[0]{{\ddot{e}}}
\newcommand{\fddot}[0]{{\ddot{f}}}
\newcommand{\gddot}[0]{{\ddot{g}}}
\newcommand{\hddot}[0]{{\ddot{h}}}
\newcommand{\iddot}[0]{{\ddot{i}}}
\newcommand{\jddot}[0]{{\ddot{j}}}
\newcommand{\kddot}[0]{{\ddot{k}}}
\newcommand{\lddot}[0]{{\ddot{l}}}
\newcommand{\mddot}[0]{{\ddot{m}}}
\newcommand{\nddot}[0]{{\ddot{n}}}
\newcommand{\oddot}[0]{{\ddot{o}}}
\newcommand{\pddot}[0]{{\ddot{p}}}
\newcommand{\qddot}[0]{{\ddot{q}}}
\newcommand{\rddot}[0]{{\ddot{r}}}
\newcommand{\sddot}[0]{{\ddot{s}}}
\newcommand{\tddot}[0]{{\ddot{t}}}
\newcommand{\uddot}[0]{{\ddot{u}}}
\newcommand{\vddot}[0]{{\ddot{v}}}
\newcommand{\wddot}[0]{{\ddot{w}}}
\newcommand{\xddot}[0]{{\ddot{x}}}
\newcommand{\yddot}[0]{{\ddot{y}}}
\newcommand{\zddot}[0]{{\ddot{z}}}

%<bold and dot greek symbols>
%

\newcommand{\Deltadot}[0]{{\dot{\Delta}}}
\newcommand{\Gammadot}[0]{{\dot{\Gamma}}}
\newcommand{\Lambdadot}[0]{{\dot{\Lambda}}}
\newcommand{\Omegadot}[0]{{\dot{\Omega}}}
\newcommand{\Phidot}[0]{{\dot{\Phi}}}
\newcommand{\Pidot}[0]{{\dot{\Pi}}}
\newcommand{\Psidot}[0]{{\dot{\Psi}}}
\newcommand{\Sigmadot}[0]{{\dot{\Sigma}}}
\newcommand{\Thetadot}[0]{{\dot{\Theta}}}
\newcommand{\Upsilondot}[0]{{\dot{\Upsilon}}}
\newcommand{\Xidot}[0]{{\dot{\Xi}}}
\newcommand{\alphadot}[0]{{\dot{\alpha}}}
\newcommand{\betadot}[0]{{\dot{\beta}}}
\newcommand{\chidot}[0]{{\dot{\chi}}}
\newcommand{\deltadot}[0]{{\dot{\delta}}}
\newcommand{\epsilondot}[0]{{\dot{\epsilon}}}
\newcommand{\etadot}[0]{{\dot{\eta}}}
\newcommand{\gammadot}[0]{{\dot{\gamma}}}
\newcommand{\kappadot}[0]{{\dot{\kappa}}}
\newcommand{\lambdadot}[0]{{\dot{\lambda}}}
\newcommand{\mudot}[0]{{\dot{\mu}}}
\newcommand{\nudot}[0]{{\dot{\nu}}}
\newcommand{\omegadot}[0]{{\dot{\omega}}}
\newcommand{\phidot}[0]{{\dot{\phi}}}
\newcommand{\pidot}[0]{{\dot{\pi}}}
\newcommand{\psidot}[0]{{\dot{\psi}}}
\newcommand{\rhodot}[0]{{\dot{\rho}}}
\newcommand{\sigmadot}[0]{{\dot{\sigma}}}
\newcommand{\taudot}[0]{{\dot{\tau}}}
\newcommand{\thetadot}[0]{{\dot{\theta}}}
\newcommand{\upsilondot}[0]{{\dot{\upsilon}}}
\newcommand{\varepsilondot}[0]{{\dot{\varepsilon}}}
\newcommand{\varphidot}[0]{{\dot{\varphi}}}
\newcommand{\varpidot}[0]{{\dot{\varpi}}}
\newcommand{\varrhodot}[0]{{\dot{\varrho}}}
\newcommand{\varsigmadot}[0]{{\dot{\varsigma}}}
\newcommand{\varthetadot}[0]{{\dot{\vartheta}}}
\newcommand{\xidot}[0]{{\dot{\xi}}}
\newcommand{\zetadot}[0]{{\dot{\zeta}}}

\newcommand{\Deltaddot}[0]{{\ddot{\Delta}}}
\newcommand{\Gammaddot}[0]{{\ddot{\Gamma}}}
\newcommand{\Lambdaddot}[0]{{\ddot{\Lambda}}}
\newcommand{\Omegaddot}[0]{{\ddot{\Omega}}}
\newcommand{\Phiddot}[0]{{\ddot{\Phi}}}
\newcommand{\Piddot}[0]{{\ddot{\Pi}}}
\newcommand{\Psiddot}[0]{{\ddot{\Psi}}}
\newcommand{\Sigmaddot}[0]{{\ddot{\Sigma}}}
\newcommand{\Thetaddot}[0]{{\ddot{\Theta}}}
\newcommand{\Upsilonddot}[0]{{\ddot{\Upsilon}}}
\newcommand{\Xiddot}[0]{{\ddot{\Xi}}}
\newcommand{\alphaddot}[0]{{\ddot{\alpha}}}
\newcommand{\betaddot}[0]{{\ddot{\beta}}}
\newcommand{\chiddot}[0]{{\ddot{\chi}}}
\newcommand{\deltaddot}[0]{{\ddot{\delta}}}
\newcommand{\epsilonddot}[0]{{\ddot{\epsilon}}}
\newcommand{\etaddot}[0]{{\ddot{\eta}}}
\newcommand{\gammaddot}[0]{{\ddot{\gamma}}}
\newcommand{\kappaddot}[0]{{\ddot{\kappa}}}
\newcommand{\lambdaddot}[0]{{\ddot{\lambda}}}
\newcommand{\muddot}[0]{{\ddot{\mu}}}
\newcommand{\nuddot}[0]{{\ddot{\nu}}}
\newcommand{\omegaddot}[0]{{\ddot{\omega}}}
\newcommand{\phiddot}[0]{{\ddot{\phi}}}
\newcommand{\piddot}[0]{{\ddot{\pi}}}
\newcommand{\psiddot}[0]{{\ddot{\psi}}}
\newcommand{\rhoddot}[0]{{\ddot{\rho}}}
\newcommand{\sigmaddot}[0]{{\ddot{\sigma}}}
\newcommand{\tauddot}[0]{{\ddot{\tau}}}
\newcommand{\thetaddot}[0]{{\ddot{\theta}}}
\newcommand{\upsilonddot}[0]{{\ddot{\upsilon}}}
\newcommand{\varepsilonddot}[0]{{\ddot{\varepsilon}}}
\newcommand{\varphiddot}[0]{{\ddot{\varphi}}}
\newcommand{\varpiddot}[0]{{\ddot{\varpi}}}
\newcommand{\varrhoddot}[0]{{\ddot{\varrho}}}
\newcommand{\varsigmaddot}[0]{{\ddot{\varsigma}}}
\newcommand{\varthetaddot}[0]{{\ddot{\vartheta}}}
\newcommand{\xiddot}[0]{{\ddot{\xi}}}
\newcommand{\zetaddot}[0]{{\ddot{\zeta}}}

\newcommand{\BDelta}[0]{\boldsymbol{\Delta}}
\newcommand{\BGamma}[0]{\boldsymbol{\Gamma}}
\newcommand{\BLambda}[0]{\boldsymbol{\Lambda}}
\newcommand{\BOmega}[0]{\boldsymbol{\Omega}}
\newcommand{\BPhi}[0]{\boldsymbol{\Phi}}
\newcommand{\BPi}[0]{\boldsymbol{\Pi}}
\newcommand{\BPsi}[0]{\boldsymbol{\Psi}}
\newcommand{\BSigma}[0]{\boldsymbol{\Sigma}}
\newcommand{\BTheta}[0]{\boldsymbol{\Theta}}
\newcommand{\BUpsilon}[0]{\boldsymbol{\Upsilon}}
\newcommand{\BXi}[0]{\boldsymbol{\Xi}}
\newcommand{\Balpha}[0]{\boldsymbol{\alpha}}
\newcommand{\Bbeta}[0]{\boldsymbol{\beta}}
\newcommand{\Bchi}[0]{\boldsymbol{\chi}}
\newcommand{\Bdelta}[0]{\boldsymbol{\delta}}
\newcommand{\Bepsilon}[0]{\boldsymbol{\epsilon}}
\newcommand{\Beta}[0]{\boldsymbol{\eta}}
\newcommand{\Bgamma}[0]{\boldsymbol{\gamma}}
\newcommand{\Bkappa}[0]{\boldsymbol{\kappa}}
\newcommand{\Blambda}[0]{\boldsymbol{\lambda}}
\newcommand{\Bmu}[0]{\boldsymbol{\mu}}
\newcommand{\Bnu}[0]{\boldsymbol{\nu}}
%\newcommand{\Bomega}[0]{\boldsymbol{\omega}}
\newcommand{\Bphi}[0]{\boldsymbol{\phi}}
\newcommand{\Bpi}[0]{\boldsymbol{\pi}}
\newcommand{\Bpsi}[0]{\boldsymbol{\psi}}
\newcommand{\Brho}[0]{\boldsymbol{\rho}}
\newcommand{\Bsigma}[0]{\boldsymbol{\sigma}}
%\newcommand{\Btau}[0]{\boldsymbol{\tau}}
%\newcommand{\Btheta}[0]{\boldsymbol{\theta}}
\newcommand{\Bupsilon}[0]{\boldsymbol{\upsilon}}
\newcommand{\Bvarepsilon}[0]{\boldsymbol{\varepsilon}}
\newcommand{\Bvarphi}[0]{\boldsymbol{\varphi}}
\newcommand{\Bvarpi}[0]{\boldsymbol{\varpi}}
\newcommand{\Bvarrho}[0]{\boldsymbol{\varrho}}
\newcommand{\Bvarsigma}[0]{\boldsymbol{\varsigma}}
\newcommand{\Bvartheta}[0]{\boldsymbol{\vartheta}}
\newcommand{\Bxi}[0]{\boldsymbol{\xi}}
\newcommand{\Bzeta}[0]{\boldsymbol{\zeta}}
%
%</bold and dot greek symbols>
%<infrequent>
%
%\newcommand{\AreaOp}[1]{\AName_{#1}}
%\newcommand{\Babs}[0]{\abs{\BB}}
%\newcommand{\Bcap}[0]{\hat{\BB}}
%\newcommand{\BrPrimeRej}[0]{\rcap(\rcap \wedge \Br')}
%\newcommand{\CA}[0]{\mathcal{A}}
%\newcommand{\Cos}[1]{\cos{\left({#1}\right)}}
%\newcommand{\Det}[1] {\abs{#1}}
%\newcommand{\Dsq}[2] {\frac {\partial^2 {#1}} {\partial {#2}^2}}
%\newcommand{\Exp}[1]{\exp{\left({#1}\right)}}
%\newcommand{\Norm}[1]{\left\lVert{#1}\right\rVert}
%\newcommand{\Sin}[1]{\sin{\left({#1}\right)}}
%\newcommand{\T}[0]{\text{T}}
%\newcommand{\VolumeOp}[1]{\VName_{#1}}
%\newcommand{\agrad}[0]{\Ba \cdot \nabla}
%\newcommand{\alphacap}[0]{\hat{\boldsymbol{\alpha}}}
%\newcommand{\Fcap}[0]{\hat{\BF}}
%\newcommand{\bithree}[0]{{\Bi}_3}
%\newcommand{\bxa}[0]{\Bx\Ba}
%\newcommand{\coordvec}[2]{
%\newcommand{\costheta}[0]{\acap \cdot \xcap}
%\newcommand{\ddt}[1]{\ddot{#1}}
%\newcommand{\ddu}[1] {\frac {d{#1}} {du}}
%\newcommand{\dsqxj}[2] {\frac {\partial^2 {#1}} {\partial {x_{#2}}^2}}
%\newcommand{\dtheta}[1]{\frac{d {#1}}{d \theta}}
%\newcommand{\dt}[1]{\dot{#1}}
%\newcommand{\dt}[1]{\frac{d {#1}}{dt}}
%\newcommand{\dxj}[2] {\frac {\partial {#1}} {\partial {x_{#2}}}}
%\newcommand{\halfPhi}[0]{\frac{\phi}{2}}
%\newcommand{\half}[0]{\inv{2}}
%\newcommand{\inv}[1]{\frac{1}{#1}}
%\newcommand{\laplacian}[0]{\nabla^2}
%\newcommand{\matrixoftx}[3]{
%\newcommand{\nrrp}[0]{\norm{\rcap \wedge \Br'}}
%\newcommand{\oiint}{\bigcirc \hspace{-1.4em} \int \hspace{-.8em} \int}
%\newcommand{\transpose}[1]{{#1}^{\text{T}}}
%\newcommand{\transpose}[1]{{{#1}^{\TextTranspose}}}
%\newcommand{\transpose}[1]{{{#1}^{\text{T}}}}
%\newcommand{\barA}[0]{\bar{A}}
%\newcommand{\qbar}[0]{\bar{q}}
%\newcommand{\qdotbar}[0]{\dot{\bar{q}}}
%
%</infrequent>





%\usepackage[bookmarks=true]{hyperref}

%\usepackage{color,cite,graphicx}
   % use colour in the document, put your citations as [1-4]
   % rather than [1,2,3,4] (it looks nicer, and the extended LaTeX2e
   % graphics package. 
%\usepackage{latexsym,amssymb,epsf} % do not remember if these are
   % needed, but their inclusion can not do any damage

\chapter{Ad-hoc motivation of some QM wave equations}
\label{chap:sch}
%\author{Peeter Joot \quad peeterjoot@protonmail.com}
\date{ Dec 13, 2008.  sch.tex }

%\begin{document}

%\maketitle{}

%\tableofcontents

\section{Motivation}

\subsection{Non-Relativistic case}

A common (cf: wikipedia and \citep{french1998iqp}) introductory motivation for the non-relativistic Schr\"{o}dinger's equation appears to follow the following lines.  Assume that
we desire a plane wave equation of the following form

\begin{equation}\label{eqn:sch:20}
\begin{aligned}
\psi = \exp(i (\Bk \cdot \Bx - \omega t))
\end{aligned}
\end{equation}

plus a requirement that we have a total energy that can be expressed in terms of kinetic plus potential

\begin{equation}\label{eqn:sch:energy}
\begin{aligned}
E = \frac{\Bp^2}{2m} + V
\end{aligned}
\end{equation}

We also have the Einstein relationship

\begin{equation}\label{eqn:sch:40}
\begin{aligned}
E = h \nu = \frac{h}{2\pi} 2 \pi \nu = \Hbar \omega
\end{aligned}
\end{equation}

and the DeBroglie Hypothesis for the magnitude of the momentum of a particle

\begin{equation}\label{eqn:sch:60}
\begin{aligned}
p = \frac{h}{\lambda}.
\end{aligned}
\end{equation}

In terms of wave number \(2 \pi k = \inv{\lambda}\) this last is

\begin{equation}\label{eqn:sch:80}
\begin{aligned}
p = \frac{h}{2\pi} k = \Hbar k.
\end{aligned}
\end{equation}

or in three dimensions

\begin{equation}\label{eqn:sch:100}
\begin{aligned}
\Bp = \Hbar \Bk.
\end{aligned}
\end{equation}

Taking derivatives of the postulated wave function one has

\begin{equation}\label{eqn:sch:timepartial}
\begin{aligned}
\frac{\partial \psi}{\partial t} = -i \omega \psi
\end{aligned}
\end{equation}

and 
\begin{equation}\label{eqn:sch:laplacian}
\begin{aligned}
\spacegrad^2 \psi = i^2 \sum_j k_j^2 \psi = - \Bk^2 \psi
\end{aligned}
\end{equation}

From the energy relationship, if one requires that
\begin{equation}\label{eqn:sch:120}
\begin{aligned}
E \psi &= \left(\frac{\Bp^2}{2m} + V\right) \psi \\
\Hbar \omega \psi &= \left(\Hbar^2 \frac{\Bk^2}{2m} + V\right) \psi \\
\end{aligned}
\end{equation}

and then substituting the derivatives from equations \eqnref{eqn:sch:timepartial} and \eqnref{eqn:sch:laplacian} we have

\begin{equation}\label{eqn:sch:140}
\begin{aligned}
i \Hbar \PD{t}{\psi} &= \left(-\frac{\Hbar^2}{2m} \spacegrad^2  + V\right) \psi \\
\end{aligned}
\end{equation}

\subsection{Relativistic case}

The relativistic force free Schr\"{o}dinger's equation is motivated by \citep{srednicki2007qft} replacing the Hamiltonian operator \(H = \BP^2/{2 m}\) with

\begin{equation}\label{eqn:sch:160}
\begin{aligned}
H = \sqrt{m^2 c^4 + \BP^2 c^2} \approx m c^2 + \BP^2/2m
\end{aligned}
\end{equation}

for

\begin{equation}\label{eqn:sch:180}
\begin{aligned}
i \Hbar \PD{t}{\psi} = \sqrt{ -\Hbar^2 c^2 \spacegrad^2 + m^2 c^4} \psi
\end{aligned}
\end{equation}

then squaring the operators on both sides, removing the root:

\begin{equation}\label{eqn:sch:200}
\begin{aligned}
- \Hbar^2 \PDsq{x^0}{\psi} &= \left(-\Hbar^2 \spacegrad^2 + m^2 c^2 \right) \psi \\
- \Hbar^2 \PDsq{x^0}{\psi} + \Hbar^2 \spacegrad^2 &= m^2 c^2 \psi \\
\end{aligned}
\end{equation}

Which is called the Klein-Gordon equation:
\begin{equation}\label{eqn:sch:220}
\begin{aligned}
\left(\frac{\Hbar^2}{2m} \grad^2 + \inv{2} m c^2\right) \psi = 0 \\
\end{aligned}
\end{equation}

The Noether's theorem wikipedia article, and Klein-Gordon pages give the action
as (removing use of natural units and changing to a \(+---\) metric) gives:

\begin{equation}\label{eqn:sch:240}
\begin{aligned}
\LL &= -\eta^{\mu\nu} \partial_\mu \psi \partial_\nu \psi^\conj + \frac{m^2 c^2}{\Hbar^2} \psi \psi^\conj \\
&= -\partial^\nu \psi \partial_\nu \psi^\conj + \frac{m^2 c^2}{\Hbar^2} \psi \psi^\conj \\
\end{aligned}
\end{equation}

This first term is a squared spacetime gradient

\begin{equation}\label{eqn:sch:260}
\begin{aligned}
(\grad \psi) \cdot (\grad \psi^\conj) 
&= (\gamma^\mu \partial_\mu \psi) \cdot (\gamma_\nu \partial^\nu \psi^\conj) \\
&= {\delta^\mu}_\nu \partial_\mu \psi \partial^\nu \psi^\conj \\
&= \partial_\mu \psi \partial^\mu \psi^\conj \\
\end{aligned}
\end{equation}

so we have
\begin{equation}\label{eqn:sch:280}
\begin{aligned}
\LL 
&= -\eta^{\mu\nu} \partial_\mu \psi \partial_\nu \psi^\conj + \frac{m^2 c^2}{\Hbar^2} \psi \psi^\conj \\
&= -(\grad \psi) \cdot (\grad \psi^\conj) + \frac{m^2 c^2}{\Hbar^2} \psi \psi^\conj \\
\end{aligned}
\end{equation}

So, here we expect to get a spacetime Laplacian, like the Maxwell potential equation, and one does:

\begin{equation}\label{eqn:sch:300}
\begin{aligned}
-\grad^2 \psi = \frac{m^2 c^2}{\Hbar^2} \psi
\end{aligned}
\end{equation}

Now, the Srednicki text indicates that Dirac linearized this equation, to get back something that was
first order in the time derivative.

I do not yet follow that description, but see that a linearization is possible by taking roots of the operators above,
undoing the somewhat fishy seeming squaring done to arrive at the Klein-Gordon equation in the first place

\begin{equation}\label{eqn:sch:dirac}
\begin{aligned}
i \Hbar \grad \psi = \pm m c \psi
\end{aligned}
\end{equation}

Is this equivalent to Dirac's formulation?  Comparison to 
\href{http://en.wikipedia.org/wiki/Dirac_equation#Covariant_form_and_relativistic_invariance}, where the Dirac equation is given in covariant
form

\begin{equation}\label{eqn:sch:320}
\begin{aligned}
i \Hbar \gamma^\mu \partial_\mu \psi - m c \psi = 0
\end{aligned}
\end{equation}

which is the positive variant of \eqnref{eqn:sch:dirac}.

What is the Lagrangian that is associated with this?  The most probable interpretation of \(i\) here is the Minkowski pseudoscalar, as opposed to
a unit bivector of spacetime vectors or some other geometrical object with \(-1\) square.

%\bibliographystyle{plainnat}
%\bibliography{myrefs}

%\end{document}

\documentclass{article}      % Specifies the document class

\usepackage{amsmath}
\usepackage{mathpazo}

%
% shorthand for bold symbols, convenient for vectors and matrices
%
\newcommand{\Ba}[0]{\mathbf{a}}
\newcommand{\Bb}[0]{\mathbf{b}}
\newcommand{\Bc}[0]{\mathbf{c}}
\newcommand{\Bd}[0]{\mathbf{d}}
\newcommand{\Be}[0]{\mathbf{e}}
\newcommand{\Bf}[0]{\mathbf{f}}
\newcommand{\Bg}[0]{\mathbf{g}}
\newcommand{\Bh}[0]{\mathbf{h}}
\newcommand{\Bi}[0]{\mathbf{i}}
\newcommand{\Bj}[0]{\mathbf{j}}
\newcommand{\Bk}[0]{\mathbf{k}}
\newcommand{\Bl}[0]{\mathbf{l}}
\newcommand{\Bm}[0]{\mathbf{m}}
\newcommand{\Bn}[0]{\mathbf{n}}
\newcommand{\Bo}[0]{\mathbf{o}}
\newcommand{\Bp}[0]{\mathbf{p}}
\newcommand{\Bq}[0]{\mathbf{q}}
\newcommand{\Br}[0]{\mathbf{r}}
\newcommand{\Bs}[0]{\mathbf{s}}
\newcommand{\Bt}[0]{\mathbf{t}}
\newcommand{\Bu}[0]{\mathbf{u}}
\newcommand{\Bv}[0]{\mathbf{v}}
\newcommand{\Bw}[0]{\mathbf{w}}
\newcommand{\Bx}[0]{\mathbf{x}}
\newcommand{\By}[0]{\mathbf{y}}
\newcommand{\Bz}[0]{\mathbf{z}}
\newcommand{\BA}[0]{\mathbf{A}}
\newcommand{\BB}[0]{\mathbf{B}}
\newcommand{\BC}[0]{\mathbf{C}}
\newcommand{\BD}[0]{\mathbf{D}}
\newcommand{\BE}[0]{\mathbf{E}}
\newcommand{\BF}[0]{\mathbf{F}}
\newcommand{\BG}[0]{\mathbf{G}}
\newcommand{\BH}[0]{\mathbf{H}}
\newcommand{\BI}[0]{\mathbf{I}}
\newcommand{\BJ}[0]{\mathbf{J}}
\newcommand{\BK}[0]{\mathbf{K}}
\newcommand{\BL}[0]{\mathbf{L}}
\newcommand{\BM}[0]{\mathbf{M}}
\newcommand{\BN}[0]{\mathbf{N}}
\newcommand{\BO}[0]{\mathbf{O}}
\newcommand{\BP}[0]{\mathbf{P}}
\newcommand{\BQ}[0]{\mathbf{Q}}
\newcommand{\BR}[0]{\mathbf{R}}
\newcommand{\BS}[0]{\mathbf{S}}
\newcommand{\BT}[0]{\mathbf{T}}
\newcommand{\BU}[0]{\mathbf{U}}
\newcommand{\BV}[0]{\mathbf{V}}
\newcommand{\BW}[0]{\mathbf{W}}
\newcommand{\BX}[0]{\mathbf{X}}
\newcommand{\BY}[0]{\mathbf{Y}}
\newcommand{\BZ}[0]{\mathbf{Z}}

\newcommand{\Bzero}[0]{\mathbf{0}}
\newcommand{\Btheta}[0]{\boldsymbol{\theta}}
\newcommand{\Btau}[0]{\boldsymbol{\tau}}
\newcommand{\Bomega}[0]{\boldsymbol{\omega}}

%
% shorthand for unit vectors
%
\newcommand{\acap}[0]{\hat{\Ba}}
\newcommand{\bcap}[0]{\hat{\Bb}}
\newcommand{\ccap}[0]{\hat{\Bc}}
\newcommand{\dcap}[0]{\hat{\Bd}}
\newcommand{\ecap}[0]{\hat{\Be}}
\newcommand{\fcap}[0]{\hat{\Bf}}
\newcommand{\gcap}[0]{\hat{\Bg}}
\newcommand{\hcap}[0]{\hat{\Bh}}
\newcommand{\icap}[0]{\hat{\Bi}}
\newcommand{\jcap}[0]{\hat{\Bj}}
\newcommand{\kcap}[0]{\hat{\Bk}}
\newcommand{\lcap}[0]{\hat{\Bl}}
\newcommand{\mcap}[0]{\hat{\Bm}}
\newcommand{\ncap}[0]{\hat{\Bn}}
\newcommand{\ocap}[0]{\hat{\Bo}}
\newcommand{\pcap}[0]{\hat{\Bp}}
\newcommand{\qcap}[0]{\hat{\Bq}}
\newcommand{\rcap}[0]{\hat{\Br}}
\newcommand{\scap}[0]{\hat{\Bs}}
\newcommand{\tcap}[0]{\hat{\Bt}}
\newcommand{\ucap}[0]{\hat{\Bu}}
\newcommand{\vcap}[0]{\hat{\Bv}}
\newcommand{\wcap}[0]{\hat{\Bw}}
\newcommand{\xcap}[0]{\hat{\Bx}}
\newcommand{\ycap}[0]{\hat{\By}}
\newcommand{\zcap}[0]{\hat{\Bz}}
\newcommand{\thetacap}[0]{\hat{\Btheta}}

%
% to write R^n and C^n in a distinguishable fashion.  Perhaps change this
% to the double lined characters upon figuring out how to do so.
%
\newcommand{\C}[1]{$\mathbb{C}^{#1}$}
\newcommand{\R}[1]{$\mathbb{R}^{#1}$}

%
% various generally useful helpers
%

% derivative of #1 wrt. #2:
\newcommand{\D}[2] {\frac {d#2} {d#1}}

\newcommand{\inv}[1]{\frac{1}{#1}}
\newcommand{\cross}[0]{\times}

\newcommand{\abs}[1]{\lvert{#1}\rvert}
\newcommand{\norm}[1]{\lVert{#1}\rVert}
\newcommand{\innerprod}[2]{\langle{#1}, {#2}\rangle}
\newcommand{\dotprod}[2]{{#1} \cdot {#2}}
\newcommand{\bdotprod}[2]{\left({#1} \cdot {#2}\right)}
\newcommand{\crossprod}[2]{{#1} \cross {#2}}
\newcommand{\tripleprod}[3]{\dotprod{\left(\crossprod{#1}{#2}\right)}{#3}}

\DeclareMathOperator{\Proj}{Proj}
\DeclareMathOperator{\Span}{span}
\DeclareMathOperator{\Sgn}{sgn}
\DeclareMathOperator{\Area}{Area}
\DeclareMathOperator{\Volume}{Volume}

%
% A few miscellaneous things specific to this document
%
\newcommand{\crossop}[1]{\crossprod{#1}{}}

% R2 vector.
\newcommand{\VectorTwo}[2]{
\begin{bmatrix}
 {#1} \\
 {#2}
\end{bmatrix}
}

\newcommand{\VectorN}[1]{
\begin{bmatrix}
{#1}_1 \\
{#1}_2 \\
\vdots \\
{#1}_N \\
\end{bmatrix}
}

\newcommand{\DETuvij}[4]{
\begin{vmatrix}
 {#1}_{#3} & {#1}_{#4} \\
 {#2}_{#3} & {#2}_{#4}
\end{vmatrix}
}

\newcommand{\DETuvwijk}[6]{
\begin{vmatrix}
 {#1}_{#4} & {#1}_{#5} & {#1}_{#6} \\
 {#2}_{#4} & {#2}_{#5} & {#2}_{#6} \\
 {#3}_{#4} & {#3}_{#5} & {#3}_{#6}
\end{vmatrix}
}

\newcommand{\DETuvwxijkl}[8]{
\begin{vmatrix}
 {#1}_{#5} & {#1}_{#6} & {#1}_{#7} & {#1}_{#8} \\
 {#2}_{#5} & {#2}_{#6} & {#2}_{#7} & {#2}_{#8} \\
 {#3}_{#5} & {#3}_{#6} & {#3}_{#7} & {#3}_{#8} \\
 {#4}_{#5} & {#4}_{#6} & {#4}_{#7} & {#4}_{#8} \\
\end{vmatrix}
}

%\newcommand{\DETuvwxyijklm}[10]{
%\begin{vmatrix}
% {#1}_{#6} & {#1}_{#7} & {#1}_{#8} & {#1}_{#9} & {#1}_{#10} \\
% {#2}_{#6} & {#2}_{#7} & {#2}_{#8} & {#2}_{#9} & {#2}_{#10} \\
% {#3}_{#6} & {#3}_{#7} & {#3}_{#8} & {#3}_{#9} & {#3}_{#10} \\
% {#4}_{#6} & {#4}_{#7} & {#4}_{#8} & {#4}_{#9} & {#4}_{#10} \\
% {#5}_{#6} & {#5}_{#7} & {#5}_{#8} & {#5}_{#9} & {#5}_{#10}
%\end{vmatrix}
%}

% R3 vector.
\newcommand{\VectorThree}[3]{
\begin{bmatrix}
 {#1} \\
 {#2} \\
 {#3}
\end{bmatrix}
}


\newcommand{\gpgrade}[2] {{\left\langle{{#1}}\right\rangle}_{#2}}

%
% The real thing:
%

                             % The preamble begins here.
\title{ Notes on shear transformation. } % Declares the document's title.
\author{Peeter Joot}         % Declares the author's name.
%\date{}        % Deleting this command produces today's date.

\begin{document}             % End of preamble and beginning of text.

\maketitle{}

\section{}

GASC and GAFP both give examples of shear transformations of the following
form:

\[
F(a) = a + \alpha(a \cdot f) e.
\]

Geometric Algebra for Physicists uses this to compute the determinant without putting the operator in matrix form.  They end up stating that 

\begin{equation}\label{eqn:shearblade}
F(A) = A + \alpha e \wedge (f \cdot A).
\end{equation}

holds for any grade blade $A$.  For grade 1 that is true since

\begin{align*}
(a \cdot f) e 
&= \gpgrade{(a \cdot f) e \rangle}{1} \\
&= (a \cdot f) \wedge e.
\end{align*}

They demonstrate equation \ref{eqn:shearblade}
holds for the grade 2 case.  To me it seems
like an induction is required to make their statement for any grade.

Question: Is there some other principle that I didn't notice in my reading that allows assertion of 
equation \ref{eqn:shearblade}
for any grade blade without the induction.

\subsection{ Proof for any grade blade. }

For $A \in {\bigwedge}^r$

\begin{align*}
F(A) \wedge F(b)
&= \left(A + \alpha e \wedge (f \cdot A)\right) \wedge
   \left(b + \alpha(b \cdot f) e \right) \\
&= A \wedge b
 + \alpha 
\left(
e \wedge (f \cdot A) \wedge b
+ (b \cdot f) A \wedge e 
\right) \\
&= A \wedge b
 + \alpha e \wedge
\underbrace{
\left(
(f \cdot A) \wedge b
+ (-1)^r (b \cdot f) A 
\right)
}_{(*)} \\
&= A \wedge b + \alpha e \wedge (f \cdot (A \wedge b))
\end{align*}

QED.

Verification below of $(*) = f \cdot (A \wedge b)$ is required to complete the proof (can probably find that in one of the books or papers but it is derivable
easily enough).

Do I have a sign mixup here somewhere?  Now that I look again I see that GAFP
has the result in different order $(A \cdot f) \wedge e$, and I get negation
reconciling the two.

\subsection{ Dot product reduction of blade by one. }

\begin{align*}
f \cdot (A \wedge b)
&= \inv{2} \gpgrade{f (A \wedge b)}{r} \\
&= \inv{2} \gpgrade{f (A b + (-1)^r b A)}{r} \\
&= \inv{2} \gpgrade{(f A) b + (-1)^r f b A}{r} \\
&= \inv{2} \gpgrade{((-1)^r A f + 2 f \cdot A) b + (-1)^r f b A}{r} \\
&= (f \cdot A) \wedge b + \frac{(-1)^r}{2} \gpgrade{A f b + f b A}{r} \\
&= (f \cdot A) \wedge b + (-1)^r (f \cdot b) A
+ \frac{(-1)^r}{2} \gpgrade{A f \wedge b + f \wedge b A}{r} \\
\end{align*}

This last term, the symmetric product of a bivector with a blade is zero.
The grades $r-2, r+4, \ldots$ terms are symmetric, and the other grades
$r, r+4, \ldots$ are antisymmetric.

Thus we have

\begin{equation}
f \cdot (A \wedge b)
= (f \cdot A) \wedge b - (-1)^r (f \cdot b) A
\end{equation}

This generalizes the familiar vector reduction formula to higher grades.
Observe that for the vector case we need the most general definition
of the wedge product for the scalar-vector wedge product (grade $1-0$ part of the product).

\end{document}               % End of document.

\include{tong_mf1}
\include{two_ninety_rotations}
\include{wave_lagrangian}
\include{wedge_norm_vs_ga_norm}

\part{lut}
\include{schwartzchild_metric}
\documentclass{article}

\usepackage{amsmath}
\usepackage{mathpazo}

%
% shorthand for bold symbols, convenient for vectors and matrices
%
\newcommand{\Ba}[0]{\mathbf{a}}
\newcommand{\Bb}[0]{\mathbf{b}}
\newcommand{\Bc}[0]{\mathbf{c}}
\newcommand{\Bd}[0]{\mathbf{d}}
\newcommand{\Be}[0]{\mathbf{e}}
\newcommand{\Bf}[0]{\mathbf{f}}
\newcommand{\Bg}[0]{\mathbf{g}}
\newcommand{\Bh}[0]{\mathbf{h}}
\newcommand{\Bi}[0]{\mathbf{i}}
\newcommand{\Bj}[0]{\mathbf{j}}
\newcommand{\Bk}[0]{\mathbf{k}}
\newcommand{\Bl}[0]{\mathbf{l}}
\newcommand{\Bm}[0]{\mathbf{m}}
\newcommand{\Bn}[0]{\mathbf{n}}
\newcommand{\Bo}[0]{\mathbf{o}}
\newcommand{\Bp}[0]{\mathbf{p}}
\newcommand{\Bq}[0]{\mathbf{q}}
\newcommand{\Br}[0]{\mathbf{r}}
\newcommand{\Bs}[0]{\mathbf{s}}
\newcommand{\Bt}[0]{\mathbf{t}}
\newcommand{\Bu}[0]{\mathbf{u}}
\newcommand{\Bv}[0]{\mathbf{v}}
\newcommand{\Bw}[0]{\mathbf{w}}
\newcommand{\Bx}[0]{\mathbf{x}}
\newcommand{\By}[0]{\mathbf{y}}
\newcommand{\Bz}[0]{\mathbf{z}}
\newcommand{\BA}[0]{\mathbf{A}}
\newcommand{\BB}[0]{\mathbf{B}}
\newcommand{\BC}[0]{\mathbf{C}}
\newcommand{\BD}[0]{\mathbf{D}}
\newcommand{\BE}[0]{\mathbf{E}}
\newcommand{\BF}[0]{\mathbf{F}}
\newcommand{\BG}[0]{\mathbf{G}}
\newcommand{\BH}[0]{\mathbf{H}}
\newcommand{\BI}[0]{\mathbf{I}}
\newcommand{\BJ}[0]{\mathbf{J}}
\newcommand{\BK}[0]{\mathbf{K}}
\newcommand{\BL}[0]{\mathbf{L}}
\newcommand{\BM}[0]{\mathbf{M}}
\newcommand{\BN}[0]{\mathbf{N}}
\newcommand{\BO}[0]{\mathbf{O}}
\newcommand{\BP}[0]{\mathbf{P}}
\newcommand{\BQ}[0]{\mathbf{Q}}
\newcommand{\BR}[0]{\mathbf{R}}
\newcommand{\BS}[0]{\mathbf{S}}
\newcommand{\BT}[0]{\mathbf{T}}
\newcommand{\BU}[0]{\mathbf{U}}
\newcommand{\BV}[0]{\mathbf{V}}
\newcommand{\BW}[0]{\mathbf{W}}
\newcommand{\BX}[0]{\mathbf{X}}
\newcommand{\BY}[0]{\mathbf{Y}}
\newcommand{\BZ}[0]{\mathbf{Z}}

\newcommand{\Bzero}[0]{\mathbf{0}}
\newcommand{\Btheta}[0]{\boldsymbol{\theta}}
\newcommand{\Btau}[0]{\boldsymbol{\tau}}
\newcommand{\Bomega}[0]{\boldsymbol{\omega}}

%
% shorthand for unit vectors
%
\newcommand{\acap}[0]{\hat{\Ba}}
\newcommand{\bcap}[0]{\hat{\Bb}}
\newcommand{\ccap}[0]{\hat{\Bc}}
\newcommand{\dcap}[0]{\hat{\Bd}}
\newcommand{\ecap}[0]{\hat{\Be}}
\newcommand{\fcap}[0]{\hat{\Bf}}
\newcommand{\gcap}[0]{\hat{\Bg}}
\newcommand{\hcap}[0]{\hat{\Bh}}
\newcommand{\icap}[0]{\hat{\Bi}}
\newcommand{\jcap}[0]{\hat{\Bj}}
\newcommand{\kcap}[0]{\hat{\Bk}}
\newcommand{\lcap}[0]{\hat{\Bl}}
\newcommand{\mcap}[0]{\hat{\Bm}}
\newcommand{\ncap}[0]{\hat{\Bn}}
\newcommand{\ocap}[0]{\hat{\Bo}}
\newcommand{\pcap}[0]{\hat{\Bp}}
\newcommand{\qcap}[0]{\hat{\Bq}}
\newcommand{\rcap}[0]{\hat{\Br}}
\newcommand{\scap}[0]{\hat{\Bs}}
\newcommand{\tcap}[0]{\hat{\Bt}}
\newcommand{\ucap}[0]{\hat{\Bu}}
\newcommand{\vcap}[0]{\hat{\Bv}}
\newcommand{\wcap}[0]{\hat{\Bw}}
\newcommand{\xcap}[0]{\hat{\Bx}}
\newcommand{\ycap}[0]{\hat{\By}}
\newcommand{\zcap}[0]{\hat{\Bz}}
\newcommand{\thetacap}[0]{\hat{\Btheta}}

%
% to write R^n and C^n in a distinguishable fashion.  Perhaps change this
% to the double lined characters upon figuring out how to do so.
%
\newcommand{\C}[1]{$\mathbb{C}^{#1}$}
\newcommand{\R}[1]{$\mathbb{R}^{#1}$}

%
% various generally useful helpers
%

% derivative of #1 wrt. #2:
\newcommand{\D}[2] {\frac {d#2} {d#1}}

\newcommand{\inv}[1]{\frac{1}{#1}}
\newcommand{\cross}[0]{\times}

\newcommand{\abs}[1]{\lvert{#1}\rvert}
\newcommand{\norm}[1]{\lVert{#1}\rVert}
\newcommand{\innerprod}[2]{\langle{#1}, {#2}\rangle}
\newcommand{\dotprod}[2]{{#1} \cdot {#2}}
\newcommand{\bdotprod}[2]{\left({#1} \cdot {#2}\right)}
\newcommand{\crossprod}[2]{{#1} \cross {#2}}
\newcommand{\tripleprod}[3]{\dotprod{\left(\crossprod{#1}{#2}\right)}{#3}}

\DeclareMathOperator{\Proj}{Proj}
\DeclareMathOperator{\Span}{span}
\DeclareMathOperator{\Sgn}{sgn}
\DeclareMathOperator{\Area}{Area}
\DeclareMathOperator{\Volume}{Volume}

%
% A few miscellaneous things specific to this document
%
\newcommand{\crossop}[1]{\crossprod{#1}{}}

% R2 vector.
\newcommand{\VectorTwo}[2]{
\begin{bmatrix}
 {#1} \\
 {#2}
\end{bmatrix}
}

\newcommand{\VectorN}[1]{
\begin{bmatrix}
{#1}_1 \\
{#1}_2 \\
\vdots \\
{#1}_N \\
\end{bmatrix}
}

\newcommand{\DETuvij}[4]{
\begin{vmatrix}
 {#1}_{#3} & {#1}_{#4} \\
 {#2}_{#3} & {#2}_{#4}
\end{vmatrix}
}

\newcommand{\DETuvwijk}[6]{
\begin{vmatrix}
 {#1}_{#4} & {#1}_{#5} & {#1}_{#6} \\
 {#2}_{#4} & {#2}_{#5} & {#2}_{#6} \\
 {#3}_{#4} & {#3}_{#5} & {#3}_{#6}
\end{vmatrix}
}

\newcommand{\DETuvwxijkl}[8]{
\begin{vmatrix}
 {#1}_{#5} & {#1}_{#6} & {#1}_{#7} & {#1}_{#8} \\
 {#2}_{#5} & {#2}_{#6} & {#2}_{#7} & {#2}_{#8} \\
 {#3}_{#5} & {#3}_{#6} & {#3}_{#7} & {#3}_{#8} \\
 {#4}_{#5} & {#4}_{#6} & {#4}_{#7} & {#4}_{#8} \\
\end{vmatrix}
}

%\newcommand{\DETuvwxyijklm}[10]{
%\begin{vmatrix}
% {#1}_{#6} & {#1}_{#7} & {#1}_{#8} & {#1}_{#9} & {#1}_{#10} \\
% {#2}_{#6} & {#2}_{#7} & {#2}_{#8} & {#2}_{#9} & {#2}_{#10} \\
% {#3}_{#6} & {#3}_{#7} & {#3}_{#8} & {#3}_{#9} & {#3}_{#10} \\
% {#4}_{#6} & {#4}_{#7} & {#4}_{#8} & {#4}_{#9} & {#4}_{#10} \\
% {#5}_{#6} & {#5}_{#7} & {#5}_{#8} & {#5}_{#9} & {#5}_{#10}
%\end{vmatrix}
%}

% R3 vector.
\newcommand{\VectorThree}[3]{
\begin{bmatrix}
 {#1} \\
 {#2} \\
 {#3}
\end{bmatrix}
}


%<misc>
%
\newcommand{\Abs}[1]{{\left\lvert{#1}\right\rvert}}
\newcommand{\spacegrad}[0]{\boldsymbol{\nabla}}
\newcommand{\grad}[0]{\nabla}
\newcommand{\LL}[0]{\mathcal{L}}

% == \partial_{#1} {#2}
\newcommand{\PD}[2]{\frac{\partial {#2}}{\partial {#1}}}
% inline variant
\newcommand{\PDi}[2]{{\partial {#2}}/{\partial {#1}}}

\newcommand{\PDD}[3]{\frac{\partial^2 {#3}}{\partial {#1}\partial {#2}}}
%\newcommand{\PDd}[2]{\frac{\partial^2 {#2}}{{\partial{#1}}^2}}
\newcommand{\PDsq}[2]{\frac{\partial^2 {#2}}{(\partial {#1})^2}}

\newcommand{\Partial}[2]{\frac{\partial {#1}}{\partial {#2}}}
\DeclareMathOperator{\RejName}{Rej}
\newcommand{\Rej}[2]{\RejName_{#1}\left( {#2} \right)}
\newcommand{\Rm}[1]{\mathbb{R}^{#1}}
\newcommand{\Cm}[1]{\mathbb{C}^{#1}}
\newcommand{\conj}[0]{{*}}

%</misc>

% <grade selection>
%
\newcommand{\gpgrade}[2] {{\left\langle{{#1}}\right\rangle}_{#2}}

\newcommand{\gpgradezero}[1] {\gpgrade{#1}{}}
%\newcommand{\gpscalargrade}[1] {{\left\langle{{#1}}\right\rangle}}
%\newcommand{\gpgradezero}[1] {\gpgrade{#1}{0}}

%\newcommand{\gpgradeone}[1] {{\left\langle{{#1}}\right\rangle}_{1}}
\newcommand{\gpgradeone}[1] {\gpgrade{#1}{1}}

\newcommand{\gpgradetwo}[1] {\gpgrade{#1}{2}}
\newcommand{\gpgradethree}[1] {\gpgrade{#1}{3}}
\newcommand{\gpgradefour}[1] {\gpgrade{#1}{4}}
%
% </grade selection>



\newcommand{\adot}[0]{{\dot{a}}}
\newcommand{\bdot}[0]{{\dot{b}}}
% taken for centered dot:
%\newcommand{\cdot}[0]{{\dot{c}}}
%\newcommand{\ddot}[0]{{\dot{d}}}
\newcommand{\edot}[0]{{\dot{e}}}
\newcommand{\fdot}[0]{{\dot{f}}}
\newcommand{\gdot}[0]{{\dot{g}}}
\newcommand{\hdot}[0]{{\dot{h}}}
\newcommand{\idot}[0]{{\dot{i}}}
\newcommand{\jdot}[0]{{\dot{j}}}
\newcommand{\kdot}[0]{{\dot{k}}}
\newcommand{\ldot}[0]{{\dot{l}}}
\newcommand{\mdot}[0]{{\dot{m}}}
\newcommand{\ndot}[0]{{\dot{n}}}
%\newcommand{\odot}[0]{{\dot{o}}}
\newcommand{\pdot}[0]{{\dot{p}}}
\newcommand{\qdot}[0]{{\dot{q}}}
\newcommand{\rdot}[0]{{\dot{r}}}
\newcommand{\sdot}[0]{{\dot{s}}}
\newcommand{\tdot}[0]{{\dot{t}}}
\newcommand{\udot}[0]{{\dot{u}}}
\newcommand{\vdot}[0]{{\dot{v}}}
\newcommand{\wdot}[0]{{\dot{w}}}
\newcommand{\xdot}[0]{{\dot{x}}}
\newcommand{\ydot}[0]{{\dot{y}}}
\newcommand{\zdot}[0]{{\dot{z}}}
\newcommand{\addot}[0]{{\ddot{a}}}
\newcommand{\bddot}[0]{{\ddot{b}}}
\newcommand{\cddot}[0]{{\ddot{c}}}
%\newcommand{\dddot}[0]{{\ddot{d}}}
\newcommand{\eddot}[0]{{\ddot{e}}}
\newcommand{\fddot}[0]{{\ddot{f}}}
\newcommand{\gddot}[0]{{\ddot{g}}}
\newcommand{\hddot}[0]{{\ddot{h}}}
\newcommand{\iddot}[0]{{\ddot{i}}}
\newcommand{\jddot}[0]{{\ddot{j}}}
\newcommand{\kddot}[0]{{\ddot{k}}}
\newcommand{\lddot}[0]{{\ddot{l}}}
\newcommand{\mddot}[0]{{\ddot{m}}}
\newcommand{\nddot}[0]{{\ddot{n}}}
\newcommand{\oddot}[0]{{\ddot{o}}}
\newcommand{\pddot}[0]{{\ddot{p}}}
\newcommand{\qddot}[0]{{\ddot{q}}}
\newcommand{\rddot}[0]{{\ddot{r}}}
\newcommand{\sddot}[0]{{\ddot{s}}}
\newcommand{\tddot}[0]{{\ddot{t}}}
\newcommand{\uddot}[0]{{\ddot{u}}}
\newcommand{\vddot}[0]{{\ddot{v}}}
\newcommand{\wddot}[0]{{\ddot{w}}}
\newcommand{\xddot}[0]{{\ddot{x}}}
\newcommand{\yddot}[0]{{\ddot{y}}}
\newcommand{\zddot}[0]{{\ddot{z}}}

%<bold and dot greek symbols>
%

\newcommand{\Deltadot}[0]{{\dot{\Delta}}}
\newcommand{\Gammadot}[0]{{\dot{\Gamma}}}
\newcommand{\Lambdadot}[0]{{\dot{\Lambda}}}
\newcommand{\Omegadot}[0]{{\dot{\Omega}}}
\newcommand{\Phidot}[0]{{\dot{\Phi}}}
\newcommand{\Pidot}[0]{{\dot{\Pi}}}
\newcommand{\Psidot}[0]{{\dot{\Psi}}}
\newcommand{\Sigmadot}[0]{{\dot{\Sigma}}}
\newcommand{\Thetadot}[0]{{\dot{\Theta}}}
\newcommand{\Upsilondot}[0]{{\dot{\Upsilon}}}
\newcommand{\Xidot}[0]{{\dot{\Xi}}}
\newcommand{\alphadot}[0]{{\dot{\alpha}}}
\newcommand{\betadot}[0]{{\dot{\beta}}}
\newcommand{\chidot}[0]{{\dot{\chi}}}
\newcommand{\deltadot}[0]{{\dot{\delta}}}
\newcommand{\epsilondot}[0]{{\dot{\epsilon}}}
\newcommand{\etadot}[0]{{\dot{\eta}}}
\newcommand{\gammadot}[0]{{\dot{\gamma}}}
\newcommand{\kappadot}[0]{{\dot{\kappa}}}
\newcommand{\lambdadot}[0]{{\dot{\lambda}}}
\newcommand{\mudot}[0]{{\dot{\mu}}}
\newcommand{\nudot}[0]{{\dot{\nu}}}
\newcommand{\omegadot}[0]{{\dot{\omega}}}
\newcommand{\phidot}[0]{{\dot{\phi}}}
\newcommand{\pidot}[0]{{\dot{\pi}}}
\newcommand{\psidot}[0]{{\dot{\psi}}}
\newcommand{\rhodot}[0]{{\dot{\rho}}}
\newcommand{\sigmadot}[0]{{\dot{\sigma}}}
\newcommand{\taudot}[0]{{\dot{\tau}}}
\newcommand{\thetadot}[0]{{\dot{\theta}}}
\newcommand{\upsilondot}[0]{{\dot{\upsilon}}}
\newcommand{\varepsilondot}[0]{{\dot{\varepsilon}}}
\newcommand{\varphidot}[0]{{\dot{\varphi}}}
\newcommand{\varpidot}[0]{{\dot{\varpi}}}
\newcommand{\varrhodot}[0]{{\dot{\varrho}}}
\newcommand{\varsigmadot}[0]{{\dot{\varsigma}}}
\newcommand{\varthetadot}[0]{{\dot{\vartheta}}}
\newcommand{\xidot}[0]{{\dot{\xi}}}
\newcommand{\zetadot}[0]{{\dot{\zeta}}}

\newcommand{\Deltaddot}[0]{{\ddot{\Delta}}}
\newcommand{\Gammaddot}[0]{{\ddot{\Gamma}}}
\newcommand{\Lambdaddot}[0]{{\ddot{\Lambda}}}
\newcommand{\Omegaddot}[0]{{\ddot{\Omega}}}
\newcommand{\Phiddot}[0]{{\ddot{\Phi}}}
\newcommand{\Piddot}[0]{{\ddot{\Pi}}}
\newcommand{\Psiddot}[0]{{\ddot{\Psi}}}
\newcommand{\Sigmaddot}[0]{{\ddot{\Sigma}}}
\newcommand{\Thetaddot}[0]{{\ddot{\Theta}}}
\newcommand{\Upsilonddot}[0]{{\ddot{\Upsilon}}}
\newcommand{\Xiddot}[0]{{\ddot{\Xi}}}
\newcommand{\alphaddot}[0]{{\ddot{\alpha}}}
\newcommand{\betaddot}[0]{{\ddot{\beta}}}
\newcommand{\chiddot}[0]{{\ddot{\chi}}}
\newcommand{\deltaddot}[0]{{\ddot{\delta}}}
\newcommand{\epsilonddot}[0]{{\ddot{\epsilon}}}
\newcommand{\etaddot}[0]{{\ddot{\eta}}}
\newcommand{\gammaddot}[0]{{\ddot{\gamma}}}
\newcommand{\kappaddot}[0]{{\ddot{\kappa}}}
\newcommand{\lambdaddot}[0]{{\ddot{\lambda}}}
\newcommand{\muddot}[0]{{\ddot{\mu}}}
\newcommand{\nuddot}[0]{{\ddot{\nu}}}
\newcommand{\omegaddot}[0]{{\ddot{\omega}}}
\newcommand{\phiddot}[0]{{\ddot{\phi}}}
\newcommand{\piddot}[0]{{\ddot{\pi}}}
\newcommand{\psiddot}[0]{{\ddot{\psi}}}
\newcommand{\rhoddot}[0]{{\ddot{\rho}}}
\newcommand{\sigmaddot}[0]{{\ddot{\sigma}}}
\newcommand{\tauddot}[0]{{\ddot{\tau}}}
\newcommand{\thetaddot}[0]{{\ddot{\theta}}}
\newcommand{\upsilonddot}[0]{{\ddot{\upsilon}}}
\newcommand{\varepsilonddot}[0]{{\ddot{\varepsilon}}}
\newcommand{\varphiddot}[0]{{\ddot{\varphi}}}
\newcommand{\varpiddot}[0]{{\ddot{\varpi}}}
\newcommand{\varrhoddot}[0]{{\ddot{\varrho}}}
\newcommand{\varsigmaddot}[0]{{\ddot{\varsigma}}}
\newcommand{\varthetaddot}[0]{{\ddot{\vartheta}}}
\newcommand{\xiddot}[0]{{\ddot{\xi}}}
\newcommand{\zetaddot}[0]{{\ddot{\zeta}}}

\newcommand{\BDelta}[0]{\boldsymbol{\Delta}}
\newcommand{\BGamma}[0]{\boldsymbol{\Gamma}}
\newcommand{\BLambda}[0]{\boldsymbol{\Lambda}}
\newcommand{\BOmega}[0]{\boldsymbol{\Omega}}
\newcommand{\BPhi}[0]{\boldsymbol{\Phi}}
\newcommand{\BPi}[0]{\boldsymbol{\Pi}}
\newcommand{\BPsi}[0]{\boldsymbol{\Psi}}
\newcommand{\BSigma}[0]{\boldsymbol{\Sigma}}
\newcommand{\BTheta}[0]{\boldsymbol{\Theta}}
\newcommand{\BUpsilon}[0]{\boldsymbol{\Upsilon}}
\newcommand{\BXi}[0]{\boldsymbol{\Xi}}
\newcommand{\Balpha}[0]{\boldsymbol{\alpha}}
\newcommand{\Bbeta}[0]{\boldsymbol{\beta}}
\newcommand{\Bchi}[0]{\boldsymbol{\chi}}
\newcommand{\Bdelta}[0]{\boldsymbol{\delta}}
\newcommand{\Bepsilon}[0]{\boldsymbol{\epsilon}}
\newcommand{\Beta}[0]{\boldsymbol{\eta}}
\newcommand{\Bgamma}[0]{\boldsymbol{\gamma}}
\newcommand{\Bkappa}[0]{\boldsymbol{\kappa}}
\newcommand{\Blambda}[0]{\boldsymbol{\lambda}}
\newcommand{\Bmu}[0]{\boldsymbol{\mu}}
\newcommand{\Bnu}[0]{\boldsymbol{\nu}}
%\newcommand{\Bomega}[0]{\boldsymbol{\omega}}
\newcommand{\Bphi}[0]{\boldsymbol{\phi}}
\newcommand{\Bpi}[0]{\boldsymbol{\pi}}
\newcommand{\Bpsi}[0]{\boldsymbol{\psi}}
\newcommand{\Brho}[0]{\boldsymbol{\rho}}
\newcommand{\Bsigma}[0]{\boldsymbol{\sigma}}
%\newcommand{\Btau}[0]{\boldsymbol{\tau}}
%\newcommand{\Btheta}[0]{\boldsymbol{\theta}}
\newcommand{\Bupsilon}[0]{\boldsymbol{\upsilon}}
\newcommand{\Bvarepsilon}[0]{\boldsymbol{\varepsilon}}
\newcommand{\Bvarphi}[0]{\boldsymbol{\varphi}}
\newcommand{\Bvarpi}[0]{\boldsymbol{\varpi}}
\newcommand{\Bvarrho}[0]{\boldsymbol{\varrho}}
\newcommand{\Bvarsigma}[0]{\boldsymbol{\varsigma}}
\newcommand{\Bvartheta}[0]{\boldsymbol{\vartheta}}
\newcommand{\Bxi}[0]{\boldsymbol{\xi}}
\newcommand{\Bzeta}[0]{\boldsymbol{\zeta}}
%
%</bold and dot greek symbols>
%<infrequent>
%
%\newcommand{\AreaOp}[1]{\AName_{#1}}
%\newcommand{\Babs}[0]{\abs{\BB}}
%\newcommand{\Bcap}[0]{\hat{\BB}}
%\newcommand{\BrPrimeRej}[0]{\rcap(\rcap \wedge \Br')}
%\newcommand{\CA}[0]{\mathcal{A}}
%\newcommand{\Cos}[1]{\cos{\left({#1}\right)}}
%\newcommand{\Det}[1] {\abs{#1}}
%\newcommand{\Dsq}[2] {\frac {\partial^2 {#1}} {\partial {#2}^2}}
%\newcommand{\Exp}[1]{\exp{\left({#1}\right)}}
%\newcommand{\Norm}[1]{\left\lVert{#1}\right\rVert}
%\newcommand{\Sin}[1]{\sin{\left({#1}\right)}}
%\newcommand{\T}[0]{\text{T}}
%\newcommand{\VolumeOp}[1]{\VName_{#1}}
%\newcommand{\agrad}[0]{\Ba \cdot \nabla}
%\newcommand{\alphacap}[0]{\hat{\boldsymbol{\alpha}}}
%\newcommand{\Fcap}[0]{\hat{\BF}}
%\newcommand{\bithree}[0]{{\Bi}_3}
%\newcommand{\bxa}[0]{\Bx\Ba}
%\newcommand{\coordvec}[2]{
%\newcommand{\costheta}[0]{\acap \cdot \xcap}
%\newcommand{\ddt}[1]{\ddot{#1}}
%\newcommand{\ddu}[1] {\frac {d{#1}} {du}}
%\newcommand{\dsqxj}[2] {\frac {\partial^2 {#1}} {\partial {x_{#2}}^2}}
%\newcommand{\dtheta}[1]{\frac{d {#1}}{d \theta}}
%\newcommand{\dt}[1]{\dot{#1}}
%\newcommand{\dt}[1]{\frac{d {#1}}{dt}}
%\newcommand{\dxj}[2] {\frac {\partial {#1}} {\partial {x_{#2}}}}
%\newcommand{\halfPhi}[0]{\frac{\phi}{2}}
%\newcommand{\half}[0]{\inv{2}}
%\newcommand{\inv}[1]{\frac{1}{#1}}
%\newcommand{\laplacian}[0]{\nabla^2}
%\newcommand{\matrixoftx}[3]{
%\newcommand{\nrrp}[0]{\norm{\rcap \wedge \Br'}}
%\newcommand{\oiint}{\bigcirc \hspace{-1.4em} \int \hspace{-.8em} \int}
%\newcommand{\transpose}[1]{{#1}^{\text{T}}}
%\newcommand{\transpose}[1]{{{#1}^{\TextTranspose}}}
%\newcommand{\transpose}[1]{{{#1}^{\text{T}}}}
%\newcommand{\barA}[0]{\bar{A}}
%\newcommand{\qbar}[0]{\bar{q}}
%\newcommand{\qdotbar}[0]{\dot{\bar{q}}}
%
%</infrequent>





%\usepackage{listings}
%\usepackage{txfonts} % for ointctr... (also appears to make "prettier" \int and \sum's)
\usepackage[bookmarks=true]{hyperref}

%\usepackage{color,cite,graphicx}
   % use colour in the document, put your citations as [1-4]
   % rather than [1,2,3,4] (it looks nicer, and the extended LaTeX2e
   % graphics package. 
\usepackage{latexsym,amssymb,epsf} % don't remember if these are
   % needed, but their inclusion can't do any damage


%\title{}
%\author{Peeter Joot \quad peeter.joot@gmail.com }
%\date{ March dd, 2009.  Last Revision: $Date: 2009/06/03 20:01:38 $ }

\begin{document}
%\maketitle{}
%\tableofcontents
%\section{}

Your equation looks something like classical rigid body motion to me. ie: 
If the position of a particle is $x$ in a frame moving along a $\lambda$ parametrized path $a(\lambda)$, and that
frame is also allowed to rotate one could write something like the following for the
composite position of the particle

\begin{align*}
\bar{x} = R(x) + a(\lambda)
\end{align*}

Or in coordinates
\begin{align*}
\bar{x^\mu} = \Lambda^\mu_\nu x^\nu + a^\mu
\end{align*}

Derivatives with respect to the parameter (which I am thinking of as time or proper time), are then

\begin{align*}
\frac{d\bar{x^\mu}}{d\lambda} 
&= 
\frac{d \Lambda^\mu_\nu}{d\lambda} x^\nu 
+ \Lambda^\mu_\nu \frac{d x^\nu}{d\lambda}
+ \frac{d a^\mu}{d\lambda}
\end{align*}

As a differential displacement, one could write

\begin{align*}
d\bar{x^\mu}
&= x^\mu(\lambda + \delta\lambda) - x^\mu(\lambda) \\
&= 
d\lambda \left(
\frac{d \Lambda^\mu_\nu}{d\lambda} x^\nu 
+ \Lambda^\mu_\nu \xdot^\nu
+ \adot^\mu
\right)
\end{align*}

Or
\begin{align}\label{eqn:part1}
\bar{x}^\mu(\lambda + \delta\lambda) 
&= \bar{x}^\mu(\lambda)
+ d\lambda \frac{d \Lambda^\mu_\nu}{d\lambda} x^\nu 
+ d\lambda \Lambda^\mu_\nu \xdot^\nu
+ d\lambda \adot^\mu
\end{align}

This gives me a couple more terms than in your email
and also requires a different identification of the positions in the two frames (ie: $\bar{x}$, and $x$ vectors).

In particular, I have a variation of the rotation along the worldline $\delta \Lambda^\mu_\nu = d\lambda d\Lambda^\mu_\nu/d\lambda$ that is missing
in your equation.  Does GR give you a constraint that allows this to vanish?

Right before sending the email with this, it occurs to me that the body frame coordinates can be eliminated.  Let the inverse transformation be given by $\Pi^\sigma_\mu$ as follows

\begin{align*}
\Pi^\sigma_\mu \Lambda^\mu_\nu = \delta^\sigma_\nu
\end{align*}

Then one can write
\begin{align*}
\Pi^\nu_\alpha(\bar{x^\alpha} - a^\alpha)
&= \Pi^\nu_\alpha \Lambda^\alpha_\beta x^\beta  \\
&= x^\nu  \\
\end{align*}

and \ref{eqn:part1} becomes

\begin{align*}
\bar{x}^\mu(\lambda + \delta\lambda) 
&= 
\bar{x}^\mu(\lambda)
+ d\lambda \frac{d \Lambda^\mu_\nu}{d\lambda} \left( \Pi^\nu_\alpha(\bar{x^\alpha} - a^\alpha) \right)
+ d\lambda \Lambda^\mu_\nu \xdot^\nu
+ d\lambda \adot^\mu \\
\end{align*}

\begin{align}\label{eqn:partII}
\bar{x}^\mu(\lambda + \delta\lambda) 
&= 
\bar{x}^\alpha(\lambda) \left( \delta^\mu_\alpha + d\lambda \Pi^\nu_\alpha \frac{d \Lambda^\mu_\nu}{d\lambda} \right)
- d\lambda \frac{d \Lambda^\mu_\nu}{d\lambda} a^\alpha
+ d\lambda \Lambda^\mu_\nu \xdot^\nu
+ d\lambda \adot^\mu \\
\end{align}

Note that since $\Pi^\sigma_\mu \Lambda^\mu_\nu = \delta^\sigma_\nu$, one has

\begin{align*}
0 
&= \frac{d}{d\lambda} \left( \Pi^\sigma_\mu \Lambda^\mu_\nu \right) \\
&= \Lambda^\mu_\nu \frac{d}{d\lambda} \Pi^\sigma_\mu + \Pi^\sigma_\mu \frac{d}{d\lambda} \Lambda^\mu_\nu 
\end{align*}

Believe this implies that the contracted inverse/derivative product $\Pi^\nu_\alpha \frac{d \Lambda^\mu_\nu}{d\lambda}$ is antisymmetric.

There's still mixed coordinates in \ref{eqn:partII} (ie: both $\bar{x}$, and $x$) and I think that the final result for an incremental displacement
should likely eliminate that too (but I haven't tried).

%\begin{figure}[htp]
%\centering
%\includegraphics[totalheight=0.4\textheight]{picturepath}
%\caption{My Caption}\label{fig:pictlabel}
%\end{figure}
%
%... see figure \ref{fig:picturepath} ...

%\bibliographystyle{plainnat}
%\bibliography{myrefs}

\end{document}


\part{maxwell}
\documentclass{article}      % Specifies the document class

\usepackage{amsmath}
\newcommand{\abs}[1]{\lvert#1\rvert}
\newcommand{\norm}[1]{\lVert#1\rVert}
\newcommand{\grad}[1]{\nabla#1}
\newcommand{\curl}[1]{\nabla \times #1}
\newcommand{\Curl}[1]{\nabla \times \mathbf{#1}}
\newcommand{\diverg}[1]{\nabla \cdot #1}
\newcommand{\Diverg}[1]{\nabla \cdot \mathbf{#1}}
\newcommand{\curlcurl}[1]{\curl\curl{#1}}
\newcommand{\curlCurl}[1]{\curl\Curl{#1}}
\newcommand{\delsquared}[1]{\nabla^2{#1}}
\newcommand{\Delsquared}[1]{{\nabla^2}{\mathbf{#1}}}
\newcommand{\ddt}[1]{ {{\partial{#1}} \over {\partial{t}}}}
\newcommand{\Ddt}[1]{ {{\partial{\mathbf{#1}}} \over {\partial{t}}}}
\newcommand{\ddts}[1]{ {{\partial^2{#1}} \over {\partial{t}^2}}}
\newcommand{\Ddts}[1]{ {{\partial^2{\mathbf{#1}}} \over {\partial{t}^2}}}
\newcommand{\Bj}[0]{\mathbf{j}}
\newcommand{\BB}[0]{\mathbf{B}}
\newcommand{\BE}[0]{\mathbf{E}}

                             % The preamble begins here.
\title{An Example Document}  % Declares the document's title.
\author{Peeter Joot}         % Declares the author's name.
\date{March 25, 2000}        % Deleting this command produces today's date.

\begin{document}             % End of preamble and beginning of text.

%\maketitle                  % Produces the title.

\section{various formulations of Maxwell's equations}

It is interesting to look at the various formulations of Maxwell's 
equations.  The standard formulation these days is the differential
form, which isn't the most intuitive form.  In cgs units, the differential
form is as follows:

\begin{equation*}
\Diverg E = 4\pi\rho
\end{equation*}
\begin{equation*}
\Diverg B = 0
\end{equation*}
\begin{equation*}
\Curl{B} = 4\pi \Bj - {1 \over c} \Ddt{E}
\end{equation*}
\begin{equation*}
\Curl{E} = {1 \over c} \Ddt{B}
\end{equation*}

It is interesting to note that these are probably not the form that Maxwell 
originally formulated his equations in.  In my 1960 version of encyclodedia 
britianica Maxwell's equations were given in a differential form.  I had 
trouble seeing how these two forms were related at first, but the relation
is via the standard integral transformations.  For example, the formula $\Diverg E = 4\pi\rho$ is an alternate formulation of Gauss' law
\begin{equation*}
\BE = \int_V{{\rho\,dV' \over \norm{\mathbf r - \mathbf r'}}}
\end{equation*}

This can be shown via application of Gauss's theorm.  Similarily the following 
integral forms of Maxwell's equations:

%[ see paper notes ].

One of the awkward things with this standard formulation is that there 
is an awkward interdependence between the electric and magnetic fields, and 
the solution of either $\BE$ or $\BB$ interdependently seems difficult.

The interdependence of the $\BE$ and $\BB$ field equations can be removed by taking the
curl of each of the curl equations, and using the fact that:

$\curlCurl{V} = \grad (\Diverg V) - \Delsquared{V}$

\begin{align*}
\curlCurl{B} &= \grad (\Diverg{B}) - \Delsquared{B} \\
\curl(4\pi\Bj - {1 \over c} \Ddt{E}) &= \\
4\pi \Curl{j} - {1 \over c^2} \Ddts{B} &= \\
      	     &= 		   - \Delsquared{B}
\end{align*}

We end up with a single differential equation for the magnetic field $\BB$.

$\delsquared{\BB} - {1 \over c^2} \Ddts{B} = - 4\pi \Curl{j}$

We can do the same calculation for the electric field.

\begin{align*}
\curlCurl{E} &= \grad (\Diverg{E}) - \Delsquared{E} \\
\curl({1 \over c} \Ddt{B}) &= \\
{1 \over c} \ddt(4\pi \Bj - {1 \over c} \Ddt{E}) &= \\
{4\pi \over c} \Ddt{j} - {1 \over c^2} \Ddts{E} &= \\
             &= \grad (4\pi\rho) - \Delsquared{E} \\
\end{align*}

which gives a single differential equation for the electric field $\BE$.

%$\Delsquared{E} - {1 \over c^2} \Ddt{E} = {4\pi \over c} \Ddt{j} - 4\pi \grad{\rho}$
$\Delsquared{E} - {1 \over c^2} \Ddt{E}$
$ = {4\pi \over c} \Ddt{j} - 4\pi \grad{\rho}$

In the absense of current density and charge these equations take the simple form of standard wave
equations.

\begin{equation*}
\Delsquared{E} - {1 \over c^2} \Ddt{E} = 0
\end{equation*}
\begin{equation*}
\Delsquared{B} - {1 \over c^2} \Ddt{B} = 0
\end{equation*}

Any solution to these is also a solution to the original form where there is current density $\Bj$, and 
charge density $\rho$ since 

\end{document}               % End of document.


\part{mechanics}
\include{goldstein_ch1_2}

% sub-sort by date
\part{Quantum Mechanics.}
\include{bohm_ch9}
\include{bohm_ch10}
\documentclass{article}

\usepackage{amsmath}
\usepackage{mathpazo}

%
% shorthand for bold symbols, convenient for vectors and matrices
%
\newcommand{\Ba}[0]{\mathbf{a}}
\newcommand{\Bb}[0]{\mathbf{b}}
\newcommand{\Bc}[0]{\mathbf{c}}
\newcommand{\Bd}[0]{\mathbf{d}}
\newcommand{\Be}[0]{\mathbf{e}}
\newcommand{\Bf}[0]{\mathbf{f}}
\newcommand{\Bg}[0]{\mathbf{g}}
\newcommand{\Bh}[0]{\mathbf{h}}
\newcommand{\Bi}[0]{\mathbf{i}}
\newcommand{\Bj}[0]{\mathbf{j}}
\newcommand{\Bk}[0]{\mathbf{k}}
\newcommand{\Bl}[0]{\mathbf{l}}
\newcommand{\Bm}[0]{\mathbf{m}}
\newcommand{\Bn}[0]{\mathbf{n}}
\newcommand{\Bo}[0]{\mathbf{o}}
\newcommand{\Bp}[0]{\mathbf{p}}
\newcommand{\Bq}[0]{\mathbf{q}}
\newcommand{\Br}[0]{\mathbf{r}}
\newcommand{\Bs}[0]{\mathbf{s}}
\newcommand{\Bt}[0]{\mathbf{t}}
\newcommand{\Bu}[0]{\mathbf{u}}
\newcommand{\Bv}[0]{\mathbf{v}}
\newcommand{\Bw}[0]{\mathbf{w}}
\newcommand{\Bx}[0]{\mathbf{x}}
\newcommand{\By}[0]{\mathbf{y}}
\newcommand{\Bz}[0]{\mathbf{z}}
\newcommand{\BA}[0]{\mathbf{A}}
\newcommand{\BB}[0]{\mathbf{B}}
\newcommand{\BC}[0]{\mathbf{C}}
\newcommand{\BD}[0]{\mathbf{D}}
\newcommand{\BE}[0]{\mathbf{E}}
\newcommand{\BF}[0]{\mathbf{F}}
\newcommand{\BG}[0]{\mathbf{G}}
\newcommand{\BH}[0]{\mathbf{H}}
\newcommand{\BI}[0]{\mathbf{I}}
\newcommand{\BJ}[0]{\mathbf{J}}
\newcommand{\BK}[0]{\mathbf{K}}
\newcommand{\BL}[0]{\mathbf{L}}
\newcommand{\BM}[0]{\mathbf{M}}
\newcommand{\BN}[0]{\mathbf{N}}
\newcommand{\BO}[0]{\mathbf{O}}
\newcommand{\BP}[0]{\mathbf{P}}
\newcommand{\BQ}[0]{\mathbf{Q}}
\newcommand{\BR}[0]{\mathbf{R}}
\newcommand{\BS}[0]{\mathbf{S}}
\newcommand{\BT}[0]{\mathbf{T}}
\newcommand{\BU}[0]{\mathbf{U}}
\newcommand{\BV}[0]{\mathbf{V}}
\newcommand{\BW}[0]{\mathbf{W}}
\newcommand{\BX}[0]{\mathbf{X}}
\newcommand{\BY}[0]{\mathbf{Y}}
\newcommand{\BZ}[0]{\mathbf{Z}}

\newcommand{\Bzero}[0]{\mathbf{0}}
\newcommand{\Btheta}[0]{\boldsymbol{\theta}}
\newcommand{\Btau}[0]{\boldsymbol{\tau}}
\newcommand{\Bomega}[0]{\boldsymbol{\omega}}

%
% shorthand for unit vectors
%
\newcommand{\acap}[0]{\hat{\Ba}}
\newcommand{\bcap}[0]{\hat{\Bb}}
\newcommand{\ccap}[0]{\hat{\Bc}}
\newcommand{\dcap}[0]{\hat{\Bd}}
\newcommand{\ecap}[0]{\hat{\Be}}
\newcommand{\fcap}[0]{\hat{\Bf}}
\newcommand{\gcap}[0]{\hat{\Bg}}
\newcommand{\hcap}[0]{\hat{\Bh}}
\newcommand{\icap}[0]{\hat{\Bi}}
\newcommand{\jcap}[0]{\hat{\Bj}}
\newcommand{\kcap}[0]{\hat{\Bk}}
\newcommand{\lcap}[0]{\hat{\Bl}}
\newcommand{\mcap}[0]{\hat{\Bm}}
\newcommand{\ncap}[0]{\hat{\Bn}}
\newcommand{\ocap}[0]{\hat{\Bo}}
\newcommand{\pcap}[0]{\hat{\Bp}}
\newcommand{\qcap}[0]{\hat{\Bq}}
\newcommand{\rcap}[0]{\hat{\Br}}
\newcommand{\scap}[0]{\hat{\Bs}}
\newcommand{\tcap}[0]{\hat{\Bt}}
\newcommand{\ucap}[0]{\hat{\Bu}}
\newcommand{\vcap}[0]{\hat{\Bv}}
\newcommand{\wcap}[0]{\hat{\Bw}}
\newcommand{\xcap}[0]{\hat{\Bx}}
\newcommand{\ycap}[0]{\hat{\By}}
\newcommand{\zcap}[0]{\hat{\Bz}}
\newcommand{\thetacap}[0]{\hat{\Btheta}}

%
% to write R^n and C^n in a distinguishable fashion.  Perhaps change this
% to the double lined characters upon figuring out how to do so.
%
\newcommand{\C}[1]{$\mathbb{C}^{#1}$}
\newcommand{\R}[1]{$\mathbb{R}^{#1}$}

%
% various generally useful helpers
%

% derivative of #1 wrt. #2:
\newcommand{\D}[2] {\frac {d#2} {d#1}}

\newcommand{\inv}[1]{\frac{1}{#1}}
\newcommand{\cross}[0]{\times}

\newcommand{\abs}[1]{\lvert{#1}\rvert}
\newcommand{\norm}[1]{\lVert{#1}\rVert}
\newcommand{\innerprod}[2]{\langle{#1}, {#2}\rangle}
\newcommand{\dotprod}[2]{{#1} \cdot {#2}}
\newcommand{\bdotprod}[2]{\left({#1} \cdot {#2}\right)}
\newcommand{\crossprod}[2]{{#1} \cross {#2}}
\newcommand{\tripleprod}[3]{\dotprod{\left(\crossprod{#1}{#2}\right)}{#3}}

\DeclareMathOperator{\Proj}{Proj}
\DeclareMathOperator{\Span}{span}
\DeclareMathOperator{\Sgn}{sgn}
\DeclareMathOperator{\Area}{Area}
\DeclareMathOperator{\Volume}{Volume}

%
% A few miscellaneous things specific to this document
%
\newcommand{\crossop}[1]{\crossprod{#1}{}}

% R2 vector.
\newcommand{\VectorTwo}[2]{
\begin{bmatrix}
 {#1} \\
 {#2}
\end{bmatrix}
}

\newcommand{\VectorN}[1]{
\begin{bmatrix}
{#1}_1 \\
{#1}_2 \\
\vdots \\
{#1}_N \\
\end{bmatrix}
}

\newcommand{\DETuvij}[4]{
\begin{vmatrix}
 {#1}_{#3} & {#1}_{#4} \\
 {#2}_{#3} & {#2}_{#4}
\end{vmatrix}
}

\newcommand{\DETuvwijk}[6]{
\begin{vmatrix}
 {#1}_{#4} & {#1}_{#5} & {#1}_{#6} \\
 {#2}_{#4} & {#2}_{#5} & {#2}_{#6} \\
 {#3}_{#4} & {#3}_{#5} & {#3}_{#6}
\end{vmatrix}
}

\newcommand{\DETuvwxijkl}[8]{
\begin{vmatrix}
 {#1}_{#5} & {#1}_{#6} & {#1}_{#7} & {#1}_{#8} \\
 {#2}_{#5} & {#2}_{#6} & {#2}_{#7} & {#2}_{#8} \\
 {#3}_{#5} & {#3}_{#6} & {#3}_{#7} & {#3}_{#8} \\
 {#4}_{#5} & {#4}_{#6} & {#4}_{#7} & {#4}_{#8} \\
\end{vmatrix}
}

%\newcommand{\DETuvwxyijklm}[10]{
%\begin{vmatrix}
% {#1}_{#6} & {#1}_{#7} & {#1}_{#8} & {#1}_{#9} & {#1}_{#10} \\
% {#2}_{#6} & {#2}_{#7} & {#2}_{#8} & {#2}_{#9} & {#2}_{#10} \\
% {#3}_{#6} & {#3}_{#7} & {#3}_{#8} & {#3}_{#9} & {#3}_{#10} \\
% {#4}_{#6} & {#4}_{#7} & {#4}_{#8} & {#4}_{#9} & {#4}_{#10} \\
% {#5}_{#6} & {#5}_{#7} & {#5}_{#8} & {#5}_{#9} & {#5}_{#10}
%\end{vmatrix}
%}

% R3 vector.
\newcommand{\VectorThree}[3]{
\begin{bmatrix}
 {#1} \\
 {#2} \\
 {#3}
\end{bmatrix}
}


%<misc>
%
\newcommand{\Abs}[1]{{\left\lvert{#1}\right\rvert}}
\newcommand{\spacegrad}[0]{\boldsymbol{\nabla}}
\newcommand{\grad}[0]{\nabla}
\newcommand{\LL}[0]{\mathcal{L}}

% == \partial_{#1} {#2}
\newcommand{\PD}[2]{\frac{\partial {#2}}{\partial {#1}}}
% inline variant
\newcommand{\PDi}[2]{{\partial {#2}}/{\partial {#1}}}

\newcommand{\PDD}[3]{\frac{\partial^2 {#3}}{\partial {#1}\partial {#2}}}
%\newcommand{\PDd}[2]{\frac{\partial^2 {#2}}{{\partial{#1}}^2}}
\newcommand{\PDsq}[2]{\frac{\partial^2 {#2}}{(\partial {#1})^2}}

\newcommand{\Partial}[2]{\frac{\partial {#1}}{\partial {#2}}}
\DeclareMathOperator{\RejName}{Rej}
\newcommand{\Rej}[2]{\RejName_{#1}\left( {#2} \right)}
\newcommand{\Rm}[1]{\mathbb{R}^{#1}}
\newcommand{\Cm}[1]{\mathbb{C}^{#1}}
\newcommand{\conj}[0]{{*}}

%</misc>

% <grade selection>
%
\newcommand{\gpgrade}[2] {{\left\langle{{#1}}\right\rangle}_{#2}}

\newcommand{\gpgradezero}[1] {\gpgrade{#1}{}}
%\newcommand{\gpscalargrade}[1] {{\left\langle{{#1}}\right\rangle}}
%\newcommand{\gpgradezero}[1] {\gpgrade{#1}{0}}

%\newcommand{\gpgradeone}[1] {{\left\langle{{#1}}\right\rangle}_{1}}
\newcommand{\gpgradeone}[1] {\gpgrade{#1}{1}}

\newcommand{\gpgradetwo}[1] {\gpgrade{#1}{2}}
\newcommand{\gpgradethree}[1] {\gpgrade{#1}{3}}
\newcommand{\gpgradefour}[1] {\gpgrade{#1}{4}}
%
% </grade selection>



\newcommand{\adot}[0]{{\dot{a}}}
\newcommand{\bdot}[0]{{\dot{b}}}
% taken for centered dot:
%\newcommand{\cdot}[0]{{\dot{c}}}
%\newcommand{\ddot}[0]{{\dot{d}}}
\newcommand{\edot}[0]{{\dot{e}}}
\newcommand{\fdot}[0]{{\dot{f}}}
\newcommand{\gdot}[0]{{\dot{g}}}
\newcommand{\hdot}[0]{{\dot{h}}}
\newcommand{\idot}[0]{{\dot{i}}}
\newcommand{\jdot}[0]{{\dot{j}}}
\newcommand{\kdot}[0]{{\dot{k}}}
\newcommand{\ldot}[0]{{\dot{l}}}
\newcommand{\mdot}[0]{{\dot{m}}}
\newcommand{\ndot}[0]{{\dot{n}}}
%\newcommand{\odot}[0]{{\dot{o}}}
\newcommand{\pdot}[0]{{\dot{p}}}
\newcommand{\qdot}[0]{{\dot{q}}}
\newcommand{\rdot}[0]{{\dot{r}}}
\newcommand{\sdot}[0]{{\dot{s}}}
\newcommand{\tdot}[0]{{\dot{t}}}
\newcommand{\udot}[0]{{\dot{u}}}
\newcommand{\vdot}[0]{{\dot{v}}}
\newcommand{\wdot}[0]{{\dot{w}}}
\newcommand{\xdot}[0]{{\dot{x}}}
\newcommand{\ydot}[0]{{\dot{y}}}
\newcommand{\zdot}[0]{{\dot{z}}}
\newcommand{\addot}[0]{{\ddot{a}}}
\newcommand{\bddot}[0]{{\ddot{b}}}
\newcommand{\cddot}[0]{{\ddot{c}}}
%\newcommand{\dddot}[0]{{\ddot{d}}}
\newcommand{\eddot}[0]{{\ddot{e}}}
\newcommand{\fddot}[0]{{\ddot{f}}}
\newcommand{\gddot}[0]{{\ddot{g}}}
\newcommand{\hddot}[0]{{\ddot{h}}}
\newcommand{\iddot}[0]{{\ddot{i}}}
\newcommand{\jddot}[0]{{\ddot{j}}}
\newcommand{\kddot}[0]{{\ddot{k}}}
\newcommand{\lddot}[0]{{\ddot{l}}}
\newcommand{\mddot}[0]{{\ddot{m}}}
\newcommand{\nddot}[0]{{\ddot{n}}}
\newcommand{\oddot}[0]{{\ddot{o}}}
\newcommand{\pddot}[0]{{\ddot{p}}}
\newcommand{\qddot}[0]{{\ddot{q}}}
\newcommand{\rddot}[0]{{\ddot{r}}}
\newcommand{\sddot}[0]{{\ddot{s}}}
\newcommand{\tddot}[0]{{\ddot{t}}}
\newcommand{\uddot}[0]{{\ddot{u}}}
\newcommand{\vddot}[0]{{\ddot{v}}}
\newcommand{\wddot}[0]{{\ddot{w}}}
\newcommand{\xddot}[0]{{\ddot{x}}}
\newcommand{\yddot}[0]{{\ddot{y}}}
\newcommand{\zddot}[0]{{\ddot{z}}}

%<bold and dot greek symbols>
%

\newcommand{\Deltadot}[0]{{\dot{\Delta}}}
\newcommand{\Gammadot}[0]{{\dot{\Gamma}}}
\newcommand{\Lambdadot}[0]{{\dot{\Lambda}}}
\newcommand{\Omegadot}[0]{{\dot{\Omega}}}
\newcommand{\Phidot}[0]{{\dot{\Phi}}}
\newcommand{\Pidot}[0]{{\dot{\Pi}}}
\newcommand{\Psidot}[0]{{\dot{\Psi}}}
\newcommand{\Sigmadot}[0]{{\dot{\Sigma}}}
\newcommand{\Thetadot}[0]{{\dot{\Theta}}}
\newcommand{\Upsilondot}[0]{{\dot{\Upsilon}}}
\newcommand{\Xidot}[0]{{\dot{\Xi}}}
\newcommand{\alphadot}[0]{{\dot{\alpha}}}
\newcommand{\betadot}[0]{{\dot{\beta}}}
\newcommand{\chidot}[0]{{\dot{\chi}}}
\newcommand{\deltadot}[0]{{\dot{\delta}}}
\newcommand{\epsilondot}[0]{{\dot{\epsilon}}}
\newcommand{\etadot}[0]{{\dot{\eta}}}
\newcommand{\gammadot}[0]{{\dot{\gamma}}}
\newcommand{\kappadot}[0]{{\dot{\kappa}}}
\newcommand{\lambdadot}[0]{{\dot{\lambda}}}
\newcommand{\mudot}[0]{{\dot{\mu}}}
\newcommand{\nudot}[0]{{\dot{\nu}}}
\newcommand{\omegadot}[0]{{\dot{\omega}}}
\newcommand{\phidot}[0]{{\dot{\phi}}}
\newcommand{\pidot}[0]{{\dot{\pi}}}
\newcommand{\psidot}[0]{{\dot{\psi}}}
\newcommand{\rhodot}[0]{{\dot{\rho}}}
\newcommand{\sigmadot}[0]{{\dot{\sigma}}}
\newcommand{\taudot}[0]{{\dot{\tau}}}
\newcommand{\thetadot}[0]{{\dot{\theta}}}
\newcommand{\upsilondot}[0]{{\dot{\upsilon}}}
\newcommand{\varepsilondot}[0]{{\dot{\varepsilon}}}
\newcommand{\varphidot}[0]{{\dot{\varphi}}}
\newcommand{\varpidot}[0]{{\dot{\varpi}}}
\newcommand{\varrhodot}[0]{{\dot{\varrho}}}
\newcommand{\varsigmadot}[0]{{\dot{\varsigma}}}
\newcommand{\varthetadot}[0]{{\dot{\vartheta}}}
\newcommand{\xidot}[0]{{\dot{\xi}}}
\newcommand{\zetadot}[0]{{\dot{\zeta}}}

\newcommand{\Deltaddot}[0]{{\ddot{\Delta}}}
\newcommand{\Gammaddot}[0]{{\ddot{\Gamma}}}
\newcommand{\Lambdaddot}[0]{{\ddot{\Lambda}}}
\newcommand{\Omegaddot}[0]{{\ddot{\Omega}}}
\newcommand{\Phiddot}[0]{{\ddot{\Phi}}}
\newcommand{\Piddot}[0]{{\ddot{\Pi}}}
\newcommand{\Psiddot}[0]{{\ddot{\Psi}}}
\newcommand{\Sigmaddot}[0]{{\ddot{\Sigma}}}
\newcommand{\Thetaddot}[0]{{\ddot{\Theta}}}
\newcommand{\Upsilonddot}[0]{{\ddot{\Upsilon}}}
\newcommand{\Xiddot}[0]{{\ddot{\Xi}}}
\newcommand{\alphaddot}[0]{{\ddot{\alpha}}}
\newcommand{\betaddot}[0]{{\ddot{\beta}}}
\newcommand{\chiddot}[0]{{\ddot{\chi}}}
\newcommand{\deltaddot}[0]{{\ddot{\delta}}}
\newcommand{\epsilonddot}[0]{{\ddot{\epsilon}}}
\newcommand{\etaddot}[0]{{\ddot{\eta}}}
\newcommand{\gammaddot}[0]{{\ddot{\gamma}}}
\newcommand{\kappaddot}[0]{{\ddot{\kappa}}}
\newcommand{\lambdaddot}[0]{{\ddot{\lambda}}}
\newcommand{\muddot}[0]{{\ddot{\mu}}}
\newcommand{\nuddot}[0]{{\ddot{\nu}}}
\newcommand{\omegaddot}[0]{{\ddot{\omega}}}
\newcommand{\phiddot}[0]{{\ddot{\phi}}}
\newcommand{\piddot}[0]{{\ddot{\pi}}}
\newcommand{\psiddot}[0]{{\ddot{\psi}}}
\newcommand{\rhoddot}[0]{{\ddot{\rho}}}
\newcommand{\sigmaddot}[0]{{\ddot{\sigma}}}
\newcommand{\tauddot}[0]{{\ddot{\tau}}}
\newcommand{\thetaddot}[0]{{\ddot{\theta}}}
\newcommand{\upsilonddot}[0]{{\ddot{\upsilon}}}
\newcommand{\varepsilonddot}[0]{{\ddot{\varepsilon}}}
\newcommand{\varphiddot}[0]{{\ddot{\varphi}}}
\newcommand{\varpiddot}[0]{{\ddot{\varpi}}}
\newcommand{\varrhoddot}[0]{{\ddot{\varrho}}}
\newcommand{\varsigmaddot}[0]{{\ddot{\varsigma}}}
\newcommand{\varthetaddot}[0]{{\ddot{\vartheta}}}
\newcommand{\xiddot}[0]{{\ddot{\xi}}}
\newcommand{\zetaddot}[0]{{\ddot{\zeta}}}

\newcommand{\BDelta}[0]{\boldsymbol{\Delta}}
\newcommand{\BGamma}[0]{\boldsymbol{\Gamma}}
\newcommand{\BLambda}[0]{\boldsymbol{\Lambda}}
\newcommand{\BOmega}[0]{\boldsymbol{\Omega}}
\newcommand{\BPhi}[0]{\boldsymbol{\Phi}}
\newcommand{\BPi}[0]{\boldsymbol{\Pi}}
\newcommand{\BPsi}[0]{\boldsymbol{\Psi}}
\newcommand{\BSigma}[0]{\boldsymbol{\Sigma}}
\newcommand{\BTheta}[0]{\boldsymbol{\Theta}}
\newcommand{\BUpsilon}[0]{\boldsymbol{\Upsilon}}
\newcommand{\BXi}[0]{\boldsymbol{\Xi}}
\newcommand{\Balpha}[0]{\boldsymbol{\alpha}}
\newcommand{\Bbeta}[0]{\boldsymbol{\beta}}
\newcommand{\Bchi}[0]{\boldsymbol{\chi}}
\newcommand{\Bdelta}[0]{\boldsymbol{\delta}}
\newcommand{\Bepsilon}[0]{\boldsymbol{\epsilon}}
\newcommand{\Beta}[0]{\boldsymbol{\eta}}
\newcommand{\Bgamma}[0]{\boldsymbol{\gamma}}
\newcommand{\Bkappa}[0]{\boldsymbol{\kappa}}
\newcommand{\Blambda}[0]{\boldsymbol{\lambda}}
\newcommand{\Bmu}[0]{\boldsymbol{\mu}}
\newcommand{\Bnu}[0]{\boldsymbol{\nu}}
%\newcommand{\Bomega}[0]{\boldsymbol{\omega}}
\newcommand{\Bphi}[0]{\boldsymbol{\phi}}
\newcommand{\Bpi}[0]{\boldsymbol{\pi}}
\newcommand{\Bpsi}[0]{\boldsymbol{\psi}}
\newcommand{\Brho}[0]{\boldsymbol{\rho}}
\newcommand{\Bsigma}[0]{\boldsymbol{\sigma}}
%\newcommand{\Btau}[0]{\boldsymbol{\tau}}
%\newcommand{\Btheta}[0]{\boldsymbol{\theta}}
\newcommand{\Bupsilon}[0]{\boldsymbol{\upsilon}}
\newcommand{\Bvarepsilon}[0]{\boldsymbol{\varepsilon}}
\newcommand{\Bvarphi}[0]{\boldsymbol{\varphi}}
\newcommand{\Bvarpi}[0]{\boldsymbol{\varpi}}
\newcommand{\Bvarrho}[0]{\boldsymbol{\varrho}}
\newcommand{\Bvarsigma}[0]{\boldsymbol{\varsigma}}
\newcommand{\Bvartheta}[0]{\boldsymbol{\vartheta}}
\newcommand{\Bxi}[0]{\boldsymbol{\xi}}
\newcommand{\Bzeta}[0]{\boldsymbol{\zeta}}
%
%</bold and dot greek symbols>
%<infrequent>
%
%\newcommand{\AreaOp}[1]{\AName_{#1}}
%\newcommand{\Babs}[0]{\abs{\BB}}
%\newcommand{\Bcap}[0]{\hat{\BB}}
%\newcommand{\BrPrimeRej}[0]{\rcap(\rcap \wedge \Br')}
%\newcommand{\CA}[0]{\mathcal{A}}
%\newcommand{\Cos}[1]{\cos{\left({#1}\right)}}
%\newcommand{\Det}[1] {\abs{#1}}
%\newcommand{\Dsq}[2] {\frac {\partial^2 {#1}} {\partial {#2}^2}}
%\newcommand{\Exp}[1]{\exp{\left({#1}\right)}}
%\newcommand{\Norm}[1]{\left\lVert{#1}\right\rVert}
%\newcommand{\Sin}[1]{\sin{\left({#1}\right)}}
%\newcommand{\T}[0]{\text{T}}
%\newcommand{\VolumeOp}[1]{\VName_{#1}}
%\newcommand{\agrad}[0]{\Ba \cdot \nabla}
%\newcommand{\alphacap}[0]{\hat{\boldsymbol{\alpha}}}
%\newcommand{\Fcap}[0]{\hat{\BF}}
%\newcommand{\bithree}[0]{{\Bi}_3}
%\newcommand{\bxa}[0]{\Bx\Ba}
%\newcommand{\coordvec}[2]{
%\newcommand{\costheta}[0]{\acap \cdot \xcap}
%\newcommand{\ddt}[1]{\ddot{#1}}
%\newcommand{\ddu}[1] {\frac {d{#1}} {du}}
%\newcommand{\dsqxj}[2] {\frac {\partial^2 {#1}} {\partial {x_{#2}}^2}}
%\newcommand{\dtheta}[1]{\frac{d {#1}}{d \theta}}
%\newcommand{\dt}[1]{\dot{#1}}
%\newcommand{\dt}[1]{\frac{d {#1}}{dt}}
%\newcommand{\dxj}[2] {\frac {\partial {#1}} {\partial {x_{#2}}}}
%\newcommand{\halfPhi}[0]{\frac{\phi}{2}}
%\newcommand{\half}[0]{\inv{2}}
%\newcommand{\inv}[1]{\frac{1}{#1}}
%\newcommand{\laplacian}[0]{\nabla^2}
%\newcommand{\matrixoftx}[3]{
%\newcommand{\nrrp}[0]{\norm{\rcap \wedge \Br'}}
%\newcommand{\oiint}{\bigcirc \hspace{-1.4em} \int \hspace{-.8em} \int}
%\newcommand{\transpose}[1]{{#1}^{\text{T}}}
%\newcommand{\transpose}[1]{{{#1}^{\TextTranspose}}}
%\newcommand{\transpose}[1]{{{#1}^{\text{T}}}}
%\newcommand{\barA}[0]{\bar{A}}
%\newcommand{\qbar}[0]{\bar{q}}
%\newcommand{\qdotbar}[0]{\dot{\bar{q}}}
%
%</infrequent>





\DeclareMathOperator{\sgn}{sgn}
\newcommand{\PDSq}[2]{\frac{\partial^2 {#2}}{\partial {#1}^2}}
\newcommand{\PDN}[3]{\frac{\partial^{#3} {#2}}{\partial {#1}^{#3}}}
\DeclareMathOperator{\sinc}{sinc}
\DeclareMathOperator{\PV}{PV}
\newcommand{\FF}[0]{\mathcal{F}}
\newcommand{\Sw}[0]{\mathcal{S}}
\newcommand{\IIinf}[0]{ \int_{-\infty}^\infty }
\newcommand{\FM}[0]{\inv{\sqrt{2\pi\hbar}}}
\newcommand{\expectation}[1]{\langle{#1}\rangle}

%\usepackage{listings}
%\usepackage{txfonts} % for ointctr... (also appears to make "prettier" \int and \sum's)
% makes \grad look funny though (almost like spacegrad, but narrower)
\usepackage[bookmarks=true]{hyperref}

\usepackage{color,cite,graphicx}
   % use colour in the document, put your citations as [1-4]
   % rather than [1,2,3,4] (it looks nicer, and the extended LaTeX2e
   % graphics package. 
\usepackage{latexsym,amssymb,epsf} % don't remember if these are
   % needed, but their inclusion can't do any damage


\title{ QM notes and problems for Bohm, chapter 11. }
\author{Peeter Joot \quad peeter.joot@gmail.com }
\date{ May 8, 2009.  Last Revision: $Date: 2009/05/09 04:33:54 $ }

\begin{document}

\maketitle{}
\tableofcontents
\section{ Bohm Chapter 11 problems. }

Problems and additional details from reading of \cite{bohm1989qt}, chapter 11.

\subsection{ Problem 1.  Probability currents for step potential. }

This problem and the associated text has a step potential $V$ for $x>0$.  The 
idea is that we have a left to right stream of particles with an associated
wave function, with reflected and transmitted coefficients.

Solutions to the stationary equation are sought in each of the intervals

\begin{align*}
-\frac{\hbar^2}{2m}\psi'' + (V-E)\psi = 0
\end{align*}

That is
\begin{align*}
\psi'' = - \frac{2m}{\hbar^2} (E-V)\psi 
\end{align*}

With an assumption of exponential solutions on the left of the barrier
(ie: no decay and associated hyperbolic solutions in the $V=0$ interval),
we must have $E>0$.

So, the solution can be written as the sum

\begin{align*}
\psi_1 = \sum A_{\pm} \exp\left( \pm i \sqrt{2mE} x / \hbar \right)
\end{align*}

Similarily for $x>0$, non-hyperbolic solutions are

\begin{align*}
\psi_2 = \sum B_{\pm} \exp\left( \pm i \sqrt{2m(E-V)} x / \hbar \right)
\end{align*}

Mathematically, there isn't anything that prevents picking $E-V <0$ solutions

\begin{align*}
\psi_2 = \sum B_{\pm}' \exp\left( \pm \sqrt{-2m(E-V)} x / \hbar \right)
\end{align*}

TODO: Come back to this and think it through.  Where will follow through using
wave function continuity and derivative continuity with these solutions
in the $x>0$ interval?

Continuity at $x=0$ ($\psi_1(0) = \psi_2(0)$) requires

\begin{align*}
A_{+} + A_{-} = B_{+} + B_{-}
\end{align*}

Whereas derivative continuity, with $p_1 = \sqrt{2mE}$, and $p_2 = \sqrt{2m (E-V)}$ requires

\begin{align*}
\frac{i p_1}{\hbar} (A_{+} - A_{-}) = \frac{i p_2}{\hbar} (B_{+} - B_{-}) 
\end{align*}

This is

\begin{align*}
\begin{bmatrix}
1 & 1 \\
p_1 & -p_1
\end{bmatrix}
\begin{bmatrix}
A_{+} \\
A_{-} 
\end{bmatrix}
=
\begin{bmatrix}
1 & 1 \\
p_2 & -p_2
\end{bmatrix}
\begin{bmatrix}
B_{+} \\
B_{-} 
\end{bmatrix}
\end{align*}

Matrix inversion gives us the $B$ coefficients in terms of $A$

\begin{align*}
\begin{bmatrix}
B_{+} \\
B_{-} 
\end{bmatrix}
&=
\inv{-2p_2}
\begin{bmatrix}
-p_2 & -1 \\
-p_2 & 1
\end{bmatrix}
\begin{bmatrix}
1 & 1 \\
p_1 & -p_1
\end{bmatrix}
\begin{bmatrix}
A_{+} \\
A_{-} 
\end{bmatrix} \\
&=
\inv{2p_2}
\begin{bmatrix}
p_2 & 1 \\
p_2 & -1
\end{bmatrix}
\begin{bmatrix}
1 & 1 \\
p_1 & -p_1
\end{bmatrix}
\begin{bmatrix}
A_{+} \\
A_{-} 
\end{bmatrix} \\
&=
\inv{2p_2}
\begin{bmatrix}
(p_1 + p_2) & (p_2 - p_1) \\
(p_2 - p_1) & (p_1 + p_2)
\end{bmatrix}
\begin{bmatrix}
A_{+} \\
A_{-} 
\end{bmatrix} \\
\end{align*}

Or equivalently

\begin{align*}
\begin{bmatrix}
A_{+} \\
A_{-} 
\end{bmatrix}
&=
\inv{2p_1}
\begin{bmatrix}
(p_1 + p_2) & (p_1 - p_2) \\
(p_1 - p_2) & (p_1 + p_2)
\end{bmatrix}
\begin{bmatrix}
B_{+} \\
B_{-} 
\end{bmatrix} \\
\end{align*}

Using this last, the 
assumption of no further barriers in the $x>0$ interval (ie: no reflection at 
$x = \infty$), allows the physical situation to dictate $B_{-} = 0$.  Then we
have

\begin{align*}
\begin{bmatrix}
A_{+} \\
A_{-} 
\end{bmatrix}
&=
\frac{B_{+} }{2p_1}
\begin{bmatrix}
p_1 + p_2 \\
p_1 - p_2 
\end{bmatrix}
\end{align*}

The first is 
\begin{align*}
A_{+} 
&=
\frac{B_{+} }{2p_1} (p_1 + p_2)
\end{align*}

Or
\begin{align*}
B_{+} 
&=
\frac{2 p_1 A_{+} }{p_1 + p_2}
\end{align*}

and the second is
\begin{align*}
A_{-} 
&=
\frac{B_{+} }{2p_1} (p_1 - p_2) \\
&=
\frac{2 p_1 A_{+} }{p_1 + p_2}
\frac{1}{2p_1} (p_1 - p_2) \\
&=
A_{+} \left(
\frac{ p_1 - p_2 }{p_1 + p_2} \right)
\end{align*}

This reduces the free parameters in the wave functions to the single 
amplitude

\begin{align*}
\psi_1 &= 
A_{+} e^{ i p_1 x/\hbar }
+A_{+} \left(
\frac{ p_1 - p_2 }{p_1 + p_2} \right)
e^{ -i p_1 x/\hbar } \\
\psi_2 &= A_{+} \frac{2 p_1 }{p_1 + p_2} e^{ i p_2 x/\hbar }
\end{align*}

Inspection provides a check that these do in fact satisfy the desired
continuity requirements.


\bibliographystyle{plainnat}
\bibliography{myrefs}

\end{document}

\include{commutator_herm}
\include{delta_ortho_series}
\documentclass{article}

\usepackage{amsmath}
\usepackage{mathpazo}

%
% shorthand for bold symbols, convenient for vectors and matrices
%
\newcommand{\Ba}[0]{\mathbf{a}}
\newcommand{\Bb}[0]{\mathbf{b}}
\newcommand{\Bc}[0]{\mathbf{c}}
\newcommand{\Bd}[0]{\mathbf{d}}
\newcommand{\Be}[0]{\mathbf{e}}
\newcommand{\Bf}[0]{\mathbf{f}}
\newcommand{\Bg}[0]{\mathbf{g}}
\newcommand{\Bh}[0]{\mathbf{h}}
\newcommand{\Bi}[0]{\mathbf{i}}
\newcommand{\Bj}[0]{\mathbf{j}}
\newcommand{\Bk}[0]{\mathbf{k}}
\newcommand{\Bl}[0]{\mathbf{l}}
\newcommand{\Bm}[0]{\mathbf{m}}
\newcommand{\Bn}[0]{\mathbf{n}}
\newcommand{\Bo}[0]{\mathbf{o}}
\newcommand{\Bp}[0]{\mathbf{p}}
\newcommand{\Bq}[0]{\mathbf{q}}
\newcommand{\Br}[0]{\mathbf{r}}
\newcommand{\Bs}[0]{\mathbf{s}}
\newcommand{\Bt}[0]{\mathbf{t}}
\newcommand{\Bu}[0]{\mathbf{u}}
\newcommand{\Bv}[0]{\mathbf{v}}
\newcommand{\Bw}[0]{\mathbf{w}}
\newcommand{\Bx}[0]{\mathbf{x}}
\newcommand{\By}[0]{\mathbf{y}}
\newcommand{\Bz}[0]{\mathbf{z}}
\newcommand{\BA}[0]{\mathbf{A}}
\newcommand{\BB}[0]{\mathbf{B}}
\newcommand{\BC}[0]{\mathbf{C}}
\newcommand{\BD}[0]{\mathbf{D}}
\newcommand{\BE}[0]{\mathbf{E}}
\newcommand{\BF}[0]{\mathbf{F}}
\newcommand{\BG}[0]{\mathbf{G}}
\newcommand{\BH}[0]{\mathbf{H}}
\newcommand{\BI}[0]{\mathbf{I}}
\newcommand{\BJ}[0]{\mathbf{J}}
\newcommand{\BK}[0]{\mathbf{K}}
\newcommand{\BL}[0]{\mathbf{L}}
\newcommand{\BM}[0]{\mathbf{M}}
\newcommand{\BN}[0]{\mathbf{N}}
\newcommand{\BO}[0]{\mathbf{O}}
\newcommand{\BP}[0]{\mathbf{P}}
\newcommand{\BQ}[0]{\mathbf{Q}}
\newcommand{\BR}[0]{\mathbf{R}}
\newcommand{\BS}[0]{\mathbf{S}}
\newcommand{\BT}[0]{\mathbf{T}}
\newcommand{\BU}[0]{\mathbf{U}}
\newcommand{\BV}[0]{\mathbf{V}}
\newcommand{\BW}[0]{\mathbf{W}}
\newcommand{\BX}[0]{\mathbf{X}}
\newcommand{\BY}[0]{\mathbf{Y}}
\newcommand{\BZ}[0]{\mathbf{Z}}

\newcommand{\Bzero}[0]{\mathbf{0}}
\newcommand{\Btheta}[0]{\boldsymbol{\theta}}
\newcommand{\Btau}[0]{\boldsymbol{\tau}}
\newcommand{\Bomega}[0]{\boldsymbol{\omega}}

%
% shorthand for unit vectors
%
\newcommand{\acap}[0]{\hat{\Ba}}
\newcommand{\bcap}[0]{\hat{\Bb}}
\newcommand{\ccap}[0]{\hat{\Bc}}
\newcommand{\dcap}[0]{\hat{\Bd}}
\newcommand{\ecap}[0]{\hat{\Be}}
\newcommand{\fcap}[0]{\hat{\Bf}}
\newcommand{\gcap}[0]{\hat{\Bg}}
\newcommand{\hcap}[0]{\hat{\Bh}}
\newcommand{\icap}[0]{\hat{\Bi}}
\newcommand{\jcap}[0]{\hat{\Bj}}
\newcommand{\kcap}[0]{\hat{\Bk}}
\newcommand{\lcap}[0]{\hat{\Bl}}
\newcommand{\mcap}[0]{\hat{\Bm}}
\newcommand{\ncap}[0]{\hat{\Bn}}
\newcommand{\ocap}[0]{\hat{\Bo}}
\newcommand{\pcap}[0]{\hat{\Bp}}
\newcommand{\qcap}[0]{\hat{\Bq}}
\newcommand{\rcap}[0]{\hat{\Br}}
\newcommand{\scap}[0]{\hat{\Bs}}
\newcommand{\tcap}[0]{\hat{\Bt}}
\newcommand{\ucap}[0]{\hat{\Bu}}
\newcommand{\vcap}[0]{\hat{\Bv}}
\newcommand{\wcap}[0]{\hat{\Bw}}
\newcommand{\xcap}[0]{\hat{\Bx}}
\newcommand{\ycap}[0]{\hat{\By}}
\newcommand{\zcap}[0]{\hat{\Bz}}
\newcommand{\thetacap}[0]{\hat{\Btheta}}

%
% to write R^n and C^n in a distinguishable fashion.  Perhaps change this
% to the double lined characters upon figuring out how to do so.
%
\newcommand{\C}[1]{$\mathbb{C}^{#1}$}
\newcommand{\R}[1]{$\mathbb{R}^{#1}$}

%
% various generally useful helpers
%

% derivative of #1 wrt. #2:
\newcommand{\D}[2] {\frac {d#2} {d#1}}

\newcommand{\inv}[1]{\frac{1}{#1}}
\newcommand{\cross}[0]{\times}

\newcommand{\abs}[1]{\lvert{#1}\rvert}
\newcommand{\norm}[1]{\lVert{#1}\rVert}
\newcommand{\innerprod}[2]{\langle{#1}, {#2}\rangle}
\newcommand{\dotprod}[2]{{#1} \cdot {#2}}
\newcommand{\bdotprod}[2]{\left({#1} \cdot {#2}\right)}
\newcommand{\crossprod}[2]{{#1} \cross {#2}}
\newcommand{\tripleprod}[3]{\dotprod{\left(\crossprod{#1}{#2}\right)}{#3}}

\DeclareMathOperator{\Proj}{Proj}
\DeclareMathOperator{\Span}{span}
\DeclareMathOperator{\Sgn}{sgn}
\DeclareMathOperator{\Area}{Area}
\DeclareMathOperator{\Volume}{Volume}

%
% A few miscellaneous things specific to this document
%
\newcommand{\crossop}[1]{\crossprod{#1}{}}

% R2 vector.
\newcommand{\VectorTwo}[2]{
\begin{bmatrix}
 {#1} \\
 {#2}
\end{bmatrix}
}

\newcommand{\VectorN}[1]{
\begin{bmatrix}
{#1}_1 \\
{#1}_2 \\
\vdots \\
{#1}_N \\
\end{bmatrix}
}

\newcommand{\DETuvij}[4]{
\begin{vmatrix}
 {#1}_{#3} & {#1}_{#4} \\
 {#2}_{#3} & {#2}_{#4}
\end{vmatrix}
}

\newcommand{\DETuvwijk}[6]{
\begin{vmatrix}
 {#1}_{#4} & {#1}_{#5} & {#1}_{#6} \\
 {#2}_{#4} & {#2}_{#5} & {#2}_{#6} \\
 {#3}_{#4} & {#3}_{#5} & {#3}_{#6}
\end{vmatrix}
}

\newcommand{\DETuvwxijkl}[8]{
\begin{vmatrix}
 {#1}_{#5} & {#1}_{#6} & {#1}_{#7} & {#1}_{#8} \\
 {#2}_{#5} & {#2}_{#6} & {#2}_{#7} & {#2}_{#8} \\
 {#3}_{#5} & {#3}_{#6} & {#3}_{#7} & {#3}_{#8} \\
 {#4}_{#5} & {#4}_{#6} & {#4}_{#7} & {#4}_{#8} \\
\end{vmatrix}
}

%\newcommand{\DETuvwxyijklm}[10]{
%\begin{vmatrix}
% {#1}_{#6} & {#1}_{#7} & {#1}_{#8} & {#1}_{#9} & {#1}_{#10} \\
% {#2}_{#6} & {#2}_{#7} & {#2}_{#8} & {#2}_{#9} & {#2}_{#10} \\
% {#3}_{#6} & {#3}_{#7} & {#3}_{#8} & {#3}_{#9} & {#3}_{#10} \\
% {#4}_{#6} & {#4}_{#7} & {#4}_{#8} & {#4}_{#9} & {#4}_{#10} \\
% {#5}_{#6} & {#5}_{#7} & {#5}_{#8} & {#5}_{#9} & {#5}_{#10}
%\end{vmatrix}
%}

% R3 vector.
\newcommand{\VectorThree}[3]{
\begin{bmatrix}
 {#1} \\
 {#2} \\
 {#3}
\end{bmatrix}
}


%<misc>
%
\newcommand{\Abs}[1]{{\left\lvert{#1}\right\rvert}}
\newcommand{\spacegrad}[0]{\boldsymbol{\nabla}}
\newcommand{\grad}[0]{\nabla}
\newcommand{\LL}[0]{\mathcal{L}}

% == \partial_{#1} {#2}
\newcommand{\PD}[2]{\frac{\partial {#2}}{\partial {#1}}}
% inline variant
\newcommand{\PDi}[2]{{\partial {#2}}/{\partial {#1}}}

\newcommand{\PDD}[3]{\frac{\partial^2 {#3}}{\partial {#1}\partial {#2}}}
%\newcommand{\PDd}[2]{\frac{\partial^2 {#2}}{{\partial{#1}}^2}}
\newcommand{\PDsq}[2]{\frac{\partial^2 {#2}}{(\partial {#1})^2}}

\newcommand{\Partial}[2]{\frac{\partial {#1}}{\partial {#2}}}
\DeclareMathOperator{\RejName}{Rej}
\newcommand{\Rej}[2]{\RejName_{#1}\left( {#2} \right)}
\newcommand{\Rm}[1]{\mathbb{R}^{#1}}
\newcommand{\Cm}[1]{\mathbb{C}^{#1}}
\newcommand{\conj}[0]{{*}}

%</misc>

% <grade selection>
%
\newcommand{\gpgrade}[2] {{\left\langle{{#1}}\right\rangle}_{#2}}

\newcommand{\gpgradezero}[1] {\gpgrade{#1}{}}
%\newcommand{\gpscalargrade}[1] {{\left\langle{{#1}}\right\rangle}}
%\newcommand{\gpgradezero}[1] {\gpgrade{#1}{0}}

%\newcommand{\gpgradeone}[1] {{\left\langle{{#1}}\right\rangle}_{1}}
\newcommand{\gpgradeone}[1] {\gpgrade{#1}{1}}

\newcommand{\gpgradetwo}[1] {\gpgrade{#1}{2}}
\newcommand{\gpgradethree}[1] {\gpgrade{#1}{3}}
\newcommand{\gpgradefour}[1] {\gpgrade{#1}{4}}
%
% </grade selection>



\newcommand{\adot}[0]{{\dot{a}}}
\newcommand{\bdot}[0]{{\dot{b}}}
% taken for centered dot:
%\newcommand{\cdot}[0]{{\dot{c}}}
%\newcommand{\ddot}[0]{{\dot{d}}}
\newcommand{\edot}[0]{{\dot{e}}}
\newcommand{\fdot}[0]{{\dot{f}}}
\newcommand{\gdot}[0]{{\dot{g}}}
\newcommand{\hdot}[0]{{\dot{h}}}
\newcommand{\idot}[0]{{\dot{i}}}
\newcommand{\jdot}[0]{{\dot{j}}}
\newcommand{\kdot}[0]{{\dot{k}}}
\newcommand{\ldot}[0]{{\dot{l}}}
\newcommand{\mdot}[0]{{\dot{m}}}
\newcommand{\ndot}[0]{{\dot{n}}}
%\newcommand{\odot}[0]{{\dot{o}}}
\newcommand{\pdot}[0]{{\dot{p}}}
\newcommand{\qdot}[0]{{\dot{q}}}
\newcommand{\rdot}[0]{{\dot{r}}}
\newcommand{\sdot}[0]{{\dot{s}}}
\newcommand{\tdot}[0]{{\dot{t}}}
\newcommand{\udot}[0]{{\dot{u}}}
\newcommand{\vdot}[0]{{\dot{v}}}
\newcommand{\wdot}[0]{{\dot{w}}}
\newcommand{\xdot}[0]{{\dot{x}}}
\newcommand{\ydot}[0]{{\dot{y}}}
\newcommand{\zdot}[0]{{\dot{z}}}
\newcommand{\addot}[0]{{\ddot{a}}}
\newcommand{\bddot}[0]{{\ddot{b}}}
\newcommand{\cddot}[0]{{\ddot{c}}}
%\newcommand{\dddot}[0]{{\ddot{d}}}
\newcommand{\eddot}[0]{{\ddot{e}}}
\newcommand{\fddot}[0]{{\ddot{f}}}
\newcommand{\gddot}[0]{{\ddot{g}}}
\newcommand{\hddot}[0]{{\ddot{h}}}
\newcommand{\iddot}[0]{{\ddot{i}}}
\newcommand{\jddot}[0]{{\ddot{j}}}
\newcommand{\kddot}[0]{{\ddot{k}}}
\newcommand{\lddot}[0]{{\ddot{l}}}
\newcommand{\mddot}[0]{{\ddot{m}}}
\newcommand{\nddot}[0]{{\ddot{n}}}
\newcommand{\oddot}[0]{{\ddot{o}}}
\newcommand{\pddot}[0]{{\ddot{p}}}
\newcommand{\qddot}[0]{{\ddot{q}}}
\newcommand{\rddot}[0]{{\ddot{r}}}
\newcommand{\sddot}[0]{{\ddot{s}}}
\newcommand{\tddot}[0]{{\ddot{t}}}
\newcommand{\uddot}[0]{{\ddot{u}}}
\newcommand{\vddot}[0]{{\ddot{v}}}
\newcommand{\wddot}[0]{{\ddot{w}}}
\newcommand{\xddot}[0]{{\ddot{x}}}
\newcommand{\yddot}[0]{{\ddot{y}}}
\newcommand{\zddot}[0]{{\ddot{z}}}

%<bold and dot greek symbols>
%

\newcommand{\Deltadot}[0]{{\dot{\Delta}}}
\newcommand{\Gammadot}[0]{{\dot{\Gamma}}}
\newcommand{\Lambdadot}[0]{{\dot{\Lambda}}}
\newcommand{\Omegadot}[0]{{\dot{\Omega}}}
\newcommand{\Phidot}[0]{{\dot{\Phi}}}
\newcommand{\Pidot}[0]{{\dot{\Pi}}}
\newcommand{\Psidot}[0]{{\dot{\Psi}}}
\newcommand{\Sigmadot}[0]{{\dot{\Sigma}}}
\newcommand{\Thetadot}[0]{{\dot{\Theta}}}
\newcommand{\Upsilondot}[0]{{\dot{\Upsilon}}}
\newcommand{\Xidot}[0]{{\dot{\Xi}}}
\newcommand{\alphadot}[0]{{\dot{\alpha}}}
\newcommand{\betadot}[0]{{\dot{\beta}}}
\newcommand{\chidot}[0]{{\dot{\chi}}}
\newcommand{\deltadot}[0]{{\dot{\delta}}}
\newcommand{\epsilondot}[0]{{\dot{\epsilon}}}
\newcommand{\etadot}[0]{{\dot{\eta}}}
\newcommand{\gammadot}[0]{{\dot{\gamma}}}
\newcommand{\kappadot}[0]{{\dot{\kappa}}}
\newcommand{\lambdadot}[0]{{\dot{\lambda}}}
\newcommand{\mudot}[0]{{\dot{\mu}}}
\newcommand{\nudot}[0]{{\dot{\nu}}}
\newcommand{\omegadot}[0]{{\dot{\omega}}}
\newcommand{\phidot}[0]{{\dot{\phi}}}
\newcommand{\pidot}[0]{{\dot{\pi}}}
\newcommand{\psidot}[0]{{\dot{\psi}}}
\newcommand{\rhodot}[0]{{\dot{\rho}}}
\newcommand{\sigmadot}[0]{{\dot{\sigma}}}
\newcommand{\taudot}[0]{{\dot{\tau}}}
\newcommand{\thetadot}[0]{{\dot{\theta}}}
\newcommand{\upsilondot}[0]{{\dot{\upsilon}}}
\newcommand{\varepsilondot}[0]{{\dot{\varepsilon}}}
\newcommand{\varphidot}[0]{{\dot{\varphi}}}
\newcommand{\varpidot}[0]{{\dot{\varpi}}}
\newcommand{\varrhodot}[0]{{\dot{\varrho}}}
\newcommand{\varsigmadot}[0]{{\dot{\varsigma}}}
\newcommand{\varthetadot}[0]{{\dot{\vartheta}}}
\newcommand{\xidot}[0]{{\dot{\xi}}}
\newcommand{\zetadot}[0]{{\dot{\zeta}}}

\newcommand{\Deltaddot}[0]{{\ddot{\Delta}}}
\newcommand{\Gammaddot}[0]{{\ddot{\Gamma}}}
\newcommand{\Lambdaddot}[0]{{\ddot{\Lambda}}}
\newcommand{\Omegaddot}[0]{{\ddot{\Omega}}}
\newcommand{\Phiddot}[0]{{\ddot{\Phi}}}
\newcommand{\Piddot}[0]{{\ddot{\Pi}}}
\newcommand{\Psiddot}[0]{{\ddot{\Psi}}}
\newcommand{\Sigmaddot}[0]{{\ddot{\Sigma}}}
\newcommand{\Thetaddot}[0]{{\ddot{\Theta}}}
\newcommand{\Upsilonddot}[0]{{\ddot{\Upsilon}}}
\newcommand{\Xiddot}[0]{{\ddot{\Xi}}}
\newcommand{\alphaddot}[0]{{\ddot{\alpha}}}
\newcommand{\betaddot}[0]{{\ddot{\beta}}}
\newcommand{\chiddot}[0]{{\ddot{\chi}}}
\newcommand{\deltaddot}[0]{{\ddot{\delta}}}
\newcommand{\epsilonddot}[0]{{\ddot{\epsilon}}}
\newcommand{\etaddot}[0]{{\ddot{\eta}}}
\newcommand{\gammaddot}[0]{{\ddot{\gamma}}}
\newcommand{\kappaddot}[0]{{\ddot{\kappa}}}
\newcommand{\lambdaddot}[0]{{\ddot{\lambda}}}
\newcommand{\muddot}[0]{{\ddot{\mu}}}
\newcommand{\nuddot}[0]{{\ddot{\nu}}}
\newcommand{\omegaddot}[0]{{\ddot{\omega}}}
\newcommand{\phiddot}[0]{{\ddot{\phi}}}
\newcommand{\piddot}[0]{{\ddot{\pi}}}
\newcommand{\psiddot}[0]{{\ddot{\psi}}}
\newcommand{\rhoddot}[0]{{\ddot{\rho}}}
\newcommand{\sigmaddot}[0]{{\ddot{\sigma}}}
\newcommand{\tauddot}[0]{{\ddot{\tau}}}
\newcommand{\thetaddot}[0]{{\ddot{\theta}}}
\newcommand{\upsilonddot}[0]{{\ddot{\upsilon}}}
\newcommand{\varepsilonddot}[0]{{\ddot{\varepsilon}}}
\newcommand{\varphiddot}[0]{{\ddot{\varphi}}}
\newcommand{\varpiddot}[0]{{\ddot{\varpi}}}
\newcommand{\varrhoddot}[0]{{\ddot{\varrho}}}
\newcommand{\varsigmaddot}[0]{{\ddot{\varsigma}}}
\newcommand{\varthetaddot}[0]{{\ddot{\vartheta}}}
\newcommand{\xiddot}[0]{{\ddot{\xi}}}
\newcommand{\zetaddot}[0]{{\ddot{\zeta}}}

\newcommand{\BDelta}[0]{\boldsymbol{\Delta}}
\newcommand{\BGamma}[0]{\boldsymbol{\Gamma}}
\newcommand{\BLambda}[0]{\boldsymbol{\Lambda}}
\newcommand{\BOmega}[0]{\boldsymbol{\Omega}}
\newcommand{\BPhi}[0]{\boldsymbol{\Phi}}
\newcommand{\BPi}[0]{\boldsymbol{\Pi}}
\newcommand{\BPsi}[0]{\boldsymbol{\Psi}}
\newcommand{\BSigma}[0]{\boldsymbol{\Sigma}}
\newcommand{\BTheta}[0]{\boldsymbol{\Theta}}
\newcommand{\BUpsilon}[0]{\boldsymbol{\Upsilon}}
\newcommand{\BXi}[0]{\boldsymbol{\Xi}}
\newcommand{\Balpha}[0]{\boldsymbol{\alpha}}
\newcommand{\Bbeta}[0]{\boldsymbol{\beta}}
\newcommand{\Bchi}[0]{\boldsymbol{\chi}}
\newcommand{\Bdelta}[0]{\boldsymbol{\delta}}
\newcommand{\Bepsilon}[0]{\boldsymbol{\epsilon}}
\newcommand{\Beta}[0]{\boldsymbol{\eta}}
\newcommand{\Bgamma}[0]{\boldsymbol{\gamma}}
\newcommand{\Bkappa}[0]{\boldsymbol{\kappa}}
\newcommand{\Blambda}[0]{\boldsymbol{\lambda}}
\newcommand{\Bmu}[0]{\boldsymbol{\mu}}
\newcommand{\Bnu}[0]{\boldsymbol{\nu}}
%\newcommand{\Bomega}[0]{\boldsymbol{\omega}}
\newcommand{\Bphi}[0]{\boldsymbol{\phi}}
\newcommand{\Bpi}[0]{\boldsymbol{\pi}}
\newcommand{\Bpsi}[0]{\boldsymbol{\psi}}
\newcommand{\Brho}[0]{\boldsymbol{\rho}}
\newcommand{\Bsigma}[0]{\boldsymbol{\sigma}}
%\newcommand{\Btau}[0]{\boldsymbol{\tau}}
%\newcommand{\Btheta}[0]{\boldsymbol{\theta}}
\newcommand{\Bupsilon}[0]{\boldsymbol{\upsilon}}
\newcommand{\Bvarepsilon}[0]{\boldsymbol{\varepsilon}}
\newcommand{\Bvarphi}[0]{\boldsymbol{\varphi}}
\newcommand{\Bvarpi}[0]{\boldsymbol{\varpi}}
\newcommand{\Bvarrho}[0]{\boldsymbol{\varrho}}
\newcommand{\Bvarsigma}[0]{\boldsymbol{\varsigma}}
\newcommand{\Bvartheta}[0]{\boldsymbol{\vartheta}}
\newcommand{\Bxi}[0]{\boldsymbol{\xi}}
\newcommand{\Bzeta}[0]{\boldsymbol{\zeta}}
%
%</bold and dot greek symbols>
%<infrequent>
%
%\newcommand{\AreaOp}[1]{\AName_{#1}}
%\newcommand{\Babs}[0]{\abs{\BB}}
%\newcommand{\Bcap}[0]{\hat{\BB}}
%\newcommand{\BrPrimeRej}[0]{\rcap(\rcap \wedge \Br')}
%\newcommand{\CA}[0]{\mathcal{A}}
%\newcommand{\Cos}[1]{\cos{\left({#1}\right)}}
%\newcommand{\Det}[1] {\abs{#1}}
%\newcommand{\Dsq}[2] {\frac {\partial^2 {#1}} {\partial {#2}^2}}
%\newcommand{\Exp}[1]{\exp{\left({#1}\right)}}
%\newcommand{\Norm}[1]{\left\lVert{#1}\right\rVert}
%\newcommand{\Sin}[1]{\sin{\left({#1}\right)}}
%\newcommand{\T}[0]{\text{T}}
%\newcommand{\VolumeOp}[1]{\VName_{#1}}
%\newcommand{\agrad}[0]{\Ba \cdot \nabla}
%\newcommand{\alphacap}[0]{\hat{\boldsymbol{\alpha}}}
%\newcommand{\Fcap}[0]{\hat{\BF}}
%\newcommand{\bithree}[0]{{\Bi}_3}
%\newcommand{\bxa}[0]{\Bx\Ba}
%\newcommand{\coordvec}[2]{
%\newcommand{\costheta}[0]{\acap \cdot \xcap}
%\newcommand{\ddt}[1]{\ddot{#1}}
%\newcommand{\ddu}[1] {\frac {d{#1}} {du}}
%\newcommand{\dsqxj}[2] {\frac {\partial^2 {#1}} {\partial {x_{#2}}^2}}
%\newcommand{\dtheta}[1]{\frac{d {#1}}{d \theta}}
%\newcommand{\dt}[1]{\dot{#1}}
%\newcommand{\dt}[1]{\frac{d {#1}}{dt}}
%\newcommand{\dxj}[2] {\frac {\partial {#1}} {\partial {x_{#2}}}}
%\newcommand{\halfPhi}[0]{\frac{\phi}{2}}
%\newcommand{\half}[0]{\inv{2}}
%\newcommand{\inv}[1]{\frac{1}{#1}}
%\newcommand{\laplacian}[0]{\nabla^2}
%\newcommand{\matrixoftx}[3]{
%\newcommand{\nrrp}[0]{\norm{\rcap \wedge \Br'}}
%\newcommand{\oiint}{\bigcirc \hspace{-1.4em} \int \hspace{-.8em} \int}
%\newcommand{\transpose}[1]{{#1}^{\text{T}}}
%\newcommand{\transpose}[1]{{{#1}^{\TextTranspose}}}
%\newcommand{\transpose}[1]{{{#1}^{\text{T}}}}
%\newcommand{\barA}[0]{\bar{A}}
%\newcommand{\qbar}[0]{\bar{q}}
%\newcommand{\qdotbar}[0]{\dot{\bar{q}}}
%
%</infrequent>




\newcommand{\PDSq}[2]{\frac{\partial^2 {#2}}{\partial {#1}^2}}

\usepackage[bookmarks=true]{hyperref}

\usepackage{color,cite,graphicx}
   % use colour in the document, put your citations as [1-4]
   % rather than [1,2,3,4] (it looks nicer, and the extended LaTeX2e
   % graphics package. 
\usepackage{latexsym,amssymb,epsf} % don't remember if these are
   % needed, but their inclusion can't do any damage

\title{ Ehrenfest's theorem. }
\author{Peeter Joot}
\date{ Jan 22, 2009.  Last Revision: $Date: 2009/01/24 02:41:28 $ }

\begin{document}

\maketitle{}

\tableofcontents
\section{ Motivation. }

\cite{mcmahon2005qmd} has a one dimensional treatment of Ehrenfest's theorem,
that the expectation values of the position and momentum operators behave
like Newton's law.

However, he makes use
of commutator and braket notation before either is defined.

That looks like a natural way to do the derivation easily, but let's try
this using instead what is defined up to this point in the text.

\section{ Review.  What do we know so far? }

\subsection{ Position and momentum operators. }

We have been given the definitions of two specific operators, position and momentum, 
whos action on a wave function is

\begin{align*}
\hat{x} \psi &= x \psi \\
\hat{p} \psi &= \left(-i \hbar \PD{x}{}\right) \psi
\end{align*}

In operator form, with the omission of the explicit wave function being operated on this is

\begin{align*}
\hat{x} &\equiv x  \\
\hat{p} &\equiv -i \hbar \PD{x}{}
\end{align*}

These are perfectly valid operator definitions, but the validity of using the 
classical names for these really comes from this upcoming Ehrenfest result where
the average of the action of these operators on a wave function is examined.

\subsection{ Expectation (average) value of an operator. }

We also have a definition for the expectation value
of an operator $\hat{A}$, given its specific action $A$.
This is defined very much like a weighted inner product
and is essentially a field weighted average of the operators action

\begin{align*}
<\hat{A}> \equiv \int \psi^\conj (A \psi)
\end{align*}

The braces show that the operator action $A$ here applies to the rightmost field variable $\psi$, and not to its conjugate.

For the position and momentum operators respectively, we have the
expectation values

\begin{align*}
<\hat{x}> &\equiv \int \psi^\conj (x \psi) \\
<\hat{p}> &\equiv \int \psi^\conj \left(-i \hbar \PD{x}{}\right) \psi
\end{align*}

\subsection{ Hermitian operator. }

The notation of a Hermitian operator as also been introduced in terms of 
left acting operators.  That is, an operator $\hat{A}$ is hermitian if

\begin{align}\label{eqn:hermitian1}
\int \psi^\conj (A \psi) = \int (\psi A)^\conj \psi
\end{align}

This is a somewhat non-Demystified seeming definition to me since I'd seen
Hermitian defined more directly in terms of ``normal'' right acting 
expectation integrals.  That is, an operator $\hat{A}$ is Hermitian if

\begin{align*}
<\hat{A}>^\conj = <\hat{A}>
\end{align*}

The conjugate of an operator's expectation value is
\begin{align*}
\left(\int \psi^\conj (A \psi)\right)^\conj 
&= \int \psi (A^\conj \psi^\conj) \\
&= \int (A^\conj \psi^\conj) \psi \\
\end{align*}

So, this second Hermitian definition means that an operator is Hermitian if

\begin{align*}
\int (A^\conj \psi^\conj) \psi &= \int \psi^\conj (A \psi)
\end{align*}

This highlights why the left acting operator notation is pretty reasonable
seeming.  Allowing the conjugation operation to switch an operators action
from right acting to left acting makes the equation prettier, and 
recovers equation \ref{eqn:hermitian1}

\begin{align*}
(\psi A)^\conj \equiv (A^\conj \psi^\conj) 
\end{align*}

Here braces have been used to express the limitation of the scope of the action of the operator.

Another way to express this is that one can say that a Hermitian operator when put 
in its wave function sandwhich has a 
conjugate action acting to the left on the conjugate wave function and a non-conjugate
action to the right.  This allows for a final notation nicety, where one can omit the 
braces entirely as in

\begin{align*}
\int \psi^\conj (A \psi) \equiv \int \psi^\conj A \psi \equiv \int (A^\conj \psi^\conj) \psi
\end{align*}
or in terms of right and left operator notation the equivalent
\begin{align*}
\int \psi^\conj (A \psi) \equiv \int \psi^\conj A \psi \equiv \int (\psi A)^\conj \psi
\end{align*}

And finally, there is one last way to express this the concept of Hermitian.
We have our definition of a left acting operator

\begin{align*}
(\psi A)^\conj = A^\conj \psi^\conj
\end{align*}

And can make the observation that conjugation of a product is the 
product of the conjugates
\begin{align*}
(\psi A)^\conj = \psi^\conj A^\conj
\end{align*}

So we must also have $A = A^\conj$ for a Hermitian operator.  

From this one can observe that the position operator $\hat{x}$ is Hermitian, but the momentum operator is not (but $\hat{p}^2$ is ).

\subsection{ Variance and Heisenberg principle. }

Various calculations have been done to calcualate expectation values.

In a few places we have had to show that the product of variances

\begin{align*}
\Delta A = \sqrt{<A^2> - <A>^2}
\end{align*}

for position and momentum all satisfy the famous Heisenberg uncertainty
principle

\begin{align*}
\Delta x \Delta p \ge \hbar/2
\end{align*}

(in a couple places this formulation is a bit fuzzy since our squared
momentum variance $(\Delta p)^2$ has been negative).

\subsection{ The wave equation. }

We are also given Schr\"{o}dinger equation in Hamiltonian form

\begin{align*}
\hat{H} \psi = i \hbar \PD{t}{\psi}
\end{align*}

and have worked with the specific form of the Hamiltonian that applies to
a non-relativistic particle (and not to photons).

\begin{align*}
\hat{H} = \frac{\hat{p}^2}{2m} + V = -\frac{\hbar^2}{2m} \grad^2 + V
\end{align*}

Most of the text up to this point has been about calculating and interpretting
specific solutions of this equation.

\subsection{ Other stuff. }

A number of other fundamental topics have been covered, probabilities, normalization, probability current, energy, phase, orthogonality, and so forth.  However, summarizing the rest of these in detail is not required as 
background for the Ehrenfest result.

\section{ Ehrenfest theorem. }

We want to calculate the time derivatives of the expectation values
for position and momentum OPERATORS, and show that these reproduce the
familiar velocity, momentum and force concepts from classical mechanics.

\subsection{ Velocity from the derivative of the position operator expectation. }

Diving straight in we have

\begin{align*}
\PD{t}{<\hat{x}>}
&= \PD{t}{} \left( \int \psi^\conj x \psi \right) \\
&= \int \PD{t}{\psi^\conj} x \psi + \int \psi^\conj x \PD{t}{\psi} 
\end{align*}

Now, here the Hamiltonian can be introduced, replacing the time derivatives.

We have
\begin{align*}
\PD{t}{\psi} &= -\frac{i}{\hbar} H \psi \\
\PD{t}{\psi^\conj} &= \frac{i}{\hbar} H \psi^\conj \\
\end{align*}

So we have
\begin{align*}
\PD{t}{<\hat{x}>}
&= \frac{i}{\hbar} \int \psi x H {\psi^\conj} - \frac{i}{\hbar}\int \psi^\conj x H {\psi} 
\end{align*}

For the Schr\"{o}dinger Hamiltonian we have

\begin{align*}
H \psi &= - \frac{\hbar^2}{2m} \PDSq{x}{\psi} + V\psi \\
H \psi^\conj &= - \frac{\hbar^2}{2m} \PDSq{x}{\psi^\conj} + V\psi^\conj \\
\end{align*}

Combining these we have
\begin{align*}
\PD{t}{<\hat{x}>}
&= 
\frac{i}{\hbar} \int \psi x \left( - \frac{\hbar^2}{2m} \PDSq{x}{\psi^\conj} + V\psi^\conj \right) 
-\frac{i}{\hbar} \int \psi^\conj x \left(- \frac{\hbar^2}{2m} \PDSq{x}{\psi} + V\psi \right) \\
&= 
\frac{i\hbar}{2m} \int \left(\psi^\conj x \PDSq{x}{\psi} -\psi x \PDSq{x}{\psi^\conj} \right)
+\frac{i}{\hbar} \int \left(\psi x V\psi^\conj -\psi^\conj x V\psi \right) 
\\
\end{align*}

The second term is zero, and by integrating the first term by parts twice we have

\begin{align*}
\PD{t}{<\hat{x}>}
&= \frac{i\hbar}{2m} \int \psi^\conj \left(x \PDSq{x}{\psi} - \PDSq{x}{(\psi x)} \right) \\
&= \frac{i\hbar}{2m} \int \psi^\conj \left(x \PDSq{x}{\psi} - \PD{x}{}\left(x \PD{x}{\psi} + \psi\right) \right) \\
&= \frac{-i\hbar}{2m} (2) \int \psi^\conj \PD{x}{\psi} \\
&= \frac{1}{m} \int \psi^\conj \left(-i\hbar \PD{x}{} \right) {\psi} \\
\end{align*}

So we now have the QM equivalent of $p = mv$, directly from the Sch\"{o}dinger equation and the definition of expectation values
of operators.

\begin{align}
\PD{t}{<\hat{x}>} &= \frac{<p>}{m} 
\end{align}

This is the first inkling that it makes sense to assign the names position and momentum to the corresponding operators
of QM!  Now the QMD derivation is way shorter and tidier, but this needed only integration by parts.  We really don't
need the more advanced operator concepts to get this important result.  

\subsection{ Force from the derivative of the momentum operator expectation. }

Now lets calculate the momentum expectation change with time.

\begin{align*}
\PD{t}{<p>} 
&= \PD{t}{} \int \psi^\conj \left(-i \hbar \PD{x}{}\right) \psi \\
&= -i \hbar \int \PD{t}{\psi^\conj} \PD{x}{\psi} +{\psi^\conj} \PD{t}{}\PD{x}{\psi} \\
&= -i \hbar \int \PD{t}{\psi^\conj} \PD{x}{\psi} +{\psi^\conj} \PD{x}{}\PD{t}{\psi} \\
&= \int \PD{x}{\psi} H \psi^\conj - {\psi^\conj} \PD{x}{} H\psi \\
&= \int \PD{x}{\psi} H \psi^\conj + \PD{x}{\psi^\conj} H\psi \\
&= 
\int \PD{x}{\psi} \left(- \frac{\hbar^2}{2m} \PDSq{x}{\psi^\conj} + V\psi^\conj \right)
+ \PD{x}{\psi^\conj} \left(- \frac{\hbar^2}{2m} \PDSq{x}{\psi} + V\psi \right) 
\\
&= 
-
\frac{\hbar^2}{2m} 
\int \PD{x}{\psi} \PDSq{x}{\psi^\conj} + \PD{x}{\psi^\conj} \PDSq{x}{\psi} 
+\int \PD{x}{\psi} V\psi^\conj + \PD{x}{\psi^\conj} V\psi 
\\
&= 
- \frac{\hbar^2}{2m} \int \PD{x}{}\left(\PD{x}{\psi} \PD{x}{\psi^\conj}\right)
+\int \PD{x}{} \left( \psi V\psi^\conj \right) -\int \psi \PD{x}{V} \psi^\conj 
\\
\end{align*}

Now, again with the assumption that $\psi$ and its derivatives are sufficiently small to vanish at the boundaries of the integration (this was also done in the integration by parts above), the first two terms are zero, and the last is an expectation value.  Specifically, we then have

\begin{align}
\PD{t}{<p>} &= - \left<\PD{x}{V}\right>
\end{align}

... which appears to be the QM equivalent to the one dimensional version of $F = -\grad V$, instead all defined in terms of expectation values.

Very cool!  Now, before learning the Lagrangian formalism, I would have been satisfied with this.  We can replace Newton's law with 
Schr\"{o}dinger's equation, and logically everything else will follow from that.  Can we apply a procedure like this to 
the Lagrangian for the wave equation, and find an expectation equivalent to the classical $\LL = m\Bv^2/2 - V$?

An additional obvious question is how to express the expectation value in the three dimensional case instead of the one dimensional case?

\bibliographystyle{plainnat}
\bibliography{myrefs}

\end{document}

%\documentclass{article}

%\usepackage{amsmath}
\usepackage{mathpazo}

%
% shorthand for bold symbols, convenient for vectors and matrices
%
\newcommand{\Ba}[0]{\mathbf{a}}
\newcommand{\Bb}[0]{\mathbf{b}}
\newcommand{\Bc}[0]{\mathbf{c}}
\newcommand{\Bd}[0]{\mathbf{d}}
\newcommand{\Be}[0]{\mathbf{e}}
\newcommand{\Bf}[0]{\mathbf{f}}
\newcommand{\Bg}[0]{\mathbf{g}}
\newcommand{\Bh}[0]{\mathbf{h}}
\newcommand{\Bi}[0]{\mathbf{i}}
\newcommand{\Bj}[0]{\mathbf{j}}
\newcommand{\Bk}[0]{\mathbf{k}}
\newcommand{\Bl}[0]{\mathbf{l}}
\newcommand{\Bm}[0]{\mathbf{m}}
\newcommand{\Bn}[0]{\mathbf{n}}
\newcommand{\Bo}[0]{\mathbf{o}}
\newcommand{\Bp}[0]{\mathbf{p}}
\newcommand{\Bq}[0]{\mathbf{q}}
\newcommand{\Br}[0]{\mathbf{r}}
\newcommand{\Bs}[0]{\mathbf{s}}
\newcommand{\Bt}[0]{\mathbf{t}}
\newcommand{\Bu}[0]{\mathbf{u}}
\newcommand{\Bv}[0]{\mathbf{v}}
\newcommand{\Bw}[0]{\mathbf{w}}
\newcommand{\Bx}[0]{\mathbf{x}}
\newcommand{\By}[0]{\mathbf{y}}
\newcommand{\Bz}[0]{\mathbf{z}}
\newcommand{\BA}[0]{\mathbf{A}}
\newcommand{\BB}[0]{\mathbf{B}}
\newcommand{\BC}[0]{\mathbf{C}}
\newcommand{\BD}[0]{\mathbf{D}}
\newcommand{\BE}[0]{\mathbf{E}}
\newcommand{\BF}[0]{\mathbf{F}}
\newcommand{\BG}[0]{\mathbf{G}}
\newcommand{\BH}[0]{\mathbf{H}}
\newcommand{\BI}[0]{\mathbf{I}}
\newcommand{\BJ}[0]{\mathbf{J}}
\newcommand{\BK}[0]{\mathbf{K}}
\newcommand{\BL}[0]{\mathbf{L}}
\newcommand{\BM}[0]{\mathbf{M}}
\newcommand{\BN}[0]{\mathbf{N}}
\newcommand{\BO}[0]{\mathbf{O}}
\newcommand{\BP}[0]{\mathbf{P}}
\newcommand{\BQ}[0]{\mathbf{Q}}
\newcommand{\BR}[0]{\mathbf{R}}
\newcommand{\BS}[0]{\mathbf{S}}
\newcommand{\BT}[0]{\mathbf{T}}
\newcommand{\BU}[0]{\mathbf{U}}
\newcommand{\BV}[0]{\mathbf{V}}
\newcommand{\BW}[0]{\mathbf{W}}
\newcommand{\BX}[0]{\mathbf{X}}
\newcommand{\BY}[0]{\mathbf{Y}}
\newcommand{\BZ}[0]{\mathbf{Z}}

\newcommand{\Bzero}[0]{\mathbf{0}}
\newcommand{\Btheta}[0]{\boldsymbol{\theta}}
\newcommand{\Btau}[0]{\boldsymbol{\tau}}
\newcommand{\Bomega}[0]{\boldsymbol{\omega}}

%
% shorthand for unit vectors
%
\newcommand{\acap}[0]{\hat{\Ba}}
\newcommand{\bcap}[0]{\hat{\Bb}}
\newcommand{\ccap}[0]{\hat{\Bc}}
\newcommand{\dcap}[0]{\hat{\Bd}}
\newcommand{\ecap}[0]{\hat{\Be}}
\newcommand{\fcap}[0]{\hat{\Bf}}
\newcommand{\gcap}[0]{\hat{\Bg}}
\newcommand{\hcap}[0]{\hat{\Bh}}
\newcommand{\icap}[0]{\hat{\Bi}}
\newcommand{\jcap}[0]{\hat{\Bj}}
\newcommand{\kcap}[0]{\hat{\Bk}}
\newcommand{\lcap}[0]{\hat{\Bl}}
\newcommand{\mcap}[0]{\hat{\Bm}}
\newcommand{\ncap}[0]{\hat{\Bn}}
\newcommand{\ocap}[0]{\hat{\Bo}}
\newcommand{\pcap}[0]{\hat{\Bp}}
\newcommand{\qcap}[0]{\hat{\Bq}}
\newcommand{\rcap}[0]{\hat{\Br}}
\newcommand{\scap}[0]{\hat{\Bs}}
\newcommand{\tcap}[0]{\hat{\Bt}}
\newcommand{\ucap}[0]{\hat{\Bu}}
\newcommand{\vcap}[0]{\hat{\Bv}}
\newcommand{\wcap}[0]{\hat{\Bw}}
\newcommand{\xcap}[0]{\hat{\Bx}}
\newcommand{\ycap}[0]{\hat{\By}}
\newcommand{\zcap}[0]{\hat{\Bz}}
\newcommand{\thetacap}[0]{\hat{\Btheta}}

%
% to write R^n and C^n in a distinguishable fashion.  Perhaps change this
% to the double lined characters upon figuring out how to do so.
%
\newcommand{\C}[1]{$\mathbb{C}^{#1}$}
\newcommand{\R}[1]{$\mathbb{R}^{#1}$}

%
% various generally useful helpers
%

% derivative of #1 wrt. #2:
\newcommand{\D}[2] {\frac {d#2} {d#1}}

\newcommand{\inv}[1]{\frac{1}{#1}}
\newcommand{\cross}[0]{\times}

\newcommand{\abs}[1]{\lvert{#1}\rvert}
\newcommand{\norm}[1]{\lVert{#1}\rVert}
\newcommand{\innerprod}[2]{\langle{#1}, {#2}\rangle}
\newcommand{\dotprod}[2]{{#1} \cdot {#2}}
\newcommand{\bdotprod}[2]{\left({#1} \cdot {#2}\right)}
\newcommand{\crossprod}[2]{{#1} \cross {#2}}
\newcommand{\tripleprod}[3]{\dotprod{\left(\crossprod{#1}{#2}\right)}{#3}}

\DeclareMathOperator{\Proj}{Proj}
\DeclareMathOperator{\Span}{span}
\DeclareMathOperator{\Sgn}{sgn}
\DeclareMathOperator{\Area}{Area}
\DeclareMathOperator{\Volume}{Volume}

%
% A few miscellaneous things specific to this document
%
\newcommand{\crossop}[1]{\crossprod{#1}{}}

% R2 vector.
\newcommand{\VectorTwo}[2]{
\begin{bmatrix}
 {#1} \\
 {#2}
\end{bmatrix}
}

\newcommand{\VectorN}[1]{
\begin{bmatrix}
{#1}_1 \\
{#1}_2 \\
\vdots \\
{#1}_N \\
\end{bmatrix}
}

\newcommand{\DETuvij}[4]{
\begin{vmatrix}
 {#1}_{#3} & {#1}_{#4} \\
 {#2}_{#3} & {#2}_{#4}
\end{vmatrix}
}

\newcommand{\DETuvwijk}[6]{
\begin{vmatrix}
 {#1}_{#4} & {#1}_{#5} & {#1}_{#6} \\
 {#2}_{#4} & {#2}_{#5} & {#2}_{#6} \\
 {#3}_{#4} & {#3}_{#5} & {#3}_{#6}
\end{vmatrix}
}

\newcommand{\DETuvwxijkl}[8]{
\begin{vmatrix}
 {#1}_{#5} & {#1}_{#6} & {#1}_{#7} & {#1}_{#8} \\
 {#2}_{#5} & {#2}_{#6} & {#2}_{#7} & {#2}_{#8} \\
 {#3}_{#5} & {#3}_{#6} & {#3}_{#7} & {#3}_{#8} \\
 {#4}_{#5} & {#4}_{#6} & {#4}_{#7} & {#4}_{#8} \\
\end{vmatrix}
}

%\newcommand{\DETuvwxyijklm}[10]{
%\begin{vmatrix}
% {#1}_{#6} & {#1}_{#7} & {#1}_{#8} & {#1}_{#9} & {#1}_{#10} \\
% {#2}_{#6} & {#2}_{#7} & {#2}_{#8} & {#2}_{#9} & {#2}_{#10} \\
% {#3}_{#6} & {#3}_{#7} & {#3}_{#8} & {#3}_{#9} & {#3}_{#10} \\
% {#4}_{#6} & {#4}_{#7} & {#4}_{#8} & {#4}_{#9} & {#4}_{#10} \\
% {#5}_{#6} & {#5}_{#7} & {#5}_{#8} & {#5}_{#9} & {#5}_{#10}
%\end{vmatrix}
%}

% R3 vector.
\newcommand{\VectorThree}[3]{
\begin{bmatrix}
 {#1} \\
 {#2} \\
 {#3}
\end{bmatrix}
}


%%<misc>
%
\newcommand{\Abs}[1]{{\left\lvert{#1}\right\rvert}}
\newcommand{\spacegrad}[0]{\boldsymbol{\nabla}}
\newcommand{\grad}[0]{\nabla}
\newcommand{\LL}[0]{\mathcal{L}}

% == \partial_{#1} {#2}
\newcommand{\PD}[2]{\frac{\partial {#2}}{\partial {#1}}}
% inline variant
\newcommand{\PDi}[2]{{\partial {#2}}/{\partial {#1}}}

\newcommand{\PDD}[3]{\frac{\partial^2 {#3}}{\partial {#1}\partial {#2}}}
%\newcommand{\PDd}[2]{\frac{\partial^2 {#2}}{{\partial{#1}}^2}}
\newcommand{\PDsq}[2]{\frac{\partial^2 {#2}}{(\partial {#1})^2}}

\newcommand{\Partial}[2]{\frac{\partial {#1}}{\partial {#2}}}
\DeclareMathOperator{\RejName}{Rej}
\newcommand{\Rej}[2]{\RejName_{#1}\left( {#2} \right)}
\newcommand{\Rm}[1]{\mathbb{R}^{#1}}
\newcommand{\Cm}[1]{\mathbb{C}^{#1}}
\newcommand{\conj}[0]{{*}}

%</misc>

% <grade selection>
%
\newcommand{\gpgrade}[2] {{\left\langle{{#1}}\right\rangle}_{#2}}

\newcommand{\gpgradezero}[1] {\gpgrade{#1}{}}
%\newcommand{\gpscalargrade}[1] {{\left\langle{{#1}}\right\rangle}}
%\newcommand{\gpgradezero}[1] {\gpgrade{#1}{0}}

%\newcommand{\gpgradeone}[1] {{\left\langle{{#1}}\right\rangle}_{1}}
\newcommand{\gpgradeone}[1] {\gpgrade{#1}{1}}

\newcommand{\gpgradetwo}[1] {\gpgrade{#1}{2}}
\newcommand{\gpgradethree}[1] {\gpgrade{#1}{3}}
\newcommand{\gpgradefour}[1] {\gpgrade{#1}{4}}
%
% </grade selection>



\newcommand{\adot}[0]{{\dot{a}}}
\newcommand{\bdot}[0]{{\dot{b}}}
% taken for centered dot:
%\newcommand{\cdot}[0]{{\dot{c}}}
%\newcommand{\ddot}[0]{{\dot{d}}}
\newcommand{\edot}[0]{{\dot{e}}}
\newcommand{\fdot}[0]{{\dot{f}}}
\newcommand{\gdot}[0]{{\dot{g}}}
\newcommand{\hdot}[0]{{\dot{h}}}
\newcommand{\idot}[0]{{\dot{i}}}
\newcommand{\jdot}[0]{{\dot{j}}}
\newcommand{\kdot}[0]{{\dot{k}}}
\newcommand{\ldot}[0]{{\dot{l}}}
\newcommand{\mdot}[0]{{\dot{m}}}
\newcommand{\ndot}[0]{{\dot{n}}}
%\newcommand{\odot}[0]{{\dot{o}}}
\newcommand{\pdot}[0]{{\dot{p}}}
\newcommand{\qdot}[0]{{\dot{q}}}
\newcommand{\rdot}[0]{{\dot{r}}}
\newcommand{\sdot}[0]{{\dot{s}}}
\newcommand{\tdot}[0]{{\dot{t}}}
\newcommand{\udot}[0]{{\dot{u}}}
\newcommand{\vdot}[0]{{\dot{v}}}
\newcommand{\wdot}[0]{{\dot{w}}}
\newcommand{\xdot}[0]{{\dot{x}}}
\newcommand{\ydot}[0]{{\dot{y}}}
\newcommand{\zdot}[0]{{\dot{z}}}
\newcommand{\addot}[0]{{\ddot{a}}}
\newcommand{\bddot}[0]{{\ddot{b}}}
\newcommand{\cddot}[0]{{\ddot{c}}}
%\newcommand{\dddot}[0]{{\ddot{d}}}
\newcommand{\eddot}[0]{{\ddot{e}}}
\newcommand{\fddot}[0]{{\ddot{f}}}
\newcommand{\gddot}[0]{{\ddot{g}}}
\newcommand{\hddot}[0]{{\ddot{h}}}
\newcommand{\iddot}[0]{{\ddot{i}}}
\newcommand{\jddot}[0]{{\ddot{j}}}
\newcommand{\kddot}[0]{{\ddot{k}}}
\newcommand{\lddot}[0]{{\ddot{l}}}
\newcommand{\mddot}[0]{{\ddot{m}}}
\newcommand{\nddot}[0]{{\ddot{n}}}
\newcommand{\oddot}[0]{{\ddot{o}}}
\newcommand{\pddot}[0]{{\ddot{p}}}
\newcommand{\qddot}[0]{{\ddot{q}}}
\newcommand{\rddot}[0]{{\ddot{r}}}
\newcommand{\sddot}[0]{{\ddot{s}}}
\newcommand{\tddot}[0]{{\ddot{t}}}
\newcommand{\uddot}[0]{{\ddot{u}}}
\newcommand{\vddot}[0]{{\ddot{v}}}
\newcommand{\wddot}[0]{{\ddot{w}}}
\newcommand{\xddot}[0]{{\ddot{x}}}
\newcommand{\yddot}[0]{{\ddot{y}}}
\newcommand{\zddot}[0]{{\ddot{z}}}

%<bold and dot greek symbols>
%

\newcommand{\Deltadot}[0]{{\dot{\Delta}}}
\newcommand{\Gammadot}[0]{{\dot{\Gamma}}}
\newcommand{\Lambdadot}[0]{{\dot{\Lambda}}}
\newcommand{\Omegadot}[0]{{\dot{\Omega}}}
\newcommand{\Phidot}[0]{{\dot{\Phi}}}
\newcommand{\Pidot}[0]{{\dot{\Pi}}}
\newcommand{\Psidot}[0]{{\dot{\Psi}}}
\newcommand{\Sigmadot}[0]{{\dot{\Sigma}}}
\newcommand{\Thetadot}[0]{{\dot{\Theta}}}
\newcommand{\Upsilondot}[0]{{\dot{\Upsilon}}}
\newcommand{\Xidot}[0]{{\dot{\Xi}}}
\newcommand{\alphadot}[0]{{\dot{\alpha}}}
\newcommand{\betadot}[0]{{\dot{\beta}}}
\newcommand{\chidot}[0]{{\dot{\chi}}}
\newcommand{\deltadot}[0]{{\dot{\delta}}}
\newcommand{\epsilondot}[0]{{\dot{\epsilon}}}
\newcommand{\etadot}[0]{{\dot{\eta}}}
\newcommand{\gammadot}[0]{{\dot{\gamma}}}
\newcommand{\kappadot}[0]{{\dot{\kappa}}}
\newcommand{\lambdadot}[0]{{\dot{\lambda}}}
\newcommand{\mudot}[0]{{\dot{\mu}}}
\newcommand{\nudot}[0]{{\dot{\nu}}}
\newcommand{\omegadot}[0]{{\dot{\omega}}}
\newcommand{\phidot}[0]{{\dot{\phi}}}
\newcommand{\pidot}[0]{{\dot{\pi}}}
\newcommand{\psidot}[0]{{\dot{\psi}}}
\newcommand{\rhodot}[0]{{\dot{\rho}}}
\newcommand{\sigmadot}[0]{{\dot{\sigma}}}
\newcommand{\taudot}[0]{{\dot{\tau}}}
\newcommand{\thetadot}[0]{{\dot{\theta}}}
\newcommand{\upsilondot}[0]{{\dot{\upsilon}}}
\newcommand{\varepsilondot}[0]{{\dot{\varepsilon}}}
\newcommand{\varphidot}[0]{{\dot{\varphi}}}
\newcommand{\varpidot}[0]{{\dot{\varpi}}}
\newcommand{\varrhodot}[0]{{\dot{\varrho}}}
\newcommand{\varsigmadot}[0]{{\dot{\varsigma}}}
\newcommand{\varthetadot}[0]{{\dot{\vartheta}}}
\newcommand{\xidot}[0]{{\dot{\xi}}}
\newcommand{\zetadot}[0]{{\dot{\zeta}}}

\newcommand{\Deltaddot}[0]{{\ddot{\Delta}}}
\newcommand{\Gammaddot}[0]{{\ddot{\Gamma}}}
\newcommand{\Lambdaddot}[0]{{\ddot{\Lambda}}}
\newcommand{\Omegaddot}[0]{{\ddot{\Omega}}}
\newcommand{\Phiddot}[0]{{\ddot{\Phi}}}
\newcommand{\Piddot}[0]{{\ddot{\Pi}}}
\newcommand{\Psiddot}[0]{{\ddot{\Psi}}}
\newcommand{\Sigmaddot}[0]{{\ddot{\Sigma}}}
\newcommand{\Thetaddot}[0]{{\ddot{\Theta}}}
\newcommand{\Upsilonddot}[0]{{\ddot{\Upsilon}}}
\newcommand{\Xiddot}[0]{{\ddot{\Xi}}}
\newcommand{\alphaddot}[0]{{\ddot{\alpha}}}
\newcommand{\betaddot}[0]{{\ddot{\beta}}}
\newcommand{\chiddot}[0]{{\ddot{\chi}}}
\newcommand{\deltaddot}[0]{{\ddot{\delta}}}
\newcommand{\epsilonddot}[0]{{\ddot{\epsilon}}}
\newcommand{\etaddot}[0]{{\ddot{\eta}}}
\newcommand{\gammaddot}[0]{{\ddot{\gamma}}}
\newcommand{\kappaddot}[0]{{\ddot{\kappa}}}
\newcommand{\lambdaddot}[0]{{\ddot{\lambda}}}
\newcommand{\muddot}[0]{{\ddot{\mu}}}
\newcommand{\nuddot}[0]{{\ddot{\nu}}}
\newcommand{\omegaddot}[0]{{\ddot{\omega}}}
\newcommand{\phiddot}[0]{{\ddot{\phi}}}
\newcommand{\piddot}[0]{{\ddot{\pi}}}
\newcommand{\psiddot}[0]{{\ddot{\psi}}}
\newcommand{\rhoddot}[0]{{\ddot{\rho}}}
\newcommand{\sigmaddot}[0]{{\ddot{\sigma}}}
\newcommand{\tauddot}[0]{{\ddot{\tau}}}
\newcommand{\thetaddot}[0]{{\ddot{\theta}}}
\newcommand{\upsilonddot}[0]{{\ddot{\upsilon}}}
\newcommand{\varepsilonddot}[0]{{\ddot{\varepsilon}}}
\newcommand{\varphiddot}[0]{{\ddot{\varphi}}}
\newcommand{\varpiddot}[0]{{\ddot{\varpi}}}
\newcommand{\varrhoddot}[0]{{\ddot{\varrho}}}
\newcommand{\varsigmaddot}[0]{{\ddot{\varsigma}}}
\newcommand{\varthetaddot}[0]{{\ddot{\vartheta}}}
\newcommand{\xiddot}[0]{{\ddot{\xi}}}
\newcommand{\zetaddot}[0]{{\ddot{\zeta}}}

\newcommand{\BDelta}[0]{\boldsymbol{\Delta}}
\newcommand{\BGamma}[0]{\boldsymbol{\Gamma}}
\newcommand{\BLambda}[0]{\boldsymbol{\Lambda}}
\newcommand{\BOmega}[0]{\boldsymbol{\Omega}}
\newcommand{\BPhi}[0]{\boldsymbol{\Phi}}
\newcommand{\BPi}[0]{\boldsymbol{\Pi}}
\newcommand{\BPsi}[0]{\boldsymbol{\Psi}}
\newcommand{\BSigma}[0]{\boldsymbol{\Sigma}}
\newcommand{\BTheta}[0]{\boldsymbol{\Theta}}
\newcommand{\BUpsilon}[0]{\boldsymbol{\Upsilon}}
\newcommand{\BXi}[0]{\boldsymbol{\Xi}}
\newcommand{\Balpha}[0]{\boldsymbol{\alpha}}
\newcommand{\Bbeta}[0]{\boldsymbol{\beta}}
\newcommand{\Bchi}[0]{\boldsymbol{\chi}}
\newcommand{\Bdelta}[0]{\boldsymbol{\delta}}
\newcommand{\Bepsilon}[0]{\boldsymbol{\epsilon}}
\newcommand{\Beta}[0]{\boldsymbol{\eta}}
\newcommand{\Bgamma}[0]{\boldsymbol{\gamma}}
\newcommand{\Bkappa}[0]{\boldsymbol{\kappa}}
\newcommand{\Blambda}[0]{\boldsymbol{\lambda}}
\newcommand{\Bmu}[0]{\boldsymbol{\mu}}
\newcommand{\Bnu}[0]{\boldsymbol{\nu}}
%\newcommand{\Bomega}[0]{\boldsymbol{\omega}}
\newcommand{\Bphi}[0]{\boldsymbol{\phi}}
\newcommand{\Bpi}[0]{\boldsymbol{\pi}}
\newcommand{\Bpsi}[0]{\boldsymbol{\psi}}
\newcommand{\Brho}[0]{\boldsymbol{\rho}}
\newcommand{\Bsigma}[0]{\boldsymbol{\sigma}}
%\newcommand{\Btau}[0]{\boldsymbol{\tau}}
%\newcommand{\Btheta}[0]{\boldsymbol{\theta}}
\newcommand{\Bupsilon}[0]{\boldsymbol{\upsilon}}
\newcommand{\Bvarepsilon}[0]{\boldsymbol{\varepsilon}}
\newcommand{\Bvarphi}[0]{\boldsymbol{\varphi}}
\newcommand{\Bvarpi}[0]{\boldsymbol{\varpi}}
\newcommand{\Bvarrho}[0]{\boldsymbol{\varrho}}
\newcommand{\Bvarsigma}[0]{\boldsymbol{\varsigma}}
\newcommand{\Bvartheta}[0]{\boldsymbol{\vartheta}}
\newcommand{\Bxi}[0]{\boldsymbol{\xi}}
\newcommand{\Bzeta}[0]{\boldsymbol{\zeta}}
%
%</bold and dot greek symbols>
%<infrequent>
%
%\newcommand{\AreaOp}[1]{\AName_{#1}}
%\newcommand{\Babs}[0]{\abs{\BB}}
%\newcommand{\Bcap}[0]{\hat{\BB}}
%\newcommand{\BrPrimeRej}[0]{\rcap(\rcap \wedge \Br')}
%\newcommand{\CA}[0]{\mathcal{A}}
%\newcommand{\Cos}[1]{\cos{\left({#1}\right)}}
%\newcommand{\Det}[1] {\abs{#1}}
%\newcommand{\Dsq}[2] {\frac {\partial^2 {#1}} {\partial {#2}^2}}
%\newcommand{\Exp}[1]{\exp{\left({#1}\right)}}
%\newcommand{\Norm}[1]{\left\lVert{#1}\right\rVert}
%\newcommand{\Sin}[1]{\sin{\left({#1}\right)}}
%\newcommand{\T}[0]{\text{T}}
%\newcommand{\VolumeOp}[1]{\VName_{#1}}
%\newcommand{\agrad}[0]{\Ba \cdot \nabla}
%\newcommand{\alphacap}[0]{\hat{\boldsymbol{\alpha}}}
%\newcommand{\Fcap}[0]{\hat{\BF}}
%\newcommand{\bithree}[0]{{\Bi}_3}
%\newcommand{\bxa}[0]{\Bx\Ba}
%\newcommand{\coordvec}[2]{
%\newcommand{\costheta}[0]{\acap \cdot \xcap}
%\newcommand{\ddt}[1]{\ddot{#1}}
%\newcommand{\ddu}[1] {\frac {d{#1}} {du}}
%\newcommand{\dsqxj}[2] {\frac {\partial^2 {#1}} {\partial {x_{#2}}^2}}
%\newcommand{\dtheta}[1]{\frac{d {#1}}{d \theta}}
%\newcommand{\dt}[1]{\dot{#1}}
%\newcommand{\dt}[1]{\frac{d {#1}}{dt}}
%\newcommand{\dxj}[2] {\frac {\partial {#1}} {\partial {x_{#2}}}}
%\newcommand{\halfPhi}[0]{\frac{\phi}{2}}
%\newcommand{\half}[0]{\inv{2}}
%\newcommand{\inv}[1]{\frac{1}{#1}}
%\newcommand{\laplacian}[0]{\nabla^2}
%\newcommand{\matrixoftx}[3]{
%\newcommand{\nrrp}[0]{\norm{\rcap \wedge \Br'}}
%\newcommand{\oiint}{\bigcirc \hspace{-1.4em} \int \hspace{-.8em} \int}
%\newcommand{\transpose}[1]{{#1}^{\text{T}}}
%\newcommand{\transpose}[1]{{{#1}^{\TextTranspose}}}
%\newcommand{\transpose}[1]{{{#1}^{\text{T}}}}
%\newcommand{\barA}[0]{\bar{A}}
%\newcommand{\qbar}[0]{\bar{q}}
%\newcommand{\qdotbar}[0]{\dot{\bar{q}}}
%
%</infrequent>





%\usepackage[bookmarks=true]{hyperref}

%\usepackage{color,cite,graphicx}
   % use colour in the document, put your citations as [1-4]
   % rather than [1,2,3,4] (it looks nicer, and the extended LaTeX2e
   % graphics package. 
%\usepackage{latexsym,amssymb,epsf} % don't remember if these are
   % needed, but their inclusion can't do any damage

\chapter{Evaluating the Gaussian integral. }
%\author{Peeter Joot \quad peeter.joot@gmail.com}
\date{ Jan 05, 2009.  $RCSfile: gaussian.tex,v $ Last $Revision: 1.12 $ $Date: 2009/06/11 17:00:37 $ }

%\begin{document}

%\maketitle{}
%\tableofcontents
\section{Plain old Gaussian. }

QM solutions appear to involve a lot of Gaussian integrals.  Looking at one
of the problems in \cite{mcmahon2005qmd} I tried to recall how to evaluate
the simplest of these.  Google says the trick is squaring and polar 
substitution.  Let's try this.

Solve
\begin{align*}
I = \int_{-\infty}^\infty e^{-\alpha x^2} dx
\end{align*}

\begin{align*}
I^2 
&= \int_{x= -\infty}^\infty e^{-\alpha x^2} dx \int_{y = -\infty}^\infty e^{-\alpha y^2} dy \\
&= \int_{\theta=0}^{2\pi}\int_{r= 0}^\infty e^{-\alpha r^2} r dr d\theta \\
&= 2\pi 
{\left.
\frac{e^{-\alpha r^2}}{-2\alpha}
\right\vert}_{r= 0}^\infty  \\
&= \frac{\pi}{\alpha}
\end{align*}

So we have

\begin{align*}
I = \sqrt{\frac{\pi}{\alpha}}
\end{align*}

\section{A couple higher order Gaussian's and normalization exercise. }

In order to do the normalization exercise for

\begin{align}\label{eqn:gaussian:exercise}
\psi = \left(A e^{-\frac{x^2}{a}} +B x e^{-\frac{x^2}{b}}\right) e^{-ict}
\end{align}

We want to calculate
\begin{align*}
\int \psi^\conj \psi = 
\Abs{A}^2 e^{-2\frac{x^2}{a}} +\Abs{B}^2 x^2 e^{-2\frac{x^2}{b}}
+ (A\bar{B} + B\bar{A}) x e^{-x^2\left(\inv{a} + \inv{b}\right)}
\end{align*}

so we need the $n=1,2$ versions of the following Gaussian integrals

\begin{align*}
I_n = \int x^n e^{-\alpha x^2} dx
\end{align*}

The $n=1$ case is directly integrable:

\begin{align*}
I_1 
&= \int x e^{-\alpha x^2} dx \\
&= \int \left(\frac{e^{-\alpha x^2}}{-2\alpha}\right)' dx \\
&= 0
\end{align*}

(integration bounds are $\pm \infty$ so the exponential vanishes).

Next.  $I_2$ follows with integration by parts

\begin{align*}
I_2 
&= \int x^2 e^{-\alpha x^2} dx \\
&= \int x \left(x e^{-\alpha x^2}\right) dx \\
&= \int x \left(\frac{e^{-\alpha x^2}}{-2\alpha}\right)' dx \\
&= -\int \frac{e^{-\alpha x^2}}{-2\alpha} dx \\
&= \inv{2\alpha}\int e^{-\alpha x^2} dx \\
&= \inv{2\alpha} \sqrt{\frac{\pi}{\alpha}}
\end{align*}

So the normalization required for \ref{eqn:gaussian:exercise} is
\begin{align*}
1 = \Abs{A}^2 \sqrt{\frac{\pi a}{2}} + \frac{\Abs{B}^2}{2} 
\sqrt{\frac{\pi b}{2}} \frac{b}{2}
\end{align*}

The values $a$, and $b$ are presumably due to boundary conditions, and this then fixes $\Abs{A}$ in terms of $\Abs{B}$ or the other way around.

\section{Generalized. }

Let 

\begin{align}
I_n = \int_{-\infty}^\infty x^n e^{-\alpha x^2} dx
\end{align}

We've solved this for $I_0 = \sqrt{\pi/\alpha}$, $I_1 = 0$, and $I_2$.  A quick calculation shows that $I_{2k+1} = 0$ too:

\begin{align*}
I_n 
&= \int_{-\infty}^\infty x^n e^{-\alpha x^2} dx \\
&= \int_{0}^\infty x^n e^{-\alpha x^2} dx +\int_{-\infty}^0 x^n e^{-\alpha x^2} dx \\
&= \int_{0}^\infty x^n e^{-\alpha x^2} dx +\int_{\infty}^0 (-x)^n e^{-\alpha x^2} (-dx) \\
&= \int_{0}^\infty x^n e^{-\alpha x^2} dx +\int_0^{\infty} (-x)^n e^{-\alpha x^2} dx \\
&= \int_{0}^\infty (x^n + (-x)^n)e^{-\alpha x^2} dx \\
\end{align*}

But if $n$ is odd $(-x)^n = -x^n$, so this is zero.

Now, for $n$ even, we can integrate by parts, as done for $I_2$.

\begin{align*}
I_{2m}
&= \int x^{2m} e^{-\alpha x^2} dx \\
&= \int x^{2m-1} \left(x e^{-\alpha x^2}\right) dx \\
&= \int x^{2m-1} \left(\frac{e^{-\alpha x^2}}{-2\alpha}\right)' dx \\
&= -\int (2m-1) x^{2m-2} \frac{e^{-\alpha x^2}}{-2\alpha} dx \\
\end{align*}

This gives us a recurrence relationship for the even order terms
\begin{align}
I_{2m} &= \frac{2m-1}{2\alpha} I_{2m-2}.
%&= \inv{2\alpha} \sqrt{\frac{\pi}{\alpha}}
\end{align}

Expanding this explicitly for the first few $m$ shows the pattern

\begin{align*}
\begin{array}{l l l}
I_2 &= \frac{2-1}{2\alpha} I_0 &= \frac{1}{2\alpha} \sqrt{\frac{\pi}{\alpha}} \\
I_4 &= \frac{4-1}{2\alpha}\frac{2-1}{2\alpha} I_0 &= \frac{3.1}{(2\alpha)^2} \sqrt{\frac{\pi}{\alpha}} \\
I_6 &= \frac{6-1}{2\alpha}\frac{4-1}{2\alpha}\frac{2-1}{2\alpha} I_0 &= \frac{5.3.1}{(2\alpha)^3} \sqrt{\frac{\pi}{\alpha}} \\
\end{array}
\end{align*}

Or
\begin{align}
I_{0} &= \sqrt{\frac{\pi}{\alpha}} \\
I_{2m-1} &= 0 \\
I_{2m} &= \frac{(2m-1)(2m-3)\cdots(3)(1)}{(2\alpha)^m} \sqrt{\frac{\pi}{\alpha}}
\end{align}

%\bibliographystyle{plainnat}
%\bibliography{myrefs}

%\end{document}

\include{harmonic_osc}
\include{klein_gordon}
\include{qm_barrier}
\include{qm_susskind}
\include{qmb_first_try}
\include{qmd_ch2quiz}
%
% Copyright � 2012 Peeter Joot.  All Rights Reserved.
% Licenced as described in the file LICENSE under the root directory of this GIT repository.
%

%
%
%\documentclass{article}

%\usepackage{amsmath}
\usepackage{mathpazo}

%
% shorthand for bold symbols, convenient for vectors and matrices
%
\newcommand{\Ba}[0]{\mathbf{a}}
\newcommand{\Bb}[0]{\mathbf{b}}
\newcommand{\Bc}[0]{\mathbf{c}}
\newcommand{\Bd}[0]{\mathbf{d}}
\newcommand{\Be}[0]{\mathbf{e}}
\newcommand{\Bf}[0]{\mathbf{f}}
\newcommand{\Bg}[0]{\mathbf{g}}
\newcommand{\Bh}[0]{\mathbf{h}}
\newcommand{\Bi}[0]{\mathbf{i}}
\newcommand{\Bj}[0]{\mathbf{j}}
\newcommand{\Bk}[0]{\mathbf{k}}
\newcommand{\Bl}[0]{\mathbf{l}}
\newcommand{\Bm}[0]{\mathbf{m}}
\newcommand{\Bn}[0]{\mathbf{n}}
\newcommand{\Bo}[0]{\mathbf{o}}
\newcommand{\Bp}[0]{\mathbf{p}}
\newcommand{\Bq}[0]{\mathbf{q}}
\newcommand{\Br}[0]{\mathbf{r}}
\newcommand{\Bs}[0]{\mathbf{s}}
\newcommand{\Bt}[0]{\mathbf{t}}
\newcommand{\Bu}[0]{\mathbf{u}}
\newcommand{\Bv}[0]{\mathbf{v}}
\newcommand{\Bw}[0]{\mathbf{w}}
\newcommand{\Bx}[0]{\mathbf{x}}
\newcommand{\By}[0]{\mathbf{y}}
\newcommand{\Bz}[0]{\mathbf{z}}
\newcommand{\BA}[0]{\mathbf{A}}
\newcommand{\BB}[0]{\mathbf{B}}
\newcommand{\BC}[0]{\mathbf{C}}
\newcommand{\BD}[0]{\mathbf{D}}
\newcommand{\BE}[0]{\mathbf{E}}
\newcommand{\BF}[0]{\mathbf{F}}
\newcommand{\BG}[0]{\mathbf{G}}
\newcommand{\BH}[0]{\mathbf{H}}
\newcommand{\BI}[0]{\mathbf{I}}
\newcommand{\BJ}[0]{\mathbf{J}}
\newcommand{\BK}[0]{\mathbf{K}}
\newcommand{\BL}[0]{\mathbf{L}}
\newcommand{\BM}[0]{\mathbf{M}}
\newcommand{\BN}[0]{\mathbf{N}}
\newcommand{\BO}[0]{\mathbf{O}}
\newcommand{\BP}[0]{\mathbf{P}}
\newcommand{\BQ}[0]{\mathbf{Q}}
\newcommand{\BR}[0]{\mathbf{R}}
\newcommand{\BS}[0]{\mathbf{S}}
\newcommand{\BT}[0]{\mathbf{T}}
\newcommand{\BU}[0]{\mathbf{U}}
\newcommand{\BV}[0]{\mathbf{V}}
\newcommand{\BW}[0]{\mathbf{W}}
\newcommand{\BX}[0]{\mathbf{X}}
\newcommand{\BY}[0]{\mathbf{Y}}
\newcommand{\BZ}[0]{\mathbf{Z}}

\newcommand{\Bzero}[0]{\mathbf{0}}
\newcommand{\Btheta}[0]{\boldsymbol{\theta}}
\newcommand{\Btau}[0]{\boldsymbol{\tau}}
\newcommand{\Bomega}[0]{\boldsymbol{\omega}}

%
% shorthand for unit vectors
%
\newcommand{\acap}[0]{\hat{\Ba}}
\newcommand{\bcap}[0]{\hat{\Bb}}
\newcommand{\ccap}[0]{\hat{\Bc}}
\newcommand{\dcap}[0]{\hat{\Bd}}
\newcommand{\ecap}[0]{\hat{\Be}}
\newcommand{\fcap}[0]{\hat{\Bf}}
\newcommand{\gcap}[0]{\hat{\Bg}}
\newcommand{\hcap}[0]{\hat{\Bh}}
\newcommand{\icap}[0]{\hat{\Bi}}
\newcommand{\jcap}[0]{\hat{\Bj}}
\newcommand{\kcap}[0]{\hat{\Bk}}
\newcommand{\lcap}[0]{\hat{\Bl}}
\newcommand{\mcap}[0]{\hat{\Bm}}
\newcommand{\ncap}[0]{\hat{\Bn}}
\newcommand{\ocap}[0]{\hat{\Bo}}
\newcommand{\pcap}[0]{\hat{\Bp}}
\newcommand{\qcap}[0]{\hat{\Bq}}
\newcommand{\rcap}[0]{\hat{\Br}}
\newcommand{\scap}[0]{\hat{\Bs}}
\newcommand{\tcap}[0]{\hat{\Bt}}
\newcommand{\ucap}[0]{\hat{\Bu}}
\newcommand{\vcap}[0]{\hat{\Bv}}
\newcommand{\wcap}[0]{\hat{\Bw}}
\newcommand{\xcap}[0]{\hat{\Bx}}
\newcommand{\ycap}[0]{\hat{\By}}
\newcommand{\zcap}[0]{\hat{\Bz}}
\newcommand{\thetacap}[0]{\hat{\Btheta}}

%
% to write R^n and C^n in a distinguishable fashion.  Perhaps change this
% to the double lined characters upon figuring out how to do so.
%
\newcommand{\C}[1]{$\mathbb{C}^{#1}$}
\newcommand{\R}[1]{$\mathbb{R}^{#1}$}

%
% various generally useful helpers
%

% derivative of #1 wrt. #2:
\newcommand{\D}[2] {\frac {d#2} {d#1}}

\newcommand{\inv}[1]{\frac{1}{#1}}
\newcommand{\cross}[0]{\times}

\newcommand{\abs}[1]{\lvert{#1}\rvert}
\newcommand{\norm}[1]{\lVert{#1}\rVert}
\newcommand{\innerprod}[2]{\langle{#1}, {#2}\rangle}
\newcommand{\dotprod}[2]{{#1} \cdot {#2}}
\newcommand{\bdotprod}[2]{\left({#1} \cdot {#2}\right)}
\newcommand{\crossprod}[2]{{#1} \cross {#2}}
\newcommand{\tripleprod}[3]{\dotprod{\left(\crossprod{#1}{#2}\right)}{#3}}

\DeclareMathOperator{\Proj}{Proj}
\DeclareMathOperator{\Span}{span}
\DeclareMathOperator{\Sgn}{sgn}
\DeclareMathOperator{\Area}{Area}
\DeclareMathOperator{\Volume}{Volume}

%
% A few miscellaneous things specific to this document
%
\newcommand{\crossop}[1]{\crossprod{#1}{}}

% R2 vector.
\newcommand{\VectorTwo}[2]{
\begin{bmatrix}
 {#1} \\
 {#2}
\end{bmatrix}
}

\newcommand{\VectorN}[1]{
\begin{bmatrix}
{#1}_1 \\
{#1}_2 \\
\vdots \\
{#1}_N \\
\end{bmatrix}
}

\newcommand{\DETuvij}[4]{
\begin{vmatrix}
 {#1}_{#3} & {#1}_{#4} \\
 {#2}_{#3} & {#2}_{#4}
\end{vmatrix}
}

\newcommand{\DETuvwijk}[6]{
\begin{vmatrix}
 {#1}_{#4} & {#1}_{#5} & {#1}_{#6} \\
 {#2}_{#4} & {#2}_{#5} & {#2}_{#6} \\
 {#3}_{#4} & {#3}_{#5} & {#3}_{#6}
\end{vmatrix}
}

\newcommand{\DETuvwxijkl}[8]{
\begin{vmatrix}
 {#1}_{#5} & {#1}_{#6} & {#1}_{#7} & {#1}_{#8} \\
 {#2}_{#5} & {#2}_{#6} & {#2}_{#7} & {#2}_{#8} \\
 {#3}_{#5} & {#3}_{#6} & {#3}_{#7} & {#3}_{#8} \\
 {#4}_{#5} & {#4}_{#6} & {#4}_{#7} & {#4}_{#8} \\
\end{vmatrix}
}

%\newcommand{\DETuvwxyijklm}[10]{
%\begin{vmatrix}
% {#1}_{#6} & {#1}_{#7} & {#1}_{#8} & {#1}_{#9} & {#1}_{#10} \\
% {#2}_{#6} & {#2}_{#7} & {#2}_{#8} & {#2}_{#9} & {#2}_{#10} \\
% {#3}_{#6} & {#3}_{#7} & {#3}_{#8} & {#3}_{#9} & {#3}_{#10} \\
% {#4}_{#6} & {#4}_{#7} & {#4}_{#8} & {#4}_{#9} & {#4}_{#10} \\
% {#5}_{#6} & {#5}_{#7} & {#5}_{#8} & {#5}_{#9} & {#5}_{#10}
%\end{vmatrix}
%}

% R3 vector.
\newcommand{\VectorThree}[3]{
\begin{bmatrix}
 {#1} \\
 {#2} \\
 {#3}
\end{bmatrix}
}


%%<misc>
%
\newcommand{\Abs}[1]{{\left\lvert{#1}\right\rvert}}
\newcommand{\spacegrad}[0]{\boldsymbol{\nabla}}
\newcommand{\grad}[0]{\nabla}
\newcommand{\LL}[0]{\mathcal{L}}

% == \partial_{#1} {#2}
\newcommand{\PD}[2]{\frac{\partial {#2}}{\partial {#1}}}
% inline variant
\newcommand{\PDi}[2]{{\partial {#2}}/{\partial {#1}}}

\newcommand{\PDD}[3]{\frac{\partial^2 {#3}}{\partial {#1}\partial {#2}}}
%\newcommand{\PDd}[2]{\frac{\partial^2 {#2}}{{\partial{#1}}^2}}
\newcommand{\PDsq}[2]{\frac{\partial^2 {#2}}{(\partial {#1})^2}}

\newcommand{\Partial}[2]{\frac{\partial {#1}}{\partial {#2}}}
\DeclareMathOperator{\RejName}{Rej}
\newcommand{\Rej}[2]{\RejName_{#1}\left( {#2} \right)}
\newcommand{\Rm}[1]{\mathbb{R}^{#1}}
\newcommand{\Cm}[1]{\mathbb{C}^{#1}}
\newcommand{\conj}[0]{{*}}

%</misc>

% <grade selection>
%
\newcommand{\gpgrade}[2] {{\left\langle{{#1}}\right\rangle}_{#2}}

\newcommand{\gpgradezero}[1] {\gpgrade{#1}{}}
%\newcommand{\gpscalargrade}[1] {{\left\langle{{#1}}\right\rangle}}
%\newcommand{\gpgradezero}[1] {\gpgrade{#1}{0}}

%\newcommand{\gpgradeone}[1] {{\left\langle{{#1}}\right\rangle}_{1}}
\newcommand{\gpgradeone}[1] {\gpgrade{#1}{1}}

\newcommand{\gpgradetwo}[1] {\gpgrade{#1}{2}}
\newcommand{\gpgradethree}[1] {\gpgrade{#1}{3}}
\newcommand{\gpgradefour}[1] {\gpgrade{#1}{4}}
%
% </grade selection>



\newcommand{\adot}[0]{{\dot{a}}}
\newcommand{\bdot}[0]{{\dot{b}}}
% taken for centered dot:
%\newcommand{\cdot}[0]{{\dot{c}}}
%\newcommand{\ddot}[0]{{\dot{d}}}
\newcommand{\edot}[0]{{\dot{e}}}
\newcommand{\fdot}[0]{{\dot{f}}}
\newcommand{\gdot}[0]{{\dot{g}}}
\newcommand{\hdot}[0]{{\dot{h}}}
\newcommand{\idot}[0]{{\dot{i}}}
\newcommand{\jdot}[0]{{\dot{j}}}
\newcommand{\kdot}[0]{{\dot{k}}}
\newcommand{\ldot}[0]{{\dot{l}}}
\newcommand{\mdot}[0]{{\dot{m}}}
\newcommand{\ndot}[0]{{\dot{n}}}
%\newcommand{\odot}[0]{{\dot{o}}}
\newcommand{\pdot}[0]{{\dot{p}}}
\newcommand{\qdot}[0]{{\dot{q}}}
\newcommand{\rdot}[0]{{\dot{r}}}
\newcommand{\sdot}[0]{{\dot{s}}}
\newcommand{\tdot}[0]{{\dot{t}}}
\newcommand{\udot}[0]{{\dot{u}}}
\newcommand{\vdot}[0]{{\dot{v}}}
\newcommand{\wdot}[0]{{\dot{w}}}
\newcommand{\xdot}[0]{{\dot{x}}}
\newcommand{\ydot}[0]{{\dot{y}}}
\newcommand{\zdot}[0]{{\dot{z}}}
\newcommand{\addot}[0]{{\ddot{a}}}
\newcommand{\bddot}[0]{{\ddot{b}}}
\newcommand{\cddot}[0]{{\ddot{c}}}
%\newcommand{\dddot}[0]{{\ddot{d}}}
\newcommand{\eddot}[0]{{\ddot{e}}}
\newcommand{\fddot}[0]{{\ddot{f}}}
\newcommand{\gddot}[0]{{\ddot{g}}}
\newcommand{\hddot}[0]{{\ddot{h}}}
\newcommand{\iddot}[0]{{\ddot{i}}}
\newcommand{\jddot}[0]{{\ddot{j}}}
\newcommand{\kddot}[0]{{\ddot{k}}}
\newcommand{\lddot}[0]{{\ddot{l}}}
\newcommand{\mddot}[0]{{\ddot{m}}}
\newcommand{\nddot}[0]{{\ddot{n}}}
\newcommand{\oddot}[0]{{\ddot{o}}}
\newcommand{\pddot}[0]{{\ddot{p}}}
\newcommand{\qddot}[0]{{\ddot{q}}}
\newcommand{\rddot}[0]{{\ddot{r}}}
\newcommand{\sddot}[0]{{\ddot{s}}}
\newcommand{\tddot}[0]{{\ddot{t}}}
\newcommand{\uddot}[0]{{\ddot{u}}}
\newcommand{\vddot}[0]{{\ddot{v}}}
\newcommand{\wddot}[0]{{\ddot{w}}}
\newcommand{\xddot}[0]{{\ddot{x}}}
\newcommand{\yddot}[0]{{\ddot{y}}}
\newcommand{\zddot}[0]{{\ddot{z}}}

%<bold and dot greek symbols>
%

\newcommand{\Deltadot}[0]{{\dot{\Delta}}}
\newcommand{\Gammadot}[0]{{\dot{\Gamma}}}
\newcommand{\Lambdadot}[0]{{\dot{\Lambda}}}
\newcommand{\Omegadot}[0]{{\dot{\Omega}}}
\newcommand{\Phidot}[0]{{\dot{\Phi}}}
\newcommand{\Pidot}[0]{{\dot{\Pi}}}
\newcommand{\Psidot}[0]{{\dot{\Psi}}}
\newcommand{\Sigmadot}[0]{{\dot{\Sigma}}}
\newcommand{\Thetadot}[0]{{\dot{\Theta}}}
\newcommand{\Upsilondot}[0]{{\dot{\Upsilon}}}
\newcommand{\Xidot}[0]{{\dot{\Xi}}}
\newcommand{\alphadot}[0]{{\dot{\alpha}}}
\newcommand{\betadot}[0]{{\dot{\beta}}}
\newcommand{\chidot}[0]{{\dot{\chi}}}
\newcommand{\deltadot}[0]{{\dot{\delta}}}
\newcommand{\epsilondot}[0]{{\dot{\epsilon}}}
\newcommand{\etadot}[0]{{\dot{\eta}}}
\newcommand{\gammadot}[0]{{\dot{\gamma}}}
\newcommand{\kappadot}[0]{{\dot{\kappa}}}
\newcommand{\lambdadot}[0]{{\dot{\lambda}}}
\newcommand{\mudot}[0]{{\dot{\mu}}}
\newcommand{\nudot}[0]{{\dot{\nu}}}
\newcommand{\omegadot}[0]{{\dot{\omega}}}
\newcommand{\phidot}[0]{{\dot{\phi}}}
\newcommand{\pidot}[0]{{\dot{\pi}}}
\newcommand{\psidot}[0]{{\dot{\psi}}}
\newcommand{\rhodot}[0]{{\dot{\rho}}}
\newcommand{\sigmadot}[0]{{\dot{\sigma}}}
\newcommand{\taudot}[0]{{\dot{\tau}}}
\newcommand{\thetadot}[0]{{\dot{\theta}}}
\newcommand{\upsilondot}[0]{{\dot{\upsilon}}}
\newcommand{\varepsilondot}[0]{{\dot{\varepsilon}}}
\newcommand{\varphidot}[0]{{\dot{\varphi}}}
\newcommand{\varpidot}[0]{{\dot{\varpi}}}
\newcommand{\varrhodot}[0]{{\dot{\varrho}}}
\newcommand{\varsigmadot}[0]{{\dot{\varsigma}}}
\newcommand{\varthetadot}[0]{{\dot{\vartheta}}}
\newcommand{\xidot}[0]{{\dot{\xi}}}
\newcommand{\zetadot}[0]{{\dot{\zeta}}}

\newcommand{\Deltaddot}[0]{{\ddot{\Delta}}}
\newcommand{\Gammaddot}[0]{{\ddot{\Gamma}}}
\newcommand{\Lambdaddot}[0]{{\ddot{\Lambda}}}
\newcommand{\Omegaddot}[0]{{\ddot{\Omega}}}
\newcommand{\Phiddot}[0]{{\ddot{\Phi}}}
\newcommand{\Piddot}[0]{{\ddot{\Pi}}}
\newcommand{\Psiddot}[0]{{\ddot{\Psi}}}
\newcommand{\Sigmaddot}[0]{{\ddot{\Sigma}}}
\newcommand{\Thetaddot}[0]{{\ddot{\Theta}}}
\newcommand{\Upsilonddot}[0]{{\ddot{\Upsilon}}}
\newcommand{\Xiddot}[0]{{\ddot{\Xi}}}
\newcommand{\alphaddot}[0]{{\ddot{\alpha}}}
\newcommand{\betaddot}[0]{{\ddot{\beta}}}
\newcommand{\chiddot}[0]{{\ddot{\chi}}}
\newcommand{\deltaddot}[0]{{\ddot{\delta}}}
\newcommand{\epsilonddot}[0]{{\ddot{\epsilon}}}
\newcommand{\etaddot}[0]{{\ddot{\eta}}}
\newcommand{\gammaddot}[0]{{\ddot{\gamma}}}
\newcommand{\kappaddot}[0]{{\ddot{\kappa}}}
\newcommand{\lambdaddot}[0]{{\ddot{\lambda}}}
\newcommand{\muddot}[0]{{\ddot{\mu}}}
\newcommand{\nuddot}[0]{{\ddot{\nu}}}
\newcommand{\omegaddot}[0]{{\ddot{\omega}}}
\newcommand{\phiddot}[0]{{\ddot{\phi}}}
\newcommand{\piddot}[0]{{\ddot{\pi}}}
\newcommand{\psiddot}[0]{{\ddot{\psi}}}
\newcommand{\rhoddot}[0]{{\ddot{\rho}}}
\newcommand{\sigmaddot}[0]{{\ddot{\sigma}}}
\newcommand{\tauddot}[0]{{\ddot{\tau}}}
\newcommand{\thetaddot}[0]{{\ddot{\theta}}}
\newcommand{\upsilonddot}[0]{{\ddot{\upsilon}}}
\newcommand{\varepsilonddot}[0]{{\ddot{\varepsilon}}}
\newcommand{\varphiddot}[0]{{\ddot{\varphi}}}
\newcommand{\varpiddot}[0]{{\ddot{\varpi}}}
\newcommand{\varrhoddot}[0]{{\ddot{\varrho}}}
\newcommand{\varsigmaddot}[0]{{\ddot{\varsigma}}}
\newcommand{\varthetaddot}[0]{{\ddot{\vartheta}}}
\newcommand{\xiddot}[0]{{\ddot{\xi}}}
\newcommand{\zetaddot}[0]{{\ddot{\zeta}}}

\newcommand{\BDelta}[0]{\boldsymbol{\Delta}}
\newcommand{\BGamma}[0]{\boldsymbol{\Gamma}}
\newcommand{\BLambda}[0]{\boldsymbol{\Lambda}}
\newcommand{\BOmega}[0]{\boldsymbol{\Omega}}
\newcommand{\BPhi}[0]{\boldsymbol{\Phi}}
\newcommand{\BPi}[0]{\boldsymbol{\Pi}}
\newcommand{\BPsi}[0]{\boldsymbol{\Psi}}
\newcommand{\BSigma}[0]{\boldsymbol{\Sigma}}
\newcommand{\BTheta}[0]{\boldsymbol{\Theta}}
\newcommand{\BUpsilon}[0]{\boldsymbol{\Upsilon}}
\newcommand{\BXi}[0]{\boldsymbol{\Xi}}
\newcommand{\Balpha}[0]{\boldsymbol{\alpha}}
\newcommand{\Bbeta}[0]{\boldsymbol{\beta}}
\newcommand{\Bchi}[0]{\boldsymbol{\chi}}
\newcommand{\Bdelta}[0]{\boldsymbol{\delta}}
\newcommand{\Bepsilon}[0]{\boldsymbol{\epsilon}}
\newcommand{\Beta}[0]{\boldsymbol{\eta}}
\newcommand{\Bgamma}[0]{\boldsymbol{\gamma}}
\newcommand{\Bkappa}[0]{\boldsymbol{\kappa}}
\newcommand{\Blambda}[0]{\boldsymbol{\lambda}}
\newcommand{\Bmu}[0]{\boldsymbol{\mu}}
\newcommand{\Bnu}[0]{\boldsymbol{\nu}}
%\newcommand{\Bomega}[0]{\boldsymbol{\omega}}
\newcommand{\Bphi}[0]{\boldsymbol{\phi}}
\newcommand{\Bpi}[0]{\boldsymbol{\pi}}
\newcommand{\Bpsi}[0]{\boldsymbol{\psi}}
\newcommand{\Brho}[0]{\boldsymbol{\rho}}
\newcommand{\Bsigma}[0]{\boldsymbol{\sigma}}
%\newcommand{\Btau}[0]{\boldsymbol{\tau}}
%\newcommand{\Btheta}[0]{\boldsymbol{\theta}}
\newcommand{\Bupsilon}[0]{\boldsymbol{\upsilon}}
\newcommand{\Bvarepsilon}[0]{\boldsymbol{\varepsilon}}
\newcommand{\Bvarphi}[0]{\boldsymbol{\varphi}}
\newcommand{\Bvarpi}[0]{\boldsymbol{\varpi}}
\newcommand{\Bvarrho}[0]{\boldsymbol{\varrho}}
\newcommand{\Bvarsigma}[0]{\boldsymbol{\varsigma}}
\newcommand{\Bvartheta}[0]{\boldsymbol{\vartheta}}
\newcommand{\Bxi}[0]{\boldsymbol{\xi}}
\newcommand{\Bzeta}[0]{\boldsymbol{\zeta}}
%
%</bold and dot greek symbols>
%<infrequent>
%
%\newcommand{\AreaOp}[1]{\AName_{#1}}
%\newcommand{\Babs}[0]{\abs{\BB}}
%\newcommand{\Bcap}[0]{\hat{\BB}}
%\newcommand{\BrPrimeRej}[0]{\rcap(\rcap \wedge \Br')}
%\newcommand{\CA}[0]{\mathcal{A}}
%\newcommand{\Cos}[1]{\cos{\left({#1}\right)}}
%\newcommand{\Det}[1] {\abs{#1}}
%\newcommand{\Dsq}[2] {\frac {\partial^2 {#1}} {\partial {#2}^2}}
%\newcommand{\Exp}[1]{\exp{\left({#1}\right)}}
%\newcommand{\Norm}[1]{\left\lVert{#1}\right\rVert}
%\newcommand{\Sin}[1]{\sin{\left({#1}\right)}}
%\newcommand{\T}[0]{\text{T}}
%\newcommand{\VolumeOp}[1]{\VName_{#1}}
%\newcommand{\agrad}[0]{\Ba \cdot \nabla}
%\newcommand{\alphacap}[0]{\hat{\boldsymbol{\alpha}}}
%\newcommand{\Fcap}[0]{\hat{\BF}}
%\newcommand{\bithree}[0]{{\Bi}_3}
%\newcommand{\bxa}[0]{\Bx\Ba}
%\newcommand{\coordvec}[2]{
%\newcommand{\costheta}[0]{\acap \cdot \xcap}
%\newcommand{\ddt}[1]{\ddot{#1}}
%\newcommand{\ddu}[1] {\frac {d{#1}} {du}}
%\newcommand{\dsqxj}[2] {\frac {\partial^2 {#1}} {\partial {x_{#2}}^2}}
%\newcommand{\dtheta}[1]{\frac{d {#1}}{d \theta}}
%\newcommand{\dt}[1]{\dot{#1}}
%\newcommand{\dt}[1]{\frac{d {#1}}{dt}}
%\newcommand{\dxj}[2] {\frac {\partial {#1}} {\partial {x_{#2}}}}
%\newcommand{\halfPhi}[0]{\frac{\phi}{2}}
%\newcommand{\half}[0]{\inv{2}}
%\newcommand{\inv}[1]{\frac{1}{#1}}
%\newcommand{\laplacian}[0]{\nabla^2}
%\newcommand{\matrixoftx}[3]{
%\newcommand{\nrrp}[0]{\norm{\rcap \wedge \Br'}}
%\newcommand{\oiint}{\bigcirc \hspace{-1.4em} \int \hspace{-.8em} \int}
%\newcommand{\transpose}[1]{{#1}^{\text{T}}}
%\newcommand{\transpose}[1]{{{#1}^{\TextTranspose}}}
%\newcommand{\transpose}[1]{{{#1}^{\text{T}}}}
%\newcommand{\barA}[0]{\bar{A}}
%\newcommand{\qbar}[0]{\bar{q}}
%\newcommand{\qdotbar}[0]{\dot{\bar{q}}}
%
%</infrequent>





%\usepackage[bookmarks=true]{hyperref}

%\usepackage{color,cite,graphicx}
   % use colour in the document, put your citations as [1-4]
   % rather than [1,2,3,4] (it looks nicer, and the extended LaTeX2e
   % graphics package.
%\usepackage{latexsym,amssymb,epsf} % do not remember if these are
   % needed, but their inclusion can not do any damage


\chapter{Simple Wave Packet Examples}
\label{chap:wavepacket}
%\author{Peeter Joot \quad peeterjoot@protonmail.com}
\date{ Feb 16, 2009.  wavepacket.tex }

%\begin{document}

%\maketitle{}
%\tableofcontents
\section{Wave packet examples}

In \citep{bohm1989qt}, chapter 3, a couple explicit wave packet examples are given.
Some of this is a little hard to follow in the small set font of the text, and some details are missing.  Here the integrals are performed
in detail.

\subsection{Unweighted example. Equation 1}

Summing plane waves over a small range of frequencies

\begin{equation}\label{eqn:wavepacket:20}
\begin{aligned}
E_z(x)
&= \int_{k_0 -\Delta k}^{k_0 -\Delta k} dk e^{ik(x-x_0)} \\
&= {\left. \frac{e^{ik(x-x_0)}}{i(x-x_0)} \right\vert}_{k_0 -\Delta k}^{k_0 -\Delta k}  \\
&= \frac{e^{i(k_0 + \Delta k)(x-x_0)}}{i(x-x_0)} -\frac{e^{i(k_0 - \Delta k)(x-x_0)}}{i(x-x_0)} \\
&= 2 {e^{i k_0 (x-x_0)}}
 \frac{{e^{i \Delta k(x-x_0)}} -{e^{-i \Delta k(x-x_0)}}}{2 i (x-x_0)} \\
\end{aligned}
\end{equation}

Which is Bohm's equation 1.
\begin{equation}\label{eqn:wavepacket:40}
\begin{aligned}
E_z(x) &= 2 {e^{i k_0 (x-x_0)}} \frac{\sin( \Delta k (x-x_0))}{ x-x_0}
\end{aligned}
\end{equation}

\subsection{Gaussian weighting example. Equation 4}

Next example, also chosen for simplicity, uses a Gaussian weighting function

\begin{equation}\label{eqn:wavepacket:60}
\begin{aligned}
\psi
&= \int_\infty^\infty
\exp\left( - \frac{(k-k_0)^2}{2(\Delta k)^2} \right)
\exp\left( i k(x-x_0) \right) dk \\
\end{aligned}
\end{equation}

Next, is the sneaky/clever step of adding and subtracting \(i k_0 (x-x_0) - (x-x_0)^2 (\Delta k)^2/2\) to the exponentials, which gives

\begin{equation}\label{eqn:wavepacket:80}
\begin{aligned}
\psi
&=
\exp\left( i k_0 (x-x_0) - \frac{(x-x_0)^2}{2} (\Delta k)^2 \right) \\
&\quad \int_\infty^\infty
\exp\left(
- \frac{(k-k_0)^2}{2(\Delta k)^2}
+i (k-k_0)(x-x_0)
+\frac{(x-x_0)^2}{2} (\Delta k)^2
\right)
dk \\
\end{aligned}
\end{equation}

Looking at just the remaining integral part, say \(I\), we have
\begin{equation}\label{eqn:wavepacket:100}
\begin{aligned}
I &=
\int_\infty^\infty
\exp\left( \inv{2} \left(
i^2 \frac{(k-k_0)^2}{(\Delta k)^2}
+2i (k-k_0)(x-x_0)
+{(x-x_0)^2}{} (\Delta k)^2
\right)
\right)
dk \\
&=
\int_\infty^\infty
\exp\left( \inv{2} \left(
i \frac{k-k_0}{\Delta k}
+(x-x_0) \Delta k
\right)^2
\right)
dk \\
&=
\int_\infty^\infty
\exp\left( -\inv{2} \left(
\frac{k-k_0}{\Delta k}
-i(x-x_0) \Delta k
\right)^2
\right)
dk \\
\end{aligned}
\end{equation}

A change of variables \(u = (k-k_0)/{\Delta k} -i(x-x_0) \Delta k\), \(du = dk/{\Delta k}\) gives

\begin{equation}\label{eqn:wavepacket:120}
\begin{aligned}
I
&= \Delta k \int_\infty^\infty e^{ -u^2/2 } du \\
&= \sqrt{2 \pi} \Delta k
\end{aligned}
\end{equation}

Which gives

\begin{equation}\label{eqn:wavepacket:140}
\begin{aligned}
\psi(x)
&= \sqrt{2 \pi} \Delta k
\exp\left( i k_0 (x-x_0) - \frac{(x-x_0)^2}{2} (\Delta k)^2 \right) \\
\end{aligned}
\end{equation}

Off by a factor of \(\sqrt{\Delta k}\) compared to the text?  Typo in the book or a mistake above?

\subsection{Gaussian weighting with angular velocity and acceleration}

Next covered is a wave packet where the angular frequency is a function of the wave number, as in equation 8

\begin{equation}\label{eqn:wavepacket:160}
\begin{aligned}
E(x,t) &= \IIinf f(k - k_0) \exp\left( i k(x-x_0) - i\omega(k) t\right) dk
\end{aligned}
\end{equation}

With a Taylor series expansion of the frequency about \(k_0\)

\begin{equation}\label{eqn:wavepacket:180}
\begin{aligned}
\omega(k)
&= \omega(k_0) + \left(\PD{k}{\omega}\right)_{k=k_0} (k-k_0) + \left(\PDSq{k}{\omega}\right)_{k=k_0} \frac{(k-k_0)^2}{2} + \cdots \\
&= \omega_0 + V_g (k-k_0) + \alpha \frac{(k-k_0)^2}{2} + \cdots \\
\end{aligned}
\end{equation}

Here \(V_g\), and \(\alpha\) are the group velocity and accelerations respectively.  Now, I had never seen the group velocity expressed
this way, which seems a particularly simple way of putting it.
The example of how \(\omega = 2 \pi c/\lambda n(\lambda)\) can vary with index of refraction and wavelength is also nice.  I
imagine a light wave going through a water oil air transition and the angular frequency in each region causing dispersion and
reflection and path alteration effects.

Back to the second order approximation of the frequency, substituting back into the wave packet integral one has

\begin{equation}\label{eqn:wavepacket:200}
\begin{aligned}
E(x,t) &\approx \IIinf f(k - k_0) \exp\left( i k(x-x_0) - i
\left(\omega_0 + V_g (k-k_0) + \alpha \frac{(k-k_0)^2}{2} \right)
t\right) dk
\end{aligned}
\end{equation}

With \(\kappa = k - k_0\), \(\Delta x = x -x_0\) and the Gaussian weighting \(f(\kappa) = e^{-\kappa^2/2(\Delta k)^2}\) this is

\begin{equation}\label{eqn:wavepacket:220}
\begin{aligned}
\exp&( i k_0 (x-x_0) )
\IIinf
\exp\left( -\frac{\kappa^2}{2(\Delta k)^2} +i \kappa (x-x_0)
- i \left(\omega_0 + V_g \kappa + \alpha \frac{\kappa^2}{2} \right) t\right) dk \\
&=
\exp( i k_0 \Delta x -i \omega_0 t)
\IIinf
\exp\left(
i \kappa ( \Delta x - V_g t )
-\frac{\kappa^2}{2} \left(\inv{(\Delta k)^2} + i \alpha t \right)
\right) d\kappa \\
\end{aligned}
\end{equation}

With \(a = \Delta x - V_g t\) and \(b = \inv{(\Delta k)^2} + i \alpha t\), the exponential in the integral takes the form
\begin{equation}\label{eqn:wavepacket:240}
\begin{aligned}
\exp\left( i \kappa a - \kappa^2 \frac{b}{2} \right)
&= \exp\left( - \frac{b}{2}\left(-2 i \kappa \frac{a}{b} + \kappa^2 \right) \right)  \\
&= \exp\left( - \frac{b}{2}\left( \kappa - i \frac{a}{b} \right)^2 + \frac{b}{2}\left(\frac{ia}{b}\right)^2 \right)  \\
&= \exp\left( - \frac{b}{2}\left( \kappa - i \frac{a}{b} \right)^2 - \frac{a^2}{2b} \right)  \\
\end{aligned}
\end{equation}

Our wave packet is now

\begin{equation}\label{eqn:wavepacket:260}
\begin{aligned}
E(x,t)
&= \exp\left( i k_0 \Delta x -i \omega_0 t - \frac{( \Delta x - V_g t)^2 (\Delta k)^2}{2(1 + i \alpha t (\Delta k)^2)} \right)
\IIinf
\exp\left( - \frac{b}{2}\left( \kappa - i \frac{a}{b} \right)^2 \right)
d\kappa \\
\end{aligned}
\end{equation}

A change of vars \(u = \sqrt{b}(\kappa - ia/b)\) gives

\begin{equation}\label{eqn:wavepacket:280}
\begin{aligned}
E(x,t)
&= \exp\left( i k_0 \Delta x -i \omega_0 t - \frac{( \Delta x - V_g t)^2 (\Delta k)^2}{2(1 + i \alpha t (\Delta k)^2)} \right)
\frac{\Delta k}{\sqrt{1 + i \alpha t (\Delta k)^2}}
\IIinf \exp\left( - \frac{u^2}{2} \right) d\kappa \\
&= \exp\left( i k_0 \Delta x -i \omega_0 t - \frac{( \Delta x - V_g t)^2 (\Delta k)^2}{2(1 + i \alpha t (\Delta k)^2)} \right)
\frac{\sqrt{2\pi}\Delta k}{\sqrt{1 + i \alpha t (\Delta k)^2}} \\
&=
\frac{\sqrt{2\pi}\Delta k}{\sqrt{1 + i \alpha t (\Delta k)^2}}
\exp\left( i \left(k_0 \Delta x - \omega_0 t
+
\alpha t (\Delta k)^2 \frac{( \Delta x - V_g t)^2 (\Delta k)^2}{2(1 + \alpha^2 t^2 (\Delta k)^4)}
\right) \right) \times \\
&\exp\left(
-
\frac{( \Delta x - V_g t)^2 (\Delta k)^2}{2(1 + \alpha^2 t^2 (\Delta k)^4)}
\right) \\
\end{aligned}
\end{equation}

This is consistent with the result in the text and with \(\alpha = 0\) confirms that equation 4 did have a typo (irrelevant to the
intensity discussion).

%gnuplot> set xlabel "x-axis"
%gnuplot> set ylabel "t-axis"
%gnuplot> splot [x=75:100] [t=-10:10] exp(-10*(x-10*t)**2/(1+10*t*t*100))
%gnuplot> splot [x=75:100] [t=-10:100] exp(-10*(x-10*t)**2/(1+10*t*t*100))
%gnuplot> splot [x=75:100] [t=-100:100] exp(-10*(x-10*t)**2/(1+10*t*t*100))
%gnuplot> splot [x=75:100] [t=-100:100] exp(-10*(x-1*t)**2/(1+10*t*t*100))
%gnuplot> splot [x=75:100] [t=-100:100] exp(-10*(x-1*t)**2/(1+10*t*t*1000))
%gnuplot> splot [x=75:100] [t=-100:100] exp(-10*(x-1*t)**2/(1+10*t*t*10000))
%gnuplot> splot [x=75:100] [t=-100:100] exp(-10*(x-1*t)**2/(1+10*t*t*2))
%gnuplot> splot [x=-75:100] [t=-100:100] exp(-10*(x-1*t)**2/(1+10*t*t*2))

%\bibliographystyle{plainnat}
%\bibliography{myrefs}

%\end{document}


\part{Relativity.}
\include{acc_four_vector}
\include{pauli_four_vector_v}
\include{pauli_qm_relativity_intro}
\include{velocity_addition}

\part{Correspondance}
\include{voltage_current_resistance}

\part{taylors}
% 
% 
% 
% Copyright � 2012 Peeter Joot
% All Rights Reserved
% 
% This file may be reproduced and distributed in whole or in part, without fee, subject to the following conditions:
% 
% o The copyright notice above and this permission notice must be preserved complete on all complete or partial copies.
% 
% o Any translation or derived work must be approved by the author in writing before distribution.
% 
% o If you distribute this work in part, instructions for obtaining the complete version of this file must be included, and a means for obtaining a complete version provided.
% 
% 
% Exceptions to these rules may be granted for academic purposes: Write to the author and ask.
% 
% 
% 
%\documentclass{article}

%\usepackage{amsmath}
\usepackage{mathpazo}

%
% shorthand for bold symbols, convenient for vectors and matrices
%
\newcommand{\Ba}[0]{\mathbf{a}}
\newcommand{\Bb}[0]{\mathbf{b}}
\newcommand{\Bc}[0]{\mathbf{c}}
\newcommand{\Bd}[0]{\mathbf{d}}
\newcommand{\Be}[0]{\mathbf{e}}
\newcommand{\Bf}[0]{\mathbf{f}}
\newcommand{\Bg}[0]{\mathbf{g}}
\newcommand{\Bh}[0]{\mathbf{h}}
\newcommand{\Bi}[0]{\mathbf{i}}
\newcommand{\Bj}[0]{\mathbf{j}}
\newcommand{\Bk}[0]{\mathbf{k}}
\newcommand{\Bl}[0]{\mathbf{l}}
\newcommand{\Bm}[0]{\mathbf{m}}
\newcommand{\Bn}[0]{\mathbf{n}}
\newcommand{\Bo}[0]{\mathbf{o}}
\newcommand{\Bp}[0]{\mathbf{p}}
\newcommand{\Bq}[0]{\mathbf{q}}
\newcommand{\Br}[0]{\mathbf{r}}
\newcommand{\Bs}[0]{\mathbf{s}}
\newcommand{\Bt}[0]{\mathbf{t}}
\newcommand{\Bu}[0]{\mathbf{u}}
\newcommand{\Bv}[0]{\mathbf{v}}
\newcommand{\Bw}[0]{\mathbf{w}}
\newcommand{\Bx}[0]{\mathbf{x}}
\newcommand{\By}[0]{\mathbf{y}}
\newcommand{\Bz}[0]{\mathbf{z}}
\newcommand{\BA}[0]{\mathbf{A}}
\newcommand{\BB}[0]{\mathbf{B}}
\newcommand{\BC}[0]{\mathbf{C}}
\newcommand{\BD}[0]{\mathbf{D}}
\newcommand{\BE}[0]{\mathbf{E}}
\newcommand{\BF}[0]{\mathbf{F}}
\newcommand{\BG}[0]{\mathbf{G}}
\newcommand{\BH}[0]{\mathbf{H}}
\newcommand{\BI}[0]{\mathbf{I}}
\newcommand{\BJ}[0]{\mathbf{J}}
\newcommand{\BK}[0]{\mathbf{K}}
\newcommand{\BL}[0]{\mathbf{L}}
\newcommand{\BM}[0]{\mathbf{M}}
\newcommand{\BN}[0]{\mathbf{N}}
\newcommand{\BO}[0]{\mathbf{O}}
\newcommand{\BP}[0]{\mathbf{P}}
\newcommand{\BQ}[0]{\mathbf{Q}}
\newcommand{\BR}[0]{\mathbf{R}}
\newcommand{\BS}[0]{\mathbf{S}}
\newcommand{\BT}[0]{\mathbf{T}}
\newcommand{\BU}[0]{\mathbf{U}}
\newcommand{\BV}[0]{\mathbf{V}}
\newcommand{\BW}[0]{\mathbf{W}}
\newcommand{\BX}[0]{\mathbf{X}}
\newcommand{\BY}[0]{\mathbf{Y}}
\newcommand{\BZ}[0]{\mathbf{Z}}

\newcommand{\Bzero}[0]{\mathbf{0}}
\newcommand{\Btheta}[0]{\boldsymbol{\theta}}
\newcommand{\Btau}[0]{\boldsymbol{\tau}}
\newcommand{\Bomega}[0]{\boldsymbol{\omega}}

%
% shorthand for unit vectors
%
\newcommand{\acap}[0]{\hat{\Ba}}
\newcommand{\bcap}[0]{\hat{\Bb}}
\newcommand{\ccap}[0]{\hat{\Bc}}
\newcommand{\dcap}[0]{\hat{\Bd}}
\newcommand{\ecap}[0]{\hat{\Be}}
\newcommand{\fcap}[0]{\hat{\Bf}}
\newcommand{\gcap}[0]{\hat{\Bg}}
\newcommand{\hcap}[0]{\hat{\Bh}}
\newcommand{\icap}[0]{\hat{\Bi}}
\newcommand{\jcap}[0]{\hat{\Bj}}
\newcommand{\kcap}[0]{\hat{\Bk}}
\newcommand{\lcap}[0]{\hat{\Bl}}
\newcommand{\mcap}[0]{\hat{\Bm}}
\newcommand{\ncap}[0]{\hat{\Bn}}
\newcommand{\ocap}[0]{\hat{\Bo}}
\newcommand{\pcap}[0]{\hat{\Bp}}
\newcommand{\qcap}[0]{\hat{\Bq}}
\newcommand{\rcap}[0]{\hat{\Br}}
\newcommand{\scap}[0]{\hat{\Bs}}
\newcommand{\tcap}[0]{\hat{\Bt}}
\newcommand{\ucap}[0]{\hat{\Bu}}
\newcommand{\vcap}[0]{\hat{\Bv}}
\newcommand{\wcap}[0]{\hat{\Bw}}
\newcommand{\xcap}[0]{\hat{\Bx}}
\newcommand{\ycap}[0]{\hat{\By}}
\newcommand{\zcap}[0]{\hat{\Bz}}
\newcommand{\thetacap}[0]{\hat{\Btheta}}

%
% to write R^n and C^n in a distinguishable fashion.  Perhaps change this
% to the double lined characters upon figuring out how to do so.
%
\newcommand{\C}[1]{$\mathbb{C}^{#1}$}
\newcommand{\R}[1]{$\mathbb{R}^{#1}$}

%
% various generally useful helpers
%

% derivative of #1 wrt. #2:
\newcommand{\D}[2] {\frac {d#2} {d#1}}

\newcommand{\inv}[1]{\frac{1}{#1}}
\newcommand{\cross}[0]{\times}

\newcommand{\abs}[1]{\lvert{#1}\rvert}
\newcommand{\norm}[1]{\lVert{#1}\rVert}
\newcommand{\innerprod}[2]{\langle{#1}, {#2}\rangle}
\newcommand{\dotprod}[2]{{#1} \cdot {#2}}
\newcommand{\bdotprod}[2]{\left({#1} \cdot {#2}\right)}
\newcommand{\crossprod}[2]{{#1} \cross {#2}}
\newcommand{\tripleprod}[3]{\dotprod{\left(\crossprod{#1}{#2}\right)}{#3}}

\DeclareMathOperator{\Proj}{Proj}
\DeclareMathOperator{\Span}{span}
\DeclareMathOperator{\Sgn}{sgn}
\DeclareMathOperator{\Area}{Area}
\DeclareMathOperator{\Volume}{Volume}

%
% A few miscellaneous things specific to this document
%
\newcommand{\crossop}[1]{\crossprod{#1}{}}

% R2 vector.
\newcommand{\VectorTwo}[2]{
\begin{bmatrix}
 {#1} \\
 {#2}
\end{bmatrix}
}

\newcommand{\VectorN}[1]{
\begin{bmatrix}
{#1}_1 \\
{#1}_2 \\
\vdots \\
{#1}_N \\
\end{bmatrix}
}

\newcommand{\DETuvij}[4]{
\begin{vmatrix}
 {#1}_{#3} & {#1}_{#4} \\
 {#2}_{#3} & {#2}_{#4}
\end{vmatrix}
}

\newcommand{\DETuvwijk}[6]{
\begin{vmatrix}
 {#1}_{#4} & {#1}_{#5} & {#1}_{#6} \\
 {#2}_{#4} & {#2}_{#5} & {#2}_{#6} \\
 {#3}_{#4} & {#3}_{#5} & {#3}_{#6}
\end{vmatrix}
}

\newcommand{\DETuvwxijkl}[8]{
\begin{vmatrix}
 {#1}_{#5} & {#1}_{#6} & {#1}_{#7} & {#1}_{#8} \\
 {#2}_{#5} & {#2}_{#6} & {#2}_{#7} & {#2}_{#8} \\
 {#3}_{#5} & {#3}_{#6} & {#3}_{#7} & {#3}_{#8} \\
 {#4}_{#5} & {#4}_{#6} & {#4}_{#7} & {#4}_{#8} \\
\end{vmatrix}
}

%\newcommand{\DETuvwxyijklm}[10]{
%\begin{vmatrix}
% {#1}_{#6} & {#1}_{#7} & {#1}_{#8} & {#1}_{#9} & {#1}_{#10} \\
% {#2}_{#6} & {#2}_{#7} & {#2}_{#8} & {#2}_{#9} & {#2}_{#10} \\
% {#3}_{#6} & {#3}_{#7} & {#3}_{#8} & {#3}_{#9} & {#3}_{#10} \\
% {#4}_{#6} & {#4}_{#7} & {#4}_{#8} & {#4}_{#9} & {#4}_{#10} \\
% {#5}_{#6} & {#5}_{#7} & {#5}_{#8} & {#5}_{#9} & {#5}_{#10}
%\end{vmatrix}
%}

% R3 vector.
\newcommand{\VectorThree}[3]{
\begin{bmatrix}
 {#1} \\
 {#2} \\
 {#3}
\end{bmatrix}
}



%\usepackage[bookmarks=true]{hyperref}

\chapter{Taylor's theorem deviation.}
\label{chap:taylors}
%\author{Peeter Joot \quad peeter.joot@gmail.com}
\date{ Feb. 2, 2008. taylors.tex }

%\begin{document}

%\maketitle{}

\citep{hestenes1999nfc} presents a very simple derivation of Taylor's Theorem,
but I feel a
slightly different (dumber, but longer) presentation would be more effective.

In the same fashion, form the integral

\[
I = \int_{t}^{t+s} F'(u) du
\]

Now, observe that the this first order derivative can be written in
terms of it's second order derivative

\[
(u F'(u))' = F'(u) + u F''(u)
\]

So we could write

\begin{align*}
I &= \int_{t}^{t+s} ((u F'(u))' - u F''(u)) du \\
  &= {u F'(u)} \vert_{u=t}^{t+s} - \int_{t}^{t+s} u F''(u)) du \\
  &= (t+s) F'(t+s) - t F'(t) - \int_{t}^{t+s} u F''(u)) du \\
\end{align*}

This is true, but not the Taylor expansion we are used to.  Adjusting things slightly leaves a zero term at $u=t+s$, as follows:

\[
\left((t + s - u) F'(u)\right)' = -F'(u) + (t+s-u) F''(u)
\]

\begin{align*}
I &= F(t+s) - F(t) \\
 &= \int_{t}^{t+s} ( - ((t + s - u) F'(u))' + (t+s-u) F''(u) ) du \\
 &= - {(t + s - u) F'(u)} \vert_{u=t}^{t+s} + \int_{t}^{t+s} (t+s-u) F''(u) du \\
 &= s F'(t) + \int_{t}^{t+s} (t+s-u) F''(u) du \\
\end{align*}

This results in the first two order terms of the Tailor series, with an explicit remainder term

\[
F(t+s) = F(t) + s F'(t) + \int_{t}^{t+s} (t+s-u) F''(u) du
\]

This process can be continued for as many terms as desired.  Doing the calculation for the second order term yields

\[
\left(\frac{(t + s - u)^2}{2} F''(u)\right)' = -( t + s - u ) F''(u) + \frac{(t+s-u)}{2} F'''(u)
\]

And substituting that into the first order expansion above we have

\[
F(t+s) = F(t) + s F'(t) + \frac{s^2}{2} F''(t) + \int_{t}^{t+s} \frac{1}{2}(t+s-u)^2 F'''(u) du
\]

Induction produces n terms of Taylor's series with an explicit remainder

\[
F(t+s) = \sum_{k=0}^{n} \frac{s^k}{k!} \frac{d^k}{dt^k} F(t) +
                        \int_{t}^{t+s} \frac{1}{n!}(t+s-u)^n \frac{d^{n+1}}{dt^{n+1}} F(u) du
\]

To truly prove the infinite series result one would have to show that the remainder term tends to zero.

%\bibliographystyle{plainnat} % supposed to allow for \url use.
%\bibliography{myrefs}      % expects file "myrefs.bib"

%\end{document}               % End of document.


% END INCLUDES.
%-------------------------------------------------------

% from the template:

%\begin{thebibliography}{99}
%  \addcontentsline{toc}{chapter}{Bibliography}
%\bibitem{lamport} L. Lamport. {\bf \LaTeX \ A Document Preparation System}
%Addison-Wesley, California 1986.
%
%\end{thebibliography}

\bibliographystyle{plainnat}
  \addcontentsline{toc}{chapter}{Bibliography}
\bibliography{myrefs}

%Note the tag used to make an index entry. You may need to consult Lamport's
%book~\cite{lamport} for details of the procedure to make the index input
%file; \LaTeX \ will create a pre-index by listing all the tagged
%items in the file {\tt bookex.idx} then you edit this into
%a {\tt theindex} environment, as {\tt index.tex}.

%\documentclass[openany]{memoir}
\usepackage[]{makeidx}

\chapterstyle{ell}

\makeindex

\begin{document}

To solve various problems in physics, it can be advantageous
to express any arbitrary piecewise-smooth function as a Fourier Series
\index{Fourier Series}
composed of multiples of sine \index{sine} and cosine \index{cosine} functions.  These are used in \cite{acheson1990elementary}.

\printindex

\bibliography{myrefs}
\bibliographystyle{unsrturl}

\end{document}

%  \addcontentsline{toc}{chapter}{Index}

\end{document}
