\documentclass[12pt,leqno]{book}

\usepackage{amsmath,amssymb,amsfonts} % Typical maths resource packages
%\usepackage{graphics}                 % Packages to allow inclusion of graphics
\usepackage{graphicx}
\usepackage{color}                   % For creating coloured text and background

% for ointctr... (also appears to make "prettier" \int and \sum's)
% ... but messes up other stuff (grad vs \spacegrad).
\usepackage{txfonts} 

% levi.tex:
\usepackage{listings}
%\usepackage{latexsym,epsf}

\usepackage[bookmarks=true]{hyperref}

\parindent 1cm
\parskip 0.2cm
\topmargin 0.2cm
\oddsidemargin 1cm
\evensidemargin 0.5cm
\textwidth 15cm
\textheight 21cm

% how do these ones work?
\newtheorem{theorem}{Theorem}[section]
\newtheorem{proposition}[theorem]{Proposition}
\newtheorem{corollary}[theorem]{Corollary}
\newtheorem{lemma}[theorem]{Lemma}
\newtheorem{remark}[theorem]{Remark}
\newtheorem{definition}[theorem]{Definition}

\usepackage{amsmath}
\usepackage{mathpazo}

%
% shorthand for bold symbols, convenient for vectors and matrices
%
\newcommand{\Ba}[0]{\mathbf{a}}
\newcommand{\Bb}[0]{\mathbf{b}}
\newcommand{\Bc}[0]{\mathbf{c}}
\newcommand{\Bd}[0]{\mathbf{d}}
\newcommand{\Be}[0]{\mathbf{e}}
\newcommand{\Bf}[0]{\mathbf{f}}
\newcommand{\Bg}[0]{\mathbf{g}}
\newcommand{\Bh}[0]{\mathbf{h}}
\newcommand{\Bi}[0]{\mathbf{i}}
\newcommand{\Bj}[0]{\mathbf{j}}
\newcommand{\Bk}[0]{\mathbf{k}}
\newcommand{\Bl}[0]{\mathbf{l}}
\newcommand{\Bm}[0]{\mathbf{m}}
\newcommand{\Bn}[0]{\mathbf{n}}
\newcommand{\Bo}[0]{\mathbf{o}}
\newcommand{\Bp}[0]{\mathbf{p}}
\newcommand{\Bq}[0]{\mathbf{q}}
\newcommand{\Br}[0]{\mathbf{r}}
\newcommand{\Bs}[0]{\mathbf{s}}
\newcommand{\Bt}[0]{\mathbf{t}}
\newcommand{\Bu}[0]{\mathbf{u}}
\newcommand{\Bv}[0]{\mathbf{v}}
\newcommand{\Bw}[0]{\mathbf{w}}
\newcommand{\Bx}[0]{\mathbf{x}}
\newcommand{\By}[0]{\mathbf{y}}
\newcommand{\Bz}[0]{\mathbf{z}}
\newcommand{\BA}[0]{\mathbf{A}}
\newcommand{\BB}[0]{\mathbf{B}}
\newcommand{\BC}[0]{\mathbf{C}}
\newcommand{\BD}[0]{\mathbf{D}}
\newcommand{\BE}[0]{\mathbf{E}}
\newcommand{\BF}[0]{\mathbf{F}}
\newcommand{\BG}[0]{\mathbf{G}}
\newcommand{\BH}[0]{\mathbf{H}}
\newcommand{\BI}[0]{\mathbf{I}}
\newcommand{\BJ}[0]{\mathbf{J}}
\newcommand{\BK}[0]{\mathbf{K}}
\newcommand{\BL}[0]{\mathbf{L}}
\newcommand{\BM}[0]{\mathbf{M}}
\newcommand{\BN}[0]{\mathbf{N}}
\newcommand{\BO}[0]{\mathbf{O}}
\newcommand{\BP}[0]{\mathbf{P}}
\newcommand{\BQ}[0]{\mathbf{Q}}
\newcommand{\BR}[0]{\mathbf{R}}
\newcommand{\BS}[0]{\mathbf{S}}
\newcommand{\BT}[0]{\mathbf{T}}
\newcommand{\BU}[0]{\mathbf{U}}
\newcommand{\BV}[0]{\mathbf{V}}
\newcommand{\BW}[0]{\mathbf{W}}
\newcommand{\BX}[0]{\mathbf{X}}
\newcommand{\BY}[0]{\mathbf{Y}}
\newcommand{\BZ}[0]{\mathbf{Z}}

\newcommand{\Bzero}[0]{\mathbf{0}}
\newcommand{\Btheta}[0]{\boldsymbol{\theta}}
\newcommand{\Btau}[0]{\boldsymbol{\tau}}
\newcommand{\Bomega}[0]{\boldsymbol{\omega}}

%
% shorthand for unit vectors
%
\newcommand{\acap}[0]{\hat{\Ba}}
\newcommand{\bcap}[0]{\hat{\Bb}}
\newcommand{\ccap}[0]{\hat{\Bc}}
\newcommand{\dcap}[0]{\hat{\Bd}}
\newcommand{\ecap}[0]{\hat{\Be}}
\newcommand{\fcap}[0]{\hat{\Bf}}
\newcommand{\gcap}[0]{\hat{\Bg}}
\newcommand{\hcap}[0]{\hat{\Bh}}
\newcommand{\icap}[0]{\hat{\Bi}}
\newcommand{\jcap}[0]{\hat{\Bj}}
\newcommand{\kcap}[0]{\hat{\Bk}}
\newcommand{\lcap}[0]{\hat{\Bl}}
\newcommand{\mcap}[0]{\hat{\Bm}}
\newcommand{\ncap}[0]{\hat{\Bn}}
\newcommand{\ocap}[0]{\hat{\Bo}}
\newcommand{\pcap}[0]{\hat{\Bp}}
\newcommand{\qcap}[0]{\hat{\Bq}}
\newcommand{\rcap}[0]{\hat{\Br}}
\newcommand{\scap}[0]{\hat{\Bs}}
\newcommand{\tcap}[0]{\hat{\Bt}}
\newcommand{\ucap}[0]{\hat{\Bu}}
\newcommand{\vcap}[0]{\hat{\Bv}}
\newcommand{\wcap}[0]{\hat{\Bw}}
\newcommand{\xcap}[0]{\hat{\Bx}}
\newcommand{\ycap}[0]{\hat{\By}}
\newcommand{\zcap}[0]{\hat{\Bz}}
\newcommand{\thetacap}[0]{\hat{\Btheta}}

%
% to write R^n and C^n in a distinguishable fashion.  Perhaps change this
% to the double lined characters upon figuring out how to do so.
%
\newcommand{\C}[1]{$\mathbb{C}^{#1}$}
\newcommand{\R}[1]{$\mathbb{R}^{#1}$}

%
% various generally useful helpers
%

% derivative of #1 wrt. #2:
\newcommand{\D}[2] {\frac {d#2} {d#1}}

\newcommand{\inv}[1]{\frac{1}{#1}}
\newcommand{\cross}[0]{\times}

\newcommand{\abs}[1]{\lvert{#1}\rvert}
\newcommand{\norm}[1]{\lVert{#1}\rVert}
\newcommand{\innerprod}[2]{\langle{#1}, {#2}\rangle}
\newcommand{\dotprod}[2]{{#1} \cdot {#2}}
\newcommand{\bdotprod}[2]{\left({#1} \cdot {#2}\right)}
\newcommand{\crossprod}[2]{{#1} \cross {#2}}
\newcommand{\tripleprod}[3]{\dotprod{\left(\crossprod{#1}{#2}\right)}{#3}}

\DeclareMathOperator{\Proj}{Proj}
\DeclareMathOperator{\Span}{span}
\DeclareMathOperator{\Sgn}{sgn}
\DeclareMathOperator{\Area}{Area}
\DeclareMathOperator{\Volume}{Volume}

%
% A few miscellaneous things specific to this document
%
\newcommand{\crossop}[1]{\crossprod{#1}{}}

% R2 vector.
\newcommand{\VectorTwo}[2]{
\begin{bmatrix}
 {#1} \\
 {#2}
\end{bmatrix}
}

\newcommand{\VectorN}[1]{
\begin{bmatrix}
{#1}_1 \\
{#1}_2 \\
\vdots \\
{#1}_N \\
\end{bmatrix}
}

\newcommand{\DETuvij}[4]{
\begin{vmatrix}
 {#1}_{#3} & {#1}_{#4} \\
 {#2}_{#3} & {#2}_{#4}
\end{vmatrix}
}

\newcommand{\DETuvwijk}[6]{
\begin{vmatrix}
 {#1}_{#4} & {#1}_{#5} & {#1}_{#6} \\
 {#2}_{#4} & {#2}_{#5} & {#2}_{#6} \\
 {#3}_{#4} & {#3}_{#5} & {#3}_{#6}
\end{vmatrix}
}

\newcommand{\DETuvwxijkl}[8]{
\begin{vmatrix}
 {#1}_{#5} & {#1}_{#6} & {#1}_{#7} & {#1}_{#8} \\
 {#2}_{#5} & {#2}_{#6} & {#2}_{#7} & {#2}_{#8} \\
 {#3}_{#5} & {#3}_{#6} & {#3}_{#7} & {#3}_{#8} \\
 {#4}_{#5} & {#4}_{#6} & {#4}_{#7} & {#4}_{#8} \\
\end{vmatrix}
}

%\newcommand{\DETuvwxyijklm}[10]{
%\begin{vmatrix}
% {#1}_{#6} & {#1}_{#7} & {#1}_{#8} & {#1}_{#9} & {#1}_{#10} \\
% {#2}_{#6} & {#2}_{#7} & {#2}_{#8} & {#2}_{#9} & {#2}_{#10} \\
% {#3}_{#6} & {#3}_{#7} & {#3}_{#8} & {#3}_{#9} & {#3}_{#10} \\
% {#4}_{#6} & {#4}_{#7} & {#4}_{#8} & {#4}_{#9} & {#4}_{#10} \\
% {#5}_{#6} & {#5}_{#7} & {#5}_{#8} & {#5}_{#9} & {#5}_{#10}
%\end{vmatrix}
%}

% R3 vector.
\newcommand{\VectorThree}[3]{
\begin{bmatrix}
 {#1} \\
 {#2} \\
 {#3}
\end{bmatrix}
}


%<misc>
%
\newcommand{\Abs}[1]{{\left\lvert{#1}\right\rvert}}
\newcommand{\spacegrad}[0]{\boldsymbol{\nabla}}
\newcommand{\grad}[0]{\nabla}
\newcommand{\LL}[0]{\mathcal{L}}

% == \partial_{#1} {#2}
\newcommand{\PD}[2]{\frac{\partial {#2}}{\partial {#1}}}
% inline variant
\newcommand{\PDi}[2]{{\partial {#2}}/{\partial {#1}}}

\newcommand{\PDD}[3]{\frac{\partial^2 {#3}}{\partial {#1}\partial {#2}}}
%\newcommand{\PDd}[2]{\frac{\partial^2 {#2}}{{\partial{#1}}^2}}
\newcommand{\PDsq}[2]{\frac{\partial^2 {#2}}{(\partial {#1})^2}}

\newcommand{\Partial}[2]{\frac{\partial {#1}}{\partial {#2}}}
\DeclareMathOperator{\RejName}{Rej}
\newcommand{\Rej}[2]{\RejName_{#1}\left( {#2} \right)}
\newcommand{\Rm}[1]{\mathbb{R}^{#1}}
\newcommand{\Cm}[1]{\mathbb{C}^{#1}}
\newcommand{\conj}[0]{{*}}

%</misc>

% <grade selection>
%
\newcommand{\gpgrade}[2] {{\left\langle{{#1}}\right\rangle}_{#2}}

\newcommand{\gpgradezero}[1] {\gpgrade{#1}{}}
%\newcommand{\gpscalargrade}[1] {{\left\langle{{#1}}\right\rangle}}
%\newcommand{\gpgradezero}[1] {\gpgrade{#1}{0}}

%\newcommand{\gpgradeone}[1] {{\left\langle{{#1}}\right\rangle}_{1}}
\newcommand{\gpgradeone}[1] {\gpgrade{#1}{1}}

\newcommand{\gpgradetwo}[1] {\gpgrade{#1}{2}}
\newcommand{\gpgradethree}[1] {\gpgrade{#1}{3}}
\newcommand{\gpgradefour}[1] {\gpgrade{#1}{4}}
%
% </grade selection>



\newcommand{\adot}[0]{{\dot{a}}}
\newcommand{\bdot}[0]{{\dot{b}}}
% taken for centered dot:
%\newcommand{\cdot}[0]{{\dot{c}}}
%\newcommand{\ddot}[0]{{\dot{d}}}
\newcommand{\edot}[0]{{\dot{e}}}
\newcommand{\fdot}[0]{{\dot{f}}}
\newcommand{\gdot}[0]{{\dot{g}}}
\newcommand{\hdot}[0]{{\dot{h}}}
\newcommand{\idot}[0]{{\dot{i}}}
\newcommand{\jdot}[0]{{\dot{j}}}
\newcommand{\kdot}[0]{{\dot{k}}}
\newcommand{\ldot}[0]{{\dot{l}}}
\newcommand{\mdot}[0]{{\dot{m}}}
\newcommand{\ndot}[0]{{\dot{n}}}
%\newcommand{\odot}[0]{{\dot{o}}}
\newcommand{\pdot}[0]{{\dot{p}}}
\newcommand{\qdot}[0]{{\dot{q}}}
\newcommand{\rdot}[0]{{\dot{r}}}
\newcommand{\sdot}[0]{{\dot{s}}}
\newcommand{\tdot}[0]{{\dot{t}}}
\newcommand{\udot}[0]{{\dot{u}}}
\newcommand{\vdot}[0]{{\dot{v}}}
\newcommand{\wdot}[0]{{\dot{w}}}
\newcommand{\xdot}[0]{{\dot{x}}}
\newcommand{\ydot}[0]{{\dot{y}}}
\newcommand{\zdot}[0]{{\dot{z}}}
\newcommand{\addot}[0]{{\ddot{a}}}
\newcommand{\bddot}[0]{{\ddot{b}}}
\newcommand{\cddot}[0]{{\ddot{c}}}
%\newcommand{\dddot}[0]{{\ddot{d}}}
\newcommand{\eddot}[0]{{\ddot{e}}}
\newcommand{\fddot}[0]{{\ddot{f}}}
\newcommand{\gddot}[0]{{\ddot{g}}}
\newcommand{\hddot}[0]{{\ddot{h}}}
\newcommand{\iddot}[0]{{\ddot{i}}}
\newcommand{\jddot}[0]{{\ddot{j}}}
\newcommand{\kddot}[0]{{\ddot{k}}}
\newcommand{\lddot}[0]{{\ddot{l}}}
\newcommand{\mddot}[0]{{\ddot{m}}}
\newcommand{\nddot}[0]{{\ddot{n}}}
\newcommand{\oddot}[0]{{\ddot{o}}}
\newcommand{\pddot}[0]{{\ddot{p}}}
\newcommand{\qddot}[0]{{\ddot{q}}}
\newcommand{\rddot}[0]{{\ddot{r}}}
\newcommand{\sddot}[0]{{\ddot{s}}}
\newcommand{\tddot}[0]{{\ddot{t}}}
\newcommand{\uddot}[0]{{\ddot{u}}}
\newcommand{\vddot}[0]{{\ddot{v}}}
\newcommand{\wddot}[0]{{\ddot{w}}}
\newcommand{\xddot}[0]{{\ddot{x}}}
\newcommand{\yddot}[0]{{\ddot{y}}}
\newcommand{\zddot}[0]{{\ddot{z}}}

%<bold and dot greek symbols>
%

\newcommand{\Deltadot}[0]{{\dot{\Delta}}}
\newcommand{\Gammadot}[0]{{\dot{\Gamma}}}
\newcommand{\Lambdadot}[0]{{\dot{\Lambda}}}
\newcommand{\Omegadot}[0]{{\dot{\Omega}}}
\newcommand{\Phidot}[0]{{\dot{\Phi}}}
\newcommand{\Pidot}[0]{{\dot{\Pi}}}
\newcommand{\Psidot}[0]{{\dot{\Psi}}}
\newcommand{\Sigmadot}[0]{{\dot{\Sigma}}}
\newcommand{\Thetadot}[0]{{\dot{\Theta}}}
\newcommand{\Upsilondot}[0]{{\dot{\Upsilon}}}
\newcommand{\Xidot}[0]{{\dot{\Xi}}}
\newcommand{\alphadot}[0]{{\dot{\alpha}}}
\newcommand{\betadot}[0]{{\dot{\beta}}}
\newcommand{\chidot}[0]{{\dot{\chi}}}
\newcommand{\deltadot}[0]{{\dot{\delta}}}
\newcommand{\epsilondot}[0]{{\dot{\epsilon}}}
\newcommand{\etadot}[0]{{\dot{\eta}}}
\newcommand{\gammadot}[0]{{\dot{\gamma}}}
\newcommand{\kappadot}[0]{{\dot{\kappa}}}
\newcommand{\lambdadot}[0]{{\dot{\lambda}}}
\newcommand{\mudot}[0]{{\dot{\mu}}}
\newcommand{\nudot}[0]{{\dot{\nu}}}
\newcommand{\omegadot}[0]{{\dot{\omega}}}
\newcommand{\phidot}[0]{{\dot{\phi}}}
\newcommand{\pidot}[0]{{\dot{\pi}}}
\newcommand{\psidot}[0]{{\dot{\psi}}}
\newcommand{\rhodot}[0]{{\dot{\rho}}}
\newcommand{\sigmadot}[0]{{\dot{\sigma}}}
\newcommand{\taudot}[0]{{\dot{\tau}}}
\newcommand{\thetadot}[0]{{\dot{\theta}}}
\newcommand{\upsilondot}[0]{{\dot{\upsilon}}}
\newcommand{\varepsilondot}[0]{{\dot{\varepsilon}}}
\newcommand{\varphidot}[0]{{\dot{\varphi}}}
\newcommand{\varpidot}[0]{{\dot{\varpi}}}
\newcommand{\varrhodot}[0]{{\dot{\varrho}}}
\newcommand{\varsigmadot}[0]{{\dot{\varsigma}}}
\newcommand{\varthetadot}[0]{{\dot{\vartheta}}}
\newcommand{\xidot}[0]{{\dot{\xi}}}
\newcommand{\zetadot}[0]{{\dot{\zeta}}}

\newcommand{\Deltaddot}[0]{{\ddot{\Delta}}}
\newcommand{\Gammaddot}[0]{{\ddot{\Gamma}}}
\newcommand{\Lambdaddot}[0]{{\ddot{\Lambda}}}
\newcommand{\Omegaddot}[0]{{\ddot{\Omega}}}
\newcommand{\Phiddot}[0]{{\ddot{\Phi}}}
\newcommand{\Piddot}[0]{{\ddot{\Pi}}}
\newcommand{\Psiddot}[0]{{\ddot{\Psi}}}
\newcommand{\Sigmaddot}[0]{{\ddot{\Sigma}}}
\newcommand{\Thetaddot}[0]{{\ddot{\Theta}}}
\newcommand{\Upsilonddot}[0]{{\ddot{\Upsilon}}}
\newcommand{\Xiddot}[0]{{\ddot{\Xi}}}
\newcommand{\alphaddot}[0]{{\ddot{\alpha}}}
\newcommand{\betaddot}[0]{{\ddot{\beta}}}
\newcommand{\chiddot}[0]{{\ddot{\chi}}}
\newcommand{\deltaddot}[0]{{\ddot{\delta}}}
\newcommand{\epsilonddot}[0]{{\ddot{\epsilon}}}
\newcommand{\etaddot}[0]{{\ddot{\eta}}}
\newcommand{\gammaddot}[0]{{\ddot{\gamma}}}
\newcommand{\kappaddot}[0]{{\ddot{\kappa}}}
\newcommand{\lambdaddot}[0]{{\ddot{\lambda}}}
\newcommand{\muddot}[0]{{\ddot{\mu}}}
\newcommand{\nuddot}[0]{{\ddot{\nu}}}
\newcommand{\omegaddot}[0]{{\ddot{\omega}}}
\newcommand{\phiddot}[0]{{\ddot{\phi}}}
\newcommand{\piddot}[0]{{\ddot{\pi}}}
\newcommand{\psiddot}[0]{{\ddot{\psi}}}
\newcommand{\rhoddot}[0]{{\ddot{\rho}}}
\newcommand{\sigmaddot}[0]{{\ddot{\sigma}}}
\newcommand{\tauddot}[0]{{\ddot{\tau}}}
\newcommand{\thetaddot}[0]{{\ddot{\theta}}}
\newcommand{\upsilonddot}[0]{{\ddot{\upsilon}}}
\newcommand{\varepsilonddot}[0]{{\ddot{\varepsilon}}}
\newcommand{\varphiddot}[0]{{\ddot{\varphi}}}
\newcommand{\varpiddot}[0]{{\ddot{\varpi}}}
\newcommand{\varrhoddot}[0]{{\ddot{\varrho}}}
\newcommand{\varsigmaddot}[0]{{\ddot{\varsigma}}}
\newcommand{\varthetaddot}[0]{{\ddot{\vartheta}}}
\newcommand{\xiddot}[0]{{\ddot{\xi}}}
\newcommand{\zetaddot}[0]{{\ddot{\zeta}}}

\newcommand{\BDelta}[0]{\boldsymbol{\Delta}}
\newcommand{\BGamma}[0]{\boldsymbol{\Gamma}}
\newcommand{\BLambda}[0]{\boldsymbol{\Lambda}}
\newcommand{\BOmega}[0]{\boldsymbol{\Omega}}
\newcommand{\BPhi}[0]{\boldsymbol{\Phi}}
\newcommand{\BPi}[0]{\boldsymbol{\Pi}}
\newcommand{\BPsi}[0]{\boldsymbol{\Psi}}
\newcommand{\BSigma}[0]{\boldsymbol{\Sigma}}
\newcommand{\BTheta}[0]{\boldsymbol{\Theta}}
\newcommand{\BUpsilon}[0]{\boldsymbol{\Upsilon}}
\newcommand{\BXi}[0]{\boldsymbol{\Xi}}
\newcommand{\Balpha}[0]{\boldsymbol{\alpha}}
\newcommand{\Bbeta}[0]{\boldsymbol{\beta}}
\newcommand{\Bchi}[0]{\boldsymbol{\chi}}
\newcommand{\Bdelta}[0]{\boldsymbol{\delta}}
\newcommand{\Bepsilon}[0]{\boldsymbol{\epsilon}}
\newcommand{\Beta}[0]{\boldsymbol{\eta}}
\newcommand{\Bgamma}[0]{\boldsymbol{\gamma}}
\newcommand{\Bkappa}[0]{\boldsymbol{\kappa}}
\newcommand{\Blambda}[0]{\boldsymbol{\lambda}}
\newcommand{\Bmu}[0]{\boldsymbol{\mu}}
\newcommand{\Bnu}[0]{\boldsymbol{\nu}}
%\newcommand{\Bomega}[0]{\boldsymbol{\omega}}
\newcommand{\Bphi}[0]{\boldsymbol{\phi}}
\newcommand{\Bpi}[0]{\boldsymbol{\pi}}
\newcommand{\Bpsi}[0]{\boldsymbol{\psi}}
\newcommand{\Brho}[0]{\boldsymbol{\rho}}
\newcommand{\Bsigma}[0]{\boldsymbol{\sigma}}
%\newcommand{\Btau}[0]{\boldsymbol{\tau}}
%\newcommand{\Btheta}[0]{\boldsymbol{\theta}}
\newcommand{\Bupsilon}[0]{\boldsymbol{\upsilon}}
\newcommand{\Bvarepsilon}[0]{\boldsymbol{\varepsilon}}
\newcommand{\Bvarphi}[0]{\boldsymbol{\varphi}}
\newcommand{\Bvarpi}[0]{\boldsymbol{\varpi}}
\newcommand{\Bvarrho}[0]{\boldsymbol{\varrho}}
\newcommand{\Bvarsigma}[0]{\boldsymbol{\varsigma}}
\newcommand{\Bvartheta}[0]{\boldsymbol{\vartheta}}
\newcommand{\Bxi}[0]{\boldsymbol{\xi}}
\newcommand{\Bzeta}[0]{\boldsymbol{\zeta}}
%
%</bold and dot greek symbols>
%<infrequent>
%
%\newcommand{\AreaOp}[1]{\AName_{#1}}
%\newcommand{\Babs}[0]{\abs{\BB}}
%\newcommand{\Bcap}[0]{\hat{\BB}}
%\newcommand{\BrPrimeRej}[0]{\rcap(\rcap \wedge \Br')}
%\newcommand{\CA}[0]{\mathcal{A}}
%\newcommand{\Cos}[1]{\cos{\left({#1}\right)}}
%\newcommand{\Det}[1] {\abs{#1}}
%\newcommand{\Dsq}[2] {\frac {\partial^2 {#1}} {\partial {#2}^2}}
%\newcommand{\Exp}[1]{\exp{\left({#1}\right)}}
%\newcommand{\Norm}[1]{\left\lVert{#1}\right\rVert}
%\newcommand{\Sin}[1]{\sin{\left({#1}\right)}}
%\newcommand{\T}[0]{\text{T}}
%\newcommand{\VolumeOp}[1]{\VName_{#1}}
%\newcommand{\agrad}[0]{\Ba \cdot \nabla}
%\newcommand{\alphacap}[0]{\hat{\boldsymbol{\alpha}}}
%\newcommand{\Fcap}[0]{\hat{\BF}}
%\newcommand{\bithree}[0]{{\Bi}_3}
%\newcommand{\bxa}[0]{\Bx\Ba}
%\newcommand{\coordvec}[2]{
%\newcommand{\costheta}[0]{\acap \cdot \xcap}
%\newcommand{\ddt}[1]{\ddot{#1}}
%\newcommand{\ddu}[1] {\frac {d{#1}} {du}}
%\newcommand{\dsqxj}[2] {\frac {\partial^2 {#1}} {\partial {x_{#2}}^2}}
%\newcommand{\dtheta}[1]{\frac{d {#1}}{d \theta}}
%\newcommand{\dt}[1]{\dot{#1}}
%\newcommand{\dt}[1]{\frac{d {#1}}{dt}}
%\newcommand{\dxj}[2] {\frac {\partial {#1}} {\partial {x_{#2}}}}
%\newcommand{\halfPhi}[0]{\frac{\phi}{2}}
%\newcommand{\half}[0]{\inv{2}}
%\newcommand{\inv}[1]{\frac{1}{#1}}
%\newcommand{\laplacian}[0]{\nabla^2}
%\newcommand{\matrixoftx}[3]{
%\newcommand{\nrrp}[0]{\norm{\rcap \wedge \Br'}}
%\newcommand{\oiint}{\bigcirc \hspace{-1.4em} \int \hspace{-.8em} \int}
%\newcommand{\transpose}[1]{{#1}^{\text{T}}}
%\newcommand{\transpose}[1]{{{#1}^{\TextTranspose}}}
%\newcommand{\transpose}[1]{{{#1}^{\text{T}}}}
%\newcommand{\barA}[0]{\bar{A}}
%\newcommand{\qbar}[0]{\bar{q}}
%\newcommand{\qdotbar}[0]{\dot{\bar{q}}}
%
%</infrequent>




\newcommand{\symmetric}[2]{{\left\{{#1},{#2}\right\}}}
\newcommand{\antisymmetric}[2]{\left[{#1},{#2}\right]}
\DeclareMathOperator{\sgn}{sgn}
\DeclareMathOperator{\something}{something}

\newcommand{\uDETuvij}[4]{
\begin{vmatrix}
 {#1}^{#3} & {#1}^{#4} \\
 {#2}^{#3} & {#2}^{#4}
\end{vmatrix}
}

\newcommand{\PDSq}[2]{\frac{\partial^2 {#2}}{\partial {#1}^2}}
\newcommand{\transpose}[1]{{#1}^{\mathrm{T}}}
\newcommand{\stardot}[0]{{*}}

% bivector.tex:
\newcommand{\laplacian}[0]{\nabla^2}
\newcommand{\Dsq}[2] {\frac {\partial^2 {#1}} {\partial {#2}^2}}
\newcommand{\dxj}[2] {\frac {\partial {#1}} {\partial {x_{#2}}}}
\newcommand{\dsqxj}[2] {\frac {\partial^2 {#1}} {\partial {x_{#2}}^2}}
\DeclareMathOperator{\ExpName}{e}
%\DeclareMathOperator{\Exp}{e}
%\newcommand{\Exp}[1]{\exp{\left({#1}\right)}}
%\DeclareMathOperator{\Rej}{Rej}
\DeclareMathOperator{\Rot}{R}
%\newcommand{\gpgrade}[2] {{\left\langle{{#1}}\right\rangle}_{#2}}
%\newcommand{\gpgradezero}[1] {\gpgrade{#1}{0}}
%\newcommand{\gpgradetwo}[1] {\gpgrade{#1}{2}}
%\newcommand{\gpgradefour}[1] {\gpgrade{#1}{4}}

% ga_wiki_torque.tex:
\newcommand{\Fcap}[0]{\hat{\BF}}
\newcommand{\bithree}[0]{{\Bi}_3}
\newcommand{\nrrp}[0]{\norm{\rcap \wedge \Br'}}
\newcommand{\dtheta}[1]{\frac{d {#1}}{d \theta}}

% ga_wiki_unit_derivative.tex
\newcommand{\dt}[1]{\frac{d {#1}}{dt}}
\newcommand{\BrPrimeRej}[0]{\rcap(\rcap \wedge \Br')}

% radial_vector_derivatives.tex:
%\newcommand{\BrPrimeRej}[0]{\rcap(\rcap \wedge \Br')}

% angular_velocity.tex

%\newcommand{\dt}[1]{\frac{d {#1}}{dt}}
%\newcommand{\Norm}[1]{\left\lVert{#1}\right\rVert}
%\newcommand{\dtheta}[1]{\frac{d {#1}}{d \theta}}

% reciprocal_frame.tex
\DeclareMathOperator{\AbsName}{abs}

%\DeclareMathOperator{\RejName}{Rej}
%\newcommand{\Rej}[2]{\RejName_{#1}\left( {#2} \right)}

\DeclareMathOperator{\AName}{A}
\newcommand{\AreaOp}[1]{\AName_{#1}}

\DeclareMathOperator{\VName}{V}
\newcommand{\VolumeOp}[1]{\VName_{#1}}

%\newcommand{\gpgrade}[2] {{\left\langle{{#1}}\right\rangle}_{#2}}
%\newcommand{\gpgradeone}[1] {{\left\langle{{#1}}\right\rangle}_{1}}


% projection_with_matrix_comparison.tex
%\DeclareMathOperator{\Transpose}{T}
\DeclareMathOperator{\rank}{rank}
%\newcommand{\transpose}[1]{{{#1}^{\TextTranspose}}}
%\newcommand{\transpose}[1]{{{#1}^{\text{T}}}}
\newcommand{\T}[0]{{\text{T}}}
%\newcommand{\BOmega}[0]{\boldsymbol{\Omega}}

%\newcommand{\Det}[1] {\abs{#1}}

% oblique_proj.tex
%\newcommand{\T}[0]{\text{T}}
%\newcommand{\Bbeta}[0]{\boldsymbol{\beta}}

% spherical_polar.tex
\newcommand{\phicap}[0]{\hat{\boldsymbol{\phi}}}
\newcommand{\Lor}[2]{{{\Lambda^{#1}}_{#2}}}
\newcommand{\ILor}[2]{{{ \{{\Lambda^{-1}\} }^{#1}}_{#2}}}

% slerp.tex
\DeclareMathOperator{\atan2}{atan2}

% kvector_exponential.tex
%\DeclareMathOperator{\Exp}{e}
%\DeclareMathOperator{\Rej}{Rej}
\newcommand{\Bcap}[0]{\hat{\BB}}
\newcommand{\Babs}[0]{\abs{\BB}}
%\newcommand{\gpgrade}[2] {{\left\langle{{#1}}\right\rangle}_{#2}}
%\newcommand{\gpgradezero}[1] {\gpgrade{#1}{0}}
%\newcommand{\gpgradetwo}[1] {\gpgrade{#1}{2}}
%\newcommand{\gpgradefour}[1] {\gpgrade{#1}{4}}

\newcommand{\ddu}[1] {\frac {d{#1}} {du}}

% vector_integral_relations.tex
%\newcommand{\Oiint}{\bigcirc \hspace{-1.4em} \int \hspace{-.8em} \int}

% legendre.tex
\newcommand{\agrad}[0]{\Ba \cdot \nabla}
\newcommand{\bxa}[0]{\Bx\Ba}
\newcommand{\costheta}[0]{\acap \cdot \xcap}
%\newcommand{\inv}[1]{\frac{1}{#1}}
\newcommand{\half}[0]{\inv{2}}

% ke_rotation.tex
\newcommand{\DotT}[1]{\dot{#1}}
\newcommand{\DDotT}[1]{\ddot{#1}}
%\newcommand{\transpose}[1]{{#1}^{\text{T}}}
%\newcommand{\Balpha}[0]{\boldsymbol{\alpha}}

%\newcommand{\gpgrade}[2] {{\left\langle{{#1}}\right\rangle}_{#2}}
%\newcommand{\gpgradeone}[1] {{\left\langle{{#1}}\right\rangle}_{1}}
\newcommand{\gpscalargrade}[1] {{\left\langle{{#1}}\right\rangle}}
%\newcommand{\BOmega}[0]{\boldsymbol{\Omega}}

% gaussian_surface.tex
%\newcommand{\phicap}[0]{\hat{\Bphi}}

% newtonian_lagrangian_and_gradient.tex
% PD macro that is backwards from current in macros2:
\newcommand{\PDb}[2]{ \frac{\partial{#1}}{\partial {#2}} }

% inertial_tensor.tex
\newcommand{\matrixoftx}[3]{
{
\begin{bmatrix}
{#1}
\end{bmatrix}
}_{#2}^{#3}
}

\newcommand{\coordvec}[2]{
{
\begin{bmatrix}
{#1}
\end{bmatrix}
}_{#2}
}

% bohr.tex
\newcommand{\K}[0]{\inv{4 \pi \epsilon_0}}

% euler_lagrange.tex
\newcommand{\qbar}[0]{\bar{q}}
\newcommand{\qdotbar}[0]{\dot{\bar{q}}}
\newcommand{\DD}[2]{\frac{d{#2}}{d{#1}}}
\newcommand{\Xdot}[0]{\dot{X}}

% rayleigh_jeans.tex
\newcommand{\EE}[0]{\boldsymbol{\mathcal{E}}}
\newcommand{\HH}[0]{\boldsymbol{\mathcal{H}}}

% 4d_fourier.tex

%\newcommand{\PDSq}[2]{\frac{\partial^2 {#2}}{\partial {#1}^2}}
\DeclareMathOperator{\sinc}{sinc}
\DeclareMathOperator{\PV}{PV}
\newcommand{\FF}[0]{\mathcal{F}}
\newcommand{\IIinf}[0]{ \int_{-\infty}^\infty }

% poisson.tex
%\newcommand{\PDSq}[2]{\frac{\partial^2 {#2}}{\partial {#1}^2}}
%\DeclareMathOperator{\sinc}{sinc}
%\DeclareMathOperator{\PV}{PV}
%\newcommand{\FF}[0]{\mathcal{F}}
%\newcommand{\IIinf}[0]{ \int_{-\infty}^\infty }

% fourier_maxwell.tex
%\newcommand{\PDSq}[2]{\frac{\partial^2 {#2}}{\partial {#1}^2}}
%\DeclareMathOperator{\sinc}{sinc}
%\DeclareMathOperator{\sgn}{sgn}
%\DeclareMathOperator{\PV}{PV}
%\newcommand{\FF}[0]{\mathcal{F}}
%\newcommand{\IIinf}[0]{ \int_{-\infty}^\infty }

% firstorder_fourier_maxwell.tex
%\newcommand{\PDSq}[2]{\frac{\partial^2 {#2}}{\partial {#1}^2}}
%\DeclareMathOperator{\sinc}{sinc}
%\DeclareMathOperator{\PV}{PV}
%\newcommand{\FF}[0]{\mathcal{F}}
%\newcommand{\IIinf}[0]{ \int_{-\infty}^\infty }

% wave_fourier.tex
%\newcommand{\PDSq}[2]{\frac{\partial^2 {#2}}{\partial {#1}^2}}
%\DeclareMathOperator{\sinc}{sinc}
%\DeclareMathOperator{\PV}{PV}
%\newcommand{\FF}[0]{\mathcal{F}}
%\newcommand{\IIinf}[0]{ \int_{-\infty}^\infty }

% heat_fourier.tex
%\newcommand{\PDSq}[2]{\frac{\partial^2 {#2}}{\partial {#1}^2}}
%\DeclareMathOperator{\sinc}{sinc}
%\newcommand{\FF}[0]{\mathcal{F}}
%\newcommand{\IIinf}[0]{ \int_{-\infty}^\infty }

% proj_generalized_dot_prod.tex
%\newcommand{\T}[0]{\text{T}}

% fourier_tx.tex
%\newcommand{\FF}[0]{\mathcal{F}}
\newcommand{\FM}[0]{\inv{\sqrt{2\pi\hbar}}}
\newcommand{\Iinf}[1]{ \int_{-\infty}^\infty {#1}}
%\DeclareMathOperator{\PV}{PV}

% fourier_notation.tex
%\newcommand{\FF}[0]{\mathcal{F}}
%\newcommand{\IIinf}[0]{ \int_{-\infty}^\infty }
%\DeclareMathOperator{\PV}{PV}
%\DeclareMathOperator{\sinc}{sinc}

% planewave.tex
%\newcommand{\EE}[0]{\boldsymbol{\mathcal{E}}}
%\newcommand{\HH}[0]{\boldsymbol{\mathcal{H}}}
%\newcommand{\IIinf}[0]{ \int_{-\infty}^\infty }

% dirac_lagrangian.tex
\newcommand{\Dslash}[0]{ \not\!D }

% pauli_matrix.tex
\newcommand{\Clifford}[2]{\mathcal{C}_{\{{#1},{#2}\}}}
\DeclareMathOperator{\tr}{Tr}
%\DeclareMathOperator{\Scalar}{Scalar}
\DeclareMathOperator{\Real}{Re}
\DeclareMathOperator{\Imag}{Im}
\newcommand{\trace}[1]{\tr{#1}}
\newcommand{\scalarProduct}[2]{{#1} \bullet {#2}}
\newcommand{\traceB}[1]{\tr\left({#1}\right)}
%\newcommand{\symmetric}[2]{{\left\{{#1},{#2}\right\}}}
%\newcommand{\antisymmetric}[2]{\left[{#1},{#2}\right]}
%\newcommand{\Bcap}[0]{\hat{\BB}}

\newcommand{\xhat}[0]{\hat{x}}

\newcommand{\PauliI}[0]{
\begin{bmatrix}
1 & 0 \\
0 & 1 \\
\end{bmatrix}
}

\newcommand{\PauliX}[0]{
\begin{bmatrix}
0 & 1 \\
1 & 0 \\
\end{bmatrix}
}

\newcommand{\PauliY}[0]{
\begin{bmatrix}
0 & -i \\
i & 0 \\
\end{bmatrix}
}

\newcommand{\PauliYNoI}[0]{
\begin{bmatrix}
0 & -1 \\
1 & 0 \\
\end{bmatrix}
}

\newcommand{\PauliZ}[0]{
\begin{bmatrix}
1 & 0 \\
0 & -1 \\
\end{bmatrix}
}

% gamma.tex
%\newcommand{\scalarProduct}[2]{{#1} \bullet {#2}}
%\newcommand{\symmetric}[2]{{\left\{{#1},{#2}\right\}}}
%\newcommand{\antisymmetric}[2]{\left[{#1},{#2}\right]}

%\newcommand{\PauliX}[0]{
%\begin{bmatrix}
%0 & 1 \\
%1 & 0 \\
%\end{bmatrix}
%}

%\newcommand{\PauliY}[0]{
%\begin{bmatrix}
%0 & -i \\
%i & 0 \\
%\end{bmatrix}
%}

%\newcommand{\PauliYNoI}[0]{
%\begin{bmatrix}
%0 & -1 \\
%1 & 0 \\
%\end{bmatrix}
%}

%\newcommand{\PauliZ}[0]{
%\begin{bmatrix}
%1 & 0 \\
%0 & -1 \\
%\end{bmatrix}
%}

% em_bivector_metric_dependencies.tex

%\newcommand{\LL}[0]{\mathcal{L}}
%\newcommand{\gpgrade}[2] {{\left\langle{{#1}}\right\rangle}_{#2}}
%\newcommand{\gpgradezero}[1] {\gpgrade{#1}{0}}
%\newcommand{\gpgradetwo}[1] {\gpgrade{#1}{2}}
%\newcommand{\gpgradeone}[1] {\gpgrade{#1}{1}}
%\newcommand{\gpgradefour}[1] {\gpgrade{#1}{4}}
%\newcommand{\grad}[0]{\nabla}
%\newcommand{\spacegrad}[0]{\boldsymbol{\nabla}}
% == \partial_{#1} {#2}
%\newcommand{\PD}[2]{\frac{\partial {#2}}{\partial {#1}}}
%\newcommand{\PDD}[3]{\frac{\partial^2 {#3}}{\partial {#1}\partial {#2}}}
\newcommand{\PDsQ}[2]{\frac{\partial^2 {#2}}{\partial^2 {#1}}}

% gem.tex
\newcommand{\barh}[0]{\bar{h}}

% mass_vary_lagrangian.tex
%\newcommand{\LL}[0]{\mathcal{L}}
%\newcommand{\grad}[0]{\nabla}
%\newcommand{\PD}[2]{\frac{\partial {#2}}{\partial {#1}}}
%\newcommand{\xdot}[0]{\dot{x}}
%\newcommand{\vdot}[0]{\dot{v}}
%\newcommand{\mdot}[0]{\dot{m}}
%\newcommand{\xddot}[0]{\ddot{x}}
%\newcommand{\spacegrad}[0]{\boldsymbol{\nabla}}

% fourvec_dotinvariance.tex
%\newcommand{\Balpha}[0]{\boldsymbol{\alpha}}
\newcommand{\alphacap}[0]{\hat{\boldsymbol{\alpha}}}
%\newcommand{\Bcap}[0]{\hat{\BB}}
%\newcommand{\gpgrade}[2] {{\left\langle{{#1}}\right\rangle}_{#2}}
%\newcommand{\gpgradezero}[1] {\gpgrade{#1}{0}}

% lorentz.tex
%\newcommand{\laplacian}[0]{\nabla^2}

% field_lagrangian.tex
%\newcommand{\LL}[0]{\mathcal{L}}
%\newcommand{\PD}[2]{\frac{\partial {#2}}{\partial {#1}}}
\newcommand{\barA}[0]{\bar{A}}
%\newcommand{\grad}[0]{\nabla}
%\newcommand{\conj}[0]{{*}}

%\newcommand{\spacegrad}[0]{\boldsymbol{\nabla}}

%\newcommand{\gpgrade}[2] {{\left\langle{{#1}}\right\rangle}_{#2}}
%\newcommand{\gpgradezero}[1] {\gpgrade{#1}{0}}
%\newcommand{\gpgradetwo}[1] {\gpgrade{#1}{2}}
%\newcommand{\gpgradefour}[1] {\gpgrade{#1}{4}}

% lagrangian_field_density.tex
%\newcommand{\LL}[0]{\mathcal{L}}
%\newcommand{\gpgrade}[2] {{\left\langle{{#1}}\right\rangle}_{#2}}
%\newcommand{\gpgradezero}[1] {\gpgrade{#1}{0}}
%\newcommand{\gpgradetwo}[1] {\gpgrade{#1}{2}}
%\newcommand{\gpgradefour}[1] {\gpgrade{#1}{4}}
%\newcommand{\grad}[0]{\nabla}
%\newcommand{\spacegrad}[0]{\boldsymbol{\nabla}}
%\newcommand{\PD}[2]{\frac{\partial {#2}}{\partial {#1}}}
\newcommand{\PDd}[2]{\frac{\partial^2 {#2}}{{\partial{#1}}^2}}
%\newcommand{\PDD}[3]{\frac{\partial^2 {#3}}{\partial {#1}\partial {#2}}}

%\newcommand{\barA}[0]{\bar{A}}

% lorentz_force.tex
%\newcommand{\grad}[0]{\nabla}
%\newcommand{\spacegrad}[0]{\boldsymbol{\nabla}}
%\newcommand{\LL}[0]{\mathcal{L}}
%\newcommand{\xdot}[0]{\dot{x}}
%\newcommand{\xddot}[0]{\ddot{x}}
%\newcommand{\pdot}[0]{\dot{p}}
%\newcommand{\pddot}[0]{\ddot{p}}
%\newcommand{\fdot}[0]{\dot{f}}
%\newcommand{\fddot}[0]{\ddot{f}}

%\newcommand{\gpgrade}[2] {{\left\langle{{#1}}\right\rangle}_{#2}}
%\newcommand{\gpgradeone}[1] {\gpgrade{#1}{1}}
%\newcommand{\gpgradezero}[1] {\gpgrade{#1}{}}
%\newcommand{\grad}[0] {\nabla}
%\newcommand{\spacegrad}[0]{\boldsymbol{\nabla}}

%\newcommand{\pdot}[0]{\dot{p}}
%\newcommand{\pddot}[0]{\ddot{p}}

%\newcommand{\xdot}[0]{\dot{x}}
%\newcommand{\xddot}[0]{\ddot{x}}
%\newcommand{\PD}[2]{\frac{\partial {#2}}{\partial {#1}}}

% stokes_maxwell_application.tex
%\newcommand{\grad}[0]{\nabla}
%\newcommand{\PD}[2]{\frac{\partial {#2}}{\partial {#1}}}
%\newcommand{\spacegrad}[0]{\boldsymbol{\nabla}}
%\newcommand{\gpgrade}[2] {{\left\langle{{#1}}\right\rangle}_{#2}}
%\newcommand{\gpgradezero}[1] {\gpgrade{#1}{0}}
%\newcommand{\gpgradeone}[1] {\gpgrade{#1}{1}}
%\newcommand{\gpgradetwo}[1] {\gpgrade{#1}{2}}
%\newcommand{\gpgradethree}[1] {\gpgrade{#1}{3}}

% lorentz_rotation.tex
%\DeclareMathOperator{\Transpose}{T}
%\newcommand{\T}[0]{\text{T}}

% electron_rotor.tex
\newcommand{\reverse}[1]{\tilde{{#1}}}
%\newcommand{\ILambda}[0]{{(\Lambda^{-1})}}
\newcommand{\ILambda}[0]{\Pi}

% em_potential.tex
%\newcommand{\spacegrad}[0]{\boldsymbol{\nabla}}
%\newcommand{\grad}[0]{\nabla}
\newcommand{\CA}[0]{\mathcal{A}}
 
% maxwell_to_tensor.tex
%\newcommand{\LL}[0]{\mathcal{L}}
%\newcommand{\gpgrade}[2] {{\left\langle{{#1}}\right\rangle}_{#2}}
%\newcommand{\gpgradezero}[1] {\gpgrade{#1}{0}}
%\newcommand{\gpgradetwo}[1] {\gpgrade{#1}{2}}
%\newcommand{\gpgradeone}[1] {\gpgrade{#1}{1}}
%\newcommand{\gpgradefour}[1] {\gpgrade{#1}{4}}
%\newcommand{\grad}[0]{\nabla}
%\newcommand{\spacegrad}[0]{\boldsymbol{\nabla}}
% == \partial_{#1} {#2}
%\newcommand{\PD}[2]{\frac{\partial {#2}}{\partial {#1}}}
%\newcommand{\PDD}[3]{\frac{\partial^2 {#3}}{\partial {#1}\partial {#2}}}
%\newcommand{\PDsQ}[2]{\frac{\partial^2 {#2}}{\partial^2 {#1}}}

%\newcommand{\EE}[0]{\boldsymbol{\mathcal{E}}}
%\newcommand{\HH}[0]{\boldsymbol{\mathcal{H}}}
\newcommand{\Vcap}[0]{\hat{\BV}}



%\newcommand{\Dslash}[0]{{\not{}}D}
\newcommand{\Dslash}[0]{D\!\!\!/}

\DeclareMathOperator{\Atan}{atan}

%%
% Copyright � 2012 Peeter Joot.  All Rights Reserved.
% Licenced as described in the file LICENSE under the root directory of this GIT repository.
%

% 
% 
\DeclareMathOperator{\Div}{div}
\DeclareMathOperator{\Mod}{mod}
\DeclareMathOperator{\PV}{PV}
\DeclareMathOperator{\Prob}{Prob}
\DeclareMathOperator{\rank}{rank}
\DeclareMathOperator{\sgn}{sgn}
\DeclareMathOperator{\sinc}{sinc}
%\DeclareMathOperator{\Atan2}{atan2}
\DeclareMathOperator{\atan}{atan}


\newcommand{\expectation}[1]{\langle{#1}\rangle}
%\newcommand{\gpgradefour}[1] {\gpgrade{#1}{4}}
%\newcommand{\gpgradeone}[1] {\gpgrade{#1}{1}}
%\newcommand{\gpgradethree}[1] {\gpgrade{#1}{3}}
%\newcommand{\gpgradetwo}[1] {\gpgrade{#1}{2}}
%\newcommand{\gpgradezero}[1] {\gpgrade{#1}{}}
%\newcommand{\gpgrade}[2] {{\left\langle{{#1}}\right\rangle}_{#2}}
%\newcommand{\grad}[0]{\boldsymbol{\nabla}}
%\newcommand{\grad}[0]{\nabla}


\newcommand{\ketbra}[2]{\ket{#1}\bra{#2}}
\newcommand{\ket}[1]{\lvert {#1} \rangle}
%\newcommand{\norm}[1]{\lVert#1\rVert}
\newcommand{\questionEquals}[0]{\stackrel{?}{=}}
\newcommand{\rightshift}[0]{\gg}
%\newcommand{\spacegrad}[0]{\boldsymbol{\nabla}}
\newcommand{\symmetric}[2]{{\left\{{#1},{#2}\right\}}}
\newcommand{\antisymmetric}[2]{\left[{#1},{#2}\right]}

%\newcommand{\Abs}[1]{\left\lvert{#1}\right\rvert}

%\newcommand{\BB}[0]{\mathbf{B}}
%\newcommand{\BE}[0]{\mathbf{E}}
%\newcommand{\BF}[0]{\mathbf{F}}
%\newcommand{\BS}[0]{\mathbf{S}}
%\newcommand{\BV}[0]{\mathbf{V}}
%\newcommand{\Bj}[0]{\mathbf{j}}

\newcommand{\BraOpKet}[3]{\bra{#1} \hat{#2} \ket{#3} }
%\newcommand{\Brho}[0]{\boldsymbol{\rho}}
\newcommand{\CC}[0]{c^2}
\newcommand{\Cos}[1]{\cos{\left({#1}\right)}}

% not working anymore.  think it's a conflicting macro for \not.
% compared to original usage in klien_gordon.ltx
%
%\newcommand{\Dslash}[0]{{\not}D}
%\newcommand{\Dslash}[0]{{\not{}}D}
% switched to cancel in macros.tex
%\newcommand{\Dslash}[0]{D\!\!\!/}

\newcommand{\Expectation}[1]{\left\langle {#1} \right\rangle}
\newcommand{\Exp}[1]{\exp{\left({#1}\right)}}
\newcommand{\FF}[0]{\mathcal{F}}
\newcommand{\FM}[0]{\inv{\sqrt{2\pi\hbar}}}
\newcommand{\IIinf}[0]{ \int_{-\infty}^\infty }
\newcommand{\Innerprod}[2]{\left\langle{#1}, {#2}\right\rangle}
%\newcommand{\LL}[0]{\mathcal{L}}

%\newcommand{\PD}[2] {\frac {\partial #2} {\partial #1}}

% backwards from ../peeterj_macros2:
\newcommand{\PDb}[2]{ \frac{\partial{#1}}{\partial {#2}} }

%\newcommand{\PDD}[3]{\frac{\partial^2 {#3}}{\partial {#1}\partial {#2}}}
\newcommand{\PDN}[3]{\frac{\partial^{#3} {#2}}{\partial {#1}^{#3}}}

\newcommand{\PDSq}[2]{\frac{\partial^2 {#2}}{\partial {#1}^2}}
\newcommand{\PDsQ}[2]{\frac{\partial^2 {#2}}{\partial^2 {#1}}}

\newcommand{\Sch}[0]{{Schr\"{o}dinger} }
\newcommand{\Sin}[1]{\sin{\left({#1}\right)}}
\newcommand{\Sw}[0]{\mathcal{S}}
%\newcommand{\T}[0]{\text{T}}
\newcommand{\T}[0]{{\text{T}}}

\newcommand{\braket}[2]{\langle{#1} \vert {#2}\rangle}
\newcommand{\bra}[1]{\langle {#1} \rvert}
\newcommand{\curl}[0]{\grad \times}
\newcommand{\delambert}[0]{\sum_{\alpha = 1}^4{\PDSq{x_\alpha}{}}}
\newcommand{\delsquared}[0]{\nabla^2}
\newcommand{\diverg}[0]{\grad \cdot}

\newcommand{\halfPhi}[0]{\frac{\phi}{2}}
\newcommand{\hatH}[0]{\hat{H}}
\newcommand{\hatS}[0]{\hat{S}}
\newcommand{\hatk}[0]{\hat{k}}
\newcommand{\hatp}[0]{\hat{p}}
\newcommand{\hatx}[0]{\hat{x}}


\newcommand{\Rdot}[0]{\dot{R}}
%\newcommand{\addot}[0]{\ddot{a}}
%\newcommand{\adot}[0]{\dot{a}}
%\newcommand{\fddot}[0]{\ddot{f}}
%\newcommand{\fdot}[0]{\dot{f}}
%\newcommand{\bddot}[0]{\ddot{b}}
%\newcommand{\bdot}[0]{\dot{b}}
\newcommand{\ddotOmega}[0]{\ddot{\Omega}}
\newcommand{\ddotalpha}[0]{\ddot{\alpha}}
\newcommand{\ddotomega}[0]{\ddot{\omega}}
\newcommand{\ddotphi}[0]{\ddot{\phi}}
\newcommand{\ddotpsi}[0]{\ddot{\psi}}
\newcommand{\ddottheta}[0]{\ddot{\theta}}
\newcommand{\dotOmega}[0]{\dot{\Omega}}
\newcommand{\dotalpha}[0]{\dot{\alpha}}
\newcommand{\dotomega}[0]{\dot{\omega}}
\newcommand{\dotphi}[0]{\dot{\phi}}
\newcommand{\dotpsi}[0]{\dot{\psi}}
\newcommand{\dottheta}[0]{\dot{\theta}}
%\newcommand{\pddot}[0]{\ddot{p}}
%\newcommand{\pdot}[0]{\dot{p}}
%\newcommand{\qddot}[0]{\ddot{q}}
%\newcommand{\qdot}[0]{\dot{q}}
%\newcommand{\rddot}[0]{\ddot{r}}
%\newcommand{\rdot}[0]{\dot{r}}
%\newcommand{\tddot}[0]{\ddot{t}}
%\newcommand{\tdot}[0]{\dot{t}}
%\newcommand{\uddot}[0]{\ddot{u}}
%\newcommand{\udot}[0]{\dot{u}}
%\newcommand{\xddot}[0]{\ddot{x}}
%\newcommand{\xdot}[0]{\dot{x}}
%\newcommand{\yddot}[0]{\ddot{y}}
%\newcommand{\ydot}[0]{\dot{y}}
%\newcommand{\zddot}[0]{\ddot{z}}
%\newcommand{\zdot}[0]{\dot{z}}








%-------------------------------------------------------------------
% ORIGINS:
%
% bohm11.tex

%\DeclareMathOperator{\sgn}{sgn}
%\newcommand{\PDSq}[2]{\frac{\partial^2 {#2}}{\partial {#1}^2}}
%\newcommand{\PDN}[3]{\frac{\partial^{#3} {#2}}{\partial {#1}^{#3}}}
%\DeclareMathOperator{\sinc}{sinc}
%\DeclareMathOperator{\PV}{PV}
%\newcommand{\FF}[0]{\mathcal{F}}
%\newcommand{\Sw}[0]{\mathcal{S}}
%\newcommand{\IIinf}[0]{ \int_{-\infty}^\infty }
%\newcommand{\FM}[0]{\inv{\sqrt{2\pi\hbar}}}
%\newcommand{\expectation}[1]{\langle{#1}\rangle}
%
%

% bohm_ch10.tex

%\DeclareMathOperator{\sgn}{sgn}
%\newcommand{\expectation}[1]{\langle{#1}\rangle}
%\newcommand{\IIinf}[0]{ \int_{-\infty}^\infty }
%\DeclareMathOperator{\PV}{PV}
%
%

% bohm_ch9.tex

%\newcommand{\PDSq}[2]{\frac{\partial^2 {#2}}{\partial {#1}^2}}
%\newcommand{\PDN}[3]{\frac{\partial^{#3} {#2}}{\partial {#1}^{#3}}}
%\DeclareMathOperator{\sinc}{sinc}
%\DeclareMathOperator{\PV}{PV}
%\newcommand{\FF}[0]{\mathcal{F}}
%\newcommand{\Sw}[0]{\mathcal{S}}
%\newcommand{\IIinf}[0]{ \int_{-\infty}^\infty }
%\newcommand{\FM}[0]{\inv{\sqrt{2\pi\hbar}}}
%\newcommand{\expectation}[1]{\langle{#1}\rangle}
%
%

% commutator_herm.tex

%\newcommand{\symmetric}[2]{{\left\{{#1},{#2}\right\}}}
%\newcommand{\antisymmetric}[2]{\left[{#1},{#2}\right]}
%
%%\newcommand{\ket}[1]{\lvert {#1} \rangle}
%%\newcommand{\bra}[1]{\langle {#1} \rvert}
%%\newcommand{\braket}[2]{\langle{#1} \vert {#2}\rangle}
%%\newcommand{\ketbra}[2]{\ket{#1}\bra{#2}}
%%\newcommand{\BraOpKet}[3]{\bra{#1} \hat{#2} \ket{#3} }
%%\newcommand{\Innerprod}[2]{\left\langle{#1}, {#2}\right\rangle}
%\newcommand{\Expectation}[1]{\left\langle {#1} \right\rangle}
%
%

% delta_ortho_series.tex

%\newcommand{\IIinf}[0]{ \int_{-\infty}^\infty }
%\newcommand{\ket}[1]{\lvert {#1} \rangle}
%\newcommand{\bra}[1]{\langle {#1} \rvert}
%\newcommand{\braket}[2]{\langle{#1} \vert {#2}\rangle}
%\newcommand{\ketbra}[2]{\ket{#1}\bra{#2}}
%\newcommand{\BraOpKet}[3]{\bra{#1} \hat{#2} \ket{#3} }
%\newcommand{\Innerprod}[2]{\left\langle{#1}, {#2}\right\rangle}
%
%

% distributions.tex

%\newcommand{\PDSq}[2]{\frac{\partial^2 {#2}}{\partial {#1}^2}}
%\DeclareMathOperator{\sinc}{sinc}
%\DeclareMathOperator{\PV}{PV}
%\newcommand{\FF}[0]{\mathcal{F}}
%\newcommand{\Sw}[0]{\mathcal{S}}
%\newcommand{\IIinf}[0]{ \int_{-\infty}^\infty }
%
%

% ehrenfest.tex

%\newcommand{\PDSq}[2]{\frac{\partial^2 {#2}}{\partial {#1}^2}}
%
%

% fletcher.tex

%\DeclareMathOperator{\Div}{div}
%\DeclareMathOperator{\Mod}{mod}
%\newcommand{\rightshift}[0]{\gg}
%\newcommand{\questionEquals}[0]{\stackrel{?}{=}}
%
%

% fvec.tex

%\newcommand{\grad}[0]{\nabla}
%\newcommand{\PD}[2]{ \frac{\partial{#1}}{\partial {#2}} }
%
%

% gacs_q8_8.tex

%\newcommand{\halfPhi}[0]{\frac{\phi}{2}}
%\newcommand{\Sin}[1]{\sin{\left({#1}\right)}}
%\newcommand{\Cos}[1]{\cos{\left({#1}\right)}}
%\newcommand{\Exp}[1]{\exp{\left({#1}\right)}}
%
%

% goldstein_ch1_2.tex

%\newcommand{\spacegrad}[0]{\boldsymbol{\nabla}}
%\newcommand{\Brho}[0]{\boldsymbol{\rho}}
%\newcommand{\LL}[0]{\mathcal{L}}
%\newcommand{\Abs}[1]{\left\lvert{#1}\right\rvert}
%\newcommand{\qdot}[0]{\dot{q}}
%\newcommand{\qddot}[0]{\ddot{q}}
%\newcommand{\xdot}[0]{\dot{x}}
%\newcommand{\xddot}[0]{\ddot{x}}
%\newcommand{\ydot}[0]{\dot{y}}
%\newcommand{\yddot}[0]{\ddot{y}}
%\newcommand{\dotalpha}[0]{\dot{\alpha}}
%\newcommand{\ddotalpha}[0]{\ddot{\alpha}}
%\newcommand{\dottheta}[0]{\dot{\theta}}
%\newcommand{\ddottheta}[0]{\ddot{\theta}}
%\newcommand{\dotphi}[0]{\dot{\phi}}
%\newcommand{\ddotphi}[0]{\ddot{\phi}}
%% == \partial_{#1} {#2}
%\newcommand{\PD}[2]{\frac{\partial {#2}}{\partial {#1}}}
%\newcommand{\PDD}[3]{\frac{\partial^2 {#3}}{\partial {#1}\partial {#2}}}
%
%% <grade selection>
%%
%\newcommand{\gpgrade}[2] {{\left\langle{{#1}}\right\rangle}_{#2}}
%
%\newcommand{\gpgradezero}[1] {\gpgrade{#1}{}}
%%\newcommand{\gpscalargrade}[1] {{\left\langle{{#1}}\right\rangle}}
%%\newcommand{\gpgradezero}[1] {\gpgrade{#1}{0}}
%
%%\newcommand{\gpgradeone}[1] {{\left\langle{{#1}}\right\rangle}_{1}}
%\newcommand{\gpgradeone}[1] {\gpgrade{#1}{1}}
%
%\newcommand{\gpgradetwo}[1] {\gpgrade{#1}{2}}
%\newcommand{\gpgradethree}[1] {\gpgrade{#1}{3}}
%\newcommand{\gpgradefour}[1] {\gpgrade{#1}{4}}
%%
%% </grade selection>
%
%
%

% harmonic_osc.tex

%\newcommand{\IIinf}[0]{ \int_{-\infty}^\infty }
%
%

% klein_gordon.tex

%\newcommand{\PDSq}[2]{\frac{\partial^2 {#2}}{\partial {#1}^2}}
%%\newcommand{\Dslash}[0]{D\!\!\!/}
%\newcommand{\Dslash}[0]{{\not}D}
%
%

% matrix_to_operator.tex

%\newcommand{\T}[0]{{\text{T}}}
%
%

% maxwell.tex

%\newcommand{\norm}[1]{\lVert#1\rVert}
%\newcommand{\grad}[0]{\boldsymbol{\nabla}}
%\newcommand{\curl}[0]{\grad \times}
%\newcommand{\diverg}[0]{\grad \cdot}
%\newcommand{\delsquared}[0]{\nabla^2}
%\newcommand{\delambert}[0]{\sum_{\alpha = 1}^4{\PDSq{x_\alpha}{}}}
%
%% partial derivative of #1 wrt. #2:
%\newcommand{\PD}[2] {\frac {\partial #2} {\partial #1}}
%% second partial derivative of #1 wrt. #2:
%\newcommand{\PDSq}[2] {\frac {\partial^2 #2} {\partial {#1}^2}}
%
%%
%% shorthand for bold symbols:
%%
%\newcommand{\Bj}[0]{\mathbf{j}}
%\newcommand{\BB}[0]{\mathbf{B}}
%\newcommand{\BE}[0]{\mathbf{E}}
%\newcommand{\BF}[0]{\mathbf{F}}
%\newcommand{\BS}[0]{\mathbf{S}}
%\newcommand{\BV}[0]{\mathbf{V}}
%
%

% mp_inverse_svd_rough_notes.tex

%\newcommand{\T}[0]{\text{T}}
%\DeclareMathOperator{\rank}{rank}
%
%

% outermorphism_det.tex

%\newcommand{\gpgrade}[2] {{\left\langle{{#1}}\right\rangle}_{#2}}
%\newcommand{\gpgradeone}[1] {\gpgrade{#1}{1}}
%\newcommand{\gpgradetwo}[1] {\gpgrade{#1}{2}}
%
%

% pauli_qm_relativity_intro.tex

%\newcommand{\Sch}[0]{{Schr\"{o}dinger} }
%
%

% pe.tex

%
%\newcommand{\grad}[0]{\nabla}

% qm_susskind.tex

%\newcommand{\ket}[1]{\lvert {#1} \rangle}
%\newcommand{\bra}[1]{\langle {#1} \rvert}
%\newcommand{\braket}[2]{\langle{#1} \vert {#2}\rangle}
%\newcommand{\ketbra}[2]{\ket{#1}\bra{#2}}
%\newcommand{\BraOpKet}[3]{\bra{#1} \hat{#2} \ket{#3} }
%\newcommand{\hatH}[0]{\hat{H}}
%\newcommand{\hatS}[0]{\hat{S}}
%\newcommand{\hatk}[0]{\hat{k}}
%\newcommand{\hatx}[0]{\hat{x}}
%\newcommand{\hatp}[0]{\hat{p}}
%\DeclareMathOperator{\Prob}{Prob}
%
%

% schwartzchild_metric.tex

%\newcommand{\grad}[0]{\nabla}
%\newcommand{\Abs}[1]{\left\lvert{#1}\right\rvert}
%\newcommand{\spacegrad}[0]{\boldsymbol{\nabla}}
%\newcommand{\LL}[0]{\mathcal{L}}
%\newcommand{\PD}[2]{\frac{\partial {#2}}{\partial {#1}}}
%\newcommand{\PDsQ}[2]{\frac{\partial^2 {#2}}{\partial^2 {#1}}}
%\newcommand{\dotalpha}[0]{\dot{\alpha}}
%\newcommand{\ddotalpha}[0]{\ddot{\alpha}}
%
%\newcommand{\dotomega}[0]{\dot{\omega}}
%\newcommand{\ddotomega}[0]{\ddot{\omega}}
%
%\newcommand{\dotOmega}[0]{\dot{\Omega}}
%\newcommand{\ddotOmega}[0]{\ddot{\Omega}}
%
%\newcommand{\CC}[0]{c^2}
%
%\newcommand{\dottheta}[0]{\dot{\theta}}
%\newcommand{\ddottheta}[0]{\ddot{\theta}}
%
%\newcommand{\dotpsi}[0]{\dot{\psi}}
%\newcommand{\ddotpsi}[0]{\ddot{\psi}}
%
%\newcommand{\adot}[0]{\dot{a}}
%\newcommand{\addot}[0]{\ddot{a}}
%\newcommand{\udot}[0]{\dot{u}}
%\newcommand{\uddot}[0]{\ddot{u}}
%\newcommand{\fdot}[0]{\dot{f}}
%\newcommand{\fddot}[0]{\ddot{f}}
%\newcommand{\bdot}[0]{\dot{b}}
%\newcommand{\bddot}[0]{\ddot{b}}
%\newcommand{\qdot}[0]{\dot{q}}
%\newcommand{\qddot}[0]{\ddot{q}}
%\newcommand{\tdot}[0]{\dot{t}}
%\newcommand{\tddot}[0]{\ddot{t}}
%
%\newcommand{\Rdot}[0]{\dot{R}}
%
%\newcommand{\pdot}[0]{\dot{p}}
%\newcommand{\pddot}[0]{\ddot{p}}
%
%\newcommand{\xdot}[0]{\dot{x}}
%\newcommand{\xddot}[0]{\ddot{x}}
%
%\newcommand{\zdot}[0]{\dot{z}}
%\newcommand{\zddot}[0]{\ddot{z}}
%
%\newcommand{\rdot}[0]{\dot{r}}
%\newcommand{\rddot}[0]{\ddot{r}}
%
%

% shear.tex

%\newcommand{\gpgrade}[2] {{\left\langle{{#1}}\right\rangle}_{#2}}
%
%

% tong_mf1.tex

%\newcommand{\Abs}[1]{\left\lvert{#1}\right\rvert}
%\newcommand{\grad}[0]{\nabla}
%\newcommand{\LL}[0]{\mathcal{L}}
%
%\newcommand{\dotalpha}[0]{\dot{\alpha}}
%\newcommand{\ddotalpha}[0]{\ddot{\alpha}}
%
%\newcommand{\dotomega}[0]{\dot{\omega}}
%\newcommand{\ddotomega}[0]{\ddot{\omega}}
%
%\newcommand{\dottheta}[0]{\dot{\theta}}
%\newcommand{\ddottheta}[0]{\ddot{\theta}}
%
%\newcommand{\dotpsi}[0]{\dot{\psi}}
%\newcommand{\ddotpsi}[0]{\ddot{\psi}}
%
%\newcommand{\qdot}[0]{\dot{q}}
%\newcommand{\qddot}[0]{\ddot{q}}
%
%\newcommand{\Rdot}[0]{\dot{R}}
%
%\newcommand{\pdot}[0]{\dot{p}}
%\newcommand{\pddot}[0]{\ddot{p}}
%
%\newcommand{\xdot}[0]{\dot{x}}
%\newcommand{\xddot}[0]{\ddot{x}}
%
%\newcommand{\zdot}[0]{\dot{z}}
%\newcommand{\zddot}[0]{\ddot{z}}
%
%\newcommand{\rdot}[0]{\dot{r}}
%\newcommand{\rddot}[0]{\ddot{r}}
%
%% == \partial_{#1} {#2}
%\newcommand{\PD}[2]{\frac{\partial {#2}}{\partial {#1}}}
%\newcommand{\PDD}[3]{\frac{\partial^2 {#3}}{\partial {#1}\partial {#2}}}
%
%

% wavepacket.tex

%\newcommand{\PDSq}[2]{\frac{\partial^2 {#2}}{\partial {#1}^2}}
%\newcommand{\IIinf}[0]{ \int_{-\infty}^\infty }
%
%

% wavevariation.tex

%\newcommand{\PDSq}[2]{\frac{\partial^2 {#2}}{\partial {#1}^2}}
%
%

% qm_barrier
%\DeclareMathOperator{\Atan2}{atan2}
%\DeclareMathOperator{\atan}{atan}

% twobodies.tex
%\DeclareMathOperator{\sgn}{sgn}


% sr_lagrangian_q.tex

%\newcommand{\PD}[2]{\frac{\partial {#2}}{\partial {#1}}}
%\newcommand{\xdot}[0]{\dot{x}}
%\newcommand{\xddot}[0]{\ddot{x}}

% stub_em_fields.tex

%\newcommand{\EE}[0]{\boldsymbol{\mathcal{E}}}
%\newcommand{\HH}[0]{\boldsymbol{\mathcal{H}}}
%\newcommand{\PDSq}[2]{\frac{\partial^2 {#2}}{\partial {#1}^2}}

% long_wire_q.tex

%\newcommand{\grad}[0]{\nabla}

% lorentz_tx_em_potential.tex
%\newcommand{\LL}[0]{\mathcal{L}}
%\newcommand{\grad}[0]{\nabla}
%\newcommand{\pdot}[0]{\dot{p}}
%\newcommand{\pddot}[0]{\ddot{p}}

%------------------------------------------------------
% cross_old.tex

%%
%% shorthand for bold symbols, convenient for vectors and matrices
%%
%\newcommand{\Ba}[0]{\mathbf{a}}
%\newcommand{\Bb}[0]{\mathbf{b}}
%\newcommand{\Bc}[0]{\mathbf{c}}
%\newcommand{\Bd}[0]{\mathbf{d}}
%\newcommand{\Be}[0]{\mathbf{e}}
%\newcommand{\Bf}[0]{\mathbf{f}}
%\newcommand{\Bg}[0]{\mathbf{g}}
%\newcommand{\Bh}[0]{\mathbf{h}}
%\newcommand{\Bi}[0]{\mathbf{i}}
%\newcommand{\Bj}[0]{\mathbf{j}}
%\newcommand{\Bk}[0]{\mathbf{k}}
%\newcommand{\Bl}[0]{\mathbf{l}}
%\newcommand{\Bm}[0]{\mathbf{m}}
%\newcommand{\Bn}[0]{\mathbf{n}}
%\newcommand{\Bo}[0]{\mathbf{o}}
%\newcommand{\Bp}[0]{\mathbf{p}}
%\newcommand{\Bq}[0]{\mathbf{q}}
%\newcommand{\Br}[0]{\mathbf{r}}
%\newcommand{\Bs}[0]{\mathbf{s}}
%\newcommand{\Bt}[0]{\mathbf{t}}
%\newcommand{\Bu}[0]{\mathbf{u}}
%\newcommand{\Bv}[0]{\mathbf{v}}
%\newcommand{\Bw}[0]{\mathbf{w}}
%\newcommand{\Bx}[0]{\mathbf{x}}
%\newcommand{\By}[0]{\mathbf{y}}
%\newcommand{\Bz}[0]{\mathbf{z}}
%\newcommand{\BA}[0]{\mathbf{A}}
%\newcommand{\BB}[0]{\mathbf{B}}
%\newcommand{\BC}[0]{\mathbf{C}}
%\newcommand{\BD}[0]{\mathbf{D}}
%\newcommand{\BE}[0]{\mathbf{E}}
%\newcommand{\BF}[0]{\mathbf{F}}
%\newcommand{\BG}[0]{\mathbf{G}}
%\newcommand{\BH}[0]{\mathbf{H}}
%\newcommand{\BI}[0]{\mathbf{I}}
%\newcommand{\BJ}[0]{\mathbf{J}}
%\newcommand{\BK}[0]{\mathbf{K}}
%\newcommand{\BL}[0]{\mathbf{L}}
%\newcommand{\BM}[0]{\mathbf{M}}
%\newcommand{\BN}[0]{\mathbf{N}}
%\newcommand{\BO}[0]{\mathbf{O}}
%\newcommand{\BP}[0]{\mathbf{P}}
%\newcommand{\BQ}[0]{\mathbf{Q}}
%\newcommand{\BR}[0]{\mathbf{R}}
%\newcommand{\BS}[0]{\mathbf{S}}
%\newcommand{\BT}[0]{\mathbf{T}}
%\newcommand{\BU}[0]{\mathbf{U}}
%\newcommand{\BV}[0]{\mathbf{V}}
%\newcommand{\BW}[0]{\mathbf{W}}
%\newcommand{\BX}[0]{\mathbf{X}}
%\newcommand{\BY}[0]{\mathbf{Y}}
%\newcommand{\BZ}[0]{\mathbf{Z}}
%
%\newcommand{\Bzero}[0]{\mathbf{0}}
%\newcommand{\Btheta}[0]{\boldsymbol{\theta}}
%\newcommand{\Btau}[0]{\boldsymbol{\tau}}
%\newcommand{\Bomega}[0]{\boldsymbol{\omega}}
%
%%
%% shorthand for unit vectors
%%
%\newcommand{\acap}[0]{\hat{\Ba}}
%\newcommand{\bcap}[0]{\hat{\Bb}}
%\newcommand{\ccap}[0]{\hat{\Bc}}
%\newcommand{\dcap}[0]{\hat{\Bd}}
%\newcommand{\ecap}[0]{\hat{\Be}}
%\newcommand{\fcap}[0]{\hat{\Bf}}
%\newcommand{\gcap}[0]{\hat{\Bg}}
%\newcommand{\hcap}[0]{\hat{\Bh}}
%\newcommand{\icap}[0]{\hat{\Bi}}
%\newcommand{\jCap}[0]{\hat{\Bj}}
%\newcommand{\kcap}[0]{\hat{\Bk}}
%\newcommand{\lcap}[0]{\hat{\Bl}}
%\newcommand{\mcap}[0]{\hat{\Bm}}
%\newcommand{\ncap}[0]{\hat{\Bn}}
%\newcommand{\ocap}[0]{\hat{\Bo}}
%\newcommand{\pcap}[0]{\hat{\Bp}}
%\newcommand{\qcap}[0]{\hat{\Bq}}
%\newcommand{\rcap}[0]{\hat{\Br}}
%\newcommand{\scap}[0]{\hat{\Bs}}
%\newcommand{\tcap}[0]{\hat{\Bt}}
%\newcommand{\ucap}[0]{\hat{\Bu}}
%\newcommand{\vcap}[0]{\hat{\Bv}}
%\newcommand{\wcap}[0]{\hat{\Bw}}
%\newcommand{\xcap}[0]{\hat{\Bx}}
%\newcommand{\ycap}[0]{\hat{\By}}
%\newcommand{\zcap}[0]{\hat{\Bz}}
%\newcommand{\thetacap}[0]{\hat{\Btheta}}
%
%%
%% to write R^n and C^n in a distinguishable fashion.  Perhaps change this
%% to the double lined characters upon figuring out how to do so.
%%
%\newcommand{\C}[1]{${\BC}^{#1}$}
%\newcommand{\R}[1]{${\BR}^{#1}$}
%
%%
%% various generally useful helpers
%%
%
%% derivative of #1 wrt. #2:
%\newcommand{\D}[2] {\frac {d#2} {d#1}}

%\newcommand{\inv}[1]{\frac{1}{#1}}
%\newcommand{\cross}[0]{\times}

%\newcommand{\abs}[1]{\lvert#1\rvert}
%\newcommand{\norm}[1]{\lVert#1\rVert}
%\newcommand{\innerprod}[2]{\langle{#1}, {#2}\rangle}
%\newcommand{\dotprod}[2]{#1 \cdot #2}
%\newcommand{\crossprod}[2]{#1 \cross #2}
%\newcommand{\tripleprod}[3]{\dotprod{\crossprod{#1}{#2}}{#3}}

%
% A few miscellaneous things specific to this document
%
%\newcommand{\crossop}[1]{\crossprod{#1}{}}

\newcommand{\PDP}[2]{\BP^{#1}\BD{\BP^{#2}}}
\newcommand{\PDPDP}[3]{\Bv^T\BP^{#1}\BD\BP^{#2}\BD\BP^{#3}\Bv}

\newcommand{\Mp}[0]{
\begin{bmatrix}
0 & 1 & 0 & 0 \\
0 & 0 & 1 & 0 \\
0 & 0 & 0 & 1 \\
1 & 0 & 0 & 0
\end{bmatrix}
}
\newcommand{\Mpp}[0]{
\begin{bmatrix}
0 & 0 & 1 & 0 \\
0 & 0 & 0 & 1 \\
1 & 0 & 0 & 0 \\
0 & 1 & 0 & 0
\end{bmatrix}
}
\newcommand{\Mppp}[0]{
\begin{bmatrix}
0 & 0 & 0 & 1 \\
1 & 0 & 0 & 0 \\
0 & 1 & 0 & 0 \\
0 & 0 & 1 & 0
\end{bmatrix}
}
\newcommand{\Mpu}[0]{
\begin{bmatrix}
u_1 & 0 & 0 & 0 \\
0 & u_2 & 0 & 0 \\
0 & 0 & u_3 & 0 \\
0 & 0 & 0 & u_4
\end{bmatrix}
}

%------------------------------------------------------



%\makeindex

\title{Misc Physics and Math Play.}

\author{Peeter Joot  \quad peeter.joot@gmail.com \\
{\small\em \copyright \  Draft date \today }}

\date{ May 31, 2009.  $RCSfile: main.tex,v $ Last $Revision: 1.8 $ $Date: 2009/06/11 17:00:37 $ }
\begin{document}
\maketitle
 \addcontentsline{toc}{chapter}{Contents}
\pagenumbering{roman}
\tableofcontents
\listoffigures
\listoftables
\chapter*{Preface}\normalsize
  \addcontentsline{toc}{chapter}{Preface}
\pagestyle{plain}

%%
% Copyright � 2015 Peeter Joot.  All Rights Reserved.
% Licenced as described in the file LICENSE under the root directory of this GIT repository.
%

% 
%\chapter{Preface}
% this suppresses an explicit chapter number for the preface.
\chapter*{Preface}%\normalsize
  \addcontentsline{toc}{chapter}{Preface}

This document was produced while taking the Spring 2016, University of Toronto Microwave Circuits course (ECE1236H), taught by Prof.\ G. V. Eleftheriades.

\paragraph{Course Syllabus}

This course outlines the principles of designing modern microwave and RF circuits.  Signal-integrity issues in high-speed digital circuits are also examined.

\begin{itemize}
\item The wave equation.
\item Ideal transmission lines.
\item Transients on transmission-lines.
\item Planar transmission lines and introduction to MMIC's.
\item Designing with scattering parameters.
\item Planar power dividers.
\item Directional couplers.
\item Microwave filters.
\item Solid-state microwave amplifiers.
\item Noise.
\item Diode-mixers.
\item RF receiver chains.
\item Oscillators.
\end{itemize}

\withproblemsetsMessage{
\textcolor{Maroon}{
\textit{THIS DOCUMENT IS REDACTED.  THE PROBLEM SET SOLUTIONS AND ASSOCIATED MATHEMATICA CODE IS NOT VISIBLE.  PLEASE EMAIL ME FOR THE FULL VERSION IF YOU ARE NOT TAKING ECE1236.}
}
}

\paragraph{This document contains:}

\begin{itemize}
\item Lecture notes.
\item Personal notes exploring auxiliary details.
\item Worked practice problems.

\ifthenelse{\boolean{redacted}}%
{%
\item Links to Mathematica notebooks associated with the course material and problems (but not problem sets).
}%
{
\item Assigned problems.%
\item Links to Mathematica notebooks associated with problems and course material.%
}
\end{itemize}

%This set of notes is significantly different from my notes for many other classes.  With the class taught on slides (and some of those slides mirroring the text closely), I did not take live notes in class.
%These notes fill in details that I felt deserved clarification, contain problem sets solutions, as well as a number of loosely related musings on Geometric Algebra equivalents to some of the generalized concepts of electromagnetic theory encountered in this class (i.e. magnetic sources).
%
My thanks go to Professor Eleftheriades for teaching this course.

Peeter Joot  \quad peeterjoot@protonmail.com 

\pagestyle{headings}
\pagenumbering{arabic}

%-------------------------------------------------------

\part{algebra}
\documentclass{article}

\usepackage{amsmath}
\usepackage{mathpazo}

%
% shorthand for bold symbols, convenient for vectors and matrices
%
\newcommand{\Ba}[0]{\mathbf{a}}
\newcommand{\Bb}[0]{\mathbf{b}}
\newcommand{\Bc}[0]{\mathbf{c}}
\newcommand{\Bd}[0]{\mathbf{d}}
\newcommand{\Be}[0]{\mathbf{e}}
\newcommand{\Bf}[0]{\mathbf{f}}
\newcommand{\Bg}[0]{\mathbf{g}}
\newcommand{\Bh}[0]{\mathbf{h}}
\newcommand{\Bi}[0]{\mathbf{i}}
\newcommand{\Bj}[0]{\mathbf{j}}
\newcommand{\Bk}[0]{\mathbf{k}}
\newcommand{\Bl}[0]{\mathbf{l}}
\newcommand{\Bm}[0]{\mathbf{m}}
\newcommand{\Bn}[0]{\mathbf{n}}
\newcommand{\Bo}[0]{\mathbf{o}}
\newcommand{\Bp}[0]{\mathbf{p}}
\newcommand{\Bq}[0]{\mathbf{q}}
\newcommand{\Br}[0]{\mathbf{r}}
\newcommand{\Bs}[0]{\mathbf{s}}
\newcommand{\Bt}[0]{\mathbf{t}}
\newcommand{\Bu}[0]{\mathbf{u}}
\newcommand{\Bv}[0]{\mathbf{v}}
\newcommand{\Bw}[0]{\mathbf{w}}
\newcommand{\Bx}[0]{\mathbf{x}}
\newcommand{\By}[0]{\mathbf{y}}
\newcommand{\Bz}[0]{\mathbf{z}}
\newcommand{\BA}[0]{\mathbf{A}}
\newcommand{\BB}[0]{\mathbf{B}}
\newcommand{\BC}[0]{\mathbf{C}}
\newcommand{\BD}[0]{\mathbf{D}}
\newcommand{\BE}[0]{\mathbf{E}}
\newcommand{\BF}[0]{\mathbf{F}}
\newcommand{\BG}[0]{\mathbf{G}}
\newcommand{\BH}[0]{\mathbf{H}}
\newcommand{\BI}[0]{\mathbf{I}}
\newcommand{\BJ}[0]{\mathbf{J}}
\newcommand{\BK}[0]{\mathbf{K}}
\newcommand{\BL}[0]{\mathbf{L}}
\newcommand{\BM}[0]{\mathbf{M}}
\newcommand{\BN}[0]{\mathbf{N}}
\newcommand{\BO}[0]{\mathbf{O}}
\newcommand{\BP}[0]{\mathbf{P}}
\newcommand{\BQ}[0]{\mathbf{Q}}
\newcommand{\BR}[0]{\mathbf{R}}
\newcommand{\BS}[0]{\mathbf{S}}
\newcommand{\BT}[0]{\mathbf{T}}
\newcommand{\BU}[0]{\mathbf{U}}
\newcommand{\BV}[0]{\mathbf{V}}
\newcommand{\BW}[0]{\mathbf{W}}
\newcommand{\BX}[0]{\mathbf{X}}
\newcommand{\BY}[0]{\mathbf{Y}}
\newcommand{\BZ}[0]{\mathbf{Z}}

\newcommand{\Bzero}[0]{\mathbf{0}}
\newcommand{\Btheta}[0]{\boldsymbol{\theta}}
\newcommand{\Btau}[0]{\boldsymbol{\tau}}
\newcommand{\Bomega}[0]{\boldsymbol{\omega}}

%
% shorthand for unit vectors
%
\newcommand{\acap}[0]{\hat{\Ba}}
\newcommand{\bcap}[0]{\hat{\Bb}}
\newcommand{\ccap}[0]{\hat{\Bc}}
\newcommand{\dcap}[0]{\hat{\Bd}}
\newcommand{\ecap}[0]{\hat{\Be}}
\newcommand{\fcap}[0]{\hat{\Bf}}
\newcommand{\gcap}[0]{\hat{\Bg}}
\newcommand{\hcap}[0]{\hat{\Bh}}
\newcommand{\icap}[0]{\hat{\Bi}}
\newcommand{\jcap}[0]{\hat{\Bj}}
\newcommand{\kcap}[0]{\hat{\Bk}}
\newcommand{\lcap}[0]{\hat{\Bl}}
\newcommand{\mcap}[0]{\hat{\Bm}}
\newcommand{\ncap}[0]{\hat{\Bn}}
\newcommand{\ocap}[0]{\hat{\Bo}}
\newcommand{\pcap}[0]{\hat{\Bp}}
\newcommand{\qcap}[0]{\hat{\Bq}}
\newcommand{\rcap}[0]{\hat{\Br}}
\newcommand{\scap}[0]{\hat{\Bs}}
\newcommand{\tcap}[0]{\hat{\Bt}}
\newcommand{\ucap}[0]{\hat{\Bu}}
\newcommand{\vcap}[0]{\hat{\Bv}}
\newcommand{\wcap}[0]{\hat{\Bw}}
\newcommand{\xcap}[0]{\hat{\Bx}}
\newcommand{\ycap}[0]{\hat{\By}}
\newcommand{\zcap}[0]{\hat{\Bz}}
\newcommand{\thetacap}[0]{\hat{\Btheta}}

%
% to write R^n and C^n in a distinguishable fashion.  Perhaps change this
% to the double lined characters upon figuring out how to do so.
%
\newcommand{\C}[1]{$\mathbb{C}^{#1}$}
\newcommand{\R}[1]{$\mathbb{R}^{#1}$}

%
% various generally useful helpers
%

% derivative of #1 wrt. #2:
\newcommand{\D}[2] {\frac {d#2} {d#1}}

\newcommand{\inv}[1]{\frac{1}{#1}}
\newcommand{\cross}[0]{\times}

\newcommand{\abs}[1]{\lvert{#1}\rvert}
\newcommand{\norm}[1]{\lVert{#1}\rVert}
\newcommand{\innerprod}[2]{\langle{#1}, {#2}\rangle}
\newcommand{\dotprod}[2]{{#1} \cdot {#2}}
\newcommand{\bdotprod}[2]{\left({#1} \cdot {#2}\right)}
\newcommand{\crossprod}[2]{{#1} \cross {#2}}
\newcommand{\tripleprod}[3]{\dotprod{\left(\crossprod{#1}{#2}\right)}{#3}}

\DeclareMathOperator{\Proj}{Proj}
\DeclareMathOperator{\Span}{span}
\DeclareMathOperator{\Sgn}{sgn}
\DeclareMathOperator{\Area}{Area}
\DeclareMathOperator{\Volume}{Volume}

%
% A few miscellaneous things specific to this document
%
\newcommand{\crossop}[1]{\crossprod{#1}{}}

% R2 vector.
\newcommand{\VectorTwo}[2]{
\begin{bmatrix}
 {#1} \\
 {#2}
\end{bmatrix}
}

\newcommand{\VectorN}[1]{
\begin{bmatrix}
{#1}_1 \\
{#1}_2 \\
\vdots \\
{#1}_N \\
\end{bmatrix}
}

\newcommand{\DETuvij}[4]{
\begin{vmatrix}
 {#1}_{#3} & {#1}_{#4} \\
 {#2}_{#3} & {#2}_{#4}
\end{vmatrix}
}

\newcommand{\DETuvwijk}[6]{
\begin{vmatrix}
 {#1}_{#4} & {#1}_{#5} & {#1}_{#6} \\
 {#2}_{#4} & {#2}_{#5} & {#2}_{#6} \\
 {#3}_{#4} & {#3}_{#5} & {#3}_{#6}
\end{vmatrix}
}

\newcommand{\DETuvwxijkl}[8]{
\begin{vmatrix}
 {#1}_{#5} & {#1}_{#6} & {#1}_{#7} & {#1}_{#8} \\
 {#2}_{#5} & {#2}_{#6} & {#2}_{#7} & {#2}_{#8} \\
 {#3}_{#5} & {#3}_{#6} & {#3}_{#7} & {#3}_{#8} \\
 {#4}_{#5} & {#4}_{#6} & {#4}_{#7} & {#4}_{#8} \\
\end{vmatrix}
}

%\newcommand{\DETuvwxyijklm}[10]{
%\begin{vmatrix}
% {#1}_{#6} & {#1}_{#7} & {#1}_{#8} & {#1}_{#9} & {#1}_{#10} \\
% {#2}_{#6} & {#2}_{#7} & {#2}_{#8} & {#2}_{#9} & {#2}_{#10} \\
% {#3}_{#6} & {#3}_{#7} & {#3}_{#8} & {#3}_{#9} & {#3}_{#10} \\
% {#4}_{#6} & {#4}_{#7} & {#4}_{#8} & {#4}_{#9} & {#4}_{#10} \\
% {#5}_{#6} & {#5}_{#7} & {#5}_{#8} & {#5}_{#9} & {#5}_{#10}
%\end{vmatrix}
%}

% R3 vector.
\newcommand{\VectorThree}[3]{
\begin{bmatrix}
 {#1} \\
 {#2} \\
 {#3}
\end{bmatrix}
}


%<misc>
%
\newcommand{\Abs}[1]{{\left\lvert{#1}\right\rvert}}
\newcommand{\spacegrad}[0]{\boldsymbol{\nabla}}
\newcommand{\grad}[0]{\nabla}
\newcommand{\LL}[0]{\mathcal{L}}

% == \partial_{#1} {#2}
\newcommand{\PD}[2]{\frac{\partial {#2}}{\partial {#1}}}
% inline variant
\newcommand{\PDi}[2]{{\partial {#2}}/{\partial {#1}}}

\newcommand{\PDD}[3]{\frac{\partial^2 {#3}}{\partial {#1}\partial {#2}}}
%\newcommand{\PDd}[2]{\frac{\partial^2 {#2}}{{\partial{#1}}^2}}
\newcommand{\PDsq}[2]{\frac{\partial^2 {#2}}{(\partial {#1})^2}}

\newcommand{\Partial}[2]{\frac{\partial {#1}}{\partial {#2}}}
\DeclareMathOperator{\RejName}{Rej}
\newcommand{\Rej}[2]{\RejName_{#1}\left( {#2} \right)}
\newcommand{\Rm}[1]{\mathbb{R}^{#1}}
\newcommand{\Cm}[1]{\mathbb{C}^{#1}}
\newcommand{\conj}[0]{{*}}

%</misc>

% <grade selection>
%
\newcommand{\gpgrade}[2] {{\left\langle{{#1}}\right\rangle}_{#2}}

\newcommand{\gpgradezero}[1] {\gpgrade{#1}{}}
%\newcommand{\gpscalargrade}[1] {{\left\langle{{#1}}\right\rangle}}
%\newcommand{\gpgradezero}[1] {\gpgrade{#1}{0}}

%\newcommand{\gpgradeone}[1] {{\left\langle{{#1}}\right\rangle}_{1}}
\newcommand{\gpgradeone}[1] {\gpgrade{#1}{1}}

\newcommand{\gpgradetwo}[1] {\gpgrade{#1}{2}}
\newcommand{\gpgradethree}[1] {\gpgrade{#1}{3}}
\newcommand{\gpgradefour}[1] {\gpgrade{#1}{4}}
%
% </grade selection>



\newcommand{\adot}[0]{{\dot{a}}}
\newcommand{\bdot}[0]{{\dot{b}}}
% taken for centered dot:
%\newcommand{\cdot}[0]{{\dot{c}}}
%\newcommand{\ddot}[0]{{\dot{d}}}
\newcommand{\edot}[0]{{\dot{e}}}
\newcommand{\fdot}[0]{{\dot{f}}}
\newcommand{\gdot}[0]{{\dot{g}}}
\newcommand{\hdot}[0]{{\dot{h}}}
\newcommand{\idot}[0]{{\dot{i}}}
\newcommand{\jdot}[0]{{\dot{j}}}
\newcommand{\kdot}[0]{{\dot{k}}}
\newcommand{\ldot}[0]{{\dot{l}}}
\newcommand{\mdot}[0]{{\dot{m}}}
\newcommand{\ndot}[0]{{\dot{n}}}
%\newcommand{\odot}[0]{{\dot{o}}}
\newcommand{\pdot}[0]{{\dot{p}}}
\newcommand{\qdot}[0]{{\dot{q}}}
\newcommand{\rdot}[0]{{\dot{r}}}
\newcommand{\sdot}[0]{{\dot{s}}}
\newcommand{\tdot}[0]{{\dot{t}}}
\newcommand{\udot}[0]{{\dot{u}}}
\newcommand{\vdot}[0]{{\dot{v}}}
\newcommand{\wdot}[0]{{\dot{w}}}
\newcommand{\xdot}[0]{{\dot{x}}}
\newcommand{\ydot}[0]{{\dot{y}}}
\newcommand{\zdot}[0]{{\dot{z}}}
\newcommand{\addot}[0]{{\ddot{a}}}
\newcommand{\bddot}[0]{{\ddot{b}}}
\newcommand{\cddot}[0]{{\ddot{c}}}
%\newcommand{\dddot}[0]{{\ddot{d}}}
\newcommand{\eddot}[0]{{\ddot{e}}}
\newcommand{\fddot}[0]{{\ddot{f}}}
\newcommand{\gddot}[0]{{\ddot{g}}}
\newcommand{\hddot}[0]{{\ddot{h}}}
\newcommand{\iddot}[0]{{\ddot{i}}}
\newcommand{\jddot}[0]{{\ddot{j}}}
\newcommand{\kddot}[0]{{\ddot{k}}}
\newcommand{\lddot}[0]{{\ddot{l}}}
\newcommand{\mddot}[0]{{\ddot{m}}}
\newcommand{\nddot}[0]{{\ddot{n}}}
\newcommand{\oddot}[0]{{\ddot{o}}}
\newcommand{\pddot}[0]{{\ddot{p}}}
\newcommand{\qddot}[0]{{\ddot{q}}}
\newcommand{\rddot}[0]{{\ddot{r}}}
\newcommand{\sddot}[0]{{\ddot{s}}}
\newcommand{\tddot}[0]{{\ddot{t}}}
\newcommand{\uddot}[0]{{\ddot{u}}}
\newcommand{\vddot}[0]{{\ddot{v}}}
\newcommand{\wddot}[0]{{\ddot{w}}}
\newcommand{\xddot}[0]{{\ddot{x}}}
\newcommand{\yddot}[0]{{\ddot{y}}}
\newcommand{\zddot}[0]{{\ddot{z}}}

%<bold and dot greek symbols>
%

\newcommand{\Deltadot}[0]{{\dot{\Delta}}}
\newcommand{\Gammadot}[0]{{\dot{\Gamma}}}
\newcommand{\Lambdadot}[0]{{\dot{\Lambda}}}
\newcommand{\Omegadot}[0]{{\dot{\Omega}}}
\newcommand{\Phidot}[0]{{\dot{\Phi}}}
\newcommand{\Pidot}[0]{{\dot{\Pi}}}
\newcommand{\Psidot}[0]{{\dot{\Psi}}}
\newcommand{\Sigmadot}[0]{{\dot{\Sigma}}}
\newcommand{\Thetadot}[0]{{\dot{\Theta}}}
\newcommand{\Upsilondot}[0]{{\dot{\Upsilon}}}
\newcommand{\Xidot}[0]{{\dot{\Xi}}}
\newcommand{\alphadot}[0]{{\dot{\alpha}}}
\newcommand{\betadot}[0]{{\dot{\beta}}}
\newcommand{\chidot}[0]{{\dot{\chi}}}
\newcommand{\deltadot}[0]{{\dot{\delta}}}
\newcommand{\epsilondot}[0]{{\dot{\epsilon}}}
\newcommand{\etadot}[0]{{\dot{\eta}}}
\newcommand{\gammadot}[0]{{\dot{\gamma}}}
\newcommand{\kappadot}[0]{{\dot{\kappa}}}
\newcommand{\lambdadot}[0]{{\dot{\lambda}}}
\newcommand{\mudot}[0]{{\dot{\mu}}}
\newcommand{\nudot}[0]{{\dot{\nu}}}
\newcommand{\omegadot}[0]{{\dot{\omega}}}
\newcommand{\phidot}[0]{{\dot{\phi}}}
\newcommand{\pidot}[0]{{\dot{\pi}}}
\newcommand{\psidot}[0]{{\dot{\psi}}}
\newcommand{\rhodot}[0]{{\dot{\rho}}}
\newcommand{\sigmadot}[0]{{\dot{\sigma}}}
\newcommand{\taudot}[0]{{\dot{\tau}}}
\newcommand{\thetadot}[0]{{\dot{\theta}}}
\newcommand{\upsilondot}[0]{{\dot{\upsilon}}}
\newcommand{\varepsilondot}[0]{{\dot{\varepsilon}}}
\newcommand{\varphidot}[0]{{\dot{\varphi}}}
\newcommand{\varpidot}[0]{{\dot{\varpi}}}
\newcommand{\varrhodot}[0]{{\dot{\varrho}}}
\newcommand{\varsigmadot}[0]{{\dot{\varsigma}}}
\newcommand{\varthetadot}[0]{{\dot{\vartheta}}}
\newcommand{\xidot}[0]{{\dot{\xi}}}
\newcommand{\zetadot}[0]{{\dot{\zeta}}}

\newcommand{\Deltaddot}[0]{{\ddot{\Delta}}}
\newcommand{\Gammaddot}[0]{{\ddot{\Gamma}}}
\newcommand{\Lambdaddot}[0]{{\ddot{\Lambda}}}
\newcommand{\Omegaddot}[0]{{\ddot{\Omega}}}
\newcommand{\Phiddot}[0]{{\ddot{\Phi}}}
\newcommand{\Piddot}[0]{{\ddot{\Pi}}}
\newcommand{\Psiddot}[0]{{\ddot{\Psi}}}
\newcommand{\Sigmaddot}[0]{{\ddot{\Sigma}}}
\newcommand{\Thetaddot}[0]{{\ddot{\Theta}}}
\newcommand{\Upsilonddot}[0]{{\ddot{\Upsilon}}}
\newcommand{\Xiddot}[0]{{\ddot{\Xi}}}
\newcommand{\alphaddot}[0]{{\ddot{\alpha}}}
\newcommand{\betaddot}[0]{{\ddot{\beta}}}
\newcommand{\chiddot}[0]{{\ddot{\chi}}}
\newcommand{\deltaddot}[0]{{\ddot{\delta}}}
\newcommand{\epsilonddot}[0]{{\ddot{\epsilon}}}
\newcommand{\etaddot}[0]{{\ddot{\eta}}}
\newcommand{\gammaddot}[0]{{\ddot{\gamma}}}
\newcommand{\kappaddot}[0]{{\ddot{\kappa}}}
\newcommand{\lambdaddot}[0]{{\ddot{\lambda}}}
\newcommand{\muddot}[0]{{\ddot{\mu}}}
\newcommand{\nuddot}[0]{{\ddot{\nu}}}
\newcommand{\omegaddot}[0]{{\ddot{\omega}}}
\newcommand{\phiddot}[0]{{\ddot{\phi}}}
\newcommand{\piddot}[0]{{\ddot{\pi}}}
\newcommand{\psiddot}[0]{{\ddot{\psi}}}
\newcommand{\rhoddot}[0]{{\ddot{\rho}}}
\newcommand{\sigmaddot}[0]{{\ddot{\sigma}}}
\newcommand{\tauddot}[0]{{\ddot{\tau}}}
\newcommand{\thetaddot}[0]{{\ddot{\theta}}}
\newcommand{\upsilonddot}[0]{{\ddot{\upsilon}}}
\newcommand{\varepsilonddot}[0]{{\ddot{\varepsilon}}}
\newcommand{\varphiddot}[0]{{\ddot{\varphi}}}
\newcommand{\varpiddot}[0]{{\ddot{\varpi}}}
\newcommand{\varrhoddot}[0]{{\ddot{\varrho}}}
\newcommand{\varsigmaddot}[0]{{\ddot{\varsigma}}}
\newcommand{\varthetaddot}[0]{{\ddot{\vartheta}}}
\newcommand{\xiddot}[0]{{\ddot{\xi}}}
\newcommand{\zetaddot}[0]{{\ddot{\zeta}}}

\newcommand{\BDelta}[0]{\boldsymbol{\Delta}}
\newcommand{\BGamma}[0]{\boldsymbol{\Gamma}}
\newcommand{\BLambda}[0]{\boldsymbol{\Lambda}}
\newcommand{\BOmega}[0]{\boldsymbol{\Omega}}
\newcommand{\BPhi}[0]{\boldsymbol{\Phi}}
\newcommand{\BPi}[0]{\boldsymbol{\Pi}}
\newcommand{\BPsi}[0]{\boldsymbol{\Psi}}
\newcommand{\BSigma}[0]{\boldsymbol{\Sigma}}
\newcommand{\BTheta}[0]{\boldsymbol{\Theta}}
\newcommand{\BUpsilon}[0]{\boldsymbol{\Upsilon}}
\newcommand{\BXi}[0]{\boldsymbol{\Xi}}
\newcommand{\Balpha}[0]{\boldsymbol{\alpha}}
\newcommand{\Bbeta}[0]{\boldsymbol{\beta}}
\newcommand{\Bchi}[0]{\boldsymbol{\chi}}
\newcommand{\Bdelta}[0]{\boldsymbol{\delta}}
\newcommand{\Bepsilon}[0]{\boldsymbol{\epsilon}}
\newcommand{\Beta}[0]{\boldsymbol{\eta}}
\newcommand{\Bgamma}[0]{\boldsymbol{\gamma}}
\newcommand{\Bkappa}[0]{\boldsymbol{\kappa}}
\newcommand{\Blambda}[0]{\boldsymbol{\lambda}}
\newcommand{\Bmu}[0]{\boldsymbol{\mu}}
\newcommand{\Bnu}[0]{\boldsymbol{\nu}}
%\newcommand{\Bomega}[0]{\boldsymbol{\omega}}
\newcommand{\Bphi}[0]{\boldsymbol{\phi}}
\newcommand{\Bpi}[0]{\boldsymbol{\pi}}
\newcommand{\Bpsi}[0]{\boldsymbol{\psi}}
\newcommand{\Brho}[0]{\boldsymbol{\rho}}
\newcommand{\Bsigma}[0]{\boldsymbol{\sigma}}
%\newcommand{\Btau}[0]{\boldsymbol{\tau}}
%\newcommand{\Btheta}[0]{\boldsymbol{\theta}}
\newcommand{\Bupsilon}[0]{\boldsymbol{\upsilon}}
\newcommand{\Bvarepsilon}[0]{\boldsymbol{\varepsilon}}
\newcommand{\Bvarphi}[0]{\boldsymbol{\varphi}}
\newcommand{\Bvarpi}[0]{\boldsymbol{\varpi}}
\newcommand{\Bvarrho}[0]{\boldsymbol{\varrho}}
\newcommand{\Bvarsigma}[0]{\boldsymbol{\varsigma}}
\newcommand{\Bvartheta}[0]{\boldsymbol{\vartheta}}
\newcommand{\Bxi}[0]{\boldsymbol{\xi}}
\newcommand{\Bzeta}[0]{\boldsymbol{\zeta}}
%
%</bold and dot greek symbols>
%<infrequent>
%
%\newcommand{\AreaOp}[1]{\AName_{#1}}
%\newcommand{\Babs}[0]{\abs{\BB}}
%\newcommand{\Bcap}[0]{\hat{\BB}}
%\newcommand{\BrPrimeRej}[0]{\rcap(\rcap \wedge \Br')}
%\newcommand{\CA}[0]{\mathcal{A}}
%\newcommand{\Cos}[1]{\cos{\left({#1}\right)}}
%\newcommand{\Det}[1] {\abs{#1}}
%\newcommand{\Dsq}[2] {\frac {\partial^2 {#1}} {\partial {#2}^2}}
%\newcommand{\Exp}[1]{\exp{\left({#1}\right)}}
%\newcommand{\Norm}[1]{\left\lVert{#1}\right\rVert}
%\newcommand{\Sin}[1]{\sin{\left({#1}\right)}}
%\newcommand{\T}[0]{\text{T}}
%\newcommand{\VolumeOp}[1]{\VName_{#1}}
%\newcommand{\agrad}[0]{\Ba \cdot \nabla}
%\newcommand{\alphacap}[0]{\hat{\boldsymbol{\alpha}}}
%\newcommand{\Fcap}[0]{\hat{\BF}}
%\newcommand{\bithree}[0]{{\Bi}_3}
%\newcommand{\bxa}[0]{\Bx\Ba}
%\newcommand{\coordvec}[2]{
%\newcommand{\costheta}[0]{\acap \cdot \xcap}
%\newcommand{\ddt}[1]{\ddot{#1}}
%\newcommand{\ddu}[1] {\frac {d{#1}} {du}}
%\newcommand{\dsqxj}[2] {\frac {\partial^2 {#1}} {\partial {x_{#2}}^2}}
%\newcommand{\dtheta}[1]{\frac{d {#1}}{d \theta}}
%\newcommand{\dt}[1]{\dot{#1}}
%\newcommand{\dt}[1]{\frac{d {#1}}{dt}}
%\newcommand{\dxj}[2] {\frac {\partial {#1}} {\partial {x_{#2}}}}
%\newcommand{\halfPhi}[0]{\frac{\phi}{2}}
%\newcommand{\half}[0]{\inv{2}}
%\newcommand{\inv}[1]{\frac{1}{#1}}
%\newcommand{\laplacian}[0]{\nabla^2}
%\newcommand{\matrixoftx}[3]{
%\newcommand{\nrrp}[0]{\norm{\rcap \wedge \Br'}}
%\newcommand{\oiint}{\bigcirc \hspace{-1.4em} \int \hspace{-.8em} \int}
%\newcommand{\transpose}[1]{{#1}^{\text{T}}}
%\newcommand{\transpose}[1]{{{#1}^{\TextTranspose}}}
%\newcommand{\transpose}[1]{{{#1}^{\text{T}}}}
%\newcommand{\barA}[0]{\bar{A}}
%\newcommand{\qbar}[0]{\bar{q}}
%\newcommand{\qdotbar}[0]{\dot{\bar{q}}}
%
%</infrequent>





%\usepackage{listings}
\usepackage{txfonts} % for ointctr... (also appears to make "prettier" \int and \sum's)
\usepackage[bookmarks=true]{hyperref}

\usepackage{color,cite,graphicx}
   % use colour in the document, put your citations as [1-4]
   % rather than [1,2,3,4] (it looks nicer, and the extended LaTeX2e
   % graphics package. 
\usepackage{latexsym,amssymb,epsf} % don't remember if these are
   % needed, but their inclusion can't do any damage


\title{ Integer binomial theorem induction, the easy dumb way. }
\author{Peeter Joot \quad peeter.joot@gmail.com }
\date{ March 26, 2009.  Last Revision: $Date: 2009/03/27 00:59:32 $ }

\begin{document}
\maketitle{}

%\tableofcontents

\section{ Motivation. }

While working a problem with an induction requirement similar to but more 
complicated than the binomial theorem, I steped back and thought I'd try
this as an easier first step.  Had some trouble doing it, until I tried it
explicitly with for the power of three case.  Ironically, working it
out for an explicit index takes the abstraction out of the problem, and
generalizing further really only requires a search and replace.

\section{ Do it. }

Want to prove

\begin{align}
(t + x)^k = \sum_{m=0}^k \binom{k}{m} t^m x^{k-m}
\end{align}

where

\begin{align}
\binom{k}{m} = \frac{k!}{(k-m)!m!}
\end{align}

In particular want to prove this for the $k+1$ case, given $k$.

\subsection{ step with k+1 = 3 }

Three isn't actually the best way to start since it is almost too trivial, but
one gets the idea easily by doing it.

\begin{align*}
(t + x)^3 
&= (t + x)(t + x)^2  \\
&= (t + x)\sum_{m=0}^2 \binom{2}{m} t^m x^{2-m} \\
&= 
\sum_{m=0}^2 \binom{2}{m} t^{m+1} x^{2-m} 
+ \sum_{m=0}^2 \binom{2}{m} t^{m} x^{2-m + 1} \\
&= 
\sum_{m=1}^{2 + 1} \binom{2}{m-1} t^{m} x^{2 - m + 1} 
+ \sum_{m=0}^2 \binom{2}{m} t^{m} x^{2-m + 1} \\
\end{align*}

Now pull the lowest and highest order terms out of the sums, and group the
remaining bits.

\begin{align*}
(t + x)^3 
&= 
 \binom{2}{0} t^{0} x^{2 - 0 + 1} 
+ \sum_{m=1}^{2} \left( \binom{2}{m} + \binom{2}{m-1} \right) t^{m} x^{2 - m + 1} 
+ \binom{2}{2 + 1 -1} t^{2 + 1} x^{2 - (2 + 1) + 1} 
\\
\end{align*}

Now, observe that in all the steps above everywhere if all places that $2$, a nice easy to think with and concrete number, we could have used some
abstract index.

\subsection{ step with k+1 }

A straight text search and replace on $2$ with $k$ gives

\begin{align*}
(t + x)^{k+1}
&= (t + x)(t + x)^k  \\
&= (t + x)\sum_{m=0}^k \binom{k}{m} t^m x^{k-m} \\
&= 
\sum_{m=0}^k \binom{k}{m} t^{m+1} x^{k-m} 
+ \sum_{m=0}^k \binom{k}{m} t^{m} x^{k-m + 1} \\
&= 
\sum_{m=1}^{k + 1} \binom{k}{m-1} t^{m} x^{k - m + 1} 
+ \sum_{m=0}^k \binom{k}{m} t^{m} x^{k-m + 1} \\
&= 
 \binom{k}{0} t^{0} x^{k - 0 + 1} 
+ \sum_{m=1}^{k} \left( \binom{k}{m} + \binom{k}{m-1} \right) t^{m} x^{k - m + 1} 
+ \binom{k}{k + 1 -1} t^{k + 1} x^{k - (k + 1) + 1} \\
\end{align*}

This is EXACTLY the same as above with $k=2$ but it sure looks more complicated with an an abstract index.

To finish off we need a couple observations, that 
$\binom{k}{0} = 1 = \binom{k+1}{0}$, and $\binom{k}{k} = 1 = \binom{k+1}{k+1}$.  This leaves us with

\begin{align*}
(t + x)^{k+1}
= \binom{k+1}{0} t^{0} x^{k + 1} 
+ \sum_{m=1}^{k} \left( \binom{k}{m} + \binom{k}{m-1} \right) t^{m} x^{k - m + 1} 
+ \binom{k + 1}{k + 1} t^{k + 1} x^{0} \\
\end{align*}

So if we can show
\begin{align*}
\binom{k}{m} + \binom{k}{m-1} = \binom{k+1}{m}
\end{align*}

then we would have

\begin{align*}
(t + x)^{k+1}
&= \sum_{m=0}^{k+1} \binom{k+1}{m} t^{m} x^{k + 1 - m} 
\end{align*}

which is what was desired.

\subsection{ That last little piece. }

To prove that last little piece, let's do it again the dumb way, and let a regular expression $s/2/k/g$ in vim do the hard work.

\begin{align*}
\binom{2}{m} + \binom{2}{m-1} 
&=
\frac{2!}{(2-m)!m!}
+ \frac{2!}{(2-m + 1)!(m-1)!} \\
&=
\frac{2!}{(2-m)!m(m-1)!}
+ \frac{2!}{(3-m)!(m-1)!} \\
&=
\frac{2!}{(2-m)!m(m-1)!}
+ \frac{2!}{(3-m)(2-m)!(m-1)!} \\
&=
\frac{2!}{(2-m)!(m-1)!} \left( \inv{m} + \inv{3-m}\right) \\
&=
\frac{2!}{(2-m)!(m-1)!} \frac{3 -m + m }{m(3-m)} \\
&=
\frac{3!}{(3-m)!(m)!} \\
\end{align*}

So, generalizing the easy way with $s/3/k+1/g$, we have 

\begin{align*}
\binom{k}{m} + \binom{k}{m-1} 
&=
\frac{k!}{(k-m)!m!}
+ \frac{k!}{(k-m + 1)!(m-1)!} \\
&=
\frac{k!}{(k-m)!m(m-1)!}
+ \frac{k!}{((k+1)-m)!(m-1)!} \\
&=
\frac{k!}{(k-m)!m(m-1)!}
+ \frac{k!}{((k+1)-m)(k-m)!(m-1)!} \\
&=
\frac{k!}{(k-m)!(m-1)!} \left( \inv{m} + \inv{(k+1)-m}\right) \\
&=
\frac{k!}{(k-m)!(m-1)!} \frac{(k+1) -m + m }{m((k+1)-m)} \\
&=
\frac{(k+1)!}{((k+1)-m)!(m)!} \\
\end{align*}

Every step is EXACTLY the same as with $k=2$, the only differences were straight text substitution.  That leaves us with

\begin{align}
\binom{k}{m} + \binom{k}{m-1} 
&=
\binom{k+1}{m}
\end{align}

That's all we needed to complete the proof.

I think this is a superior way to do inductive proofs.  Just do the absolute easiest case and do it with a number that is easy to think with.  Search and replace in an editor does all the bits that would make you look clever if you were to leave off the fact that you were really only doing the easy version!

%\bibliographystyle{plainnat}
%\bibliography{myrefs}

\end{document}

%\documentclass{article}

%\usepackage{amsmath}
\usepackage{mathpazo}

%
% shorthand for bold symbols, convenient for vectors and matrices
%
\newcommand{\Ba}[0]{\mathbf{a}}
\newcommand{\Bb}[0]{\mathbf{b}}
\newcommand{\Bc}[0]{\mathbf{c}}
\newcommand{\Bd}[0]{\mathbf{d}}
\newcommand{\Be}[0]{\mathbf{e}}
\newcommand{\Bf}[0]{\mathbf{f}}
\newcommand{\Bg}[0]{\mathbf{g}}
\newcommand{\Bh}[0]{\mathbf{h}}
\newcommand{\Bi}[0]{\mathbf{i}}
\newcommand{\Bj}[0]{\mathbf{j}}
\newcommand{\Bk}[0]{\mathbf{k}}
\newcommand{\Bl}[0]{\mathbf{l}}
\newcommand{\Bm}[0]{\mathbf{m}}
\newcommand{\Bn}[0]{\mathbf{n}}
\newcommand{\Bo}[0]{\mathbf{o}}
\newcommand{\Bp}[0]{\mathbf{p}}
\newcommand{\Bq}[0]{\mathbf{q}}
\newcommand{\Br}[0]{\mathbf{r}}
\newcommand{\Bs}[0]{\mathbf{s}}
\newcommand{\Bt}[0]{\mathbf{t}}
\newcommand{\Bu}[0]{\mathbf{u}}
\newcommand{\Bv}[0]{\mathbf{v}}
\newcommand{\Bw}[0]{\mathbf{w}}
\newcommand{\Bx}[0]{\mathbf{x}}
\newcommand{\By}[0]{\mathbf{y}}
\newcommand{\Bz}[0]{\mathbf{z}}
\newcommand{\BA}[0]{\mathbf{A}}
\newcommand{\BB}[0]{\mathbf{B}}
\newcommand{\BC}[0]{\mathbf{C}}
\newcommand{\BD}[0]{\mathbf{D}}
\newcommand{\BE}[0]{\mathbf{E}}
\newcommand{\BF}[0]{\mathbf{F}}
\newcommand{\BG}[0]{\mathbf{G}}
\newcommand{\BH}[0]{\mathbf{H}}
\newcommand{\BI}[0]{\mathbf{I}}
\newcommand{\BJ}[0]{\mathbf{J}}
\newcommand{\BK}[0]{\mathbf{K}}
\newcommand{\BL}[0]{\mathbf{L}}
\newcommand{\BM}[0]{\mathbf{M}}
\newcommand{\BN}[0]{\mathbf{N}}
\newcommand{\BO}[0]{\mathbf{O}}
\newcommand{\BP}[0]{\mathbf{P}}
\newcommand{\BQ}[0]{\mathbf{Q}}
\newcommand{\BR}[0]{\mathbf{R}}
\newcommand{\BS}[0]{\mathbf{S}}
\newcommand{\BT}[0]{\mathbf{T}}
\newcommand{\BU}[0]{\mathbf{U}}
\newcommand{\BV}[0]{\mathbf{V}}
\newcommand{\BW}[0]{\mathbf{W}}
\newcommand{\BX}[0]{\mathbf{X}}
\newcommand{\BY}[0]{\mathbf{Y}}
\newcommand{\BZ}[0]{\mathbf{Z}}

\newcommand{\Bzero}[0]{\mathbf{0}}
\newcommand{\Btheta}[0]{\boldsymbol{\theta}}
\newcommand{\Btau}[0]{\boldsymbol{\tau}}
\newcommand{\Bomega}[0]{\boldsymbol{\omega}}

%
% shorthand for unit vectors
%
\newcommand{\acap}[0]{\hat{\Ba}}
\newcommand{\bcap}[0]{\hat{\Bb}}
\newcommand{\ccap}[0]{\hat{\Bc}}
\newcommand{\dcap}[0]{\hat{\Bd}}
\newcommand{\ecap}[0]{\hat{\Be}}
\newcommand{\fcap}[0]{\hat{\Bf}}
\newcommand{\gcap}[0]{\hat{\Bg}}
\newcommand{\hcap}[0]{\hat{\Bh}}
\newcommand{\icap}[0]{\hat{\Bi}}
\newcommand{\jcap}[0]{\hat{\Bj}}
\newcommand{\kcap}[0]{\hat{\Bk}}
\newcommand{\lcap}[0]{\hat{\Bl}}
\newcommand{\mcap}[0]{\hat{\Bm}}
\newcommand{\ncap}[0]{\hat{\Bn}}
\newcommand{\ocap}[0]{\hat{\Bo}}
\newcommand{\pcap}[0]{\hat{\Bp}}
\newcommand{\qcap}[0]{\hat{\Bq}}
\newcommand{\rcap}[0]{\hat{\Br}}
\newcommand{\scap}[0]{\hat{\Bs}}
\newcommand{\tcap}[0]{\hat{\Bt}}
\newcommand{\ucap}[0]{\hat{\Bu}}
\newcommand{\vcap}[0]{\hat{\Bv}}
\newcommand{\wcap}[0]{\hat{\Bw}}
\newcommand{\xcap}[0]{\hat{\Bx}}
\newcommand{\ycap}[0]{\hat{\By}}
\newcommand{\zcap}[0]{\hat{\Bz}}
\newcommand{\thetacap}[0]{\hat{\Btheta}}

%
% to write R^n and C^n in a distinguishable fashion.  Perhaps change this
% to the double lined characters upon figuring out how to do so.
%
\newcommand{\C}[1]{$\mathbb{C}^{#1}$}
\newcommand{\R}[1]{$\mathbb{R}^{#1}$}

%
% various generally useful helpers
%

% derivative of #1 wrt. #2:
\newcommand{\D}[2] {\frac {d#2} {d#1}}

\newcommand{\inv}[1]{\frac{1}{#1}}
\newcommand{\cross}[0]{\times}

\newcommand{\abs}[1]{\lvert{#1}\rvert}
\newcommand{\norm}[1]{\lVert{#1}\rVert}
\newcommand{\innerprod}[2]{\langle{#1}, {#2}\rangle}
\newcommand{\dotprod}[2]{{#1} \cdot {#2}}
\newcommand{\bdotprod}[2]{\left({#1} \cdot {#2}\right)}
\newcommand{\crossprod}[2]{{#1} \cross {#2}}
\newcommand{\tripleprod}[3]{\dotprod{\left(\crossprod{#1}{#2}\right)}{#3}}

\DeclareMathOperator{\Proj}{Proj}
\DeclareMathOperator{\Span}{span}
\DeclareMathOperator{\Sgn}{sgn}
\DeclareMathOperator{\Area}{Area}
\DeclareMathOperator{\Volume}{Volume}

%
% A few miscellaneous things specific to this document
%
\newcommand{\crossop}[1]{\crossprod{#1}{}}

% R2 vector.
\newcommand{\VectorTwo}[2]{
\begin{bmatrix}
 {#1} \\
 {#2}
\end{bmatrix}
}

\newcommand{\VectorN}[1]{
\begin{bmatrix}
{#1}_1 \\
{#1}_2 \\
\vdots \\
{#1}_N \\
\end{bmatrix}
}

\newcommand{\DETuvij}[4]{
\begin{vmatrix}
 {#1}_{#3} & {#1}_{#4} \\
 {#2}_{#3} & {#2}_{#4}
\end{vmatrix}
}

\newcommand{\DETuvwijk}[6]{
\begin{vmatrix}
 {#1}_{#4} & {#1}_{#5} & {#1}_{#6} \\
 {#2}_{#4} & {#2}_{#5} & {#2}_{#6} \\
 {#3}_{#4} & {#3}_{#5} & {#3}_{#6}
\end{vmatrix}
}

\newcommand{\DETuvwxijkl}[8]{
\begin{vmatrix}
 {#1}_{#5} & {#1}_{#6} & {#1}_{#7} & {#1}_{#8} \\
 {#2}_{#5} & {#2}_{#6} & {#2}_{#7} & {#2}_{#8} \\
 {#3}_{#5} & {#3}_{#6} & {#3}_{#7} & {#3}_{#8} \\
 {#4}_{#5} & {#4}_{#6} & {#4}_{#7} & {#4}_{#8} \\
\end{vmatrix}
}

%\newcommand{\DETuvwxyijklm}[10]{
%\begin{vmatrix}
% {#1}_{#6} & {#1}_{#7} & {#1}_{#8} & {#1}_{#9} & {#1}_{#10} \\
% {#2}_{#6} & {#2}_{#7} & {#2}_{#8} & {#2}_{#9} & {#2}_{#10} \\
% {#3}_{#6} & {#3}_{#7} & {#3}_{#8} & {#3}_{#9} & {#3}_{#10} \\
% {#4}_{#6} & {#4}_{#7} & {#4}_{#8} & {#4}_{#9} & {#4}_{#10} \\
% {#5}_{#6} & {#5}_{#7} & {#5}_{#8} & {#5}_{#9} & {#5}_{#10}
%\end{vmatrix}
%}

% R3 vector.
\newcommand{\VectorThree}[3]{
\begin{bmatrix}
 {#1} \\
 {#2} \\
 {#3}
\end{bmatrix}
}


%%<misc>
%
\newcommand{\Abs}[1]{{\left\lvert{#1}\right\rvert}}
\newcommand{\spacegrad}[0]{\boldsymbol{\nabla}}
\newcommand{\grad}[0]{\nabla}
\newcommand{\LL}[0]{\mathcal{L}}

% == \partial_{#1} {#2}
\newcommand{\PD}[2]{\frac{\partial {#2}}{\partial {#1}}}
% inline variant
\newcommand{\PDi}[2]{{\partial {#2}}/{\partial {#1}}}

\newcommand{\PDD}[3]{\frac{\partial^2 {#3}}{\partial {#1}\partial {#2}}}
%\newcommand{\PDd}[2]{\frac{\partial^2 {#2}}{{\partial{#1}}^2}}
\newcommand{\PDsq}[2]{\frac{\partial^2 {#2}}{(\partial {#1})^2}}

\newcommand{\Partial}[2]{\frac{\partial {#1}}{\partial {#2}}}
\DeclareMathOperator{\RejName}{Rej}
\newcommand{\Rej}[2]{\RejName_{#1}\left( {#2} \right)}
\newcommand{\Rm}[1]{\mathbb{R}^{#1}}
\newcommand{\Cm}[1]{\mathbb{C}^{#1}}
\newcommand{\conj}[0]{{*}}

%</misc>

% <grade selection>
%
\newcommand{\gpgrade}[2] {{\left\langle{{#1}}\right\rangle}_{#2}}

\newcommand{\gpgradezero}[1] {\gpgrade{#1}{}}
%\newcommand{\gpscalargrade}[1] {{\left\langle{{#1}}\right\rangle}}
%\newcommand{\gpgradezero}[1] {\gpgrade{#1}{0}}

%\newcommand{\gpgradeone}[1] {{\left\langle{{#1}}\right\rangle}_{1}}
\newcommand{\gpgradeone}[1] {\gpgrade{#1}{1}}

\newcommand{\gpgradetwo}[1] {\gpgrade{#1}{2}}
\newcommand{\gpgradethree}[1] {\gpgrade{#1}{3}}
\newcommand{\gpgradefour}[1] {\gpgrade{#1}{4}}
%
% </grade selection>



\newcommand{\adot}[0]{{\dot{a}}}
\newcommand{\bdot}[0]{{\dot{b}}}
% taken for centered dot:
%\newcommand{\cdot}[0]{{\dot{c}}}
%\newcommand{\ddot}[0]{{\dot{d}}}
\newcommand{\edot}[0]{{\dot{e}}}
\newcommand{\fdot}[0]{{\dot{f}}}
\newcommand{\gdot}[0]{{\dot{g}}}
\newcommand{\hdot}[0]{{\dot{h}}}
\newcommand{\idot}[0]{{\dot{i}}}
\newcommand{\jdot}[0]{{\dot{j}}}
\newcommand{\kdot}[0]{{\dot{k}}}
\newcommand{\ldot}[0]{{\dot{l}}}
\newcommand{\mdot}[0]{{\dot{m}}}
\newcommand{\ndot}[0]{{\dot{n}}}
%\newcommand{\odot}[0]{{\dot{o}}}
\newcommand{\pdot}[0]{{\dot{p}}}
\newcommand{\qdot}[0]{{\dot{q}}}
\newcommand{\rdot}[0]{{\dot{r}}}
\newcommand{\sdot}[0]{{\dot{s}}}
\newcommand{\tdot}[0]{{\dot{t}}}
\newcommand{\udot}[0]{{\dot{u}}}
\newcommand{\vdot}[0]{{\dot{v}}}
\newcommand{\wdot}[0]{{\dot{w}}}
\newcommand{\xdot}[0]{{\dot{x}}}
\newcommand{\ydot}[0]{{\dot{y}}}
\newcommand{\zdot}[0]{{\dot{z}}}
\newcommand{\addot}[0]{{\ddot{a}}}
\newcommand{\bddot}[0]{{\ddot{b}}}
\newcommand{\cddot}[0]{{\ddot{c}}}
%\newcommand{\dddot}[0]{{\ddot{d}}}
\newcommand{\eddot}[0]{{\ddot{e}}}
\newcommand{\fddot}[0]{{\ddot{f}}}
\newcommand{\gddot}[0]{{\ddot{g}}}
\newcommand{\hddot}[0]{{\ddot{h}}}
\newcommand{\iddot}[0]{{\ddot{i}}}
\newcommand{\jddot}[0]{{\ddot{j}}}
\newcommand{\kddot}[0]{{\ddot{k}}}
\newcommand{\lddot}[0]{{\ddot{l}}}
\newcommand{\mddot}[0]{{\ddot{m}}}
\newcommand{\nddot}[0]{{\ddot{n}}}
\newcommand{\oddot}[0]{{\ddot{o}}}
\newcommand{\pddot}[0]{{\ddot{p}}}
\newcommand{\qddot}[0]{{\ddot{q}}}
\newcommand{\rddot}[0]{{\ddot{r}}}
\newcommand{\sddot}[0]{{\ddot{s}}}
\newcommand{\tddot}[0]{{\ddot{t}}}
\newcommand{\uddot}[0]{{\ddot{u}}}
\newcommand{\vddot}[0]{{\ddot{v}}}
\newcommand{\wddot}[0]{{\ddot{w}}}
\newcommand{\xddot}[0]{{\ddot{x}}}
\newcommand{\yddot}[0]{{\ddot{y}}}
\newcommand{\zddot}[0]{{\ddot{z}}}

%<bold and dot greek symbols>
%

\newcommand{\Deltadot}[0]{{\dot{\Delta}}}
\newcommand{\Gammadot}[0]{{\dot{\Gamma}}}
\newcommand{\Lambdadot}[0]{{\dot{\Lambda}}}
\newcommand{\Omegadot}[0]{{\dot{\Omega}}}
\newcommand{\Phidot}[0]{{\dot{\Phi}}}
\newcommand{\Pidot}[0]{{\dot{\Pi}}}
\newcommand{\Psidot}[0]{{\dot{\Psi}}}
\newcommand{\Sigmadot}[0]{{\dot{\Sigma}}}
\newcommand{\Thetadot}[0]{{\dot{\Theta}}}
\newcommand{\Upsilondot}[0]{{\dot{\Upsilon}}}
\newcommand{\Xidot}[0]{{\dot{\Xi}}}
\newcommand{\alphadot}[0]{{\dot{\alpha}}}
\newcommand{\betadot}[0]{{\dot{\beta}}}
\newcommand{\chidot}[0]{{\dot{\chi}}}
\newcommand{\deltadot}[0]{{\dot{\delta}}}
\newcommand{\epsilondot}[0]{{\dot{\epsilon}}}
\newcommand{\etadot}[0]{{\dot{\eta}}}
\newcommand{\gammadot}[0]{{\dot{\gamma}}}
\newcommand{\kappadot}[0]{{\dot{\kappa}}}
\newcommand{\lambdadot}[0]{{\dot{\lambda}}}
\newcommand{\mudot}[0]{{\dot{\mu}}}
\newcommand{\nudot}[0]{{\dot{\nu}}}
\newcommand{\omegadot}[0]{{\dot{\omega}}}
\newcommand{\phidot}[0]{{\dot{\phi}}}
\newcommand{\pidot}[0]{{\dot{\pi}}}
\newcommand{\psidot}[0]{{\dot{\psi}}}
\newcommand{\rhodot}[0]{{\dot{\rho}}}
\newcommand{\sigmadot}[0]{{\dot{\sigma}}}
\newcommand{\taudot}[0]{{\dot{\tau}}}
\newcommand{\thetadot}[0]{{\dot{\theta}}}
\newcommand{\upsilondot}[0]{{\dot{\upsilon}}}
\newcommand{\varepsilondot}[0]{{\dot{\varepsilon}}}
\newcommand{\varphidot}[0]{{\dot{\varphi}}}
\newcommand{\varpidot}[0]{{\dot{\varpi}}}
\newcommand{\varrhodot}[0]{{\dot{\varrho}}}
\newcommand{\varsigmadot}[0]{{\dot{\varsigma}}}
\newcommand{\varthetadot}[0]{{\dot{\vartheta}}}
\newcommand{\xidot}[0]{{\dot{\xi}}}
\newcommand{\zetadot}[0]{{\dot{\zeta}}}

\newcommand{\Deltaddot}[0]{{\ddot{\Delta}}}
\newcommand{\Gammaddot}[0]{{\ddot{\Gamma}}}
\newcommand{\Lambdaddot}[0]{{\ddot{\Lambda}}}
\newcommand{\Omegaddot}[0]{{\ddot{\Omega}}}
\newcommand{\Phiddot}[0]{{\ddot{\Phi}}}
\newcommand{\Piddot}[0]{{\ddot{\Pi}}}
\newcommand{\Psiddot}[0]{{\ddot{\Psi}}}
\newcommand{\Sigmaddot}[0]{{\ddot{\Sigma}}}
\newcommand{\Thetaddot}[0]{{\ddot{\Theta}}}
\newcommand{\Upsilonddot}[0]{{\ddot{\Upsilon}}}
\newcommand{\Xiddot}[0]{{\ddot{\Xi}}}
\newcommand{\alphaddot}[0]{{\ddot{\alpha}}}
\newcommand{\betaddot}[0]{{\ddot{\beta}}}
\newcommand{\chiddot}[0]{{\ddot{\chi}}}
\newcommand{\deltaddot}[0]{{\ddot{\delta}}}
\newcommand{\epsilonddot}[0]{{\ddot{\epsilon}}}
\newcommand{\etaddot}[0]{{\ddot{\eta}}}
\newcommand{\gammaddot}[0]{{\ddot{\gamma}}}
\newcommand{\kappaddot}[0]{{\ddot{\kappa}}}
\newcommand{\lambdaddot}[0]{{\ddot{\lambda}}}
\newcommand{\muddot}[0]{{\ddot{\mu}}}
\newcommand{\nuddot}[0]{{\ddot{\nu}}}
\newcommand{\omegaddot}[0]{{\ddot{\omega}}}
\newcommand{\phiddot}[0]{{\ddot{\phi}}}
\newcommand{\piddot}[0]{{\ddot{\pi}}}
\newcommand{\psiddot}[0]{{\ddot{\psi}}}
\newcommand{\rhoddot}[0]{{\ddot{\rho}}}
\newcommand{\sigmaddot}[0]{{\ddot{\sigma}}}
\newcommand{\tauddot}[0]{{\ddot{\tau}}}
\newcommand{\thetaddot}[0]{{\ddot{\theta}}}
\newcommand{\upsilonddot}[0]{{\ddot{\upsilon}}}
\newcommand{\varepsilonddot}[0]{{\ddot{\varepsilon}}}
\newcommand{\varphiddot}[0]{{\ddot{\varphi}}}
\newcommand{\varpiddot}[0]{{\ddot{\varpi}}}
\newcommand{\varrhoddot}[0]{{\ddot{\varrho}}}
\newcommand{\varsigmaddot}[0]{{\ddot{\varsigma}}}
\newcommand{\varthetaddot}[0]{{\ddot{\vartheta}}}
\newcommand{\xiddot}[0]{{\ddot{\xi}}}
\newcommand{\zetaddot}[0]{{\ddot{\zeta}}}

\newcommand{\BDelta}[0]{\boldsymbol{\Delta}}
\newcommand{\BGamma}[0]{\boldsymbol{\Gamma}}
\newcommand{\BLambda}[0]{\boldsymbol{\Lambda}}
\newcommand{\BOmega}[0]{\boldsymbol{\Omega}}
\newcommand{\BPhi}[0]{\boldsymbol{\Phi}}
\newcommand{\BPi}[0]{\boldsymbol{\Pi}}
\newcommand{\BPsi}[0]{\boldsymbol{\Psi}}
\newcommand{\BSigma}[0]{\boldsymbol{\Sigma}}
\newcommand{\BTheta}[0]{\boldsymbol{\Theta}}
\newcommand{\BUpsilon}[0]{\boldsymbol{\Upsilon}}
\newcommand{\BXi}[0]{\boldsymbol{\Xi}}
\newcommand{\Balpha}[0]{\boldsymbol{\alpha}}
\newcommand{\Bbeta}[0]{\boldsymbol{\beta}}
\newcommand{\Bchi}[0]{\boldsymbol{\chi}}
\newcommand{\Bdelta}[0]{\boldsymbol{\delta}}
\newcommand{\Bepsilon}[0]{\boldsymbol{\epsilon}}
\newcommand{\Beta}[0]{\boldsymbol{\eta}}
\newcommand{\Bgamma}[0]{\boldsymbol{\gamma}}
\newcommand{\Bkappa}[0]{\boldsymbol{\kappa}}
\newcommand{\Blambda}[0]{\boldsymbol{\lambda}}
\newcommand{\Bmu}[0]{\boldsymbol{\mu}}
\newcommand{\Bnu}[0]{\boldsymbol{\nu}}
%\newcommand{\Bomega}[0]{\boldsymbol{\omega}}
\newcommand{\Bphi}[0]{\boldsymbol{\phi}}
\newcommand{\Bpi}[0]{\boldsymbol{\pi}}
\newcommand{\Bpsi}[0]{\boldsymbol{\psi}}
\newcommand{\Brho}[0]{\boldsymbol{\rho}}
\newcommand{\Bsigma}[0]{\boldsymbol{\sigma}}
%\newcommand{\Btau}[0]{\boldsymbol{\tau}}
%\newcommand{\Btheta}[0]{\boldsymbol{\theta}}
\newcommand{\Bupsilon}[0]{\boldsymbol{\upsilon}}
\newcommand{\Bvarepsilon}[0]{\boldsymbol{\varepsilon}}
\newcommand{\Bvarphi}[0]{\boldsymbol{\varphi}}
\newcommand{\Bvarpi}[0]{\boldsymbol{\varpi}}
\newcommand{\Bvarrho}[0]{\boldsymbol{\varrho}}
\newcommand{\Bvarsigma}[0]{\boldsymbol{\varsigma}}
\newcommand{\Bvartheta}[0]{\boldsymbol{\vartheta}}
\newcommand{\Bxi}[0]{\boldsymbol{\xi}}
\newcommand{\Bzeta}[0]{\boldsymbol{\zeta}}
%
%</bold and dot greek symbols>
%<infrequent>
%
%\newcommand{\AreaOp}[1]{\AName_{#1}}
%\newcommand{\Babs}[0]{\abs{\BB}}
%\newcommand{\Bcap}[0]{\hat{\BB}}
%\newcommand{\BrPrimeRej}[0]{\rcap(\rcap \wedge \Br')}
%\newcommand{\CA}[0]{\mathcal{A}}
%\newcommand{\Cos}[1]{\cos{\left({#1}\right)}}
%\newcommand{\Det}[1] {\abs{#1}}
%\newcommand{\Dsq}[2] {\frac {\partial^2 {#1}} {\partial {#2}^2}}
%\newcommand{\Exp}[1]{\exp{\left({#1}\right)}}
%\newcommand{\Norm}[1]{\left\lVert{#1}\right\rVert}
%\newcommand{\Sin}[1]{\sin{\left({#1}\right)}}
%\newcommand{\T}[0]{\text{T}}
%\newcommand{\VolumeOp}[1]{\VName_{#1}}
%\newcommand{\agrad}[0]{\Ba \cdot \nabla}
%\newcommand{\alphacap}[0]{\hat{\boldsymbol{\alpha}}}
%\newcommand{\Fcap}[0]{\hat{\BF}}
%\newcommand{\bithree}[0]{{\Bi}_3}
%\newcommand{\bxa}[0]{\Bx\Ba}
%\newcommand{\coordvec}[2]{
%\newcommand{\costheta}[0]{\acap \cdot \xcap}
%\newcommand{\ddt}[1]{\ddot{#1}}
%\newcommand{\ddu}[1] {\frac {d{#1}} {du}}
%\newcommand{\dsqxj}[2] {\frac {\partial^2 {#1}} {\partial {x_{#2}}^2}}
%\newcommand{\dtheta}[1]{\frac{d {#1}}{d \theta}}
%\newcommand{\dt}[1]{\dot{#1}}
%\newcommand{\dt}[1]{\frac{d {#1}}{dt}}
%\newcommand{\dxj}[2] {\frac {\partial {#1}} {\partial {x_{#2}}}}
%\newcommand{\halfPhi}[0]{\frac{\phi}{2}}
%\newcommand{\half}[0]{\inv{2}}
%\newcommand{\inv}[1]{\frac{1}{#1}}
%\newcommand{\laplacian}[0]{\nabla^2}
%\newcommand{\matrixoftx}[3]{
%\newcommand{\nrrp}[0]{\norm{\rcap \wedge \Br'}}
%\newcommand{\oiint}{\bigcirc \hspace{-1.4em} \int \hspace{-.8em} \int}
%\newcommand{\transpose}[1]{{#1}^{\text{T}}}
%\newcommand{\transpose}[1]{{{#1}^{\TextTranspose}}}
%\newcommand{\transpose}[1]{{{#1}^{\text{T}}}}
%\newcommand{\barA}[0]{\bar{A}}
%\newcommand{\qbar}[0]{\bar{q}}
%\newcommand{\qdotbar}[0]{\dot{\bar{q}}}
%
%</infrequent>





%\usepackage[bookmarks=true]{hyperref}

%\usepackage{color,cite,graphicx}
   % use colour in the document, put your citations as [1-4]
   % rather than [1,2,3,4] (it looks nicer, and the extended LaTeX2e
   % graphics package. 
%\usepackage{latexsym,amssymb,epsf} % don't remember if these are
   % needed, but their inclusion can't do any damage


\chapter{Dot product linearity by construction. }
%\author{Peeter Joot \quad peeter.joot@gmail.com }
%\date{ March 13, 2009.  Last Revision: $Date: 2009/06/03 22:13:06 $ }

%\begin{document}

%\maketitle{}
%\tableofcontents
\section{Motivation. }

Reading of \cite{byron1992mca} it is observed that the dot product when defined via geometrical constructs such as

\begin{align*}
\Bx \cdot \By = \Abs{\Bx} \Abs{\By} \cos\theta
\end{align*}

is linear

\begin{align*}
\Bx \cdot (\By + \Bz) = \Bx \cdot \By + \Bx \cdot \Bz 
\end{align*}

Despite the fact that this is obvious when the dot product is given in algebraic form, this doesn't look obvious geometrically, so my 
immediate thought was ``how would you show this geometrically''.  Sure enough, in the next paragraph is the statement that the reader will
want to show this by construction.  Here's such a demonstration and construction.

\begin{figure}[htp]
\centering
\includegraphics[totalheight=0.4\textheight]{dot_lin}
\caption{Sum of two vectors and their angles with another.}\label{fig:dot_linearity}
\end{figure}

\section{Info from the figure. }
\subsection{Law of cosines. }

From 
figure \ref{fig:dot_linearity}, with $\Be_1 = \xcap$ we have

\begin{align*}
\By &= \Abs{\By} \cos\theta \Be_1 + \Abs{\By} \sin\theta \Be_2 \\
\Bz &= \Abs{\Bz} \cos\alpha \Be_1 + \Abs{\Bz} \sin\alpha \Be_2
\end{align*}

The vector sum $\By + \Bz$ is therefore

\begin{align*}
\By + \Bz &= (\Abs{\By} \cos\theta + \Abs{\Bz} \cos\alpha )\Be_1 + (\Abs{\By} \sin\theta + \Abs{\Bz} \sin\alpha ) \Be_2
\end{align*}

By using Pythagoras's law, a calculation of the squared length, produces the law of cosines

\begin{align*}
\Abs{\By + \Bz}^2 
&= (\Abs{\By} \cos\theta + \Abs{\Bz} \cos\alpha )^2 + (\Abs{\By} \sin\theta + \Abs{\Bz} \sin\alpha )^2 \\
&= \Abs{\By}^2 + \Abs{\Bz}^2 + 2 \Abs{\By} \Abs{\Bz} (\cos\theta \cos\alpha + \sin\theta \sin\alpha) \\
&= \Abs{\By}^2 + \Abs{\Bz}^2 + 2 \Abs{\By} \Abs{\Bz} \cos(\theta -\alpha) \\
\end{align*}

\subsection{Linearity by construction. }

Okay, that is a digression, ... back to the original problem.  We get the dot product linearity follows from direct calculation
of the cosine of the angle between $\xcap$ and $\By + \Bz$.  Again from the figure we have

\begin{align*}
\cos\beta &= \frac{\Abs{\By}\cos\theta + \Abs{\By}\cos\alpha}{\Abs{\By + \Bz}}
\end{align*}

or
\begin{align*}
{\Abs{\By + \Bz}} \cos\beta &= {\Abs{\By}\cos\theta + \Abs{\By}\cos\alpha} \\
\end{align*}

But $\xcap \cdot \By = \Abs{\By}\cos\theta $, and $\xcap \cdot \Bz = \Abs{\By}\cos\alpha $, so we have

\begin{align*}
{\Abs{\By + \Bz}} \cos\beta &= \xcap \cdot \By + \xcap \cdot \Bz
\end{align*}

Multiplying by $\Abs{\Bx}$ we have

\begin{align*}
\Abs{\Bx} {\Abs{\By + \Bz}} \cos\beta &= \Bx \cdot \By + \Bx \cdot \Bz
\end{align*}

The left hand side is $\Bx \cdot (\By + \Bz)$, which completes the desired demonstration.

%\bibliographystyle{plainnat}
%\bibliography{myrefs}

%\end{document}

%
% Copyright � 2012 Peeter Joot.  All Rights Reserved.
% Licenced as described in the file LICENSE under the root directory of this GIT repository.
%

% 
% 
%\documentclass{article}      % Specifies the document class

%\usepackage{amsmath}
\usepackage{mathpazo}

%
% shorthand for bold symbols, convenient for vectors and matrices
%
\newcommand{\Ba}[0]{\mathbf{a}}
\newcommand{\Bb}[0]{\mathbf{b}}
\newcommand{\Bc}[0]{\mathbf{c}}
\newcommand{\Bd}[0]{\mathbf{d}}
\newcommand{\Be}[0]{\mathbf{e}}
\newcommand{\Bf}[0]{\mathbf{f}}
\newcommand{\Bg}[0]{\mathbf{g}}
\newcommand{\Bh}[0]{\mathbf{h}}
\newcommand{\Bi}[0]{\mathbf{i}}
\newcommand{\Bj}[0]{\mathbf{j}}
\newcommand{\Bk}[0]{\mathbf{k}}
\newcommand{\Bl}[0]{\mathbf{l}}
\newcommand{\Bm}[0]{\mathbf{m}}
\newcommand{\Bn}[0]{\mathbf{n}}
\newcommand{\Bo}[0]{\mathbf{o}}
\newcommand{\Bp}[0]{\mathbf{p}}
\newcommand{\Bq}[0]{\mathbf{q}}
\newcommand{\Br}[0]{\mathbf{r}}
\newcommand{\Bs}[0]{\mathbf{s}}
\newcommand{\Bt}[0]{\mathbf{t}}
\newcommand{\Bu}[0]{\mathbf{u}}
\newcommand{\Bv}[0]{\mathbf{v}}
\newcommand{\Bw}[0]{\mathbf{w}}
\newcommand{\Bx}[0]{\mathbf{x}}
\newcommand{\By}[0]{\mathbf{y}}
\newcommand{\Bz}[0]{\mathbf{z}}
\newcommand{\BA}[0]{\mathbf{A}}
\newcommand{\BB}[0]{\mathbf{B}}
\newcommand{\BC}[0]{\mathbf{C}}
\newcommand{\BD}[0]{\mathbf{D}}
\newcommand{\BE}[0]{\mathbf{E}}
\newcommand{\BF}[0]{\mathbf{F}}
\newcommand{\BG}[0]{\mathbf{G}}
\newcommand{\BH}[0]{\mathbf{H}}
\newcommand{\BI}[0]{\mathbf{I}}
\newcommand{\BJ}[0]{\mathbf{J}}
\newcommand{\BK}[0]{\mathbf{K}}
\newcommand{\BL}[0]{\mathbf{L}}
\newcommand{\BM}[0]{\mathbf{M}}
\newcommand{\BN}[0]{\mathbf{N}}
\newcommand{\BO}[0]{\mathbf{O}}
\newcommand{\BP}[0]{\mathbf{P}}
\newcommand{\BQ}[0]{\mathbf{Q}}
\newcommand{\BR}[0]{\mathbf{R}}
\newcommand{\BS}[0]{\mathbf{S}}
\newcommand{\BT}[0]{\mathbf{T}}
\newcommand{\BU}[0]{\mathbf{U}}
\newcommand{\BV}[0]{\mathbf{V}}
\newcommand{\BW}[0]{\mathbf{W}}
\newcommand{\BX}[0]{\mathbf{X}}
\newcommand{\BY}[0]{\mathbf{Y}}
\newcommand{\BZ}[0]{\mathbf{Z}}

\newcommand{\Bzero}[0]{\mathbf{0}}
\newcommand{\Btheta}[0]{\boldsymbol{\theta}}
\newcommand{\Btau}[0]{\boldsymbol{\tau}}
\newcommand{\Bomega}[0]{\boldsymbol{\omega}}

%
% shorthand for unit vectors
%
\newcommand{\acap}[0]{\hat{\Ba}}
\newcommand{\bcap}[0]{\hat{\Bb}}
\newcommand{\ccap}[0]{\hat{\Bc}}
\newcommand{\dcap}[0]{\hat{\Bd}}
\newcommand{\ecap}[0]{\hat{\Be}}
\newcommand{\fcap}[0]{\hat{\Bf}}
\newcommand{\gcap}[0]{\hat{\Bg}}
\newcommand{\hcap}[0]{\hat{\Bh}}
\newcommand{\icap}[0]{\hat{\Bi}}
\newcommand{\jcap}[0]{\hat{\Bj}}
\newcommand{\kcap}[0]{\hat{\Bk}}
\newcommand{\lcap}[0]{\hat{\Bl}}
\newcommand{\mcap}[0]{\hat{\Bm}}
\newcommand{\ncap}[0]{\hat{\Bn}}
\newcommand{\ocap}[0]{\hat{\Bo}}
\newcommand{\pcap}[0]{\hat{\Bp}}
\newcommand{\qcap}[0]{\hat{\Bq}}
\newcommand{\rcap}[0]{\hat{\Br}}
\newcommand{\scap}[0]{\hat{\Bs}}
\newcommand{\tcap}[0]{\hat{\Bt}}
\newcommand{\ucap}[0]{\hat{\Bu}}
\newcommand{\vcap}[0]{\hat{\Bv}}
\newcommand{\wcap}[0]{\hat{\Bw}}
\newcommand{\xcap}[0]{\hat{\Bx}}
\newcommand{\ycap}[0]{\hat{\By}}
\newcommand{\zcap}[0]{\hat{\Bz}}
\newcommand{\thetacap}[0]{\hat{\Btheta}}

%
% to write R^n and C^n in a distinguishable fashion.  Perhaps change this
% to the double lined characters upon figuring out how to do so.
%
\newcommand{\C}[1]{$\mathbb{C}^{#1}$}
\newcommand{\R}[1]{$\mathbb{R}^{#1}$}

%
% various generally useful helpers
%

% derivative of #1 wrt. #2:
\newcommand{\D}[2] {\frac {d#2} {d#1}}

\newcommand{\inv}[1]{\frac{1}{#1}}
\newcommand{\cross}[0]{\times}

\newcommand{\abs}[1]{\lvert{#1}\rvert}
\newcommand{\norm}[1]{\lVert{#1}\rVert}
\newcommand{\innerprod}[2]{\langle{#1}, {#2}\rangle}
\newcommand{\dotprod}[2]{{#1} \cdot {#2}}
\newcommand{\bdotprod}[2]{\left({#1} \cdot {#2}\right)}
\newcommand{\crossprod}[2]{{#1} \cross {#2}}
\newcommand{\tripleprod}[3]{\dotprod{\left(\crossprod{#1}{#2}\right)}{#3}}

\DeclareMathOperator{\Proj}{Proj}
\DeclareMathOperator{\Span}{span}
\DeclareMathOperator{\Sgn}{sgn}
\DeclareMathOperator{\Area}{Area}
\DeclareMathOperator{\Volume}{Volume}

%
% A few miscellaneous things specific to this document
%
\newcommand{\crossop}[1]{\crossprod{#1}{}}

% R2 vector.
\newcommand{\VectorTwo}[2]{
\begin{bmatrix}
 {#1} \\
 {#2}
\end{bmatrix}
}

\newcommand{\VectorN}[1]{
\begin{bmatrix}
{#1}_1 \\
{#1}_2 \\
\vdots \\
{#1}_N \\
\end{bmatrix}
}

\newcommand{\DETuvij}[4]{
\begin{vmatrix}
 {#1}_{#3} & {#1}_{#4} \\
 {#2}_{#3} & {#2}_{#4}
\end{vmatrix}
}

\newcommand{\DETuvwijk}[6]{
\begin{vmatrix}
 {#1}_{#4} & {#1}_{#5} & {#1}_{#6} \\
 {#2}_{#4} & {#2}_{#5} & {#2}_{#6} \\
 {#3}_{#4} & {#3}_{#5} & {#3}_{#6}
\end{vmatrix}
}

\newcommand{\DETuvwxijkl}[8]{
\begin{vmatrix}
 {#1}_{#5} & {#1}_{#6} & {#1}_{#7} & {#1}_{#8} \\
 {#2}_{#5} & {#2}_{#6} & {#2}_{#7} & {#2}_{#8} \\
 {#3}_{#5} & {#3}_{#6} & {#3}_{#7} & {#3}_{#8} \\
 {#4}_{#5} & {#4}_{#6} & {#4}_{#7} & {#4}_{#8} \\
\end{vmatrix}
}

%\newcommand{\DETuvwxyijklm}[10]{
%\begin{vmatrix}
% {#1}_{#6} & {#1}_{#7} & {#1}_{#8} & {#1}_{#9} & {#1}_{#10} \\
% {#2}_{#6} & {#2}_{#7} & {#2}_{#8} & {#2}_{#9} & {#2}_{#10} \\
% {#3}_{#6} & {#3}_{#7} & {#3}_{#8} & {#3}_{#9} & {#3}_{#10} \\
% {#4}_{#6} & {#4}_{#7} & {#4}_{#8} & {#4}_{#9} & {#4}_{#10} \\
% {#5}_{#6} & {#5}_{#7} & {#5}_{#8} & {#5}_{#9} & {#5}_{#10}
%\end{vmatrix}
%}

% R3 vector.
\newcommand{\VectorThree}[3]{
\begin{bmatrix}
 {#1} \\
 {#2} \\
 {#3}
\end{bmatrix}
}



%\usepackage{color,cite,graphicx}
   % use colour in the document, put your citations as [1-4]
   % rather than [1,2,3,4] (it looks nicer, and the extended LaTeX2e
   % graphics package. 
%\usepackage{latexsym,amssymb,epsf} % do not remember if these are
   % needed, but their inclusion can not do any damage


%
% The real thing:
%

                             % The preamble begins here.
\chapter{Pythagoras law}
\label{chap:pythagoras}
%\author{Peeter Joot}         % Declares the author's name.
\date{ March 17, 2008.  pythagoras.tex }

%\begin{document}             % End of preamble and beginning of text.

%\maketitle{}

\section{Length}

To base vector multiplication on length, and examine all the consequences of having done so, it is first necessary to 
The geometrical definition of length for vectors generalizes Pythagoras theorem to higher dimensions.

In two dimensions this theorem can be proved with the aid of the following diagram

\imageFigure{../../figures/miscphysics/square_in_square}{Geometrical Proof of Pythagoras Theorem for Right Triangle}{fig:phthagoras}{0.3}

The area of the interior and exterior squares is \(c^2\), and \((a+b)^2\) respectively.  The interior area can also be calculated by subtracting the area of the triangles from the exterior area:

\begin{equation}\label{eqn:pythagoras:20}
(a+b)^2 - 4(ab/2) = a^2 + b^2 + 2ab - 2ab
\end{equation}

Thus proving Pythagoras theorem for the length of the diagonal in a right angle triangle

\begin{equation}\label{eqn:pythagoras:40}
c^2 = a^2 + b^2
\end{equation}

%for a geometrical proof like this one should perhaps show that the inscribed shape is a square ; all the lengths being equal is sufficient IMO.

The length of a vector in three dimensions can be found by repeated application of Pythagoras theorem, as in the following figure 

\imageFigure{../../figures/miscphysics/3d_vector_len}{Length of vector in three dimensions}{fig:3dveclen}{0.3}

The vector \(\Be = \Ba + \Bb + \Bc\), where each of the vectors \(\Ba\), \(\Bb\), and \(\Bc\) are mutually perpendicular can be found by first calculating

\begin{equation}\label{eqn:pythagoras:60}
d^2 = a^2 + b^2
\end{equation}

Then

\begin{equation}\label{eqn:pythagoras:80}
e^2 = d^2 + c^2 = a^2 + b^2 + c^2
\end{equation}

This process can be repeated for any number of higher dimensions.  Having calculated the length of a \(N-1\) dimensional vector

\begin{equation}\label{eqn:pythagoras:100}
{L(\Bv)}^2 = \sum_{i=1}^{N-1} {{l_i}^2}
\end{equation}

Once an additional component of length \(l_N\) is added to that vector in a direction mutually perpendicular to all previous components the new length of this vector becomes

\begin{equation}\label{eqn:pythagoras:120}
\sum_{i=1}^{N-1} {{l_i}^2} + l_N^2 = \sum_{i=0}^{N} {{l_i}^2}
\end{equation}

This is what we mean by the geometrical length of a vector.

%\subsection{Pythagoras Law, and the vector product}
%
%We have a rule for vector multiplication when two vectors are collinear.  Comparison to Pythagoras law will provide an additional rule for vector multiplication when the vectors are completely perpendicular.
%

%\end{document}               % End of document.

%
% Copyright � 2012 Peeter Joot.  All Rights Reserved.
% Licenced as described in the file LICENSE under the root directory of this GIT repository.
%

% 
% 
\chapter{Singular Value Decomposition}
\label{chap:mpInverseSvdRoughNotes}
\date{ May 15, 2008.  mpInverseSvdRoughNotes.tex }

\section{blah}

\subsection{Application of projection as left pseudoinverse (ie: linear fitting)}

%We have shown that the left pseudoinverse product with the matrix can
%be expressed as a projection matrix (sum of the projection matrices
%associated with a set of orthonormal vectors)

%\begin{equation}\label{eqn:mpInvRough:pseudoprojmatsum}
%A^{+} A =
%\sum_{k=1}^r {u_k}u_k^\T
%\end{equation}
%

%(note this is a different ``\(V\)'' than the \(V\) in \(A = U \Sigma V^\T\) since it only includes the first \(r\) columns).
%This allows us to write the matrix of \eqnref{eqn:mpInvRough:pseudoprojmatsum} as

%\begin{equation}
%A^{+} A = V V^\T
%\end{equation}

Equation (FIXME)
%\eqnref{eqn:mpInvRough:projectiongeneralmatrix} %% ??
provides us a way to find best solutions to general equations of the form:


\begin{equation}\label{eqn:mpInverseSvdRoughNotes:20}
A x = b
\end{equation}

Here \(A\) is the matrix of a linear transformation, \(A : \mathbb{R}^k \rightarrow \mathbb{R}^n\), for some \(k<n\).
By ``best solutions'' here, we give this the geometrical meaning, namely, the solution matching the projection of \(b\) onto the space.

If b is not completely in the column space \(C(A)\) of \(A\), this can have no solution.  However, writing

\begin{equation}\label{eqn:mpInverseSvdRoughNotes:40}
b = \Proj_A(b) + b_\perp
\end{equation}

as the components of \(b\) in \(C(A)\) and not in \(C(A)\) respectively we can at least solve the reduced equation for \(\hat{x}\):


\begin{equation}\label{eqn:mpInvRough:reducedinverseproblem}
A \hat{x} = \Proj_A(b)
\end{equation}


This will be possible even in circumstances that the original equation had no solution.  Specifically, the vector b when projected onto the plane can be expressed as some
linear combination of the columns of \(A\) (a basis for the subspace).

Substitution of our projection result into \eqnref{eqn:mpInvRough:reducedinverseproblem} yields:

\begin{equation}\label{eqn:mpInverseSvdRoughNotes:180}
\begin{aligned}
A \hat{x} 
&= \Proj_{A}\left(b\right) = A (A^\T A)^{-1} A^\T b
\end{aligned}
\end{equation}

The simplest case here is when \(A\) is of full column rank since one can pre-multiply this complete equation by \(A^\T\) without any possibility of nulling
\(A \hat{x}\).

\begin{equation}\label{eqn:mpInverseSvdRoughNotes:200}
\begin{aligned}
A^\T A \hat{x} 
&= A^\T A (A^\T A)^{-1} A^\T b \\
&= A^\T b \\
\end{aligned}
\end{equation}

Thus our best fit vector is

\begin{equation}
\hat{x} 
= (A^\T A)^{-1} A^\T b
\end{equation}

Another way to view this is for any vector \(x\) that is not in the null space \(N(A)\), then the matrix:

\begin{equation}
A^{+}= (A^\T A)^{-1} A^\T
\end{equation}

has the action of a left inverse for any full column rank matrix \(A\).  Thus when there is a solution to:

\begin{equation}
A x = b.
\end{equation}

It can be obtained by pre-multiplication using this "left" inverse.

\begin{equation}
A^{+} A x = x = A^{+} b
\end{equation}











































\section{SVD connection}


SVT decomposition is an factoring of \(A \in M^{m \times n}\) with orthonormal matrices \(U \in M^{m \times m}\)

and \(V \in M^{n \times n} \) producing the following form:

\begin{equation}\label{eqn:mpInverseSvdRoughNotes:60}
A = U \Sigma V^\T
\end{equation}

Sigma has the form:

\begin{equation}\label{eqn:mpInverseSvdRoughNotes:80}
\Sigma = 
\begin{bmatrix}
D_{r,r} & 0_{r,n-r} \\
0_{m-r,r} & 0_{m-r,n-r} \\
\end{bmatrix}
\end{equation}

where \(r = \rank(A)\), and \(D\) is a diagonal matrix with the root of the (positive) eigenvalues of \(A^\T A\).

This provides a generalized spectral decomposition and similarity that applies to both non-square matrices and matrices not otherwise diagonalizable
(ie: square matrix with similarity to a Jordon form matrix).  Given this decomposition we can write:

\begin{equation}\label{eqn:mpInverseSvdRoughNotes:100}
\Sigma = U^\T A V
\end{equation}

If one were to ask the question of what is the closest that one could get to inverting such a matrix.  It is pretty clear that the closest one could get to
identity will be with multiplication of a \(\Sigma^{+}\) of the following form:

\begin{equation}\label{eqn:mpInverseSvdRoughNotes:120}
\Sigma^{+} \Sigma
=
\begin{bmatrix}
(D_{r,r})^{-1} & 0_{r,m-r} \\
0_{n-r,r} & 0_{n-r,m-r} \\
\end{bmatrix}
\begin{bmatrix}
D_{r,r} & 0_{r,n-r} \\
0_{m-r,r} & 0_{m-r,n-r} \\
\end{bmatrix}
=
\begin{bmatrix}
I_{r,r} & 0_{r,n-r} \\
0_{n-r,r} & 0_{n-r,n-r} \\
\end{bmatrix}
\end{equation}

For a right pseudoinverse we have a similar result:

\begin{equation}\label{eqn:mpInverseSvdRoughNotes:140}
\Sigma
\Sigma^{+}
=
\begin{bmatrix}
D_{r,r} & 0_{r,n-r} \\
0_{m-r,r} & 0_{m-r,n-r} \\
\end{bmatrix}
\begin{bmatrix}
(D_{r,r})^{-1} & 0_{r,m-r} \\
0_{n-r,r} & 0_{n-r,m-r} \\
\end{bmatrix}
=
\begin{bmatrix}
I_{r,r} & 0_{r,m-r} \\
0_{m-r,r} & 0_{m-r,m-r} \\
\end{bmatrix}
\end{equation}

With either of these one can define a corresponding pseudoinverse (left or right) as:

\begin{equation}
A^{+} = V \Sigma^{+} U^\T
\end{equation}

This is a logical definition, but how close is it to the projective
left inverse we calculated above in the case where \(A\) is not of full column 
rank?

Multiplication gives: 

\begin{equation}\label{eqn:mpInverseSvdRoughNotes:220}
\begin{aligned}
A^{+} A 
&= V \Sigma^{+} U^\T U \Sigma V^\T \\
&= V \Sigma^{+} \Sigma V^\T \\
&=
\begin{bmatrix}
v_1 & v_2 & \cdots & v_r & v_{r+1} & \cdots & v_n \\
\end{bmatrix}
\begin{bmatrix}
(D_{r,r})^{-1} & 0_{r,m-r} \\
0_{n-r,r} & 0_{n-r,m-r} \\
\end{bmatrix}
\begin{bmatrix}
D_{r,r} & 0_{r,n-r} \\
0_{m-r,r} & 0_{m-r,n-r} \\
\end{bmatrix}
\begin{bmatrix}
v_1^\T \\ v_2^\T \\ \vdots \\ v_r^\T \\ {v_{r+1}}^\T \\ \vdots \\ v_n^\T \\
\end{bmatrix}
\end{aligned}
\end{equation}
%Embeded in that is the same "as-close-to" identity as calculated above.

Writing \(D_{r,r} = [\delta_{ij}\sigma_i]_{ij}\), we have:

\begin{equation}\label{eqn:mpInvRough:VIrVt}
V \Sigma^{+} \Sigma V^\T 
=
\begin{bmatrix}
\frac{v_1}{\sigma_1} & \frac{v_2}{\sigma_2} & \cdots & \frac{v_r}{\sigma_2} & 0 & \cdots & 0 
\end{bmatrix}
\begin{bmatrix}
v_1^\T \sigma_1 \\ v_2^\T \sigma_2 \\ \vdots \\ v_r^\T \sigma_r \\ 0 \\ \vdots \\ 0 \\
\end{bmatrix}
\end{equation}

Considering this as the product of block matrices we have a product here of the form

\begin{equation}\label{eqn:mpInverseSvdRoughNotes:240}
\begin{aligned}
\begin{bmatrix}
A_{n,r} & 0_{n,n-r}
\end{bmatrix}
\begin{bmatrix}
B_{r,n} \\ 0_{n-r,n}
\end{bmatrix} \\
&=
\begin{bmatrix}
A_{n,r} B_{r,n} + 0_{n,n-r} 0_{n-r,n}
\end{bmatrix} \\
&=
\begin{bmatrix}
A_{n,r} B_{r,n} + 0_{n,n}
\end{bmatrix} \\
&=
\begin{bmatrix}
A_{n,r} B_{r,n}
\end{bmatrix}
\end{aligned}
\end{equation}

Thus we can strip the block zero matrices from \eqnref{eqn:mpInvRough:VIrVt} and write

\begin{equation}\label{eqn:mpInvRough:pseudoinversetimesmatrix}
A^{+} A =
V \Sigma^{+} \Sigma V^\T 
=
\begin{bmatrix}
\frac{v_1}{\sigma_1} & \frac{v_2}{\sigma_2} & \cdots & \frac{v_r}{\sigma_2} 
\end{bmatrix}
\begin{bmatrix}
v_1^\T \sigma_1 \\ v_2^\T \sigma_2 \\ \vdots \\ v_r^\T \sigma_r 
\end{bmatrix}
\end{equation}

Eliminating the \(\sigma\) terms we have:

\begin{equation}\label{eqn:mpInvRough:pseudoinversetimesmatrixsum}
A^{+} A =
\begin{bmatrix}
\sum_{k=1}^r {v_k}v_k^\T
\end{bmatrix}
=
\begin{bmatrix}
v_1 & v_2 & \cdots & v_r 
\end{bmatrix}
\begin{bmatrix}
v_1^\T \\ v_2^\T \\ \vdots \\ v_r^\T 
\end{bmatrix}
\end{equation}

We previously calculated a left inverse using the projection matrix associated with a full column rank matrix.  For this product to have the properties of a
left acting inverse we also expect it to be a projection.
Let us digress
slightly before looking at whether equation
\eqnref{eqn:mpInvRough:pseudoinversetimesmatrixsum} satisfies this expectation.

\subsection{Correlating the SVD derived projection matrix back to \texorpdfstring{\(A\)}{A}}

We now have to show that this is also the projection matrix associated

with the columns of the 
original matrix that we have an SVD factorization for

\begin{equation}\label{eqn:mpInverseSvdRoughNotes:160}
A = U \Sigma V^\T
\end{equation}

Once we show this, then we have also demonstrated that the first \(r\) 
(orthonormal) column vectors in the matrix \(V\) of this decomposition
are a basis for the column space of \(A\) itself.  Note that we are
switching back to the original definition of \(V \in M^{n,n}\) here, and
not the \(V \in M^{n,r}\) of equation (FIXME)
%\eqnref{eqn:mpInvRough:projOrthonormal}. % ???


\part{calculus}
\documentclass{article}

\usepackage{amsmath}
\usepackage{mathpazo}

%
% shorthand for bold symbols, convenient for vectors and matrices
%
\newcommand{\Ba}[0]{\mathbf{a}}
\newcommand{\Bb}[0]{\mathbf{b}}
\newcommand{\Bc}[0]{\mathbf{c}}
\newcommand{\Bd}[0]{\mathbf{d}}
\newcommand{\Be}[0]{\mathbf{e}}
\newcommand{\Bf}[0]{\mathbf{f}}
\newcommand{\Bg}[0]{\mathbf{g}}
\newcommand{\Bh}[0]{\mathbf{h}}
\newcommand{\Bi}[0]{\mathbf{i}}
\newcommand{\Bj}[0]{\mathbf{j}}
\newcommand{\Bk}[0]{\mathbf{k}}
\newcommand{\Bl}[0]{\mathbf{l}}
\newcommand{\Bm}[0]{\mathbf{m}}
\newcommand{\Bn}[0]{\mathbf{n}}
\newcommand{\Bo}[0]{\mathbf{o}}
\newcommand{\Bp}[0]{\mathbf{p}}
\newcommand{\Bq}[0]{\mathbf{q}}
\newcommand{\Br}[0]{\mathbf{r}}
\newcommand{\Bs}[0]{\mathbf{s}}
\newcommand{\Bt}[0]{\mathbf{t}}
\newcommand{\Bu}[0]{\mathbf{u}}
\newcommand{\Bv}[0]{\mathbf{v}}
\newcommand{\Bw}[0]{\mathbf{w}}
\newcommand{\Bx}[0]{\mathbf{x}}
\newcommand{\By}[0]{\mathbf{y}}
\newcommand{\Bz}[0]{\mathbf{z}}
\newcommand{\BA}[0]{\mathbf{A}}
\newcommand{\BB}[0]{\mathbf{B}}
\newcommand{\BC}[0]{\mathbf{C}}
\newcommand{\BD}[0]{\mathbf{D}}
\newcommand{\BE}[0]{\mathbf{E}}
\newcommand{\BF}[0]{\mathbf{F}}
\newcommand{\BG}[0]{\mathbf{G}}
\newcommand{\BH}[0]{\mathbf{H}}
\newcommand{\BI}[0]{\mathbf{I}}
\newcommand{\BJ}[0]{\mathbf{J}}
\newcommand{\BK}[0]{\mathbf{K}}
\newcommand{\BL}[0]{\mathbf{L}}
\newcommand{\BM}[0]{\mathbf{M}}
\newcommand{\BN}[0]{\mathbf{N}}
\newcommand{\BO}[0]{\mathbf{O}}
\newcommand{\BP}[0]{\mathbf{P}}
\newcommand{\BQ}[0]{\mathbf{Q}}
\newcommand{\BR}[0]{\mathbf{R}}
\newcommand{\BS}[0]{\mathbf{S}}
\newcommand{\BT}[0]{\mathbf{T}}
\newcommand{\BU}[0]{\mathbf{U}}
\newcommand{\BV}[0]{\mathbf{V}}
\newcommand{\BW}[0]{\mathbf{W}}
\newcommand{\BX}[0]{\mathbf{X}}
\newcommand{\BY}[0]{\mathbf{Y}}
\newcommand{\BZ}[0]{\mathbf{Z}}

\newcommand{\Bzero}[0]{\mathbf{0}}
\newcommand{\Btheta}[0]{\boldsymbol{\theta}}
\newcommand{\Btau}[0]{\boldsymbol{\tau}}
\newcommand{\Bomega}[0]{\boldsymbol{\omega}}

%
% shorthand for unit vectors
%
\newcommand{\acap}[0]{\hat{\Ba}}
\newcommand{\bcap}[0]{\hat{\Bb}}
\newcommand{\ccap}[0]{\hat{\Bc}}
\newcommand{\dcap}[0]{\hat{\Bd}}
\newcommand{\ecap}[0]{\hat{\Be}}
\newcommand{\fcap}[0]{\hat{\Bf}}
\newcommand{\gcap}[0]{\hat{\Bg}}
\newcommand{\hcap}[0]{\hat{\Bh}}
\newcommand{\icap}[0]{\hat{\Bi}}
\newcommand{\jcap}[0]{\hat{\Bj}}
\newcommand{\kcap}[0]{\hat{\Bk}}
\newcommand{\lcap}[0]{\hat{\Bl}}
\newcommand{\mcap}[0]{\hat{\Bm}}
\newcommand{\ncap}[0]{\hat{\Bn}}
\newcommand{\ocap}[0]{\hat{\Bo}}
\newcommand{\pcap}[0]{\hat{\Bp}}
\newcommand{\qcap}[0]{\hat{\Bq}}
\newcommand{\rcap}[0]{\hat{\Br}}
\newcommand{\scap}[0]{\hat{\Bs}}
\newcommand{\tcap}[0]{\hat{\Bt}}
\newcommand{\ucap}[0]{\hat{\Bu}}
\newcommand{\vcap}[0]{\hat{\Bv}}
\newcommand{\wcap}[0]{\hat{\Bw}}
\newcommand{\xcap}[0]{\hat{\Bx}}
\newcommand{\ycap}[0]{\hat{\By}}
\newcommand{\zcap}[0]{\hat{\Bz}}
\newcommand{\thetacap}[0]{\hat{\Btheta}}

%
% to write R^n and C^n in a distinguishable fashion.  Perhaps change this
% to the double lined characters upon figuring out how to do so.
%
\newcommand{\C}[1]{$\mathbb{C}^{#1}$}
\newcommand{\R}[1]{$\mathbb{R}^{#1}$}

%
% various generally useful helpers
%

% derivative of #1 wrt. #2:
\newcommand{\D}[2] {\frac {d#2} {d#1}}

\newcommand{\inv}[1]{\frac{1}{#1}}
\newcommand{\cross}[0]{\times}

\newcommand{\abs}[1]{\lvert{#1}\rvert}
\newcommand{\norm}[1]{\lVert{#1}\rVert}
\newcommand{\innerprod}[2]{\langle{#1}, {#2}\rangle}
\newcommand{\dotprod}[2]{{#1} \cdot {#2}}
\newcommand{\bdotprod}[2]{\left({#1} \cdot {#2}\right)}
\newcommand{\crossprod}[2]{{#1} \cross {#2}}
\newcommand{\tripleprod}[3]{\dotprod{\left(\crossprod{#1}{#2}\right)}{#3}}

\DeclareMathOperator{\Proj}{Proj}
\DeclareMathOperator{\Span}{span}
\DeclareMathOperator{\Sgn}{sgn}
\DeclareMathOperator{\Area}{Area}
\DeclareMathOperator{\Volume}{Volume}

%
% A few miscellaneous things specific to this document
%
\newcommand{\crossop}[1]{\crossprod{#1}{}}

% R2 vector.
\newcommand{\VectorTwo}[2]{
\begin{bmatrix}
 {#1} \\
 {#2}
\end{bmatrix}
}

\newcommand{\VectorN}[1]{
\begin{bmatrix}
{#1}_1 \\
{#1}_2 \\
\vdots \\
{#1}_N \\
\end{bmatrix}
}

\newcommand{\DETuvij}[4]{
\begin{vmatrix}
 {#1}_{#3} & {#1}_{#4} \\
 {#2}_{#3} & {#2}_{#4}
\end{vmatrix}
}

\newcommand{\DETuvwijk}[6]{
\begin{vmatrix}
 {#1}_{#4} & {#1}_{#5} & {#1}_{#6} \\
 {#2}_{#4} & {#2}_{#5} & {#2}_{#6} \\
 {#3}_{#4} & {#3}_{#5} & {#3}_{#6}
\end{vmatrix}
}

\newcommand{\DETuvwxijkl}[8]{
\begin{vmatrix}
 {#1}_{#5} & {#1}_{#6} & {#1}_{#7} & {#1}_{#8} \\
 {#2}_{#5} & {#2}_{#6} & {#2}_{#7} & {#2}_{#8} \\
 {#3}_{#5} & {#3}_{#6} & {#3}_{#7} & {#3}_{#8} \\
 {#4}_{#5} & {#4}_{#6} & {#4}_{#7} & {#4}_{#8} \\
\end{vmatrix}
}

%\newcommand{\DETuvwxyijklm}[10]{
%\begin{vmatrix}
% {#1}_{#6} & {#1}_{#7} & {#1}_{#8} & {#1}_{#9} & {#1}_{#10} \\
% {#2}_{#6} & {#2}_{#7} & {#2}_{#8} & {#2}_{#9} & {#2}_{#10} \\
% {#3}_{#6} & {#3}_{#7} & {#3}_{#8} & {#3}_{#9} & {#3}_{#10} \\
% {#4}_{#6} & {#4}_{#7} & {#4}_{#8} & {#4}_{#9} & {#4}_{#10} \\
% {#5}_{#6} & {#5}_{#7} & {#5}_{#8} & {#5}_{#9} & {#5}_{#10}
%\end{vmatrix}
%}

% R3 vector.
\newcommand{\VectorThree}[3]{
\begin{bmatrix}
 {#1} \\
 {#2} \\
 {#3}
\end{bmatrix}
}


%<misc>
%
\newcommand{\Abs}[1]{{\left\lvert{#1}\right\rvert}}
\newcommand{\spacegrad}[0]{\boldsymbol{\nabla}}
\newcommand{\grad}[0]{\nabla}
\newcommand{\LL}[0]{\mathcal{L}}

% == \partial_{#1} {#2}
\newcommand{\PD}[2]{\frac{\partial {#2}}{\partial {#1}}}
% inline variant
\newcommand{\PDi}[2]{{\partial {#2}}/{\partial {#1}}}

\newcommand{\PDD}[3]{\frac{\partial^2 {#3}}{\partial {#1}\partial {#2}}}
%\newcommand{\PDd}[2]{\frac{\partial^2 {#2}}{{\partial{#1}}^2}}
\newcommand{\PDsq}[2]{\frac{\partial^2 {#2}}{(\partial {#1})^2}}

\newcommand{\Partial}[2]{\frac{\partial {#1}}{\partial {#2}}}
\DeclareMathOperator{\RejName}{Rej}
\newcommand{\Rej}[2]{\RejName_{#1}\left( {#2} \right)}
\newcommand{\Rm}[1]{\mathbb{R}^{#1}}
\newcommand{\Cm}[1]{\mathbb{C}^{#1}}
\newcommand{\conj}[0]{{*}}

%</misc>

% <grade selection>
%
\newcommand{\gpgrade}[2] {{\left\langle{{#1}}\right\rangle}_{#2}}

\newcommand{\gpgradezero}[1] {\gpgrade{#1}{}}
%\newcommand{\gpscalargrade}[1] {{\left\langle{{#1}}\right\rangle}}
%\newcommand{\gpgradezero}[1] {\gpgrade{#1}{0}}

%\newcommand{\gpgradeone}[1] {{\left\langle{{#1}}\right\rangle}_{1}}
\newcommand{\gpgradeone}[1] {\gpgrade{#1}{1}}

\newcommand{\gpgradetwo}[1] {\gpgrade{#1}{2}}
\newcommand{\gpgradethree}[1] {\gpgrade{#1}{3}}
\newcommand{\gpgradefour}[1] {\gpgrade{#1}{4}}
%
% </grade selection>



\newcommand{\adot}[0]{{\dot{a}}}
\newcommand{\bdot}[0]{{\dot{b}}}
% taken for centered dot:
%\newcommand{\cdot}[0]{{\dot{c}}}
%\newcommand{\ddot}[0]{{\dot{d}}}
\newcommand{\edot}[0]{{\dot{e}}}
\newcommand{\fdot}[0]{{\dot{f}}}
\newcommand{\gdot}[0]{{\dot{g}}}
\newcommand{\hdot}[0]{{\dot{h}}}
\newcommand{\idot}[0]{{\dot{i}}}
\newcommand{\jdot}[0]{{\dot{j}}}
\newcommand{\kdot}[0]{{\dot{k}}}
\newcommand{\ldot}[0]{{\dot{l}}}
\newcommand{\mdot}[0]{{\dot{m}}}
\newcommand{\ndot}[0]{{\dot{n}}}
%\newcommand{\odot}[0]{{\dot{o}}}
\newcommand{\pdot}[0]{{\dot{p}}}
\newcommand{\qdot}[0]{{\dot{q}}}
\newcommand{\rdot}[0]{{\dot{r}}}
\newcommand{\sdot}[0]{{\dot{s}}}
\newcommand{\tdot}[0]{{\dot{t}}}
\newcommand{\udot}[0]{{\dot{u}}}
\newcommand{\vdot}[0]{{\dot{v}}}
\newcommand{\wdot}[0]{{\dot{w}}}
\newcommand{\xdot}[0]{{\dot{x}}}
\newcommand{\ydot}[0]{{\dot{y}}}
\newcommand{\zdot}[0]{{\dot{z}}}
\newcommand{\addot}[0]{{\ddot{a}}}
\newcommand{\bddot}[0]{{\ddot{b}}}
\newcommand{\cddot}[0]{{\ddot{c}}}
%\newcommand{\dddot}[0]{{\ddot{d}}}
\newcommand{\eddot}[0]{{\ddot{e}}}
\newcommand{\fddot}[0]{{\ddot{f}}}
\newcommand{\gddot}[0]{{\ddot{g}}}
\newcommand{\hddot}[0]{{\ddot{h}}}
\newcommand{\iddot}[0]{{\ddot{i}}}
\newcommand{\jddot}[0]{{\ddot{j}}}
\newcommand{\kddot}[0]{{\ddot{k}}}
\newcommand{\lddot}[0]{{\ddot{l}}}
\newcommand{\mddot}[0]{{\ddot{m}}}
\newcommand{\nddot}[0]{{\ddot{n}}}
\newcommand{\oddot}[0]{{\ddot{o}}}
\newcommand{\pddot}[0]{{\ddot{p}}}
\newcommand{\qddot}[0]{{\ddot{q}}}
\newcommand{\rddot}[0]{{\ddot{r}}}
\newcommand{\sddot}[0]{{\ddot{s}}}
\newcommand{\tddot}[0]{{\ddot{t}}}
\newcommand{\uddot}[0]{{\ddot{u}}}
\newcommand{\vddot}[0]{{\ddot{v}}}
\newcommand{\wddot}[0]{{\ddot{w}}}
\newcommand{\xddot}[0]{{\ddot{x}}}
\newcommand{\yddot}[0]{{\ddot{y}}}
\newcommand{\zddot}[0]{{\ddot{z}}}

%<bold and dot greek symbols>
%

\newcommand{\Deltadot}[0]{{\dot{\Delta}}}
\newcommand{\Gammadot}[0]{{\dot{\Gamma}}}
\newcommand{\Lambdadot}[0]{{\dot{\Lambda}}}
\newcommand{\Omegadot}[0]{{\dot{\Omega}}}
\newcommand{\Phidot}[0]{{\dot{\Phi}}}
\newcommand{\Pidot}[0]{{\dot{\Pi}}}
\newcommand{\Psidot}[0]{{\dot{\Psi}}}
\newcommand{\Sigmadot}[0]{{\dot{\Sigma}}}
\newcommand{\Thetadot}[0]{{\dot{\Theta}}}
\newcommand{\Upsilondot}[0]{{\dot{\Upsilon}}}
\newcommand{\Xidot}[0]{{\dot{\Xi}}}
\newcommand{\alphadot}[0]{{\dot{\alpha}}}
\newcommand{\betadot}[0]{{\dot{\beta}}}
\newcommand{\chidot}[0]{{\dot{\chi}}}
\newcommand{\deltadot}[0]{{\dot{\delta}}}
\newcommand{\epsilondot}[0]{{\dot{\epsilon}}}
\newcommand{\etadot}[0]{{\dot{\eta}}}
\newcommand{\gammadot}[0]{{\dot{\gamma}}}
\newcommand{\kappadot}[0]{{\dot{\kappa}}}
\newcommand{\lambdadot}[0]{{\dot{\lambda}}}
\newcommand{\mudot}[0]{{\dot{\mu}}}
\newcommand{\nudot}[0]{{\dot{\nu}}}
\newcommand{\omegadot}[0]{{\dot{\omega}}}
\newcommand{\phidot}[0]{{\dot{\phi}}}
\newcommand{\pidot}[0]{{\dot{\pi}}}
\newcommand{\psidot}[0]{{\dot{\psi}}}
\newcommand{\rhodot}[0]{{\dot{\rho}}}
\newcommand{\sigmadot}[0]{{\dot{\sigma}}}
\newcommand{\taudot}[0]{{\dot{\tau}}}
\newcommand{\thetadot}[0]{{\dot{\theta}}}
\newcommand{\upsilondot}[0]{{\dot{\upsilon}}}
\newcommand{\varepsilondot}[0]{{\dot{\varepsilon}}}
\newcommand{\varphidot}[0]{{\dot{\varphi}}}
\newcommand{\varpidot}[0]{{\dot{\varpi}}}
\newcommand{\varrhodot}[0]{{\dot{\varrho}}}
\newcommand{\varsigmadot}[0]{{\dot{\varsigma}}}
\newcommand{\varthetadot}[0]{{\dot{\vartheta}}}
\newcommand{\xidot}[0]{{\dot{\xi}}}
\newcommand{\zetadot}[0]{{\dot{\zeta}}}

\newcommand{\Deltaddot}[0]{{\ddot{\Delta}}}
\newcommand{\Gammaddot}[0]{{\ddot{\Gamma}}}
\newcommand{\Lambdaddot}[0]{{\ddot{\Lambda}}}
\newcommand{\Omegaddot}[0]{{\ddot{\Omega}}}
\newcommand{\Phiddot}[0]{{\ddot{\Phi}}}
\newcommand{\Piddot}[0]{{\ddot{\Pi}}}
\newcommand{\Psiddot}[0]{{\ddot{\Psi}}}
\newcommand{\Sigmaddot}[0]{{\ddot{\Sigma}}}
\newcommand{\Thetaddot}[0]{{\ddot{\Theta}}}
\newcommand{\Upsilonddot}[0]{{\ddot{\Upsilon}}}
\newcommand{\Xiddot}[0]{{\ddot{\Xi}}}
\newcommand{\alphaddot}[0]{{\ddot{\alpha}}}
\newcommand{\betaddot}[0]{{\ddot{\beta}}}
\newcommand{\chiddot}[0]{{\ddot{\chi}}}
\newcommand{\deltaddot}[0]{{\ddot{\delta}}}
\newcommand{\epsilonddot}[0]{{\ddot{\epsilon}}}
\newcommand{\etaddot}[0]{{\ddot{\eta}}}
\newcommand{\gammaddot}[0]{{\ddot{\gamma}}}
\newcommand{\kappaddot}[0]{{\ddot{\kappa}}}
\newcommand{\lambdaddot}[0]{{\ddot{\lambda}}}
\newcommand{\muddot}[0]{{\ddot{\mu}}}
\newcommand{\nuddot}[0]{{\ddot{\nu}}}
\newcommand{\omegaddot}[0]{{\ddot{\omega}}}
\newcommand{\phiddot}[0]{{\ddot{\phi}}}
\newcommand{\piddot}[0]{{\ddot{\pi}}}
\newcommand{\psiddot}[0]{{\ddot{\psi}}}
\newcommand{\rhoddot}[0]{{\ddot{\rho}}}
\newcommand{\sigmaddot}[0]{{\ddot{\sigma}}}
\newcommand{\tauddot}[0]{{\ddot{\tau}}}
\newcommand{\thetaddot}[0]{{\ddot{\theta}}}
\newcommand{\upsilonddot}[0]{{\ddot{\upsilon}}}
\newcommand{\varepsilonddot}[0]{{\ddot{\varepsilon}}}
\newcommand{\varphiddot}[0]{{\ddot{\varphi}}}
\newcommand{\varpiddot}[0]{{\ddot{\varpi}}}
\newcommand{\varrhoddot}[0]{{\ddot{\varrho}}}
\newcommand{\varsigmaddot}[0]{{\ddot{\varsigma}}}
\newcommand{\varthetaddot}[0]{{\ddot{\vartheta}}}
\newcommand{\xiddot}[0]{{\ddot{\xi}}}
\newcommand{\zetaddot}[0]{{\ddot{\zeta}}}

\newcommand{\BDelta}[0]{\boldsymbol{\Delta}}
\newcommand{\BGamma}[0]{\boldsymbol{\Gamma}}
\newcommand{\BLambda}[0]{\boldsymbol{\Lambda}}
\newcommand{\BOmega}[0]{\boldsymbol{\Omega}}
\newcommand{\BPhi}[0]{\boldsymbol{\Phi}}
\newcommand{\BPi}[0]{\boldsymbol{\Pi}}
\newcommand{\BPsi}[0]{\boldsymbol{\Psi}}
\newcommand{\BSigma}[0]{\boldsymbol{\Sigma}}
\newcommand{\BTheta}[0]{\boldsymbol{\Theta}}
\newcommand{\BUpsilon}[0]{\boldsymbol{\Upsilon}}
\newcommand{\BXi}[0]{\boldsymbol{\Xi}}
\newcommand{\Balpha}[0]{\boldsymbol{\alpha}}
\newcommand{\Bbeta}[0]{\boldsymbol{\beta}}
\newcommand{\Bchi}[0]{\boldsymbol{\chi}}
\newcommand{\Bdelta}[0]{\boldsymbol{\delta}}
\newcommand{\Bepsilon}[0]{\boldsymbol{\epsilon}}
\newcommand{\Beta}[0]{\boldsymbol{\eta}}
\newcommand{\Bgamma}[0]{\boldsymbol{\gamma}}
\newcommand{\Bkappa}[0]{\boldsymbol{\kappa}}
\newcommand{\Blambda}[0]{\boldsymbol{\lambda}}
\newcommand{\Bmu}[0]{\boldsymbol{\mu}}
\newcommand{\Bnu}[0]{\boldsymbol{\nu}}
%\newcommand{\Bomega}[0]{\boldsymbol{\omega}}
\newcommand{\Bphi}[0]{\boldsymbol{\phi}}
\newcommand{\Bpi}[0]{\boldsymbol{\pi}}
\newcommand{\Bpsi}[0]{\boldsymbol{\psi}}
\newcommand{\Brho}[0]{\boldsymbol{\rho}}
\newcommand{\Bsigma}[0]{\boldsymbol{\sigma}}
%\newcommand{\Btau}[0]{\boldsymbol{\tau}}
%\newcommand{\Btheta}[0]{\boldsymbol{\theta}}
\newcommand{\Bupsilon}[0]{\boldsymbol{\upsilon}}
\newcommand{\Bvarepsilon}[0]{\boldsymbol{\varepsilon}}
\newcommand{\Bvarphi}[0]{\boldsymbol{\varphi}}
\newcommand{\Bvarpi}[0]{\boldsymbol{\varpi}}
\newcommand{\Bvarrho}[0]{\boldsymbol{\varrho}}
\newcommand{\Bvarsigma}[0]{\boldsymbol{\varsigma}}
\newcommand{\Bvartheta}[0]{\boldsymbol{\vartheta}}
\newcommand{\Bxi}[0]{\boldsymbol{\xi}}
\newcommand{\Bzeta}[0]{\boldsymbol{\zeta}}
%
%</bold and dot greek symbols>
%<infrequent>
%
%\newcommand{\AreaOp}[1]{\AName_{#1}}
%\newcommand{\Babs}[0]{\abs{\BB}}
%\newcommand{\Bcap}[0]{\hat{\BB}}
%\newcommand{\BrPrimeRej}[0]{\rcap(\rcap \wedge \Br')}
%\newcommand{\CA}[0]{\mathcal{A}}
%\newcommand{\Cos}[1]{\cos{\left({#1}\right)}}
%\newcommand{\Det}[1] {\abs{#1}}
%\newcommand{\Dsq}[2] {\frac {\partial^2 {#1}} {\partial {#2}^2}}
%\newcommand{\Exp}[1]{\exp{\left({#1}\right)}}
%\newcommand{\Norm}[1]{\left\lVert{#1}\right\rVert}
%\newcommand{\Sin}[1]{\sin{\left({#1}\right)}}
%\newcommand{\T}[0]{\text{T}}
%\newcommand{\VolumeOp}[1]{\VName_{#1}}
%\newcommand{\agrad}[0]{\Ba \cdot \nabla}
%\newcommand{\alphacap}[0]{\hat{\boldsymbol{\alpha}}}
%\newcommand{\Fcap}[0]{\hat{\BF}}
%\newcommand{\bithree}[0]{{\Bi}_3}
%\newcommand{\bxa}[0]{\Bx\Ba}
%\newcommand{\coordvec}[2]{
%\newcommand{\costheta}[0]{\acap \cdot \xcap}
%\newcommand{\ddt}[1]{\ddot{#1}}
%\newcommand{\ddu}[1] {\frac {d{#1}} {du}}
%\newcommand{\dsqxj}[2] {\frac {\partial^2 {#1}} {\partial {x_{#2}}^2}}
%\newcommand{\dtheta}[1]{\frac{d {#1}}{d \theta}}
%\newcommand{\dt}[1]{\dot{#1}}
%\newcommand{\dt}[1]{\frac{d {#1}}{dt}}
%\newcommand{\dxj}[2] {\frac {\partial {#1}} {\partial {x_{#2}}}}
%\newcommand{\halfPhi}[0]{\frac{\phi}{2}}
%\newcommand{\half}[0]{\inv{2}}
%\newcommand{\inv}[1]{\frac{1}{#1}}
%\newcommand{\laplacian}[0]{\nabla^2}
%\newcommand{\matrixoftx}[3]{
%\newcommand{\nrrp}[0]{\norm{\rcap \wedge \Br'}}
%\newcommand{\oiint}{\bigcirc \hspace{-1.4em} \int \hspace{-.8em} \int}
%\newcommand{\transpose}[1]{{#1}^{\text{T}}}
%\newcommand{\transpose}[1]{{{#1}^{\TextTranspose}}}
%\newcommand{\transpose}[1]{{{#1}^{\text{T}}}}
%\newcommand{\barA}[0]{\bar{A}}
%\newcommand{\qbar}[0]{\bar{q}}
%\newcommand{\qdotbar}[0]{\dot{\bar{q}}}
%
%</infrequent>





%\usepackage{listings}
\usepackage{txfonts} % for ointctr... (also appears to make "prettier" \int and \sum's)
\usepackage[bookmarks=true]{hyperref}

\usepackage{color,cite,graphicx}
   % use colour in the document, put your citations as [1-4]
   % rather than [1,2,3,4] (it looks nicer, and the extended LaTeX2e
   % graphics package. 
\usepackage{latexsym,amssymb,epsf} % don't remember if these are
   % needed, but their inclusion can't do any damage


\title{ Worked calculus of variations problems from Byron and Fuller. }
\author{Peeter Joot \quad peeter.joot@gmail.com }
\date{ March 21, 2009.  Last Revision: $Date: 2009/03/22 05:32:43 $ }

\begin{document}

\maketitle{}
\tableofcontents

\section{ Worked calculus of variations problems. }

Select problems from chapter II of \cite{byron1992mca}.

\subsection{ Problem 1.  Shortest line between points in polar coordinates. }

Problem.  Variational calculus exersize to find shortest distance between two points using polar coordinates.
 
The line element is:
 
\begin{align*}
ds^2 = r^2 d\theta^2 + {r'}^2
\end{align*}
 
So, the integral to minimize is
 
\begin{align*}
I = \int \sqrt{ r^2 + {r'}^2 } d\theta
\end{align*}
 
Application of the Euler-Lagrange equations yields
 
\begin{align*}
0 
&= \left( \PD{r}{} - \frac{d}{d\theta} \PD{r'}{} \right) \sqrt{ r^2 + {r'}^2 }  \\
&= \frac{r}{\sqrt{{r'}^2 + r^2}} - \frac{d}{d\theta} \left( \frac{{r'}}{\sqrt{{r'}^2 + r^2}}\right) \\
\end{align*}
 
Dividing through by $r$ and writing $v = u' = {r'}/r$ this is
 
\begin{align*}
\frac{1}{\sqrt{v^2 + 1}}
&= \frac{d}{d\theta} \left( \frac{v}{\sqrt{v^2 + 1}}\right) \\
&= \frac{v'}{\sqrt{v^2 + 1}} -\frac{v^2 v'}{(\sqrt{v^2 + 1})^3} \\
\end{align*}
 
\begin{align*}
1 
&= v' \left( 1 -\frac{v^2 }{v^2 + 1} \right) \\
&= \frac{v'}{v^2 + 1} \\
\end{align*}

This is now separable, and can be integrated directly

\begin{align*}
\theta - \theta_0
&= \int \frac{dv}{v^2 + 1} \\
&= \arctan(v) \\
\end{align*}
 
\begin{align*}
\tan(\theta - \theta_0)
&= \frac{{r'}}{r} \\
&= \frac{d \ln(r) }{d\theta}
\end{align*}

That solves the first of the second order differential equations resulting from the Euler-Lagrange equations, and the 
last becomes
\begin{align*}
\ln(r) &= \int \tan(\theta - \theta_0) d\theta \\
&= -\ln(\cos(\theta - \theta_0)) + \ln(r_0)
\end{align*}

Finally a polar parametric equation is obtained
 
\begin{align*}
\frac{r}{r_0} \cos(\theta - \theta_0) = 1
\end{align*}
 
If all went right, this should be the equation for a straight line in polar form.

It doesn't look like one, but if the cosine is expanded 

\begin{align*}
\cos(\theta - \theta_0) 
&= \Re\left( e^{i\theta}e^{-i\theta_0} \right) \\
&= \Re\left( (\cos\theta + i\sin\theta)(\cos\theta_0 - i\sin\theta_0) \right) \\
&= \cos\theta\cos\theta_0 + \sin\theta\sin\theta_0 \\
\end{align*}

With $x = r\cos\theta$, and $y = r\sin\theta$ this gives

\begin{align*}
r_0 
&= r \left( \cos\theta\cos\theta_0 + \sin\theta\sin\theta_0 \right) \\
&= x \cos\theta_0 + y\sin\theta_0 \\
\end{align*}

So, sure enough, following the math gives an equation for a straight line in a recognizable form.

\subsection{ Problem 2. Shortest line, in 3D. }

Did this one in \cite{PJgoldch1}

\subsection{ Problem 3.  Spherical geodesics. }

First calculate the line element (this was given in the problem, but I feel like working it out).
The position vector with $i = \Be_1 \Be_2$, is given by

\begin{align*}
\Br = a ( \sin\theta \Be_1 e^{i\phi} + \Be_3 \cos\theta )
\end{align*}

So, the differential given constant radius $a$ is

\begin{align*}
d\Br 
&= 
a \thetadot ( \cos\theta \Be_1 e^{i\phi} - \Be_3 \sin\theta )
+ a \phidot ( \sin\theta \Be_1 \Be_1 \Be_2 e^{i\phi} ) \\
&= 
a \left( \thetadot \cos\theta \Be_1 + \phidot \sin\theta \Be_2 \right) e^{i\phi} - a \thetadot \Be_3 \sin\theta \\
\end{align*}

And the square is
\begin{align*}
d\Br^2
&= a^2 \left( \thetadot^2 \cos^2\theta + \phidot^2 \sin^2\theta + \thetadot^2 \sin^2\theta \right) \\
&= a^2 \left( \thetadot^2 + \phidot^2 \sin^2\theta \right) \\
\end{align*}

Here the derivatives are with respect to some implicit variable that parameterizes the differential displacement.
This can be taken to be $\theta$, which gives the distance along any two points on the sphere as

\begin{align*}
S &= a^2 \int d\theta \sqrt{1 + \left(\frac{d\phi}{d\theta}\right)^2 \sin^2\theta } \\
\end{align*}

Writing $f(\theta, \phi, \phidot) = \sqrt{1 + \phidot^2 \sin^2\theta}$, the Euler-Lagrange equations can be
applied

\begin{align*}
0 
&= \left( \PD{\phi}{} - \frac{d}{d\theta} \PD{\phidot}{} \right) f \\
&= 0 - \frac{d}{d\theta} \frac{(1/2)(2\phidot) \sin^2\theta}{\sqrt{1 + \phidot^2 \sin^2\theta}} \\
\end{align*}

Introducing an integration constant $\kappa$, this is

\begin{align*}
\phidot \sin^2\theta = \kappa \sqrt{1 + \phidot^2 \sin^2\theta}
\end{align*}

squaring
\begin{align*}
\phidot^2 \sin^4\theta &= \kappa^2 \left(1 + \phidot^2 \sin^2\theta \right) \\
\phidot^2 \sin^2\theta \left( \sin^2 \theta - \kappa^2 \right) &= \kappa^2 \\
\end{align*}

\begin{align*}
\phi - \phi_0 &= \kappa \int \frac{d\theta}{\sin\theta \sqrt{ \sin^2 \theta - \kappa^2 }} \\
\end{align*}

This doesn't look particularily nice to integrate.  Instead, let's try writing the arc length integral as

\begin{align*}
\frac{S}{a^2} &= \int d\phi \sqrt{{\frac{d\theta}{d\phi}}^2 + \sin^2\theta } \\
\end{align*}

\begin{align*}
0 
&= \left( \PD{\theta}{} - \frac{d}{d\phi} \PD{\thetadot}{} \right) \sqrt{\thetadot^2 + \sin^2\theta } \\
&= \frac{\sin\theta \cos\theta}{ \sqrt{\thetadot^2 + \sin^2\theta } } -
\frac{d}{d\phi} \frac{\thetadot} { \sqrt{\thetadot^2 + \sin^2\theta } } \\
\end{align*}

\bibliographystyle{plainnat}
\bibliography{myrefs}

\end{document}

%
% Copyright � 2012 Peeter Joot.  All Rights Reserved.
% Licenced as described in the file LICENSE under the root directory of this GIT repository.
%

% 
% 
%\documentclass{article}

%\usepackage{amsmath}
\usepackage{mathpazo}

%
% shorthand for bold symbols, convenient for vectors and matrices
%
\newcommand{\Ba}[0]{\mathbf{a}}
\newcommand{\Bb}[0]{\mathbf{b}}
\newcommand{\Bc}[0]{\mathbf{c}}
\newcommand{\Bd}[0]{\mathbf{d}}
\newcommand{\Be}[0]{\mathbf{e}}
\newcommand{\Bf}[0]{\mathbf{f}}
\newcommand{\Bg}[0]{\mathbf{g}}
\newcommand{\Bh}[0]{\mathbf{h}}
\newcommand{\Bi}[0]{\mathbf{i}}
\newcommand{\Bj}[0]{\mathbf{j}}
\newcommand{\Bk}[0]{\mathbf{k}}
\newcommand{\Bl}[0]{\mathbf{l}}
\newcommand{\Bm}[0]{\mathbf{m}}
\newcommand{\Bn}[0]{\mathbf{n}}
\newcommand{\Bo}[0]{\mathbf{o}}
\newcommand{\Bp}[0]{\mathbf{p}}
\newcommand{\Bq}[0]{\mathbf{q}}
\newcommand{\Br}[0]{\mathbf{r}}
\newcommand{\Bs}[0]{\mathbf{s}}
\newcommand{\Bt}[0]{\mathbf{t}}
\newcommand{\Bu}[0]{\mathbf{u}}
\newcommand{\Bv}[0]{\mathbf{v}}
\newcommand{\Bw}[0]{\mathbf{w}}
\newcommand{\Bx}[0]{\mathbf{x}}
\newcommand{\By}[0]{\mathbf{y}}
\newcommand{\Bz}[0]{\mathbf{z}}
\newcommand{\BA}[0]{\mathbf{A}}
\newcommand{\BB}[0]{\mathbf{B}}
\newcommand{\BC}[0]{\mathbf{C}}
\newcommand{\BD}[0]{\mathbf{D}}
\newcommand{\BE}[0]{\mathbf{E}}
\newcommand{\BF}[0]{\mathbf{F}}
\newcommand{\BG}[0]{\mathbf{G}}
\newcommand{\BH}[0]{\mathbf{H}}
\newcommand{\BI}[0]{\mathbf{I}}
\newcommand{\BJ}[0]{\mathbf{J}}
\newcommand{\BK}[0]{\mathbf{K}}
\newcommand{\BL}[0]{\mathbf{L}}
\newcommand{\BM}[0]{\mathbf{M}}
\newcommand{\BN}[0]{\mathbf{N}}
\newcommand{\BO}[0]{\mathbf{O}}
\newcommand{\BP}[0]{\mathbf{P}}
\newcommand{\BQ}[0]{\mathbf{Q}}
\newcommand{\BR}[0]{\mathbf{R}}
\newcommand{\BS}[0]{\mathbf{S}}
\newcommand{\BT}[0]{\mathbf{T}}
\newcommand{\BU}[0]{\mathbf{U}}
\newcommand{\BV}[0]{\mathbf{V}}
\newcommand{\BW}[0]{\mathbf{W}}
\newcommand{\BX}[0]{\mathbf{X}}
\newcommand{\BY}[0]{\mathbf{Y}}
\newcommand{\BZ}[0]{\mathbf{Z}}

\newcommand{\Bzero}[0]{\mathbf{0}}
\newcommand{\Btheta}[0]{\boldsymbol{\theta}}
\newcommand{\Btau}[0]{\boldsymbol{\tau}}
\newcommand{\Bomega}[0]{\boldsymbol{\omega}}

%
% shorthand for unit vectors
%
\newcommand{\acap}[0]{\hat{\Ba}}
\newcommand{\bcap}[0]{\hat{\Bb}}
\newcommand{\ccap}[0]{\hat{\Bc}}
\newcommand{\dcap}[0]{\hat{\Bd}}
\newcommand{\ecap}[0]{\hat{\Be}}
\newcommand{\fcap}[0]{\hat{\Bf}}
\newcommand{\gcap}[0]{\hat{\Bg}}
\newcommand{\hcap}[0]{\hat{\Bh}}
\newcommand{\icap}[0]{\hat{\Bi}}
\newcommand{\jcap}[0]{\hat{\Bj}}
\newcommand{\kcap}[0]{\hat{\Bk}}
\newcommand{\lcap}[0]{\hat{\Bl}}
\newcommand{\mcap}[0]{\hat{\Bm}}
\newcommand{\ncap}[0]{\hat{\Bn}}
\newcommand{\ocap}[0]{\hat{\Bo}}
\newcommand{\pcap}[0]{\hat{\Bp}}
\newcommand{\qcap}[0]{\hat{\Bq}}
\newcommand{\rcap}[0]{\hat{\Br}}
\newcommand{\scap}[0]{\hat{\Bs}}
\newcommand{\tcap}[0]{\hat{\Bt}}
\newcommand{\ucap}[0]{\hat{\Bu}}
\newcommand{\vcap}[0]{\hat{\Bv}}
\newcommand{\wcap}[0]{\hat{\Bw}}
\newcommand{\xcap}[0]{\hat{\Bx}}
\newcommand{\ycap}[0]{\hat{\By}}
\newcommand{\zcap}[0]{\hat{\Bz}}
\newcommand{\thetacap}[0]{\hat{\Btheta}}

%
% to write R^n and C^n in a distinguishable fashion.  Perhaps change this
% to the double lined characters upon figuring out how to do so.
%
\newcommand{\C}[1]{$\mathbb{C}^{#1}$}
\newcommand{\R}[1]{$\mathbb{R}^{#1}$}

%
% various generally useful helpers
%

% derivative of #1 wrt. #2:
\newcommand{\D}[2] {\frac {d#2} {d#1}}

\newcommand{\inv}[1]{\frac{1}{#1}}
\newcommand{\cross}[0]{\times}

\newcommand{\abs}[1]{\lvert{#1}\rvert}
\newcommand{\norm}[1]{\lVert{#1}\rVert}
\newcommand{\innerprod}[2]{\langle{#1}, {#2}\rangle}
\newcommand{\dotprod}[2]{{#1} \cdot {#2}}
\newcommand{\bdotprod}[2]{\left({#1} \cdot {#2}\right)}
\newcommand{\crossprod}[2]{{#1} \cross {#2}}
\newcommand{\tripleprod}[3]{\dotprod{\left(\crossprod{#1}{#2}\right)}{#3}}

\DeclareMathOperator{\Proj}{Proj}
\DeclareMathOperator{\Span}{span}
\DeclareMathOperator{\Sgn}{sgn}
\DeclareMathOperator{\Area}{Area}
\DeclareMathOperator{\Volume}{Volume}

%
% A few miscellaneous things specific to this document
%
\newcommand{\crossop}[1]{\crossprod{#1}{}}

% R2 vector.
\newcommand{\VectorTwo}[2]{
\begin{bmatrix}
 {#1} \\
 {#2}
\end{bmatrix}
}

\newcommand{\VectorN}[1]{
\begin{bmatrix}
{#1}_1 \\
{#1}_2 \\
\vdots \\
{#1}_N \\
\end{bmatrix}
}

\newcommand{\DETuvij}[4]{
\begin{vmatrix}
 {#1}_{#3} & {#1}_{#4} \\
 {#2}_{#3} & {#2}_{#4}
\end{vmatrix}
}

\newcommand{\DETuvwijk}[6]{
\begin{vmatrix}
 {#1}_{#4} & {#1}_{#5} & {#1}_{#6} \\
 {#2}_{#4} & {#2}_{#5} & {#2}_{#6} \\
 {#3}_{#4} & {#3}_{#5} & {#3}_{#6}
\end{vmatrix}
}

\newcommand{\DETuvwxijkl}[8]{
\begin{vmatrix}
 {#1}_{#5} & {#1}_{#6} & {#1}_{#7} & {#1}_{#8} \\
 {#2}_{#5} & {#2}_{#6} & {#2}_{#7} & {#2}_{#8} \\
 {#3}_{#5} & {#3}_{#6} & {#3}_{#7} & {#3}_{#8} \\
 {#4}_{#5} & {#4}_{#6} & {#4}_{#7} & {#4}_{#8} \\
\end{vmatrix}
}

%\newcommand{\DETuvwxyijklm}[10]{
%\begin{vmatrix}
% {#1}_{#6} & {#1}_{#7} & {#1}_{#8} & {#1}_{#9} & {#1}_{#10} \\
% {#2}_{#6} & {#2}_{#7} & {#2}_{#8} & {#2}_{#9} & {#2}_{#10} \\
% {#3}_{#6} & {#3}_{#7} & {#3}_{#8} & {#3}_{#9} & {#3}_{#10} \\
% {#4}_{#6} & {#4}_{#7} & {#4}_{#8} & {#4}_{#9} & {#4}_{#10} \\
% {#5}_{#6} & {#5}_{#7} & {#5}_{#8} & {#5}_{#9} & {#5}_{#10}
%\end{vmatrix}
%}

% R3 vector.
\newcommand{\VectorThree}[3]{
\begin{bmatrix}
 {#1} \\
 {#2} \\
 {#3}
\end{bmatrix}
}


%%<misc>
%
\newcommand{\Abs}[1]{{\left\lvert{#1}\right\rvert}}
\newcommand{\spacegrad}[0]{\boldsymbol{\nabla}}
\newcommand{\grad}[0]{\nabla}
\newcommand{\LL}[0]{\mathcal{L}}

% == \partial_{#1} {#2}
\newcommand{\PD}[2]{\frac{\partial {#2}}{\partial {#1}}}
% inline variant
\newcommand{\PDi}[2]{{\partial {#2}}/{\partial {#1}}}

\newcommand{\PDD}[3]{\frac{\partial^2 {#3}}{\partial {#1}\partial {#2}}}
%\newcommand{\PDd}[2]{\frac{\partial^2 {#2}}{{\partial{#1}}^2}}
\newcommand{\PDsq}[2]{\frac{\partial^2 {#2}}{(\partial {#1})^2}}

\newcommand{\Partial}[2]{\frac{\partial {#1}}{\partial {#2}}}
\DeclareMathOperator{\RejName}{Rej}
\newcommand{\Rej}[2]{\RejName_{#1}\left( {#2} \right)}
\newcommand{\Rm}[1]{\mathbb{R}^{#1}}
\newcommand{\Cm}[1]{\mathbb{C}^{#1}}
\newcommand{\conj}[0]{{*}}

%</misc>

% <grade selection>
%
\newcommand{\gpgrade}[2] {{\left\langle{{#1}}\right\rangle}_{#2}}

\newcommand{\gpgradezero}[1] {\gpgrade{#1}{}}
%\newcommand{\gpscalargrade}[1] {{\left\langle{{#1}}\right\rangle}}
%\newcommand{\gpgradezero}[1] {\gpgrade{#1}{0}}

%\newcommand{\gpgradeone}[1] {{\left\langle{{#1}}\right\rangle}_{1}}
\newcommand{\gpgradeone}[1] {\gpgrade{#1}{1}}

\newcommand{\gpgradetwo}[1] {\gpgrade{#1}{2}}
\newcommand{\gpgradethree}[1] {\gpgrade{#1}{3}}
\newcommand{\gpgradefour}[1] {\gpgrade{#1}{4}}
%
% </grade selection>



\newcommand{\adot}[0]{{\dot{a}}}
\newcommand{\bdot}[0]{{\dot{b}}}
% taken for centered dot:
%\newcommand{\cdot}[0]{{\dot{c}}}
%\newcommand{\ddot}[0]{{\dot{d}}}
\newcommand{\edot}[0]{{\dot{e}}}
\newcommand{\fdot}[0]{{\dot{f}}}
\newcommand{\gdot}[0]{{\dot{g}}}
\newcommand{\hdot}[0]{{\dot{h}}}
\newcommand{\idot}[0]{{\dot{i}}}
\newcommand{\jdot}[0]{{\dot{j}}}
\newcommand{\kdot}[0]{{\dot{k}}}
\newcommand{\ldot}[0]{{\dot{l}}}
\newcommand{\mdot}[0]{{\dot{m}}}
\newcommand{\ndot}[0]{{\dot{n}}}
%\newcommand{\odot}[0]{{\dot{o}}}
\newcommand{\pdot}[0]{{\dot{p}}}
\newcommand{\qdot}[0]{{\dot{q}}}
\newcommand{\rdot}[0]{{\dot{r}}}
\newcommand{\sdot}[0]{{\dot{s}}}
\newcommand{\tdot}[0]{{\dot{t}}}
\newcommand{\udot}[0]{{\dot{u}}}
\newcommand{\vdot}[0]{{\dot{v}}}
\newcommand{\wdot}[0]{{\dot{w}}}
\newcommand{\xdot}[0]{{\dot{x}}}
\newcommand{\ydot}[0]{{\dot{y}}}
\newcommand{\zdot}[0]{{\dot{z}}}
\newcommand{\addot}[0]{{\ddot{a}}}
\newcommand{\bddot}[0]{{\ddot{b}}}
\newcommand{\cddot}[0]{{\ddot{c}}}
%\newcommand{\dddot}[0]{{\ddot{d}}}
\newcommand{\eddot}[0]{{\ddot{e}}}
\newcommand{\fddot}[0]{{\ddot{f}}}
\newcommand{\gddot}[0]{{\ddot{g}}}
\newcommand{\hddot}[0]{{\ddot{h}}}
\newcommand{\iddot}[0]{{\ddot{i}}}
\newcommand{\jddot}[0]{{\ddot{j}}}
\newcommand{\kddot}[0]{{\ddot{k}}}
\newcommand{\lddot}[0]{{\ddot{l}}}
\newcommand{\mddot}[0]{{\ddot{m}}}
\newcommand{\nddot}[0]{{\ddot{n}}}
\newcommand{\oddot}[0]{{\ddot{o}}}
\newcommand{\pddot}[0]{{\ddot{p}}}
\newcommand{\qddot}[0]{{\ddot{q}}}
\newcommand{\rddot}[0]{{\ddot{r}}}
\newcommand{\sddot}[0]{{\ddot{s}}}
\newcommand{\tddot}[0]{{\ddot{t}}}
\newcommand{\uddot}[0]{{\ddot{u}}}
\newcommand{\vddot}[0]{{\ddot{v}}}
\newcommand{\wddot}[0]{{\ddot{w}}}
\newcommand{\xddot}[0]{{\ddot{x}}}
\newcommand{\yddot}[0]{{\ddot{y}}}
\newcommand{\zddot}[0]{{\ddot{z}}}

%<bold and dot greek symbols>
%

\newcommand{\Deltadot}[0]{{\dot{\Delta}}}
\newcommand{\Gammadot}[0]{{\dot{\Gamma}}}
\newcommand{\Lambdadot}[0]{{\dot{\Lambda}}}
\newcommand{\Omegadot}[0]{{\dot{\Omega}}}
\newcommand{\Phidot}[0]{{\dot{\Phi}}}
\newcommand{\Pidot}[0]{{\dot{\Pi}}}
\newcommand{\Psidot}[0]{{\dot{\Psi}}}
\newcommand{\Sigmadot}[0]{{\dot{\Sigma}}}
\newcommand{\Thetadot}[0]{{\dot{\Theta}}}
\newcommand{\Upsilondot}[0]{{\dot{\Upsilon}}}
\newcommand{\Xidot}[0]{{\dot{\Xi}}}
\newcommand{\alphadot}[0]{{\dot{\alpha}}}
\newcommand{\betadot}[0]{{\dot{\beta}}}
\newcommand{\chidot}[0]{{\dot{\chi}}}
\newcommand{\deltadot}[0]{{\dot{\delta}}}
\newcommand{\epsilondot}[0]{{\dot{\epsilon}}}
\newcommand{\etadot}[0]{{\dot{\eta}}}
\newcommand{\gammadot}[0]{{\dot{\gamma}}}
\newcommand{\kappadot}[0]{{\dot{\kappa}}}
\newcommand{\lambdadot}[0]{{\dot{\lambda}}}
\newcommand{\mudot}[0]{{\dot{\mu}}}
\newcommand{\nudot}[0]{{\dot{\nu}}}
\newcommand{\omegadot}[0]{{\dot{\omega}}}
\newcommand{\phidot}[0]{{\dot{\phi}}}
\newcommand{\pidot}[0]{{\dot{\pi}}}
\newcommand{\psidot}[0]{{\dot{\psi}}}
\newcommand{\rhodot}[0]{{\dot{\rho}}}
\newcommand{\sigmadot}[0]{{\dot{\sigma}}}
\newcommand{\taudot}[0]{{\dot{\tau}}}
\newcommand{\thetadot}[0]{{\dot{\theta}}}
\newcommand{\upsilondot}[0]{{\dot{\upsilon}}}
\newcommand{\varepsilondot}[0]{{\dot{\varepsilon}}}
\newcommand{\varphidot}[0]{{\dot{\varphi}}}
\newcommand{\varpidot}[0]{{\dot{\varpi}}}
\newcommand{\varrhodot}[0]{{\dot{\varrho}}}
\newcommand{\varsigmadot}[0]{{\dot{\varsigma}}}
\newcommand{\varthetadot}[0]{{\dot{\vartheta}}}
\newcommand{\xidot}[0]{{\dot{\xi}}}
\newcommand{\zetadot}[0]{{\dot{\zeta}}}

\newcommand{\Deltaddot}[0]{{\ddot{\Delta}}}
\newcommand{\Gammaddot}[0]{{\ddot{\Gamma}}}
\newcommand{\Lambdaddot}[0]{{\ddot{\Lambda}}}
\newcommand{\Omegaddot}[0]{{\ddot{\Omega}}}
\newcommand{\Phiddot}[0]{{\ddot{\Phi}}}
\newcommand{\Piddot}[0]{{\ddot{\Pi}}}
\newcommand{\Psiddot}[0]{{\ddot{\Psi}}}
\newcommand{\Sigmaddot}[0]{{\ddot{\Sigma}}}
\newcommand{\Thetaddot}[0]{{\ddot{\Theta}}}
\newcommand{\Upsilonddot}[0]{{\ddot{\Upsilon}}}
\newcommand{\Xiddot}[0]{{\ddot{\Xi}}}
\newcommand{\alphaddot}[0]{{\ddot{\alpha}}}
\newcommand{\betaddot}[0]{{\ddot{\beta}}}
\newcommand{\chiddot}[0]{{\ddot{\chi}}}
\newcommand{\deltaddot}[0]{{\ddot{\delta}}}
\newcommand{\epsilonddot}[0]{{\ddot{\epsilon}}}
\newcommand{\etaddot}[0]{{\ddot{\eta}}}
\newcommand{\gammaddot}[0]{{\ddot{\gamma}}}
\newcommand{\kappaddot}[0]{{\ddot{\kappa}}}
\newcommand{\lambdaddot}[0]{{\ddot{\lambda}}}
\newcommand{\muddot}[0]{{\ddot{\mu}}}
\newcommand{\nuddot}[0]{{\ddot{\nu}}}
\newcommand{\omegaddot}[0]{{\ddot{\omega}}}
\newcommand{\phiddot}[0]{{\ddot{\phi}}}
\newcommand{\piddot}[0]{{\ddot{\pi}}}
\newcommand{\psiddot}[0]{{\ddot{\psi}}}
\newcommand{\rhoddot}[0]{{\ddot{\rho}}}
\newcommand{\sigmaddot}[0]{{\ddot{\sigma}}}
\newcommand{\tauddot}[0]{{\ddot{\tau}}}
\newcommand{\thetaddot}[0]{{\ddot{\theta}}}
\newcommand{\upsilonddot}[0]{{\ddot{\upsilon}}}
\newcommand{\varepsilonddot}[0]{{\ddot{\varepsilon}}}
\newcommand{\varphiddot}[0]{{\ddot{\varphi}}}
\newcommand{\varpiddot}[0]{{\ddot{\varpi}}}
\newcommand{\varrhoddot}[0]{{\ddot{\varrho}}}
\newcommand{\varsigmaddot}[0]{{\ddot{\varsigma}}}
\newcommand{\varthetaddot}[0]{{\ddot{\vartheta}}}
\newcommand{\xiddot}[0]{{\ddot{\xi}}}
\newcommand{\zetaddot}[0]{{\ddot{\zeta}}}

\newcommand{\BDelta}[0]{\boldsymbol{\Delta}}
\newcommand{\BGamma}[0]{\boldsymbol{\Gamma}}
\newcommand{\BLambda}[0]{\boldsymbol{\Lambda}}
\newcommand{\BOmega}[0]{\boldsymbol{\Omega}}
\newcommand{\BPhi}[0]{\boldsymbol{\Phi}}
\newcommand{\BPi}[0]{\boldsymbol{\Pi}}
\newcommand{\BPsi}[0]{\boldsymbol{\Psi}}
\newcommand{\BSigma}[0]{\boldsymbol{\Sigma}}
\newcommand{\BTheta}[0]{\boldsymbol{\Theta}}
\newcommand{\BUpsilon}[0]{\boldsymbol{\Upsilon}}
\newcommand{\BXi}[0]{\boldsymbol{\Xi}}
\newcommand{\Balpha}[0]{\boldsymbol{\alpha}}
\newcommand{\Bbeta}[0]{\boldsymbol{\beta}}
\newcommand{\Bchi}[0]{\boldsymbol{\chi}}
\newcommand{\Bdelta}[0]{\boldsymbol{\delta}}
\newcommand{\Bepsilon}[0]{\boldsymbol{\epsilon}}
\newcommand{\Beta}[0]{\boldsymbol{\eta}}
\newcommand{\Bgamma}[0]{\boldsymbol{\gamma}}
\newcommand{\Bkappa}[0]{\boldsymbol{\kappa}}
\newcommand{\Blambda}[0]{\boldsymbol{\lambda}}
\newcommand{\Bmu}[0]{\boldsymbol{\mu}}
\newcommand{\Bnu}[0]{\boldsymbol{\nu}}
%\newcommand{\Bomega}[0]{\boldsymbol{\omega}}
\newcommand{\Bphi}[0]{\boldsymbol{\phi}}
\newcommand{\Bpi}[0]{\boldsymbol{\pi}}
\newcommand{\Bpsi}[0]{\boldsymbol{\psi}}
\newcommand{\Brho}[0]{\boldsymbol{\rho}}
\newcommand{\Bsigma}[0]{\boldsymbol{\sigma}}
%\newcommand{\Btau}[0]{\boldsymbol{\tau}}
%\newcommand{\Btheta}[0]{\boldsymbol{\theta}}
\newcommand{\Bupsilon}[0]{\boldsymbol{\upsilon}}
\newcommand{\Bvarepsilon}[0]{\boldsymbol{\varepsilon}}
\newcommand{\Bvarphi}[0]{\boldsymbol{\varphi}}
\newcommand{\Bvarpi}[0]{\boldsymbol{\varpi}}
\newcommand{\Bvarrho}[0]{\boldsymbol{\varrho}}
\newcommand{\Bvarsigma}[0]{\boldsymbol{\varsigma}}
\newcommand{\Bvartheta}[0]{\boldsymbol{\vartheta}}
\newcommand{\Bxi}[0]{\boldsymbol{\xi}}
\newcommand{\Bzeta}[0]{\boldsymbol{\zeta}}
%
%</bold and dot greek symbols>
%<infrequent>
%
%\newcommand{\AreaOp}[1]{\AName_{#1}}
%\newcommand{\Babs}[0]{\abs{\BB}}
%\newcommand{\Bcap}[0]{\hat{\BB}}
%\newcommand{\BrPrimeRej}[0]{\rcap(\rcap \wedge \Br')}
%\newcommand{\CA}[0]{\mathcal{A}}
%\newcommand{\Cos}[1]{\cos{\left({#1}\right)}}
%\newcommand{\Det}[1] {\abs{#1}}
%\newcommand{\Dsq}[2] {\frac {\partial^2 {#1}} {\partial {#2}^2}}
%\newcommand{\Exp}[1]{\exp{\left({#1}\right)}}
%\newcommand{\Norm}[1]{\left\lVert{#1}\right\rVert}
%\newcommand{\Sin}[1]{\sin{\left({#1}\right)}}
%\newcommand{\T}[0]{\text{T}}
%\newcommand{\VolumeOp}[1]{\VName_{#1}}
%\newcommand{\agrad}[0]{\Ba \cdot \nabla}
%\newcommand{\alphacap}[0]{\hat{\boldsymbol{\alpha}}}
%\newcommand{\Fcap}[0]{\hat{\BF}}
%\newcommand{\bithree}[0]{{\Bi}_3}
%\newcommand{\bxa}[0]{\Bx\Ba}
%\newcommand{\coordvec}[2]{
%\newcommand{\costheta}[0]{\acap \cdot \xcap}
%\newcommand{\ddt}[1]{\ddot{#1}}
%\newcommand{\ddu}[1] {\frac {d{#1}} {du}}
%\newcommand{\dsqxj}[2] {\frac {\partial^2 {#1}} {\partial {x_{#2}}^2}}
%\newcommand{\dtheta}[1]{\frac{d {#1}}{d \theta}}
%\newcommand{\dt}[1]{\dot{#1}}
%\newcommand{\dt}[1]{\frac{d {#1}}{dt}}
%\newcommand{\dxj}[2] {\frac {\partial {#1}} {\partial {x_{#2}}}}
%\newcommand{\halfPhi}[0]{\frac{\phi}{2}}
%\newcommand{\half}[0]{\inv{2}}
%\newcommand{\inv}[1]{\frac{1}{#1}}
%\newcommand{\laplacian}[0]{\nabla^2}
%\newcommand{\matrixoftx}[3]{
%\newcommand{\nrrp}[0]{\norm{\rcap \wedge \Br'}}
%\newcommand{\oiint}{\bigcirc \hspace{-1.4em} \int \hspace{-.8em} \int}
%\newcommand{\transpose}[1]{{#1}^{\text{T}}}
%\newcommand{\transpose}[1]{{{#1}^{\TextTranspose}}}
%\newcommand{\transpose}[1]{{{#1}^{\text{T}}}}
%\newcommand{\barA}[0]{\bar{A}}
%\newcommand{\qbar}[0]{\bar{q}}
%\newcommand{\qdotbar}[0]{\dot{\bar{q}}}
%
%</infrequent>




%\usepackage{listings}
%\usepackage{txfonts} % for ointctr... (also appears to make "prettier" \int and \sum's)
%\usepackage[bookmarks=true]{hyperref}

%\usepackage{color,cite,graphicx}
   % use colour in the document, put your citations as [1-4]
   % rather than [1,2,3,4] (it looks nicer, and the extended LaTeX2e
   % graphics package. 
%\usepackage{latexsym,amssymb,epsf} % do not remember if these are
   % needed, but their inclusion can not do any damage


\chapter{Simple minded variation of one dimensional wave equation Lagrangian}
\label{chap:wavevariation}
%\author{Peeter Joot \quad peeterjoot@protonmail.com }
\date{ April 27, 2009.  wavevariation.tex }

%\begin{document}

%\maketitle{}
%\tableofcontents
%\section{}

From the action

\begin{equation}\label{eqn:wavevariation:20}
\begin{aligned}
S_\eta = \int dx dt \inv{2} \left( \left(\PD{x}{\eta}\right)^2 - \inv{v^2} \left(\PD{t}{\eta}\right)^2 \right)
\end{aligned}
\end{equation}

We can vary the field \(\eta = \psi + \epsilon\), where \(\psi\) is the field variable to be determined, and \(\epsilon\) is the field variable allowed to vary within the volume of integration.

Forming the difference, subtracts off the ``constant'' parts of the action due only to the optimal field variable \(\psi\)

\begin{equation}\label{eqn:wavevariation:40}
\begin{aligned}
S_{\psi + \epsilon} - S_\psi
&=
\int dx dt \inv{2} \left( \left(\PD{x}{(\psi + \epsilon)}\right)^2 - \inv{v^2} \left(\PD{t}{(\psi + \epsilon)}\right)^2 \right)
-\int dx dt \inv{2} \left( \left(\PD{x}{\psi}\right)^2 - \inv{v^2} \left(\PD{t}{\psi}\right)^2 \right) \\
&=
\int dx dt \left( \PD{x}{\psi}\PD{x}{\epsilon} - \PD{t}{\psi}\PD{t}{\epsilon} \right)
+ \int dx dt \inv{2} \left( \left(\PD{x}{\epsilon }\right)^2 - \inv{v^2} \left(\PD{t}{\epsilon }\right)^2 \right) \\
\end{aligned}
\end{equation}

Now integrating by parts

\begin{equation}\label{eqn:wavevariation:60}
\begin{aligned}
S_{\psi + \epsilon} - S_\psi
&=
\int dx dt \left( \inv{v^2} \PDSq{t}{\psi} -\PDSq{x}{\psi} \right) {\epsilon} 
+ \int dx dt \inv{2} \left( -\PDSq{x}{\epsilon } + \inv{v^2} \PDSq{t}{\epsilon } \right) \epsilon \\
\end{aligned}
\end{equation}

Roughly speaking the terms that are quadratic in \(\epsilon\) can be discarded as small, and if the remaining differential
is to be zero for all \(\epsilon\), we are left with the wave equation

\begin{equation}\label{eqn:wavevariation:80}
\begin{aligned}
\inv{v^2} \PDSq{t}{\psi} -\PDSq{x}{\psi} = 0 
\end{aligned}
\end{equation}

with solutions 
%(like your exponential) 
of the form 

\begin{equation}\label{eqn:wavevariation:100}
\begin{aligned}
\psi = f(x \pm vt)
\end{aligned}
\end{equation}

%\bibliographystyle{plainnat}
%\bibliography{myrefs}

%\end{document}

% 
% 
% 
% Copyright � 2012 Peeter Joot
% All Rights Reserved
% 
% This file may be reproduced and distributed in whole or in part, without fee, subject to the following conditions:
% 
% o The copyright notice above and this permission notice must be preserved complete on all complete or partial copies.
% 
% o Any translation or derived work must be approved by the author in writing before distribution.
% 
% o If you distribute this work in part, instructions for obtaining the complete version of this file must be included, and a means for obtaining a complete version provided.
% 
% 
% Exceptions to these rules may be granted for academic purposes: Write to the author and ask.
% 
% 
% 
%\documentclass{article}

%\usepackage{amsmath}
\usepackage{mathpazo}

%
% shorthand for bold symbols, convenient for vectors and matrices
%
\newcommand{\Ba}[0]{\mathbf{a}}
\newcommand{\Bb}[0]{\mathbf{b}}
\newcommand{\Bc}[0]{\mathbf{c}}
\newcommand{\Bd}[0]{\mathbf{d}}
\newcommand{\Be}[0]{\mathbf{e}}
\newcommand{\Bf}[0]{\mathbf{f}}
\newcommand{\Bg}[0]{\mathbf{g}}
\newcommand{\Bh}[0]{\mathbf{h}}
\newcommand{\Bi}[0]{\mathbf{i}}
\newcommand{\Bj}[0]{\mathbf{j}}
\newcommand{\Bk}[0]{\mathbf{k}}
\newcommand{\Bl}[0]{\mathbf{l}}
\newcommand{\Bm}[0]{\mathbf{m}}
\newcommand{\Bn}[0]{\mathbf{n}}
\newcommand{\Bo}[0]{\mathbf{o}}
\newcommand{\Bp}[0]{\mathbf{p}}
\newcommand{\Bq}[0]{\mathbf{q}}
\newcommand{\Br}[0]{\mathbf{r}}
\newcommand{\Bs}[0]{\mathbf{s}}
\newcommand{\Bt}[0]{\mathbf{t}}
\newcommand{\Bu}[0]{\mathbf{u}}
\newcommand{\Bv}[0]{\mathbf{v}}
\newcommand{\Bw}[0]{\mathbf{w}}
\newcommand{\Bx}[0]{\mathbf{x}}
\newcommand{\By}[0]{\mathbf{y}}
\newcommand{\Bz}[0]{\mathbf{z}}
\newcommand{\BA}[0]{\mathbf{A}}
\newcommand{\BB}[0]{\mathbf{B}}
\newcommand{\BC}[0]{\mathbf{C}}
\newcommand{\BD}[0]{\mathbf{D}}
\newcommand{\BE}[0]{\mathbf{E}}
\newcommand{\BF}[0]{\mathbf{F}}
\newcommand{\BG}[0]{\mathbf{G}}
\newcommand{\BH}[0]{\mathbf{H}}
\newcommand{\BI}[0]{\mathbf{I}}
\newcommand{\BJ}[0]{\mathbf{J}}
\newcommand{\BK}[0]{\mathbf{K}}
\newcommand{\BL}[0]{\mathbf{L}}
\newcommand{\BM}[0]{\mathbf{M}}
\newcommand{\BN}[0]{\mathbf{N}}
\newcommand{\BO}[0]{\mathbf{O}}
\newcommand{\BP}[0]{\mathbf{P}}
\newcommand{\BQ}[0]{\mathbf{Q}}
\newcommand{\BR}[0]{\mathbf{R}}
\newcommand{\BS}[0]{\mathbf{S}}
\newcommand{\BT}[0]{\mathbf{T}}
\newcommand{\BU}[0]{\mathbf{U}}
\newcommand{\BV}[0]{\mathbf{V}}
\newcommand{\BW}[0]{\mathbf{W}}
\newcommand{\BX}[0]{\mathbf{X}}
\newcommand{\BY}[0]{\mathbf{Y}}
\newcommand{\BZ}[0]{\mathbf{Z}}

\newcommand{\Bzero}[0]{\mathbf{0}}
\newcommand{\Btheta}[0]{\boldsymbol{\theta}}
\newcommand{\Btau}[0]{\boldsymbol{\tau}}
\newcommand{\Bomega}[0]{\boldsymbol{\omega}}

%
% shorthand for unit vectors
%
\newcommand{\acap}[0]{\hat{\Ba}}
\newcommand{\bcap}[0]{\hat{\Bb}}
\newcommand{\ccap}[0]{\hat{\Bc}}
\newcommand{\dcap}[0]{\hat{\Bd}}
\newcommand{\ecap}[0]{\hat{\Be}}
\newcommand{\fcap}[0]{\hat{\Bf}}
\newcommand{\gcap}[0]{\hat{\Bg}}
\newcommand{\hcap}[0]{\hat{\Bh}}
\newcommand{\icap}[0]{\hat{\Bi}}
\newcommand{\jcap}[0]{\hat{\Bj}}
\newcommand{\kcap}[0]{\hat{\Bk}}
\newcommand{\lcap}[0]{\hat{\Bl}}
\newcommand{\mcap}[0]{\hat{\Bm}}
\newcommand{\ncap}[0]{\hat{\Bn}}
\newcommand{\ocap}[0]{\hat{\Bo}}
\newcommand{\pcap}[0]{\hat{\Bp}}
\newcommand{\qcap}[0]{\hat{\Bq}}
\newcommand{\rcap}[0]{\hat{\Br}}
\newcommand{\scap}[0]{\hat{\Bs}}
\newcommand{\tcap}[0]{\hat{\Bt}}
\newcommand{\ucap}[0]{\hat{\Bu}}
\newcommand{\vcap}[0]{\hat{\Bv}}
\newcommand{\wcap}[0]{\hat{\Bw}}
\newcommand{\xcap}[0]{\hat{\Bx}}
\newcommand{\ycap}[0]{\hat{\By}}
\newcommand{\zcap}[0]{\hat{\Bz}}
\newcommand{\thetacap}[0]{\hat{\Btheta}}

%
% to write R^n and C^n in a distinguishable fashion.  Perhaps change this
% to the double lined characters upon figuring out how to do so.
%
\newcommand{\C}[1]{$\mathbb{C}^{#1}$}
\newcommand{\R}[1]{$\mathbb{R}^{#1}$}

%
% various generally useful helpers
%

% derivative of #1 wrt. #2:
\newcommand{\D}[2] {\frac {d#2} {d#1}}

\newcommand{\inv}[1]{\frac{1}{#1}}
\newcommand{\cross}[0]{\times}

\newcommand{\abs}[1]{\lvert{#1}\rvert}
\newcommand{\norm}[1]{\lVert{#1}\rVert}
\newcommand{\innerprod}[2]{\langle{#1}, {#2}\rangle}
\newcommand{\dotprod}[2]{{#1} \cdot {#2}}
\newcommand{\bdotprod}[2]{\left({#1} \cdot {#2}\right)}
\newcommand{\crossprod}[2]{{#1} \cross {#2}}
\newcommand{\tripleprod}[3]{\dotprod{\left(\crossprod{#1}{#2}\right)}{#3}}

\DeclareMathOperator{\Proj}{Proj}
\DeclareMathOperator{\Span}{span}
\DeclareMathOperator{\Sgn}{sgn}
\DeclareMathOperator{\Area}{Area}
\DeclareMathOperator{\Volume}{Volume}

%
% A few miscellaneous things specific to this document
%
\newcommand{\crossop}[1]{\crossprod{#1}{}}

% R2 vector.
\newcommand{\VectorTwo}[2]{
\begin{bmatrix}
 {#1} \\
 {#2}
\end{bmatrix}
}

\newcommand{\VectorN}[1]{
\begin{bmatrix}
{#1}_1 \\
{#1}_2 \\
\vdots \\
{#1}_N \\
\end{bmatrix}
}

\newcommand{\DETuvij}[4]{
\begin{vmatrix}
 {#1}_{#3} & {#1}_{#4} \\
 {#2}_{#3} & {#2}_{#4}
\end{vmatrix}
}

\newcommand{\DETuvwijk}[6]{
\begin{vmatrix}
 {#1}_{#4} & {#1}_{#5} & {#1}_{#6} \\
 {#2}_{#4} & {#2}_{#5} & {#2}_{#6} \\
 {#3}_{#4} & {#3}_{#5} & {#3}_{#6}
\end{vmatrix}
}

\newcommand{\DETuvwxijkl}[8]{
\begin{vmatrix}
 {#1}_{#5} & {#1}_{#6} & {#1}_{#7} & {#1}_{#8} \\
 {#2}_{#5} & {#2}_{#6} & {#2}_{#7} & {#2}_{#8} \\
 {#3}_{#5} & {#3}_{#6} & {#3}_{#7} & {#3}_{#8} \\
 {#4}_{#5} & {#4}_{#6} & {#4}_{#7} & {#4}_{#8} \\
\end{vmatrix}
}

%\newcommand{\DETuvwxyijklm}[10]{
%\begin{vmatrix}
% {#1}_{#6} & {#1}_{#7} & {#1}_{#8} & {#1}_{#9} & {#1}_{#10} \\
% {#2}_{#6} & {#2}_{#7} & {#2}_{#8} & {#2}_{#9} & {#2}_{#10} \\
% {#3}_{#6} & {#3}_{#7} & {#3}_{#8} & {#3}_{#9} & {#3}_{#10} \\
% {#4}_{#6} & {#4}_{#7} & {#4}_{#8} & {#4}_{#9} & {#4}_{#10} \\
% {#5}_{#6} & {#5}_{#7} & {#5}_{#8} & {#5}_{#9} & {#5}_{#10}
%\end{vmatrix}
%}

% R3 vector.
\newcommand{\VectorThree}[3]{
\begin{bmatrix}
 {#1} \\
 {#2} \\
 {#3}
\end{bmatrix}
}



%\usepackage[bookmarks=true]{hyperref}

\chapter{Taylor's theorem deviation.}
\label{chap:taylors}
%\author{Peeter Joot \quad peeter.joot@gmail.com}
\date{ Feb. 2, 2008. taylors.tex }

%\begin{document}

%\maketitle{}

\citep{hestenes1999nfc} presents a very simple derivation of Taylor's Theorem,
but I feel a
slightly different (dumber, but longer) presentation would be more effective.

In the same fashion, form the integral

\[
I = \int_{t}^{t+s} F'(u) du
\]

Now, observe that the this first order derivative can be written in
terms of it's second order derivative

\[
(u F'(u))' = F'(u) + u F''(u)
\]

So we could write

\begin{align*}
I &= \int_{t}^{t+s} ((u F'(u))' - u F''(u)) du \\
  &= {u F'(u)} \vert_{u=t}^{t+s} - \int_{t}^{t+s} u F''(u)) du \\
  &= (t+s) F'(t+s) - t F'(t) - \int_{t}^{t+s} u F''(u)) du \\
\end{align*}

This is true, but not the Taylor expansion we are used to.  Adjusting things slightly leaves a zero term at $u=t+s$, as follows:

\[
\left((t + s - u) F'(u)\right)' = -F'(u) + (t+s-u) F''(u)
\]

\begin{align*}
I &= F(t+s) - F(t) \\
 &= \int_{t}^{t+s} ( - ((t + s - u) F'(u))' + (t+s-u) F''(u) ) du \\
 &= - {(t + s - u) F'(u)} \vert_{u=t}^{t+s} + \int_{t}^{t+s} (t+s-u) F''(u) du \\
 &= s F'(t) + \int_{t}^{t+s} (t+s-u) F''(u) du \\
\end{align*}

This results in the first two order terms of the Tailor series, with an explicit remainder term

\[
F(t+s) = F(t) + s F'(t) + \int_{t}^{t+s} (t+s-u) F''(u) du
\]

This process can be continued for as many terms as desired.  Doing the calculation for the second order term yields

\[
\left(\frac{(t + s - u)^2}{2} F''(u)\right)' = -( t + s - u ) F''(u) + \frac{(t+s-u)}{2} F'''(u)
\]

And substituting that into the first order expansion above we have

\[
F(t+s) = F(t) + s F'(t) + \frac{s^2}{2} F''(t) + \int_{t}^{t+s} \frac{1}{2}(t+s-u)^2 F'''(u) du
\]

Induction produces n terms of Taylor's series with an explicit remainder

\[
F(t+s) = \sum_{k=0}^{n} \frac{s^k}{k!} \frac{d^k}{dt^k} F(t) +
                        \int_{t}^{t+s} \frac{1}{n!}(t+s-u)^n \frac{d^{n+1}}{dt^{n+1}} F(u) du
\]

To truly prove the infinite series result one would have to show that the remainder term tends to zero.

%\bibliographystyle{plainnat} % supposed to allow for \url use.
%\bibliography{myrefs}      % expects file "myrefs.bib"

%\end{document}               % End of document.

%
% Copyright � 2012 Peeter Joot.  All Rights Reserved.
% Licenced as described in the file LICENSE under the root directory of this GIT repository.
%

% 
% 
%\documentclass{article}

%\usepackage{amsmath}
\usepackage{mathpazo}

%
% shorthand for bold symbols, convenient for vectors and matrices
%
\newcommand{\Ba}[0]{\mathbf{a}}
\newcommand{\Bb}[0]{\mathbf{b}}
\newcommand{\Bc}[0]{\mathbf{c}}
\newcommand{\Bd}[0]{\mathbf{d}}
\newcommand{\Be}[0]{\mathbf{e}}
\newcommand{\Bf}[0]{\mathbf{f}}
\newcommand{\Bg}[0]{\mathbf{g}}
\newcommand{\Bh}[0]{\mathbf{h}}
\newcommand{\Bi}[0]{\mathbf{i}}
\newcommand{\Bj}[0]{\mathbf{j}}
\newcommand{\Bk}[0]{\mathbf{k}}
\newcommand{\Bl}[0]{\mathbf{l}}
\newcommand{\Bm}[0]{\mathbf{m}}
\newcommand{\Bn}[0]{\mathbf{n}}
\newcommand{\Bo}[0]{\mathbf{o}}
\newcommand{\Bp}[0]{\mathbf{p}}
\newcommand{\Bq}[0]{\mathbf{q}}
\newcommand{\Br}[0]{\mathbf{r}}
\newcommand{\Bs}[0]{\mathbf{s}}
\newcommand{\Bt}[0]{\mathbf{t}}
\newcommand{\Bu}[0]{\mathbf{u}}
\newcommand{\Bv}[0]{\mathbf{v}}
\newcommand{\Bw}[0]{\mathbf{w}}
\newcommand{\Bx}[0]{\mathbf{x}}
\newcommand{\By}[0]{\mathbf{y}}
\newcommand{\Bz}[0]{\mathbf{z}}
\newcommand{\BA}[0]{\mathbf{A}}
\newcommand{\BB}[0]{\mathbf{B}}
\newcommand{\BC}[0]{\mathbf{C}}
\newcommand{\BD}[0]{\mathbf{D}}
\newcommand{\BE}[0]{\mathbf{E}}
\newcommand{\BF}[0]{\mathbf{F}}
\newcommand{\BG}[0]{\mathbf{G}}
\newcommand{\BH}[0]{\mathbf{H}}
\newcommand{\BI}[0]{\mathbf{I}}
\newcommand{\BJ}[0]{\mathbf{J}}
\newcommand{\BK}[0]{\mathbf{K}}
\newcommand{\BL}[0]{\mathbf{L}}
\newcommand{\BM}[0]{\mathbf{M}}
\newcommand{\BN}[0]{\mathbf{N}}
\newcommand{\BO}[0]{\mathbf{O}}
\newcommand{\BP}[0]{\mathbf{P}}
\newcommand{\BQ}[0]{\mathbf{Q}}
\newcommand{\BR}[0]{\mathbf{R}}
\newcommand{\BS}[0]{\mathbf{S}}
\newcommand{\BT}[0]{\mathbf{T}}
\newcommand{\BU}[0]{\mathbf{U}}
\newcommand{\BV}[0]{\mathbf{V}}
\newcommand{\BW}[0]{\mathbf{W}}
\newcommand{\BX}[0]{\mathbf{X}}
\newcommand{\BY}[0]{\mathbf{Y}}
\newcommand{\BZ}[0]{\mathbf{Z}}

\newcommand{\Bzero}[0]{\mathbf{0}}
\newcommand{\Btheta}[0]{\boldsymbol{\theta}}
\newcommand{\Btau}[0]{\boldsymbol{\tau}}
\newcommand{\Bomega}[0]{\boldsymbol{\omega}}

%
% shorthand for unit vectors
%
\newcommand{\acap}[0]{\hat{\Ba}}
\newcommand{\bcap}[0]{\hat{\Bb}}
\newcommand{\ccap}[0]{\hat{\Bc}}
\newcommand{\dcap}[0]{\hat{\Bd}}
\newcommand{\ecap}[0]{\hat{\Be}}
\newcommand{\fcap}[0]{\hat{\Bf}}
\newcommand{\gcap}[0]{\hat{\Bg}}
\newcommand{\hcap}[0]{\hat{\Bh}}
\newcommand{\icap}[0]{\hat{\Bi}}
\newcommand{\jcap}[0]{\hat{\Bj}}
\newcommand{\kcap}[0]{\hat{\Bk}}
\newcommand{\lcap}[0]{\hat{\Bl}}
\newcommand{\mcap}[0]{\hat{\Bm}}
\newcommand{\ncap}[0]{\hat{\Bn}}
\newcommand{\ocap}[0]{\hat{\Bo}}
\newcommand{\pcap}[0]{\hat{\Bp}}
\newcommand{\qcap}[0]{\hat{\Bq}}
\newcommand{\rcap}[0]{\hat{\Br}}
\newcommand{\scap}[0]{\hat{\Bs}}
\newcommand{\tcap}[0]{\hat{\Bt}}
\newcommand{\ucap}[0]{\hat{\Bu}}
\newcommand{\vcap}[0]{\hat{\Bv}}
\newcommand{\wcap}[0]{\hat{\Bw}}
\newcommand{\xcap}[0]{\hat{\Bx}}
\newcommand{\ycap}[0]{\hat{\By}}
\newcommand{\zcap}[0]{\hat{\Bz}}
\newcommand{\thetacap}[0]{\hat{\Btheta}}

%
% to write R^n and C^n in a distinguishable fashion.  Perhaps change this
% to the double lined characters upon figuring out how to do so.
%
\newcommand{\C}[1]{$\mathbb{C}^{#1}$}
\newcommand{\R}[1]{$\mathbb{R}^{#1}$}

%
% various generally useful helpers
%

% derivative of #1 wrt. #2:
\newcommand{\D}[2] {\frac {d#2} {d#1}}

\newcommand{\inv}[1]{\frac{1}{#1}}
\newcommand{\cross}[0]{\times}

\newcommand{\abs}[1]{\lvert{#1}\rvert}
\newcommand{\norm}[1]{\lVert{#1}\rVert}
\newcommand{\innerprod}[2]{\langle{#1}, {#2}\rangle}
\newcommand{\dotprod}[2]{{#1} \cdot {#2}}
\newcommand{\bdotprod}[2]{\left({#1} \cdot {#2}\right)}
\newcommand{\crossprod}[2]{{#1} \cross {#2}}
\newcommand{\tripleprod}[3]{\dotprod{\left(\crossprod{#1}{#2}\right)}{#3}}

\DeclareMathOperator{\Proj}{Proj}
\DeclareMathOperator{\Span}{span}
\DeclareMathOperator{\Sgn}{sgn}
\DeclareMathOperator{\Area}{Area}
\DeclareMathOperator{\Volume}{Volume}

%
% A few miscellaneous things specific to this document
%
\newcommand{\crossop}[1]{\crossprod{#1}{}}

% R2 vector.
\newcommand{\VectorTwo}[2]{
\begin{bmatrix}
 {#1} \\
 {#2}
\end{bmatrix}
}

\newcommand{\VectorN}[1]{
\begin{bmatrix}
{#1}_1 \\
{#1}_2 \\
\vdots \\
{#1}_N \\
\end{bmatrix}
}

\newcommand{\DETuvij}[4]{
\begin{vmatrix}
 {#1}_{#3} & {#1}_{#4} \\
 {#2}_{#3} & {#2}_{#4}
\end{vmatrix}
}

\newcommand{\DETuvwijk}[6]{
\begin{vmatrix}
 {#1}_{#4} & {#1}_{#5} & {#1}_{#6} \\
 {#2}_{#4} & {#2}_{#5} & {#2}_{#6} \\
 {#3}_{#4} & {#3}_{#5} & {#3}_{#6}
\end{vmatrix}
}

\newcommand{\DETuvwxijkl}[8]{
\begin{vmatrix}
 {#1}_{#5} & {#1}_{#6} & {#1}_{#7} & {#1}_{#8} \\
 {#2}_{#5} & {#2}_{#6} & {#2}_{#7} & {#2}_{#8} \\
 {#3}_{#5} & {#3}_{#6} & {#3}_{#7} & {#3}_{#8} \\
 {#4}_{#5} & {#4}_{#6} & {#4}_{#7} & {#4}_{#8} \\
\end{vmatrix}
}

%\newcommand{\DETuvwxyijklm}[10]{
%\begin{vmatrix}
% {#1}_{#6} & {#1}_{#7} & {#1}_{#8} & {#1}_{#9} & {#1}_{#10} \\
% {#2}_{#6} & {#2}_{#7} & {#2}_{#8} & {#2}_{#9} & {#2}_{#10} \\
% {#3}_{#6} & {#3}_{#7} & {#3}_{#8} & {#3}_{#9} & {#3}_{#10} \\
% {#4}_{#6} & {#4}_{#7} & {#4}_{#8} & {#4}_{#9} & {#4}_{#10} \\
% {#5}_{#6} & {#5}_{#7} & {#5}_{#8} & {#5}_{#9} & {#5}_{#10}
%\end{vmatrix}
%}

% R3 vector.
\newcommand{\VectorThree}[3]{
\begin{bmatrix}
 {#1} \\
 {#2} \\
 {#3}
\end{bmatrix}
}


%%<misc>
%
\newcommand{\Abs}[1]{{\left\lvert{#1}\right\rvert}}
\newcommand{\spacegrad}[0]{\boldsymbol{\nabla}}
\newcommand{\grad}[0]{\nabla}
\newcommand{\LL}[0]{\mathcal{L}}

% == \partial_{#1} {#2}
\newcommand{\PD}[2]{\frac{\partial {#2}}{\partial {#1}}}
% inline variant
\newcommand{\PDi}[2]{{\partial {#2}}/{\partial {#1}}}

\newcommand{\PDD}[3]{\frac{\partial^2 {#3}}{\partial {#1}\partial {#2}}}
%\newcommand{\PDd}[2]{\frac{\partial^2 {#2}}{{\partial{#1}}^2}}
\newcommand{\PDsq}[2]{\frac{\partial^2 {#2}}{(\partial {#1})^2}}

\newcommand{\Partial}[2]{\frac{\partial {#1}}{\partial {#2}}}
\DeclareMathOperator{\RejName}{Rej}
\newcommand{\Rej}[2]{\RejName_{#1}\left( {#2} \right)}
\newcommand{\Rm}[1]{\mathbb{R}^{#1}}
\newcommand{\Cm}[1]{\mathbb{C}^{#1}}
\newcommand{\conj}[0]{{*}}

%</misc>

% <grade selection>
%
\newcommand{\gpgrade}[2] {{\left\langle{{#1}}\right\rangle}_{#2}}

\newcommand{\gpgradezero}[1] {\gpgrade{#1}{}}
%\newcommand{\gpscalargrade}[1] {{\left\langle{{#1}}\right\rangle}}
%\newcommand{\gpgradezero}[1] {\gpgrade{#1}{0}}

%\newcommand{\gpgradeone}[1] {{\left\langle{{#1}}\right\rangle}_{1}}
\newcommand{\gpgradeone}[1] {\gpgrade{#1}{1}}

\newcommand{\gpgradetwo}[1] {\gpgrade{#1}{2}}
\newcommand{\gpgradethree}[1] {\gpgrade{#1}{3}}
\newcommand{\gpgradefour}[1] {\gpgrade{#1}{4}}
%
% </grade selection>



\newcommand{\adot}[0]{{\dot{a}}}
\newcommand{\bdot}[0]{{\dot{b}}}
% taken for centered dot:
%\newcommand{\cdot}[0]{{\dot{c}}}
%\newcommand{\ddot}[0]{{\dot{d}}}
\newcommand{\edot}[0]{{\dot{e}}}
\newcommand{\fdot}[0]{{\dot{f}}}
\newcommand{\gdot}[0]{{\dot{g}}}
\newcommand{\hdot}[0]{{\dot{h}}}
\newcommand{\idot}[0]{{\dot{i}}}
\newcommand{\jdot}[0]{{\dot{j}}}
\newcommand{\kdot}[0]{{\dot{k}}}
\newcommand{\ldot}[0]{{\dot{l}}}
\newcommand{\mdot}[0]{{\dot{m}}}
\newcommand{\ndot}[0]{{\dot{n}}}
%\newcommand{\odot}[0]{{\dot{o}}}
\newcommand{\pdot}[0]{{\dot{p}}}
\newcommand{\qdot}[0]{{\dot{q}}}
\newcommand{\rdot}[0]{{\dot{r}}}
\newcommand{\sdot}[0]{{\dot{s}}}
\newcommand{\tdot}[0]{{\dot{t}}}
\newcommand{\udot}[0]{{\dot{u}}}
\newcommand{\vdot}[0]{{\dot{v}}}
\newcommand{\wdot}[0]{{\dot{w}}}
\newcommand{\xdot}[0]{{\dot{x}}}
\newcommand{\ydot}[0]{{\dot{y}}}
\newcommand{\zdot}[0]{{\dot{z}}}
\newcommand{\addot}[0]{{\ddot{a}}}
\newcommand{\bddot}[0]{{\ddot{b}}}
\newcommand{\cddot}[0]{{\ddot{c}}}
%\newcommand{\dddot}[0]{{\ddot{d}}}
\newcommand{\eddot}[0]{{\ddot{e}}}
\newcommand{\fddot}[0]{{\ddot{f}}}
\newcommand{\gddot}[0]{{\ddot{g}}}
\newcommand{\hddot}[0]{{\ddot{h}}}
\newcommand{\iddot}[0]{{\ddot{i}}}
\newcommand{\jddot}[0]{{\ddot{j}}}
\newcommand{\kddot}[0]{{\ddot{k}}}
\newcommand{\lddot}[0]{{\ddot{l}}}
\newcommand{\mddot}[0]{{\ddot{m}}}
\newcommand{\nddot}[0]{{\ddot{n}}}
\newcommand{\oddot}[0]{{\ddot{o}}}
\newcommand{\pddot}[0]{{\ddot{p}}}
\newcommand{\qddot}[0]{{\ddot{q}}}
\newcommand{\rddot}[0]{{\ddot{r}}}
\newcommand{\sddot}[0]{{\ddot{s}}}
\newcommand{\tddot}[0]{{\ddot{t}}}
\newcommand{\uddot}[0]{{\ddot{u}}}
\newcommand{\vddot}[0]{{\ddot{v}}}
\newcommand{\wddot}[0]{{\ddot{w}}}
\newcommand{\xddot}[0]{{\ddot{x}}}
\newcommand{\yddot}[0]{{\ddot{y}}}
\newcommand{\zddot}[0]{{\ddot{z}}}

%<bold and dot greek symbols>
%

\newcommand{\Deltadot}[0]{{\dot{\Delta}}}
\newcommand{\Gammadot}[0]{{\dot{\Gamma}}}
\newcommand{\Lambdadot}[0]{{\dot{\Lambda}}}
\newcommand{\Omegadot}[0]{{\dot{\Omega}}}
\newcommand{\Phidot}[0]{{\dot{\Phi}}}
\newcommand{\Pidot}[0]{{\dot{\Pi}}}
\newcommand{\Psidot}[0]{{\dot{\Psi}}}
\newcommand{\Sigmadot}[0]{{\dot{\Sigma}}}
\newcommand{\Thetadot}[0]{{\dot{\Theta}}}
\newcommand{\Upsilondot}[0]{{\dot{\Upsilon}}}
\newcommand{\Xidot}[0]{{\dot{\Xi}}}
\newcommand{\alphadot}[0]{{\dot{\alpha}}}
\newcommand{\betadot}[0]{{\dot{\beta}}}
\newcommand{\chidot}[0]{{\dot{\chi}}}
\newcommand{\deltadot}[0]{{\dot{\delta}}}
\newcommand{\epsilondot}[0]{{\dot{\epsilon}}}
\newcommand{\etadot}[0]{{\dot{\eta}}}
\newcommand{\gammadot}[0]{{\dot{\gamma}}}
\newcommand{\kappadot}[0]{{\dot{\kappa}}}
\newcommand{\lambdadot}[0]{{\dot{\lambda}}}
\newcommand{\mudot}[0]{{\dot{\mu}}}
\newcommand{\nudot}[0]{{\dot{\nu}}}
\newcommand{\omegadot}[0]{{\dot{\omega}}}
\newcommand{\phidot}[0]{{\dot{\phi}}}
\newcommand{\pidot}[0]{{\dot{\pi}}}
\newcommand{\psidot}[0]{{\dot{\psi}}}
\newcommand{\rhodot}[0]{{\dot{\rho}}}
\newcommand{\sigmadot}[0]{{\dot{\sigma}}}
\newcommand{\taudot}[0]{{\dot{\tau}}}
\newcommand{\thetadot}[0]{{\dot{\theta}}}
\newcommand{\upsilondot}[0]{{\dot{\upsilon}}}
\newcommand{\varepsilondot}[0]{{\dot{\varepsilon}}}
\newcommand{\varphidot}[0]{{\dot{\varphi}}}
\newcommand{\varpidot}[0]{{\dot{\varpi}}}
\newcommand{\varrhodot}[0]{{\dot{\varrho}}}
\newcommand{\varsigmadot}[0]{{\dot{\varsigma}}}
\newcommand{\varthetadot}[0]{{\dot{\vartheta}}}
\newcommand{\xidot}[0]{{\dot{\xi}}}
\newcommand{\zetadot}[0]{{\dot{\zeta}}}

\newcommand{\Deltaddot}[0]{{\ddot{\Delta}}}
\newcommand{\Gammaddot}[0]{{\ddot{\Gamma}}}
\newcommand{\Lambdaddot}[0]{{\ddot{\Lambda}}}
\newcommand{\Omegaddot}[0]{{\ddot{\Omega}}}
\newcommand{\Phiddot}[0]{{\ddot{\Phi}}}
\newcommand{\Piddot}[0]{{\ddot{\Pi}}}
\newcommand{\Psiddot}[0]{{\ddot{\Psi}}}
\newcommand{\Sigmaddot}[0]{{\ddot{\Sigma}}}
\newcommand{\Thetaddot}[0]{{\ddot{\Theta}}}
\newcommand{\Upsilonddot}[0]{{\ddot{\Upsilon}}}
\newcommand{\Xiddot}[0]{{\ddot{\Xi}}}
\newcommand{\alphaddot}[0]{{\ddot{\alpha}}}
\newcommand{\betaddot}[0]{{\ddot{\beta}}}
\newcommand{\chiddot}[0]{{\ddot{\chi}}}
\newcommand{\deltaddot}[0]{{\ddot{\delta}}}
\newcommand{\epsilonddot}[0]{{\ddot{\epsilon}}}
\newcommand{\etaddot}[0]{{\ddot{\eta}}}
\newcommand{\gammaddot}[0]{{\ddot{\gamma}}}
\newcommand{\kappaddot}[0]{{\ddot{\kappa}}}
\newcommand{\lambdaddot}[0]{{\ddot{\lambda}}}
\newcommand{\muddot}[0]{{\ddot{\mu}}}
\newcommand{\nuddot}[0]{{\ddot{\nu}}}
\newcommand{\omegaddot}[0]{{\ddot{\omega}}}
\newcommand{\phiddot}[0]{{\ddot{\phi}}}
\newcommand{\piddot}[0]{{\ddot{\pi}}}
\newcommand{\psiddot}[0]{{\ddot{\psi}}}
\newcommand{\rhoddot}[0]{{\ddot{\rho}}}
\newcommand{\sigmaddot}[0]{{\ddot{\sigma}}}
\newcommand{\tauddot}[0]{{\ddot{\tau}}}
\newcommand{\thetaddot}[0]{{\ddot{\theta}}}
\newcommand{\upsilonddot}[0]{{\ddot{\upsilon}}}
\newcommand{\varepsilonddot}[0]{{\ddot{\varepsilon}}}
\newcommand{\varphiddot}[0]{{\ddot{\varphi}}}
\newcommand{\varpiddot}[0]{{\ddot{\varpi}}}
\newcommand{\varrhoddot}[0]{{\ddot{\varrho}}}
\newcommand{\varsigmaddot}[0]{{\ddot{\varsigma}}}
\newcommand{\varthetaddot}[0]{{\ddot{\vartheta}}}
\newcommand{\xiddot}[0]{{\ddot{\xi}}}
\newcommand{\zetaddot}[0]{{\ddot{\zeta}}}

\newcommand{\BDelta}[0]{\boldsymbol{\Delta}}
\newcommand{\BGamma}[0]{\boldsymbol{\Gamma}}
\newcommand{\BLambda}[0]{\boldsymbol{\Lambda}}
\newcommand{\BOmega}[0]{\boldsymbol{\Omega}}
\newcommand{\BPhi}[0]{\boldsymbol{\Phi}}
\newcommand{\BPi}[0]{\boldsymbol{\Pi}}
\newcommand{\BPsi}[0]{\boldsymbol{\Psi}}
\newcommand{\BSigma}[0]{\boldsymbol{\Sigma}}
\newcommand{\BTheta}[0]{\boldsymbol{\Theta}}
\newcommand{\BUpsilon}[0]{\boldsymbol{\Upsilon}}
\newcommand{\BXi}[0]{\boldsymbol{\Xi}}
\newcommand{\Balpha}[0]{\boldsymbol{\alpha}}
\newcommand{\Bbeta}[0]{\boldsymbol{\beta}}
\newcommand{\Bchi}[0]{\boldsymbol{\chi}}
\newcommand{\Bdelta}[0]{\boldsymbol{\delta}}
\newcommand{\Bepsilon}[0]{\boldsymbol{\epsilon}}
\newcommand{\Beta}[0]{\boldsymbol{\eta}}
\newcommand{\Bgamma}[0]{\boldsymbol{\gamma}}
\newcommand{\Bkappa}[0]{\boldsymbol{\kappa}}
\newcommand{\Blambda}[0]{\boldsymbol{\lambda}}
\newcommand{\Bmu}[0]{\boldsymbol{\mu}}
\newcommand{\Bnu}[0]{\boldsymbol{\nu}}
%\newcommand{\Bomega}[0]{\boldsymbol{\omega}}
\newcommand{\Bphi}[0]{\boldsymbol{\phi}}
\newcommand{\Bpi}[0]{\boldsymbol{\pi}}
\newcommand{\Bpsi}[0]{\boldsymbol{\psi}}
\newcommand{\Brho}[0]{\boldsymbol{\rho}}
\newcommand{\Bsigma}[0]{\boldsymbol{\sigma}}
%\newcommand{\Btau}[0]{\boldsymbol{\tau}}
%\newcommand{\Btheta}[0]{\boldsymbol{\theta}}
\newcommand{\Bupsilon}[0]{\boldsymbol{\upsilon}}
\newcommand{\Bvarepsilon}[0]{\boldsymbol{\varepsilon}}
\newcommand{\Bvarphi}[0]{\boldsymbol{\varphi}}
\newcommand{\Bvarpi}[0]{\boldsymbol{\varpi}}
\newcommand{\Bvarrho}[0]{\boldsymbol{\varrho}}
\newcommand{\Bvarsigma}[0]{\boldsymbol{\varsigma}}
\newcommand{\Bvartheta}[0]{\boldsymbol{\vartheta}}
\newcommand{\Bxi}[0]{\boldsymbol{\xi}}
\newcommand{\Bzeta}[0]{\boldsymbol{\zeta}}
%
%</bold and dot greek symbols>
%<infrequent>
%
%\newcommand{\AreaOp}[1]{\AName_{#1}}
%\newcommand{\Babs}[0]{\abs{\BB}}
%\newcommand{\Bcap}[0]{\hat{\BB}}
%\newcommand{\BrPrimeRej}[0]{\rcap(\rcap \wedge \Br')}
%\newcommand{\CA}[0]{\mathcal{A}}
%\newcommand{\Cos}[1]{\cos{\left({#1}\right)}}
%\newcommand{\Det}[1] {\abs{#1}}
%\newcommand{\Dsq}[2] {\frac {\partial^2 {#1}} {\partial {#2}^2}}
%\newcommand{\Exp}[1]{\exp{\left({#1}\right)}}
%\newcommand{\Norm}[1]{\left\lVert{#1}\right\rVert}
%\newcommand{\Sin}[1]{\sin{\left({#1}\right)}}
%\newcommand{\T}[0]{\text{T}}
%\newcommand{\VolumeOp}[1]{\VName_{#1}}
%\newcommand{\agrad}[0]{\Ba \cdot \nabla}
%\newcommand{\alphacap}[0]{\hat{\boldsymbol{\alpha}}}
%\newcommand{\Fcap}[0]{\hat{\BF}}
%\newcommand{\bithree}[0]{{\Bi}_3}
%\newcommand{\bxa}[0]{\Bx\Ba}
%\newcommand{\coordvec}[2]{
%\newcommand{\costheta}[0]{\acap \cdot \xcap}
%\newcommand{\ddt}[1]{\ddot{#1}}
%\newcommand{\ddu}[1] {\frac {d{#1}} {du}}
%\newcommand{\dsqxj}[2] {\frac {\partial^2 {#1}} {\partial {x_{#2}}^2}}
%\newcommand{\dtheta}[1]{\frac{d {#1}}{d \theta}}
%\newcommand{\dt}[1]{\dot{#1}}
%\newcommand{\dt}[1]{\frac{d {#1}}{dt}}
%\newcommand{\dxj}[2] {\frac {\partial {#1}} {\partial {x_{#2}}}}
%\newcommand{\halfPhi}[0]{\frac{\phi}{2}}
%\newcommand{\half}[0]{\inv{2}}
%\newcommand{\inv}[1]{\frac{1}{#1}}
%\newcommand{\laplacian}[0]{\nabla^2}
%\newcommand{\matrixoftx}[3]{
%\newcommand{\nrrp}[0]{\norm{\rcap \wedge \Br'}}
%\newcommand{\oiint}{\bigcirc \hspace{-1.4em} \int \hspace{-.8em} \int}
%\newcommand{\transpose}[1]{{#1}^{\text{T}}}
%\newcommand{\transpose}[1]{{{#1}^{\TextTranspose}}}
%\newcommand{\transpose}[1]{{{#1}^{\text{T}}}}
%\newcommand{\barA}[0]{\bar{A}}
%\newcommand{\qbar}[0]{\bar{q}}
%\newcommand{\qdotbar}[0]{\dot{\bar{q}}}
%
%</infrequent>





%\usepackage[bookmarks=true]{hyperref}

%\usepackage{color,cite,graphicx}
   % use colour in the document, put your citations as [1-4]
   % rather than [1,2,3,4] (it looks nicer, and the extended LaTeX2e
   % graphics package. 
%\usepackage{latexsym,amssymb,epsf} % don't remember if these are
   % needed, but their inclusion can't do any damage


\chapter{Green's functions}
\label{chap:greens}
%\author{Peeter Joot \quad peeter.joot@gmail.com}
\date{ Dec 19, 2008.  $RCSfile: greens.tex,v $ Last $Revision: 1.10 $ $Date: 2009/08/20 02:24:45 $ }

%\begin{document}
%\maketitle{}
%\tableofcontents

\section{Motivation}

For electromagnetism the vector potential equation to solve is

\begin{align*}
\grad (\grad \wedge A) = \frac{J}{c \epsilon_0}
\end{align*}

Imposing a gauge condition $\grad \cdot A$ on the solutions $A$, we have

\begin{align*}
\grad^2 A = \frac{J}{c \epsilon_0}
\end{align*}

Which gives us our four scalar potential equations

\begin{align*}
\left(\inv{c^2}\PDsq{t}{} - \sum_k \PDsq{x^k}{} \right) A^0 &= \frac{\rho}{\epsilon_0} \\
\left(\inv{c^2}\PDsq{t}{} - \sum_k \PDsq{x^k}{} \right) A^i &= \frac{J^i}{c \epsilon_0} \\
\end{align*}

One form of solution, found for example in \citep{feynman1963flp} for these equations is the retarded time potentials.  For example, with $\phi = A^0$, the retarded solution for $\phi$ at some time $t$, is expressed as an integral over all space

\begin{align*}
\phi(\Bx, t) &= \inv{4 \pi \epsilon_0} \int \frac{\rho(\Bx', t - \Abs{\Bx-\Bx'}/c)}{\Abs{\Bx - \Bx'}} dV'
\end{align*}

The first time I saw this equation, I was struck by the relativistic nature of the solution.  Only the charges with speed of light separation from the point of interest has an effect on the field at that point.

Unfortunately understanding the origin of those solutions, as covered in \citep{FitzRelEandM}, appears to require Green's functions, Fourier transforms and a whole mess of complex seeming stuff.

Intuition tells me that relativistic treatment of the electrostatics potential solution (ie: Lorentz boosting the Coulomb statics potential) can be used to develop the retarded time solutions, and that will be worth a try.  However, it also appears that direct study of the Green's function tools will be worthwhile.

These notes will examine topics related to Green's functions, with the eventual goal of building towards these retarded time potential solutions.

\section{Green's functions}

The basic idea is given a linear differential operator $\LL$, and the Dirac delta function $\delta$, a Green's function solution to the operator equation is

\begin{align*}
\LL G(x,s) = \delta(x-s)
\end{align*}

Using convolution the impulse (delta) function can be used to form arbitrary driving functions, and the general solution, by superposition then only requires an equivalent convolution integration.

Examples of possible differential operators are

\begin{align*}
\LL &= \frac{d}{dx} + a \\
\LL &= a \PDsq{x}{} + b \PDsq{y}{} \\
\LL &= \spacegrad^2 - \inv{v^2}\PDsq{t}{}\\
\LL &= \grad^2 \\
\end{align*}

The last of which is the desired operator for the electromagnetism case.

\section{Solving linear differential equations}

We don't necessarily need Green's functions and transform theory explicitly to solve linear equations.  Lets review some simple cases to see how to directly solve some of these.

\subsection{Simplest case}

The simplest case is a homogeneous linear first order equation of one variable.

\begin{align*}
f' + a f = 0
\end{align*}

The usual exponential characteristic solution method solves this one

\begin{align*}
f &= e^{rx} \\
\implies \\
(r + a) e^{rx} &= 0
\end{align*}

So a homogeneous solution is

\begin{align*}
f = e^{-ax}
\end{align*}

The next level of complexity is the inhomogeneous case, with a constant forcing function

\begin{align*}
f' + a f = b
\end{align*}

This is separable and the solution follows from direct integration

\begin{align*}
\int \frac{df}{b - a f} &= \int dx \\
\inv{-a}\ln(b - af) &= x - \inv{a} ln(-B) \\
\end{align*}

So our complete solution is

\begin{align*}
f &= A e^{-ax} + \inv{a}\left(b + B e^{-a x}\right)
\end{align*}

\subsubsection{With a sampling impulse function}

Now consider a unit area rectangular spike driving function of width $\epsilon$, and height $1/\epsilon$.  Will this be a reasonable approximation of a delta function when $\epsilon$ is let approach zero?

\begin{align*}
b(x) = 
\left\{
\begin{array}{l l}
1/{\epsilon} & \quad \mbox{if $x \in \left[-\epsilon/2, \epsilon/2 \right]$} \\
0              & \quad \mbox{otherwise}
\end{array} \right.
\end{align*}

With a degree of freedom for the homogeneous solution in each of the three ranges, and one degree of freedom for the inhomogeneous part in the impulse interval we have the following general solution

\begin{align*}
f(x) = 
\left\{
\begin{array}{l l}
A_{-} e^{-a(x + \epsilon/2)} & \quad x < -\epsilon/2 \\
%\inv{a \epsilon}
B e^{-ax} + \inv{a}\left(\inv{\epsilon} + C e^{-a x}\right)& \quad x \in \left[-\epsilon/2, \epsilon/2 \right] \\
A_{+} e^{-a(x - \epsilon/2)} & \quad x > \epsilon/2 \\
\end{array} \right.
\end{align*}

%There's actually one degree of freedom too many in the middle term as can be seen by regrouping slightly
%\begin{align*}
%\inv{a \epsilon} B e^{-ax} + \inv{a}\left(\inv{\epsilon} + C e^{-a x}\right) 
%&= \inv{a \epsilon} (B+C) e^{-ax} + \inv{a \epsilon} \\
%&= \inv{a \epsilon} \left( (B + C)e^{-ax} + 1 \right)
%\end{align*}

This isn't necessarily a continuous function, and there is in fact a degree of freedom too many in the middle term above.  Suppose that one wanted this term
to take values $B_{-}$, and $B_{+}$ at the boundaries of the impulse interval, one is then left with the following set of simultaneous equations

\begin{align*}
B_{-} &= \left(B + \inv{a \epsilon } C \right) e^{a \epsilon/2} + \inv{a \epsilon} \\
B_{+} &= \left(B + \inv{a \epsilon } C \right) e^{-a \epsilon/2} + \inv{a \epsilon} \\
\end{align*}

As a vector equation this is
\begin{align*}
a \epsilon
\begin{bmatrix}
B_{-} \\
B_{+} 
\end{bmatrix}
=
\left(a \epsilon B + C \right) 
\begin{bmatrix}
e^{a \epsilon/2} \\
e^{-a \epsilon/2} 
\end{bmatrix}
+
\begin{bmatrix}
1 \\
1
\end{bmatrix}
\end{align*}

and we can wedge with $(1,1)$ to eliminate it, leaving

\begin{align*}
a \epsilon
(B_{-} - B_{+})
=
\left(a \epsilon B + C \right) 
(e^{a \epsilon/2} - e^{-a \epsilon/2})
\end{align*}

So we have

\begin{align*}
B + \inv{a \epsilon}C = \frac{B_{-} - B_{+}}{2 \sinh(a \epsilon/2)}
\end{align*}

and the solution to our equation is therefore

\begin{align*}
f(x) = 
\left\{
\begin{array}{l l}
A_{-} e^{-a(x + \epsilon/2)} & \quad x < -\epsilon/2 \\
\frac{B_{-} - B_{+}}{2 \sinh(a \epsilon/2)} e^{-ax} + \inv{a \epsilon} & \quad x \in \left[-\epsilon/2, \epsilon/2 \right] \\
A_{+} e^{-a(x - \epsilon/2)} & \quad x > \epsilon/2 \\
\end{array} \right.
\end{align*}

However, with the middle term being a function of only the difference between the endpoints, means that we must have an additional constraint on the endpoints.  In general one could make this match the a desired value at only one of the endpoints, and then the other is then determined.  How about picking the middle term so that its value at $x=0$ is the midpoint between $A_{-}$ and $A_{+}$.  If one writes $f = \mu e^{-ax} + 1/a \epsilon$ we have

\begin{align*}
\mu + \inv{a\epsilon} &= \inv{2}\left(A_{-} + A_{+}\right) \\
\implies \\
f &= \inv{2}\left(A_{-} + A_{+}\right) e^{-ax} + \inv{a \epsilon} \left(1 - e^{-a x}\right)
\end{align*}

Now, this was for only the impulse interval.  If however, we have $A = A_{-} = A_{+}$, then writing $u(x)$ for the unit step function, we have for the complete interval the following general solution

\begin{align*}
f &= A e^{-ax} + \inv{2 a \epsilon} \left(1 - e^{-a x}\right)\left( u(\epsilon/2 + x) + u(\epsilon/2 -x)\right)
\end{align*}

This finally has a desirable form, with the sum of a regular old homogeneous solution, plus a specific solution.  The specific solution here also has the role of the Green's function for our delta like impulse function.  Doing a convolution sum with this impulse function has a sampling like action, but it isn't clear how to take the limit of the Green's like function as $\epsilon \rightarrow 0$.

Seeing the delta like function here suggests that the proper Green's function for this operator may in fact be a weighted delta function.  Specifically if we had

\begin{align*}
f' + a f &= \delta(x) \\
\end{align*}

Then is the corresponding general solution in terms of homogeneous solution and a Green's function like so:
\begin{align*}
f &= A e^{-ax} + \inv{a} \left(1 - e^{-a x}\right) \delta(x) \\
\end{align*}

How would one even see if this makes sense (ie: how does one take the derivative of a delta function)?  Probably need the convolution before this would make sense.  Let's try this.

%\bibliographystyle{plainnat}
%\bibliography{myrefs}

%\end{document}


\part{dynamics}
\documentclass{article}

\usepackage{amsmath}
\usepackage{mathpazo}

%
% shorthand for bold symbols, convenient for vectors and matrices
%
\newcommand{\Ba}[0]{\mathbf{a}}
\newcommand{\Bb}[0]{\mathbf{b}}
\newcommand{\Bc}[0]{\mathbf{c}}
\newcommand{\Bd}[0]{\mathbf{d}}
\newcommand{\Be}[0]{\mathbf{e}}
\newcommand{\Bf}[0]{\mathbf{f}}
\newcommand{\Bg}[0]{\mathbf{g}}
\newcommand{\Bh}[0]{\mathbf{h}}
\newcommand{\Bi}[0]{\mathbf{i}}
\newcommand{\Bj}[0]{\mathbf{j}}
\newcommand{\Bk}[0]{\mathbf{k}}
\newcommand{\Bl}[0]{\mathbf{l}}
\newcommand{\Bm}[0]{\mathbf{m}}
\newcommand{\Bn}[0]{\mathbf{n}}
\newcommand{\Bo}[0]{\mathbf{o}}
\newcommand{\Bp}[0]{\mathbf{p}}
\newcommand{\Bq}[0]{\mathbf{q}}
\newcommand{\Br}[0]{\mathbf{r}}
\newcommand{\Bs}[0]{\mathbf{s}}
\newcommand{\Bt}[0]{\mathbf{t}}
\newcommand{\Bu}[0]{\mathbf{u}}
\newcommand{\Bv}[0]{\mathbf{v}}
\newcommand{\Bw}[0]{\mathbf{w}}
\newcommand{\Bx}[0]{\mathbf{x}}
\newcommand{\By}[0]{\mathbf{y}}
\newcommand{\Bz}[0]{\mathbf{z}}
\newcommand{\BA}[0]{\mathbf{A}}
\newcommand{\BB}[0]{\mathbf{B}}
\newcommand{\BC}[0]{\mathbf{C}}
\newcommand{\BD}[0]{\mathbf{D}}
\newcommand{\BE}[0]{\mathbf{E}}
\newcommand{\BF}[0]{\mathbf{F}}
\newcommand{\BG}[0]{\mathbf{G}}
\newcommand{\BH}[0]{\mathbf{H}}
\newcommand{\BI}[0]{\mathbf{I}}
\newcommand{\BJ}[0]{\mathbf{J}}
\newcommand{\BK}[0]{\mathbf{K}}
\newcommand{\BL}[0]{\mathbf{L}}
\newcommand{\BM}[0]{\mathbf{M}}
\newcommand{\BN}[0]{\mathbf{N}}
\newcommand{\BO}[0]{\mathbf{O}}
\newcommand{\BP}[0]{\mathbf{P}}
\newcommand{\BQ}[0]{\mathbf{Q}}
\newcommand{\BR}[0]{\mathbf{R}}
\newcommand{\BS}[0]{\mathbf{S}}
\newcommand{\BT}[0]{\mathbf{T}}
\newcommand{\BU}[0]{\mathbf{U}}
\newcommand{\BV}[0]{\mathbf{V}}
\newcommand{\BW}[0]{\mathbf{W}}
\newcommand{\BX}[0]{\mathbf{X}}
\newcommand{\BY}[0]{\mathbf{Y}}
\newcommand{\BZ}[0]{\mathbf{Z}}

\newcommand{\Bzero}[0]{\mathbf{0}}
\newcommand{\Btheta}[0]{\boldsymbol{\theta}}
\newcommand{\Btau}[0]{\boldsymbol{\tau}}
\newcommand{\Bomega}[0]{\boldsymbol{\omega}}

%
% shorthand for unit vectors
%
\newcommand{\acap}[0]{\hat{\Ba}}
\newcommand{\bcap}[0]{\hat{\Bb}}
\newcommand{\ccap}[0]{\hat{\Bc}}
\newcommand{\dcap}[0]{\hat{\Bd}}
\newcommand{\ecap}[0]{\hat{\Be}}
\newcommand{\fcap}[0]{\hat{\Bf}}
\newcommand{\gcap}[0]{\hat{\Bg}}
\newcommand{\hcap}[0]{\hat{\Bh}}
\newcommand{\icap}[0]{\hat{\Bi}}
\newcommand{\jcap}[0]{\hat{\Bj}}
\newcommand{\kcap}[0]{\hat{\Bk}}
\newcommand{\lcap}[0]{\hat{\Bl}}
\newcommand{\mcap}[0]{\hat{\Bm}}
\newcommand{\ncap}[0]{\hat{\Bn}}
\newcommand{\ocap}[0]{\hat{\Bo}}
\newcommand{\pcap}[0]{\hat{\Bp}}
\newcommand{\qcap}[0]{\hat{\Bq}}
\newcommand{\rcap}[0]{\hat{\Br}}
\newcommand{\scap}[0]{\hat{\Bs}}
\newcommand{\tcap}[0]{\hat{\Bt}}
\newcommand{\ucap}[0]{\hat{\Bu}}
\newcommand{\vcap}[0]{\hat{\Bv}}
\newcommand{\wcap}[0]{\hat{\Bw}}
\newcommand{\xcap}[0]{\hat{\Bx}}
\newcommand{\ycap}[0]{\hat{\By}}
\newcommand{\zcap}[0]{\hat{\Bz}}
\newcommand{\thetacap}[0]{\hat{\Btheta}}

%
% to write R^n and C^n in a distinguishable fashion.  Perhaps change this
% to the double lined characters upon figuring out how to do so.
%
\newcommand{\C}[1]{$\mathbb{C}^{#1}$}
\newcommand{\R}[1]{$\mathbb{R}^{#1}$}

%
% various generally useful helpers
%

% derivative of #1 wrt. #2:
\newcommand{\D}[2] {\frac {d#2} {d#1}}

\newcommand{\inv}[1]{\frac{1}{#1}}
\newcommand{\cross}[0]{\times}

\newcommand{\abs}[1]{\lvert{#1}\rvert}
\newcommand{\norm}[1]{\lVert{#1}\rVert}
\newcommand{\innerprod}[2]{\langle{#1}, {#2}\rangle}
\newcommand{\dotprod}[2]{{#1} \cdot {#2}}
\newcommand{\bdotprod}[2]{\left({#1} \cdot {#2}\right)}
\newcommand{\crossprod}[2]{{#1} \cross {#2}}
\newcommand{\tripleprod}[3]{\dotprod{\left(\crossprod{#1}{#2}\right)}{#3}}

\DeclareMathOperator{\Proj}{Proj}
\DeclareMathOperator{\Span}{span}
\DeclareMathOperator{\Sgn}{sgn}
\DeclareMathOperator{\Area}{Area}
\DeclareMathOperator{\Volume}{Volume}

%
% A few miscellaneous things specific to this document
%
\newcommand{\crossop}[1]{\crossprod{#1}{}}

% R2 vector.
\newcommand{\VectorTwo}[2]{
\begin{bmatrix}
 {#1} \\
 {#2}
\end{bmatrix}
}

\newcommand{\VectorN}[1]{
\begin{bmatrix}
{#1}_1 \\
{#1}_2 \\
\vdots \\
{#1}_N \\
\end{bmatrix}
}

\newcommand{\DETuvij}[4]{
\begin{vmatrix}
 {#1}_{#3} & {#1}_{#4} \\
 {#2}_{#3} & {#2}_{#4}
\end{vmatrix}
}

\newcommand{\DETuvwijk}[6]{
\begin{vmatrix}
 {#1}_{#4} & {#1}_{#5} & {#1}_{#6} \\
 {#2}_{#4} & {#2}_{#5} & {#2}_{#6} \\
 {#3}_{#4} & {#3}_{#5} & {#3}_{#6}
\end{vmatrix}
}

\newcommand{\DETuvwxijkl}[8]{
\begin{vmatrix}
 {#1}_{#5} & {#1}_{#6} & {#1}_{#7} & {#1}_{#8} \\
 {#2}_{#5} & {#2}_{#6} & {#2}_{#7} & {#2}_{#8} \\
 {#3}_{#5} & {#3}_{#6} & {#3}_{#7} & {#3}_{#8} \\
 {#4}_{#5} & {#4}_{#6} & {#4}_{#7} & {#4}_{#8} \\
\end{vmatrix}
}

%\newcommand{\DETuvwxyijklm}[10]{
%\begin{vmatrix}
% {#1}_{#6} & {#1}_{#7} & {#1}_{#8} & {#1}_{#9} & {#1}_{#10} \\
% {#2}_{#6} & {#2}_{#7} & {#2}_{#8} & {#2}_{#9} & {#2}_{#10} \\
% {#3}_{#6} & {#3}_{#7} & {#3}_{#8} & {#3}_{#9} & {#3}_{#10} \\
% {#4}_{#6} & {#4}_{#7} & {#4}_{#8} & {#4}_{#9} & {#4}_{#10} \\
% {#5}_{#6} & {#5}_{#7} & {#5}_{#8} & {#5}_{#9} & {#5}_{#10}
%\end{vmatrix}
%}

% R3 vector.
\newcommand{\VectorThree}[3]{
\begin{bmatrix}
 {#1} \\
 {#2} \\
 {#3}
\end{bmatrix}
}


%<misc>
%
\newcommand{\Abs}[1]{{\left\lvert{#1}\right\rvert}}
\newcommand{\spacegrad}[0]{\boldsymbol{\nabla}}
\newcommand{\grad}[0]{\nabla}
\newcommand{\LL}[0]{\mathcal{L}}

% == \partial_{#1} {#2}
\newcommand{\PD}[2]{\frac{\partial {#2}}{\partial {#1}}}
% inline variant
\newcommand{\PDi}[2]{{\partial {#2}}/{\partial {#1}}}

\newcommand{\PDD}[3]{\frac{\partial^2 {#3}}{\partial {#1}\partial {#2}}}
%\newcommand{\PDd}[2]{\frac{\partial^2 {#2}}{{\partial{#1}}^2}}
\newcommand{\PDsq}[2]{\frac{\partial^2 {#2}}{(\partial {#1})^2}}

\newcommand{\Partial}[2]{\frac{\partial {#1}}{\partial {#2}}}
\DeclareMathOperator{\RejName}{Rej}
\newcommand{\Rej}[2]{\RejName_{#1}\left( {#2} \right)}
\newcommand{\Rm}[1]{\mathbb{R}^{#1}}
\newcommand{\Cm}[1]{\mathbb{C}^{#1}}
\newcommand{\conj}[0]{{*}}

%</misc>

% <grade selection>
%
\newcommand{\gpgrade}[2] {{\left\langle{{#1}}\right\rangle}_{#2}}

\newcommand{\gpgradezero}[1] {\gpgrade{#1}{}}
%\newcommand{\gpscalargrade}[1] {{\left\langle{{#1}}\right\rangle}}
%\newcommand{\gpgradezero}[1] {\gpgrade{#1}{0}}

%\newcommand{\gpgradeone}[1] {{\left\langle{{#1}}\right\rangle}_{1}}
\newcommand{\gpgradeone}[1] {\gpgrade{#1}{1}}

\newcommand{\gpgradetwo}[1] {\gpgrade{#1}{2}}
\newcommand{\gpgradethree}[1] {\gpgrade{#1}{3}}
\newcommand{\gpgradefour}[1] {\gpgrade{#1}{4}}
%
% </grade selection>



\newcommand{\adot}[0]{{\dot{a}}}
\newcommand{\bdot}[0]{{\dot{b}}}
% taken for centered dot:
%\newcommand{\cdot}[0]{{\dot{c}}}
%\newcommand{\ddot}[0]{{\dot{d}}}
\newcommand{\edot}[0]{{\dot{e}}}
\newcommand{\fdot}[0]{{\dot{f}}}
\newcommand{\gdot}[0]{{\dot{g}}}
\newcommand{\hdot}[0]{{\dot{h}}}
\newcommand{\idot}[0]{{\dot{i}}}
\newcommand{\jdot}[0]{{\dot{j}}}
\newcommand{\kdot}[0]{{\dot{k}}}
\newcommand{\ldot}[0]{{\dot{l}}}
\newcommand{\mdot}[0]{{\dot{m}}}
\newcommand{\ndot}[0]{{\dot{n}}}
%\newcommand{\odot}[0]{{\dot{o}}}
\newcommand{\pdot}[0]{{\dot{p}}}
\newcommand{\qdot}[0]{{\dot{q}}}
\newcommand{\rdot}[0]{{\dot{r}}}
\newcommand{\sdot}[0]{{\dot{s}}}
\newcommand{\tdot}[0]{{\dot{t}}}
\newcommand{\udot}[0]{{\dot{u}}}
\newcommand{\vdot}[0]{{\dot{v}}}
\newcommand{\wdot}[0]{{\dot{w}}}
\newcommand{\xdot}[0]{{\dot{x}}}
\newcommand{\ydot}[0]{{\dot{y}}}
\newcommand{\zdot}[0]{{\dot{z}}}
\newcommand{\addot}[0]{{\ddot{a}}}
\newcommand{\bddot}[0]{{\ddot{b}}}
\newcommand{\cddot}[0]{{\ddot{c}}}
%\newcommand{\dddot}[0]{{\ddot{d}}}
\newcommand{\eddot}[0]{{\ddot{e}}}
\newcommand{\fddot}[0]{{\ddot{f}}}
\newcommand{\gddot}[0]{{\ddot{g}}}
\newcommand{\hddot}[0]{{\ddot{h}}}
\newcommand{\iddot}[0]{{\ddot{i}}}
\newcommand{\jddot}[0]{{\ddot{j}}}
\newcommand{\kddot}[0]{{\ddot{k}}}
\newcommand{\lddot}[0]{{\ddot{l}}}
\newcommand{\mddot}[0]{{\ddot{m}}}
\newcommand{\nddot}[0]{{\ddot{n}}}
\newcommand{\oddot}[0]{{\ddot{o}}}
\newcommand{\pddot}[0]{{\ddot{p}}}
\newcommand{\qddot}[0]{{\ddot{q}}}
\newcommand{\rddot}[0]{{\ddot{r}}}
\newcommand{\sddot}[0]{{\ddot{s}}}
\newcommand{\tddot}[0]{{\ddot{t}}}
\newcommand{\uddot}[0]{{\ddot{u}}}
\newcommand{\vddot}[0]{{\ddot{v}}}
\newcommand{\wddot}[0]{{\ddot{w}}}
\newcommand{\xddot}[0]{{\ddot{x}}}
\newcommand{\yddot}[0]{{\ddot{y}}}
\newcommand{\zddot}[0]{{\ddot{z}}}

%<bold and dot greek symbols>
%

\newcommand{\Deltadot}[0]{{\dot{\Delta}}}
\newcommand{\Gammadot}[0]{{\dot{\Gamma}}}
\newcommand{\Lambdadot}[0]{{\dot{\Lambda}}}
\newcommand{\Omegadot}[0]{{\dot{\Omega}}}
\newcommand{\Phidot}[0]{{\dot{\Phi}}}
\newcommand{\Pidot}[0]{{\dot{\Pi}}}
\newcommand{\Psidot}[0]{{\dot{\Psi}}}
\newcommand{\Sigmadot}[0]{{\dot{\Sigma}}}
\newcommand{\Thetadot}[0]{{\dot{\Theta}}}
\newcommand{\Upsilondot}[0]{{\dot{\Upsilon}}}
\newcommand{\Xidot}[0]{{\dot{\Xi}}}
\newcommand{\alphadot}[0]{{\dot{\alpha}}}
\newcommand{\betadot}[0]{{\dot{\beta}}}
\newcommand{\chidot}[0]{{\dot{\chi}}}
\newcommand{\deltadot}[0]{{\dot{\delta}}}
\newcommand{\epsilondot}[0]{{\dot{\epsilon}}}
\newcommand{\etadot}[0]{{\dot{\eta}}}
\newcommand{\gammadot}[0]{{\dot{\gamma}}}
\newcommand{\kappadot}[0]{{\dot{\kappa}}}
\newcommand{\lambdadot}[0]{{\dot{\lambda}}}
\newcommand{\mudot}[0]{{\dot{\mu}}}
\newcommand{\nudot}[0]{{\dot{\nu}}}
\newcommand{\omegadot}[0]{{\dot{\omega}}}
\newcommand{\phidot}[0]{{\dot{\phi}}}
\newcommand{\pidot}[0]{{\dot{\pi}}}
\newcommand{\psidot}[0]{{\dot{\psi}}}
\newcommand{\rhodot}[0]{{\dot{\rho}}}
\newcommand{\sigmadot}[0]{{\dot{\sigma}}}
\newcommand{\taudot}[0]{{\dot{\tau}}}
\newcommand{\thetadot}[0]{{\dot{\theta}}}
\newcommand{\upsilondot}[0]{{\dot{\upsilon}}}
\newcommand{\varepsilondot}[0]{{\dot{\varepsilon}}}
\newcommand{\varphidot}[0]{{\dot{\varphi}}}
\newcommand{\varpidot}[0]{{\dot{\varpi}}}
\newcommand{\varrhodot}[0]{{\dot{\varrho}}}
\newcommand{\varsigmadot}[0]{{\dot{\varsigma}}}
\newcommand{\varthetadot}[0]{{\dot{\vartheta}}}
\newcommand{\xidot}[0]{{\dot{\xi}}}
\newcommand{\zetadot}[0]{{\dot{\zeta}}}

\newcommand{\Deltaddot}[0]{{\ddot{\Delta}}}
\newcommand{\Gammaddot}[0]{{\ddot{\Gamma}}}
\newcommand{\Lambdaddot}[0]{{\ddot{\Lambda}}}
\newcommand{\Omegaddot}[0]{{\ddot{\Omega}}}
\newcommand{\Phiddot}[0]{{\ddot{\Phi}}}
\newcommand{\Piddot}[0]{{\ddot{\Pi}}}
\newcommand{\Psiddot}[0]{{\ddot{\Psi}}}
\newcommand{\Sigmaddot}[0]{{\ddot{\Sigma}}}
\newcommand{\Thetaddot}[0]{{\ddot{\Theta}}}
\newcommand{\Upsilonddot}[0]{{\ddot{\Upsilon}}}
\newcommand{\Xiddot}[0]{{\ddot{\Xi}}}
\newcommand{\alphaddot}[0]{{\ddot{\alpha}}}
\newcommand{\betaddot}[0]{{\ddot{\beta}}}
\newcommand{\chiddot}[0]{{\ddot{\chi}}}
\newcommand{\deltaddot}[0]{{\ddot{\delta}}}
\newcommand{\epsilonddot}[0]{{\ddot{\epsilon}}}
\newcommand{\etaddot}[0]{{\ddot{\eta}}}
\newcommand{\gammaddot}[0]{{\ddot{\gamma}}}
\newcommand{\kappaddot}[0]{{\ddot{\kappa}}}
\newcommand{\lambdaddot}[0]{{\ddot{\lambda}}}
\newcommand{\muddot}[0]{{\ddot{\mu}}}
\newcommand{\nuddot}[0]{{\ddot{\nu}}}
\newcommand{\omegaddot}[0]{{\ddot{\omega}}}
\newcommand{\phiddot}[0]{{\ddot{\phi}}}
\newcommand{\piddot}[0]{{\ddot{\pi}}}
\newcommand{\psiddot}[0]{{\ddot{\psi}}}
\newcommand{\rhoddot}[0]{{\ddot{\rho}}}
\newcommand{\sigmaddot}[0]{{\ddot{\sigma}}}
\newcommand{\tauddot}[0]{{\ddot{\tau}}}
\newcommand{\thetaddot}[0]{{\ddot{\theta}}}
\newcommand{\upsilonddot}[0]{{\ddot{\upsilon}}}
\newcommand{\varepsilonddot}[0]{{\ddot{\varepsilon}}}
\newcommand{\varphiddot}[0]{{\ddot{\varphi}}}
\newcommand{\varpiddot}[0]{{\ddot{\varpi}}}
\newcommand{\varrhoddot}[0]{{\ddot{\varrho}}}
\newcommand{\varsigmaddot}[0]{{\ddot{\varsigma}}}
\newcommand{\varthetaddot}[0]{{\ddot{\vartheta}}}
\newcommand{\xiddot}[0]{{\ddot{\xi}}}
\newcommand{\zetaddot}[0]{{\ddot{\zeta}}}

\newcommand{\BDelta}[0]{\boldsymbol{\Delta}}
\newcommand{\BGamma}[0]{\boldsymbol{\Gamma}}
\newcommand{\BLambda}[0]{\boldsymbol{\Lambda}}
\newcommand{\BOmega}[0]{\boldsymbol{\Omega}}
\newcommand{\BPhi}[0]{\boldsymbol{\Phi}}
\newcommand{\BPi}[0]{\boldsymbol{\Pi}}
\newcommand{\BPsi}[0]{\boldsymbol{\Psi}}
\newcommand{\BSigma}[0]{\boldsymbol{\Sigma}}
\newcommand{\BTheta}[0]{\boldsymbol{\Theta}}
\newcommand{\BUpsilon}[0]{\boldsymbol{\Upsilon}}
\newcommand{\BXi}[0]{\boldsymbol{\Xi}}
\newcommand{\Balpha}[0]{\boldsymbol{\alpha}}
\newcommand{\Bbeta}[0]{\boldsymbol{\beta}}
\newcommand{\Bchi}[0]{\boldsymbol{\chi}}
\newcommand{\Bdelta}[0]{\boldsymbol{\delta}}
\newcommand{\Bepsilon}[0]{\boldsymbol{\epsilon}}
\newcommand{\Beta}[0]{\boldsymbol{\eta}}
\newcommand{\Bgamma}[0]{\boldsymbol{\gamma}}
\newcommand{\Bkappa}[0]{\boldsymbol{\kappa}}
\newcommand{\Blambda}[0]{\boldsymbol{\lambda}}
\newcommand{\Bmu}[0]{\boldsymbol{\mu}}
\newcommand{\Bnu}[0]{\boldsymbol{\nu}}
%\newcommand{\Bomega}[0]{\boldsymbol{\omega}}
\newcommand{\Bphi}[0]{\boldsymbol{\phi}}
\newcommand{\Bpi}[0]{\boldsymbol{\pi}}
\newcommand{\Bpsi}[0]{\boldsymbol{\psi}}
\newcommand{\Brho}[0]{\boldsymbol{\rho}}
\newcommand{\Bsigma}[0]{\boldsymbol{\sigma}}
%\newcommand{\Btau}[0]{\boldsymbol{\tau}}
%\newcommand{\Btheta}[0]{\boldsymbol{\theta}}
\newcommand{\Bupsilon}[0]{\boldsymbol{\upsilon}}
\newcommand{\Bvarepsilon}[0]{\boldsymbol{\varepsilon}}
\newcommand{\Bvarphi}[0]{\boldsymbol{\varphi}}
\newcommand{\Bvarpi}[0]{\boldsymbol{\varpi}}
\newcommand{\Bvarrho}[0]{\boldsymbol{\varrho}}
\newcommand{\Bvarsigma}[0]{\boldsymbol{\varsigma}}
\newcommand{\Bvartheta}[0]{\boldsymbol{\vartheta}}
\newcommand{\Bxi}[0]{\boldsymbol{\xi}}
\newcommand{\Bzeta}[0]{\boldsymbol{\zeta}}
%
%</bold and dot greek symbols>
%<infrequent>
%
%\newcommand{\AreaOp}[1]{\AName_{#1}}
%\newcommand{\Babs}[0]{\abs{\BB}}
%\newcommand{\Bcap}[0]{\hat{\BB}}
%\newcommand{\BrPrimeRej}[0]{\rcap(\rcap \wedge \Br')}
%\newcommand{\CA}[0]{\mathcal{A}}
%\newcommand{\Cos}[1]{\cos{\left({#1}\right)}}
%\newcommand{\Det}[1] {\abs{#1}}
%\newcommand{\Dsq}[2] {\frac {\partial^2 {#1}} {\partial {#2}^2}}
%\newcommand{\Exp}[1]{\exp{\left({#1}\right)}}
%\newcommand{\Norm}[1]{\left\lVert{#1}\right\rVert}
%\newcommand{\Sin}[1]{\sin{\left({#1}\right)}}
%\newcommand{\T}[0]{\text{T}}
%\newcommand{\VolumeOp}[1]{\VName_{#1}}
%\newcommand{\agrad}[0]{\Ba \cdot \nabla}
%\newcommand{\alphacap}[0]{\hat{\boldsymbol{\alpha}}}
%\newcommand{\Fcap}[0]{\hat{\BF}}
%\newcommand{\bithree}[0]{{\Bi}_3}
%\newcommand{\bxa}[0]{\Bx\Ba}
%\newcommand{\coordvec}[2]{
%\newcommand{\costheta}[0]{\acap \cdot \xcap}
%\newcommand{\ddt}[1]{\ddot{#1}}
%\newcommand{\ddu}[1] {\frac {d{#1}} {du}}
%\newcommand{\dsqxj}[2] {\frac {\partial^2 {#1}} {\partial {x_{#2}}^2}}
%\newcommand{\dtheta}[1]{\frac{d {#1}}{d \theta}}
%\newcommand{\dt}[1]{\dot{#1}}
%\newcommand{\dt}[1]{\frac{d {#1}}{dt}}
%\newcommand{\dxj}[2] {\frac {\partial {#1}} {\partial {x_{#2}}}}
%\newcommand{\halfPhi}[0]{\frac{\phi}{2}}
%\newcommand{\half}[0]{\inv{2}}
%\newcommand{\inv}[1]{\frac{1}{#1}}
%\newcommand{\laplacian}[0]{\nabla^2}
%\newcommand{\matrixoftx}[3]{
%\newcommand{\nrrp}[0]{\norm{\rcap \wedge \Br'}}
%\newcommand{\oiint}{\bigcirc \hspace{-1.4em} \int \hspace{-.8em} \int}
%\newcommand{\transpose}[1]{{#1}^{\text{T}}}
%\newcommand{\transpose}[1]{{{#1}^{\TextTranspose}}}
%\newcommand{\transpose}[1]{{{#1}^{\text{T}}}}
%\newcommand{\barA}[0]{\bar{A}}
%\newcommand{\qbar}[0]{\bar{q}}
%\newcommand{\qdotbar}[0]{\dot{\bar{q}}}
%
%</infrequent>





\usepackage[bookmarks=true]{hyperref}

\usepackage{color,cite,graphicx}
   % use colour in the document, put your citations as [1-4]
   % rather than [1,2,3,4] (it looks nicer, and the extended LaTeX2e
   % graphics package. 
\usepackage{latexsym,amssymb,epsf} % don't remember if these are
   % needed, but their inclusion can't do any damage


\title{ Time for two bodies to converge in space. }
\author{Peeter Joot}
\date{ Dec 15, 2008.  Last Revision: $Date: 2008/12/16 04:45:14 $ }

\begin{document}

\maketitle{}

%\tableofcontents
%
%\section{}

Assuming that the initial relative velocities of the two bodies lies along
the line between them this can be treated as a one dimensional problem.

We have two force equations

\begin{align*}
\text{Force on} \quad m_2 &= - G m_1 m_2 (x_2 - x_1)^{-2} = m_2 {x_2}'' \\
\text{Force on} \quad m_1 &=  G m_1 m_2 (x_2 - x_1)^{-2} = m_1 {x_1}''
\end{align*}

Subtraction (scaled) of these two equations leaves an integration problem for a difference of position variable $u = x_2 - x_1$

\begin{align*}
- G (x_2 - x_1)^{-2} (m_1 + m_2)= (x_2 - x_1)'' \\
\implies
- G (m_1 + m_2) \inv{u^2} = u'' \\
\end{align*}

The chain rule can be used for a first elimination of a time derivative

\begin{align*}
\frac{d}{dt} = \frac{du}{dt} \frac{d}{du}
\end{align*}

With $k = G(m_1 + m_2)$, this is

\begin{align*}
\frac{du}{dt} \frac{d}{du} \left(\frac{du}{dt}\right) &= -k u^{-2}
\end{align*}

Introducing a difference of position "velocity" the LHS of this is

\begin{align*}
\frac{du}{dt} \frac{d}{du} \left(\frac{du}{dt}\right)
&= v \frac{dv}{du} \\
&= \frac{d}{du} \left( \inv{2}v^2 \right) \\
\end{align*}

So now we have something that can be directly integrated:

\begin{align*}
\inv{2}v^2 &= k \inv{u} + \inv{2}C
\end{align*}

Fixing the constant in terms of the initial position and speed this is

\begin{align}\label{eqn:velocity}
\frac{du}{dt} &= \pm \sqrt{v_0^2 + 2 G M \left(\inv{u} - \inv{u_0}\right)}
\end{align}

and we can integrate for the time

\begin{align*}
\delta t &= \pm \int_{u_0}^{u_1} du \left(v_0^2 + 2 G M \left(\inv{u} - \inv{u_0}\right)\right)^{-1/2}
\end{align*}

The integral here is of the form

\begin{align*}
\int \frac{\sqrt{u}}{\sqrt{u + \alpha}} du
\end{align*}

and now it's time to pull out the table of integrals, and put in some numbers.

\href{http://calc101.com/webMathematica/integrals.jsp}{webmathematica} gives, in figure \ref{fig:msp}

\begin{figure}[htp]
\centering
\includegraphics[totalheight=0.4\textheight]{msp}
\caption{the integral}\label{fig:msp}
\end{figure}

More interesting than the numbers themselves is to think through the physical meaning for the $\pm$.  It makes sense in the velocity equation \ref{eqn:velocity} since the particles can be initially moving towards or away from each other (or stationary).  However, do negative time "solutions" indicate the particles can never converge to the desired separation?

%\bibliographystyle{plainnat}
%\bibliography{myrefs}

\end{document}


\part{feynman}
% 
% 
% 
% Copyright � 2012 Peeter Joot
% All Rights Reserved
% 
% This file may be reproduced and distributed in whole or in part, without fee, subject to the following conditions:
% 
% o The copyright notice above and this permission notice must be preserved complete on all complete or partial copies.
% 
% o Any translation or derived work must be approved by the author in writing before distribution.
% 
% o If you distribute this work in part, instructions for obtaining the complete version of this file must be included, and a means for obtaining a complete version provided.
% 
% 
% Exceptions to these rules may be granted for academic purposes: Write to the author and ask.
% 
% 
% 
%\documentclass{article}

%\usepackage{amsmath}
\usepackage{mathpazo}

%
% shorthand for bold symbols, convenient for vectors and matrices
%
\newcommand{\Ba}[0]{\mathbf{a}}
\newcommand{\Bb}[0]{\mathbf{b}}
\newcommand{\Bc}[0]{\mathbf{c}}
\newcommand{\Bd}[0]{\mathbf{d}}
\newcommand{\Be}[0]{\mathbf{e}}
\newcommand{\Bf}[0]{\mathbf{f}}
\newcommand{\Bg}[0]{\mathbf{g}}
\newcommand{\Bh}[0]{\mathbf{h}}
\newcommand{\Bi}[0]{\mathbf{i}}
\newcommand{\Bj}[0]{\mathbf{j}}
\newcommand{\Bk}[0]{\mathbf{k}}
\newcommand{\Bl}[0]{\mathbf{l}}
\newcommand{\Bm}[0]{\mathbf{m}}
\newcommand{\Bn}[0]{\mathbf{n}}
\newcommand{\Bo}[0]{\mathbf{o}}
\newcommand{\Bp}[0]{\mathbf{p}}
\newcommand{\Bq}[0]{\mathbf{q}}
\newcommand{\Br}[0]{\mathbf{r}}
\newcommand{\Bs}[0]{\mathbf{s}}
\newcommand{\Bt}[0]{\mathbf{t}}
\newcommand{\Bu}[0]{\mathbf{u}}
\newcommand{\Bv}[0]{\mathbf{v}}
\newcommand{\Bw}[0]{\mathbf{w}}
\newcommand{\Bx}[0]{\mathbf{x}}
\newcommand{\By}[0]{\mathbf{y}}
\newcommand{\Bz}[0]{\mathbf{z}}
\newcommand{\BA}[0]{\mathbf{A}}
\newcommand{\BB}[0]{\mathbf{B}}
\newcommand{\BC}[0]{\mathbf{C}}
\newcommand{\BD}[0]{\mathbf{D}}
\newcommand{\BE}[0]{\mathbf{E}}
\newcommand{\BF}[0]{\mathbf{F}}
\newcommand{\BG}[0]{\mathbf{G}}
\newcommand{\BH}[0]{\mathbf{H}}
\newcommand{\BI}[0]{\mathbf{I}}
\newcommand{\BJ}[0]{\mathbf{J}}
\newcommand{\BK}[0]{\mathbf{K}}
\newcommand{\BL}[0]{\mathbf{L}}
\newcommand{\BM}[0]{\mathbf{M}}
\newcommand{\BN}[0]{\mathbf{N}}
\newcommand{\BO}[0]{\mathbf{O}}
\newcommand{\BP}[0]{\mathbf{P}}
\newcommand{\BQ}[0]{\mathbf{Q}}
\newcommand{\BR}[0]{\mathbf{R}}
\newcommand{\BS}[0]{\mathbf{S}}
\newcommand{\BT}[0]{\mathbf{T}}
\newcommand{\BU}[0]{\mathbf{U}}
\newcommand{\BV}[0]{\mathbf{V}}
\newcommand{\BW}[0]{\mathbf{W}}
\newcommand{\BX}[0]{\mathbf{X}}
\newcommand{\BY}[0]{\mathbf{Y}}
\newcommand{\BZ}[0]{\mathbf{Z}}

\newcommand{\Bzero}[0]{\mathbf{0}}
\newcommand{\Btheta}[0]{\boldsymbol{\theta}}
\newcommand{\Btau}[0]{\boldsymbol{\tau}}
\newcommand{\Bomega}[0]{\boldsymbol{\omega}}

%
% shorthand for unit vectors
%
\newcommand{\acap}[0]{\hat{\Ba}}
\newcommand{\bcap}[0]{\hat{\Bb}}
\newcommand{\ccap}[0]{\hat{\Bc}}
\newcommand{\dcap}[0]{\hat{\Bd}}
\newcommand{\ecap}[0]{\hat{\Be}}
\newcommand{\fcap}[0]{\hat{\Bf}}
\newcommand{\gcap}[0]{\hat{\Bg}}
\newcommand{\hcap}[0]{\hat{\Bh}}
\newcommand{\icap}[0]{\hat{\Bi}}
\newcommand{\jcap}[0]{\hat{\Bj}}
\newcommand{\kcap}[0]{\hat{\Bk}}
\newcommand{\lcap}[0]{\hat{\Bl}}
\newcommand{\mcap}[0]{\hat{\Bm}}
\newcommand{\ncap}[0]{\hat{\Bn}}
\newcommand{\ocap}[0]{\hat{\Bo}}
\newcommand{\pcap}[0]{\hat{\Bp}}
\newcommand{\qcap}[0]{\hat{\Bq}}
\newcommand{\rcap}[0]{\hat{\Br}}
\newcommand{\scap}[0]{\hat{\Bs}}
\newcommand{\tcap}[0]{\hat{\Bt}}
\newcommand{\ucap}[0]{\hat{\Bu}}
\newcommand{\vcap}[0]{\hat{\Bv}}
\newcommand{\wcap}[0]{\hat{\Bw}}
\newcommand{\xcap}[0]{\hat{\Bx}}
\newcommand{\ycap}[0]{\hat{\By}}
\newcommand{\zcap}[0]{\hat{\Bz}}
\newcommand{\thetacap}[0]{\hat{\Btheta}}

%
% to write R^n and C^n in a distinguishable fashion.  Perhaps change this
% to the double lined characters upon figuring out how to do so.
%
\newcommand{\C}[1]{$\mathbb{C}^{#1}$}
\newcommand{\R}[1]{$\mathbb{R}^{#1}$}

%
% various generally useful helpers
%

% derivative of #1 wrt. #2:
\newcommand{\D}[2] {\frac {d#2} {d#1}}

\newcommand{\inv}[1]{\frac{1}{#1}}
\newcommand{\cross}[0]{\times}

\newcommand{\abs}[1]{\lvert{#1}\rvert}
\newcommand{\norm}[1]{\lVert{#1}\rVert}
\newcommand{\innerprod}[2]{\langle{#1}, {#2}\rangle}
\newcommand{\dotprod}[2]{{#1} \cdot {#2}}
\newcommand{\bdotprod}[2]{\left({#1} \cdot {#2}\right)}
\newcommand{\crossprod}[2]{{#1} \cross {#2}}
\newcommand{\tripleprod}[3]{\dotprod{\left(\crossprod{#1}{#2}\right)}{#3}}

\DeclareMathOperator{\Proj}{Proj}
\DeclareMathOperator{\Span}{span}
\DeclareMathOperator{\Sgn}{sgn}
\DeclareMathOperator{\Area}{Area}
\DeclareMathOperator{\Volume}{Volume}

%
% A few miscellaneous things specific to this document
%
\newcommand{\crossop}[1]{\crossprod{#1}{}}

% R2 vector.
\newcommand{\VectorTwo}[2]{
\begin{bmatrix}
 {#1} \\
 {#2}
\end{bmatrix}
}

\newcommand{\VectorN}[1]{
\begin{bmatrix}
{#1}_1 \\
{#1}_2 \\
\vdots \\
{#1}_N \\
\end{bmatrix}
}

\newcommand{\DETuvij}[4]{
\begin{vmatrix}
 {#1}_{#3} & {#1}_{#4} \\
 {#2}_{#3} & {#2}_{#4}
\end{vmatrix}
}

\newcommand{\DETuvwijk}[6]{
\begin{vmatrix}
 {#1}_{#4} & {#1}_{#5} & {#1}_{#6} \\
 {#2}_{#4} & {#2}_{#5} & {#2}_{#6} \\
 {#3}_{#4} & {#3}_{#5} & {#3}_{#6}
\end{vmatrix}
}

\newcommand{\DETuvwxijkl}[8]{
\begin{vmatrix}
 {#1}_{#5} & {#1}_{#6} & {#1}_{#7} & {#1}_{#8} \\
 {#2}_{#5} & {#2}_{#6} & {#2}_{#7} & {#2}_{#8} \\
 {#3}_{#5} & {#3}_{#6} & {#3}_{#7} & {#3}_{#8} \\
 {#4}_{#5} & {#4}_{#6} & {#4}_{#7} & {#4}_{#8} \\
\end{vmatrix}
}

%\newcommand{\DETuvwxyijklm}[10]{
%\begin{vmatrix}
% {#1}_{#6} & {#1}_{#7} & {#1}_{#8} & {#1}_{#9} & {#1}_{#10} \\
% {#2}_{#6} & {#2}_{#7} & {#2}_{#8} & {#2}_{#9} & {#2}_{#10} \\
% {#3}_{#6} & {#3}_{#7} & {#3}_{#8} & {#3}_{#9} & {#3}_{#10} \\
% {#4}_{#6} & {#4}_{#7} & {#4}_{#8} & {#4}_{#9} & {#4}_{#10} \\
% {#5}_{#6} & {#5}_{#7} & {#5}_{#8} & {#5}_{#9} & {#5}_{#10}
%\end{vmatrix}
%}

% R3 vector.
\newcommand{\VectorThree}[3]{
\begin{bmatrix}
 {#1} \\
 {#2} \\
 {#3}
\end{bmatrix}
}


%%<misc>
%
\newcommand{\Abs}[1]{{\left\lvert{#1}\right\rvert}}
\newcommand{\spacegrad}[0]{\boldsymbol{\nabla}}
\newcommand{\grad}[0]{\nabla}
\newcommand{\LL}[0]{\mathcal{L}}

% == \partial_{#1} {#2}
\newcommand{\PD}[2]{\frac{\partial {#2}}{\partial {#1}}}
% inline variant
\newcommand{\PDi}[2]{{\partial {#2}}/{\partial {#1}}}

\newcommand{\PDD}[3]{\frac{\partial^2 {#3}}{\partial {#1}\partial {#2}}}
%\newcommand{\PDd}[2]{\frac{\partial^2 {#2}}{{\partial{#1}}^2}}
\newcommand{\PDsq}[2]{\frac{\partial^2 {#2}}{(\partial {#1})^2}}

\newcommand{\Partial}[2]{\frac{\partial {#1}}{\partial {#2}}}
\DeclareMathOperator{\RejName}{Rej}
\newcommand{\Rej}[2]{\RejName_{#1}\left( {#2} \right)}
\newcommand{\Rm}[1]{\mathbb{R}^{#1}}
\newcommand{\Cm}[1]{\mathbb{C}^{#1}}
\newcommand{\conj}[0]{{*}}

%</misc>

% <grade selection>
%
\newcommand{\gpgrade}[2] {{\left\langle{{#1}}\right\rangle}_{#2}}

\newcommand{\gpgradezero}[1] {\gpgrade{#1}{}}
%\newcommand{\gpscalargrade}[1] {{\left\langle{{#1}}\right\rangle}}
%\newcommand{\gpgradezero}[1] {\gpgrade{#1}{0}}

%\newcommand{\gpgradeone}[1] {{\left\langle{{#1}}\right\rangle}_{1}}
\newcommand{\gpgradeone}[1] {\gpgrade{#1}{1}}

\newcommand{\gpgradetwo}[1] {\gpgrade{#1}{2}}
\newcommand{\gpgradethree}[1] {\gpgrade{#1}{3}}
\newcommand{\gpgradefour}[1] {\gpgrade{#1}{4}}
%
% </grade selection>



\newcommand{\adot}[0]{{\dot{a}}}
\newcommand{\bdot}[0]{{\dot{b}}}
% taken for centered dot:
%\newcommand{\cdot}[0]{{\dot{c}}}
%\newcommand{\ddot}[0]{{\dot{d}}}
\newcommand{\edot}[0]{{\dot{e}}}
\newcommand{\fdot}[0]{{\dot{f}}}
\newcommand{\gdot}[0]{{\dot{g}}}
\newcommand{\hdot}[0]{{\dot{h}}}
\newcommand{\idot}[0]{{\dot{i}}}
\newcommand{\jdot}[0]{{\dot{j}}}
\newcommand{\kdot}[0]{{\dot{k}}}
\newcommand{\ldot}[0]{{\dot{l}}}
\newcommand{\mdot}[0]{{\dot{m}}}
\newcommand{\ndot}[0]{{\dot{n}}}
%\newcommand{\odot}[0]{{\dot{o}}}
\newcommand{\pdot}[0]{{\dot{p}}}
\newcommand{\qdot}[0]{{\dot{q}}}
\newcommand{\rdot}[0]{{\dot{r}}}
\newcommand{\sdot}[0]{{\dot{s}}}
\newcommand{\tdot}[0]{{\dot{t}}}
\newcommand{\udot}[0]{{\dot{u}}}
\newcommand{\vdot}[0]{{\dot{v}}}
\newcommand{\wdot}[0]{{\dot{w}}}
\newcommand{\xdot}[0]{{\dot{x}}}
\newcommand{\ydot}[0]{{\dot{y}}}
\newcommand{\zdot}[0]{{\dot{z}}}
\newcommand{\addot}[0]{{\ddot{a}}}
\newcommand{\bddot}[0]{{\ddot{b}}}
\newcommand{\cddot}[0]{{\ddot{c}}}
%\newcommand{\dddot}[0]{{\ddot{d}}}
\newcommand{\eddot}[0]{{\ddot{e}}}
\newcommand{\fddot}[0]{{\ddot{f}}}
\newcommand{\gddot}[0]{{\ddot{g}}}
\newcommand{\hddot}[0]{{\ddot{h}}}
\newcommand{\iddot}[0]{{\ddot{i}}}
\newcommand{\jddot}[0]{{\ddot{j}}}
\newcommand{\kddot}[0]{{\ddot{k}}}
\newcommand{\lddot}[0]{{\ddot{l}}}
\newcommand{\mddot}[0]{{\ddot{m}}}
\newcommand{\nddot}[0]{{\ddot{n}}}
\newcommand{\oddot}[0]{{\ddot{o}}}
\newcommand{\pddot}[0]{{\ddot{p}}}
\newcommand{\qddot}[0]{{\ddot{q}}}
\newcommand{\rddot}[0]{{\ddot{r}}}
\newcommand{\sddot}[0]{{\ddot{s}}}
\newcommand{\tddot}[0]{{\ddot{t}}}
\newcommand{\uddot}[0]{{\ddot{u}}}
\newcommand{\vddot}[0]{{\ddot{v}}}
\newcommand{\wddot}[0]{{\ddot{w}}}
\newcommand{\xddot}[0]{{\ddot{x}}}
\newcommand{\yddot}[0]{{\ddot{y}}}
\newcommand{\zddot}[0]{{\ddot{z}}}

%<bold and dot greek symbols>
%

\newcommand{\Deltadot}[0]{{\dot{\Delta}}}
\newcommand{\Gammadot}[0]{{\dot{\Gamma}}}
\newcommand{\Lambdadot}[0]{{\dot{\Lambda}}}
\newcommand{\Omegadot}[0]{{\dot{\Omega}}}
\newcommand{\Phidot}[0]{{\dot{\Phi}}}
\newcommand{\Pidot}[0]{{\dot{\Pi}}}
\newcommand{\Psidot}[0]{{\dot{\Psi}}}
\newcommand{\Sigmadot}[0]{{\dot{\Sigma}}}
\newcommand{\Thetadot}[0]{{\dot{\Theta}}}
\newcommand{\Upsilondot}[0]{{\dot{\Upsilon}}}
\newcommand{\Xidot}[0]{{\dot{\Xi}}}
\newcommand{\alphadot}[0]{{\dot{\alpha}}}
\newcommand{\betadot}[0]{{\dot{\beta}}}
\newcommand{\chidot}[0]{{\dot{\chi}}}
\newcommand{\deltadot}[0]{{\dot{\delta}}}
\newcommand{\epsilondot}[0]{{\dot{\epsilon}}}
\newcommand{\etadot}[0]{{\dot{\eta}}}
\newcommand{\gammadot}[0]{{\dot{\gamma}}}
\newcommand{\kappadot}[0]{{\dot{\kappa}}}
\newcommand{\lambdadot}[0]{{\dot{\lambda}}}
\newcommand{\mudot}[0]{{\dot{\mu}}}
\newcommand{\nudot}[0]{{\dot{\nu}}}
\newcommand{\omegadot}[0]{{\dot{\omega}}}
\newcommand{\phidot}[0]{{\dot{\phi}}}
\newcommand{\pidot}[0]{{\dot{\pi}}}
\newcommand{\psidot}[0]{{\dot{\psi}}}
\newcommand{\rhodot}[0]{{\dot{\rho}}}
\newcommand{\sigmadot}[0]{{\dot{\sigma}}}
\newcommand{\taudot}[0]{{\dot{\tau}}}
\newcommand{\thetadot}[0]{{\dot{\theta}}}
\newcommand{\upsilondot}[0]{{\dot{\upsilon}}}
\newcommand{\varepsilondot}[0]{{\dot{\varepsilon}}}
\newcommand{\varphidot}[0]{{\dot{\varphi}}}
\newcommand{\varpidot}[0]{{\dot{\varpi}}}
\newcommand{\varrhodot}[0]{{\dot{\varrho}}}
\newcommand{\varsigmadot}[0]{{\dot{\varsigma}}}
\newcommand{\varthetadot}[0]{{\dot{\vartheta}}}
\newcommand{\xidot}[0]{{\dot{\xi}}}
\newcommand{\zetadot}[0]{{\dot{\zeta}}}

\newcommand{\Deltaddot}[0]{{\ddot{\Delta}}}
\newcommand{\Gammaddot}[0]{{\ddot{\Gamma}}}
\newcommand{\Lambdaddot}[0]{{\ddot{\Lambda}}}
\newcommand{\Omegaddot}[0]{{\ddot{\Omega}}}
\newcommand{\Phiddot}[0]{{\ddot{\Phi}}}
\newcommand{\Piddot}[0]{{\ddot{\Pi}}}
\newcommand{\Psiddot}[0]{{\ddot{\Psi}}}
\newcommand{\Sigmaddot}[0]{{\ddot{\Sigma}}}
\newcommand{\Thetaddot}[0]{{\ddot{\Theta}}}
\newcommand{\Upsilonddot}[0]{{\ddot{\Upsilon}}}
\newcommand{\Xiddot}[0]{{\ddot{\Xi}}}
\newcommand{\alphaddot}[0]{{\ddot{\alpha}}}
\newcommand{\betaddot}[0]{{\ddot{\beta}}}
\newcommand{\chiddot}[0]{{\ddot{\chi}}}
\newcommand{\deltaddot}[0]{{\ddot{\delta}}}
\newcommand{\epsilonddot}[0]{{\ddot{\epsilon}}}
\newcommand{\etaddot}[0]{{\ddot{\eta}}}
\newcommand{\gammaddot}[0]{{\ddot{\gamma}}}
\newcommand{\kappaddot}[0]{{\ddot{\kappa}}}
\newcommand{\lambdaddot}[0]{{\ddot{\lambda}}}
\newcommand{\muddot}[0]{{\ddot{\mu}}}
\newcommand{\nuddot}[0]{{\ddot{\nu}}}
\newcommand{\omegaddot}[0]{{\ddot{\omega}}}
\newcommand{\phiddot}[0]{{\ddot{\phi}}}
\newcommand{\piddot}[0]{{\ddot{\pi}}}
\newcommand{\psiddot}[0]{{\ddot{\psi}}}
\newcommand{\rhoddot}[0]{{\ddot{\rho}}}
\newcommand{\sigmaddot}[0]{{\ddot{\sigma}}}
\newcommand{\tauddot}[0]{{\ddot{\tau}}}
\newcommand{\thetaddot}[0]{{\ddot{\theta}}}
\newcommand{\upsilonddot}[0]{{\ddot{\upsilon}}}
\newcommand{\varepsilonddot}[0]{{\ddot{\varepsilon}}}
\newcommand{\varphiddot}[0]{{\ddot{\varphi}}}
\newcommand{\varpiddot}[0]{{\ddot{\varpi}}}
\newcommand{\varrhoddot}[0]{{\ddot{\varrho}}}
\newcommand{\varsigmaddot}[0]{{\ddot{\varsigma}}}
\newcommand{\varthetaddot}[0]{{\ddot{\vartheta}}}
\newcommand{\xiddot}[0]{{\ddot{\xi}}}
\newcommand{\zetaddot}[0]{{\ddot{\zeta}}}

\newcommand{\BDelta}[0]{\boldsymbol{\Delta}}
\newcommand{\BGamma}[0]{\boldsymbol{\Gamma}}
\newcommand{\BLambda}[0]{\boldsymbol{\Lambda}}
\newcommand{\BOmega}[0]{\boldsymbol{\Omega}}
\newcommand{\BPhi}[0]{\boldsymbol{\Phi}}
\newcommand{\BPi}[0]{\boldsymbol{\Pi}}
\newcommand{\BPsi}[0]{\boldsymbol{\Psi}}
\newcommand{\BSigma}[0]{\boldsymbol{\Sigma}}
\newcommand{\BTheta}[0]{\boldsymbol{\Theta}}
\newcommand{\BUpsilon}[0]{\boldsymbol{\Upsilon}}
\newcommand{\BXi}[0]{\boldsymbol{\Xi}}
\newcommand{\Balpha}[0]{\boldsymbol{\alpha}}
\newcommand{\Bbeta}[0]{\boldsymbol{\beta}}
\newcommand{\Bchi}[0]{\boldsymbol{\chi}}
\newcommand{\Bdelta}[0]{\boldsymbol{\delta}}
\newcommand{\Bepsilon}[0]{\boldsymbol{\epsilon}}
\newcommand{\Beta}[0]{\boldsymbol{\eta}}
\newcommand{\Bgamma}[0]{\boldsymbol{\gamma}}
\newcommand{\Bkappa}[0]{\boldsymbol{\kappa}}
\newcommand{\Blambda}[0]{\boldsymbol{\lambda}}
\newcommand{\Bmu}[0]{\boldsymbol{\mu}}
\newcommand{\Bnu}[0]{\boldsymbol{\nu}}
%\newcommand{\Bomega}[0]{\boldsymbol{\omega}}
\newcommand{\Bphi}[0]{\boldsymbol{\phi}}
\newcommand{\Bpi}[0]{\boldsymbol{\pi}}
\newcommand{\Bpsi}[0]{\boldsymbol{\psi}}
\newcommand{\Brho}[0]{\boldsymbol{\rho}}
\newcommand{\Bsigma}[0]{\boldsymbol{\sigma}}
%\newcommand{\Btau}[0]{\boldsymbol{\tau}}
%\newcommand{\Btheta}[0]{\boldsymbol{\theta}}
\newcommand{\Bupsilon}[0]{\boldsymbol{\upsilon}}
\newcommand{\Bvarepsilon}[0]{\boldsymbol{\varepsilon}}
\newcommand{\Bvarphi}[0]{\boldsymbol{\varphi}}
\newcommand{\Bvarpi}[0]{\boldsymbol{\varpi}}
\newcommand{\Bvarrho}[0]{\boldsymbol{\varrho}}
\newcommand{\Bvarsigma}[0]{\boldsymbol{\varsigma}}
\newcommand{\Bvartheta}[0]{\boldsymbol{\vartheta}}
\newcommand{\Bxi}[0]{\boldsymbol{\xi}}
\newcommand{\Bzeta}[0]{\boldsymbol{\zeta}}
%
%</bold and dot greek symbols>
%<infrequent>
%
%\newcommand{\AreaOp}[1]{\AName_{#1}}
%\newcommand{\Babs}[0]{\abs{\BB}}
%\newcommand{\Bcap}[0]{\hat{\BB}}
%\newcommand{\BrPrimeRej}[0]{\rcap(\rcap \wedge \Br')}
%\newcommand{\CA}[0]{\mathcal{A}}
%\newcommand{\Cos}[1]{\cos{\left({#1}\right)}}
%\newcommand{\Det}[1] {\abs{#1}}
%\newcommand{\Dsq}[2] {\frac {\partial^2 {#1}} {\partial {#2}^2}}
%\newcommand{\Exp}[1]{\exp{\left({#1}\right)}}
%\newcommand{\Norm}[1]{\left\lVert{#1}\right\rVert}
%\newcommand{\Sin}[1]{\sin{\left({#1}\right)}}
%\newcommand{\T}[0]{\text{T}}
%\newcommand{\VolumeOp}[1]{\VName_{#1}}
%\newcommand{\agrad}[0]{\Ba \cdot \nabla}
%\newcommand{\alphacap}[0]{\hat{\boldsymbol{\alpha}}}
%\newcommand{\Fcap}[0]{\hat{\BF}}
%\newcommand{\bithree}[0]{{\Bi}_3}
%\newcommand{\bxa}[0]{\Bx\Ba}
%\newcommand{\coordvec}[2]{
%\newcommand{\costheta}[0]{\acap \cdot \xcap}
%\newcommand{\ddt}[1]{\ddot{#1}}
%\newcommand{\ddu}[1] {\frac {d{#1}} {du}}
%\newcommand{\dsqxj}[2] {\frac {\partial^2 {#1}} {\partial {x_{#2}}^2}}
%\newcommand{\dtheta}[1]{\frac{d {#1}}{d \theta}}
%\newcommand{\dt}[1]{\dot{#1}}
%\newcommand{\dt}[1]{\frac{d {#1}}{dt}}
%\newcommand{\dxj}[2] {\frac {\partial {#1}} {\partial {x_{#2}}}}
%\newcommand{\halfPhi}[0]{\frac{\phi}{2}}
%\newcommand{\half}[0]{\inv{2}}
%\newcommand{\inv}[1]{\frac{1}{#1}}
%\newcommand{\laplacian}[0]{\nabla^2}
%\newcommand{\matrixoftx}[3]{
%\newcommand{\nrrp}[0]{\norm{\rcap \wedge \Br'}}
%\newcommand{\oiint}{\bigcirc \hspace{-1.4em} \int \hspace{-.8em} \int}
%\newcommand{\transpose}[1]{{#1}^{\text{T}}}
%\newcommand{\transpose}[1]{{{#1}^{\TextTranspose}}}
%\newcommand{\transpose}[1]{{{#1}^{\text{T}}}}
%\newcommand{\barA}[0]{\bar{A}}
%\newcommand{\qbar}[0]{\bar{q}}
%\newcommand{\qdotbar}[0]{\dot{\bar{q}}}
%
%</infrequent>





%\usepackage[bookmarks=true]{hyperref}

%\usepackage{color,cite,graphicx}
   % use colour in the document, put your citations as [1-4]
   % rather than [1,2,3,4] (it looks nicer, and the extended LaTeX2e
   % graphics package. 
%\usepackage{latexsym,amssymb,epsf} % don't remember if these are
   % needed, but their inclusion can't do any damage

\chapter{Derive the star distance calculation in the Feynman lectures. }
\label{chap:starDistance}
%\author{Peeter Joot \quad peeter.joot@gmail.com}
\date{ Jan 17, 2000.  starDistance.tex }

%\begin{document}

%\maketitle{}

%\tableofcontents
\section{Motivation. }

\begin{figure}[htp]
\centering
\includegraphics[totalheight=0.4\textheight]{feynman_star}
\caption{Triangulating the distance to a star}\label{fig:feynman_star}
\end{figure}

Derivation of the "locate Sputnik" formula of Fig 5-5 in \citep{feynman1963flp}, as illustrated in \ref{fig:feynman_star}.

\section{}

Elliptical orbit $x^2/a^2 + y^2/b^2 = c^2$, with orbital diameter $2ac = L$.  Angles to the star, measured relative to the Sun are $\alpha$, and $\pi - (\alpha+\epsilon)$.  Using the sine rule for triangles, 

\begin{displaymath}
  \frac{ l_1}{\sin(\pi - (\alpha + \epsilon))} 
= \frac{ l_2 }{\sin{\alpha}}
= \frac{L}{\sin{\epsilon}}
\end{displaymath}

So the lengths to the star from each end of the orbit we have 
\begin{eqnarray*}
l_1 & = & L \frac{ \sin(\alpha + \epsilon) }   { \sin{\epsilon}} \\
l_2 & = & L \frac{ \sin{\alpha} }   { \sin{\epsilon}}
\end{eqnarray*}

and so the average length $\overline{l}$ to the star is 
\begin{eqnarray*}
\overline{l} & = & L \frac{ \sin(\alpha + \epsilon) + \sin{\alpha} }{    \sin{\epsilon}} \\
             & = & 2L \frac{ \sin(\alpha + \epsilon/2)\cos(\epsilon/2) }{    \sin{\epsilon}}
\end{eqnarray*}

Since the angular difference $\epsilon << 0$, the average distance to the star can be approximated as
\begin{displaymath}
\overline{l} = 2L \frac{ \sin{\alpha} }{\epsilon }
\end{displaymath}
%\end{multicols}

%\pagebreak

%\bibliographystyle{plainnat}
%\bibliography{myrefs}

%\end{document}


\part{fletcher}
%
% Copyright � 2012 Peeter Joot.  All Rights Reserved.
% Licenced as described in the file LICENSE under the root directory of this GIT repository.
%

% 
% 
%\documentclass{article}

%\usepackage{amsmath}
\usepackage{mathpazo}

%
% shorthand for bold symbols, convenient for vectors and matrices
%
\newcommand{\Ba}[0]{\mathbf{a}}
\newcommand{\Bb}[0]{\mathbf{b}}
\newcommand{\Bc}[0]{\mathbf{c}}
\newcommand{\Bd}[0]{\mathbf{d}}
\newcommand{\Be}[0]{\mathbf{e}}
\newcommand{\Bf}[0]{\mathbf{f}}
\newcommand{\Bg}[0]{\mathbf{g}}
\newcommand{\Bh}[0]{\mathbf{h}}
\newcommand{\Bi}[0]{\mathbf{i}}
\newcommand{\Bj}[0]{\mathbf{j}}
\newcommand{\Bk}[0]{\mathbf{k}}
\newcommand{\Bl}[0]{\mathbf{l}}
\newcommand{\Bm}[0]{\mathbf{m}}
\newcommand{\Bn}[0]{\mathbf{n}}
\newcommand{\Bo}[0]{\mathbf{o}}
\newcommand{\Bp}[0]{\mathbf{p}}
\newcommand{\Bq}[0]{\mathbf{q}}
\newcommand{\Br}[0]{\mathbf{r}}
\newcommand{\Bs}[0]{\mathbf{s}}
\newcommand{\Bt}[0]{\mathbf{t}}
\newcommand{\Bu}[0]{\mathbf{u}}
\newcommand{\Bv}[0]{\mathbf{v}}
\newcommand{\Bw}[0]{\mathbf{w}}
\newcommand{\Bx}[0]{\mathbf{x}}
\newcommand{\By}[0]{\mathbf{y}}
\newcommand{\Bz}[0]{\mathbf{z}}
\newcommand{\BA}[0]{\mathbf{A}}
\newcommand{\BB}[0]{\mathbf{B}}
\newcommand{\BC}[0]{\mathbf{C}}
\newcommand{\BD}[0]{\mathbf{D}}
\newcommand{\BE}[0]{\mathbf{E}}
\newcommand{\BF}[0]{\mathbf{F}}
\newcommand{\BG}[0]{\mathbf{G}}
\newcommand{\BH}[0]{\mathbf{H}}
\newcommand{\BI}[0]{\mathbf{I}}
\newcommand{\BJ}[0]{\mathbf{J}}
\newcommand{\BK}[0]{\mathbf{K}}
\newcommand{\BL}[0]{\mathbf{L}}
\newcommand{\BM}[0]{\mathbf{M}}
\newcommand{\BN}[0]{\mathbf{N}}
\newcommand{\BO}[0]{\mathbf{O}}
\newcommand{\BP}[0]{\mathbf{P}}
\newcommand{\BQ}[0]{\mathbf{Q}}
\newcommand{\BR}[0]{\mathbf{R}}
\newcommand{\BS}[0]{\mathbf{S}}
\newcommand{\BT}[0]{\mathbf{T}}
\newcommand{\BU}[0]{\mathbf{U}}
\newcommand{\BV}[0]{\mathbf{V}}
\newcommand{\BW}[0]{\mathbf{W}}
\newcommand{\BX}[0]{\mathbf{X}}
\newcommand{\BY}[0]{\mathbf{Y}}
\newcommand{\BZ}[0]{\mathbf{Z}}

\newcommand{\Bzero}[0]{\mathbf{0}}
\newcommand{\Btheta}[0]{\boldsymbol{\theta}}
\newcommand{\Btau}[0]{\boldsymbol{\tau}}
\newcommand{\Bomega}[0]{\boldsymbol{\omega}}

%
% shorthand for unit vectors
%
\newcommand{\acap}[0]{\hat{\Ba}}
\newcommand{\bcap}[0]{\hat{\Bb}}
\newcommand{\ccap}[0]{\hat{\Bc}}
\newcommand{\dcap}[0]{\hat{\Bd}}
\newcommand{\ecap}[0]{\hat{\Be}}
\newcommand{\fcap}[0]{\hat{\Bf}}
\newcommand{\gcap}[0]{\hat{\Bg}}
\newcommand{\hcap}[0]{\hat{\Bh}}
\newcommand{\icap}[0]{\hat{\Bi}}
\newcommand{\jcap}[0]{\hat{\Bj}}
\newcommand{\kcap}[0]{\hat{\Bk}}
\newcommand{\lcap}[0]{\hat{\Bl}}
\newcommand{\mcap}[0]{\hat{\Bm}}
\newcommand{\ncap}[0]{\hat{\Bn}}
\newcommand{\ocap}[0]{\hat{\Bo}}
\newcommand{\pcap}[0]{\hat{\Bp}}
\newcommand{\qcap}[0]{\hat{\Bq}}
\newcommand{\rcap}[0]{\hat{\Br}}
\newcommand{\scap}[0]{\hat{\Bs}}
\newcommand{\tcap}[0]{\hat{\Bt}}
\newcommand{\ucap}[0]{\hat{\Bu}}
\newcommand{\vcap}[0]{\hat{\Bv}}
\newcommand{\wcap}[0]{\hat{\Bw}}
\newcommand{\xcap}[0]{\hat{\Bx}}
\newcommand{\ycap}[0]{\hat{\By}}
\newcommand{\zcap}[0]{\hat{\Bz}}
\newcommand{\thetacap}[0]{\hat{\Btheta}}

%
% to write R^n and C^n in a distinguishable fashion.  Perhaps change this
% to the double lined characters upon figuring out how to do so.
%
\newcommand{\C}[1]{$\mathbb{C}^{#1}$}
\newcommand{\R}[1]{$\mathbb{R}^{#1}$}

%
% various generally useful helpers
%

% derivative of #1 wrt. #2:
\newcommand{\D}[2] {\frac {d#2} {d#1}}

\newcommand{\inv}[1]{\frac{1}{#1}}
\newcommand{\cross}[0]{\times}

\newcommand{\abs}[1]{\lvert{#1}\rvert}
\newcommand{\norm}[1]{\lVert{#1}\rVert}
\newcommand{\innerprod}[2]{\langle{#1}, {#2}\rangle}
\newcommand{\dotprod}[2]{{#1} \cdot {#2}}
\newcommand{\bdotprod}[2]{\left({#1} \cdot {#2}\right)}
\newcommand{\crossprod}[2]{{#1} \cross {#2}}
\newcommand{\tripleprod}[3]{\dotprod{\left(\crossprod{#1}{#2}\right)}{#3}}

\DeclareMathOperator{\Proj}{Proj}
\DeclareMathOperator{\Span}{span}
\DeclareMathOperator{\Sgn}{sgn}
\DeclareMathOperator{\Area}{Area}
\DeclareMathOperator{\Volume}{Volume}

%
% A few miscellaneous things specific to this document
%
\newcommand{\crossop}[1]{\crossprod{#1}{}}

% R2 vector.
\newcommand{\VectorTwo}[2]{
\begin{bmatrix}
 {#1} \\
 {#2}
\end{bmatrix}
}

\newcommand{\VectorN}[1]{
\begin{bmatrix}
{#1}_1 \\
{#1}_2 \\
\vdots \\
{#1}_N \\
\end{bmatrix}
}

\newcommand{\DETuvij}[4]{
\begin{vmatrix}
 {#1}_{#3} & {#1}_{#4} \\
 {#2}_{#3} & {#2}_{#4}
\end{vmatrix}
}

\newcommand{\DETuvwijk}[6]{
\begin{vmatrix}
 {#1}_{#4} & {#1}_{#5} & {#1}_{#6} \\
 {#2}_{#4} & {#2}_{#5} & {#2}_{#6} \\
 {#3}_{#4} & {#3}_{#5} & {#3}_{#6}
\end{vmatrix}
}

\newcommand{\DETuvwxijkl}[8]{
\begin{vmatrix}
 {#1}_{#5} & {#1}_{#6} & {#1}_{#7} & {#1}_{#8} \\
 {#2}_{#5} & {#2}_{#6} & {#2}_{#7} & {#2}_{#8} \\
 {#3}_{#5} & {#3}_{#6} & {#3}_{#7} & {#3}_{#8} \\
 {#4}_{#5} & {#4}_{#6} & {#4}_{#7} & {#4}_{#8} \\
\end{vmatrix}
}

%\newcommand{\DETuvwxyijklm}[10]{
%\begin{vmatrix}
% {#1}_{#6} & {#1}_{#7} & {#1}_{#8} & {#1}_{#9} & {#1}_{#10} \\
% {#2}_{#6} & {#2}_{#7} & {#2}_{#8} & {#2}_{#9} & {#2}_{#10} \\
% {#3}_{#6} & {#3}_{#7} & {#3}_{#8} & {#3}_{#9} & {#3}_{#10} \\
% {#4}_{#6} & {#4}_{#7} & {#4}_{#8} & {#4}_{#9} & {#4}_{#10} \\
% {#5}_{#6} & {#5}_{#7} & {#5}_{#8} & {#5}_{#9} & {#5}_{#10}
%\end{vmatrix}
%}

% R3 vector.
\newcommand{\VectorThree}[3]{
\begin{bmatrix}
 {#1} \\
 {#2} \\
 {#3}
\end{bmatrix}
}


%%<misc>
%
\newcommand{\Abs}[1]{{\left\lvert{#1}\right\rvert}}
\newcommand{\spacegrad}[0]{\boldsymbol{\nabla}}
\newcommand{\grad}[0]{\nabla}
\newcommand{\LL}[0]{\mathcal{L}}

% == \partial_{#1} {#2}
\newcommand{\PD}[2]{\frac{\partial {#2}}{\partial {#1}}}
% inline variant
\newcommand{\PDi}[2]{{\partial {#2}}/{\partial {#1}}}

\newcommand{\PDD}[3]{\frac{\partial^2 {#3}}{\partial {#1}\partial {#2}}}
%\newcommand{\PDd}[2]{\frac{\partial^2 {#2}}{{\partial{#1}}^2}}
\newcommand{\PDsq}[2]{\frac{\partial^2 {#2}}{(\partial {#1})^2}}

\newcommand{\Partial}[2]{\frac{\partial {#1}}{\partial {#2}}}
\DeclareMathOperator{\RejName}{Rej}
\newcommand{\Rej}[2]{\RejName_{#1}\left( {#2} \right)}
\newcommand{\Rm}[1]{\mathbb{R}^{#1}}
\newcommand{\Cm}[1]{\mathbb{C}^{#1}}
\newcommand{\conj}[0]{{*}}

%</misc>

% <grade selection>
%
\newcommand{\gpgrade}[2] {{\left\langle{{#1}}\right\rangle}_{#2}}

\newcommand{\gpgradezero}[1] {\gpgrade{#1}{}}
%\newcommand{\gpscalargrade}[1] {{\left\langle{{#1}}\right\rangle}}
%\newcommand{\gpgradezero}[1] {\gpgrade{#1}{0}}

%\newcommand{\gpgradeone}[1] {{\left\langle{{#1}}\right\rangle}_{1}}
\newcommand{\gpgradeone}[1] {\gpgrade{#1}{1}}

\newcommand{\gpgradetwo}[1] {\gpgrade{#1}{2}}
\newcommand{\gpgradethree}[1] {\gpgrade{#1}{3}}
\newcommand{\gpgradefour}[1] {\gpgrade{#1}{4}}
%
% </grade selection>



\newcommand{\adot}[0]{{\dot{a}}}
\newcommand{\bdot}[0]{{\dot{b}}}
% taken for centered dot:
%\newcommand{\cdot}[0]{{\dot{c}}}
%\newcommand{\ddot}[0]{{\dot{d}}}
\newcommand{\edot}[0]{{\dot{e}}}
\newcommand{\fdot}[0]{{\dot{f}}}
\newcommand{\gdot}[0]{{\dot{g}}}
\newcommand{\hdot}[0]{{\dot{h}}}
\newcommand{\idot}[0]{{\dot{i}}}
\newcommand{\jdot}[0]{{\dot{j}}}
\newcommand{\kdot}[0]{{\dot{k}}}
\newcommand{\ldot}[0]{{\dot{l}}}
\newcommand{\mdot}[0]{{\dot{m}}}
\newcommand{\ndot}[0]{{\dot{n}}}
%\newcommand{\odot}[0]{{\dot{o}}}
\newcommand{\pdot}[0]{{\dot{p}}}
\newcommand{\qdot}[0]{{\dot{q}}}
\newcommand{\rdot}[0]{{\dot{r}}}
\newcommand{\sdot}[0]{{\dot{s}}}
\newcommand{\tdot}[0]{{\dot{t}}}
\newcommand{\udot}[0]{{\dot{u}}}
\newcommand{\vdot}[0]{{\dot{v}}}
\newcommand{\wdot}[0]{{\dot{w}}}
\newcommand{\xdot}[0]{{\dot{x}}}
\newcommand{\ydot}[0]{{\dot{y}}}
\newcommand{\zdot}[0]{{\dot{z}}}
\newcommand{\addot}[0]{{\ddot{a}}}
\newcommand{\bddot}[0]{{\ddot{b}}}
\newcommand{\cddot}[0]{{\ddot{c}}}
%\newcommand{\dddot}[0]{{\ddot{d}}}
\newcommand{\eddot}[0]{{\ddot{e}}}
\newcommand{\fddot}[0]{{\ddot{f}}}
\newcommand{\gddot}[0]{{\ddot{g}}}
\newcommand{\hddot}[0]{{\ddot{h}}}
\newcommand{\iddot}[0]{{\ddot{i}}}
\newcommand{\jddot}[0]{{\ddot{j}}}
\newcommand{\kddot}[0]{{\ddot{k}}}
\newcommand{\lddot}[0]{{\ddot{l}}}
\newcommand{\mddot}[0]{{\ddot{m}}}
\newcommand{\nddot}[0]{{\ddot{n}}}
\newcommand{\oddot}[0]{{\ddot{o}}}
\newcommand{\pddot}[0]{{\ddot{p}}}
\newcommand{\qddot}[0]{{\ddot{q}}}
\newcommand{\rddot}[0]{{\ddot{r}}}
\newcommand{\sddot}[0]{{\ddot{s}}}
\newcommand{\tddot}[0]{{\ddot{t}}}
\newcommand{\uddot}[0]{{\ddot{u}}}
\newcommand{\vddot}[0]{{\ddot{v}}}
\newcommand{\wddot}[0]{{\ddot{w}}}
\newcommand{\xddot}[0]{{\ddot{x}}}
\newcommand{\yddot}[0]{{\ddot{y}}}
\newcommand{\zddot}[0]{{\ddot{z}}}

%<bold and dot greek symbols>
%

\newcommand{\Deltadot}[0]{{\dot{\Delta}}}
\newcommand{\Gammadot}[0]{{\dot{\Gamma}}}
\newcommand{\Lambdadot}[0]{{\dot{\Lambda}}}
\newcommand{\Omegadot}[0]{{\dot{\Omega}}}
\newcommand{\Phidot}[0]{{\dot{\Phi}}}
\newcommand{\Pidot}[0]{{\dot{\Pi}}}
\newcommand{\Psidot}[0]{{\dot{\Psi}}}
\newcommand{\Sigmadot}[0]{{\dot{\Sigma}}}
\newcommand{\Thetadot}[0]{{\dot{\Theta}}}
\newcommand{\Upsilondot}[0]{{\dot{\Upsilon}}}
\newcommand{\Xidot}[0]{{\dot{\Xi}}}
\newcommand{\alphadot}[0]{{\dot{\alpha}}}
\newcommand{\betadot}[0]{{\dot{\beta}}}
\newcommand{\chidot}[0]{{\dot{\chi}}}
\newcommand{\deltadot}[0]{{\dot{\delta}}}
\newcommand{\epsilondot}[0]{{\dot{\epsilon}}}
\newcommand{\etadot}[0]{{\dot{\eta}}}
\newcommand{\gammadot}[0]{{\dot{\gamma}}}
\newcommand{\kappadot}[0]{{\dot{\kappa}}}
\newcommand{\lambdadot}[0]{{\dot{\lambda}}}
\newcommand{\mudot}[0]{{\dot{\mu}}}
\newcommand{\nudot}[0]{{\dot{\nu}}}
\newcommand{\omegadot}[0]{{\dot{\omega}}}
\newcommand{\phidot}[0]{{\dot{\phi}}}
\newcommand{\pidot}[0]{{\dot{\pi}}}
\newcommand{\psidot}[0]{{\dot{\psi}}}
\newcommand{\rhodot}[0]{{\dot{\rho}}}
\newcommand{\sigmadot}[0]{{\dot{\sigma}}}
\newcommand{\taudot}[0]{{\dot{\tau}}}
\newcommand{\thetadot}[0]{{\dot{\theta}}}
\newcommand{\upsilondot}[0]{{\dot{\upsilon}}}
\newcommand{\varepsilondot}[0]{{\dot{\varepsilon}}}
\newcommand{\varphidot}[0]{{\dot{\varphi}}}
\newcommand{\varpidot}[0]{{\dot{\varpi}}}
\newcommand{\varrhodot}[0]{{\dot{\varrho}}}
\newcommand{\varsigmadot}[0]{{\dot{\varsigma}}}
\newcommand{\varthetadot}[0]{{\dot{\vartheta}}}
\newcommand{\xidot}[0]{{\dot{\xi}}}
\newcommand{\zetadot}[0]{{\dot{\zeta}}}

\newcommand{\Deltaddot}[0]{{\ddot{\Delta}}}
\newcommand{\Gammaddot}[0]{{\ddot{\Gamma}}}
\newcommand{\Lambdaddot}[0]{{\ddot{\Lambda}}}
\newcommand{\Omegaddot}[0]{{\ddot{\Omega}}}
\newcommand{\Phiddot}[0]{{\ddot{\Phi}}}
\newcommand{\Piddot}[0]{{\ddot{\Pi}}}
\newcommand{\Psiddot}[0]{{\ddot{\Psi}}}
\newcommand{\Sigmaddot}[0]{{\ddot{\Sigma}}}
\newcommand{\Thetaddot}[0]{{\ddot{\Theta}}}
\newcommand{\Upsilonddot}[0]{{\ddot{\Upsilon}}}
\newcommand{\Xiddot}[0]{{\ddot{\Xi}}}
\newcommand{\alphaddot}[0]{{\ddot{\alpha}}}
\newcommand{\betaddot}[0]{{\ddot{\beta}}}
\newcommand{\chiddot}[0]{{\ddot{\chi}}}
\newcommand{\deltaddot}[0]{{\ddot{\delta}}}
\newcommand{\epsilonddot}[0]{{\ddot{\epsilon}}}
\newcommand{\etaddot}[0]{{\ddot{\eta}}}
\newcommand{\gammaddot}[0]{{\ddot{\gamma}}}
\newcommand{\kappaddot}[0]{{\ddot{\kappa}}}
\newcommand{\lambdaddot}[0]{{\ddot{\lambda}}}
\newcommand{\muddot}[0]{{\ddot{\mu}}}
\newcommand{\nuddot}[0]{{\ddot{\nu}}}
\newcommand{\omegaddot}[0]{{\ddot{\omega}}}
\newcommand{\phiddot}[0]{{\ddot{\phi}}}
\newcommand{\piddot}[0]{{\ddot{\pi}}}
\newcommand{\psiddot}[0]{{\ddot{\psi}}}
\newcommand{\rhoddot}[0]{{\ddot{\rho}}}
\newcommand{\sigmaddot}[0]{{\ddot{\sigma}}}
\newcommand{\tauddot}[0]{{\ddot{\tau}}}
\newcommand{\thetaddot}[0]{{\ddot{\theta}}}
\newcommand{\upsilonddot}[0]{{\ddot{\upsilon}}}
\newcommand{\varepsilonddot}[0]{{\ddot{\varepsilon}}}
\newcommand{\varphiddot}[0]{{\ddot{\varphi}}}
\newcommand{\varpiddot}[0]{{\ddot{\varpi}}}
\newcommand{\varrhoddot}[0]{{\ddot{\varrho}}}
\newcommand{\varsigmaddot}[0]{{\ddot{\varsigma}}}
\newcommand{\varthetaddot}[0]{{\ddot{\vartheta}}}
\newcommand{\xiddot}[0]{{\ddot{\xi}}}
\newcommand{\zetaddot}[0]{{\ddot{\zeta}}}

\newcommand{\BDelta}[0]{\boldsymbol{\Delta}}
\newcommand{\BGamma}[0]{\boldsymbol{\Gamma}}
\newcommand{\BLambda}[0]{\boldsymbol{\Lambda}}
\newcommand{\BOmega}[0]{\boldsymbol{\Omega}}
\newcommand{\BPhi}[0]{\boldsymbol{\Phi}}
\newcommand{\BPi}[0]{\boldsymbol{\Pi}}
\newcommand{\BPsi}[0]{\boldsymbol{\Psi}}
\newcommand{\BSigma}[0]{\boldsymbol{\Sigma}}
\newcommand{\BTheta}[0]{\boldsymbol{\Theta}}
\newcommand{\BUpsilon}[0]{\boldsymbol{\Upsilon}}
\newcommand{\BXi}[0]{\boldsymbol{\Xi}}
\newcommand{\Balpha}[0]{\boldsymbol{\alpha}}
\newcommand{\Bbeta}[0]{\boldsymbol{\beta}}
\newcommand{\Bchi}[0]{\boldsymbol{\chi}}
\newcommand{\Bdelta}[0]{\boldsymbol{\delta}}
\newcommand{\Bepsilon}[0]{\boldsymbol{\epsilon}}
\newcommand{\Beta}[0]{\boldsymbol{\eta}}
\newcommand{\Bgamma}[0]{\boldsymbol{\gamma}}
\newcommand{\Bkappa}[0]{\boldsymbol{\kappa}}
\newcommand{\Blambda}[0]{\boldsymbol{\lambda}}
\newcommand{\Bmu}[0]{\boldsymbol{\mu}}
\newcommand{\Bnu}[0]{\boldsymbol{\nu}}
%\newcommand{\Bomega}[0]{\boldsymbol{\omega}}
\newcommand{\Bphi}[0]{\boldsymbol{\phi}}
\newcommand{\Bpi}[0]{\boldsymbol{\pi}}
\newcommand{\Bpsi}[0]{\boldsymbol{\psi}}
\newcommand{\Brho}[0]{\boldsymbol{\rho}}
\newcommand{\Bsigma}[0]{\boldsymbol{\sigma}}
%\newcommand{\Btau}[0]{\boldsymbol{\tau}}
%\newcommand{\Btheta}[0]{\boldsymbol{\theta}}
\newcommand{\Bupsilon}[0]{\boldsymbol{\upsilon}}
\newcommand{\Bvarepsilon}[0]{\boldsymbol{\varepsilon}}
\newcommand{\Bvarphi}[0]{\boldsymbol{\varphi}}
\newcommand{\Bvarpi}[0]{\boldsymbol{\varpi}}
\newcommand{\Bvarrho}[0]{\boldsymbol{\varrho}}
\newcommand{\Bvarsigma}[0]{\boldsymbol{\varsigma}}
\newcommand{\Bvartheta}[0]{\boldsymbol{\vartheta}}
\newcommand{\Bxi}[0]{\boldsymbol{\xi}}
\newcommand{\Bzeta}[0]{\boldsymbol{\zeta}}
%
%</bold and dot greek symbols>
%<infrequent>
%
%\newcommand{\AreaOp}[1]{\AName_{#1}}
%\newcommand{\Babs}[0]{\abs{\BB}}
%\newcommand{\Bcap}[0]{\hat{\BB}}
%\newcommand{\BrPrimeRej}[0]{\rcap(\rcap \wedge \Br')}
%\newcommand{\CA}[0]{\mathcal{A}}
%\newcommand{\Cos}[1]{\cos{\left({#1}\right)}}
%\newcommand{\Det}[1] {\abs{#1}}
%\newcommand{\Dsq}[2] {\frac {\partial^2 {#1}} {\partial {#2}^2}}
%\newcommand{\Exp}[1]{\exp{\left({#1}\right)}}
%\newcommand{\Norm}[1]{\left\lVert{#1}\right\rVert}
%\newcommand{\Sin}[1]{\sin{\left({#1}\right)}}
%\newcommand{\T}[0]{\text{T}}
%\newcommand{\VolumeOp}[1]{\VName_{#1}}
%\newcommand{\agrad}[0]{\Ba \cdot \nabla}
%\newcommand{\alphacap}[0]{\hat{\boldsymbol{\alpha}}}
%\newcommand{\Fcap}[0]{\hat{\BF}}
%\newcommand{\bithree}[0]{{\Bi}_3}
%\newcommand{\bxa}[0]{\Bx\Ba}
%\newcommand{\coordvec}[2]{
%\newcommand{\costheta}[0]{\acap \cdot \xcap}
%\newcommand{\ddt}[1]{\ddot{#1}}
%\newcommand{\ddu}[1] {\frac {d{#1}} {du}}
%\newcommand{\dsqxj}[2] {\frac {\partial^2 {#1}} {\partial {x_{#2}}^2}}
%\newcommand{\dtheta}[1]{\frac{d {#1}}{d \theta}}
%\newcommand{\dt}[1]{\dot{#1}}
%\newcommand{\dt}[1]{\frac{d {#1}}{dt}}
%\newcommand{\dxj}[2] {\frac {\partial {#1}} {\partial {x_{#2}}}}
%\newcommand{\halfPhi}[0]{\frac{\phi}{2}}
%\newcommand{\half}[0]{\inv{2}}
%\newcommand{\inv}[1]{\frac{1}{#1}}
%\newcommand{\laplacian}[0]{\nabla^2}
%\newcommand{\matrixoftx}[3]{
%\newcommand{\nrrp}[0]{\norm{\rcap \wedge \Br'}}
%\newcommand{\oiint}{\bigcirc \hspace{-1.4em} \int \hspace{-.8em} \int}
%\newcommand{\transpose}[1]{{#1}^{\text{T}}}
%\newcommand{\transpose}[1]{{{#1}^{\TextTranspose}}}
%\newcommand{\transpose}[1]{{{#1}^{\text{T}}}}
%\newcommand{\barA}[0]{\bar{A}}
%\newcommand{\qbar}[0]{\bar{q}}
%\newcommand{\qdotbar}[0]{\dot{\bar{q}}}
%
%</infrequent>





%\usepackage[bookmarks=true]{hyperref}

%\usepackage{color,cite,graphicx}
   % use colour in the document, put your citations as [1-4]
   % rather than [1,2,3,4] (it looks nicer, and the extended LaTeX2e
   % graphics package. 
%\usepackage{latexsym,amssymb,epsf} % do not remember if these are
   % needed, but their inclusion can not do any damage


\chapter{fletcher64}
\label{chap:fletcher}
%\author{Peeter Joot \quad peeter.joot@gmail.com }
\date{ March dd, 2009.  fletcher.tex }

%\begin{document}

%\maketitle{}
%\tableofcontents
\section{Fast modulus by one less than power of 2?}

Statement in some code is

``\((x \& 0xFFFFFFFF) + (x \rightshift 32)\) is a faster way to do \(x \Mod 0xFFFFFFFF\), except the max result may be
\(0x1FFFFFFFE\), so need another round of this.''

How can this be justified?

\subsection{Translating the bit ops to math ops}

Suppose we have 

\begin{equation}\label{eqn:fletcher:modPower2}
\begin{aligned}
x = y \times 2^B + r \quad \mbox{where \(r \le 2^{B-1}\)}
\end{aligned}
\end{equation}

then we have

\begin{equation}\label{eqn:fletcher:22}
\begin{aligned}
y &= x \Div 2^{B} = x \rightshift B \\
r &= x \Mod 2^B = x \& 2^{B-1}
\end{aligned}
\end{equation}

So we have
\begin{equation}\label{eqn:fletcher:42}
\begin{aligned}
x 
&= (x \& 2^{B-1}) \times 2^B + x \rightshift B \\
&= (x \Mod 2^B) \times 2^B + x \Div 2^B \\
\end{aligned}
\end{equation}

\subsection{The shift expression}

From \eqnref{eqn:fletcher:modPower2} we have

\begin{equation}\label{eqn:fletcher:62}
\begin{aligned}
x 
&= y \times 2^B + r \\
&= y \times (2^B -1) + y + r \\
\end{aligned}
\end{equation}

Therefore we have

\begin{equation}\label{eqn:fletcher:82}
\begin{aligned}
x \Mod (2^B -1) &= (y + r) \Mod (2^B -1) \\
\end{aligned}
\end{equation}

The net effect is that the original desired modulus calculation has been reduced to one of a lesser numerical order.  In particular, for the
code in question we have \(B=32\), and an unsigned 64-bit value \(x\).  This means the value \(y \le 0xFFFFFFFF\), whereas \(r \le 0xFFFFFFFF\), so the new value
\(y + r \le 0x1FFFFFFFE\) with up to 31 bits knocked off.  A second round of reduction can give a value no more than \(0xFFFFFFFE + 1 = 0xFFFFFFFF\).

The conclusion is that this ``fast modulus'' is not exactly a modulus replacement.  Instead it is either the true modulus (ie: \(x \Mod 0xFFFFFFFF < 0xFFFFFFFF\)), but may for some \(x\) actually be equal to \(0xFFFFFFFF\).  This almost modulus operation was however sufficient in the code in question, since the idea was only to reduce the magnitude to a 32-bit quantity.

\section{Fletcher64}

A 32-bit word extension of the \href{http://en.wikipedia.org/wiki/Fletcher%27s_checksum}{wikipedia fletcher algorithm} is

\begin{equation}\label{eqn:fletcher:102}
\begin{aligned}
&\text{while} ( i < n )  \\
&\quad   S_1 += A_i \\
&\quad   S_2 += S_1 \\
\\
&C = \Mod_{32}(S_2) | \Mod_{32}(S_1)
\end{aligned}
\end{equation}

Where \(\Mod_{32}(x)\) is the fast modulus operation described above.

The summation equivalent to the loop above is
\begin{equation}\label{eqn:fletcher:122}
\begin{aligned}
S_1 &= \sum_{i=0}^{n-1} A_i \\
S_2 &= \sum_{i=0}^{n-1} (n-i)A_i \\
\end{aligned}
\end{equation}

Note that the above loop assumes that the data is constrained enough that no arithmetic overflows occur within the loop iteration.
Strictly speaking a more accurate transcription of the wiki fletcher32 algorithm using 64-bit accumulators is

\begin{equation}\label{eqn:fletcher:142}
\begin{aligned}
&\text{while} ( i < n )  \\
&\quad   S_1 += A_i \\
&\quad   S_2 += S_1 \\
&\quad   S_1 = \Mod_{32}(S_1) \\
&\quad   S_2 = \Mod_{32}(S_2) \\
\\
&C = S_2 | S_1
\end{aligned}
\end{equation}

Optimizations (like the use of the magic number 360 in the wikipedia article) are possible to avoid the modulus reduction in each iteration of the loop.

In particular, the modulus accumulation of residues is equivalent to the modulus of the sums.  Say

\begin{equation}\label{eqn:fletcher:162}
\begin{aligned}
a_1 &= (a_1 \Div m ) m + r_1 \\
a_2 &= (a_2 \Div m ) m + r_2
\end{aligned}
\end{equation}

So we have
\begin{equation}\label{eqn:fletcher:182}
\begin{aligned}
a_1 + a_2 &= (a_1 \Div m + a_2 \Div m ) m + r_1 + r_2 \\
\implies \\
(a_1 + a_2) \Mod m &= (r_1 + r_2) \Mod m \\
\end{aligned}
\end{equation}

Or more generally

\begin{equation}\label{eqn:fletcher:202}
\begin{aligned}
\left(\sum_i a_i \right) \Mod m &= \sum_i (a_i \Mod m) \Mod m \\
\end{aligned}
\end{equation}

\subsection{Max loops before truncation}

With 32K page sizes and 4 byte words we have a max number of iterations for the loop of

\begin{equation}\label{eqn:fletcher:222}
\begin{aligned}
\frac{32 \times 1024 }{4} = 8192 
\end{aligned}
\end{equation}

Assuming the biggest size value of \(M = A_i = 0xFFFFFFFF\) when do our summation variables wrap?

\begin{equation}\label{eqn:fletcher:242}
\begin{aligned}
S_1 &= \sum_{i=0}^{n-1} A_i \le n M \\
S_2 &= \sum_{i=0}^{n-1} (n-i)A_i <= M \sum_{i=0}^{n-1} (n-i) = M n(n-1)/2 < M n^2/2
\end{aligned}
\end{equation}

The overflow point is dominated by the \(S_2\) sum, so if our max accumulator value is \(M_a\) we want

\begin{equation}\label{eqn:fletcher:262}
\begin{aligned}
M_a \ge M n^2/2 \\
\implies \\
\sqrt{\frac{ 2 M_a }{ M}} \ge n \\
\end{aligned}
\end{equation}

For 64-bit max accumulator and 32-bit words we have
\begin{equation}\label{eqn:fletcher:282}
\begin{aligned}
\sqrt{2 \frac{2^{64} - 1}{2^{32} - 1}} = 92 681.9
\end{aligned}
\end{equation}

which is far bigger than \(8192\).

% google calc:
%log( (2^64-1)/8192 + 1)/log( 2 )

The biggest number of bits in the word that we can use without having to worry about carries are as follows

\begin{tabular}{|l|l|l|}
\hline
Page Size & N & bits \\
\hline
4K & 712 & 46 \\
8K & 1524 & 43 \\
16K & 3196 & 41 \\
32K & 6721 & 39 \\
\hline
\end{tabular}


%\section{Scratch notes}
%
%\begin{align*}
%f(x)
%&= (x \& 2^{B-1}) + x \rightshift B \\
%&= (x \Mod 2^B) + x \Div 2^B \\
%&\questionEquals x \Mod (2^B-1)
%\end{align*}

%\bibliographystyle{plainnat}
%\bibliography{myrefs}

%\end{document}


\part{Fourier treatments}
\documentclass{article}

\usepackage{amsmath}
\usepackage{mathpazo}

%
% shorthand for bold symbols, convenient for vectors and matrices
%
\newcommand{\Ba}[0]{\mathbf{a}}
\newcommand{\Bb}[0]{\mathbf{b}}
\newcommand{\Bc}[0]{\mathbf{c}}
\newcommand{\Bd}[0]{\mathbf{d}}
\newcommand{\Be}[0]{\mathbf{e}}
\newcommand{\Bf}[0]{\mathbf{f}}
\newcommand{\Bg}[0]{\mathbf{g}}
\newcommand{\Bh}[0]{\mathbf{h}}
\newcommand{\Bi}[0]{\mathbf{i}}
\newcommand{\Bj}[0]{\mathbf{j}}
\newcommand{\Bk}[0]{\mathbf{k}}
\newcommand{\Bl}[0]{\mathbf{l}}
\newcommand{\Bm}[0]{\mathbf{m}}
\newcommand{\Bn}[0]{\mathbf{n}}
\newcommand{\Bo}[0]{\mathbf{o}}
\newcommand{\Bp}[0]{\mathbf{p}}
\newcommand{\Bq}[0]{\mathbf{q}}
\newcommand{\Br}[0]{\mathbf{r}}
\newcommand{\Bs}[0]{\mathbf{s}}
\newcommand{\Bt}[0]{\mathbf{t}}
\newcommand{\Bu}[0]{\mathbf{u}}
\newcommand{\Bv}[0]{\mathbf{v}}
\newcommand{\Bw}[0]{\mathbf{w}}
\newcommand{\Bx}[0]{\mathbf{x}}
\newcommand{\By}[0]{\mathbf{y}}
\newcommand{\Bz}[0]{\mathbf{z}}
\newcommand{\BA}[0]{\mathbf{A}}
\newcommand{\BB}[0]{\mathbf{B}}
\newcommand{\BC}[0]{\mathbf{C}}
\newcommand{\BD}[0]{\mathbf{D}}
\newcommand{\BE}[0]{\mathbf{E}}
\newcommand{\BF}[0]{\mathbf{F}}
\newcommand{\BG}[0]{\mathbf{G}}
\newcommand{\BH}[0]{\mathbf{H}}
\newcommand{\BI}[0]{\mathbf{I}}
\newcommand{\BJ}[0]{\mathbf{J}}
\newcommand{\BK}[0]{\mathbf{K}}
\newcommand{\BL}[0]{\mathbf{L}}
\newcommand{\BM}[0]{\mathbf{M}}
\newcommand{\BN}[0]{\mathbf{N}}
\newcommand{\BO}[0]{\mathbf{O}}
\newcommand{\BP}[0]{\mathbf{P}}
\newcommand{\BQ}[0]{\mathbf{Q}}
\newcommand{\BR}[0]{\mathbf{R}}
\newcommand{\BS}[0]{\mathbf{S}}
\newcommand{\BT}[0]{\mathbf{T}}
\newcommand{\BU}[0]{\mathbf{U}}
\newcommand{\BV}[0]{\mathbf{V}}
\newcommand{\BW}[0]{\mathbf{W}}
\newcommand{\BX}[0]{\mathbf{X}}
\newcommand{\BY}[0]{\mathbf{Y}}
\newcommand{\BZ}[0]{\mathbf{Z}}

\newcommand{\Bzero}[0]{\mathbf{0}}
\newcommand{\Btheta}[0]{\boldsymbol{\theta}}
\newcommand{\Btau}[0]{\boldsymbol{\tau}}
\newcommand{\Bomega}[0]{\boldsymbol{\omega}}

%
% shorthand for unit vectors
%
\newcommand{\acap}[0]{\hat{\Ba}}
\newcommand{\bcap}[0]{\hat{\Bb}}
\newcommand{\ccap}[0]{\hat{\Bc}}
\newcommand{\dcap}[0]{\hat{\Bd}}
\newcommand{\ecap}[0]{\hat{\Be}}
\newcommand{\fcap}[0]{\hat{\Bf}}
\newcommand{\gcap}[0]{\hat{\Bg}}
\newcommand{\hcap}[0]{\hat{\Bh}}
\newcommand{\icap}[0]{\hat{\Bi}}
\newcommand{\jcap}[0]{\hat{\Bj}}
\newcommand{\kcap}[0]{\hat{\Bk}}
\newcommand{\lcap}[0]{\hat{\Bl}}
\newcommand{\mcap}[0]{\hat{\Bm}}
\newcommand{\ncap}[0]{\hat{\Bn}}
\newcommand{\ocap}[0]{\hat{\Bo}}
\newcommand{\pcap}[0]{\hat{\Bp}}
\newcommand{\qcap}[0]{\hat{\Bq}}
\newcommand{\rcap}[0]{\hat{\Br}}
\newcommand{\scap}[0]{\hat{\Bs}}
\newcommand{\tcap}[0]{\hat{\Bt}}
\newcommand{\ucap}[0]{\hat{\Bu}}
\newcommand{\vcap}[0]{\hat{\Bv}}
\newcommand{\wcap}[0]{\hat{\Bw}}
\newcommand{\xcap}[0]{\hat{\Bx}}
\newcommand{\ycap}[0]{\hat{\By}}
\newcommand{\zcap}[0]{\hat{\Bz}}
\newcommand{\thetacap}[0]{\hat{\Btheta}}

%
% to write R^n and C^n in a distinguishable fashion.  Perhaps change this
% to the double lined characters upon figuring out how to do so.
%
\newcommand{\C}[1]{$\mathbb{C}^{#1}$}
\newcommand{\R}[1]{$\mathbb{R}^{#1}$}

%
% various generally useful helpers
%

% derivative of #1 wrt. #2:
\newcommand{\D}[2] {\frac {d#2} {d#1}}

\newcommand{\inv}[1]{\frac{1}{#1}}
\newcommand{\cross}[0]{\times}

\newcommand{\abs}[1]{\lvert{#1}\rvert}
\newcommand{\norm}[1]{\lVert{#1}\rVert}
\newcommand{\innerprod}[2]{\langle{#1}, {#2}\rangle}
\newcommand{\dotprod}[2]{{#1} \cdot {#2}}
\newcommand{\bdotprod}[2]{\left({#1} \cdot {#2}\right)}
\newcommand{\crossprod}[2]{{#1} \cross {#2}}
\newcommand{\tripleprod}[3]{\dotprod{\left(\crossprod{#1}{#2}\right)}{#3}}

\DeclareMathOperator{\Proj}{Proj}
\DeclareMathOperator{\Span}{span}
\DeclareMathOperator{\Sgn}{sgn}
\DeclareMathOperator{\Area}{Area}
\DeclareMathOperator{\Volume}{Volume}

%
% A few miscellaneous things specific to this document
%
\newcommand{\crossop}[1]{\crossprod{#1}{}}

% R2 vector.
\newcommand{\VectorTwo}[2]{
\begin{bmatrix}
 {#1} \\
 {#2}
\end{bmatrix}
}

\newcommand{\VectorN}[1]{
\begin{bmatrix}
{#1}_1 \\
{#1}_2 \\
\vdots \\
{#1}_N \\
\end{bmatrix}
}

\newcommand{\DETuvij}[4]{
\begin{vmatrix}
 {#1}_{#3} & {#1}_{#4} \\
 {#2}_{#3} & {#2}_{#4}
\end{vmatrix}
}

\newcommand{\DETuvwijk}[6]{
\begin{vmatrix}
 {#1}_{#4} & {#1}_{#5} & {#1}_{#6} \\
 {#2}_{#4} & {#2}_{#5} & {#2}_{#6} \\
 {#3}_{#4} & {#3}_{#5} & {#3}_{#6}
\end{vmatrix}
}

\newcommand{\DETuvwxijkl}[8]{
\begin{vmatrix}
 {#1}_{#5} & {#1}_{#6} & {#1}_{#7} & {#1}_{#8} \\
 {#2}_{#5} & {#2}_{#6} & {#2}_{#7} & {#2}_{#8} \\
 {#3}_{#5} & {#3}_{#6} & {#3}_{#7} & {#3}_{#8} \\
 {#4}_{#5} & {#4}_{#6} & {#4}_{#7} & {#4}_{#8} \\
\end{vmatrix}
}

%\newcommand{\DETuvwxyijklm}[10]{
%\begin{vmatrix}
% {#1}_{#6} & {#1}_{#7} & {#1}_{#8} & {#1}_{#9} & {#1}_{#10} \\
% {#2}_{#6} & {#2}_{#7} & {#2}_{#8} & {#2}_{#9} & {#2}_{#10} \\
% {#3}_{#6} & {#3}_{#7} & {#3}_{#8} & {#3}_{#9} & {#3}_{#10} \\
% {#4}_{#6} & {#4}_{#7} & {#4}_{#8} & {#4}_{#9} & {#4}_{#10} \\
% {#5}_{#6} & {#5}_{#7} & {#5}_{#8} & {#5}_{#9} & {#5}_{#10}
%\end{vmatrix}
%}

% R3 vector.
\newcommand{\VectorThree}[3]{
\begin{bmatrix}
 {#1} \\
 {#2} \\
 {#3}
\end{bmatrix}
}


%<misc>
%
\newcommand{\Abs}[1]{{\left\lvert{#1}\right\rvert}}
\newcommand{\spacegrad}[0]{\boldsymbol{\nabla}}
\newcommand{\grad}[0]{\nabla}
\newcommand{\LL}[0]{\mathcal{L}}

% == \partial_{#1} {#2}
\newcommand{\PD}[2]{\frac{\partial {#2}}{\partial {#1}}}
% inline variant
\newcommand{\PDi}[2]{{\partial {#2}}/{\partial {#1}}}

\newcommand{\PDD}[3]{\frac{\partial^2 {#3}}{\partial {#1}\partial {#2}}}
%\newcommand{\PDd}[2]{\frac{\partial^2 {#2}}{{\partial{#1}}^2}}
\newcommand{\PDsq}[2]{\frac{\partial^2 {#2}}{(\partial {#1})^2}}

\newcommand{\Partial}[2]{\frac{\partial {#1}}{\partial {#2}}}
\DeclareMathOperator{\RejName}{Rej}
\newcommand{\Rej}[2]{\RejName_{#1}\left( {#2} \right)}
\newcommand{\Rm}[1]{\mathbb{R}^{#1}}
\newcommand{\Cm}[1]{\mathbb{C}^{#1}}
\newcommand{\conj}[0]{{*}}

%</misc>

% <grade selection>
%
\newcommand{\gpgrade}[2] {{\left\langle{{#1}}\right\rangle}_{#2}}

\newcommand{\gpgradezero}[1] {\gpgrade{#1}{}}
%\newcommand{\gpscalargrade}[1] {{\left\langle{{#1}}\right\rangle}}
%\newcommand{\gpgradezero}[1] {\gpgrade{#1}{0}}

%\newcommand{\gpgradeone}[1] {{\left\langle{{#1}}\right\rangle}_{1}}
\newcommand{\gpgradeone}[1] {\gpgrade{#1}{1}}

\newcommand{\gpgradetwo}[1] {\gpgrade{#1}{2}}
\newcommand{\gpgradethree}[1] {\gpgrade{#1}{3}}
\newcommand{\gpgradefour}[1] {\gpgrade{#1}{4}}
%
% </grade selection>



\newcommand{\adot}[0]{{\dot{a}}}
\newcommand{\bdot}[0]{{\dot{b}}}
% taken for centered dot:
%\newcommand{\cdot}[0]{{\dot{c}}}
%\newcommand{\ddot}[0]{{\dot{d}}}
\newcommand{\edot}[0]{{\dot{e}}}
\newcommand{\fdot}[0]{{\dot{f}}}
\newcommand{\gdot}[0]{{\dot{g}}}
\newcommand{\hdot}[0]{{\dot{h}}}
\newcommand{\idot}[0]{{\dot{i}}}
\newcommand{\jdot}[0]{{\dot{j}}}
\newcommand{\kdot}[0]{{\dot{k}}}
\newcommand{\ldot}[0]{{\dot{l}}}
\newcommand{\mdot}[0]{{\dot{m}}}
\newcommand{\ndot}[0]{{\dot{n}}}
%\newcommand{\odot}[0]{{\dot{o}}}
\newcommand{\pdot}[0]{{\dot{p}}}
\newcommand{\qdot}[0]{{\dot{q}}}
\newcommand{\rdot}[0]{{\dot{r}}}
\newcommand{\sdot}[0]{{\dot{s}}}
\newcommand{\tdot}[0]{{\dot{t}}}
\newcommand{\udot}[0]{{\dot{u}}}
\newcommand{\vdot}[0]{{\dot{v}}}
\newcommand{\wdot}[0]{{\dot{w}}}
\newcommand{\xdot}[0]{{\dot{x}}}
\newcommand{\ydot}[0]{{\dot{y}}}
\newcommand{\zdot}[0]{{\dot{z}}}
\newcommand{\addot}[0]{{\ddot{a}}}
\newcommand{\bddot}[0]{{\ddot{b}}}
\newcommand{\cddot}[0]{{\ddot{c}}}
%\newcommand{\dddot}[0]{{\ddot{d}}}
\newcommand{\eddot}[0]{{\ddot{e}}}
\newcommand{\fddot}[0]{{\ddot{f}}}
\newcommand{\gddot}[0]{{\ddot{g}}}
\newcommand{\hddot}[0]{{\ddot{h}}}
\newcommand{\iddot}[0]{{\ddot{i}}}
\newcommand{\jddot}[0]{{\ddot{j}}}
\newcommand{\kddot}[0]{{\ddot{k}}}
\newcommand{\lddot}[0]{{\ddot{l}}}
\newcommand{\mddot}[0]{{\ddot{m}}}
\newcommand{\nddot}[0]{{\ddot{n}}}
\newcommand{\oddot}[0]{{\ddot{o}}}
\newcommand{\pddot}[0]{{\ddot{p}}}
\newcommand{\qddot}[0]{{\ddot{q}}}
\newcommand{\rddot}[0]{{\ddot{r}}}
\newcommand{\sddot}[0]{{\ddot{s}}}
\newcommand{\tddot}[0]{{\ddot{t}}}
\newcommand{\uddot}[0]{{\ddot{u}}}
\newcommand{\vddot}[0]{{\ddot{v}}}
\newcommand{\wddot}[0]{{\ddot{w}}}
\newcommand{\xddot}[0]{{\ddot{x}}}
\newcommand{\yddot}[0]{{\ddot{y}}}
\newcommand{\zddot}[0]{{\ddot{z}}}

%<bold and dot greek symbols>
%

\newcommand{\Deltadot}[0]{{\dot{\Delta}}}
\newcommand{\Gammadot}[0]{{\dot{\Gamma}}}
\newcommand{\Lambdadot}[0]{{\dot{\Lambda}}}
\newcommand{\Omegadot}[0]{{\dot{\Omega}}}
\newcommand{\Phidot}[0]{{\dot{\Phi}}}
\newcommand{\Pidot}[0]{{\dot{\Pi}}}
\newcommand{\Psidot}[0]{{\dot{\Psi}}}
\newcommand{\Sigmadot}[0]{{\dot{\Sigma}}}
\newcommand{\Thetadot}[0]{{\dot{\Theta}}}
\newcommand{\Upsilondot}[0]{{\dot{\Upsilon}}}
\newcommand{\Xidot}[0]{{\dot{\Xi}}}
\newcommand{\alphadot}[0]{{\dot{\alpha}}}
\newcommand{\betadot}[0]{{\dot{\beta}}}
\newcommand{\chidot}[0]{{\dot{\chi}}}
\newcommand{\deltadot}[0]{{\dot{\delta}}}
\newcommand{\epsilondot}[0]{{\dot{\epsilon}}}
\newcommand{\etadot}[0]{{\dot{\eta}}}
\newcommand{\gammadot}[0]{{\dot{\gamma}}}
\newcommand{\kappadot}[0]{{\dot{\kappa}}}
\newcommand{\lambdadot}[0]{{\dot{\lambda}}}
\newcommand{\mudot}[0]{{\dot{\mu}}}
\newcommand{\nudot}[0]{{\dot{\nu}}}
\newcommand{\omegadot}[0]{{\dot{\omega}}}
\newcommand{\phidot}[0]{{\dot{\phi}}}
\newcommand{\pidot}[0]{{\dot{\pi}}}
\newcommand{\psidot}[0]{{\dot{\psi}}}
\newcommand{\rhodot}[0]{{\dot{\rho}}}
\newcommand{\sigmadot}[0]{{\dot{\sigma}}}
\newcommand{\taudot}[0]{{\dot{\tau}}}
\newcommand{\thetadot}[0]{{\dot{\theta}}}
\newcommand{\upsilondot}[0]{{\dot{\upsilon}}}
\newcommand{\varepsilondot}[0]{{\dot{\varepsilon}}}
\newcommand{\varphidot}[0]{{\dot{\varphi}}}
\newcommand{\varpidot}[0]{{\dot{\varpi}}}
\newcommand{\varrhodot}[0]{{\dot{\varrho}}}
\newcommand{\varsigmadot}[0]{{\dot{\varsigma}}}
\newcommand{\varthetadot}[0]{{\dot{\vartheta}}}
\newcommand{\xidot}[0]{{\dot{\xi}}}
\newcommand{\zetadot}[0]{{\dot{\zeta}}}

\newcommand{\Deltaddot}[0]{{\ddot{\Delta}}}
\newcommand{\Gammaddot}[0]{{\ddot{\Gamma}}}
\newcommand{\Lambdaddot}[0]{{\ddot{\Lambda}}}
\newcommand{\Omegaddot}[0]{{\ddot{\Omega}}}
\newcommand{\Phiddot}[0]{{\ddot{\Phi}}}
\newcommand{\Piddot}[0]{{\ddot{\Pi}}}
\newcommand{\Psiddot}[0]{{\ddot{\Psi}}}
\newcommand{\Sigmaddot}[0]{{\ddot{\Sigma}}}
\newcommand{\Thetaddot}[0]{{\ddot{\Theta}}}
\newcommand{\Upsilonddot}[0]{{\ddot{\Upsilon}}}
\newcommand{\Xiddot}[0]{{\ddot{\Xi}}}
\newcommand{\alphaddot}[0]{{\ddot{\alpha}}}
\newcommand{\betaddot}[0]{{\ddot{\beta}}}
\newcommand{\chiddot}[0]{{\ddot{\chi}}}
\newcommand{\deltaddot}[0]{{\ddot{\delta}}}
\newcommand{\epsilonddot}[0]{{\ddot{\epsilon}}}
\newcommand{\etaddot}[0]{{\ddot{\eta}}}
\newcommand{\gammaddot}[0]{{\ddot{\gamma}}}
\newcommand{\kappaddot}[0]{{\ddot{\kappa}}}
\newcommand{\lambdaddot}[0]{{\ddot{\lambda}}}
\newcommand{\muddot}[0]{{\ddot{\mu}}}
\newcommand{\nuddot}[0]{{\ddot{\nu}}}
\newcommand{\omegaddot}[0]{{\ddot{\omega}}}
\newcommand{\phiddot}[0]{{\ddot{\phi}}}
\newcommand{\piddot}[0]{{\ddot{\pi}}}
\newcommand{\psiddot}[0]{{\ddot{\psi}}}
\newcommand{\rhoddot}[0]{{\ddot{\rho}}}
\newcommand{\sigmaddot}[0]{{\ddot{\sigma}}}
\newcommand{\tauddot}[0]{{\ddot{\tau}}}
\newcommand{\thetaddot}[0]{{\ddot{\theta}}}
\newcommand{\upsilonddot}[0]{{\ddot{\upsilon}}}
\newcommand{\varepsilonddot}[0]{{\ddot{\varepsilon}}}
\newcommand{\varphiddot}[0]{{\ddot{\varphi}}}
\newcommand{\varpiddot}[0]{{\ddot{\varpi}}}
\newcommand{\varrhoddot}[0]{{\ddot{\varrho}}}
\newcommand{\varsigmaddot}[0]{{\ddot{\varsigma}}}
\newcommand{\varthetaddot}[0]{{\ddot{\vartheta}}}
\newcommand{\xiddot}[0]{{\ddot{\xi}}}
\newcommand{\zetaddot}[0]{{\ddot{\zeta}}}

\newcommand{\BDelta}[0]{\boldsymbol{\Delta}}
\newcommand{\BGamma}[0]{\boldsymbol{\Gamma}}
\newcommand{\BLambda}[0]{\boldsymbol{\Lambda}}
\newcommand{\BOmega}[0]{\boldsymbol{\Omega}}
\newcommand{\BPhi}[0]{\boldsymbol{\Phi}}
\newcommand{\BPi}[0]{\boldsymbol{\Pi}}
\newcommand{\BPsi}[0]{\boldsymbol{\Psi}}
\newcommand{\BSigma}[0]{\boldsymbol{\Sigma}}
\newcommand{\BTheta}[0]{\boldsymbol{\Theta}}
\newcommand{\BUpsilon}[0]{\boldsymbol{\Upsilon}}
\newcommand{\BXi}[0]{\boldsymbol{\Xi}}
\newcommand{\Balpha}[0]{\boldsymbol{\alpha}}
\newcommand{\Bbeta}[0]{\boldsymbol{\beta}}
\newcommand{\Bchi}[0]{\boldsymbol{\chi}}
\newcommand{\Bdelta}[0]{\boldsymbol{\delta}}
\newcommand{\Bepsilon}[0]{\boldsymbol{\epsilon}}
\newcommand{\Beta}[0]{\boldsymbol{\eta}}
\newcommand{\Bgamma}[0]{\boldsymbol{\gamma}}
\newcommand{\Bkappa}[0]{\boldsymbol{\kappa}}
\newcommand{\Blambda}[0]{\boldsymbol{\lambda}}
\newcommand{\Bmu}[0]{\boldsymbol{\mu}}
\newcommand{\Bnu}[0]{\boldsymbol{\nu}}
%\newcommand{\Bomega}[0]{\boldsymbol{\omega}}
\newcommand{\Bphi}[0]{\boldsymbol{\phi}}
\newcommand{\Bpi}[0]{\boldsymbol{\pi}}
\newcommand{\Bpsi}[0]{\boldsymbol{\psi}}
\newcommand{\Brho}[0]{\boldsymbol{\rho}}
\newcommand{\Bsigma}[0]{\boldsymbol{\sigma}}
%\newcommand{\Btau}[0]{\boldsymbol{\tau}}
%\newcommand{\Btheta}[0]{\boldsymbol{\theta}}
\newcommand{\Bupsilon}[0]{\boldsymbol{\upsilon}}
\newcommand{\Bvarepsilon}[0]{\boldsymbol{\varepsilon}}
\newcommand{\Bvarphi}[0]{\boldsymbol{\varphi}}
\newcommand{\Bvarpi}[0]{\boldsymbol{\varpi}}
\newcommand{\Bvarrho}[0]{\boldsymbol{\varrho}}
\newcommand{\Bvarsigma}[0]{\boldsymbol{\varsigma}}
\newcommand{\Bvartheta}[0]{\boldsymbol{\vartheta}}
\newcommand{\Bxi}[0]{\boldsymbol{\xi}}
\newcommand{\Bzeta}[0]{\boldsymbol{\zeta}}
%
%</bold and dot greek symbols>
%<infrequent>
%
%\newcommand{\AreaOp}[1]{\AName_{#1}}
%\newcommand{\Babs}[0]{\abs{\BB}}
%\newcommand{\Bcap}[0]{\hat{\BB}}
%\newcommand{\BrPrimeRej}[0]{\rcap(\rcap \wedge \Br')}
%\newcommand{\CA}[0]{\mathcal{A}}
%\newcommand{\Cos}[1]{\cos{\left({#1}\right)}}
%\newcommand{\Det}[1] {\abs{#1}}
%\newcommand{\Dsq}[2] {\frac {\partial^2 {#1}} {\partial {#2}^2}}
%\newcommand{\Exp}[1]{\exp{\left({#1}\right)}}
%\newcommand{\Norm}[1]{\left\lVert{#1}\right\rVert}
%\newcommand{\Sin}[1]{\sin{\left({#1}\right)}}
%\newcommand{\T}[0]{\text{T}}
%\newcommand{\VolumeOp}[1]{\VName_{#1}}
%\newcommand{\agrad}[0]{\Ba \cdot \nabla}
%\newcommand{\alphacap}[0]{\hat{\boldsymbol{\alpha}}}
%\newcommand{\Fcap}[0]{\hat{\BF}}
%\newcommand{\bithree}[0]{{\Bi}_3}
%\newcommand{\bxa}[0]{\Bx\Ba}
%\newcommand{\coordvec}[2]{
%\newcommand{\costheta}[0]{\acap \cdot \xcap}
%\newcommand{\ddt}[1]{\ddot{#1}}
%\newcommand{\ddu}[1] {\frac {d{#1}} {du}}
%\newcommand{\dsqxj}[2] {\frac {\partial^2 {#1}} {\partial {x_{#2}}^2}}
%\newcommand{\dtheta}[1]{\frac{d {#1}}{d \theta}}
%\newcommand{\dt}[1]{\dot{#1}}
%\newcommand{\dt}[1]{\frac{d {#1}}{dt}}
%\newcommand{\dxj}[2] {\frac {\partial {#1}} {\partial {x_{#2}}}}
%\newcommand{\halfPhi}[0]{\frac{\phi}{2}}
%\newcommand{\half}[0]{\inv{2}}
%\newcommand{\inv}[1]{\frac{1}{#1}}
%\newcommand{\laplacian}[0]{\nabla^2}
%\newcommand{\matrixoftx}[3]{
%\newcommand{\nrrp}[0]{\norm{\rcap \wedge \Br'}}
%\newcommand{\oiint}{\bigcirc \hspace{-1.4em} \int \hspace{-.8em} \int}
%\newcommand{\transpose}[1]{{#1}^{\text{T}}}
%\newcommand{\transpose}[1]{{{#1}^{\TextTranspose}}}
%\newcommand{\transpose}[1]{{{#1}^{\text{T}}}}
%\newcommand{\barA}[0]{\bar{A}}
%\newcommand{\qbar}[0]{\bar{q}}
%\newcommand{\qdotbar}[0]{\dot{\bar{q}}}
%
%</infrequent>





\newcommand{\PDSq}[2]{\frac{\partial^2 {#2}}{\partial {#1}^2}}
\DeclareMathOperator{\sinc}{sinc}
\DeclareMathOperator{\PV}{PV}
\newcommand{\FF}[0]{\mathcal{F}}
\newcommand{\IIinf}[0]{ \int_{-\infty}^\infty }

\usepackage[bookmarks=true]{hyperref}

\usepackage{color,cite,graphicx}
   % use colour in the document, put your citations as [1-4]
   % rather than [1,2,3,4] (it looks nicer, and the extended LaTeX2e
   % graphics package. 
\usepackage{latexsym,amssymb,epsf} % don't remember if these are
   % needed, but their inclusion can't do any damage


\title{ Applications of Fourier distribution theory to some PDEs. }
\author{Peeter Joot \quad peeter.joot@gmail.com }
\date{ March 4, 2009.  Last Revision: $Date: 2009/03/04 22:35:55 $ }

\begin{document}
\maketitle{}
\tableofcontents

\section{ Motivation. }

I recently listened to Prof Brad Osgood's lectures on distributions in Fourier
transform theory and read the associated lecture notes
\cite{osgoodFourier}.  Here is an attempt to apply these ideas to solution
of some of the common PDEs of physics (wave and Poisson equations).

Some of these were tackled recently using ``classical'' Fourier methods
in \cite{PJpoisson}. 
This requires ad-hoc PV associations of integrals with delta and step
functions.  Such solutions do not inspire confidence.  Without
validation of the solutions by substituion back into the generating PDE
or comparison to a known solution one is left wondering
if all the right fudges were actually performed to get the answer.

\section{ Conventions and definitions. }

\subsection{ Transform pairs. }

The form of the Fourier transform pairs used here are
\begin{align*}
\hat{f}(\mathbf{k}) &= \frac{1}{(\sqrt{2\pi})^n} \iiint f(\mathbf{x}) e^{-i \mathbf{k} \cdot \mathbf{x} } d^n x \\
{f}(\mathbf{x}) &= \frac{1}{(\sqrt{2\pi})^n} \iiint \hat{f}(\mathbf{k}) e^{i \mathbf{k} \cdot \mathbf{x} } d^n k \\
\end{align*}

\subsection{ Space of Schwarz functions. }

\section{ 1D first order homogeneous wave equation. }

This is the simplest PDE that I can think of that one should be able 
to apply Fourier techniques to.  We seek solutions $f(x,t)$ of

\begin{align}
\inv{v} \PD{t}{f} - \PD{x}{f} = 0
\end{align}

\subsection{ Classical way. }

\subsection{ Using distributions. }

\bibliographystyle{plainnat}
\bibliography{myrefs}

\end{document}


\part{Old cross product musings}
%
% Copyright � 2012 Peeter Joot.  All Rights Reserved.
% Licenced as described in the file LICENSE under the root directory of this GIT repository.
%

% 
% 
%\documentclass{article}      

%\usepackage{amsmath}
\usepackage{mathpazo}

%
% shorthand for bold symbols, convenient for vectors and matrices
%
\newcommand{\Ba}[0]{\mathbf{a}}
\newcommand{\Bb}[0]{\mathbf{b}}
\newcommand{\Bc}[0]{\mathbf{c}}
\newcommand{\Bd}[0]{\mathbf{d}}
\newcommand{\Be}[0]{\mathbf{e}}
\newcommand{\Bf}[0]{\mathbf{f}}
\newcommand{\Bg}[0]{\mathbf{g}}
\newcommand{\Bh}[0]{\mathbf{h}}
\newcommand{\Bi}[0]{\mathbf{i}}
\newcommand{\Bj}[0]{\mathbf{j}}
\newcommand{\Bk}[0]{\mathbf{k}}
\newcommand{\Bl}[0]{\mathbf{l}}
\newcommand{\Bm}[0]{\mathbf{m}}
\newcommand{\Bn}[0]{\mathbf{n}}
\newcommand{\Bo}[0]{\mathbf{o}}
\newcommand{\Bp}[0]{\mathbf{p}}
\newcommand{\Bq}[0]{\mathbf{q}}
\newcommand{\Br}[0]{\mathbf{r}}
\newcommand{\Bs}[0]{\mathbf{s}}
\newcommand{\Bt}[0]{\mathbf{t}}
\newcommand{\Bu}[0]{\mathbf{u}}
\newcommand{\Bv}[0]{\mathbf{v}}
\newcommand{\Bw}[0]{\mathbf{w}}
\newcommand{\Bx}[0]{\mathbf{x}}
\newcommand{\By}[0]{\mathbf{y}}
\newcommand{\Bz}[0]{\mathbf{z}}
\newcommand{\BA}[0]{\mathbf{A}}
\newcommand{\BB}[0]{\mathbf{B}}
\newcommand{\BC}[0]{\mathbf{C}}
\newcommand{\BD}[0]{\mathbf{D}}
\newcommand{\BE}[0]{\mathbf{E}}
\newcommand{\BF}[0]{\mathbf{F}}
\newcommand{\BG}[0]{\mathbf{G}}
\newcommand{\BH}[0]{\mathbf{H}}
\newcommand{\BI}[0]{\mathbf{I}}
\newcommand{\BJ}[0]{\mathbf{J}}
\newcommand{\BK}[0]{\mathbf{K}}
\newcommand{\BL}[0]{\mathbf{L}}
\newcommand{\BM}[0]{\mathbf{M}}
\newcommand{\BN}[0]{\mathbf{N}}
\newcommand{\BO}[0]{\mathbf{O}}
\newcommand{\BP}[0]{\mathbf{P}}
\newcommand{\BQ}[0]{\mathbf{Q}}
\newcommand{\BR}[0]{\mathbf{R}}
\newcommand{\BS}[0]{\mathbf{S}}
\newcommand{\BT}[0]{\mathbf{T}}
\newcommand{\BU}[0]{\mathbf{U}}
\newcommand{\BV}[0]{\mathbf{V}}
\newcommand{\BW}[0]{\mathbf{W}}
\newcommand{\BX}[0]{\mathbf{X}}
\newcommand{\BY}[0]{\mathbf{Y}}
\newcommand{\BZ}[0]{\mathbf{Z}}

\newcommand{\Bzero}[0]{\mathbf{0}}
\newcommand{\Btheta}[0]{\boldsymbol{\theta}}
\newcommand{\Btau}[0]{\boldsymbol{\tau}}
\newcommand{\Bomega}[0]{\boldsymbol{\omega}}

%
% shorthand for unit vectors
%
\newcommand{\acap}[0]{\hat{\Ba}}
\newcommand{\bcap}[0]{\hat{\Bb}}
\newcommand{\ccap}[0]{\hat{\Bc}}
\newcommand{\dcap}[0]{\hat{\Bd}}
\newcommand{\ecap}[0]{\hat{\Be}}
\newcommand{\fcap}[0]{\hat{\Bf}}
\newcommand{\gcap}[0]{\hat{\Bg}}
\newcommand{\hcap}[0]{\hat{\Bh}}
\newcommand{\icap}[0]{\hat{\Bi}}
\newcommand{\jcap}[0]{\hat{\Bj}}
\newcommand{\kcap}[0]{\hat{\Bk}}
\newcommand{\lcap}[0]{\hat{\Bl}}
\newcommand{\mcap}[0]{\hat{\Bm}}
\newcommand{\ncap}[0]{\hat{\Bn}}
\newcommand{\ocap}[0]{\hat{\Bo}}
\newcommand{\pcap}[0]{\hat{\Bp}}
\newcommand{\qcap}[0]{\hat{\Bq}}
\newcommand{\rcap}[0]{\hat{\Br}}
\newcommand{\scap}[0]{\hat{\Bs}}
\newcommand{\tcap}[0]{\hat{\Bt}}
\newcommand{\ucap}[0]{\hat{\Bu}}
\newcommand{\vcap}[0]{\hat{\Bv}}
\newcommand{\wcap}[0]{\hat{\Bw}}
\newcommand{\xcap}[0]{\hat{\Bx}}
\newcommand{\ycap}[0]{\hat{\By}}
\newcommand{\zcap}[0]{\hat{\Bz}}
\newcommand{\thetacap}[0]{\hat{\Btheta}}

%
% to write R^n and C^n in a distinguishable fashion.  Perhaps change this
% to the double lined characters upon figuring out how to do so.
%
\newcommand{\C}[1]{$\mathbb{C}^{#1}$}
\newcommand{\R}[1]{$\mathbb{R}^{#1}$}

%
% various generally useful helpers
%

% derivative of #1 wrt. #2:
\newcommand{\D}[2] {\frac {d#2} {d#1}}

\newcommand{\inv}[1]{\frac{1}{#1}}
\newcommand{\cross}[0]{\times}

\newcommand{\abs}[1]{\lvert{#1}\rvert}
\newcommand{\norm}[1]{\lVert{#1}\rVert}
\newcommand{\innerprod}[2]{\langle{#1}, {#2}\rangle}
\newcommand{\dotprod}[2]{{#1} \cdot {#2}}
\newcommand{\bdotprod}[2]{\left({#1} \cdot {#2}\right)}
\newcommand{\crossprod}[2]{{#1} \cross {#2}}
\newcommand{\tripleprod}[3]{\dotprod{\left(\crossprod{#1}{#2}\right)}{#3}}

\DeclareMathOperator{\Proj}{Proj}
\DeclareMathOperator{\Span}{span}
\DeclareMathOperator{\Sgn}{sgn}
\DeclareMathOperator{\Area}{Area}
\DeclareMathOperator{\Volume}{Volume}

%
% A few miscellaneous things specific to this document
%
\newcommand{\crossop}[1]{\crossprod{#1}{}}

% R2 vector.
\newcommand{\VectorTwo}[2]{
\begin{bmatrix}
 {#1} \\
 {#2}
\end{bmatrix}
}

\newcommand{\VectorN}[1]{
\begin{bmatrix}
{#1}_1 \\
{#1}_2 \\
\vdots \\
{#1}_N \\
\end{bmatrix}
}

\newcommand{\DETuvij}[4]{
\begin{vmatrix}
 {#1}_{#3} & {#1}_{#4} \\
 {#2}_{#3} & {#2}_{#4}
\end{vmatrix}
}

\newcommand{\DETuvwijk}[6]{
\begin{vmatrix}
 {#1}_{#4} & {#1}_{#5} & {#1}_{#6} \\
 {#2}_{#4} & {#2}_{#5} & {#2}_{#6} \\
 {#3}_{#4} & {#3}_{#5} & {#3}_{#6}
\end{vmatrix}
}

\newcommand{\DETuvwxijkl}[8]{
\begin{vmatrix}
 {#1}_{#5} & {#1}_{#6} & {#1}_{#7} & {#1}_{#8} \\
 {#2}_{#5} & {#2}_{#6} & {#2}_{#7} & {#2}_{#8} \\
 {#3}_{#5} & {#3}_{#6} & {#3}_{#7} & {#3}_{#8} \\
 {#4}_{#5} & {#4}_{#6} & {#4}_{#7} & {#4}_{#8} \\
\end{vmatrix}
}

%\newcommand{\DETuvwxyijklm}[10]{
%\begin{vmatrix}
% {#1}_{#6} & {#1}_{#7} & {#1}_{#8} & {#1}_{#9} & {#1}_{#10} \\
% {#2}_{#6} & {#2}_{#7} & {#2}_{#8} & {#2}_{#9} & {#2}_{#10} \\
% {#3}_{#6} & {#3}_{#7} & {#3}_{#8} & {#3}_{#9} & {#3}_{#10} \\
% {#4}_{#6} & {#4}_{#7} & {#4}_{#8} & {#4}_{#9} & {#4}_{#10} \\
% {#5}_{#6} & {#5}_{#7} & {#5}_{#8} & {#5}_{#9} & {#5}_{#10}
%\end{vmatrix}
%}

% R3 vector.
\newcommand{\VectorThree}[3]{
\begin{bmatrix}
 {#1} \\
 {#2} \\
 {#3}
\end{bmatrix}
}







                             
\chapter{The cross product in three and more dimensions} 
\label{chap:cross}
%\author{Peeter Joot \quad peeter.joot@gmail.com }
\date{ October 12, 2007.  cross.tex }

%\begin{document}             

%\maketitle{}

\section{Introduction}

We are initially taught that there are two ways to multiply vectors.  One is the dot product and one is the cross product.  However, we are also taught
how to work with higher dimensional vectors.  There is lots of real applications of higher dimensional vectors that do not require one to waste time puzzling about what the fourth dimension is or how to visualize it.  All we need are four, or N measured qualities for something and we have higher dimensions (position, color, texture, ...).

We see the dot product in a lot of very basic math and physics.  It is inherently simple to understand and to teach, and work with it for a variety of
sorts of calculations.

On the other hand, the cross product is an ugly arbitrary seeming sort of beast.  However, it is a beast that
describes many sorts of physical and mathematical situations.  In vector calculus
cross product terms and it relative the determinant end up occurring all over the place,
and in physics the cross product also occurs in many contexts.
Examples are Stokes theorem, Jacobian transformations, normal equations, the
curl operator, Maxwell's equations, torque, and the list goes on.

The cross product and the dot product have some similarities in form
yet the cross product is only defined for \R{3}, while the dot product
can be defined for \R{n} including \(n > 3\), and even be extended easily in many other ways.  Examples are complex vector spaces, and
the integral dot products that underpin the simple and beautiful Fourier theory.
Note that we call the dot product the inner product when being general, leaving dot or scalar product for the our standard Cartesian vector space beastie.

In many of these cases the mathematics ought to have no logical tie to three dimensions, yet
the cross product is an explicitly three dimensional sort of beast.

It is a bit surprising that one can get all the way through four years of engineering school and still not have an answer about how to do all the sort
of routine vector calculations that we can do in \R{3} in \R{N} (or even \R{4}).
With the second of our two ways to multiply vectors valid only in \R{3}, there is something wrong or
at least missing.

This paper was initially a write up of my scribblings on dot and cross products, where I tried to go back to some of the basic
physical situations that we first the dot and cross products and see for myself how each of these surface in a natural way.  I felt this ought to help indicate where the
explicit three dimensionality of the cross product really came from.

I had some success with this and would say that I had a better feel for the underlying structure of the cross product,
so next came an attempt to investigate possible generalizations of the cross
product to higher dimensions and other mathematical fields.

Since writing some initial notes on this (and producing a generalized cross product that is probably
not entirely useful), I have stumbled across the calculus of differential forms and its wedge product.
I have found, unsurprisingly in retrospect, that I am not the first one to try to
generalize the cross product and the math related to it.
However, differential forms (or calculus on manifolds) can be presented in a horrendously abstract unfriendly fashion, with few of the
geometric considerations that
likely inspired the subject in the first place.
It would be interesting to look at some of the original works by the founders of the subject and see how they presented it.
What we find in texts now is a presentation of the subject in the fashion required to demonstrate its proof from principle to principle.  This hides the natural progression of the subject and is hard to learn from in my opinion.  Obviously there is value to having a thoroughly proven set of
theorems underlying the subject, but that should not inhibit learning or teaching something that appears to have the potential to simplify
the way we work with vector calculus and physics in general.
I have also found a severe lack of clearly worked examples and easy to understand spelled out details in any of my books on this subject, and I have accumulated some of these for myself here.  It may not all make sense without also reading a or some books on differential forms too, but if you are reading this and you are not me, then I hope you get something out of it!

What is this wedge product?  The wedge product is not a mapping into \R{1} or \R{N} from the product of two (or more) vectors in \R{N}, but is instead a value in a ``product space'' that has a dimension usually different from the original.  This is not unreasonable, since we should not have any specific reason to require the product of two vectors itself be a vector of the same dimension.  In fact this is one of the aspects of the cross product that is particularly awkward since we should expect that we can take two vectors in \R{2} and multiply them without having to introduce a normal in a direction that was originally not defined.

We are shown that the wedge product is like the \R{3} cross or triple product in many ways.   It can be used to express the area of the
parallelogram formed by two vectors, or the parallelepiped formed by three, and can also be used to express normals.
My texts on the subject either leave it to you to demonstrate
these for yourself for \R{2} or \R{3} cases, or do not touch on the details at all.  Perhaps it was too ``obvious'' to the author to bother with.

This paper will provide a natural progressive discussion on some of the ways we can end up with a cross product.  An explicit calculation
of the area and volume of an \R{N} parallelogram and parallelepiped will be presented.  Details on how to calculate and express normals to lines,
planes, and volumes will be given in a few different ways.

In the end all this will be tied to the wedge product, and I will hopefully have
an intuitive underpinning that will help me learn differential calculus a bit better.  I had like to see how basic vector math and physics
ends up expressed in the language of differential forms and whether or not it simplifies things and ends up as an easier way to
calculate (I have the feeling it will).

\section{Vector products in geometry}

One can define a product of two vectors in any way you see fit.  The dot product provides a mapping
from \R{n} \(\otimes\) \R{n} to \R{1}, whereas the cross product is a mapping from
from \R{3} \(\otimes\) \R{3} to \R{3}.

The introduction of the cross product with a direction that seems so arbitrarily picked in a normal direction to both vectors was one of my reasons for needing to examine the underlying structure.
It is not unreasonable to ask for a vector product definition that
maps a pair of vectors in \R{2} into \R{2}, without introducing a third dimension to express the result
as is required by the cross product.

It seems irregular to have to introduce a product that is in a space different from the original.  In retrospect that is not too irregular, the dot product does just that, as it provides a mapping to a scalar space.

We should expect to be able to define vector products in many different spaces, and the dot and cross products are just two such potential vector products.

Much more general than either of the dot or cross products would be a product that included all the possible pairs of products of the components

\begin{equation}
\Bu \times \Bv = \sum_{i,j}{u_i v_j F_{i,j}}
\end{equation}

Here, \(F_{i,j}\) is some arbitrary function of the indices, perhaps a vector or component of a matrix or a value in some arbitrary field.  One also does not have to assume that there is any relation between \(F_{i,j}\) and \(F_{j,i}\).
This is generally called a tensor product I believe and like the wedge product it is in a space different than the original.  One could for example, express such a product as a NxN matrix or a vector in \R{N^2}.

Let us start with the saner (well, at least more regular and intuitive) older brother of the cross product, the dot product.  Later, examinations of physical concepts, area's, normals, and
volumes will lead to the cross and wedge products.

\subsection{The dot product}

The first time
we see the dot product in physics is in the context of vector projections onto two or three axis.  For example, when drawing
vector force diagrams and calculating work done against a force applied in a direction different than the motion.

Quantifying this projective operation requires nothing more than basic trigonometry, and we can
express the component of a vector
\(\Bv\) in the direction of \(\Bu\) , as:

\begin{equation}\label{eqn:cross:21}
\Proj_{\ucap}(\Bv) = \norm{\Bv} \cos(\theta) \ucap
\end{equation}

Here \(\ucap = \Bu/\norm{\Bu}\) is the unit vector in the direction of \(\Bu\), and \(\theta\) is the angle between the vector \(\Bu\) and \(\Bv\) where the angle is measured with counterclockwise rotation positive as usual.

Unlike the
cross product, both vector projection, and the lengths of the sides of
a triangle defined by vectors in a plane are just as valid in \R{n}
as in \R{3}, and in fact do not even have a particularly strong tie to
the field of real numbers.  There is however a requirement for both concepts
to introduce a distance metric.

The length of a line from the origin \((0, 0)\) to a point \((v_1, v_2)\), can be
shown to be equal to \(\sqrt{v_1^2 + v_2^2}\) (sophisticated math is not required for this and one can show this with the classic square inscribed in a square diagram ... probably dating back to Pythagoras).
Successive applications of this result shows that this length for a point \((v_1, v_2, v_3)\) equals \(\sqrt{v_1^2 + v_2^2 + v_3^2}\).  It is thus natural to define the length of an \R{n} vector in the same fashion: \(\norm{\Bv} = \sqrt{\sum_{i}{v_i^2}}\).

Taking the length of a vector sum (the opposing side of a triangle formed by these two vectors end to end) we have:

\begin{equation}\label{eqn:cross:901}
\begin{aligned}
\norm{\Bu + \Bv}^2 &= \sum_i{(u_i + v_i)(u_i + v_i)} \\
                   &= \sum_i{{u_i}^2 + 2 u_i v_i + {v_i}^2} \\
                   &= \sum_i{{u_i}^2} + 2 \sum_i{u_i v_i} + \sum{{v_i}^2} \\
                   &= \norm{\Bu}^2 + 2 \sum_i{u_i v_i} + \norm{\Bv}^2
\end{aligned}
\end{equation}

So, if these vectors form a right triangle, the middle term \(\sum_i{u_i v_i}\) must equal zero.

Similarly, for a triangle formed by the difference of two vectors both at the origin the length of the opposing side is:

\begin{equation}\label{eqn:cross:921}
\begin{aligned}
\norm{\Bv - \Bu}^2 &= \sum_i{(v_i - u_i)(v_i - u_i)} \\
                   &= \sum_i{{v_i}^2 - 2 u_i v_i + {u_i}^2} \\
                   &= \sum_i{{v_i}^2} - 2 \sum_i{u_i v_i} + \sum{{u_i}^2} \\
                   &= \norm{\Bv}^2 - 2 \sum_i{u_i v_i} + \norm{\Bu}^2
\end{aligned}
\end{equation}

Note that this is just the triangle law.

\begin{equation}\label{eqn:cross:trianglelaw}
\norm{\Bu - \Bv}^2
= \norm{\Bu}^2 + \norm{\Bv}^2 - 2 \norm{\Bu} \norm{\Bv} \cos(\theta)
\end{equation}

This middle term
\(\sum_i{u_i v_i}\)
in both expansions, we give the name (ie: define) ``dot product'', and write that as:

\begin{equation}
\dotprod{\Bu}{\Bv} = \sum_i{u_i v_i}
\end{equation}

By the triangle law comparison above this can also be expressed, writing \(\theta = (\Bu, \Bv)\), in the projective form

\begin{equation}
\dotprod{\Bu}{\Bv} = \norm{\Bu} \norm{\Bv} \cos(\Bu, \Bv)
\end{equation}

Alternatively the projective operation itself can be expressed in terms of the dot product:

\begin{equation}\label{eqn:cross:941}
\begin{aligned}
\Proj_{\ucap}(\Bv) &= (\norm{\Bv} \cos(\theta)) \ucap \\
                  &= \left(\frac{\dotprod{\Bu}{\Bv}}{\norm{\Bu}}\right) \ucap \\
                  &= \bdotprod{\ucap}{\Bv} \ucap
\end{aligned}
\end{equation}

\subsection{Proof of the projective form of the dot product}

Since the proof of the triangle law was not given, our result for the projective form of the dot product is also unproven.
However, a proof of this would also implicitly prove the triangle law
by comparison as above.

Let us do that proof of the projective form of the dot product without requiring a previous (trigonometric) proof of the triangle law.  To calculate \(\cos(\theta)\) where \(\theta\) is the angle between two vectors \(\Bu\) and \(\Bv\), we let

\begin{equation}\label{eqn:cross:961}
\begin{aligned}
\Bx &= \Proj_{\ucap}(\Bv) \\
    &= \alpha\Bu
\end{aligned}
\end{equation}

and,
\begin{equation}\label{eqn:cross:41}
\By = \Bv - \alpha\Bu
\end{equation}

Temporarily imposing a restriction \(\theta \in [0, \pi/2]\) so that \(\alpha\) is positive, we can now express the vector \(\Bv\) in terms of its perpendicular components \(\Bx\) and \(\By\).

\begin{equation}\label{eqn:cross:dotcosine}
\cos(\theta) = \frac{\alpha \norm{\Bu}}{\norm{\Bv}}
\end{equation}

\begin{equation}\label{eqn:cross:981}
\begin{aligned}
\norm{\Bv}^2       &= \\
\norm{\Bx + \By}^2 &= \norm{\alpha\Bu}^2 + \norm{\Bv - \alpha\Bu}^2 \\
                   &= 2 \alpha^2 \norm{\Bu}^2 + \norm{\Bv}^2 - 2\alpha\dotprod{\Bu}{\Bv}
\end{aligned}
\end{equation}

So,
\begin{equation}\label{eqn:cross:1001}
\begin{aligned}
\alpha\dotprod{\Bu}{\Bv} &= \alpha^2 \norm{\Bu}^2 \\
      \dotprod{\Bu}{\Bv} &= \alpha \norm{\Bu}^2 \\
                         &= (\alpha \norm{\Bu}) \norm{\Bu} \\
                         &= \left(\frac{\alpha \norm{\Bu}}{\norm{\Bv}}\right) \norm{\Bu} \norm{\Bv} \\
                         &= \cos(\theta) \norm{\Bu} \norm{\Bv}
\end{aligned}
\end{equation}

Lifting the restriction and considering the \(\theta \in [\pi/2, \pi]\) range, then by the above:

\begin{equation}\label{eqn:cross:1021}
\begin{aligned}
      \dotprod{-\Bu}{\Bv} &= \alpha \norm{\Bu}^2 \\
                          &= (\cos(\pi - \theta)) \norm{-\Bu} \norm{\Bv}
\end{aligned}
\end{equation}

So, again we have:

\begin{equation}\label{eqn:cross:projectivedotprod}
      \dotprod{\Bu}{\Bv}  = \cos(\theta) \norm{\Bu} \norm{\Bv}
\end{equation}

Proof of \eqnref{eqn:cross:projectivedotprod} for the third and fourth quadrants is similar, proving the following:

\begin{equation}\label{eqn:cross:1041}
\begin{aligned}
\cos(\theta)      &= \cos(\Bu,\Bv) \\
                  &= \frac{\dotprod{\Bu}{\Bv}}{\norm{\Bu} \norm{\Bv}} \\
\cos(\ucap,\vcap) &= \dotprod{\ucap}{\vcap}
\end{aligned}
\end{equation}

Now a triangle law, which gave us the significance of the dot product before even naming it, is proven as a side effect.

\subsection{Projective form of the cross product.  Part I.  Normal to vector in direction of other vector}

Since we started with the projective form of the dot product, it is natural to also start with the
projective form of the cross product.

The cross product is first seen (by me at least), in high school was in the following projective form:

\begin{equation}\label{eqn:cross:61}
\crossprod{\Bu}{\Bv} = \norm{\Bu}\norm{\Bv} \sin(\theta) \ncap
\end{equation}

Now, comparing to the projective form of the dot product we can expect that this is going to be
related to the
component of \(\Bv\) that is perpendicular to \(\Bu\), since that vector has magnitude:

\begin{equation}\label{eqn:cross:81}
\norm{\Bv - \Proj_{\ucap}(\Bv)} = \norm{\Bv} | \sin(\theta) |
\end{equation}

Let us calculate the Cartesian representation for the component of \(\Bv\) normal to \(\Bu\), a
perpendicular projection, \(\Proj_{\perp\ucap}(\Bv) = \Bv - \Proj_{\ucap}(\Bv)\):

\begin{equation}\label{eqn:cross:1061}
\begin{aligned}
\Proj_{\perp\ucap}(\Bv) = \Bv - \Proj_{\ucap}(\Bv) &= \Bv - \bdotprod{\ucap}{\Bv} \ucap \\
                                                 &= \frac{1}{\norm{\Bu}^2} \left(\Bv \norm{\Bu}^2 - \bdotprod{\Bu}{\Bv} \Bu \right) \\
                                                 &= \frac{1}{\norm{\Bu}^2} \sum_{i,j}(v_i \ecap_i u_j u_j - u_j v_j u_i \ecap_i) \\
                                                 &= -\frac{1}{\norm{\Bu}^2} \sum_{i,j}u_j \ecap_i (u_i v_j - u_j v_i) \\
                                                 &= -\frac{1}{\norm{\Bu}^2} \sum_{i,j}u_j \ecap_i \DETuvij{u}{v}{i}{j}
\end{aligned}
\end{equation}

For brevity, let us introduce a shorthand notation for this determinant:

\begin{equation}
D_{ij}^{\Bu \Bv} = \DETuvij{u}{v}{i}{j}
\end{equation}

Since \(D_{ii}^{\Bu \Bv} = 0\), we can write:
\begin{equation}\label{eqn:cross:1081}
\begin{aligned}
\Proj_{\perp\ucap}(\Bv) = \Bv - \Proj_{\ucap}(\Bv)
   &= -\frac{1}{\norm{\Bu}^2} \sum_{i,j} u_j \ecap_i D_{ij}^{\Bu \Bv} \\
   &= -\frac{1}{\norm{\Bu}^2} \left(\sum_{i<j} u_j \ecap_i D_{ij}^{\Bu \Bv} + \sum_{j<i} u_{j} \ecap_{i} D_{ij}^{\Bu \Bv}\right) \\
   &= -\frac{1}{\norm{\Bu}^2} \left(\sum_{i<j} u_j \ecap_i D_{ij}^{\Bu \Bv} + \sum_{j'<i'} u_{j'} \ecap_{i'} D_{i'j'}^{\Bu \Bv}\right) \\
   &= -\frac{1}{\norm{\Bu}^2} \left(\sum_{i<j} u_j \ecap_i D_{ij}^{\Bu \Bv} + \sum_{i<j} u_i \ecap_j D_{ji}^{\Bu \Bv}\right) \\
   &= \frac{1}{\norm{\Bu}^2} \sum_{i<j} (u_i \ecap_j - u_j \ecap_i) D_{ij}^{\Bu \Bv}
\end{aligned}
\end{equation}

But \(u_i \ecap_j - u_j \ecap_i\) is also a determinant, so writing \(\Be = (\ecap_1, \dots, \ecap_N)\), we have:

\begin{equation}\label{eqn:cross:1101}
\begin{aligned}
\Proj_{\perp\ucap}(\Bv) = \Bv - \Proj_{\ucap}(\Bv)
   &= \frac{1}{\norm{\Bu}^2} \sum_{i<j} (u_i \ecap_j - u_j \ecap_i) D_{ij}^{\Bu \Bv} \\
   &= \frac{1}{\norm{\Bu}^2} \sum_{i<j} D_{ij}^{\Bu \Bv} D_{ij}^{\Bu \Be}
\end{aligned}
\end{equation}

This has squared magnitude:

\begin{equation}\label{eqn:cross:1121}
\begin{aligned}
\norm{\Proj_{\perp\ucap}(\Bv)}^2
   &= \dotprod{\Bv}{\left(\Bv - \Proj_{\ucap}(\Bv)\right)} \\
   &= \frac{1}{\norm{\Bu}^2} \sum_{i<j} (D_{ij}^{\Bu \Bv})^2
\end{aligned}
\end{equation}

Taking the root:
\begin{equation}\label{eqn:cross:101}
\norm{\Proj_{\perp\ucap}(\Bv)}
   = \frac{1}{\norm{\Bu}} \left(\sum_{i<j} (D_{ij}^{\Bu \Bv})^2\right)^{1/2}
\end{equation}

This also yields the area of the \R{N} parallelogram, with the two vectors \(\Bu\), \(\Bv\) as edges:
\begin{equation}\label{eqn:cross:1141}
\begin{aligned}
\Area(\Bu,\Bv) &= \norm{\Bu} \norm{\Proj_{\perp\ucap}(\Bv)} \\
              &= \left(\sum_{i<j} (D_{ij}^{\Bu \Bv})^2\right)^{1/2}
\end{aligned}
\end{equation}

\subsection{Projective form of the cross product.  Part II.  Normal to two vectors in direction of other vector}

The result \(\Proj_{\perp\ucap}(\Bv) = \frac{1}{\norm{\Bu}^2} \sum_{i<j} D_{ij}^{\Bu \Bv} D_{ij}^{\Bu \Be}\)
does not look like the cross product, and it is not.  However, it is also not a normal to two vectors as the cross product is, only one.

If we continue with the calculation of the normal to two vectors (in the direction of a third) something that we can
calculate in \R{N}, it is expected that the result will have similar aspects to the cross product, especially for \R{3}.

Like the one vector normal
\(\Proj_{\perp\ucap}(\Bv)\)
, we can only calculate this definitively for most dimensions when that calculation is with respect to an additional
vector.  Without a reference vector, we can calculate only specific cases.  These are the \R{2} normal to one vector, and a \R{3} normal to two vectors (cross product), or the \R{N} normal to \(N-1\) vectors (and for all of these the result can vary by an arbitrary scalar multiplier).

It is a bit laborious, but let us calculate the normal to two vectors in the direction of a third (the component that is perpendicular to the plane formed by all the linear combinations of the first two vectors).

Let \(\Bm = \Proj_{\perp\ucap}(\Bv)\)

\begin{equation}\label{eqn:cross:1161}
\begin{aligned}
\Proj_{\perp\ucap,\vcap}(\Bw) &= \Bw - \bdotprod{\Bw}{\ucap}
 \ucap - \bdotprod{\Bw}{\mcap} \mcap \\
                             &= \Bw - \frac{1}{\norm{\Bu}^2}
\bdotprod{\Bu}{\Bw}
 \Bu - \frac{1}{\norm{\Bm}^2}\bdotprod{\Bw}{\Bm} \Bm \\
                             &= \frac{1}{\norm{\Bu}^2 \norm{\Bm}^2}\left(\Bw\norm{\Bu}^2\norm{\Bm}^2 - \norm{\Bm}^2
\bdotprod{\Bu}{\Bw}
 \Bu - {\norm{\Bu}^2}\bdotprod{\Bw}{\Bm} \Bm\right)
\end{aligned}
\end{equation}

Expanding, \(\norm{\Bm}^2\), yields:

\begin{equation}\label{eqn:cross:1181}
\begin{aligned}
\norm{\Bm}^2 &=
\dotprod{\left( \Bv - \bdotprod{\ucap}{\Bv} \ucap \right)} {\Bv} \\
             &= \frac{1}{\norm{\Bu}^2}
\bdotprod{\Bv \norm{\Bu}^2 - \bdotprod{\Bu}{\Bv} \Bu}{\Bv} \\
             &= \frac{1}{\norm{\Bu}^2}
\left( \norm{\Bu}^2 \norm{\Bv}^2 - {\bdotprod{\Bu}{\Bv}}^2 \right) \\
\end{aligned}
\end{equation}
%\norm{\Bm}^2 \norm{\Bu}^2 = \norm{\Bu}^2 \norm{\Bv}^2 - {\bdotprod{\Bu}{\Bv}}^2
%\Bm = \Bv - \bdotprod{\ucap}{\Bv}\ucap

\begin{equation}\label{eqn:cross:1201}
\begin{aligned}
\Rightarrow \Proj_{\perp\ucap,\vcap}(\Bw)
&= \frac{1}{\sum_{i<j} (D_{ij}^{\Bu \Bv})^2}
   \left(\Bw\norm{\Bu}^2\norm{\Bm}^2 - \norm{\Bm}^2
\bdotprod{\Bu}{\Bw}
\Bu - {\norm{\Bu}^2}\bdotprod{\Bw}{\Bm} \Bm\right) \\
\end{aligned}
\end{equation}

\begin{equation}\label{eqn:cross:1221}
\begin{aligned}
\Rightarrow \Proj_{\perp\ucap,\vcap}(\Bw) {\sum_{i<j} (D_{ij}^{\Bu \Bv})^2} &=
\Bw \left(\norm{\Bu}^2 \norm{\Bv}^2 - {\bdotprod{\Bu}{\Bv}}^2\right) \\
&- \left(\frac{1}{\norm{\Bu}^2}\left( \norm{\Bu}^2 \norm{\Bv}^2 -
{\bdotprod{\Bu}{\Bv}}^2\right)\right) {\bdotprod{\Bu}{\Bw}} \Bu \\
&- {\norm{\Bu}^2}
\bdotprod{\Bw}{(\Bv - {\bdotprod{\ucap}{\Bv}}\ucap)}
 \left( \Bv - {\bdotprod{\ucap}{\Bv}}\ucap \right)
\end{aligned}
\end{equation}

\begin{equation}\label{eqn:cross:1241}
\begin{aligned}
\Rightarrow \Proj_{\perp\ucap,\vcap}(\Bw) {\sum_{i<j} (D_{ij}^{\Bu \Bv})^2} \norm{\Bu}^2
&= \Bw \left(\norm{\Bu}^4 \norm{\Bv}^2 - {\bdotprod{\Bu}{\Bv}}^2\norm{\Bu}^2\right) \\
&- \left( \norm{\Bu}^2 \norm{\Bv}^2 - {\bdotprod{\Bu}{\Bv}}
^2\right) {\bdotprod{\Bu}{\Bw}} \Bu \\
&- \left( \dotprod{\Bw}({ {\norm{\Bu}^2} \Bv -
{\bdotprod{\Bu}{\Bv}}
\Bu})\right) \left( \Bv {\norm{\Bu}^2} - {\bdotprod{\Bu}{\Bv}}\Bu\right)
\end{aligned}
\end{equation}

\begin{equation}\label{eqn:cross:1261}
\begin{aligned}
&=
   \left( \norm{\Bu}^4 \norm{\Bv}^2 - {\bdotprod{\Bu}{\Bv}}^2\norm{\Bu}^2 \right)                                             \Bw \\
&- \left( {\bdotprod{\Bv}{\Bw}}
 \norm{\Bu}^4 - {\bdotprod{\Bu}{\Bw}}
{\bdotprod{\Bu}{\Bv}} {\norm{\Bu}^2} \right)                \Bv \\
&+ \left( {\bdotprod{\Bv}{\Bw}}
 {\bdotprod{\Bu}{\Bv}}
 {\norm{\Bu}^2} - {\bdotprod{\Bu}{\Bw}} \norm{\Bu}^2 \norm{\Bv}^2 \right)  \Bu
\end{aligned}
\end{equation}

\begin{equation}\label{eqn:cross:1281}
\begin{aligned}
\Rightarrow \Proj_{\perp\ucap,\vcap}(\Bw) {\sum_{i<j} (D_{ij}^{\Bu \Bv})^2}
&= \left( \norm{\Bu}^2 \norm{\Bv}^2 - {\bdotprod{\Bu}{\Bv}}^2 \right)                              \Bw \\
&- \left( \norm{\Bu}^2 {\bdotprod{\Bv}{\Bw}}
 - {\bdotprod{\Bu}{\Bv}}
 {\bdotprod{\Bu}{\Bw}}\right)    \Bv \\
&+ \left( {\bdotprod{\Bu}{\Bv}}
 {\bdotprod{\Bv}{\Bw}}
 - {\bdotprod{\Bu}{\Bw}} \norm{\Bv}^2 \right)   \Bu \\
\end{aligned}
\end{equation}
\begin{equation}\label{eqn:cross:1301}
\begin{aligned}
= \sum_{ijk} \ecap_i ( &\left( u_j u_j v_k v_k - u_j v_j u_k v_k \right) w_i \\
                      +&\left( u_j w_j u_k v_k - u_j u_j v_k w_k \right) v_i \\
                      +&\left( u_j v_j v_k w_k - u_j w_j v_k v_k \right) u_i )
\end{aligned}
\end{equation}
\begin{equation}\label{eqn:cross:1321}
\begin{aligned}
= \sum_{ijk} u_j v_k \ecap_i ( &\left( u_j v_k - v_j u_k \right) w_i \\
                              +&\left( w_j u_k - u_j w_k \right) v_i \\
                              +&\left( v_j w_k - w_j v_k \right) u_i )
\end{aligned}
\end{equation}
\begin{equation}\label{eqn:cross:1341}
\begin{aligned}
&= \sum_{ijk} u_j v_k \ecap_i \left(
u_i D_{jk}^{\Bv \Bw}
+v_i D_{jk}^{\Bw \Bu}
+w_i D_{jk}^{\Bu \Bv}\right)  \\
&= \sum_{i,j<k} \left(u_j v_k - u_k v_j\right) \ecap_i \left( u_i D_{jk}^{\Bv \Bw} + v_i D_{jk}^{\Bw \Bu} + w_i D_{jk}^{\Bu \Bv} \right) \\
&= \sum_{i,j<k} \ecap_i D_{jk}^{\Bu \Bv} \left( u_i D_{jk}^{\Bv \Bw} + v_i D_{jk}^{\Bw \Bu} + w_i D_{jk}^{\Bu \Bv} \right) \\
&= \sum_{i,j<k} \ecap_i D_{jk}^{\Bu \Bv} D_{ijk}^{\Bu \Bv \Bw} \\
&= \sum_{i<j<k} \ecap_i D_{jk}^{\Bu \Bv} D_{ijk}^{\Bu \Bv \Bw}
+ \sum_{j'<i'<k} \ecap_i' D_{j'k}^{\Bu \Bv} D_{i'j'k}^{\Bu \Bv \Bw}
+ \sum_{j'<k'<i'} \ecap_i' D_{j'k'}^{\Bu \Bv} D_{i'j'k'}^{\Bu \Bv \Bw} \\
&= \sum_{i<j<k} \ecap_i D_{jk}^{\Bu \Bv} D_{ijk}^{\Bu \Bv \Bw}
+ \sum_{i<j<k} \ecap_j D_{ik}^{\Bu \Bv} D_{jik}^{\Bu \Bv \Bw}
+ \sum_{i<j<k} \ecap_k D_{ij}^{\Bu \Bv} D_{kij}^{\Bu \Bv \Bw} \\
\end{aligned}
\end{equation}
\begin{equation}\label{eqn:cross:1361}
\begin{aligned}
&= \sum_{i<j<k} \left(\ecap_i D_{jk}^{\Bu \Bv} + \ecap_j D_{ki}^{\Bu \Bv} + \ecap_k D_{ij}^{\Bu \Bv}\right) D_{ijk}^{\Bu \Bv \Bw} \\
&= \sum_{i<j<k} D_{ijk}^{\Bu \Bv \Bw} D_{ijk}^{\Bu \Bv \Be} \\
\end{aligned}
\end{equation}
\begin{equation}\label{eqn:cross:121}
\Rightarrow \Proj_{\perp\ucap,\vcap}(\Bw) =
\frac{1}{\sum_{i<j} \left(D_{ij}^{\Bu \Bv}\right)^2}\sum_{i<j<k} D_{ijk}^{\Bu \Bv \Bw} D_{ijk}^{\Bu \Bv \Be} \\
\end{equation}

Calculating the magnitude of this vector yields a formula for the volume of the \R{N} parallelepiped spanned by three vectors \(\Bu\), \(\Bv\), and \(\Bw\).

\begin{equation}\label{eqn:cross:1381}
\begin{aligned}
\Volume(\Bu, \Bv, \Bw) &= \Area(\Bu, \Bv) \norm{\Proj_{\perp\ucap, \vcap}(\Bw)} \\
                      &= \left(\sum_{i<j<k} \left(D_{ijk}^{\Bu \Bv \Bw}\right)^{2}\right)^{1/2}
\end{aligned}
\end{equation}

\subsection{Summary of \texorpdfstring{\R{N}}{ND} directed normal results}

\begin{equation}\label{eqn:cross:1401}
\begin{aligned}
\Proj_{\perp\ucap}(\Bv)
   &= \Bv - {\bdotprod{\ucap}{\Bv}}\ucap \\
   &= \Bv - \norm{\Bu}^{-2}{\bdotprod{\Bu}{\Bv}}\Bu \\
   &= \frac{1}{\norm{\Bu}^2} \sum_{i<j} D_{ij}^{\Bu \Bv} D_{ij}^{\Bu \Be} \\
   &= \norm{\Bu}^{-2} \sum_{i<j} \DETuvij{u}{v}{i}{j} \DETuvij{u}{\ecap}{i}{j}
\end{aligned}
\end{equation}
\begin{equation}\label{eqn:cross:1421}
\begin{aligned}
\Proj_{\perp\ucap,\vcap}(\Bw)
&= \Bw - \norm{\Bu}^{-2} {\bdotprod{\Bu}{\Bw}}\Bu \\
&- \norm{\Proj_{\perp\ucap}\left(\Bv\right)}^{-2}
\bdotprod{\Proj_{\perp\ucap}\left(\Bv\right)}{\Bw}
 \Proj_{\perp\ucap}(\Bv) \\
&= \frac{1}{\sum_{i<j} \left(D_{ij}^{\Bu \Bv}\right)^2}\sum_{i<j<k} D_{ijk}^{\Bu \Bv \Bw} D_{ijk}^{\Bu \Bv \Be} \\
&= \left(\sum_{i<j} {\DETuvij{u}{v}{i}{j}}^2\right)^{-1} \sum_{i<j<k} \DETuvwijk{u}{v}{w}{i}{j}{k} \DETuvwijk{u}{v}{\ecap}{i}{j}{k} \\
\end{aligned}
\end{equation}

These results are valid for \R{3}, as well as other dimensions \R{N} (including \R{2} which the cross product is not).

For \R{2} the one vector normal, and for \R{3} the two vector normals become:

\begin{equation}\label{eqn:cross:1441}
\begin{aligned}
\Proj_{\perp\ucap}(\Bv)        &= \left(\norm{\Bu}^{-2} \DETuvij{u}{v}{1}{2}\right) \DETuvij{u}{\ecap}{1}{2} \\
                              &= (scalar value) \DETuvij{u}{\ecap}{1}{2} \\
                              &= (scalar value) \VectorTwo{-u_2}{u_1} \\
\Proj_{\perp\ucap,\vcap}(\Bw)
&= \left(\left(\sum_{1 \leq i<j \leq 3} {\DETuvij{u}{v}{i}{j}}^2\right)^{-1} \DETuvwijk{u}{v}{w}{1}{2}{3}\right) \DETuvwijk{u}{v}{\ecap}{1}{2}{3} \\
&= (scalar value) \DETuvwijk{u}{v}{\ecap}{1}{2}{3} \\
%&= (scalar value) \VectorThree {D_{23}^{\Bu\Bv}} {D_{31}^{\Bu\Bv}} {D_{12}^{\Bu\Bv}} \\
&= (scalar value) \VectorThree {u_2 v_3 - u_3 v_2} {u_3 v_1 - u_1 v_3} {u_1 v_2 - u_2 v_1} \\
%&= (scalar value) (\crossprod{\Bu}{\Bv})
\end{aligned}
\end{equation}

The first is a (scaled) normal vector to \(\Bu\), and the second is a (scaled) normal vector to \(\Bu\) and \(\Bv\) (ie: cross product)

\subsection{Summary of \texorpdfstring{\R{N}}{ND} Area, Volume results}

\begin{equation}\label{eqn:cross:1461}
\begin{aligned}
\Area(\Bu, \Bv)
   &= \left(\sum_{i<j} \left(D_{ij}^{\Bu \Bv}\right)^2\right)^{1/2} \\
   &= \left(\sum_{i<j} {\DETuvij{u}{v}{i}{j}}^2\right)^{1/2} \\
\end{aligned}
\end{equation}
\begin{equation}\label{eqn:cross:1481}
\begin{aligned}
\Volume(\Bu, \Bv, \Bw)
&= \left(\sum_{i<j<k} \left(D_{ijk}^{\Bu \Bv \Bw}\right)^2\right)^{1/2} \\
&= \left(\sum_{i<j<k} {\DETuvwijk{u}{v}{w}{i}{j}{k}}^2\right)^{1/2}
\end{aligned}
\end{equation}

For the \R{2} Area, and \R{3} Volume these becomes the familiar determinant and triplet product results:

\begin{equation}\label{eqn:cross:1501}
\begin{aligned}
\Area(\Bu, \Bv)
&= \abs{ D_{ij}^{uv} } \\
&= abs \DETuvij{u}{v}{1}{2}
\end{aligned}
\end{equation}

\begin{equation}\label{eqn:cross:1521}
\begin{aligned}
\Volume(\Bu, \Bv, \Bw)
&= \abs{\dotprod{(\crossprod{\Bu}{\Bv})}{\Bw}} \\
&= abs \DETuvwijk{u}{v}{w}{1}{2}{3}
\end{aligned}
\end{equation}

What was not proved is that this generalizes, and that the m-parallelepiped volume is what we would expect:

\begin{equation}\label{eqn:cross:141}
\Volume(\Bu_1, \dotsc, \Bu_{m}) =
\sum_{i_1 < \dotsb < i_m} \left(D_{i_1,\dotsc,i_m}^{\Bu_1, \dotsc, \Bu_{m}}\right)^2
\end{equation}

This \(\Volume()\) result corresponds to
the length of a vector, area of a parallelogram, and the volume of a parallelepiped for the 1 vector, 2 vector and 3 vector cases respectively, and this has been proved for \R{N} (not just \R{2} or \R{3}).  To prove it for a
\(m+1\) vector parallelogram, in terms of the
\(\Volume(\Bu_1, \dotsc, \Bu_{m})\)
 we need to take the component of this \(m+1\)'th vector that is perpendicular to the the span of all the vectors in the m-parallelepiped and multiply the length of that projection by \(\Volume(\Bu_1, \dotsc, \Bu_{m})\)

Without calculation, it is expected that this perpendicular projection is:

\begin{equation}\label{eqn:cross:1541}
\begin{aligned}
\Proj_{\perp \ucap_1\dotsb\ucap_{m-1}}(\Bu_m)
&=
\Bu_m - \Proj_{\ucap_1\dotsb\ucap_{m-1}}(\Bu_m) \\
&=
\left(\Volume(\Bu_1, \dotsc, \Bu_{m-1})\right)^{-2}
\sum_{i_1 < \dotsb < i_m}
D_{i_1,\dotsc,i_m}^{\Bu_1, \dotsc, \Bu_{m}}
D_{i_1,\dotsc,i_m}^{\Bu_1, \dotsc, \Bu_{m-1}, \Be}
\end{aligned}
\end{equation}

From which the \(m>3\) volume result would follow.  I have not been successful proving this for myself inductively even for
\(m=3\) based on the \(m=2\) result in the form above.  I also note that the books that I have also do not prove this.  Some have a kind of
sneaky way of dealing with this by defining the generalized volume in terms of the wedge product, and never really
demonstrating the geometrical validity of doing so except for \(m=2\) or \(m=3\), or in some cases only for \R{3}.  Since the generalized volume result seems to be uniformly accepted, I am sure some
sufficiently talented mathematician has done the inductive proof for this, perhaps tackling the problem
from some other direction where the result follows more easily.

\section{Normals without a reference vector}

Calculation of a normal above required a reference vector, since there can be normals in many different directions to a set of \(n \in\) \R{N} vectors unless \(n = N-1\).  In the \(n = N-1\) case, the normal only varies by a scalar multiplier.

In this section the normal to a set of vectors will be calculated without introducing a
reference vector.  This is closer to the formulation one would expect of a generalized cross
product, but does generally introduce a set of undetermined coefficients.  We can then
compare the to results for the normals taken in the direction of a reference vector.

\subsection{orthogonality and the Null Space of two vectors}

Using just an
orthogonality condition is not enough to uniquely define a ``cross product''
even in \R{3}, but for \R{3} that is good within at least a scalar multiple.

For \R{N}, lets calculate via row reduction the Null Space of a matrix with rows formed of the elements of the two vectors \(\Bu\) and \(\Bv\), and then solve for \(\Bn\).

\begin{equation}\label{eqn:cross:161}
\begin{bmatrix}
u_1 & u_2 & \dotsb & u_N \\
v_1 & v_2 & \dotsb & v_N
\end{bmatrix}
\VectorN{n}
= \Bzero
\end{equation}

The row reduction can be performed with any set of two columns, not just the first two (the first two could be all zeros for example).  So for generality, we Row reduce based on columns \(i\) and \(j\):

\begin{equation}\label{eqn:cross:181}
\begin{bmatrix}
v_j & -u_j \\
-v_i & u_i
\end{bmatrix}
\begin{bmatrix}
u_1 & u_2 & \dotsb & u_N \\
v_1 & v_2 & \dotsb & v_N
\end{bmatrix}
\VectorN{n} = \Bzero
\end{equation}

For column \(k\) this is:
\begin{equation}\label{eqn:cross:201}
\begin{bmatrix}
v_j & -u_j \\
-v_i & u_i
\end{bmatrix}
\VectorTwo{u_k}{v_k}
=
\VectorTwo{ u_k v_j - u_j v_k }{ u_i v_k - u_k v_i }
=
\VectorTwo{ D_{kj}^{\Bu\Bv} }{ D_{ik}^{\Bu\Bv} }
\end{equation}

In particular, for columns \(i\), and \(j\) respectively this is:

\begin{equation}\label{eqn:cross:221}
\VectorTwo{ D_{ij}^{\Bu\Bv} }{ 0 }
,
\VectorTwo{ 0 }{ D_{ij}^{\Bu\Bv} }
\end{equation}

And we are left with two sets of equations in \(N-2\) dependent variables.

\begin{equation}\label{eqn:cross:1561}
\begin{aligned}
D_{ij}^{\Bu\Bv} n_i &= \sum_{k \neq i,j}D_{jk}^{\Bu\Bv} n_k \\
D_{ij}^{\Bu\Bv} n_j &= \sum_{k \neq i,j}D_{ki}^{\Bu\Bv} n_k \\
\end{aligned}
\end{equation}

Now, let \(n_k = t_k\) for \(k \neq i,j\), and \(t_k\) is an arbitrary constant, and combining these equations:

\begin{equation}\label{eqn:cross:1581}
\begin{aligned}
D_{ij}^{\Bu\Bv} \Bn
&= \sum_s D_{ij}^{\Bu\Bv} n_s \ecap_s \\
&= D_{ij}^{\Bu\Bv} n_i \ecap_i + D_{ij}^{\Bu\Bv} n_j \ecap_j +
\sum_{k \neq i,j} n_k \ecap_k D_{ij}^{\Bu\Bv} \\
&=
\ecap_i \sum_{k \neq i,j}{D_{jk}^{\Bu\Bv} t_k} +
\ecap_j \sum_{k \neq i,j}{D_{ki}^{\Bu\Bv} t_k} +
\sum_{k \neq i,j} t_k \ecap_k D_{ij}^{\Bu\Bv} \\
&=
\sum_{k \neq i,j} t_k
\left(
\ecap_i {D_{jk}^{\Bu\Bv}} +
\ecap_j {D_{ki}^{\Bu\Bv}} +
\ecap_k {D_{ij}^{\Bu\Bv}}
\right) \\
&=
\sum_{k \neq i,j} t_k D_{ijk}^{\Bu\Bv\Be}
\end{aligned}
\end{equation}

Since \(i\) and \(j\) were chosen arbitrarily, this is really the sum over all sets of unique combinations of \(i\), \(j\), and \(k\), so we can write the most
generic normal to a pair of vectors in \R{N} as
\begin{equation}\label{eqn:cross:1601}
\begin{aligned}
\Bn
&= \sum_{i,j,k} s_{ijk} D_{ijk}^{\Bu\Bv\Be} \\
&= \sum_{i<j<k} \left( s_{ijk} - s_{ikj} + s_{kij} - s_{kji} + s_{jki} - s_{jik} \right) D_{ijk}^{\Bu\Bv\Be} \\
&= \sum_{i<j<k} \left( \sum_{\pi_x(i,j,k)} s_{\pi_x}\Sgn(\pi_x) \right) D_{ijk}^{\Bu\Bv\Be} \\
&= \sum_{i<j<k} \left( \sum_{\pi_x \in \pi(i,j,k)} s_{\pi_x}\Sgn(\pi_x) \right) \DETuvwijk{u}{v}{\ecap}{i}{j}{k} \\
\end{aligned}
\end{equation}

Here \(s_{ijk} = t_k/{D_{ij}^{\Bu\Bv}}\), and \(\pi(i,j,k)\) are the permutations of the indices \(i\), \(j\), and \(k\), and \(\Sgn\) is the sign of the individual permutation \(\pi_x\) in that set (-1 for odd numbers of index switches, 1 for even numbers of switches).

For \R{3}, we once again have a scaled cross product.  Also observe that the coefficient term is much like a determinant, the sign alternates with the switch of any two indices and zero if any indices match.  This is not surprising given the earlier calculation of the normal in the direction of a reference vector, was in fact a determinant.

Because the values \(s_{ijk}\) were arbitrary constants, so is the composite value \(s_{ijk}' = \sum_{\pi_x \in \pi(i,j,k)} s_{\pi_x}\Sgn(\pi_x)\), so really this is just a statement that:

\begin{equation}\label{eqn:cross:241}
\Bn \in \Span \left\lbrace
\DETuvwijk{u}{v}{\ecap}{i}{j}{k}
\right\rbrace
\end{equation}

A small note here about the dependence of this result on the field of real numbers.  The normal that was calculated here is not normal for \C{N}, but is a sort of
conjugate normal.  One would have to row reduce the complex conjugates of the vectors to produce a result that is valid for \C{N}.  If one replaces the components in the determinants with their conjugates that should correct the result.

\subsection{orthogonality and the Null Space of one vector}

Having done the calculation for the two vector case, a result like \(\Proj_{\perp\ucap}(\Bv)\) is expected.  Here is the calculation that verifies this:

\begin{equation}\label{eqn:cross:1621}
\begin{aligned}
 (n_i u_i) \ecap_i &= \left( -\sum_{j \neq i}n_j u_j \right) \ecap_i \\
                   &= \left( -\sum_{j \neq i}t_j u_j \right) \ecap_i
\end{aligned}
\end{equation}
\begin{equation}\label{eqn:cross:1641}
\begin{aligned}
 u_i (n_j \ecap_j) &= u_i ( t_j \ecap_j)
\end{aligned}
\end{equation}
\begin{equation}\label{eqn:cross:1661}
\begin{aligned}
 \Rightarrow  u_i \Bn  &= \sum_{j \neq i}( u_i t_j \ecap_j - t_j u_j \ecap_i ) \\
                       &= \sum_{j \neq i} t_j ( u_i \ecap_j - u_j \ecap_i ) \\
                       &= \sum_{j \neq i} t_j D_{ij}^{\Bu\Be}
\end{aligned}
\end{equation}
Let \(s_{ij} = t_j/u_i\),
\begin{equation}\label{eqn:cross:1681}
\begin{aligned}
 \Rightarrow \Bn       &= \sum_{i<j} (s_{ij} - s_{ji}) D_{ij}^{\Bu\Be} \\
                       &= \sum_{i<j} \left(\sum_{\pi_x \in \pi(i,j)}s_{\pi_x} \Sgn(\pi_x)\right) D_{ij}^{\Bu\Be} \\
                       &= \sum_{i<j} s_{ij}' D_{ij}^{\Bu\Be}
\end{aligned}
\end{equation}

With the expected result:

\begin{equation}\label{eqn:cross:261}
\Bn \in \Span \left\lbrace
\DETuvij{u}{\ecap}{i}{j}
\right\rbrace
\end{equation}

\subsection{orthogonality and the Null Space of three vectors}

The calculation of \(\Proj_{\perp\ucap\vcap}(\Bw)\) was pretty laborious.  Without actually calculating it
the expected result for, it is expected that:

\begin{equation}\label{eqn:cross:281}
\Proj_{\perp\ucap\vcap\wcap}(\Bx) = {\Volume(\Bu, \Bv, \Bw)}^{-2} \sum_{i<j<k<l} D_{ijkl}^{\Bu\Bv\Bw\Bx} D_{ijkl}^{\Bu\Bv\Bw\Be}
\end{equation}

There is probably a way to verify this inductively, but it is not obvious to me how to approach this.  One the
other hand the calculation of the Null Space of three vectors is not too hard.

First calculate the Cofactor matrix for
\begin{equation}\label{eqn:cross:301}
\begin{bmatrix}
u_i & u_j & u_k \\
v_i & v_j & v_k \\
w_i & w_j & w_k \\
\end{bmatrix}
\xrightarrow{Transpose}
\begin{bmatrix}
u_i & v_i & w_i \\
u_j & v_j & w_j \\
u_k & v_k & w_k \\
\end{bmatrix}
\xrightarrow{Cofactors}
\begin{bmatrix}
D_{jk}^{\Bv\Bw} & -D_{jk}^{\Bu\Bw} & D_{jk}^{\Bu\Bv} \\
-D_{ik}^{\Bv\Bw} & D_{ik}^{\Bu\Bw} & -D_{ik}^{\Bu\Bv} \\
D_{ij}^{\Bv\Bw} & -D_{ij}^{\Bu\Bw} & D_{ij}^{\Bu\Bv}
\end{bmatrix}
\end{equation}
\begin{equation}\label{eqn:cross:321}
\Rightarrow
\begin{bmatrix}
D_{jk}^{\Bv\Bw} & -D_{jk}^{\Bu\Bw} & D_{jk}^{\Bu\Bv} \\
-D_{ik}^{\Bv\Bw} & D_{ik}^{\Bu\Bw} & -D_{ik}^{\Bu\Bv} \\
D_{ij}^{\Bv\Bw} & -D_{ij}^{\Bu\Bw} & D_{ij}^{\Bu\Bv}
\end{bmatrix}
\VectorThree{u_m}{v_m}{w_m}
=
\begin{bmatrix}
u_m D_{jk}^{\Bv\Bw} & -v_m D_{jk}^{\Bu\Bw} & w_m D_{jk}^{\Bu\Bv} \\
u_m D_{ki}^{\Bv\Bw} & -v_m D_{ki}^{\Bu\Bw} & w_m D_{ki}^{\Bu\Bv} \\
u_m D_{ij}^{\Bv\Bw} & -v_m D_{ij}^{\Bu\Bw} & w_m D_{ij}^{\Bu\Bv}
\end{bmatrix}
=
\VectorThree
{D_{mjk}^{\Bu\Bv\Bw}}
{D_{mki}^{\Bu\Bv\Bw}}
{D_{mij}^{\Bu\Bv\Bw}}
\end{equation}

Columns \(m=i\), \(m=j\), and \(m=k\) are respectively,
\begin{equation}\label{eqn:cross:341}
\VectorThree
{D_{ijk}^{\Bu\Bv\Bw}}
{0}
{0}
,
\VectorThree
{0}
{D_{ijk}^{\Bu\Bv\Bw}}
{0}
,
\VectorThree
{0}
{0}
{D_{ijk}^{\Bu\Bv\Bw}}
\end{equation}

With the introduction of free parameters \(n_m = t_m\) we have four equations,

\begin{equation}\label{eqn:cross:1701}
\begin{aligned}
{D_{ijk}^{\Bu\Bv\Bw}} n_i \ecap_i &= - \ecap_i \sum_{m \neq i,j,k} t_m {D_{mjk}^{\Bu\Bv\Bw}} \\
{D_{ijk}^{\Bu\Bv\Bw}} n_j \ecap_j &= - \ecap_j \sum_{m \neq i,j,k} t_m {D_{mki}^{\Bu\Bv\Bw}} \\
{D_{ijk}^{\Bu\Bv\Bw}} n_k \ecap_k &= - \ecap_k \sum_{m \neq i,j,k} t_m {D_{mij}^{\Bu\Bv\Bw}} \\
\sum_{m \neq i,j,k} {D_{ijk}^{\Bu\Bv\Bw}} n_m \ecap_m &= \sum_{m \neq i,j,k} {D_{ijk}^{\Bu\Bv\Bw}} t_m \ecap_m
\end{aligned}
\end{equation}

Adding these
\begin{equation}\label{eqn:cross:1721}
\begin{aligned}
{D_{ijk}^{\Bu\Bv\Bw}} \Bn
&= \sum_{m} {D_{ijk}^{\Bu\Bv\Bw}} n_m \ecap_m \\
&=
\sum_{m \neq i,j,k} t_m \left(
  \ecap_m {D_{ijk}^{\Bu\Bv\Bw}}
- \ecap_i {D_{jkm}^{\Bu\Bv\Bw}}
+ \ecap_j {D_{kmi}^{\Bu\Bv\Bw}}
- \ecap_k {D_{mij}^{\Bu\Bv\Bw}}
\right) \\
&=
\sum_{m \neq i,j,k} t_m D_{mijk}^{\Bu\Bv\Bw\Be}
\end{aligned}
\end{equation}

A result exactly like the one and two vector cases, with the same conclusion:

\begin{equation}\label{eqn:cross:361}
\Rightarrow
\Bn \in \Span \left\lbrace
\DETuvwxijkl{u}{v}{w}{\ecap}{i}{j}{k}{l}
\right\rbrace
\end{equation}

\subsection{orthogonality and complex vector products}
The same thing can be done for the complex inner product, where
for orthogonality the term,
\begin{equation}\label{eqn:cross:381}
\sum_i{ u_i \overline{v_i} + v_i \overline{u_i}}
\end{equation}
must be zero.

If \(\sum_i{ u_i \overline{v_i}} = 0\), this implies
\(\overline{\sum_i{ u_i \overline{v_i}}} = \sum_i{v_i \overline{u_i}} = 0\), so the definitions of both the
complex and the real inner products arise naturally from an examination of orthogonality constraints.

\section{Introducing the wedge product}

Now, some clever mathematicians have observed that the underlying properties of the determinant is what is important in all these problems of normal, area, and volume.

In particular the property that it changes sign if two of its elements (rows or columns) are switched, and is zero if any of its elements are the same, and it is linear in either variable.

%linearity demo/proof (using cofactor expansion on first row ... any other would do).
% det(a+b, c) = (a_1 + b_1)c_2 - (a_2 + b_2)c_1 = det(a,c) + det(b,c)
% det(a+b, ..., c) = \sum(a_i + b_i)C_1i = \sum(a_i)C_1i +\sum(b_i)C_1i = det(a,...c) + det(b,...,c)

The wedge product is an operator defined based on these two properties.  Symbolically, for two vectors \(\Bu\) and \(\Bv\), these rules are:

\begin{equation}\label{eqn:cross:1741}
\begin{aligned}
\Bu \wedge \Bv &= - (\Bv \wedge \Bu) \\
\Bu \wedge \Bu &= \Bzero \\
(\Bu + \Bv) \wedge \Bw &= \Bu \wedge \Bw + \Bv \wedge \Bw
\end{aligned}
\end{equation}

And for a third vector the wedge defined as:
\begin{equation}\label{eqn:cross:401}
\Bu \wedge \Bv \wedge \Bw = (\Bu \wedge \Bv) \wedge \Bw = \Bu \wedge (\Bv \wedge \Bw)
\end{equation}

Now, intuitively this is a more awkward sort of product than the dot or cross product.  It is it is own thing, in general having no direct mapping to a vector (like the cross product), nor to a scalar (like the dot product).  We will see the dimensions of this beast is not even necessarily close to the dimensions of the original vector space (and that dimension varies according to how many wedge products are composed).

To get a feel for this quantity, an expansion of this in terms of components is helpful (I would have liked to have seen this spelled out in my books for the dumb reader like me).

With
\( \Bu = \sum_i{u_i \ecap_i} \), and \( \Bv = \sum_i{v_i \ecap_i}\), here is the expansion of \(\Bu \wedge \Bv\) .

\begin{equation}\label{eqn:cross:1761}
\begin{aligned}
\Bu \wedge \Bv
&= \left(\sum_i{u_i \ecap_i}\right) \wedge \left(\sum_j{v_j \ecap_j}\right) \\
&= \sum_{i,j}{u_i v_j \left(\ecap_i \wedge \ecap_j\right)} \\
&= \sum_{i \neq j}{u_i v_j \left(\ecap_i \wedge \ecap_j\right)} \\
&= \sum_{i < j}{u_i v_j \left(\ecap_i \wedge \ecap_j\right)} +
   \sum_{j' < i'}{u_{i'} v_{j'} \left(\ecap_{i'} \wedge \ecap_{j'}\right)} \\
&= \sum_{i < j}\left(u_i v_j - u_{j} v_{i}\right)\left(\ecap_i \wedge \ecap_j\right) \\
&= \sum_{i < j}\DETuvij{u}{v}{i}{j}\left(\ecap_i \wedge \ecap_j\right) \\
&= \sum_{i < j}D_{ij}^{\Bu\Bv}\left(\ecap_i \wedge \ecap_j\right)
\end{aligned}
\end{equation}

Introducing a third wedge:
\begin{equation}\label{eqn:cross:1781}
\begin{aligned}
\Bu \wedge \Bv \wedge \Bw
&= \left(\sum_{i < j}D_{ij}^{\Bu\Bv}\left(\ecap_i \wedge \ecap_j\right)\right) \wedge
\Bw = \sum_k{w_k \ecap_k} \\
&= \sum_{i < j, k}w_k D_{ij}^{\Bu\Bv}\left(\ecap_i \wedge \ecap_j \wedge \ecap_k\right) \\
&= \sum_{i < j, k \neq i,j}w_k D_{ij}^{\Bu\Bv}\left(\ecap_i \wedge \ecap_j \wedge \ecap_k\right) \\
&=
\left(\sum_{i < j < k} +
\sum_{i < k < j} +
\sum_{k < i < j}\right)
w_k D_{ij}^{\Bu\Bv}\left(\ecap_i \wedge \ecap_j \wedge \ecap_k\right) \\
&=
\sum_{i < j < k} \left(w_k D_{ij}^{\Bu\Bv}\left(\ecap_i \wedge \ecap_j \wedge \ecap_k\right) +
w_j D_{ik}^{\Bu\Bv}\left(\ecap_i \wedge \ecap_k \wedge \ecap_j\right) +
w_i D_{jk}^{\Bu\Bv}\left(\ecap_j \wedge \ecap_k \wedge \ecap_i\right)\right)
 \\
&=
\sum_{i < j < k} \left(w_k D_{ij}^{\Bu\Bv} - w_j D_{ik}^{\Bu\Bv} + w_i D_{jk}^{\Bu\Bv}\right)
\left(\ecap_i \wedge \ecap_j \wedge \ecap_k\right)
 \\
&=
\sum_{i < j < k} D_{ijk}^{\Bu\Bv\Bw}\left(\ecap_i \wedge \ecap_j \wedge \ecap_k\right)
 \\
\end{aligned}
\end{equation}

Written out in full, these are:
\begin{equation}\label{eqn:cross:1801}
\begin{aligned}
\Bu \wedge \Bv
&=
\sum_{i < j} \DETuvij{u}{v}{i}{j}\left(\ecap_i \wedge \ecap_j\right)  \\
\Bu \wedge \Bv \wedge \Bw
&=
\sum_{i < j < k} \DETuvwijk{u}{v}{w}{i}{j}{k}\left(\ecap_i \wedge \ecap_j \wedge \ecap_k\right)
 \\
\end{aligned}
\end{equation}

A couple observations.

The set
$\left\lbrace \ecap_i \wedge \ecap_j
| i < j
 \right\rbrace
$ is
a basis for the two vector ``wedge product space'', and the set
$\left\lbrace \ecap_i \wedge \ecap_j \wedge \ecap_k
| i < j < k
\right\rbrace
$ is
a basis for the three vector ``wedge product space''.

With these values taken as the basis ``vectors'' for the product space, the natural dot product of two
vectors would be:

\begin{equation}\label{eqn:cross:1821}
\begin{aligned}
\dotprod{(\Bu \wedge \Bv)}{(\Bw \wedge \Bx)}
&=
\dotprod{\left(\sum_{i < j}D_{ij}^{\Bu\Bv}\left(\ecap_i \wedge \ecap_j\right)\right)}
{\left(\sum_{{s} < {t}}D_{{s}{t}}^{\Bw\Bx}\left(\ecap_{s} \wedge \ecap_{t}\right)\right)} \\
&=
\sum_{i < j}\sum_{s < t} D_{ij}^{\Bu\Bv} D_{{s}{t}}^{\Bw\Bx}
\dotprod{\left(\ecap_i \wedge \ecap_j\right)}{\left(\ecap_{s} \wedge \ecap_{t}\right)} \\
&=
\sum_{i < j}\sum_{s < t} D_{ij}^{\Bu\Bv} D_{{s}{t}}^{\Bw\Bx} \delta_{ij,st} \\
&=
\sum_{i < j}D_{ij}^{\Bu\Bv} D_{ij}^{\Bw\Bx}
\end{aligned}
\end{equation}

And in particular, this defines the length of a wedge product, which can be written in terms of the area of the parallelogram spanned by the two ``wedged'' vectors.

\begin{equation}\label{eqn:cross:1841}
\begin{aligned}
\norm{(\Bu \wedge \Bv)}^2
&= \sum_{i < j}\left(D_{ij}^{\Bu\Bv}\right)^2 \\
&= \left(\Area(\Bu,\Bv)\right)^2
\end{aligned}
\end{equation}

For the length of a three vector wedge, we have a length equivalent to the volume of the parallelepiped spanned by the three vectors:
% I think of a proof of this is not neccessary.  Exactly like the above.
\begin{equation}\label{eqn:cross:1861}
\begin{aligned}
\norm{(\Bu \wedge \Bv \wedge \Bw)}^2
&= \sum_{i < j < k}\left(D_{ijk}^{\Bu\Bv\Bw}\right)^2 \\
&= \left(\Volume(\Bu,\Bv,\Bw)\right)^2
\end{aligned}
\end{equation}

So, the elements
\(D_{i_1 \dotsb i_M}^{\Bu^1 \dotsb \Bu^M}\left( \ecap_{i_1} \wedge \dotsb \wedge \ecap_{i_M} \right)\)
of the wedge product
can be thought of as oriented ``\(\Volume\)'' elements of the subspace spanned by the \(M\) vectors.

\subsection{Comparing the wedge product to the normal of \texorpdfstring{\(N-1\)}{N - 1} independent vectors in \texorpdfstring{\R{N}}{R N}}

We have three variations now that generalize the cross product of two vectors in different ways:

\begin{equation}\label{eqn:cross:1881}
\begin{aligned}
\Bu \wedge \Bv &= \sum_{i < j} \DETuvij{u}{v}{i}{j}\left(\ecap_i \wedge \ecap_j\right)  \\
\Proj_{\perp\ucap,\vcap}(\Bw)
&= \left({\sum_{i<j} \left(D_{ij}^{\Bu \Bv}\right)^2}\right)^{-2}\sum_{i<j<k} \DETuvwijk{u}{v}{w}{i}{j}{k} \DETuvwijk{u}{v}{\ecap}{i}{j}{k} \\
\Bn(\Bu,\Bv)
&= \sum_{i<j<k} \left( \sum_{\pi_x \in \pi(i,j,k)} s_{\pi_x}\Sgn(\pi_x) \right) \DETuvwijk{u}{v}{\ecap}{i}{j}{k} \\
\end{aligned}
\end{equation}

The wedge product of \(N-1\) \R{N} vectors and the normal those \(N-1\) vectors (assuming that they are all linearly independent) are a very close match.  Let us compare these for \R{3}, \R{4} and \R{5}.

%\R{2}:
%\begin{align*}
%\Bn(\Bu)
%& \propto \DETuvij{u}{\ecap}{1}{2} \\
%&= (+\ecap_1) |u_2| \\
%&+ (-\ecap_2) |u_1|
%\end{align*}

For \R{3} the normal to two vectors is of the following form:
\begin{equation}\label{eqn:cross:1901}
\begin{aligned}
\Bn(\Bu,\Bv)
&\propto \DETuvwijk{u}{v}{\ecap}{1}{2}{3} \\
&= (+\ecap_1) \DETuvij{u}{v}{2}{3} \\
&+ (-\ecap_2) \DETuvij{u}{v}{1}{3} \\
&+ (+\ecap_3) \DETuvij{u}{v}{1}{2} \\
\end{aligned}
\end{equation}

Compare this to the wedge product of two \R{3} vectors:
\begin{equation}\label{eqn:cross:1921}
\begin{aligned}
\Bu \wedge \Bv
&= (\ecap_2 \wedge \ecap_3) \DETuvij{u}{v}{2}{3} \\
&+ (\ecap_1 \wedge \ecap_3) \DETuvij{u}{v}{1}{3} \\
&+ (\ecap_1 \wedge \ecap_2) \DETuvij{u}{v}{1}{2} \\
\end{aligned}
\end{equation}

Similarly, for \R{4} the normal to three vectors is of the following form:

\R{4}:
\begin{equation}\label{eqn:cross:1941}
\begin{aligned}
\Bn(\Bu,\Bv,\Bw)
&\propto \DETuvwxijkl{u}{v}{w}{\ecap}{1}{2}{3}{4} \\
&= (+\ecap_1) \DETuvwijk{u}{v}{w}{2}{3}{4} + (-\ecap_2) \DETuvwijk{u}{v}{w}{1}{3}{4} \\
&+ (+\ecap_3) \DETuvwijk{u}{v}{w}{1}{2}{4} + (-\ecap_4) \DETuvwijk{u}{v}{w}{1}{2}{3} \\
\end{aligned}
\end{equation}

Compare this to the wedge product of three \R{4} vectors:

\begin{equation}\label{eqn:cross:1961}
\begin{aligned}
\Bu \wedge \Bv \wedge \Bw
&= (\ecap_{2} \wedge \ecap_{3} \wedge \ecap_{4}) \DETuvwijk{u}{v}{w}{2}{3}{4}
+ (\ecap_{1} \wedge \ecap_{3} \wedge \ecap_{4}) \DETuvwijk{u}{v}{w}{1}{3}{4} \\
&+ (\ecap_{1} \wedge \ecap_{2} \wedge \ecap_{4}) \DETuvwijk{u}{v}{w}{1}{2}{4}
+ (\ecap_{1} \wedge \ecap_{2} \wedge \ecap_{3}) \DETuvwijk{u}{v}{w}{1}{2}{3} \\
%\sum_{i < j < k} \DETuvwijk{u}{v}{w}{i}{j}{k}\left(\ecap_i \wedge \ecap_j \wedge \ecap_k\right)
\end{aligned}
\end{equation}

And finally, for \R{5} the normal to four vectors is of the following form:
\begin{equation}\label{eqn:cross:1981}
\begin{aligned}
\Bn(\Bu,\Bv,\Bw,\Bx)
&\propto
%\DETuvwxyijklm{u}{v}{w}{x}{\ecap}{1}{2}{3}{4}{5} \\
\begin{vmatrix}
 {u}_{1} & {u}_{2} & {u}_{3} & {u}_{4} & {u}_{5} \\
 {v}_{1} & {v}_{2} & {v}_{3} & {v}_{4} & {v}_{5} \\
 {w}_{1} & {w}_{2} & {w}_{3} & {w}_{4} & {w}_{5} \\
 {x}_{1} & {x}_{2} & {x}_{3} & {x}_{4} & {x}_{5} \\
 {\ecap}_{1} & {\ecap}_{2} & {\ecap}_{3} & {\ecap}_{4} & {\ecap}_{5}
\end{vmatrix} \\
&= (+\ecap_1) \DETuvwxijkl{u}{v}{w}{x}{2}{3}{4}{5} \\
&+ (-\ecap_2) \DETuvwxijkl{u}{v}{w}{x}{1}{3}{4}{5} \\
&+ (+\ecap_3) \DETuvwxijkl{u}{v}{w}{x}{1}{2}{4}{5} \\
&+ (-\ecap_4) \DETuvwxijkl{u}{v}{w}{x}{1}{2}{3}{4} \\
&+ (+\ecap_5) \DETuvwxijkl{u}{v}{w}{x}{1}{2}{3}{4} \\
\end{aligned}
\end{equation}

Compare this to the wedge product of four \R{5} vectors:
\begin{equation}\label{eqn:cross:2001}
\begin{aligned}
\Bu \wedge \Bv \wedge \Bw \wedge \Bx
%%% \sum_{i < j < k < l} \DETuvwijkl{u}{v}{w}{x}{i}{j}{k}{l}\left(\ecap_i \wedge \ecap_j \wedge \ecap_k \wedge \ecap_l \right)
&= \left(\ecap_{2} \wedge \ecap_{3} \wedge \ecap_{4} \wedge \ecap_{5} \right) \DETuvwxijkl{u}{v}{w}{x}{2}{3}{4}{5} \\
&+ \left(\ecap_{1} \wedge \ecap_{3} \wedge \ecap_{4} \wedge \ecap_{5} \right) \DETuvwxijkl{u}{v}{w}{x}{1}{3}{4}{5} \\
&+ \left(\ecap_{1} \wedge \ecap_{2} \wedge \ecap_{4} \wedge \ecap_{5} \right) \DETuvwxijkl{u}{v}{w}{x}{1}{2}{4}{5} \\
&+ \left(\ecap_{1} \wedge \ecap_{2} \wedge \ecap_{3} \wedge \ecap_{4} \right) \DETuvwxijkl{u}{v}{w}{x}{1}{2}{3}{4} \\
&+ \left(\ecap_{1} \wedge \ecap_{2} \wedge \ecap_{3} \wedge \ecap_{4} \right) \DETuvwxijkl{u}{v}{w}{x}{1}{2}{3}{4} \\
\end{aligned}
\end{equation}

We see that the wedge product is a normal to two vectors in \R{3} (in fact is the cross product) if we make the identities:
\begin{equation}\label{eqn:cross:2021}
\begin{aligned}
 \ecap_1 &= \ecap_2 \wedge \ecap_3 \\
-\ecap_2 &= \ecap_1 \wedge \ecap_3 \\
 \ecap_3 &= \ecap_1 \wedge \ecap_2 \\
\end{aligned}
\end{equation}

And, the wedge product is a normal to three vectors in \R{4} if we make the identities:
\begin{equation}\label{eqn:cross:2041}
\begin{aligned}
  \ecap_1 &= \ecap_{2} \wedge \ecap_{3} \wedge \ecap_{4} \\
 -\ecap_2 &= \ecap_{1} \wedge \ecap_{3} \wedge \ecap_{4} \\
  \ecap_3 &= \ecap_{1} \wedge \ecap_{2} \wedge \ecap_{4} \\
 -\ecap_4 &= \ecap_{1} \wedge \ecap_{2} \wedge \ecap_{3} \\
\end{aligned}
\end{equation}

And, the wedge product is a normal to four vectors in \R{5} if we make the identities:
\begin{equation}\label{eqn:cross:2061}
\begin{aligned}
  \ecap_1 &= \ecap_{2} \wedge \ecap_{3} \wedge \ecap_{4} \wedge \ecap_{5}   \\
 -\ecap_2 &= \ecap_{1} \wedge \ecap_{3} \wedge \ecap_{4} \wedge \ecap_{5}   \\
  \ecap_3 &= \ecap_{1} \wedge \ecap_{2} \wedge \ecap_{4} \wedge \ecap_{5}   \\
 -\ecap_4 &= \ecap_{1} \wedge \ecap_{2} \wedge \ecap_{3} \wedge \ecap_{4}   \\
  \ecap_5 &= \ecap_{1} \wedge \ecap_{2} \wedge \ecap_{3} \wedge \ecap_{4}   \\
\end{aligned}
\end{equation}

So, as well as \(\Bu \wedge \Bv\) being an oriented parallelogram area vector, and \(\Bu \wedge \Bv \wedge \Bw\) being an oriented parallelepiped volume vector, 
the wedge product of \(N-1\) linearly independent vectors in \R{N} 
is also normal to those vectors if we make the identity:

\begin{equation}\label{eqn:cross:421}
  \ecap_j = \Sgn(j, i_1, \dotsc, i_{N-1}) \ecap_{i_1} \wedge \dotsb \wedge \ecap_{i_{N-1}}
\end{equation}
Or,
\begin{equation}\label{eqn:cross:2081}
\begin{aligned}
\ecap_{i_1} \wedge \dotsb \wedge \ecap_{i_{N-1}} &= \Sgn(j, i_1, \dotsc, i_{N-1}) \ecap_j \\
                                                 &= (-1)^{j+1} \ecap_j 
\end{aligned}
\end{equation}

Where \(i_1 < i_2 < \dotsb < i_{N-1}\) and \(j \notin \lbrace i_1, \dotsc, i_{N-1} \rbrace\).

\subsection{Comparing the wedge product and the cross product and triple product}

For the wedge of \(N-1\) vectors in \R{N}, and using the mapping from this wedge product to a normal to these vectors we have:

\begin{equation}\label{eqn:cross:2101}
\begin{aligned}
\Bu^{1} \wedge \dotsb \wedge \Bu^{N-1} &=
\sum_{i_1<\dotsb <i_{N-1}}{ D_{i_1 \dotsb i_{N-1}}^{\Bu^{1}\dotsc\Bu^{N-1}}\left( \ecap_{i_1} \wedge \dotsb \wedge \ecap_{i_{N-1}} \right) } \\
&=
\sum_{i_1<\dotsb <i_{N-1}}{ D_{i_1 \dotsb i_{N-1}}^{\Bu^{1}\dotsc\Bu^{N-1}}
\left( 
%\ecap_{i_1} \wedge \dotsb \wedge \ecap_{i_{N-1}} 
\Sgn(j, i_1, \dotsc, i_{N-1}) \ecap_j
\right)
} \\
\end{aligned}
\end{equation}

\begin{equation}\label{eqn:cross:Nminus1Product}
\Rightarrow
\Bu^{1} \wedge \dotsb \wedge \Bu^{N-1} =
\sum_{i_1<\dotsb <i_{N-1}}{ D_{i_1 \dotsb i_{N-1}}^{\Bu^{1}\dotsc\Bu^{N-1}}
\left( 
%\Sgn(j, i_1, \dotsc, i_{N-1}) \ecap_j
(-1)^{j+1} \ecap_j 
\right)
}
\end{equation}

Where, as before \(j \notin \lbrace i_1, \dotsc, i_{N-1} \rbrace\).

(Note that for \R{3} this is exactly the cross product \(\crossprod{\Bu^1}{\Bu^2}\).)

Continuing with the wedge of \(N-1\) vectors as a normal in \R{N} we can write:

\begin{equation}\label{eqn:cross:2121}
\begin{aligned}
\dotprod{\left(\Bu^{1} \wedge \dotsb \wedge \Bu^{N-1}\right)}{\Bu^N}
&=
\sum_{j,i_1<\dotsb <i_{N-1}}
   { 
      D_{i_1 \dotsb i_{N-1}}^{\Bu^{1}\dotsc\Bu^{N-1}}
      \left( (-1)^{j+1} u_j^N \right)
   } \\
&=
D_{1 \dotsb N}^{\Bu^{1}\dotsc\Bu^{N}}
\end{aligned}
\end{equation}

But,
\begin{equation}\label{eqn:cross:441}
\Bu^{1} \wedge \dotsb \wedge \Bu^{N} =
D_{1 \dotsb N}^{\Bu^{1}\dotsc\Bu^{N}}
\left(\ecap_{1} \wedge \dotsb \wedge \ecap_{N}\right)
\end{equation}
\begin{equation}\label{eqn:cross:461}
\Rightarrow
\Bu^{1} \wedge \dotsb \wedge \Bu^{N} =
\left(\dotprod{\left(\Bu^{1} \wedge \dotsb \wedge \Bu^{N-1}\right)}{\Bu^N} \right)
\left(\ecap_{1} \wedge \dotsb \wedge \ecap_{N}\right)
\end{equation}

This result is very much like the \R{3} triple product:

\begin{equation}\label{eqn:cross:481}
\tripleprod{\Bu}{\Bv}{\Bw} = \DETuvwijk{u}{v}{w}{1}{2}{3}
\end{equation}

This should not be surprising since \eqnref{eqn:cross:Nminus1Product} was the \R{3} cross product.



Explicitly expanding this \R{N} wedge of \(N\) vectors for a couple of dimensions to illustrate.

For \R{2}:
\begin{equation}\label{eqn:cross:501}
\Bu \wedge \Bv = \DETuvij{u}{v}{1}{2} (\ecap_1 \wedge \ecap_2)
\end{equation}
(Here the determinant is the oriented area of the \R{2} parallelogram formed by vectors \(\Bu\) and \(\Bv\).)

For \R{3}:
\begin{equation}\label{eqn:cross:521}
\Bu \wedge \Bv \wedge \Bw = \DETuvwijk{u}{v}{w}{1}{2}{3} (\ecap_1 \wedge \ecap_2 \wedge \ecap_3)
\end{equation}
(Here the determinant is the oriented volume of the \R{3} parallelepiped formed by vectors \(\Bu\), \(\Bv\), and \(\Bw\).)

And for \R{4}:
\begin{equation}\label{eqn:cross:541}
\Bu \wedge \Bv \wedge \Bw \wedge \Bx = \DETuvwxijkl{u}{v}{w}{x}{1}{2}{3}{4} (\ecap_1 \wedge \ecap_2 \wedge \ecap_3 \wedge \ecap_4)
\end{equation}
(Here the determinant is the oriented \(4\Volume\) of the \R{4} parallelo-4gram formed by vectors \(\Bu\), \(\Bv\), \(\Bw\), and \(\Bx\))

Each of these are one dimensional wedge product space vectors can be thought of as directed 
areas, volumes, or N-volumes.  There are two ways that the sign can vary, one is due to the ordering of the vectors themselves, and the 
other is due to the ordering of the \(\ecap_{i_{1}} \wedge \dotsb \wedge \ecap_{i_{N}}\) term. Picking the ordering of that
term is equivalent to picking a basis for the wedge product space.  Once that basis is picked it defines an isomorphism with \R{1}.


What this does do is put the wedge product into a context that we are used to.  

We can identify the \(N-1\) vector wedge product with the cross product at least with respect to its normal properties.

We can identify the wedge product of \(N\) vectors with the triple product (which is better described as an \(N\) product, valid for \(N \geq 2\)).
This \(N\) product is a scalar, and in particular has 
a geometric meaning which is the area, volume, ... of the parallelo-Ngram formed by the span of the vectors in question.

\subsection{Comparing the wedge product and the general normal to some vectors}

In general can we put the normal equations in a form closer to that of the wedge product?

\begin{equation}\label{eqn:cross:2141}
\begin{aligned}
\Proj_{\perp\ucap,\vcap}(\Bw)
\left({\sum_{i<j} \left(D_{ij}^{\Bu \Bv}\right)^2}\right)^2
&=
\sum_{i<j<k} \DETuvwijk{u}{v}{w}{i}{j}{k} \DETuvwijk{u}{v}{\ecap}{i}{j}{k} \\
&=
\sum_{i<j<k} \DETuvwijk{u}{v}{w}{i}{j}{k}
\left(\ecap_i \DETuvij{u}{v}{j}{k}
+\ecap_j \DETuvij{u}{v}{k}{i}
+\ecap_k \DETuvij{u}{v}{i}{j} \right) \\
&=
\sum_{t<i<j} \DETuvwijk{u}{v}{w}{t}{i}{j}
\ecap_t \DETuvij{u}{v}{i}{j} \\
&-
\sum_{i<t<j} \DETuvwijk{u}{v}{w}{i}{t}{j}
\ecap_t \DETuvij{u}{v}{i}{j} \\
&+
\sum_{i<j<t} \DETuvwijk{u}{v}{w}{i}{j}{t}
\ecap_t \DETuvij{u}{v}{i}{j} \\
&=
\sum_{i<j} \left( \sum_t \DETuvwijk{u}{v}{w}{t}{i}{j} \ecap_t\right) \DETuvij{u}{v}{i}{j} \\
\end{aligned}
\end{equation}

The same can be done for the general normal to two vectors,

\begin{equation}\label{eqn:cross:2161}
\begin{aligned}
\Bn(\Bu,\Bv)
&= \sum_{i<j<k} \left( \sum_{\pi_x \in \pi(i,j,k)} s_{\pi_x}\Sgn(\pi_x) \right)
\left(\ecap_i \DETuvij{u}{v}{j}{k}
+\ecap_j \DETuvij{u}{v}{k}{i}
+\ecap_k \DETuvij{u}{v}{i}{j} \right) \\
&=
\sum_{t<i<j}
\left( \sum_{\pi_x \in \pi(t,i,j)} s_{\pi_x}\Sgn(\pi_x) \right)
\ecap_t \DETuvij{u}{v}{i}{j} \\
&- \sum_{i<t<j}
\left( \sum_{\pi_x \in \pi(i,t,j)} s_{\pi_x}\Sgn(\pi_x) \right)
\ecap_t \DETuvij{u}{v}{i}{j} \\
&+\sum_{i<j<t}
\left( \sum_{\pi_x \in \pi(i,j,t)} s_{\pi_x}\Sgn(\pi_x) \right)
\ecap_t \DETuvij{u}{v}{i}{j} \\
&=
\sum_{i<j}
\left( \sum_{t=1}^{n} \ecap_t \sum_{\pi_x \in \pi(t,i,j)} s_{\pi_x}\Sgn(\pi_x) \right)
\DETuvij{u}{v}{i}{j} \\
\end{aligned}
\end{equation}

Let us expand the \(\Proj_{\perp\ucap,\vcap}(\Bw)\) result for a few specific examples \R{3}, \R{4} and \R{5} to get a feel for it.

Here is the \R{3} case:
\begin{equation}\label{eqn:cross:2181}
\begin{aligned}
\Bn_{(\Bu,\Bv)}(\Bw)
&=
\Proj_{\perp\ucap,\vcap}(\Bw)
\left({\sum_{ij=12,23,13} \left(D_{ij}^{\Bu \Bv}\right)^2}\right)^2 \\
&=
\sum_{ij=12,23,13}
\left(\sum_{t=1}^{3} \ecap_t \DETuvwijk{u}{v}{w}{t}{i}{j} \right)\DETuvij{u}{v}{i}{j} \\
&=
\sum_{ij=12,23,13}
\left(\sum_{t=1}^{3} \ecap_t D_{{t}{i}{j}}^{\Bu\Bv\Bw} \right)
\DETuvij{u}{v}{i}{j} \\
&=
\sum_{t=1}^{3} \ecap_t D_{{t}{1}{2}}^{\Bu\Bv\Bw} \DETuvij{u}{v}{1}{2}
+ \sum_{t=1}^{3} \ecap_t D_{{t}{2}{3}}^{\Bu\Bv\Bw} \DETuvij{u}{v}{2}{3}
+ \sum_{t=1}^{3} \ecap_t D_{{t}{1}{3}}^{\Bu\Bv\Bw} \DETuvij{u}{v}{1}{3} \\
&=
 \ecap_3 D_{{3}{1}{2}}^{\Bu\Bv\Bw} \DETuvij{u}{v}{1}{2}
+ \ecap_1 D_{{1}{2}{3}}^{\Bu\Bv\Bw} \DETuvij{u}{v}{2}{3}
+ \ecap_2 D_{{2}{1}{3}}^{\Bu\Bv\Bw} \DETuvij{u}{v}{1}{3} \\
&=
D_{{1}{2}{3}}^{\Bu\Bv\Bw} \left(
 \ecap_3 \DETuvij{u}{v}{1}{2}
+\ecap_1 \DETuvij{u}{v}{2}{3}
+\ecap_2 \DETuvij{u}{v}{3}{1}
\right) \\
&=
\DETuvwijk{u}{v}{w}{1}{2}{3} \left(
 \ecap_3 \DETuvij{u}{v}{1}{2}
+\ecap_1 \DETuvij{u}{v}{2}{3}
+\ecap_2 \DETuvij{u}{v}{3}{1}
\right)
\end{aligned}
\end{equation}

And here is the \R{4} case:
\begin{equation}\label{eqn:cross:2201}
\begin{aligned}
\Bn_{(\Bu,\Bv)}(\Bw)
&=
\Proj_{\perp\ucap,\vcap}(\Bw)
\left({\sum_{ij=12,13,14,23,24,34} \left(D_{ij}^{\Bu \Bv}\right)^2}\right)^2 \\
&=
\sum_{ij=12,13,14,23,24,34} \left( \sum_t \DETuvwijk{u}{v}{w}{t}{i}{j} \ecap_t\right) \DETuvij{u}{v}{i}{j} \\
&=
\sum_{t=1}^{4} \ecap_t D_{{t}{1}{2}}^{\Bu\Bv\Bw} \DETuvij{u}{v}{1}{2}
+
\sum_{t=1}^{4} \ecap_t D_{{t}{1}{3}}^{\Bu\Bv\Bw} \DETuvij{u}{v}{1}{3} \\
&+
\sum_{t=1}^{4} \ecap_t D_{{t}{1}{4}}^{\Bu\Bv\Bw} \DETuvij{u}{v}{1}{4}
+
\sum_{t=1}^{4} \ecap_t D_{{t}{2}{3}}^{\Bu\Bv\Bw} \DETuvij{u}{v}{2}{3} \\
&+
\sum_{t=1}^{4} \ecap_t D_{{t}{2}{4}}^{\Bu\Bv\Bw} \DETuvij{u}{v}{2}{4}
+
\sum_{t=1}^{4} \ecap_t D_{{t}{3}{4}}^{\Bu\Bv\Bw} \DETuvij{u}{v}{3}{4} \\
&=
\left(
\ecap_{3} D_{{3}{1}{2}}^{\Bu\Bv\Bw}
+\ecap_{4} D_{{4}{1}{2}}^{\Bu\Bv\Bw}
\right)
\DETuvij{u}{v}{1}{2}
+
\left(
\ecap_{2} D_{{2}{1}{3}}^{\Bu\Bv\Bw}
+\ecap_{4} D_{{4}{1}{3}}^{\Bu\Bv\Bw}
\right)
\DETuvij{u}{v}{1}{3} \\
&+
\left(
\ecap_{2} D_{{2}{1}{4}}^{\Bu\Bv\Bw}
+\ecap_{3} D_{{3}{1}{4}}^{\Bu\Bv\Bw}
\right)
\DETuvij{u}{v}{1}{4}
+
\left(
\ecap_{1} D_{{1}{2}{3}}^{\Bu\Bv\Bw}
+\ecap_{4} D_{{4}{2}{3}}^{\Bu\Bv\Bw}
\right)
\DETuvij{u}{v}{2}{3} \\
&+
\left(
\ecap_{1} D_{{1}{2}{4}}^{\Bu\Bv\Bw}
+\ecap_{3} D_{{3}{2}{4}}^{\Bu\Bv\Bw}
\right)
\DETuvij{u}{v}{2}{4}
+
\left(
\ecap_{1} D_{{1}{3}{4}}^{\Bu\Bv\Bw}
+\ecap_{2} D_{{2}{3}{4}}^{\Bu\Bv\Bw}
\right)
\DETuvij{u}{v}{3}{4} \\
\end{aligned}
\end{equation}

For \R{5} the set of indices \(\lbrace{ij}\rbrace = \lbrace{12,13,14,15,23,24,25,34,35,45}\rbrace\).  The
coefficients of the \(\DETuvij{u}{v}{i}{j}\) terms are \(\sum_{t \neq i,j} \ecap_t D_{tij}^{\Bu\Bv\Bw}\), so for
\R{5} this will be three terms.

So, we can equate the vector normal to the plane of \(\Bu\) and \(\Bv\) in the direction of \(\Bw\) if one introduces the following mapping:

\begin{equation}\label{eqn:cross:561}
\ecap_i \wedge \ecap_j = \frac{\sum_{t \neq i,j} \ecap_t D_{tij}^{\Bu\Bv\Bw}}
{\sum_{i<j} \left(D_{ij}^{\Bu\Bv}\right)^2}
\end{equation}

Comparing to the normal to the \(\Bu,\Bv\) plane in an unspecified direction:
\(\Bn = \sum s_{ijk}D_{ijk}^{\Bu\Bv\Be}\)
we can treat \(\Bu \wedge \Bv\) as a normal if we write:

\begin{equation}\label{eqn:cross:581}
\ecap_i \wedge \ecap_j =
\left( \sum_{t \neq i \neq j} \ecap_t \sum_{\pi_x \in \pi(t,i,j)} s_{\pi_x}\Sgn(\pi_x) \right)
\end{equation}

Similarly we can think of
\(\Bu \wedge \Bv \wedge \Bw\) as a normal to the \(\Bu,\Bv,\Bw\) volume, \(\Bn = \sum{s_{ijkl}D_{ijkl}^{\Bu\Bv\Bw\Be}}\),  if we write:

\begin{equation}\label{eqn:cross:601}
\ecap_i \wedge \ecap_j \wedge \ecap_k =
\left( \sum_{t \neq i \neq j \neq k} \ecap_t \sum_{\pi_x \in \pi(t,i,j,k)} s_{\pi_x}\Sgn(\pi_x) \right)
\end{equation}

%  Example for \R{4}
%
%\begin{align*}
%\Bn(\Bu,\Bv,\Bw) =
%\left( \sum_{t \neq i \neq j \neq k} \sum_{\pi_x \in \pi(t,i,j,k)} s_{\pi_x}\Sgn(\pi_x) \right)
%\left( \sum_{i<j<k<l} \ecap_
%\sum_{\pi_x \in \pi(t,i,j,k)} s_{\pi_x}\Sgn(\pi_x) \right)
%\end{align*}

This is not exactly a natural seeming correspondence, unlike the \(N-1\) vector case.  There it did make some sense to treat
the wedge product as a normal form, but only in the \(N-1\) vector case is there an unambiguous normal form (apart from the scalar multiplier).

\subsection{A first attempt to tie in to differential forms.  Dot product of wedge products as normal times magnitude}

There is a natural scenario where it does make sense to equate the normal and the wedge.

By example, if one has a surface function with components f, and g in the \(x_1,x_2\) and \(x_2,x_3\) planes respectively, one could in \R{3} express this as:

\begin{equation}\label{eqn:cross:621}
\Bv(\Br) = f(\Br) \ecap_3 + g(\Br) \ecap_1
\end{equation}

As illustrated by the generic normal calculations, in \R{4}, 
the normal to the \(x_1,x_2\) plane does not have a unique direction and potentially has components in one or more of the \(\ecap_3\) and \(\ecap_4\) directions, and
the normal to the \(x_2,x_3\) plane potentially has \(\ecap_1\) and \(\ecap_4\) components.

So, it is much more natural to not try to express a directed surface function in terms of a normal that is not well defined unless the ``surface'' is of dimension \(N-1\).  Instead such a directed surface function is best expressed exclusively in terms of its components in each of its planes.  Restating the above in such terms we have:

\begin{equation}\label{eqn:cross:641}
\Bv(\Br) = f(\Br) \ecap_{1,2} + g(\Br) \ecap_{2,3}
\end{equation}

Here \(\ecap_{1,2}\) indicates that \(f(\Br)\) is the component of the surface function \(\Bv\) that is in the \(x_1,x_2\) plane and \(\ecap_{2,3}\) means that \(g(\Br)\) is the component of \(\Bv\) in the \(x_2,x_3\) plane.

It is natural to use the wedge product to express each of these components, writing:

\begin{equation}\label{eqn:cross:661}
\Bv(\Br) = f(\Br) (\ecap_1 \wedge \ecap_2) + g(\Br) (\ecap_2 \wedge \ecap_3)
\end{equation}

A formulation like this is well regardless of whether the space is \R{3}, \R{4}, \R{5}, or \R{N}, though it does not have any specific physical interpretation.

More generally, a surface function in \R{N} can be expressed as
\begin{equation}\label{eqn:cross:681}
\Bv(\Br) = \sum_{i \neq j} f_{ij}(\Br) \ecap_{i,j}
\end{equation}

and write this as:

\begin{equation}\label{eqn:cross:701}
\Bv(\Br) = \sum_{i<j} f_{ij}(\Br) \ecap_i \wedge \ecap_j
\end{equation}

This and the fact that we can express length, area, volume, n-volumes, as wedge products thus gives us a way to describe flux and work like quantities.  

Recall that a line integral of the following form describes phenomena such as ``work done''.

\begin{equation}\label{eqn:cross:702}
\int_{\Br} \dotprod{\Bv(\Br)}{d\Br} 
\end{equation}

This is the component of a directed field and multiply by the length of a vector in the specified direction.

Similarly a surface integral of the following form describes ``flux through a surface'' like quantities.  We write this as:

\begin{equation}\label{eqn:cross:721}
\int_{\BA} \dotprod{\Bv(\Br)}{d\BA} 
\end{equation}

This is the component of a field in the direction of a surface, is multiplied by the area element for that surface.

Using the example above with components f, and g in the \(x_1,x_2\) and \(x_2,x_3\) planes respectively, one could in \R{3} express the flux for this function as:
 
\begin{equation}\label{eqn:cross:741}
\int_{\Br} \dotprod{\left(f(\Br) \ecap_3 + g(\Br) \ecap_1\right)}{\left(dx_1 dx_2 \ecap_3 + dx_2 dx_3 \ecap_1 + dx_3 dx_1 \ecap_2\right)}
\end{equation}

But again, this is a formulation that is only good for \R{3}.  Also note that there is an implied common orientation
of the normal and the area elements.

Because we can express the component of a surface function as a wedge product and can also express an oriented area element as a wedge product we can 
express this flux quantity in terms of the wedge product and have a formulation that is valid for \R{N} as well as \R{3}.

\begin{equation}\label{eqn:cross:761}
\int
\dotprod{\left(f(\Br) (\ecap_1 \wedge \ecap_3) + g(\Br) (\ecap_2 \wedge \ecap_3)\right)}{\left(\sum_{i<j} d\ecap_i \wedge d\ecap_j\right)}
=
\int f(\Br) dx_1 dx_3 + g(\Br) dx_2 dx_3
\end{equation}

Similarly, say one has a volume function in \R{4} with components f, and g with \(x_1,x_2,x_3\) and \(x_1,x_2,x_4\) cubes, one could express it as:

\begin{equation}\label{eqn:cross:781}
\Bv(\Br) = f(\Br) \ecap_4 + g(\Br) \ecap_3
\end{equation}

But this is only a good description for \R{4}.  If one expressed the same thing as:

\begin{equation}\label{eqn:cross:801}
\Bv(\Br) = f(\Br) (\ecap_1 \wedge \ecap_2 \wedge \ecap_3) + g(\Br) (\ecap_1 \wedge \ecap_2 \wedge \ecap_4)
\end{equation}

This is a good description of the directed volume function for \R{4} as well as any other dimension \R{N}, and one could write a \R{N} volume ``flux'' like integral
for this function as:

\begin{equation}\label{eqn:cross:821}
\int
\dotprod{\left(
f(\Br) (\ecap_1 \wedge \ecap_2 \wedge \ecap_3) + g(\Br) (\ecap_1 \wedge \ecap_2 \wedge \ecap_4)
\right)}
{\left(\sum_{i<j<k} d\ecap_i \wedge d\ecap_j \wedge d\ecap_k\right)}
\end{equation}
\begin{equation}\label{eqn:cross:841}
=
\int f(\Br) dx_1 dx_2 dx_3 + g(\Br) dx_1 dx_2 dx_4
\end{equation}

Here like the surface flux formulation, the volume function has a specific orientation.  That orientation has been defined by 
considering its \(i,j,k\) element as positive in the ``direction'' of \(\ecap_i \wedge \ecap_j \wedge \ecap_k\).

Now, it may not look like much has been gained by introducing the wedge product, but when it comes to parameterizing the surface-integral, or volume-integral 
this allows for a great deal of flexibility.  For example, if the piece of the surface element can be parametrized as a parallelogram in terms of two vectors \(d\Bu\) and \(d\Bv\), then that element of the flux integral above can be expressed as

\begin{equation}\label{eqn:cross:861}
\int
\dotprod{\left(f(\Br) (\ecap_1 \wedge \ecap_3) + g(\Br) (\ecap_2 \wedge \ecap_3)\right)}{\left(d\Bu \wedge d\Bv\right)}
\end{equation}

and using the volume flux example above, if a parallelepiped volume element is parametrized in terms of three vectors 
\(d\Bu\), \(d\Bv\) and \(d\Bw\), then the flux element can be expressed as

\begin{equation}\label{eqn:cross:881}
\int
\dotprod{\left(
f(\Br) (\ecap_1 \wedge \ecap_2 \wedge \ecap_3) + g(\Br) (\ecap_1 \wedge \ecap_2 \wedge \ecap_4)
\right)}{\left(d\Bu \wedge d\Bv \wedge d\Bw\right)}
\end{equation}

%\end{document}               % End of document.

%
% Copyright � 2012 Peeter Joot.  All Rights Reserved.
% Licenced as described in the file LICENSE under the root directory of this GIT repository.
%

%
%
% forked from here everything after 1.15 version of cross.ltx
% this is the original bits.

%\documentclass{article}      % Specifies the document class

%\usepackage{amsmath}

%
% The real thing:
%

\chapter{Early cross product generalization and motivation attempt}
\label{chap:crossOld}
%\author{Peeter Joot}         % Declares the author's name.
\date{ May 2000.  crossOld.tex }

%\begin{document}             % End of preamble and beginning of text.

%\maketitle{}

\section{Introduction/Abstract}

The cross product is an ugly arbitrary seeming sort of beast, but it is a beast that
describes many sorts of physical and mathematical situations.  In vector calculus
cross product terms and it relative the determinant end up occurring all over the place,
and in physics the cross product also occurs in many contexts.
Examples are Stokes theorem, Jacobian transformations, normal equations, the
curl operator, Maxwell's equations, torque, and the list goes on.
In many of
these cases the mathematics has no logical tie to three dimensions, yet
the cross product is an explicitly three dimensional sort of beast.
The cross product and the dot product have some similarities in form
yet the cross product is only defined for \R{3}, while the dot product
can be defined for \R{n} including \(n < 3\), and even extended easily to \C{n}.
The open question remains of how to generalize it and the math that
is related to it to higher dimensions and other mathematical fields.

\section{The cross product in physical situations}

On of the common places where the cross product appears naturally is in the
definition of torque.
The basic definition of torque as a scalar quantity is the product of the radial distance times
the perpendicular force.  The formula in terms of components in three dimensions given a force vector
\(\BF = (F_x, F_y, F_z)\) and the
radial distance \(\Br = (x, y, z)\) is pretty messy, which is the reason it
is typically described by means of a cross product, and
a generalized torque ``vector'' with a magnitude and direction.

%My Feynman book gives a derivation of for the formula for torque in one dimension as
%the differential work per unit rotation.  This derivation is interesting
%because it yields in a simple fashion a torque formula without having to
%introduce the complexities of the cross product or the torque pseudo-vector.  I will
%not reproduce it here, but will go through a generalized derivation for the torque
%equation when the plane of rotation has an arbitrary orientation in space, rather
%than being restricted to the x,y plane (or y,z or z,x).

The torque expression can be seen to be a natural result of the examination
of the differential work per unit rotation.
\footnote{The Feynman lectures, where a one dimension derivation
of torque is given in this fashion for the (x,y) plane.}
A derivation of this torque expression for an arbitrary rotation
in space will be given in the following sections, first in two dimensions then in three.
The expression for angular velocity for a rotational motion will also be derived.  In each of
these physical scenarios it will be seen that the expression for the cross product arises.
These physical preliminaries will
lead to a technique for which a possible higher dimensional cross product can
be formed and also show how a cross product operator can be defined in a convenient
and natural matrix formulation.

%To start things off, some basic vector algebra results will be presented.

\subsection{torque in two dimensions}

%Feynman's derivation of torque in two dimensions was geometrical and also was quite simple.
In
modern physics where torque is a vector in three dimensions
%even if the rotation is constrained to two dimensions
it does not make sense to talk of a two dimensional torque, but
%There is no reason
the magnitude of the torque for a rotation confined to a plane can be defined without
reference to the plane's normal (ie: the third dimension).
%in two dimensions can not be determined even if the
%the third dimension or rotational axis has not been defined.
This
%, strictly speaking,
is what is meant by torque
in this section.
Application of transformations to and from a rotated frame will be used to define an expression for torque in
\R{2}.  This approach can be applied to do the same in \R{3}, yielding a natural occurrence of the mathematical
form known as the cross product.  The cross product is typically first introduced from its projective
definition, but this form does not easily lead to generalizations in higher dimensions.  Using this procedure
the cross product will be shown to be an expression of incremental rotation, and an \R{n} cross product will
be defined by examination of what we will call an \R{n} rotation.

%arrive at the same result, but have the additional benefit of indicating an approach for the same
%problem for \R{3} and paves the way for generalizing the cross product for \R{n}.

If a rotation inducing force \(\BF\) is applied to an object in space with position \(\Br\) then the only component of the
force that will do work is the component perpendicular to the direction \(\Br\).  Let a new coordinate system
with unit vectors \(\{\rcap = \Br/\norm{\Br}\), \(\thetacap\}\) be defined where \(\thetacap\) is the unit vector perpendicular to \(\Br\) in the direction of positive angular increase.  In this coordinate system
for the force \(\BF' = (F_r, F_\theta)\), only the \(F_\theta\)
component does any work.

To transform to the \(r,\theta\) basis, it can be noted that \(\thetacap \propto (-y, x)\).  Thus
\(\rcap = \inv{r}(x,y)\), \(\thetacap = \inv{r}(-y,x)\), and
\footnote{see appendix for a refresh on change of basis calculations and for the \(\BM\) notation used here}
$\BM =
\inv{r}
\Bigl[
\begin{smallmatrix}
 x & y \\
-y & x
\end{smallmatrix}
\Bigr]
$

The work done \(dW\) is
\begin{equation}\label{eqn:crossOld:20}
\begin{aligned}
dW &=\dotprod{\BF}{d\Bl} \\
   &=\dotprod{\BF'} r d\thetacap \\
   &= F_\theta r d\theta
\end{aligned}
\end{equation}

and
\begin{equation}\label{eqn:crossOld:40}
\begin{aligned}
\BF' &= \BM \BF \\
     &=
\inv{r}
\begin{bmatrix}
 x & y \\
-y & x
\end{bmatrix}
\begin{bmatrix}
 F_x \\
 F_y
\end{bmatrix} \\
     &=
\inv{r}
\begin{bmatrix}
 x Fx - y F_y \\
-y Fx + x F_y
\end{bmatrix}
\end{aligned}
\end{equation}

so
\begin{equation*}
dW = (x F_y - y F_x) d\theta
\end{equation*}

What is being called the torque
\(\tau\)
is this scalar quantity
\(\tau = \D{\theta}{W} = x F_y - y F_x\),
the work per unit rotation for a force \(\BF = (F_x, F_y)\)
applied at a point \(\Br = (x, y)\) from the origin about which the rotation occurs.

It is also easily noted that the transformation
\begin{equation*}
\BM =
\inv{r}
\begin{bmatrix}
 x & y \\
-y & x
\end{bmatrix}
=
\begin{bmatrix}
 x/r & y/r \\
-y/r & x/r
\end{bmatrix}
=
\begin{bmatrix}
 \cos\theta & \sin\theta \\
-\sin\theta & \cos\theta
\end{bmatrix}
=-\BR_\theta
={\BR_\theta}^T
={\BR_\theta}^{-1}
\end{equation*}

Where
\(\BR_\theta\) is the transformation matrix for a rotation through an angle \(\theta\).

The torque can also be calculated in an alternate fashion by using the
rotation matrix.
For a rotation
through a small angle \(d\theta\) this transformation becomes,

\begin{equation*}
\BR_{d\theta} =
\begin{bmatrix}
 \cos{d\theta} & -\sin{d\theta} \\
 \sin{d\theta} & \cos{d\theta}
\end{bmatrix}
=
\begin{bmatrix}
 1 & -d\theta \\
 d\theta & 1
\end{bmatrix}
\end{equation*}

and so the displaced vector is
\begin{equation*}
\Br' = \BR_{d\theta} \Br =
\Bigr( \BI +
\begin{bmatrix}
 0 & -d\theta \\
 d\theta & 0
\end{bmatrix}
\Bigl) \Br
\end{equation*}

which gives the differential change in position
\begin{equation*}
d\Br = \Br' - \Br =
(\BR_{d\theta}-\BI) \Br
=
\begin{bmatrix}
 0 & -d\theta \\
 d\theta & 0
\end{bmatrix}
\Br
=
\begin{bmatrix}
 0 & -1 \\
 1 & 0
\end{bmatrix}
\Br d\theta
\end{equation*}

and the work done is

\begin{equation}\label{eqn:crossOld:60}
\begin{aligned}
dW &=\dotprod{\BF}{d\Br} \\
   &=
\begin{bmatrix}
F_x & F_y
\end{bmatrix}
\begin{bmatrix}
 0 & -1 \\
 1 & 0
\end{bmatrix}
\Br d\theta \\
   &=
\begin{bmatrix}
F_y & -F_x
\end{bmatrix}
\begin{bmatrix}
x \\
y
\end{bmatrix}
d\theta \\
   &=
(x F_y - y F_x) d\theta
\end{aligned}
\end{equation}

This same technique can be applied in three and more dimensions, which will be done in
the following sections.

\subsection{torque in three dimensions}
For three dimensions successive rotations in the \(xy, yz\) and \(zx\) planes can be applied

\begin{equation}\label{eqn:crossOld:80}
\begin{aligned}
\BR_{d\theta_{xy}}
&=
\BR_{d\theta_z}
&=
\begin{bmatrix}
1 & -d\theta_z & 0 \\
 d\theta_z & 1 & 0 \\
0 & 0 & 1
\end{bmatrix} \\
\BR_{d\theta_{yz}}
&=
\BR_{d\theta_x}
&=
\begin{bmatrix}
 1 & 0 & 0 \\
 0 & 1 & -d\theta_x \\
 0 & d\theta_x & 1
\end{bmatrix} \\
\BR_{d\theta_{zx}}
&=
\BR_{d\theta_y}
&=
\begin{bmatrix}
 1 & 0 & d\theta_y \\
 0 & 1 & 0 \\
 -d\theta_y & 0 & 1
\end{bmatrix} \\
\end{aligned}
\end{equation}

Applying these transformations in sequence is a bit messy, but certainly easier than
applying three successive large rotations in sequence.  The mess of sine and cosine terms
for that is horrendous if you care to try!

The calculation for the sequential application of
\(\BR_{d\theta_{xy}}\)
,
\(\BR_{d\theta_{yz}}\)
and
\(\BR_{d\theta_{zx}}\)
is below.

\begin{multline*}
\BR_{d\theta_{zx}}
\BR_{d\theta_{yz}}
\BR_{d\theta_{xy}}
%
=
%
\BR_{d\theta_y}
\BR_{d\theta_x}
\BR_{d\theta_z} \\
%
=
%
% \BR_\theta_y
\begin{bmatrix}
 1 & 0 & d\theta_y \\
 0 & 1 & 0 \\
 -d\theta_y & 0 & 1
\end{bmatrix}
% \BR_{d\theta_x}
\begin{bmatrix}
 1 & 0 & 0 \\
 0 & 1 & -d\theta_x \\
 0 & d\theta_x & 1
\end{bmatrix}
% \BR_{d\theta_z}
\begin{bmatrix}
1 & -d\theta_z & 0 \\
d\theta_z & 1 & 0 \\
0 & 0 & 1
\end{bmatrix} \\
%
=
%
\begin{bmatrix}
1 & d\theta_x\,d\theta_y & d\theta_y \\
0 & 1 & -d\theta_x \\
-d\theta_y & 0 & 1
\end{bmatrix}
% \BR_{d\theta_z}
\begin{bmatrix}
1 & -d\theta_z & 0 \\
d\theta_z & 1 & 0 \\
0 & 0 & 1
\end{bmatrix} \\
%
=
%
\begin{bmatrix}
1+d\theta_x\,d\theta_y\,d\theta_z & -d\theta_z + d\theta_x\,d\theta_y & d\theta_y \\
d\theta_z & 1 & -d\theta_x \\
-d\theta_y + d\theta_x\,d\theta_z & d\theta_y\,d\theta_z d\theta_x & 1
\end{bmatrix} \\
%
=
%
\BI
+
\begin{bmatrix}
0 & -d\theta_z & d\theta_y \\
d\theta_z & 0 & -d\theta_x \\
-d\theta_y & d\theta_x & 0
\end{bmatrix}
+
\begin{bmatrix}
0 & d\theta_x\,d\theta_y & 0 \\
0 & 0 & 0 \\
d\theta_x\,d\theta_z & d\theta_y\,d\theta_z & 0
\end{bmatrix} \\
+
\begin{bmatrix}
d\theta_x\,d\theta_y\,d\theta_z & 0 & 0 \\
0 & 0 & 0 \\
0 & 0 & 0
\end{bmatrix} \\
\end{multline*}

Note that if the second and third order terms are neglected then
\begin{equation*}
\BR_{d\theta_{zx}}
\BR_{d\theta_{yz}}
\BR_{d\theta_{xy}} - \BI
%
=
%
\BR_{d\theta_y}
\BR_{d\theta_x}
\BR_{d\theta_z} - \BI
%
\approx
%
\begin{bmatrix}
0 & -d\theta_z & d\theta_y \\
d\theta_z & 0 & -d\theta_x \\
-d\theta_y & d\theta_x & 0
\end{bmatrix}
\end{equation*}

and that

\begin{multline*}
\begin{bmatrix}
0 & -d\theta_z & d\theta_y \\
d\theta_z & 0 & -d\theta_x \\
-d\theta_y & d\theta_x & 0
\end{bmatrix}
= \\
% \BR_\theta_y
\begin{bmatrix}
 0 & 0 & d\theta_y \\
 0 & 0 & 0 \\
-d\theta_y & 0 & 0
\end{bmatrix}
+
% \BR_{d\theta_x}
\begin{bmatrix}
 0 & 0 & 0 \\
 0 & 0 & -d\theta_x \\
 0 & d\theta_x & 0
\end{bmatrix}
% \BR_{d\theta_z}
+
\begin{bmatrix}
0 & -d\theta_z & 0 \\
d\theta_z & 0 & 0 \\
0 & 0 & 0
\end{bmatrix}
\end{multline*}

The following result

\begin{multline*}
\BR_{d\theta_{zx}}
\BR_{d\theta_{yz}}
\BR_{d\theta_{xy}}
%
=
%
\BR_{d\theta_y}
\BR_{d\theta_x}
\BR_{d\theta_z} \\
\approx
\BI +
(\BR_{d\theta_y} - \BI)
+
(\BR_{d\theta_x} - \BI)
+
(\BR_{d\theta_z} - \BI)
=
\BR_{d\theta_{xyz}}
\end{multline*}
is independent of the order of application of the rotations, which is not true for the case
when the rotations are not infinitesimal.

Using this a differential change in \(\Br\) due to the rotation is

\begin{equation}\label{eqn:crossOld:100}
\begin{aligned}
d\Br = \Br' - \Br =
(\BR_{d\theta_{xyx}}-\BI) \Br
&=
\begin{bmatrix}
0 & -d\theta_z & d\theta_y \\
d\theta_z & 0 & -d\theta_x \\
-d\theta_y & d\theta_x & 0
\end{bmatrix} \Br \\
&=
-
\begin{bmatrix}
0 & -z & y \\
z & 0 & -x \\
-y & x & 0
\end{bmatrix}
\begin{bmatrix}
d\theta_x \\
d\theta_y \\
d\theta_z
\end{bmatrix}
\end{aligned}
\end{equation}

This vector of \(d\theta_i\) components can be written
\begin{equation*}
d\Btheta =
\begin{bmatrix}
d\theta_x \\
d\theta_y \\
d\theta_z
\end{bmatrix}
\end{equation*}

In the same fashion as in the two dimensional case above,
this can be applied
to
calculate the work done,

\begin{equation}\label{eqn:crossOld:120}
\begin{aligned}
dW &=\dotprod{\BF}{d\Br} \\
   &=
\begin{bmatrix}
F_x & F_y & F_z
\end{bmatrix}
\Biggl(
   -
\begin{bmatrix}
0 & -z & y \\
z & 0 & -x \\
-y & x & 0
\end{bmatrix} d\Btheta
\Biggr)
\\
 &=
-
{\Biggl(
{\begin{bmatrix}
0 & -z & y \\
z & 0 & -x \\
-y & x & 0
\end{bmatrix}}^T
\begin{bmatrix}
F_x \\
F_y \\
F_z
\end{bmatrix}
\Biggr)}^T
d\Btheta
\\
 &=
 \Biggl(
\begin{bmatrix}
0 & -z & y \\
z & 0 & -x \\
-y & x & 0
\end{bmatrix}
\BF
\Biggr) \cdot d\Btheta
\end{aligned}
\end{equation}

So in the three dimensional case we can write
\begin{equation*}
dW =\dotprod{\BF}{d\Br} =\dotprod{\Btau}{d\Btheta}
\end{equation*}

Where in analogy with the two dimensional case
\(dW =\dotprod{\BF}{d\Br} = \tau d\theta\) we can define the torque as this
quantity \(\Btau\),

\begin{equation*}
\Btau =
\begin{bmatrix}
0 & -z & y \\
z & 0 & -x \\
-y & x & 0
\end{bmatrix}
\BF
=
\begin{bmatrix}
y F_z - z F_y \\
z F_x - x F_z \\
x F_y - y F_x
\end{bmatrix}
=
\crossprod{\Br}{\BF}
\end{equation*}

the work per unit ``angle'' of rotation in space.

\subsection{angular velocity in three dimensions}

One of the formulas that I recall was always just presented and never derived (for the three dimensional case)
was that for
\(\D{t}{\Br}\) in terms of a vector angular velocity.  This is another equation that the cross product
comes up, and an equation whose derivation is easily done given some of the work above.

Let \(\Bomega = \D{t}{}\Bigl(\theta_x, \theta_y, \theta_z\Bigr)\) then in the limit

\begin{equation}\label{eqn:crossOld:140}
\begin{aligned}
\D{t}{\Br} &= \frac{\Br(t + dt) - \Br(t)}{dt} \\
%           &= \frac{\Br' - \Br}{dt} \\
           &= \frac{\BR_{d\theta_{xyx}}\Br -\Br}{dt} \\
           &= \frac{(\BR_{d\theta_{xyx}}-\BI) \Br}{dt} \\
	   &= \inv{dt}
\begin{bmatrix}
0 & -d\theta_z & d\theta_y \\
d\theta_z & 0 & -d\theta_x \\
-d\theta_y & d\theta_x & 0
\end{bmatrix} \Br \\
           &=
\begin{bmatrix}
\omega_y z -\omega_z y \\
\omega_z x -\omega_x z \\
\omega_x y -\omega_y x
\end{bmatrix}
\end{aligned}
\end{equation}

Thus
\begin{equation*}
\Bv = \crossprod{\Bomega}{\Br}
\end{equation*}

Note that these were not derivations of the cross product, but just
physical situations in which the cross product occurs.

\section{generalizing the cross product}

\subsection{defining a cross product operator}

In each of the physical situations where the cross product occurs
above (ie: torque and angular velocity) the derivation of
these formulas was closely tied to the
differential change in position \(d\Br\) after application of a rotation
transformation \(\BR_{d\theta_{xyz}}\),
%limiting expression for
%a three dimensional differential rotation
%\(\BR_{d\theta_{xyz}}\)
%applied to a vector \(\Br\)

\begin{equation*}
d\Br =
(\BR_{d\theta_{xyz}} - \BI) \Br =
\begin{bmatrix}
0 & -d\theta_z & d\theta_y \\
d\theta_z & 0 & -d\theta_x \\
-d\theta_y & d\theta_x & 0
\end{bmatrix} \Br
=
%-
\begin{bmatrix}
0 & -z & y \\
z & 0 & -x \\
-y & x & 0
\end{bmatrix} d\Btheta
\end{equation*}

As a side effect of the derivations above the cross
product shows up
tied
to a
matrix form since the rotation transformation was defined in matrix form.
Expressing the cross product in this matrix form has a certain aesthetic pleasantness
that the
component form lacks.  It can also be seen that
as an operational quantity
the matrices above,
whether they contain the \(d\theta_i\) or the \(r_i\) terms,
are the important part of the equation, and we can drop the vector multiplication
part of the expression.

A cross product operator \(\Bu \cross\) can be defined for and vector \(\Bu\)
in \R{3},

\begin{equation*}
\crossop{\Bu}=
\begin{bmatrix}
0 & -u_z & u_y \\
u_z & 0 & -u_x \\
-u_y & u_x & 0
\end{bmatrix}
\end{equation*}

When applied to a vector \(\Bv\)
\begin{equation*}
(\Bu \cross) \Bv =
\begin{bmatrix}
0 & -u_z & u_y \\
u_z & 0 & -u_x \\
-u_y & u_x & 0
\end{bmatrix}
\begin{bmatrix}
v_x \\
v_y \\
v_z
\end{bmatrix}
= \crossprod{\Bu}{\Bv}
\end{equation*}

which is what we typically define as \(\cross{\Bu}{\Bv}\).  Using this new notation the change
in position can be written,

\begin{equation*}
d\Br
=  (\crossprod{d\Btheta}{})\Br
= -(\crossprod{\Br}{})d\Btheta
\end{equation*}

By defining the cross product in this fashion, the rotational and physical origins have been discarded,
%As an operational entity there is no reason to refer
%to the differential rotation vector \(d\Btheta = (d\theta_x, d\theta_y, d\theta_z)\) anymore,
but it is interesting to note the way that the cross product
and a rotation in space are related in a fundamental way.

\subsection{decomposition of the cross product operator}

The cross product operator as defined above can be antisymetrically
decomposed into its positive and negative portions as follows
\begin{equation*}
\crossop{\Bu}=
\begin{bmatrix}
0 & -u_z & u_y \\
u_z & 0 & -u_x \\
-u_y & u_x & 0
\end{bmatrix}
=
\begin{bmatrix}
0 & 0 & u_y \\
u_z & 0 & 0 \\
0 & u_x & 0
\end{bmatrix}
-
\begin{bmatrix}
0 & 0 & u_y \\
u_z & 0 & 0 \\
0 & u_x & 0
\end{bmatrix}^T
\end{equation*}

Each half of the right hand side can be diagonalized, but not via a change of basis
\footnote
{
I was playing around with this at one point
and I believe it was possible to diagonalized the
matrix, but not in a simple fashion, as it required complex eigenvalues and
eigenvectors, and the solution of a cubic equation.
}
, as follows:
\begin{multline*}
\crossop{\Bu}=
\begin{bmatrix}
0 & 1 & 0 \\
0 & 0 & 1 \\
1 & 0 & 0
\end{bmatrix}
\begin{bmatrix}
u_x & 0 & 0 \\
0 & u_y & 0 \\
0 & 0 & u_z
\end{bmatrix}
\begin{bmatrix}
0 & 1 & 0 \\
0 & 0 & 1 \\
1 & 0 & 0
\end{bmatrix} \\
-
\begin{bmatrix}
0 & 0 & 1 \\
1 & 0 & 0 \\
0 & 1 & 0
\end{bmatrix}
\begin{bmatrix}
u_x & 0 & 0 \\
0 & u_y & 0 \\
0 & 0 & u_z
\end{bmatrix}
\begin{bmatrix}
0 & 0 & 1 \\
1 & 0 & 0 \\
0 & 1 & 0
\end{bmatrix}
\end{multline*}

The matrix
\( \BP =
\Bigl[
\begin{smallmatrix}
0 & 1 & 0 \\
0 & 0 & 1 \\
1 & 0 & 0
\end{smallmatrix}
\Bigr]
\) above is an interesting one, as \(\BP^{-1} = \BP^T = \BP^2\).

%From this it can be seen that one can write
%
%\begin{multline*}
%\begin{bmatrix}
%0 & 0 & 1 \\
%1 & 0 & 0 \\
%0 & 1 & 0
%\end{bmatrix}
%\begin{bmatrix}
%u_x & 0 & 0 \\
%0 & u_y & 0 \\
%0 & 0 & u_z
%%\end{bmatrix}
%\begin{bmatrix}
%0 & 0 & 1 \\
%1 & 0 & 0 \\
%0 & 1 & 0
%\end{bmatrix} \\
%=
%\begin{bmatrix}
%0 & 1 & 0 \\
%0 & 0 & 1 \\
%1 & 0 & 0
%\end{bmatrix}
%\Biggl(
%\begin{bmatrix}
%0 & 1 & 0 \\
%0 & 0 & 1 \\
%1 & 0 & 0
%\end{bmatrix}
%\begin{bmatrix}
%u_x & 0 & 0 \\
%0 & u_y & 0 \\
%0 & 0 & u_z
%\end{bmatrix}
%\begin{bmatrix}
%0 & 0 & 1 \\
%1 & 0 & 0 \\
%0 & 1 & 0
%\end{bmatrix}
%\Biggr) \\
%=
%\Biggl(
%\begin{bmatrix}
%0 & 0 & 1 \\
%1 & 0 & 0 \\
%%0 & 1 & 0
%\end{bmatrix}
%\begin{bmatrix}
%u_x & 0 & 0 \\
%0 & u_y & 0 \\
%0 & 0 & u_z
%\end{bmatrix}
%\begin{bmatrix}
%0 & 1 & 0 \\
%0 & 0 & 1 \\
%1 & 0 & 0
%\end{bmatrix}
%\Biggr)
%\begin{bmatrix}
%0 & 1 & 0 \\
%0 & 0 & 1 \\
%1 & 0 & 0
%\end{bmatrix}
%\end{multline*}
%
%where the terms inside the braces represent a change of basis transformation.

If one defines \(D(\Bu)\) as the matrix with the terms of \(\Bu\) are along the diagonal, then
there is now a nice and concise way of writing the cross product operator.

\begin{equation*}
\crossop{\Bu}= \BP D(\Bu) \BP - \BP^T D(\Bu) \BP^T
\end{equation*}

This is also
possibly suggestive of how to define the cross product in greater than three dimensions, for
it could possibly be of the same form where \(\BP\) is a permutation or some other transformation.

%An alternative form is also possible, by taking advantage of the \(\BP\) decomposition noted above.
%Let \(G(\Bu) = \BP^T D(\Bu) \BP = G(\Bu)^T\), then
%\begin{align*}
%\crossop{\Bu}&= \BP^T (\BP^T D(\Bu) \BP) - (\BP^T D(\Bu) \BP) \BP \\
%             &= \BP^T G(\Bu) - G(\Bu)^T \BP
%\end{align*}
%
% -- this does not really buy us anything to mention.
%

\subsection{cross product via four dimensional rotation}

As was noted above the cross product is closely related to a rotation in space.
This leads to a possible means for generalizing the cross product to
higher dimensions by examining the four dimensional rotation operator.

Some clarification here is probably in order.  What is meant by a four
dimensional rotation?  When the three dimensional rotation
operator was defined, it was the product in the limit of applying a possible rotation
\(d\theta_{xy}\) around the \(z\) axis, a possible rotation
\(d\theta_{yz}\) around the \(x\) axis, and a possible rotation
\(d\theta_{zx}\) around the \(y\) axis.  It is important to note that the
magnitude of these rotations is small, because otherwise the result
is different according to which order each of the rotations is applied.
\footnote{
I am not even sure that this composite rotation is a
single rotation in the traditional sense of a rotation along a plane in
space.
}
A four dimension differential rotation
operator can be defined in the same fashion as the limit of the products of the rotations
in each plane.  This is slightly more complicated in four dimensions than
in three since a rotation can be
simultaneously
perpendicular to two different axis, rather than one.  For example,
a rotation in the \(xy\) plane is perpendicular to both the \(z\) axis and the
\(w\) axis, if the space is defined as having \(w\), \(x\), \(y\), and \(z\)
components.

Each of the possible differential rotations will be enumerated below.  Let
\(\Bu = (d\theta_w, d\theta_x, d\theta_y, d\theta_z)\)
where each of the \(d\theta_i\) terms is a rotation perpendicular to the \(i\)
axis.

Differential rotations in the \(1, 2, 3\) subspace,
in the \(1, 2\) plane perpendicular to \(3\),
in the \(2, 3\) plane perpendicular to \(1\),
in the \(3, 1\) plane perpendicular to \(2\), respectively.
\begin{equation*}
\begin{bmatrix}
 1   &-u_3  & 0    & 0   \\
 u_3 & 1    & 0    & 0   \\
 0   & 0    & 1    & 0   \\
 0   & 0    & 0    & 1
\end{bmatrix}
%\end{equation*}
,
%\begin{equation*}
\begin{bmatrix}
   1 &  0   & 0    & 0    \\
   0 &  1   &-u_1  & 0    \\
   0 &  u_1 & 1    & 0    \\
   0 &  0   & 0    & 1
\end{bmatrix}
%\end{equation*}
,
%\begin{equation*}
\begin{bmatrix}
 1   & 0    &  u_2 & 0    \\
 0   & 1    & 0    & 0    \\
-u_2 & 0    & 1    & 0    \\
   0 &  0   & 0    & 1
\end{bmatrix}
\end{equation*}

Differential rotations in the \(1, 2, 4\) subspace,
in the \(1, 2\) plane perpendicular to \(4\),
in the \(2, 4\) plane perpendicular to \(1\),
in the \(4, 1\) plane perpendicular to \(2\), respectively.
\begin{equation*}
\begin{bmatrix}
 1   &-u_4  & 0    & 0   \\
 u_4 & 1    & 0    & 0   \\
 0   & 0    & 1    & 0   \\
 0   & 0    & 0    & 1
\end{bmatrix}
%\end{equation*}
,
%\begin{equation*}
\begin{bmatrix}
   1 &  0   & 0    & 0   \\
   0 &  1   & 0    &-u_1 \\
 0   & 0    & 1    & 0   \\
   0 &  u_1 & 0    & 1
\end{bmatrix}
%\end{equation*}
,
%\begin{equation*}
\begin{bmatrix}
 1   & 0    & 0    & u_2 \\
 0   & 1    & 0    & 0    \\
 0   & 0    & 1    & 0    \\
-u_2 & 0    & 0    & 1
\end{bmatrix}
\end{equation*}

Differential rotations in the \(1, 3, 4\) subspace,
in the \(1, 3\) plane perpendicular to \(4\),
in the \(3, 4\) plane perpendicular to \(1\),
in the \(4, 1\) plane perpendicular to \(3\), respectively.
\begin{equation*}
\begin{bmatrix}
 1   & 0    &-u_4  & 0   \\
 0   & 1    & 0    & 0   \\
 u_4 & 0    & 1    & 0   \\
 0   & 0    & 0    & 1
\end{bmatrix}
%\end{equation*}
,
%\begin{equation*}
\begin{bmatrix}
   1 & 0    &  0   & 0   \\
 0   & 1    & 0    & 0   \\
   0 & 0    &  1   &-u_1 \\
   0 & 0    &  u_1 & 1
\end{bmatrix}
%\end{equation*}
,
%\begin{equation*}
\begin{bmatrix}
 1   & 0    & 0    &  u_3 \\
 0   & 1    & 0    & 0    \\
 0   & 0    & 1    & 0    \\
-u_3 & 0    & 0    & 1
\end{bmatrix}
\end{equation*}

Differential rotations in the \(2, 3, 4\) subspace
in the \(2, 3\) plane perpendicular to \(4\),
in the \(3, 4\) plane perpendicular to \(2\),
in the \(4, 2\) plane perpendicular to \(3\), respectively.
\begin{equation*}
\begin{bmatrix}
   1 & 0    &  0   & 0   \\
   0 & 1    &-u_4  & 0   \\
   0 &  u_4 & 1    & 0   \\
   0 & 0    & 0    & 1
\end{bmatrix}
%\end{equation*}
,
%\begin{equation*}
\begin{bmatrix}
   1 & 0 &  0   & 0   \\
   0 & 1 &  0   & 0   \\
   0 & 0 &  1   &-u_2 \\
   0 & 0 &  u_2 & 1
\end{bmatrix}
%\end{equation*}
,
%\begin{equation*}
\begin{bmatrix}
 1   & 0   & 0    & 0   \\
 0   & 1   & 0    & u_3 \\
 0   & 0   & 1    & 0    \\
 0   &-u_3 & 0    & 1
\end{bmatrix}
\end{equation*}

In the limit, where any \(d\theta_i d\theta_j\) or higher order terms are
neglected, the product of all of these matrices \(\BR_i\) is

\begin{multline*}
\BR = \prod_i\BR_i \approx \BI + \sum_i(\BR_i-\BI) = \\
\begin{bmatrix}
%
%=================================
% 0   &-u_3  & 0    & 0   \\
% 0   & 0    &  u_2 & 0    \\
% 0   &-u_4  & 0    & 0   \\
% 0   & 0    & 0    &  u_2 \\
% 0   & 0    &-u_4  & 0   \\
% 0   & 0    & 0    &  u_3 \\
% ---------------------
1 &-u_3-u_4 &u_2-u_4 & u_2+u_3 \\
%=================================
% u_3 & 0    & 0    & 0   \\
%   0 &  0   &-u_1  & 0    \\
% u_4 & 0    & 0    & 0   \\
%   0 &  0   & 0    &-u_1 \\
%   0 & 0    &-u_4  & 0   \\
% 0   & 0   & 0    &  u_3 \\
% ---------------------
 u_3+u_4 & 1 &-u_1-u_4 &-u_1+u_3 \\
%=================================
%   0 &  u_1 & 0    & 0    \\
%-u_2 & 0    & 0    & 0    \\
% u_4 & 0    & 0    & 0   \\
%   0 & 0    &  0   &-u_1 \\
%   0 &  u_4 & 0    & 0   \\
%   0 & 0 &  0   &-u_2 \\
% ---------------------
u_4-u_2 &  u_1+u_4 & 1 &-u_1-u_2 \\
%=================================
%   0 &  u_1 & 0    & 0
%-u_2 & 0    & 0    & 0
%   0 & 0    &  u_1 & 0
%-u_3 & 0    & 0    & 0
%   0 & 0 &  u_2 & 0
% 0   &-u_3 & 0    & 0
% ---------------------
-u_2-u_3 &u_1-u_3 &  u_1+u_2 & 1
\end{bmatrix}
\end{multline*}

From this in the same fashion as was done for three dimensions
a cross product operator can be defined for \(\Bu\) as \(\crossop{\Bu}= \BR - \BI\)
to give a cross product definition for \R{4}

\begin{equation*}
\crossop{\Bu}=
\begin{bmatrix}
           0 & - u_3 - u_4 &   u_2 - u_4 &   u_2 + u_3 \\
   u_3 + u_4 & 0           & - u_1 - u_4 &   u_3 - u_1 \\
 - u_2 + u_4 & u_1 + u_4   & 0           & - u_1 - u_2 \\
 - u_2 - u_3 & u_1 - u_3   & u_1 + u_2   &   0
\end{bmatrix}
\end{equation*}

It has yet to be shown that this ``cross product'' has characteristics
similar to the three dimension cross product.  It can be verified that
it satisfies at least the following orthogonality conditions as does
the standard \R{3} cross product.

\begin{equation}\label{eqn:crossOld:160}
\begin{aligned}
\tripleprod{\Bu}{\Bv}{\Bu} &= 0 \\
\tripleprod{\Bu}{\Bv}{\Bv} &= 0 \\
      \crossprod{\Bu}{\Bu} &= \Bzero \\
      \crossprod{\Bu}{\Bv} &= -\crossprod{\Bv}{\Bu} \\
      \crossprod{\Bu}{(a\Bv + b\Bw)} &= a\crossprod{\Bu}{\Bv} + b\crossprod{\Bu}{\Bw} \\
      \crossprod{(a\Bu + b\Bv)}{\Bw} &= a\crossprod{\Bu}{\Bw} + b\crossprod{\Bv}{\Bw} \\
\tripleprod{\Bu}{\Bv}{\Bw} &= -\tripleprod{\Bu}{\Bw}{\Bv}
\end{aligned}
\end{equation}

Not all of these conditions are independent,
\(\crossprod{\Bu}{\Bv}\) is implied by \(\crossprod{\Bu}{\Bu} = -\crossprod{\Bu}{\Bu}\),
and \(\tripleprod{\Bu}{\Bv}{\Bu} = 0\) is implied since,
\begin{equation}\label{eqn:crossOld:180}
\begin{aligned}
\tripleprod{\Bu}{\Bv}{\Bu} &= -\tripleprod{\Bu}{\Bu}{\Bv} \\
                           &= -\dotprod{(\crossprod{\Bu}{\Bu})}{\Bv} \\
                           &= 0
\end{aligned}
\end{equation}
The key properties are probably the bilinearity, and the negation on exchange, but I have not yet
spent the time proving that all the rest follow.
% TODO...

\subsection{orthogonality and vector products}
I had arrived at the above result for a \R{4} cross product in a few
different ways, where this was one of the later methods.
The first ways that I arrived at this result was
by looking at orthogonality conditions and trying to extend the three
dimensional cross product in component form.  Using just the
orthogonality conditions is not enough to uniquely define a ``cross product''
even in \R{3}.

It is interesting to note that the dot product can be seen to be a statement of
the orthogonality conditions of Pythagoras law
%(the sum of
%squares of the lengths of two perpendicular line segments
%is the square of the length of the hypotonus).
\begin{equation*}
\norm{\Bu + \Bv}^2 = \norm{\Bu}^2 + \norm{\Bv}^2
\end{equation*}

In terms of components this is
\begin{equation}\label{eqn:crossOld:200}
\begin{aligned}
\norm{\Bu + \Bv}^2 &= \sum_i{(u_i + v_i)(u_i + v_i)} \\
                   &= \sum_i{{u_i}^2 + 2 u_i v_i + {v_i}^2} \\
                   &= \sum_i{{u_i}^2} + 2 \sum_i{u_i v_i} + \sum{{v_i}^2} \\
                   &= \norm{\Bu}^2 + 2 \sum_i{u_i v_i} + \norm{\Bv}^2
\end{aligned}
\end{equation}

So, if the Pythagorean condition is to hold the term, the dot product,
\begin{equation*}
\sum_i{u_i v_i}
\end{equation*}
must be zero.

The same thing can be done for the complex inner product, where
for orthogonality the term,
\begin{equation*}
\sum_i{ u_i \overline{v_i} + v_i \overline{u_i}}
\end{equation*}
must be zero.

If \(\sum_i{ u_i \overline{v_i}} = 0\), this implies
\(\overline{\sum_i{ u_i \overline{v_i}}} = \sum_i{v_i \overline{u_i}} = 0\), so the definitions of both the
complex and the real inner products arise naturally from an examination of orthogonality constraints.

The cross product is also closely related to orthogonality constraints and the \R{3}
cross product can be derived by looking specifically at these constraints.
This can be seen by
calculating the null space of a matrix with rows formed of the elements of two vectors
\(\Bu\) and \(\Bv\)

\begin{equation*}
\begin{bmatrix}
u_1 & u_2 & u_3 \\
v_1 & v_2 & v_3
\end{bmatrix}
\end{equation*}

Any vector that is in the null space is not a linear combination of the two vectors and then
must be perpendicular to it.
\footnote{a proof of this should be inserted.  Note the implicit dependence on the real
inner product here.}
Row reducing this matrix gives

\begin{equation*}
\begin{bmatrix}
u_1(u_2 v_1 - u_1 v_2) & 0                      & u_1(u_2 v_3 - u_3 v_2) \\
0                      & u_2(u_2 v_1 - u_1 v_2) & u_2(u_3 v_1 - u_1 v_3)
\end{bmatrix}
\end{equation*}

Provided that \(u_1 \neq 0\), \(u_2 \neq 0\), and \(u_2 v_1 - u_1 v_2 \neq 0\), then the
fully reduced form of this matrix is
\begin{equation*}
\begin{bmatrix}
1 & 0 & (u_2 v_3 - u_3 v_2)/(u_2 v_1 - u_1 v_2) \\
0 & 1 & (u_3 v_1 - u_1 v_3)/(u_2 v_1 - u_1 v_2)
\end{bmatrix}
\end{equation*}

So
%, where \(t\) is an arbitrary constant,
the null space is composed of the set of scalar multiples of
the vector
\begin{equation*}
\begin{bmatrix}
(u_2 v_3 - u_3 v_2)/(u_2 v_1 - u_1 v_2) \\
(u_3 v_1 - u_1 v_3)/(u_2 v_1 - u_1 v_2) \\
-1
\end{bmatrix}
\end{equation*}

of which, provided the \R{3} cross product \(\crossprod{\Bu}{\Bv}\) is one of
\begin{equation*}
\begin{bmatrix}
{u_2 v_3 - u_3 v_2} \\
{u_3 v_1 - u_1 v_3} \\
{u_1 v_2 - u_2 v_1}
\end{bmatrix}
\end{equation*}

This shows that orthogonality is not enough to uniquely define the cross product.

Doing the same null space calculations in \R{n} for the two vectors \(\Bu\) and \(\Bv\) gives the
null space as the set of vectors \({\Bn}\), where \(t_i\) are all arbitrary constants and \(\Bn\) is
defined as follows

\begin{equation*}
\Bn =
t_1
\begin{bmatrix}
u_2 v_3 - u_3 v_2 \\
u_3 v_1 - u_1 v_3 \\
u_1 v_2 - u_2 v_1 \\
0 \\
0 \\
\vdots \\
0
\end{bmatrix}
+ t_2
\begin{bmatrix}
u_2 v_4 - u_4 v_2 \\
u_4 v_1 - u_1 v_4 \\
0 \\
u_1 v_2 - u_2 v_1 \\
0 \\
\vdots \\
0
\end{bmatrix}
\hdots + t_{n-2}
\begin{bmatrix}
u_2 v_n - u_n v_2 \\
u_n v_1 - u_1 v_n \\
0 \\
\vdots \\
0 \\
0 \\
u_1 v_2 - u_2 v_1
\end{bmatrix}
\end{equation*}

This is the set of vectors that are orthogonal to both \(\Bu\) and \(\Bv\), but at a glance
no particular vector from that
set is appealing as a choice for a vector product.

Suppose, for \R{4}, setting \(t_1 = 1\) and \(t_2 = 1\), then a vector from the
null space is

\begin{equation*}
\begin{bmatrix}
u_2 v_3 - u_3 v_2 \\
u_3 v_1 - u_1 v_3 \\
u_1 v_2 - u_2 v_1 \\
0
\end{bmatrix}
+
\begin{bmatrix}
u_2 v_4 - u_4 v_2 \\
u_4 v_1 - u_1 v_4 \\
0 \\
u_1 v_2 - u_2 v_1
\end{bmatrix}
\end{equation*}

If this is compared to what was called the \R{4} cross product above, it can be seen that
the cross product has these two terms, plus two more.

\begin{equation*}
\begin{bmatrix}
u_3 v_4 - u_4 v_3 \\
0 \\
u_4 v_1 - u_1 v_4 \\
u_1 v_3 - u_3 v_1
\end{bmatrix}
+
\begin{bmatrix}
0 \\
u_3 v_4 - u_4 v_3 \\
u_4 v_2 - u_2 v_4 \\
u_2 v_3 - u_3 v_2
\end{bmatrix}
\end{equation*}

It should be possible to form these last two terms via a linear combination
of the first two, but this has not been tried.
%\footnote{
%an exercise for the reader ;)}

\subsection{more on the cross product in four dimensions}
Going back to the original decomposition of the three dimensional
cross product, a possible higher dimensional
cross product can be defined in the same fashion
$
\Bu \cross_4 = \BP_4 D(\Bu) \BP_4 - {\BP_4}^T D(\Bu) {\BP_4}^T
$

or perhaps as some more general quantity
$
\Bu \cross_4 = G(\Bu) - G(\Bu)^T
$

but how are the \(\BP_4\) or \(G\) matrices selected, so that
the result
has properties comparable to the \R{3}
cross product.

It can be noted above that the \(\BP\) matrix above is a permutation matrix. This is the identity
matrix with its rows shifted
up by one, or it columns shifted over right by one.

In four dimensions there are three permutation matrices that be can created by similarly shifting the
rows or columns of the identity matrix.  These are

\begin{equation*}
\BP = \Mp
\end{equation*}
\begin{equation*}
\BP^2 = \Mpp
\end{equation*}
\begin{equation*}
\BP^3 = \Mppp
\end{equation*}

They matrices are unitary, so for each the inverse is the transpose
\begin{equation}\label{eqn:crossOld:220}
\begin{aligned}
{\BP}^{-1} &= {\BP}^T = \BP^3 \\
{\BP^2}^{-1} &= {\BP^2}^T = \BP^2 \\
{\BP^3}^{-1} &= {\BP^3}^T = \BP
\end{aligned}
\end{equation}

With the hopes of discovering a suitable cross product operator with the form of the \R{3}
cross product operator, calculation for \({\BP^i} D(\Bu) {\BP^i}\) follows.

\begin{equation}\label{eqn:crossOld:240}
\begin{aligned}
{\BP} D(\Bu) {\BP} &=
\Mp
\Mpu
\Mp \\
& =
\begin{bmatrix}
0 & u_2 & 0 & 0 \\
0 & 0 & u_3 & 0 \\
0 & 0 & 0 & u_4 \\
u_1 & 0 & 0 & 0
\end{bmatrix}
\Mp
=
\begin{bmatrix}
0   & 0   & u_2 & 0   \\
0   & 0   & 0   & u_3 \\
u_4 & 0   & 0   & 0   \\
0   & u_1 & 0   & 0
\end{bmatrix}
\end{aligned}
\end{equation}

\begin{equation}\label{eqn:crossOld:260}
\begin{aligned}
{\BP^2} D(\Bu) {\BP^2} &=
\Mpp
\Mpu
\Mpp \\
&=
\begin{bmatrix}
0   & 0   & u_3 & 0   \\
0   & 0   & 0   & u_4 \\
u_1 & 0   & 0   & 0   \\
0   & u_2 & 0   & 0
\end{bmatrix}
\Mpp
=
\begin{bmatrix}
u_3 & 0   & 0   & 0   \\
0   & u_4 & 0   & 0   \\
0   & 0   & u_1 & 0   \\
0   & 0   & 0   & u_2
\end{bmatrix}
\end{aligned}
\end{equation}

\begin{equation}\label{eqn:crossOld:280}
\begin{aligned}
{\BP^3} D(\Bu) {\BP^3} &=
\Mppp
\Mpu
\Mppp \\
&=
\begin{bmatrix}
0   & 0   & 0   & u_4 \\
u_1 & 0   & 0   & 0   \\
0   & u_2 & 0   & 0   \\
0   & 0   & u_3 & 0
\end{bmatrix}
\Mppp
=
\begin{bmatrix}
0   & 0   & u_4 & 0   \\
0   & 0   & 0   & u_1 \\
u_2 & 0   & 0   & 0   \\
0   & u_3 & 0   & 0
\end{bmatrix}
\end{aligned}
\end{equation}

The potential cross product operators can be defined as
\begin{equation*}
\crossop{\Bu} = {\BP^i} D(\Bu) {\BP^i} - (\BP^i)^T D(\Bu) (\BP^i)^T
\end{equation*}

For \(\BP\) the following cross product operator is generated
\begin{multline*}
\crossop{\Bu} = {\BP} D(\Bu) {\BP} - {\BP}^T D(\Bu) {\BP}^T \\
=
\begin{bmatrix}
0         & 0         & u_2 - u_4 & 0         \\
0         & 0         & 0         & u_3 - u_1 \\
u_4 - u_2 & 0         & 0         & 0         \\
0         & u_1 - u_3 & 0         & 0
\end{bmatrix}
\end{multline*}

For \(\BP^2\) a trivial cross product operator is generated
\begin{equation*}
\crossop{\Bu} = {\BP^2} D(\Bu) {\BP^2} - (\BP^2)^T D(\Bu) (\BP^2)^T = \Bzero
\end{equation*}

And the cross product generated by \(\BP^3\) is just the transpose of that
for \(\BP\)
\begin{multline*}
\crossop{\Bu} = {\BP^3} D(\Bu) {\BP^3} - (\BP^3)^T D(\Bu) (\BP^3)^T \\
=
\begin{bmatrix}
0         & 0         & u_4 - u_2 & 0         \\
0         & 0         & 0         & u_1 - u_3 \\
u_2 - u_4 & 0         & 0         & 0         \\
0         & u_3 - u_1 & 0         & 0
\end{bmatrix}
\end{multline*}

Obviously the second of these does not generate a useful cross product.  Since the other two are
transposes of each other, either of those can be chosen for investigation.
Examining the first of these relations shows that
\begin{multline*}
\crossprod{\Bu}{\Bv} = ({\BP} D(\Bu) {\BP} - {\BP}^T D(\Bu) {\BP}^T) \Bv
= \\
\begin{bmatrix}
0         & 0         & u_2 - u_4 & 0         \\
0         & 0         & 0         & u_3 - u_1 \\
u_4 - u_2 & 0         & 0         & 0         \\
0         & u_1 - u_3 & 0         & 0
\end{bmatrix}
\begin{bmatrix}
v_1 \\
v_2 \\
v_3 \\
v_4
\end{bmatrix}
=
\begin{bmatrix}
(u_2 - u_4)v_3 \\
(u_3 - u_1)v_4 \\
(u_4 - u_2)v_1 \\
(u_1 - u_3)v_2
\end{bmatrix}
\end{multline*}

One of the properties that the cross product in three dimensions had was
\(\tripleprod{\Bu}{\Bv}{\Bv} = 0\) and
\(\tripleprod{\Bu}{\Bv}{\Bu} = 0\).  Does this potential cross product have the same properties?
\begin{equation*}
\tripleprod{\Bu}{\Bv}{\Bv} =
%\\
\begin{matrix}
  &v_3 v_1(u_2 - u_4) \\
+ &v_4 v_2(u_3 - u_1) \\
+ &v_1 v_3(u_4 - u_2) \\
+ &v_2 v_4(u_1 - u_3)
\end{matrix}
=
\begin{matrix}
  & v_1 v_3 ( u_2 - u_4 + u_4 - u_2) \\
+ & v_2 v_4 ( u_3 - u_1 + u_1 - u_3) \\
\end{matrix}
= 0
\end{equation*}
\begin{equation*}
\tripleprod{\Bu}{\Bv}{\Bu} =
%\\
\begin{matrix}
  &(u_2 - u_4)v_3 u_1 \\
+ &(u_3 - u_1)v_4 u_2 \\
+ &(u_4 - u_2)v_1 u_3 \\
+ &(u_1 - u_3)v_2 u_4
\end{matrix}
=
\begin{matrix}
  &u_1 u_2 ( v_3 - v_4 ) \\
+ &u_1 u_4 ( v_2 - v_3 ) \\
+ &u_2 u_3 ( v_4 - v_1 ) \\
+ &u_3 u_4 ( v_1 - v_2 )
\end{matrix}
\neq 0
\end{equation*}

Although
\(\tripleprod{\Bu}{\Bv}{\Bv} = 0\), and
\(\tripleprod{\Bu}{\Bv}{\Bw} = -\tripleprod{\Bu}{\Bw}{\Bv}\) as the \R{3}
cross product, this product seems incomplete.  There are no
\(v_1 v_2\), \(v_1 v_4\), or \(v_2 v_3\) terms.  In
\(\tripleprod{\Bu}{\Bv}{\Bu}\) there are no
\(u_1 u_3\) or \(u_2 u_4\) terms and the result is not zero as would be expected in a cross product.
\footnote{
The fact that
\(\tripleprod{\Bu}{\Bv}{\Bu} \ne 0\) can also be viewed as a consequence
of \(\crossprod{\Bu}{\Bu} \ne 0\) for this cross product, given that
\(\tripleprod{\Bu}{\Bv}{\Bu} = -\tripleprod{\Bu}{\Bu}{\Bv} \ne 0\).
}

Some of the terms that are missing can be added to generate a cross product which satisfy the same
orthogonality conditions that are true for \R{3}.
For example a \(u_1 v_3 \xcap_1\) term and a
\(u_3 v_1 \xcap_1\) term could be added.  Similarly a \(-u_3 v_2 \xcap_1\) term and a
\(u_2 v_3 \xcap_1\) term can be added.  The result for
\(\tripleprod{\Bu}{\Bv}{\Bu} = \Bu\)
had a
\(u_1 u_4 v_2\) term that resulted from the \(u_1 v_2 \xcap_4\) term of \(\crossprod{\Bu}{\Bv}\).  If
a \(-u_4 v_2 \xcap_1\) term is added then it would have canceled out.
If terms are added until each term has a ``match'' and each term of \(\tripleprod{\Bu}{\Bv}{\Bu}\)
cancels out leaving zero then the following revised cross product is generated.

\begin{equation*}
\begin{matrix}
 &(u_2 v_3 - u_3 v_2)\xcap_1 \\
+&(u_3 v_4 - u_4 v_3)\xcap_2 \\
+&(u_4 v_1 - u_1 v_4)\xcap_3 \\
+&(u_1 v_2 - u_2 v_1)\xcap_4 \\
 & \\
+&(u_3 v_4 - u_4 v_3)\xcap_1 \\
+&(u_4 v_1 - u_1 v_4)\xcap_2 \\
+&(u_1 v_2 - u_2 v_1)\xcap_3 \\
+&(u_2 v_3 - u_3 v_2)\xcap_4 \\
 & \\
+&(u_2 v_4 - u_4 v_2)\xcap_1 \\
+&(u_3 v_1 - u_1 v_3)\xcap_2 \\
+&(u_4 v_2 - u_2 v_4)\xcap_3 \\
+&(u_1 v_3 - u_3 v_1)\xcap_4
\end{matrix}
\end{equation*}

Note that this can also written as
\begin{equation*}
\begin{matrix}
% 1, 2, 3
 &(u_2 v_3 - u_3 v_2)\xcap_1 \\
+&(u_3 v_1 - u_1 v_3)\xcap_2 \\
+&(u_1 v_2 - u_2 v_1)\xcap_3 \\
 & \\
% 2, 3, 4
+&(u_3 v_4 - u_4 v_3)\xcap_2 \\
+&(u_4 v_2 - u_2 v_4)\xcap_3 \\
+&(u_2 v_3 - u_3 v_2)\xcap_4 \\
 & \\
% 1, 2, 4
+&(u_2 v_4 - u_4 v_2)\xcap_1 \\
+&(u_4 v_1 - u_1 v_4)\xcap_2 \\
+&(u_1 v_2 - u_2 v_1)\xcap_4 \\
 & \\
% 1, 3, 4
+&(u_3 v_4 - u_4 v_3)\xcap_1 \\
+&(u_4 v_1 - u_1 v_4)\xcap_3 \\
+&(u_1 v_3 - u_3 v_1)\xcap_4
\end{matrix}
\end{equation*}
where the terms are grouped into 4 sets of the three dimensional cross products on the
\((1, 2, 3)\),
\((2, 3, 4)\),
\((1, 2, 4)\), and
\((1, 3, 4)\) subspaces.

If this is put back into the matrix form \(\crossprod{\Bu}{}\) as
\begin{equation*}
\begin{bmatrix}
  0         & - u_3 - u_4 &   u_2 - u_4 &   u_2 + u_3 \\
  u_3 + u_4 &   0         & - u_1 - u_4 &   u_3 - u_1 \\
  u_4 - u_2 &   u_1 + u_4 &   0         & - u_1 - u_2 \\
- u_2 - u_3 &   u_1 - u_3 &   u_1 + u_2 &   0
\end{bmatrix}
\begin{bmatrix}
v_1 \\
v_2 \\
v_3 \\
v_4
\end{bmatrix}
\end{equation*}

Then the left hand side is the same as obtained via the \R{4} rotation method.

%%%%%% verification calculations:
%%%%%\begin{equation*}
%%%%%\crossop{\Bu}=
%%%%%\begin{bmatrix}
%%%%%  0         & - u_3 - u_4 &   u_2 - u_4 &   u_2 + u_3 \\
%%%%%  u_3 + u_4 &   0         & - u_1 - u_4 &   u_3 - u_1 \\
%%%%%  u_4 - u_2 &   u_1 + u_4 &   0         & - u_1 - u_2 \\
%%%%%- u_2 - u_3 &   u_1 - u_3 &   u_1 + u_2 &   0
%%%%%\end{bmatrix}
%%%%%\end{equation*}
%%%%%
%%%%%\begin{equation*}
%%%%%\crossop{\Bu}= \Bv
%%%%%\begin{bmatrix}
%%%%%(  0        )v_1 +(- u_3 - u_4)v_2 +(  u_2 - u_4)v_3 +(  u_2 + u_3)v_4 \\
%%%%%(  u_3 + u_4)v_1 +(  0        )v_2 +(- u_1 - u_4)v_3 +(  u_3 - u_1)v_4 \\
%%%%%(  u_4 - u_2)v_1 +(  u_1 + u_4)v_2 +(  0        )v_3 +(- u_1 - u_2)v_4 \\
%%%%%(- u_2 - u_3)v_1 +(  u_1 - u_3)v_2 +(  u_1 + u_2)v_3 +(  0        )v_4
%%%%%\end{bmatrix}
%%%%%\end{equation*}
%%%%%
%%%%%\begin{multline*}
%%%%%\tripleprod{\Bu}{\Bv}{\Bv} = \\
%%%%%\begin{bmatrix}
%%%%%(- u_3 - u_4)v_2 v_1 +(  u_2 - u_4)v_3 v_1 +(  u_2 + u_3)v_4 v_1 \\
%%%%%(  u_3 + u_4)v_1 v_2 +(- u_1 - u_4)v_3 v_2 +(  u_3 - u_1)v_4 v_2 \\
%%%%%(  u_4 - u_2)v_1 v_3 +(  u_1 + u_4)v_2 v_3 +(- u_1 - u_2)v_4 v_3 \\
%%%%%(- u_2 - u_3)v_1 v_4 +(  u_1 - u_3)v_2 v_4 +(  u_1 + u_2)v_3 v_4
%%%%%\end{bmatrix}
%%%%%= \\
%%%%%(  u_1 + u_2)v_3 v_4 + \\
%%%%%(- u_1 - u_2)v_4 v_3 + \\
%%%%%(  u_1 + u_4)v_2 v_3 + \\
%%%%%(- u_1 - u_4)v_3 v_2 + \\
%%%%%(  u_1 - u_3)v_2 v_4 + \\
%%%%%(  u_3 - u_1)v_4 v_2 + \\
%%%%%(  u_2 + u_3)v_4 v_1 + \\
%%%%%(- u_2 - u_3)v_1 v_4 + \\
%%%%%(  u_2 - u_4)v_3 v_1 + \\
%%%%%(  u_4 - u_2)v_1 v_3 + \\
%%%%%(  u_3 + u_4)v_1 v_2 + \\
%%%%%(- u_3 - u_4)v_2 v_1 \\
%%%%%= 0
%%%%%\end{multline*}
%%%%%
%%%%%\begin{multline*}
%%%%%\tripleprod{\Bu}{\Bv}{\Bu} = \\
%%%%%\begin{bmatrix}
%%%%%(- u_3 - u_4)v_2 u_1 +(  u_2 - u_4)v_3 u_1 +(  u_2 + u_3)v_4 u_1 \\
%%%%%(  u_3 + u_4)v_1 u_2 +(- u_1 - u_4)v_3 u_2 +(  u_3 - u_1)v_4 u_2 \\
%%%%%(  u_4 - u_2)v_1 u_3 +(  u_1 + u_4)v_2 u_3 +(- u_1 - u_2)v_4 u_3 \\
%%%%%(- u_2 - u_3)v_1 u_4 +(  u_1 - u_3)v_2 u_4 +(  u_1 + u_2)v_3 u_4
%%%%%\end{bmatrix} \\
%%%%% = \\
%%%%%u_1 u_2 v_3 + \\
%%%%%u_1 u_2 v_3 (-1) + \\
%%%%%u_1 u_2 v_4 + \\ + \\
%%%%%u_1 u_2 v_4 (-1) + \\
%%%%%u_1 u_3 v_2 + \\
%%%%%u_1 u_3 v_2 (-1) + \\
%%%%%u_1 u_3 v_4 + \\
%%%%%u_1 u_3 v_4 (-1) + \\
%%%%%u_1 u_4 v_2 + \\
%%%%%u_1 u_4 v_2 (-1) + \\
%%%%%u_1 u_4 v_3 + \\
%%%%%u_1 u_4 v_3 (-1) + \\
%%%%%u_2 u_3 v_1 + \\
%%%%%u_2 u_3 v_1 (-1) + \\
%%%%%u_2 u_3 v_4 + \\
%%%%%u_2 u_3 v_4 (-1) + \\
%%%%%u_2 u_4 v_1 + \\
%%%%%u_2 u_4 v_1 (-1) + \\
%%%%%u_2 u_4 v_3 + \\
%%%%%u_2 u_4 v_3 (-1) + \\
%%%%%u_3 u_4 v_1 + \\
%%%%%u_3 u_4 v_1 (-1) + \\
%%%%%u_3 u_4 v_2 + \\
%%%%%u_3 u_4 v_2 (-1) \\
%%%%% = 0
%%%%%\end{multline*}
%%%%%
%%%%%\begin{multline*}
%%%%%\crossprod{\Bu}{\Bu} = \\
%%%%%\begin{bmatrix}
%%%%%(- u_3 - u_4)u_2 +(  u_2 - u_4)u_3 +(  u_2 + u_3)u_4 \\
%%%%%(  u_3 + u_4)u_1 +(- u_1 - u_4)u_3 +(  u_3 - u_1)u_4 \\
%%%%%(  u_4 - u_2)u_1 +(  u_1 + u_4)u_2 +(- u_1 - u_2)u_4 \\
%%%%%(- u_2 - u_3)u_1 +(  u_1 - u_3)u_2 +(  u_1 + u_2)u_3
%%%%%\end{bmatrix} \\
%%%%%=0
%%%%%\end{multline*}
%%%%%% end verification

\subsection{components of the \texorpdfstring{\R{4}}{4D} cross product operator}

The \R{4} cross product operator that has been defined above was
arrived at by two different methods.  One was via an \R{4} rotation, and
the second was by starting with the decomposed form of $\Bu \cross_3 =
\BP\BD\BP - (\BP\BD\BP)^T$ and adding terms until it was ``complete'' with
respect to various orthogonality conditions that hold in \R{3}.  A
additional method of arriving at the same operator can be seen by
decomposing this operator.

To do so, \(\crossop{\Bu}\) can be written
\(G(\Bu) - {G(\Bu)}^T\) where
\begin{equation}\label{eqn:crossOld:300}
\begin{aligned}
G(\Bu)
=&
\begin{bmatrix}
  0         &   0         &   u_2       &   u_2 + u_3 \\
  u_3 + u_4 &   0         &   0         &   u_3       \\
  u_4       &   u_1 + u_4 &   0         &   0         \\
  0         &   u_1       &   u_1 + u_2 &   0
\end{bmatrix} \\
=&
\begin{bmatrix}
  0         &   0         &   u_2       &   0         \\
  0         &   0         &   0         &   u_3       \\
  u_4       &   0         &   0         &   0         \\
  0         &   u_1       &   0         &   0
\end{bmatrix} \\
+&
\begin{bmatrix}
  0         &   0         &   0         &   u_3       \\
  u_4       &   0         &   0         &   0         \\
  0         &   u_1       &   0         &   0         \\
  0         &   0         &   u_2       &   0
\end{bmatrix} \\
+&
\begin{bmatrix}
  0         &   0         &   0         &   u_2       \\
  u_3       &   0         &   0         &   0         \\
  0         &   u_4       &   0         &   0         \\
  0         &   0         &   u_1       &   0
\end{bmatrix}
\end{aligned}
\end{equation}

If this is decomposed into four sets of three dimension cross product operators on each of
the subspaces where one component is removed then
\begin{equation}\label{eqn:crossOld:320}
\begin{aligned}
G(\Bu)
=&
\begin{bmatrix}
  0         &   0         &   u_2       &   0         \\
  u_3       &   0         &   0         &   0         \\
  0         &   u_1       &   0         &   0         \\
  0         &   0         &   0         &   0
\end{bmatrix} \\
 +&
\begin{bmatrix}
  0         &   0         &   0         &   0         \\
  0         &   0         &   0         &   u_3       \\
  0         &   u_4       &   0         &   0         \\
  0         &             &   u_2       &   0
\end{bmatrix} \\
+&
\begin{bmatrix}
  0         &   0         &   0         &   u_2       \\
  u_4       &   0         &   0         &   0         \\
  0         &   0         &   0         &   0         \\
  0         &   u_1       &   0         &   0
\end{bmatrix} \\
+&
\begin{bmatrix}
  0         &   0         &   0         &   u_3       \\
  0         &   0         &   0         &   0         \\
  u_4       &   0         &   0         &   0         \\
  0         &   0         &   u_1       &   0
\end{bmatrix}
\end{aligned}
\end{equation}

Thus the \R{4} cross product operator can be generated by adding all of the \R{3} cross product
operators for each subspace where one component is removed or zeroed out.

Each of the terms of the sum for \(G(\Bu)\) above can be decomposed using the
\(\BP\), \(\BP^2\), and \(\BP^3\)
permutation matrices
\begin{equation}\label{eqn:crossOld:340}
\begin{aligned}
\begin{bmatrix}
  0         &   0         &   u_2       &   0         \\
  0         &   0         &   0         &   u_3       \\
  u_4       &   0         &   0         &   0         \\
  0         &   u_1       &   0         &   0
\end{bmatrix}
&=
\Mp \Mpu \Mp \\
&=
\BP D(\Bu) \BP
\end{aligned}
\end{equation}
\begin{equation}\label{eqn:crossOld:360}
\begin{aligned}
\begin{bmatrix}
  0         &   0         &   0         &   u_3       \\
  u_4       &   0         &   0         &   0         \\
  0         &   u_1       &   0         &   0         \\
  0         &   0         &   u_2       &   0
\end{bmatrix}
&=
\Mpp \Mpu \Mp \\
&=
\BP^2 D(\Bu) \BP
%=
%\BP \BP D(\Bu) \BP
\end{aligned}
\end{equation}
\begin{equation}\label{eqn:crossOld:380}
\begin{aligned}
\begin{bmatrix}
  0         &   0         &   0         &   u_2       \\
  u_3       &   0         &   0         &   0         \\
  0         &   u_4       &   0         &   0         \\
  0         &   0         &   u_1       &   0
\end{bmatrix}
&=
\Mp \Mpu \Mpp \\
&=
\BP D(\Bu) \BP^2
%=
%\BP D(\Bu) \BP \BP
\end{aligned}
\end{equation}

And can write in summary, that the four dimensional cross product operator is
\begin{equation*}
G(\Bu)
=
\BP D(\Bu) \BP +
\BP D(\Bu) \BP^2
+
\BP^2 D(\Bu) \BP
\end{equation*}

Defining,
\begin{equation*}
F(\Bu) = \BP D(\Bu) \BP,
\end{equation*}
the three and four dimensional cross products operator matrices can be written,

\begin{multline*}
\Bu \cross_3 =
F(\Bu) - {F(\Bu)}^T \\
\text{where}
\BP =
\begin{bmatrix}
0 & 1 & 0 \\
0 & 0 & 1 \\
1 & 0 & 0
\end{bmatrix}
\end{multline*}
\begin{multline*}
\Bu \cross_4 =
      F(\Bu)     -         {F(\Bu)}^T \\
+ \BP F(\Bu)     -         {F(\Bu)}^T {\BP}^T
    + F(\Bu) \BP - {\BP}^T {F(\Bu)}^T
\\
\text{where}
\BP = \Mp
\end{multline*}

\subsection{possible cross products in five dimensions}
Possible cross products were constructed earlier
for which the
\(\tripleprod{\Bu}{\Bv}{\Bv} = 0\)
and for which
\(\tripleprod{\Bu}{\Bv}{\Bu}\) could possibly be zero under certain conditions, but was not true generally.
It was shown that the sum of the cross products of the 4 possible \R{3} subspaces of \R{4}
was a more suitable choice for a cross product than the other construction as it generates a
result that is orthogonal to both of its component vectors.  That result could have been obtained
more directly, but the process used to arrive at it indirectly was useful or at least interesting.

Let us form the sum of the \R{4} cross products of the five possible \R{4} subspaces of \R{5} and see what the result is.
It will be simpler to use just the positive parts of the \R{4} cross product operator,
then to expand this all out explicitly.

For the \((1,2,3,4)\) subspace
\begin{equation*}
\begin{bmatrix}
  0         &   0         &   u_2       &   u_2 + u_3 &   0         \\
  u_3 + u_4 &   0         &   0         &   u_3       &   0         \\
  u_4       &   u_1 + u_4 &   0         &   0         &   0         \\
  0         &   u_1       &   u_1 + u_2 &   0         &   0         \\
  0         &   0         &   0         &   0         &   0
\end{bmatrix}
\end{equation*}
For the \((1,2,3,5)\) subspace
\begin{equation*}
\begin{bmatrix}
  0         &   0         &   u_2       &   0         &   u_2 + u_3 \\
  u_3 + u_5 &   0         &   0         &   0         &   u_3       \\
  u_5       &   u_1 + u_5 &   0         &   0         &   0         \\
  0         &   0         &   0         &   0         &   0         \\
  0         &   u_1       &   u_1 + u_2 &   0         &   0
\end{bmatrix}
\end{equation*}
For the \((1,2,4,5)\) subspace
\begin{equation*}
\begin{bmatrix}
  0         &   0         &   0         &   u_2       &   u_2 + u_4 \\
  u_4 + u_5 &   0         &   0         &   0         &   u_4       \\
  0         &   0         &   0         &   0         &   0         \\
  u_5       &   u_1 + u_5 &   0         &   0         &   0         \\
  0         &   u_1       &   0         &   u_1 + u_2 &   0
\end{bmatrix}
\end{equation*}
For the \((1,3,4,5)\) subspace
\begin{equation*}
\begin{bmatrix}
  0         &   0         &   0         &   u_3       &   u_3 + u_4 \\
  0         &   0         &   0         &   0         &   0         \\
  u_4 + u_5 &   0         &   0         &   0         &   u_4       \\
  u_5       &   0         &   u_1 + u_5 &   0         &   0         \\
  0         &   0         &   u_1       &   u_1 + u_3 &   0
\end{bmatrix}
\end{equation*}
For the \((2,3,4,5)\) subspace
\begin{equation*}
\begin{bmatrix}
  0         & 0         &   0         &   0         &   0         \\
  0         & 0         &   0         &   u_3       &   u_3 + u_4 \\
  0         & u_4 + u_5 &   0         &   0         &   u_4       \\
  0         & u_5       &   u_2 + u_5 &   0         &   0         \\
  0         & 0         &   u_2       &   u_2 + u_3 &   0
\end{bmatrix}
\end{equation*}

%\begin{bmatrix}
%  0         &   0         &   u_2       &   u_2 + u_3 &   0         \\
%  0         &   0         &   u_2       &   0         &   u_2 + u_3 \\
%  0         &   0         &   0         &   u_2       &   u_2 + u_4 \\
%  0         &   0         &   0         &   u_3       &   u_3 + u_4 \\
%  0         &   0         &   0         &   0         &   0         \\
%----------------------------------------------------------------------
%  0         &   0         &   2 u_2     &   2 u_2     &   2 u_2     \\
%                                          + 2 u_3       + 2 u_3
%                                                        + 2 u_4
%----------------------------------------------------------------------
%  0         &   0         &   2 u_2     & 2 u_2 +2u_3&2u_2+2u_3+2u_4 \\
%
%
%  u_3 + u_4 &   0         &   0         &   u_3       &   0         \\
%  u_3 + u_5 &   0         &   0         &   0         &   u_3       \\
%  u_4 + u_5 &   0         &   0         &   0         &   u_4       \\
%  0         &   0         &   0         &   0         &   0         \\
%  0         &   0         &   0         &   u_3       &   u_3 + u_4 \\
%----------------------------------------------------------------------
%  2 u_3     &   0         &   0         &   2 u_3     &   2 u_3     \\
%+ 2 u_4                                                 + 2 u_4
%+ 2 u_5
%----------------------------------------------------------------------
%2u_3+2u_4+2u_5& 0         &   0         &   2 u_3     &   2 u_3 +2u_4    \\
%
%
%  u_4       &   u_1 + u_4 &   0         &   0         &   0         \\
%  u_5       &   u_1 + u_5 &   0         &   0         &   0         \\
%  0         &   0         &   0         &   0         &   0         \\
%  u_4 + u_5 &   0         &   0         &   0         &   u_4       \\
%  0         &   u_4 + u_5 &   0         &   0         &   u_4       \\
%----------------------------------------------------------------------
%  2 u_4     &   2 u_4     &   0         &   0         &   2 u_4     \\
%+ 2 u_5       + 2 u_5
%              + 2 u_1
%----------------------------------------------------------------------
%  2 u_4+2u_5&2u_1+2u_4+2u_5 &  0        &   0         &   2 u_4     \\
%
%
%  0         &   u_1       &   u_1 + u_2 &   0         &   0         \\
%  0         &   0         &   0         &   0         &   0         \\
%  u_5       &   u_1 + u_5 &   0         &   0         &   0         \\
%  u_5       &   0         &   u_1 + u_5 &   0         &   0         \\
%  0         &   u_5       &   u_2 + u_5 &   0         &   0         \\
%----------------------------------------------------------------------
%  2 u_5     &   2 u_1     &   2 u_1     &   0         &   0         \\
%              + 2 u_5     & + 2 u_2                                 \\
%                            + 2 u_5                                 \\
%----------------------------------------------------------------------
%  2 u_5     & 2 u_1 + 2u_5&2u_1+2u_2+2u_5&  0         &   0         \\
%
%
%  0         &   0         &   0         &   0         &   0
%  0         &   u_1       &   u_1 + u_2 &   0         &   0
%  0         &   u_1       &   0         &   u_1 + u_2 &   0
%  0         &   0         &   u_1       &   u_1 + u_3 &   0
%  0         &   0         &   u_2       &   u_2 + u_3 &   0
%----------------------------------------------------------------------
%  0         &   2 u_1     &  2u_1+2u_2  &2u_1+2u_2+2u_3 &   0
%\end{bmatrix}
%
%
%
%Adding these yields (too wide to fit across one line)
%\begin{bmatrix}
%0                     & 0                     &         2 u_2         &         2 u_2 + 2 u_3 & 2 u_2 + 2 u_3 + 2 u_4 \\
%2 u_3 + 2 u_4 + 2 u_5 & 0                     & 0                     &         2 u_3         &         2 u_3 + 2 u_4 \\
%        2 u_4 + 2 u_5 & 2 u_1 + 2 u_4 + 2 u_5 & 0                     & 0                     &                 2 u_4 \\
%                2 u_5 & 2 u_1 +         2 u_5 & 2 u_1 + 2 u_2 + 2 u_5 & 0                     & 0                     \\
%0                     & 2 u_1                 & 2 u_1 + 2 u_2         & 2 u_1 + 2 u_2 + 2 u_3 & 0
%\end{bmatrix}
%
%
%
%Adding these yields (where the first matrix is the first three columns and the second is the last two)
%\begin{multline*}
%\begin{bmatrix}
%%0                     & 0                     &         2 u_2         &\\
%2 u_3 + 2 u_4 + 2 u_5 & 0                     & 0                     &\\
%        2 u_4 + 2 u_5 & 2 u_1 + 2 u_4 + 2 u_5 & 0                     &\\
%                2 u_5 & 2 u_1 +         2 u_5 & 2 u_1 + 2 u_2 + 2 u_5 &\\
%0                     & 2 u_1                 & 2 u_1 + 2 u_2         &
%\end{bmatrix} \\
%\begin{bmatrix}
%&         2 u_2 + 2 u_3 & 2 u_2 + 2 u_3 + 2 u_4 \\
%&         2 u_3         &         2 u_3 + 2 u_4 \\
%& 0                     &                 2 u_4 \\
%& 0                     & 0                     \\
%& 2 u_1 + 2 u_2 + 2 u_3 & 0
%\end{bmatrix}
%\end{multline*}

Which sums to the following
\begin{equation}\label{eqn:crossOld:400}
\begin{aligned}
2&
\begin{bmatrix}
0                     & 0                     &           u_2         &           u_2 +   u_3 &   u_2 +   u_3 +   u_4 \\
  u_3 +   u_4 +   u_5 & 0                     & 0                     &           u_3         &           u_3 +   u_4 \\
          u_4 +   u_5 &   u_1 +   u_4 +   u_5 & 0                     & 0                     &                   u_4 \\
                  u_5 &   u_1 +           u_5 &   u_1 +   u_2 +   u_5 & 0                     & 0                     \\
0                     &   u_1                 &   u_1 +   u_2         &   u_1 +   u_2 +   u_3 & 0
\end{bmatrix} \\
=&
2
\begin{bmatrix}
 0   & 0   & 0   & 0   & u_2 \\
 u_3 & 0   & 0   & 0   & 0 \\
 0   & u_4 & 0   & 0   & 0 \\
 0   & 0   & u_5 & 0   & 0 \\
 0   & 0   & 0   & u_1 & 0
\end{bmatrix}
+2
\begin{bmatrix}
 0   & 0   & 0   & 0   & u_3 \\
 u_4 & 0   & 0   & 0   & 0 \\
 0   & u_5 & 0   & 0   & 0 \\
 0   & 0   & u_1 & 0   & 0 \\
 0   & 0   & 0   & u_2 & 0
\end{bmatrix} \\
+&2
\begin{bmatrix}
 0   & 0   & 0   & 0   & u_4 \\
 u_5 & 0   & 0   & 0   & 0 \\
 0   & u_1 & 0   & 0   & 0 \\
 0   & 0   & u_2 & 0   & 0 \\
 0   & 0   & 0   & u_3 & 0
\end{bmatrix}
+2
\begin{bmatrix}
 0   & 0   & 0   & u_2 & 0 \\
 0   & 0   & 0   & 0   & u_3 \\
 u_4 & 0   & 0   & 0   & 0 \\
 0   & u_5 & 0   & 0   & 0 \\
 0   & 0   & u_1 & 0   & 0
\end{bmatrix} \\
+&2
\begin{bmatrix}
 0   & 0   & 0   & u_3 & 0 \\
 0   & 0   & 0   & 0   & u_4 \\
 u_5 & 0   & 0   & 0   & 0 \\
 0   & u_1 & 0   & 0   & 0 \\
 0   & 0   & u_2 & 0   & 0
\end{bmatrix}
+2
\begin{bmatrix}
 0   & 0   & u_2 & 0   & 0 \\
 0   & 0   & 0   & u_3 & 0 \\
 0   & 0   & 0   & 0   & u_4 \\
 u_5 & 0   & 0   & 0   & 0 \\
 0   & u_1 & 0   & 0   & 0
\end{bmatrix}
\end{aligned}
\end{equation}

If \(\BP\) is defined as
\begin{equation*}
\begin{bmatrix}
 0 & 1 & 0 & 0 & 0 \\
 0 & 0 & 1 & 0 & 0 \\
 0 & 0 & 0 & 1 & 0 \\
 0 & 0 & 0 & 0 & 1 \\
 1 & 0 & 0 & 0 & 0
\end{bmatrix}
\end{equation*}
then the \R{5} cross product operator can be written as
\begin{equation*}
\Bu \cross_5 = G(\Bu) - {G(\Bu)}^T
\end{equation*}
where
\begin{multline*}
G(\Bu) = 2 (
\BP D(\Bu) \BP
+ \BP D(\Bu) \BP^2
+ \BP^2 D(\Bu) \BP^2
+ \BP^2 D(\Bu) \BP \\
+ \BP D(\Bu) \BP^3
+ \BP^3 D(\Bu) \BP
)
\end{multline*}

Reiterating the results for each of
\R{3},
\R{4}, and
\R{5}, where \(\BD = D(\Bu)\)

\begin{equation*}
\Bu \cross_3 =
\BP\BD\BP
-
(\BP\BD\BP)^T
\end{equation*}
\begin{equation*}
\BP =
\begin{bmatrix}
0 & 1 & 0 \\
0 & 0 & 1 \\
1 & 0 & 0
\end{bmatrix}
\end{equation*}
\begin{equation*}
\Bu \cross_4 =
\BP\BD\BP + \BP\BD\BP^2 + \BP^2\BD\BP
-
(\BP\BD\BP + \BP\BD\BP^2 + \BP^2\BD\BP)^T
\end{equation*}
\begin{equation*}
\BP = \Mp
\end{equation*}
\begin{multline*}
\Bu \cross_5 =
2 (
  \BP\BD\BP
+ \BP\BD\BP^2
+ \BP^2\BD\BP^2
+ \BP^2\BD\BP
+ \BP\BD\BP^3
+ \BP^3\BD\BP
)
- \\
2 (
  \BP\BD\BP
+ \BP\BD\BP^2
+ \BP^2\BD\BP^2
+ \BP^2\BD\BP
+ \BP\BD\BP^3
+ \BP^3\BD\BP
)^T
\end{multline*}
\begin{equation*}
\BP =
\begin{bmatrix}
 0 & 1 & 0 & 0 & 0 \\
 0 & 0 & 1 & 0 & 0 \\
 0 & 0 & 0 & 1 & 0 \\
 0 & 0 & 0 & 0 & 1 \\
 1 & 0 & 0 & 0 & 0
\end{bmatrix}
\end{equation*}

There is a definite pattern here.  Looking at the positive parts, this pattern can extrapolated to higher dimensions
\begin{equation}\label{eqn:crossOld:420}
\begin{aligned}
\BR^3 : &\PDP{1}{1} \\
\BR^4 : &\PDP{1}{1} + \PDP{1}{2} \\
      + &\PDP{2}{1} \\
\BR^5 : &\PDP{1}{1} + \PDP{1}{2} + \PDP{1}{3} \\
      + &\PDP{2}{1} + \PDP{2}{2} \\
      + &\PDP{3}{1} \\
\BR^6 : &\PDP{1}{1} + \PDP{1}{2} + \PDP{1}{3} + \PDP{1}{4} \\
      + &\PDP{2}{1} + \PDP{2}{2} + \PDP{2}{3} \\
      + &\PDP{3}{1} + \PDP{3}{2} \\
      + &\PDP{4}{1} \\
\BR^7 : &\hdots
\end{aligned}
\end{equation}
In the above all of the sets of \(\PDP{i}{j}\) have been included such that \(i+j < n\), the dimension of the
vector space.  The number of these matrices is \(\sum_{i=1}^{n-2}{i} = \frac{1}{2}(n-2)(n-1)\).
Because of the way these operators have been constructed, for \(\Bu \cross\)
in \R{4} there are \((4) (3) = 12 \) positive terms (in \R{3} there are \(3\) positive terms),
in \R{5} there are \((5) (12)\) positive terms, and
in \R{n} it can be expected that there will be \((n) (n-1) ... (3) = n!/2\) positive terms.

Each of the matrices \(\PDP{i}{j}\) contributes \(n\) positive terms to
the cross product, and so the following multiplicative factor can be added to the terms above
to have a definition consistent with the \R{5} cross product operator derived above.
\begin{equation*}
\frac{\frac{1}{n}\frac{n!}{2}}{\frac{1}{2}(n-2)(n-1)} = (n-3)!
\end{equation*}

Using this pattern, the general cross product operator matrix for \R{n} can be written as
\begin{equation}\label{eqn:crossOld:440}
\begin{aligned}
\Bu \cross_n &= (n-3)! \sum_{i=1}^{n-2}\sum_{j=1}^{n-1-i}(\PDP{i}{j} - (\PDP{i}{j})^T)
\\
%\end{equation*}
%\begin{equation*}
\text{where}\:
\BP &=
\begin{bmatrix}
0 & 1 & 0 & \hdots & 0 \\
0 & 0 & 1 & 0      & 0 \\
0 & 0 & 0 & \ddots & 0 \\
0 & 0 & 0 & 0      & 1 \\
1 & 0 & 0 & \hdots & 0
\end{bmatrix}
\\
%\end{equation*}
%\begin{equation*}
\text{and}\:
\BD &=
\begin{bmatrix}
u_1 & 0   & \hdots & 0       & 0 \\
0   & u_2 & 0      & 0       & 0 \\
0   & 0   & \ddots & 0       & 0 \\
0   & 0   & 0      & u_{n-1} & 0 \\
0   & 0   & \hdots & 0       & u_n
\end{bmatrix}
\end{aligned}
\end{equation}

Note that for higher than \R{5} it has not yet been verified that
\(\tripleprod{\Bu}{\Bv}{\Bu} = 0\)
or that
\(\tripleprod{\Bu}{\Bv}{\Bv} = 0\)
or that
\(\crossprod{\Bu}{\Bu} = \Bzero\)
.  These have been verified explicitly for \R{4} and \R{5}, but the calculations are too tedious to show.

%%%%%% verification:
%%%%%\subsection{validity check for generalized cross product operator}
%%%%%
%%%%%For \R{5} let us calculate \(\crossprod{\Bu}{\Bv}\) and see if this is orthogonal to both \(\Bu\) and \(\Bv\),
%%%%%and see if \(\crossprod{\Bu}{\Bu} = 0\).
%%%%%\begin{multline*}
%%%%%\frac{1}{2} \Bu \cross_5 = \\
%%%%%\begin{bmatrix}
%%%%% 0           &-u_3-u_4-u_5 & u_2-u_4-u_5 & u_2+u_3-u_5 & u_2+u_3+u_4 \\
%%%%% u_3+u_4+u_5 & 0           &-u_1-u_4-u_5 & u_3-u_1-u_5 & u_3+u_4-u_1 \\
%%%%% u_4+u_5-u_2 & u_1+u_4+u_5 & 0           &-u_1-u_2-u_5 & u_4-u_1-u_2 \\
%%%%% u_5-u_2-u_3 & u_1+u_5-u_3 & u_1+u_2+u_5 & 0           &-u_1-u_2-u_3 \\
%%%%%-u_2-u_3-u_4 & u_1-u_3-u_4 & u_1+u_2-u_4 & u_1+u_2+u_3 & 0
%%%%%\end{bmatrix}
%%%%%\end{multline*}
%%%%%
%%%%%\begin{multline*}
%%%%%\frac{1}{2} \Bu \cross_5 \Bv = \\
%%%%%\begin{bmatrix}
%%%%%(-u_3-u_4-u_5)v_2 +( u_2-u_4-u_5)v_3 +( u_2+u_3-u_5)v_4 +( u_2+u_3+u_4)v_5 \\
%%%%%( u_3+u_4+u_5)v_1 +(-u_1-u_4-u_5)v_3 +( u_3-u_1-u_5)v_4 +( u_3+u_4-u_1)v_5 \\
%%%%%( u_4+u_5-u_2)v_1 +( u_1+u_4+u_5)v_2 +(-u_1-u_2-u_5)v_4 +( u_4-u_1-u_2)v_5 \\
%%%%%( u_5-u_2-u_3)v_1 +( u_1+u_5-u_3)v_2 +( u_1+u_2+u_5)v_3 +(-u_1-u_2-u_3)v_5 \\
%%%%%(-u_2-u_3-u_4)v_1 +( u_1-u_3-u_4)v_2 +( u_1+u_2-u_4)v_3 +( u_1+u_2+u_3)v_4
%%%%%\end{bmatrix}
%%%%%\end{multline*}
%%%%%
%%%%%\begin{multline*}
%%%%%\frac{1}{2} \Bu \cross_5 \Bv \cdot \Bv = \\
%%%%%(-u_3-u_4-u_5)v_2 v_1 +( u_2-u_4-u_5)v_3 v_1 +( u_2+u_3-u_5)v_4 v_1 +( u_2+u_3+u_4)v_5 v_1 \\
%%%%%( u_3+u_4+u_5)v_1 v_2 +(-u_1-u_4-u_5)v_3 v_2 +( u_3-u_1-u_5)v_4 v_2 +( u_3+u_4-u_1)v_5 v_2 \\
%%%%%( u_4+u_5-u_2)v_1 v_3 +( u_1+u_4+u_5)v_2 v_3 +(-u_1-u_2-u_5)v_4 v_3 +( u_4-u_1-u_2)v_5 v_3 \\
%%%%%( u_5-u_2-u_3)v_1 v_4 +( u_1+u_5-u_3)v_2 v_4 +( u_1+u_2+u_5)v_3 v_4 +(-u_1-u_2-u_3)v_5 v_4 \\
%%%%%(-u_2-u_3-u_4)v_1 v_5 +( u_1-u_3-u_4)v_2 v_5 +( u_1+u_2-u_4)v_3 v_5 +( u_1+u_2+u_3)v_4 v_5 \\
%%%%%= \\
%%%%%v_1 v_2( u_3+u_4+u_5)) + \\
%%%%%v_1 v_2(-u_3-u_4-u_5)) + \\
%%%%%v_1 v_3( u_2-u_4-u_5)) + \\
%%%%%v_1 v_3( u_4+u_5-u_2)) + \\
%%%%%v_1 v_4( u_2+u_3-u_5)) + \\
%%%%%v_1 v_4( u_5-u_2-u_3)) + \\
%%%%%v_1 v_5( u_2+u_3+u_4)) + \\
%%%%%v_1 v_5(-u_2-u_3-u_4)) + \\
%%%%%v_2 v_3( u_1+u_4+u_5)) + \\
%%%%%v_2 v_3(-u_1-u_4-u_5)) + \\
%%%%%v_2 v_4( u_1+u_5-u_3)) + \\
%%%%%v_2 v_4( u_3-u_1-u_5)) + \\
%%%%%v_2 v_5( u_1-u_3-u_4)) + \\
%%%%%v_2 v_5( u_3+u_4-u_1)) + \\
%%%%%v_3 v_4( u_1+u_2+u_5)) + \\
%%%%%v_3 v_4(-u_1-u_2-u_5)) + \\
%%%%%v_3 v_5( u_1+u_2-u_4)) + \\
%%%%%v_3 v_5( u_4-u_1-u_2)) + \\
%%%%%v_4 v_5( u_1+u_2+u_3)) + \\
%%%%%v_4 v_5(-u_1-u_2-u_3)) \\
%%%%%= 0
%%%%%\end{multline*}
%%%%%
%%%%%\begin{multline*}
%%%%%\frac{1}{2} \Bu \cross_5 \Bv \cdot \Bu = \\
%%%%%(-u_3-u_4-u_5)v_2 u_1 +( u_2-u_4-u_5)v_3 u_1 +( u_2+u_3-u_5)v_4 u_1 +( u_2+u_3+u_4)v_5 u_1 \\
%%%%%( u_3+u_4+u_5)v_1 u_2 +(-u_1-u_4-u_5)v_3 u_2 +( u_3-u_1-u_5)v_4 u_2 +( u_3+u_4-u_1)v_5 u_2 \\
%%%%%( u_4+u_5-u_2)v_1 u_3 +( u_1+u_4+u_5)v_2 u_3 +(-u_1-u_2-u_5)v_4 u_3 +( u_4-u_1-u_2)v_5 u_3 \\
%%%%%( u_5-u_2-u_3)v_1 u_4 +( u_1+u_5-u_3)v_2 u_4 +( u_1+u_2+u_5)v_3 u_4 +(-u_1-u_2-u_3)v_5 u_4 \\
%%%%%(-u_2-u_3-u_4)v_1 u_5 +( u_1-u_3-u_4)v_2 u_5 +( u_1+u_2-u_4)v_3 u_5 +( u_1+u_2+u_3)v_4 u_5 \\
%%%%%= \\
%%%%%(-u_3-u_4-u_5)v_2 u_1 + \\
%%%%%(+u_2-u_4-u_5)v_3 u_1 + \\
%%%%%(+u_2+u_3-u_5)v_4 u_1 + \\
%%%%%(+u_2+u_3+u_4)v_5 u_1 + \\
%%%%%(+u_3+u_4+u_5)v_1 u_2 + \\
%%%%%(-u_1-u_4-u_5)v_3 u_2 + \\
%%%%%(+u_3-u_1-u_5)v_4 u_2 + \\
%%%%%(+u_3+u_4-u_1)v_5 u_2 + \\
%%%%%(+u_4+u_5-u_2)v_1 u_3 + \\
%%%%%(+u_1+u_4+u_5)v_2 u_3 + \\
%%%%%(-u_1-u_2-u_5)v_4 u_3 + \\
%%%%%(+u_4-u_1-u_2)v_5 u_3 + \\
%%%%%(+u_5-u_2-u_3)v_1 u_4 + \\
%%%%%(+u_1+u_5-u_3)v_2 u_4 + \\
%%%%%(+u_1+u_2+u_5)v_3 u_4 + \\
%%%%%(-u_1-u_2-u_3)v_5 u_4 + \\
%%%%%(-u_2-u_3-u_4)v_1 u_5 + \\
%%%%%(+u_1-u_3-u_4)v_2 u_5 + \\
%%%%%(+u_1+u_2-u_4)v_3 u_5 + \\
%%%%%(+u_1+u_2+u_3)v_4 u_5 + \\
%%%%%= \\
%%%%%u_1 u_2 v_3 + \\
%%%%%u_1 u_2 v_4 + \\
%%%%%u_1 u_2 v_5 + \\
%%%%%u_1 u_2(-1) v_3 + \\
%%%%%u_1 u_2(-1) v_4 + \\
%%%%%u_1 u_2(-1) v_5 + \\
%%%%%u_1 u_3 v_2 + \\
%%%%%u_1 u_3 v_4 + \\
%%%%%u_1 u_3 v_5 + \\
%%%%%u_1 u_3(-1) v_2 + \\
%%%%%u_1 u_3(-1) v_4 + \\
%%%%%u_1 u_3(-1) v_5 + \\
%%%%%u_1 u_4 v_2 + \\
%%%%%u_1 u_4 v_3 + \\
%%%%%u_1 u_4 v_5 + \\
%%%%%u_1 u_4(-1) v_2 + \\
%%%%%u_1 u_4(-1) v_3 + \\
%%%%%u_1 u_4(-1) v_5 + \\
%%%%%u_1 u_5 v_2 + \\
%%%%%u_1 u_5 v_3 + \\
%%%%%u_1 u_5 v_4 + \\
%%%%%u_1 u_5(-1) v_2 + \\
%%%%%u_1 u_5(-1) v_3 + \\
%%%%%u_1 u_5(-1) v_4 + \\
%%%%%u_2 u_3 v_1 + \\
%%%%%u_2 u_3 v_4 + \\
%%%%%u_2 u_3 v_5 + \\
%%%%%u_2 u_3(-1) v_1 + \\
%%%%%u_2 u_3(-1) v_4 + \\
%%%%%u_2 u_3(-1) v_5 + \\
%%%%%u_2 u_4 v_1 + \\
%%%%%u_2 u_4 v_3 + \\
%%%%%u_2 u_4 v_5 + \\
%%%%%u_2 u_4(-1) v_1 + \\
%%%%%u_2 u_4(-1) v_3 + \\
%%%%%u_2 u_4(-1) v_5 + \\
%%%%%u_2 u_5 v_1 + \\
%%%%%u_2 u_5 v_3 + \\
%%%%%u_2 u_5 v_4 + \\
%%%%%u_2 u_5(-1) v_1 + \\
%%%%%u_2 u_5(-1) v_3 + \\
%%%%%u_2 u_5(-1) v_4 + \\
%%%%%u_3 u_4 v_1 + \\
%%%%%u_3 u_4 v_2 + \\
%%%%%u_3 u_4 v_5 + \\
%%%%%u_3 u_4(-1) v_1 + \\
%%%%%u_3 u_4(-1) v_2 + \\
%%%%%u_3 u_4(-1) v_5 + \\
%%%%%u_3 u_5 v_1 + \\
%%%%%u_3 u_5 v_2 + \\
%%%%%u_3 u_5 v_4 + \\
%%%%%u_3 u_5(-1) v_1 + \\
%%%%%u_3 u_5(-1) v_2 + \\
%%%%%u_3 u_5(-1) v_4 + \\
%%%%%u_4 u_5 v_1 + \\
%%%%%u_4 u_5 v_2 + \\
%%%%%u_4 u_5 v_3 + \\
%%%%%u_4 u_5(-1) v_1 + \\
%%%%%u_4 u_5(-1) v_2 + \\
%%%%%u_4 u_5(-1) v_3
%%%%%\end{multline*}
%%%%%
%%%%%\begin{multline*}
%%%%%\frac{1}{2} \Bu \cross_5 \Bu = \\
%%%%%\begin{bmatrix}
%%%%%(-u_3-u_4-u_5)u_2 +( u_2-u_4-u_5)u_3 +( u_2+u_3-u_5)u_4 +( u_2+u_3+u_4)u_5 \\
%%%%%( u_3+u_4+u_5)u_1 +(-u_1-u_4-u_5)u_3 +( u_3-u_1-u_5)u_4 +( u_3+u_4-u_1)u_5 \\
%%%%%( u_4+u_5-u_2)u_1 +( u_1+u_4+u_5)u_2 +(-u_1-u_2-u_5)u_4 +( u_4-u_1-u_2)u_5 \\
%%%%%( u_5-u_2-u_3)u_1 +( u_1+u_5-u_3)u_2 +( u_1+u_2+u_5)u_3 +(-u_1-u_2-u_3)u_5 \\
%%%%%(-u_2-u_3-u_4)u_1 +( u_1-u_3-u_4)u_2 +( u_1+u_2-u_4)u_3 +( u_1+u_2+u_3)u_4
%%%%%\end{bmatrix}
%%%%%= 0
%%%%%\end{multline*}
%%%%%% end verification.

One additional property that holds for the three dimensional cross product
that also holds for the \R{n} version is
\(\tripleprod{\Bu}{\Bv}{\Bw} = -\tripleprod{\Bu}{\Bw}{\Bv}\).
If \(\crossprod{\Bu}{\Bu} = 0\) is true for the \R{n} cross product
defined above, then this implies that
\(\tripleprod{\Bu}{\Bv}{\Bu} = 0\) too.  This first property can be shown by
writing \(\crossop{\Bu} = \BG - {\BG}^T = [g_{ij}]\), so that

\begin{equation*}
(\crossprod{\Bu}{\Bv})_i = \sum_{s=1}^{n}{g_{is}v_s}
\end{equation*}

thus for the triple-product
\begin{equation}\label{eqn:crossOld:460}
\begin{aligned}
\tripleprod{\Bu}{\Bv}{\Bw} &= \sum_{t=1}^{n}{(\sum_{s=1}^{n}g_{ts}v_s)w_t} \\
                           &= \sum_{s=1}^{n}{(\sum_{t=1}^{n}g_{ts}w_t)v_s} \\
			   &= \dotprod{{([g_{ij}]}^T \Bw)}{\Bv} \\
			   &= \dotprod{(\BG-\BG^T)^T \Bw}{\Bv} \\
			   &= -\dotprod{(\BG-\BG^T) \Bw}{\Bv} \\
			   &= -\tripleprod{\Bu}{\Bw}{\Bv}
\end{aligned}
\end{equation}

I suspect that \(\crossprod{\Bu}{\Bu} = \Bzero\) and that \(\tripleprod{\Bu}{\Bv}{\Bv} = 0\) also hold for \(n>5\) in the \R{n} cross product as defined above.
Some sort of recursive proof for this is probably required to show this.

\subsection{on the magnitude of the cross product operator}

The orthogonality properties of the cross product operator are not the only
ones of interest, since the cross product in \R{3} has a specific
magnitude as well as direction.

The projective form
\(\crossprod{\Bu}{\Bv} = \norm{\Bu}\norm{\Bv}\sin(\Bu,\Bv) \ncap\) may give
some indication of what to expect for \R{n}, where \((\Bu,\Bv)\) is the
angle between the two vectors \(\Bu\) and \(\Bv\), and \(\ncap\) is a unit
vector in the direction of \(\crossprod{\Bu}{\Bv}\).  However, how would the
angle be defined for \R{n}.

For this we can go to the projective form of the dot product
\(\dotprod{\Bu}{\Bv} = \norm{\Bu}\norm{\Bv}\cos(\Bu,\Bv)\).

Note that this form of the dot product comes directly from the
triangle law of trigonometry.

\begin{equation}\label{eqn:crossOld:480}
\begin{aligned}
\norm{\Bu - \Bv}^2 &= \norm{\Bu}^2 + \norm{\Bv}^2 -2\norm{\Bu}\norm{\Bv}\cos(\Bu,\Bv) \\
                  &= \innerprod{\Bu-\Bv}{\Bu-\Bv} \\
                  &=
		  \innerprod{\Bu}{\Bu}
		  +\innerprod{\Bv}{\Bv}
		  -\innerprod{\Bu}{\Bv}
		  -\innerprod{\Bv}{\Bu} \\
                  &= \norm{\Bu}^2 + \norm{\Bv}^2 -\innerprod{\Bu}{\Bv} -\innerprod{\Bv}{\Bu}
\end{aligned}
\end{equation}

The result follows, since for the real case, \(\innerprod{\Bu}{\Bv} + \innerprod{\Bv}{\Bu} = 2\dotprod{\Bu}{\Bv}\).

If \(\cos(\Bu,\Bv) = \frac{\dotprod{\Bu}{\Bv}}{\norm{\Bu}\norm{\Bv}}\) is taken to implicitly define the
angle between two vectors in \R{n}, then the magnitude of the \R{n} cross product could be defined in
the following fashion as is true for \R{3}

\begin{dmath}\label{eqn:crossOld:500}
\norm{\crossprod{\Bu}{\Bv}}^2
=
\norm{\Bu}^2\norm{\Bv}^2
\lr{
1-
\lr{
\frac{\dotprod{\Bu}{\Bv}}{\norm{\Bu}\norm{\Bv}}
}^2
}
= \norm{\Bu}^2\norm{\Bv}^2 - (\dotprod{\Bu}{\Bv})^2
\end{dmath}

Note
that using the norm squared as a measure of magnitude looses the sign of the magnitude.  There may be
a better way of defining \(\sin(\Bu,\Bv) = \sqrt{1 - \frac{\dotprod{\Bu}{\Bv}}{\norm{\Bu}\norm{\Bv}}}\)
because this has an implied sign ambiguity.  Then again the \(\ncap\) term has also been ignored, so
perhaps the positive root is an acceptable angular measure.

I still need to check if this is true for the \R{n} cross product operator as defined above for \(n>3\).
In order to calculate
\(\norm{\Bu \cross_n \Bv}^2\)
the following sum has to be evaluated

\begin{multline*}
((n-3)!)^2
\Bigl(
\sum_{i=1}^{n-2}\sum_{j=1}^{n-1-i}(\PDP{i}{j}\Bv - (\PDP{i}{j})^T\Bv)
\Bigr)
\cdot \\
\Bigl(
\sum_{i'=1}^{n-2}\sum_{j'=1}^{n-1-i'}(\PDP{i'}{j'}\Bv - (\PDP{i'}{j'})^T\Bv)
\Bigr)
\end{multline*}

The dot product can be brought into the sum
\begin{multline*}
((n-3)!)^2
\sum_{i=1}^{n-2}\sum_{j=1}^{n-1-i}
\sum_{i'=1}^{n-2}\sum_{j'=1}^{n-1-i'}
\Bigl(
(\PDP{i}{j}\Bv - (\PDP{i}{j})^T\Bv)
\\
\cdot
(\PDP{i'}{j'}\Bv - (\PDP{i'}{j'})^T\Bv)
\Bigr)
\end{multline*}

and this can be expanded
\begin{multline*}
((n-3)!)^2
\sum_{i=1}^{n-2}\sum_{j=1}^{n-1-i}
\sum_{i'=1}^{n-2}\sum_{j'=1}^{n-1-i'} \\
  \dotprod
   {\PDP{i}{j}\Bv}
   {\PDP{i'}{j'}\Bv}
 -\dotprod
   {\PDP{i}{j}\Bv}
   {(\PDP{i'}{j'})^T\Bv} \\
 +\dotprod
   {(\PDP{i}{j})^T\Bv}
   {(\PDP{i'}{j'})^T\Bv}
 -\dotprod
   {(\PDP{i}{j})^T\Bv}
   {\PDP{i'}{j'}\Bv}
\end{multline*}

There are a couple further manipulations that can be done, since
\(\dotprod{\Ba}{\Bb} = \Ba^T\Bv\), \(\BP^T = \BP^{-1}\), and
\(\BP^{-i} = \BP^{n-i}\).

\begin{multline*}
((n-3)!)^2
\sum_{i=1}^{n-2}\sum_{j=1}^{n-1-i}
\sum_{i'=1}^{n-2}\sum_{j'=1}^{n-1-i'} \\
   \PDPDP{-j}{i'-i}{j'}
  -\PDPDP{-j}{-i-j'}{-i'} \\
  +\PDPDP{i}{j-j'}{-i'}
  -\PDPDP{i}{j+i'}{j'}
\end{multline*}

Well, after all this, I am not actually any closer to getting \(\norm{\crossprod{\Bu}{\Bv}}^2\) evaluated.
Perhaps there is an easier way.  This matrix formulation for \(\crossop{\Bu}\) is a nice way for expression
but it has turned out to be a bit awkward for manipulation, at least without a way to expand \(\PDP{i}{j}\),
which I have not tried for the general case.

\section{Appendix 1}
\subsection{Change of basis, transformations, and rotations}

Given an orthogonal basis \((\ucap_i)_i\) in one coordinate system and an
orthogonal basis \(({\ucap_i}')_i\) for the same coordinate system, how are
the two related?

The two sets of unit vectors can be related by a set of linear equations

\begin{equation}\label{eqn:crossOld:520}
\begin{aligned}
{\ucap_i}' &= \sum_{s=1}^n{a_{is}\ucap_s} \\
\ucap_i &= \sum_{s=1}^n{b_{is}{\ucap_s}'}
\end{aligned}
\end{equation}

What the values of \(a_{ij}\) or \(b_{ij}\) are can be determined by taking inner products and by using the
orthogonality constraints.

\begin{equation}\label{eqn:crossOld:540}
\begin{aligned}
\innerprod{{\ucap_i}'}{\ucap_j} &= \sum_{s=1}^n{a_{is}\innerprod{\ucap_s}{\ucap_j}} \\
                                &= \sum_{s=1}^n{a_{is}\delta_{sj}} \\
                                &= a_{ij} \\
\innerprod{\ucap_i}{{\ucap_j}'} &= \sum_{s=1}^n{b_{is}\innerprod{{\ucap_s}'}{{\ucap_j}'}} \\
                                &= \sum_{s=1}^n{b_{is}\delta_{sj}} \\
                                &= b_{ij} \\
                                &= \overline{a_{ji}} \\
\end{aligned}
\end{equation}

So the relationships between the two sets of basis vectors \({\ucap_i}'\) and \(\ucap_i\) are

\begin{equation}\label{eqn:crossOld:560}
\begin{aligned}
{\ucap_i}'
&= \sum_{s=1}^n{
a_{is}
\ucap_s
}
&=
\sum_{s=1}^n{
\innerprod{{\ucap_i}'}{\ucap_s}
\ucap_s
}
\\
\ucap_i
&= \sum_{s=1}^n{
\overline{a_{si}}
{\ucap_s}'
}
&=
\sum_{s=1}^n{
\innerprod{\ucap_i}{{\ucap_s}'}
{\ucap_s}'
}
\end{aligned}
\end{equation}

Note that these two relationships can be expressed with a transformation
matrix \(\BM\) and its Hermitian transpose \(\BM^*\)

\begin{equation*}
\begin{bmatrix}
{\ucap_1}' \\
{\ucap_2}' \\
\vdots	  \\
{\ucap_n}'
\end{bmatrix}
=
\begin{bmatrix}
	a_{11} & a_{12} & \dots  & a_{1n} \\
        a_{21} & a_{22} & 	  &        \\
	\vdots &        & \ddots &        \\
	a_{n1} & \dots  &        & a_{nn}
\end{bmatrix}
\begin{bmatrix}
\ucap_1  \\
\ucap_2  \\
\vdots	  \\
\ucap_n
\end{bmatrix}
= \BM
\begin{bmatrix}
\ucap_1  \\
\ucap_2  \\
\vdots	  \\
\ucap_n
\end{bmatrix}
\end{equation*}

\begin{equation*}
\begin{bmatrix}
\ucap_1 \\
\ucap_2 \\
\vdots	  \\
\ucap_n
\end{bmatrix}
=
\begin{bmatrix}
	\overline{a_{11}} & \overline{a_{21}} & \dots  & \overline{a_{n1}} \\
        \overline{a_{12}} & \overline{a_{22}} & 	  &        \\
	\vdots &        & \ddots &        \\
	\overline{a_{1n}} & \dots  &        & \overline{a_{nn}}
\end{bmatrix}
\begin{bmatrix}
{\ucap_1}' \\
{\ucap_2}' \\
\vdots	  \\
{\ucap_n}'
\end{bmatrix}
= \BM^*
\begin{bmatrix}
{\ucap_1}' \\
{\ucap_2}' \\
\vdots	  \\
{\ucap_n}'
\end{bmatrix}
\end{equation*}

or

\begin{equation*}
\begin{bmatrix}
{\ucap_1}' \\
{\ucap_2}' \\
\vdots	  \\
{\ucap_n}'
\end{bmatrix}
=
\begin{bmatrix}
	\innerprod{{\ucap_1}'}{\ucap_1} & \innerprod{{\ucap_1}'}{\ucap_2} & \dots  & \innerprod{{\ucap_1}'}{\ucap_n} \\
        \innerprod{{\ucap_2}'}{\ucap_1} & \innerprod{{\ucap_2}'}{\ucap_2} & 	  &        \\
	\vdots &        & \ddots &        \\
	\innerprod{{\ucap_n}'}{\ucap_1} & \dots  &        & \innerprod{{\ucap_n}'}{\ucap_n}
\end{bmatrix}
\begin{bmatrix}
\ucap_1  \\
\ucap_2  \\
\vdots	  \\
\ucap_n
\end{bmatrix}
\end{equation*}

Given an arbitrary vector \(\Br = [r_j]_j\) in the primary coordinate system, one
can express this vector \(\Br' = [r_j']_j\) in the secondary coordinate system using
the same sort procedure used to derive the transformation matrix \(\BM\).

\begin{equation}\label{eqn:crossOld:580}
\begin{aligned}
\Br' &=
      \sum_{s=1}^n
      {
       r_s
       \ucap_s
      } \\
      &=
      \sum_{s=1}^n
      {
       r_s
\sum_{t=1}^n
{
\overline{a_{ts}}
{\ucap_t}'
}
      } \\
      &=
\sum_{t=1}^n
      {
{\ucap_t}'
      \sum_{s=1}^n
{
\overline{a_{ts}}
       r_s
}
      } \\
      &=
\sum_{t=1}^n
      {
{\ucap_t}'
r_t'
      }
\end{aligned}
\end{equation}

Since $r_i' =
      \sum_{s=1}^n
{
\overline{a_{is}}
       r_s
}
$
one can see that the components of the vectors transform
in a similar fashion the
basis vectors, and this can be written \(\Br = \BM^T \Br'\) and \(\Br' = \overline{\BM} \Br\).

When deriving this this result seemed odd at first, and found myself wondering if have I messed up despite the fact everything looked okay?  On paper I had only derived this case for \R{n} and not \C{n}.\footnote{
A worked example showed that transformation of the coordinate vectors and the basis vectors do differ by a complex conjugate factor.

Setting \({\ucap_1}' = \inv{\sqrt{2}}(1,i), {\ucap_2}'=\inv{\sqrt{2}}(1,-i)\),
\(\ucap_i = \ecap_i\) the unit vectors in \R{2}, then
$\BM =
\inv{\sqrt{2}}
\Bigl[
\begin{smallmatrix}
1 & i \\
1 & -i
\end{smallmatrix}
\Bigr]
$.  Picking an arbitrary test vector
\(\Br = (1,1) = \ucap_1 + \ucap_2 = \inv{\sqrt{2}}((1-i){\ucap_1}' + (1+i){\ucap_2}')\) the application of the
transformation formulas shows $\Br = \BM^T \Br' =
\inv{\sqrt{2}}
\Bigl[
\begin{smallmatrix}
1 & 1 \\
i & -i
\end{smallmatrix}
\Bigr]
\inv{\sqrt{2}}
\Bigl[
\begin{smallmatrix}
1 - i \\
1 + i
\end{smallmatrix}
\Bigr]
=
\Bigl[
\begin{smallmatrix}
1 \\
1
\end{smallmatrix}
\Bigr]
$ as expected.
}

It does not matter too much, because I do not need the result for the general case in the torque examination anyhow.

%\end{document}               % End of document.


\part{geometric-algebra}
\documentclass{article}      % Specifies the document class

\usepackage{amsmath}
\usepackage{mathpazo}

%
% shorthand for bold symbols, convenient for vectors and matrices
%
\newcommand{\Ba}[0]{\mathbf{a}}
\newcommand{\Bb}[0]{\mathbf{b}}
\newcommand{\Bc}[0]{\mathbf{c}}
\newcommand{\Bd}[0]{\mathbf{d}}
\newcommand{\Be}[0]{\mathbf{e}}
\newcommand{\Bf}[0]{\mathbf{f}}
\newcommand{\Bg}[0]{\mathbf{g}}
\newcommand{\Bh}[0]{\mathbf{h}}
\newcommand{\Bi}[0]{\mathbf{i}}
\newcommand{\Bj}[0]{\mathbf{j}}
\newcommand{\Bk}[0]{\mathbf{k}}
\newcommand{\Bl}[0]{\mathbf{l}}
\newcommand{\Bm}[0]{\mathbf{m}}
\newcommand{\Bn}[0]{\mathbf{n}}
\newcommand{\Bo}[0]{\mathbf{o}}
\newcommand{\Bp}[0]{\mathbf{p}}
\newcommand{\Bq}[0]{\mathbf{q}}
\newcommand{\Br}[0]{\mathbf{r}}
\newcommand{\Bs}[0]{\mathbf{s}}
\newcommand{\Bt}[0]{\mathbf{t}}
\newcommand{\Bu}[0]{\mathbf{u}}
\newcommand{\Bv}[0]{\mathbf{v}}
\newcommand{\Bw}[0]{\mathbf{w}}
\newcommand{\Bx}[0]{\mathbf{x}}
\newcommand{\By}[0]{\mathbf{y}}
\newcommand{\Bz}[0]{\mathbf{z}}
\newcommand{\BA}[0]{\mathbf{A}}
\newcommand{\BB}[0]{\mathbf{B}}
\newcommand{\BC}[0]{\mathbf{C}}
\newcommand{\BD}[0]{\mathbf{D}}
\newcommand{\BE}[0]{\mathbf{E}}
\newcommand{\BF}[0]{\mathbf{F}}
\newcommand{\BG}[0]{\mathbf{G}}
\newcommand{\BH}[0]{\mathbf{H}}
\newcommand{\BI}[0]{\mathbf{I}}
\newcommand{\BJ}[0]{\mathbf{J}}
\newcommand{\BK}[0]{\mathbf{K}}
\newcommand{\BL}[0]{\mathbf{L}}
\newcommand{\BM}[0]{\mathbf{M}}
\newcommand{\BN}[0]{\mathbf{N}}
\newcommand{\BO}[0]{\mathbf{O}}
\newcommand{\BP}[0]{\mathbf{P}}
\newcommand{\BQ}[0]{\mathbf{Q}}
\newcommand{\BR}[0]{\mathbf{R}}
\newcommand{\BS}[0]{\mathbf{S}}
\newcommand{\BT}[0]{\mathbf{T}}
\newcommand{\BU}[0]{\mathbf{U}}
\newcommand{\BV}[0]{\mathbf{V}}
\newcommand{\BW}[0]{\mathbf{W}}
\newcommand{\BX}[0]{\mathbf{X}}
\newcommand{\BY}[0]{\mathbf{Y}}
\newcommand{\BZ}[0]{\mathbf{Z}}

\newcommand{\Bzero}[0]{\mathbf{0}}
\newcommand{\Btheta}[0]{\boldsymbol{\theta}}
\newcommand{\Btau}[0]{\boldsymbol{\tau}}
\newcommand{\Bomega}[0]{\boldsymbol{\omega}}

%
% shorthand for unit vectors
%
\newcommand{\acap}[0]{\hat{\Ba}}
\newcommand{\bcap}[0]{\hat{\Bb}}
\newcommand{\ccap}[0]{\hat{\Bc}}
\newcommand{\dcap}[0]{\hat{\Bd}}
\newcommand{\ecap}[0]{\hat{\Be}}
\newcommand{\fcap}[0]{\hat{\Bf}}
\newcommand{\gcap}[0]{\hat{\Bg}}
\newcommand{\hcap}[0]{\hat{\Bh}}
\newcommand{\icap}[0]{\hat{\Bi}}
\newcommand{\jcap}[0]{\hat{\Bj}}
\newcommand{\kcap}[0]{\hat{\Bk}}
\newcommand{\lcap}[0]{\hat{\Bl}}
\newcommand{\mcap}[0]{\hat{\Bm}}
\newcommand{\ncap}[0]{\hat{\Bn}}
\newcommand{\ocap}[0]{\hat{\Bo}}
\newcommand{\pcap}[0]{\hat{\Bp}}
\newcommand{\qcap}[0]{\hat{\Bq}}
\newcommand{\rcap}[0]{\hat{\Br}}
\newcommand{\scap}[0]{\hat{\Bs}}
\newcommand{\tcap}[0]{\hat{\Bt}}
\newcommand{\ucap}[0]{\hat{\Bu}}
\newcommand{\vcap}[0]{\hat{\Bv}}
\newcommand{\wcap}[0]{\hat{\Bw}}
\newcommand{\xcap}[0]{\hat{\Bx}}
\newcommand{\ycap}[0]{\hat{\By}}
\newcommand{\zcap}[0]{\hat{\Bz}}
\newcommand{\thetacap}[0]{\hat{\Btheta}}

%
% to write R^n and C^n in a distinguishable fashion.  Perhaps change this
% to the double lined characters upon figuring out how to do so.
%
\newcommand{\C}[1]{$\mathbb{C}^{#1}$}
\newcommand{\R}[1]{$\mathbb{R}^{#1}$}

%
% various generally useful helpers
%

% derivative of #1 wrt. #2:
\newcommand{\D}[2] {\frac {d#2} {d#1}}

\newcommand{\inv}[1]{\frac{1}{#1}}
\newcommand{\cross}[0]{\times}

\newcommand{\abs}[1]{\lvert{#1}\rvert}
\newcommand{\norm}[1]{\lVert{#1}\rVert}
\newcommand{\innerprod}[2]{\langle{#1}, {#2}\rangle}
\newcommand{\dotprod}[2]{{#1} \cdot {#2}}
\newcommand{\bdotprod}[2]{\left({#1} \cdot {#2}\right)}
\newcommand{\crossprod}[2]{{#1} \cross {#2}}
\newcommand{\tripleprod}[3]{\dotprod{\left(\crossprod{#1}{#2}\right)}{#3}}

\DeclareMathOperator{\Proj}{Proj}
\DeclareMathOperator{\Span}{span}
\DeclareMathOperator{\Sgn}{sgn}
\DeclareMathOperator{\Area}{Area}
\DeclareMathOperator{\Volume}{Volume}

%
% A few miscellaneous things specific to this document
%
\newcommand{\crossop}[1]{\crossprod{#1}{}}

% R2 vector.
\newcommand{\VectorTwo}[2]{
\begin{bmatrix}
 {#1} \\
 {#2}
\end{bmatrix}
}

\newcommand{\VectorN}[1]{
\begin{bmatrix}
{#1}_1 \\
{#1}_2 \\
\vdots \\
{#1}_N \\
\end{bmatrix}
}

\newcommand{\DETuvij}[4]{
\begin{vmatrix}
 {#1}_{#3} & {#1}_{#4} \\
 {#2}_{#3} & {#2}_{#4}
\end{vmatrix}
}

\newcommand{\DETuvwijk}[6]{
\begin{vmatrix}
 {#1}_{#4} & {#1}_{#5} & {#1}_{#6} \\
 {#2}_{#4} & {#2}_{#5} & {#2}_{#6} \\
 {#3}_{#4} & {#3}_{#5} & {#3}_{#6}
\end{vmatrix}
}

\newcommand{\DETuvwxijkl}[8]{
\begin{vmatrix}
 {#1}_{#5} & {#1}_{#6} & {#1}_{#7} & {#1}_{#8} \\
 {#2}_{#5} & {#2}_{#6} & {#2}_{#7} & {#2}_{#8} \\
 {#3}_{#5} & {#3}_{#6} & {#3}_{#7} & {#3}_{#8} \\
 {#4}_{#5} & {#4}_{#6} & {#4}_{#7} & {#4}_{#8} \\
\end{vmatrix}
}

%\newcommand{\DETuvwxyijklm}[10]{
%\begin{vmatrix}
% {#1}_{#6} & {#1}_{#7} & {#1}_{#8} & {#1}_{#9} & {#1}_{#10} \\
% {#2}_{#6} & {#2}_{#7} & {#2}_{#8} & {#2}_{#9} & {#2}_{#10} \\
% {#3}_{#6} & {#3}_{#7} & {#3}_{#8} & {#3}_{#9} & {#3}_{#10} \\
% {#4}_{#6} & {#4}_{#7} & {#4}_{#8} & {#4}_{#9} & {#4}_{#10} \\
% {#5}_{#6} & {#5}_{#7} & {#5}_{#8} & {#5}_{#9} & {#5}_{#10}
%\end{vmatrix}
%}

% R3 vector.
\newcommand{\VectorThree}[3]{
\begin{bmatrix}
 {#1} \\
 {#2} \\
 {#3}
\end{bmatrix}
}


\newcommand{\halfPhi}[0]{\frac{\phi}{2}}
\newcommand{\Sin}[1]{\sin{\left({#1}\right)}}
\newcommand{\Cos}[1]{\cos{\left({#1}\right)}}
\newcommand{\Exp}[1]{\exp{\left({#1}\right)}}

%
% The real thing:
%

                             % The preamble begins here.
\title{ Questioning equation 8.8 from GASC. } % Declares the document's title.
\author{Peeter Joot \quad peeter.joot@gmail.com}         % Declares the author's name.
\date{ May 9, 2008.  Last Revision: $Date: 2009/02/22 15:35:07 $ }

\begin{document}             % End of preamble and beginning of text.

\maketitle{}

\section{ email 1. }

In formula 8.8 you take the scalar derivative of a rotor applied to a blade, where ``$I \phi$ is a function of $\tau$ so that both rotation plane and rotation angle may vary''.  I'm not sure about the end result, but one of the intermediate steps looks wrong to me.  It appears that you implicitly use the following to arrive at your second line:

\[
\Exp{ I \phi }' = ( I \phi )' \Exp{ I \phi }
\]

(writing primes for your $\partial_\tau$).

If I calculate a bivector exponential I get:

\begin{align*}
\Exp{ I \phi }'
&= (\Cos{\phi} + I \Sin{\phi})' \\
&= (-\Sin{\phi} + I \Cos{\phi}) \phi' + I' \Sin{\phi} \\
&= I \phi' \Exp{ I \phi } + I' \Sin{\phi} \\
\end{align*}

I believe this is generally only $(I \phi)' \Exp{ I \phi }$ when $I' = 0$, which contradicts the description where you allow for variation of the rotation plane?
%  If I misunderstand I'd appreciate if you could point out how (and consider this point for clarification in a future edition).  Thanks!

\section{ email 2. } 

I've answered my own question about what equation 8.8 should look like if one allows for a general rotation (ie: allowing for wobble), as implied by the text.

Applying logic from the rigid body treatment in a book like ``GA for Physicists'' then you can arrive at a commutator result for blades similar to your equation 8.8:

\[
X = R X_0 R^\dagger
\]
\begin{align*}
\implies
X' &= R' X_0 R^\dagger + R X_0 {R'}^\dagger \\
   &= R' (R^\dagger R) X_0 R^\dagger + R X_0 (R^\dagger R) {R'}^\dagger \\
   &= R' R^\dagger X + X R {R'}^\dagger \\
\end{align*}

Until this point the procedure isn't that much different than your 8.8 treatment, but to get a commutator result, you need the rigid body trick:

\[
(R R^\dagger)' = 1' = 0 = R {R'}^\dagger + R' R^\dagger
\]

That trick you can also find in non-GA rigid body texts with matrixes $R R^T = I$.  The rotor deriviative will also only have 0,2 grades, so the product potentially has 0,2,4 grades.  Since it value is the negation of its reverse, it must be strictly grade 2.  Introducing $R' R^\dagger = \inv{2} \Omega$, and using the result above, this gives:

\[
X' = \inv{2}(\Omega X - X \Omega) = \Omega \times X
\]

%Hindsight shows that incorporating the factor of two in $\Omega$ would be cleaner, but that's sneaky:)

I'm pretty sure that this is how 8.8 has to look if you allow for wobbly rotation ($\frac{dI}{d\tau} \ne 0$), so the question part of my original note is resolved to my satisfaction.  Writing this up, it occurs to me that, that your equation 8.8, while exact for a fixed orientation rotation, is probably accurate up to a first order differential approximation of a wobbly rotation.

\section{ Post emails. }

Let's try this.  With the derivative calculated above we have for $R'R^\dagger$, where $R = \Exp{-I \halfPhi}$ :

\begin{align*}
R'R^\dagger 
&= \left(-I \phi'/2 \Exp{ -I \halfPhi } - I' \Sin{\halfPhi} \right)\Exp{ I \halfPhi } \\ 
&= -I \phi'/2 - I' \Sin{\halfPhi} \Exp{ I \halfPhi } \\
&= -I \phi'/2 - I' \Sin{\halfPhi}\Cos{\halfPhi} + I'I \Sin{\halfPhi} \\
&= -I \phi'/2 - I'/2 \Sin{\phi} + I'I \Sin{\halfPhi} \\
\end{align*}

\[
\implies
\Omega = -I \phi' - I' \Sin{\phi} + 2 I'I \Sin{\halfPhi}
\]
\begin{align*}
X'
&= \left( -I \phi' - I' \Sin{\phi} + 2 I'I \Sin{\halfPhi} \right) \times X \\
&= \inv{2} \left( \left( -I \phi' - I' \Sin{\phi} + 2 I'I \Sin{\halfPhi} \right) X - X \left( -I \phi' - I' \Sin{\phi} + 2 I'I \Sin{\halfPhi} \right) \right) \\
&= -I \phi' \times X  - I'\Sin{\phi} \times X + \Sin{\halfPhi} \left( I'I X - X I' I \right)  \\
&= X \times \left(I \phi' + I'\Sin{\phi} \right ) + 2 \Sin{\halfPhi} \left( I'I \right) \times X \\
&= X \times \left(I \phi' + I'\Sin{\phi} - 2 \Sin{\halfPhi} I'I \right) \\
\end{align*}

Now for very small $\phi$, $\Sin{\phi} = \phi$, so we have:

\[
X' \approx X \times (I \phi)' + \phi ( I'I ) \times X
\]
or:
\[
X' \approx X \times \left( (I \phi)' - \phi ( I'I ) \right)
\]

In the GASC notation this is
\[
\partial_\tau X \approx X \times \partial_\tau (I \phi) - X \times (\partial_\tau(I) I \phi)
\]

The first part of this is equation 8.8, but only holds when the angle $\phi$ is small.  We also have a corrective term that comes only from variation in the bivector orientation, and is also an approximation that holds only when the angle $\phi$ is small.

\end{document}               % End of document.

\documentclass{article}      % Specifies the document class

\usepackage{amsmath}
\usepackage{mathpazo}

%
% shorthand for bold symbols, convenient for vectors and matrices
%
\newcommand{\Ba}[0]{\mathbf{a}}
\newcommand{\Bb}[0]{\mathbf{b}}
\newcommand{\Bc}[0]{\mathbf{c}}
\newcommand{\Bd}[0]{\mathbf{d}}
\newcommand{\Be}[0]{\mathbf{e}}
\newcommand{\Bf}[0]{\mathbf{f}}
\newcommand{\Bg}[0]{\mathbf{g}}
\newcommand{\Bh}[0]{\mathbf{h}}
\newcommand{\Bi}[0]{\mathbf{i}}
\newcommand{\Bj}[0]{\mathbf{j}}
\newcommand{\Bk}[0]{\mathbf{k}}
\newcommand{\Bl}[0]{\mathbf{l}}
\newcommand{\Bm}[0]{\mathbf{m}}
\newcommand{\Bn}[0]{\mathbf{n}}
\newcommand{\Bo}[0]{\mathbf{o}}
\newcommand{\Bp}[0]{\mathbf{p}}
\newcommand{\Bq}[0]{\mathbf{q}}
\newcommand{\Br}[0]{\mathbf{r}}
\newcommand{\Bs}[0]{\mathbf{s}}
\newcommand{\Bt}[0]{\mathbf{t}}
\newcommand{\Bu}[0]{\mathbf{u}}
\newcommand{\Bv}[0]{\mathbf{v}}
\newcommand{\Bw}[0]{\mathbf{w}}
\newcommand{\Bx}[0]{\mathbf{x}}
\newcommand{\By}[0]{\mathbf{y}}
\newcommand{\Bz}[0]{\mathbf{z}}
\newcommand{\BA}[0]{\mathbf{A}}
\newcommand{\BB}[0]{\mathbf{B}}
\newcommand{\BC}[0]{\mathbf{C}}
\newcommand{\BD}[0]{\mathbf{D}}
\newcommand{\BE}[0]{\mathbf{E}}
\newcommand{\BF}[0]{\mathbf{F}}
\newcommand{\BG}[0]{\mathbf{G}}
\newcommand{\BH}[0]{\mathbf{H}}
\newcommand{\BI}[0]{\mathbf{I}}
\newcommand{\BJ}[0]{\mathbf{J}}
\newcommand{\BK}[0]{\mathbf{K}}
\newcommand{\BL}[0]{\mathbf{L}}
\newcommand{\BM}[0]{\mathbf{M}}
\newcommand{\BN}[0]{\mathbf{N}}
\newcommand{\BO}[0]{\mathbf{O}}
\newcommand{\BP}[0]{\mathbf{P}}
\newcommand{\BQ}[0]{\mathbf{Q}}
\newcommand{\BR}[0]{\mathbf{R}}
\newcommand{\BS}[0]{\mathbf{S}}
\newcommand{\BT}[0]{\mathbf{T}}
\newcommand{\BU}[0]{\mathbf{U}}
\newcommand{\BV}[0]{\mathbf{V}}
\newcommand{\BW}[0]{\mathbf{W}}
\newcommand{\BX}[0]{\mathbf{X}}
\newcommand{\BY}[0]{\mathbf{Y}}
\newcommand{\BZ}[0]{\mathbf{Z}}

\newcommand{\Bzero}[0]{\mathbf{0}}
\newcommand{\Btheta}[0]{\boldsymbol{\theta}}
\newcommand{\Btau}[0]{\boldsymbol{\tau}}
\newcommand{\Bomega}[0]{\boldsymbol{\omega}}

%
% shorthand for unit vectors
%
\newcommand{\acap}[0]{\hat{\Ba}}
\newcommand{\bcap}[0]{\hat{\Bb}}
\newcommand{\ccap}[0]{\hat{\Bc}}
\newcommand{\dcap}[0]{\hat{\Bd}}
\newcommand{\ecap}[0]{\hat{\Be}}
\newcommand{\fcap}[0]{\hat{\Bf}}
\newcommand{\gcap}[0]{\hat{\Bg}}
\newcommand{\hcap}[0]{\hat{\Bh}}
\newcommand{\icap}[0]{\hat{\Bi}}
\newcommand{\jcap}[0]{\hat{\Bj}}
\newcommand{\kcap}[0]{\hat{\Bk}}
\newcommand{\lcap}[0]{\hat{\Bl}}
\newcommand{\mcap}[0]{\hat{\Bm}}
\newcommand{\ncap}[0]{\hat{\Bn}}
\newcommand{\ocap}[0]{\hat{\Bo}}
\newcommand{\pcap}[0]{\hat{\Bp}}
\newcommand{\qcap}[0]{\hat{\Bq}}
\newcommand{\rcap}[0]{\hat{\Br}}
\newcommand{\scap}[0]{\hat{\Bs}}
\newcommand{\tcap}[0]{\hat{\Bt}}
\newcommand{\ucap}[0]{\hat{\Bu}}
\newcommand{\vcap}[0]{\hat{\Bv}}
\newcommand{\wcap}[0]{\hat{\Bw}}
\newcommand{\xcap}[0]{\hat{\Bx}}
\newcommand{\ycap}[0]{\hat{\By}}
\newcommand{\zcap}[0]{\hat{\Bz}}
\newcommand{\thetacap}[0]{\hat{\Btheta}}

%
% to write R^n and C^n in a distinguishable fashion.  Perhaps change this
% to the double lined characters upon figuring out how to do so.
%
\newcommand{\C}[1]{$\mathbb{C}^{#1}$}
\newcommand{\R}[1]{$\mathbb{R}^{#1}$}

%
% various generally useful helpers
%

% derivative of #1 wrt. #2:
\newcommand{\D}[2] {\frac {d#2} {d#1}}

\newcommand{\inv}[1]{\frac{1}{#1}}
\newcommand{\cross}[0]{\times}

\newcommand{\abs}[1]{\lvert{#1}\rvert}
\newcommand{\norm}[1]{\lVert{#1}\rVert}
\newcommand{\innerprod}[2]{\langle{#1}, {#2}\rangle}
\newcommand{\dotprod}[2]{{#1} \cdot {#2}}
\newcommand{\bdotprod}[2]{\left({#1} \cdot {#2}\right)}
\newcommand{\crossprod}[2]{{#1} \cross {#2}}
\newcommand{\tripleprod}[3]{\dotprod{\left(\crossprod{#1}{#2}\right)}{#3}}

\DeclareMathOperator{\Proj}{Proj}
\DeclareMathOperator{\Span}{span}
\DeclareMathOperator{\Sgn}{sgn}
\DeclareMathOperator{\Area}{Area}
\DeclareMathOperator{\Volume}{Volume}

%
% A few miscellaneous things specific to this document
%
\newcommand{\crossop}[1]{\crossprod{#1}{}}

% R2 vector.
\newcommand{\VectorTwo}[2]{
\begin{bmatrix}
 {#1} \\
 {#2}
\end{bmatrix}
}

\newcommand{\VectorN}[1]{
\begin{bmatrix}
{#1}_1 \\
{#1}_2 \\
\vdots \\
{#1}_N \\
\end{bmatrix}
}

\newcommand{\DETuvij}[4]{
\begin{vmatrix}
 {#1}_{#3} & {#1}_{#4} \\
 {#2}_{#3} & {#2}_{#4}
\end{vmatrix}
}

\newcommand{\DETuvwijk}[6]{
\begin{vmatrix}
 {#1}_{#4} & {#1}_{#5} & {#1}_{#6} \\
 {#2}_{#4} & {#2}_{#5} & {#2}_{#6} \\
 {#3}_{#4} & {#3}_{#5} & {#3}_{#6}
\end{vmatrix}
}

\newcommand{\DETuvwxijkl}[8]{
\begin{vmatrix}
 {#1}_{#5} & {#1}_{#6} & {#1}_{#7} & {#1}_{#8} \\
 {#2}_{#5} & {#2}_{#6} & {#2}_{#7} & {#2}_{#8} \\
 {#3}_{#5} & {#3}_{#6} & {#3}_{#7} & {#3}_{#8} \\
 {#4}_{#5} & {#4}_{#6} & {#4}_{#7} & {#4}_{#8} \\
\end{vmatrix}
}

%\newcommand{\DETuvwxyijklm}[10]{
%\begin{vmatrix}
% {#1}_{#6} & {#1}_{#7} & {#1}_{#8} & {#1}_{#9} & {#1}_{#10} \\
% {#2}_{#6} & {#2}_{#7} & {#2}_{#8} & {#2}_{#9} & {#2}_{#10} \\
% {#3}_{#6} & {#3}_{#7} & {#3}_{#8} & {#3}_{#9} & {#3}_{#10} \\
% {#4}_{#6} & {#4}_{#7} & {#4}_{#8} & {#4}_{#9} & {#4}_{#10} \\
% {#5}_{#6} & {#5}_{#7} & {#5}_{#8} & {#5}_{#9} & {#5}_{#10}
%\end{vmatrix}
%}

% R3 vector.
\newcommand{\VectorThree}[3]{
\begin{bmatrix}
 {#1} \\
 {#2} \\
 {#3}
\end{bmatrix}
}


\newcommand{\gpgrade}[2] {{\left\langle{{#1}}\right\rangle}_{#2}}

%
% The real thing:
%

                             % The preamble begins here.
\title{ Notes on shear transformation. } % Declares the document's title.
\author{Peeter Joot}         % Declares the author's name.
%\date{}        % Deleting this command produces today's date.

\begin{document}             % End of preamble and beginning of text.

\maketitle{}

\section{}

GASC and GAFP both give examples of shear transformations of the following
form:

\[
F(a) = a + \alpha(a \cdot f) e.
\]

Geometric Algebra for Physicists uses this to compute the determinant without putting the operator in matrix form.  They end up stating that 

\begin{equation}\label{eqn:shearblade}
F(A) = A + \alpha e \wedge (f \cdot A).
\end{equation}

holds for any grade blade $A$.  For grade 1 that is true since

\begin{align*}
(a \cdot f) e 
&= \gpgrade{(a \cdot f) e \rangle}{1} \\
&= (a \cdot f) \wedge e.
\end{align*}

They demonstrate equation \ref{eqn:shearblade}
holds for the grade 2 case.  To me it seems
like an induction is required to make their statement for any grade.

Question: Is there some other principle that I didn't notice in my reading that allows assertion of 
equation \ref{eqn:shearblade}
for any grade blade without the induction.

\subsection{ Proof for any grade blade. }

For $A \in {\bigwedge}^r$

\begin{align*}
F(A) \wedge F(b)
&= \left(A + \alpha e \wedge (f \cdot A)\right) \wedge
   \left(b + \alpha(b \cdot f) e \right) \\
&= A \wedge b
 + \alpha 
\left(
e \wedge (f \cdot A) \wedge b
+ (b \cdot f) A \wedge e 
\right) \\
&= A \wedge b
 + \alpha e \wedge
\underbrace{
\left(
(f \cdot A) \wedge b
+ (-1)^r (b \cdot f) A 
\right)
}_{(*)} \\
&= A \wedge b + \alpha e \wedge (f \cdot (A \wedge b))
\end{align*}

QED.

Verification below of $(*) = f \cdot (A \wedge b)$ is required to complete the proof (can probably find that in one of the books or papers but it is derivable
easily enough).

Do I have a sign mixup here somewhere?  Now that I look again I see that GAFP
has the result in different order $(A \cdot f) \wedge e$, and I get negation
reconciling the two.

\subsection{ Dot product reduction of blade by one. }

\begin{align*}
f \cdot (A \wedge b)
&= \inv{2} \gpgrade{f (A \wedge b)}{r} \\
&= \inv{2} \gpgrade{f (A b + (-1)^r b A)}{r} \\
&= \inv{2} \gpgrade{(f A) b + (-1)^r f b A}{r} \\
&= \inv{2} \gpgrade{((-1)^r A f + 2 f \cdot A) b + (-1)^r f b A}{r} \\
&= (f \cdot A) \wedge b + \frac{(-1)^r}{2} \gpgrade{A f b + f b A}{r} \\
&= (f \cdot A) \wedge b + (-1)^r (f \cdot b) A
+ \frac{(-1)^r}{2} \gpgrade{A f \wedge b + f \wedge b A}{r} \\
\end{align*}

This last term, the symmetric product of a bivector with a blade is zero.
The grades $r-2, r+4, \ldots$ terms are symmetric, and the other grades
$r, r+4, \ldots$ are antisymmetric.

Thus we have

\begin{equation}
f \cdot (A \wedge b)
= (f \cdot A) \wedge b - (-1)^r (f \cdot b) A
\end{equation}

This generalizes the familiar vector reduction formula to higher grades.
Observe that for the vector case we need the most general definition
of the wedge product for the scalar-vector wedge product (grade $1-0$ part of the product).

\end{document}               % End of document.

%
% Copyright � 2012 Peeter Joot.  All Rights Reserved.
% Licenced as described in the file LICENSE under the root directory of this GIT repository.
%

% 
% 
%\documentclass{article}

%\usepackage{amsmath}
\usepackage{mathpazo}

%
% shorthand for bold symbols, convenient for vectors and matrices
%
\newcommand{\Ba}[0]{\mathbf{a}}
\newcommand{\Bb}[0]{\mathbf{b}}
\newcommand{\Bc}[0]{\mathbf{c}}
\newcommand{\Bd}[0]{\mathbf{d}}
\newcommand{\Be}[0]{\mathbf{e}}
\newcommand{\Bf}[0]{\mathbf{f}}
\newcommand{\Bg}[0]{\mathbf{g}}
\newcommand{\Bh}[0]{\mathbf{h}}
\newcommand{\Bi}[0]{\mathbf{i}}
\newcommand{\Bj}[0]{\mathbf{j}}
\newcommand{\Bk}[0]{\mathbf{k}}
\newcommand{\Bl}[0]{\mathbf{l}}
\newcommand{\Bm}[0]{\mathbf{m}}
\newcommand{\Bn}[0]{\mathbf{n}}
\newcommand{\Bo}[0]{\mathbf{o}}
\newcommand{\Bp}[0]{\mathbf{p}}
\newcommand{\Bq}[0]{\mathbf{q}}
\newcommand{\Br}[0]{\mathbf{r}}
\newcommand{\Bs}[0]{\mathbf{s}}
\newcommand{\Bt}[0]{\mathbf{t}}
\newcommand{\Bu}[0]{\mathbf{u}}
\newcommand{\Bv}[0]{\mathbf{v}}
\newcommand{\Bw}[0]{\mathbf{w}}
\newcommand{\Bx}[0]{\mathbf{x}}
\newcommand{\By}[0]{\mathbf{y}}
\newcommand{\Bz}[0]{\mathbf{z}}
\newcommand{\BA}[0]{\mathbf{A}}
\newcommand{\BB}[0]{\mathbf{B}}
\newcommand{\BC}[0]{\mathbf{C}}
\newcommand{\BD}[0]{\mathbf{D}}
\newcommand{\BE}[0]{\mathbf{E}}
\newcommand{\BF}[0]{\mathbf{F}}
\newcommand{\BG}[0]{\mathbf{G}}
\newcommand{\BH}[0]{\mathbf{H}}
\newcommand{\BI}[0]{\mathbf{I}}
\newcommand{\BJ}[0]{\mathbf{J}}
\newcommand{\BK}[0]{\mathbf{K}}
\newcommand{\BL}[0]{\mathbf{L}}
\newcommand{\BM}[0]{\mathbf{M}}
\newcommand{\BN}[0]{\mathbf{N}}
\newcommand{\BO}[0]{\mathbf{O}}
\newcommand{\BP}[0]{\mathbf{P}}
\newcommand{\BQ}[0]{\mathbf{Q}}
\newcommand{\BR}[0]{\mathbf{R}}
\newcommand{\BS}[0]{\mathbf{S}}
\newcommand{\BT}[0]{\mathbf{T}}
\newcommand{\BU}[0]{\mathbf{U}}
\newcommand{\BV}[0]{\mathbf{V}}
\newcommand{\BW}[0]{\mathbf{W}}
\newcommand{\BX}[0]{\mathbf{X}}
\newcommand{\BY}[0]{\mathbf{Y}}
\newcommand{\BZ}[0]{\mathbf{Z}}

\newcommand{\Bzero}[0]{\mathbf{0}}
\newcommand{\Btheta}[0]{\boldsymbol{\theta}}
\newcommand{\Btau}[0]{\boldsymbol{\tau}}
\newcommand{\Bomega}[0]{\boldsymbol{\omega}}

%
% shorthand for unit vectors
%
\newcommand{\acap}[0]{\hat{\Ba}}
\newcommand{\bcap}[0]{\hat{\Bb}}
\newcommand{\ccap}[0]{\hat{\Bc}}
\newcommand{\dcap}[0]{\hat{\Bd}}
\newcommand{\ecap}[0]{\hat{\Be}}
\newcommand{\fcap}[0]{\hat{\Bf}}
\newcommand{\gcap}[0]{\hat{\Bg}}
\newcommand{\hcap}[0]{\hat{\Bh}}
\newcommand{\icap}[0]{\hat{\Bi}}
\newcommand{\jcap}[0]{\hat{\Bj}}
\newcommand{\kcap}[0]{\hat{\Bk}}
\newcommand{\lcap}[0]{\hat{\Bl}}
\newcommand{\mcap}[0]{\hat{\Bm}}
\newcommand{\ncap}[0]{\hat{\Bn}}
\newcommand{\ocap}[0]{\hat{\Bo}}
\newcommand{\pcap}[0]{\hat{\Bp}}
\newcommand{\qcap}[0]{\hat{\Bq}}
\newcommand{\rcap}[0]{\hat{\Br}}
\newcommand{\scap}[0]{\hat{\Bs}}
\newcommand{\tcap}[0]{\hat{\Bt}}
\newcommand{\ucap}[0]{\hat{\Bu}}
\newcommand{\vcap}[0]{\hat{\Bv}}
\newcommand{\wcap}[0]{\hat{\Bw}}
\newcommand{\xcap}[0]{\hat{\Bx}}
\newcommand{\ycap}[0]{\hat{\By}}
\newcommand{\zcap}[0]{\hat{\Bz}}
\newcommand{\thetacap}[0]{\hat{\Btheta}}

%
% to write R^n and C^n in a distinguishable fashion.  Perhaps change this
% to the double lined characters upon figuring out how to do so.
%
\newcommand{\C}[1]{$\mathbb{C}^{#1}$}
\newcommand{\R}[1]{$\mathbb{R}^{#1}$}

%
% various generally useful helpers
%

% derivative of #1 wrt. #2:
\newcommand{\D}[2] {\frac {d#2} {d#1}}

\newcommand{\inv}[1]{\frac{1}{#1}}
\newcommand{\cross}[0]{\times}

\newcommand{\abs}[1]{\lvert{#1}\rvert}
\newcommand{\norm}[1]{\lVert{#1}\rVert}
\newcommand{\innerprod}[2]{\langle{#1}, {#2}\rangle}
\newcommand{\dotprod}[2]{{#1} \cdot {#2}}
\newcommand{\bdotprod}[2]{\left({#1} \cdot {#2}\right)}
\newcommand{\crossprod}[2]{{#1} \cross {#2}}
\newcommand{\tripleprod}[3]{\dotprod{\left(\crossprod{#1}{#2}\right)}{#3}}

\DeclareMathOperator{\Proj}{Proj}
\DeclareMathOperator{\Span}{span}
\DeclareMathOperator{\Sgn}{sgn}
\DeclareMathOperator{\Area}{Area}
\DeclareMathOperator{\Volume}{Volume}

%
% A few miscellaneous things specific to this document
%
\newcommand{\crossop}[1]{\crossprod{#1}{}}

% R2 vector.
\newcommand{\VectorTwo}[2]{
\begin{bmatrix}
 {#1} \\
 {#2}
\end{bmatrix}
}

\newcommand{\VectorN}[1]{
\begin{bmatrix}
{#1}_1 \\
{#1}_2 \\
\vdots \\
{#1}_N \\
\end{bmatrix}
}

\newcommand{\DETuvij}[4]{
\begin{vmatrix}
 {#1}_{#3} & {#1}_{#4} \\
 {#2}_{#3} & {#2}_{#4}
\end{vmatrix}
}

\newcommand{\DETuvwijk}[6]{
\begin{vmatrix}
 {#1}_{#4} & {#1}_{#5} & {#1}_{#6} \\
 {#2}_{#4} & {#2}_{#5} & {#2}_{#6} \\
 {#3}_{#4} & {#3}_{#5} & {#3}_{#6}
\end{vmatrix}
}

\newcommand{\DETuvwxijkl}[8]{
\begin{vmatrix}
 {#1}_{#5} & {#1}_{#6} & {#1}_{#7} & {#1}_{#8} \\
 {#2}_{#5} & {#2}_{#6} & {#2}_{#7} & {#2}_{#8} \\
 {#3}_{#5} & {#3}_{#6} & {#3}_{#7} & {#3}_{#8} \\
 {#4}_{#5} & {#4}_{#6} & {#4}_{#7} & {#4}_{#8} \\
\end{vmatrix}
}

%\newcommand{\DETuvwxyijklm}[10]{
%\begin{vmatrix}
% {#1}_{#6} & {#1}_{#7} & {#1}_{#8} & {#1}_{#9} & {#1}_{#10} \\
% {#2}_{#6} & {#2}_{#7} & {#2}_{#8} & {#2}_{#9} & {#2}_{#10} \\
% {#3}_{#6} & {#3}_{#7} & {#3}_{#8} & {#3}_{#9} & {#3}_{#10} \\
% {#4}_{#6} & {#4}_{#7} & {#4}_{#8} & {#4}_{#9} & {#4}_{#10} \\
% {#5}_{#6} & {#5}_{#7} & {#5}_{#8} & {#5}_{#9} & {#5}_{#10}
%\end{vmatrix}
%}

% R3 vector.
\newcommand{\VectorThree}[3]{
\begin{bmatrix}
 {#1} \\
 {#2} \\
 {#3}
\end{bmatrix}
}


%%<misc>
%
\newcommand{\Abs}[1]{{\left\lvert{#1}\right\rvert}}
\newcommand{\spacegrad}[0]{\boldsymbol{\nabla}}
\newcommand{\grad}[0]{\nabla}
\newcommand{\LL}[0]{\mathcal{L}}

% == \partial_{#1} {#2}
\newcommand{\PD}[2]{\frac{\partial {#2}}{\partial {#1}}}
% inline variant
\newcommand{\PDi}[2]{{\partial {#2}}/{\partial {#1}}}

\newcommand{\PDD}[3]{\frac{\partial^2 {#3}}{\partial {#1}\partial {#2}}}
%\newcommand{\PDd}[2]{\frac{\partial^2 {#2}}{{\partial{#1}}^2}}
\newcommand{\PDsq}[2]{\frac{\partial^2 {#2}}{(\partial {#1})^2}}

\newcommand{\Partial}[2]{\frac{\partial {#1}}{\partial {#2}}}
\DeclareMathOperator{\RejName}{Rej}
\newcommand{\Rej}[2]{\RejName_{#1}\left( {#2} \right)}
\newcommand{\Rm}[1]{\mathbb{R}^{#1}}
\newcommand{\Cm}[1]{\mathbb{C}^{#1}}
\newcommand{\conj}[0]{{*}}

%</misc>

% <grade selection>
%
\newcommand{\gpgrade}[2] {{\left\langle{{#1}}\right\rangle}_{#2}}

\newcommand{\gpgradezero}[1] {\gpgrade{#1}{}}
%\newcommand{\gpscalargrade}[1] {{\left\langle{{#1}}\right\rangle}}
%\newcommand{\gpgradezero}[1] {\gpgrade{#1}{0}}

%\newcommand{\gpgradeone}[1] {{\left\langle{{#1}}\right\rangle}_{1}}
\newcommand{\gpgradeone}[1] {\gpgrade{#1}{1}}

\newcommand{\gpgradetwo}[1] {\gpgrade{#1}{2}}
\newcommand{\gpgradethree}[1] {\gpgrade{#1}{3}}
\newcommand{\gpgradefour}[1] {\gpgrade{#1}{4}}
%
% </grade selection>



\newcommand{\adot}[0]{{\dot{a}}}
\newcommand{\bdot}[0]{{\dot{b}}}
% taken for centered dot:
%\newcommand{\cdot}[0]{{\dot{c}}}
%\newcommand{\ddot}[0]{{\dot{d}}}
\newcommand{\edot}[0]{{\dot{e}}}
\newcommand{\fdot}[0]{{\dot{f}}}
\newcommand{\gdot}[0]{{\dot{g}}}
\newcommand{\hdot}[0]{{\dot{h}}}
\newcommand{\idot}[0]{{\dot{i}}}
\newcommand{\jdot}[0]{{\dot{j}}}
\newcommand{\kdot}[0]{{\dot{k}}}
\newcommand{\ldot}[0]{{\dot{l}}}
\newcommand{\mdot}[0]{{\dot{m}}}
\newcommand{\ndot}[0]{{\dot{n}}}
%\newcommand{\odot}[0]{{\dot{o}}}
\newcommand{\pdot}[0]{{\dot{p}}}
\newcommand{\qdot}[0]{{\dot{q}}}
\newcommand{\rdot}[0]{{\dot{r}}}
\newcommand{\sdot}[0]{{\dot{s}}}
\newcommand{\tdot}[0]{{\dot{t}}}
\newcommand{\udot}[0]{{\dot{u}}}
\newcommand{\vdot}[0]{{\dot{v}}}
\newcommand{\wdot}[0]{{\dot{w}}}
\newcommand{\xdot}[0]{{\dot{x}}}
\newcommand{\ydot}[0]{{\dot{y}}}
\newcommand{\zdot}[0]{{\dot{z}}}
\newcommand{\addot}[0]{{\ddot{a}}}
\newcommand{\bddot}[0]{{\ddot{b}}}
\newcommand{\cddot}[0]{{\ddot{c}}}
%\newcommand{\dddot}[0]{{\ddot{d}}}
\newcommand{\eddot}[0]{{\ddot{e}}}
\newcommand{\fddot}[0]{{\ddot{f}}}
\newcommand{\gddot}[0]{{\ddot{g}}}
\newcommand{\hddot}[0]{{\ddot{h}}}
\newcommand{\iddot}[0]{{\ddot{i}}}
\newcommand{\jddot}[0]{{\ddot{j}}}
\newcommand{\kddot}[0]{{\ddot{k}}}
\newcommand{\lddot}[0]{{\ddot{l}}}
\newcommand{\mddot}[0]{{\ddot{m}}}
\newcommand{\nddot}[0]{{\ddot{n}}}
\newcommand{\oddot}[0]{{\ddot{o}}}
\newcommand{\pddot}[0]{{\ddot{p}}}
\newcommand{\qddot}[0]{{\ddot{q}}}
\newcommand{\rddot}[0]{{\ddot{r}}}
\newcommand{\sddot}[0]{{\ddot{s}}}
\newcommand{\tddot}[0]{{\ddot{t}}}
\newcommand{\uddot}[0]{{\ddot{u}}}
\newcommand{\vddot}[0]{{\ddot{v}}}
\newcommand{\wddot}[0]{{\ddot{w}}}
\newcommand{\xddot}[0]{{\ddot{x}}}
\newcommand{\yddot}[0]{{\ddot{y}}}
\newcommand{\zddot}[0]{{\ddot{z}}}

%<bold and dot greek symbols>
%

\newcommand{\Deltadot}[0]{{\dot{\Delta}}}
\newcommand{\Gammadot}[0]{{\dot{\Gamma}}}
\newcommand{\Lambdadot}[0]{{\dot{\Lambda}}}
\newcommand{\Omegadot}[0]{{\dot{\Omega}}}
\newcommand{\Phidot}[0]{{\dot{\Phi}}}
\newcommand{\Pidot}[0]{{\dot{\Pi}}}
\newcommand{\Psidot}[0]{{\dot{\Psi}}}
\newcommand{\Sigmadot}[0]{{\dot{\Sigma}}}
\newcommand{\Thetadot}[0]{{\dot{\Theta}}}
\newcommand{\Upsilondot}[0]{{\dot{\Upsilon}}}
\newcommand{\Xidot}[0]{{\dot{\Xi}}}
\newcommand{\alphadot}[0]{{\dot{\alpha}}}
\newcommand{\betadot}[0]{{\dot{\beta}}}
\newcommand{\chidot}[0]{{\dot{\chi}}}
\newcommand{\deltadot}[0]{{\dot{\delta}}}
\newcommand{\epsilondot}[0]{{\dot{\epsilon}}}
\newcommand{\etadot}[0]{{\dot{\eta}}}
\newcommand{\gammadot}[0]{{\dot{\gamma}}}
\newcommand{\kappadot}[0]{{\dot{\kappa}}}
\newcommand{\lambdadot}[0]{{\dot{\lambda}}}
\newcommand{\mudot}[0]{{\dot{\mu}}}
\newcommand{\nudot}[0]{{\dot{\nu}}}
\newcommand{\omegadot}[0]{{\dot{\omega}}}
\newcommand{\phidot}[0]{{\dot{\phi}}}
\newcommand{\pidot}[0]{{\dot{\pi}}}
\newcommand{\psidot}[0]{{\dot{\psi}}}
\newcommand{\rhodot}[0]{{\dot{\rho}}}
\newcommand{\sigmadot}[0]{{\dot{\sigma}}}
\newcommand{\taudot}[0]{{\dot{\tau}}}
\newcommand{\thetadot}[0]{{\dot{\theta}}}
\newcommand{\upsilondot}[0]{{\dot{\upsilon}}}
\newcommand{\varepsilondot}[0]{{\dot{\varepsilon}}}
\newcommand{\varphidot}[0]{{\dot{\varphi}}}
\newcommand{\varpidot}[0]{{\dot{\varpi}}}
\newcommand{\varrhodot}[0]{{\dot{\varrho}}}
\newcommand{\varsigmadot}[0]{{\dot{\varsigma}}}
\newcommand{\varthetadot}[0]{{\dot{\vartheta}}}
\newcommand{\xidot}[0]{{\dot{\xi}}}
\newcommand{\zetadot}[0]{{\dot{\zeta}}}

\newcommand{\Deltaddot}[0]{{\ddot{\Delta}}}
\newcommand{\Gammaddot}[0]{{\ddot{\Gamma}}}
\newcommand{\Lambdaddot}[0]{{\ddot{\Lambda}}}
\newcommand{\Omegaddot}[0]{{\ddot{\Omega}}}
\newcommand{\Phiddot}[0]{{\ddot{\Phi}}}
\newcommand{\Piddot}[0]{{\ddot{\Pi}}}
\newcommand{\Psiddot}[0]{{\ddot{\Psi}}}
\newcommand{\Sigmaddot}[0]{{\ddot{\Sigma}}}
\newcommand{\Thetaddot}[0]{{\ddot{\Theta}}}
\newcommand{\Upsilonddot}[0]{{\ddot{\Upsilon}}}
\newcommand{\Xiddot}[0]{{\ddot{\Xi}}}
\newcommand{\alphaddot}[0]{{\ddot{\alpha}}}
\newcommand{\betaddot}[0]{{\ddot{\beta}}}
\newcommand{\chiddot}[0]{{\ddot{\chi}}}
\newcommand{\deltaddot}[0]{{\ddot{\delta}}}
\newcommand{\epsilonddot}[0]{{\ddot{\epsilon}}}
\newcommand{\etaddot}[0]{{\ddot{\eta}}}
\newcommand{\gammaddot}[0]{{\ddot{\gamma}}}
\newcommand{\kappaddot}[0]{{\ddot{\kappa}}}
\newcommand{\lambdaddot}[0]{{\ddot{\lambda}}}
\newcommand{\muddot}[0]{{\ddot{\mu}}}
\newcommand{\nuddot}[0]{{\ddot{\nu}}}
\newcommand{\omegaddot}[0]{{\ddot{\omega}}}
\newcommand{\phiddot}[0]{{\ddot{\phi}}}
\newcommand{\piddot}[0]{{\ddot{\pi}}}
\newcommand{\psiddot}[0]{{\ddot{\psi}}}
\newcommand{\rhoddot}[0]{{\ddot{\rho}}}
\newcommand{\sigmaddot}[0]{{\ddot{\sigma}}}
\newcommand{\tauddot}[0]{{\ddot{\tau}}}
\newcommand{\thetaddot}[0]{{\ddot{\theta}}}
\newcommand{\upsilonddot}[0]{{\ddot{\upsilon}}}
\newcommand{\varepsilonddot}[0]{{\ddot{\varepsilon}}}
\newcommand{\varphiddot}[0]{{\ddot{\varphi}}}
\newcommand{\varpiddot}[0]{{\ddot{\varpi}}}
\newcommand{\varrhoddot}[0]{{\ddot{\varrho}}}
\newcommand{\varsigmaddot}[0]{{\ddot{\varsigma}}}
\newcommand{\varthetaddot}[0]{{\ddot{\vartheta}}}
\newcommand{\xiddot}[0]{{\ddot{\xi}}}
\newcommand{\zetaddot}[0]{{\ddot{\zeta}}}

\newcommand{\BDelta}[0]{\boldsymbol{\Delta}}
\newcommand{\BGamma}[0]{\boldsymbol{\Gamma}}
\newcommand{\BLambda}[0]{\boldsymbol{\Lambda}}
\newcommand{\BOmega}[0]{\boldsymbol{\Omega}}
\newcommand{\BPhi}[0]{\boldsymbol{\Phi}}
\newcommand{\BPi}[0]{\boldsymbol{\Pi}}
\newcommand{\BPsi}[0]{\boldsymbol{\Psi}}
\newcommand{\BSigma}[0]{\boldsymbol{\Sigma}}
\newcommand{\BTheta}[0]{\boldsymbol{\Theta}}
\newcommand{\BUpsilon}[0]{\boldsymbol{\Upsilon}}
\newcommand{\BXi}[0]{\boldsymbol{\Xi}}
\newcommand{\Balpha}[0]{\boldsymbol{\alpha}}
\newcommand{\Bbeta}[0]{\boldsymbol{\beta}}
\newcommand{\Bchi}[0]{\boldsymbol{\chi}}
\newcommand{\Bdelta}[0]{\boldsymbol{\delta}}
\newcommand{\Bepsilon}[0]{\boldsymbol{\epsilon}}
\newcommand{\Beta}[0]{\boldsymbol{\eta}}
\newcommand{\Bgamma}[0]{\boldsymbol{\gamma}}
\newcommand{\Bkappa}[0]{\boldsymbol{\kappa}}
\newcommand{\Blambda}[0]{\boldsymbol{\lambda}}
\newcommand{\Bmu}[0]{\boldsymbol{\mu}}
\newcommand{\Bnu}[0]{\boldsymbol{\nu}}
%\newcommand{\Bomega}[0]{\boldsymbol{\omega}}
\newcommand{\Bphi}[0]{\boldsymbol{\phi}}
\newcommand{\Bpi}[0]{\boldsymbol{\pi}}
\newcommand{\Bpsi}[0]{\boldsymbol{\psi}}
\newcommand{\Brho}[0]{\boldsymbol{\rho}}
\newcommand{\Bsigma}[0]{\boldsymbol{\sigma}}
%\newcommand{\Btau}[0]{\boldsymbol{\tau}}
%\newcommand{\Btheta}[0]{\boldsymbol{\theta}}
\newcommand{\Bupsilon}[0]{\boldsymbol{\upsilon}}
\newcommand{\Bvarepsilon}[0]{\boldsymbol{\varepsilon}}
\newcommand{\Bvarphi}[0]{\boldsymbol{\varphi}}
\newcommand{\Bvarpi}[0]{\boldsymbol{\varpi}}
\newcommand{\Bvarrho}[0]{\boldsymbol{\varrho}}
\newcommand{\Bvarsigma}[0]{\boldsymbol{\varsigma}}
\newcommand{\Bvartheta}[0]{\boldsymbol{\vartheta}}
\newcommand{\Bxi}[0]{\boldsymbol{\xi}}
\newcommand{\Bzeta}[0]{\boldsymbol{\zeta}}
%
%</bold and dot greek symbols>
%<infrequent>
%
%\newcommand{\AreaOp}[1]{\AName_{#1}}
%\newcommand{\Babs}[0]{\abs{\BB}}
%\newcommand{\Bcap}[0]{\hat{\BB}}
%\newcommand{\BrPrimeRej}[0]{\rcap(\rcap \wedge \Br')}
%\newcommand{\CA}[0]{\mathcal{A}}
%\newcommand{\Cos}[1]{\cos{\left({#1}\right)}}
%\newcommand{\Det}[1] {\abs{#1}}
%\newcommand{\Dsq}[2] {\frac {\partial^2 {#1}} {\partial {#2}^2}}
%\newcommand{\Exp}[1]{\exp{\left({#1}\right)}}
%\newcommand{\Norm}[1]{\left\lVert{#1}\right\rVert}
%\newcommand{\Sin}[1]{\sin{\left({#1}\right)}}
%\newcommand{\T}[0]{\text{T}}
%\newcommand{\VolumeOp}[1]{\VName_{#1}}
%\newcommand{\agrad}[0]{\Ba \cdot \nabla}
%\newcommand{\alphacap}[0]{\hat{\boldsymbol{\alpha}}}
%\newcommand{\Fcap}[0]{\hat{\BF}}
%\newcommand{\bithree}[0]{{\Bi}_3}
%\newcommand{\bxa}[0]{\Bx\Ba}
%\newcommand{\coordvec}[2]{
%\newcommand{\costheta}[0]{\acap \cdot \xcap}
%\newcommand{\ddt}[1]{\ddot{#1}}
%\newcommand{\ddu}[1] {\frac {d{#1}} {du}}
%\newcommand{\dsqxj}[2] {\frac {\partial^2 {#1}} {\partial {x_{#2}}^2}}
%\newcommand{\dtheta}[1]{\frac{d {#1}}{d \theta}}
%\newcommand{\dt}[1]{\dot{#1}}
%\newcommand{\dt}[1]{\frac{d {#1}}{dt}}
%\newcommand{\dxj}[2] {\frac {\partial {#1}} {\partial {x_{#2}}}}
%\newcommand{\halfPhi}[0]{\frac{\phi}{2}}
%\newcommand{\half}[0]{\inv{2}}
%\newcommand{\inv}[1]{\frac{1}{#1}}
%\newcommand{\laplacian}[0]{\nabla^2}
%\newcommand{\matrixoftx}[3]{
%\newcommand{\nrrp}[0]{\norm{\rcap \wedge \Br'}}
%\newcommand{\oiint}{\bigcirc \hspace{-1.4em} \int \hspace{-.8em} \int}
%\newcommand{\transpose}[1]{{#1}^{\text{T}}}
%\newcommand{\transpose}[1]{{{#1}^{\TextTranspose}}}
%\newcommand{\transpose}[1]{{{#1}^{\text{T}}}}
%\newcommand{\barA}[0]{\bar{A}}
%\newcommand{\qbar}[0]{\bar{q}}
%\newcommand{\qdotbar}[0]{\dot{\bar{q}}}
%
%</infrequent>





%\usepackage[bookmarks=true]{hyperref}

%\usepackage{color,cite,graphicx}
   % use colour in the document, put your citations as [1-4]
   % rather than [1,2,3,4] (it looks nicer, and the extended LaTeX2e
   % graphics package. 
%\usepackage{latexsym,amssymb,epsf} % do not remember if these are
   % needed, but their inclusion can not do any damage


\chapter{Composition of rotations exercise.  Two nineties}
\label{chap:twoNinetyRotations}
%\author{Peeter Joot \quad peeterjoot@protonmail.com}
\date{ Jan 17, 2009.  \(RCSfile: twoNinetyRotations.tex,v \) Last \(Revision: 1.8 \) \(Date: 2009/06/14 23:51:45 \) }

%\begin{document}

%\maketitle{}
%\tableofcontents
\section{Problem}

Rotate 90 about the z-axis, and then 90 about the new x-axis (problem from Alan M's book draft).

\section{Solution}

The z-axis rotation is 

\begin{equation}\label{eqn:twoNinetyRotations:20}
\begin{aligned}
R_{z,90}(\Bx) &= e^{-\Be_{12}} \Bx e^{\Be_{12}}
\end{aligned}
\end{equation}

and the rotation about the new x-axis (ie: in the old -1,3 plane) is

\begin{equation}\label{eqn:twoNinetyRotations:40}
\begin{aligned}
R_{x',90}(\Bx') &= e^{\Be_{13}} \Bx' e^{-\Be_{13}}
\end{aligned}
\end{equation}

Therefore the composite rotation is

\begin{equation}\label{eqn:twoNinetyRotations:60}
\begin{aligned}
R(\Bx) &= e^{\Be_{13}} e^{-\Be_{12}} \Bx e^{\Be_{12}} e^{-\Be_{13}}
\end{aligned}
\end{equation}

We want to expand the product

\begin{equation}\label{eqn:twoNinetyRotations:80}
\begin{aligned}
R 
&= e^{\Be_{13}} e^{-\Be_{12}} \\
&= \inv{2} (1 + \Be_{13}) (1 - \Be_{12}) \\
&= \inv{2} (\Be_1 - \Be_{3}) \Be_1 \Be_1 (\Be_1 - \Be_{2}) \\
&= \inv{2} (\Be_1 - \Be_{3}) \cdot (\Be_1 - \Be_{2}) +\inv{2} (\Be_1 - \Be_{3}) \wedge (\Be_1 - \Be_{2}) \\
&= \inv{2} + \frac{\sqrt{3}}{2} \frac{(\Be_1 - \Be_{3}) \wedge (\Be_1 - \Be_{2})}{\sqrt{3}} \\
\end{aligned}
\end{equation}

Letting \(i = ((\Be_1 - \Be_{2}) \wedge (\Be_1 - \Be_{3}))/\sqrt{3}\) we have

\begin{equation}\label{eqn:twoNinetyRotations:100}
\begin{aligned}
R 
&= \cos(\pi/3) - i\sin(\pi/3) \\
&= e^{-i \pi/3}
\end{aligned}
\end{equation}

So, the composite rotation will take vectors that lie in the \((\Be_1 - \Be_{2}) \wedge (\Be_1 - \Be_{3})\) plane, and rotate them by \(2\pi/3 = 120^\circ\).

In terms of a normal we can write the plane in its dual form 

\begin{equation}\label{eqn:twoNinetyRotations:120}
\begin{aligned}
i &= \tilde{I}(I i) = -I \Bn
\end{aligned}
\end{equation}

So the normal of the rotational plane is

\begin{equation}\label{eqn:twoNinetyRotations:140}
\begin{aligned}
\Bn 
&= \inv{\sqrt{3}} \Be_{123} \left( -\Be_{13} - \Be_{21} + \Be_{23} \right) \\
&= \frac{-1}{\sqrt{3}} \left( \Be_{2} + \Be_{3} + \Be_{1} \right) \\
\end{aligned}
\end{equation}

So we can also write this rotation as a rotation about the \(\Be_1 + \Be_2 + \Be_3\) axis (with a sense that I had have to think about to get right).

%\bibliographystyle{plainnat}
%\bibliography{myrefs}

%\end{document}

%
% Copyright � 2012 Peeter Joot.  All Rights Reserved.
% Licenced as described in the file LICENSE under the root directory of this GIT repository.
%

%
%
\chapter{Wedge product norm and GA bivector norm comparison}
\label{chap:wedgeNormVsGaNorm}
\date{Jan 2008 or so.  \(RCSfile: wedgeNormVsGaNorm.tex,v \) Last \(Revision: 1.9 \) \(Date: 2009/08/20 02:24:45 \) }

\section{Motivation}

Informative to look at the bivector formed by two unit perpendicular vectors in a plane.

\begin{equation}\label{eqn:wedgeNormVsGaNorm:20}
\ucap \wedge
\frac{
\left(\Bv - \bdotprod{\ucap}{\Bv} \ucap \right)
}
{
\left(\frac{1}{\norm{\Bu}^2} \sum_{i<j} {\left(D_{ij}^{\Bu \Bv}\right)}^2\right)^{1/2}
}
\end{equation}
\begin{equation}\label{eqn:wedgeNormVsGaNorm:40}
=
\frac{\Bu \wedge \Bv}
{
{\left(\sum_{i<j} {\left(D_{ij}^{\Bu \Bv}\right)}^2\right)}^{1/2}
}
\end{equation}
\begin{equation}\label{eqn:wedgeNormVsGaNorm:60}
=
\frac{\Bu \wedge \Bv}
{
\norm{\Bu \wedge \Bv}
}
\end{equation}

Here, \(\norm{ bivector }\) is taken with the most obvious definition.  For an arbitrary bivector we define:

\begin{equation}\label{eqn:wedgeNormVsGaNorm:80}
\norm{\sum_{i<j}g_{ij} \ecap_i \wedge \ecap_j}^2
=
\sum_{i<j}{g_{ij}}^2
\end{equation}

Let us see how this compares with the GA norm (square) of a bivector.
\begin{equation}\label{eqn:wedgeNormVsGaNorm:220}
\begin{aligned}
(\Bu \wedge \Bv)^2 &= (\Bu \Bv - \dotprod{\Bu}{\Bv})^2 \\
                   &= (\Bu \Bv - \dotprod{\Bu}{\Bv})(\Bu \Bv - \dotprod{\Bu}{\Bv}) \\
                   &= \Bu \Bv \Bu \Bv - 2 (\dotprod{\Bu}{\Bv}) \Bu \Bv + {(\dotprod{\Bu}{\Bv})}^2
\end{aligned}
\end{equation}

Since \(\dotprod{\Bu}{\Bv} = (\Bu \Bv + \Bv \Bu)/2\), then
\(\Bv \Bu = 2 \dotprod{\Bu}{\Bv} - \Bu \Bv\).

\begin{equation}\label{eqn:wedgeNormVsGaNorm:240}
\begin{aligned}
(\Bu \wedge \Bv)^2 &= \Bu ( 2 \dotprod{\Bu}{\Bv} - \Bu \Bv ) \Bv - 2 (\dotprod{\Bu}{\Bv})\Bu \Bv + {(\dotprod{\Bu}{\Bv})}^2) \\
                   &= - \Bu \Bu \Bv \Bv + {(\dotprod{\Bu}{\Bv})}^2 \\
                   &= {(\dotprod{\Bu}{\Bv})}^2 - \norm{\Bu}^2 \norm{\Bv}^2 \\
                   &= - \sum_{i<j} {\left(D_{ij}^{\Bu \Bv}\right)}^2 \\
                   &= - \norm{\Bu \wedge \Bv}^2
\end{aligned}
\end{equation}

So, we can also define the norm of a bivector in terms of its GA product:
\begin{equation}\label{eqn:wedgeNormVsGaNorm:100}
\norm{\Bu \wedge \Bv}^2 = -(\Bu \wedge \Bv)^2
\end{equation}

And, with implications to Rotors, we can see that the role of the GA product \(\BI = \ecap_i \ecap_j, i \neq j\) can also be filled by the wedge product of any two perpendicular unit vectors in a plane, or equivalently any unit bivector.  So, we have \(-1\) as the GA magnitude (square) of any unit bivector:

\begin{equation}\label{eqn:wedgeNormVsGaNorm:120}
\left(\frac{\Bu \wedge \Bv}
{
\norm{\Bu \wedge \Bv}
}\right)^2 = -1
\end{equation}

\subsection{Perpendicular to vector in direction of second expressed as GA product}

Starting with the component of the vector \(\Bv\) that is perpendicular to \(\Bu\), we have:

\begin{equation}\label{eqn:wedgeNormVsGaNorm:140}
\Bv' = \left(\Bv - \bdotprod{\ucap}{\Bv} \ucap \right)
= {\frac{1}{\norm{\Bu}^2}}\left(\norm{\Bu}^2 \Bv - \Bu \bdotprod{\Bu}{\Bv}  \right)
\end{equation}

Using the GA product \(\Bu^2 = \norm{\Bu}^2\), and \(\Bu \wedge \Bv = \Bu \Bv - \dotprod{\Bu}{\Bv}\),

\begin{equation}\label{eqn:wedgeNormVsGaNorm:160}
\Rightarrow
\Bv' = \frac{\Bu}{\Bu^2} \left(\Bu \Bv - \bdotprod{\Bu}{\Bv} \right)
=
\frac{1}{\Bu} \left(\Bu \wedge \Bv \right)
=
\ucap \left(\ucap \wedge \Bv \right)
\end{equation}

Thus we can write the decomposition of the vector \(\Bv\) into components parallel and perpendicular to \(\Bu\) as:

\begin{equation}\label{eqn:wedgeNormVsGaNorm:180}
\Bv = \ucap \left(\dotprod{\ucap}{\Bv}\right) + \ucap \left(\ucap \wedge \Bv \right)
\end{equation}
Or,
\begin{equation}\label{eqn:wedgeNormVsGaNorm:200}
\Bv = \frac{1}{\Bu} \left(\dotprod{\Bu}{\Bv}\right) + \frac{1}{\Bu} \left(\Bu \wedge \Bv \right)
\end{equation}

Without the GA product formulation we did not have a way without messy determinant sums to formulate the perpendicular component of this vector.

%\documentclass{article}      % Specifies the document class

%\usepackage{amsmath}
\usepackage{mathpazo}

%
% shorthand for bold symbols, convenient for vectors and matrices
%
\newcommand{\Ba}[0]{\mathbf{a}}
\newcommand{\Bb}[0]{\mathbf{b}}
\newcommand{\Bc}[0]{\mathbf{c}}
\newcommand{\Bd}[0]{\mathbf{d}}
\newcommand{\Be}[0]{\mathbf{e}}
\newcommand{\Bf}[0]{\mathbf{f}}
\newcommand{\Bg}[0]{\mathbf{g}}
\newcommand{\Bh}[0]{\mathbf{h}}
\newcommand{\Bi}[0]{\mathbf{i}}
\newcommand{\Bj}[0]{\mathbf{j}}
\newcommand{\Bk}[0]{\mathbf{k}}
\newcommand{\Bl}[0]{\mathbf{l}}
\newcommand{\Bm}[0]{\mathbf{m}}
\newcommand{\Bn}[0]{\mathbf{n}}
\newcommand{\Bo}[0]{\mathbf{o}}
\newcommand{\Bp}[0]{\mathbf{p}}
\newcommand{\Bq}[0]{\mathbf{q}}
\newcommand{\Br}[0]{\mathbf{r}}
\newcommand{\Bs}[0]{\mathbf{s}}
\newcommand{\Bt}[0]{\mathbf{t}}
\newcommand{\Bu}[0]{\mathbf{u}}
\newcommand{\Bv}[0]{\mathbf{v}}
\newcommand{\Bw}[0]{\mathbf{w}}
\newcommand{\Bx}[0]{\mathbf{x}}
\newcommand{\By}[0]{\mathbf{y}}
\newcommand{\Bz}[0]{\mathbf{z}}
\newcommand{\BA}[0]{\mathbf{A}}
\newcommand{\BB}[0]{\mathbf{B}}
\newcommand{\BC}[0]{\mathbf{C}}
\newcommand{\BD}[0]{\mathbf{D}}
\newcommand{\BE}[0]{\mathbf{E}}
\newcommand{\BF}[0]{\mathbf{F}}
\newcommand{\BG}[0]{\mathbf{G}}
\newcommand{\BH}[0]{\mathbf{H}}
\newcommand{\BI}[0]{\mathbf{I}}
\newcommand{\BJ}[0]{\mathbf{J}}
\newcommand{\BK}[0]{\mathbf{K}}
\newcommand{\BL}[0]{\mathbf{L}}
\newcommand{\BM}[0]{\mathbf{M}}
\newcommand{\BN}[0]{\mathbf{N}}
\newcommand{\BO}[0]{\mathbf{O}}
\newcommand{\BP}[0]{\mathbf{P}}
\newcommand{\BQ}[0]{\mathbf{Q}}
\newcommand{\BR}[0]{\mathbf{R}}
\newcommand{\BS}[0]{\mathbf{S}}
\newcommand{\BT}[0]{\mathbf{T}}
\newcommand{\BU}[0]{\mathbf{U}}
\newcommand{\BV}[0]{\mathbf{V}}
\newcommand{\BW}[0]{\mathbf{W}}
\newcommand{\BX}[0]{\mathbf{X}}
\newcommand{\BY}[0]{\mathbf{Y}}
\newcommand{\BZ}[0]{\mathbf{Z}}

\newcommand{\Bzero}[0]{\mathbf{0}}
\newcommand{\Btheta}[0]{\boldsymbol{\theta}}
\newcommand{\Btau}[0]{\boldsymbol{\tau}}
\newcommand{\Bomega}[0]{\boldsymbol{\omega}}

%
% shorthand for unit vectors
%
\newcommand{\acap}[0]{\hat{\Ba}}
\newcommand{\bcap}[0]{\hat{\Bb}}
\newcommand{\ccap}[0]{\hat{\Bc}}
\newcommand{\dcap}[0]{\hat{\Bd}}
\newcommand{\ecap}[0]{\hat{\Be}}
\newcommand{\fcap}[0]{\hat{\Bf}}
\newcommand{\gcap}[0]{\hat{\Bg}}
\newcommand{\hcap}[0]{\hat{\Bh}}
\newcommand{\icap}[0]{\hat{\Bi}}
\newcommand{\jcap}[0]{\hat{\Bj}}
\newcommand{\kcap}[0]{\hat{\Bk}}
\newcommand{\lcap}[0]{\hat{\Bl}}
\newcommand{\mcap}[0]{\hat{\Bm}}
\newcommand{\ncap}[0]{\hat{\Bn}}
\newcommand{\ocap}[0]{\hat{\Bo}}
\newcommand{\pcap}[0]{\hat{\Bp}}
\newcommand{\qcap}[0]{\hat{\Bq}}
\newcommand{\rcap}[0]{\hat{\Br}}
\newcommand{\scap}[0]{\hat{\Bs}}
\newcommand{\tcap}[0]{\hat{\Bt}}
\newcommand{\ucap}[0]{\hat{\Bu}}
\newcommand{\vcap}[0]{\hat{\Bv}}
\newcommand{\wcap}[0]{\hat{\Bw}}
\newcommand{\xcap}[0]{\hat{\Bx}}
\newcommand{\ycap}[0]{\hat{\By}}
\newcommand{\zcap}[0]{\hat{\Bz}}
\newcommand{\thetacap}[0]{\hat{\Btheta}}

%
% to write R^n and C^n in a distinguishable fashion.  Perhaps change this
% to the double lined characters upon figuring out how to do so.
%
\newcommand{\C}[1]{$\mathbb{C}^{#1}$}
\newcommand{\R}[1]{$\mathbb{R}^{#1}$}

%
% various generally useful helpers
%

% derivative of #1 wrt. #2:
\newcommand{\D}[2] {\frac {d#2} {d#1}}

\newcommand{\inv}[1]{\frac{1}{#1}}
\newcommand{\cross}[0]{\times}

\newcommand{\abs}[1]{\lvert{#1}\rvert}
\newcommand{\norm}[1]{\lVert{#1}\rVert}
\newcommand{\innerprod}[2]{\langle{#1}, {#2}\rangle}
\newcommand{\dotprod}[2]{{#1} \cdot {#2}}
\newcommand{\bdotprod}[2]{\left({#1} \cdot {#2}\right)}
\newcommand{\crossprod}[2]{{#1} \cross {#2}}
\newcommand{\tripleprod}[3]{\dotprod{\left(\crossprod{#1}{#2}\right)}{#3}}

\DeclareMathOperator{\Proj}{Proj}
\DeclareMathOperator{\Span}{span}
\DeclareMathOperator{\Sgn}{sgn}
\DeclareMathOperator{\Area}{Area}
\DeclareMathOperator{\Volume}{Volume}

%
% A few miscellaneous things specific to this document
%
\newcommand{\crossop}[1]{\crossprod{#1}{}}

% R2 vector.
\newcommand{\VectorTwo}[2]{
\begin{bmatrix}
 {#1} \\
 {#2}
\end{bmatrix}
}

\newcommand{\VectorN}[1]{
\begin{bmatrix}
{#1}_1 \\
{#1}_2 \\
\vdots \\
{#1}_N \\
\end{bmatrix}
}

\newcommand{\DETuvij}[4]{
\begin{vmatrix}
 {#1}_{#3} & {#1}_{#4} \\
 {#2}_{#3} & {#2}_{#4}
\end{vmatrix}
}

\newcommand{\DETuvwijk}[6]{
\begin{vmatrix}
 {#1}_{#4} & {#1}_{#5} & {#1}_{#6} \\
 {#2}_{#4} & {#2}_{#5} & {#2}_{#6} \\
 {#3}_{#4} & {#3}_{#5} & {#3}_{#6}
\end{vmatrix}
}

\newcommand{\DETuvwxijkl}[8]{
\begin{vmatrix}
 {#1}_{#5} & {#1}_{#6} & {#1}_{#7} & {#1}_{#8} \\
 {#2}_{#5} & {#2}_{#6} & {#2}_{#7} & {#2}_{#8} \\
 {#3}_{#5} & {#3}_{#6} & {#3}_{#7} & {#3}_{#8} \\
 {#4}_{#5} & {#4}_{#6} & {#4}_{#7} & {#4}_{#8} \\
\end{vmatrix}
}

%\newcommand{\DETuvwxyijklm}[10]{
%\begin{vmatrix}
% {#1}_{#6} & {#1}_{#7} & {#1}_{#8} & {#1}_{#9} & {#1}_{#10} \\
% {#2}_{#6} & {#2}_{#7} & {#2}_{#8} & {#2}_{#9} & {#2}_{#10} \\
% {#3}_{#6} & {#3}_{#7} & {#3}_{#8} & {#3}_{#9} & {#3}_{#10} \\
% {#4}_{#6} & {#4}_{#7} & {#4}_{#8} & {#4}_{#9} & {#4}_{#10} \\
% {#5}_{#6} & {#5}_{#7} & {#5}_{#8} & {#5}_{#9} & {#5}_{#10}
%\end{vmatrix}
%}

% R3 vector.
\newcommand{\VectorThree}[3]{
\begin{bmatrix}
 {#1} \\
 {#2} \\
 {#3}
\end{bmatrix}
}



%
% The real thing:
%

\chapter{Geometric Algebra: Signs of electromagnetic field tensor components? }
%\author{Peeter Joot \quad peeter.joot@gmail.com}         % Declares the author's name.
%\date{ May 30, 2008.  Last Revision: $Date: 2009/06/04 03:00:07 $ }

%\begin{document}             % End of preamble and beginning of text.

%\maketitle{}

\section{}

Here's a question that may look like an E \& M question, but is really just a geometric algebra question.  In particular, I've got a sign off by 1 somewhere I think and I wonder if somebody can spot it.

Doran/Lasenby (Geometric Algebra for Physicists) writes for the electromagnetic field:

\[
F = \mathbf{E} + I\mathbf{B}
\]

\[
\mathbf{E} = \sum \sigma_i E_i
\]

\[
\mathbf{B} = \sum \sigma_i B_i
\]

\[
\sigma_i = \gamma_i \gamma_0
\]

\[
I = \gamma_0 \gamma_1 \gamma_2 \gamma_3
\]

Here, the $\{\gamma_\mu\}$ vectors are an ``orthonormal'' basis, with respect to a $(+,-,-,-)$ metric dot product.

They point out that the coordinates of the bivector F can be calculated by taking dot products with the reciprocal frame vectors:

\[
F^{\mu\nu} = (\gamma^\nu \wedge \gamma^\mu) \cdot F
\]

Where the reciprocal frame vectors $\{\gamma^i\}$ are those vectors defined by:

\[
\gamma^\mu \cdot \gamma_\nu = \delta_{\mu\nu}
\]

($\gamma^0 = \gamma_0$, and $\gamma^i = -\gamma_i$).

Expressed as a matrix this bivector coordinates are the tensor:

\[
F^{\mu\nu} =
\begin{bmatrix}
0 & -E_1 & -E_2 & -E_3 \\
E_1 & 0 & -B_3 & B_2 \\
E_2 & B_3 & 0 & -B_1 \\
E_3 & -B_2 & B_1 & 0 \\
\end{bmatrix}
\]

They don't actually do anything with this tensor in the text, since they operate on the bivector form directly.  It is essentially written out for in matrix form for comparison to other relativistic electrodynamics texts.

If I try calculating this I get different signs for the $B_i$ terms in the matrix above.  The 
calculation of the $E$ components is straightforward and I get the same answer.  $\mathbf{E}$ explicitly is:

\[
\mathbf{E} = E_1 \gamma_{10} + E_2 \gamma_{20} + E_3 \gamma_{30}
\]

Calculation of the $\nu = 0$ terms is:

\begin{align*}
F^{\mu 0} 
&= (\gamma^0 \wedge \gamma^\mu) \cdot F \\
&= (\gamma_0 \wedge (-\gamma_\mu)) \cdot \sum E_i \gamma_{i0} \\
&= -E_\mu (\gamma_0 \wedge \gamma_\mu) \cdot (\gamma_\mu \wedge \gamma_{0}) \\
&= -E_\mu \gamma_0 (\gamma_\mu \cdot (\gamma_\mu \wedge \gamma_{0})) \\
&= -E_\mu \gamma_0 ( \underbrace{\gamma_\mu \cdot \gamma_\mu}_{=-1} \gamma_{0} - \underbrace{\gamma_\mu \cdot \gamma_{0}}_{=0} \gamma_\mu) \\
&= E_\mu \gamma_0 \cdot \gamma_0 \\
&= E_\mu \\
\end{align*}

This is consistent with column zero of their matrix of tensor components above.

For the $B$ components, first I expanded out $I\mathbf{B}$ explicitly:

\begin{align*}
I\mathbf{B} 
&= \sum \gamma_0 \gamma_1 \gamma_2 \gamma_3 \gamma_i \gamma_0 B_i \\
&= \sum \gamma_1 \gamma_2 \gamma_3 \gamma_i B_i \\
&= B_1 \gamma_{23} + B_2 \gamma_{31} + B_3 \gamma_{12}
\end{align*}

This calculation is easier for just one pair of index, and for example, calculating the $12$ component I get:

\begin{align*}
F^{12} 
&= (\gamma^2 \wedge \gamma^1) \cdot ( \gamma_1 \wedge \gamma_2 ) B_3 \\
&= \gamma^2 \cdot (\gamma^1 \cdot ( \gamma_1 \wedge \gamma_2 )) B_3 \\
&= \gamma^2 \cdot ( \gamma^1 \cdot \gamma_1 \gamma_2 - \gamma^1 \cdot \gamma_2 \gamma_1 ) B_3 \\
&= \gamma^2 \cdot ( -\gamma_1 \cdot \gamma_1 \gamma_2 ) B_3 \\
&= \gamma^2 \cdot \gamma_2 B_3 \\
&= - \gamma_2 \cdot \gamma_2 B_3 \\
&= - (-1) B_3 \\
&= B_3 \\
\end{align*}

Observe that the sign is opposite for this compared to what's in the matrix above.  I don't see a mistake in my calculation, but this isn't listed in the errata even after two editions
so I'm assuming I have one hiding in there somewhere.

%\end{document}               % End of document.

\documentclass{article}

\usepackage{amsmath}
\usepackage{mathpazo}

%
% shorthand for bold symbols, convenient for vectors and matrices
%
\newcommand{\Ba}[0]{\mathbf{a}}
\newcommand{\Bb}[0]{\mathbf{b}}
\newcommand{\Bc}[0]{\mathbf{c}}
\newcommand{\Bd}[0]{\mathbf{d}}
\newcommand{\Be}[0]{\mathbf{e}}
\newcommand{\Bf}[0]{\mathbf{f}}
\newcommand{\Bg}[0]{\mathbf{g}}
\newcommand{\Bh}[0]{\mathbf{h}}
\newcommand{\Bi}[0]{\mathbf{i}}
\newcommand{\Bj}[0]{\mathbf{j}}
\newcommand{\Bk}[0]{\mathbf{k}}
\newcommand{\Bl}[0]{\mathbf{l}}
\newcommand{\Bm}[0]{\mathbf{m}}
\newcommand{\Bn}[0]{\mathbf{n}}
\newcommand{\Bo}[0]{\mathbf{o}}
\newcommand{\Bp}[0]{\mathbf{p}}
\newcommand{\Bq}[0]{\mathbf{q}}
\newcommand{\Br}[0]{\mathbf{r}}
\newcommand{\Bs}[0]{\mathbf{s}}
\newcommand{\Bt}[0]{\mathbf{t}}
\newcommand{\Bu}[0]{\mathbf{u}}
\newcommand{\Bv}[0]{\mathbf{v}}
\newcommand{\Bw}[0]{\mathbf{w}}
\newcommand{\Bx}[0]{\mathbf{x}}
\newcommand{\By}[0]{\mathbf{y}}
\newcommand{\Bz}[0]{\mathbf{z}}
\newcommand{\BA}[0]{\mathbf{A}}
\newcommand{\BB}[0]{\mathbf{B}}
\newcommand{\BC}[0]{\mathbf{C}}
\newcommand{\BD}[0]{\mathbf{D}}
\newcommand{\BE}[0]{\mathbf{E}}
\newcommand{\BF}[0]{\mathbf{F}}
\newcommand{\BG}[0]{\mathbf{G}}
\newcommand{\BH}[0]{\mathbf{H}}
\newcommand{\BI}[0]{\mathbf{I}}
\newcommand{\BJ}[0]{\mathbf{J}}
\newcommand{\BK}[0]{\mathbf{K}}
\newcommand{\BL}[0]{\mathbf{L}}
\newcommand{\BM}[0]{\mathbf{M}}
\newcommand{\BN}[0]{\mathbf{N}}
\newcommand{\BO}[0]{\mathbf{O}}
\newcommand{\BP}[0]{\mathbf{P}}
\newcommand{\BQ}[0]{\mathbf{Q}}
\newcommand{\BR}[0]{\mathbf{R}}
\newcommand{\BS}[0]{\mathbf{S}}
\newcommand{\BT}[0]{\mathbf{T}}
\newcommand{\BU}[0]{\mathbf{U}}
\newcommand{\BV}[0]{\mathbf{V}}
\newcommand{\BW}[0]{\mathbf{W}}
\newcommand{\BX}[0]{\mathbf{X}}
\newcommand{\BY}[0]{\mathbf{Y}}
\newcommand{\BZ}[0]{\mathbf{Z}}

\newcommand{\Bzero}[0]{\mathbf{0}}
\newcommand{\Btheta}[0]{\boldsymbol{\theta}}
\newcommand{\Btau}[0]{\boldsymbol{\tau}}
\newcommand{\Bomega}[0]{\boldsymbol{\omega}}

%
% shorthand for unit vectors
%
\newcommand{\acap}[0]{\hat{\Ba}}
\newcommand{\bcap}[0]{\hat{\Bb}}
\newcommand{\ccap}[0]{\hat{\Bc}}
\newcommand{\dcap}[0]{\hat{\Bd}}
\newcommand{\ecap}[0]{\hat{\Be}}
\newcommand{\fcap}[0]{\hat{\Bf}}
\newcommand{\gcap}[0]{\hat{\Bg}}
\newcommand{\hcap}[0]{\hat{\Bh}}
\newcommand{\icap}[0]{\hat{\Bi}}
\newcommand{\jcap}[0]{\hat{\Bj}}
\newcommand{\kcap}[0]{\hat{\Bk}}
\newcommand{\lcap}[0]{\hat{\Bl}}
\newcommand{\mcap}[0]{\hat{\Bm}}
\newcommand{\ncap}[0]{\hat{\Bn}}
\newcommand{\ocap}[0]{\hat{\Bo}}
\newcommand{\pcap}[0]{\hat{\Bp}}
\newcommand{\qcap}[0]{\hat{\Bq}}
\newcommand{\rcap}[0]{\hat{\Br}}
\newcommand{\scap}[0]{\hat{\Bs}}
\newcommand{\tcap}[0]{\hat{\Bt}}
\newcommand{\ucap}[0]{\hat{\Bu}}
\newcommand{\vcap}[0]{\hat{\Bv}}
\newcommand{\wcap}[0]{\hat{\Bw}}
\newcommand{\xcap}[0]{\hat{\Bx}}
\newcommand{\ycap}[0]{\hat{\By}}
\newcommand{\zcap}[0]{\hat{\Bz}}
\newcommand{\thetacap}[0]{\hat{\Btheta}}

%
% to write R^n and C^n in a distinguishable fashion.  Perhaps change this
% to the double lined characters upon figuring out how to do so.
%
\newcommand{\C}[1]{$\mathbb{C}^{#1}$}
\newcommand{\R}[1]{$\mathbb{R}^{#1}$}

%
% various generally useful helpers
%

% derivative of #1 wrt. #2:
\newcommand{\D}[2] {\frac {d#2} {d#1}}

\newcommand{\inv}[1]{\frac{1}{#1}}
\newcommand{\cross}[0]{\times}

\newcommand{\abs}[1]{\lvert{#1}\rvert}
\newcommand{\norm}[1]{\lVert{#1}\rVert}
\newcommand{\innerprod}[2]{\langle{#1}, {#2}\rangle}
\newcommand{\dotprod}[2]{{#1} \cdot {#2}}
\newcommand{\bdotprod}[2]{\left({#1} \cdot {#2}\right)}
\newcommand{\crossprod}[2]{{#1} \cross {#2}}
\newcommand{\tripleprod}[3]{\dotprod{\left(\crossprod{#1}{#2}\right)}{#3}}

\DeclareMathOperator{\Proj}{Proj}
\DeclareMathOperator{\Span}{span}
\DeclareMathOperator{\Sgn}{sgn}
\DeclareMathOperator{\Area}{Area}
\DeclareMathOperator{\Volume}{Volume}

%
% A few miscellaneous things specific to this document
%
\newcommand{\crossop}[1]{\crossprod{#1}{}}

% R2 vector.
\newcommand{\VectorTwo}[2]{
\begin{bmatrix}
 {#1} \\
 {#2}
\end{bmatrix}
}

\newcommand{\VectorN}[1]{
\begin{bmatrix}
{#1}_1 \\
{#1}_2 \\
\vdots \\
{#1}_N \\
\end{bmatrix}
}

\newcommand{\DETuvij}[4]{
\begin{vmatrix}
 {#1}_{#3} & {#1}_{#4} \\
 {#2}_{#3} & {#2}_{#4}
\end{vmatrix}
}

\newcommand{\DETuvwijk}[6]{
\begin{vmatrix}
 {#1}_{#4} & {#1}_{#5} & {#1}_{#6} \\
 {#2}_{#4} & {#2}_{#5} & {#2}_{#6} \\
 {#3}_{#4} & {#3}_{#5} & {#3}_{#6}
\end{vmatrix}
}

\newcommand{\DETuvwxijkl}[8]{
\begin{vmatrix}
 {#1}_{#5} & {#1}_{#6} & {#1}_{#7} & {#1}_{#8} \\
 {#2}_{#5} & {#2}_{#6} & {#2}_{#7} & {#2}_{#8} \\
 {#3}_{#5} & {#3}_{#6} & {#3}_{#7} & {#3}_{#8} \\
 {#4}_{#5} & {#4}_{#6} & {#4}_{#7} & {#4}_{#8} \\
\end{vmatrix}
}

%\newcommand{\DETuvwxyijklm}[10]{
%\begin{vmatrix}
% {#1}_{#6} & {#1}_{#7} & {#1}_{#8} & {#1}_{#9} & {#1}_{#10} \\
% {#2}_{#6} & {#2}_{#7} & {#2}_{#8} & {#2}_{#9} & {#2}_{#10} \\
% {#3}_{#6} & {#3}_{#7} & {#3}_{#8} & {#3}_{#9} & {#3}_{#10} \\
% {#4}_{#6} & {#4}_{#7} & {#4}_{#8} & {#4}_{#9} & {#4}_{#10} \\
% {#5}_{#6} & {#5}_{#7} & {#5}_{#8} & {#5}_{#9} & {#5}_{#10}
%\end{vmatrix}
%}

% R3 vector.
\newcommand{\VectorThree}[3]{
\begin{bmatrix}
 {#1} \\
 {#2} \\
 {#3}
\end{bmatrix}
}


%<misc>
%
\newcommand{\Abs}[1]{{\left\lvert{#1}\right\rvert}}
\newcommand{\spacegrad}[0]{\boldsymbol{\nabla}}
\newcommand{\grad}[0]{\nabla}
\newcommand{\LL}[0]{\mathcal{L}}

% == \partial_{#1} {#2}
\newcommand{\PD}[2]{\frac{\partial {#2}}{\partial {#1}}}
% inline variant
\newcommand{\PDi}[2]{{\partial {#2}}/{\partial {#1}}}

\newcommand{\PDD}[3]{\frac{\partial^2 {#3}}{\partial {#1}\partial {#2}}}
%\newcommand{\PDd}[2]{\frac{\partial^2 {#2}}{{\partial{#1}}^2}}
\newcommand{\PDsq}[2]{\frac{\partial^2 {#2}}{(\partial {#1})^2}}

\newcommand{\Partial}[2]{\frac{\partial {#1}}{\partial {#2}}}
\DeclareMathOperator{\RejName}{Rej}
\newcommand{\Rej}[2]{\RejName_{#1}\left( {#2} \right)}
\newcommand{\Rm}[1]{\mathbb{R}^{#1}}
\newcommand{\Cm}[1]{\mathbb{C}^{#1}}
\newcommand{\conj}[0]{{*}}

%</misc>

% <grade selection>
%
\newcommand{\gpgrade}[2] {{\left\langle{{#1}}\right\rangle}_{#2}}

\newcommand{\gpgradezero}[1] {\gpgrade{#1}{}}
%\newcommand{\gpscalargrade}[1] {{\left\langle{{#1}}\right\rangle}}
%\newcommand{\gpgradezero}[1] {\gpgrade{#1}{0}}

%\newcommand{\gpgradeone}[1] {{\left\langle{{#1}}\right\rangle}_{1}}
\newcommand{\gpgradeone}[1] {\gpgrade{#1}{1}}

\newcommand{\gpgradetwo}[1] {\gpgrade{#1}{2}}
\newcommand{\gpgradethree}[1] {\gpgrade{#1}{3}}
\newcommand{\gpgradefour}[1] {\gpgrade{#1}{4}}
%
% </grade selection>



\newcommand{\adot}[0]{{\dot{a}}}
\newcommand{\bdot}[0]{{\dot{b}}}
% taken for centered dot:
%\newcommand{\cdot}[0]{{\dot{c}}}
%\newcommand{\ddot}[0]{{\dot{d}}}
\newcommand{\edot}[0]{{\dot{e}}}
\newcommand{\fdot}[0]{{\dot{f}}}
\newcommand{\gdot}[0]{{\dot{g}}}
\newcommand{\hdot}[0]{{\dot{h}}}
\newcommand{\idot}[0]{{\dot{i}}}
\newcommand{\jdot}[0]{{\dot{j}}}
\newcommand{\kdot}[0]{{\dot{k}}}
\newcommand{\ldot}[0]{{\dot{l}}}
\newcommand{\mdot}[0]{{\dot{m}}}
\newcommand{\ndot}[0]{{\dot{n}}}
%\newcommand{\odot}[0]{{\dot{o}}}
\newcommand{\pdot}[0]{{\dot{p}}}
\newcommand{\qdot}[0]{{\dot{q}}}
\newcommand{\rdot}[0]{{\dot{r}}}
\newcommand{\sdot}[0]{{\dot{s}}}
\newcommand{\tdot}[0]{{\dot{t}}}
\newcommand{\udot}[0]{{\dot{u}}}
\newcommand{\vdot}[0]{{\dot{v}}}
\newcommand{\wdot}[0]{{\dot{w}}}
\newcommand{\xdot}[0]{{\dot{x}}}
\newcommand{\ydot}[0]{{\dot{y}}}
\newcommand{\zdot}[0]{{\dot{z}}}
\newcommand{\addot}[0]{{\ddot{a}}}
\newcommand{\bddot}[0]{{\ddot{b}}}
\newcommand{\cddot}[0]{{\ddot{c}}}
%\newcommand{\dddot}[0]{{\ddot{d}}}
\newcommand{\eddot}[0]{{\ddot{e}}}
\newcommand{\fddot}[0]{{\ddot{f}}}
\newcommand{\gddot}[0]{{\ddot{g}}}
\newcommand{\hddot}[0]{{\ddot{h}}}
\newcommand{\iddot}[0]{{\ddot{i}}}
\newcommand{\jddot}[0]{{\ddot{j}}}
\newcommand{\kddot}[0]{{\ddot{k}}}
\newcommand{\lddot}[0]{{\ddot{l}}}
\newcommand{\mddot}[0]{{\ddot{m}}}
\newcommand{\nddot}[0]{{\ddot{n}}}
\newcommand{\oddot}[0]{{\ddot{o}}}
\newcommand{\pddot}[0]{{\ddot{p}}}
\newcommand{\qddot}[0]{{\ddot{q}}}
\newcommand{\rddot}[0]{{\ddot{r}}}
\newcommand{\sddot}[0]{{\ddot{s}}}
\newcommand{\tddot}[0]{{\ddot{t}}}
\newcommand{\uddot}[0]{{\ddot{u}}}
\newcommand{\vddot}[0]{{\ddot{v}}}
\newcommand{\wddot}[0]{{\ddot{w}}}
\newcommand{\xddot}[0]{{\ddot{x}}}
\newcommand{\yddot}[0]{{\ddot{y}}}
\newcommand{\zddot}[0]{{\ddot{z}}}

%<bold and dot greek symbols>
%

\newcommand{\Deltadot}[0]{{\dot{\Delta}}}
\newcommand{\Gammadot}[0]{{\dot{\Gamma}}}
\newcommand{\Lambdadot}[0]{{\dot{\Lambda}}}
\newcommand{\Omegadot}[0]{{\dot{\Omega}}}
\newcommand{\Phidot}[0]{{\dot{\Phi}}}
\newcommand{\Pidot}[0]{{\dot{\Pi}}}
\newcommand{\Psidot}[0]{{\dot{\Psi}}}
\newcommand{\Sigmadot}[0]{{\dot{\Sigma}}}
\newcommand{\Thetadot}[0]{{\dot{\Theta}}}
\newcommand{\Upsilondot}[0]{{\dot{\Upsilon}}}
\newcommand{\Xidot}[0]{{\dot{\Xi}}}
\newcommand{\alphadot}[0]{{\dot{\alpha}}}
\newcommand{\betadot}[0]{{\dot{\beta}}}
\newcommand{\chidot}[0]{{\dot{\chi}}}
\newcommand{\deltadot}[0]{{\dot{\delta}}}
\newcommand{\epsilondot}[0]{{\dot{\epsilon}}}
\newcommand{\etadot}[0]{{\dot{\eta}}}
\newcommand{\gammadot}[0]{{\dot{\gamma}}}
\newcommand{\kappadot}[0]{{\dot{\kappa}}}
\newcommand{\lambdadot}[0]{{\dot{\lambda}}}
\newcommand{\mudot}[0]{{\dot{\mu}}}
\newcommand{\nudot}[0]{{\dot{\nu}}}
\newcommand{\omegadot}[0]{{\dot{\omega}}}
\newcommand{\phidot}[0]{{\dot{\phi}}}
\newcommand{\pidot}[0]{{\dot{\pi}}}
\newcommand{\psidot}[0]{{\dot{\psi}}}
\newcommand{\rhodot}[0]{{\dot{\rho}}}
\newcommand{\sigmadot}[0]{{\dot{\sigma}}}
\newcommand{\taudot}[0]{{\dot{\tau}}}
\newcommand{\thetadot}[0]{{\dot{\theta}}}
\newcommand{\upsilondot}[0]{{\dot{\upsilon}}}
\newcommand{\varepsilondot}[0]{{\dot{\varepsilon}}}
\newcommand{\varphidot}[0]{{\dot{\varphi}}}
\newcommand{\varpidot}[0]{{\dot{\varpi}}}
\newcommand{\varrhodot}[0]{{\dot{\varrho}}}
\newcommand{\varsigmadot}[0]{{\dot{\varsigma}}}
\newcommand{\varthetadot}[0]{{\dot{\vartheta}}}
\newcommand{\xidot}[0]{{\dot{\xi}}}
\newcommand{\zetadot}[0]{{\dot{\zeta}}}

\newcommand{\Deltaddot}[0]{{\ddot{\Delta}}}
\newcommand{\Gammaddot}[0]{{\ddot{\Gamma}}}
\newcommand{\Lambdaddot}[0]{{\ddot{\Lambda}}}
\newcommand{\Omegaddot}[0]{{\ddot{\Omega}}}
\newcommand{\Phiddot}[0]{{\ddot{\Phi}}}
\newcommand{\Piddot}[0]{{\ddot{\Pi}}}
\newcommand{\Psiddot}[0]{{\ddot{\Psi}}}
\newcommand{\Sigmaddot}[0]{{\ddot{\Sigma}}}
\newcommand{\Thetaddot}[0]{{\ddot{\Theta}}}
\newcommand{\Upsilonddot}[0]{{\ddot{\Upsilon}}}
\newcommand{\Xiddot}[0]{{\ddot{\Xi}}}
\newcommand{\alphaddot}[0]{{\ddot{\alpha}}}
\newcommand{\betaddot}[0]{{\ddot{\beta}}}
\newcommand{\chiddot}[0]{{\ddot{\chi}}}
\newcommand{\deltaddot}[0]{{\ddot{\delta}}}
\newcommand{\epsilonddot}[0]{{\ddot{\epsilon}}}
\newcommand{\etaddot}[0]{{\ddot{\eta}}}
\newcommand{\gammaddot}[0]{{\ddot{\gamma}}}
\newcommand{\kappaddot}[0]{{\ddot{\kappa}}}
\newcommand{\lambdaddot}[0]{{\ddot{\lambda}}}
\newcommand{\muddot}[0]{{\ddot{\mu}}}
\newcommand{\nuddot}[0]{{\ddot{\nu}}}
\newcommand{\omegaddot}[0]{{\ddot{\omega}}}
\newcommand{\phiddot}[0]{{\ddot{\phi}}}
\newcommand{\piddot}[0]{{\ddot{\pi}}}
\newcommand{\psiddot}[0]{{\ddot{\psi}}}
\newcommand{\rhoddot}[0]{{\ddot{\rho}}}
\newcommand{\sigmaddot}[0]{{\ddot{\sigma}}}
\newcommand{\tauddot}[0]{{\ddot{\tau}}}
\newcommand{\thetaddot}[0]{{\ddot{\theta}}}
\newcommand{\upsilonddot}[0]{{\ddot{\upsilon}}}
\newcommand{\varepsilonddot}[0]{{\ddot{\varepsilon}}}
\newcommand{\varphiddot}[0]{{\ddot{\varphi}}}
\newcommand{\varpiddot}[0]{{\ddot{\varpi}}}
\newcommand{\varrhoddot}[0]{{\ddot{\varrho}}}
\newcommand{\varsigmaddot}[0]{{\ddot{\varsigma}}}
\newcommand{\varthetaddot}[0]{{\ddot{\vartheta}}}
\newcommand{\xiddot}[0]{{\ddot{\xi}}}
\newcommand{\zetaddot}[0]{{\ddot{\zeta}}}

\newcommand{\BDelta}[0]{\boldsymbol{\Delta}}
\newcommand{\BGamma}[0]{\boldsymbol{\Gamma}}
\newcommand{\BLambda}[0]{\boldsymbol{\Lambda}}
\newcommand{\BOmega}[0]{\boldsymbol{\Omega}}
\newcommand{\BPhi}[0]{\boldsymbol{\Phi}}
\newcommand{\BPi}[0]{\boldsymbol{\Pi}}
\newcommand{\BPsi}[0]{\boldsymbol{\Psi}}
\newcommand{\BSigma}[0]{\boldsymbol{\Sigma}}
\newcommand{\BTheta}[0]{\boldsymbol{\Theta}}
\newcommand{\BUpsilon}[0]{\boldsymbol{\Upsilon}}
\newcommand{\BXi}[0]{\boldsymbol{\Xi}}
\newcommand{\Balpha}[0]{\boldsymbol{\alpha}}
\newcommand{\Bbeta}[0]{\boldsymbol{\beta}}
\newcommand{\Bchi}[0]{\boldsymbol{\chi}}
\newcommand{\Bdelta}[0]{\boldsymbol{\delta}}
\newcommand{\Bepsilon}[0]{\boldsymbol{\epsilon}}
\newcommand{\Beta}[0]{\boldsymbol{\eta}}
\newcommand{\Bgamma}[0]{\boldsymbol{\gamma}}
\newcommand{\Bkappa}[0]{\boldsymbol{\kappa}}
\newcommand{\Blambda}[0]{\boldsymbol{\lambda}}
\newcommand{\Bmu}[0]{\boldsymbol{\mu}}
\newcommand{\Bnu}[0]{\boldsymbol{\nu}}
%\newcommand{\Bomega}[0]{\boldsymbol{\omega}}
\newcommand{\Bphi}[0]{\boldsymbol{\phi}}
\newcommand{\Bpi}[0]{\boldsymbol{\pi}}
\newcommand{\Bpsi}[0]{\boldsymbol{\psi}}
\newcommand{\Brho}[0]{\boldsymbol{\rho}}
\newcommand{\Bsigma}[0]{\boldsymbol{\sigma}}
%\newcommand{\Btau}[0]{\boldsymbol{\tau}}
%\newcommand{\Btheta}[0]{\boldsymbol{\theta}}
\newcommand{\Bupsilon}[0]{\boldsymbol{\upsilon}}
\newcommand{\Bvarepsilon}[0]{\boldsymbol{\varepsilon}}
\newcommand{\Bvarphi}[0]{\boldsymbol{\varphi}}
\newcommand{\Bvarpi}[0]{\boldsymbol{\varpi}}
\newcommand{\Bvarrho}[0]{\boldsymbol{\varrho}}
\newcommand{\Bvarsigma}[0]{\boldsymbol{\varsigma}}
\newcommand{\Bvartheta}[0]{\boldsymbol{\vartheta}}
\newcommand{\Bxi}[0]{\boldsymbol{\xi}}
\newcommand{\Bzeta}[0]{\boldsymbol{\zeta}}
%
%</bold and dot greek symbols>
%<infrequent>
%
%\newcommand{\AreaOp}[1]{\AName_{#1}}
%\newcommand{\Babs}[0]{\abs{\BB}}
%\newcommand{\Bcap}[0]{\hat{\BB}}
%\newcommand{\BrPrimeRej}[0]{\rcap(\rcap \wedge \Br')}
%\newcommand{\CA}[0]{\mathcal{A}}
%\newcommand{\Cos}[1]{\cos{\left({#1}\right)}}
%\newcommand{\Det}[1] {\abs{#1}}
%\newcommand{\Dsq}[2] {\frac {\partial^2 {#1}} {\partial {#2}^2}}
%\newcommand{\Exp}[1]{\exp{\left({#1}\right)}}
%\newcommand{\Norm}[1]{\left\lVert{#1}\right\rVert}
%\newcommand{\Sin}[1]{\sin{\left({#1}\right)}}
%\newcommand{\T}[0]{\text{T}}
%\newcommand{\VolumeOp}[1]{\VName_{#1}}
%\newcommand{\agrad}[0]{\Ba \cdot \nabla}
%\newcommand{\alphacap}[0]{\hat{\boldsymbol{\alpha}}}
%\newcommand{\Fcap}[0]{\hat{\BF}}
%\newcommand{\bithree}[0]{{\Bi}_3}
%\newcommand{\bxa}[0]{\Bx\Ba}
%\newcommand{\coordvec}[2]{
%\newcommand{\costheta}[0]{\acap \cdot \xcap}
%\newcommand{\ddt}[1]{\ddot{#1}}
%\newcommand{\ddu}[1] {\frac {d{#1}} {du}}
%\newcommand{\dsqxj}[2] {\frac {\partial^2 {#1}} {\partial {x_{#2}}^2}}
%\newcommand{\dtheta}[1]{\frac{d {#1}}{d \theta}}
%\newcommand{\dt}[1]{\dot{#1}}
%\newcommand{\dt}[1]{\frac{d {#1}}{dt}}
%\newcommand{\dxj}[2] {\frac {\partial {#1}} {\partial {x_{#2}}}}
%\newcommand{\halfPhi}[0]{\frac{\phi}{2}}
%\newcommand{\half}[0]{\inv{2}}
%\newcommand{\inv}[1]{\frac{1}{#1}}
%\newcommand{\laplacian}[0]{\nabla^2}
%\newcommand{\matrixoftx}[3]{
%\newcommand{\nrrp}[0]{\norm{\rcap \wedge \Br'}}
%\newcommand{\oiint}{\bigcirc \hspace{-1.4em} \int \hspace{-.8em} \int}
%\newcommand{\transpose}[1]{{#1}^{\text{T}}}
%\newcommand{\transpose}[1]{{{#1}^{\TextTranspose}}}
%\newcommand{\transpose}[1]{{{#1}^{\text{T}}}}
%\newcommand{\barA}[0]{\bar{A}}
%\newcommand{\qbar}[0]{\bar{q}}
%\newcommand{\qdotbar}[0]{\dot{\bar{q}}}
%
%</infrequent>





\newcommand{\T}[0]{{\text{T}}}

\usepackage[bookmarks=true]{hyperref}

\title{ 2D matrix vs. GA wedge-dot vector to vector function. }
\author{Peeter Joot \quad peeter.joot@gmail.com}
\date{ Nov 25, 2008.  Last Revision: $Date: 2009/02/22 15:11:52 $ }

\begin{document}

\maketitle{}

%\tableofcontents
\section{ Motivation. }

Persuing an email thread with Lut.  
Consider a simpler case of Lut's mapping of matrix to wedge-dot operator form.

GA has natural operator representations of a number of common geometric
operations that can also be expressed as matrix transformations.  Examples
are reflection, rotation, projection and rejection:

Examples of GA linear functions are

\begin{itemize}
\item reflection
\begin{align*}
R(x) = -n x \inv{n}
\end{align*}

\item rotation, boost (composition of two reflections)
\begin{align*}
R(x)
%&= \phi x \phi^\dagger \quad \text{where $\phi$ has only even grades.} \\
&= m n x \inv{n} \inv{m} \\
\end{align*}

\item projection onto direction vector

\begin{align*}
\Proj_v(x) 
&= \inv{v} v \cdot x \\
&= \inv{2v} (v x + x v) \\
\end{align*}

with matrix equivalent
\begin{align*}
\Proj_v(x) = \left(v \inv{v^\T v} v^\T \right) x
\end{align*}

(subspace projection takes similar to the vector forms for both GA and matrixes).

\item rejection from direction vector

\begin{align*}
\Proj_v(x) 
&= \inv{v} v \wedge x \\
&= \inv{2v} (v x - x v) \\
\end{align*}

matrix equivalent?

\end{itemize}

Given an arbitrary general matrix such as 

\begin{align*}
M = 
\begin{bmatrix}
a & b & c \\
d & e & f \\
g & h & i \\
\end{bmatrix}
\end{align*}

is there a natural way to represent this operation as a GA operation, perhaps first decomposing it into symmetric and antisymmetric parts?

\section{ Simpler case. }

A generalization of the rejection operation of the form

\begin{align*}
(a \wedge x ) \cdot b 
&= \inv{4} ( a x b - x a b - b a x + b x a )
\end{align*}

is a natural operator to consider as a relativity simple form that neccessarily maps vectors to vectors.  What is the matrix equivalent of this?

Consider to start just the 2D case.  Expanding this by coordinates one has for the wedge

\begin{align*}
(a \wedge x)
&= ((a_1 e_1 + a_2 e_2) \wedge (x_1 e_1 + x_2 e_2) \\
&= (a_1 x_2 - a_2 x_1) e_1 \wedge e_2
\end{align*}

so taking dot products one has

\begin{align*}
(a \wedge x ) \cdot b
&= (a_1 x_2 - a_2 x_1) (e_1 \wedge e_2) \cdot (b_1 e_1 + b_2 e_2) \\
&= (a_1 x_2 - a_2 x_1) (e_1 b_2 - e_2 b_1) \\
\end{align*}

For the matrix with respect to the standard basis of this linear transformation we then have

\begin{align*}
\begin{bmatrix}
(a_1 x_2 - a_2 x_1) b_2 \\
(a_1 x_2 - a_2 x_1) (-b_1) \\
\end{bmatrix}
=
\begin{bmatrix}
- a_2 b_2 & a_1 b_2  \\
a_2 b_1 & -a_1 b_1 \\
\end{bmatrix}
\begin{bmatrix}
x_1 \\
x_2 \\
\end{bmatrix}
\end{align*}

This matrix can be factored, which highlights some of the structure
\begin{align}\label{eqn:matrixOfOperator}
\begin{bmatrix}
- a_2 b_2 & a_1 b_2  \\
a_2 b_1 & -a_1 b_1 \\
\end{bmatrix}
=
\begin{bmatrix}
0 & -1 \\
1 & 0 \\
\end{bmatrix}
\begin{bmatrix}
b_1 \\
b_2 \\
\end{bmatrix}
\begin{bmatrix}
a_1 & a_2 \\
\end{bmatrix}
\begin{bmatrix}
0 & -1 \\
1 & 0 \\
\end{bmatrix}
\end{align}

Writing $A$, and $B$ for the coordinate column vectors, and $R = R_{\pi/2}$ for the antisymetric permutation matrix one has 

\begin{align*}
\begin{bmatrix}
(a \wedge x) \cdot b
\end{bmatrix}
=
R B A^\T R
\end{align*}

This seems to be a pretty specific form and I would guess that one can't go the other way around to find a wedge-dot operator representation of a general two by two matrix such as

\begin{align*}
\begin{bmatrix}
a & b \\
c & d \\
\end{bmatrix}
\end{align*}

To demonstrate this equate the two
\begin{align*}
\begin{bmatrix}
a & b \\
c & d \\
\end{bmatrix}
=
\begin{bmatrix}
- a_2 b_2 & a_1 b_2  \\
a_2 b_1 & -a_1 b_1 \\
\end{bmatrix}
\end{align*}

writing $a_2 = -a/b_2$, and $a_1 = b/b_2$ we have

\begin{align*}
\begin{bmatrix}
a & b \\
c & d \\
\end{bmatrix}
=
\begin{bmatrix}
a & b \\
-a (b_1/b_2) & -b (b_1/b_2) \\
\end{bmatrix}
\end{align*}

With $c = -a(b_1/b_2)$ or $b_1 = -c b_2 /a$ the $2,2$ term of the matrix
is left with the value $-b (-c b_2 /a / b_2 ) = -b (-c /a )$.  Therefore 
this matrix can only represent the
polynomial vector function $f(x) = (a \wedge x) \cdot b$ if 
$d = bc/a$, or $ad - bc = 0$.  The matrix must must have zero determinant
to have a representation of this form.

This can be seen directly by taking the determinant of 
or matrix in equation \ref{eqn:matrixOfOperator}

\begin{align*}
a_2 b_2 a_1 b_1 - a_2 b_1 a_1 b_2 = 0
\end{align*}

Now, is there a natural representation of an arbitrary matrix in polynomial form.  There are many possible
vector polynomials that map vectors to vectors.  Seeing the form of the polynomials for reflection, rotation, projection, rejection, and this wedge-dot operation (name?), one could guess that some hybrid that includes some subset of the 
all possible such variations would do.  Perhaps the best way to followup on this idea would be to consider the eigenvector and generalized eigenvector (Jordan form) decomposition of the general matrix to be considered.  The eigenvectors give a projective breakdown of the matrix, and each of those projections can be represented in GA form.  For the jordon blocks in the generalized eigenvalue decomposition, perhaps null determinant operators such as this wedge-dot function can represent those?

%\bibliographystyle{plainnat}
%\bibliography{myrefs}

\end{document}


\part{lut}
\documentclass{article}      % Specifies the document class

\usepackage{amsmath}
\usepackage{mathpazo}

%
% shorthand for bold symbols, convenient for vectors and matrices
%
\newcommand{\Ba}[0]{\mathbf{a}}
\newcommand{\Bb}[0]{\mathbf{b}}
\newcommand{\Bc}[0]{\mathbf{c}}
\newcommand{\Bd}[0]{\mathbf{d}}
\newcommand{\Be}[0]{\mathbf{e}}
\newcommand{\Bf}[0]{\mathbf{f}}
\newcommand{\Bg}[0]{\mathbf{g}}
\newcommand{\Bh}[0]{\mathbf{h}}
\newcommand{\Bi}[0]{\mathbf{i}}
\newcommand{\Bj}[0]{\mathbf{j}}
\newcommand{\Bk}[0]{\mathbf{k}}
\newcommand{\Bl}[0]{\mathbf{l}}
\newcommand{\Bm}[0]{\mathbf{m}}
\newcommand{\Bn}[0]{\mathbf{n}}
\newcommand{\Bo}[0]{\mathbf{o}}
\newcommand{\Bp}[0]{\mathbf{p}}
\newcommand{\Bq}[0]{\mathbf{q}}
\newcommand{\Br}[0]{\mathbf{r}}
\newcommand{\Bs}[0]{\mathbf{s}}
\newcommand{\Bt}[0]{\mathbf{t}}
\newcommand{\Bu}[0]{\mathbf{u}}
\newcommand{\Bv}[0]{\mathbf{v}}
\newcommand{\Bw}[0]{\mathbf{w}}
\newcommand{\Bx}[0]{\mathbf{x}}
\newcommand{\By}[0]{\mathbf{y}}
\newcommand{\Bz}[0]{\mathbf{z}}
\newcommand{\BA}[0]{\mathbf{A}}
\newcommand{\BB}[0]{\mathbf{B}}
\newcommand{\BC}[0]{\mathbf{C}}
\newcommand{\BD}[0]{\mathbf{D}}
\newcommand{\BE}[0]{\mathbf{E}}
\newcommand{\BF}[0]{\mathbf{F}}
\newcommand{\BG}[0]{\mathbf{G}}
\newcommand{\BH}[0]{\mathbf{H}}
\newcommand{\BI}[0]{\mathbf{I}}
\newcommand{\BJ}[0]{\mathbf{J}}
\newcommand{\BK}[0]{\mathbf{K}}
\newcommand{\BL}[0]{\mathbf{L}}
\newcommand{\BM}[0]{\mathbf{M}}
\newcommand{\BN}[0]{\mathbf{N}}
\newcommand{\BO}[0]{\mathbf{O}}
\newcommand{\BP}[0]{\mathbf{P}}
\newcommand{\BQ}[0]{\mathbf{Q}}
\newcommand{\BR}[0]{\mathbf{R}}
\newcommand{\BS}[0]{\mathbf{S}}
\newcommand{\BT}[0]{\mathbf{T}}
\newcommand{\BU}[0]{\mathbf{U}}
\newcommand{\BV}[0]{\mathbf{V}}
\newcommand{\BW}[0]{\mathbf{W}}
\newcommand{\BX}[0]{\mathbf{X}}
\newcommand{\BY}[0]{\mathbf{Y}}
\newcommand{\BZ}[0]{\mathbf{Z}}

\newcommand{\Bzero}[0]{\mathbf{0}}
\newcommand{\Btheta}[0]{\boldsymbol{\theta}}
\newcommand{\Btau}[0]{\boldsymbol{\tau}}
\newcommand{\Bomega}[0]{\boldsymbol{\omega}}

%
% shorthand for unit vectors
%
\newcommand{\acap}[0]{\hat{\Ba}}
\newcommand{\bcap}[0]{\hat{\Bb}}
\newcommand{\ccap}[0]{\hat{\Bc}}
\newcommand{\dcap}[0]{\hat{\Bd}}
\newcommand{\ecap}[0]{\hat{\Be}}
\newcommand{\fcap}[0]{\hat{\Bf}}
\newcommand{\gcap}[0]{\hat{\Bg}}
\newcommand{\hcap}[0]{\hat{\Bh}}
\newcommand{\icap}[0]{\hat{\Bi}}
\newcommand{\jcap}[0]{\hat{\Bj}}
\newcommand{\kcap}[0]{\hat{\Bk}}
\newcommand{\lcap}[0]{\hat{\Bl}}
\newcommand{\mcap}[0]{\hat{\Bm}}
\newcommand{\ncap}[0]{\hat{\Bn}}
\newcommand{\ocap}[0]{\hat{\Bo}}
\newcommand{\pcap}[0]{\hat{\Bp}}
\newcommand{\qcap}[0]{\hat{\Bq}}
\newcommand{\rcap}[0]{\hat{\Br}}
\newcommand{\scap}[0]{\hat{\Bs}}
\newcommand{\tcap}[0]{\hat{\Bt}}
\newcommand{\ucap}[0]{\hat{\Bu}}
\newcommand{\vcap}[0]{\hat{\Bv}}
\newcommand{\wcap}[0]{\hat{\Bw}}
\newcommand{\xcap}[0]{\hat{\Bx}}
\newcommand{\ycap}[0]{\hat{\By}}
\newcommand{\zcap}[0]{\hat{\Bz}}
\newcommand{\thetacap}[0]{\hat{\Btheta}}

%
% to write R^n and C^n in a distinguishable fashion.  Perhaps change this
% to the double lined characters upon figuring out how to do so.
%
\newcommand{\C}[1]{$\mathbb{C}^{#1}$}
\newcommand{\R}[1]{$\mathbb{R}^{#1}$}

%
% various generally useful helpers
%

% derivative of #1 wrt. #2:
\newcommand{\D}[2] {\frac {d#2} {d#1}}

\newcommand{\inv}[1]{\frac{1}{#1}}
\newcommand{\cross}[0]{\times}

\newcommand{\abs}[1]{\lvert{#1}\rvert}
\newcommand{\norm}[1]{\lVert{#1}\rVert}
\newcommand{\innerprod}[2]{\langle{#1}, {#2}\rangle}
\newcommand{\dotprod}[2]{{#1} \cdot {#2}}
\newcommand{\bdotprod}[2]{\left({#1} \cdot {#2}\right)}
\newcommand{\crossprod}[2]{{#1} \cross {#2}}
\newcommand{\tripleprod}[3]{\dotprod{\left(\crossprod{#1}{#2}\right)}{#3}}

\DeclareMathOperator{\Proj}{Proj}
\DeclareMathOperator{\Span}{span}
\DeclareMathOperator{\Sgn}{sgn}
\DeclareMathOperator{\Area}{Area}
\DeclareMathOperator{\Volume}{Volume}

%
% A few miscellaneous things specific to this document
%
\newcommand{\crossop}[1]{\crossprod{#1}{}}

% R2 vector.
\newcommand{\VectorTwo}[2]{
\begin{bmatrix}
 {#1} \\
 {#2}
\end{bmatrix}
}

\newcommand{\VectorN}[1]{
\begin{bmatrix}
{#1}_1 \\
{#1}_2 \\
\vdots \\
{#1}_N \\
\end{bmatrix}
}

\newcommand{\DETuvij}[4]{
\begin{vmatrix}
 {#1}_{#3} & {#1}_{#4} \\
 {#2}_{#3} & {#2}_{#4}
\end{vmatrix}
}

\newcommand{\DETuvwijk}[6]{
\begin{vmatrix}
 {#1}_{#4} & {#1}_{#5} & {#1}_{#6} \\
 {#2}_{#4} & {#2}_{#5} & {#2}_{#6} \\
 {#3}_{#4} & {#3}_{#5} & {#3}_{#6}
\end{vmatrix}
}

\newcommand{\DETuvwxijkl}[8]{
\begin{vmatrix}
 {#1}_{#5} & {#1}_{#6} & {#1}_{#7} & {#1}_{#8} \\
 {#2}_{#5} & {#2}_{#6} & {#2}_{#7} & {#2}_{#8} \\
 {#3}_{#5} & {#3}_{#6} & {#3}_{#7} & {#3}_{#8} \\
 {#4}_{#5} & {#4}_{#6} & {#4}_{#7} & {#4}_{#8} \\
\end{vmatrix}
}

%\newcommand{\DETuvwxyijklm}[10]{
%\begin{vmatrix}
% {#1}_{#6} & {#1}_{#7} & {#1}_{#8} & {#1}_{#9} & {#1}_{#10} \\
% {#2}_{#6} & {#2}_{#7} & {#2}_{#8} & {#2}_{#9} & {#2}_{#10} \\
% {#3}_{#6} & {#3}_{#7} & {#3}_{#8} & {#3}_{#9} & {#3}_{#10} \\
% {#4}_{#6} & {#4}_{#7} & {#4}_{#8} & {#4}_{#9} & {#4}_{#10} \\
% {#5}_{#6} & {#5}_{#7} & {#5}_{#8} & {#5}_{#9} & {#5}_{#10}
%\end{vmatrix}
%}

% R3 vector.
\newcommand{\VectorThree}[3]{
\begin{bmatrix}
 {#1} \\
 {#2} \\
 {#3}
\end{bmatrix}
}


\newcommand{\LL}[0]{\mathcal{L}}
\newcommand{\PD}[2]{\frac{\partial {#2}}{\partial {#1}}}
\newcommand{\dotalpha}[0]{\dot{\alpha}}
\newcommand{\ddotalpha}[0]{\ddot{\alpha}}

\newcommand{\dotomega}[0]{\dot{\omega}}
\newcommand{\ddotomega}[0]{\ddot{\omega}}

\newcommand{\dotOmega}[0]{\dot{\Omega}}
\newcommand{\ddotOmega}[0]{\ddot{\Omega}}

\newcommand{\CC}[0]{c^2}

\newcommand{\dottheta}[0]{\dot{\theta}}
\newcommand{\ddottheta}[0]{\ddot{\theta}}

\newcommand{\dotpsi}[0]{\dot{\psi}}
\newcommand{\ddotpsi}[0]{\ddot{\psi}}

\newcommand{\adot}[0]{\dot{a}}
\newcommand{\addot}[0]{\ddot{a}}
\newcommand{\qdot}[0]{\dot{q}}
\newcommand{\qddot}[0]{\ddot{q}}
\newcommand{\tdot}[0]{\dot{t}}
\newcommand{\tddot}[0]{\ddot{t}}

\newcommand{\Rdot}[0]{\dot{R}}

\newcommand{\pdot}[0]{\dot{p}}
\newcommand{\pddot}[0]{\ddot{p}}

\newcommand{\xdot}[0]{\dot{x}}
\newcommand{\xddot}[0]{\ddot{x}}

\newcommand{\zdot}[0]{\dot{z}}
\newcommand{\zddot}[0]{\ddot{z}}

\newcommand{\rdot}[0]{\dot{r}}
\newcommand{\rddot}[0]{\ddot{r}}

%
% The real thing:
%

\usepackage[bookmarks=true]{hyperref}

                             % The preamble begins here.
\title{ Some GR Notes. } % Declares the document's title.
\author{Peeter Joot}         % Declares the author's name.
\date{ October 2, 2008.  Last Revision: $Date: 2008/10/04 04:01:18 $ } % Deleting this command produces today's date.

\begin{document}             % End of preamble and beginning of text.

\maketitle{}

\tableofcontents

\section{ Motivation. }

Some General relativity notes exploring ideas from emails with Lut Mentz.
Prior to this I was under the impression that I had zero knowledge of GR,
but it turns out that many of the ideas are really action based. 
Allowing the spacetime unit vectors to vary, something we are free to
do in SR or newtonian mechanics too, results in a more 
general metric in a purely kinetic Lagrangian.  This metric variation
can be interpretted as a mechanism for introducing more general
accererations very similar to fictious forces that one sees in a
rotating frame or other non-uniform coordinate system.

These notes contain my attempt to walk through some of these ideas, to see
if I can coherently explain them to myself.  If I can't do so then I don't
understand things sufficently.  Being able to produce such an explaination
may not mean that I truely understand the issues, but it is a required
first step.

Useful references are Lut's writeup \cite{lutSchwarzChildRadial}, 
and the schwarzchild calculation \cite{mathpagesSchwarzChildRadial}
from the online text reflections on relativity.

FIXME: Ask Lut if he put his Rindler derivation online.

FIXME: Original PF thread and email with Lut was associated with 
a Lagragian where mass was allowed to vary as described in \cite{PJMassVary}.  He was
investigating the similarities between varying mass directly and the spatial/metric
variation of GR.

\subsection{ Lagrangian for General Relativity. }

The equations of motion resulting from a purely kinetic Lagrangian

\begin{equation}\label{eqn:keLagrangian}
\LL = \inv{2} \sum  g_{b c}(q^a) \qdot^b \qdot^c,
\end{equation}

can be found to be

\begin{align}
0
&= \qddot^a + \qdot^b \qdot^c {\Gamma^a}_{b c} \\
{\Gamma^a}_{b c} &= \inv{2} g^{a d} \left( 
\PD{q^b}{g_{c d}} 
+ \PD{q^c}{g_{d b}} 
- \PD{q^d}{g_{b c}} 
\right)
\end{align}

One such derivation can be found in
the solution of problem one \cite{PJTongMf1}, associated with the 
Lagrangian problem set for
Dr. David Tong's online mechanics text \cite{TongDynamics}.

This is the Lagrangian for general relativity, once the metric tensor $g_{a b}$ is specified.

\subsection{ General kinetic Lagrangian for fixed frame basis. }

One doesn't have to go to GR to find Kinetic energy expressions of the form in equation \ref{eqn:keLagrangian}.

A simple example of a more general Kinetic energy description can be found by any use of non-orthonormal basis vectors, say $\{\Be_i\}$,
for the space.

Given such a non-orthonormal frame, the trick to calculating the coordinates is tied to an alternate set of basis vectors, called the reciprocal frame.
Provided the initial set of vectors spans the space, one can always calculate this second pair such that they meet the following relationships:

\begin{equation*}
\Be^i \cdot \Be_j = {\delta^i}_j
\end{equation*}

Calculating these reciprocal frame vectors is a linear algebra problem, essentially requiring a matrix inversion.  Here we just assume they can be calculated and use them as a convienent way to find the coordinates in this non-orthogonal frame

\begin{align*}
\Bx &= \sum \Be_j a_j \\
\Bx \cdot \Be^i &= \sum (\Be_j a_j) \cdot \Be^i = \sum {\delta^j}_i a_j = a_i \\
\implies \\
\Bx &= \sum \Be_j (\Bx \cdot \Be^j)
\end{align*}

It is customary to write $a_i = \Bx \cdot \Be^i = x^i$, in order to have
mixed upper and lower indexes for implied summation.

\begin{equation*}
\Bx = \sum \Be_j x^j = \Be_j x^j
\end{equation*}

Once one has a way of calculating coordinates for an arbitrary basis, other quantities such as velocity
can then be calculated

\begin{equation*}
\Bv^2 = \dot{\Bx} \cdot \dot{\Bx} = (\Be_j \cdot \Be_k) \xdot^j \xdot^k
\end{equation*}

This
$\Be_j \cdot \Be_k$ coefficient of the coordinates
gets a special name, the metric tensor
$g_{j k} = \Be_j \cdot \Be_k$.  It is a symmetric and invertable quantity and can be employed to 
express the Kinetic Energy term of a single particle Lagrangian in the general tensor form 

\begin{equation*}
\inv{2} m \Bv^2 = \inv{2} m g_{j k} \xdot^j \xdot^k
\end{equation*}

Now, in this case $g_{j k}$ is not a function of the coordinates (ie: of position) as in
equation \ref{eqn:keLagrangian}.

\subsection{ General kinetic Lagrangian for a basis frame. }

Now, in the GR case the metric varies (or may) with position.  What do we need to observe this same form in Newtonian
physics?  The immediate thing that comes to mind is the use of a curvalinear basis, where the basis vectors
for a space are allowed to vary direction with position along some path.  Initially that seemed reasonable to
me but for some arbitrary parameterized path, wouldn't the metric then also vary with the path parameterization?  If that
was the case, the Lagrangian in equation \ref{eqn:keLagrangian} does not have the form of a kinetic energy expression.

To resolve this I considered an example.  I can lay out two directions in my backyard, one along the vegetable garden
parallel to the house roughly pointing north, and another diagonally across to my gate.  This logically defines a coordinate
system or set of frame vectors that I can make local measurements with respect to.
Now, translation of this coordinate system
to my dad's house 40 km to the south won't be a particularily logical for measuring there.  He lives on a very 
steep hill.

I can pace out distances in my backyard without having to consider the curvature
of the Earth and my dad can do the same for a long stretch of the hill walking up the street towards his house.
The local frame vectors can be considered to lie along a flat surface if that surface area is small enough.
Alternately we can say the associated metric associated with a surface coordinate system for a point on surface of the Earth can be considered constant for small enough measurements.

Now, to express the same ideas mathematically, consider a curve expressed parametrically between two points, such that all
points along the path take the values

\begin{equation*}
\Bx(\lambda) = x^i(\lambda) \Be_i(\Bx)
\end{equation*}

The vector distance between two points on this path is

\begin{equation*}
\Bx(\lambda_2) - \Bx(\lambda_1) = x^i(\lambda_2) \Be_i(\Bx(\lambda_2)) - x^i(\lambda_1) \Be_i(\Bx(\lambda_1))
\end{equation*}

but this is the direct difference in position between these two points, not the distance along the curve.
To be a true measure of the distance the difference in position has to also be small enough that the frame vectors lie in the same direction at both points to some approximation.

Given such an approximation one can then write
\begin{align*}
\Bx(\lambda_2) - \Bx(\lambda_1) &= \left(x^i(\lambda_2) - x^i(\lambda_1)\right) \Be_i(\Bx) \\
d\Bx &= \frac{d x^i}{d\lambda} \Be_i(\Bx) d\lambda \\
\end{align*}

For such a representation to be valid, the variation of $\Be_i$ at the point $\Bx$ has to be small enough that $\Be_i$ can be 
considered constant.  This is still not a very well defined statement mathematically, and it is not too hard to imagine 
scenerios where it totally fails.  An example is a fractal like curve, something continuous but not differtiable at
any point.

Assuming a sufficently differentiable curve then the distance along the curve between two points can be obtained from the
integral

\begin{align*}
ds
&= \int_{\lambda_1}^{\lambda_2} \sqrt{\left(\frac{d\Bx}{d\lambda}\right)^2} d\lambda \\
&= \int_{\lambda_1}^{\lambda_2} \sqrt{ g_{ i j } \frac{dx^i}{d\lambda} \frac{dx^j}{d\lambda} } d\lambda \\
\end{align*}

Or
\begin{align*}
\left(\frac{ds}{d\lambda}\right)^2 = g_{ i j } \frac{dx^i}{d\lambda} \frac{dx^j}{d\lambda}
\end{align*}

Now, what is the most natural parameterization?  For physical situations time comes to mind, but if the particle stops for a
while on the path, then this derivative goes to zero for a while even if the curve is continuous and has derivatives of all
orders at all points.  Falling back to the most simple curve as a motivator, the circle, 
the natural parameterization is using fractions $\theta$ of the total arc length of the circle $2\pi$.  If the
curve is parameterized by arc length we have

\begin{align*}
\left(\frac{ds}{ds}\right)^2 = 1 = g_{ i j } \frac{dx^i}{ds} \frac{dx^j}{ds}
\end{align*}

%there the natural parameterization is the arc length itself
%Just as in geometry the most natural mathematic parameterization variable is often the arc length itself (


\begin{align*}
t &= t(\lambda) \\
r &= r(\lambda) \\
\Omega &= \Omega(\lambda)
\end{align*}

An element of event arc length will then be

\begin{align*}
ds^2 = \left(\frac{ds}{d\lambda}\right)^2 {d\lambda}^2
&=
\left( -\CC a \left(\frac{dt}{d\lambda}\right)^2 + {b} \left(\frac{dr}{d\lambda}\right)^2 + r^2\left(\frac{d\Omega}{d\lambda}\right)^2 \right) {d\lambda}^2
\end{align*}


\section{ Rindler Metric. }

Lut described the 

\section{ Schwartzchild Metric. } 

Work through the Euler-Lagrange equations for what Lut calls the 
Schwartzchild metric.

\subsection{ Affine Coordinates. }

%\begin{align}
%\CC (d\tau)^2 &= -\CC a(r) (dt)^2 + {b(r)} (dr)^2 + r^2(d\Omega)^2 \\
%a(r) &= 1 - \kappa/r \\
%b(r) &= \inv{a(r)} \\
%\PD{r}{a} &= \kappa/r^2 \\
%\PD{r}{b} &= -\kappa/(r-k)^2 \\
%\adot &= \frac{\kappa \rdot}{r^2}
%\bdot &= \frac{-\kappa \rdot}{(r-k)^2}
%\end{align}

I would interpret the affine coordinates this way.  If we have two ``close'' events described by our $t,r,\Omega$ coordinates, then the path and corresponding
element of arc length between two events can be described parameterically
with an arbitrary parameter $\lambda$

\begin{align*}
t &= t(\lambda) \\
r &= r(\lambda) \\
\Omega &= \Omega(\lambda)
\end{align*}

An element of event arc length will then be

\begin{align*}
ds^2 = \left(\frac{ds}{d\lambda}\right)^2 {d\lambda}^2
&=
\left( -\CC a \left(\frac{dt}{d\lambda}\right)^2 + {b} \left(\frac{dr}{d\lambda}\right)^2 + r^2\left(\frac{d\Omega}{d\lambda}\right)^2 \right) {d\lambda}^2
\end{align*}


Introducing such a natural parameterization with units of time means that we need a dimensional fudge factor, so we write $ds = c d\tau$, so that $\tau$ is the unique parameterization for which $(ds/d\tau)^2 = c^2$.
This implicitly defines the ``proper time'', and we can write our Lagrangian to minimize in terms of that variable

\begin{align*}%\label{eqn:action}
\LL
&= \left(\frac{ds}{d\tau}\right)^2 \\
&= c^2 \\
&= \left( -\CC a \left(\frac{dt}{d\tau}\right)^2 + {b} \left(\frac{dr}{d\tau}\right)^2 + r^2\left(\frac{d\Omega}{d\tau}\right)^2 \right) \\
&= -\CC a \tdot^2 + {b} \rdot^2 + r^2\dotOmega^2 \\
\end{align*}

%Derivatives for $\Omega$
%
%\begin{align*}
%\PD{\Omega}{\LL} &= \left(\PD{\dotOmega}{\LL}\right)' \\
%0 &= (2 r^2 \dotOmega)' \\
%\end{align*}
%
%Observing that $\Omega$ is cyclic, we can write for some constant $A$
%
%\begin{align*}
%\dotOmega = \frac{A}{r^2}
%\end{align*}
%
%From \ref{eqn:action} this provides a relationship between $\rdot$ and $\tdot$
%
%\begin{align*}
%0 = -\CC (a \tdot^2 + 1) + \inv{a} \rdot^2 + A^2/r^2
%\end{align*}
%
%Calculating the derivatives for $t$
%
%\begin{align*}
%\PD{t}{\LL} &= \left(\PD{\tdot}{\LL}\right)' \\
%0 
%&= \left( - 2 \CC a \tdot \right)' \\
%%&= - 2 \CC \adot \tdot - 2 \CC a \tddot \\
%%\implies \\
%%\tddot &= \frac{\kappa \rdot \tdot}{a r^2} \\
%\end{align*}
%
%Again observing that $t$ is cyclic, we can write for some constant $B$
%
%\begin{align*}
%\tdot = B/a
%\end{align*}
%
%Or
%\begin{align*}
%\rdot^2 = \CC ({B^2} + a) - a A^2/r^2
%\end{align*}
%
%Last the derivatives for $r$
%
%%\LL = \CC  = -\CC a \tdot^2 + \inv{a} \rdot^2 + r^2\dotOmega^2
%\begin{align*}
%\PD{r}{\LL} &= \left(\PD{\rdot}{\LL}\right)' \\
%-\CC \PD{r}{a} \tdot^2 - \frac{\adot \rdot^2}{a^2} + 2 r \dotOmega^2 
%&= \left(\frac{2 \rdot}{a}\right)' \\
%&= -2 \frac{ \adot \rdot }{a^2} + \frac{2 \rddot}{a} \\ 
%\implies \\
%{\rddot} 
%&= - \inv{2} \CC \frac{\kappa}{r^2} a \tdot^2 -  \frac{\adot \rdot^2}{2 a} + r a \dotOmega^2 + \frac{ \adot \rdot }{a} \\
%&= - \inv{2} \CC \frac{\kappa B^2}{a r^2} -  \frac{\kappa \rdot^3}{2 a r^2} + a \frac{A^2}{r^3} + \frac{ \kappa \rdot^2 }{a r^2} \\
%\end{align*}
%
%The $\rdot^3$ term doesn't look right dimensionally.



%\bibliographystyle{plain}
\bibliographystyle{plainnat} % supposed to allow for \url use.
\bibliography{myrefs}      % expects file "myrefs.bib"

\end{document}               % End of document.

\documentclass{article}

\usepackage{amsmath}
\usepackage{mathpazo}

%
% shorthand for bold symbols, convenient for vectors and matrices
%
\newcommand{\Ba}[0]{\mathbf{a}}
\newcommand{\Bb}[0]{\mathbf{b}}
\newcommand{\Bc}[0]{\mathbf{c}}
\newcommand{\Bd}[0]{\mathbf{d}}
\newcommand{\Be}[0]{\mathbf{e}}
\newcommand{\Bf}[0]{\mathbf{f}}
\newcommand{\Bg}[0]{\mathbf{g}}
\newcommand{\Bh}[0]{\mathbf{h}}
\newcommand{\Bi}[0]{\mathbf{i}}
\newcommand{\Bj}[0]{\mathbf{j}}
\newcommand{\Bk}[0]{\mathbf{k}}
\newcommand{\Bl}[0]{\mathbf{l}}
\newcommand{\Bm}[0]{\mathbf{m}}
\newcommand{\Bn}[0]{\mathbf{n}}
\newcommand{\Bo}[0]{\mathbf{o}}
\newcommand{\Bp}[0]{\mathbf{p}}
\newcommand{\Bq}[0]{\mathbf{q}}
\newcommand{\Br}[0]{\mathbf{r}}
\newcommand{\Bs}[0]{\mathbf{s}}
\newcommand{\Bt}[0]{\mathbf{t}}
\newcommand{\Bu}[0]{\mathbf{u}}
\newcommand{\Bv}[0]{\mathbf{v}}
\newcommand{\Bw}[0]{\mathbf{w}}
\newcommand{\Bx}[0]{\mathbf{x}}
\newcommand{\By}[0]{\mathbf{y}}
\newcommand{\Bz}[0]{\mathbf{z}}
\newcommand{\BA}[0]{\mathbf{A}}
\newcommand{\BB}[0]{\mathbf{B}}
\newcommand{\BC}[0]{\mathbf{C}}
\newcommand{\BD}[0]{\mathbf{D}}
\newcommand{\BE}[0]{\mathbf{E}}
\newcommand{\BF}[0]{\mathbf{F}}
\newcommand{\BG}[0]{\mathbf{G}}
\newcommand{\BH}[0]{\mathbf{H}}
\newcommand{\BI}[0]{\mathbf{I}}
\newcommand{\BJ}[0]{\mathbf{J}}
\newcommand{\BK}[0]{\mathbf{K}}
\newcommand{\BL}[0]{\mathbf{L}}
\newcommand{\BM}[0]{\mathbf{M}}
\newcommand{\BN}[0]{\mathbf{N}}
\newcommand{\BO}[0]{\mathbf{O}}
\newcommand{\BP}[0]{\mathbf{P}}
\newcommand{\BQ}[0]{\mathbf{Q}}
\newcommand{\BR}[0]{\mathbf{R}}
\newcommand{\BS}[0]{\mathbf{S}}
\newcommand{\BT}[0]{\mathbf{T}}
\newcommand{\BU}[0]{\mathbf{U}}
\newcommand{\BV}[0]{\mathbf{V}}
\newcommand{\BW}[0]{\mathbf{W}}
\newcommand{\BX}[0]{\mathbf{X}}
\newcommand{\BY}[0]{\mathbf{Y}}
\newcommand{\BZ}[0]{\mathbf{Z}}

\newcommand{\Bzero}[0]{\mathbf{0}}
\newcommand{\Btheta}[0]{\boldsymbol{\theta}}
\newcommand{\Btau}[0]{\boldsymbol{\tau}}
\newcommand{\Bomega}[0]{\boldsymbol{\omega}}

%
% shorthand for unit vectors
%
\newcommand{\acap}[0]{\hat{\Ba}}
\newcommand{\bcap}[0]{\hat{\Bb}}
\newcommand{\ccap}[0]{\hat{\Bc}}
\newcommand{\dcap}[0]{\hat{\Bd}}
\newcommand{\ecap}[0]{\hat{\Be}}
\newcommand{\fcap}[0]{\hat{\Bf}}
\newcommand{\gcap}[0]{\hat{\Bg}}
\newcommand{\hcap}[0]{\hat{\Bh}}
\newcommand{\icap}[0]{\hat{\Bi}}
\newcommand{\jcap}[0]{\hat{\Bj}}
\newcommand{\kcap}[0]{\hat{\Bk}}
\newcommand{\lcap}[0]{\hat{\Bl}}
\newcommand{\mcap}[0]{\hat{\Bm}}
\newcommand{\ncap}[0]{\hat{\Bn}}
\newcommand{\ocap}[0]{\hat{\Bo}}
\newcommand{\pcap}[0]{\hat{\Bp}}
\newcommand{\qcap}[0]{\hat{\Bq}}
\newcommand{\rcap}[0]{\hat{\Br}}
\newcommand{\scap}[0]{\hat{\Bs}}
\newcommand{\tcap}[0]{\hat{\Bt}}
\newcommand{\ucap}[0]{\hat{\Bu}}
\newcommand{\vcap}[0]{\hat{\Bv}}
\newcommand{\wcap}[0]{\hat{\Bw}}
\newcommand{\xcap}[0]{\hat{\Bx}}
\newcommand{\ycap}[0]{\hat{\By}}
\newcommand{\zcap}[0]{\hat{\Bz}}
\newcommand{\thetacap}[0]{\hat{\Btheta}}

%
% to write R^n and C^n in a distinguishable fashion.  Perhaps change this
% to the double lined characters upon figuring out how to do so.
%
\newcommand{\C}[1]{$\mathbb{C}^{#1}$}
\newcommand{\R}[1]{$\mathbb{R}^{#1}$}

%
% various generally useful helpers
%

% derivative of #1 wrt. #2:
\newcommand{\D}[2] {\frac {d#2} {d#1}}

\newcommand{\inv}[1]{\frac{1}{#1}}
\newcommand{\cross}[0]{\times}

\newcommand{\abs}[1]{\lvert{#1}\rvert}
\newcommand{\norm}[1]{\lVert{#1}\rVert}
\newcommand{\innerprod}[2]{\langle{#1}, {#2}\rangle}
\newcommand{\dotprod}[2]{{#1} \cdot {#2}}
\newcommand{\bdotprod}[2]{\left({#1} \cdot {#2}\right)}
\newcommand{\crossprod}[2]{{#1} \cross {#2}}
\newcommand{\tripleprod}[3]{\dotprod{\left(\crossprod{#1}{#2}\right)}{#3}}

\DeclareMathOperator{\Proj}{Proj}
\DeclareMathOperator{\Span}{span}
\DeclareMathOperator{\Sgn}{sgn}
\DeclareMathOperator{\Area}{Area}
\DeclareMathOperator{\Volume}{Volume}

%
% A few miscellaneous things specific to this document
%
\newcommand{\crossop}[1]{\crossprod{#1}{}}

% R2 vector.
\newcommand{\VectorTwo}[2]{
\begin{bmatrix}
 {#1} \\
 {#2}
\end{bmatrix}
}

\newcommand{\VectorN}[1]{
\begin{bmatrix}
{#1}_1 \\
{#1}_2 \\
\vdots \\
{#1}_N \\
\end{bmatrix}
}

\newcommand{\DETuvij}[4]{
\begin{vmatrix}
 {#1}_{#3} & {#1}_{#4} \\
 {#2}_{#3} & {#2}_{#4}
\end{vmatrix}
}

\newcommand{\DETuvwijk}[6]{
\begin{vmatrix}
 {#1}_{#4} & {#1}_{#5} & {#1}_{#6} \\
 {#2}_{#4} & {#2}_{#5} & {#2}_{#6} \\
 {#3}_{#4} & {#3}_{#5} & {#3}_{#6}
\end{vmatrix}
}

\newcommand{\DETuvwxijkl}[8]{
\begin{vmatrix}
 {#1}_{#5} & {#1}_{#6} & {#1}_{#7} & {#1}_{#8} \\
 {#2}_{#5} & {#2}_{#6} & {#2}_{#7} & {#2}_{#8} \\
 {#3}_{#5} & {#3}_{#6} & {#3}_{#7} & {#3}_{#8} \\
 {#4}_{#5} & {#4}_{#6} & {#4}_{#7} & {#4}_{#8} \\
\end{vmatrix}
}

%\newcommand{\DETuvwxyijklm}[10]{
%\begin{vmatrix}
% {#1}_{#6} & {#1}_{#7} & {#1}_{#8} & {#1}_{#9} & {#1}_{#10} \\
% {#2}_{#6} & {#2}_{#7} & {#2}_{#8} & {#2}_{#9} & {#2}_{#10} \\
% {#3}_{#6} & {#3}_{#7} & {#3}_{#8} & {#3}_{#9} & {#3}_{#10} \\
% {#4}_{#6} & {#4}_{#7} & {#4}_{#8} & {#4}_{#9} & {#4}_{#10} \\
% {#5}_{#6} & {#5}_{#7} & {#5}_{#8} & {#5}_{#9} & {#5}_{#10}
%\end{vmatrix}
%}

% R3 vector.
\newcommand{\VectorThree}[3]{
\begin{bmatrix}
 {#1} \\
 {#2} \\
 {#3}
\end{bmatrix}
}


%<misc>
%
\newcommand{\Abs}[1]{{\left\lvert{#1}\right\rvert}}
\newcommand{\spacegrad}[0]{\boldsymbol{\nabla}}
\newcommand{\grad}[0]{\nabla}
\newcommand{\LL}[0]{\mathcal{L}}

% == \partial_{#1} {#2}
\newcommand{\PD}[2]{\frac{\partial {#2}}{\partial {#1}}}
% inline variant
\newcommand{\PDi}[2]{{\partial {#2}}/{\partial {#1}}}

\newcommand{\PDD}[3]{\frac{\partial^2 {#3}}{\partial {#1}\partial {#2}}}
%\newcommand{\PDd}[2]{\frac{\partial^2 {#2}}{{\partial{#1}}^2}}
\newcommand{\PDsq}[2]{\frac{\partial^2 {#2}}{(\partial {#1})^2}}

\newcommand{\Partial}[2]{\frac{\partial {#1}}{\partial {#2}}}
\DeclareMathOperator{\RejName}{Rej}
\newcommand{\Rej}[2]{\RejName_{#1}\left( {#2} \right)}
\newcommand{\Rm}[1]{\mathbb{R}^{#1}}
\newcommand{\Cm}[1]{\mathbb{C}^{#1}}
\newcommand{\conj}[0]{{*}}

%</misc>

% <grade selection>
%
\newcommand{\gpgrade}[2] {{\left\langle{{#1}}\right\rangle}_{#2}}

\newcommand{\gpgradezero}[1] {\gpgrade{#1}{}}
%\newcommand{\gpscalargrade}[1] {{\left\langle{{#1}}\right\rangle}}
%\newcommand{\gpgradezero}[1] {\gpgrade{#1}{0}}

%\newcommand{\gpgradeone}[1] {{\left\langle{{#1}}\right\rangle}_{1}}
\newcommand{\gpgradeone}[1] {\gpgrade{#1}{1}}

\newcommand{\gpgradetwo}[1] {\gpgrade{#1}{2}}
\newcommand{\gpgradethree}[1] {\gpgrade{#1}{3}}
\newcommand{\gpgradefour}[1] {\gpgrade{#1}{4}}
%
% </grade selection>



\newcommand{\adot}[0]{{\dot{a}}}
\newcommand{\bdot}[0]{{\dot{b}}}
% taken for centered dot:
%\newcommand{\cdot}[0]{{\dot{c}}}
%\newcommand{\ddot}[0]{{\dot{d}}}
\newcommand{\edot}[0]{{\dot{e}}}
\newcommand{\fdot}[0]{{\dot{f}}}
\newcommand{\gdot}[0]{{\dot{g}}}
\newcommand{\hdot}[0]{{\dot{h}}}
\newcommand{\idot}[0]{{\dot{i}}}
\newcommand{\jdot}[0]{{\dot{j}}}
\newcommand{\kdot}[0]{{\dot{k}}}
\newcommand{\ldot}[0]{{\dot{l}}}
\newcommand{\mdot}[0]{{\dot{m}}}
\newcommand{\ndot}[0]{{\dot{n}}}
%\newcommand{\odot}[0]{{\dot{o}}}
\newcommand{\pdot}[0]{{\dot{p}}}
\newcommand{\qdot}[0]{{\dot{q}}}
\newcommand{\rdot}[0]{{\dot{r}}}
\newcommand{\sdot}[0]{{\dot{s}}}
\newcommand{\tdot}[0]{{\dot{t}}}
\newcommand{\udot}[0]{{\dot{u}}}
\newcommand{\vdot}[0]{{\dot{v}}}
\newcommand{\wdot}[0]{{\dot{w}}}
\newcommand{\xdot}[0]{{\dot{x}}}
\newcommand{\ydot}[0]{{\dot{y}}}
\newcommand{\zdot}[0]{{\dot{z}}}
\newcommand{\addot}[0]{{\ddot{a}}}
\newcommand{\bddot}[0]{{\ddot{b}}}
\newcommand{\cddot}[0]{{\ddot{c}}}
%\newcommand{\dddot}[0]{{\ddot{d}}}
\newcommand{\eddot}[0]{{\ddot{e}}}
\newcommand{\fddot}[0]{{\ddot{f}}}
\newcommand{\gddot}[0]{{\ddot{g}}}
\newcommand{\hddot}[0]{{\ddot{h}}}
\newcommand{\iddot}[0]{{\ddot{i}}}
\newcommand{\jddot}[0]{{\ddot{j}}}
\newcommand{\kddot}[0]{{\ddot{k}}}
\newcommand{\lddot}[0]{{\ddot{l}}}
\newcommand{\mddot}[0]{{\ddot{m}}}
\newcommand{\nddot}[0]{{\ddot{n}}}
\newcommand{\oddot}[0]{{\ddot{o}}}
\newcommand{\pddot}[0]{{\ddot{p}}}
\newcommand{\qddot}[0]{{\ddot{q}}}
\newcommand{\rddot}[0]{{\ddot{r}}}
\newcommand{\sddot}[0]{{\ddot{s}}}
\newcommand{\tddot}[0]{{\ddot{t}}}
\newcommand{\uddot}[0]{{\ddot{u}}}
\newcommand{\vddot}[0]{{\ddot{v}}}
\newcommand{\wddot}[0]{{\ddot{w}}}
\newcommand{\xddot}[0]{{\ddot{x}}}
\newcommand{\yddot}[0]{{\ddot{y}}}
\newcommand{\zddot}[0]{{\ddot{z}}}

%<bold and dot greek symbols>
%

\newcommand{\Deltadot}[0]{{\dot{\Delta}}}
\newcommand{\Gammadot}[0]{{\dot{\Gamma}}}
\newcommand{\Lambdadot}[0]{{\dot{\Lambda}}}
\newcommand{\Omegadot}[0]{{\dot{\Omega}}}
\newcommand{\Phidot}[0]{{\dot{\Phi}}}
\newcommand{\Pidot}[0]{{\dot{\Pi}}}
\newcommand{\Psidot}[0]{{\dot{\Psi}}}
\newcommand{\Sigmadot}[0]{{\dot{\Sigma}}}
\newcommand{\Thetadot}[0]{{\dot{\Theta}}}
\newcommand{\Upsilondot}[0]{{\dot{\Upsilon}}}
\newcommand{\Xidot}[0]{{\dot{\Xi}}}
\newcommand{\alphadot}[0]{{\dot{\alpha}}}
\newcommand{\betadot}[0]{{\dot{\beta}}}
\newcommand{\chidot}[0]{{\dot{\chi}}}
\newcommand{\deltadot}[0]{{\dot{\delta}}}
\newcommand{\epsilondot}[0]{{\dot{\epsilon}}}
\newcommand{\etadot}[0]{{\dot{\eta}}}
\newcommand{\gammadot}[0]{{\dot{\gamma}}}
\newcommand{\kappadot}[0]{{\dot{\kappa}}}
\newcommand{\lambdadot}[0]{{\dot{\lambda}}}
\newcommand{\mudot}[0]{{\dot{\mu}}}
\newcommand{\nudot}[0]{{\dot{\nu}}}
\newcommand{\omegadot}[0]{{\dot{\omega}}}
\newcommand{\phidot}[0]{{\dot{\phi}}}
\newcommand{\pidot}[0]{{\dot{\pi}}}
\newcommand{\psidot}[0]{{\dot{\psi}}}
\newcommand{\rhodot}[0]{{\dot{\rho}}}
\newcommand{\sigmadot}[0]{{\dot{\sigma}}}
\newcommand{\taudot}[0]{{\dot{\tau}}}
\newcommand{\thetadot}[0]{{\dot{\theta}}}
\newcommand{\upsilondot}[0]{{\dot{\upsilon}}}
\newcommand{\varepsilondot}[0]{{\dot{\varepsilon}}}
\newcommand{\varphidot}[0]{{\dot{\varphi}}}
\newcommand{\varpidot}[0]{{\dot{\varpi}}}
\newcommand{\varrhodot}[0]{{\dot{\varrho}}}
\newcommand{\varsigmadot}[0]{{\dot{\varsigma}}}
\newcommand{\varthetadot}[0]{{\dot{\vartheta}}}
\newcommand{\xidot}[0]{{\dot{\xi}}}
\newcommand{\zetadot}[0]{{\dot{\zeta}}}

\newcommand{\Deltaddot}[0]{{\ddot{\Delta}}}
\newcommand{\Gammaddot}[0]{{\ddot{\Gamma}}}
\newcommand{\Lambdaddot}[0]{{\ddot{\Lambda}}}
\newcommand{\Omegaddot}[0]{{\ddot{\Omega}}}
\newcommand{\Phiddot}[0]{{\ddot{\Phi}}}
\newcommand{\Piddot}[0]{{\ddot{\Pi}}}
\newcommand{\Psiddot}[0]{{\ddot{\Psi}}}
\newcommand{\Sigmaddot}[0]{{\ddot{\Sigma}}}
\newcommand{\Thetaddot}[0]{{\ddot{\Theta}}}
\newcommand{\Upsilonddot}[0]{{\ddot{\Upsilon}}}
\newcommand{\Xiddot}[0]{{\ddot{\Xi}}}
\newcommand{\alphaddot}[0]{{\ddot{\alpha}}}
\newcommand{\betaddot}[0]{{\ddot{\beta}}}
\newcommand{\chiddot}[0]{{\ddot{\chi}}}
\newcommand{\deltaddot}[0]{{\ddot{\delta}}}
\newcommand{\epsilonddot}[0]{{\ddot{\epsilon}}}
\newcommand{\etaddot}[0]{{\ddot{\eta}}}
\newcommand{\gammaddot}[0]{{\ddot{\gamma}}}
\newcommand{\kappaddot}[0]{{\ddot{\kappa}}}
\newcommand{\lambdaddot}[0]{{\ddot{\lambda}}}
\newcommand{\muddot}[0]{{\ddot{\mu}}}
\newcommand{\nuddot}[0]{{\ddot{\nu}}}
\newcommand{\omegaddot}[0]{{\ddot{\omega}}}
\newcommand{\phiddot}[0]{{\ddot{\phi}}}
\newcommand{\piddot}[0]{{\ddot{\pi}}}
\newcommand{\psiddot}[0]{{\ddot{\psi}}}
\newcommand{\rhoddot}[0]{{\ddot{\rho}}}
\newcommand{\sigmaddot}[0]{{\ddot{\sigma}}}
\newcommand{\tauddot}[0]{{\ddot{\tau}}}
\newcommand{\thetaddot}[0]{{\ddot{\theta}}}
\newcommand{\upsilonddot}[0]{{\ddot{\upsilon}}}
\newcommand{\varepsilonddot}[0]{{\ddot{\varepsilon}}}
\newcommand{\varphiddot}[0]{{\ddot{\varphi}}}
\newcommand{\varpiddot}[0]{{\ddot{\varpi}}}
\newcommand{\varrhoddot}[0]{{\ddot{\varrho}}}
\newcommand{\varsigmaddot}[0]{{\ddot{\varsigma}}}
\newcommand{\varthetaddot}[0]{{\ddot{\vartheta}}}
\newcommand{\xiddot}[0]{{\ddot{\xi}}}
\newcommand{\zetaddot}[0]{{\ddot{\zeta}}}

\newcommand{\BDelta}[0]{\boldsymbol{\Delta}}
\newcommand{\BGamma}[0]{\boldsymbol{\Gamma}}
\newcommand{\BLambda}[0]{\boldsymbol{\Lambda}}
\newcommand{\BOmega}[0]{\boldsymbol{\Omega}}
\newcommand{\BPhi}[0]{\boldsymbol{\Phi}}
\newcommand{\BPi}[0]{\boldsymbol{\Pi}}
\newcommand{\BPsi}[0]{\boldsymbol{\Psi}}
\newcommand{\BSigma}[0]{\boldsymbol{\Sigma}}
\newcommand{\BTheta}[0]{\boldsymbol{\Theta}}
\newcommand{\BUpsilon}[0]{\boldsymbol{\Upsilon}}
\newcommand{\BXi}[0]{\boldsymbol{\Xi}}
\newcommand{\Balpha}[0]{\boldsymbol{\alpha}}
\newcommand{\Bbeta}[0]{\boldsymbol{\beta}}
\newcommand{\Bchi}[0]{\boldsymbol{\chi}}
\newcommand{\Bdelta}[0]{\boldsymbol{\delta}}
\newcommand{\Bepsilon}[0]{\boldsymbol{\epsilon}}
\newcommand{\Beta}[0]{\boldsymbol{\eta}}
\newcommand{\Bgamma}[0]{\boldsymbol{\gamma}}
\newcommand{\Bkappa}[0]{\boldsymbol{\kappa}}
\newcommand{\Blambda}[0]{\boldsymbol{\lambda}}
\newcommand{\Bmu}[0]{\boldsymbol{\mu}}
\newcommand{\Bnu}[0]{\boldsymbol{\nu}}
%\newcommand{\Bomega}[0]{\boldsymbol{\omega}}
\newcommand{\Bphi}[0]{\boldsymbol{\phi}}
\newcommand{\Bpi}[0]{\boldsymbol{\pi}}
\newcommand{\Bpsi}[0]{\boldsymbol{\psi}}
\newcommand{\Brho}[0]{\boldsymbol{\rho}}
\newcommand{\Bsigma}[0]{\boldsymbol{\sigma}}
%\newcommand{\Btau}[0]{\boldsymbol{\tau}}
%\newcommand{\Btheta}[0]{\boldsymbol{\theta}}
\newcommand{\Bupsilon}[0]{\boldsymbol{\upsilon}}
\newcommand{\Bvarepsilon}[0]{\boldsymbol{\varepsilon}}
\newcommand{\Bvarphi}[0]{\boldsymbol{\varphi}}
\newcommand{\Bvarpi}[0]{\boldsymbol{\varpi}}
\newcommand{\Bvarrho}[0]{\boldsymbol{\varrho}}
\newcommand{\Bvarsigma}[0]{\boldsymbol{\varsigma}}
\newcommand{\Bvartheta}[0]{\boldsymbol{\vartheta}}
\newcommand{\Bxi}[0]{\boldsymbol{\xi}}
\newcommand{\Bzeta}[0]{\boldsymbol{\zeta}}
%
%</bold and dot greek symbols>
%<infrequent>
%
%\newcommand{\AreaOp}[1]{\AName_{#1}}
%\newcommand{\Babs}[0]{\abs{\BB}}
%\newcommand{\Bcap}[0]{\hat{\BB}}
%\newcommand{\BrPrimeRej}[0]{\rcap(\rcap \wedge \Br')}
%\newcommand{\CA}[0]{\mathcal{A}}
%\newcommand{\Cos}[1]{\cos{\left({#1}\right)}}
%\newcommand{\Det}[1] {\abs{#1}}
%\newcommand{\Dsq}[2] {\frac {\partial^2 {#1}} {\partial {#2}^2}}
%\newcommand{\Exp}[1]{\exp{\left({#1}\right)}}
%\newcommand{\Norm}[1]{\left\lVert{#1}\right\rVert}
%\newcommand{\Sin}[1]{\sin{\left({#1}\right)}}
%\newcommand{\T}[0]{\text{T}}
%\newcommand{\VolumeOp}[1]{\VName_{#1}}
%\newcommand{\agrad}[0]{\Ba \cdot \nabla}
%\newcommand{\alphacap}[0]{\hat{\boldsymbol{\alpha}}}
%\newcommand{\Fcap}[0]{\hat{\BF}}
%\newcommand{\bithree}[0]{{\Bi}_3}
%\newcommand{\bxa}[0]{\Bx\Ba}
%\newcommand{\coordvec}[2]{
%\newcommand{\costheta}[0]{\acap \cdot \xcap}
%\newcommand{\ddt}[1]{\ddot{#1}}
%\newcommand{\ddu}[1] {\frac {d{#1}} {du}}
%\newcommand{\dsqxj}[2] {\frac {\partial^2 {#1}} {\partial {x_{#2}}^2}}
%\newcommand{\dtheta}[1]{\frac{d {#1}}{d \theta}}
%\newcommand{\dt}[1]{\dot{#1}}
%\newcommand{\dt}[1]{\frac{d {#1}}{dt}}
%\newcommand{\dxj}[2] {\frac {\partial {#1}} {\partial {x_{#2}}}}
%\newcommand{\halfPhi}[0]{\frac{\phi}{2}}
%\newcommand{\half}[0]{\inv{2}}
%\newcommand{\inv}[1]{\frac{1}{#1}}
%\newcommand{\laplacian}[0]{\nabla^2}
%\newcommand{\matrixoftx}[3]{
%\newcommand{\nrrp}[0]{\norm{\rcap \wedge \Br'}}
%\newcommand{\oiint}{\bigcirc \hspace{-1.4em} \int \hspace{-.8em} \int}
%\newcommand{\transpose}[1]{{#1}^{\text{T}}}
%\newcommand{\transpose}[1]{{{#1}^{\TextTranspose}}}
%\newcommand{\transpose}[1]{{{#1}^{\text{T}}}}
%\newcommand{\barA}[0]{\bar{A}}
%\newcommand{\qbar}[0]{\bar{q}}
%\newcommand{\qdotbar}[0]{\dot{\bar{q}}}
%
%</infrequent>





%\usepackage{listings}
%\usepackage{txfonts} % for ointctr... (also appears to make "prettier" \int and \sum's)
\usepackage[bookmarks=true]{hyperref}

%\usepackage{color,cite,graphicx}
   % use colour in the document, put your citations as [1-4]
   % rather than [1,2,3,4] (it looks nicer, and the extended LaTeX2e
   % graphics package. 
\usepackage{latexsym,amssymb,epsf} % don't remember if these are
   % needed, but their inclusion can't do any damage


%\title{}
%\author{Peeter Joot \quad peeter.joot@gmail.com }
%\date{ March dd, 2009.  Last Revision: $Date: 2009/06/03 20:01:38 $ }

\begin{document}
%\maketitle{}
%\tableofcontents
%\section{}

Your equation looks something like classical rigid body motion to me. ie: 
If the position of a particle is $x$ in a frame moving along a $\lambda$ parametrized path $a(\lambda)$, and that
frame is also allowed to rotate one could write something like the following for the
composite position of the particle

\begin{align*}
\bar{x} = R(x) + a(\lambda)
\end{align*}

Or in coordinates
\begin{align*}
\bar{x^\mu} = \Lambda^\mu_\nu x^\nu + a^\mu
\end{align*}

Derivatives with respect to the parameter (which I am thinking of as time or proper time), are then

\begin{align*}
\frac{d\bar{x^\mu}}{d\lambda} 
&= 
\frac{d \Lambda^\mu_\nu}{d\lambda} x^\nu 
+ \Lambda^\mu_\nu \frac{d x^\nu}{d\lambda}
+ \frac{d a^\mu}{d\lambda}
\end{align*}

As a differential displacement, one could write

\begin{align*}
d\bar{x^\mu}
&= x^\mu(\lambda + \delta\lambda) - x^\mu(\lambda) \\
&= 
d\lambda \left(
\frac{d \Lambda^\mu_\nu}{d\lambda} x^\nu 
+ \Lambda^\mu_\nu \xdot^\nu
+ \adot^\mu
\right)
\end{align*}

Or
\begin{align}\label{eqn:part1}
\bar{x}^\mu(\lambda + \delta\lambda) 
&= \bar{x}^\mu(\lambda)
+ d\lambda \frac{d \Lambda^\mu_\nu}{d\lambda} x^\nu 
+ d\lambda \Lambda^\mu_\nu \xdot^\nu
+ d\lambda \adot^\mu
\end{align}

This gives me a couple more terms than in your email
and also requires a different identification of the positions in the two frames (ie: $\bar{x}$, and $x$ vectors).

In particular, I have a variation of the rotation along the worldline $\delta \Lambda^\mu_\nu = d\lambda d\Lambda^\mu_\nu/d\lambda$ that is missing
in your equation.  Does GR give you a constraint that allows this to vanish?

Right before sending the email with this, it occurs to me that the body frame coordinates can be eliminated.  Let the inverse transformation be given by $\Pi^\sigma_\mu$ as follows

\begin{align*}
\Pi^\sigma_\mu \Lambda^\mu_\nu = \delta^\sigma_\nu
\end{align*}

Then one can write
\begin{align*}
\Pi^\nu_\alpha(\bar{x^\alpha} - a^\alpha)
&= \Pi^\nu_\alpha \Lambda^\alpha_\beta x^\beta  \\
&= x^\nu  \\
\end{align*}

and \ref{eqn:part1} becomes

\begin{align*}
\bar{x}^\mu(\lambda + \delta\lambda) 
&= 
\bar{x}^\mu(\lambda)
+ d\lambda \frac{d \Lambda^\mu_\nu}{d\lambda} \left( \Pi^\nu_\alpha(\bar{x^\alpha} - a^\alpha) \right)
+ d\lambda \Lambda^\mu_\nu \xdot^\nu
+ d\lambda \adot^\mu \\
\end{align*}

\begin{align}\label{eqn:partII}
\bar{x}^\mu(\lambda + \delta\lambda) 
&= 
\bar{x}^\alpha(\lambda) \left( \delta^\mu_\alpha + d\lambda \Pi^\nu_\alpha \frac{d \Lambda^\mu_\nu}{d\lambda} \right)
- d\lambda \frac{d \Lambda^\mu_\nu}{d\lambda} a^\alpha
+ d\lambda \Lambda^\mu_\nu \xdot^\nu
+ d\lambda \adot^\mu \\
\end{align}

Note that since $\Pi^\sigma_\mu \Lambda^\mu_\nu = \delta^\sigma_\nu$, one has

\begin{align*}
0 
&= \frac{d}{d\lambda} \left( \Pi^\sigma_\mu \Lambda^\mu_\nu \right) \\
&= \Lambda^\mu_\nu \frac{d}{d\lambda} \Pi^\sigma_\mu + \Pi^\sigma_\mu \frac{d}{d\lambda} \Lambda^\mu_\nu 
\end{align*}

Believe this implies that the contracted inverse/derivative product $\Pi^\nu_\alpha \frac{d \Lambda^\mu_\nu}{d\lambda}$ is antisymmetric.

There's still mixed coordinates in \ref{eqn:partII} (ie: both $\bar{x}$, and $x$) and I think that the final result for an incremental displacement
should likely eliminate that too (but I haven't tried).

%\begin{figure}[htp]
%\centering
%\includegraphics[totalheight=0.4\textheight]{picturepath}
%\caption{My Caption}\label{fig:pictlabel}
%\end{figure}
%
%... see figure \ref{fig:picturepath} ...

%\bibliographystyle{plainnat}
%\bibliography{myrefs}

\end{document}


\part{maxwell}
\documentclass{article}      % Specifies the document class

\usepackage{amsmath}
\newcommand{\abs}[1]{\lvert#1\rvert}
\newcommand{\norm}[1]{\lVert#1\rVert}
\newcommand{\grad}[1]{\nabla#1}
\newcommand{\curl}[1]{\nabla \times #1}
\newcommand{\Curl}[1]{\nabla \times \mathbf{#1}}
\newcommand{\diverg}[1]{\nabla \cdot #1}
\newcommand{\Diverg}[1]{\nabla \cdot \mathbf{#1}}
\newcommand{\curlcurl}[1]{\curl\curl{#1}}
\newcommand{\curlCurl}[1]{\curl\Curl{#1}}
\newcommand{\delsquared}[1]{\nabla^2{#1}}
\newcommand{\Delsquared}[1]{{\nabla^2}{\mathbf{#1}}}
\newcommand{\ddt}[1]{ {{\partial{#1}} \over {\partial{t}}}}
\newcommand{\Ddt}[1]{ {{\partial{\mathbf{#1}}} \over {\partial{t}}}}
\newcommand{\ddts}[1]{ {{\partial^2{#1}} \over {\partial{t}^2}}}
\newcommand{\Ddts}[1]{ {{\partial^2{\mathbf{#1}}} \over {\partial{t}^2}}}
\newcommand{\Bj}[0]{\mathbf{j}}
\newcommand{\BB}[0]{\mathbf{B}}
\newcommand{\BE}[0]{\mathbf{E}}

                             % The preamble begins here.
\title{An Example Document}  % Declares the document's title.
\author{Peeter Joot}         % Declares the author's name.
\date{March 25, 2000}        % Deleting this command produces today's date.

\begin{document}             % End of preamble and beginning of text.

%\maketitle                  % Produces the title.

\section{various formulations of Maxwell's equations}

It is interesting to look at the various formulations of Maxwell's 
equations.  The standard formulation these days is the differential
form, which isn't the most intuitive form.  In cgs units, the differential
form is as follows:

\begin{equation*}
\Diverg E = 4\pi\rho
\end{equation*}
\begin{equation*}
\Diverg B = 0
\end{equation*}
\begin{equation*}
\Curl{B} = 4\pi \Bj - {1 \over c} \Ddt{E}
\end{equation*}
\begin{equation*}
\Curl{E} = {1 \over c} \Ddt{B}
\end{equation*}

It is interesting to note that these are probably not the form that Maxwell 
originally formulated his equations in.  In my 1960 version of encyclodedia 
britianica Maxwell's equations were given in a differential form.  I had 
trouble seeing how these two forms were related at first, but the relation
is via the standard integral transformations.  For example, the formula $\Diverg E = 4\pi\rho$ is an alternate formulation of Gauss' law
\begin{equation*}
\BE = \int_V{{\rho\,dV' \over \norm{\mathbf r - \mathbf r'}}}
\end{equation*}

This can be shown via application of Gauss's theorm.  Similarily the following 
integral forms of Maxwell's equations:

%[ see paper notes ].

One of the awkward things with this standard formulation is that there 
is an awkward interdependence between the electric and magnetic fields, and 
the solution of either $\BE$ or $\BB$ interdependently seems difficult.

The interdependence of the $\BE$ and $\BB$ field equations can be removed by taking the
curl of each of the curl equations, and using the fact that:

$\curlCurl{V} = \grad (\Diverg V) - \Delsquared{V}$

\begin{align*}
\curlCurl{B} &= \grad (\Diverg{B}) - \Delsquared{B} \\
\curl(4\pi\Bj - {1 \over c} \Ddt{E}) &= \\
4\pi \Curl{j} - {1 \over c^2} \Ddts{B} &= \\
      	     &= 		   - \Delsquared{B}
\end{align*}

We end up with a single differential equation for the magnetic field $\BB$.

$\delsquared{\BB} - {1 \over c^2} \Ddts{B} = - 4\pi \Curl{j}$

We can do the same calculation for the electric field.

\begin{align*}
\curlCurl{E} &= \grad (\Diverg{E}) - \Delsquared{E} \\
\curl({1 \over c} \Ddt{B}) &= \\
{1 \over c} \ddt(4\pi \Bj - {1 \over c} \Ddt{E}) &= \\
{4\pi \over c} \Ddt{j} - {1 \over c^2} \Ddts{E} &= \\
             &= \grad (4\pi\rho) - \Delsquared{E} \\
\end{align*}

which gives a single differential equation for the electric field $\BE$.

%$\Delsquared{E} - {1 \over c^2} \Ddt{E} = {4\pi \over c} \Ddt{j} - 4\pi \grad{\rho}$
$\Delsquared{E} - {1 \over c^2} \Ddt{E}$
$ = {4\pi \over c} \Ddt{j} - 4\pi \grad{\rho}$

In the absense of current density and charge these equations take the simple form of standard wave
equations.

\begin{equation*}
\Delsquared{E} - {1 \over c^2} \Ddt{E} = 0
\end{equation*}
\begin{equation*}
\Delsquared{B} - {1 \over c^2} \Ddt{B} = 0
\end{equation*}

Any solution to these is also a solution to the original form where there is current density $\Bj$, and 
charge density $\rho$ since 

\end{document}               % End of document.


\part{mechanics}
\documentclass{article}      % Specifies the document class

\usepackage{amsmath}
\usepackage{mathpazo}

%
% shorthand for bold symbols, convenient for vectors and matrices
%
\newcommand{\Ba}[0]{\mathbf{a}}
\newcommand{\Bb}[0]{\mathbf{b}}
\newcommand{\Bc}[0]{\mathbf{c}}
\newcommand{\Bd}[0]{\mathbf{d}}
\newcommand{\Be}[0]{\mathbf{e}}
\newcommand{\Bf}[0]{\mathbf{f}}
\newcommand{\Bg}[0]{\mathbf{g}}
\newcommand{\Bh}[0]{\mathbf{h}}
\newcommand{\Bi}[0]{\mathbf{i}}
\newcommand{\Bj}[0]{\mathbf{j}}
\newcommand{\Bk}[0]{\mathbf{k}}
\newcommand{\Bl}[0]{\mathbf{l}}
\newcommand{\Bm}[0]{\mathbf{m}}
\newcommand{\Bn}[0]{\mathbf{n}}
\newcommand{\Bo}[0]{\mathbf{o}}
\newcommand{\Bp}[0]{\mathbf{p}}
\newcommand{\Bq}[0]{\mathbf{q}}
\newcommand{\Br}[0]{\mathbf{r}}
\newcommand{\Bs}[0]{\mathbf{s}}
\newcommand{\Bt}[0]{\mathbf{t}}
\newcommand{\Bu}[0]{\mathbf{u}}
\newcommand{\Bv}[0]{\mathbf{v}}
\newcommand{\Bw}[0]{\mathbf{w}}
\newcommand{\Bx}[0]{\mathbf{x}}
\newcommand{\By}[0]{\mathbf{y}}
\newcommand{\Bz}[0]{\mathbf{z}}
\newcommand{\BA}[0]{\mathbf{A}}
\newcommand{\BB}[0]{\mathbf{B}}
\newcommand{\BC}[0]{\mathbf{C}}
\newcommand{\BD}[0]{\mathbf{D}}
\newcommand{\BE}[0]{\mathbf{E}}
\newcommand{\BF}[0]{\mathbf{F}}
\newcommand{\BG}[0]{\mathbf{G}}
\newcommand{\BH}[0]{\mathbf{H}}
\newcommand{\BI}[0]{\mathbf{I}}
\newcommand{\BJ}[0]{\mathbf{J}}
\newcommand{\BK}[0]{\mathbf{K}}
\newcommand{\BL}[0]{\mathbf{L}}
\newcommand{\BM}[0]{\mathbf{M}}
\newcommand{\BN}[0]{\mathbf{N}}
\newcommand{\BO}[0]{\mathbf{O}}
\newcommand{\BP}[0]{\mathbf{P}}
\newcommand{\BQ}[0]{\mathbf{Q}}
\newcommand{\BR}[0]{\mathbf{R}}
\newcommand{\BS}[0]{\mathbf{S}}
\newcommand{\BT}[0]{\mathbf{T}}
\newcommand{\BU}[0]{\mathbf{U}}
\newcommand{\BV}[0]{\mathbf{V}}
\newcommand{\BW}[0]{\mathbf{W}}
\newcommand{\BX}[0]{\mathbf{X}}
\newcommand{\BY}[0]{\mathbf{Y}}
\newcommand{\BZ}[0]{\mathbf{Z}}

\newcommand{\Bzero}[0]{\mathbf{0}}
\newcommand{\Btheta}[0]{\boldsymbol{\theta}}
\newcommand{\Btau}[0]{\boldsymbol{\tau}}
\newcommand{\Bomega}[0]{\boldsymbol{\omega}}

%
% shorthand for unit vectors
%
\newcommand{\acap}[0]{\hat{\Ba}}
\newcommand{\bcap}[0]{\hat{\Bb}}
\newcommand{\ccap}[0]{\hat{\Bc}}
\newcommand{\dcap}[0]{\hat{\Bd}}
\newcommand{\ecap}[0]{\hat{\Be}}
\newcommand{\fcap}[0]{\hat{\Bf}}
\newcommand{\gcap}[0]{\hat{\Bg}}
\newcommand{\hcap}[0]{\hat{\Bh}}
\newcommand{\icap}[0]{\hat{\Bi}}
\newcommand{\jcap}[0]{\hat{\Bj}}
\newcommand{\kcap}[0]{\hat{\Bk}}
\newcommand{\lcap}[0]{\hat{\Bl}}
\newcommand{\mcap}[0]{\hat{\Bm}}
\newcommand{\ncap}[0]{\hat{\Bn}}
\newcommand{\ocap}[0]{\hat{\Bo}}
\newcommand{\pcap}[0]{\hat{\Bp}}
\newcommand{\qcap}[0]{\hat{\Bq}}
\newcommand{\rcap}[0]{\hat{\Br}}
\newcommand{\scap}[0]{\hat{\Bs}}
\newcommand{\tcap}[0]{\hat{\Bt}}
\newcommand{\ucap}[0]{\hat{\Bu}}
\newcommand{\vcap}[0]{\hat{\Bv}}
\newcommand{\wcap}[0]{\hat{\Bw}}
\newcommand{\xcap}[0]{\hat{\Bx}}
\newcommand{\ycap}[0]{\hat{\By}}
\newcommand{\zcap}[0]{\hat{\Bz}}
\newcommand{\thetacap}[0]{\hat{\Btheta}}

%
% to write R^n and C^n in a distinguishable fashion.  Perhaps change this
% to the double lined characters upon figuring out how to do so.
%
\newcommand{\C}[1]{$\mathbb{C}^{#1}$}
\newcommand{\R}[1]{$\mathbb{R}^{#1}$}

%
% various generally useful helpers
%

% derivative of #1 wrt. #2:
\newcommand{\D}[2] {\frac {d#2} {d#1}}

\newcommand{\inv}[1]{\frac{1}{#1}}
\newcommand{\cross}[0]{\times}

\newcommand{\abs}[1]{\lvert{#1}\rvert}
\newcommand{\norm}[1]{\lVert{#1}\rVert}
\newcommand{\innerprod}[2]{\langle{#1}, {#2}\rangle}
\newcommand{\dotprod}[2]{{#1} \cdot {#2}}
\newcommand{\bdotprod}[2]{\left({#1} \cdot {#2}\right)}
\newcommand{\crossprod}[2]{{#1} \cross {#2}}
\newcommand{\tripleprod}[3]{\dotprod{\left(\crossprod{#1}{#2}\right)}{#3}}

\DeclareMathOperator{\Proj}{Proj}
\DeclareMathOperator{\Span}{span}
\DeclareMathOperator{\Sgn}{sgn}
\DeclareMathOperator{\Area}{Area}
\DeclareMathOperator{\Volume}{Volume}

%
% A few miscellaneous things specific to this document
%
\newcommand{\crossop}[1]{\crossprod{#1}{}}

% R2 vector.
\newcommand{\VectorTwo}[2]{
\begin{bmatrix}
 {#1} \\
 {#2}
\end{bmatrix}
}

\newcommand{\VectorN}[1]{
\begin{bmatrix}
{#1}_1 \\
{#1}_2 \\
\vdots \\
{#1}_N \\
\end{bmatrix}
}

\newcommand{\DETuvij}[4]{
\begin{vmatrix}
 {#1}_{#3} & {#1}_{#4} \\
 {#2}_{#3} & {#2}_{#4}
\end{vmatrix}
}

\newcommand{\DETuvwijk}[6]{
\begin{vmatrix}
 {#1}_{#4} & {#1}_{#5} & {#1}_{#6} \\
 {#2}_{#4} & {#2}_{#5} & {#2}_{#6} \\
 {#3}_{#4} & {#3}_{#5} & {#3}_{#6}
\end{vmatrix}
}

\newcommand{\DETuvwxijkl}[8]{
\begin{vmatrix}
 {#1}_{#5} & {#1}_{#6} & {#1}_{#7} & {#1}_{#8} \\
 {#2}_{#5} & {#2}_{#6} & {#2}_{#7} & {#2}_{#8} \\
 {#3}_{#5} & {#3}_{#6} & {#3}_{#7} & {#3}_{#8} \\
 {#4}_{#5} & {#4}_{#6} & {#4}_{#7} & {#4}_{#8} \\
\end{vmatrix}
}

%\newcommand{\DETuvwxyijklm}[10]{
%\begin{vmatrix}
% {#1}_{#6} & {#1}_{#7} & {#1}_{#8} & {#1}_{#9} & {#1}_{#10} \\
% {#2}_{#6} & {#2}_{#7} & {#2}_{#8} & {#2}_{#9} & {#2}_{#10} \\
% {#3}_{#6} & {#3}_{#7} & {#3}_{#8} & {#3}_{#9} & {#3}_{#10} \\
% {#4}_{#6} & {#4}_{#7} & {#4}_{#8} & {#4}_{#9} & {#4}_{#10} \\
% {#5}_{#6} & {#5}_{#7} & {#5}_{#8} & {#5}_{#9} & {#5}_{#10}
%\end{vmatrix}
%}

% R3 vector.
\newcommand{\VectorThree}[3]{
\begin{bmatrix}
 {#1} \\
 {#2} \\
 {#3}
\end{bmatrix}
}


\newcommand{\grad}[0]{\nabla}

%
% The real thing:
%

\title{ Potential and Kinetic Energy.} 
\author{Peeter Joot}         
\date{}

\begin{document}             

\maketitle{}

\section{}

Attempting some Lagrangian calculation problems I found I got all the signs of my potential energy terms wrong.  Here's a quick step back to basics to clarify for myself what the definition of potential energy is, and thus implicitly determine the correct signs.

Starting with kinetic energy, expressed in vector form:

\begin{equation*}
K 
= \inv{2} m \Br' \cdot \Br' 
= \inv{2} \Bp \cdot \Br',
\end{equation*}

one can calculate the rate of change of that energy:

\begin{align*}
\frac{dK }{dt}
&= \inv{2} \left(\Bp' \cdot \Br' + \Bp \cdot \Br''\right) \\
&= \inv{2} \left(\Bp' \cdot \Br' + \Br' \cdot \Bp'\right) \\
&= \Bp' \cdot \Br'.
\end{align*}

Note that the mass has been assumed constant above.

Integrating this time rate of change of kinetic energy produces a force
line 
integral:

\begin{align*}
K_2 - K_1 
&= \int_{t1}^{t2} \frac{dK}{dt} dt \\
&= \int_{t1}^{t2} \Bp' \cdot \Br' dt \\
&= \int_{t1}^{t2} \Bp' \cdot \frac{d\Br'}{dt} dt \\
&= \int_{\Br_1}^{\Br_2} \BF \cdot d\Br
\end{align*}

For the path integral to depend on only the end points or the corresponding end times requires a conservative force that can be expressed as a gradient.
Let's say that $\BF = \grad f$, then integrating:

\begin{align*}
K_2 - K_1 
&= \int_{\Br_1}^{\Br_2} \BF \cdot d\Br \\
&= \int_{\Br_1}^{\Br_2} \grad f \cdot d\Br \\
&= {\text{limit}}_{\epsilon \rightarrow 0} \int_{\Br_1}^{\Br_1 + \epsilon\rcap}
   \left(\rcap \frac{f(\Br + \epsilon \rcap)}{\epsilon}\right)
      \cdot d\Br \\
&= \text{handwaving} \\
&= f(\Br_2) - f(\Br_1).
\end{align*}

Assembling the quantities for times $1$, and $2$, we have

\begin{equation}
K_2 -f(\Br_2) = K_1 - f(\Br_1) = \text{constant}.
\end{equation}

This constant is what we give the name Energy.  The quantities $-f(\Br_i)$ we label potential energy $V_i$, and finally write the total energy as the sum of the kinetic and potential energies for a particle at a point in time and space:

\begin{equation}
K_2 + V_2 = K_1 + V_1 = E
\end{equation}
\begin{equation}
\BF = -\grad V
\end{equation}

\subsection{ Work with a specific example.  Newtonian gravitational force.}

Take the gravitional force:

\begin{equation}
F = -\frac{GmM}{r^2} \rcap
\end{equation}

The rate of change of kinetic energy with respect to such a force (FIXME: think though signs ... with or against?), is:

\begin{align*}
\frac{dK}{dt} 
&= \Bp' \cdot \Br' \\
&= -\frac{GmM}{r^2} \rcap \cdot \frac{d\Br}{dt} \\
&= -\frac{GmM}{r^3} \Br \cdot \frac{d\Br}{dt}.
\end{align*}

The vector dot products above can be eliminated with the standard trick:

\begin{align*}
\frac{dr^2}{dt} 
&= \frac{\Br \cdot \Br}{dt} \\
&= 2 \frac{d\Br}{dt} \cdot \Br.
\end{align*}

Thus,
\begin{align*}
\frac{dK}{dt} 
&= -\frac{GmM }{2r^3} \frac{dr^2}{dt} \\
&= -\frac{GmM }{r^2} \frac{dr}{dt} \\
&= \frac{d}{dt} \left( \frac{GmM }{r} \right).
\end{align*}

This can be integrated to find the kinetic energy difference associated with a change of position in a gravitational field:

\begin{align*}
K_2 - K_1 
&= \int_{t_1}^{t_2} \frac{d}{dt} \left( \frac{GmM }{r} \right) dt \\
&= GmM \left( \inv{r_2} - \inv{r_1} \right).
\end{align*}

Or, 

\begin{align*}
K_2 - \frac{GmM}{r_2} = K_1 - \frac{GmM}{r_1} = E.
\end{align*}

Taking gradients of this negative term:

\begin{align*}
\grad \left( - \frac{GmM}{r} \right)
&= \rcap \frac{\partial}{\partial r} \left( - \frac{GmM}{r} \right) \\
&= \rcap \frac{GmM}{r^2},
\end{align*}

returns the negation of the original force, so if we write $V = -Gmm/r^2$, it implies the force is:

\begin{equation}
\BF = -\grad V.
\end{equation}

By this example we see how one arrives at the negative sign convention for the potential energy, and also how we end up with strictly positive terms in the Lagrangian associated with a gravitational force:

\begin{equation}
L = \inv{2} m \Bv^2 + \frac{GmM}{r}.
\end{equation}

\end{document}               

\documentclass{article}      % Specifies the document class

\usepackage{amsmath}
\usepackage{mathpazo}

%
% shorthand for bold symbols, convenient for vectors and matrices
%
\newcommand{\Ba}[0]{\mathbf{a}}
\newcommand{\Bb}[0]{\mathbf{b}}
\newcommand{\Bc}[0]{\mathbf{c}}
\newcommand{\Bd}[0]{\mathbf{d}}
\newcommand{\Be}[0]{\mathbf{e}}
\newcommand{\Bf}[0]{\mathbf{f}}
\newcommand{\Bg}[0]{\mathbf{g}}
\newcommand{\Bh}[0]{\mathbf{h}}
\newcommand{\Bi}[0]{\mathbf{i}}
\newcommand{\Bj}[0]{\mathbf{j}}
\newcommand{\Bk}[0]{\mathbf{k}}
\newcommand{\Bl}[0]{\mathbf{l}}
\newcommand{\Bm}[0]{\mathbf{m}}
\newcommand{\Bn}[0]{\mathbf{n}}
\newcommand{\Bo}[0]{\mathbf{o}}
\newcommand{\Bp}[0]{\mathbf{p}}
\newcommand{\Bq}[0]{\mathbf{q}}
\newcommand{\Br}[0]{\mathbf{r}}
\newcommand{\Bs}[0]{\mathbf{s}}
\newcommand{\Bt}[0]{\mathbf{t}}
\newcommand{\Bu}[0]{\mathbf{u}}
\newcommand{\Bv}[0]{\mathbf{v}}
\newcommand{\Bw}[0]{\mathbf{w}}
\newcommand{\Bx}[0]{\mathbf{x}}
\newcommand{\By}[0]{\mathbf{y}}
\newcommand{\Bz}[0]{\mathbf{z}}
\newcommand{\BA}[0]{\mathbf{A}}
\newcommand{\BB}[0]{\mathbf{B}}
\newcommand{\BC}[0]{\mathbf{C}}
\newcommand{\BD}[0]{\mathbf{D}}
\newcommand{\BE}[0]{\mathbf{E}}
\newcommand{\BF}[0]{\mathbf{F}}
\newcommand{\BG}[0]{\mathbf{G}}
\newcommand{\BH}[0]{\mathbf{H}}
\newcommand{\BI}[0]{\mathbf{I}}
\newcommand{\BJ}[0]{\mathbf{J}}
\newcommand{\BK}[0]{\mathbf{K}}
\newcommand{\BL}[0]{\mathbf{L}}
\newcommand{\BM}[0]{\mathbf{M}}
\newcommand{\BN}[0]{\mathbf{N}}
\newcommand{\BO}[0]{\mathbf{O}}
\newcommand{\BP}[0]{\mathbf{P}}
\newcommand{\BQ}[0]{\mathbf{Q}}
\newcommand{\BR}[0]{\mathbf{R}}
\newcommand{\BS}[0]{\mathbf{S}}
\newcommand{\BT}[0]{\mathbf{T}}
\newcommand{\BU}[0]{\mathbf{U}}
\newcommand{\BV}[0]{\mathbf{V}}
\newcommand{\BW}[0]{\mathbf{W}}
\newcommand{\BX}[0]{\mathbf{X}}
\newcommand{\BY}[0]{\mathbf{Y}}
\newcommand{\BZ}[0]{\mathbf{Z}}

\newcommand{\Bzero}[0]{\mathbf{0}}
\newcommand{\Btheta}[0]{\boldsymbol{\theta}}
\newcommand{\Btau}[0]{\boldsymbol{\tau}}
\newcommand{\Bomega}[0]{\boldsymbol{\omega}}

%
% shorthand for unit vectors
%
\newcommand{\acap}[0]{\hat{\Ba}}
\newcommand{\bcap}[0]{\hat{\Bb}}
\newcommand{\ccap}[0]{\hat{\Bc}}
\newcommand{\dcap}[0]{\hat{\Bd}}
\newcommand{\ecap}[0]{\hat{\Be}}
\newcommand{\fcap}[0]{\hat{\Bf}}
\newcommand{\gcap}[0]{\hat{\Bg}}
\newcommand{\hcap}[0]{\hat{\Bh}}
\newcommand{\icap}[0]{\hat{\Bi}}
\newcommand{\jcap}[0]{\hat{\Bj}}
\newcommand{\kcap}[0]{\hat{\Bk}}
\newcommand{\lcap}[0]{\hat{\Bl}}
\newcommand{\mcap}[0]{\hat{\Bm}}
\newcommand{\ncap}[0]{\hat{\Bn}}
\newcommand{\ocap}[0]{\hat{\Bo}}
\newcommand{\pcap}[0]{\hat{\Bp}}
\newcommand{\qcap}[0]{\hat{\Bq}}
\newcommand{\rcap}[0]{\hat{\Br}}
\newcommand{\scap}[0]{\hat{\Bs}}
\newcommand{\tcap}[0]{\hat{\Bt}}
\newcommand{\ucap}[0]{\hat{\Bu}}
\newcommand{\vcap}[0]{\hat{\Bv}}
\newcommand{\wcap}[0]{\hat{\Bw}}
\newcommand{\xcap}[0]{\hat{\Bx}}
\newcommand{\ycap}[0]{\hat{\By}}
\newcommand{\zcap}[0]{\hat{\Bz}}
\newcommand{\thetacap}[0]{\hat{\Btheta}}

%
% to write R^n and C^n in a distinguishable fashion.  Perhaps change this
% to the double lined characters upon figuring out how to do so.
%
\newcommand{\C}[1]{$\mathbb{C}^{#1}$}
\newcommand{\R}[1]{$\mathbb{R}^{#1}$}

%
% various generally useful helpers
%

% derivative of #1 wrt. #2:
\newcommand{\D}[2] {\frac {d#2} {d#1}}

\newcommand{\inv}[1]{\frac{1}{#1}}
\newcommand{\cross}[0]{\times}

\newcommand{\abs}[1]{\lvert{#1}\rvert}
\newcommand{\norm}[1]{\lVert{#1}\rVert}
\newcommand{\innerprod}[2]{\langle{#1}, {#2}\rangle}
\newcommand{\dotprod}[2]{{#1} \cdot {#2}}
\newcommand{\bdotprod}[2]{\left({#1} \cdot {#2}\right)}
\newcommand{\crossprod}[2]{{#1} \cross {#2}}
\newcommand{\tripleprod}[3]{\dotprod{\left(\crossprod{#1}{#2}\right)}{#3}}

\DeclareMathOperator{\Proj}{Proj}
\DeclareMathOperator{\Span}{span}
\DeclareMathOperator{\Sgn}{sgn}
\DeclareMathOperator{\Area}{Area}
\DeclareMathOperator{\Volume}{Volume}

%
% A few miscellaneous things specific to this document
%
\newcommand{\crossop}[1]{\crossprod{#1}{}}

% R2 vector.
\newcommand{\VectorTwo}[2]{
\begin{bmatrix}
 {#1} \\
 {#2}
\end{bmatrix}
}

\newcommand{\VectorN}[1]{
\begin{bmatrix}
{#1}_1 \\
{#1}_2 \\
\vdots \\
{#1}_N \\
\end{bmatrix}
}

\newcommand{\DETuvij}[4]{
\begin{vmatrix}
 {#1}_{#3} & {#1}_{#4} \\
 {#2}_{#3} & {#2}_{#4}
\end{vmatrix}
}

\newcommand{\DETuvwijk}[6]{
\begin{vmatrix}
 {#1}_{#4} & {#1}_{#5} & {#1}_{#6} \\
 {#2}_{#4} & {#2}_{#5} & {#2}_{#6} \\
 {#3}_{#4} & {#3}_{#5} & {#3}_{#6}
\end{vmatrix}
}

\newcommand{\DETuvwxijkl}[8]{
\begin{vmatrix}
 {#1}_{#5} & {#1}_{#6} & {#1}_{#7} & {#1}_{#8} \\
 {#2}_{#5} & {#2}_{#6} & {#2}_{#7} & {#2}_{#8} \\
 {#3}_{#5} & {#3}_{#6} & {#3}_{#7} & {#3}_{#8} \\
 {#4}_{#5} & {#4}_{#6} & {#4}_{#7} & {#4}_{#8} \\
\end{vmatrix}
}

%\newcommand{\DETuvwxyijklm}[10]{
%\begin{vmatrix}
% {#1}_{#6} & {#1}_{#7} & {#1}_{#8} & {#1}_{#9} & {#1}_{#10} \\
% {#2}_{#6} & {#2}_{#7} & {#2}_{#8} & {#2}_{#9} & {#2}_{#10} \\
% {#3}_{#6} & {#3}_{#7} & {#3}_{#8} & {#3}_{#9} & {#3}_{#10} \\
% {#4}_{#6} & {#4}_{#7} & {#4}_{#8} & {#4}_{#9} & {#4}_{#10} \\
% {#5}_{#6} & {#5}_{#7} & {#5}_{#8} & {#5}_{#9} & {#5}_{#10}
%\end{vmatrix}
%}

% R3 vector.
\newcommand{\VectorThree}[3]{
\begin{bmatrix}
 {#1} \\
 {#2} \\
 {#3}
\end{bmatrix}
}


\newcommand{\Brho}[0]{\boldsymbol{\rho}}
\newcommand{\LL}[0]{\mathcal{L}}
\newcommand{\Abs}[1]{\left\lvert{#1}\right\rvert}
\newcommand{\qdot}[0]{\dot{q}}
\newcommand{\qddot}[0]{\ddot{q}}
\newcommand{\xdot}[0]{\dot{x}}
\newcommand{\xddot}[0]{\ddot{x}}
\newcommand{\ydot}[0]{\dot{y}}
\newcommand{\yddot}[0]{\ddot{y}}
\newcommand{\dotalpha}[0]{\dot{\alpha}}
\newcommand{\ddotalpha}[0]{\ddot{\alpha}}
\newcommand{\dottheta}[0]{\dot{\theta}}
\newcommand{\ddottheta}[0]{\ddot{\theta}}
\newcommand{\dotphi}[0]{\dot{\phi}}
\newcommand{\ddotphi}[0]{\ddot{\phi}}
% == \partial_{#1} {#2}
\newcommand{\PD}[2]{\frac{\partial {#2}}{\partial {#1}}}
\newcommand{\PDD}[3]{\frac{\partial^2 {#3}}{\partial {#1}\partial {#2}}}

%
% The real thing:
%

                             % The preamble begins here.
\title{Attempts at solutions for some Goldstein Mechanics problems.} % Declares the document's title.
\author{Peeter Joot}         % Declares the author's name.
\date{ }        % Deleting this command produces today's date.

\begin{document}             % End of preamble and beginning of text.

\maketitle{}

\section{ Problem 1.7 }

Barbell shape, equal masses.  center of rod between masses constrained to circular motion.

Assuming motion in a plane, the equation for the center of the rod is:

\begin{equation*}
c = a e^{i\theta}
\end{equation*}

and the two mass points positions are:
\begin{align*}
q_1 &= c + (l/2) e^{i\alpha} \\
q_2 &= c - (l/2) e^{i\alpha}
\end{align*}

taking derivatives:
\begin{align*}
\qdot_1 &= a i \dottheta e^{i\theta} + (l/2) i \dotalpha e^{i\alpha} \\
\qdot_2 &= a i \dottheta e^{i\theta} - (l/2) i \dotalpha e^{i\alpha} \\
\end{align*}

and squared magnitudes:

\begin{align*}
\qdot_{\pm}
&= \Abs{a \dottheta \pm (l/2) \dotalpha e^{i(\alpha - \theta)}}^2 \\
&= \left(a \dottheta   \pm   \inv{2} l \dotalpha \cos(\alpha - \theta)\right)^2 + \left(\inv{2} l \dotalpha \sin(\alpha - \theta)\right)^2
\end{align*}

Summing the kinetic terms yeilds

\begin{equation*}
K = m \left(a \dottheta \right)^2 + m \left(\inv{2} l \dotalpha\right)^2
\end{equation*}

Summing the potential energies, presuming that the motion is verticle, we have:

\begin{equation*}
V = m g (l/2) \cos\theta - m g (l/2) \cos \theta
\end{equation*}

So, the Lagrangian is just the Kinetic energy.

Taking derivatives to get the OEMs we have:

\begin{align*}
(m a^2 \dottheta)' &= 0 \\
\left(\inv{4} m l^2 \dotalpha \right)' &= 0
\end{align*}

This is suprising seeming.  Is this correct?

\section{ Problem 1.8 }

Hopefully, not a copyright violation, but here is the problem verbatim:

A system is composed of three particles of equal mass m.  Between any two of them there are forces derivable from a potential

\begin{equation*}
V = -g e^{-\mu r}
\end{equation*}

where r is the disance between the two particles.  In addition, two of the particles each exert a force on the third which can be obtained from a generalized potential of the form

\begin{equation*}
U = -f \Bv \cdot \Br
\end{equation*}

$\Bv$ being the relative velocity of the interacting particles and f a constant.  Set up the Lagragian for the system, using as coordinates the radius vector $\BR$ of the center of mass and the two vectors

\begin{align*}
\Brho_1 &= \Br_1 - \Br_3 \\
\Brho_2 &= \Br_2 - \Br_3
\end{align*}

Is the total angular momentum of the system conserved?

\subsection{ Solution attempt. }

The center of mass vector is:

\begin{equation*}
\BR = \inv{3}(\Br_1 + \Br_2 + \Br_3)
\end{equation*}

This can be used to express each of the position vectors in terms of the $\Brho_i$ vectors:

\begin{align*}
3 m \BR &= m (\Brho_1 + \Br_3) + m(\Brho_2 + \Br_3) + m \Br_3 \\
        &= 2 m (\Brho_1 + \Brho_2) + 3 m \Br_3 \\
  \Br_3 &= \BR - \inv{3}(\Brho_1 + \Brho_2) \\
\Br_2 = \Brho_2 + \Br_3 &= \Brho_2 + \Br_3 = \frac{2}{3} \Brho_2 - \inv{2} \Brho_1 + \BR \\
\Br_1 = \Brho_1 + \Br_3 &= \frac{2}{3} \Brho_1 - \inv{2} \Brho_2 + \BR \\
\end{align*}

Now, that is enough to specify the part of the Lagrangian from the potentials that act between all the particles

\begin{equation*}
\LL_U = \sum -U_{ij} = g \left( e^{-\mu \Abs{\Brho_1}} + e^{-\mu \Abs{\Brho_2}} + e^{-\mu \Abs{ \Brho_1 - \Brho_2 }} \right)
\end{equation*}

Now, we need to calculate the two $V$ potentials in terms.  If we consider with positions $\Br_1$, and $\Br_2$ to be the ones
that can exert a force on the third, the velocities of those masses relative to $\Br_3$ are:

\begin{equation*}
(\Br_3 - \Br_i)' = \dot{\Brho_i}
\end{equation*}

Adding this to the first half of the Lagrangian we have:
So, for the second half of the Lagrangian we have:

\begin{equation*}
\LL =
g \left( e^{-\mu \Abs{\Brho_1}} + e^{-\mu \Abs{\Brho_2}} + e^{-\mu \Abs{ \Brho_1 - \Brho_2 }} \right)
+ f \left(\BR - \inv{3}(\Brho_1 + \Brho_2) \right) \cdot \left( \dot{\Brho_1} + \dot{\Brho_2} \right)
\end{equation*}

So, there's the Lagrangian.

How about the angular momentum conservation question?  How to answer that?  One way would be to compute the forces from the Lagrangian, and take cross products but is that really the best way?  Perhaps the answer is as simple as observing that there are no external torque's on the system, thus $d\BL/dt = 0$, or angular momentum for the system is constant (conserved).

FIXME: FOLLOWUP: it has been suggested to me on PF that I should look at how this Lagrangian transforms under rotation.  The relative vectors (both speed and velocity) between the different points will be rotation invarient.  Think that is the case for the CM too.

\section{ Problem 2.1 }

Prove that the shortest length curve between two points in space is a straight line.

A first attempt of this I used:

\begin{equation*}
ds = \sqrt{ 1 + (dy/dx)^2 + (dz/dx)^2 } dx
\end{equation*}

Application of the Euler-Lagrange equations does show that one ends up with a linear relation between the y and z coordinates, but no mention of x.  Rather than write that up, consider instead a parameterization of the coordinates:

\begin{align*}
x &= x_1(\lambda) \\
y &= x_2(\lambda) \\
z &= x_3(\lambda)
\end{align*}

in terms of this arbitrary parameterization we have a segment length of:

\begin{equation*}
ds = \sqrt{ \sum \left(\frac{d x_i}{d\lambda}\right)^2 } d \lambda = f\left(x_i\right) d\lambda
\end{equation*}

Application of the Euler-Lagrange equation to $f$ we have:

\begin{align*}
\PD{x_i}{f} 
&= 0 \\
&= \frac{d}{d\lambda} \PD{\xdot_i}{} \sqrt{ \sum {\xdot_j}^2 } \\
&= \frac{d}{d\lambda} \frac{ \xdot_i }{\sqrt{ \sum {\xdot_j}^2 }}
\end{align*}

Therefore each of these quotients can be equated to a constant:

\begin{align*}
\frac{ \xdot_i }{\sqrt{ \sum {\xdot_j}^2 }} &= {c_i}^{-2} \\
{c_i}^2 \xdot_i^2 &= \sum {\xdot_j}^2 \\
({c_i}^2 -1)\xdot_i^2 &= \sum_{j \ne i} {\xdot_j}^2 \\
(1 - {c_i}^2)\xdot_i^2 + \sum_{j \ne i} {\xdot_j}^2 &= 0 
\end{align*}

This last form shows explicitly that not all of these squared derivative terms can be linearly independent.  In particular, we have a
zero determinant:

\begin{equation*}
0 =
\begin{vmatrix}
1 - c_1^2   & 1            & 1         & 1 & \hdots \\
1           & 1 - c_2^2    & 1         & 1 & \vdots \\
1           & 1            & 1 - c_3^2 & 1 & \\
            &              &           & \ddots & \\
            &              &           &        & 1 - {c_n}^2
\end{vmatrix}
\end{equation*}

Now, expanding this for a couple specific cases isn't too hard.  For $n=2$ we have:

\begin{align*}
0 &= (1 - c_1^2)(1-c_2^2) - 1 \\
c_1^2 + c_2^2 &= c_1^2 c_2^2 \\
c_1^2 &= \frac{c_2^2}{ c_2^2 - 1 } \\
c_2^2 - 1 &= \frac{c_2^2}{ c_1^2 }
\end{align*}

This can be substuited back into one our $c_2^2$ equation:

\begin{align*}
({c_2}^2 -1)\xdot_2^2 &= {\xdot_1}^2 \\
\frac{c_2^2}{ c_1^2 } \xdot_2^2 &= {\xdot_1}^2 \\
\pm \frac{c_2}{ c_1 } \xdot_2 &= {\xdot_1} \\
\pm \frac{c_2}{ c_1 } x_2 &= x_1 + \kappa \\
\end{align*}

This is precisely the straight line that was desired, but we have setup for proving that consideration of all path variations from two points 
in \R{N} space has the shortest distance when that path is a straight line.

Despite the general setup, I'm going to chicken out and show this only for the \R{3} case.  In that case our determinant expands to:

\begin{equation*}
c_1^2 + c_2^2 + c_3^2 = c_1^2 c_2^2 c_3^2
\end{equation*}

Since not all of the $\xdot_i^2$ can be linearly independent, one can be eliminated:

\begin{align*}
(1 - c_1^2) \xdot_1^2 + \xdot_2^2 + \xdot_3^2 &= 0 \\
(1 - c_2^2) \xdot_2^2 + \xdot_3^2 + \xdot_1^2 &= 0 \\
(1 - c_3^2) \xdot_3^2 + \xdot_1^2 + \xdot_2^2 &= 0
\end{align*}

Let's pick $\xdot_1^2$ to eliminate, and subst 2 into 3:

\begin{align*}
%(1 - c_1^2) (-(1 - c_2^2) \xdot_2^2 - \xdot_3^2) + \xdot_2^2 + \xdot_3^2 &= 0 \\
(1 - c_3^2) \xdot_3^2 + (-(1 - c_2^2) \xdot_2^2 - \xdot_3^2) + \xdot_2^2 &= 0
\implies \\
%\xdot_2^2 ( 1 - (1 - c_1^2)(1 - c_2^2) ) + \xdot_3^2 ( 1 - (1 - c_1^2) ) &= 0 \\
- c_3^2 \xdot_3^2 + c_2^2 \xdot_2 &= 0 \\
\pm c_3 \xdot_3 &= c_2 \xdot_2 \\
\end{align*}

%Which is, once again a straight line:
%
%\begin{equation*}
%\pm c_3 x_3 = c_2 x_2 + \kappa
%\end{equation*}

Since these equations are symmetric, we can do this for all, with the result:
\begin{align*}
\pm c_3 \xdot_3 &= c_2 \xdot_2 \\
\pm c_3 \xdot_3 &= c_1 \xdot_1 \\
\pm c_2 \xdot_2 &= c_1 \xdot_1 \\
\end{align*}

Since the $c_i$ constants are arbitrary, then we can for example pick the negative sign for $\pm c_2$, and the positive for the rest, then add all of these, and scale by two:

\begin{equation*}
c_3 \xdot_3 - c_2 \xdot_2 = c_1 \xdot_1
\end{equation*}

and integrating:

\begin{equation*}
c_3 x_3 - c_2 x_2 = c_1 x_1 + \kappa
\end{equation*}

Again, we have the general equation of a line, subject to the desired constraints on the end points.  In the end we didn't need to 
evaluate the determinant after all, as done in the 
\R{2} case.

\section{ Problem 2.2 }

Prove that the geodesics (shortest length paths) on a spherical surface are great circles.

As a variational problem, the first step is to formulate an element of length on the surface.  If we write our vector in spherical coordinates ($\phi$ on the equator, and $\theta$ measuring from the north pole) we have:

FIXME: Scan picture.

\begin{equation*}
\Br = (x, y, z) = R( \sin\theta cos\phi, \sin\theta \sin\phi, \cos\theta)
\end{equation*}

A differential vector element on the surface is (set $R=1$ without loss of generality) :

\begin{align*}
d \Br 
&= \frac{d\Br}{d \theta} \frac{d \theta}{d \lambda} d \lambda + \frac{d\Br}{d \phi} \frac{d \phi}{d \lambda} d \lambda \\
&=
 ( \cos\theta \cos\phi, \cos\theta \sin\phi, -\sin\theta) \dottheta d\lambda
+( -\sin\theta \sin\phi, \sin\theta \cos\phi, 0) \dotphi d\lambda \\
&=
 ( \cos\theta \cos\phi \dottheta - \sin\theta \sin\phi \dotphi,
   \cos\theta \sin\phi \dottheta + \sin\theta \cos\phi \dotphi,
  -\sin\theta \dottheta) d\lambda
\end{align*}

Calculation of the length $ds$ of this vector yields:

\begin{equation*}
ds = \Abs{ d\Br} = \sqrt{\dottheta^2 + (\sin\theta)^2 \dotphi^2} d\lambda
\end{equation*}

This completes the setup for the minimization problem, and we want to 
minimize:

\begin{equation*}
s = \int \sqrt{\dottheta^2 + ( \dotphi \sin\theta )^2 } d\lambda
\end{equation*}

and can therefore apply the Euler-Lagrange equations to the function

\begin{equation*}
f(\theta, \phi, \dottheta, \dotphi, \lambda) = 
\sqrt{\dottheta^2 + ( \dotphi \sin\theta )^2 }
\end{equation*}

The $\phi$ is cyclic, and we have:

\begin{equation*}
\PD{\phi}{f} = 0 = \frac{d}{d\lambda} \frac{\dotphi \sin^2\theta}{f}
\end{equation*}

Thus we have:
\begin{align*}
\dotphi^2 \sin^4\theta &= K^2 \left(\dottheta^2 + ( \dotphi \sin\theta )^2 \right) \\
\dotphi^2 \sin^2\theta( \sin^2\theta - K^2 ) &= K^2 \dottheta^2 \\
\dotphi^2 
&= \frac{K^2 \dottheta^2 }{ \sin^2\theta ( \sin^2\theta - K^2 ) } \\
\dotphi
&= \frac{K \dottheta }{ \sin\theta \sqrt{ \sin^2\theta - K^2 } } \\
\end{align*}

This is in a nicely separated form, but it is not obvious that this describes paths that are great circles.

Let's have a look at the second equation.
\begin{align*}
\PD{\theta}{f} &= \frac{d}{d\lambda} \PD{\dottheta}{f} \\
\frac{\sin\theta\cos\theta \dotphi^2}{f}
&= \frac{d}{d\lambda} \frac{\dottheta}{f} \\
&= \frac{\ddottheta}{f} - \inv{2} \frac{ (\dottheta^2 + ( \dotphi \sin\theta )^2 )' }{f^3} \\
&= \frac{\ddottheta}{f} - \frac{ \dottheta \ddottheta + \dotphi \sin\theta ( \ddotphi \sin\theta + \dotphi \cos\theta \dottheta ) }{f^3} \\
\implies
-\sin\theta\cos\theta \dotphi^2 ( \dottheta^2 + ( \dotphi \sin\theta )^2 )
&= -\ddottheta ( \dottheta^2 + ( \dotphi \sin\theta )^2 )
   + \dottheta \ddottheta 
   + \dotphi \sin\theta ( \ddotphi \sin\theta + \dotphi \cos\theta \dottheta ) \\
\end{align*}

Or,
\begin{equation*}
- \ddottheta \dottheta^2 
- \ddottheta \dotphi^2 \sin^2\theta 
+ \dottheta \ddottheta 
+ \dotphi \ddotphi \sin^2\theta
+ \dotphi^2 \dottheta \sin\theta \cos\theta
+ \dotphi^2 \dottheta^2 \sin\theta \cos\theta 
+ \dotphi^4 \sin^3\theta \cos\theta 
= 0
\end{equation*}

What a mess!  I don't feel inclined to try to reduce this at the moment.  I'll come back to this problem later.  Perhaps there's a better parameterization?

\section{ Problem 2.3 }

For $f = f( y, \ydot, \yddot, x )$, find the equations for extreme values of

\begin{equation*}
I = \int_a^b f dx
\end{equation*}

Here we want $y$ and $\ydot$ fixed at the end points.  Following the first derivative derivation write the 
functions in terms of the desired extremum functions plus a set of arbitrary functions:

\begin{align*}
y( x, \alpha ) &= y( x, 0 ) + \alpha n(x) \\
\ydot( x, \alpha ) &= \ydot( x, 0 ) + \alpha m(x)
\end{align*}

Here we specify that these arbitrary variational functions vanish at the endpoints:

\begin{equation*}
n(a) = n(b) = m(a) = m(b) = 0
\end{equation*}

The functions $y(x, 0)$, and $\ydot(x, 0)$ are the functions we are looking for as solutions to the min/max problem.

Calculating derivatives we have:

\begin{equation*}
\frac{dI}{d\alpha} = 
\int \left( 
\PD{y}{f} \PD{\alpha}{y}
+\PD{\ydot}{f} \PD{\alpha}{\ydot}
+\PD{\yddot}{f} \PD{\alpha}{\yddot}
\right) d x
\end{equation*}

Assuming sufficient continuity look at the second term where we have:

\begin{align*}
\PD{\alpha}{\ydot} 
&= \PD{\alpha}{} \PD{x}{y} \\
&= \PD{x}{} \PD{\alpha}{y} \\
&= \PD{x}{} n(x) \\
&= \frac{d}{ d x} n(x) \\
&= \frac{d}{ d x} \PD{\alpha}{y} \\
\end{align*}

Similarily for the third term we have:

\begin{equation*}
\PD{\alpha}{\ydot} = \frac{d}{ d x} \PD{\alpha}{\ydot}
\end{equation*}

\begin{equation*}
\frac{dI}{d\alpha} = 
\int \PD{y}{f} \PD{\alpha}{y} d x +
\underbrace{\PD{\ydot}{f} \frac{d}{ d x} \PD{\alpha}{y}}_{ u v' = (u v)' - u' v } d x
+\PD{\yddot}{f} \frac{d}{ d x} \PD{\alpha}{\ydot} d x
\end{equation*}

Now integrating by parts:
\begin{align*}
\frac{dI}{d\alpha} &= 
 \int \PD{y}{f} \PD{\alpha}{y} d x
+\int \PD{\ydot}{f} \frac{d}{ d x} \PD{\alpha}{y} d x
+\int \PD{\yddot}{f} \frac{d}{ d x} \PD{\alpha}{\ydot} d x \\
\frac{dI}{d\alpha} &= 
 \int \PD{y}{f} \PD{\alpha}{y} d x
+\left(\PD{\ydot}{f} \underbrace{\PD{\alpha}{y}}_{n(x)}\right)_a^b - \int \PD{\alpha}{y} \frac{d}{ d x} \PD{\ydot}{f} d x
+\left(\PD{\yddot}{f} \underbrace{\PD{\alpha}{\ydot}}_{m(x)} \right)_a^b
-\int \PD{\alpha}{\ydot} \frac{d}{ d x} \PD{\yddot}{f} d x
\end{align*}

Because $m(a) = m(b) = n(a) = n(b)$ the non-integral terms are all zero, leaving:

\begin{align*}
\frac{dI}{d\alpha} &= 
  \int \PD{y}{f} \PD{\alpha}{y} d x
- \int \PD{\alpha}{y} \frac{d}{ d x} \PD{\ydot}{f} d x
- \int \PD{\alpha}{\ydot} \frac{d}{ d x} \PD{\yddot}{f} d x
\end{align*}

Now consider just this last integral, which we can again integrate by parts:
\begin{align*}
\int \PD{\alpha}{\ydot} \frac{d}{ d x} \PD{\yddot}{f} d x
&= \int \underbrace{\frac{d}{dx} \PD{\alpha}{y}}_{u'} \underbrace{\frac{d}{ d x} \PD{\yddot}{f}}_{v} d x \\
&= 
\left( \underbrace{\PD{\alpha}{y}}_{n(x)} {\frac{d}{ d x} \PD{\yddot}{f}} \right)_a^b
-\int \PD{\alpha}{y} \frac{d}{dx} {\frac{d}{ d x} \PD{\yddot}{f}} d x \\
&= 
-\int \PD{\alpha}{y} \frac{d^2}{dx^2} \PD{\yddot}{f} d x \\
\end{align*}

This gives:
\begin{align*}
\frac{dI}{d\alpha} &= 
  \int \PD{y}{f} \PD{\alpha}{y} d x
- \int \PD{\alpha}{y} \frac{d}{ d x} \PD{\ydot}{f} d x
+ \int \PD{\alpha}{y} \frac{d^2}{dx^2} \PD{\yddot}{f} d x \\
\frac{dI}{d\alpha} 
&= \int d x \PD{\alpha}{y} \left( \PD{y}{f} - \frac{d}{ d x} \PD{\ydot}{f} + \frac{d^2}{dx^2} \PD{\yddot}{f} \right) \\
&= \int d x n(x) \left( \PD{y}{f} - \frac{d}{ d x} \PD{\ydot}{f} + \frac{d^2}{dx^2} \PD{\yddot}{f} \right)
\end{align*}

So, if we want this derivative to equal zero for any $n(x)$ we require the inner quantity to by zero:

\begin{equation}
\PD{y}{f} - \frac{d}{ d x} \PD{\ydot}{f} + \frac{d^2}{dx^2} \PD{\yddot}{f} = 0
\end{equation}

Question.  Goldstein writes this in total differential form instead of a derivative:

\begin{align*}
dI &= \frac{dI}{d\alpha} d\alpha \\
&= \int d x \left(\PD{\alpha}{y} d \alpha\right) \left( \PD{y}{f} - \frac{d}{ d x} \PD{\ydot}{f} + \frac{d^2}{dx^2} \PD{\yddot}{f} \right) \\
\end{align*}

and then calls this quantity $\PD{\alpha}{y} d \alpha = \delta y$, the variation of $y$.  There must be a mathematical subtlety which motivates this
but it isn't clear to me what that is.  Since the variational calculus texts go a different route, with norms on functional spaces and so forth, perhaps
understanding that motivation isn't worthwhile.  In the end, the conclusion is the same, namely that the inner expression must equal zero for the extremum
condition.

\end{document}               % End of document.

%
% Copyright � 2012 Peeter Joot.  All Rights Reserved.
% Licenced as described in the file LICENSE under the root directory of this GIT repository.
%

\chapter{Compare some wave equation's and their Lagrangians}
\index{wave equation}
\label{chap:waveLagrangian}
%\date{ Dec 02, 2008.  waveLagrangian.tex }

\section{Motivation}

Compare the Lagrangians for the classical wave equation of a vibrating string/film with the wave equation Lagrangian for electromagnetism and Quantum mechanics.

Observe the similarities and differences, and come back to this later after grasping some of the concepts of Field energy and momentum (energy in vibration and electromagnetism and momentum in quantum mechanics).  Do the ideas of field momentum carry in quantum have equivalents in electromagnetism?

\section{Vibrating object equations}

\subsection{One dimensional wave equation}

\citep{goldstein1951cm} does a nice derivation of the one dimensional wave
equation Lagrangian, using a limiting argument applied to an infinite
sequence of connected masses on springs.

\begin{equation}\label{eqn:wave_lagrangian:oneDimensionalWaveLagrangian}
\begin{aligned}
\LL = \inv{2} \left(\mu \left(\PD{t}{\eta}\right)^2 - Y \left(\PD{x}{\eta}\right)^2 \right)
\end{aligned}
\end{equation}

here \(\eta\) was the displacement from the equilibrium position, \(\mu\) is the mass line density and \(Y\) is Young's modulus.

Taking derivatives confirms that this is the correct form.  The Euler-Lagrange
equations for this equation are:

\begin{equation}\label{eqn:waveLagrangian:20}
\begin{aligned}
\PD{\eta}{\LL} &= \PD{t}{} \PD{\PD{t}{\eta}}{\LL} +\PD{x}{} \PD{\PD{x}{\eta}}{\LL} \\
0 &= \PD{t}{} \mu \PD{t}{\eta} -\PD{x}{} Y \PD{x}{\eta} \\
\end{aligned}
\end{equation}

Which has the expected form

\begin{equation}\label{eqn:waveLagrangian:40}
\begin{aligned}
\mu \PDsq{t}{\eta} - Y \PDsq{x}{\eta} &= 0 \\
\end{aligned}
\end{equation}

\subsection{Higher dimension wave equation}

For a string or film or other wavy material with more degrees of freedom than a string with back and forth motion one can guess the Lagrangian from \eqnref{eqn:wave_lagrangian:oneDimensionalWaveLagrangian}.

\begin{equation}\label{eqn:wave_lagrangian:moreDimensionalWaveLagrangian}
\begin{aligned}
\LL = \inv{2} \left(\mu \left(\PD{t}{\eta}\right)^2 - Y \sum_i \left(\PD{x^i}{\eta}\right)^2 \right)
\end{aligned}
\end{equation}

Calculating the Euler-Lagrange equations gives
\begin{equation}\label{eqn:waveLagrangian:60}
\begin{aligned}
\PD{\eta}{\LL} &= \PD{t}{} \PD{\PD{t}{\eta}}{\LL} +\sum_i \PD{x^i}{} \PD{\PD{x^i}{\eta}}{\LL} \\
0 &= \PD{t}{} \mu \PD{t}{\eta} - \sum_i \PD{x^i}{} Y \PD{x^i}{\eta} \\
\end{aligned}
\end{equation}

Which also has the expected form

\begin{equation}\label{eqn:waveLagrangian:80}
\begin{aligned}
\mu \PDsq{t}{\eta} - Y \sum_i \PDsq{x^i}{\eta} &= 0 \\
\end{aligned}
\end{equation}

\section{Electrodynamics wave equation}

From \eqnref{eqn:wave_lagrangian:moreDimensionalWaveLagrangian} one can guess the Lagrangian for the electrodynamic potential wave equations.  Maxwell's equation in potential form are:

\begin{equation}\label{eqn:wave_lagrangian:maxwellPotential}
\begin{aligned}
\grad^2 A &= J/\epsilon_0 c
\end{aligned}
\end{equation}

Which has the following split into four scalar equations
\begin{equation}\label{eqn:waveLagrangian:100}
\begin{aligned}
\grad^2 A^\mu \gamma_\mu &= J^\mu \gamma_\mu/\epsilon_0 c \\
\grad^2 A^\mu &= J^\mu /\epsilon_0 c
\end{aligned}
\end{equation}

For the \(A^\mu\) coordinate try the Lagrangian

\begin{equation}\label{eqn:waveLagrangian:120}
\begin{aligned}
\LL
&= \sum_\nu \inv{2} (\gamma^\nu)^2 \left( \PD{x^\nu}{A^\mu} \right)^2 + J^\nu A^\nu / \epsilon_0 c \\
&= \sum_\nu \inv{2} (\gamma^\nu)^2 \left(\partial_{\nu}{A^\mu}\right)^2 + J^\nu A^\nu / \epsilon_0 c \\
\end{aligned}
\end{equation}

With evaluation of the Euler-Lagrange equations we have

\begin{equation}\label{eqn:waveLagrangian:140}
\begin{aligned}
\PD{A^\mu}{\LL} &= \sum_\alpha \partial_{\alpha} \PD{(\partial{\alpha}{A^\mu})}{\LL} \\
\implies \\
J^\mu/\epsilon_0 c
&= \sum \partial_\alpha (\gamma^\alpha)^2 \partial_{\alpha}{A^\mu} \\
&= \partial^\alpha \partial_{\alpha}{A^\mu} \\
&= \grad^2 A^\mu \\
\end{aligned}
\end{equation}

Which recovers Maxwell's equation.  Having done that the Lagrangian can be tidied slightly introducing the spacetime gradient:

\begin{equation}\label{eqn:wave_lagrangian:potentialLagrangianWithGrad}
\begin{aligned}
\LL &= \inv{2} \left( \grad {A^\alpha} \right)^2 + J^\alpha A^\alpha / \epsilon_0 c
\end{aligned}
\end{equation}

\subsection{Comparing with complex (bivector) version of Maxwell Lagrangian}
Previously, in \bookchapcite{PJMaxwellLagrangian}{phy354} and \bookchapcite{PJMaxwellLagrangian}{phy354}, Maxwell's equation

\begin{equation}\label{eqn:wave_lagrangian:maxwell}
\begin{aligned}
\grad (\grad \wedge A) &= J/ \epsilon_0 c
\end{aligned}
\end{equation}

was seen as the result of evaluating the Lagrangian

\begin{equation}\label{eqn:wave_lagrangian:maxlag}
\begin{aligned}
\LL &= -\frac{\epsilon_0 c}{2} (\grad \wedge A)^2 + J \cdot A
\end{aligned}
\end{equation}

\Eqnref{eqn:wave_lagrangian:maxwell} with the gauge condition \(\grad \cdot A = 0\)
is where we get the potential form \eqnref{eqn:wave_lagrangian:maxwellPotential} from.

For comparison it should be possible to reconcile this with
\eqnref{eqn:wave_lagrangian:potentialLagrangianWithGrad}.  We can multiply by \((\gamma_\alpha)^2\), which is \((\pm 1)\) dependent on \(\alpha\), as well as multiply by \(\epsilon_0 c\)

\begin{equation}\label{eqn:waveLagrangian:160}
\begin{aligned}
\LL &= \frac{\epsilon_0 c}{2} \grad {A^\alpha} \grad {A_\alpha} + J^\alpha A_\alpha \\
\end{aligned}
\end{equation}

No sum need be implied here, but since the field variables are independent we can sum them without changing the field equations.  So, instead of having four independent Lagrangians, we are now left with a (sums now implied) single density that can be evaluated for each of the potential coordinate variables:

\begin{equation}\label{eqn:waveLagrangian:180}
\begin{aligned}
\LL &= \frac{\epsilon_0 c}{2} \grad {A^\alpha} \grad {A_\alpha} + J \cdot A \\
\end{aligned}
\end{equation}

This is looking more like \eqnref{eqn:wave_lagrangian:maxlag} now.  It is expected that the gauge condition can be used to complete the reconciliation.  However, I have had trouble actually doing this, despite the fact that both Lagrangians appear to correctly
lead to equivalent results.

Also notable perhaps is a comparison to the four potential Lagrangian in Goldstein:

\begin{equation}\label{eqn:waveLagrangian:200}
\begin{aligned}
\LL =
-\inv{16\pi} \sum_{\mu,\nu} \left(
\PD{x_\nu}{A_\mu} - \PD{x_\mu}{A_\nu}
\right)^2
-\inv{8\pi}
\sum_\mu
\left(
\PD{x_\mu}{A_\mu}
\right)^2
+
\sum_\mu \frac{j_\mu A_\mu}{c}
\end{aligned}
\end{equation}

This one is considerably more complex looking, and
it should be possible to see how exactly this is related to the wave
equation guessed by comparison to the vibrating string.

\section{Quantum Mechanics}

\subsection{Non-relativistic case}

The non-relativistic Lagrangian given by Goldstein (problem 11.3) is

\begin{equation}\label{eqn:waveLagrangian:220}
\begin{aligned}
\LL = \frac{\Hbar^2}{2m}
(\spacegrad \psi) \cdot (\spacegrad \psi^\conj) + V \psi \psi^\conj + {i \Hbar} \left( \psi \partial_t \psi^\conj - \psi^\conj \partial_t \psi \right)
\end{aligned}
\end{equation}

Again we see the square of the spatial gradient so we expect a (spatial) Laplacian
in the field equation, which one has:

\begin{equation}\label{eqn:waveLagrangian:240}
\begin{aligned}
\left( \frac{-\Hbar^2}{2m} \spacegrad^2 + V \right) \psi = i \Hbar \PD{t}{\psi}
\end{aligned}
\end{equation}

\subsection{Relativistic case. Klein-Gordon}

The Klein-Gordon Lagrangian is
\begin{equation}\label{eqn:waveLagrangian:260}
\begin{aligned}
\LL
&= -(\grad \psi) \cdot (\grad \psi^\conj) + \frac{m^2 c^2}{\Hbar^2} \psi \psi^\conj \\
\end{aligned}
\end{equation}

from which we can recover the Klein-Gordon scalar wave equation which applies to a
specific subset of quantum phenomena (what exactly?)

\begin{equation}\label{eqn:waveLagrangian:280}
\begin{aligned}
\left(\frac{\Hbar^2}{2m} \grad^2 + \inv{2} m c^2\right) \psi = 0 \\
\end{aligned}
\end{equation}

\subsection{Dirac wave equation}

The Dirac wave equation, for vector wave function \(\psi\) can be formally
obtained by
taking vector roots of the scalar operators in the Klein-Gordon equation
to yield:

\begin{equation}\label{eqn:waveLagrangian:300}
\begin{aligned}
i \Hbar \grad \psi = \pm m c \psi
\end{aligned}
\end{equation}

The Lagrangian for this field equation is

\begin{equation}\label{eqn:waveLagrangian:320}
\begin{aligned}
\LL = mc \overbar{\psi}\psi - {\inv{2}i\Hbar}(\overbar{\psi}\gamma^\mu (\partial_\mu\psi) - (\partial_\mu\overbar{\psi})\gamma^\mu \psi)
\end{aligned}
\end{equation}

Where \(\overbar{\psi} = \gamma_0 \tilde{\psi}\), and \(\tilde{\psi}\) is the reversed field spinor.

\section{Summary comparison of all the second order wave equations}

\begin{itemize}

\item Vibration wave equation.

\begin{equation}\label{eqn:waveLagrangian:340}
\begin{aligned}
\LL &= \mu \left( \PD{t}{\eta} \right)^2 - Y \left( \spacegrad \eta \right)^2 \\
0 &= \mu \PDsq{t}{\eta} - Y \spacegrad^2 {\eta}
\end{aligned}
\end{equation}

\item Maxwell wave equation.

\begin{equation}\label{eqn:waveLagrangian:360}
\begin{aligned}
\LL &= \inv{2} \left( \grad {A^\alpha} \right)^2 + J^\alpha A^\alpha / \epsilon_0 c \\
\grad^2 A^\alpha &= J^\alpha/\epsilon_0 c
\end{aligned}
\end{equation}

\item Schr\"{o}dinger non-relativistic wave equation.

\begin{equation}\label{eqn:waveLagrangian:380}
\begin{aligned}
\LL &= \frac{\Hbar^2}{2m}
(\spacegrad \psi) \cdot (\spacegrad \psi^\conj) + V \psi \psi^\conj + {i \Hbar} \left( \psi \partial_t \psi^\conj - \psi^\conj \partial_t \psi \right) \\
\left( \frac{-\Hbar}{2m} \spacegrad^2 + V \right) \psi &= \Hbar i \PD{t}{\psi}
\end{aligned}
\end{equation}

\item Klein-Gordon wave equation.

\begin{equation}\label{eqn:waveLagrangian:400}
\begin{aligned}
\LL &= -(\grad \psi) \cdot (\grad \psi^\conj) + \frac{m^2 c^2}{\Hbar^2} \psi \psi^\conj \\
-\grad^2 \psi &= \frac{m^2 c^2}{\Hbar^2} \psi
\end{aligned}
\end{equation}

\end{itemize}

%
% Copyright � 2012 Peeter Joot.  All Rights Reserved.
% Licenced as described in the file LICENSE under the root directory of this GIT repository.
%

\chapter{Short metric tensor explanation}
\index{metric tensor}
\label{chap:lorentzMetricTensor}
%\date{August 30, 2008.  lorentzMetricTensor.tex }

\section{}

\href{http://www.physicsforums.com/showthread.php?p=1853416}{PF thread.}

I have found it helpful to think about the metric tensor in terms of vector dot products, and a corresponding basis.

You can cut relativity completely out of the question, and ask the same question for Euclidean space, where the metric tensor it the identity matrix when you pick an orthonormal basis.

That diagonality is due to orthogonality conditions of the basis chosen.  For, example, in 3D we can express vectors in terms of an
orthonormal frame, but if we choose not to, say picking \(e_1 + e_2\), \(e_1-e_2\), and \(e_1 + e_3\) as our basis vectors then how do we calculate the coordinates?

The trick is to calculate, or assume calculated, an alternate set of basis vectors, called the reciprocal frame.  Provided the initial set of vectors spans the space, one can always calculate (and that part is a linear algebra exercise) this second pair such that they meet the following relationships:

\begin{equation*}
e^i \cdot e_j = {\delta^i}_j
\end{equation*}

So, if a vector is specified in terms of the \(e_i\)
\begin{equation*}
x = \sum e_j a_j
\end{equation*}

Dotting with \(e^i\) one has:

\begin{equation*}
x \cdot e^i = \sum (e_j a_j) \cdot e^i = \sum {\delta^j}_i a_j = a_i
\end{equation*}

It is customary to write \(a_i = x^i\), which allows for the entire vector to
be written in the mixed upper and lower index method where sums are assumed:

\begin{equation*}
x = \sum e_j x^j = e_j x^j
\end{equation*}

Now, if one calculates dot product here, say with \(x\), and a second vector

\begin{equation*}
y = \sum e_j y^j
\end{equation*}

you have:

\begin{equation*}
x \cdot y = \sum (e_j \cdot e_k) x^j y^k
\end{equation*}

The coefficient of this \(x^j y^k\) term is symmetric, and if you choose, you
can write \(g_{jk} = e_j \cdot e_k\), and you have the dot product in
tensor form:

\begin{equation*}
x \cdot y = \sum g_{jk} x^j y^k = g_{jk} x^j y^k
\end{equation*}

Now, for relativity, you have four instead of three basis vectors, so if you choose your spatial basis vectors orthonormally, and a timelike basis vector normal to all of those (ie: no mixing of space and time vectors in anything but a Lorentz fashion), then you get a diagonal metric tensor.  You can choose not to work in an "orthonormal" spacetime basis, and a non-diagonal metric tensor will show up in all your dot products.  That decision is perfectly valid, just makes everything harder.  When it comes down to why, it all boils down to your choice of basis.

Now, just like you can think of a rotation as a linear transformation that preserves angles in Euclidean space, the Lorentz transformation preserves the spacetime relationships appropriately.  So, if one transforms from a "orthonormal" spacetime frame to an alternate "orthonormal" spacetime frame (and a Lorentz transformation is just that) you still have the same "angles" (ie: dot products) between an event coordinates, and the metric will still be diagonal as described.  This could be viewed as just a rather long winded way of saying exactly what jdstokes said, but its the explanation coming from somebody who is also just learning this (so I had need such a longer explanation if I was explaining to myself).


% sub-sort by date
\part{Quantum Mechanics.}
% 
% 
% 
% Copyright � 2012 Peeter Joot
% All Rights Reserved
% 
% This file may be reproduced and distributed in whole or in part, without fee, subject to the following conditions:
% 
% o The copyright notice above and this permission notice must be preserved complete on all complete or partial copies.
% 
% o Any translation or derived work must be approved by the author in writing before distribution.
% 
% o If you distribute this work in part, instructions for obtaining the complete version of this file must be included, and a means for obtaining a complete version provided.
% 
% 
% Exceptions to these rules may be granted for academic purposes: Write to the author and ask.
% 
% 
% 
%\documentclass{article}

%\usepackage{amsmath}
\usepackage{mathpazo}

%
% shorthand for bold symbols, convenient for vectors and matrices
%
\newcommand{\Ba}[0]{\mathbf{a}}
\newcommand{\Bb}[0]{\mathbf{b}}
\newcommand{\Bc}[0]{\mathbf{c}}
\newcommand{\Bd}[0]{\mathbf{d}}
\newcommand{\Be}[0]{\mathbf{e}}
\newcommand{\Bf}[0]{\mathbf{f}}
\newcommand{\Bg}[0]{\mathbf{g}}
\newcommand{\Bh}[0]{\mathbf{h}}
\newcommand{\Bi}[0]{\mathbf{i}}
\newcommand{\Bj}[0]{\mathbf{j}}
\newcommand{\Bk}[0]{\mathbf{k}}
\newcommand{\Bl}[0]{\mathbf{l}}
\newcommand{\Bm}[0]{\mathbf{m}}
\newcommand{\Bn}[0]{\mathbf{n}}
\newcommand{\Bo}[0]{\mathbf{o}}
\newcommand{\Bp}[0]{\mathbf{p}}
\newcommand{\Bq}[0]{\mathbf{q}}
\newcommand{\Br}[0]{\mathbf{r}}
\newcommand{\Bs}[0]{\mathbf{s}}
\newcommand{\Bt}[0]{\mathbf{t}}
\newcommand{\Bu}[0]{\mathbf{u}}
\newcommand{\Bv}[0]{\mathbf{v}}
\newcommand{\Bw}[0]{\mathbf{w}}
\newcommand{\Bx}[0]{\mathbf{x}}
\newcommand{\By}[0]{\mathbf{y}}
\newcommand{\Bz}[0]{\mathbf{z}}
\newcommand{\BA}[0]{\mathbf{A}}
\newcommand{\BB}[0]{\mathbf{B}}
\newcommand{\BC}[0]{\mathbf{C}}
\newcommand{\BD}[0]{\mathbf{D}}
\newcommand{\BE}[0]{\mathbf{E}}
\newcommand{\BF}[0]{\mathbf{F}}
\newcommand{\BG}[0]{\mathbf{G}}
\newcommand{\BH}[0]{\mathbf{H}}
\newcommand{\BI}[0]{\mathbf{I}}
\newcommand{\BJ}[0]{\mathbf{J}}
\newcommand{\BK}[0]{\mathbf{K}}
\newcommand{\BL}[0]{\mathbf{L}}
\newcommand{\BM}[0]{\mathbf{M}}
\newcommand{\BN}[0]{\mathbf{N}}
\newcommand{\BO}[0]{\mathbf{O}}
\newcommand{\BP}[0]{\mathbf{P}}
\newcommand{\BQ}[0]{\mathbf{Q}}
\newcommand{\BR}[0]{\mathbf{R}}
\newcommand{\BS}[0]{\mathbf{S}}
\newcommand{\BT}[0]{\mathbf{T}}
\newcommand{\BU}[0]{\mathbf{U}}
\newcommand{\BV}[0]{\mathbf{V}}
\newcommand{\BW}[0]{\mathbf{W}}
\newcommand{\BX}[0]{\mathbf{X}}
\newcommand{\BY}[0]{\mathbf{Y}}
\newcommand{\BZ}[0]{\mathbf{Z}}

\newcommand{\Bzero}[0]{\mathbf{0}}
\newcommand{\Btheta}[0]{\boldsymbol{\theta}}
\newcommand{\Btau}[0]{\boldsymbol{\tau}}
\newcommand{\Bomega}[0]{\boldsymbol{\omega}}

%
% shorthand for unit vectors
%
\newcommand{\acap}[0]{\hat{\Ba}}
\newcommand{\bcap}[0]{\hat{\Bb}}
\newcommand{\ccap}[0]{\hat{\Bc}}
\newcommand{\dcap}[0]{\hat{\Bd}}
\newcommand{\ecap}[0]{\hat{\Be}}
\newcommand{\fcap}[0]{\hat{\Bf}}
\newcommand{\gcap}[0]{\hat{\Bg}}
\newcommand{\hcap}[0]{\hat{\Bh}}
\newcommand{\icap}[0]{\hat{\Bi}}
\newcommand{\jcap}[0]{\hat{\Bj}}
\newcommand{\kcap}[0]{\hat{\Bk}}
\newcommand{\lcap}[0]{\hat{\Bl}}
\newcommand{\mcap}[0]{\hat{\Bm}}
\newcommand{\ncap}[0]{\hat{\Bn}}
\newcommand{\ocap}[0]{\hat{\Bo}}
\newcommand{\pcap}[0]{\hat{\Bp}}
\newcommand{\qcap}[0]{\hat{\Bq}}
\newcommand{\rcap}[0]{\hat{\Br}}
\newcommand{\scap}[0]{\hat{\Bs}}
\newcommand{\tcap}[0]{\hat{\Bt}}
\newcommand{\ucap}[0]{\hat{\Bu}}
\newcommand{\vcap}[0]{\hat{\Bv}}
\newcommand{\wcap}[0]{\hat{\Bw}}
\newcommand{\xcap}[0]{\hat{\Bx}}
\newcommand{\ycap}[0]{\hat{\By}}
\newcommand{\zcap}[0]{\hat{\Bz}}
\newcommand{\thetacap}[0]{\hat{\Btheta}}

%
% to write R^n and C^n in a distinguishable fashion.  Perhaps change this
% to the double lined characters upon figuring out how to do so.
%
\newcommand{\C}[1]{$\mathbb{C}^{#1}$}
\newcommand{\R}[1]{$\mathbb{R}^{#1}$}

%
% various generally useful helpers
%

% derivative of #1 wrt. #2:
\newcommand{\D}[2] {\frac {d#2} {d#1}}

\newcommand{\inv}[1]{\frac{1}{#1}}
\newcommand{\cross}[0]{\times}

\newcommand{\abs}[1]{\lvert{#1}\rvert}
\newcommand{\norm}[1]{\lVert{#1}\rVert}
\newcommand{\innerprod}[2]{\langle{#1}, {#2}\rangle}
\newcommand{\dotprod}[2]{{#1} \cdot {#2}}
\newcommand{\bdotprod}[2]{\left({#1} \cdot {#2}\right)}
\newcommand{\crossprod}[2]{{#1} \cross {#2}}
\newcommand{\tripleprod}[3]{\dotprod{\left(\crossprod{#1}{#2}\right)}{#3}}

\DeclareMathOperator{\Proj}{Proj}
\DeclareMathOperator{\Span}{span}
\DeclareMathOperator{\Sgn}{sgn}
\DeclareMathOperator{\Area}{Area}
\DeclareMathOperator{\Volume}{Volume}

%
% A few miscellaneous things specific to this document
%
\newcommand{\crossop}[1]{\crossprod{#1}{}}

% R2 vector.
\newcommand{\VectorTwo}[2]{
\begin{bmatrix}
 {#1} \\
 {#2}
\end{bmatrix}
}

\newcommand{\VectorN}[1]{
\begin{bmatrix}
{#1}_1 \\
{#1}_2 \\
\vdots \\
{#1}_N \\
\end{bmatrix}
}

\newcommand{\DETuvij}[4]{
\begin{vmatrix}
 {#1}_{#3} & {#1}_{#4} \\
 {#2}_{#3} & {#2}_{#4}
\end{vmatrix}
}

\newcommand{\DETuvwijk}[6]{
\begin{vmatrix}
 {#1}_{#4} & {#1}_{#5} & {#1}_{#6} \\
 {#2}_{#4} & {#2}_{#5} & {#2}_{#6} \\
 {#3}_{#4} & {#3}_{#5} & {#3}_{#6}
\end{vmatrix}
}

\newcommand{\DETuvwxijkl}[8]{
\begin{vmatrix}
 {#1}_{#5} & {#1}_{#6} & {#1}_{#7} & {#1}_{#8} \\
 {#2}_{#5} & {#2}_{#6} & {#2}_{#7} & {#2}_{#8} \\
 {#3}_{#5} & {#3}_{#6} & {#3}_{#7} & {#3}_{#8} \\
 {#4}_{#5} & {#4}_{#6} & {#4}_{#7} & {#4}_{#8} \\
\end{vmatrix}
}

%\newcommand{\DETuvwxyijklm}[10]{
%\begin{vmatrix}
% {#1}_{#6} & {#1}_{#7} & {#1}_{#8} & {#1}_{#9} & {#1}_{#10} \\
% {#2}_{#6} & {#2}_{#7} & {#2}_{#8} & {#2}_{#9} & {#2}_{#10} \\
% {#3}_{#6} & {#3}_{#7} & {#3}_{#8} & {#3}_{#9} & {#3}_{#10} \\
% {#4}_{#6} & {#4}_{#7} & {#4}_{#8} & {#4}_{#9} & {#4}_{#10} \\
% {#5}_{#6} & {#5}_{#7} & {#5}_{#8} & {#5}_{#9} & {#5}_{#10}
%\end{vmatrix}
%}

% R3 vector.
\newcommand{\VectorThree}[3]{
\begin{bmatrix}
 {#1} \\
 {#2} \\
 {#3}
\end{bmatrix}
}


%%<misc>
%
\newcommand{\Abs}[1]{{\left\lvert{#1}\right\rvert}}
\newcommand{\spacegrad}[0]{\boldsymbol{\nabla}}
\newcommand{\grad}[0]{\nabla}
\newcommand{\LL}[0]{\mathcal{L}}

% == \partial_{#1} {#2}
\newcommand{\PD}[2]{\frac{\partial {#2}}{\partial {#1}}}
% inline variant
\newcommand{\PDi}[2]{{\partial {#2}}/{\partial {#1}}}

\newcommand{\PDD}[3]{\frac{\partial^2 {#3}}{\partial {#1}\partial {#2}}}
%\newcommand{\PDd}[2]{\frac{\partial^2 {#2}}{{\partial{#1}}^2}}
\newcommand{\PDsq}[2]{\frac{\partial^2 {#2}}{(\partial {#1})^2}}

\newcommand{\Partial}[2]{\frac{\partial {#1}}{\partial {#2}}}
\DeclareMathOperator{\RejName}{Rej}
\newcommand{\Rej}[2]{\RejName_{#1}\left( {#2} \right)}
\newcommand{\Rm}[1]{\mathbb{R}^{#1}}
\newcommand{\Cm}[1]{\mathbb{C}^{#1}}
\newcommand{\conj}[0]{{*}}

%</misc>

% <grade selection>
%
\newcommand{\gpgrade}[2] {{\left\langle{{#1}}\right\rangle}_{#2}}

\newcommand{\gpgradezero}[1] {\gpgrade{#1}{}}
%\newcommand{\gpscalargrade}[1] {{\left\langle{{#1}}\right\rangle}}
%\newcommand{\gpgradezero}[1] {\gpgrade{#1}{0}}

%\newcommand{\gpgradeone}[1] {{\left\langle{{#1}}\right\rangle}_{1}}
\newcommand{\gpgradeone}[1] {\gpgrade{#1}{1}}

\newcommand{\gpgradetwo}[1] {\gpgrade{#1}{2}}
\newcommand{\gpgradethree}[1] {\gpgrade{#1}{3}}
\newcommand{\gpgradefour}[1] {\gpgrade{#1}{4}}
%
% </grade selection>



\newcommand{\adot}[0]{{\dot{a}}}
\newcommand{\bdot}[0]{{\dot{b}}}
% taken for centered dot:
%\newcommand{\cdot}[0]{{\dot{c}}}
%\newcommand{\ddot}[0]{{\dot{d}}}
\newcommand{\edot}[0]{{\dot{e}}}
\newcommand{\fdot}[0]{{\dot{f}}}
\newcommand{\gdot}[0]{{\dot{g}}}
\newcommand{\hdot}[0]{{\dot{h}}}
\newcommand{\idot}[0]{{\dot{i}}}
\newcommand{\jdot}[0]{{\dot{j}}}
\newcommand{\kdot}[0]{{\dot{k}}}
\newcommand{\ldot}[0]{{\dot{l}}}
\newcommand{\mdot}[0]{{\dot{m}}}
\newcommand{\ndot}[0]{{\dot{n}}}
%\newcommand{\odot}[0]{{\dot{o}}}
\newcommand{\pdot}[0]{{\dot{p}}}
\newcommand{\qdot}[0]{{\dot{q}}}
\newcommand{\rdot}[0]{{\dot{r}}}
\newcommand{\sdot}[0]{{\dot{s}}}
\newcommand{\tdot}[0]{{\dot{t}}}
\newcommand{\udot}[0]{{\dot{u}}}
\newcommand{\vdot}[0]{{\dot{v}}}
\newcommand{\wdot}[0]{{\dot{w}}}
\newcommand{\xdot}[0]{{\dot{x}}}
\newcommand{\ydot}[0]{{\dot{y}}}
\newcommand{\zdot}[0]{{\dot{z}}}
\newcommand{\addot}[0]{{\ddot{a}}}
\newcommand{\bddot}[0]{{\ddot{b}}}
\newcommand{\cddot}[0]{{\ddot{c}}}
%\newcommand{\dddot}[0]{{\ddot{d}}}
\newcommand{\eddot}[0]{{\ddot{e}}}
\newcommand{\fddot}[0]{{\ddot{f}}}
\newcommand{\gddot}[0]{{\ddot{g}}}
\newcommand{\hddot}[0]{{\ddot{h}}}
\newcommand{\iddot}[0]{{\ddot{i}}}
\newcommand{\jddot}[0]{{\ddot{j}}}
\newcommand{\kddot}[0]{{\ddot{k}}}
\newcommand{\lddot}[0]{{\ddot{l}}}
\newcommand{\mddot}[0]{{\ddot{m}}}
\newcommand{\nddot}[0]{{\ddot{n}}}
\newcommand{\oddot}[0]{{\ddot{o}}}
\newcommand{\pddot}[0]{{\ddot{p}}}
\newcommand{\qddot}[0]{{\ddot{q}}}
\newcommand{\rddot}[0]{{\ddot{r}}}
\newcommand{\sddot}[0]{{\ddot{s}}}
\newcommand{\tddot}[0]{{\ddot{t}}}
\newcommand{\uddot}[0]{{\ddot{u}}}
\newcommand{\vddot}[0]{{\ddot{v}}}
\newcommand{\wddot}[0]{{\ddot{w}}}
\newcommand{\xddot}[0]{{\ddot{x}}}
\newcommand{\yddot}[0]{{\ddot{y}}}
\newcommand{\zddot}[0]{{\ddot{z}}}

%<bold and dot greek symbols>
%

\newcommand{\Deltadot}[0]{{\dot{\Delta}}}
\newcommand{\Gammadot}[0]{{\dot{\Gamma}}}
\newcommand{\Lambdadot}[0]{{\dot{\Lambda}}}
\newcommand{\Omegadot}[0]{{\dot{\Omega}}}
\newcommand{\Phidot}[0]{{\dot{\Phi}}}
\newcommand{\Pidot}[0]{{\dot{\Pi}}}
\newcommand{\Psidot}[0]{{\dot{\Psi}}}
\newcommand{\Sigmadot}[0]{{\dot{\Sigma}}}
\newcommand{\Thetadot}[0]{{\dot{\Theta}}}
\newcommand{\Upsilondot}[0]{{\dot{\Upsilon}}}
\newcommand{\Xidot}[0]{{\dot{\Xi}}}
\newcommand{\alphadot}[0]{{\dot{\alpha}}}
\newcommand{\betadot}[0]{{\dot{\beta}}}
\newcommand{\chidot}[0]{{\dot{\chi}}}
\newcommand{\deltadot}[0]{{\dot{\delta}}}
\newcommand{\epsilondot}[0]{{\dot{\epsilon}}}
\newcommand{\etadot}[0]{{\dot{\eta}}}
\newcommand{\gammadot}[0]{{\dot{\gamma}}}
\newcommand{\kappadot}[0]{{\dot{\kappa}}}
\newcommand{\lambdadot}[0]{{\dot{\lambda}}}
\newcommand{\mudot}[0]{{\dot{\mu}}}
\newcommand{\nudot}[0]{{\dot{\nu}}}
\newcommand{\omegadot}[0]{{\dot{\omega}}}
\newcommand{\phidot}[0]{{\dot{\phi}}}
\newcommand{\pidot}[0]{{\dot{\pi}}}
\newcommand{\psidot}[0]{{\dot{\psi}}}
\newcommand{\rhodot}[0]{{\dot{\rho}}}
\newcommand{\sigmadot}[0]{{\dot{\sigma}}}
\newcommand{\taudot}[0]{{\dot{\tau}}}
\newcommand{\thetadot}[0]{{\dot{\theta}}}
\newcommand{\upsilondot}[0]{{\dot{\upsilon}}}
\newcommand{\varepsilondot}[0]{{\dot{\varepsilon}}}
\newcommand{\varphidot}[0]{{\dot{\varphi}}}
\newcommand{\varpidot}[0]{{\dot{\varpi}}}
\newcommand{\varrhodot}[0]{{\dot{\varrho}}}
\newcommand{\varsigmadot}[0]{{\dot{\varsigma}}}
\newcommand{\varthetadot}[0]{{\dot{\vartheta}}}
\newcommand{\xidot}[0]{{\dot{\xi}}}
\newcommand{\zetadot}[0]{{\dot{\zeta}}}

\newcommand{\Deltaddot}[0]{{\ddot{\Delta}}}
\newcommand{\Gammaddot}[0]{{\ddot{\Gamma}}}
\newcommand{\Lambdaddot}[0]{{\ddot{\Lambda}}}
\newcommand{\Omegaddot}[0]{{\ddot{\Omega}}}
\newcommand{\Phiddot}[0]{{\ddot{\Phi}}}
\newcommand{\Piddot}[0]{{\ddot{\Pi}}}
\newcommand{\Psiddot}[0]{{\ddot{\Psi}}}
\newcommand{\Sigmaddot}[0]{{\ddot{\Sigma}}}
\newcommand{\Thetaddot}[0]{{\ddot{\Theta}}}
\newcommand{\Upsilonddot}[0]{{\ddot{\Upsilon}}}
\newcommand{\Xiddot}[0]{{\ddot{\Xi}}}
\newcommand{\alphaddot}[0]{{\ddot{\alpha}}}
\newcommand{\betaddot}[0]{{\ddot{\beta}}}
\newcommand{\chiddot}[0]{{\ddot{\chi}}}
\newcommand{\deltaddot}[0]{{\ddot{\delta}}}
\newcommand{\epsilonddot}[0]{{\ddot{\epsilon}}}
\newcommand{\etaddot}[0]{{\ddot{\eta}}}
\newcommand{\gammaddot}[0]{{\ddot{\gamma}}}
\newcommand{\kappaddot}[0]{{\ddot{\kappa}}}
\newcommand{\lambdaddot}[0]{{\ddot{\lambda}}}
\newcommand{\muddot}[0]{{\ddot{\mu}}}
\newcommand{\nuddot}[0]{{\ddot{\nu}}}
\newcommand{\omegaddot}[0]{{\ddot{\omega}}}
\newcommand{\phiddot}[0]{{\ddot{\phi}}}
\newcommand{\piddot}[0]{{\ddot{\pi}}}
\newcommand{\psiddot}[0]{{\ddot{\psi}}}
\newcommand{\rhoddot}[0]{{\ddot{\rho}}}
\newcommand{\sigmaddot}[0]{{\ddot{\sigma}}}
\newcommand{\tauddot}[0]{{\ddot{\tau}}}
\newcommand{\thetaddot}[0]{{\ddot{\theta}}}
\newcommand{\upsilonddot}[0]{{\ddot{\upsilon}}}
\newcommand{\varepsilonddot}[0]{{\ddot{\varepsilon}}}
\newcommand{\varphiddot}[0]{{\ddot{\varphi}}}
\newcommand{\varpiddot}[0]{{\ddot{\varpi}}}
\newcommand{\varrhoddot}[0]{{\ddot{\varrho}}}
\newcommand{\varsigmaddot}[0]{{\ddot{\varsigma}}}
\newcommand{\varthetaddot}[0]{{\ddot{\vartheta}}}
\newcommand{\xiddot}[0]{{\ddot{\xi}}}
\newcommand{\zetaddot}[0]{{\ddot{\zeta}}}

\newcommand{\BDelta}[0]{\boldsymbol{\Delta}}
\newcommand{\BGamma}[0]{\boldsymbol{\Gamma}}
\newcommand{\BLambda}[0]{\boldsymbol{\Lambda}}
\newcommand{\BOmega}[0]{\boldsymbol{\Omega}}
\newcommand{\BPhi}[0]{\boldsymbol{\Phi}}
\newcommand{\BPi}[0]{\boldsymbol{\Pi}}
\newcommand{\BPsi}[0]{\boldsymbol{\Psi}}
\newcommand{\BSigma}[0]{\boldsymbol{\Sigma}}
\newcommand{\BTheta}[0]{\boldsymbol{\Theta}}
\newcommand{\BUpsilon}[0]{\boldsymbol{\Upsilon}}
\newcommand{\BXi}[0]{\boldsymbol{\Xi}}
\newcommand{\Balpha}[0]{\boldsymbol{\alpha}}
\newcommand{\Bbeta}[0]{\boldsymbol{\beta}}
\newcommand{\Bchi}[0]{\boldsymbol{\chi}}
\newcommand{\Bdelta}[0]{\boldsymbol{\delta}}
\newcommand{\Bepsilon}[0]{\boldsymbol{\epsilon}}
\newcommand{\Beta}[0]{\boldsymbol{\eta}}
\newcommand{\Bgamma}[0]{\boldsymbol{\gamma}}
\newcommand{\Bkappa}[0]{\boldsymbol{\kappa}}
\newcommand{\Blambda}[0]{\boldsymbol{\lambda}}
\newcommand{\Bmu}[0]{\boldsymbol{\mu}}
\newcommand{\Bnu}[0]{\boldsymbol{\nu}}
%\newcommand{\Bomega}[0]{\boldsymbol{\omega}}
\newcommand{\Bphi}[0]{\boldsymbol{\phi}}
\newcommand{\Bpi}[0]{\boldsymbol{\pi}}
\newcommand{\Bpsi}[0]{\boldsymbol{\psi}}
\newcommand{\Brho}[0]{\boldsymbol{\rho}}
\newcommand{\Bsigma}[0]{\boldsymbol{\sigma}}
%\newcommand{\Btau}[0]{\boldsymbol{\tau}}
%\newcommand{\Btheta}[0]{\boldsymbol{\theta}}
\newcommand{\Bupsilon}[0]{\boldsymbol{\upsilon}}
\newcommand{\Bvarepsilon}[0]{\boldsymbol{\varepsilon}}
\newcommand{\Bvarphi}[0]{\boldsymbol{\varphi}}
\newcommand{\Bvarpi}[0]{\boldsymbol{\varpi}}
\newcommand{\Bvarrho}[0]{\boldsymbol{\varrho}}
\newcommand{\Bvarsigma}[0]{\boldsymbol{\varsigma}}
\newcommand{\Bvartheta}[0]{\boldsymbol{\vartheta}}
\newcommand{\Bxi}[0]{\boldsymbol{\xi}}
\newcommand{\Bzeta}[0]{\boldsymbol{\zeta}}
%
%</bold and dot greek symbols>
%<infrequent>
%
%\newcommand{\AreaOp}[1]{\AName_{#1}}
%\newcommand{\Babs}[0]{\abs{\BB}}
%\newcommand{\Bcap}[0]{\hat{\BB}}
%\newcommand{\BrPrimeRej}[0]{\rcap(\rcap \wedge \Br')}
%\newcommand{\CA}[0]{\mathcal{A}}
%\newcommand{\Cos}[1]{\cos{\left({#1}\right)}}
%\newcommand{\Det}[1] {\abs{#1}}
%\newcommand{\Dsq}[2] {\frac {\partial^2 {#1}} {\partial {#2}^2}}
%\newcommand{\Exp}[1]{\exp{\left({#1}\right)}}
%\newcommand{\Norm}[1]{\left\lVert{#1}\right\rVert}
%\newcommand{\Sin}[1]{\sin{\left({#1}\right)}}
%\newcommand{\T}[0]{\text{T}}
%\newcommand{\VolumeOp}[1]{\VName_{#1}}
%\newcommand{\agrad}[0]{\Ba \cdot \nabla}
%\newcommand{\alphacap}[0]{\hat{\boldsymbol{\alpha}}}
%\newcommand{\Fcap}[0]{\hat{\BF}}
%\newcommand{\bithree}[0]{{\Bi}_3}
%\newcommand{\bxa}[0]{\Bx\Ba}
%\newcommand{\coordvec}[2]{
%\newcommand{\costheta}[0]{\acap \cdot \xcap}
%\newcommand{\ddt}[1]{\ddot{#1}}
%\newcommand{\ddu}[1] {\frac {d{#1}} {du}}
%\newcommand{\dsqxj}[2] {\frac {\partial^2 {#1}} {\partial {x_{#2}}^2}}
%\newcommand{\dtheta}[1]{\frac{d {#1}}{d \theta}}
%\newcommand{\dt}[1]{\dot{#1}}
%\newcommand{\dt}[1]{\frac{d {#1}}{dt}}
%\newcommand{\dxj}[2] {\frac {\partial {#1}} {\partial {x_{#2}}}}
%\newcommand{\halfPhi}[0]{\frac{\phi}{2}}
%\newcommand{\half}[0]{\inv{2}}
%\newcommand{\inv}[1]{\frac{1}{#1}}
%\newcommand{\laplacian}[0]{\nabla^2}
%\newcommand{\matrixoftx}[3]{
%\newcommand{\nrrp}[0]{\norm{\rcap \wedge \Br'}}
%\newcommand{\oiint}{\bigcirc \hspace{-1.4em} \int \hspace{-.8em} \int}
%\newcommand{\transpose}[1]{{#1}^{\text{T}}}
%\newcommand{\transpose}[1]{{{#1}^{\TextTranspose}}}
%\newcommand{\transpose}[1]{{{#1}^{\text{T}}}}
%\newcommand{\barA}[0]{\bar{A}}
%\newcommand{\qbar}[0]{\bar{q}}
%\newcommand{\qdotbar}[0]{\dot{\bar{q}}}
%
%</infrequent>





%\usepackage[bookmarks=true]{hyperref}

\chapter{Some notes on DeBroglie paper.}
\label{chap:debroglie}
%\author{Peeter Joot \quad peeter.joot@gmail.com}
\date{ Oct. 25, 2008.  debroglie.tex }

%\begin{document}

%\maketitle{}
%\tableofcontents

\section{Motivation. }

The translation of the DeBroglie thesis \citep{AFkracklauerDeBroglie}
appears to have a quite readable introduction to many relativity and 
quantum phenomena.  Here I collect additional notes on things that were
not clear to me.

\section{Chapter 2. }

\subsection{Equation 2.2.10}

Let 

\begin{align*}
\LL = k = g_{ij} \qdot^i \qdot^j
\end{align*}
\begin{align*}
\PD{\qdot^k}{\LL} = 2 g_{ik} \qdot^i \\
\PD{\qdot^k}{\LL} \qdot^k = 2 g_{ik} \qdot^i \qdot^k = 2 K
\end{align*}


\subsection{Equation 2.3.3}

A worldline velocity with respect to some parametrization is

\begin{align*}
\left(\frac{ds}{d\lambda}\right)^2 = \left( \frac{dx}{d\lambda} \cdot \frac{dx}{d\lambda} \right)^2
&= \left(\frac{dx^4}{d\lambda}\right)^2 - \sum_i \left(\frac{dx^i}{d\lambda}\right)^2
\end{align*}

For $\lambda = s$, we can therefore calculate $u^4$:

\begin{align*}
\left(\frac{ds}{ds}\right)^2 = 1 
&= \left(\frac{dx^4}{ds}\right)^2 - \sum_i \left(\frac{dx^i}{ds}\right)^2 \\
&= \left(\frac{dx^4}{ds}\right)^2 - \sum_i \left( \frac{dx^i}{dx^4} \frac{dx^4}{ds} \right)^2 \\
&= \left(\frac{dx^4}{ds}\right)^2 \left( 1 - \sum_i \left( \frac{dx^i}{dx^4} \right)^2 \right) \\
\end{align*}

Or
\begin{align}\label{eqn:debroglie:timearc}
\left(\frac{dx^4}{ds}\right)^2 = \inv{1 - \Bv^2/c^2 } \\
\end{align}

There is a freedom to pick either plus or minus here.  Returning to that later, first 
calculate the remainder of this table of derivatives.  Picking $x^1$ as representative

\begin{align*}
1 &= \inv{1 - \Bv^2/c^2 } 
- \left(\frac{dx^1}{ds}\right)^2
- \left( \frac{dy}{dx^4} \frac{dx^4}{ds} \right)^2
- \left( \frac{dz}{dx^4} \frac{dx^4}{ds} \right)^2 \\
\left(\frac{dx^1}{ds}\right)^2
&= \frac{\Bv^2/c^2 }{1 - \Bv^2/c^2 } 
-\inv{1 - \Bv^2/c^2 } \left( \left( \frac{dy}{dx^4} \right)^2 + \left( \frac{dz}{dx^4} \right)^2 \right) \\
&= \inv{c^2 (1 - \Bv^2/c^2) } \left( \Bv^2 - \left( \frac{dy}{dt} \right)^2 - \left( \frac{dz}{dt} \right)^2 \right) \\
&= \frac{v_x^2/c^2}{1 - \Bv^2/c^2}
\end{align*}

Now, express the coordinate vector for the worldline differential in its entirety:

\begin{align*}
\frac{dx}{ds} =
\frac{d}{ds}(x^1, x^2, x^3, x^4)
&= \frac{dx^4}{ds} \left( \frac{dx^1}{dx^4}, \frac{dx^2}{dx^4}, \frac{dx^3}{dx^4}, 1 \right) \\
&= \frac{dx^4}{ds} ( v_x/c, v_y/c, v_z/c, 1) \\
\end{align*}

This shows that the flexibility to choose a sign for the square roots to obtain $dx^\mu/ds$ must all match the sign for the $dx^4/ds$ term.  Considering a particle at rest in the implied frame associated with these coordinates, one has, by \ref{eqn:debroglie:timearc}

\begin{align*}
\frac{dx}{ds} 
&= \pm \left(0, 0, 0, \inv{\sqrt{1 - \Bv^2/c^2}} \right) \\
&= \pm \left(0, 0, 0, 1 \right) \\
\end{align*}

If we take positive $ds$ to measure increase of time in the rest frame, then there is some sense to picking the positive root.  One
wouldn't have to, since there is also a corresponding freedom to bury a sign adjustment in the $dx_\mu/ds$ derivatives.

%\bibliographystyle{plainnat}
%\bibliography{myrefs}

%\end{document}

%
% Copyright � 2012 Peeter Joot.  All Rights Reserved.
% Licenced as described in the file LICENSE under the root directory of this GIT repository.
%

% 
% 
%\documentclass{article}

%\usepackage{amsmath}
\usepackage{mathpazo}

%
% shorthand for bold symbols, convenient for vectors and matrices
%
\newcommand{\Ba}[0]{\mathbf{a}}
\newcommand{\Bb}[0]{\mathbf{b}}
\newcommand{\Bc}[0]{\mathbf{c}}
\newcommand{\Bd}[0]{\mathbf{d}}
\newcommand{\Be}[0]{\mathbf{e}}
\newcommand{\Bf}[0]{\mathbf{f}}
\newcommand{\Bg}[0]{\mathbf{g}}
\newcommand{\Bh}[0]{\mathbf{h}}
\newcommand{\Bi}[0]{\mathbf{i}}
\newcommand{\Bj}[0]{\mathbf{j}}
\newcommand{\Bk}[0]{\mathbf{k}}
\newcommand{\Bl}[0]{\mathbf{l}}
\newcommand{\Bm}[0]{\mathbf{m}}
\newcommand{\Bn}[0]{\mathbf{n}}
\newcommand{\Bo}[0]{\mathbf{o}}
\newcommand{\Bp}[0]{\mathbf{p}}
\newcommand{\Bq}[0]{\mathbf{q}}
\newcommand{\Br}[0]{\mathbf{r}}
\newcommand{\Bs}[0]{\mathbf{s}}
\newcommand{\Bt}[0]{\mathbf{t}}
\newcommand{\Bu}[0]{\mathbf{u}}
\newcommand{\Bv}[0]{\mathbf{v}}
\newcommand{\Bw}[0]{\mathbf{w}}
\newcommand{\Bx}[0]{\mathbf{x}}
\newcommand{\By}[0]{\mathbf{y}}
\newcommand{\Bz}[0]{\mathbf{z}}
\newcommand{\BA}[0]{\mathbf{A}}
\newcommand{\BB}[0]{\mathbf{B}}
\newcommand{\BC}[0]{\mathbf{C}}
\newcommand{\BD}[0]{\mathbf{D}}
\newcommand{\BE}[0]{\mathbf{E}}
\newcommand{\BF}[0]{\mathbf{F}}
\newcommand{\BG}[0]{\mathbf{G}}
\newcommand{\BH}[0]{\mathbf{H}}
\newcommand{\BI}[0]{\mathbf{I}}
\newcommand{\BJ}[0]{\mathbf{J}}
\newcommand{\BK}[0]{\mathbf{K}}
\newcommand{\BL}[0]{\mathbf{L}}
\newcommand{\BM}[0]{\mathbf{M}}
\newcommand{\BN}[0]{\mathbf{N}}
\newcommand{\BO}[0]{\mathbf{O}}
\newcommand{\BP}[0]{\mathbf{P}}
\newcommand{\BQ}[0]{\mathbf{Q}}
\newcommand{\BR}[0]{\mathbf{R}}
\newcommand{\BS}[0]{\mathbf{S}}
\newcommand{\BT}[0]{\mathbf{T}}
\newcommand{\BU}[0]{\mathbf{U}}
\newcommand{\BV}[0]{\mathbf{V}}
\newcommand{\BW}[0]{\mathbf{W}}
\newcommand{\BX}[0]{\mathbf{X}}
\newcommand{\BY}[0]{\mathbf{Y}}
\newcommand{\BZ}[0]{\mathbf{Z}}

\newcommand{\Bzero}[0]{\mathbf{0}}
\newcommand{\Btheta}[0]{\boldsymbol{\theta}}
\newcommand{\Btau}[0]{\boldsymbol{\tau}}
\newcommand{\Bomega}[0]{\boldsymbol{\omega}}

%
% shorthand for unit vectors
%
\newcommand{\acap}[0]{\hat{\Ba}}
\newcommand{\bcap}[0]{\hat{\Bb}}
\newcommand{\ccap}[0]{\hat{\Bc}}
\newcommand{\dcap}[0]{\hat{\Bd}}
\newcommand{\ecap}[0]{\hat{\Be}}
\newcommand{\fcap}[0]{\hat{\Bf}}
\newcommand{\gcap}[0]{\hat{\Bg}}
\newcommand{\hcap}[0]{\hat{\Bh}}
\newcommand{\icap}[0]{\hat{\Bi}}
\newcommand{\jcap}[0]{\hat{\Bj}}
\newcommand{\kcap}[0]{\hat{\Bk}}
\newcommand{\lcap}[0]{\hat{\Bl}}
\newcommand{\mcap}[0]{\hat{\Bm}}
\newcommand{\ncap}[0]{\hat{\Bn}}
\newcommand{\ocap}[0]{\hat{\Bo}}
\newcommand{\pcap}[0]{\hat{\Bp}}
\newcommand{\qcap}[0]{\hat{\Bq}}
\newcommand{\rcap}[0]{\hat{\Br}}
\newcommand{\scap}[0]{\hat{\Bs}}
\newcommand{\tcap}[0]{\hat{\Bt}}
\newcommand{\ucap}[0]{\hat{\Bu}}
\newcommand{\vcap}[0]{\hat{\Bv}}
\newcommand{\wcap}[0]{\hat{\Bw}}
\newcommand{\xcap}[0]{\hat{\Bx}}
\newcommand{\ycap}[0]{\hat{\By}}
\newcommand{\zcap}[0]{\hat{\Bz}}
\newcommand{\thetacap}[0]{\hat{\Btheta}}

%
% to write R^n and C^n in a distinguishable fashion.  Perhaps change this
% to the double lined characters upon figuring out how to do so.
%
\newcommand{\C}[1]{$\mathbb{C}^{#1}$}
\newcommand{\R}[1]{$\mathbb{R}^{#1}$}

%
% various generally useful helpers
%

% derivative of #1 wrt. #2:
\newcommand{\D}[2] {\frac {d#2} {d#1}}

\newcommand{\inv}[1]{\frac{1}{#1}}
\newcommand{\cross}[0]{\times}

\newcommand{\abs}[1]{\lvert{#1}\rvert}
\newcommand{\norm}[1]{\lVert{#1}\rVert}
\newcommand{\innerprod}[2]{\langle{#1}, {#2}\rangle}
\newcommand{\dotprod}[2]{{#1} \cdot {#2}}
\newcommand{\bdotprod}[2]{\left({#1} \cdot {#2}\right)}
\newcommand{\crossprod}[2]{{#1} \cross {#2}}
\newcommand{\tripleprod}[3]{\dotprod{\left(\crossprod{#1}{#2}\right)}{#3}}

\DeclareMathOperator{\Proj}{Proj}
\DeclareMathOperator{\Span}{span}
\DeclareMathOperator{\Sgn}{sgn}
\DeclareMathOperator{\Area}{Area}
\DeclareMathOperator{\Volume}{Volume}

%
% A few miscellaneous things specific to this document
%
\newcommand{\crossop}[1]{\crossprod{#1}{}}

% R2 vector.
\newcommand{\VectorTwo}[2]{
\begin{bmatrix}
 {#1} \\
 {#2}
\end{bmatrix}
}

\newcommand{\VectorN}[1]{
\begin{bmatrix}
{#1}_1 \\
{#1}_2 \\
\vdots \\
{#1}_N \\
\end{bmatrix}
}

\newcommand{\DETuvij}[4]{
\begin{vmatrix}
 {#1}_{#3} & {#1}_{#4} \\
 {#2}_{#3} & {#2}_{#4}
\end{vmatrix}
}

\newcommand{\DETuvwijk}[6]{
\begin{vmatrix}
 {#1}_{#4} & {#1}_{#5} & {#1}_{#6} \\
 {#2}_{#4} & {#2}_{#5} & {#2}_{#6} \\
 {#3}_{#4} & {#3}_{#5} & {#3}_{#6}
\end{vmatrix}
}

\newcommand{\DETuvwxijkl}[8]{
\begin{vmatrix}
 {#1}_{#5} & {#1}_{#6} & {#1}_{#7} & {#1}_{#8} \\
 {#2}_{#5} & {#2}_{#6} & {#2}_{#7} & {#2}_{#8} \\
 {#3}_{#5} & {#3}_{#6} & {#3}_{#7} & {#3}_{#8} \\
 {#4}_{#5} & {#4}_{#6} & {#4}_{#7} & {#4}_{#8} \\
\end{vmatrix}
}

%\newcommand{\DETuvwxyijklm}[10]{
%\begin{vmatrix}
% {#1}_{#6} & {#1}_{#7} & {#1}_{#8} & {#1}_{#9} & {#1}_{#10} \\
% {#2}_{#6} & {#2}_{#7} & {#2}_{#8} & {#2}_{#9} & {#2}_{#10} \\
% {#3}_{#6} & {#3}_{#7} & {#3}_{#8} & {#3}_{#9} & {#3}_{#10} \\
% {#4}_{#6} & {#4}_{#7} & {#4}_{#8} & {#4}_{#9} & {#4}_{#10} \\
% {#5}_{#6} & {#5}_{#7} & {#5}_{#8} & {#5}_{#9} & {#5}_{#10}
%\end{vmatrix}
%}

% R3 vector.
\newcommand{\VectorThree}[3]{
\begin{bmatrix}
 {#1} \\
 {#2} \\
 {#3}
\end{bmatrix}
}


%%<misc>
%
\newcommand{\Abs}[1]{{\left\lvert{#1}\right\rvert}}
\newcommand{\spacegrad}[0]{\boldsymbol{\nabla}}
\newcommand{\grad}[0]{\nabla}
\newcommand{\LL}[0]{\mathcal{L}}

% == \partial_{#1} {#2}
\newcommand{\PD}[2]{\frac{\partial {#2}}{\partial {#1}}}
% inline variant
\newcommand{\PDi}[2]{{\partial {#2}}/{\partial {#1}}}

\newcommand{\PDD}[3]{\frac{\partial^2 {#3}}{\partial {#1}\partial {#2}}}
%\newcommand{\PDd}[2]{\frac{\partial^2 {#2}}{{\partial{#1}}^2}}
\newcommand{\PDsq}[2]{\frac{\partial^2 {#2}}{(\partial {#1})^2}}

\newcommand{\Partial}[2]{\frac{\partial {#1}}{\partial {#2}}}
\DeclareMathOperator{\RejName}{Rej}
\newcommand{\Rej}[2]{\RejName_{#1}\left( {#2} \right)}
\newcommand{\Rm}[1]{\mathbb{R}^{#1}}
\newcommand{\Cm}[1]{\mathbb{C}^{#1}}
\newcommand{\conj}[0]{{*}}

%</misc>

% <grade selection>
%
\newcommand{\gpgrade}[2] {{\left\langle{{#1}}\right\rangle}_{#2}}

\newcommand{\gpgradezero}[1] {\gpgrade{#1}{}}
%\newcommand{\gpscalargrade}[1] {{\left\langle{{#1}}\right\rangle}}
%\newcommand{\gpgradezero}[1] {\gpgrade{#1}{0}}

%\newcommand{\gpgradeone}[1] {{\left\langle{{#1}}\right\rangle}_{1}}
\newcommand{\gpgradeone}[1] {\gpgrade{#1}{1}}

\newcommand{\gpgradetwo}[1] {\gpgrade{#1}{2}}
\newcommand{\gpgradethree}[1] {\gpgrade{#1}{3}}
\newcommand{\gpgradefour}[1] {\gpgrade{#1}{4}}
%
% </grade selection>



\newcommand{\adot}[0]{{\dot{a}}}
\newcommand{\bdot}[0]{{\dot{b}}}
% taken for centered dot:
%\newcommand{\cdot}[0]{{\dot{c}}}
%\newcommand{\ddot}[0]{{\dot{d}}}
\newcommand{\edot}[0]{{\dot{e}}}
\newcommand{\fdot}[0]{{\dot{f}}}
\newcommand{\gdot}[0]{{\dot{g}}}
\newcommand{\hdot}[0]{{\dot{h}}}
\newcommand{\idot}[0]{{\dot{i}}}
\newcommand{\jdot}[0]{{\dot{j}}}
\newcommand{\kdot}[0]{{\dot{k}}}
\newcommand{\ldot}[0]{{\dot{l}}}
\newcommand{\mdot}[0]{{\dot{m}}}
\newcommand{\ndot}[0]{{\dot{n}}}
%\newcommand{\odot}[0]{{\dot{o}}}
\newcommand{\pdot}[0]{{\dot{p}}}
\newcommand{\qdot}[0]{{\dot{q}}}
\newcommand{\rdot}[0]{{\dot{r}}}
\newcommand{\sdot}[0]{{\dot{s}}}
\newcommand{\tdot}[0]{{\dot{t}}}
\newcommand{\udot}[0]{{\dot{u}}}
\newcommand{\vdot}[0]{{\dot{v}}}
\newcommand{\wdot}[0]{{\dot{w}}}
\newcommand{\xdot}[0]{{\dot{x}}}
\newcommand{\ydot}[0]{{\dot{y}}}
\newcommand{\zdot}[0]{{\dot{z}}}
\newcommand{\addot}[0]{{\ddot{a}}}
\newcommand{\bddot}[0]{{\ddot{b}}}
\newcommand{\cddot}[0]{{\ddot{c}}}
%\newcommand{\dddot}[0]{{\ddot{d}}}
\newcommand{\eddot}[0]{{\ddot{e}}}
\newcommand{\fddot}[0]{{\ddot{f}}}
\newcommand{\gddot}[0]{{\ddot{g}}}
\newcommand{\hddot}[0]{{\ddot{h}}}
\newcommand{\iddot}[0]{{\ddot{i}}}
\newcommand{\jddot}[0]{{\ddot{j}}}
\newcommand{\kddot}[0]{{\ddot{k}}}
\newcommand{\lddot}[0]{{\ddot{l}}}
\newcommand{\mddot}[0]{{\ddot{m}}}
\newcommand{\nddot}[0]{{\ddot{n}}}
\newcommand{\oddot}[0]{{\ddot{o}}}
\newcommand{\pddot}[0]{{\ddot{p}}}
\newcommand{\qddot}[0]{{\ddot{q}}}
\newcommand{\rddot}[0]{{\ddot{r}}}
\newcommand{\sddot}[0]{{\ddot{s}}}
\newcommand{\tddot}[0]{{\ddot{t}}}
\newcommand{\uddot}[0]{{\ddot{u}}}
\newcommand{\vddot}[0]{{\ddot{v}}}
\newcommand{\wddot}[0]{{\ddot{w}}}
\newcommand{\xddot}[0]{{\ddot{x}}}
\newcommand{\yddot}[0]{{\ddot{y}}}
\newcommand{\zddot}[0]{{\ddot{z}}}

%<bold and dot greek symbols>
%

\newcommand{\Deltadot}[0]{{\dot{\Delta}}}
\newcommand{\Gammadot}[0]{{\dot{\Gamma}}}
\newcommand{\Lambdadot}[0]{{\dot{\Lambda}}}
\newcommand{\Omegadot}[0]{{\dot{\Omega}}}
\newcommand{\Phidot}[0]{{\dot{\Phi}}}
\newcommand{\Pidot}[0]{{\dot{\Pi}}}
\newcommand{\Psidot}[0]{{\dot{\Psi}}}
\newcommand{\Sigmadot}[0]{{\dot{\Sigma}}}
\newcommand{\Thetadot}[0]{{\dot{\Theta}}}
\newcommand{\Upsilondot}[0]{{\dot{\Upsilon}}}
\newcommand{\Xidot}[0]{{\dot{\Xi}}}
\newcommand{\alphadot}[0]{{\dot{\alpha}}}
\newcommand{\betadot}[0]{{\dot{\beta}}}
\newcommand{\chidot}[0]{{\dot{\chi}}}
\newcommand{\deltadot}[0]{{\dot{\delta}}}
\newcommand{\epsilondot}[0]{{\dot{\epsilon}}}
\newcommand{\etadot}[0]{{\dot{\eta}}}
\newcommand{\gammadot}[0]{{\dot{\gamma}}}
\newcommand{\kappadot}[0]{{\dot{\kappa}}}
\newcommand{\lambdadot}[0]{{\dot{\lambda}}}
\newcommand{\mudot}[0]{{\dot{\mu}}}
\newcommand{\nudot}[0]{{\dot{\nu}}}
\newcommand{\omegadot}[0]{{\dot{\omega}}}
\newcommand{\phidot}[0]{{\dot{\phi}}}
\newcommand{\pidot}[0]{{\dot{\pi}}}
\newcommand{\psidot}[0]{{\dot{\psi}}}
\newcommand{\rhodot}[0]{{\dot{\rho}}}
\newcommand{\sigmadot}[0]{{\dot{\sigma}}}
\newcommand{\taudot}[0]{{\dot{\tau}}}
\newcommand{\thetadot}[0]{{\dot{\theta}}}
\newcommand{\upsilondot}[0]{{\dot{\upsilon}}}
\newcommand{\varepsilondot}[0]{{\dot{\varepsilon}}}
\newcommand{\varphidot}[0]{{\dot{\varphi}}}
\newcommand{\varpidot}[0]{{\dot{\varpi}}}
\newcommand{\varrhodot}[0]{{\dot{\varrho}}}
\newcommand{\varsigmadot}[0]{{\dot{\varsigma}}}
\newcommand{\varthetadot}[0]{{\dot{\vartheta}}}
\newcommand{\xidot}[0]{{\dot{\xi}}}
\newcommand{\zetadot}[0]{{\dot{\zeta}}}

\newcommand{\Deltaddot}[0]{{\ddot{\Delta}}}
\newcommand{\Gammaddot}[0]{{\ddot{\Gamma}}}
\newcommand{\Lambdaddot}[0]{{\ddot{\Lambda}}}
\newcommand{\Omegaddot}[0]{{\ddot{\Omega}}}
\newcommand{\Phiddot}[0]{{\ddot{\Phi}}}
\newcommand{\Piddot}[0]{{\ddot{\Pi}}}
\newcommand{\Psiddot}[0]{{\ddot{\Psi}}}
\newcommand{\Sigmaddot}[0]{{\ddot{\Sigma}}}
\newcommand{\Thetaddot}[0]{{\ddot{\Theta}}}
\newcommand{\Upsilonddot}[0]{{\ddot{\Upsilon}}}
\newcommand{\Xiddot}[0]{{\ddot{\Xi}}}
\newcommand{\alphaddot}[0]{{\ddot{\alpha}}}
\newcommand{\betaddot}[0]{{\ddot{\beta}}}
\newcommand{\chiddot}[0]{{\ddot{\chi}}}
\newcommand{\deltaddot}[0]{{\ddot{\delta}}}
\newcommand{\epsilonddot}[0]{{\ddot{\epsilon}}}
\newcommand{\etaddot}[0]{{\ddot{\eta}}}
\newcommand{\gammaddot}[0]{{\ddot{\gamma}}}
\newcommand{\kappaddot}[0]{{\ddot{\kappa}}}
\newcommand{\lambdaddot}[0]{{\ddot{\lambda}}}
\newcommand{\muddot}[0]{{\ddot{\mu}}}
\newcommand{\nuddot}[0]{{\ddot{\nu}}}
\newcommand{\omegaddot}[0]{{\ddot{\omega}}}
\newcommand{\phiddot}[0]{{\ddot{\phi}}}
\newcommand{\piddot}[0]{{\ddot{\pi}}}
\newcommand{\psiddot}[0]{{\ddot{\psi}}}
\newcommand{\rhoddot}[0]{{\ddot{\rho}}}
\newcommand{\sigmaddot}[0]{{\ddot{\sigma}}}
\newcommand{\tauddot}[0]{{\ddot{\tau}}}
\newcommand{\thetaddot}[0]{{\ddot{\theta}}}
\newcommand{\upsilonddot}[0]{{\ddot{\upsilon}}}
\newcommand{\varepsilonddot}[0]{{\ddot{\varepsilon}}}
\newcommand{\varphiddot}[0]{{\ddot{\varphi}}}
\newcommand{\varpiddot}[0]{{\ddot{\varpi}}}
\newcommand{\varrhoddot}[0]{{\ddot{\varrho}}}
\newcommand{\varsigmaddot}[0]{{\ddot{\varsigma}}}
\newcommand{\varthetaddot}[0]{{\ddot{\vartheta}}}
\newcommand{\xiddot}[0]{{\ddot{\xi}}}
\newcommand{\zetaddot}[0]{{\ddot{\zeta}}}

\newcommand{\BDelta}[0]{\boldsymbol{\Delta}}
\newcommand{\BGamma}[0]{\boldsymbol{\Gamma}}
\newcommand{\BLambda}[0]{\boldsymbol{\Lambda}}
\newcommand{\BOmega}[0]{\boldsymbol{\Omega}}
\newcommand{\BPhi}[0]{\boldsymbol{\Phi}}
\newcommand{\BPi}[0]{\boldsymbol{\Pi}}
\newcommand{\BPsi}[0]{\boldsymbol{\Psi}}
\newcommand{\BSigma}[0]{\boldsymbol{\Sigma}}
\newcommand{\BTheta}[0]{\boldsymbol{\Theta}}
\newcommand{\BUpsilon}[0]{\boldsymbol{\Upsilon}}
\newcommand{\BXi}[0]{\boldsymbol{\Xi}}
\newcommand{\Balpha}[0]{\boldsymbol{\alpha}}
\newcommand{\Bbeta}[0]{\boldsymbol{\beta}}
\newcommand{\Bchi}[0]{\boldsymbol{\chi}}
\newcommand{\Bdelta}[0]{\boldsymbol{\delta}}
\newcommand{\Bepsilon}[0]{\boldsymbol{\epsilon}}
\newcommand{\Beta}[0]{\boldsymbol{\eta}}
\newcommand{\Bgamma}[0]{\boldsymbol{\gamma}}
\newcommand{\Bkappa}[0]{\boldsymbol{\kappa}}
\newcommand{\Blambda}[0]{\boldsymbol{\lambda}}
\newcommand{\Bmu}[0]{\boldsymbol{\mu}}
\newcommand{\Bnu}[0]{\boldsymbol{\nu}}
%\newcommand{\Bomega}[0]{\boldsymbol{\omega}}
\newcommand{\Bphi}[0]{\boldsymbol{\phi}}
\newcommand{\Bpi}[0]{\boldsymbol{\pi}}
\newcommand{\Bpsi}[0]{\boldsymbol{\psi}}
\newcommand{\Brho}[0]{\boldsymbol{\rho}}
\newcommand{\Bsigma}[0]{\boldsymbol{\sigma}}
%\newcommand{\Btau}[0]{\boldsymbol{\tau}}
%\newcommand{\Btheta}[0]{\boldsymbol{\theta}}
\newcommand{\Bupsilon}[0]{\boldsymbol{\upsilon}}
\newcommand{\Bvarepsilon}[0]{\boldsymbol{\varepsilon}}
\newcommand{\Bvarphi}[0]{\boldsymbol{\varphi}}
\newcommand{\Bvarpi}[0]{\boldsymbol{\varpi}}
\newcommand{\Bvarrho}[0]{\boldsymbol{\varrho}}
\newcommand{\Bvarsigma}[0]{\boldsymbol{\varsigma}}
\newcommand{\Bvartheta}[0]{\boldsymbol{\vartheta}}
\newcommand{\Bxi}[0]{\boldsymbol{\xi}}
\newcommand{\Bzeta}[0]{\boldsymbol{\zeta}}
%
%</bold and dot greek symbols>
%<infrequent>
%
%\newcommand{\AreaOp}[1]{\AName_{#1}}
%\newcommand{\Babs}[0]{\abs{\BB}}
%\newcommand{\Bcap}[0]{\hat{\BB}}
%\newcommand{\BrPrimeRej}[0]{\rcap(\rcap \wedge \Br')}
%\newcommand{\CA}[0]{\mathcal{A}}
%\newcommand{\Cos}[1]{\cos{\left({#1}\right)}}
%\newcommand{\Det}[1] {\abs{#1}}
%\newcommand{\Dsq}[2] {\frac {\partial^2 {#1}} {\partial {#2}^2}}
%\newcommand{\Exp}[1]{\exp{\left({#1}\right)}}
%\newcommand{\Norm}[1]{\left\lVert{#1}\right\rVert}
%\newcommand{\Sin}[1]{\sin{\left({#1}\right)}}
%\newcommand{\T}[0]{\text{T}}
%\newcommand{\VolumeOp}[1]{\VName_{#1}}
%\newcommand{\agrad}[0]{\Ba \cdot \nabla}
%\newcommand{\alphacap}[0]{\hat{\boldsymbol{\alpha}}}
%\newcommand{\Fcap}[0]{\hat{\BF}}
%\newcommand{\bithree}[0]{{\Bi}_3}
%\newcommand{\bxa}[0]{\Bx\Ba}
%\newcommand{\coordvec}[2]{
%\newcommand{\costheta}[0]{\acap \cdot \xcap}
%\newcommand{\ddt}[1]{\ddot{#1}}
%\newcommand{\ddu}[1] {\frac {d{#1}} {du}}
%\newcommand{\dsqxj}[2] {\frac {\partial^2 {#1}} {\partial {x_{#2}}^2}}
%\newcommand{\dtheta}[1]{\frac{d {#1}}{d \theta}}
%\newcommand{\dt}[1]{\dot{#1}}
%\newcommand{\dt}[1]{\frac{d {#1}}{dt}}
%\newcommand{\dxj}[2] {\frac {\partial {#1}} {\partial {x_{#2}}}}
%\newcommand{\halfPhi}[0]{\frac{\phi}{2}}
%\newcommand{\half}[0]{\inv{2}}
%\newcommand{\inv}[1]{\frac{1}{#1}}
%\newcommand{\laplacian}[0]{\nabla^2}
%\newcommand{\matrixoftx}[3]{
%\newcommand{\nrrp}[0]{\norm{\rcap \wedge \Br'}}
%\newcommand{\oiint}{\bigcirc \hspace{-1.4em} \int \hspace{-.8em} \int}
%\newcommand{\transpose}[1]{{#1}^{\text{T}}}
%\newcommand{\transpose}[1]{{{#1}^{\TextTranspose}}}
%\newcommand{\transpose}[1]{{{#1}^{\text{T}}}}
%\newcommand{\barA}[0]{\bar{A}}
%\newcommand{\qbar}[0]{\bar{q}}
%\newcommand{\qdotbar}[0]{\dot{\bar{q}}}
%
%</infrequent>





%\usepackage[bookmarks=true]{hyperref}

%\usepackage{color,cite,graphicx}
   % use colour in the document, put your citations as [1-4]
   % rather than [1,2,3,4] (it looks nicer, and the extended LaTeX2e
   % graphics package. 
%\usepackage{latexsym,amssymb,epsf} % do not remember if these are
   % needed, but their inclusion can not do any damage

\chapter{Ad-hoc motivation of some QM wave equations}
\label{chap:sch}
%\author{Peeter Joot \quad peeterjoot@protonmail.com}
\date{ Dec 13, 2008.  sch.tex }

%\begin{document}

%\maketitle{}

%\tableofcontents

\section{Motivation}

\subsection{Non-Relativistic case}

A common (cf: wikipedia and \citep{french1998iqp}) introductory motivation for the non-relativistic Schr\"{o}dinger's equation appears to follow the following lines.  Assume that
we desire a plane wave equation of the following form

\begin{equation}\label{eqn:sch:20}
\begin{aligned}
\psi = \exp(i (\Bk \cdot \Bx - \omega t))
\end{aligned}
\end{equation}

plus a requirement that we have a total energy that can be expressed in terms of kinetic plus potential

\begin{equation}\label{eqn:sch:energy}
\begin{aligned}
E = \frac{\Bp^2}{2m} + V
\end{aligned}
\end{equation}

We also have the Einstein relationship

\begin{equation}\label{eqn:sch:40}
\begin{aligned}
E = h \nu = \frac{h}{2\pi} 2 \pi \nu = \Hbar \omega
\end{aligned}
\end{equation}

and the DeBroglie Hypothesis for the magnitude of the momentum of a particle

\begin{equation}\label{eqn:sch:60}
\begin{aligned}
p = \frac{h}{\lambda}.
\end{aligned}
\end{equation}

In terms of wave number \(2 \pi k = \inv{\lambda}\) this last is

\begin{equation}\label{eqn:sch:80}
\begin{aligned}
p = \frac{h}{2\pi} k = \Hbar k.
\end{aligned}
\end{equation}

or in three dimensions

\begin{equation}\label{eqn:sch:100}
\begin{aligned}
\Bp = \Hbar \Bk.
\end{aligned}
\end{equation}

Taking derivatives of the postulated wave function one has

\begin{equation}\label{eqn:sch:timepartial}
\begin{aligned}
\frac{\partial \psi}{\partial t} = -i \omega \psi
\end{aligned}
\end{equation}

and 
\begin{equation}\label{eqn:sch:laplacian}
\begin{aligned}
\spacegrad^2 \psi = i^2 \sum_j k_j^2 \psi = - \Bk^2 \psi
\end{aligned}
\end{equation}

From the energy relationship, if one requires that
\begin{equation}\label{eqn:sch:120}
\begin{aligned}
E \psi &= \left(\frac{\Bp^2}{2m} + V\right) \psi \\
\Hbar \omega \psi &= \left(\Hbar^2 \frac{\Bk^2}{2m} + V\right) \psi \\
\end{aligned}
\end{equation}

and then substituting the derivatives from equations \eqnref{eqn:sch:timepartial} and \eqnref{eqn:sch:laplacian} we have

\begin{equation}\label{eqn:sch:140}
\begin{aligned}
i \Hbar \PD{t}{\psi} &= \left(-\frac{\Hbar^2}{2m} \spacegrad^2  + V\right) \psi \\
\end{aligned}
\end{equation}

\subsection{Relativistic case}

The relativistic force free Schr\"{o}dinger's equation is motivated by \citep{srednicki2007qft} replacing the Hamiltonian operator \(H = \BP^2/{2 m}\) with

\begin{equation}\label{eqn:sch:160}
\begin{aligned}
H = \sqrt{m^2 c^4 + \BP^2 c^2} \approx m c^2 + \BP^2/2m
\end{aligned}
\end{equation}

for

\begin{equation}\label{eqn:sch:180}
\begin{aligned}
i \Hbar \PD{t}{\psi} = \sqrt{ -\Hbar^2 c^2 \spacegrad^2 + m^2 c^4} \psi
\end{aligned}
\end{equation}

then squaring the operators on both sides, removing the root:

\begin{equation}\label{eqn:sch:200}
\begin{aligned}
- \Hbar^2 \PDsq{x^0}{\psi} &= \left(-\Hbar^2 \spacegrad^2 + m^2 c^2 \right) \psi \\
- \Hbar^2 \PDsq{x^0}{\psi} + \Hbar^2 \spacegrad^2 &= m^2 c^2 \psi \\
\end{aligned}
\end{equation}

Which is called the Klein-Gordon equation:
\begin{equation}\label{eqn:sch:220}
\begin{aligned}
\left(\frac{\Hbar^2}{2m} \grad^2 + \inv{2} m c^2\right) \psi = 0 \\
\end{aligned}
\end{equation}

The Noether's theorem wikipedia article, and Klein-Gordon pages give the action
as (removing use of natural units and changing to a \(+---\) metric) gives:

\begin{equation}\label{eqn:sch:240}
\begin{aligned}
\LL &= -\eta^{\mu\nu} \partial_\mu \psi \partial_\nu \psi^\conj + \frac{m^2 c^2}{\Hbar^2} \psi \psi^\conj \\
&= -\partial^\nu \psi \partial_\nu \psi^\conj + \frac{m^2 c^2}{\Hbar^2} \psi \psi^\conj \\
\end{aligned}
\end{equation}

This first term is a squared spacetime gradient

\begin{equation}\label{eqn:sch:260}
\begin{aligned}
(\grad \psi) \cdot (\grad \psi^\conj) 
&= (\gamma^\mu \partial_\mu \psi) \cdot (\gamma_\nu \partial^\nu \psi^\conj) \\
&= {\delta^\mu}_\nu \partial_\mu \psi \partial^\nu \psi^\conj \\
&= \partial_\mu \psi \partial^\mu \psi^\conj \\
\end{aligned}
\end{equation}

so we have
\begin{equation}\label{eqn:sch:280}
\begin{aligned}
\LL 
&= -\eta^{\mu\nu} \partial_\mu \psi \partial_\nu \psi^\conj + \frac{m^2 c^2}{\Hbar^2} \psi \psi^\conj \\
&= -(\grad \psi) \cdot (\grad \psi^\conj) + \frac{m^2 c^2}{\Hbar^2} \psi \psi^\conj \\
\end{aligned}
\end{equation}

So, here we expect to get a spacetime Laplacian, like the Maxwell potential equation, and one does:

\begin{equation}\label{eqn:sch:300}
\begin{aligned}
-\grad^2 \psi = \frac{m^2 c^2}{\Hbar^2} \psi
\end{aligned}
\end{equation}

Now, the Srednicki text indicates that Dirac linearized this equation, to get back something that was
first order in the time derivative.

I do not yet follow that description, but see that a linearization is possible by taking roots of the operators above,
undoing the somewhat fishy seeming squaring done to arrive at the Klein-Gordon equation in the first place

\begin{equation}\label{eqn:sch:dirac}
\begin{aligned}
i \Hbar \grad \psi = \pm m c \psi
\end{aligned}
\end{equation}

Is this equivalent to Dirac's formulation?  Comparison to 
\href{http://en.wikipedia.org/wiki/Dirac_equation#Covariant_form_and_relativistic_invariance}, where the Dirac equation is given in covariant
form

\begin{equation}\label{eqn:sch:320}
\begin{aligned}
i \Hbar \gamma^\mu \partial_\mu \psi - m c \psi = 0
\end{aligned}
\end{equation}

which is the positive variant of \eqnref{eqn:sch:dirac}.

What is the Lagrangian that is associated with this?  The most probable interpretation of \(i\) here is the Minkowski pseudoscalar, as opposed to
a unit bivector of spacetime vectors or some other geometrical object with \(-1\) square.

%\bibliographystyle{plainnat}
%\bibliography{myrefs}

%\end{document}

%\documentclass{article}

%\usepackage{amsmath}
\usepackage{mathpazo}

%
% shorthand for bold symbols, convenient for vectors and matrices
%
\newcommand{\Ba}[0]{\mathbf{a}}
\newcommand{\Bb}[0]{\mathbf{b}}
\newcommand{\Bc}[0]{\mathbf{c}}
\newcommand{\Bd}[0]{\mathbf{d}}
\newcommand{\Be}[0]{\mathbf{e}}
\newcommand{\Bf}[0]{\mathbf{f}}
\newcommand{\Bg}[0]{\mathbf{g}}
\newcommand{\Bh}[0]{\mathbf{h}}
\newcommand{\Bi}[0]{\mathbf{i}}
\newcommand{\Bj}[0]{\mathbf{j}}
\newcommand{\Bk}[0]{\mathbf{k}}
\newcommand{\Bl}[0]{\mathbf{l}}
\newcommand{\Bm}[0]{\mathbf{m}}
\newcommand{\Bn}[0]{\mathbf{n}}
\newcommand{\Bo}[0]{\mathbf{o}}
\newcommand{\Bp}[0]{\mathbf{p}}
\newcommand{\Bq}[0]{\mathbf{q}}
\newcommand{\Br}[0]{\mathbf{r}}
\newcommand{\Bs}[0]{\mathbf{s}}
\newcommand{\Bt}[0]{\mathbf{t}}
\newcommand{\Bu}[0]{\mathbf{u}}
\newcommand{\Bv}[0]{\mathbf{v}}
\newcommand{\Bw}[0]{\mathbf{w}}
\newcommand{\Bx}[0]{\mathbf{x}}
\newcommand{\By}[0]{\mathbf{y}}
\newcommand{\Bz}[0]{\mathbf{z}}
\newcommand{\BA}[0]{\mathbf{A}}
\newcommand{\BB}[0]{\mathbf{B}}
\newcommand{\BC}[0]{\mathbf{C}}
\newcommand{\BD}[0]{\mathbf{D}}
\newcommand{\BE}[0]{\mathbf{E}}
\newcommand{\BF}[0]{\mathbf{F}}
\newcommand{\BG}[0]{\mathbf{G}}
\newcommand{\BH}[0]{\mathbf{H}}
\newcommand{\BI}[0]{\mathbf{I}}
\newcommand{\BJ}[0]{\mathbf{J}}
\newcommand{\BK}[0]{\mathbf{K}}
\newcommand{\BL}[0]{\mathbf{L}}
\newcommand{\BM}[0]{\mathbf{M}}
\newcommand{\BN}[0]{\mathbf{N}}
\newcommand{\BO}[0]{\mathbf{O}}
\newcommand{\BP}[0]{\mathbf{P}}
\newcommand{\BQ}[0]{\mathbf{Q}}
\newcommand{\BR}[0]{\mathbf{R}}
\newcommand{\BS}[0]{\mathbf{S}}
\newcommand{\BT}[0]{\mathbf{T}}
\newcommand{\BU}[0]{\mathbf{U}}
\newcommand{\BV}[0]{\mathbf{V}}
\newcommand{\BW}[0]{\mathbf{W}}
\newcommand{\BX}[0]{\mathbf{X}}
\newcommand{\BY}[0]{\mathbf{Y}}
\newcommand{\BZ}[0]{\mathbf{Z}}

\newcommand{\Bzero}[0]{\mathbf{0}}
\newcommand{\Btheta}[0]{\boldsymbol{\theta}}
\newcommand{\Btau}[0]{\boldsymbol{\tau}}
\newcommand{\Bomega}[0]{\boldsymbol{\omega}}

%
% shorthand for unit vectors
%
\newcommand{\acap}[0]{\hat{\Ba}}
\newcommand{\bcap}[0]{\hat{\Bb}}
\newcommand{\ccap}[0]{\hat{\Bc}}
\newcommand{\dcap}[0]{\hat{\Bd}}
\newcommand{\ecap}[0]{\hat{\Be}}
\newcommand{\fcap}[0]{\hat{\Bf}}
\newcommand{\gcap}[0]{\hat{\Bg}}
\newcommand{\hcap}[0]{\hat{\Bh}}
\newcommand{\icap}[0]{\hat{\Bi}}
\newcommand{\jcap}[0]{\hat{\Bj}}
\newcommand{\kcap}[0]{\hat{\Bk}}
\newcommand{\lcap}[0]{\hat{\Bl}}
\newcommand{\mcap}[0]{\hat{\Bm}}
\newcommand{\ncap}[0]{\hat{\Bn}}
\newcommand{\ocap}[0]{\hat{\Bo}}
\newcommand{\pcap}[0]{\hat{\Bp}}
\newcommand{\qcap}[0]{\hat{\Bq}}
\newcommand{\rcap}[0]{\hat{\Br}}
\newcommand{\scap}[0]{\hat{\Bs}}
\newcommand{\tcap}[0]{\hat{\Bt}}
\newcommand{\ucap}[0]{\hat{\Bu}}
\newcommand{\vcap}[0]{\hat{\Bv}}
\newcommand{\wcap}[0]{\hat{\Bw}}
\newcommand{\xcap}[0]{\hat{\Bx}}
\newcommand{\ycap}[0]{\hat{\By}}
\newcommand{\zcap}[0]{\hat{\Bz}}
\newcommand{\thetacap}[0]{\hat{\Btheta}}

%
% to write R^n and C^n in a distinguishable fashion.  Perhaps change this
% to the double lined characters upon figuring out how to do so.
%
\newcommand{\C}[1]{$\mathbb{C}^{#1}$}
\newcommand{\R}[1]{$\mathbb{R}^{#1}$}

%
% various generally useful helpers
%

% derivative of #1 wrt. #2:
\newcommand{\D}[2] {\frac {d#2} {d#1}}

\newcommand{\inv}[1]{\frac{1}{#1}}
\newcommand{\cross}[0]{\times}

\newcommand{\abs}[1]{\lvert{#1}\rvert}
\newcommand{\norm}[1]{\lVert{#1}\rVert}
\newcommand{\innerprod}[2]{\langle{#1}, {#2}\rangle}
\newcommand{\dotprod}[2]{{#1} \cdot {#2}}
\newcommand{\bdotprod}[2]{\left({#1} \cdot {#2}\right)}
\newcommand{\crossprod}[2]{{#1} \cross {#2}}
\newcommand{\tripleprod}[3]{\dotprod{\left(\crossprod{#1}{#2}\right)}{#3}}

\DeclareMathOperator{\Proj}{Proj}
\DeclareMathOperator{\Span}{span}
\DeclareMathOperator{\Sgn}{sgn}
\DeclareMathOperator{\Area}{Area}
\DeclareMathOperator{\Volume}{Volume}

%
% A few miscellaneous things specific to this document
%
\newcommand{\crossop}[1]{\crossprod{#1}{}}

% R2 vector.
\newcommand{\VectorTwo}[2]{
\begin{bmatrix}
 {#1} \\
 {#2}
\end{bmatrix}
}

\newcommand{\VectorN}[1]{
\begin{bmatrix}
{#1}_1 \\
{#1}_2 \\
\vdots \\
{#1}_N \\
\end{bmatrix}
}

\newcommand{\DETuvij}[4]{
\begin{vmatrix}
 {#1}_{#3} & {#1}_{#4} \\
 {#2}_{#3} & {#2}_{#4}
\end{vmatrix}
}

\newcommand{\DETuvwijk}[6]{
\begin{vmatrix}
 {#1}_{#4} & {#1}_{#5} & {#1}_{#6} \\
 {#2}_{#4} & {#2}_{#5} & {#2}_{#6} \\
 {#3}_{#4} & {#3}_{#5} & {#3}_{#6}
\end{vmatrix}
}

\newcommand{\DETuvwxijkl}[8]{
\begin{vmatrix}
 {#1}_{#5} & {#1}_{#6} & {#1}_{#7} & {#1}_{#8} \\
 {#2}_{#5} & {#2}_{#6} & {#2}_{#7} & {#2}_{#8} \\
 {#3}_{#5} & {#3}_{#6} & {#3}_{#7} & {#3}_{#8} \\
 {#4}_{#5} & {#4}_{#6} & {#4}_{#7} & {#4}_{#8} \\
\end{vmatrix}
}

%\newcommand{\DETuvwxyijklm}[10]{
%\begin{vmatrix}
% {#1}_{#6} & {#1}_{#7} & {#1}_{#8} & {#1}_{#9} & {#1}_{#10} \\
% {#2}_{#6} & {#2}_{#7} & {#2}_{#8} & {#2}_{#9} & {#2}_{#10} \\
% {#3}_{#6} & {#3}_{#7} & {#3}_{#8} & {#3}_{#9} & {#3}_{#10} \\
% {#4}_{#6} & {#4}_{#7} & {#4}_{#8} & {#4}_{#9} & {#4}_{#10} \\
% {#5}_{#6} & {#5}_{#7} & {#5}_{#8} & {#5}_{#9} & {#5}_{#10}
%\end{vmatrix}
%}

% R3 vector.
\newcommand{\VectorThree}[3]{
\begin{bmatrix}
 {#1} \\
 {#2} \\
 {#3}
\end{bmatrix}
}


%%<misc>
%
\newcommand{\Abs}[1]{{\left\lvert{#1}\right\rvert}}
\newcommand{\spacegrad}[0]{\boldsymbol{\nabla}}
\newcommand{\grad}[0]{\nabla}
\newcommand{\LL}[0]{\mathcal{L}}

% == \partial_{#1} {#2}
\newcommand{\PD}[2]{\frac{\partial {#2}}{\partial {#1}}}
% inline variant
\newcommand{\PDi}[2]{{\partial {#2}}/{\partial {#1}}}

\newcommand{\PDD}[3]{\frac{\partial^2 {#3}}{\partial {#1}\partial {#2}}}
%\newcommand{\PDd}[2]{\frac{\partial^2 {#2}}{{\partial{#1}}^2}}
\newcommand{\PDsq}[2]{\frac{\partial^2 {#2}}{(\partial {#1})^2}}

\newcommand{\Partial}[2]{\frac{\partial {#1}}{\partial {#2}}}
\DeclareMathOperator{\RejName}{Rej}
\newcommand{\Rej}[2]{\RejName_{#1}\left( {#2} \right)}
\newcommand{\Rm}[1]{\mathbb{R}^{#1}}
\newcommand{\Cm}[1]{\mathbb{C}^{#1}}
\newcommand{\conj}[0]{{*}}

%</misc>

% <grade selection>
%
\newcommand{\gpgrade}[2] {{\left\langle{{#1}}\right\rangle}_{#2}}

\newcommand{\gpgradezero}[1] {\gpgrade{#1}{}}
%\newcommand{\gpscalargrade}[1] {{\left\langle{{#1}}\right\rangle}}
%\newcommand{\gpgradezero}[1] {\gpgrade{#1}{0}}

%\newcommand{\gpgradeone}[1] {{\left\langle{{#1}}\right\rangle}_{1}}
\newcommand{\gpgradeone}[1] {\gpgrade{#1}{1}}

\newcommand{\gpgradetwo}[1] {\gpgrade{#1}{2}}
\newcommand{\gpgradethree}[1] {\gpgrade{#1}{3}}
\newcommand{\gpgradefour}[1] {\gpgrade{#1}{4}}
%
% </grade selection>



\newcommand{\adot}[0]{{\dot{a}}}
\newcommand{\bdot}[0]{{\dot{b}}}
% taken for centered dot:
%\newcommand{\cdot}[0]{{\dot{c}}}
%\newcommand{\ddot}[0]{{\dot{d}}}
\newcommand{\edot}[0]{{\dot{e}}}
\newcommand{\fdot}[0]{{\dot{f}}}
\newcommand{\gdot}[0]{{\dot{g}}}
\newcommand{\hdot}[0]{{\dot{h}}}
\newcommand{\idot}[0]{{\dot{i}}}
\newcommand{\jdot}[0]{{\dot{j}}}
\newcommand{\kdot}[0]{{\dot{k}}}
\newcommand{\ldot}[0]{{\dot{l}}}
\newcommand{\mdot}[0]{{\dot{m}}}
\newcommand{\ndot}[0]{{\dot{n}}}
%\newcommand{\odot}[0]{{\dot{o}}}
\newcommand{\pdot}[0]{{\dot{p}}}
\newcommand{\qdot}[0]{{\dot{q}}}
\newcommand{\rdot}[0]{{\dot{r}}}
\newcommand{\sdot}[0]{{\dot{s}}}
\newcommand{\tdot}[0]{{\dot{t}}}
\newcommand{\udot}[0]{{\dot{u}}}
\newcommand{\vdot}[0]{{\dot{v}}}
\newcommand{\wdot}[0]{{\dot{w}}}
\newcommand{\xdot}[0]{{\dot{x}}}
\newcommand{\ydot}[0]{{\dot{y}}}
\newcommand{\zdot}[0]{{\dot{z}}}
\newcommand{\addot}[0]{{\ddot{a}}}
\newcommand{\bddot}[0]{{\ddot{b}}}
\newcommand{\cddot}[0]{{\ddot{c}}}
%\newcommand{\dddot}[0]{{\ddot{d}}}
\newcommand{\eddot}[0]{{\ddot{e}}}
\newcommand{\fddot}[0]{{\ddot{f}}}
\newcommand{\gddot}[0]{{\ddot{g}}}
\newcommand{\hddot}[0]{{\ddot{h}}}
\newcommand{\iddot}[0]{{\ddot{i}}}
\newcommand{\jddot}[0]{{\ddot{j}}}
\newcommand{\kddot}[0]{{\ddot{k}}}
\newcommand{\lddot}[0]{{\ddot{l}}}
\newcommand{\mddot}[0]{{\ddot{m}}}
\newcommand{\nddot}[0]{{\ddot{n}}}
\newcommand{\oddot}[0]{{\ddot{o}}}
\newcommand{\pddot}[0]{{\ddot{p}}}
\newcommand{\qddot}[0]{{\ddot{q}}}
\newcommand{\rddot}[0]{{\ddot{r}}}
\newcommand{\sddot}[0]{{\ddot{s}}}
\newcommand{\tddot}[0]{{\ddot{t}}}
\newcommand{\uddot}[0]{{\ddot{u}}}
\newcommand{\vddot}[0]{{\ddot{v}}}
\newcommand{\wddot}[0]{{\ddot{w}}}
\newcommand{\xddot}[0]{{\ddot{x}}}
\newcommand{\yddot}[0]{{\ddot{y}}}
\newcommand{\zddot}[0]{{\ddot{z}}}

%<bold and dot greek symbols>
%

\newcommand{\Deltadot}[0]{{\dot{\Delta}}}
\newcommand{\Gammadot}[0]{{\dot{\Gamma}}}
\newcommand{\Lambdadot}[0]{{\dot{\Lambda}}}
\newcommand{\Omegadot}[0]{{\dot{\Omega}}}
\newcommand{\Phidot}[0]{{\dot{\Phi}}}
\newcommand{\Pidot}[0]{{\dot{\Pi}}}
\newcommand{\Psidot}[0]{{\dot{\Psi}}}
\newcommand{\Sigmadot}[0]{{\dot{\Sigma}}}
\newcommand{\Thetadot}[0]{{\dot{\Theta}}}
\newcommand{\Upsilondot}[0]{{\dot{\Upsilon}}}
\newcommand{\Xidot}[0]{{\dot{\Xi}}}
\newcommand{\alphadot}[0]{{\dot{\alpha}}}
\newcommand{\betadot}[0]{{\dot{\beta}}}
\newcommand{\chidot}[0]{{\dot{\chi}}}
\newcommand{\deltadot}[0]{{\dot{\delta}}}
\newcommand{\epsilondot}[0]{{\dot{\epsilon}}}
\newcommand{\etadot}[0]{{\dot{\eta}}}
\newcommand{\gammadot}[0]{{\dot{\gamma}}}
\newcommand{\kappadot}[0]{{\dot{\kappa}}}
\newcommand{\lambdadot}[0]{{\dot{\lambda}}}
\newcommand{\mudot}[0]{{\dot{\mu}}}
\newcommand{\nudot}[0]{{\dot{\nu}}}
\newcommand{\omegadot}[0]{{\dot{\omega}}}
\newcommand{\phidot}[0]{{\dot{\phi}}}
\newcommand{\pidot}[0]{{\dot{\pi}}}
\newcommand{\psidot}[0]{{\dot{\psi}}}
\newcommand{\rhodot}[0]{{\dot{\rho}}}
\newcommand{\sigmadot}[0]{{\dot{\sigma}}}
\newcommand{\taudot}[0]{{\dot{\tau}}}
\newcommand{\thetadot}[0]{{\dot{\theta}}}
\newcommand{\upsilondot}[0]{{\dot{\upsilon}}}
\newcommand{\varepsilondot}[0]{{\dot{\varepsilon}}}
\newcommand{\varphidot}[0]{{\dot{\varphi}}}
\newcommand{\varpidot}[0]{{\dot{\varpi}}}
\newcommand{\varrhodot}[0]{{\dot{\varrho}}}
\newcommand{\varsigmadot}[0]{{\dot{\varsigma}}}
\newcommand{\varthetadot}[0]{{\dot{\vartheta}}}
\newcommand{\xidot}[0]{{\dot{\xi}}}
\newcommand{\zetadot}[0]{{\dot{\zeta}}}

\newcommand{\Deltaddot}[0]{{\ddot{\Delta}}}
\newcommand{\Gammaddot}[0]{{\ddot{\Gamma}}}
\newcommand{\Lambdaddot}[0]{{\ddot{\Lambda}}}
\newcommand{\Omegaddot}[0]{{\ddot{\Omega}}}
\newcommand{\Phiddot}[0]{{\ddot{\Phi}}}
\newcommand{\Piddot}[0]{{\ddot{\Pi}}}
\newcommand{\Psiddot}[0]{{\ddot{\Psi}}}
\newcommand{\Sigmaddot}[0]{{\ddot{\Sigma}}}
\newcommand{\Thetaddot}[0]{{\ddot{\Theta}}}
\newcommand{\Upsilonddot}[0]{{\ddot{\Upsilon}}}
\newcommand{\Xiddot}[0]{{\ddot{\Xi}}}
\newcommand{\alphaddot}[0]{{\ddot{\alpha}}}
\newcommand{\betaddot}[0]{{\ddot{\beta}}}
\newcommand{\chiddot}[0]{{\ddot{\chi}}}
\newcommand{\deltaddot}[0]{{\ddot{\delta}}}
\newcommand{\epsilonddot}[0]{{\ddot{\epsilon}}}
\newcommand{\etaddot}[0]{{\ddot{\eta}}}
\newcommand{\gammaddot}[0]{{\ddot{\gamma}}}
\newcommand{\kappaddot}[0]{{\ddot{\kappa}}}
\newcommand{\lambdaddot}[0]{{\ddot{\lambda}}}
\newcommand{\muddot}[0]{{\ddot{\mu}}}
\newcommand{\nuddot}[0]{{\ddot{\nu}}}
\newcommand{\omegaddot}[0]{{\ddot{\omega}}}
\newcommand{\phiddot}[0]{{\ddot{\phi}}}
\newcommand{\piddot}[0]{{\ddot{\pi}}}
\newcommand{\psiddot}[0]{{\ddot{\psi}}}
\newcommand{\rhoddot}[0]{{\ddot{\rho}}}
\newcommand{\sigmaddot}[0]{{\ddot{\sigma}}}
\newcommand{\tauddot}[0]{{\ddot{\tau}}}
\newcommand{\thetaddot}[0]{{\ddot{\theta}}}
\newcommand{\upsilonddot}[0]{{\ddot{\upsilon}}}
\newcommand{\varepsilonddot}[0]{{\ddot{\varepsilon}}}
\newcommand{\varphiddot}[0]{{\ddot{\varphi}}}
\newcommand{\varpiddot}[0]{{\ddot{\varpi}}}
\newcommand{\varrhoddot}[0]{{\ddot{\varrho}}}
\newcommand{\varsigmaddot}[0]{{\ddot{\varsigma}}}
\newcommand{\varthetaddot}[0]{{\ddot{\vartheta}}}
\newcommand{\xiddot}[0]{{\ddot{\xi}}}
\newcommand{\zetaddot}[0]{{\ddot{\zeta}}}

\newcommand{\BDelta}[0]{\boldsymbol{\Delta}}
\newcommand{\BGamma}[0]{\boldsymbol{\Gamma}}
\newcommand{\BLambda}[0]{\boldsymbol{\Lambda}}
\newcommand{\BOmega}[0]{\boldsymbol{\Omega}}
\newcommand{\BPhi}[0]{\boldsymbol{\Phi}}
\newcommand{\BPi}[0]{\boldsymbol{\Pi}}
\newcommand{\BPsi}[0]{\boldsymbol{\Psi}}
\newcommand{\BSigma}[0]{\boldsymbol{\Sigma}}
\newcommand{\BTheta}[0]{\boldsymbol{\Theta}}
\newcommand{\BUpsilon}[0]{\boldsymbol{\Upsilon}}
\newcommand{\BXi}[0]{\boldsymbol{\Xi}}
\newcommand{\Balpha}[0]{\boldsymbol{\alpha}}
\newcommand{\Bbeta}[0]{\boldsymbol{\beta}}
\newcommand{\Bchi}[0]{\boldsymbol{\chi}}
\newcommand{\Bdelta}[0]{\boldsymbol{\delta}}
\newcommand{\Bepsilon}[0]{\boldsymbol{\epsilon}}
\newcommand{\Beta}[0]{\boldsymbol{\eta}}
\newcommand{\Bgamma}[0]{\boldsymbol{\gamma}}
\newcommand{\Bkappa}[0]{\boldsymbol{\kappa}}
\newcommand{\Blambda}[0]{\boldsymbol{\lambda}}
\newcommand{\Bmu}[0]{\boldsymbol{\mu}}
\newcommand{\Bnu}[0]{\boldsymbol{\nu}}
%\newcommand{\Bomega}[0]{\boldsymbol{\omega}}
\newcommand{\Bphi}[0]{\boldsymbol{\phi}}
\newcommand{\Bpi}[0]{\boldsymbol{\pi}}
\newcommand{\Bpsi}[0]{\boldsymbol{\psi}}
\newcommand{\Brho}[0]{\boldsymbol{\rho}}
\newcommand{\Bsigma}[0]{\boldsymbol{\sigma}}
%\newcommand{\Btau}[0]{\boldsymbol{\tau}}
%\newcommand{\Btheta}[0]{\boldsymbol{\theta}}
\newcommand{\Bupsilon}[0]{\boldsymbol{\upsilon}}
\newcommand{\Bvarepsilon}[0]{\boldsymbol{\varepsilon}}
\newcommand{\Bvarphi}[0]{\boldsymbol{\varphi}}
\newcommand{\Bvarpi}[0]{\boldsymbol{\varpi}}
\newcommand{\Bvarrho}[0]{\boldsymbol{\varrho}}
\newcommand{\Bvarsigma}[0]{\boldsymbol{\varsigma}}
\newcommand{\Bvartheta}[0]{\boldsymbol{\vartheta}}
\newcommand{\Bxi}[0]{\boldsymbol{\xi}}
\newcommand{\Bzeta}[0]{\boldsymbol{\zeta}}
%
%</bold and dot greek symbols>
%<infrequent>
%
%\newcommand{\AreaOp}[1]{\AName_{#1}}
%\newcommand{\Babs}[0]{\abs{\BB}}
%\newcommand{\Bcap}[0]{\hat{\BB}}
%\newcommand{\BrPrimeRej}[0]{\rcap(\rcap \wedge \Br')}
%\newcommand{\CA}[0]{\mathcal{A}}
%\newcommand{\Cos}[1]{\cos{\left({#1}\right)}}
%\newcommand{\Det}[1] {\abs{#1}}
%\newcommand{\Dsq}[2] {\frac {\partial^2 {#1}} {\partial {#2}^2}}
%\newcommand{\Exp}[1]{\exp{\left({#1}\right)}}
%\newcommand{\Norm}[1]{\left\lVert{#1}\right\rVert}
%\newcommand{\Sin}[1]{\sin{\left({#1}\right)}}
%\newcommand{\T}[0]{\text{T}}
%\newcommand{\VolumeOp}[1]{\VName_{#1}}
%\newcommand{\agrad}[0]{\Ba \cdot \nabla}
%\newcommand{\alphacap}[0]{\hat{\boldsymbol{\alpha}}}
%\newcommand{\Fcap}[0]{\hat{\BF}}
%\newcommand{\bithree}[0]{{\Bi}_3}
%\newcommand{\bxa}[0]{\Bx\Ba}
%\newcommand{\coordvec}[2]{
%\newcommand{\costheta}[0]{\acap \cdot \xcap}
%\newcommand{\ddt}[1]{\ddot{#1}}
%\newcommand{\ddu}[1] {\frac {d{#1}} {du}}
%\newcommand{\dsqxj}[2] {\frac {\partial^2 {#1}} {\partial {x_{#2}}^2}}
%\newcommand{\dtheta}[1]{\frac{d {#1}}{d \theta}}
%\newcommand{\dt}[1]{\dot{#1}}
%\newcommand{\dt}[1]{\frac{d {#1}}{dt}}
%\newcommand{\dxj}[2] {\frac {\partial {#1}} {\partial {x_{#2}}}}
%\newcommand{\halfPhi}[0]{\frac{\phi}{2}}
%\newcommand{\half}[0]{\inv{2}}
%\newcommand{\inv}[1]{\frac{1}{#1}}
%\newcommand{\laplacian}[0]{\nabla^2}
%\newcommand{\matrixoftx}[3]{
%\newcommand{\nrrp}[0]{\norm{\rcap \wedge \Br'}}
%\newcommand{\oiint}{\bigcirc \hspace{-1.4em} \int \hspace{-.8em} \int}
%\newcommand{\transpose}[1]{{#1}^{\text{T}}}
%\newcommand{\transpose}[1]{{{#1}^{\TextTranspose}}}
%\newcommand{\transpose}[1]{{{#1}^{\text{T}}}}
%\newcommand{\barA}[0]{\bar{A}}
%\newcommand{\qbar}[0]{\bar{q}}
%\newcommand{\qdotbar}[0]{\dot{\bar{q}}}
%
%</infrequent>





%\usepackage[bookmarks=true]{hyperref}

%\usepackage{color,cite,graphicx}
   % use colour in the document, put your citations as [1-4]
   % rather than [1,2,3,4] (it looks nicer, and the extended LaTeX2e
   % graphics package. 
%\usepackage{latexsym,amssymb,epsf} % don't remember if these are
   % needed, but their inclusion can't do any damage


\chapter{Bohm Chapter 9 problems. }
%\author{Peeter Joot \quad peeter.joot@gmail.com }
%\date{ March 6, 2009.  Last Revision: $Date: 2009/06/03 22:13:06 $ }

%\begin{document}

%\maketitle{}

%\tableofcontents

\section{Bohm Chapter 9 problems. }

Problems and additional details from reading of \cite{bohm1989qt}, chapter 9.

\subsection{P1. Momentum wave function normalization. }

Given a normalized wave function

\begin{align*}
\IIinf \psi^\conj(x) \psi(x) dx = 1
\end{align*}

Show that the wave function $\phi(k)$ is also normalized, and find the normalization factor for $\Phi(p)$.

\begin{align*}
\IIinf \phi^\conj(k) \phi(k) dk 
&= 
\IIinf \phi^\conj(k) \left( \inv{\sqrt{2\pi}} \IIinf \psi(x) e^{-i k x} dx \right) dk  \\
&= 
\IIinf \left( \inv{\sqrt{2\pi}} \IIinf \phi^\conj(k) e^{-i k x} dk \right) \psi(x) dx  \\
&= 
\IIinf {\left( \inv{\sqrt{2\pi}} \IIinf \phi(k) e^{i k x} dk \right)}^\conj \psi(x) dx  \\
&= 
\IIinf \psi^\conj(x) \psi(x) dx  \\
&= 1 \quad\quad\quad \square
\end{align*}

Bohm defines $\Phi(p) \propto \phi(k)$ with the normalization constant determined by $\int \Abs{\Phi(p)} dp = 1$.  Suppose we 
let $\Phi(p) = \alpha \phi(k)$, then we have

\begin{align*}
1 
&= \int \Phi^\conj(p) \Phi(p) dp \\
&= \int \alpha^2 \phi^\conj(k) \phi(k) \hbar d k
\end{align*}

So we want $\alpha^2 \hbar = 1$, and therefore $\Phi(p) = \inv{\sqrt{\hbar}} \phi(k)$.

In \cite{mcmahon2005qmd}, with followup in \cite{PJqmFourier} we've seen that an alternate Fourier transform pair can be used in terms of
momentum variables.  That is

\begin{align*}
\Phi(p) &= \FM \IIinf \psi(x) e^{-ipx/\hbar} dx \\
\psi(x) &= \FM \IIinf \Phi(p) e^{ipx/\hbar} dp \\
\end{align*}

Observe that this is consistent with Bohm's notation, since one can read off 
$\Phi(p)$ in terms of $\phi(k)$.
by inspection

\begin{align*}
\Phi(p) &= \FM \IIinf \psi(x) e^{-ipx/\hbar} dx = \inv{\sqrt{\hbar}} \phi(k)
\end{align*}

\subsection{P2. Expectation of polynomial momentum function. }

Given a function of momentum 

\begin{align*}
f(p) &= \sum C_n p^n
\end{align*}

Express the average, or expectation value of $f(p)$.  It is sufficient to consider one of the monomial terms, say $p^n$.  A translation 
to position basis via Fourier transformation produces the desired result

\begin{align*}
\expectation{p^n} 
&= \int \Phi^\conj(p) p^n \Phi(p) dp \\
&= \inv{2\pi \hbar} \iiint \left( \psi^\conj(x') e^{ipx'/\hbar} dx' \right) (\hbar k)^n \left( \psi(x) e^{-ipx/\hbar} dx \right) (\hbar dk) \\
&= \frac{\hbar^n}{2\pi } \iiint \psi^\conj(x') e^{ikx'} dx' k^n e^{-i k x} \psi(x) dx dk \\
\end{align*}

The $k^n$ can be reduced to differential form as Bohm did for the $\expectation{p}$ case

\begin{align*}
k^n e^{-i k x} 
&= k^{n-1} k e^{-i k x} \\
&= k^{n-1} i \PD{x}{}e^{-i k x} \\
&= k^{n-m} i^m \PDN{x}{}{m}e^{-i k x} \\
&= i^n \PDN{x}{}{n}e^{-i k x} \\
\end{align*}

This leaves something that's in shape for integration by parts

\begin{align*}
\expectation{p^n} 
&= \frac{(i\hbar)^n}{2\pi} \iiint \psi^\conj(x') e^{ikx'} dx' \left( \PDN{x}{}{n}e^{-i k x} \right) \psi(x) dx dk \\
&= \frac{(-i\hbar)^n}{2\pi} \iiint \psi^\conj(x') e^{ikx'} dx' \PDN{x}{\psi(x)}{n} e^{-i k x} dx dk \\
&= \frac{(-i\hbar)^n}{2\pi} \iiint \psi^\conj(x') e^{ik(x'-x)} \PDN{x}{\psi(x)}{n} dx' dx dk \\
&= {(-i\hbar)^n}{} \iint \psi^\conj(x') \PDN{x}{\psi(x)}{n} dx' dx \inv{2\pi}\int e^{ik(x'-x)} dk \\
\end{align*}

This last integral is really a distribution, and can be identified with the delta function $\delta(x'-x)$ operating on, in this case, the preceding integral.
%, and operates on a test function.  Suppose we designate such a test function as $a(u)$, then we have
%
%\int \inv{2\pi}\int e^{iku} dk a(u) du
%&= \int \inv{2\pi}\int e^{iku} a(u) du dk \\

%Now, we can apply the distribution theory as covered in \cite{osgoodFourier} to do a delta function reduction of the exponentials in 
%this integral.  Specifically pick a function $a(k)$

So we have
\begin{align*}
\expectation{p^n} 
&= {(-i\hbar)^n}{} \iint \psi^\conj(x') \PDN{x}{\psi(x)}{n} dx' dx \delta(x'-x) \\
&= {(-i\hbar)^n}{} \int \psi^\conj(x) \PDN{x}{\psi(x)}{n} dx \\
\end{align*}

We can put this into explicit operator form, nicely motivating the identification of $-i\hbar \PDi{x}{}$ with the momentum by virtue 
of the definition of the average or expectation value.

\begin{align*}
\expectation{p^n} 
&= \int \psi^\conj(x) {\left( -i \hbar \PD{x}{} \right)}^n {\psi(x)} dx \\
\end{align*}

\subsection{P3.  Expectation of position in momentum space. }

\begin{align*}
\expectation{x} 
&= \int \psi^\conj(x) x \psi(x) dx \\
&= 
\inv{2\pi\hbar} \iiint \Phi^\conj(p) e^{-ipx/\hbar} dp x \Phi(p') e^{ip'x/\hbar} dp' dx \\
&= 
\inv{2\pi\hbar} \iiint \Phi^\conj(p) e^{-ipx/\hbar} dp \left( -i \PD{p'}{} e^{ip'x/\hbar} \right) \Phi(p') dp' dx \\
&= 
\inv{2\pi\hbar} \iiint \Phi^\conj(p) e^{-ipx/\hbar} dp \left( i \PD{p'}{\Phi(p')} \right) e^{ip'x/\hbar} dp' dx \\
&= 
\iint \Phi^\conj(p) \left( i \PD{p'}{} \right) {\Phi(p')} dp dp' \inv{2\pi\hbar} \int e^{i(p'-p)x/\hbar} dx  \\
&= 
\iint \Phi^\conj(p) \left( i \PD{p'}{} \right) {\Phi(p')} dp dp' \delta(p'-p)  \\
\end{align*}

This is

\begin{align*}
\expectation{x} &= \int \Phi^\conj(p) \left( i \PD{p}{} \right) {\Phi(p)} dp 
\end{align*}

We see that expressing momentum in position space and position in momentum space both result in differential
operator forms in calculations of expected values

\begin{align}\label{eqn:bohm_ch9:operatorCorrespondance}
p &\sim -i \hbar \PD{x}{} \\
x &\sim i \hbar \PD{p}{}
\end{align}

Observe the Hamiltonian and Poisson equation structure in these two sets of operators.

\subsection{P4. Expectation of polynomial position function. }

This problem follows just as P2, and I'm not going to bother typing it up for myself.  For validity, we require
$x^n \phi(x) \rightarrow 0$ as $x \rightarrow \pm \infty$, or equivalently that $\PDN{p}{\Phi}{n} \rightarrow 0$.

\subsection{P5. Some commutator calculations. }

\subsubsection{P5. Position momentum moment commutators. }

Evaluate

\begin{align*}
f(x,p) = x^n p^m - p^m x^n
\end{align*}

Up to this point we've only seen operators in expectation values.  Let's look the simplest case with $n = m = 1$ in that
context

\begin{align*}
\expectation{f} 
&= \frac{\hbar}{i} \int \psi^\conj(x) \left(x \PD{x}{} - \PD{x}{} x \right) \psi(x) dx \\
&= \frac{\hbar}{i} \int \psi^\conj(x) \left(x \PD{x}{\psi(x)} - \psi(x) - x \PD{x}{\psi(x)} \right) dx \\
&= -\frac{\hbar}{i} \int \psi^\conj(x) \psi(x) dx \\
&= {i\hbar}
\end{align*}

So in the same way that the operator correspondence between momentum and the derivative as summarized in 
\ref{eqn:bohm_ch9:operatorCorrespondance}, one can associate the commutator operator with its action in the expectation value and
say

\begin{align}\label{eqn:bohm_ch9:commutator}
x p - p x \sim  i\hbar
\end{align}

The higher order commutator expansions could also be evaluated this way, but exploiting the operator nature directly 
makes this easier.  For the first order moment commutator above one can write

\begin{align*}
f(x,p) \psi(x) 
&= (x p - p x) \psi(x) \\
&= -i \hbar \left(x \PD{x}{} - \PD{x}{} x\right) \psi(x) \\
&= -i \hbar \left(x \PD{x}{\psi}(x) - \PD{x}{x \psi(x)} \right) \\
&= -i \hbar \left(x \PD{x}{\psi}(x) - \PD{x}{\psi(x)} -\psi(x) \right) \\
&= i \hbar \psi(x) \\
\end{align*}

So again we see that as a right acting operator the net effect on any wave function is the following action

\begin{align*}
(x p - p x) \psi = i \hbar \psi \\
\end{align*}

If one starts from this point and then calculates the expectation value the result will still be $i \hbar$, but working
with the probability integrals from the get go is just additional complication.

Building on this result we can then calculate the higher order moment differences of the problem by using the commutator
to change the order of operations

\begin{align*}
p x \sim -i \hbar + x p
\end{align*}

Let's use this for a couple simple examples to start
\begin{align*}
x^2 p - p x^2
&=
x^2 p - ( -i \hbar + x p) x \\
&=
x^2 p + i \hbar x - x ( -i \hbar + x p) \\
&=
x^2 p + 2 i \hbar x - x^2 p \\
&=
2 i \hbar x \\
\end{align*}

\begin{align*}
x p^2 - p^2 x
&=
x p^2 - p ( -i \hbar + x p) \\
&=
x p^2 + i \hbar p - p x p \\
&=
x p^2 + i \hbar p + ( +i \hbar - x p) p \\
&=
2 i \hbar p \\
\end{align*}

Calculation of third powers shows a pattern, and one can guess at an induction hypothesis

\begin{align*}
x p^n &= p^n x + n i \hbar p^{n-1} \\
-p x^n &= -x^n p + n i \hbar x^{n-1} \\
\end{align*}

The $n=1$ cases follow from $xp - px = i\hbar$, leaving only the induction on $n$.  For the momentum powers we have

\begin{align*}
x p^n p 
&= p^n x p + n i \hbar p^{n} \\
&= p^n (p x + i \hbar) + n i \hbar p^{n} \\
&= p^{n+1} x + (n+1) i \hbar p^{n} \quad\quad\quad\square \\
\end{align*}

For the position powers we have
\begin{align*}
-p x^n x 
&= -x^n p x + n i \hbar x^{n} \\
&= x^n (-x p + i \hbar) + n i \hbar x^{n} \\
&= -x^{n+1} p + (n+1) i \hbar x^{n} \quad\quad\quad\square \\
\end{align*}

This completes the proof for a first order version of the problem

\begin{align*}
x p^n - p^n x &= n i \hbar p^{n-1} \\
x^n p -p x^n &=  n i \hbar x^{n-1} \\
\end{align*}

Observe that working with the operator form changes the calculation of derivatives problem in the original
commutator evaluation to nothing more than an algebraic exercise.
% (but one where there's a requirement to not accidentally invert the product order).

The general case still remains.  Building up to that let's do a couple examples

%x p^n = 
%p^n x +  n i \hbar p^{n-1} \\
%
%x^n p = 
%p x^n  +  n i \hbar x^{n-1} \\
%
%x^{n-1} p = 
%(p x^{n-1}  +  (n-1) i \hbar x^{n-2})
%x^{n-2} p = 
%(p x^{n-2}  +  (n-2) i \hbar x^{n-3})

\begin{align*}
x^n p^2
&= (x^n p) p \\
&= (p x^n  +  n i \hbar x^{n-1} ) p \\
&= p (x^n p) +  n i \hbar (x^{n-1} p ) \\
&= p^2 x^n  +  2 n i \hbar p x^{n-1} +  n (n-1) (i \hbar)^2 x^{n-2} \\
\end{align*}

\begin{align*}
x^n p^3
&=
(x^n p^2) p \\
&=
(p^2 x^n  +  2 n i \hbar p x^{n-1} +  n (n-1) (i \hbar)^2 x^{n-2} ) p \\
&=
  p^2 ( p x^n  +  n i \hbar x^{n-1} ) 
+ 2 n i \hbar p ( p x^{n-1}  +  (n-1) i \hbar x^{n-2} ) 
+ n (n-1) (i \hbar)^2 ( p x^{n-2}  +  (n-2) i \hbar x^{n-3}) 
\\
&=
  p^3 x^n
+ 3 n (i \hbar) p^2 x^{n-1}  
+ 3 n(n-1) (i \hbar)^2 p x^{n-2} 
+ n (n-1)(n-2) (i \hbar)^3 x^{n-3}
\\
\end{align*}

We see what looks like binomial coefficients, so a reasonable inductive hypothesis, for $m \le n$

\begin{align}\label{eqn:bohm_ch9:commutatorMomentMlessThanN}
x^n p^m
&= \sum_{j=0}^m \binom{m}{j} (i \hbar)^j p^{m-j} x^{n-j} (n)(n-1)\cdots(n-j+1)
\end{align}

And in particular, for $m \le n$
\begin{align}
x^n p^m - p^m x^n
&= \sum_{j=1}^m \binom{m}{j} (i \hbar)^j p^{m-j} x^{n-j} (n)(n-1)\cdots(n-j+1) 
\end{align}

For $m \ge n$, let's start with

\begin{align*}
p^m x = x p^m -  m i \hbar p^{m-1} 
\end{align*}
%p^m x = x p^m -  m (i \hbar) p^{m-1} 
%p^{m-1} x = x p^{m-1} -  (m-1) (i \hbar) p^{m-2} 
%p^{m-2} x = x p^{m-2} -  (m-2) (i \hbar) p^{m-3} 

First do the $x^2$
\begin{align*}
p^m x^2 
&= x p^m x -  m (i \hbar) p^{m-1} x \\
&= x (p^m x) -  m (i \hbar) (p^{m-1} x) \\
&= x^2 p^m - 2 m (i \hbar) x p^{m-1} +  m (m-1)(i \hbar)^2 p^{m-2} \\
\end{align*}

And for the cube $x^3$
\begin{align*}
p^m x^3  
&= 
( p^m x^2 ) x \\
&= 
( x^2 p^m - 2 m (i \hbar) x p^{m-1} +  m (m-1)(i \hbar)^2 p^{m-2} ) x \\
&= 
x^2 (p^m x )
- 2 m (i \hbar) x (p^{m-1} x )
+ m (m-1)(i \hbar)^2 (p^{m-2} x) \\
&= 
x^2 ( x p^m -  m (i \hbar) p^{m-1} ) \\
&\quad- 2 m (i \hbar) x ( x p^{m-1} -  (m-1) (i \hbar) p^{m-2} ) \\
&\quad+ m (m-1)(i \hbar)^2 ( x p^{m-2} -  (m-2) (i \hbar) p^{m-3} ) \\
&= 
x^3 p^m 
- 3 m (i \hbar) x^2 p^{m-1} 
+ 3 m (m-1) (i \hbar)^2 x p^{m-2} 
- m (m-1)(i \hbar)^2 (m-2) (i \hbar) p^{m-3} \\
\end{align*}

It appears that in this case with $m \ge n$, like \ref{eqn:bohm_ch9:commutatorMomentMlessThanN}, we want as the induction statement 

\begin{align}
p^m x^n &= \sum_{j=0}^n \binom{n}{j} (-i \hbar)^j x^{n-j} p^{m-j} (m)(m-1)\cdots(m-j+1) 
\end{align}

And for the commutator moment the expected result, pending induction on the above, is
\begin{align}
x^n p^m - p^m x^n &= -\sum_{j=1}^n \binom{n}{j} (-i \hbar)^j x^{n-j} p^{m-j} (m)(m-1)\cdots(m-j+1) 
\end{align}

Summarizing, this is
\begin{align*}
&x^n p^m - p^m x^n
= \\
&\left\{
\begin{array}{l l}
\sum_{j=1}^m \binom{m}{j} (i \hbar)^j p^{m-j} x^{n-j} (n)(n-1)\cdots(n-j+1) & \quad \mbox{if $m \le n$} \\
-\sum_{j=1}^n \binom{n}{j} (-i \hbar)^j x^{n-j} p^{m-j} (m)(m-1)\cdots(m-j+1) & \quad \mbox{if $m \ge n$}
\end{array}
\right.
\end{align*}

\subsubsection{P5.b }

\begin{align*}
e^{i k x} p - 
p e^{ i k x}
\end{align*}

Reversing the second term via power series expansion we have

\begin{align*}
p e^{ i k x}
&=
p \sum_{n=0}^\infty \frac{( i k x )^n}{n!} \\
&=
\sum_{n=0}^\infty \frac{( i k )^n}{n!} (x^n p - n i \hbar x^{n-1} )
\\
&=
e^{ i k x} p
-\sum_{n=1}^\infty \frac{( i k )^n}{n!} (n i \hbar x^{n-1} )
\\
&=
e^{ i k x} p
-(i k)(i\hbar) \sum_{n=1}^\infty \frac{( i k x)^{n-1}}{(n-1)!} 
\\
&=
e^{ i k x} p
+ (k\hbar) e^{i k x}
\\
\end{align*}

So we have

\begin{align*}
e^{ i k x} p - p e^{ i k x} &= - (k\hbar) e^{i k x}
\end{align*}

\subsection{P6. Hermitian operators. Powers of momentum operators. }

Show that $p^n$ is Hermitian

\begin{align*}
\expectation{p^n}^\conj 
&=
\left( \int \psi^\conj (-i \hbar)^n \frac{d^n}{dx^n} \psi \right)^\conj \\
&=
\int \psi (i \hbar)^n \frac{d^n}{dx^n} \psi^\conj \\
&=
\int \left( (-1)^n \frac{d^n}{dx^n} \psi \right) (i \hbar)^n \psi^\conj \\
&=
\int \left( p^n \psi \right) \psi^\conj \\
&=
\expectation{p^n}
\end{align*}

Thus, by the definition of equation (13) in the text, this operator is
Hermitian.

Next is to show that 

\begin{align*}
f(p) = \sum_k A_k p^k
\end{align*}

is Hermitian, provided $A_k$ are all real.  This part is clear by inspection.

\subsection{P7. Hermitian operators. Powers of position operators. }

Want to show that the following is Hermitian

\begin{align*}
f(x) &= \sum A_k x^k
\end{align*}

If the conjugate of the expectation equals itself we required only $A_k = A_k^\conj$, so $A_k$ must be strictly real, and we are done.

\subsection{P8. Non Hermitian momentum power operators if derivative doesn't vanish. }

Show that if $\partial^n \psi/\partial x^n$ doesn't vanish then $(-i\hbar)^{n+1} \partial^{n+1}/\partial x^{n+1}$ is not Hermitian.

We want to evaluate the following and compare it to its conjugate

\begin{align*}
\expectation{p^{n+1}} 
&= (-i\hbar)^{n+1} \IIinf \psi^\conj \PDN{x}{\psi}{n+1} dx \\
&= 
(-i\hbar)^{n+1} \left. \psi^\conj \PDN{x}{\psi}{n} \right\vert_{-\infty}^{\infty}
+(-1)^{1}(-i\hbar)^{n+1} \IIinf \PDN{x}{\psi^\conj}{1} \PDN{x}{\psi}{n} dx 
\\
&= 
(-i\hbar)^{n+1} \left. \psi^\conj \PDN{x}{\psi}{n} \right\vert_{-\infty}^{\infty}
+(-1)^{1}(-i\hbar)^{n+1} \left. \PDN{x}{\psi^\conj}{1} \PDN{x}{\psi}{n-1} \right\vert_{-\infty}^{\infty}
+(-1)^{2}(-i\hbar)^{n+1} \IIinf \PDN{x}{\psi^\conj}{2} \PDN{x}{\psi}{n-1} dx 
\\
&= 
(-i\hbar)^{n+1} \sum_{k=0}^1
(-1)^{k}
\left. \PDN{x}{\psi^\conj}{k} \PDN{x}{\psi}{n-k} \right\vert_{-\infty}^{\infty}
+(-1)^{2}(-i\hbar)^{n+1} \IIinf \PDN{x}{\psi^\conj}{2} \PDN{x}{\psi}{n-1} dx  
\\
&= 
(-i\hbar)^{n+1} \sum_{k=0}^m
(-1)^{k}
\left. \PDN{x}{\psi^\conj}{k} \PDN{x}{\psi}{n-k} \right\vert_{-\infty}^{\infty}
+(-1)^{m+1}(-i\hbar)^{n+1} \IIinf \PDN{x}{\psi^\conj}{m+1} \PDN{x}{\psi}{n-m)} dx 
\\
&= 
(-i\hbar)^{n+1} \sum_{k=0}^n
(-1)^{k}
\left. \PDN{x}{\psi^\conj}{k} \PDN{x}{\psi}{n-k} \right\vert_{-\infty}^{\infty}
+(i\hbar)^{n+1} \IIinf \PDN{x}{\psi^\conj}{n+1} \psi dx 
\\
&= 
(-i\hbar)^{n+1} \sum_{k=0}^n
(-1)^{k}
\left. \PDN{x}{\psi^\conj}{k} \PDN{x}{\psi}{n-k} \right\vert_{-\infty}^{\infty}
+
\expectation{p^{n+1}}^\conj
\\
\end{align*}
%\newcommand{\PDN}[3]{\frac{\partial^{#3} {#2}}{\partial {#1}^{#3}}}

So we have
\begin{align*}
\expectation{p^{n+1}} - \expectation{p^{n+1}}^\conj
&= 
(-i\hbar)^{n+1} \sum_{k=0}^n
(-1)^{k}
\left. \PDN{x}{\psi^\conj}{k} \PDN{x}{\psi}{n-k} \right\vert_{-\infty}^{\infty}
\\
\end{align*}

If $p^{n+1}$ is Hermitian, then this difference should be zero, but if the indicated partial doesn't vanish this
remainder bit can be non-zero.

\subsection{P9. m, n'th moment is Hermitian. }

Consider the operator 

\begin{align*}
\sum A_{nm} \left(\frac{p^n x^m + x^m p^n}{2}\right) 
\end{align*}

Show that this is Hermitian if all $A_{nm}$ are real.

Consider first one specific term with $A_{nm}$, calculate the conjugate of the expectation value, and integrate
by parts

\begin{align*}
\left(
\int \psi^\conj \inv{2}(p^n x^m + x^m p^n) \psi
\right)^\conj
&=
\inv{2} (-1)^n
\int \psi (p^n x^m + x^m p^n) \psi^\conj \\
&=
\inv{2} (i\hbar)^n 
\int \psi \left(\frac{d^n (x^m \psi^\conj)}{dx^n} + x^m \frac{d^n \psi^\conj}{dx^n} \right) \\
&=
\inv{2} (-i\hbar)^n 
\int x^m \psi^\conj \left(\frac{d^n \psi }{dx^n} + \psi^\conj \frac{d^n (x^m \psi)}{dx^n} \right) \\
&=
\inv{2}
\int \psi^\conj ( x^m p^n + p^n x^m ) \psi
\end{align*}

This shows that $(p^n x^m + x^m p^n)/2$ is Hermitian, and the conjugation requires $A_{nm}$ to be real for the
product of the two to be Hermitian.

\subsection{P10. Hermitizing Classical operator $(px)^2$. }

Show that 
\begin{align*}
\inv{2}\left(x^2 p^2 + p^2 x^2 \right)
\end{align*}

and 
\begin{align*}
\inv{4}\left(x p + p x \right)^2
\end{align*}

lead to results that differ by a factor of $\hbar^2$.

To do so consider the difference of the expectation of this operator, first calculating this difference.  We will
want to use the commutator relation, in a few equivalent forms

\begin{align*}
xp - p x &= i \hbar \\
xp &= p x + i \hbar \\
px &= x p - i \hbar \\
x p + p x &= 2 p x + i \hbar \\
          &= 2 x p - i \hbar \\
\end{align*}

This gives us

\begin{align*}
\inv{4}\left(x p + p x \right)^2 
&= \inv{4}(2 p x + i \hbar)( 2 x p - i \hbar ) \\
&= p x^2 p + i\hbar \inv{2}(x p - p x) + \inv{4} \hbar^2 \\
&= p x^2 p - \hbar^2 \inv{2} + \inv{4} \hbar^2 \\
&= p x^2 p - \hbar^2 \inv{4} \\
\end{align*}

For the other operator, reduction to a form that also contains $p x^2 p$, we have

\begin{align*}
\inv{2}\left(x^2 p^2 + p^2 x^2 \right)
&=
\inv{2}\left(x (x p) p + p (p x) x \right) \\
&=
\inv{2}\left(x (p x + i \hbar) p + p (x p -i\hbar) x \right) \\
&=
\inv{2}\left((x p) x p + p x (p x) + i \hbar ( x p - p x ) \right) \\
&=
\inv{2}\left((p x + i\hbar) x p + p x ( x p - i\hbar) + (i \hbar)^2 \right) \\
&=
\inv{2}\left( 2 p x^2 p + i \hbar ( x p - p x) + -\hbar^2 \right) \\
&=
p x^2 p + -\hbar^2 
\end{align*}

So now, if we take the difference 
\begin{align*}
\inv{2}\left(x^2 p^2 + p^2 x^2 \right) - \inv{4}\left(x p + p x \right)^2 
&= (p x^2 p + -\hbar^2 ) - (p x^2 p - \hbar^2 \inv{4}) 
 \\
&= -\frac{3}{4}\hbar^2 
 \\
\end{align*}

The difference of the expectation values of these operators is thus of the order $\hbar^2$ as was to be calculated.

\subsection{P11.  An explicit calculation of a Hermitian operator. }

Show by integration by parts that $(xp)^\dagger = p x$.

The defining relation for the Hermitian conjugation operation is equation 16 in the text.

\begin{align*}
\int \psi^\conj (O^\dagger \psi) dx &= \int \psi (O^\conj \psi^\conj) dx
\end{align*}

For the operator $xp$, we have

\begin{align*}
\int \psi^\conj (xp)^\dagger \psi dx 
&= \int \psi (xp)^\conj \psi^\conj dx \\
&= (-i\hbar)^\conj \int \psi x \PD{x}{\psi^\conj} dx \\
&= - i\hbar \int \PD{x}{x \psi} \psi^\conj dx \\
&= \int \psi^\conj (p x) \psi dx \\
\end{align*}

So we have, as desired

\begin{align*}
(xp)^\dagger = p x
\end{align*}

\subsection{P12. Hermitian operator from antisymmetric difference. }

Show that $H = i (O - O^\dagger)$ is a Hermitian operator.

This follows directly from the definition, calculating the expectation

\begin{align*}
\expectation{H} 
&=\int \psi^\conj (i (O - O^\dagger) \psi ) \\
&=
i \int \psi^\conj (O \psi )
-i \int \psi^\conj (O^\dagger \psi ) \\
&=
i \int \psi^\conj (O \psi )
-i \int \psi (O^\conj \psi^\conj) \\
\end{align*}

Taking conjugates we have

\begin{align*}
\expectation{H}^\conj
&=
-i \int \psi (O^\conj \psi^\conj )
+i \int \psi^\conj (O \psi ) \\
&=
\expectation{H} \\
\quad\quad\quad \square
\end{align*}


\subsection{P13. When product of operators is Hermitian. }

What relation must exist between Hermitian $B$ and $A$ must exist for $AB$ to be Hermitian.

TODO:
Am guessing that this has something to do with the commutator of the operators.  This one I don't have a check mark
besides in my text, so did I ever figure it out?

\subsection{P14. Show directly that $i(p^2 x - xp^2)$ is Hermitian }

This follows from $(AB)^\dagger = BA$ in the text above.  We have

\begin{align*}
(i (p^2 x - x p^2))^\dagger 
&=
(x^\dagger p^\dagger p^\dagger - p^\dagger p^\dagger x^\dagger)(-i) \\
&=
-i (x p^2 - p^2 x) \\
&=
i (p^2 x - x p^2 ) \quad\quad\quad \square
\end{align*}

%\bibliographystyle{plainnat}
%\bibliography{myrefs}

%\end{document}

%
% Copyright � 2012 Peeter Joot.  All Rights Reserved.
% Licenced as described in the file LICENSE under the root directory of this GIT repository.
%

% 
% 
%\documentclass{article}

%\usepackage{amsmath}
\usepackage{mathpazo}

%
% shorthand for bold symbols, convenient for vectors and matrices
%
\newcommand{\Ba}[0]{\mathbf{a}}
\newcommand{\Bb}[0]{\mathbf{b}}
\newcommand{\Bc}[0]{\mathbf{c}}
\newcommand{\Bd}[0]{\mathbf{d}}
\newcommand{\Be}[0]{\mathbf{e}}
\newcommand{\Bf}[0]{\mathbf{f}}
\newcommand{\Bg}[0]{\mathbf{g}}
\newcommand{\Bh}[0]{\mathbf{h}}
\newcommand{\Bi}[0]{\mathbf{i}}
\newcommand{\Bj}[0]{\mathbf{j}}
\newcommand{\Bk}[0]{\mathbf{k}}
\newcommand{\Bl}[0]{\mathbf{l}}
\newcommand{\Bm}[0]{\mathbf{m}}
\newcommand{\Bn}[0]{\mathbf{n}}
\newcommand{\Bo}[0]{\mathbf{o}}
\newcommand{\Bp}[0]{\mathbf{p}}
\newcommand{\Bq}[0]{\mathbf{q}}
\newcommand{\Br}[0]{\mathbf{r}}
\newcommand{\Bs}[0]{\mathbf{s}}
\newcommand{\Bt}[0]{\mathbf{t}}
\newcommand{\Bu}[0]{\mathbf{u}}
\newcommand{\Bv}[0]{\mathbf{v}}
\newcommand{\Bw}[0]{\mathbf{w}}
\newcommand{\Bx}[0]{\mathbf{x}}
\newcommand{\By}[0]{\mathbf{y}}
\newcommand{\Bz}[0]{\mathbf{z}}
\newcommand{\BA}[0]{\mathbf{A}}
\newcommand{\BB}[0]{\mathbf{B}}
\newcommand{\BC}[0]{\mathbf{C}}
\newcommand{\BD}[0]{\mathbf{D}}
\newcommand{\BE}[0]{\mathbf{E}}
\newcommand{\BF}[0]{\mathbf{F}}
\newcommand{\BG}[0]{\mathbf{G}}
\newcommand{\BH}[0]{\mathbf{H}}
\newcommand{\BI}[0]{\mathbf{I}}
\newcommand{\BJ}[0]{\mathbf{J}}
\newcommand{\BK}[0]{\mathbf{K}}
\newcommand{\BL}[0]{\mathbf{L}}
\newcommand{\BM}[0]{\mathbf{M}}
\newcommand{\BN}[0]{\mathbf{N}}
\newcommand{\BO}[0]{\mathbf{O}}
\newcommand{\BP}[0]{\mathbf{P}}
\newcommand{\BQ}[0]{\mathbf{Q}}
\newcommand{\BR}[0]{\mathbf{R}}
\newcommand{\BS}[0]{\mathbf{S}}
\newcommand{\BT}[0]{\mathbf{T}}
\newcommand{\BU}[0]{\mathbf{U}}
\newcommand{\BV}[0]{\mathbf{V}}
\newcommand{\BW}[0]{\mathbf{W}}
\newcommand{\BX}[0]{\mathbf{X}}
\newcommand{\BY}[0]{\mathbf{Y}}
\newcommand{\BZ}[0]{\mathbf{Z}}

\newcommand{\Bzero}[0]{\mathbf{0}}
\newcommand{\Btheta}[0]{\boldsymbol{\theta}}
\newcommand{\Btau}[0]{\boldsymbol{\tau}}
\newcommand{\Bomega}[0]{\boldsymbol{\omega}}

%
% shorthand for unit vectors
%
\newcommand{\acap}[0]{\hat{\Ba}}
\newcommand{\bcap}[0]{\hat{\Bb}}
\newcommand{\ccap}[0]{\hat{\Bc}}
\newcommand{\dcap}[0]{\hat{\Bd}}
\newcommand{\ecap}[0]{\hat{\Be}}
\newcommand{\fcap}[0]{\hat{\Bf}}
\newcommand{\gcap}[0]{\hat{\Bg}}
\newcommand{\hcap}[0]{\hat{\Bh}}
\newcommand{\icap}[0]{\hat{\Bi}}
\newcommand{\jcap}[0]{\hat{\Bj}}
\newcommand{\kcap}[0]{\hat{\Bk}}
\newcommand{\lcap}[0]{\hat{\Bl}}
\newcommand{\mcap}[0]{\hat{\Bm}}
\newcommand{\ncap}[0]{\hat{\Bn}}
\newcommand{\ocap}[0]{\hat{\Bo}}
\newcommand{\pcap}[0]{\hat{\Bp}}
\newcommand{\qcap}[0]{\hat{\Bq}}
\newcommand{\rcap}[0]{\hat{\Br}}
\newcommand{\scap}[0]{\hat{\Bs}}
\newcommand{\tcap}[0]{\hat{\Bt}}
\newcommand{\ucap}[0]{\hat{\Bu}}
\newcommand{\vcap}[0]{\hat{\Bv}}
\newcommand{\wcap}[0]{\hat{\Bw}}
\newcommand{\xcap}[0]{\hat{\Bx}}
\newcommand{\ycap}[0]{\hat{\By}}
\newcommand{\zcap}[0]{\hat{\Bz}}
\newcommand{\thetacap}[0]{\hat{\Btheta}}

%
% to write R^n and C^n in a distinguishable fashion.  Perhaps change this
% to the double lined characters upon figuring out how to do so.
%
\newcommand{\C}[1]{$\mathbb{C}^{#1}$}
\newcommand{\R}[1]{$\mathbb{R}^{#1}$}

%
% various generally useful helpers
%

% derivative of #1 wrt. #2:
\newcommand{\D}[2] {\frac {d#2} {d#1}}

\newcommand{\inv}[1]{\frac{1}{#1}}
\newcommand{\cross}[0]{\times}

\newcommand{\abs}[1]{\lvert{#1}\rvert}
\newcommand{\norm}[1]{\lVert{#1}\rVert}
\newcommand{\innerprod}[2]{\langle{#1}, {#2}\rangle}
\newcommand{\dotprod}[2]{{#1} \cdot {#2}}
\newcommand{\bdotprod}[2]{\left({#1} \cdot {#2}\right)}
\newcommand{\crossprod}[2]{{#1} \cross {#2}}
\newcommand{\tripleprod}[3]{\dotprod{\left(\crossprod{#1}{#2}\right)}{#3}}

\DeclareMathOperator{\Proj}{Proj}
\DeclareMathOperator{\Span}{span}
\DeclareMathOperator{\Sgn}{sgn}
\DeclareMathOperator{\Area}{Area}
\DeclareMathOperator{\Volume}{Volume}

%
% A few miscellaneous things specific to this document
%
\newcommand{\crossop}[1]{\crossprod{#1}{}}

% R2 vector.
\newcommand{\VectorTwo}[2]{
\begin{bmatrix}
 {#1} \\
 {#2}
\end{bmatrix}
}

\newcommand{\VectorN}[1]{
\begin{bmatrix}
{#1}_1 \\
{#1}_2 \\
\vdots \\
{#1}_N \\
\end{bmatrix}
}

\newcommand{\DETuvij}[4]{
\begin{vmatrix}
 {#1}_{#3} & {#1}_{#4} \\
 {#2}_{#3} & {#2}_{#4}
\end{vmatrix}
}

\newcommand{\DETuvwijk}[6]{
\begin{vmatrix}
 {#1}_{#4} & {#1}_{#5} & {#1}_{#6} \\
 {#2}_{#4} & {#2}_{#5} & {#2}_{#6} \\
 {#3}_{#4} & {#3}_{#5} & {#3}_{#6}
\end{vmatrix}
}

\newcommand{\DETuvwxijkl}[8]{
\begin{vmatrix}
 {#1}_{#5} & {#1}_{#6} & {#1}_{#7} & {#1}_{#8} \\
 {#2}_{#5} & {#2}_{#6} & {#2}_{#7} & {#2}_{#8} \\
 {#3}_{#5} & {#3}_{#6} & {#3}_{#7} & {#3}_{#8} \\
 {#4}_{#5} & {#4}_{#6} & {#4}_{#7} & {#4}_{#8} \\
\end{vmatrix}
}

%\newcommand{\DETuvwxyijklm}[10]{
%\begin{vmatrix}
% {#1}_{#6} & {#1}_{#7} & {#1}_{#8} & {#1}_{#9} & {#1}_{#10} \\
% {#2}_{#6} & {#2}_{#7} & {#2}_{#8} & {#2}_{#9} & {#2}_{#10} \\
% {#3}_{#6} & {#3}_{#7} & {#3}_{#8} & {#3}_{#9} & {#3}_{#10} \\
% {#4}_{#6} & {#4}_{#7} & {#4}_{#8} & {#4}_{#9} & {#4}_{#10} \\
% {#5}_{#6} & {#5}_{#7} & {#5}_{#8} & {#5}_{#9} & {#5}_{#10}
%\end{vmatrix}
%}

% R3 vector.
\newcommand{\VectorThree}[3]{
\begin{bmatrix}
 {#1} \\
 {#2} \\
 {#3}
\end{bmatrix}
}


%%<misc>
%
\newcommand{\Abs}[1]{{\left\lvert{#1}\right\rvert}}
\newcommand{\spacegrad}[0]{\boldsymbol{\nabla}}
\newcommand{\grad}[0]{\nabla}
\newcommand{\LL}[0]{\mathcal{L}}

% == \partial_{#1} {#2}
\newcommand{\PD}[2]{\frac{\partial {#2}}{\partial {#1}}}
% inline variant
\newcommand{\PDi}[2]{{\partial {#2}}/{\partial {#1}}}

\newcommand{\PDD}[3]{\frac{\partial^2 {#3}}{\partial {#1}\partial {#2}}}
%\newcommand{\PDd}[2]{\frac{\partial^2 {#2}}{{\partial{#1}}^2}}
\newcommand{\PDsq}[2]{\frac{\partial^2 {#2}}{(\partial {#1})^2}}

\newcommand{\Partial}[2]{\frac{\partial {#1}}{\partial {#2}}}
\DeclareMathOperator{\RejName}{Rej}
\newcommand{\Rej}[2]{\RejName_{#1}\left( {#2} \right)}
\newcommand{\Rm}[1]{\mathbb{R}^{#1}}
\newcommand{\Cm}[1]{\mathbb{C}^{#1}}
\newcommand{\conj}[0]{{*}}

%</misc>

% <grade selection>
%
\newcommand{\gpgrade}[2] {{\left\langle{{#1}}\right\rangle}_{#2}}

\newcommand{\gpgradezero}[1] {\gpgrade{#1}{}}
%\newcommand{\gpscalargrade}[1] {{\left\langle{{#1}}\right\rangle}}
%\newcommand{\gpgradezero}[1] {\gpgrade{#1}{0}}

%\newcommand{\gpgradeone}[1] {{\left\langle{{#1}}\right\rangle}_{1}}
\newcommand{\gpgradeone}[1] {\gpgrade{#1}{1}}

\newcommand{\gpgradetwo}[1] {\gpgrade{#1}{2}}
\newcommand{\gpgradethree}[1] {\gpgrade{#1}{3}}
\newcommand{\gpgradefour}[1] {\gpgrade{#1}{4}}
%
% </grade selection>



\newcommand{\adot}[0]{{\dot{a}}}
\newcommand{\bdot}[0]{{\dot{b}}}
% taken for centered dot:
%\newcommand{\cdot}[0]{{\dot{c}}}
%\newcommand{\ddot}[0]{{\dot{d}}}
\newcommand{\edot}[0]{{\dot{e}}}
\newcommand{\fdot}[0]{{\dot{f}}}
\newcommand{\gdot}[0]{{\dot{g}}}
\newcommand{\hdot}[0]{{\dot{h}}}
\newcommand{\idot}[0]{{\dot{i}}}
\newcommand{\jdot}[0]{{\dot{j}}}
\newcommand{\kdot}[0]{{\dot{k}}}
\newcommand{\ldot}[0]{{\dot{l}}}
\newcommand{\mdot}[0]{{\dot{m}}}
\newcommand{\ndot}[0]{{\dot{n}}}
%\newcommand{\odot}[0]{{\dot{o}}}
\newcommand{\pdot}[0]{{\dot{p}}}
\newcommand{\qdot}[0]{{\dot{q}}}
\newcommand{\rdot}[0]{{\dot{r}}}
\newcommand{\sdot}[0]{{\dot{s}}}
\newcommand{\tdot}[0]{{\dot{t}}}
\newcommand{\udot}[0]{{\dot{u}}}
\newcommand{\vdot}[0]{{\dot{v}}}
\newcommand{\wdot}[0]{{\dot{w}}}
\newcommand{\xdot}[0]{{\dot{x}}}
\newcommand{\ydot}[0]{{\dot{y}}}
\newcommand{\zdot}[0]{{\dot{z}}}
\newcommand{\addot}[0]{{\ddot{a}}}
\newcommand{\bddot}[0]{{\ddot{b}}}
\newcommand{\cddot}[0]{{\ddot{c}}}
%\newcommand{\dddot}[0]{{\ddot{d}}}
\newcommand{\eddot}[0]{{\ddot{e}}}
\newcommand{\fddot}[0]{{\ddot{f}}}
\newcommand{\gddot}[0]{{\ddot{g}}}
\newcommand{\hddot}[0]{{\ddot{h}}}
\newcommand{\iddot}[0]{{\ddot{i}}}
\newcommand{\jddot}[0]{{\ddot{j}}}
\newcommand{\kddot}[0]{{\ddot{k}}}
\newcommand{\lddot}[0]{{\ddot{l}}}
\newcommand{\mddot}[0]{{\ddot{m}}}
\newcommand{\nddot}[0]{{\ddot{n}}}
\newcommand{\oddot}[0]{{\ddot{o}}}
\newcommand{\pddot}[0]{{\ddot{p}}}
\newcommand{\qddot}[0]{{\ddot{q}}}
\newcommand{\rddot}[0]{{\ddot{r}}}
\newcommand{\sddot}[0]{{\ddot{s}}}
\newcommand{\tddot}[0]{{\ddot{t}}}
\newcommand{\uddot}[0]{{\ddot{u}}}
\newcommand{\vddot}[0]{{\ddot{v}}}
\newcommand{\wddot}[0]{{\ddot{w}}}
\newcommand{\xddot}[0]{{\ddot{x}}}
\newcommand{\yddot}[0]{{\ddot{y}}}
\newcommand{\zddot}[0]{{\ddot{z}}}

%<bold and dot greek symbols>
%

\newcommand{\Deltadot}[0]{{\dot{\Delta}}}
\newcommand{\Gammadot}[0]{{\dot{\Gamma}}}
\newcommand{\Lambdadot}[0]{{\dot{\Lambda}}}
\newcommand{\Omegadot}[0]{{\dot{\Omega}}}
\newcommand{\Phidot}[0]{{\dot{\Phi}}}
\newcommand{\Pidot}[0]{{\dot{\Pi}}}
\newcommand{\Psidot}[0]{{\dot{\Psi}}}
\newcommand{\Sigmadot}[0]{{\dot{\Sigma}}}
\newcommand{\Thetadot}[0]{{\dot{\Theta}}}
\newcommand{\Upsilondot}[0]{{\dot{\Upsilon}}}
\newcommand{\Xidot}[0]{{\dot{\Xi}}}
\newcommand{\alphadot}[0]{{\dot{\alpha}}}
\newcommand{\betadot}[0]{{\dot{\beta}}}
\newcommand{\chidot}[0]{{\dot{\chi}}}
\newcommand{\deltadot}[0]{{\dot{\delta}}}
\newcommand{\epsilondot}[0]{{\dot{\epsilon}}}
\newcommand{\etadot}[0]{{\dot{\eta}}}
\newcommand{\gammadot}[0]{{\dot{\gamma}}}
\newcommand{\kappadot}[0]{{\dot{\kappa}}}
\newcommand{\lambdadot}[0]{{\dot{\lambda}}}
\newcommand{\mudot}[0]{{\dot{\mu}}}
\newcommand{\nudot}[0]{{\dot{\nu}}}
\newcommand{\omegadot}[0]{{\dot{\omega}}}
\newcommand{\phidot}[0]{{\dot{\phi}}}
\newcommand{\pidot}[0]{{\dot{\pi}}}
\newcommand{\psidot}[0]{{\dot{\psi}}}
\newcommand{\rhodot}[0]{{\dot{\rho}}}
\newcommand{\sigmadot}[0]{{\dot{\sigma}}}
\newcommand{\taudot}[0]{{\dot{\tau}}}
\newcommand{\thetadot}[0]{{\dot{\theta}}}
\newcommand{\upsilondot}[0]{{\dot{\upsilon}}}
\newcommand{\varepsilondot}[0]{{\dot{\varepsilon}}}
\newcommand{\varphidot}[0]{{\dot{\varphi}}}
\newcommand{\varpidot}[0]{{\dot{\varpi}}}
\newcommand{\varrhodot}[0]{{\dot{\varrho}}}
\newcommand{\varsigmadot}[0]{{\dot{\varsigma}}}
\newcommand{\varthetadot}[0]{{\dot{\vartheta}}}
\newcommand{\xidot}[0]{{\dot{\xi}}}
\newcommand{\zetadot}[0]{{\dot{\zeta}}}

\newcommand{\Deltaddot}[0]{{\ddot{\Delta}}}
\newcommand{\Gammaddot}[0]{{\ddot{\Gamma}}}
\newcommand{\Lambdaddot}[0]{{\ddot{\Lambda}}}
\newcommand{\Omegaddot}[0]{{\ddot{\Omega}}}
\newcommand{\Phiddot}[0]{{\ddot{\Phi}}}
\newcommand{\Piddot}[0]{{\ddot{\Pi}}}
\newcommand{\Psiddot}[0]{{\ddot{\Psi}}}
\newcommand{\Sigmaddot}[0]{{\ddot{\Sigma}}}
\newcommand{\Thetaddot}[0]{{\ddot{\Theta}}}
\newcommand{\Upsilonddot}[0]{{\ddot{\Upsilon}}}
\newcommand{\Xiddot}[0]{{\ddot{\Xi}}}
\newcommand{\alphaddot}[0]{{\ddot{\alpha}}}
\newcommand{\betaddot}[0]{{\ddot{\beta}}}
\newcommand{\chiddot}[0]{{\ddot{\chi}}}
\newcommand{\deltaddot}[0]{{\ddot{\delta}}}
\newcommand{\epsilonddot}[0]{{\ddot{\epsilon}}}
\newcommand{\etaddot}[0]{{\ddot{\eta}}}
\newcommand{\gammaddot}[0]{{\ddot{\gamma}}}
\newcommand{\kappaddot}[0]{{\ddot{\kappa}}}
\newcommand{\lambdaddot}[0]{{\ddot{\lambda}}}
\newcommand{\muddot}[0]{{\ddot{\mu}}}
\newcommand{\nuddot}[0]{{\ddot{\nu}}}
\newcommand{\omegaddot}[0]{{\ddot{\omega}}}
\newcommand{\phiddot}[0]{{\ddot{\phi}}}
\newcommand{\piddot}[0]{{\ddot{\pi}}}
\newcommand{\psiddot}[0]{{\ddot{\psi}}}
\newcommand{\rhoddot}[0]{{\ddot{\rho}}}
\newcommand{\sigmaddot}[0]{{\ddot{\sigma}}}
\newcommand{\tauddot}[0]{{\ddot{\tau}}}
\newcommand{\thetaddot}[0]{{\ddot{\theta}}}
\newcommand{\upsilonddot}[0]{{\ddot{\upsilon}}}
\newcommand{\varepsilonddot}[0]{{\ddot{\varepsilon}}}
\newcommand{\varphiddot}[0]{{\ddot{\varphi}}}
\newcommand{\varpiddot}[0]{{\ddot{\varpi}}}
\newcommand{\varrhoddot}[0]{{\ddot{\varrho}}}
\newcommand{\varsigmaddot}[0]{{\ddot{\varsigma}}}
\newcommand{\varthetaddot}[0]{{\ddot{\vartheta}}}
\newcommand{\xiddot}[0]{{\ddot{\xi}}}
\newcommand{\zetaddot}[0]{{\ddot{\zeta}}}

\newcommand{\BDelta}[0]{\boldsymbol{\Delta}}
\newcommand{\BGamma}[0]{\boldsymbol{\Gamma}}
\newcommand{\BLambda}[0]{\boldsymbol{\Lambda}}
\newcommand{\BOmega}[0]{\boldsymbol{\Omega}}
\newcommand{\BPhi}[0]{\boldsymbol{\Phi}}
\newcommand{\BPi}[0]{\boldsymbol{\Pi}}
\newcommand{\BPsi}[0]{\boldsymbol{\Psi}}
\newcommand{\BSigma}[0]{\boldsymbol{\Sigma}}
\newcommand{\BTheta}[0]{\boldsymbol{\Theta}}
\newcommand{\BUpsilon}[0]{\boldsymbol{\Upsilon}}
\newcommand{\BXi}[0]{\boldsymbol{\Xi}}
\newcommand{\Balpha}[0]{\boldsymbol{\alpha}}
\newcommand{\Bbeta}[0]{\boldsymbol{\beta}}
\newcommand{\Bchi}[0]{\boldsymbol{\chi}}
\newcommand{\Bdelta}[0]{\boldsymbol{\delta}}
\newcommand{\Bepsilon}[0]{\boldsymbol{\epsilon}}
\newcommand{\Beta}[0]{\boldsymbol{\eta}}
\newcommand{\Bgamma}[0]{\boldsymbol{\gamma}}
\newcommand{\Bkappa}[0]{\boldsymbol{\kappa}}
\newcommand{\Blambda}[0]{\boldsymbol{\lambda}}
\newcommand{\Bmu}[0]{\boldsymbol{\mu}}
\newcommand{\Bnu}[0]{\boldsymbol{\nu}}
%\newcommand{\Bomega}[0]{\boldsymbol{\omega}}
\newcommand{\Bphi}[0]{\boldsymbol{\phi}}
\newcommand{\Bpi}[0]{\boldsymbol{\pi}}
\newcommand{\Bpsi}[0]{\boldsymbol{\psi}}
\newcommand{\Brho}[0]{\boldsymbol{\rho}}
\newcommand{\Bsigma}[0]{\boldsymbol{\sigma}}
%\newcommand{\Btau}[0]{\boldsymbol{\tau}}
%\newcommand{\Btheta}[0]{\boldsymbol{\theta}}
\newcommand{\Bupsilon}[0]{\boldsymbol{\upsilon}}
\newcommand{\Bvarepsilon}[0]{\boldsymbol{\varepsilon}}
\newcommand{\Bvarphi}[0]{\boldsymbol{\varphi}}
\newcommand{\Bvarpi}[0]{\boldsymbol{\varpi}}
\newcommand{\Bvarrho}[0]{\boldsymbol{\varrho}}
\newcommand{\Bvarsigma}[0]{\boldsymbol{\varsigma}}
\newcommand{\Bvartheta}[0]{\boldsymbol{\vartheta}}
\newcommand{\Bxi}[0]{\boldsymbol{\xi}}
\newcommand{\Bzeta}[0]{\boldsymbol{\zeta}}
%
%</bold and dot greek symbols>
%<infrequent>
%
%\newcommand{\AreaOp}[1]{\AName_{#1}}
%\newcommand{\Babs}[0]{\abs{\BB}}
%\newcommand{\Bcap}[0]{\hat{\BB}}
%\newcommand{\BrPrimeRej}[0]{\rcap(\rcap \wedge \Br')}
%\newcommand{\CA}[0]{\mathcal{A}}
%\newcommand{\Cos}[1]{\cos{\left({#1}\right)}}
%\newcommand{\Det}[1] {\abs{#1}}
%\newcommand{\Dsq}[2] {\frac {\partial^2 {#1}} {\partial {#2}^2}}
%\newcommand{\Exp}[1]{\exp{\left({#1}\right)}}
%\newcommand{\Norm}[1]{\left\lVert{#1}\right\rVert}
%\newcommand{\Sin}[1]{\sin{\left({#1}\right)}}
%\newcommand{\T}[0]{\text{T}}
%\newcommand{\VolumeOp}[1]{\VName_{#1}}
%\newcommand{\agrad}[0]{\Ba \cdot \nabla}
%\newcommand{\alphacap}[0]{\hat{\boldsymbol{\alpha}}}
%\newcommand{\Fcap}[0]{\hat{\BF}}
%\newcommand{\bithree}[0]{{\Bi}_3}
%\newcommand{\bxa}[0]{\Bx\Ba}
%\newcommand{\coordvec}[2]{
%\newcommand{\costheta}[0]{\acap \cdot \xcap}
%\newcommand{\ddt}[1]{\ddot{#1}}
%\newcommand{\ddu}[1] {\frac {d{#1}} {du}}
%\newcommand{\dsqxj}[2] {\frac {\partial^2 {#1}} {\partial {x_{#2}}^2}}
%\newcommand{\dtheta}[1]{\frac{d {#1}}{d \theta}}
%\newcommand{\dt}[1]{\dot{#1}}
%\newcommand{\dt}[1]{\frac{d {#1}}{dt}}
%\newcommand{\dxj}[2] {\frac {\partial {#1}} {\partial {x_{#2}}}}
%\newcommand{\halfPhi}[0]{\frac{\phi}{2}}
%\newcommand{\half}[0]{\inv{2}}
%\newcommand{\inv}[1]{\frac{1}{#1}}
%\newcommand{\laplacian}[0]{\nabla^2}
%\newcommand{\matrixoftx}[3]{
%\newcommand{\nrrp}[0]{\norm{\rcap \wedge \Br'}}
%\newcommand{\oiint}{\bigcirc \hspace{-1.4em} \int \hspace{-.8em} \int}
%\newcommand{\transpose}[1]{{#1}^{\text{T}}}
%\newcommand{\transpose}[1]{{{#1}^{\TextTranspose}}}
%\newcommand{\transpose}[1]{{{#1}^{\text{T}}}}
%\newcommand{\barA}[0]{\bar{A}}
%\newcommand{\qbar}[0]{\bar{q}}
%\newcommand{\qdotbar}[0]{\dot{\bar{q}}}
%
%</infrequent>





%\usepackage{listings}
%\usepackage{txfonts} % for ointctr... (also appears to make "prettier" \int and \sum's)
%\usepackage[bookmarks=true]{hyperref}

%\usepackage{color,cite,graphicx}
   % use colour in the document, put your citations as [1-4]
   % rather than [1,2,3,4] (it looks nicer, and the extended LaTeX2e
   % graphics package. 
%\usepackage{latexsym,amssymb,epsf} % do not remember if these are
   % needed, but their inclusion can not do any damage


\chapter{Bohm Chapter 10 problems}
\label{chap:bohmCh10}
%\author{Peeter Joot \quad peeter.joot@gmail.com }
\date{ April 23, 2009.  bohmCh10.tex }

%\begin{document}

%\maketitle{}
%\tableofcontents
\section{Bohm Chapter 10 problems}

Problems and additional details from reading of \citep{bohm1989qt}, chapter 10.

Differing from the text, the notation \(\expectation{O}\) has been used instead \(\overbar{O}\) mostly due to not knowing how to format the a wide overbar, and getting peculiar looking results.

\subsection{P1. Uncertainty calculations}

Calculate \(\Delta x \Delta p\) for a few wave functions

\subsubsection{Gaussian wave function}

\begin{equation}\label{eqn:bohmCh10:20}
\begin{aligned}
\psi = \alpha_1 e^{-\alpha x^2/2}
\end{aligned}
\end{equation}

Normalization

\begin{equation}\label{eqn:bohmCh10:40}
\begin{aligned}
1 
&= \Abs{\alpha_1}^2 \int e^{-\alpha x^2} dx \\
%&= \Abs{\alpha_1}^2 \sqrt{\pi/\alpha}
\end{aligned}
\end{equation}

Position expectation is zero, since it is odd:

\begin{equation}\label{eqn:bohmCh10:60}
\begin{aligned}
\expectation{x} \propto \int x e^{-\alpha x^2} dx = 0
\end{aligned}
\end{equation}

And the second power

\begin{equation}\label{eqn:bohmCh10:80}
\begin{aligned}
\expectation{x^2} 
&= \Abs{\alpha_1}^2 \int x^2 e^{-\alpha x^2} dx \\
&= \Abs{\alpha_1}^2 \int x (e^{-\alpha x^2}/-2\alpha)' dx \\
&= \Abs{\alpha_1}^2 \int (e^{-\alpha x^2}/2\alpha) dx \\
&= \inv{2\alpha} \mathLabelBox{\Abs{\alpha_1}^2 \int e^{-\alpha x^2} dx}{\(=1\)} \\
&= \inv{2\alpha}
\end{aligned}
\end{equation}

For the first momentum expectation, we have zero again since we end up with an odd integral:

\begin{equation}\label{eqn:bohmCh10:100}
\begin{aligned}
\expectation{p} 
&= -i \Hbar \Abs{\alpha_1}^2 \int e^{-\alpha x^2 /2} \frac{d}{dx} e^{-\alpha x^2 /2} dx \\
&= -i \Hbar \Abs{\alpha_1}^2 \int e^{-\alpha x^2 /2} (-\alpha x) e^{-\alpha x^2 /2} dx \\
&= 0
\end{aligned}
\end{equation}

And for the second power

\begin{equation}\label{eqn:bohmCh10:120}
\begin{aligned}
\expectation{p^2} 
&= - \Hbar^2 \Abs{\alpha_1}^2 \int e^{-\alpha x^2 /2} \frac{d}{dx} ((-\alpha x) e^{-\alpha x^2 /2}) dx \\
&= \Hbar^2 \Abs{\alpha_1}^2 \alpha \int e^{-\alpha x^2 /2} \frac{d}{dx} (x e^{-\alpha x^2 /2}) dx \\
&= -\Hbar^2 \Abs{\alpha_1}^2 \alpha \int \left(\frac{d}{dx}e^{-\alpha x^2 /2}\right) (x e^{-\alpha x^2 /2}) dx \\
&= \Hbar^2 \alpha^2 \mathLabelBox{\Abs{\alpha_1}^2 \int x^2 e^{-\alpha x^2} dx}{\(\expectation{x^2} = 1/2\alpha\)} \\
&= \Hbar^2 \alpha \inv{2}
\end{aligned}
\end{equation}

Assembling results

\begin{equation}\label{eqn:bohmCh10:140}
\begin{aligned}
\Delta x \Delta p 
&= \sqrt{\Hbar^2 \alpha \inv{2} \inv{2\alpha}} \\
&= \sqrt{\Hbar^2 \inv{4}} \\
&= \frac{\Hbar}{2}
\end{aligned}
\end{equation}

This is the expected result since equality with \(\Hbar/2\) occurs only with the Gaussian.

\subsubsection{Absolute valued exponential wave function}

\begin{equation}\label{eqn:bohmCh10:160}
\begin{aligned}
\psi &= \alpha_2 e^{-\alpha\Abs{x}}
\end{aligned}
\end{equation}

Normalization

\begin{equation}\label{eqn:bohmCh10:180}
\begin{aligned}
1 
&= 2 \Abs{\alpha_2}^2 \int_0^\infty e^{-2 \alpha{x}} dx \\
&= 2 \Abs{\alpha_2}^2 \left. \frac{e^{-2 \alpha{x}}}{-2\alpha} \right\vert_0^\infty \\
&= 2 \Abs{\alpha_2}^2 \inv{2\alpha}
\end{aligned}
\end{equation}

\begin{equation}\label{eqn:bohmCh10:200}
\begin{aligned}
\expectation{x} = 0
\end{aligned}
\end{equation}

(odd).

\begin{equation}\label{eqn:bohmCh10:220}
\begin{aligned}
\expectation{x^2} 
&= 2 \Abs{\alpha_2}^2 \int_0^\infty x^2 e^{-2\alpha x} dx \\
&= \Abs{\alpha_2}^2 \inv{2 \alpha^3}
\end{aligned}
\end{equation}

For the momentum we need derivatives

\begin{equation}\label{eqn:bohmCh10:240}
\begin{aligned}
\frac{d}{dx} e^{-\alpha \Abs{x}} 
&=
\left\{
\begin{array}{l l}
\frac{d}{dx}(e^{-\alpha x} & \quad \mbox{\(x>0\)} \\
\frac{d}{dx}(e^{\alpha x} & \quad \mbox{\(x<0\)} \\
\end{array}
\right. \\
&=
\left\{
\begin{array}{l l}
-\alpha (e^{-\alpha x} & \quad \mbox{\(x>0\)} \\
\alpha (e^{\alpha x} & \quad \mbox{\(x<0\)} \\
\end{array}
\right. \\
&=
-\alpha \sgn(x) e^{-\alpha \Abs{x}}
\end{aligned}
\end{equation}

\begin{equation}\label{eqn:bohmCh10:260}
\begin{aligned}
\expectation{p} 
&= i \Hbar \Abs{\alpha_2}^2 \alpha \int \sgn(x) e^{-2\alpha\Abs{x}} dx  \\
&= 0
\end{aligned}
\end{equation}

(odd)

\begin{equation}\label{eqn:bohmCh10:280}
\begin{aligned}
\expectation{p^2} 
&= (-i \Hbar)^2 \Abs{\alpha_2}^2 \alpha^2 \int (\sgn(x))^2 e^{-2\alpha\Abs{x}} dx  \\
&= - 2(\Hbar)^2 \Abs{\alpha_2}^2 \alpha^2 \int_0^\infty e^{-2\alpha\Abs{x}} dx  \\
&= 2(\Hbar)^2 \Abs{\alpha_2}^2 \alpha^2 \inv{2\alpha} \\
&= (\Hbar)^2 \Abs{\alpha_2}^2 \alpha \\
\end{aligned}
\end{equation}

Putting results together we have

\begin{equation}\label{eqn:bohmCh10:300}
\begin{aligned}
\expectation{p^2} \expectation{x^2} 
&= (\Hbar)^2 \alpha_2^4 \inv{2 \alpha^2} \\
&= (\Hbar)^2/2
\end{aligned}
\end{equation}

Same as in the Gaussian.

\subsubsection{Squared polynomial}

\begin{equation}\label{eqn:bohmCh10:320}
\begin{aligned}
\psi = \frac{\alpha_3}{(\alpha^2 + x^2)^2}
\end{aligned}
\end{equation}

For this one, the integrals were evaluated with Mathematica online integrator, where the contributions at \(\infty\) were scaled \(\arctan(x/\alpha)\) values.

\begin{equation}\label{eqn:bohmCh10:340}
\begin{aligned}
1 
&= \Abs{\alpha_3}^2 \int \frac{dx}{(\alpha^2 + x^2)^4} \\
&= \Abs{\alpha_3}^2 \frac{15 \pi}{48 \alpha^7}
\end{aligned}
\end{equation}

\begin{equation}\label{eqn:bohmCh10:360}
\begin{aligned}
\expectation{x} &= 0
\end{aligned}
\end{equation}

\begin{equation}\label{eqn:bohmCh10:380}
\begin{aligned}
\expectation{x^2} 
&= \Abs{\alpha_3}^2 \int \frac{x^2 dx}{(\alpha^2 + x^2)^4} \\
&= \Abs{\alpha_3}^2 \frac{3 \pi}{48 \alpha^5}
\end{aligned}
\end{equation}

\begin{equation}\label{eqn:bohmCh10:400}
\begin{aligned}
\expectation{p} 
&= -i \Hbar \Abs{\alpha_3}^2 \int \frac{dx}{(\alpha^2 + x^2)^2} \frac{d}{dx} \frac{1}{(\alpha^2 + x^2)^2}  \\
&= -i \Hbar \Abs{\alpha_3}^2 \int \frac{dx}{(\alpha^2 + x^2)^2} \frac{-4x}{(\alpha^2 + x^2)^2} \\
&= 0
\end{aligned}
\end{equation}

\begin{equation}\label{eqn:bohmCh10:420}
\begin{aligned}
\expectation{p^2} 
&= (-i \Hbar)^2 \Abs{\alpha_3}^2 \int \frac{dx}{(\alpha^2 + x^2)^2} \frac{d^2}{dx^2} \frac{1}{(\alpha^2 + x^2)^2}  \\
&= 4 \Hbar^2 \Abs{\alpha_3}^2 \int \frac{dx}{(\alpha^2 + x^2)^2} \frac{d}{dx} \frac{x}{(\alpha^2 + x^2)^3} \\
&= 4 \Hbar^2 \Abs{\alpha_3}^2 \int dx
\left( \frac{1}{(\alpha^2 + x^2)^5} \frac{x(-3)(2x)}{(\alpha^2 + x^2)^6} \right)
\\
&= 4 \Hbar^2 \Abs{\alpha_3}^2 \int dx \frac{\alpha^2 - 5x^2}{(\alpha^2 + x^2)^6} 
\\
&= 4 \Hbar^2 \Abs{\alpha_3}^2 \frac{105 \pi}{960 \alpha^9}
\\
\end{aligned}
\end{equation}

Assembling

\begin{equation}\label{eqn:bohmCh10:440}
\begin{aligned}
\expectation{p^2} \expectation{x^2} 
&= 4 \Hbar^2 \alpha_3^4 \frac{105 \pi}{960 \alpha^9} \frac{3 \pi}{48 \alpha^5} \\
&= \Hbar^2 \frac{7}{25}
\end{aligned}
\end{equation}

For
\begin{equation}\label{eqn:bohmCh10:460}
\begin{aligned}
\Delta{p} \Delta{x} 
&= \Hbar \frac{\sqrt{7}}{5} \\
&\approx 0.52 \Hbar  \\
&> \Hbar/2
\end{aligned}
\end{equation}

\subsection{P2. Correlation coefficients}

It is noted that a classical correlation coefficient for random variables \(x\), and \(p\) has the form

\begin{equation}\label{eqn:bohmCh10:480}
\begin{aligned}
C_{n,m}
&= \expectation{x^n p^m} - \expectation{x^n}\expectation{p^m}
\end{aligned}
\end{equation}

However, for operator expectation values and average of both orderings is more reasonable

\begin{equation}\label{eqn:bohmCh10:500}
\begin{aligned}
C_{n,m}
&= \inv{2} \left( 
\expectation{x^n p^m} - \expectation{x^n}\expectation{p^m} 
+ \expectation{p^m x^n} - \expectation{p^m}\expectation{x^n} 
\right) \\
&= \inv{2} \left( \expectation{x^n p^m} + \expectation{p^m x^n} \right) - \expectation{x^n}\expectation{p^m} 
\end{aligned}
\end{equation}

With the operator substitution \(p \rightarrow -i \Hbar d/dx\) this provides equation (7) in the text.

\subsubsection{first correlations}

Calculate \(C_{1,1}\), and \(C_{2,2}\) for 

\begin{equation}\label{eqn:bohmCh10:520}
\begin{aligned}
\psi = \alpha e^{-\alpha x^2/2}
\end{aligned}
\end{equation}

First the normalization and first and second order expectations.

\begin{equation}\label{eqn:bohmCh10:540}
\begin{aligned}
1 
&= \alpha^2 \int e^{-\alpha x^2} dx \\
&= \alpha^2 \sqrt{\pi/\alpha} \\
\implies \\
\alpha &= \pi^{1/3}
\end{aligned}
\end{equation}

\begin{equation}\label{eqn:bohmCh10:560}
\begin{aligned}
\expectation{x} &= 0
\end{aligned}
\end{equation}

\begin{equation}\label{eqn:bohmCh10:580}
\begin{aligned}
\expectation{p} &= 0
\end{aligned}
\end{equation}

And in particular

\begin{equation}\label{eqn:bohmCh10:600}
\begin{aligned}
\expectation{p} \expectation{x} &= 0 \\
\end{aligned}
\end{equation}

\subsubsection{second correlations}

\begin{equation}\label{eqn:bohmCh10:620}
\begin{aligned}
\expectation{x^2} 
&= \alpha^2 \int x^2 e^{-\alpha x^2} dx \\
&= \alpha^2 \int x x e^{-\alpha x^2} dx \\
&= \alpha^2 \int x (e^{-\alpha x^2}/-\alpha)' dx \\
&= \alpha^2 \int e^{-\alpha x^2}/\alpha dx \\
&= \alpha \int e^{-\alpha x^2} dx \\
&= \alpha \sqrt{\pi/\alpha} \\
&= \sqrt{\alpha \pi} \\
&= \pi^{2/3} \\
\end{aligned}
\end{equation}

\begin{equation}\label{eqn:bohmCh10:640}
\begin{aligned}
\expectation{p^2} 
&= -\alpha^2 \Hbar^2 \int e^{-\alpha x^2/2} \frac{d^2}{dx^2} e^{-\alpha x^2/2} dx \\
&= \alpha^2 \Hbar^2 \int \frac{d}{dx} e^{-\alpha x^2/2} \frac{d}{dx} e^{-\alpha x^2/2} dx \\
&= \alpha^2 \Hbar^2 \int (-\alpha x)^2 e^{-\alpha x^2} dx \\
&= \alpha^4 \Hbar^2 \int x^2 e^{-\alpha x^2} dx \\
&= \alpha^4 \Hbar^2 \int x x e^{-\alpha x^2} dx \\
&= \alpha^4 \Hbar^2 \int x (e^{-\alpha x^2}/-\alpha)' dx \\
&= \alpha^3 \Hbar^2 \int e^{-\alpha x^2} dx \\
&= \alpha^3 \Hbar^2 \inv{2\alpha} \\
&= \alpha^2 \Hbar^2  \\
&= \pi^{2/3} \Hbar^2 
\end{aligned}
\end{equation}

\begin{equation}\label{eqn:bohmCh10:660}
\begin{aligned}
\expectation{p^2} \expectation{x^2} &= \Hbar^2 \pi^{4/3}
\end{aligned}
\end{equation}

For the first terms we want

\begin{equation}\label{eqn:bohmCh10:680}
\begin{aligned}
\inv{2} \left( \expectation{x^2 p^2} + \expectation{p^2 x^2} \right)
&=
\frac{-\Hbar^2 \alpha^2}{2} \int \left( 
e^{-\alpha x^2/2} x^2 \frac{d^2}{dx^2} (e^{-\alpha x^2/2} )
+ e^{-\alpha x^2/2} \frac{d^2}{dx^2} ( x^2 e^{-\alpha x^2/2} ) \right) dx \\
&=
\frac{-\Hbar^2 \alpha^2}{2} \int \left( 
 \frac{d^2}{dx^2} ( e^{-\alpha x^2/2} x^2 ) e^{-\alpha x^2/2} 
+ e^{-\alpha x^2/2} \frac{d^2}{dx^2} ( x^2 e^{-\alpha x^2/2} ) \right) dx \\
&=
{-\Hbar^2 \alpha^2} \int e^{-\alpha x^2/2} \frac{d^2}{dx^2} ( e^{-\alpha x^2/2} x^2 ) dx \\
&=
{\Hbar^2 \alpha^2} \int \left( \frac{d}{dx} e^{-\alpha x^2/2} \right) \left( \frac{d}{dx} ( e^{-\alpha x^2/2} x^2 ) \right) dx \\
&=
{\Hbar^2 \alpha^2} \int (-\alpha x) e^{-\alpha x^2} (2x + x^2(-\alpha x)) dx \\
&=
{\Hbar^2 \alpha^3} \int x^2 e^{-\alpha x^2} (-2 + x^2 \alpha ) dx \\
&=
{-\Hbar^2 \alpha^2} \sqrt{\pi/\alpha}  \\
&=
-\Hbar^2 \pi
\end{aligned}
\end{equation}

This leaves 

\begin{equation}\label{eqn:bohmCh10:700}
\begin{aligned}
C_{2,2} &=
-\Hbar^2 \left( \pi + \pi^{2/3} \right)
\end{aligned}
\end{equation}

\subsection{P3. First correlations zero for real wave function}

Show that the first correlation coefficient is zero for any real wave function.

\begin{equation}\label{eqn:bohmCh10:720}
\begin{aligned}
C_{1,1} &= \inv{2}\left( \expectation{x p} + \expectation{p x} \right) - \expectation{x}\expectation{p}
\end{aligned}
\end{equation}

Calculate instead the equivalent problem

\begin{equation}\label{eqn:bohmCh10:740}
\begin{aligned}
2 C_{1,1}/(-i\Hbar) &= \left( \expectation{x \frac{d}{dx}} + \expectation{\frac{d}{dx} x} \right) - 2 \expectation{x}\expectation{\frac{d}{dx}}
\end{aligned}
\end{equation}

For the anti-commutator part we have

\begin{equation}\label{eqn:bohmCh10:760}
\begin{aligned}
\expectation{x \frac{d}{dx}} + \expectation{\frac{d}{dx} x} 
&=
\int \psi x \psi' + \psi (x \psi)' \\
&=
\int \psi x \psi' - \psi' x \psi \\
&= 0
\end{aligned}
\end{equation}

and for the remainder if one is zero then the sum is.  In particular

\begin{equation}\label{eqn:bohmCh10:780}
\begin{aligned}
\expectation{\frac{d}{dx}} 
&= \int_{-\infty}^\infty dx \psi \psi' \\
&= \inv{2} \int_{-\infty}^\infty dx (\psi^2)' \\
&= \inv{2} \left. \psi^2 \right\vert_{-\infty}^\infty
\end{aligned}
\end{equation}

Provided the wave function vanishes in the square at \(\pm \infty\), then we are done.

\subsection{P4}
\subsection{P5}
\subsection{P6}

The phase space text on this page is not clear to me.  Revisit after study
of phase space, Poisson brackets, and Liouville's theorem in a classical
context.

\subsection{P7. wave function for the position and momentum operators for the equality uncertainty case}

Note that in the definitions of \(\alpha\) and \(\beta\) right before equation (25) in the text, the symbols are reversed.  For consistency with condition (1)
this should be

\begin{equation}\label{eqn:bohmCh10:800}
\begin{aligned}
\beta &= (x - \overbar{x}) \\
\alpha &= (p - \overbar{p})
\end{aligned}
\end{equation}

Where condition (1) for equality in the Schwartz inequality for this 
generalized uncertainty principle is

\begin{equation}\label{eqn:bohmCh10:820}
\begin{aligned}
\alpha \psi &= C \beta \psi
\end{aligned}
\end{equation}

Putting the two together for this problem one has

\begin{equation}\label{eqn:bohmCh10:840}
\begin{aligned}
(p - \overbar{p}) \psi &= C (x - \overbar{x}) \psi \\
\implies \\
p \psi &= \overbar{p} \psi + C (x - \overbar{x}) \psi \\
\end{aligned}
\end{equation}

or
\begin{equation}\label{eqn:bohmCh10:860}
\begin{aligned}
\frac{\Hbar}{i} \frac{d\psi}{dx} &= \overbar{p} \psi + C (x - \overbar{x}) \psi \\
\end{aligned}
\end{equation}

Integrating, as was done in the \(\overbar{x} = \overbar{p} = 0\) case in the text, 
one has

\begin{equation}\label{eqn:bohmCh10:880}
\begin{aligned}
\ln \psi &= \frac{i}{\Hbar}( \overbar{p} x + C (x - \overbar{x})^2/2 ) + \ln D \\
\end{aligned}
\end{equation}

or
\begin{equation}\label{eqn:bohmCh10:900}
\begin{aligned}
\psi &= D e^{i\overbar{p} x/\Hbar} e^{ i C (x - \overbar{x})^2/2\Hbar } \\
\end{aligned}
\end{equation}

Note that this shows there is a typo in equation (26) in the text too (\(i\overbar{p} x\) needs the \(1/\Hbar\) factor).  References to \(C\) in the text preceding this should be \(C/\Hbar\) in a few cases too.

It was shown above that real wave functions have the vanishing \(C_{1,1}\) coefficient required for uncorrelated operators, and if that is the case \(i C/\Hbar\), must be a real negative constant.  Denoting that as \(-a\) as in the text one has

\begin{equation}\label{eqn:bohmCh10:920}
\begin{aligned}
\psi \propto e^{i\overbar{p} x/\Hbar} e^{ -a(x - \overbar{x})^2/2} \\
\end{aligned}
\end{equation}

\subsection{P8. Calculate uncertainty for the initially Gaussian wave function}

Wave function for the problem is

\begin{equation}\label{eqn:bohmCh10:940}
\begin{aligned}
\psi &= \alpha \exp\left( -(A - iB) \frac{x^2}{2} \right) \\
A &= \frac{(\Delta k)^2}{ 1 + \frac{\Hbar^2 t^2}{m^2}(\Delta k)^4 } \\
B &= (\Delta k)^4 \frac{\Hbar t}{m} \frac{1}{ 1 + \frac{\Hbar^2 t^2}{m^2}(\Delta k)^4 } \\
\end{aligned}
\end{equation}

The normalization is

\begin{equation}\label{eqn:bohmCh10:960}
\begin{aligned}
1 
&= \Abs{\alpha}^2 \int \exp\left( -A {x^2} \right) \\
&= \Abs{\alpha}^2 \sqrt{\frac{\pi}{A}}
\end{aligned}
\end{equation}

First moment
\begin{equation}\label{eqn:bohmCh10:980}
\begin{aligned}
\expectation{x} = 0
\end{aligned}
\end{equation}

Second moment
\begin{equation}\label{eqn:bohmCh10:1000}
\begin{aligned}
\expectation{x^2} 
&= 
\Abs{\alpha}^2 \int x^2 \exp\left( -A x^2 \right) \\
&= 
\Abs{\alpha}^2 \int x (\exp\left( -A x^2 \right)/(-2A))' \\
&= 
\Abs{\alpha}^2 \int \exp\left( -A x^2 \right)/2A \\
&= 
\Abs{\alpha}^2 \inv{2A} \sqrt{ \frac{\pi}{A}} \\
&= 
\inv{2A} \\
&=
\inv{2} \frac{ 1 + \frac{\Hbar^2 t^2}{m^2}(\Delta k)^4 }{(\Delta k)^2}
 \\
&=
\inv{2}\left(\inv{(\Delta k)^2} + \frac{\Hbar^2 t^2}{m^2}(\Delta k)^2 \right) \\
\end{aligned}
\end{equation}

For the momentum expectation

\begin{equation}\label{eqn:bohmCh10:1020}
\begin{aligned}
\expectation{p}/(-i\Hbar\Abs{\alpha}^2) 
&= \int dx
\exp\left( -(A + iB) \frac{x^2}{2} \right) 
\frac{d}{dx}
\exp\left( -(A - iB) \frac{x^2}{2} \right)  \\
&= \int dx
\exp\left( -(A + iB) \frac{x^2}{2} \right) 
(-(A -iB)x)
\exp\left( -(A - iB) \frac{x^2}{2} \right)  \\
&= 
-(A -iB)
\int x \exp\left( -A x^2 \right) dx
\\
&= 0
\end{aligned}
\end{equation}

Second moment
\begin{equation}\label{eqn:bohmCh10:1040}
\begin{aligned}
\expectation{p^2}/((-i\Hbar)^2\Abs{\alpha}^2) 
&= \int dx
\exp\left( -(A + iB) \frac{x^2}{2} \right) 
\frac{d}{dx}
(-(A -iB)x) \exp\left( -(A - iB) \frac{x^2}{2} \right)  \\
&= -\int dx
(-(A -iB)x) \exp\left( -(A - iB) \frac{x^2}{2} \right)  
\frac{d}{dx}
\exp\left( -(A + iB) \frac{x^2}{2} \right) 
\\
&= -\int dx
(-(A -iB)x) \exp\left( -(A - iB) \frac{x^2}{2} \right) 
(-(A +iB)x) \exp\left( -(A - iB) \frac{x^2}{2} \right)  \\
&= -\int x^2 dx
(A^2 + B^2) \exp\left( -(A - iB) \frac{x^2}{2} \right) 
\exp\left( -(A - iB) \frac{x^2}{2} \right)  \\
&= -\int x^2 dx (A^2 + B^2) \exp\left( -A x^2 \right) \\
&= - (A^2 + B^2) \int x (\exp\left( -A x^2 \right)/(-2A))' \\
&= -\frac{A^2 + B^2}{2A} \int \exp\left( -A x^2 \right) \\
\end{aligned}
\end{equation}

So we have
\begin{equation}\label{eqn:bohmCh10:1060}
\begin{aligned}
\expectation{p^2} &= \Hbar^2 \frac{A^2 + B^2}{2A} 
\end{aligned}
\end{equation}

But 
\begin{equation}\label{eqn:bohmCh10:1080}
\begin{aligned}
A^2 + B^2 
&= 
\inv{(1 + \frac{\Hbar^2 t^2}{m^2}(\Delta k)^4)^2} ((\Delta k)^4 + (\Delta k)^8 \left(\frac{\Hbar t}{m}\right)^2 ) \\
&= 
\inv{(1 + \frac{\Hbar^2 t^2}{m^2}(\Delta k)^4)^2} (\Delta k)^4(1 + (\Delta k)^4 \left(\frac{\Hbar t}{m}\right)^2 ) \\
&= 
\frac{(\Delta k)^4}{1 + \frac{\Hbar^2 t^2}{m^2}(\Delta k)^4} \\
&= (\Delta k)^2 A
%A &= \frac{(\Delta k)^2}{ 1 + \frac{\Hbar^2 t^2}{m^2}(\Delta k)^4 } \\
\end{aligned}
\end{equation}

So we have the constant second moment as desired

\begin{equation}\label{eqn:bohmCh10:1100}
\begin{aligned}
\expectation{p^2} &= \Hbar^2 (\Delta k)^2/2
\end{aligned}
\end{equation}

\begin{equation}\label{eqn:bohmCh10:1120}
\begin{aligned}
(\Delta x \Delta p)^2
&= \Hbar^2 \frac{(\Delta k)^2}{4} \left( \inv{(\Delta k)^2} + \frac{\Hbar^2 t^2}{m^2}(\Delta k)^2 \right) \\
&= \left(\frac{\Hbar}{2}\right)^2 \left( 1 + \frac{\Hbar^2 t^2}{m^2}(\Delta k)^4 \right) \\
\end{aligned}
\end{equation}

This matches all the expectations.  At \(t=0\) we have equality for the minimum uncertainty, and it grows as
time increases.

\subsection{P9. (first P9 of the chapter.)}

if \(\psi_A(x_1)\) and \(\psi_B(x_2)\) are independent, then the probability integral over all space is

\begin{equation}\label{eqn:bohmCh10:1140}
\begin{aligned}
\int P(x_1, x_2) dx_1 dx_2
&=
\iint \psi_A(x_1) \psi_B(x_2) dx_1 dx_2 \\
&=
\int \psi_A(x_1) dx_1 \int \psi_B(x_2) dx_2 \\
\end{aligned}
\end{equation}

So, if they are also separately normalized then so is this one.

\subsection{P9. (second P9 of the chapter.)}

Prove, ``It may be shown that if \(\psi\) is not an eigenfunction of the operator \(O\) must show some fluctuation.''

By fluctuation, I assume he means deviation of the moments, 
as in the variational differences

\begin{equation}\label{eqn:bohmCh10:1160}
\begin{aligned}
\expectation{O^n} - (\expectation{O})^n 
\end{aligned}
\end{equation}

How to prove this?  Suppose that the wave function is almost an eigenfunction
differing by a bit

\begin{equation}\label{eqn:bohmCh10:1180}
\begin{aligned}
O \psi = \lambda \psi + \epsilon
\end{aligned}
\end{equation}

Then we have for the moments

\begin{equation}\label{eqn:bohmCh10:1200}
\begin{aligned}
\expectation{O}
&= \int \psi^\conj O \psi \\
&= \int \psi^\conj ( \lambda \psi + \epsilon ) \\
&= \lambda + \int \psi^\conj \epsilon \\
\end{aligned}
\end{equation}

\begin{equation}\label{eqn:bohmCh10:1220}
\begin{aligned}
\expectation{O^2}
&= \int \psi^\conj O^2 \psi \\
&= \int \psi^\conj O \left( \lambda \psi + \epsilon \right) \\
&= \lambda \expectation{O} + \int \psi^\conj O \epsilon \\
&= \lambda \expectation{O} + \int \epsilon^\conj O \psi \\
&= \lambda \expectation{O} + \int \epsilon^\conj \left(\lambda \psi + \epsilon\right) \\
&= \left(\expectation{O} - \int \psi^\conj \epsilon \right)\expectation{O} + 
\lambda \int \epsilon^\conj \psi 
+ \int \epsilon^\conj \epsilon 
\\
&= \expectation{O}^2 - \int \psi^\conj \epsilon \expectation{O} + \left( \int \psi^\conj \epsilon - \expectation{O}\right) \int \epsilon^\conj \psi + \int \epsilon^\conj \epsilon 
\\
&= \expectation{O}^2 
- \expectation{O} \left( \int \epsilon^\conj \psi + \int \psi^\conj \epsilon \right)
+ \Abs{\int \psi^\conj \epsilon}^2
+ \int \epsilon^\conj \epsilon 
\\
\end{aligned}
\end{equation}

So we have

\begin{equation}\label{eqn:bohmCh10:1240}
\begin{aligned}
\expectation{O^2} - \expectation{O}^2 
&= 
\Abs{\int \psi^\conj \epsilon}^2 + \int \epsilon^\conj \epsilon - 2 \expectation{O} \Re \int \epsilon^\conj \psi \\
\end{aligned}
\end{equation}

There is no good reason to assume that this RHS should be zero in general, so at least for the second order moment this shows that we have the fluctuation when the wave
function is not an eigenfunction.

\subsection{P10}

\begin{equation}\label{eqn:bohmCh10:1260}
\begin{aligned}
\psi = \frac{A}{(x-x_0)^2 + (\Delta x)^2}
\end{aligned}
\end{equation}

Normalizing, picking the upper plane contour around \(i \Delta x\), we have

\begin{equation}\label{eqn:bohmCh10:1280}
\begin{aligned}
1 &= \int \psi^\conj \psi \\
&= A^2 \int_{-\infty}^\infty \frac{dx}{(x-x_0)^2 + (\Delta x)^2} \\
&= A^2 \int_{-\infty}^\infty \frac{dx}{x^2 + (\Delta x)^2} \\
&= A^2 \int_{-\infty}^\infty \frac{dx}{(x + i\Delta x)(x - i \Delta x)} \\
&= A^2 \frac{2 \pi i }{2 i\Delta x} \\
&= A^2 \frac{\pi}{\Delta x} \\
\end{aligned}
\end{equation}

So we have the desired normalization

\begin{equation}\label{eqn:bohmCh10:1300}
\begin{aligned}
A &= \sqrt{\frac{\Delta x}{\pi}} \\
\end{aligned}
\end{equation}

So how do you show that the now normalized wave function is an eigenfunction of \(x\) 

\begin{equation}\label{eqn:bohmCh10:1320}
\begin{aligned}
\psi &= \sqrt{\frac{\Delta x}{\pi}} \frac{1}{(x-x_0)^2 + (\Delta x)^2} \\
\end{aligned}
\end{equation}

Would a calculation of the expectation value for the position operator be sufficient?  That is

\begin{equation}\label{eqn:bohmCh10:1340}
\begin{aligned}
\expectation{x} 
&= A^2 \IIinf \frac{x dx }{(x-x_0)^2 + (\Delta x)^2} \\
&= A^2 \IIinf \frac{(u + x_0) du }{u^2 + (\Delta x)^2} \\
&= x_0 + A^2 \IIinf \frac{u du }{u^2 + (\Delta x)^2} \\
\end{aligned}
\end{equation}

Now \(u/(u^2 + \alpha^2)\) has antiderivative \(\ln(x^2 + \alpha^2)/2\), so the PV value of this integral for \(\Delta x \ne 0\)
is

\begin{equation}\label{eqn:bohmCh10:1360}
\begin{aligned}
PV A^2 \IIinf \frac{u du }{u^2 + (\Delta x)^2} 
PV \frac{\Delta x}{\pi} \IIinf \frac{u du }{u^2 + (\Delta x)^2} 
&= \inv{2} \lim_{R\rightarrow \infty} \frac{\Delta x}{\pi} \ln\left( \frac{R^2 + (\Delta x)^2}{(-R)^2 + (\Delta x)^2} \right) \\
&= 0
\end{aligned}
\end{equation}

So, for all \(\Delta x \ne 0\) (where that \(\PV\) integral goes messy), we have

\begin{equation}\label{eqn:bohmCh10:1380}
\begin{aligned}
\expectation{x} &= x_0 
\end{aligned}
\end{equation}

Is this sufficient to show that this wave function approaches an eigenfunction as \(\Delta x \rightarrow 0\).
I suppose that one could loosely argue that the \(A^2\) term kills off the log term at the limit of \(\Delta x = 0\) (if you are careful in the argument about how exactly \(\Delta x \rightarrow 0\) with \(R \rightarrow \infty\)).

\subsection{P11. Delta function example as limit}

\subsubsection{The problem}

Show that for

\begin{equation}\label{eqn:bohmCh10:1400}
\begin{aligned}
\delta_\epsilon(x - x_0) &= \frac{A}{(x-x_0)^2 + \epsilon^2}
\end{aligned}
\end{equation}

The limit has a delta function action

\begin{equation}\label{eqn:bohmCh10:1420}
\begin{aligned}
\delta(x - x_0) &= \lim_{\epsilon \rightarrow 0} \delta_\epsilon(x - x_0)
\end{aligned}
\end{equation}

Calculate \(A\), and explain why it is different than \(A\) in problem 10.

\subsubsection{Normalization}

First for the constant

\begin{equation}\label{eqn:bohmCh10:1440}
\begin{aligned}
1 &= \IIinf dx \delta_\epsilon(x - x_0) \\
&= A \IIinf \frac{dx}{x^2 + \epsilon^2} \\
&= A \IIinf \frac{dx}{(x + i\epsilon)(x - i \epsilon)} \\
&= A \frac{2 \pi i}{ 2 i \epsilon } \\
&= A \frac{\pi}{\epsilon } \\
\end{aligned}
\end{equation}

So we have

\begin{equation}\label{eqn:bohmCh10:1460}
\begin{aligned}
A = \frac{\epsilon }{\pi} \\
\end{aligned}
\end{equation}

This constant is necessarily different from the eigenfunction normalization, since that involved normalization in the square.  The resulting 
nascent delta function is

\begin{equation}\label{eqn:bohmCh10:1480}
\begin{aligned}
\delta_\epsilon(x - x_0) &= \frac{\epsilon }{\pi} \frac{1}{(x-x_0)^2 + \epsilon^2}
\end{aligned}
\end{equation}

\subsubsection{Delta function action}

How do we show that \(\delta_\epsilon\) behaves as a delta function in the limit?  The delta function is really defined by how it
acts on a test function in an integration operation, so let us calculate 

\begin{equation}\label{eqn:bohmCh10:1500}
\begin{aligned}
\IIinf \delta_\epsilon(x - x_0) f(x) dx 
&=
\frac{\epsilon }{\pi} \IIinf \frac{f(x) dx}{(x-x_0)^2 + \epsilon^2} \\
&=
\frac{\epsilon }{\pi} \IIinf \frac{f(u + x_0) du}{u^2 + \epsilon^2} \\
&=
\frac{\epsilon }{\pi} \IIinf \frac{f(u + x_0) du}{(u + i \epsilon)(u - i \epsilon)} \\
\end{aligned}
\end{equation}

An upper half plane contour, assuming that \(f(x_0 + i\epsilon)\) is a regular point (and that f(z) has no poles in the upper half plane) gives us

\begin{equation}\label{eqn:bohmCh10:1520}
\begin{aligned}
\IIinf \delta_\epsilon(x - x_0) f(x) dx 
&=
\frac{\epsilon }{\pi} 2 \pi i \frac{f(i \epsilon + x_0) }{2 i \epsilon} \\
&=
f(i \epsilon + x_0)
\end{aligned}
\end{equation}

So in the limit if \(f(x)\) is regular all the way down the \(x_0 + i\epsilon\) trajectory, we have

\begin{equation}\label{eqn:bohmCh10:1540}
\begin{aligned}
\lim_{\epsilon \rightarrow 0} \IIinf \delta_\epsilon(x - x_0) f(x) dx &= f(x_0)
\end{aligned}
\end{equation}

which is precisely the operational definition of the delta function.

\subsection{P12. Delta function differentiation}

Prove by successive differentiation that

\begin{equation}\label{eqn:bohmCh10:1560}
\begin{aligned}
\frac{d^n f(x)}{dx^n} 
&= \int_{-\infty}^{\infty} \delta(x - x_0) \frac{d^n f(x_0)}{dx_0} dx_0 \\
\end{aligned}
\end{equation}

Doing the integration by parts, and change of variables for the delta function derivatives:

\begin{equation}\label{eqn:bohmCh10:1580}
\begin{aligned}
\frac{d^n f(x)}{dx^n} 
&= \IIinf \frac{d^n \delta(x - x_0)}{dx^n} f(x_0) dx_0 \\
&= \IIinf \frac{d^n \delta(x-x_0)}{{dx_0}^n} (-1)^n f(x_0) dx_0 \\
%&= -(-1)^n \int_\infty^{-\infty} \frac{d^n \delta(x - x_0)}{dx_0^n} f(x_0) dx_0 \\
%&= (-1)^n \int_{-\infty}^{\infty} \frac{d^n \delta(x - x_0)}{dx_0^n} f(x_0) dx_0 \\
&= -(-1)^n \int_{-\infty}^{\infty} \frac{d^{n-1} \delta(x - x_0)}{dx_0^{n-1}} \frac{df(x_0)}{dx_0} dx_0 \\
&= (-1)^2 (-1)^n \int_{-\infty}^{\infty} \frac{d^{n-2} \delta(x - x_0)}{dx_0^{n-2}} \frac{d^2 f(x_0)}{dx_0^2} dx_0 \\
&= \cdots \\
&= (-1)^{n-1} (-1)^n \int_{-\infty}^{\infty} \frac{d^{n-(n-1)} \delta(x - x_0)}{dx_0^{n-(n-1)}} \frac{d^{n-1} f(x_0)}{dx_0^{n-1}} dx_0 \\
&= (-1)^{n-1} (-1)^n \int_{-\infty}^{\infty} \frac{d \delta(x - x_0)}{dx_0} \frac{d^{n-1} f(x_0)}{dx_0^{n-1}} dx_0 \\
&= (-1)^n (-1)^n \int_{-\infty}^{\infty} \delta(x - x_0) \frac{d^n f(x_0)}{dx_0} dx_0 \\
&= \int_{-\infty}^{\infty} \delta(x - x_0) \frac{d^n f(x_0)}{dx_0} dx_0 \\
\end{aligned}
\end{equation}

\subsection{P13. Eigenfunctions for particle in one dimensional box}

Box of side \(L\).  Eigenfunction of \(p\) are given as exponentials in the problem.  Stepping back slightly to see where these come from 
consider the operator eigenvalue statement itself.  This will in fact indirectly solve the problem

\begin{equation}\label{eqn:bohmCh10:1600}
\begin{aligned}
\lambda \psi 
&= p \psi_\lambda \\
&= \left(-i \Hbar \PD{x}{} \right) \psi_\lambda \\
\end{aligned}
\end{equation}

This can be integrated

\begin{equation}\label{eqn:bohmCh10:1620}
\begin{aligned}
(\ln \psi_\lambda)' &= i \lambda/\Hbar \\
\implies \\
\ln \psi_\lambda &= \frac{i \lambda x }{\Hbar} + \ln A \\
\implies \\
\psi_\lambda &= A \exp\left( \frac{i \lambda x }{\Hbar} \right) \\
\end{aligned}
\end{equation}

Considering two such eigenfunctions (normalization omitted) with an orthogonality requirement in the \([0,L]\) interval we have for \(\lambda \ne \mu\)

\begin{equation}\label{eqn:bohmCh10:1640}
\begin{aligned}
\int_0^L (\psi_\mu)^\conj \psi_\lambda dx 
&=
\int_0^L 
\exp\left( \frac{-i \mu x }{\Hbar} \right) \exp\left( \frac{i \lambda x }{\Hbar} \right) \\
&=
\int_0^L \exp\left( \frac{i (\lambda -\mu) x }{\Hbar} \right) \\
&=
\left. \inv{ i (\lambda -\mu)/\Hbar} \exp\left( \frac{i (\lambda -\mu) x }{\Hbar} \right) \right\vert_{0}^L
 \\
&=
\inv{ i (\lambda -\mu)/\Hbar} \left( \exp\left( \frac{i (\lambda -\mu) L }{\Hbar} \right) - 1 \right)
 \\
\end{aligned}
\end{equation}

So, for orthogonality we need \(\lambda L/\Hbar = 2\pi n_\lambda\), or more simply

\begin{equation}\label{eqn:bohmCh10:1660}
\begin{aligned}
\psi_n = \inv{\sqrt{L}} \exp\left( \frac{ 2 \pi i n x }{L } \right) \\
\end{aligned}
\end{equation}

It is interesting to see that the one dimensional particle in a box can be reduced to a first order differential equation or 
eigenvalue problem, instead of looking for solutions to the energy operator equation \((p^2/2m) \psi = E \psi\).

\subsection{P14. Fourier series representation of delta function}

TODO.

\subsection{P15}

This one has a prereq on the ch3 problems, which I did not do.  Revisit.

%\bibliographystyle{plainnat}
%\bibliography{myrefs}

%\end{document}

\documentclass{article}

\usepackage{amsmath}
\usepackage{mathpazo}

%
% shorthand for bold symbols, convenient for vectors and matrices
%
\newcommand{\Ba}[0]{\mathbf{a}}
\newcommand{\Bb}[0]{\mathbf{b}}
\newcommand{\Bc}[0]{\mathbf{c}}
\newcommand{\Bd}[0]{\mathbf{d}}
\newcommand{\Be}[0]{\mathbf{e}}
\newcommand{\Bf}[0]{\mathbf{f}}
\newcommand{\Bg}[0]{\mathbf{g}}
\newcommand{\Bh}[0]{\mathbf{h}}
\newcommand{\Bi}[0]{\mathbf{i}}
\newcommand{\Bj}[0]{\mathbf{j}}
\newcommand{\Bk}[0]{\mathbf{k}}
\newcommand{\Bl}[0]{\mathbf{l}}
\newcommand{\Bm}[0]{\mathbf{m}}
\newcommand{\Bn}[0]{\mathbf{n}}
\newcommand{\Bo}[0]{\mathbf{o}}
\newcommand{\Bp}[0]{\mathbf{p}}
\newcommand{\Bq}[0]{\mathbf{q}}
\newcommand{\Br}[0]{\mathbf{r}}
\newcommand{\Bs}[0]{\mathbf{s}}
\newcommand{\Bt}[0]{\mathbf{t}}
\newcommand{\Bu}[0]{\mathbf{u}}
\newcommand{\Bv}[0]{\mathbf{v}}
\newcommand{\Bw}[0]{\mathbf{w}}
\newcommand{\Bx}[0]{\mathbf{x}}
\newcommand{\By}[0]{\mathbf{y}}
\newcommand{\Bz}[0]{\mathbf{z}}
\newcommand{\BA}[0]{\mathbf{A}}
\newcommand{\BB}[0]{\mathbf{B}}
\newcommand{\BC}[0]{\mathbf{C}}
\newcommand{\BD}[0]{\mathbf{D}}
\newcommand{\BE}[0]{\mathbf{E}}
\newcommand{\BF}[0]{\mathbf{F}}
\newcommand{\BG}[0]{\mathbf{G}}
\newcommand{\BH}[0]{\mathbf{H}}
\newcommand{\BI}[0]{\mathbf{I}}
\newcommand{\BJ}[0]{\mathbf{J}}
\newcommand{\BK}[0]{\mathbf{K}}
\newcommand{\BL}[0]{\mathbf{L}}
\newcommand{\BM}[0]{\mathbf{M}}
\newcommand{\BN}[0]{\mathbf{N}}
\newcommand{\BO}[0]{\mathbf{O}}
\newcommand{\BP}[0]{\mathbf{P}}
\newcommand{\BQ}[0]{\mathbf{Q}}
\newcommand{\BR}[0]{\mathbf{R}}
\newcommand{\BS}[0]{\mathbf{S}}
\newcommand{\BT}[0]{\mathbf{T}}
\newcommand{\BU}[0]{\mathbf{U}}
\newcommand{\BV}[0]{\mathbf{V}}
\newcommand{\BW}[0]{\mathbf{W}}
\newcommand{\BX}[0]{\mathbf{X}}
\newcommand{\BY}[0]{\mathbf{Y}}
\newcommand{\BZ}[0]{\mathbf{Z}}

\newcommand{\Bzero}[0]{\mathbf{0}}
\newcommand{\Btheta}[0]{\boldsymbol{\theta}}
\newcommand{\Btau}[0]{\boldsymbol{\tau}}
\newcommand{\Bomega}[0]{\boldsymbol{\omega}}

%
% shorthand for unit vectors
%
\newcommand{\acap}[0]{\hat{\Ba}}
\newcommand{\bcap}[0]{\hat{\Bb}}
\newcommand{\ccap}[0]{\hat{\Bc}}
\newcommand{\dcap}[0]{\hat{\Bd}}
\newcommand{\ecap}[0]{\hat{\Be}}
\newcommand{\fcap}[0]{\hat{\Bf}}
\newcommand{\gcap}[0]{\hat{\Bg}}
\newcommand{\hcap}[0]{\hat{\Bh}}
\newcommand{\icap}[0]{\hat{\Bi}}
\newcommand{\jcap}[0]{\hat{\Bj}}
\newcommand{\kcap}[0]{\hat{\Bk}}
\newcommand{\lcap}[0]{\hat{\Bl}}
\newcommand{\mcap}[0]{\hat{\Bm}}
\newcommand{\ncap}[0]{\hat{\Bn}}
\newcommand{\ocap}[0]{\hat{\Bo}}
\newcommand{\pcap}[0]{\hat{\Bp}}
\newcommand{\qcap}[0]{\hat{\Bq}}
\newcommand{\rcap}[0]{\hat{\Br}}
\newcommand{\scap}[0]{\hat{\Bs}}
\newcommand{\tcap}[0]{\hat{\Bt}}
\newcommand{\ucap}[0]{\hat{\Bu}}
\newcommand{\vcap}[0]{\hat{\Bv}}
\newcommand{\wcap}[0]{\hat{\Bw}}
\newcommand{\xcap}[0]{\hat{\Bx}}
\newcommand{\ycap}[0]{\hat{\By}}
\newcommand{\zcap}[0]{\hat{\Bz}}
\newcommand{\thetacap}[0]{\hat{\Btheta}}

%
% to write R^n and C^n in a distinguishable fashion.  Perhaps change this
% to the double lined characters upon figuring out how to do so.
%
\newcommand{\C}[1]{$\mathbb{C}^{#1}$}
\newcommand{\R}[1]{$\mathbb{R}^{#1}$}

%
% various generally useful helpers
%

% derivative of #1 wrt. #2:
\newcommand{\D}[2] {\frac {d#2} {d#1}}

\newcommand{\inv}[1]{\frac{1}{#1}}
\newcommand{\cross}[0]{\times}

\newcommand{\abs}[1]{\lvert{#1}\rvert}
\newcommand{\norm}[1]{\lVert{#1}\rVert}
\newcommand{\innerprod}[2]{\langle{#1}, {#2}\rangle}
\newcommand{\dotprod}[2]{{#1} \cdot {#2}}
\newcommand{\bdotprod}[2]{\left({#1} \cdot {#2}\right)}
\newcommand{\crossprod}[2]{{#1} \cross {#2}}
\newcommand{\tripleprod}[3]{\dotprod{\left(\crossprod{#1}{#2}\right)}{#3}}

\DeclareMathOperator{\Proj}{Proj}
\DeclareMathOperator{\Span}{span}
\DeclareMathOperator{\Sgn}{sgn}
\DeclareMathOperator{\Area}{Area}
\DeclareMathOperator{\Volume}{Volume}

%
% A few miscellaneous things specific to this document
%
\newcommand{\crossop}[1]{\crossprod{#1}{}}

% R2 vector.
\newcommand{\VectorTwo}[2]{
\begin{bmatrix}
 {#1} \\
 {#2}
\end{bmatrix}
}

\newcommand{\VectorN}[1]{
\begin{bmatrix}
{#1}_1 \\
{#1}_2 \\
\vdots \\
{#1}_N \\
\end{bmatrix}
}

\newcommand{\DETuvij}[4]{
\begin{vmatrix}
 {#1}_{#3} & {#1}_{#4} \\
 {#2}_{#3} & {#2}_{#4}
\end{vmatrix}
}

\newcommand{\DETuvwijk}[6]{
\begin{vmatrix}
 {#1}_{#4} & {#1}_{#5} & {#1}_{#6} \\
 {#2}_{#4} & {#2}_{#5} & {#2}_{#6} \\
 {#3}_{#4} & {#3}_{#5} & {#3}_{#6}
\end{vmatrix}
}

\newcommand{\DETuvwxijkl}[8]{
\begin{vmatrix}
 {#1}_{#5} & {#1}_{#6} & {#1}_{#7} & {#1}_{#8} \\
 {#2}_{#5} & {#2}_{#6} & {#2}_{#7} & {#2}_{#8} \\
 {#3}_{#5} & {#3}_{#6} & {#3}_{#7} & {#3}_{#8} \\
 {#4}_{#5} & {#4}_{#6} & {#4}_{#7} & {#4}_{#8} \\
\end{vmatrix}
}

%\newcommand{\DETuvwxyijklm}[10]{
%\begin{vmatrix}
% {#1}_{#6} & {#1}_{#7} & {#1}_{#8} & {#1}_{#9} & {#1}_{#10} \\
% {#2}_{#6} & {#2}_{#7} & {#2}_{#8} & {#2}_{#9} & {#2}_{#10} \\
% {#3}_{#6} & {#3}_{#7} & {#3}_{#8} & {#3}_{#9} & {#3}_{#10} \\
% {#4}_{#6} & {#4}_{#7} & {#4}_{#8} & {#4}_{#9} & {#4}_{#10} \\
% {#5}_{#6} & {#5}_{#7} & {#5}_{#8} & {#5}_{#9} & {#5}_{#10}
%\end{vmatrix}
%}

% R3 vector.
\newcommand{\VectorThree}[3]{
\begin{bmatrix}
 {#1} \\
 {#2} \\
 {#3}
\end{bmatrix}
}


%<misc>
%
\newcommand{\Abs}[1]{{\left\lvert{#1}\right\rvert}}
\newcommand{\spacegrad}[0]{\boldsymbol{\nabla}}
\newcommand{\grad}[0]{\nabla}
\newcommand{\LL}[0]{\mathcal{L}}

% == \partial_{#1} {#2}
\newcommand{\PD}[2]{\frac{\partial {#2}}{\partial {#1}}}
% inline variant
\newcommand{\PDi}[2]{{\partial {#2}}/{\partial {#1}}}

\newcommand{\PDD}[3]{\frac{\partial^2 {#3}}{\partial {#1}\partial {#2}}}
%\newcommand{\PDd}[2]{\frac{\partial^2 {#2}}{{\partial{#1}}^2}}
\newcommand{\PDsq}[2]{\frac{\partial^2 {#2}}{(\partial {#1})^2}}

\newcommand{\Partial}[2]{\frac{\partial {#1}}{\partial {#2}}}
\DeclareMathOperator{\RejName}{Rej}
\newcommand{\Rej}[2]{\RejName_{#1}\left( {#2} \right)}
\newcommand{\Rm}[1]{\mathbb{R}^{#1}}
\newcommand{\Cm}[1]{\mathbb{C}^{#1}}
\newcommand{\conj}[0]{{*}}

%</misc>

% <grade selection>
%
\newcommand{\gpgrade}[2] {{\left\langle{{#1}}\right\rangle}_{#2}}

\newcommand{\gpgradezero}[1] {\gpgrade{#1}{}}
%\newcommand{\gpscalargrade}[1] {{\left\langle{{#1}}\right\rangle}}
%\newcommand{\gpgradezero}[1] {\gpgrade{#1}{0}}

%\newcommand{\gpgradeone}[1] {{\left\langle{{#1}}\right\rangle}_{1}}
\newcommand{\gpgradeone}[1] {\gpgrade{#1}{1}}

\newcommand{\gpgradetwo}[1] {\gpgrade{#1}{2}}
\newcommand{\gpgradethree}[1] {\gpgrade{#1}{3}}
\newcommand{\gpgradefour}[1] {\gpgrade{#1}{4}}
%
% </grade selection>



\newcommand{\adot}[0]{{\dot{a}}}
\newcommand{\bdot}[0]{{\dot{b}}}
% taken for centered dot:
%\newcommand{\cdot}[0]{{\dot{c}}}
%\newcommand{\ddot}[0]{{\dot{d}}}
\newcommand{\edot}[0]{{\dot{e}}}
\newcommand{\fdot}[0]{{\dot{f}}}
\newcommand{\gdot}[0]{{\dot{g}}}
\newcommand{\hdot}[0]{{\dot{h}}}
\newcommand{\idot}[0]{{\dot{i}}}
\newcommand{\jdot}[0]{{\dot{j}}}
\newcommand{\kdot}[0]{{\dot{k}}}
\newcommand{\ldot}[0]{{\dot{l}}}
\newcommand{\mdot}[0]{{\dot{m}}}
\newcommand{\ndot}[0]{{\dot{n}}}
%\newcommand{\odot}[0]{{\dot{o}}}
\newcommand{\pdot}[0]{{\dot{p}}}
\newcommand{\qdot}[0]{{\dot{q}}}
\newcommand{\rdot}[0]{{\dot{r}}}
\newcommand{\sdot}[0]{{\dot{s}}}
\newcommand{\tdot}[0]{{\dot{t}}}
\newcommand{\udot}[0]{{\dot{u}}}
\newcommand{\vdot}[0]{{\dot{v}}}
\newcommand{\wdot}[0]{{\dot{w}}}
\newcommand{\xdot}[0]{{\dot{x}}}
\newcommand{\ydot}[0]{{\dot{y}}}
\newcommand{\zdot}[0]{{\dot{z}}}
\newcommand{\addot}[0]{{\ddot{a}}}
\newcommand{\bddot}[0]{{\ddot{b}}}
\newcommand{\cddot}[0]{{\ddot{c}}}
%\newcommand{\dddot}[0]{{\ddot{d}}}
\newcommand{\eddot}[0]{{\ddot{e}}}
\newcommand{\fddot}[0]{{\ddot{f}}}
\newcommand{\gddot}[0]{{\ddot{g}}}
\newcommand{\hddot}[0]{{\ddot{h}}}
\newcommand{\iddot}[0]{{\ddot{i}}}
\newcommand{\jddot}[0]{{\ddot{j}}}
\newcommand{\kddot}[0]{{\ddot{k}}}
\newcommand{\lddot}[0]{{\ddot{l}}}
\newcommand{\mddot}[0]{{\ddot{m}}}
\newcommand{\nddot}[0]{{\ddot{n}}}
\newcommand{\oddot}[0]{{\ddot{o}}}
\newcommand{\pddot}[0]{{\ddot{p}}}
\newcommand{\qddot}[0]{{\ddot{q}}}
\newcommand{\rddot}[0]{{\ddot{r}}}
\newcommand{\sddot}[0]{{\ddot{s}}}
\newcommand{\tddot}[0]{{\ddot{t}}}
\newcommand{\uddot}[0]{{\ddot{u}}}
\newcommand{\vddot}[0]{{\ddot{v}}}
\newcommand{\wddot}[0]{{\ddot{w}}}
\newcommand{\xddot}[0]{{\ddot{x}}}
\newcommand{\yddot}[0]{{\ddot{y}}}
\newcommand{\zddot}[0]{{\ddot{z}}}

%<bold and dot greek symbols>
%

\newcommand{\Deltadot}[0]{{\dot{\Delta}}}
\newcommand{\Gammadot}[0]{{\dot{\Gamma}}}
\newcommand{\Lambdadot}[0]{{\dot{\Lambda}}}
\newcommand{\Omegadot}[0]{{\dot{\Omega}}}
\newcommand{\Phidot}[0]{{\dot{\Phi}}}
\newcommand{\Pidot}[0]{{\dot{\Pi}}}
\newcommand{\Psidot}[0]{{\dot{\Psi}}}
\newcommand{\Sigmadot}[0]{{\dot{\Sigma}}}
\newcommand{\Thetadot}[0]{{\dot{\Theta}}}
\newcommand{\Upsilondot}[0]{{\dot{\Upsilon}}}
\newcommand{\Xidot}[0]{{\dot{\Xi}}}
\newcommand{\alphadot}[0]{{\dot{\alpha}}}
\newcommand{\betadot}[0]{{\dot{\beta}}}
\newcommand{\chidot}[0]{{\dot{\chi}}}
\newcommand{\deltadot}[0]{{\dot{\delta}}}
\newcommand{\epsilondot}[0]{{\dot{\epsilon}}}
\newcommand{\etadot}[0]{{\dot{\eta}}}
\newcommand{\gammadot}[0]{{\dot{\gamma}}}
\newcommand{\kappadot}[0]{{\dot{\kappa}}}
\newcommand{\lambdadot}[0]{{\dot{\lambda}}}
\newcommand{\mudot}[0]{{\dot{\mu}}}
\newcommand{\nudot}[0]{{\dot{\nu}}}
\newcommand{\omegadot}[0]{{\dot{\omega}}}
\newcommand{\phidot}[0]{{\dot{\phi}}}
\newcommand{\pidot}[0]{{\dot{\pi}}}
\newcommand{\psidot}[0]{{\dot{\psi}}}
\newcommand{\rhodot}[0]{{\dot{\rho}}}
\newcommand{\sigmadot}[0]{{\dot{\sigma}}}
\newcommand{\taudot}[0]{{\dot{\tau}}}
\newcommand{\thetadot}[0]{{\dot{\theta}}}
\newcommand{\upsilondot}[0]{{\dot{\upsilon}}}
\newcommand{\varepsilondot}[0]{{\dot{\varepsilon}}}
\newcommand{\varphidot}[0]{{\dot{\varphi}}}
\newcommand{\varpidot}[0]{{\dot{\varpi}}}
\newcommand{\varrhodot}[0]{{\dot{\varrho}}}
\newcommand{\varsigmadot}[0]{{\dot{\varsigma}}}
\newcommand{\varthetadot}[0]{{\dot{\vartheta}}}
\newcommand{\xidot}[0]{{\dot{\xi}}}
\newcommand{\zetadot}[0]{{\dot{\zeta}}}

\newcommand{\Deltaddot}[0]{{\ddot{\Delta}}}
\newcommand{\Gammaddot}[0]{{\ddot{\Gamma}}}
\newcommand{\Lambdaddot}[0]{{\ddot{\Lambda}}}
\newcommand{\Omegaddot}[0]{{\ddot{\Omega}}}
\newcommand{\Phiddot}[0]{{\ddot{\Phi}}}
\newcommand{\Piddot}[0]{{\ddot{\Pi}}}
\newcommand{\Psiddot}[0]{{\ddot{\Psi}}}
\newcommand{\Sigmaddot}[0]{{\ddot{\Sigma}}}
\newcommand{\Thetaddot}[0]{{\ddot{\Theta}}}
\newcommand{\Upsilonddot}[0]{{\ddot{\Upsilon}}}
\newcommand{\Xiddot}[0]{{\ddot{\Xi}}}
\newcommand{\alphaddot}[0]{{\ddot{\alpha}}}
\newcommand{\betaddot}[0]{{\ddot{\beta}}}
\newcommand{\chiddot}[0]{{\ddot{\chi}}}
\newcommand{\deltaddot}[0]{{\ddot{\delta}}}
\newcommand{\epsilonddot}[0]{{\ddot{\epsilon}}}
\newcommand{\etaddot}[0]{{\ddot{\eta}}}
\newcommand{\gammaddot}[0]{{\ddot{\gamma}}}
\newcommand{\kappaddot}[0]{{\ddot{\kappa}}}
\newcommand{\lambdaddot}[0]{{\ddot{\lambda}}}
\newcommand{\muddot}[0]{{\ddot{\mu}}}
\newcommand{\nuddot}[0]{{\ddot{\nu}}}
\newcommand{\omegaddot}[0]{{\ddot{\omega}}}
\newcommand{\phiddot}[0]{{\ddot{\phi}}}
\newcommand{\piddot}[0]{{\ddot{\pi}}}
\newcommand{\psiddot}[0]{{\ddot{\psi}}}
\newcommand{\rhoddot}[0]{{\ddot{\rho}}}
\newcommand{\sigmaddot}[0]{{\ddot{\sigma}}}
\newcommand{\tauddot}[0]{{\ddot{\tau}}}
\newcommand{\thetaddot}[0]{{\ddot{\theta}}}
\newcommand{\upsilonddot}[0]{{\ddot{\upsilon}}}
\newcommand{\varepsilonddot}[0]{{\ddot{\varepsilon}}}
\newcommand{\varphiddot}[0]{{\ddot{\varphi}}}
\newcommand{\varpiddot}[0]{{\ddot{\varpi}}}
\newcommand{\varrhoddot}[0]{{\ddot{\varrho}}}
\newcommand{\varsigmaddot}[0]{{\ddot{\varsigma}}}
\newcommand{\varthetaddot}[0]{{\ddot{\vartheta}}}
\newcommand{\xiddot}[0]{{\ddot{\xi}}}
\newcommand{\zetaddot}[0]{{\ddot{\zeta}}}

\newcommand{\BDelta}[0]{\boldsymbol{\Delta}}
\newcommand{\BGamma}[0]{\boldsymbol{\Gamma}}
\newcommand{\BLambda}[0]{\boldsymbol{\Lambda}}
\newcommand{\BOmega}[0]{\boldsymbol{\Omega}}
\newcommand{\BPhi}[0]{\boldsymbol{\Phi}}
\newcommand{\BPi}[0]{\boldsymbol{\Pi}}
\newcommand{\BPsi}[0]{\boldsymbol{\Psi}}
\newcommand{\BSigma}[0]{\boldsymbol{\Sigma}}
\newcommand{\BTheta}[0]{\boldsymbol{\Theta}}
\newcommand{\BUpsilon}[0]{\boldsymbol{\Upsilon}}
\newcommand{\BXi}[0]{\boldsymbol{\Xi}}
\newcommand{\Balpha}[0]{\boldsymbol{\alpha}}
\newcommand{\Bbeta}[0]{\boldsymbol{\beta}}
\newcommand{\Bchi}[0]{\boldsymbol{\chi}}
\newcommand{\Bdelta}[0]{\boldsymbol{\delta}}
\newcommand{\Bepsilon}[0]{\boldsymbol{\epsilon}}
\newcommand{\Beta}[0]{\boldsymbol{\eta}}
\newcommand{\Bgamma}[0]{\boldsymbol{\gamma}}
\newcommand{\Bkappa}[0]{\boldsymbol{\kappa}}
\newcommand{\Blambda}[0]{\boldsymbol{\lambda}}
\newcommand{\Bmu}[0]{\boldsymbol{\mu}}
\newcommand{\Bnu}[0]{\boldsymbol{\nu}}
%\newcommand{\Bomega}[0]{\boldsymbol{\omega}}
\newcommand{\Bphi}[0]{\boldsymbol{\phi}}
\newcommand{\Bpi}[0]{\boldsymbol{\pi}}
\newcommand{\Bpsi}[0]{\boldsymbol{\psi}}
\newcommand{\Brho}[0]{\boldsymbol{\rho}}
\newcommand{\Bsigma}[0]{\boldsymbol{\sigma}}
%\newcommand{\Btau}[0]{\boldsymbol{\tau}}
%\newcommand{\Btheta}[0]{\boldsymbol{\theta}}
\newcommand{\Bupsilon}[0]{\boldsymbol{\upsilon}}
\newcommand{\Bvarepsilon}[0]{\boldsymbol{\varepsilon}}
\newcommand{\Bvarphi}[0]{\boldsymbol{\varphi}}
\newcommand{\Bvarpi}[0]{\boldsymbol{\varpi}}
\newcommand{\Bvarrho}[0]{\boldsymbol{\varrho}}
\newcommand{\Bvarsigma}[0]{\boldsymbol{\varsigma}}
\newcommand{\Bvartheta}[0]{\boldsymbol{\vartheta}}
\newcommand{\Bxi}[0]{\boldsymbol{\xi}}
\newcommand{\Bzeta}[0]{\boldsymbol{\zeta}}
%
%</bold and dot greek symbols>
%<infrequent>
%
%\newcommand{\AreaOp}[1]{\AName_{#1}}
%\newcommand{\Babs}[0]{\abs{\BB}}
%\newcommand{\Bcap}[0]{\hat{\BB}}
%\newcommand{\BrPrimeRej}[0]{\rcap(\rcap \wedge \Br')}
%\newcommand{\CA}[0]{\mathcal{A}}
%\newcommand{\Cos}[1]{\cos{\left({#1}\right)}}
%\newcommand{\Det}[1] {\abs{#1}}
%\newcommand{\Dsq}[2] {\frac {\partial^2 {#1}} {\partial {#2}^2}}
%\newcommand{\Exp}[1]{\exp{\left({#1}\right)}}
%\newcommand{\Norm}[1]{\left\lVert{#1}\right\rVert}
%\newcommand{\Sin}[1]{\sin{\left({#1}\right)}}
%\newcommand{\T}[0]{\text{T}}
%\newcommand{\VolumeOp}[1]{\VName_{#1}}
%\newcommand{\agrad}[0]{\Ba \cdot \nabla}
%\newcommand{\alphacap}[0]{\hat{\boldsymbol{\alpha}}}
%\newcommand{\Fcap}[0]{\hat{\BF}}
%\newcommand{\bithree}[0]{{\Bi}_3}
%\newcommand{\bxa}[0]{\Bx\Ba}
%\newcommand{\coordvec}[2]{
%\newcommand{\costheta}[0]{\acap \cdot \xcap}
%\newcommand{\ddt}[1]{\ddot{#1}}
%\newcommand{\ddu}[1] {\frac {d{#1}} {du}}
%\newcommand{\dsqxj}[2] {\frac {\partial^2 {#1}} {\partial {x_{#2}}^2}}
%\newcommand{\dtheta}[1]{\frac{d {#1}}{d \theta}}
%\newcommand{\dt}[1]{\dot{#1}}
%\newcommand{\dt}[1]{\frac{d {#1}}{dt}}
%\newcommand{\dxj}[2] {\frac {\partial {#1}} {\partial {x_{#2}}}}
%\newcommand{\halfPhi}[0]{\frac{\phi}{2}}
%\newcommand{\half}[0]{\inv{2}}
%\newcommand{\inv}[1]{\frac{1}{#1}}
%\newcommand{\laplacian}[0]{\nabla^2}
%\newcommand{\matrixoftx}[3]{
%\newcommand{\nrrp}[0]{\norm{\rcap \wedge \Br'}}
%\newcommand{\oiint}{\bigcirc \hspace{-1.4em} \int \hspace{-.8em} \int}
%\newcommand{\transpose}[1]{{#1}^{\text{T}}}
%\newcommand{\transpose}[1]{{{#1}^{\TextTranspose}}}
%\newcommand{\transpose}[1]{{{#1}^{\text{T}}}}
%\newcommand{\barA}[0]{\bar{A}}
%\newcommand{\qbar}[0]{\bar{q}}
%\newcommand{\qdotbar}[0]{\dot{\bar{q}}}
%
%</infrequent>





\DeclareMathOperator{\sgn}{sgn}
\newcommand{\PDSq}[2]{\frac{\partial^2 {#2}}{\partial {#1}^2}}
\newcommand{\PDN}[3]{\frac{\partial^{#3} {#2}}{\partial {#1}^{#3}}}
\DeclareMathOperator{\sinc}{sinc}
\DeclareMathOperator{\PV}{PV}
\newcommand{\FF}[0]{\mathcal{F}}
\newcommand{\Sw}[0]{\mathcal{S}}
\newcommand{\IIinf}[0]{ \int_{-\infty}^\infty }
\newcommand{\FM}[0]{\inv{\sqrt{2\pi\hbar}}}
\newcommand{\expectation}[1]{\langle{#1}\rangle}

%\usepackage{listings}
%\usepackage{txfonts} % for ointctr... (also appears to make "prettier" \int and \sum's)
% makes \grad look funny though (almost like spacegrad, but narrower)
\usepackage[bookmarks=true]{hyperref}

\usepackage{color,cite,graphicx}
   % use colour in the document, put your citations as [1-4]
   % rather than [1,2,3,4] (it looks nicer, and the extended LaTeX2e
   % graphics package. 
\usepackage{latexsym,amssymb,epsf} % don't remember if these are
   % needed, but their inclusion can't do any damage


\title{ QM notes and problems for Bohm, chapter 11. }
\author{Peeter Joot \quad peeter.joot@gmail.com }
\date{ May 8, 2009.  Last Revision: $Date: 2009/05/09 04:33:54 $ }

\begin{document}

\maketitle{}
\tableofcontents
\section{ Bohm Chapter 11 problems. }

Problems and additional details from reading of \cite{bohm1989qt}, chapter 11.

\subsection{ Problem 1.  Probability currents for step potential. }

This problem and the associated text has a step potential $V$ for $x>0$.  The 
idea is that we have a left to right stream of particles with an associated
wave function, with reflected and transmitted coefficients.

Solutions to the stationary equation are sought in each of the intervals

\begin{align*}
-\frac{\hbar^2}{2m}\psi'' + (V-E)\psi = 0
\end{align*}

That is
\begin{align*}
\psi'' = - \frac{2m}{\hbar^2} (E-V)\psi 
\end{align*}

With an assumption of exponential solutions on the left of the barrier
(ie: no decay and associated hyperbolic solutions in the $V=0$ interval),
we must have $E>0$.

So, the solution can be written as the sum

\begin{align*}
\psi_1 = \sum A_{\pm} \exp\left( \pm i \sqrt{2mE} x / \hbar \right)
\end{align*}

Similarily for $x>0$, non-hyperbolic solutions are

\begin{align*}
\psi_2 = \sum B_{\pm} \exp\left( \pm i \sqrt{2m(E-V)} x / \hbar \right)
\end{align*}

Mathematically, there isn't anything that prevents picking $E-V <0$ solutions

\begin{align*}
\psi_2 = \sum B_{\pm}' \exp\left( \pm \sqrt{-2m(E-V)} x / \hbar \right)
\end{align*}

TODO: Come back to this and think it through.  Where will follow through using
wave function continuity and derivative continuity with these solutions
in the $x>0$ interval?

Continuity at $x=0$ ($\psi_1(0) = \psi_2(0)$) requires

\begin{align*}
A_{+} + A_{-} = B_{+} + B_{-}
\end{align*}

Whereas derivative continuity, with $p_1 = \sqrt{2mE}$, and $p_2 = \sqrt{2m (E-V)}$ requires

\begin{align*}
\frac{i p_1}{\hbar} (A_{+} - A_{-}) = \frac{i p_2}{\hbar} (B_{+} - B_{-}) 
\end{align*}

This is

\begin{align*}
\begin{bmatrix}
1 & 1 \\
p_1 & -p_1
\end{bmatrix}
\begin{bmatrix}
A_{+} \\
A_{-} 
\end{bmatrix}
=
\begin{bmatrix}
1 & 1 \\
p_2 & -p_2
\end{bmatrix}
\begin{bmatrix}
B_{+} \\
B_{-} 
\end{bmatrix}
\end{align*}

Matrix inversion gives us the $B$ coefficients in terms of $A$

\begin{align*}
\begin{bmatrix}
B_{+} \\
B_{-} 
\end{bmatrix}
&=
\inv{-2p_2}
\begin{bmatrix}
-p_2 & -1 \\
-p_2 & 1
\end{bmatrix}
\begin{bmatrix}
1 & 1 \\
p_1 & -p_1
\end{bmatrix}
\begin{bmatrix}
A_{+} \\
A_{-} 
\end{bmatrix} \\
&=
\inv{2p_2}
\begin{bmatrix}
p_2 & 1 \\
p_2 & -1
\end{bmatrix}
\begin{bmatrix}
1 & 1 \\
p_1 & -p_1
\end{bmatrix}
\begin{bmatrix}
A_{+} \\
A_{-} 
\end{bmatrix} \\
&=
\inv{2p_2}
\begin{bmatrix}
(p_1 + p_2) & (p_2 - p_1) \\
(p_2 - p_1) & (p_1 + p_2)
\end{bmatrix}
\begin{bmatrix}
A_{+} \\
A_{-} 
\end{bmatrix} \\
\end{align*}

Or equivalently

\begin{align*}
\begin{bmatrix}
A_{+} \\
A_{-} 
\end{bmatrix}
&=
\inv{2p_1}
\begin{bmatrix}
(p_1 + p_2) & (p_1 - p_2) \\
(p_1 - p_2) & (p_1 + p_2)
\end{bmatrix}
\begin{bmatrix}
B_{+} \\
B_{-} 
\end{bmatrix} \\
\end{align*}

Using this last, the 
assumption of no further barriers in the $x>0$ interval (ie: no reflection at 
$x = \infty$), allows the physical situation to dictate $B_{-} = 0$.  Then we
have

\begin{align*}
\begin{bmatrix}
A_{+} \\
A_{-} 
\end{bmatrix}
&=
\frac{B_{+} }{2p_1}
\begin{bmatrix}
p_1 + p_2 \\
p_1 - p_2 
\end{bmatrix}
\end{align*}

The first is 
\begin{align*}
A_{+} 
&=
\frac{B_{+} }{2p_1} (p_1 + p_2)
\end{align*}

Or
\begin{align*}
B_{+} 
&=
\frac{2 p_1 A_{+} }{p_1 + p_2}
\end{align*}

and the second is
\begin{align*}
A_{-} 
&=
\frac{B_{+} }{2p_1} (p_1 - p_2) \\
&=
\frac{2 p_1 A_{+} }{p_1 + p_2}
\frac{1}{2p_1} (p_1 - p_2) \\
&=
A_{+} \left(
\frac{ p_1 - p_2 }{p_1 + p_2} \right)
\end{align*}

This reduces the free parameters in the wave functions to the single 
amplitude

\begin{align*}
\psi_1 &= 
A_{+} e^{ i p_1 x/\hbar }
+A_{+} \left(
\frac{ p_1 - p_2 }{p_1 + p_2} \right)
e^{ -i p_1 x/\hbar } \\
\psi_2 &= A_{+} \frac{2 p_1 }{p_1 + p_2} e^{ i p_2 x/\hbar }
\end{align*}

Inspection provides a check that these do in fact satisfy the desired
continuity requirements.


\bibliographystyle{plainnat}
\bibliography{myrefs}

\end{document}

%
% Copyright � 2012 Peeter Joot.  All Rights Reserved.
% Licenced as described in the file LICENSE under the root directory of this GIT repository.
%

% 
% 
%\documentclass{article}

%\usepackage{amsmath}
\usepackage{mathpazo}

%
% shorthand for bold symbols, convenient for vectors and matrices
%
\newcommand{\Ba}[0]{\mathbf{a}}
\newcommand{\Bb}[0]{\mathbf{b}}
\newcommand{\Bc}[0]{\mathbf{c}}
\newcommand{\Bd}[0]{\mathbf{d}}
\newcommand{\Be}[0]{\mathbf{e}}
\newcommand{\Bf}[0]{\mathbf{f}}
\newcommand{\Bg}[0]{\mathbf{g}}
\newcommand{\Bh}[0]{\mathbf{h}}
\newcommand{\Bi}[0]{\mathbf{i}}
\newcommand{\Bj}[0]{\mathbf{j}}
\newcommand{\Bk}[0]{\mathbf{k}}
\newcommand{\Bl}[0]{\mathbf{l}}
\newcommand{\Bm}[0]{\mathbf{m}}
\newcommand{\Bn}[0]{\mathbf{n}}
\newcommand{\Bo}[0]{\mathbf{o}}
\newcommand{\Bp}[0]{\mathbf{p}}
\newcommand{\Bq}[0]{\mathbf{q}}
\newcommand{\Br}[0]{\mathbf{r}}
\newcommand{\Bs}[0]{\mathbf{s}}
\newcommand{\Bt}[0]{\mathbf{t}}
\newcommand{\Bu}[0]{\mathbf{u}}
\newcommand{\Bv}[0]{\mathbf{v}}
\newcommand{\Bw}[0]{\mathbf{w}}
\newcommand{\Bx}[0]{\mathbf{x}}
\newcommand{\By}[0]{\mathbf{y}}
\newcommand{\Bz}[0]{\mathbf{z}}
\newcommand{\BA}[0]{\mathbf{A}}
\newcommand{\BB}[0]{\mathbf{B}}
\newcommand{\BC}[0]{\mathbf{C}}
\newcommand{\BD}[0]{\mathbf{D}}
\newcommand{\BE}[0]{\mathbf{E}}
\newcommand{\BF}[0]{\mathbf{F}}
\newcommand{\BG}[0]{\mathbf{G}}
\newcommand{\BH}[0]{\mathbf{H}}
\newcommand{\BI}[0]{\mathbf{I}}
\newcommand{\BJ}[0]{\mathbf{J}}
\newcommand{\BK}[0]{\mathbf{K}}
\newcommand{\BL}[0]{\mathbf{L}}
\newcommand{\BM}[0]{\mathbf{M}}
\newcommand{\BN}[0]{\mathbf{N}}
\newcommand{\BO}[0]{\mathbf{O}}
\newcommand{\BP}[0]{\mathbf{P}}
\newcommand{\BQ}[0]{\mathbf{Q}}
\newcommand{\BR}[0]{\mathbf{R}}
\newcommand{\BS}[0]{\mathbf{S}}
\newcommand{\BT}[0]{\mathbf{T}}
\newcommand{\BU}[0]{\mathbf{U}}
\newcommand{\BV}[0]{\mathbf{V}}
\newcommand{\BW}[0]{\mathbf{W}}
\newcommand{\BX}[0]{\mathbf{X}}
\newcommand{\BY}[0]{\mathbf{Y}}
\newcommand{\BZ}[0]{\mathbf{Z}}

\newcommand{\Bzero}[0]{\mathbf{0}}
\newcommand{\Btheta}[0]{\boldsymbol{\theta}}
\newcommand{\Btau}[0]{\boldsymbol{\tau}}
\newcommand{\Bomega}[0]{\boldsymbol{\omega}}

%
% shorthand for unit vectors
%
\newcommand{\acap}[0]{\hat{\Ba}}
\newcommand{\bcap}[0]{\hat{\Bb}}
\newcommand{\ccap}[0]{\hat{\Bc}}
\newcommand{\dcap}[0]{\hat{\Bd}}
\newcommand{\ecap}[0]{\hat{\Be}}
\newcommand{\fcap}[0]{\hat{\Bf}}
\newcommand{\gcap}[0]{\hat{\Bg}}
\newcommand{\hcap}[0]{\hat{\Bh}}
\newcommand{\icap}[0]{\hat{\Bi}}
\newcommand{\jcap}[0]{\hat{\Bj}}
\newcommand{\kcap}[0]{\hat{\Bk}}
\newcommand{\lcap}[0]{\hat{\Bl}}
\newcommand{\mcap}[0]{\hat{\Bm}}
\newcommand{\ncap}[0]{\hat{\Bn}}
\newcommand{\ocap}[0]{\hat{\Bo}}
\newcommand{\pcap}[0]{\hat{\Bp}}
\newcommand{\qcap}[0]{\hat{\Bq}}
\newcommand{\rcap}[0]{\hat{\Br}}
\newcommand{\scap}[0]{\hat{\Bs}}
\newcommand{\tcap}[0]{\hat{\Bt}}
\newcommand{\ucap}[0]{\hat{\Bu}}
\newcommand{\vcap}[0]{\hat{\Bv}}
\newcommand{\wcap}[0]{\hat{\Bw}}
\newcommand{\xcap}[0]{\hat{\Bx}}
\newcommand{\ycap}[0]{\hat{\By}}
\newcommand{\zcap}[0]{\hat{\Bz}}
\newcommand{\thetacap}[0]{\hat{\Btheta}}

%
% to write R^n and C^n in a distinguishable fashion.  Perhaps change this
% to the double lined characters upon figuring out how to do so.
%
\newcommand{\C}[1]{$\mathbb{C}^{#1}$}
\newcommand{\R}[1]{$\mathbb{R}^{#1}$}

%
% various generally useful helpers
%

% derivative of #1 wrt. #2:
\newcommand{\D}[2] {\frac {d#2} {d#1}}

\newcommand{\inv}[1]{\frac{1}{#1}}
\newcommand{\cross}[0]{\times}

\newcommand{\abs}[1]{\lvert{#1}\rvert}
\newcommand{\norm}[1]{\lVert{#1}\rVert}
\newcommand{\innerprod}[2]{\langle{#1}, {#2}\rangle}
\newcommand{\dotprod}[2]{{#1} \cdot {#2}}
\newcommand{\bdotprod}[2]{\left({#1} \cdot {#2}\right)}
\newcommand{\crossprod}[2]{{#1} \cross {#2}}
\newcommand{\tripleprod}[3]{\dotprod{\left(\crossprod{#1}{#2}\right)}{#3}}

\DeclareMathOperator{\Proj}{Proj}
\DeclareMathOperator{\Span}{span}
\DeclareMathOperator{\Sgn}{sgn}
\DeclareMathOperator{\Area}{Area}
\DeclareMathOperator{\Volume}{Volume}

%
% A few miscellaneous things specific to this document
%
\newcommand{\crossop}[1]{\crossprod{#1}{}}

% R2 vector.
\newcommand{\VectorTwo}[2]{
\begin{bmatrix}
 {#1} \\
 {#2}
\end{bmatrix}
}

\newcommand{\VectorN}[1]{
\begin{bmatrix}
{#1}_1 \\
{#1}_2 \\
\vdots \\
{#1}_N \\
\end{bmatrix}
}

\newcommand{\DETuvij}[4]{
\begin{vmatrix}
 {#1}_{#3} & {#1}_{#4} \\
 {#2}_{#3} & {#2}_{#4}
\end{vmatrix}
}

\newcommand{\DETuvwijk}[6]{
\begin{vmatrix}
 {#1}_{#4} & {#1}_{#5} & {#1}_{#6} \\
 {#2}_{#4} & {#2}_{#5} & {#2}_{#6} \\
 {#3}_{#4} & {#3}_{#5} & {#3}_{#6}
\end{vmatrix}
}

\newcommand{\DETuvwxijkl}[8]{
\begin{vmatrix}
 {#1}_{#5} & {#1}_{#6} & {#1}_{#7} & {#1}_{#8} \\
 {#2}_{#5} & {#2}_{#6} & {#2}_{#7} & {#2}_{#8} \\
 {#3}_{#5} & {#3}_{#6} & {#3}_{#7} & {#3}_{#8} \\
 {#4}_{#5} & {#4}_{#6} & {#4}_{#7} & {#4}_{#8} \\
\end{vmatrix}
}

%\newcommand{\DETuvwxyijklm}[10]{
%\begin{vmatrix}
% {#1}_{#6} & {#1}_{#7} & {#1}_{#8} & {#1}_{#9} & {#1}_{#10} \\
% {#2}_{#6} & {#2}_{#7} & {#2}_{#8} & {#2}_{#9} & {#2}_{#10} \\
% {#3}_{#6} & {#3}_{#7} & {#3}_{#8} & {#3}_{#9} & {#3}_{#10} \\
% {#4}_{#6} & {#4}_{#7} & {#4}_{#8} & {#4}_{#9} & {#4}_{#10} \\
% {#5}_{#6} & {#5}_{#7} & {#5}_{#8} & {#5}_{#9} & {#5}_{#10}
%\end{vmatrix}
%}

% R3 vector.
\newcommand{\VectorThree}[3]{
\begin{bmatrix}
 {#1} \\
 {#2} \\
 {#3}
\end{bmatrix}
}


%%<misc>
%
\newcommand{\Abs}[1]{{\left\lvert{#1}\right\rvert}}
\newcommand{\spacegrad}[0]{\boldsymbol{\nabla}}
\newcommand{\grad}[0]{\nabla}
\newcommand{\LL}[0]{\mathcal{L}}

% == \partial_{#1} {#2}
\newcommand{\PD}[2]{\frac{\partial {#2}}{\partial {#1}}}
% inline variant
\newcommand{\PDi}[2]{{\partial {#2}}/{\partial {#1}}}

\newcommand{\PDD}[3]{\frac{\partial^2 {#3}}{\partial {#1}\partial {#2}}}
%\newcommand{\PDd}[2]{\frac{\partial^2 {#2}}{{\partial{#1}}^2}}
\newcommand{\PDsq}[2]{\frac{\partial^2 {#2}}{(\partial {#1})^2}}

\newcommand{\Partial}[2]{\frac{\partial {#1}}{\partial {#2}}}
\DeclareMathOperator{\RejName}{Rej}
\newcommand{\Rej}[2]{\RejName_{#1}\left( {#2} \right)}
\newcommand{\Rm}[1]{\mathbb{R}^{#1}}
\newcommand{\Cm}[1]{\mathbb{C}^{#1}}
\newcommand{\conj}[0]{{*}}

%</misc>

% <grade selection>
%
\newcommand{\gpgrade}[2] {{\left\langle{{#1}}\right\rangle}_{#2}}

\newcommand{\gpgradezero}[1] {\gpgrade{#1}{}}
%\newcommand{\gpscalargrade}[1] {{\left\langle{{#1}}\right\rangle}}
%\newcommand{\gpgradezero}[1] {\gpgrade{#1}{0}}

%\newcommand{\gpgradeone}[1] {{\left\langle{{#1}}\right\rangle}_{1}}
\newcommand{\gpgradeone}[1] {\gpgrade{#1}{1}}

\newcommand{\gpgradetwo}[1] {\gpgrade{#1}{2}}
\newcommand{\gpgradethree}[1] {\gpgrade{#1}{3}}
\newcommand{\gpgradefour}[1] {\gpgrade{#1}{4}}
%
% </grade selection>



\newcommand{\adot}[0]{{\dot{a}}}
\newcommand{\bdot}[0]{{\dot{b}}}
% taken for centered dot:
%\newcommand{\cdot}[0]{{\dot{c}}}
%\newcommand{\ddot}[0]{{\dot{d}}}
\newcommand{\edot}[0]{{\dot{e}}}
\newcommand{\fdot}[0]{{\dot{f}}}
\newcommand{\gdot}[0]{{\dot{g}}}
\newcommand{\hdot}[0]{{\dot{h}}}
\newcommand{\idot}[0]{{\dot{i}}}
\newcommand{\jdot}[0]{{\dot{j}}}
\newcommand{\kdot}[0]{{\dot{k}}}
\newcommand{\ldot}[0]{{\dot{l}}}
\newcommand{\mdot}[0]{{\dot{m}}}
\newcommand{\ndot}[0]{{\dot{n}}}
%\newcommand{\odot}[0]{{\dot{o}}}
\newcommand{\pdot}[0]{{\dot{p}}}
\newcommand{\qdot}[0]{{\dot{q}}}
\newcommand{\rdot}[0]{{\dot{r}}}
\newcommand{\sdot}[0]{{\dot{s}}}
\newcommand{\tdot}[0]{{\dot{t}}}
\newcommand{\udot}[0]{{\dot{u}}}
\newcommand{\vdot}[0]{{\dot{v}}}
\newcommand{\wdot}[0]{{\dot{w}}}
\newcommand{\xdot}[0]{{\dot{x}}}
\newcommand{\ydot}[0]{{\dot{y}}}
\newcommand{\zdot}[0]{{\dot{z}}}
\newcommand{\addot}[0]{{\ddot{a}}}
\newcommand{\bddot}[0]{{\ddot{b}}}
\newcommand{\cddot}[0]{{\ddot{c}}}
%\newcommand{\dddot}[0]{{\ddot{d}}}
\newcommand{\eddot}[0]{{\ddot{e}}}
\newcommand{\fddot}[0]{{\ddot{f}}}
\newcommand{\gddot}[0]{{\ddot{g}}}
\newcommand{\hddot}[0]{{\ddot{h}}}
\newcommand{\iddot}[0]{{\ddot{i}}}
\newcommand{\jddot}[0]{{\ddot{j}}}
\newcommand{\kddot}[0]{{\ddot{k}}}
\newcommand{\lddot}[0]{{\ddot{l}}}
\newcommand{\mddot}[0]{{\ddot{m}}}
\newcommand{\nddot}[0]{{\ddot{n}}}
\newcommand{\oddot}[0]{{\ddot{o}}}
\newcommand{\pddot}[0]{{\ddot{p}}}
\newcommand{\qddot}[0]{{\ddot{q}}}
\newcommand{\rddot}[0]{{\ddot{r}}}
\newcommand{\sddot}[0]{{\ddot{s}}}
\newcommand{\tddot}[0]{{\ddot{t}}}
\newcommand{\uddot}[0]{{\ddot{u}}}
\newcommand{\vddot}[0]{{\ddot{v}}}
\newcommand{\wddot}[0]{{\ddot{w}}}
\newcommand{\xddot}[0]{{\ddot{x}}}
\newcommand{\yddot}[0]{{\ddot{y}}}
\newcommand{\zddot}[0]{{\ddot{z}}}

%<bold and dot greek symbols>
%

\newcommand{\Deltadot}[0]{{\dot{\Delta}}}
\newcommand{\Gammadot}[0]{{\dot{\Gamma}}}
\newcommand{\Lambdadot}[0]{{\dot{\Lambda}}}
\newcommand{\Omegadot}[0]{{\dot{\Omega}}}
\newcommand{\Phidot}[0]{{\dot{\Phi}}}
\newcommand{\Pidot}[0]{{\dot{\Pi}}}
\newcommand{\Psidot}[0]{{\dot{\Psi}}}
\newcommand{\Sigmadot}[0]{{\dot{\Sigma}}}
\newcommand{\Thetadot}[0]{{\dot{\Theta}}}
\newcommand{\Upsilondot}[0]{{\dot{\Upsilon}}}
\newcommand{\Xidot}[0]{{\dot{\Xi}}}
\newcommand{\alphadot}[0]{{\dot{\alpha}}}
\newcommand{\betadot}[0]{{\dot{\beta}}}
\newcommand{\chidot}[0]{{\dot{\chi}}}
\newcommand{\deltadot}[0]{{\dot{\delta}}}
\newcommand{\epsilondot}[0]{{\dot{\epsilon}}}
\newcommand{\etadot}[0]{{\dot{\eta}}}
\newcommand{\gammadot}[0]{{\dot{\gamma}}}
\newcommand{\kappadot}[0]{{\dot{\kappa}}}
\newcommand{\lambdadot}[0]{{\dot{\lambda}}}
\newcommand{\mudot}[0]{{\dot{\mu}}}
\newcommand{\nudot}[0]{{\dot{\nu}}}
\newcommand{\omegadot}[0]{{\dot{\omega}}}
\newcommand{\phidot}[0]{{\dot{\phi}}}
\newcommand{\pidot}[0]{{\dot{\pi}}}
\newcommand{\psidot}[0]{{\dot{\psi}}}
\newcommand{\rhodot}[0]{{\dot{\rho}}}
\newcommand{\sigmadot}[0]{{\dot{\sigma}}}
\newcommand{\taudot}[0]{{\dot{\tau}}}
\newcommand{\thetadot}[0]{{\dot{\theta}}}
\newcommand{\upsilondot}[0]{{\dot{\upsilon}}}
\newcommand{\varepsilondot}[0]{{\dot{\varepsilon}}}
\newcommand{\varphidot}[0]{{\dot{\varphi}}}
\newcommand{\varpidot}[0]{{\dot{\varpi}}}
\newcommand{\varrhodot}[0]{{\dot{\varrho}}}
\newcommand{\varsigmadot}[0]{{\dot{\varsigma}}}
\newcommand{\varthetadot}[0]{{\dot{\vartheta}}}
\newcommand{\xidot}[0]{{\dot{\xi}}}
\newcommand{\zetadot}[0]{{\dot{\zeta}}}

\newcommand{\Deltaddot}[0]{{\ddot{\Delta}}}
\newcommand{\Gammaddot}[0]{{\ddot{\Gamma}}}
\newcommand{\Lambdaddot}[0]{{\ddot{\Lambda}}}
\newcommand{\Omegaddot}[0]{{\ddot{\Omega}}}
\newcommand{\Phiddot}[0]{{\ddot{\Phi}}}
\newcommand{\Piddot}[0]{{\ddot{\Pi}}}
\newcommand{\Psiddot}[0]{{\ddot{\Psi}}}
\newcommand{\Sigmaddot}[0]{{\ddot{\Sigma}}}
\newcommand{\Thetaddot}[0]{{\ddot{\Theta}}}
\newcommand{\Upsilonddot}[0]{{\ddot{\Upsilon}}}
\newcommand{\Xiddot}[0]{{\ddot{\Xi}}}
\newcommand{\alphaddot}[0]{{\ddot{\alpha}}}
\newcommand{\betaddot}[0]{{\ddot{\beta}}}
\newcommand{\chiddot}[0]{{\ddot{\chi}}}
\newcommand{\deltaddot}[0]{{\ddot{\delta}}}
\newcommand{\epsilonddot}[0]{{\ddot{\epsilon}}}
\newcommand{\etaddot}[0]{{\ddot{\eta}}}
\newcommand{\gammaddot}[0]{{\ddot{\gamma}}}
\newcommand{\kappaddot}[0]{{\ddot{\kappa}}}
\newcommand{\lambdaddot}[0]{{\ddot{\lambda}}}
\newcommand{\muddot}[0]{{\ddot{\mu}}}
\newcommand{\nuddot}[0]{{\ddot{\nu}}}
\newcommand{\omegaddot}[0]{{\ddot{\omega}}}
\newcommand{\phiddot}[0]{{\ddot{\phi}}}
\newcommand{\piddot}[0]{{\ddot{\pi}}}
\newcommand{\psiddot}[0]{{\ddot{\psi}}}
\newcommand{\rhoddot}[0]{{\ddot{\rho}}}
\newcommand{\sigmaddot}[0]{{\ddot{\sigma}}}
\newcommand{\tauddot}[0]{{\ddot{\tau}}}
\newcommand{\thetaddot}[0]{{\ddot{\theta}}}
\newcommand{\upsilonddot}[0]{{\ddot{\upsilon}}}
\newcommand{\varepsilonddot}[0]{{\ddot{\varepsilon}}}
\newcommand{\varphiddot}[0]{{\ddot{\varphi}}}
\newcommand{\varpiddot}[0]{{\ddot{\varpi}}}
\newcommand{\varrhoddot}[0]{{\ddot{\varrho}}}
\newcommand{\varsigmaddot}[0]{{\ddot{\varsigma}}}
\newcommand{\varthetaddot}[0]{{\ddot{\vartheta}}}
\newcommand{\xiddot}[0]{{\ddot{\xi}}}
\newcommand{\zetaddot}[0]{{\ddot{\zeta}}}

\newcommand{\BDelta}[0]{\boldsymbol{\Delta}}
\newcommand{\BGamma}[0]{\boldsymbol{\Gamma}}
\newcommand{\BLambda}[0]{\boldsymbol{\Lambda}}
\newcommand{\BOmega}[0]{\boldsymbol{\Omega}}
\newcommand{\BPhi}[0]{\boldsymbol{\Phi}}
\newcommand{\BPi}[0]{\boldsymbol{\Pi}}
\newcommand{\BPsi}[0]{\boldsymbol{\Psi}}
\newcommand{\BSigma}[0]{\boldsymbol{\Sigma}}
\newcommand{\BTheta}[0]{\boldsymbol{\Theta}}
\newcommand{\BUpsilon}[0]{\boldsymbol{\Upsilon}}
\newcommand{\BXi}[0]{\boldsymbol{\Xi}}
\newcommand{\Balpha}[0]{\boldsymbol{\alpha}}
\newcommand{\Bbeta}[0]{\boldsymbol{\beta}}
\newcommand{\Bchi}[0]{\boldsymbol{\chi}}
\newcommand{\Bdelta}[0]{\boldsymbol{\delta}}
\newcommand{\Bepsilon}[0]{\boldsymbol{\epsilon}}
\newcommand{\Beta}[0]{\boldsymbol{\eta}}
\newcommand{\Bgamma}[0]{\boldsymbol{\gamma}}
\newcommand{\Bkappa}[0]{\boldsymbol{\kappa}}
\newcommand{\Blambda}[0]{\boldsymbol{\lambda}}
\newcommand{\Bmu}[0]{\boldsymbol{\mu}}
\newcommand{\Bnu}[0]{\boldsymbol{\nu}}
%\newcommand{\Bomega}[0]{\boldsymbol{\omega}}
\newcommand{\Bphi}[0]{\boldsymbol{\phi}}
\newcommand{\Bpi}[0]{\boldsymbol{\pi}}
\newcommand{\Bpsi}[0]{\boldsymbol{\psi}}
\newcommand{\Brho}[0]{\boldsymbol{\rho}}
\newcommand{\Bsigma}[0]{\boldsymbol{\sigma}}
%\newcommand{\Btau}[0]{\boldsymbol{\tau}}
%\newcommand{\Btheta}[0]{\boldsymbol{\theta}}
\newcommand{\Bupsilon}[0]{\boldsymbol{\upsilon}}
\newcommand{\Bvarepsilon}[0]{\boldsymbol{\varepsilon}}
\newcommand{\Bvarphi}[0]{\boldsymbol{\varphi}}
\newcommand{\Bvarpi}[0]{\boldsymbol{\varpi}}
\newcommand{\Bvarrho}[0]{\boldsymbol{\varrho}}
\newcommand{\Bvarsigma}[0]{\boldsymbol{\varsigma}}
\newcommand{\Bvartheta}[0]{\boldsymbol{\vartheta}}
\newcommand{\Bxi}[0]{\boldsymbol{\xi}}
\newcommand{\Bzeta}[0]{\boldsymbol{\zeta}}
%
%</bold and dot greek symbols>
%<infrequent>
%
%\newcommand{\AreaOp}[1]{\AName_{#1}}
%\newcommand{\Babs}[0]{\abs{\BB}}
%\newcommand{\Bcap}[0]{\hat{\BB}}
%\newcommand{\BrPrimeRej}[0]{\rcap(\rcap \wedge \Br')}
%\newcommand{\CA}[0]{\mathcal{A}}
%\newcommand{\Cos}[1]{\cos{\left({#1}\right)}}
%\newcommand{\Det}[1] {\abs{#1}}
%\newcommand{\Dsq}[2] {\frac {\partial^2 {#1}} {\partial {#2}^2}}
%\newcommand{\Exp}[1]{\exp{\left({#1}\right)}}
%\newcommand{\Norm}[1]{\left\lVert{#1}\right\rVert}
%\newcommand{\Sin}[1]{\sin{\left({#1}\right)}}
%\newcommand{\T}[0]{\text{T}}
%\newcommand{\VolumeOp}[1]{\VName_{#1}}
%\newcommand{\agrad}[0]{\Ba \cdot \nabla}
%\newcommand{\alphacap}[0]{\hat{\boldsymbol{\alpha}}}
%\newcommand{\Fcap}[0]{\hat{\BF}}
%\newcommand{\bithree}[0]{{\Bi}_3}
%\newcommand{\bxa}[0]{\Bx\Ba}
%\newcommand{\coordvec}[2]{
%\newcommand{\costheta}[0]{\acap \cdot \xcap}
%\newcommand{\ddt}[1]{\ddot{#1}}
%\newcommand{\ddu}[1] {\frac {d{#1}} {du}}
%\newcommand{\dsqxj}[2] {\frac {\partial^2 {#1}} {\partial {x_{#2}}^2}}
%\newcommand{\dtheta}[1]{\frac{d {#1}}{d \theta}}
%\newcommand{\dt}[1]{\dot{#1}}
%\newcommand{\dt}[1]{\frac{d {#1}}{dt}}
%\newcommand{\dxj}[2] {\frac {\partial {#1}} {\partial {x_{#2}}}}
%\newcommand{\halfPhi}[0]{\frac{\phi}{2}}
%\newcommand{\half}[0]{\inv{2}}
%\newcommand{\inv}[1]{\frac{1}{#1}}
%\newcommand{\laplacian}[0]{\nabla^2}
%\newcommand{\matrixoftx}[3]{
%\newcommand{\nrrp}[0]{\norm{\rcap \wedge \Br'}}
%\newcommand{\oiint}{\bigcirc \hspace{-1.4em} \int \hspace{-.8em} \int}
%\newcommand{\transpose}[1]{{#1}^{\text{T}}}
%\newcommand{\transpose}[1]{{{#1}^{\TextTranspose}}}
%\newcommand{\transpose}[1]{{{#1}^{\text{T}}}}
%\newcommand{\barA}[0]{\bar{A}}
%\newcommand{\qbar}[0]{\bar{q}}
%\newcommand{\qdotbar}[0]{\dot{\bar{q}}}
%
%</infrequent>





%\usepackage{listings}
%\usepackage{txfonts} % for ointctr... (also appears to make "prettier" \int and \sum's)
%\usepackage[bookmarks=true]{hyperref}

%\usepackage{color,cite,graphicx}
   % use colour in the document, put your citations as [1-4]
   % rather than [1,2,3,4] (it looks nicer, and the extended LaTeX2e
   % graphics package. 
%\usepackage{latexsym,amssymb,epsf} % do not remember if these are
   % needed, but their inclusion can not do any damage


\chapter{Commutator and Anti-Commutator Hermitian-ness}
\label{chap:commutatorHerm}
%\author{Peeter Joot \quad peeter.joot@gmail.com }
\date{ April 13, 2009.  commutatorHerm.tex }

%\begin{document}

%\maketitle{}

%\tableofcontents

\section{Motivation}

Reading of the proof of chapter 3, equation 12.7, in \citep{pauli2000wm},
that the anti-commutator 

\begin{equation}\label{eqn:commutatorHerm:20}
\begin{aligned}
\symmetric{F}{G} = {F G + G F}
\end{aligned}
\end{equation}

is Hermitian, is unclear to me.  Fill in the missing details.  Also prove 12.8, that 

\begin{equation}\label{eqn:commutatorHerm:40}
\begin{aligned}
i \antisymmetric{F}{G} = i (F G - G F)
\end{aligned}
\end{equation}

is Hermitian.

\section{Hermitian operator examples}

\subsection{Hermitian definition}

Pauli defines Hermitian in terms of the operator expectation value.  An operator \(H\) is Hermitian if

\begin{equation}\label{eqn:commutator_herm:Hermitian}
\begin{aligned}
\Expectation{H} = \int \psi^\conj (H \psi) d^3 x = \int \psi (H \psi)^\conj d^3 x
\end{aligned}
\end{equation}

Or
\begin{equation}\label{eqn:commutator_herm:HermitianInt}
\begin{aligned}
0 &= \Expectation{H} - \Expectation{H}^\conj = \int d^3 x \left( \psi^\conj (H \psi) - \psi (H \psi)^\conj \right)
\end{aligned}
\end{equation}

\subsection{Two operator form}

For completeness, let us derive the two wave function form of the Hermitian operator definition in full detail (omitted from the text).  With \(\psi = \psi_1 + \psi_2\)
\eqnref{eqn:commutator_herm:HermitianInt} becomes

\begin{equation}\label{eqn:commutatorHerm:60}
\begin{aligned}
\Expectation{H} - \Expectation{H}^\conj
&= \int (\psi_1 + \psi_2)^\conj (H (\psi_1 + \psi_2)) d^3 x - \int (\psi_1 + \psi_2) (H (\psi_1 + \psi_2))^\conj d^3 x \\
&= \int d^3 x \\
&\psi_1^\conj (H \psi_1) - \psi_1 (H \psi_1)^\conj 
+\psi_1^\conj (H \psi_2) - \psi_1 (H \psi_2)^\conj \\
&+\psi_2^\conj (H \psi_2) - \psi_2 (H \psi_2)^\conj 
+\psi_2^\conj (H \psi_1) - \psi_2 (H \psi_1)^\conj \\
&= \int d^3 x
\left( \psi_1^\conj (H \psi_2) - \psi_1 (H \psi_2)^\conj 
+\psi_2^\conj (H \psi_1) - \psi_2 (H \psi_1)^\conj \right) \\
\end{aligned}
\end{equation}

Grouping terms, we have 

\begin{equation}\label{eqn:commutatorHerm:80}
\begin{aligned}
\int d^3 x \left( \psi_1^\conj (H \psi_2) - \psi_2 (H \psi_1)^\conj \right) = \int d^3 x \left( \psi_1 (H \psi_2)^\conj - \psi_2^\conj (H \psi_1) \right)
\end{aligned}
\end{equation}

This is quite a bit different than both sides being separately zero, and the key to that further statement (as pointed out in 9.17 in \citep{bohm1989qt}) 
is that this is also true if the two wave function are adjusted by constant phase factors
\(\psi_1 \rightarrow \psi_1 e^{ia}\), 
\(\psi_2 \rightarrow \psi_2 e^{ib}\).  Doing so we have

\begin{equation}\label{eqn:commutatorHerm:100}
\begin{aligned}
e^{i(a-b)} \int d^3 x \left( \psi_1^\conj (H \psi_2) - \psi_2 (H \psi_1)^\conj \right) = e^{i(b-a)} \int d^3 x \left( \psi_1 (H \psi_2)^\conj - \psi_2^\conj (H \psi_1) \right)
\end{aligned}
\end{equation}

For this to hold for any \(a\), \(b\) both sides of the equation must separately equal zero, and we have

\begin{equation}\label{eqn:commutator_herm:HermitianTwo}
\begin{aligned}
\int d^3 x \psi_1^\conj (H \psi_2) = \int d^3 x \psi_2 (H \psi_1)^\conj
\end{aligned}
\end{equation}

\subsection{Lemma for repeated operators}

Next, examine the reversion behavior of repeated operators.  This appears to be used in the text (or is proved implicitly via some other operation not explained).

Given an pair of Hermitian operators, \(H_1\), and \(H_2\), 

\begin{equation}\label{eqn:commutatorHerm:120}
\begin{aligned}
\int d^3 x \psi_1^\conj (H_1 H_2 \psi_2)
\end{aligned}
\end{equation}

what do we get by reversing the operator action?  With the introduction of a couple of helper wave function variables, \(\epsilon = H_2 \psi_2\), and \(\beta = H_1 \psi_1\), 
this becomes straight forward to determine

\begin{equation}\label{eqn:commutatorHerm:140}
\begin{aligned}
\int d^3 x \psi_1^\conj (H_1 \epsilon)
&=
\int d^3 x \epsilon (H_1 \psi_1)^\conj \\
&=
\int d^3 x (H_2 \psi_2) (H_1 \psi_1)^\conj \\
&=
\int d^3 x \beta^\conj (H_2 \psi_2) \\
&=
\int d^3 x \psi_2 (H_2 \beta)^\conj \\
\end{aligned}
\end{equation}

So we have
\begin{equation}\label{eqn:commutator_herm:reverse}
\begin{aligned}
\int d^3 x \psi_1^\conj (H_1 H_2 \psi_2) &= \int d^3 x \psi_2 (H_2 H_1 \psi_1)^\conj 
\end{aligned}
\end{equation}

\subsection{Anti-commutator}

The statement that the anti-commutator is Hermitian means that we have

\begin{equation}\label{eqn:commutatorHerm:160}
\begin{aligned}
\int d^3 x \psi_2^\conj \left( \symmetric{F}{G} \psi_1 \right) &= \int d^3 x \psi_1 \left(\symmetric{F}{G} \psi_2 \right)^\conj \\
\text{or} \\
\int d^3 x \psi_2^\conj \left( (F G + G F) \psi_1 \right) &= \int d^3 x \psi_1 \left((F G + G F) \psi_2 \right)^\conj \\
\end{aligned}
\end{equation}

Let us expand the right hand side and see if we can get back the LHS
Or
\begin{equation}\label{eqn:commutatorHerm:180}
\begin{aligned}
\int d^3 x \psi_1 \left(\symmetric{F}{G} \psi_2 \right)^\conj 
&=
\int d^3 x \psi_1 \left((F G + G F) \psi_2 \right)^\conj \\
&=
\int d^3 x \psi_2^\conj \left((G F + F G) \psi_1 \right) \quad\quad \text{(applying \eqnref{eqn:commutator_herm:reverse} twice)} \\
&=
\int d^3 x \psi_2^\conj \left( \symmetric{F}{G} \psi_1 \right) \\
\end{aligned}
\end{equation}

Hmm.  That is the Hermitian identity of \eqnref{eqn:commutator_herm:HermitianTwo}, so we are done.  Not at all complicated after all (albeit less
general than the text where a result for a more general pair of operators was given).

\subsection{Commutator}

Now, how about the (imaginary scaled) commutator case?

\begin{equation}\label{eqn:commutatorHerm:200}
\begin{aligned}
\int d^3 x \psi_2^\conj \left( i \antisymmetric{F}{G} \psi_1 \right) &= \int d^3 x \psi_1 \left( i \antisymmetric{F}{G} \psi_2 \right)^\conj \\
\text{or} \\
\int d^3 x \psi_2^\conj \left( i(F G - G F) \psi_1 \right) &= \int d^3 x \psi_1 \left(i(F G - G F) \psi_2 \right)^\conj \\
\end{aligned}
\end{equation}

Again, let us try just expanding out the RHS

\begin{equation}\label{eqn:commutatorHerm:220}
\begin{aligned}
\int d^3 x \psi_1 \left( i \antisymmetric{F}{G} \psi_2 \right)^\conj 
&= \int d^3 x \psi_1 \left(i(F G - G F) \psi_2 \right)^\conj \\
&= -i \int d^3 x \psi_1 \left((F G - G F) \psi_2 \right)^\conj \\
&= -i \int d^3 x \psi_2^\conj \left((G F - F G) \psi_1 \right) \\
&= \int d^3 x \psi_2^\conj \left(i (F G - G F) \psi_1 \right) \\
&= \int d^3 x \psi_2^\conj \left(i \antisymmetric{F}{G}\psi_1 \right) \\
\end{aligned}
\end{equation}

QED.

\section{Future: Relation to Clifford product}

For vector spaces, as noted in \citep{gabook:PJpauliMatrix},
we can write the Clifford product of two \R{N} vectors in terms of commutators and anti-commutators

\begin{equation}\label{eqn:commutatorHerm:240}
\begin{aligned}
\Bf \Bg = \inv{2} \left( \symmetric{\Bf}{\Bg} + i \antisymmetric{\Bf}{\Bg} \right)
\end{aligned}
\end{equation}

where \(i\) is the pseudoscalar for the space.  So, while \(FG\) is not necessarily Hermitian, it is interesting that the composite operator
\(\symmetric{F}{G} + i \antisymmetric{F}{G}\), which is so close to the product operator of Euclidean vector spaces, is Hermitian.
Explore this geometric analogy later.

%\bibliographystyle{plainnat}
%\bibliography{myrefs}

%\end{document}

%
% Copyright � 2012 Peeter Joot.  All Rights Reserved.
% Licenced as described in the file LICENSE under the root directory of this GIT repository.
%

%
%
%\documentclass{article}

%\usepackage{amsmath}
\usepackage{mathpazo}

%
% shorthand for bold symbols, convenient for vectors and matrices
%
\newcommand{\Ba}[0]{\mathbf{a}}
\newcommand{\Bb}[0]{\mathbf{b}}
\newcommand{\Bc}[0]{\mathbf{c}}
\newcommand{\Bd}[0]{\mathbf{d}}
\newcommand{\Be}[0]{\mathbf{e}}
\newcommand{\Bf}[0]{\mathbf{f}}
\newcommand{\Bg}[0]{\mathbf{g}}
\newcommand{\Bh}[0]{\mathbf{h}}
\newcommand{\Bi}[0]{\mathbf{i}}
\newcommand{\Bj}[0]{\mathbf{j}}
\newcommand{\Bk}[0]{\mathbf{k}}
\newcommand{\Bl}[0]{\mathbf{l}}
\newcommand{\Bm}[0]{\mathbf{m}}
\newcommand{\Bn}[0]{\mathbf{n}}
\newcommand{\Bo}[0]{\mathbf{o}}
\newcommand{\Bp}[0]{\mathbf{p}}
\newcommand{\Bq}[0]{\mathbf{q}}
\newcommand{\Br}[0]{\mathbf{r}}
\newcommand{\Bs}[0]{\mathbf{s}}
\newcommand{\Bt}[0]{\mathbf{t}}
\newcommand{\Bu}[0]{\mathbf{u}}
\newcommand{\Bv}[0]{\mathbf{v}}
\newcommand{\Bw}[0]{\mathbf{w}}
\newcommand{\Bx}[0]{\mathbf{x}}
\newcommand{\By}[0]{\mathbf{y}}
\newcommand{\Bz}[0]{\mathbf{z}}
\newcommand{\BA}[0]{\mathbf{A}}
\newcommand{\BB}[0]{\mathbf{B}}
\newcommand{\BC}[0]{\mathbf{C}}
\newcommand{\BD}[0]{\mathbf{D}}
\newcommand{\BE}[0]{\mathbf{E}}
\newcommand{\BF}[0]{\mathbf{F}}
\newcommand{\BG}[0]{\mathbf{G}}
\newcommand{\BH}[0]{\mathbf{H}}
\newcommand{\BI}[0]{\mathbf{I}}
\newcommand{\BJ}[0]{\mathbf{J}}
\newcommand{\BK}[0]{\mathbf{K}}
\newcommand{\BL}[0]{\mathbf{L}}
\newcommand{\BM}[0]{\mathbf{M}}
\newcommand{\BN}[0]{\mathbf{N}}
\newcommand{\BO}[0]{\mathbf{O}}
\newcommand{\BP}[0]{\mathbf{P}}
\newcommand{\BQ}[0]{\mathbf{Q}}
\newcommand{\BR}[0]{\mathbf{R}}
\newcommand{\BS}[0]{\mathbf{S}}
\newcommand{\BT}[0]{\mathbf{T}}
\newcommand{\BU}[0]{\mathbf{U}}
\newcommand{\BV}[0]{\mathbf{V}}
\newcommand{\BW}[0]{\mathbf{W}}
\newcommand{\BX}[0]{\mathbf{X}}
\newcommand{\BY}[0]{\mathbf{Y}}
\newcommand{\BZ}[0]{\mathbf{Z}}

\newcommand{\Bzero}[0]{\mathbf{0}}
\newcommand{\Btheta}[0]{\boldsymbol{\theta}}
\newcommand{\Btau}[0]{\boldsymbol{\tau}}
\newcommand{\Bomega}[0]{\boldsymbol{\omega}}

%
% shorthand for unit vectors
%
\newcommand{\acap}[0]{\hat{\Ba}}
\newcommand{\bcap}[0]{\hat{\Bb}}
\newcommand{\ccap}[0]{\hat{\Bc}}
\newcommand{\dcap}[0]{\hat{\Bd}}
\newcommand{\ecap}[0]{\hat{\Be}}
\newcommand{\fcap}[0]{\hat{\Bf}}
\newcommand{\gcap}[0]{\hat{\Bg}}
\newcommand{\hcap}[0]{\hat{\Bh}}
\newcommand{\icap}[0]{\hat{\Bi}}
\newcommand{\jcap}[0]{\hat{\Bj}}
\newcommand{\kcap}[0]{\hat{\Bk}}
\newcommand{\lcap}[0]{\hat{\Bl}}
\newcommand{\mcap}[0]{\hat{\Bm}}
\newcommand{\ncap}[0]{\hat{\Bn}}
\newcommand{\ocap}[0]{\hat{\Bo}}
\newcommand{\pcap}[0]{\hat{\Bp}}
\newcommand{\qcap}[0]{\hat{\Bq}}
\newcommand{\rcap}[0]{\hat{\Br}}
\newcommand{\scap}[0]{\hat{\Bs}}
\newcommand{\tcap}[0]{\hat{\Bt}}
\newcommand{\ucap}[0]{\hat{\Bu}}
\newcommand{\vcap}[0]{\hat{\Bv}}
\newcommand{\wcap}[0]{\hat{\Bw}}
\newcommand{\xcap}[0]{\hat{\Bx}}
\newcommand{\ycap}[0]{\hat{\By}}
\newcommand{\zcap}[0]{\hat{\Bz}}
\newcommand{\thetacap}[0]{\hat{\Btheta}}

%
% to write R^n and C^n in a distinguishable fashion.  Perhaps change this
% to the double lined characters upon figuring out how to do so.
%
\newcommand{\C}[1]{$\mathbb{C}^{#1}$}
\newcommand{\R}[1]{$\mathbb{R}^{#1}$}

%
% various generally useful helpers
%

% derivative of #1 wrt. #2:
\newcommand{\D}[2] {\frac {d#2} {d#1}}

\newcommand{\inv}[1]{\frac{1}{#1}}
\newcommand{\cross}[0]{\times}

\newcommand{\abs}[1]{\lvert{#1}\rvert}
\newcommand{\norm}[1]{\lVert{#1}\rVert}
\newcommand{\innerprod}[2]{\langle{#1}, {#2}\rangle}
\newcommand{\dotprod}[2]{{#1} \cdot {#2}}
\newcommand{\bdotprod}[2]{\left({#1} \cdot {#2}\right)}
\newcommand{\crossprod}[2]{{#1} \cross {#2}}
\newcommand{\tripleprod}[3]{\dotprod{\left(\crossprod{#1}{#2}\right)}{#3}}

\DeclareMathOperator{\Proj}{Proj}
\DeclareMathOperator{\Span}{span}
\DeclareMathOperator{\Sgn}{sgn}
\DeclareMathOperator{\Area}{Area}
\DeclareMathOperator{\Volume}{Volume}

%
% A few miscellaneous things specific to this document
%
\newcommand{\crossop}[1]{\crossprod{#1}{}}

% R2 vector.
\newcommand{\VectorTwo}[2]{
\begin{bmatrix}
 {#1} \\
 {#2}
\end{bmatrix}
}

\newcommand{\VectorN}[1]{
\begin{bmatrix}
{#1}_1 \\
{#1}_2 \\
\vdots \\
{#1}_N \\
\end{bmatrix}
}

\newcommand{\DETuvij}[4]{
\begin{vmatrix}
 {#1}_{#3} & {#1}_{#4} \\
 {#2}_{#3} & {#2}_{#4}
\end{vmatrix}
}

\newcommand{\DETuvwijk}[6]{
\begin{vmatrix}
 {#1}_{#4} & {#1}_{#5} & {#1}_{#6} \\
 {#2}_{#4} & {#2}_{#5} & {#2}_{#6} \\
 {#3}_{#4} & {#3}_{#5} & {#3}_{#6}
\end{vmatrix}
}

\newcommand{\DETuvwxijkl}[8]{
\begin{vmatrix}
 {#1}_{#5} & {#1}_{#6} & {#1}_{#7} & {#1}_{#8} \\
 {#2}_{#5} & {#2}_{#6} & {#2}_{#7} & {#2}_{#8} \\
 {#3}_{#5} & {#3}_{#6} & {#3}_{#7} & {#3}_{#8} \\
 {#4}_{#5} & {#4}_{#6} & {#4}_{#7} & {#4}_{#8} \\
\end{vmatrix}
}

%\newcommand{\DETuvwxyijklm}[10]{
%\begin{vmatrix}
% {#1}_{#6} & {#1}_{#7} & {#1}_{#8} & {#1}_{#9} & {#1}_{#10} \\
% {#2}_{#6} & {#2}_{#7} & {#2}_{#8} & {#2}_{#9} & {#2}_{#10} \\
% {#3}_{#6} & {#3}_{#7} & {#3}_{#8} & {#3}_{#9} & {#3}_{#10} \\
% {#4}_{#6} & {#4}_{#7} & {#4}_{#8} & {#4}_{#9} & {#4}_{#10} \\
% {#5}_{#6} & {#5}_{#7} & {#5}_{#8} & {#5}_{#9} & {#5}_{#10}
%\end{vmatrix}
%}

% R3 vector.
\newcommand{\VectorThree}[3]{
\begin{bmatrix}
 {#1} \\
 {#2} \\
 {#3}
\end{bmatrix}
}


%%<misc>
%
\newcommand{\Abs}[1]{{\left\lvert{#1}\right\rvert}}
\newcommand{\spacegrad}[0]{\boldsymbol{\nabla}}
\newcommand{\grad}[0]{\nabla}
\newcommand{\LL}[0]{\mathcal{L}}

% == \partial_{#1} {#2}
\newcommand{\PD}[2]{\frac{\partial {#2}}{\partial {#1}}}
% inline variant
\newcommand{\PDi}[2]{{\partial {#2}}/{\partial {#1}}}

\newcommand{\PDD}[3]{\frac{\partial^2 {#3}}{\partial {#1}\partial {#2}}}
%\newcommand{\PDd}[2]{\frac{\partial^2 {#2}}{{\partial{#1}}^2}}
\newcommand{\PDsq}[2]{\frac{\partial^2 {#2}}{(\partial {#1})^2}}

\newcommand{\Partial}[2]{\frac{\partial {#1}}{\partial {#2}}}
\DeclareMathOperator{\RejName}{Rej}
\newcommand{\Rej}[2]{\RejName_{#1}\left( {#2} \right)}
\newcommand{\Rm}[1]{\mathbb{R}^{#1}}
\newcommand{\Cm}[1]{\mathbb{C}^{#1}}
\newcommand{\conj}[0]{{*}}

%</misc>

% <grade selection>
%
\newcommand{\gpgrade}[2] {{\left\langle{{#1}}\right\rangle}_{#2}}

\newcommand{\gpgradezero}[1] {\gpgrade{#1}{}}
%\newcommand{\gpscalargrade}[1] {{\left\langle{{#1}}\right\rangle}}
%\newcommand{\gpgradezero}[1] {\gpgrade{#1}{0}}

%\newcommand{\gpgradeone}[1] {{\left\langle{{#1}}\right\rangle}_{1}}
\newcommand{\gpgradeone}[1] {\gpgrade{#1}{1}}

\newcommand{\gpgradetwo}[1] {\gpgrade{#1}{2}}
\newcommand{\gpgradethree}[1] {\gpgrade{#1}{3}}
\newcommand{\gpgradefour}[1] {\gpgrade{#1}{4}}
%
% </grade selection>



\newcommand{\adot}[0]{{\dot{a}}}
\newcommand{\bdot}[0]{{\dot{b}}}
% taken for centered dot:
%\newcommand{\cdot}[0]{{\dot{c}}}
%\newcommand{\ddot}[0]{{\dot{d}}}
\newcommand{\edot}[0]{{\dot{e}}}
\newcommand{\fdot}[0]{{\dot{f}}}
\newcommand{\gdot}[0]{{\dot{g}}}
\newcommand{\hdot}[0]{{\dot{h}}}
\newcommand{\idot}[0]{{\dot{i}}}
\newcommand{\jdot}[0]{{\dot{j}}}
\newcommand{\kdot}[0]{{\dot{k}}}
\newcommand{\ldot}[0]{{\dot{l}}}
\newcommand{\mdot}[0]{{\dot{m}}}
\newcommand{\ndot}[0]{{\dot{n}}}
%\newcommand{\odot}[0]{{\dot{o}}}
\newcommand{\pdot}[0]{{\dot{p}}}
\newcommand{\qdot}[0]{{\dot{q}}}
\newcommand{\rdot}[0]{{\dot{r}}}
\newcommand{\sdot}[0]{{\dot{s}}}
\newcommand{\tdot}[0]{{\dot{t}}}
\newcommand{\udot}[0]{{\dot{u}}}
\newcommand{\vdot}[0]{{\dot{v}}}
\newcommand{\wdot}[0]{{\dot{w}}}
\newcommand{\xdot}[0]{{\dot{x}}}
\newcommand{\ydot}[0]{{\dot{y}}}
\newcommand{\zdot}[0]{{\dot{z}}}
\newcommand{\addot}[0]{{\ddot{a}}}
\newcommand{\bddot}[0]{{\ddot{b}}}
\newcommand{\cddot}[0]{{\ddot{c}}}
%\newcommand{\dddot}[0]{{\ddot{d}}}
\newcommand{\eddot}[0]{{\ddot{e}}}
\newcommand{\fddot}[0]{{\ddot{f}}}
\newcommand{\gddot}[0]{{\ddot{g}}}
\newcommand{\hddot}[0]{{\ddot{h}}}
\newcommand{\iddot}[0]{{\ddot{i}}}
\newcommand{\jddot}[0]{{\ddot{j}}}
\newcommand{\kddot}[0]{{\ddot{k}}}
\newcommand{\lddot}[0]{{\ddot{l}}}
\newcommand{\mddot}[0]{{\ddot{m}}}
\newcommand{\nddot}[0]{{\ddot{n}}}
\newcommand{\oddot}[0]{{\ddot{o}}}
\newcommand{\pddot}[0]{{\ddot{p}}}
\newcommand{\qddot}[0]{{\ddot{q}}}
\newcommand{\rddot}[0]{{\ddot{r}}}
\newcommand{\sddot}[0]{{\ddot{s}}}
\newcommand{\tddot}[0]{{\ddot{t}}}
\newcommand{\uddot}[0]{{\ddot{u}}}
\newcommand{\vddot}[0]{{\ddot{v}}}
\newcommand{\wddot}[0]{{\ddot{w}}}
\newcommand{\xddot}[0]{{\ddot{x}}}
\newcommand{\yddot}[0]{{\ddot{y}}}
\newcommand{\zddot}[0]{{\ddot{z}}}

%<bold and dot greek symbols>
%

\newcommand{\Deltadot}[0]{{\dot{\Delta}}}
\newcommand{\Gammadot}[0]{{\dot{\Gamma}}}
\newcommand{\Lambdadot}[0]{{\dot{\Lambda}}}
\newcommand{\Omegadot}[0]{{\dot{\Omega}}}
\newcommand{\Phidot}[0]{{\dot{\Phi}}}
\newcommand{\Pidot}[0]{{\dot{\Pi}}}
\newcommand{\Psidot}[0]{{\dot{\Psi}}}
\newcommand{\Sigmadot}[0]{{\dot{\Sigma}}}
\newcommand{\Thetadot}[0]{{\dot{\Theta}}}
\newcommand{\Upsilondot}[0]{{\dot{\Upsilon}}}
\newcommand{\Xidot}[0]{{\dot{\Xi}}}
\newcommand{\alphadot}[0]{{\dot{\alpha}}}
\newcommand{\betadot}[0]{{\dot{\beta}}}
\newcommand{\chidot}[0]{{\dot{\chi}}}
\newcommand{\deltadot}[0]{{\dot{\delta}}}
\newcommand{\epsilondot}[0]{{\dot{\epsilon}}}
\newcommand{\etadot}[0]{{\dot{\eta}}}
\newcommand{\gammadot}[0]{{\dot{\gamma}}}
\newcommand{\kappadot}[0]{{\dot{\kappa}}}
\newcommand{\lambdadot}[0]{{\dot{\lambda}}}
\newcommand{\mudot}[0]{{\dot{\mu}}}
\newcommand{\nudot}[0]{{\dot{\nu}}}
\newcommand{\omegadot}[0]{{\dot{\omega}}}
\newcommand{\phidot}[0]{{\dot{\phi}}}
\newcommand{\pidot}[0]{{\dot{\pi}}}
\newcommand{\psidot}[0]{{\dot{\psi}}}
\newcommand{\rhodot}[0]{{\dot{\rho}}}
\newcommand{\sigmadot}[0]{{\dot{\sigma}}}
\newcommand{\taudot}[0]{{\dot{\tau}}}
\newcommand{\thetadot}[0]{{\dot{\theta}}}
\newcommand{\upsilondot}[0]{{\dot{\upsilon}}}
\newcommand{\varepsilondot}[0]{{\dot{\varepsilon}}}
\newcommand{\varphidot}[0]{{\dot{\varphi}}}
\newcommand{\varpidot}[0]{{\dot{\varpi}}}
\newcommand{\varrhodot}[0]{{\dot{\varrho}}}
\newcommand{\varsigmadot}[0]{{\dot{\varsigma}}}
\newcommand{\varthetadot}[0]{{\dot{\vartheta}}}
\newcommand{\xidot}[0]{{\dot{\xi}}}
\newcommand{\zetadot}[0]{{\dot{\zeta}}}

\newcommand{\Deltaddot}[0]{{\ddot{\Delta}}}
\newcommand{\Gammaddot}[0]{{\ddot{\Gamma}}}
\newcommand{\Lambdaddot}[0]{{\ddot{\Lambda}}}
\newcommand{\Omegaddot}[0]{{\ddot{\Omega}}}
\newcommand{\Phiddot}[0]{{\ddot{\Phi}}}
\newcommand{\Piddot}[0]{{\ddot{\Pi}}}
\newcommand{\Psiddot}[0]{{\ddot{\Psi}}}
\newcommand{\Sigmaddot}[0]{{\ddot{\Sigma}}}
\newcommand{\Thetaddot}[0]{{\ddot{\Theta}}}
\newcommand{\Upsilonddot}[0]{{\ddot{\Upsilon}}}
\newcommand{\Xiddot}[0]{{\ddot{\Xi}}}
\newcommand{\alphaddot}[0]{{\ddot{\alpha}}}
\newcommand{\betaddot}[0]{{\ddot{\beta}}}
\newcommand{\chiddot}[0]{{\ddot{\chi}}}
\newcommand{\deltaddot}[0]{{\ddot{\delta}}}
\newcommand{\epsilonddot}[0]{{\ddot{\epsilon}}}
\newcommand{\etaddot}[0]{{\ddot{\eta}}}
\newcommand{\gammaddot}[0]{{\ddot{\gamma}}}
\newcommand{\kappaddot}[0]{{\ddot{\kappa}}}
\newcommand{\lambdaddot}[0]{{\ddot{\lambda}}}
\newcommand{\muddot}[0]{{\ddot{\mu}}}
\newcommand{\nuddot}[0]{{\ddot{\nu}}}
\newcommand{\omegaddot}[0]{{\ddot{\omega}}}
\newcommand{\phiddot}[0]{{\ddot{\phi}}}
\newcommand{\piddot}[0]{{\ddot{\pi}}}
\newcommand{\psiddot}[0]{{\ddot{\psi}}}
\newcommand{\rhoddot}[0]{{\ddot{\rho}}}
\newcommand{\sigmaddot}[0]{{\ddot{\sigma}}}
\newcommand{\tauddot}[0]{{\ddot{\tau}}}
\newcommand{\thetaddot}[0]{{\ddot{\theta}}}
\newcommand{\upsilonddot}[0]{{\ddot{\upsilon}}}
\newcommand{\varepsilonddot}[0]{{\ddot{\varepsilon}}}
\newcommand{\varphiddot}[0]{{\ddot{\varphi}}}
\newcommand{\varpiddot}[0]{{\ddot{\varpi}}}
\newcommand{\varrhoddot}[0]{{\ddot{\varrho}}}
\newcommand{\varsigmaddot}[0]{{\ddot{\varsigma}}}
\newcommand{\varthetaddot}[0]{{\ddot{\vartheta}}}
\newcommand{\xiddot}[0]{{\ddot{\xi}}}
\newcommand{\zetaddot}[0]{{\ddot{\zeta}}}

\newcommand{\BDelta}[0]{\boldsymbol{\Delta}}
\newcommand{\BGamma}[0]{\boldsymbol{\Gamma}}
\newcommand{\BLambda}[0]{\boldsymbol{\Lambda}}
\newcommand{\BOmega}[0]{\boldsymbol{\Omega}}
\newcommand{\BPhi}[0]{\boldsymbol{\Phi}}
\newcommand{\BPi}[0]{\boldsymbol{\Pi}}
\newcommand{\BPsi}[0]{\boldsymbol{\Psi}}
\newcommand{\BSigma}[0]{\boldsymbol{\Sigma}}
\newcommand{\BTheta}[0]{\boldsymbol{\Theta}}
\newcommand{\BUpsilon}[0]{\boldsymbol{\Upsilon}}
\newcommand{\BXi}[0]{\boldsymbol{\Xi}}
\newcommand{\Balpha}[0]{\boldsymbol{\alpha}}
\newcommand{\Bbeta}[0]{\boldsymbol{\beta}}
\newcommand{\Bchi}[0]{\boldsymbol{\chi}}
\newcommand{\Bdelta}[0]{\boldsymbol{\delta}}
\newcommand{\Bepsilon}[0]{\boldsymbol{\epsilon}}
\newcommand{\Beta}[0]{\boldsymbol{\eta}}
\newcommand{\Bgamma}[0]{\boldsymbol{\gamma}}
\newcommand{\Bkappa}[0]{\boldsymbol{\kappa}}
\newcommand{\Blambda}[0]{\boldsymbol{\lambda}}
\newcommand{\Bmu}[0]{\boldsymbol{\mu}}
\newcommand{\Bnu}[0]{\boldsymbol{\nu}}
%\newcommand{\Bomega}[0]{\boldsymbol{\omega}}
\newcommand{\Bphi}[0]{\boldsymbol{\phi}}
\newcommand{\Bpi}[0]{\boldsymbol{\pi}}
\newcommand{\Bpsi}[0]{\boldsymbol{\psi}}
\newcommand{\Brho}[0]{\boldsymbol{\rho}}
\newcommand{\Bsigma}[0]{\boldsymbol{\sigma}}
%\newcommand{\Btau}[0]{\boldsymbol{\tau}}
%\newcommand{\Btheta}[0]{\boldsymbol{\theta}}
\newcommand{\Bupsilon}[0]{\boldsymbol{\upsilon}}
\newcommand{\Bvarepsilon}[0]{\boldsymbol{\varepsilon}}
\newcommand{\Bvarphi}[0]{\boldsymbol{\varphi}}
\newcommand{\Bvarpi}[0]{\boldsymbol{\varpi}}
\newcommand{\Bvarrho}[0]{\boldsymbol{\varrho}}
\newcommand{\Bvarsigma}[0]{\boldsymbol{\varsigma}}
\newcommand{\Bvartheta}[0]{\boldsymbol{\vartheta}}
\newcommand{\Bxi}[0]{\boldsymbol{\xi}}
\newcommand{\Bzeta}[0]{\boldsymbol{\zeta}}
%
%</bold and dot greek symbols>
%<infrequent>
%
%\newcommand{\AreaOp}[1]{\AName_{#1}}
%\newcommand{\Babs}[0]{\abs{\BB}}
%\newcommand{\Bcap}[0]{\hat{\BB}}
%\newcommand{\BrPrimeRej}[0]{\rcap(\rcap \wedge \Br')}
%\newcommand{\CA}[0]{\mathcal{A}}
%\newcommand{\Cos}[1]{\cos{\left({#1}\right)}}
%\newcommand{\Det}[1] {\abs{#1}}
%\newcommand{\Dsq}[2] {\frac {\partial^2 {#1}} {\partial {#2}^2}}
%\newcommand{\Exp}[1]{\exp{\left({#1}\right)}}
%\newcommand{\Norm}[1]{\left\lVert{#1}\right\rVert}
%\newcommand{\Sin}[1]{\sin{\left({#1}\right)}}
%\newcommand{\T}[0]{\text{T}}
%\newcommand{\VolumeOp}[1]{\VName_{#1}}
%\newcommand{\agrad}[0]{\Ba \cdot \nabla}
%\newcommand{\alphacap}[0]{\hat{\boldsymbol{\alpha}}}
%\newcommand{\Fcap}[0]{\hat{\BF}}
%\newcommand{\bithree}[0]{{\Bi}_3}
%\newcommand{\bxa}[0]{\Bx\Ba}
%\newcommand{\coordvec}[2]{
%\newcommand{\costheta}[0]{\acap \cdot \xcap}
%\newcommand{\ddt}[1]{\ddot{#1}}
%\newcommand{\ddu}[1] {\frac {d{#1}} {du}}
%\newcommand{\dsqxj}[2] {\frac {\partial^2 {#1}} {\partial {x_{#2}}^2}}
%\newcommand{\dtheta}[1]{\frac{d {#1}}{d \theta}}
%\newcommand{\dt}[1]{\dot{#1}}
%\newcommand{\dt}[1]{\frac{d {#1}}{dt}}
%\newcommand{\dxj}[2] {\frac {\partial {#1}} {\partial {x_{#2}}}}
%\newcommand{\halfPhi}[0]{\frac{\phi}{2}}
%\newcommand{\half}[0]{\inv{2}}
%\newcommand{\inv}[1]{\frac{1}{#1}}
%\newcommand{\laplacian}[0]{\nabla^2}
%\newcommand{\matrixoftx}[3]{
%\newcommand{\nrrp}[0]{\norm{\rcap \wedge \Br'}}
%\newcommand{\oiint}{\bigcirc \hspace{-1.4em} \int \hspace{-.8em} \int}
%\newcommand{\transpose}[1]{{#1}^{\text{T}}}
%\newcommand{\transpose}[1]{{{#1}^{\TextTranspose}}}
%\newcommand{\transpose}[1]{{{#1}^{\text{T}}}}
%\newcommand{\barA}[0]{\bar{A}}
%\newcommand{\qbar}[0]{\bar{q}}
%\newcommand{\qdotbar}[0]{\dot{\bar{q}}}
%
%</infrequent>





%\usepackage{txfonts} % for ointctr... (also appears to make "prettier" \int and \sum's)

%\usepackage[bookmarks=true]{hyperref}

%\usepackage{color,cite,graphicx}
   % use colour in the document, put your citations as [1-4]
   % rather than [1,2,3,4] (it looks nicer, and the extended LaTeX2e
   % graphics package.
%\usepackage{latexsym,amssymb,epsf} % do not remember if these are
   % needed, but their inclusion can not do any damage


\chapter{Dirac delta function in terms of orthogonal functions}
\label{chap:deltaOrthoSeries}
%\author{Peeter Joot \quad peeterjoot@protonmail.com }
\date{ March 8, 2009.  deltaOrthoSeries.tex }

%\begin{document}

%\maketitle{}
%\tableofcontents

\section{Motivation}

Chapter II of \citep{pauli2000wm} expresses the delta function in terms of
orthonormal basis functions, but the treatment is slightly
hard to follow.
Re-express some of this in my own words the slow and dumb way to get an
understanding of the ideas.  Also explore the summation representation of
the delta function and use it to relate Fourier series and transforms.

\section{Fourier coefficients}

Given an orthonormal basis

\begin{equation}\label{eqn:deltaOrthoSeries:20}
\begin{aligned}
\int u_m^\conj(x) u_n(x) = \delta_{mn}
\end{aligned}
\end{equation}

For a function that can be expressed entirely in this basis, such as

\begin{equation}\label{eqn:deltaOrthoSeries:40}
\begin{aligned}
f(x) = \sum_k a_k u_k(x)
\end{aligned}
\end{equation}

We can then compute the Fourier coefficients \(a_k\) in the normal fashion

\begin{equation}\label{eqn:deltaOrthoSeries:60}
\begin{aligned}
\int u_k^\conj(x) f(x) dx
&= \sum_n a_n \int u_k^\conj(x) u_n(x) dx \\
&= \sum_n a_n \delta_{kn} \\
&= a_k \\
\end{aligned}
\end{equation}

So we have
\begin{equation}\label{eqn:deltaOrthoSeries:80}
\begin{aligned}
f(x) = \sum_k a_k u_k(x)  = \sum_k u_k(x) \int u_k^\conj(x') f(x') dx'
\end{aligned}
\end{equation}

\subsection{Mean square convergence}

How good of a match is a subset of such a sum?  Pauli considers a mean convergence.

\begin{equation}\label{eqn:deltaOrthoSeries:100}
\begin{aligned}
0 &= \lim_{N \rightarrow \infty}\int
{\Abs{f(x') -\sum_{k=1}^N a_k u_k(x') }}^2 dx'  \\
&=
\int \left(f^\conj(x') -\sum_{k=1}^N a_k^\conj u_k^\conj(x') \right) \left(f(x') - \sum_{m=1}^N a_m u_m(x') \right)
dx' \\
&=
\int
\left( f^\conj(x') f(x')
-f^\conj(x') \sum_{m=1}^N a_m u_m(x')
- \sum_{k=1}^N a_k^\conj u_k^\conj(x') f(x')
+ \sum_{m=1}^N a_m u_m(x') \sum_{k=1}^N a_k^\conj u_k^\conj(x')  \right)
dx' \\
&=
\int f^\conj(x') f(x') dx'
- \sum_{m=1}^N a_m a_m^\conj
- \sum_{k=1}^N a_k^\conj a_k
+ \sum_{m=1}^N \sum_{k=1}^N a_m a_k^\conj \delta_{km} \\
%\int u_m(x') u_k^\conj(x') dx' \\
&= \int \Abs{f(x')}^2 dx' - \sum_{m=1}^N \Abs{a_m}^2 \\
\end{aligned}
\end{equation}

So if we have mean square equality in the limit as \(N \rightarrow \infty\), then it must also be true that

\begin{equation}\label{eqn:deltaOrthoSeries:120}
\begin{aligned}
\int \Abs{f(x')}^2 dx' = \sum_{m=1}^\infty \Abs{a_m}^2 \\
\end{aligned}
\end{equation}

He calls this the completeness relation.  If the orthonormal basis is sufficient to express the set of desired functions, then
the squared absolute value of such functions can be expressed entirely in terms of the Fourier coefficients.  The mean square
equality is weaker in the sense that a function can be mismatched to its Fourier representation at a set (of ``measure zero'') points,
and still meet the mean square equality statement.

\subsection{Generalizing the inner product}

Pauli next introduces the an inner product on functions (without calling it that)
in a somewhat indirect
fashion (ie: in terms of Fourier components instead of by definition).

Supposing that one has two functions built up by Fourier components

\begin{equation}\label{eqn:deltaOrthoSeries:140}
\begin{aligned}
f(x) &= \sum_k a_k u_k(x) \\
g(x) &= \sum_k b_k u_k(x) \\
\end{aligned}
\end{equation}

Then we have
\begin{equation}\label{eqn:deltaOrthoSeries:160}
\begin{aligned}
\int f^\conj(x) g(x) &= \sum_{k,m} a_k^\conj b_m \int u_k^\conj(x) u_m(x) = \sum_k a_k^\conj b_k \\
\int g^\conj(x) f(x) &= \sum_{k,m} a_k b_m^\conj \int u_m^\conj(x) u_k(x) = \sum_k b_k^\conj a_k \\
\end{aligned}
\end{equation}

This is something that is familiar to anybody who has taken a linear
algebra course, but perhaps had to be motivated when he wrote the book?

\subsection{Delta function as a sum}

Perhaps Pauli wrote this general function inner product that way to show a natural way that a sum of the
form

\begin{equation}\label{eqn:deltaOrthoSeries:180}
\begin{aligned}
\sum u_m^\conj(x) u_k(x)
\end{aligned}
\end{equation}

arises in use, because he now writes the completeness relation using a sum similar to that above

\begin{equation}\label{eqn:delta_ortho_series:deltaSum}
\begin{aligned}
\sum_{k} u_k^\conj(x') u_k(x) \equiv \delta(x-x')
\end{aligned}
\end{equation}

I had seen this in bra ket notation, in Susskind's lectures as noted in \chapcite{PJQmSusskind}, and also in \citep{mcmahon2005qmd} as the
identity operator

\begin{equation}\label{eqn:deltaOrthoSeries:200}
\begin{aligned}
\sum_{k} \ketbra{k}{k} \equiv \delta(x-x')
\end{aligned}
\end{equation}

From neither of those two sources did I understand where it came from (in Susskind's lectures it appeared to be
related to Fourier transforms).
As Pauli did, let us verify that this works, and try to relate this to a few specific choices of inner products (covering at
least classical Fourier series and the Fourier transform).

The relation of \eqnref{eqn:delta_ortho_series:deltaSum} can be shown to have delta function behavior by integration

\begin{equation}\label{eqn:deltaOrthoSeries:220}
\begin{aligned}
\int \sum_{k} u_k^\conj(x') u_k(x) f(x') dx'
&=
\sum_{k,m} u_k(x) a_m \int u_k^\conj(x') u_m(x') dx' \\
&=
\sum_{k,m} u_k(x) a_m \delta_{km} \\
&=
\sum_{k} u_k(x) a_k \\
&=
f(x)
\end{aligned}
\end{equation}

Strictly speaking this ought to be formulated in terms of mean square convergence since an arbitrary function f(x)
may differ from its Fourier sum at specific points (for example at points of discontinuity).

\subsubsection{Fourier series example}

Suppose the inner product is defined for the range \(I = [a, a+T]\).

\begin{equation}\label{eqn:deltaOrthoSeries:240}
\begin{aligned}
\Innerprod{f}{g} &= \int_{\partial I} f^\conj(x) g(x) dx
\end{aligned}
\end{equation}

What is the identity operator representation in the Fourier series basis \({u'}_k(x) = e^{ 2 \pi i k x / T}\)?  First the
normalization is required.

\begin{equation}\label{eqn:deltaOrthoSeries:260}
\begin{aligned}
\Innerprod{{u'}_k}{{u'}_m}
&= \int_{\partial I} e^{ 2 \pi i (m-k) x /T } dx  \\
&= \delta_{km} T
\end{aligned}
\end{equation}

So our orthonormalized basis is

\begin{equation}\label{eqn:deltaOrthoSeries:280}
\begin{aligned}
u_k(x) = \inv{\sqrt{T}} e^{ 2 \pi i k x / T}
\end{aligned}
\end{equation}

Given this orthonormal basis we can write

\begin{equation}\label{eqn:deltaOrthoSeries:300}
\begin{aligned}
f(x)
&= \sum_k a_k u_k(x) \\
a_k &= \int_{\partial I} u_k^\conj(x) f(x) dx = \Innerprod{u_k(x)}{f(x)} \\
\end{aligned}
\end{equation}

Or in a vector like notation

\begin{equation}\label{eqn:deltaOrthoSeries:320}
\begin{aligned}
f(x) &= \sum_k u_k(x) \Innerprod{u_k(x)}{f(x)}
\end{aligned}
\end{equation}

In this basis the
delta function (identity operator) form of \eqnref{eqn:delta_ortho_series:deltaSum}
becomes

\begin{equation}\label{eqn:deltaOrthoSeries:340}
\begin{aligned}
\delta(x- x') = \inv{{T}} \sum_k e^{ 2 \pi i k (x-x') / T}
\end{aligned}
\end{equation}

\subsubsection{Fourier transform inner-product}

For the Fourier transform we have an infinite range inner product

\begin{equation}\label{eqn:deltaOrthoSeries:360}
\begin{aligned}
\Innerprod{f}{g} &= \IIinf f^\conj(x) g(x) dx
\end{aligned}
\end{equation}

With a Fourier transform pair

\begin{equation}\label{eqn:deltaOrthoSeries:380}
\begin{aligned}
\hat{f}(k) &= \frac{1}{\sqrt{2\pi}} \int f(x) e^{-i k x} dx \\
{f}(x) &= \frac{1}{\sqrt{2\pi}} \int \hat{f}(k) e^{i k x } dk \\
\end{aligned}
\end{equation}

It appears that a natural choice of basis functions is actually \(u_k\) from the
Fourier series above with \(T=2\pi\).  That is

\begin{equation}\label{eqn:deltaOrthoSeries:400}
\begin{aligned}
u_k = \inv{\sqrt{2\pi}} e^{i k x}
\end{aligned}
\end{equation}

Our Fourier coefficients are now continuous and we have a form that
is very close to the discrete Fourier series

\begin{equation}\label{eqn:deltaOrthoSeries:420}
\begin{aligned}
f(x)
&= \int dk a_k u_k \\
a_k &= \int u_k^\conj(x) f(x) dx = \Innerprod{u_k(x)}{f(x)} \\
\end{aligned}
\end{equation}

Besides the inner product range difference from the discrete frequency case
the only other difference in this formulation is that we have a
\(\sum_k \rightarrow \int dk\) replacement.

What is the delta function representation in this inner product space?

A continuous variation of the summation delta function representation
in the Fourier series basis is

\begin{equation}\label{eqn:deltaOrthoSeries:440}
\begin{aligned}
\int dk u_k^\conj(x) u_k(x')
&=
\int dk \inv{2\pi} e^{ i k (x' - x)}
\end{aligned}
\end{equation}

Okay, cool.  The principle value of this integral is the sinc function
that is the familiar limiting form of the delta function.

This is an interesting and unifying way of expressing these Fourier
relationships.  The inner product is seen here to provide a more general
structure that is common to both the Fourier series and Fourier transform.
It is not surprising that the physicists rightly pick the algebraic
orthonormal function representation as fundamental ... too bad they do it
all with the braket notation that automatically obfuscates the subject.
%(somebody like me familiar with inner product spaces still can not look at
%a QM book and without wondering what sort of drug needs to be splook
%at a only they know anything about.

This also clarifies for me what Susskind did in his QM lectures.  There
he used the identity operator representation to express the Fourier transform
without ever touching on the tricky aspects of Fourier inversion.  That is a
tricky but interesting approach.

\subsubsection{Legendre polynomials}

Let us see how one non-Fourier like inner product function space
representation works out this way.

Using the Legendre inner product

\begin{equation}\label{eqn:deltaOrthoSeries:460}
\begin{aligned}
\Innerprod{f}{g} &= \int_{-1}^1 f(x) g(x) dx
\end{aligned}
\end{equation}

An orthonormal basis can be had by normalizing the
Legendre polynomials.
\href{ http://mathworld.wolfram.com/LegendrePolynomial.html }{Wolfram's Legendre Polynomial page} lists these in a number of closed forms

\begin{equation}\label{eqn:deltaOrthoSeries:480}
\begin{aligned}
P_n(x)
&= \inv{2 \pi i} \ointctrclockwise \frac{dt}{t^{n+1}\sqrt{1 - 2t x +t^2}} \\
&= \inv{2^n}\sum_{k=0}^n {\binom{n}{k}}^2 (x-1)^{n-k}(x+1)^k
\end{aligned}
\end{equation}

The first of these uses a closed contour around the origin.

These polynomials are not orthonormal, having
\begin{equation}\label{eqn:deltaOrthoSeries:500}
\begin{aligned}
\Innerprod{P_n}{P_m} &= \frac{2}{2 n + 1}\delta_{mn}
\end{aligned}
\end{equation}

So we have an orthonormal basis if we pick
\begin{equation}\label{eqn:deltaOrthoSeries:520}
\begin{aligned}
u_n(x) &= P_n(x) \sqrt{n + 1/2}
\end{aligned}
\end{equation}

Our delta function representation in this basis becomes

\begin{equation}\label{eqn:deltaOrthoSeries:540}
\begin{aligned}
\delta(x-x')
&\sim \sum_{n=0}^\infty \left(n+ \inv{2}\right) P_n(x') P_n(x) \\
&= -\inv{4\pi^2} \sum_{n=0}^\infty \left(n+ \inv{2}\right) \ointctrclockwise \frac{du}{u^{n+1}\sqrt{1 - 2u x' + u^2}} \ointctrclockwise \frac{dt}{t^{n+1}\sqrt{1 - 2t x + t^2}} \\
&= \sum_{n=0}^\infty \sum_{m=0}^n \sum_{k=0}^n \frac{n+ \inv{2}}{2^{2n}} {\binom{n}{m}}^2 (x'-1)^{n-m}(x'+1)^m {\binom{n}{k}}^2 (x-1)^{n-k}(x+1)^k
\end{aligned}
\end{equation}

Neither of these are familiar looking to me, but I was mostly curious to see one of these delta representations for a non-Fourier-ish basis.  A number of other
orthogonal polynomials can be found detailed in \href{http://mathworld.wolfram.com/OrthogonalPolynomials.html}{Wolfram's orthogonal polynomial article.}

%\bibliographystyle{plainnat}
%\bibliography{myrefs}

%\end{document}

\documentclass{article}

\usepackage{amsmath}
\usepackage{mathpazo}

%
% shorthand for bold symbols, convenient for vectors and matrices
%
\newcommand{\Ba}[0]{\mathbf{a}}
\newcommand{\Bb}[0]{\mathbf{b}}
\newcommand{\Bc}[0]{\mathbf{c}}
\newcommand{\Bd}[0]{\mathbf{d}}
\newcommand{\Be}[0]{\mathbf{e}}
\newcommand{\Bf}[0]{\mathbf{f}}
\newcommand{\Bg}[0]{\mathbf{g}}
\newcommand{\Bh}[0]{\mathbf{h}}
\newcommand{\Bi}[0]{\mathbf{i}}
\newcommand{\Bj}[0]{\mathbf{j}}
\newcommand{\Bk}[0]{\mathbf{k}}
\newcommand{\Bl}[0]{\mathbf{l}}
\newcommand{\Bm}[0]{\mathbf{m}}
\newcommand{\Bn}[0]{\mathbf{n}}
\newcommand{\Bo}[0]{\mathbf{o}}
\newcommand{\Bp}[0]{\mathbf{p}}
\newcommand{\Bq}[0]{\mathbf{q}}
\newcommand{\Br}[0]{\mathbf{r}}
\newcommand{\Bs}[0]{\mathbf{s}}
\newcommand{\Bt}[0]{\mathbf{t}}
\newcommand{\Bu}[0]{\mathbf{u}}
\newcommand{\Bv}[0]{\mathbf{v}}
\newcommand{\Bw}[0]{\mathbf{w}}
\newcommand{\Bx}[0]{\mathbf{x}}
\newcommand{\By}[0]{\mathbf{y}}
\newcommand{\Bz}[0]{\mathbf{z}}
\newcommand{\BA}[0]{\mathbf{A}}
\newcommand{\BB}[0]{\mathbf{B}}
\newcommand{\BC}[0]{\mathbf{C}}
\newcommand{\BD}[0]{\mathbf{D}}
\newcommand{\BE}[0]{\mathbf{E}}
\newcommand{\BF}[0]{\mathbf{F}}
\newcommand{\BG}[0]{\mathbf{G}}
\newcommand{\BH}[0]{\mathbf{H}}
\newcommand{\BI}[0]{\mathbf{I}}
\newcommand{\BJ}[0]{\mathbf{J}}
\newcommand{\BK}[0]{\mathbf{K}}
\newcommand{\BL}[0]{\mathbf{L}}
\newcommand{\BM}[0]{\mathbf{M}}
\newcommand{\BN}[0]{\mathbf{N}}
\newcommand{\BO}[0]{\mathbf{O}}
\newcommand{\BP}[0]{\mathbf{P}}
\newcommand{\BQ}[0]{\mathbf{Q}}
\newcommand{\BR}[0]{\mathbf{R}}
\newcommand{\BS}[0]{\mathbf{S}}
\newcommand{\BT}[0]{\mathbf{T}}
\newcommand{\BU}[0]{\mathbf{U}}
\newcommand{\BV}[0]{\mathbf{V}}
\newcommand{\BW}[0]{\mathbf{W}}
\newcommand{\BX}[0]{\mathbf{X}}
\newcommand{\BY}[0]{\mathbf{Y}}
\newcommand{\BZ}[0]{\mathbf{Z}}

\newcommand{\Bzero}[0]{\mathbf{0}}
\newcommand{\Btheta}[0]{\boldsymbol{\theta}}
\newcommand{\Btau}[0]{\boldsymbol{\tau}}
\newcommand{\Bomega}[0]{\boldsymbol{\omega}}

%
% shorthand for unit vectors
%
\newcommand{\acap}[0]{\hat{\Ba}}
\newcommand{\bcap}[0]{\hat{\Bb}}
\newcommand{\ccap}[0]{\hat{\Bc}}
\newcommand{\dcap}[0]{\hat{\Bd}}
\newcommand{\ecap}[0]{\hat{\Be}}
\newcommand{\fcap}[0]{\hat{\Bf}}
\newcommand{\gcap}[0]{\hat{\Bg}}
\newcommand{\hcap}[0]{\hat{\Bh}}
\newcommand{\icap}[0]{\hat{\Bi}}
\newcommand{\jcap}[0]{\hat{\Bj}}
\newcommand{\kcap}[0]{\hat{\Bk}}
\newcommand{\lcap}[0]{\hat{\Bl}}
\newcommand{\mcap}[0]{\hat{\Bm}}
\newcommand{\ncap}[0]{\hat{\Bn}}
\newcommand{\ocap}[0]{\hat{\Bo}}
\newcommand{\pcap}[0]{\hat{\Bp}}
\newcommand{\qcap}[0]{\hat{\Bq}}
\newcommand{\rcap}[0]{\hat{\Br}}
\newcommand{\scap}[0]{\hat{\Bs}}
\newcommand{\tcap}[0]{\hat{\Bt}}
\newcommand{\ucap}[0]{\hat{\Bu}}
\newcommand{\vcap}[0]{\hat{\Bv}}
\newcommand{\wcap}[0]{\hat{\Bw}}
\newcommand{\xcap}[0]{\hat{\Bx}}
\newcommand{\ycap}[0]{\hat{\By}}
\newcommand{\zcap}[0]{\hat{\Bz}}
\newcommand{\thetacap}[0]{\hat{\Btheta}}

%
% to write R^n and C^n in a distinguishable fashion.  Perhaps change this
% to the double lined characters upon figuring out how to do so.
%
\newcommand{\C}[1]{$\mathbb{C}^{#1}$}
\newcommand{\R}[1]{$\mathbb{R}^{#1}$}

%
% various generally useful helpers
%

% derivative of #1 wrt. #2:
\newcommand{\D}[2] {\frac {d#2} {d#1}}

\newcommand{\inv}[1]{\frac{1}{#1}}
\newcommand{\cross}[0]{\times}

\newcommand{\abs}[1]{\lvert{#1}\rvert}
\newcommand{\norm}[1]{\lVert{#1}\rVert}
\newcommand{\innerprod}[2]{\langle{#1}, {#2}\rangle}
\newcommand{\dotprod}[2]{{#1} \cdot {#2}}
\newcommand{\bdotprod}[2]{\left({#1} \cdot {#2}\right)}
\newcommand{\crossprod}[2]{{#1} \cross {#2}}
\newcommand{\tripleprod}[3]{\dotprod{\left(\crossprod{#1}{#2}\right)}{#3}}

\DeclareMathOperator{\Proj}{Proj}
\DeclareMathOperator{\Span}{span}
\DeclareMathOperator{\Sgn}{sgn}
\DeclareMathOperator{\Area}{Area}
\DeclareMathOperator{\Volume}{Volume}

%
% A few miscellaneous things specific to this document
%
\newcommand{\crossop}[1]{\crossprod{#1}{}}

% R2 vector.
\newcommand{\VectorTwo}[2]{
\begin{bmatrix}
 {#1} \\
 {#2}
\end{bmatrix}
}

\newcommand{\VectorN}[1]{
\begin{bmatrix}
{#1}_1 \\
{#1}_2 \\
\vdots \\
{#1}_N \\
\end{bmatrix}
}

\newcommand{\DETuvij}[4]{
\begin{vmatrix}
 {#1}_{#3} & {#1}_{#4} \\
 {#2}_{#3} & {#2}_{#4}
\end{vmatrix}
}

\newcommand{\DETuvwijk}[6]{
\begin{vmatrix}
 {#1}_{#4} & {#1}_{#5} & {#1}_{#6} \\
 {#2}_{#4} & {#2}_{#5} & {#2}_{#6} \\
 {#3}_{#4} & {#3}_{#5} & {#3}_{#6}
\end{vmatrix}
}

\newcommand{\DETuvwxijkl}[8]{
\begin{vmatrix}
 {#1}_{#5} & {#1}_{#6} & {#1}_{#7} & {#1}_{#8} \\
 {#2}_{#5} & {#2}_{#6} & {#2}_{#7} & {#2}_{#8} \\
 {#3}_{#5} & {#3}_{#6} & {#3}_{#7} & {#3}_{#8} \\
 {#4}_{#5} & {#4}_{#6} & {#4}_{#7} & {#4}_{#8} \\
\end{vmatrix}
}

%\newcommand{\DETuvwxyijklm}[10]{
%\begin{vmatrix}
% {#1}_{#6} & {#1}_{#7} & {#1}_{#8} & {#1}_{#9} & {#1}_{#10} \\
% {#2}_{#6} & {#2}_{#7} & {#2}_{#8} & {#2}_{#9} & {#2}_{#10} \\
% {#3}_{#6} & {#3}_{#7} & {#3}_{#8} & {#3}_{#9} & {#3}_{#10} \\
% {#4}_{#6} & {#4}_{#7} & {#4}_{#8} & {#4}_{#9} & {#4}_{#10} \\
% {#5}_{#6} & {#5}_{#7} & {#5}_{#8} & {#5}_{#9} & {#5}_{#10}
%\end{vmatrix}
%}

% R3 vector.
\newcommand{\VectorThree}[3]{
\begin{bmatrix}
 {#1} \\
 {#2} \\
 {#3}
\end{bmatrix}
}


%<misc>
%
\newcommand{\Abs}[1]{{\left\lvert{#1}\right\rvert}}
\newcommand{\spacegrad}[0]{\boldsymbol{\nabla}}
\newcommand{\grad}[0]{\nabla}
\newcommand{\LL}[0]{\mathcal{L}}

% == \partial_{#1} {#2}
\newcommand{\PD}[2]{\frac{\partial {#2}}{\partial {#1}}}
% inline variant
\newcommand{\PDi}[2]{{\partial {#2}}/{\partial {#1}}}

\newcommand{\PDD}[3]{\frac{\partial^2 {#3}}{\partial {#1}\partial {#2}}}
%\newcommand{\PDd}[2]{\frac{\partial^2 {#2}}{{\partial{#1}}^2}}
\newcommand{\PDsq}[2]{\frac{\partial^2 {#2}}{(\partial {#1})^2}}

\newcommand{\Partial}[2]{\frac{\partial {#1}}{\partial {#2}}}
\DeclareMathOperator{\RejName}{Rej}
\newcommand{\Rej}[2]{\RejName_{#1}\left( {#2} \right)}
\newcommand{\Rm}[1]{\mathbb{R}^{#1}}
\newcommand{\Cm}[1]{\mathbb{C}^{#1}}
\newcommand{\conj}[0]{{*}}

%</misc>

% <grade selection>
%
\newcommand{\gpgrade}[2] {{\left\langle{{#1}}\right\rangle}_{#2}}

\newcommand{\gpgradezero}[1] {\gpgrade{#1}{}}
%\newcommand{\gpscalargrade}[1] {{\left\langle{{#1}}\right\rangle}}
%\newcommand{\gpgradezero}[1] {\gpgrade{#1}{0}}

%\newcommand{\gpgradeone}[1] {{\left\langle{{#1}}\right\rangle}_{1}}
\newcommand{\gpgradeone}[1] {\gpgrade{#1}{1}}

\newcommand{\gpgradetwo}[1] {\gpgrade{#1}{2}}
\newcommand{\gpgradethree}[1] {\gpgrade{#1}{3}}
\newcommand{\gpgradefour}[1] {\gpgrade{#1}{4}}
%
% </grade selection>



\newcommand{\adot}[0]{{\dot{a}}}
\newcommand{\bdot}[0]{{\dot{b}}}
% taken for centered dot:
%\newcommand{\cdot}[0]{{\dot{c}}}
%\newcommand{\ddot}[0]{{\dot{d}}}
\newcommand{\edot}[0]{{\dot{e}}}
\newcommand{\fdot}[0]{{\dot{f}}}
\newcommand{\gdot}[0]{{\dot{g}}}
\newcommand{\hdot}[0]{{\dot{h}}}
\newcommand{\idot}[0]{{\dot{i}}}
\newcommand{\jdot}[0]{{\dot{j}}}
\newcommand{\kdot}[0]{{\dot{k}}}
\newcommand{\ldot}[0]{{\dot{l}}}
\newcommand{\mdot}[0]{{\dot{m}}}
\newcommand{\ndot}[0]{{\dot{n}}}
%\newcommand{\odot}[0]{{\dot{o}}}
\newcommand{\pdot}[0]{{\dot{p}}}
\newcommand{\qdot}[0]{{\dot{q}}}
\newcommand{\rdot}[0]{{\dot{r}}}
\newcommand{\sdot}[0]{{\dot{s}}}
\newcommand{\tdot}[0]{{\dot{t}}}
\newcommand{\udot}[0]{{\dot{u}}}
\newcommand{\vdot}[0]{{\dot{v}}}
\newcommand{\wdot}[0]{{\dot{w}}}
\newcommand{\xdot}[0]{{\dot{x}}}
\newcommand{\ydot}[0]{{\dot{y}}}
\newcommand{\zdot}[0]{{\dot{z}}}
\newcommand{\addot}[0]{{\ddot{a}}}
\newcommand{\bddot}[0]{{\ddot{b}}}
\newcommand{\cddot}[0]{{\ddot{c}}}
%\newcommand{\dddot}[0]{{\ddot{d}}}
\newcommand{\eddot}[0]{{\ddot{e}}}
\newcommand{\fddot}[0]{{\ddot{f}}}
\newcommand{\gddot}[0]{{\ddot{g}}}
\newcommand{\hddot}[0]{{\ddot{h}}}
\newcommand{\iddot}[0]{{\ddot{i}}}
\newcommand{\jddot}[0]{{\ddot{j}}}
\newcommand{\kddot}[0]{{\ddot{k}}}
\newcommand{\lddot}[0]{{\ddot{l}}}
\newcommand{\mddot}[0]{{\ddot{m}}}
\newcommand{\nddot}[0]{{\ddot{n}}}
\newcommand{\oddot}[0]{{\ddot{o}}}
\newcommand{\pddot}[0]{{\ddot{p}}}
\newcommand{\qddot}[0]{{\ddot{q}}}
\newcommand{\rddot}[0]{{\ddot{r}}}
\newcommand{\sddot}[0]{{\ddot{s}}}
\newcommand{\tddot}[0]{{\ddot{t}}}
\newcommand{\uddot}[0]{{\ddot{u}}}
\newcommand{\vddot}[0]{{\ddot{v}}}
\newcommand{\wddot}[0]{{\ddot{w}}}
\newcommand{\xddot}[0]{{\ddot{x}}}
\newcommand{\yddot}[0]{{\ddot{y}}}
\newcommand{\zddot}[0]{{\ddot{z}}}

%<bold and dot greek symbols>
%

\newcommand{\Deltadot}[0]{{\dot{\Delta}}}
\newcommand{\Gammadot}[0]{{\dot{\Gamma}}}
\newcommand{\Lambdadot}[0]{{\dot{\Lambda}}}
\newcommand{\Omegadot}[0]{{\dot{\Omega}}}
\newcommand{\Phidot}[0]{{\dot{\Phi}}}
\newcommand{\Pidot}[0]{{\dot{\Pi}}}
\newcommand{\Psidot}[0]{{\dot{\Psi}}}
\newcommand{\Sigmadot}[0]{{\dot{\Sigma}}}
\newcommand{\Thetadot}[0]{{\dot{\Theta}}}
\newcommand{\Upsilondot}[0]{{\dot{\Upsilon}}}
\newcommand{\Xidot}[0]{{\dot{\Xi}}}
\newcommand{\alphadot}[0]{{\dot{\alpha}}}
\newcommand{\betadot}[0]{{\dot{\beta}}}
\newcommand{\chidot}[0]{{\dot{\chi}}}
\newcommand{\deltadot}[0]{{\dot{\delta}}}
\newcommand{\epsilondot}[0]{{\dot{\epsilon}}}
\newcommand{\etadot}[0]{{\dot{\eta}}}
\newcommand{\gammadot}[0]{{\dot{\gamma}}}
\newcommand{\kappadot}[0]{{\dot{\kappa}}}
\newcommand{\lambdadot}[0]{{\dot{\lambda}}}
\newcommand{\mudot}[0]{{\dot{\mu}}}
\newcommand{\nudot}[0]{{\dot{\nu}}}
\newcommand{\omegadot}[0]{{\dot{\omega}}}
\newcommand{\phidot}[0]{{\dot{\phi}}}
\newcommand{\pidot}[0]{{\dot{\pi}}}
\newcommand{\psidot}[0]{{\dot{\psi}}}
\newcommand{\rhodot}[0]{{\dot{\rho}}}
\newcommand{\sigmadot}[0]{{\dot{\sigma}}}
\newcommand{\taudot}[0]{{\dot{\tau}}}
\newcommand{\thetadot}[0]{{\dot{\theta}}}
\newcommand{\upsilondot}[0]{{\dot{\upsilon}}}
\newcommand{\varepsilondot}[0]{{\dot{\varepsilon}}}
\newcommand{\varphidot}[0]{{\dot{\varphi}}}
\newcommand{\varpidot}[0]{{\dot{\varpi}}}
\newcommand{\varrhodot}[0]{{\dot{\varrho}}}
\newcommand{\varsigmadot}[0]{{\dot{\varsigma}}}
\newcommand{\varthetadot}[0]{{\dot{\vartheta}}}
\newcommand{\xidot}[0]{{\dot{\xi}}}
\newcommand{\zetadot}[0]{{\dot{\zeta}}}

\newcommand{\Deltaddot}[0]{{\ddot{\Delta}}}
\newcommand{\Gammaddot}[0]{{\ddot{\Gamma}}}
\newcommand{\Lambdaddot}[0]{{\ddot{\Lambda}}}
\newcommand{\Omegaddot}[0]{{\ddot{\Omega}}}
\newcommand{\Phiddot}[0]{{\ddot{\Phi}}}
\newcommand{\Piddot}[0]{{\ddot{\Pi}}}
\newcommand{\Psiddot}[0]{{\ddot{\Psi}}}
\newcommand{\Sigmaddot}[0]{{\ddot{\Sigma}}}
\newcommand{\Thetaddot}[0]{{\ddot{\Theta}}}
\newcommand{\Upsilonddot}[0]{{\ddot{\Upsilon}}}
\newcommand{\Xiddot}[0]{{\ddot{\Xi}}}
\newcommand{\alphaddot}[0]{{\ddot{\alpha}}}
\newcommand{\betaddot}[0]{{\ddot{\beta}}}
\newcommand{\chiddot}[0]{{\ddot{\chi}}}
\newcommand{\deltaddot}[0]{{\ddot{\delta}}}
\newcommand{\epsilonddot}[0]{{\ddot{\epsilon}}}
\newcommand{\etaddot}[0]{{\ddot{\eta}}}
\newcommand{\gammaddot}[0]{{\ddot{\gamma}}}
\newcommand{\kappaddot}[0]{{\ddot{\kappa}}}
\newcommand{\lambdaddot}[0]{{\ddot{\lambda}}}
\newcommand{\muddot}[0]{{\ddot{\mu}}}
\newcommand{\nuddot}[0]{{\ddot{\nu}}}
\newcommand{\omegaddot}[0]{{\ddot{\omega}}}
\newcommand{\phiddot}[0]{{\ddot{\phi}}}
\newcommand{\piddot}[0]{{\ddot{\pi}}}
\newcommand{\psiddot}[0]{{\ddot{\psi}}}
\newcommand{\rhoddot}[0]{{\ddot{\rho}}}
\newcommand{\sigmaddot}[0]{{\ddot{\sigma}}}
\newcommand{\tauddot}[0]{{\ddot{\tau}}}
\newcommand{\thetaddot}[0]{{\ddot{\theta}}}
\newcommand{\upsilonddot}[0]{{\ddot{\upsilon}}}
\newcommand{\varepsilonddot}[0]{{\ddot{\varepsilon}}}
\newcommand{\varphiddot}[0]{{\ddot{\varphi}}}
\newcommand{\varpiddot}[0]{{\ddot{\varpi}}}
\newcommand{\varrhoddot}[0]{{\ddot{\varrho}}}
\newcommand{\varsigmaddot}[0]{{\ddot{\varsigma}}}
\newcommand{\varthetaddot}[0]{{\ddot{\vartheta}}}
\newcommand{\xiddot}[0]{{\ddot{\xi}}}
\newcommand{\zetaddot}[0]{{\ddot{\zeta}}}

\newcommand{\BDelta}[0]{\boldsymbol{\Delta}}
\newcommand{\BGamma}[0]{\boldsymbol{\Gamma}}
\newcommand{\BLambda}[0]{\boldsymbol{\Lambda}}
\newcommand{\BOmega}[0]{\boldsymbol{\Omega}}
\newcommand{\BPhi}[0]{\boldsymbol{\Phi}}
\newcommand{\BPi}[0]{\boldsymbol{\Pi}}
\newcommand{\BPsi}[0]{\boldsymbol{\Psi}}
\newcommand{\BSigma}[0]{\boldsymbol{\Sigma}}
\newcommand{\BTheta}[0]{\boldsymbol{\Theta}}
\newcommand{\BUpsilon}[0]{\boldsymbol{\Upsilon}}
\newcommand{\BXi}[0]{\boldsymbol{\Xi}}
\newcommand{\Balpha}[0]{\boldsymbol{\alpha}}
\newcommand{\Bbeta}[0]{\boldsymbol{\beta}}
\newcommand{\Bchi}[0]{\boldsymbol{\chi}}
\newcommand{\Bdelta}[0]{\boldsymbol{\delta}}
\newcommand{\Bepsilon}[0]{\boldsymbol{\epsilon}}
\newcommand{\Beta}[0]{\boldsymbol{\eta}}
\newcommand{\Bgamma}[0]{\boldsymbol{\gamma}}
\newcommand{\Bkappa}[0]{\boldsymbol{\kappa}}
\newcommand{\Blambda}[0]{\boldsymbol{\lambda}}
\newcommand{\Bmu}[0]{\boldsymbol{\mu}}
\newcommand{\Bnu}[0]{\boldsymbol{\nu}}
%\newcommand{\Bomega}[0]{\boldsymbol{\omega}}
\newcommand{\Bphi}[0]{\boldsymbol{\phi}}
\newcommand{\Bpi}[0]{\boldsymbol{\pi}}
\newcommand{\Bpsi}[0]{\boldsymbol{\psi}}
\newcommand{\Brho}[0]{\boldsymbol{\rho}}
\newcommand{\Bsigma}[0]{\boldsymbol{\sigma}}
%\newcommand{\Btau}[0]{\boldsymbol{\tau}}
%\newcommand{\Btheta}[0]{\boldsymbol{\theta}}
\newcommand{\Bupsilon}[0]{\boldsymbol{\upsilon}}
\newcommand{\Bvarepsilon}[0]{\boldsymbol{\varepsilon}}
\newcommand{\Bvarphi}[0]{\boldsymbol{\varphi}}
\newcommand{\Bvarpi}[0]{\boldsymbol{\varpi}}
\newcommand{\Bvarrho}[0]{\boldsymbol{\varrho}}
\newcommand{\Bvarsigma}[0]{\boldsymbol{\varsigma}}
\newcommand{\Bvartheta}[0]{\boldsymbol{\vartheta}}
\newcommand{\Bxi}[0]{\boldsymbol{\xi}}
\newcommand{\Bzeta}[0]{\boldsymbol{\zeta}}
%
%</bold and dot greek symbols>
%<infrequent>
%
%\newcommand{\AreaOp}[1]{\AName_{#1}}
%\newcommand{\Babs}[0]{\abs{\BB}}
%\newcommand{\Bcap}[0]{\hat{\BB}}
%\newcommand{\BrPrimeRej}[0]{\rcap(\rcap \wedge \Br')}
%\newcommand{\CA}[0]{\mathcal{A}}
%\newcommand{\Cos}[1]{\cos{\left({#1}\right)}}
%\newcommand{\Det}[1] {\abs{#1}}
%\newcommand{\Dsq}[2] {\frac {\partial^2 {#1}} {\partial {#2}^2}}
%\newcommand{\Exp}[1]{\exp{\left({#1}\right)}}
%\newcommand{\Norm}[1]{\left\lVert{#1}\right\rVert}
%\newcommand{\Sin}[1]{\sin{\left({#1}\right)}}
%\newcommand{\T}[0]{\text{T}}
%\newcommand{\VolumeOp}[1]{\VName_{#1}}
%\newcommand{\agrad}[0]{\Ba \cdot \nabla}
%\newcommand{\alphacap}[0]{\hat{\boldsymbol{\alpha}}}
%\newcommand{\Fcap}[0]{\hat{\BF}}
%\newcommand{\bithree}[0]{{\Bi}_3}
%\newcommand{\bxa}[0]{\Bx\Ba}
%\newcommand{\coordvec}[2]{
%\newcommand{\costheta}[0]{\acap \cdot \xcap}
%\newcommand{\ddt}[1]{\ddot{#1}}
%\newcommand{\ddu}[1] {\frac {d{#1}} {du}}
%\newcommand{\dsqxj}[2] {\frac {\partial^2 {#1}} {\partial {x_{#2}}^2}}
%\newcommand{\dtheta}[1]{\frac{d {#1}}{d \theta}}
%\newcommand{\dt}[1]{\dot{#1}}
%\newcommand{\dt}[1]{\frac{d {#1}}{dt}}
%\newcommand{\dxj}[2] {\frac {\partial {#1}} {\partial {x_{#2}}}}
%\newcommand{\halfPhi}[0]{\frac{\phi}{2}}
%\newcommand{\half}[0]{\inv{2}}
%\newcommand{\inv}[1]{\frac{1}{#1}}
%\newcommand{\laplacian}[0]{\nabla^2}
%\newcommand{\matrixoftx}[3]{
%\newcommand{\nrrp}[0]{\norm{\rcap \wedge \Br'}}
%\newcommand{\oiint}{\bigcirc \hspace{-1.4em} \int \hspace{-.8em} \int}
%\newcommand{\transpose}[1]{{#1}^{\text{T}}}
%\newcommand{\transpose}[1]{{{#1}^{\TextTranspose}}}
%\newcommand{\transpose}[1]{{{#1}^{\text{T}}}}
%\newcommand{\barA}[0]{\bar{A}}
%\newcommand{\qbar}[0]{\bar{q}}
%\newcommand{\qdotbar}[0]{\dot{\bar{q}}}
%
%</infrequent>




\newcommand{\PDSq}[2]{\frac{\partial^2 {#2}}{\partial {#1}^2}}

\usepackage[bookmarks=true]{hyperref}

\usepackage{color,cite,graphicx}
   % use colour in the document, put your citations as [1-4]
   % rather than [1,2,3,4] (it looks nicer, and the extended LaTeX2e
   % graphics package. 
\usepackage{latexsym,amssymb,epsf} % don't remember if these are
   % needed, but their inclusion can't do any damage

\title{ Ehrenfest's theorem. }
\author{Peeter Joot}
\date{ Jan 22, 2009.  Last Revision: $Date: 2009/01/24 02:41:28 $ }

\begin{document}

\maketitle{}

\tableofcontents
\section{ Motivation. }

\cite{mcmahon2005qmd} has a one dimensional treatment of Ehrenfest's theorem,
that the expectation values of the position and momentum operators behave
like Newton's law.

However, he makes use
of commutator and braket notation before either is defined.

That looks like a natural way to do the derivation easily, but let's try
this using instead what is defined up to this point in the text.

\section{ Review.  What do we know so far? }

\subsection{ Position and momentum operators. }

We have been given the definitions of two specific operators, position and momentum, 
whos action on a wave function is

\begin{align*}
\hat{x} \psi &= x \psi \\
\hat{p} \psi &= \left(-i \hbar \PD{x}{}\right) \psi
\end{align*}

In operator form, with the omission of the explicit wave function being operated on this is

\begin{align*}
\hat{x} &\equiv x  \\
\hat{p} &\equiv -i \hbar \PD{x}{}
\end{align*}

These are perfectly valid operator definitions, but the validity of using the 
classical names for these really comes from this upcoming Ehrenfest result where
the average of the action of these operators on a wave function is examined.

\subsection{ Expectation (average) value of an operator. }

We also have a definition for the expectation value
of an operator $\hat{A}$, given its specific action $A$.
This is defined very much like a weighted inner product
and is essentially a field weighted average of the operators action

\begin{align*}
<\hat{A}> \equiv \int \psi^\conj (A \psi)
\end{align*}

The braces show that the operator action $A$ here applies to the rightmost field variable $\psi$, and not to its conjugate.

For the position and momentum operators respectively, we have the
expectation values

\begin{align*}
<\hat{x}> &\equiv \int \psi^\conj (x \psi) \\
<\hat{p}> &\equiv \int \psi^\conj \left(-i \hbar \PD{x}{}\right) \psi
\end{align*}

\subsection{ Hermitian operator. }

The notation of a Hermitian operator as also been introduced in terms of 
left acting operators.  That is, an operator $\hat{A}$ is hermitian if

\begin{align}\label{eqn:hermitian1}
\int \psi^\conj (A \psi) = \int (\psi A)^\conj \psi
\end{align}

This is a somewhat non-Demystified seeming definition to me since I'd seen
Hermitian defined more directly in terms of ``normal'' right acting 
expectation integrals.  That is, an operator $\hat{A}$ is Hermitian if

\begin{align*}
<\hat{A}>^\conj = <\hat{A}>
\end{align*}

The conjugate of an operator's expectation value is
\begin{align*}
\left(\int \psi^\conj (A \psi)\right)^\conj 
&= \int \psi (A^\conj \psi^\conj) \\
&= \int (A^\conj \psi^\conj) \psi \\
\end{align*}

So, this second Hermitian definition means that an operator is Hermitian if

\begin{align*}
\int (A^\conj \psi^\conj) \psi &= \int \psi^\conj (A \psi)
\end{align*}

This highlights why the left acting operator notation is pretty reasonable
seeming.  Allowing the conjugation operation to switch an operators action
from right acting to left acting makes the equation prettier, and 
recovers equation \ref{eqn:hermitian1}

\begin{align*}
(\psi A)^\conj \equiv (A^\conj \psi^\conj) 
\end{align*}

Here braces have been used to express the limitation of the scope of the action of the operator.

Another way to express this is that one can say that a Hermitian operator when put 
in its wave function sandwhich has a 
conjugate action acting to the left on the conjugate wave function and a non-conjugate
action to the right.  This allows for a final notation nicety, where one can omit the 
braces entirely as in

\begin{align*}
\int \psi^\conj (A \psi) \equiv \int \psi^\conj A \psi \equiv \int (A^\conj \psi^\conj) \psi
\end{align*}
or in terms of right and left operator notation the equivalent
\begin{align*}
\int \psi^\conj (A \psi) \equiv \int \psi^\conj A \psi \equiv \int (\psi A)^\conj \psi
\end{align*}

And finally, there is one last way to express this the concept of Hermitian.
We have our definition of a left acting operator

\begin{align*}
(\psi A)^\conj = A^\conj \psi^\conj
\end{align*}

And can make the observation that conjugation of a product is the 
product of the conjugates
\begin{align*}
(\psi A)^\conj = \psi^\conj A^\conj
\end{align*}

So we must also have $A = A^\conj$ for a Hermitian operator.  

From this one can observe that the position operator $\hat{x}$ is Hermitian, but the momentum operator is not (but $\hat{p}^2$ is ).

\subsection{ Variance and Heisenberg principle. }

Various calculations have been done to calcualate expectation values.

In a few places we have had to show that the product of variances

\begin{align*}
\Delta A = \sqrt{<A^2> - <A>^2}
\end{align*}

for position and momentum all satisfy the famous Heisenberg uncertainty
principle

\begin{align*}
\Delta x \Delta p \ge \hbar/2
\end{align*}

(in a couple places this formulation is a bit fuzzy since our squared
momentum variance $(\Delta p)^2$ has been negative).

\subsection{ The wave equation. }

We are also given Schr\"{o}dinger equation in Hamiltonian form

\begin{align*}
\hat{H} \psi = i \hbar \PD{t}{\psi}
\end{align*}

and have worked with the specific form of the Hamiltonian that applies to
a non-relativistic particle (and not to photons).

\begin{align*}
\hat{H} = \frac{\hat{p}^2}{2m} + V = -\frac{\hbar^2}{2m} \grad^2 + V
\end{align*}

Most of the text up to this point has been about calculating and interpretting
specific solutions of this equation.

\subsection{ Other stuff. }

A number of other fundamental topics have been covered, probabilities, normalization, probability current, energy, phase, orthogonality, and so forth.  However, summarizing the rest of these in detail is not required as 
background for the Ehrenfest result.

\section{ Ehrenfest theorem. }

We want to calculate the time derivatives of the expectation values
for position and momentum OPERATORS, and show that these reproduce the
familiar velocity, momentum and force concepts from classical mechanics.

\subsection{ Velocity from the derivative of the position operator expectation. }

Diving straight in we have

\begin{align*}
\PD{t}{<\hat{x}>}
&= \PD{t}{} \left( \int \psi^\conj x \psi \right) \\
&= \int \PD{t}{\psi^\conj} x \psi + \int \psi^\conj x \PD{t}{\psi} 
\end{align*}

Now, here the Hamiltonian can be introduced, replacing the time derivatives.

We have
\begin{align*}
\PD{t}{\psi} &= -\frac{i}{\hbar} H \psi \\
\PD{t}{\psi^\conj} &= \frac{i}{\hbar} H \psi^\conj \\
\end{align*}

So we have
\begin{align*}
\PD{t}{<\hat{x}>}
&= \frac{i}{\hbar} \int \psi x H {\psi^\conj} - \frac{i}{\hbar}\int \psi^\conj x H {\psi} 
\end{align*}

For the Schr\"{o}dinger Hamiltonian we have

\begin{align*}
H \psi &= - \frac{\hbar^2}{2m} \PDSq{x}{\psi} + V\psi \\
H \psi^\conj &= - \frac{\hbar^2}{2m} \PDSq{x}{\psi^\conj} + V\psi^\conj \\
\end{align*}

Combining these we have
\begin{align*}
\PD{t}{<\hat{x}>}
&= 
\frac{i}{\hbar} \int \psi x \left( - \frac{\hbar^2}{2m} \PDSq{x}{\psi^\conj} + V\psi^\conj \right) 
-\frac{i}{\hbar} \int \psi^\conj x \left(- \frac{\hbar^2}{2m} \PDSq{x}{\psi} + V\psi \right) \\
&= 
\frac{i\hbar}{2m} \int \left(\psi^\conj x \PDSq{x}{\psi} -\psi x \PDSq{x}{\psi^\conj} \right)
+\frac{i}{\hbar} \int \left(\psi x V\psi^\conj -\psi^\conj x V\psi \right) 
\\
\end{align*}

The second term is zero, and by integrating the first term by parts twice we have

\begin{align*}
\PD{t}{<\hat{x}>}
&= \frac{i\hbar}{2m} \int \psi^\conj \left(x \PDSq{x}{\psi} - \PDSq{x}{(\psi x)} \right) \\
&= \frac{i\hbar}{2m} \int \psi^\conj \left(x \PDSq{x}{\psi} - \PD{x}{}\left(x \PD{x}{\psi} + \psi\right) \right) \\
&= \frac{-i\hbar}{2m} (2) \int \psi^\conj \PD{x}{\psi} \\
&= \frac{1}{m} \int \psi^\conj \left(-i\hbar \PD{x}{} \right) {\psi} \\
\end{align*}

So we now have the QM equivalent of $p = mv$, directly from the Sch\"{o}dinger equation and the definition of expectation values
of operators.

\begin{align}
\PD{t}{<\hat{x}>} &= \frac{<p>}{m} 
\end{align}

This is the first inkling that it makes sense to assign the names position and momentum to the corresponding operators
of QM!  Now the QMD derivation is way shorter and tidier, but this needed only integration by parts.  We really don't
need the more advanced operator concepts to get this important result.  

\subsection{ Force from the derivative of the momentum operator expectation. }

Now lets calculate the momentum expectation change with time.

\begin{align*}
\PD{t}{<p>} 
&= \PD{t}{} \int \psi^\conj \left(-i \hbar \PD{x}{}\right) \psi \\
&= -i \hbar \int \PD{t}{\psi^\conj} \PD{x}{\psi} +{\psi^\conj} \PD{t}{}\PD{x}{\psi} \\
&= -i \hbar \int \PD{t}{\psi^\conj} \PD{x}{\psi} +{\psi^\conj} \PD{x}{}\PD{t}{\psi} \\
&= \int \PD{x}{\psi} H \psi^\conj - {\psi^\conj} \PD{x}{} H\psi \\
&= \int \PD{x}{\psi} H \psi^\conj + \PD{x}{\psi^\conj} H\psi \\
&= 
\int \PD{x}{\psi} \left(- \frac{\hbar^2}{2m} \PDSq{x}{\psi^\conj} + V\psi^\conj \right)
+ \PD{x}{\psi^\conj} \left(- \frac{\hbar^2}{2m} \PDSq{x}{\psi} + V\psi \right) 
\\
&= 
-
\frac{\hbar^2}{2m} 
\int \PD{x}{\psi} \PDSq{x}{\psi^\conj} + \PD{x}{\psi^\conj} \PDSq{x}{\psi} 
+\int \PD{x}{\psi} V\psi^\conj + \PD{x}{\psi^\conj} V\psi 
\\
&= 
- \frac{\hbar^2}{2m} \int \PD{x}{}\left(\PD{x}{\psi} \PD{x}{\psi^\conj}\right)
+\int \PD{x}{} \left( \psi V\psi^\conj \right) -\int \psi \PD{x}{V} \psi^\conj 
\\
\end{align*}

Now, again with the assumption that $\psi$ and its derivatives are sufficiently small to vanish at the boundaries of the integration (this was also done in the integration by parts above), the first two terms are zero, and the last is an expectation value.  Specifically, we then have

\begin{align}
\PD{t}{<p>} &= - \left<\PD{x}{V}\right>
\end{align}

... which appears to be the QM equivalent to the one dimensional version of $F = -\grad V$, instead all defined in terms of expectation values.

Very cool!  Now, before learning the Lagrangian formalism, I would have been satisfied with this.  We can replace Newton's law with 
Schr\"{o}dinger's equation, and logically everything else will follow from that.  Can we apply a procedure like this to 
the Lagrangian for the wave equation, and find an expectation equivalent to the classical $\LL = m\Bv^2/2 - V$?

An additional obvious question is how to express the expectation value in the three dimensional case instead of the one dimensional case?

\bibliographystyle{plainnat}
\bibliography{myrefs}

\end{document}

%\documentclass{article}

%\usepackage{amsmath}
\usepackage{mathpazo}

%
% shorthand for bold symbols, convenient for vectors and matrices
%
\newcommand{\Ba}[0]{\mathbf{a}}
\newcommand{\Bb}[0]{\mathbf{b}}
\newcommand{\Bc}[0]{\mathbf{c}}
\newcommand{\Bd}[0]{\mathbf{d}}
\newcommand{\Be}[0]{\mathbf{e}}
\newcommand{\Bf}[0]{\mathbf{f}}
\newcommand{\Bg}[0]{\mathbf{g}}
\newcommand{\Bh}[0]{\mathbf{h}}
\newcommand{\Bi}[0]{\mathbf{i}}
\newcommand{\Bj}[0]{\mathbf{j}}
\newcommand{\Bk}[0]{\mathbf{k}}
\newcommand{\Bl}[0]{\mathbf{l}}
\newcommand{\Bm}[0]{\mathbf{m}}
\newcommand{\Bn}[0]{\mathbf{n}}
\newcommand{\Bo}[0]{\mathbf{o}}
\newcommand{\Bp}[0]{\mathbf{p}}
\newcommand{\Bq}[0]{\mathbf{q}}
\newcommand{\Br}[0]{\mathbf{r}}
\newcommand{\Bs}[0]{\mathbf{s}}
\newcommand{\Bt}[0]{\mathbf{t}}
\newcommand{\Bu}[0]{\mathbf{u}}
\newcommand{\Bv}[0]{\mathbf{v}}
\newcommand{\Bw}[0]{\mathbf{w}}
\newcommand{\Bx}[0]{\mathbf{x}}
\newcommand{\By}[0]{\mathbf{y}}
\newcommand{\Bz}[0]{\mathbf{z}}
\newcommand{\BA}[0]{\mathbf{A}}
\newcommand{\BB}[0]{\mathbf{B}}
\newcommand{\BC}[0]{\mathbf{C}}
\newcommand{\BD}[0]{\mathbf{D}}
\newcommand{\BE}[0]{\mathbf{E}}
\newcommand{\BF}[0]{\mathbf{F}}
\newcommand{\BG}[0]{\mathbf{G}}
\newcommand{\BH}[0]{\mathbf{H}}
\newcommand{\BI}[0]{\mathbf{I}}
\newcommand{\BJ}[0]{\mathbf{J}}
\newcommand{\BK}[0]{\mathbf{K}}
\newcommand{\BL}[0]{\mathbf{L}}
\newcommand{\BM}[0]{\mathbf{M}}
\newcommand{\BN}[0]{\mathbf{N}}
\newcommand{\BO}[0]{\mathbf{O}}
\newcommand{\BP}[0]{\mathbf{P}}
\newcommand{\BQ}[0]{\mathbf{Q}}
\newcommand{\BR}[0]{\mathbf{R}}
\newcommand{\BS}[0]{\mathbf{S}}
\newcommand{\BT}[0]{\mathbf{T}}
\newcommand{\BU}[0]{\mathbf{U}}
\newcommand{\BV}[0]{\mathbf{V}}
\newcommand{\BW}[0]{\mathbf{W}}
\newcommand{\BX}[0]{\mathbf{X}}
\newcommand{\BY}[0]{\mathbf{Y}}
\newcommand{\BZ}[0]{\mathbf{Z}}

\newcommand{\Bzero}[0]{\mathbf{0}}
\newcommand{\Btheta}[0]{\boldsymbol{\theta}}
\newcommand{\Btau}[0]{\boldsymbol{\tau}}
\newcommand{\Bomega}[0]{\boldsymbol{\omega}}

%
% shorthand for unit vectors
%
\newcommand{\acap}[0]{\hat{\Ba}}
\newcommand{\bcap}[0]{\hat{\Bb}}
\newcommand{\ccap}[0]{\hat{\Bc}}
\newcommand{\dcap}[0]{\hat{\Bd}}
\newcommand{\ecap}[0]{\hat{\Be}}
\newcommand{\fcap}[0]{\hat{\Bf}}
\newcommand{\gcap}[0]{\hat{\Bg}}
\newcommand{\hcap}[0]{\hat{\Bh}}
\newcommand{\icap}[0]{\hat{\Bi}}
\newcommand{\jcap}[0]{\hat{\Bj}}
\newcommand{\kcap}[0]{\hat{\Bk}}
\newcommand{\lcap}[0]{\hat{\Bl}}
\newcommand{\mcap}[0]{\hat{\Bm}}
\newcommand{\ncap}[0]{\hat{\Bn}}
\newcommand{\ocap}[0]{\hat{\Bo}}
\newcommand{\pcap}[0]{\hat{\Bp}}
\newcommand{\qcap}[0]{\hat{\Bq}}
\newcommand{\rcap}[0]{\hat{\Br}}
\newcommand{\scap}[0]{\hat{\Bs}}
\newcommand{\tcap}[0]{\hat{\Bt}}
\newcommand{\ucap}[0]{\hat{\Bu}}
\newcommand{\vcap}[0]{\hat{\Bv}}
\newcommand{\wcap}[0]{\hat{\Bw}}
\newcommand{\xcap}[0]{\hat{\Bx}}
\newcommand{\ycap}[0]{\hat{\By}}
\newcommand{\zcap}[0]{\hat{\Bz}}
\newcommand{\thetacap}[0]{\hat{\Btheta}}

%
% to write R^n and C^n in a distinguishable fashion.  Perhaps change this
% to the double lined characters upon figuring out how to do so.
%
\newcommand{\C}[1]{$\mathbb{C}^{#1}$}
\newcommand{\R}[1]{$\mathbb{R}^{#1}$}

%
% various generally useful helpers
%

% derivative of #1 wrt. #2:
\newcommand{\D}[2] {\frac {d#2} {d#1}}

\newcommand{\inv}[1]{\frac{1}{#1}}
\newcommand{\cross}[0]{\times}

\newcommand{\abs}[1]{\lvert{#1}\rvert}
\newcommand{\norm}[1]{\lVert{#1}\rVert}
\newcommand{\innerprod}[2]{\langle{#1}, {#2}\rangle}
\newcommand{\dotprod}[2]{{#1} \cdot {#2}}
\newcommand{\bdotprod}[2]{\left({#1} \cdot {#2}\right)}
\newcommand{\crossprod}[2]{{#1} \cross {#2}}
\newcommand{\tripleprod}[3]{\dotprod{\left(\crossprod{#1}{#2}\right)}{#3}}

\DeclareMathOperator{\Proj}{Proj}
\DeclareMathOperator{\Span}{span}
\DeclareMathOperator{\Sgn}{sgn}
\DeclareMathOperator{\Area}{Area}
\DeclareMathOperator{\Volume}{Volume}

%
% A few miscellaneous things specific to this document
%
\newcommand{\crossop}[1]{\crossprod{#1}{}}

% R2 vector.
\newcommand{\VectorTwo}[2]{
\begin{bmatrix}
 {#1} \\
 {#2}
\end{bmatrix}
}

\newcommand{\VectorN}[1]{
\begin{bmatrix}
{#1}_1 \\
{#1}_2 \\
\vdots \\
{#1}_N \\
\end{bmatrix}
}

\newcommand{\DETuvij}[4]{
\begin{vmatrix}
 {#1}_{#3} & {#1}_{#4} \\
 {#2}_{#3} & {#2}_{#4}
\end{vmatrix}
}

\newcommand{\DETuvwijk}[6]{
\begin{vmatrix}
 {#1}_{#4} & {#1}_{#5} & {#1}_{#6} \\
 {#2}_{#4} & {#2}_{#5} & {#2}_{#6} \\
 {#3}_{#4} & {#3}_{#5} & {#3}_{#6}
\end{vmatrix}
}

\newcommand{\DETuvwxijkl}[8]{
\begin{vmatrix}
 {#1}_{#5} & {#1}_{#6} & {#1}_{#7} & {#1}_{#8} \\
 {#2}_{#5} & {#2}_{#6} & {#2}_{#7} & {#2}_{#8} \\
 {#3}_{#5} & {#3}_{#6} & {#3}_{#7} & {#3}_{#8} \\
 {#4}_{#5} & {#4}_{#6} & {#4}_{#7} & {#4}_{#8} \\
\end{vmatrix}
}

%\newcommand{\DETuvwxyijklm}[10]{
%\begin{vmatrix}
% {#1}_{#6} & {#1}_{#7} & {#1}_{#8} & {#1}_{#9} & {#1}_{#10} \\
% {#2}_{#6} & {#2}_{#7} & {#2}_{#8} & {#2}_{#9} & {#2}_{#10} \\
% {#3}_{#6} & {#3}_{#7} & {#3}_{#8} & {#3}_{#9} & {#3}_{#10} \\
% {#4}_{#6} & {#4}_{#7} & {#4}_{#8} & {#4}_{#9} & {#4}_{#10} \\
% {#5}_{#6} & {#5}_{#7} & {#5}_{#8} & {#5}_{#9} & {#5}_{#10}
%\end{vmatrix}
%}

% R3 vector.
\newcommand{\VectorThree}[3]{
\begin{bmatrix}
 {#1} \\
 {#2} \\
 {#3}
\end{bmatrix}
}


%%<misc>
%
\newcommand{\Abs}[1]{{\left\lvert{#1}\right\rvert}}
\newcommand{\spacegrad}[0]{\boldsymbol{\nabla}}
\newcommand{\grad}[0]{\nabla}
\newcommand{\LL}[0]{\mathcal{L}}

% == \partial_{#1} {#2}
\newcommand{\PD}[2]{\frac{\partial {#2}}{\partial {#1}}}
% inline variant
\newcommand{\PDi}[2]{{\partial {#2}}/{\partial {#1}}}

\newcommand{\PDD}[3]{\frac{\partial^2 {#3}}{\partial {#1}\partial {#2}}}
%\newcommand{\PDd}[2]{\frac{\partial^2 {#2}}{{\partial{#1}}^2}}
\newcommand{\PDsq}[2]{\frac{\partial^2 {#2}}{(\partial {#1})^2}}

\newcommand{\Partial}[2]{\frac{\partial {#1}}{\partial {#2}}}
\DeclareMathOperator{\RejName}{Rej}
\newcommand{\Rej}[2]{\RejName_{#1}\left( {#2} \right)}
\newcommand{\Rm}[1]{\mathbb{R}^{#1}}
\newcommand{\Cm}[1]{\mathbb{C}^{#1}}
\newcommand{\conj}[0]{{*}}

%</misc>

% <grade selection>
%
\newcommand{\gpgrade}[2] {{\left\langle{{#1}}\right\rangle}_{#2}}

\newcommand{\gpgradezero}[1] {\gpgrade{#1}{}}
%\newcommand{\gpscalargrade}[1] {{\left\langle{{#1}}\right\rangle}}
%\newcommand{\gpgradezero}[1] {\gpgrade{#1}{0}}

%\newcommand{\gpgradeone}[1] {{\left\langle{{#1}}\right\rangle}_{1}}
\newcommand{\gpgradeone}[1] {\gpgrade{#1}{1}}

\newcommand{\gpgradetwo}[1] {\gpgrade{#1}{2}}
\newcommand{\gpgradethree}[1] {\gpgrade{#1}{3}}
\newcommand{\gpgradefour}[1] {\gpgrade{#1}{4}}
%
% </grade selection>



\newcommand{\adot}[0]{{\dot{a}}}
\newcommand{\bdot}[0]{{\dot{b}}}
% taken for centered dot:
%\newcommand{\cdot}[0]{{\dot{c}}}
%\newcommand{\ddot}[0]{{\dot{d}}}
\newcommand{\edot}[0]{{\dot{e}}}
\newcommand{\fdot}[0]{{\dot{f}}}
\newcommand{\gdot}[0]{{\dot{g}}}
\newcommand{\hdot}[0]{{\dot{h}}}
\newcommand{\idot}[0]{{\dot{i}}}
\newcommand{\jdot}[0]{{\dot{j}}}
\newcommand{\kdot}[0]{{\dot{k}}}
\newcommand{\ldot}[0]{{\dot{l}}}
\newcommand{\mdot}[0]{{\dot{m}}}
\newcommand{\ndot}[0]{{\dot{n}}}
%\newcommand{\odot}[0]{{\dot{o}}}
\newcommand{\pdot}[0]{{\dot{p}}}
\newcommand{\qdot}[0]{{\dot{q}}}
\newcommand{\rdot}[0]{{\dot{r}}}
\newcommand{\sdot}[0]{{\dot{s}}}
\newcommand{\tdot}[0]{{\dot{t}}}
\newcommand{\udot}[0]{{\dot{u}}}
\newcommand{\vdot}[0]{{\dot{v}}}
\newcommand{\wdot}[0]{{\dot{w}}}
\newcommand{\xdot}[0]{{\dot{x}}}
\newcommand{\ydot}[0]{{\dot{y}}}
\newcommand{\zdot}[0]{{\dot{z}}}
\newcommand{\addot}[0]{{\ddot{a}}}
\newcommand{\bddot}[0]{{\ddot{b}}}
\newcommand{\cddot}[0]{{\ddot{c}}}
%\newcommand{\dddot}[0]{{\ddot{d}}}
\newcommand{\eddot}[0]{{\ddot{e}}}
\newcommand{\fddot}[0]{{\ddot{f}}}
\newcommand{\gddot}[0]{{\ddot{g}}}
\newcommand{\hddot}[0]{{\ddot{h}}}
\newcommand{\iddot}[0]{{\ddot{i}}}
\newcommand{\jddot}[0]{{\ddot{j}}}
\newcommand{\kddot}[0]{{\ddot{k}}}
\newcommand{\lddot}[0]{{\ddot{l}}}
\newcommand{\mddot}[0]{{\ddot{m}}}
\newcommand{\nddot}[0]{{\ddot{n}}}
\newcommand{\oddot}[0]{{\ddot{o}}}
\newcommand{\pddot}[0]{{\ddot{p}}}
\newcommand{\qddot}[0]{{\ddot{q}}}
\newcommand{\rddot}[0]{{\ddot{r}}}
\newcommand{\sddot}[0]{{\ddot{s}}}
\newcommand{\tddot}[0]{{\ddot{t}}}
\newcommand{\uddot}[0]{{\ddot{u}}}
\newcommand{\vddot}[0]{{\ddot{v}}}
\newcommand{\wddot}[0]{{\ddot{w}}}
\newcommand{\xddot}[0]{{\ddot{x}}}
\newcommand{\yddot}[0]{{\ddot{y}}}
\newcommand{\zddot}[0]{{\ddot{z}}}

%<bold and dot greek symbols>
%

\newcommand{\Deltadot}[0]{{\dot{\Delta}}}
\newcommand{\Gammadot}[0]{{\dot{\Gamma}}}
\newcommand{\Lambdadot}[0]{{\dot{\Lambda}}}
\newcommand{\Omegadot}[0]{{\dot{\Omega}}}
\newcommand{\Phidot}[0]{{\dot{\Phi}}}
\newcommand{\Pidot}[0]{{\dot{\Pi}}}
\newcommand{\Psidot}[0]{{\dot{\Psi}}}
\newcommand{\Sigmadot}[0]{{\dot{\Sigma}}}
\newcommand{\Thetadot}[0]{{\dot{\Theta}}}
\newcommand{\Upsilondot}[0]{{\dot{\Upsilon}}}
\newcommand{\Xidot}[0]{{\dot{\Xi}}}
\newcommand{\alphadot}[0]{{\dot{\alpha}}}
\newcommand{\betadot}[0]{{\dot{\beta}}}
\newcommand{\chidot}[0]{{\dot{\chi}}}
\newcommand{\deltadot}[0]{{\dot{\delta}}}
\newcommand{\epsilondot}[0]{{\dot{\epsilon}}}
\newcommand{\etadot}[0]{{\dot{\eta}}}
\newcommand{\gammadot}[0]{{\dot{\gamma}}}
\newcommand{\kappadot}[0]{{\dot{\kappa}}}
\newcommand{\lambdadot}[0]{{\dot{\lambda}}}
\newcommand{\mudot}[0]{{\dot{\mu}}}
\newcommand{\nudot}[0]{{\dot{\nu}}}
\newcommand{\omegadot}[0]{{\dot{\omega}}}
\newcommand{\phidot}[0]{{\dot{\phi}}}
\newcommand{\pidot}[0]{{\dot{\pi}}}
\newcommand{\psidot}[0]{{\dot{\psi}}}
\newcommand{\rhodot}[0]{{\dot{\rho}}}
\newcommand{\sigmadot}[0]{{\dot{\sigma}}}
\newcommand{\taudot}[0]{{\dot{\tau}}}
\newcommand{\thetadot}[0]{{\dot{\theta}}}
\newcommand{\upsilondot}[0]{{\dot{\upsilon}}}
\newcommand{\varepsilondot}[0]{{\dot{\varepsilon}}}
\newcommand{\varphidot}[0]{{\dot{\varphi}}}
\newcommand{\varpidot}[0]{{\dot{\varpi}}}
\newcommand{\varrhodot}[0]{{\dot{\varrho}}}
\newcommand{\varsigmadot}[0]{{\dot{\varsigma}}}
\newcommand{\varthetadot}[0]{{\dot{\vartheta}}}
\newcommand{\xidot}[0]{{\dot{\xi}}}
\newcommand{\zetadot}[0]{{\dot{\zeta}}}

\newcommand{\Deltaddot}[0]{{\ddot{\Delta}}}
\newcommand{\Gammaddot}[0]{{\ddot{\Gamma}}}
\newcommand{\Lambdaddot}[0]{{\ddot{\Lambda}}}
\newcommand{\Omegaddot}[0]{{\ddot{\Omega}}}
\newcommand{\Phiddot}[0]{{\ddot{\Phi}}}
\newcommand{\Piddot}[0]{{\ddot{\Pi}}}
\newcommand{\Psiddot}[0]{{\ddot{\Psi}}}
\newcommand{\Sigmaddot}[0]{{\ddot{\Sigma}}}
\newcommand{\Thetaddot}[0]{{\ddot{\Theta}}}
\newcommand{\Upsilonddot}[0]{{\ddot{\Upsilon}}}
\newcommand{\Xiddot}[0]{{\ddot{\Xi}}}
\newcommand{\alphaddot}[0]{{\ddot{\alpha}}}
\newcommand{\betaddot}[0]{{\ddot{\beta}}}
\newcommand{\chiddot}[0]{{\ddot{\chi}}}
\newcommand{\deltaddot}[0]{{\ddot{\delta}}}
\newcommand{\epsilonddot}[0]{{\ddot{\epsilon}}}
\newcommand{\etaddot}[0]{{\ddot{\eta}}}
\newcommand{\gammaddot}[0]{{\ddot{\gamma}}}
\newcommand{\kappaddot}[0]{{\ddot{\kappa}}}
\newcommand{\lambdaddot}[0]{{\ddot{\lambda}}}
\newcommand{\muddot}[0]{{\ddot{\mu}}}
\newcommand{\nuddot}[0]{{\ddot{\nu}}}
\newcommand{\omegaddot}[0]{{\ddot{\omega}}}
\newcommand{\phiddot}[0]{{\ddot{\phi}}}
\newcommand{\piddot}[0]{{\ddot{\pi}}}
\newcommand{\psiddot}[0]{{\ddot{\psi}}}
\newcommand{\rhoddot}[0]{{\ddot{\rho}}}
\newcommand{\sigmaddot}[0]{{\ddot{\sigma}}}
\newcommand{\tauddot}[0]{{\ddot{\tau}}}
\newcommand{\thetaddot}[0]{{\ddot{\theta}}}
\newcommand{\upsilonddot}[0]{{\ddot{\upsilon}}}
\newcommand{\varepsilonddot}[0]{{\ddot{\varepsilon}}}
\newcommand{\varphiddot}[0]{{\ddot{\varphi}}}
\newcommand{\varpiddot}[0]{{\ddot{\varpi}}}
\newcommand{\varrhoddot}[0]{{\ddot{\varrho}}}
\newcommand{\varsigmaddot}[0]{{\ddot{\varsigma}}}
\newcommand{\varthetaddot}[0]{{\ddot{\vartheta}}}
\newcommand{\xiddot}[0]{{\ddot{\xi}}}
\newcommand{\zetaddot}[0]{{\ddot{\zeta}}}

\newcommand{\BDelta}[0]{\boldsymbol{\Delta}}
\newcommand{\BGamma}[0]{\boldsymbol{\Gamma}}
\newcommand{\BLambda}[0]{\boldsymbol{\Lambda}}
\newcommand{\BOmega}[0]{\boldsymbol{\Omega}}
\newcommand{\BPhi}[0]{\boldsymbol{\Phi}}
\newcommand{\BPi}[0]{\boldsymbol{\Pi}}
\newcommand{\BPsi}[0]{\boldsymbol{\Psi}}
\newcommand{\BSigma}[0]{\boldsymbol{\Sigma}}
\newcommand{\BTheta}[0]{\boldsymbol{\Theta}}
\newcommand{\BUpsilon}[0]{\boldsymbol{\Upsilon}}
\newcommand{\BXi}[0]{\boldsymbol{\Xi}}
\newcommand{\Balpha}[0]{\boldsymbol{\alpha}}
\newcommand{\Bbeta}[0]{\boldsymbol{\beta}}
\newcommand{\Bchi}[0]{\boldsymbol{\chi}}
\newcommand{\Bdelta}[0]{\boldsymbol{\delta}}
\newcommand{\Bepsilon}[0]{\boldsymbol{\epsilon}}
\newcommand{\Beta}[0]{\boldsymbol{\eta}}
\newcommand{\Bgamma}[0]{\boldsymbol{\gamma}}
\newcommand{\Bkappa}[0]{\boldsymbol{\kappa}}
\newcommand{\Blambda}[0]{\boldsymbol{\lambda}}
\newcommand{\Bmu}[0]{\boldsymbol{\mu}}
\newcommand{\Bnu}[0]{\boldsymbol{\nu}}
%\newcommand{\Bomega}[0]{\boldsymbol{\omega}}
\newcommand{\Bphi}[0]{\boldsymbol{\phi}}
\newcommand{\Bpi}[0]{\boldsymbol{\pi}}
\newcommand{\Bpsi}[0]{\boldsymbol{\psi}}
\newcommand{\Brho}[0]{\boldsymbol{\rho}}
\newcommand{\Bsigma}[0]{\boldsymbol{\sigma}}
%\newcommand{\Btau}[0]{\boldsymbol{\tau}}
%\newcommand{\Btheta}[0]{\boldsymbol{\theta}}
\newcommand{\Bupsilon}[0]{\boldsymbol{\upsilon}}
\newcommand{\Bvarepsilon}[0]{\boldsymbol{\varepsilon}}
\newcommand{\Bvarphi}[0]{\boldsymbol{\varphi}}
\newcommand{\Bvarpi}[0]{\boldsymbol{\varpi}}
\newcommand{\Bvarrho}[0]{\boldsymbol{\varrho}}
\newcommand{\Bvarsigma}[0]{\boldsymbol{\varsigma}}
\newcommand{\Bvartheta}[0]{\boldsymbol{\vartheta}}
\newcommand{\Bxi}[0]{\boldsymbol{\xi}}
\newcommand{\Bzeta}[0]{\boldsymbol{\zeta}}
%
%</bold and dot greek symbols>
%<infrequent>
%
%\newcommand{\AreaOp}[1]{\AName_{#1}}
%\newcommand{\Babs}[0]{\abs{\BB}}
%\newcommand{\Bcap}[0]{\hat{\BB}}
%\newcommand{\BrPrimeRej}[0]{\rcap(\rcap \wedge \Br')}
%\newcommand{\CA}[0]{\mathcal{A}}
%\newcommand{\Cos}[1]{\cos{\left({#1}\right)}}
%\newcommand{\Det}[1] {\abs{#1}}
%\newcommand{\Dsq}[2] {\frac {\partial^2 {#1}} {\partial {#2}^2}}
%\newcommand{\Exp}[1]{\exp{\left({#1}\right)}}
%\newcommand{\Norm}[1]{\left\lVert{#1}\right\rVert}
%\newcommand{\Sin}[1]{\sin{\left({#1}\right)}}
%\newcommand{\T}[0]{\text{T}}
%\newcommand{\VolumeOp}[1]{\VName_{#1}}
%\newcommand{\agrad}[0]{\Ba \cdot \nabla}
%\newcommand{\alphacap}[0]{\hat{\boldsymbol{\alpha}}}
%\newcommand{\Fcap}[0]{\hat{\BF}}
%\newcommand{\bithree}[0]{{\Bi}_3}
%\newcommand{\bxa}[0]{\Bx\Ba}
%\newcommand{\coordvec}[2]{
%\newcommand{\costheta}[0]{\acap \cdot \xcap}
%\newcommand{\ddt}[1]{\ddot{#1}}
%\newcommand{\ddu}[1] {\frac {d{#1}} {du}}
%\newcommand{\dsqxj}[2] {\frac {\partial^2 {#1}} {\partial {x_{#2}}^2}}
%\newcommand{\dtheta}[1]{\frac{d {#1}}{d \theta}}
%\newcommand{\dt}[1]{\dot{#1}}
%\newcommand{\dt}[1]{\frac{d {#1}}{dt}}
%\newcommand{\dxj}[2] {\frac {\partial {#1}} {\partial {x_{#2}}}}
%\newcommand{\halfPhi}[0]{\frac{\phi}{2}}
%\newcommand{\half}[0]{\inv{2}}
%\newcommand{\inv}[1]{\frac{1}{#1}}
%\newcommand{\laplacian}[0]{\nabla^2}
%\newcommand{\matrixoftx}[3]{
%\newcommand{\nrrp}[0]{\norm{\rcap \wedge \Br'}}
%\newcommand{\oiint}{\bigcirc \hspace{-1.4em} \int \hspace{-.8em} \int}
%\newcommand{\transpose}[1]{{#1}^{\text{T}}}
%\newcommand{\transpose}[1]{{{#1}^{\TextTranspose}}}
%\newcommand{\transpose}[1]{{{#1}^{\text{T}}}}
%\newcommand{\barA}[0]{\bar{A}}
%\newcommand{\qbar}[0]{\bar{q}}
%\newcommand{\qdotbar}[0]{\dot{\bar{q}}}
%
%</infrequent>





%\usepackage[bookmarks=true]{hyperref}

%\usepackage{color,cite,graphicx}
   % use colour in the document, put your citations as [1-4]
   % rather than [1,2,3,4] (it looks nicer, and the extended LaTeX2e
   % graphics package. 
%\usepackage{latexsym,amssymb,epsf} % don't remember if these are
   % needed, but their inclusion can't do any damage

\chapter{Evaluating the Gaussian integral. }
%\author{Peeter Joot \quad peeter.joot@gmail.com}
\date{ Jan 05, 2009.  $RCSfile: gaussian.tex,v $ Last $Revision: 1.12 $ $Date: 2009/06/11 17:00:37 $ }

%\begin{document}

%\maketitle{}
%\tableofcontents
\section{Plain old Gaussian. }

QM solutions appear to involve a lot of Gaussian integrals.  Looking at one
of the problems in \cite{mcmahon2005qmd} I tried to recall how to evaluate
the simplest of these.  Google says the trick is squaring and polar 
substitution.  Let's try this.

Solve
\begin{align*}
I = \int_{-\infty}^\infty e^{-\alpha x^2} dx
\end{align*}

\begin{align*}
I^2 
&= \int_{x= -\infty}^\infty e^{-\alpha x^2} dx \int_{y = -\infty}^\infty e^{-\alpha y^2} dy \\
&= \int_{\theta=0}^{2\pi}\int_{r= 0}^\infty e^{-\alpha r^2} r dr d\theta \\
&= 2\pi 
{\left.
\frac{e^{-\alpha r^2}}{-2\alpha}
\right\vert}_{r= 0}^\infty  \\
&= \frac{\pi}{\alpha}
\end{align*}

So we have

\begin{align*}
I = \sqrt{\frac{\pi}{\alpha}}
\end{align*}

\section{A couple higher order Gaussian's and normalization exercise. }

In order to do the normalization exercise for

\begin{align}\label{eqn:gaussian:exercise}
\psi = \left(A e^{-\frac{x^2}{a}} +B x e^{-\frac{x^2}{b}}\right) e^{-ict}
\end{align}

We want to calculate
\begin{align*}
\int \psi^\conj \psi = 
\Abs{A}^2 e^{-2\frac{x^2}{a}} +\Abs{B}^2 x^2 e^{-2\frac{x^2}{b}}
+ (A\bar{B} + B\bar{A}) x e^{-x^2\left(\inv{a} + \inv{b}\right)}
\end{align*}

so we need the $n=1,2$ versions of the following Gaussian integrals

\begin{align*}
I_n = \int x^n e^{-\alpha x^2} dx
\end{align*}

The $n=1$ case is directly integrable:

\begin{align*}
I_1 
&= \int x e^{-\alpha x^2} dx \\
&= \int \left(\frac{e^{-\alpha x^2}}{-2\alpha}\right)' dx \\
&= 0
\end{align*}

(integration bounds are $\pm \infty$ so the exponential vanishes).

Next.  $I_2$ follows with integration by parts

\begin{align*}
I_2 
&= \int x^2 e^{-\alpha x^2} dx \\
&= \int x \left(x e^{-\alpha x^2}\right) dx \\
&= \int x \left(\frac{e^{-\alpha x^2}}{-2\alpha}\right)' dx \\
&= -\int \frac{e^{-\alpha x^2}}{-2\alpha} dx \\
&= \inv{2\alpha}\int e^{-\alpha x^2} dx \\
&= \inv{2\alpha} \sqrt{\frac{\pi}{\alpha}}
\end{align*}

So the normalization required for \ref{eqn:gaussian:exercise} is
\begin{align*}
1 = \Abs{A}^2 \sqrt{\frac{\pi a}{2}} + \frac{\Abs{B}^2}{2} 
\sqrt{\frac{\pi b}{2}} \frac{b}{2}
\end{align*}

The values $a$, and $b$ are presumably due to boundary conditions, and this then fixes $\Abs{A}$ in terms of $\Abs{B}$ or the other way around.

\section{Generalized. }

Let 

\begin{align}
I_n = \int_{-\infty}^\infty x^n e^{-\alpha x^2} dx
\end{align}

We've solved this for $I_0 = \sqrt{\pi/\alpha}$, $I_1 = 0$, and $I_2$.  A quick calculation shows that $I_{2k+1} = 0$ too:

\begin{align*}
I_n 
&= \int_{-\infty}^\infty x^n e^{-\alpha x^2} dx \\
&= \int_{0}^\infty x^n e^{-\alpha x^2} dx +\int_{-\infty}^0 x^n e^{-\alpha x^2} dx \\
&= \int_{0}^\infty x^n e^{-\alpha x^2} dx +\int_{\infty}^0 (-x)^n e^{-\alpha x^2} (-dx) \\
&= \int_{0}^\infty x^n e^{-\alpha x^2} dx +\int_0^{\infty} (-x)^n e^{-\alpha x^2} dx \\
&= \int_{0}^\infty (x^n + (-x)^n)e^{-\alpha x^2} dx \\
\end{align*}

But if $n$ is odd $(-x)^n = -x^n$, so this is zero.

Now, for $n$ even, we can integrate by parts, as done for $I_2$.

\begin{align*}
I_{2m}
&= \int x^{2m} e^{-\alpha x^2} dx \\
&= \int x^{2m-1} \left(x e^{-\alpha x^2}\right) dx \\
&= \int x^{2m-1} \left(\frac{e^{-\alpha x^2}}{-2\alpha}\right)' dx \\
&= -\int (2m-1) x^{2m-2} \frac{e^{-\alpha x^2}}{-2\alpha} dx \\
\end{align*}

This gives us a recurrence relationship for the even order terms
\begin{align}
I_{2m} &= \frac{2m-1}{2\alpha} I_{2m-2}.
%&= \inv{2\alpha} \sqrt{\frac{\pi}{\alpha}}
\end{align}

Expanding this explicitly for the first few $m$ shows the pattern

\begin{align*}
\begin{array}{l l l}
I_2 &= \frac{2-1}{2\alpha} I_0 &= \frac{1}{2\alpha} \sqrt{\frac{\pi}{\alpha}} \\
I_4 &= \frac{4-1}{2\alpha}\frac{2-1}{2\alpha} I_0 &= \frac{3.1}{(2\alpha)^2} \sqrt{\frac{\pi}{\alpha}} \\
I_6 &= \frac{6-1}{2\alpha}\frac{4-1}{2\alpha}\frac{2-1}{2\alpha} I_0 &= \frac{5.3.1}{(2\alpha)^3} \sqrt{\frac{\pi}{\alpha}} \\
\end{array}
\end{align*}

Or
\begin{align}
I_{0} &= \sqrt{\frac{\pi}{\alpha}} \\
I_{2m-1} &= 0 \\
I_{2m} &= \frac{(2m-1)(2m-3)\cdots(3)(1)}{(2\alpha)^m} \sqrt{\frac{\pi}{\alpha}}
\end{align}

%\bibliographystyle{plainnat}
%\bibliography{myrefs}

%\end{document}

\documentclass{article}

\usepackage{amsmath}
\usepackage{mathpazo}

%
% shorthand for bold symbols, convenient for vectors and matrices
%
\newcommand{\Ba}[0]{\mathbf{a}}
\newcommand{\Bb}[0]{\mathbf{b}}
\newcommand{\Bc}[0]{\mathbf{c}}
\newcommand{\Bd}[0]{\mathbf{d}}
\newcommand{\Be}[0]{\mathbf{e}}
\newcommand{\Bf}[0]{\mathbf{f}}
\newcommand{\Bg}[0]{\mathbf{g}}
\newcommand{\Bh}[0]{\mathbf{h}}
\newcommand{\Bi}[0]{\mathbf{i}}
\newcommand{\Bj}[0]{\mathbf{j}}
\newcommand{\Bk}[0]{\mathbf{k}}
\newcommand{\Bl}[0]{\mathbf{l}}
\newcommand{\Bm}[0]{\mathbf{m}}
\newcommand{\Bn}[0]{\mathbf{n}}
\newcommand{\Bo}[0]{\mathbf{o}}
\newcommand{\Bp}[0]{\mathbf{p}}
\newcommand{\Bq}[0]{\mathbf{q}}
\newcommand{\Br}[0]{\mathbf{r}}
\newcommand{\Bs}[0]{\mathbf{s}}
\newcommand{\Bt}[0]{\mathbf{t}}
\newcommand{\Bu}[0]{\mathbf{u}}
\newcommand{\Bv}[0]{\mathbf{v}}
\newcommand{\Bw}[0]{\mathbf{w}}
\newcommand{\Bx}[0]{\mathbf{x}}
\newcommand{\By}[0]{\mathbf{y}}
\newcommand{\Bz}[0]{\mathbf{z}}
\newcommand{\BA}[0]{\mathbf{A}}
\newcommand{\BB}[0]{\mathbf{B}}
\newcommand{\BC}[0]{\mathbf{C}}
\newcommand{\BD}[0]{\mathbf{D}}
\newcommand{\BE}[0]{\mathbf{E}}
\newcommand{\BF}[0]{\mathbf{F}}
\newcommand{\BG}[0]{\mathbf{G}}
\newcommand{\BH}[0]{\mathbf{H}}
\newcommand{\BI}[0]{\mathbf{I}}
\newcommand{\BJ}[0]{\mathbf{J}}
\newcommand{\BK}[0]{\mathbf{K}}
\newcommand{\BL}[0]{\mathbf{L}}
\newcommand{\BM}[0]{\mathbf{M}}
\newcommand{\BN}[0]{\mathbf{N}}
\newcommand{\BO}[0]{\mathbf{O}}
\newcommand{\BP}[0]{\mathbf{P}}
\newcommand{\BQ}[0]{\mathbf{Q}}
\newcommand{\BR}[0]{\mathbf{R}}
\newcommand{\BS}[0]{\mathbf{S}}
\newcommand{\BT}[0]{\mathbf{T}}
\newcommand{\BU}[0]{\mathbf{U}}
\newcommand{\BV}[0]{\mathbf{V}}
\newcommand{\BW}[0]{\mathbf{W}}
\newcommand{\BX}[0]{\mathbf{X}}
\newcommand{\BY}[0]{\mathbf{Y}}
\newcommand{\BZ}[0]{\mathbf{Z}}

\newcommand{\Bzero}[0]{\mathbf{0}}
\newcommand{\Btheta}[0]{\boldsymbol{\theta}}
\newcommand{\Btau}[0]{\boldsymbol{\tau}}
\newcommand{\Bomega}[0]{\boldsymbol{\omega}}

%
% shorthand for unit vectors
%
\newcommand{\acap}[0]{\hat{\Ba}}
\newcommand{\bcap}[0]{\hat{\Bb}}
\newcommand{\ccap}[0]{\hat{\Bc}}
\newcommand{\dcap}[0]{\hat{\Bd}}
\newcommand{\ecap}[0]{\hat{\Be}}
\newcommand{\fcap}[0]{\hat{\Bf}}
\newcommand{\gcap}[0]{\hat{\Bg}}
\newcommand{\hcap}[0]{\hat{\Bh}}
\newcommand{\icap}[0]{\hat{\Bi}}
\newcommand{\jcap}[0]{\hat{\Bj}}
\newcommand{\kcap}[0]{\hat{\Bk}}
\newcommand{\lcap}[0]{\hat{\Bl}}
\newcommand{\mcap}[0]{\hat{\Bm}}
\newcommand{\ncap}[0]{\hat{\Bn}}
\newcommand{\ocap}[0]{\hat{\Bo}}
\newcommand{\pcap}[0]{\hat{\Bp}}
\newcommand{\qcap}[0]{\hat{\Bq}}
\newcommand{\rcap}[0]{\hat{\Br}}
\newcommand{\scap}[0]{\hat{\Bs}}
\newcommand{\tcap}[0]{\hat{\Bt}}
\newcommand{\ucap}[0]{\hat{\Bu}}
\newcommand{\vcap}[0]{\hat{\Bv}}
\newcommand{\wcap}[0]{\hat{\Bw}}
\newcommand{\xcap}[0]{\hat{\Bx}}
\newcommand{\ycap}[0]{\hat{\By}}
\newcommand{\zcap}[0]{\hat{\Bz}}
\newcommand{\thetacap}[0]{\hat{\Btheta}}

%
% to write R^n and C^n in a distinguishable fashion.  Perhaps change this
% to the double lined characters upon figuring out how to do so.
%
\newcommand{\C}[1]{$\mathbb{C}^{#1}$}
\newcommand{\R}[1]{$\mathbb{R}^{#1}$}

%
% various generally useful helpers
%

% derivative of #1 wrt. #2:
\newcommand{\D}[2] {\frac {d#2} {d#1}}

\newcommand{\inv}[1]{\frac{1}{#1}}
\newcommand{\cross}[0]{\times}

\newcommand{\abs}[1]{\lvert{#1}\rvert}
\newcommand{\norm}[1]{\lVert{#1}\rVert}
\newcommand{\innerprod}[2]{\langle{#1}, {#2}\rangle}
\newcommand{\dotprod}[2]{{#1} \cdot {#2}}
\newcommand{\bdotprod}[2]{\left({#1} \cdot {#2}\right)}
\newcommand{\crossprod}[2]{{#1} \cross {#2}}
\newcommand{\tripleprod}[3]{\dotprod{\left(\crossprod{#1}{#2}\right)}{#3}}

\DeclareMathOperator{\Proj}{Proj}
\DeclareMathOperator{\Span}{span}
\DeclareMathOperator{\Sgn}{sgn}
\DeclareMathOperator{\Area}{Area}
\DeclareMathOperator{\Volume}{Volume}

%
% A few miscellaneous things specific to this document
%
\newcommand{\crossop}[1]{\crossprod{#1}{}}

% R2 vector.
\newcommand{\VectorTwo}[2]{
\begin{bmatrix}
 {#1} \\
 {#2}
\end{bmatrix}
}

\newcommand{\VectorN}[1]{
\begin{bmatrix}
{#1}_1 \\
{#1}_2 \\
\vdots \\
{#1}_N \\
\end{bmatrix}
}

\newcommand{\DETuvij}[4]{
\begin{vmatrix}
 {#1}_{#3} & {#1}_{#4} \\
 {#2}_{#3} & {#2}_{#4}
\end{vmatrix}
}

\newcommand{\DETuvwijk}[6]{
\begin{vmatrix}
 {#1}_{#4} & {#1}_{#5} & {#1}_{#6} \\
 {#2}_{#4} & {#2}_{#5} & {#2}_{#6} \\
 {#3}_{#4} & {#3}_{#5} & {#3}_{#6}
\end{vmatrix}
}

\newcommand{\DETuvwxijkl}[8]{
\begin{vmatrix}
 {#1}_{#5} & {#1}_{#6} & {#1}_{#7} & {#1}_{#8} \\
 {#2}_{#5} & {#2}_{#6} & {#2}_{#7} & {#2}_{#8} \\
 {#3}_{#5} & {#3}_{#6} & {#3}_{#7} & {#3}_{#8} \\
 {#4}_{#5} & {#4}_{#6} & {#4}_{#7} & {#4}_{#8} \\
\end{vmatrix}
}

%\newcommand{\DETuvwxyijklm}[10]{
%\begin{vmatrix}
% {#1}_{#6} & {#1}_{#7} & {#1}_{#8} & {#1}_{#9} & {#1}_{#10} \\
% {#2}_{#6} & {#2}_{#7} & {#2}_{#8} & {#2}_{#9} & {#2}_{#10} \\
% {#3}_{#6} & {#3}_{#7} & {#3}_{#8} & {#3}_{#9} & {#3}_{#10} \\
% {#4}_{#6} & {#4}_{#7} & {#4}_{#8} & {#4}_{#9} & {#4}_{#10} \\
% {#5}_{#6} & {#5}_{#7} & {#5}_{#8} & {#5}_{#9} & {#5}_{#10}
%\end{vmatrix}
%}

% R3 vector.
\newcommand{\VectorThree}[3]{
\begin{bmatrix}
 {#1} \\
 {#2} \\
 {#3}
\end{bmatrix}
}


%<misc>
%
\newcommand{\Abs}[1]{{\left\lvert{#1}\right\rvert}}
\newcommand{\spacegrad}[0]{\boldsymbol{\nabla}}
\newcommand{\grad}[0]{\nabla}
\newcommand{\LL}[0]{\mathcal{L}}

% == \partial_{#1} {#2}
\newcommand{\PD}[2]{\frac{\partial {#2}}{\partial {#1}}}
% inline variant
\newcommand{\PDi}[2]{{\partial {#2}}/{\partial {#1}}}

\newcommand{\PDD}[3]{\frac{\partial^2 {#3}}{\partial {#1}\partial {#2}}}
%\newcommand{\PDd}[2]{\frac{\partial^2 {#2}}{{\partial{#1}}^2}}
\newcommand{\PDsq}[2]{\frac{\partial^2 {#2}}{(\partial {#1})^2}}

\newcommand{\Partial}[2]{\frac{\partial {#1}}{\partial {#2}}}
\DeclareMathOperator{\RejName}{Rej}
\newcommand{\Rej}[2]{\RejName_{#1}\left( {#2} \right)}
\newcommand{\Rm}[1]{\mathbb{R}^{#1}}
\newcommand{\Cm}[1]{\mathbb{C}^{#1}}
\newcommand{\conj}[0]{{*}}

%</misc>

% <grade selection>
%
\newcommand{\gpgrade}[2] {{\left\langle{{#1}}\right\rangle}_{#2}}

\newcommand{\gpgradezero}[1] {\gpgrade{#1}{}}
%\newcommand{\gpscalargrade}[1] {{\left\langle{{#1}}\right\rangle}}
%\newcommand{\gpgradezero}[1] {\gpgrade{#1}{0}}

%\newcommand{\gpgradeone}[1] {{\left\langle{{#1}}\right\rangle}_{1}}
\newcommand{\gpgradeone}[1] {\gpgrade{#1}{1}}

\newcommand{\gpgradetwo}[1] {\gpgrade{#1}{2}}
\newcommand{\gpgradethree}[1] {\gpgrade{#1}{3}}
\newcommand{\gpgradefour}[1] {\gpgrade{#1}{4}}
%
% </grade selection>



\newcommand{\adot}[0]{{\dot{a}}}
\newcommand{\bdot}[0]{{\dot{b}}}
% taken for centered dot:
%\newcommand{\cdot}[0]{{\dot{c}}}
%\newcommand{\ddot}[0]{{\dot{d}}}
\newcommand{\edot}[0]{{\dot{e}}}
\newcommand{\fdot}[0]{{\dot{f}}}
\newcommand{\gdot}[0]{{\dot{g}}}
\newcommand{\hdot}[0]{{\dot{h}}}
\newcommand{\idot}[0]{{\dot{i}}}
\newcommand{\jdot}[0]{{\dot{j}}}
\newcommand{\kdot}[0]{{\dot{k}}}
\newcommand{\ldot}[0]{{\dot{l}}}
\newcommand{\mdot}[0]{{\dot{m}}}
\newcommand{\ndot}[0]{{\dot{n}}}
%\newcommand{\odot}[0]{{\dot{o}}}
\newcommand{\pdot}[0]{{\dot{p}}}
\newcommand{\qdot}[0]{{\dot{q}}}
\newcommand{\rdot}[0]{{\dot{r}}}
\newcommand{\sdot}[0]{{\dot{s}}}
\newcommand{\tdot}[0]{{\dot{t}}}
\newcommand{\udot}[0]{{\dot{u}}}
\newcommand{\vdot}[0]{{\dot{v}}}
\newcommand{\wdot}[0]{{\dot{w}}}
\newcommand{\xdot}[0]{{\dot{x}}}
\newcommand{\ydot}[0]{{\dot{y}}}
\newcommand{\zdot}[0]{{\dot{z}}}
\newcommand{\addot}[0]{{\ddot{a}}}
\newcommand{\bddot}[0]{{\ddot{b}}}
\newcommand{\cddot}[0]{{\ddot{c}}}
%\newcommand{\dddot}[0]{{\ddot{d}}}
\newcommand{\eddot}[0]{{\ddot{e}}}
\newcommand{\fddot}[0]{{\ddot{f}}}
\newcommand{\gddot}[0]{{\ddot{g}}}
\newcommand{\hddot}[0]{{\ddot{h}}}
\newcommand{\iddot}[0]{{\ddot{i}}}
\newcommand{\jddot}[0]{{\ddot{j}}}
\newcommand{\kddot}[0]{{\ddot{k}}}
\newcommand{\lddot}[0]{{\ddot{l}}}
\newcommand{\mddot}[0]{{\ddot{m}}}
\newcommand{\nddot}[0]{{\ddot{n}}}
\newcommand{\oddot}[0]{{\ddot{o}}}
\newcommand{\pddot}[0]{{\ddot{p}}}
\newcommand{\qddot}[0]{{\ddot{q}}}
\newcommand{\rddot}[0]{{\ddot{r}}}
\newcommand{\sddot}[0]{{\ddot{s}}}
\newcommand{\tddot}[0]{{\ddot{t}}}
\newcommand{\uddot}[0]{{\ddot{u}}}
\newcommand{\vddot}[0]{{\ddot{v}}}
\newcommand{\wddot}[0]{{\ddot{w}}}
\newcommand{\xddot}[0]{{\ddot{x}}}
\newcommand{\yddot}[0]{{\ddot{y}}}
\newcommand{\zddot}[0]{{\ddot{z}}}

%<bold and dot greek symbols>
%

\newcommand{\Deltadot}[0]{{\dot{\Delta}}}
\newcommand{\Gammadot}[0]{{\dot{\Gamma}}}
\newcommand{\Lambdadot}[0]{{\dot{\Lambda}}}
\newcommand{\Omegadot}[0]{{\dot{\Omega}}}
\newcommand{\Phidot}[0]{{\dot{\Phi}}}
\newcommand{\Pidot}[0]{{\dot{\Pi}}}
\newcommand{\Psidot}[0]{{\dot{\Psi}}}
\newcommand{\Sigmadot}[0]{{\dot{\Sigma}}}
\newcommand{\Thetadot}[0]{{\dot{\Theta}}}
\newcommand{\Upsilondot}[0]{{\dot{\Upsilon}}}
\newcommand{\Xidot}[0]{{\dot{\Xi}}}
\newcommand{\alphadot}[0]{{\dot{\alpha}}}
\newcommand{\betadot}[0]{{\dot{\beta}}}
\newcommand{\chidot}[0]{{\dot{\chi}}}
\newcommand{\deltadot}[0]{{\dot{\delta}}}
\newcommand{\epsilondot}[0]{{\dot{\epsilon}}}
\newcommand{\etadot}[0]{{\dot{\eta}}}
\newcommand{\gammadot}[0]{{\dot{\gamma}}}
\newcommand{\kappadot}[0]{{\dot{\kappa}}}
\newcommand{\lambdadot}[0]{{\dot{\lambda}}}
\newcommand{\mudot}[0]{{\dot{\mu}}}
\newcommand{\nudot}[0]{{\dot{\nu}}}
\newcommand{\omegadot}[0]{{\dot{\omega}}}
\newcommand{\phidot}[0]{{\dot{\phi}}}
\newcommand{\pidot}[0]{{\dot{\pi}}}
\newcommand{\psidot}[0]{{\dot{\psi}}}
\newcommand{\rhodot}[0]{{\dot{\rho}}}
\newcommand{\sigmadot}[0]{{\dot{\sigma}}}
\newcommand{\taudot}[0]{{\dot{\tau}}}
\newcommand{\thetadot}[0]{{\dot{\theta}}}
\newcommand{\upsilondot}[0]{{\dot{\upsilon}}}
\newcommand{\varepsilondot}[0]{{\dot{\varepsilon}}}
\newcommand{\varphidot}[0]{{\dot{\varphi}}}
\newcommand{\varpidot}[0]{{\dot{\varpi}}}
\newcommand{\varrhodot}[0]{{\dot{\varrho}}}
\newcommand{\varsigmadot}[0]{{\dot{\varsigma}}}
\newcommand{\varthetadot}[0]{{\dot{\vartheta}}}
\newcommand{\xidot}[0]{{\dot{\xi}}}
\newcommand{\zetadot}[0]{{\dot{\zeta}}}

\newcommand{\Deltaddot}[0]{{\ddot{\Delta}}}
\newcommand{\Gammaddot}[0]{{\ddot{\Gamma}}}
\newcommand{\Lambdaddot}[0]{{\ddot{\Lambda}}}
\newcommand{\Omegaddot}[0]{{\ddot{\Omega}}}
\newcommand{\Phiddot}[0]{{\ddot{\Phi}}}
\newcommand{\Piddot}[0]{{\ddot{\Pi}}}
\newcommand{\Psiddot}[0]{{\ddot{\Psi}}}
\newcommand{\Sigmaddot}[0]{{\ddot{\Sigma}}}
\newcommand{\Thetaddot}[0]{{\ddot{\Theta}}}
\newcommand{\Upsilonddot}[0]{{\ddot{\Upsilon}}}
\newcommand{\Xiddot}[0]{{\ddot{\Xi}}}
\newcommand{\alphaddot}[0]{{\ddot{\alpha}}}
\newcommand{\betaddot}[0]{{\ddot{\beta}}}
\newcommand{\chiddot}[0]{{\ddot{\chi}}}
\newcommand{\deltaddot}[0]{{\ddot{\delta}}}
\newcommand{\epsilonddot}[0]{{\ddot{\epsilon}}}
\newcommand{\etaddot}[0]{{\ddot{\eta}}}
\newcommand{\gammaddot}[0]{{\ddot{\gamma}}}
\newcommand{\kappaddot}[0]{{\ddot{\kappa}}}
\newcommand{\lambdaddot}[0]{{\ddot{\lambda}}}
\newcommand{\muddot}[0]{{\ddot{\mu}}}
\newcommand{\nuddot}[0]{{\ddot{\nu}}}
\newcommand{\omegaddot}[0]{{\ddot{\omega}}}
\newcommand{\phiddot}[0]{{\ddot{\phi}}}
\newcommand{\piddot}[0]{{\ddot{\pi}}}
\newcommand{\psiddot}[0]{{\ddot{\psi}}}
\newcommand{\rhoddot}[0]{{\ddot{\rho}}}
\newcommand{\sigmaddot}[0]{{\ddot{\sigma}}}
\newcommand{\tauddot}[0]{{\ddot{\tau}}}
\newcommand{\thetaddot}[0]{{\ddot{\theta}}}
\newcommand{\upsilonddot}[0]{{\ddot{\upsilon}}}
\newcommand{\varepsilonddot}[0]{{\ddot{\varepsilon}}}
\newcommand{\varphiddot}[0]{{\ddot{\varphi}}}
\newcommand{\varpiddot}[0]{{\ddot{\varpi}}}
\newcommand{\varrhoddot}[0]{{\ddot{\varrho}}}
\newcommand{\varsigmaddot}[0]{{\ddot{\varsigma}}}
\newcommand{\varthetaddot}[0]{{\ddot{\vartheta}}}
\newcommand{\xiddot}[0]{{\ddot{\xi}}}
\newcommand{\zetaddot}[0]{{\ddot{\zeta}}}

\newcommand{\BDelta}[0]{\boldsymbol{\Delta}}
\newcommand{\BGamma}[0]{\boldsymbol{\Gamma}}
\newcommand{\BLambda}[0]{\boldsymbol{\Lambda}}
\newcommand{\BOmega}[0]{\boldsymbol{\Omega}}
\newcommand{\BPhi}[0]{\boldsymbol{\Phi}}
\newcommand{\BPi}[0]{\boldsymbol{\Pi}}
\newcommand{\BPsi}[0]{\boldsymbol{\Psi}}
\newcommand{\BSigma}[0]{\boldsymbol{\Sigma}}
\newcommand{\BTheta}[0]{\boldsymbol{\Theta}}
\newcommand{\BUpsilon}[0]{\boldsymbol{\Upsilon}}
\newcommand{\BXi}[0]{\boldsymbol{\Xi}}
\newcommand{\Balpha}[0]{\boldsymbol{\alpha}}
\newcommand{\Bbeta}[0]{\boldsymbol{\beta}}
\newcommand{\Bchi}[0]{\boldsymbol{\chi}}
\newcommand{\Bdelta}[0]{\boldsymbol{\delta}}
\newcommand{\Bepsilon}[0]{\boldsymbol{\epsilon}}
\newcommand{\Beta}[0]{\boldsymbol{\eta}}
\newcommand{\Bgamma}[0]{\boldsymbol{\gamma}}
\newcommand{\Bkappa}[0]{\boldsymbol{\kappa}}
\newcommand{\Blambda}[0]{\boldsymbol{\lambda}}
\newcommand{\Bmu}[0]{\boldsymbol{\mu}}
\newcommand{\Bnu}[0]{\boldsymbol{\nu}}
%\newcommand{\Bomega}[0]{\boldsymbol{\omega}}
\newcommand{\Bphi}[0]{\boldsymbol{\phi}}
\newcommand{\Bpi}[0]{\boldsymbol{\pi}}
\newcommand{\Bpsi}[0]{\boldsymbol{\psi}}
\newcommand{\Brho}[0]{\boldsymbol{\rho}}
\newcommand{\Bsigma}[0]{\boldsymbol{\sigma}}
%\newcommand{\Btau}[0]{\boldsymbol{\tau}}
%\newcommand{\Btheta}[0]{\boldsymbol{\theta}}
\newcommand{\Bupsilon}[0]{\boldsymbol{\upsilon}}
\newcommand{\Bvarepsilon}[0]{\boldsymbol{\varepsilon}}
\newcommand{\Bvarphi}[0]{\boldsymbol{\varphi}}
\newcommand{\Bvarpi}[0]{\boldsymbol{\varpi}}
\newcommand{\Bvarrho}[0]{\boldsymbol{\varrho}}
\newcommand{\Bvarsigma}[0]{\boldsymbol{\varsigma}}
\newcommand{\Bvartheta}[0]{\boldsymbol{\vartheta}}
\newcommand{\Bxi}[0]{\boldsymbol{\xi}}
\newcommand{\Bzeta}[0]{\boldsymbol{\zeta}}
%
%</bold and dot greek symbols>
%<infrequent>
%
%\newcommand{\AreaOp}[1]{\AName_{#1}}
%\newcommand{\Babs}[0]{\abs{\BB}}
%\newcommand{\Bcap}[0]{\hat{\BB}}
%\newcommand{\BrPrimeRej}[0]{\rcap(\rcap \wedge \Br')}
%\newcommand{\CA}[0]{\mathcal{A}}
%\newcommand{\Cos}[1]{\cos{\left({#1}\right)}}
%\newcommand{\Det}[1] {\abs{#1}}
%\newcommand{\Dsq}[2] {\frac {\partial^2 {#1}} {\partial {#2}^2}}
%\newcommand{\Exp}[1]{\exp{\left({#1}\right)}}
%\newcommand{\Norm}[1]{\left\lVert{#1}\right\rVert}
%\newcommand{\Sin}[1]{\sin{\left({#1}\right)}}
%\newcommand{\T}[0]{\text{T}}
%\newcommand{\VolumeOp}[1]{\VName_{#1}}
%\newcommand{\agrad}[0]{\Ba \cdot \nabla}
%\newcommand{\alphacap}[0]{\hat{\boldsymbol{\alpha}}}
%\newcommand{\Fcap}[0]{\hat{\BF}}
%\newcommand{\bithree}[0]{{\Bi}_3}
%\newcommand{\bxa}[0]{\Bx\Ba}
%\newcommand{\coordvec}[2]{
%\newcommand{\costheta}[0]{\acap \cdot \xcap}
%\newcommand{\ddt}[1]{\ddot{#1}}
%\newcommand{\ddu}[1] {\frac {d{#1}} {du}}
%\newcommand{\dsqxj}[2] {\frac {\partial^2 {#1}} {\partial {x_{#2}}^2}}
%\newcommand{\dtheta}[1]{\frac{d {#1}}{d \theta}}
%\newcommand{\dt}[1]{\dot{#1}}
%\newcommand{\dt}[1]{\frac{d {#1}}{dt}}
%\newcommand{\dxj}[2] {\frac {\partial {#1}} {\partial {x_{#2}}}}
%\newcommand{\halfPhi}[0]{\frac{\phi}{2}}
%\newcommand{\half}[0]{\inv{2}}
%\newcommand{\inv}[1]{\frac{1}{#1}}
%\newcommand{\laplacian}[0]{\nabla^2}
%\newcommand{\matrixoftx}[3]{
%\newcommand{\nrrp}[0]{\norm{\rcap \wedge \Br'}}
%\newcommand{\oiint}{\bigcirc \hspace{-1.4em} \int \hspace{-.8em} \int}
%\newcommand{\transpose}[1]{{#1}^{\text{T}}}
%\newcommand{\transpose}[1]{{{#1}^{\TextTranspose}}}
%\newcommand{\transpose}[1]{{{#1}^{\text{T}}}}
%\newcommand{\barA}[0]{\bar{A}}
%\newcommand{\qbar}[0]{\bar{q}}
%\newcommand{\qdotbar}[0]{\dot{\bar{q}}}
%
%</infrequent>





%\usepackage{listings}
\usepackage{txfonts} % for ointctr... (also appears to make "prettier" \int and \sum's)
\usepackage[bookmarks=true]{hyperref}

\usepackage{color,cite,graphicx}
   % use colour in the document, put your citations as [1-4]
   % rather than [1,2,3,4] (it looks nicer, and the extended LaTeX2e
   % graphics package.
\usepackage{latexsym,amssymb,epsf} % don't remember if these are
   % needed, but their inclusion can't do any damage


\title{ Quantum Harmonic Oscillator. }
\author{Peeter Joot \quad peeter.joot@gmail.com }
\date{ April 19, 2009.  Last Revision: $Date: 2009/04/20 04:09:27 $ }

\begin{document}

\maketitle{}
\tableofcontents
\section{ Motivation. }

In \cite{byron1992mca} (chapter II), an operator solution to the
(one dimensional) quantum
harmonic oscillator problem is presented.  Try this in a more old fashioned way,
as a comparison.

We want to solve the Schr\"{o}dinger equation for a quadratic potential

\begin{align}\label{eqn:toSolve}
-\frac{\hbar^2}{2m} \psi + \inv{2} m \omega^2 x^2 \psi = i \hbar \PD{t}{\psi}
\end{align}

\section{ Setup. }

\subsection{ Separation of variables. }

Equation \ref{eqn:toSolve} is separable, and to do so we can write

\begin{align*}
\psi(x,t) = \phi(x) T(t)
\end{align*}

and proceed to form the separated equation

\begin{align*}
-\frac{\hbar^2}{2m} \frac{\phi''}{\phi} + \inv{2} m \omega^2 x^2 = i \hbar \frac{T'}{T} = \text{constant}
\end{align*}

Writing $E$ for the constant, the and solving for the time function we have

\begin{align*}
(\ln(T))' &= -i \frac{E}{\hbar} \\
\implies \\
\ln(T) &= -i \frac{Et}{\hbar} + \ln(A)
\end{align*}

where $A$ is some constant.  This yields

\begin{align*}
T(t) = A e^{ -i E t/\hbar }
\end{align*}

What remains is now to solve the spatial wave equation

\begin{align}\label{eqn:spatial}
\phi'' + \left( \frac{m^2 \omega^2}{\hbar^2} x^2 - \frac{2m E}{\hbar^2}\right) \phi = 0
\end{align}

This doesn't really look too much like the harmonic oscillator problem of classical physics.  Let's remind ourself
what that was like before continuing.

\subsection{ Mass on a spring. }

The harmonic oscillator problem from classical physics
shows up many times, and is usually first seen when examining motion
of a mass on a spring.  There we have a restoring force that accelerates the
mass in the opposite direction from its equilibrium position

\begin{align}
m \xddot = -k x
\end{align}

This has two complex exponential solutions.  With a substitution of the test function

\begin{align*}
x = e^{i \omega t}
\end{align*}

we have

\begin{align*}
(-m \omega^2 + k) e^{i \omega t} = 0
\end{align*}

So the test function is a solution provided

\begin{align*}
\omega^2 = \frac{k}{m}
\end{align*}

That doesn't really help understanding why \ref{eqn:spatial} is labeled the Harmonic oscillator.  Let's instead put the equation into an energy form.  The work done against the spring
(potential energy to be returned when the mass is released) is

\begin{align*}
W 
&= - \int F \cdot dx \\
&= - \int -kx dx \\
&= \inv{2} k (x^2 - x_0^2)
\end{align*}

So, our Lagrangian is

\begin{align*}
\LL = \inv{2}m \xdot^2 - \inv{2} k (x^2 - x_0^2)
\end{align*}

The constant term in the potential can be dropped, since it won't contribute to the equations of motion.  Our conjugate momentum $\PDi{\xdot}{\LL}$ is just $m \xdot$, and the Hamiltonian
is therefore

\begin{align*}
H 
&= \xdot \PD{\xdot}{\LL} - \LL \\
&= m \xdot^2  - \left( \inv{2}m \xdot^2 - \inv{2} k x^2 \right) \\
&= \inv{2} m \xdot^2  + \inv{2} k x^2 \\
\end{align*}

Or

\begin{align*}
H &= \frac{p^2}{2 m } + \inv{2} m \omega^2 x^2 \\
\end{align*}

Ah.  In this form we see the structure of the QM Harmonic oscillator equation.  With the position space representation of the momentum operator $p \sim -i \hbar \PDi{x}{}$ we have
something now similar to \ref{eqn:toSolve}.

\section{ Series solution. }

\subsection{ Assuming Gaussian solutions. }

Having seen the operator solution of the QM harmonic oscillator problem, we will cheat, and use that as a starting point.  Assume that
the solution can be expressed as a scaled Gaussian as in

\begin{align}
\phi(x) &= f(x) e^{ - \alpha x^2/2 }
\end{align}

\begin{align*}
\phi'(x) &= \left( f'(x) - \alpha x f(x) \right) e^{ - \alpha x^2/2 } \\
\phi''(x)
&=
\left( f''(x) - \alpha f(x) -\alpha x f'(x) \right) e^{ - \alpha x^2/2 }
\left( f'(x) - \alpha x f(x) \right) (-\alpha x) e^{ - \alpha x^2/2 } \\
&=
\left( f''(x) - \alpha f(x) - 2 \alpha x f'(x) + \alpha^2 x^2 f(x) \right) e^{ - \alpha x^2/2 }
\end{align*}

Substitution of the scaled Gaussian test solution, and its derivatives, gives us

\begin{align*}
\left( f'' - \alpha f - 2 \alpha x f' + \alpha^2 x^2 f + \left( \frac{m^2 \omega^2}{\hbar^2} x^2 - \frac{2 m E}{\hbar^2} \right) f \right) e^{ -\alpha x^2 /2} = 0
\end{align*}

Since the exponential is never zero, this requires a zero for the differential equation

\begin{align}
f'' - 2 \alpha x f' + \left( \left(\alpha^2 + \frac{m^2 \omega^2}{\hbar^2} \right) x^2 - \left(\frac{2 m E}{\hbar^2} + \alpha \right) \right) f = 0
\end{align}

Compared to \ref{eqn:spatial}, this doesn't really appear to be much of an improvement, but let's work with it, looking for a term by term power series solution.

Before doing so, a couple helper variable substitutions appear to be in order.  Let

\begin{align*}
\beta^2 &= \alpha^2 + \frac{m^2 \omega^2}{\hbar^2}  \\
\sigma &= \frac{2 m E}{\hbar^2} + \alpha 
\end{align*}

So the new differential equation to solve is

\begin{align}
f'' - 2 \alpha x f' + \beta^2 x^2 f - \sigma f = 0
\end{align}

Assuming various power series solutions of the form

\begin{align*}
f_n(x) &= \sum_{r=0}^n a_r x^r
\end{align*}

derivatives are
\begin{align*}
f_n' &= \sum_{r=0}^n r a_r x^{r-1} \\
f_n'' &= \sum_{r=0}^n r(r-1) a_r x^{r-2}
\end{align*}

We have 
\begin{align*}
0 &= \sum_{r=0}^{n-2} (r+2)(r+1) a_{r+2} x^{r}
 - 2 \alpha \sum_{r=1}^n r a_r x^{r}
+ \beta^2
\sum_{r=0}^n a_r x^{r+2} 
- \sigma \sum_{r=0}^n a_r x^r
\end{align*}

This should be enough to figure out recurrence relations for the various constants in the polynomials.

However, before trying to acquire the recurrence relations in their most general form, an attempt at a few simple cases, looking for the lowest order
polynomial solutions explicitly, gets into trouble.

Specifically, if I try $n=0$, $n=1$, $n=2$ equating each of the polynomial coefficients to zero keeps killing all the coefficients in sequence.  What's 
gone wrong?

\subsection{ Try zeroth order polynomial scaled Gaussian explicitly. }

Somewhere above things went wrong.  How about a plain old Gaussian?  Let's substitute 

\begin{align*}
\psi = A e^{\alpha x^2/2}
\end{align*}

into 

\begin{align*}
-\frac{\hbar^2}{2m} \psi'' + \left( \inv{2}m \omega^2 x^2 - E \right) \psi = 0
\end{align*}

Dropping the constant $A$ temporarily the derivatives are
\begin{align*}
\psi' &= \alpha x e^{\alpha x^2/2} \\
\psi'' &= ( \alpha^2 x^2 + \alpha ) e^{\alpha x^2/2} \\
\end{align*}

This gives
\begin{align*}
\left( -( \alpha^2 x^2 + \alpha )\frac{\hbar^2}{2 m} + \left( \inv{2}m \omega^2 x^2 - E \right) \right) e^{\alpha x^2/2} = 0
\end{align*}

Equating $x^2$ and $x^0$ terms we have

\begin{align*}
\frac{\alpha^2 \hbar^2}{2 m} &= \inv{2}m \omega^2  \\
\frac{\alpha \hbar^2}{2 m} &= -E 
\end{align*}

Or
\begin{align*}
\alpha &= \pm \frac{ m \omega }{ \hbar }  \\
{\alpha } &= - \frac{2 m E }{\hbar^2}
\end{align*}

Since $\omega = \sqrt{k/m}$ is a given, we want $E$ in terms of $\omega$.  Picking $\alpha$ negative for positive energy, this is

\begin{align*}
E = \frac{ \omega \hbar }{ 2 } 
\end{align*}

The solution to the $T'/T$ equation is thus

\begin{align*}
T \propto e^{-i \omega t/2}
\end{align*}

So, except for the undetermined constant normalization factor, we have one full solution of the wave equation, 

\begin{align*}
\psi(x,t) = A \exp\left( - \frac{m \omega x^2 }{2 \hbar} - i \frac{\omega t}{2} \right)
\end{align*}

The normalization

\begin{align*}
1 
&= \int \psi \psi^\conj \\
&= A^2 \int \exp\left( - \frac{m \omega x^2 }{\hbar} \right) \\
&= A^2 \sqrt{ \frac{\hbar \pi}{ m \omega } }
\end{align*}

For a normalized solution
\begin{align}
\psi(x,t) = \left( 
\frac{ m \omega }{\hbar \pi}
\right)^{1/4} \exp\left( - \frac{m \omega x^2 }{2 \hbar} - i \frac{\omega t}{2} \right)
\end{align}

\subsection{ Try first order polynomial scaled Gaussian. }

Next, let's try 

\begin{align*}
\psi = (x + A) e^{-\alpha x^2/2}
\end{align*}

No coefficient for the first order nomial has been used since we will need a scale factor in the end for normalization anyways.  Taking derivatives

\begin{align*}
\psi' 
&= (1 + (x + A)(-\alpha x)) e^{-\alpha x^2/2} \\
&= (1 - \alpha A x - \alpha x^2 ) e^{-\alpha x^2/2} \\
\end{align*}

\begin{align*}
\psi'' 
&= 
\left( - 3 \alpha x - \alpha A + \alpha^2 A x^2 + \alpha^2 x^3 \right) e^{-\alpha x^2/2} \\
\end{align*}

So we have

\begin{align*}
-\frac{\hbar^2}{2m}\left( - 3 \alpha x - \alpha A + \alpha^2 A x^2 + \alpha^2 x^3 \right) 
+ \left( \inv{2}m \omega^2 x^2 - E \right) (x + A) = 0
\end{align*}

Equating either cubic or squared terms provides $\alpha$

\begin{align*}
\alpha = \pm \frac{m \omega}{\hbar}
\end{align*}

This is what we had for the plain old Gaussian as well.

Equating either the first and scalar terms gives us the energy 

\begin{align*}
E = \frac{3 \alpha \hbar^2}{2m} = \frac{3 \omega \hbar}{2}
\end{align*}

which differs from the zeroth order case by a factor of three.

It's curious that the coefficient $A$ cannot be determined.  I didn't expect it to be a free parameter.  Is the normalization enough to
fix this and any other leading factor?

With
\begin{align*}
\psi = (B x + A) \exp( - m \omega x^2/2 \hbar - 3 i \omega t / 2 )
\end{align*}

\begin{align*}
1 
&= \int \psi\psi^\conj \\
&= \int (B^2 x^2 + 2 A B x + A^2 ) e^{- m \omega x^2/ \hbar } \\
&= \int (B^2 x^2 + A^2 ) e^{- m \omega x^2/ \hbar } \\
&= A^2 \sqrt{\frac{\pi \hbar}{m \omega}} + B^2 \inv{2 \pi} \left( \frac{\pi \hbar}{ m \omega} \right)^{3/2} \\
&= \sqrt{\frac{\pi \hbar}{m \omega}} \left( A^2 + B^2 \frac{ \hbar}{ 2 m \omega} \right) \\
\end{align*}

In terms of an arbitrary constant $B$, this gives

\begin{align*}
\end{align*}

It is perhaps reasonable to pick $B=1$.  Regardless, the general solution for this first order polynomial scaled Gaussian is

\begin{align*}
\psi(x,t) = 
\left(B x \pm \sqrt{\sqrt{\frac{m \omega}{\pi \hbar}} - B^2 \frac{ \hbar}{ 2 m \omega} }\right) \exp\left( - \frac{m \omega x^2}{2 \hbar} - \frac{3 i \omega t }{ 2} \right)
\end{align*}

I expected something a bit more simple, without this extra degree of freedom.  That probably has to come from the orthonormality conditions on the
Hermite polynomials.

Now, there isn't anything here that is particularly special in these two cases, so I'd expect the error has to be a plain old algebra problem 
hiding in there somewhere.

\bibliographystyle{plainnat}
\bibliography{myrefs}

\end{document}

\documentclass{article}

\usepackage{amsmath}
\usepackage{mathpazo}

%
% shorthand for bold symbols, convenient for vectors and matrices
%
\newcommand{\Ba}[0]{\mathbf{a}}
\newcommand{\Bb}[0]{\mathbf{b}}
\newcommand{\Bc}[0]{\mathbf{c}}
\newcommand{\Bd}[0]{\mathbf{d}}
\newcommand{\Be}[0]{\mathbf{e}}
\newcommand{\Bf}[0]{\mathbf{f}}
\newcommand{\Bg}[0]{\mathbf{g}}
\newcommand{\Bh}[0]{\mathbf{h}}
\newcommand{\Bi}[0]{\mathbf{i}}
\newcommand{\Bj}[0]{\mathbf{j}}
\newcommand{\Bk}[0]{\mathbf{k}}
\newcommand{\Bl}[0]{\mathbf{l}}
\newcommand{\Bm}[0]{\mathbf{m}}
\newcommand{\Bn}[0]{\mathbf{n}}
\newcommand{\Bo}[0]{\mathbf{o}}
\newcommand{\Bp}[0]{\mathbf{p}}
\newcommand{\Bq}[0]{\mathbf{q}}
\newcommand{\Br}[0]{\mathbf{r}}
\newcommand{\Bs}[0]{\mathbf{s}}
\newcommand{\Bt}[0]{\mathbf{t}}
\newcommand{\Bu}[0]{\mathbf{u}}
\newcommand{\Bv}[0]{\mathbf{v}}
\newcommand{\Bw}[0]{\mathbf{w}}
\newcommand{\Bx}[0]{\mathbf{x}}
\newcommand{\By}[0]{\mathbf{y}}
\newcommand{\Bz}[0]{\mathbf{z}}
\newcommand{\BA}[0]{\mathbf{A}}
\newcommand{\BB}[0]{\mathbf{B}}
\newcommand{\BC}[0]{\mathbf{C}}
\newcommand{\BD}[0]{\mathbf{D}}
\newcommand{\BE}[0]{\mathbf{E}}
\newcommand{\BF}[0]{\mathbf{F}}
\newcommand{\BG}[0]{\mathbf{G}}
\newcommand{\BH}[0]{\mathbf{H}}
\newcommand{\BI}[0]{\mathbf{I}}
\newcommand{\BJ}[0]{\mathbf{J}}
\newcommand{\BK}[0]{\mathbf{K}}
\newcommand{\BL}[0]{\mathbf{L}}
\newcommand{\BM}[0]{\mathbf{M}}
\newcommand{\BN}[0]{\mathbf{N}}
\newcommand{\BO}[0]{\mathbf{O}}
\newcommand{\BP}[0]{\mathbf{P}}
\newcommand{\BQ}[0]{\mathbf{Q}}
\newcommand{\BR}[0]{\mathbf{R}}
\newcommand{\BS}[0]{\mathbf{S}}
\newcommand{\BT}[0]{\mathbf{T}}
\newcommand{\BU}[0]{\mathbf{U}}
\newcommand{\BV}[0]{\mathbf{V}}
\newcommand{\BW}[0]{\mathbf{W}}
\newcommand{\BX}[0]{\mathbf{X}}
\newcommand{\BY}[0]{\mathbf{Y}}
\newcommand{\BZ}[0]{\mathbf{Z}}

\newcommand{\Bzero}[0]{\mathbf{0}}
\newcommand{\Btheta}[0]{\boldsymbol{\theta}}
\newcommand{\Btau}[0]{\boldsymbol{\tau}}
\newcommand{\Bomega}[0]{\boldsymbol{\omega}}

%
% shorthand for unit vectors
%
\newcommand{\acap}[0]{\hat{\Ba}}
\newcommand{\bcap}[0]{\hat{\Bb}}
\newcommand{\ccap}[0]{\hat{\Bc}}
\newcommand{\dcap}[0]{\hat{\Bd}}
\newcommand{\ecap}[0]{\hat{\Be}}
\newcommand{\fcap}[0]{\hat{\Bf}}
\newcommand{\gcap}[0]{\hat{\Bg}}
\newcommand{\hcap}[0]{\hat{\Bh}}
\newcommand{\icap}[0]{\hat{\Bi}}
\newcommand{\jcap}[0]{\hat{\Bj}}
\newcommand{\kcap}[0]{\hat{\Bk}}
\newcommand{\lcap}[0]{\hat{\Bl}}
\newcommand{\mcap}[0]{\hat{\Bm}}
\newcommand{\ncap}[0]{\hat{\Bn}}
\newcommand{\ocap}[0]{\hat{\Bo}}
\newcommand{\pcap}[0]{\hat{\Bp}}
\newcommand{\qcap}[0]{\hat{\Bq}}
\newcommand{\rcap}[0]{\hat{\Br}}
\newcommand{\scap}[0]{\hat{\Bs}}
\newcommand{\tcap}[0]{\hat{\Bt}}
\newcommand{\ucap}[0]{\hat{\Bu}}
\newcommand{\vcap}[0]{\hat{\Bv}}
\newcommand{\wcap}[0]{\hat{\Bw}}
\newcommand{\xcap}[0]{\hat{\Bx}}
\newcommand{\ycap}[0]{\hat{\By}}
\newcommand{\zcap}[0]{\hat{\Bz}}
\newcommand{\thetacap}[0]{\hat{\Btheta}}

%
% to write R^n and C^n in a distinguishable fashion.  Perhaps change this
% to the double lined characters upon figuring out how to do so.
%
\newcommand{\C}[1]{$\mathbb{C}^{#1}$}
\newcommand{\R}[1]{$\mathbb{R}^{#1}$}

%
% various generally useful helpers
%

% derivative of #1 wrt. #2:
\newcommand{\D}[2] {\frac {d#2} {d#1}}

\newcommand{\inv}[1]{\frac{1}{#1}}
\newcommand{\cross}[0]{\times}

\newcommand{\abs}[1]{\lvert{#1}\rvert}
\newcommand{\norm}[1]{\lVert{#1}\rVert}
\newcommand{\innerprod}[2]{\langle{#1}, {#2}\rangle}
\newcommand{\dotprod}[2]{{#1} \cdot {#2}}
\newcommand{\bdotprod}[2]{\left({#1} \cdot {#2}\right)}
\newcommand{\crossprod}[2]{{#1} \cross {#2}}
\newcommand{\tripleprod}[3]{\dotprod{\left(\crossprod{#1}{#2}\right)}{#3}}

\DeclareMathOperator{\Proj}{Proj}
\DeclareMathOperator{\Span}{span}
\DeclareMathOperator{\Sgn}{sgn}
\DeclareMathOperator{\Area}{Area}
\DeclareMathOperator{\Volume}{Volume}

%
% A few miscellaneous things specific to this document
%
\newcommand{\crossop}[1]{\crossprod{#1}{}}

% R2 vector.
\newcommand{\VectorTwo}[2]{
\begin{bmatrix}
 {#1} \\
 {#2}
\end{bmatrix}
}

\newcommand{\VectorN}[1]{
\begin{bmatrix}
{#1}_1 \\
{#1}_2 \\
\vdots \\
{#1}_N \\
\end{bmatrix}
}

\newcommand{\DETuvij}[4]{
\begin{vmatrix}
 {#1}_{#3} & {#1}_{#4} \\
 {#2}_{#3} & {#2}_{#4}
\end{vmatrix}
}

\newcommand{\DETuvwijk}[6]{
\begin{vmatrix}
 {#1}_{#4} & {#1}_{#5} & {#1}_{#6} \\
 {#2}_{#4} & {#2}_{#5} & {#2}_{#6} \\
 {#3}_{#4} & {#3}_{#5} & {#3}_{#6}
\end{vmatrix}
}

\newcommand{\DETuvwxijkl}[8]{
\begin{vmatrix}
 {#1}_{#5} & {#1}_{#6} & {#1}_{#7} & {#1}_{#8} \\
 {#2}_{#5} & {#2}_{#6} & {#2}_{#7} & {#2}_{#8} \\
 {#3}_{#5} & {#3}_{#6} & {#3}_{#7} & {#3}_{#8} \\
 {#4}_{#5} & {#4}_{#6} & {#4}_{#7} & {#4}_{#8} \\
\end{vmatrix}
}

%\newcommand{\DETuvwxyijklm}[10]{
%\begin{vmatrix}
% {#1}_{#6} & {#1}_{#7} & {#1}_{#8} & {#1}_{#9} & {#1}_{#10} \\
% {#2}_{#6} & {#2}_{#7} & {#2}_{#8} & {#2}_{#9} & {#2}_{#10} \\
% {#3}_{#6} & {#3}_{#7} & {#3}_{#8} & {#3}_{#9} & {#3}_{#10} \\
% {#4}_{#6} & {#4}_{#7} & {#4}_{#8} & {#4}_{#9} & {#4}_{#10} \\
% {#5}_{#6} & {#5}_{#7} & {#5}_{#8} & {#5}_{#9} & {#5}_{#10}
%\end{vmatrix}
%}

% R3 vector.
\newcommand{\VectorThree}[3]{
\begin{bmatrix}
 {#1} \\
 {#2} \\
 {#3}
\end{bmatrix}
}


%<misc>
%
\newcommand{\Abs}[1]{{\left\lvert{#1}\right\rvert}}
\newcommand{\spacegrad}[0]{\boldsymbol{\nabla}}
\newcommand{\grad}[0]{\nabla}
\newcommand{\LL}[0]{\mathcal{L}}

% == \partial_{#1} {#2}
\newcommand{\PD}[2]{\frac{\partial {#2}}{\partial {#1}}}
% inline variant
\newcommand{\PDi}[2]{{\partial {#2}}/{\partial {#1}}}

\newcommand{\PDD}[3]{\frac{\partial^2 {#3}}{\partial {#1}\partial {#2}}}
%\newcommand{\PDd}[2]{\frac{\partial^2 {#2}}{{\partial{#1}}^2}}
\newcommand{\PDsq}[2]{\frac{\partial^2 {#2}}{(\partial {#1})^2}}

\newcommand{\Partial}[2]{\frac{\partial {#1}}{\partial {#2}}}
\DeclareMathOperator{\RejName}{Rej}
\newcommand{\Rej}[2]{\RejName_{#1}\left( {#2} \right)}
\newcommand{\Rm}[1]{\mathbb{R}^{#1}}
\newcommand{\Cm}[1]{\mathbb{C}^{#1}}
\newcommand{\conj}[0]{{*}}

%</misc>

% <grade selection>
%
\newcommand{\gpgrade}[2] {{\left\langle{{#1}}\right\rangle}_{#2}}

\newcommand{\gpgradezero}[1] {\gpgrade{#1}{}}
%\newcommand{\gpscalargrade}[1] {{\left\langle{{#1}}\right\rangle}}
%\newcommand{\gpgradezero}[1] {\gpgrade{#1}{0}}

%\newcommand{\gpgradeone}[1] {{\left\langle{{#1}}\right\rangle}_{1}}
\newcommand{\gpgradeone}[1] {\gpgrade{#1}{1}}

\newcommand{\gpgradetwo}[1] {\gpgrade{#1}{2}}
\newcommand{\gpgradethree}[1] {\gpgrade{#1}{3}}
\newcommand{\gpgradefour}[1] {\gpgrade{#1}{4}}
%
% </grade selection>



\newcommand{\adot}[0]{{\dot{a}}}
\newcommand{\bdot}[0]{{\dot{b}}}
% taken for centered dot:
%\newcommand{\cdot}[0]{{\dot{c}}}
%\newcommand{\ddot}[0]{{\dot{d}}}
\newcommand{\edot}[0]{{\dot{e}}}
\newcommand{\fdot}[0]{{\dot{f}}}
\newcommand{\gdot}[0]{{\dot{g}}}
\newcommand{\hdot}[0]{{\dot{h}}}
\newcommand{\idot}[0]{{\dot{i}}}
\newcommand{\jdot}[0]{{\dot{j}}}
\newcommand{\kdot}[0]{{\dot{k}}}
\newcommand{\ldot}[0]{{\dot{l}}}
\newcommand{\mdot}[0]{{\dot{m}}}
\newcommand{\ndot}[0]{{\dot{n}}}
%\newcommand{\odot}[0]{{\dot{o}}}
\newcommand{\pdot}[0]{{\dot{p}}}
\newcommand{\qdot}[0]{{\dot{q}}}
\newcommand{\rdot}[0]{{\dot{r}}}
\newcommand{\sdot}[0]{{\dot{s}}}
\newcommand{\tdot}[0]{{\dot{t}}}
\newcommand{\udot}[0]{{\dot{u}}}
\newcommand{\vdot}[0]{{\dot{v}}}
\newcommand{\wdot}[0]{{\dot{w}}}
\newcommand{\xdot}[0]{{\dot{x}}}
\newcommand{\ydot}[0]{{\dot{y}}}
\newcommand{\zdot}[0]{{\dot{z}}}
\newcommand{\addot}[0]{{\ddot{a}}}
\newcommand{\bddot}[0]{{\ddot{b}}}
\newcommand{\cddot}[0]{{\ddot{c}}}
%\newcommand{\dddot}[0]{{\ddot{d}}}
\newcommand{\eddot}[0]{{\ddot{e}}}
\newcommand{\fddot}[0]{{\ddot{f}}}
\newcommand{\gddot}[0]{{\ddot{g}}}
\newcommand{\hddot}[0]{{\ddot{h}}}
\newcommand{\iddot}[0]{{\ddot{i}}}
\newcommand{\jddot}[0]{{\ddot{j}}}
\newcommand{\kddot}[0]{{\ddot{k}}}
\newcommand{\lddot}[0]{{\ddot{l}}}
\newcommand{\mddot}[0]{{\ddot{m}}}
\newcommand{\nddot}[0]{{\ddot{n}}}
\newcommand{\oddot}[0]{{\ddot{o}}}
\newcommand{\pddot}[0]{{\ddot{p}}}
\newcommand{\qddot}[0]{{\ddot{q}}}
\newcommand{\rddot}[0]{{\ddot{r}}}
\newcommand{\sddot}[0]{{\ddot{s}}}
\newcommand{\tddot}[0]{{\ddot{t}}}
\newcommand{\uddot}[0]{{\ddot{u}}}
\newcommand{\vddot}[0]{{\ddot{v}}}
\newcommand{\wddot}[0]{{\ddot{w}}}
\newcommand{\xddot}[0]{{\ddot{x}}}
\newcommand{\yddot}[0]{{\ddot{y}}}
\newcommand{\zddot}[0]{{\ddot{z}}}

%<bold and dot greek symbols>
%

\newcommand{\Deltadot}[0]{{\dot{\Delta}}}
\newcommand{\Gammadot}[0]{{\dot{\Gamma}}}
\newcommand{\Lambdadot}[0]{{\dot{\Lambda}}}
\newcommand{\Omegadot}[0]{{\dot{\Omega}}}
\newcommand{\Phidot}[0]{{\dot{\Phi}}}
\newcommand{\Pidot}[0]{{\dot{\Pi}}}
\newcommand{\Psidot}[0]{{\dot{\Psi}}}
\newcommand{\Sigmadot}[0]{{\dot{\Sigma}}}
\newcommand{\Thetadot}[0]{{\dot{\Theta}}}
\newcommand{\Upsilondot}[0]{{\dot{\Upsilon}}}
\newcommand{\Xidot}[0]{{\dot{\Xi}}}
\newcommand{\alphadot}[0]{{\dot{\alpha}}}
\newcommand{\betadot}[0]{{\dot{\beta}}}
\newcommand{\chidot}[0]{{\dot{\chi}}}
\newcommand{\deltadot}[0]{{\dot{\delta}}}
\newcommand{\epsilondot}[0]{{\dot{\epsilon}}}
\newcommand{\etadot}[0]{{\dot{\eta}}}
\newcommand{\gammadot}[0]{{\dot{\gamma}}}
\newcommand{\kappadot}[0]{{\dot{\kappa}}}
\newcommand{\lambdadot}[0]{{\dot{\lambda}}}
\newcommand{\mudot}[0]{{\dot{\mu}}}
\newcommand{\nudot}[0]{{\dot{\nu}}}
\newcommand{\omegadot}[0]{{\dot{\omega}}}
\newcommand{\phidot}[0]{{\dot{\phi}}}
\newcommand{\pidot}[0]{{\dot{\pi}}}
\newcommand{\psidot}[0]{{\dot{\psi}}}
\newcommand{\rhodot}[0]{{\dot{\rho}}}
\newcommand{\sigmadot}[0]{{\dot{\sigma}}}
\newcommand{\taudot}[0]{{\dot{\tau}}}
\newcommand{\thetadot}[0]{{\dot{\theta}}}
\newcommand{\upsilondot}[0]{{\dot{\upsilon}}}
\newcommand{\varepsilondot}[0]{{\dot{\varepsilon}}}
\newcommand{\varphidot}[0]{{\dot{\varphi}}}
\newcommand{\varpidot}[0]{{\dot{\varpi}}}
\newcommand{\varrhodot}[0]{{\dot{\varrho}}}
\newcommand{\varsigmadot}[0]{{\dot{\varsigma}}}
\newcommand{\varthetadot}[0]{{\dot{\vartheta}}}
\newcommand{\xidot}[0]{{\dot{\xi}}}
\newcommand{\zetadot}[0]{{\dot{\zeta}}}

\newcommand{\Deltaddot}[0]{{\ddot{\Delta}}}
\newcommand{\Gammaddot}[0]{{\ddot{\Gamma}}}
\newcommand{\Lambdaddot}[0]{{\ddot{\Lambda}}}
\newcommand{\Omegaddot}[0]{{\ddot{\Omega}}}
\newcommand{\Phiddot}[0]{{\ddot{\Phi}}}
\newcommand{\Piddot}[0]{{\ddot{\Pi}}}
\newcommand{\Psiddot}[0]{{\ddot{\Psi}}}
\newcommand{\Sigmaddot}[0]{{\ddot{\Sigma}}}
\newcommand{\Thetaddot}[0]{{\ddot{\Theta}}}
\newcommand{\Upsilonddot}[0]{{\ddot{\Upsilon}}}
\newcommand{\Xiddot}[0]{{\ddot{\Xi}}}
\newcommand{\alphaddot}[0]{{\ddot{\alpha}}}
\newcommand{\betaddot}[0]{{\ddot{\beta}}}
\newcommand{\chiddot}[0]{{\ddot{\chi}}}
\newcommand{\deltaddot}[0]{{\ddot{\delta}}}
\newcommand{\epsilonddot}[0]{{\ddot{\epsilon}}}
\newcommand{\etaddot}[0]{{\ddot{\eta}}}
\newcommand{\gammaddot}[0]{{\ddot{\gamma}}}
\newcommand{\kappaddot}[0]{{\ddot{\kappa}}}
\newcommand{\lambdaddot}[0]{{\ddot{\lambda}}}
\newcommand{\muddot}[0]{{\ddot{\mu}}}
\newcommand{\nuddot}[0]{{\ddot{\nu}}}
\newcommand{\omegaddot}[0]{{\ddot{\omega}}}
\newcommand{\phiddot}[0]{{\ddot{\phi}}}
\newcommand{\piddot}[0]{{\ddot{\pi}}}
\newcommand{\psiddot}[0]{{\ddot{\psi}}}
\newcommand{\rhoddot}[0]{{\ddot{\rho}}}
\newcommand{\sigmaddot}[0]{{\ddot{\sigma}}}
\newcommand{\tauddot}[0]{{\ddot{\tau}}}
\newcommand{\thetaddot}[0]{{\ddot{\theta}}}
\newcommand{\upsilonddot}[0]{{\ddot{\upsilon}}}
\newcommand{\varepsilonddot}[0]{{\ddot{\varepsilon}}}
\newcommand{\varphiddot}[0]{{\ddot{\varphi}}}
\newcommand{\varpiddot}[0]{{\ddot{\varpi}}}
\newcommand{\varrhoddot}[0]{{\ddot{\varrho}}}
\newcommand{\varsigmaddot}[0]{{\ddot{\varsigma}}}
\newcommand{\varthetaddot}[0]{{\ddot{\vartheta}}}
\newcommand{\xiddot}[0]{{\ddot{\xi}}}
\newcommand{\zetaddot}[0]{{\ddot{\zeta}}}

\newcommand{\BDelta}[0]{\boldsymbol{\Delta}}
\newcommand{\BGamma}[0]{\boldsymbol{\Gamma}}
\newcommand{\BLambda}[0]{\boldsymbol{\Lambda}}
\newcommand{\BOmega}[0]{\boldsymbol{\Omega}}
\newcommand{\BPhi}[0]{\boldsymbol{\Phi}}
\newcommand{\BPi}[0]{\boldsymbol{\Pi}}
\newcommand{\BPsi}[0]{\boldsymbol{\Psi}}
\newcommand{\BSigma}[0]{\boldsymbol{\Sigma}}
\newcommand{\BTheta}[0]{\boldsymbol{\Theta}}
\newcommand{\BUpsilon}[0]{\boldsymbol{\Upsilon}}
\newcommand{\BXi}[0]{\boldsymbol{\Xi}}
\newcommand{\Balpha}[0]{\boldsymbol{\alpha}}
\newcommand{\Bbeta}[0]{\boldsymbol{\beta}}
\newcommand{\Bchi}[0]{\boldsymbol{\chi}}
\newcommand{\Bdelta}[0]{\boldsymbol{\delta}}
\newcommand{\Bepsilon}[0]{\boldsymbol{\epsilon}}
\newcommand{\Beta}[0]{\boldsymbol{\eta}}
\newcommand{\Bgamma}[0]{\boldsymbol{\gamma}}
\newcommand{\Bkappa}[0]{\boldsymbol{\kappa}}
\newcommand{\Blambda}[0]{\boldsymbol{\lambda}}
\newcommand{\Bmu}[0]{\boldsymbol{\mu}}
\newcommand{\Bnu}[0]{\boldsymbol{\nu}}
%\newcommand{\Bomega}[0]{\boldsymbol{\omega}}
\newcommand{\Bphi}[0]{\boldsymbol{\phi}}
\newcommand{\Bpi}[0]{\boldsymbol{\pi}}
\newcommand{\Bpsi}[0]{\boldsymbol{\psi}}
\newcommand{\Brho}[0]{\boldsymbol{\rho}}
\newcommand{\Bsigma}[0]{\boldsymbol{\sigma}}
%\newcommand{\Btau}[0]{\boldsymbol{\tau}}
%\newcommand{\Btheta}[0]{\boldsymbol{\theta}}
\newcommand{\Bupsilon}[0]{\boldsymbol{\upsilon}}
\newcommand{\Bvarepsilon}[0]{\boldsymbol{\varepsilon}}
\newcommand{\Bvarphi}[0]{\boldsymbol{\varphi}}
\newcommand{\Bvarpi}[0]{\boldsymbol{\varpi}}
\newcommand{\Bvarrho}[0]{\boldsymbol{\varrho}}
\newcommand{\Bvarsigma}[0]{\boldsymbol{\varsigma}}
\newcommand{\Bvartheta}[0]{\boldsymbol{\vartheta}}
\newcommand{\Bxi}[0]{\boldsymbol{\xi}}
\newcommand{\Bzeta}[0]{\boldsymbol{\zeta}}
%
%</bold and dot greek symbols>
%<infrequent>
%
%\newcommand{\AreaOp}[1]{\AName_{#1}}
%\newcommand{\Babs}[0]{\abs{\BB}}
%\newcommand{\Bcap}[0]{\hat{\BB}}
%\newcommand{\BrPrimeRej}[0]{\rcap(\rcap \wedge \Br')}
%\newcommand{\CA}[0]{\mathcal{A}}
%\newcommand{\Cos}[1]{\cos{\left({#1}\right)}}
%\newcommand{\Det}[1] {\abs{#1}}
%\newcommand{\Dsq}[2] {\frac {\partial^2 {#1}} {\partial {#2}^2}}
%\newcommand{\Exp}[1]{\exp{\left({#1}\right)}}
%\newcommand{\Norm}[1]{\left\lVert{#1}\right\rVert}
%\newcommand{\Sin}[1]{\sin{\left({#1}\right)}}
%\newcommand{\T}[0]{\text{T}}
%\newcommand{\VolumeOp}[1]{\VName_{#1}}
%\newcommand{\agrad}[0]{\Ba \cdot \nabla}
%\newcommand{\alphacap}[0]{\hat{\boldsymbol{\alpha}}}
%\newcommand{\Fcap}[0]{\hat{\BF}}
%\newcommand{\bithree}[0]{{\Bi}_3}
%\newcommand{\bxa}[0]{\Bx\Ba}
%\newcommand{\coordvec}[2]{
%\newcommand{\costheta}[0]{\acap \cdot \xcap}
%\newcommand{\ddt}[1]{\ddot{#1}}
%\newcommand{\ddu}[1] {\frac {d{#1}} {du}}
%\newcommand{\dsqxj}[2] {\frac {\partial^2 {#1}} {\partial {x_{#2}}^2}}
%\newcommand{\dtheta}[1]{\frac{d {#1}}{d \theta}}
%\newcommand{\dt}[1]{\dot{#1}}
%\newcommand{\dt}[1]{\frac{d {#1}}{dt}}
%\newcommand{\dxj}[2] {\frac {\partial {#1}} {\partial {x_{#2}}}}
%\newcommand{\halfPhi}[0]{\frac{\phi}{2}}
%\newcommand{\half}[0]{\inv{2}}
%\newcommand{\inv}[1]{\frac{1}{#1}}
%\newcommand{\laplacian}[0]{\nabla^2}
%\newcommand{\matrixoftx}[3]{
%\newcommand{\nrrp}[0]{\norm{\rcap \wedge \Br'}}
%\newcommand{\oiint}{\bigcirc \hspace{-1.4em} \int \hspace{-.8em} \int}
%\newcommand{\transpose}[1]{{#1}^{\text{T}}}
%\newcommand{\transpose}[1]{{{#1}^{\TextTranspose}}}
%\newcommand{\transpose}[1]{{{#1}^{\text{T}}}}
%\newcommand{\barA}[0]{\bar{A}}
%\newcommand{\qbar}[0]{\bar{q}}
%\newcommand{\qdotbar}[0]{\dot{\bar{q}}}
%
%</infrequent>





%\usepackage{listings}
%\usepackage{txfonts} % for ointctr... (also appears to make "prettier" \int and \sum's)
\usepackage[bookmarks=true]{hyperref}

\usepackage{color,cite,graphicx}
   % use colour in the document, put your citations as [1-4]
   % rather than [1,2,3,4] (it looks nicer, and the extended LaTeX2e
   % graphics package. 
\usepackage{latexsym,amssymb,epsf} % don't remember if these are
   % needed, but their inclusion can't do any damage


\title{ Some Klien-Gordon equation notes. }
\author{Peeter Joot \quad peeter.joot@gmail.com }
\date{ March 27, 2009.  Last Revision: $Date: 2009/03/27 22:53:49 $ }

\begin{document}

\maketitle{}
\tableofcontents
\section{ Motivation }

Want to explore the ideas of global and local gauge invariance.  I seem to recall that Susskind
used the Klien-Gordon Lagrangian, which had a form something like

\begin{align}\label{eqn:densityUndeterminedCoeff}
\LL = \partial^\mu \psi \partial_\mu \psi^\conj + \alpha m^2 \psi \psi^\conj
\end{align}

Since this was one of the simplest forms to apply the 
a relativisitic gauge transformation concept to.

\subsection{ Determine that constant. }

We want

\begin{align*}
\grad^2 psi = \frac{m^2 c^2}{\hbar^2} \psi
\end{align*}

So, to start things off let's do the variation on the Langrangian density of \ref{eqn:densityUndeterminedCoeff}.  For fun, let's try it with Feynman's method (first order Taylor expansion, and no memorization
of the field form of the Euler-Lagrange equations).

Write $\psi = \phi + \epsilon$, where $\phi$ is the desired solution and $\epsilon$ is a field that
vanishes on the boundaries of the action integral

\begin{align*}
S 
&= \int \LL d^4 x \\
%&= 
%\int \left( \partial^\mu (\phi + \epsilon) \partial_\mu (\phi + \epsilon)^\conj + \alpha m^2 (\phi + \epsilon) (\phi + \epsilon)^\conj \right) d^4 x \\
%&= 
%\int d^4 x \partial^\mu \phi \partial_\mu \phi^\conj 
%+ \alpha m^2 \int \phi \phi^\conj \right) d^4 x \\
%&+ \int d^4 x \partial^\mu \epsilon \partial_\mu \phi^\conj 
%+ \int d^4 x \partial^\mu \phi \partial_\mu \epsilon^\conj  \\
%&+ \alpha m^2 \int \phi \epsilon^\conj \right) d^4 x
%+ \alpha m^2 \int \epsilon \phi^\conj \right) d^4 x \\
%&+ \alpha m^2 \int \epsilon \epsilon^\conj \right) d^4 x
%+ \int d^4 x \partial^\mu \epsilon \partial_\mu \epsilon^\conj \\
\end{align*}

\bibliographystyle{plainnat}
\bibliography{myrefs}

\end{document}

\documentclass{article}

\usepackage{amsmath}
\usepackage{mathpazo}

%
% shorthand for bold symbols, convenient for vectors and matrices
%
\newcommand{\Ba}[0]{\mathbf{a}}
\newcommand{\Bb}[0]{\mathbf{b}}
\newcommand{\Bc}[0]{\mathbf{c}}
\newcommand{\Bd}[0]{\mathbf{d}}
\newcommand{\Be}[0]{\mathbf{e}}
\newcommand{\Bf}[0]{\mathbf{f}}
\newcommand{\Bg}[0]{\mathbf{g}}
\newcommand{\Bh}[0]{\mathbf{h}}
\newcommand{\Bi}[0]{\mathbf{i}}
\newcommand{\Bj}[0]{\mathbf{j}}
\newcommand{\Bk}[0]{\mathbf{k}}
\newcommand{\Bl}[0]{\mathbf{l}}
\newcommand{\Bm}[0]{\mathbf{m}}
\newcommand{\Bn}[0]{\mathbf{n}}
\newcommand{\Bo}[0]{\mathbf{o}}
\newcommand{\Bp}[0]{\mathbf{p}}
\newcommand{\Bq}[0]{\mathbf{q}}
\newcommand{\Br}[0]{\mathbf{r}}
\newcommand{\Bs}[0]{\mathbf{s}}
\newcommand{\Bt}[0]{\mathbf{t}}
\newcommand{\Bu}[0]{\mathbf{u}}
\newcommand{\Bv}[0]{\mathbf{v}}
\newcommand{\Bw}[0]{\mathbf{w}}
\newcommand{\Bx}[0]{\mathbf{x}}
\newcommand{\By}[0]{\mathbf{y}}
\newcommand{\Bz}[0]{\mathbf{z}}
\newcommand{\BA}[0]{\mathbf{A}}
\newcommand{\BB}[0]{\mathbf{B}}
\newcommand{\BC}[0]{\mathbf{C}}
\newcommand{\BD}[0]{\mathbf{D}}
\newcommand{\BE}[0]{\mathbf{E}}
\newcommand{\BF}[0]{\mathbf{F}}
\newcommand{\BG}[0]{\mathbf{G}}
\newcommand{\BH}[0]{\mathbf{H}}
\newcommand{\BI}[0]{\mathbf{I}}
\newcommand{\BJ}[0]{\mathbf{J}}
\newcommand{\BK}[0]{\mathbf{K}}
\newcommand{\BL}[0]{\mathbf{L}}
\newcommand{\BM}[0]{\mathbf{M}}
\newcommand{\BN}[0]{\mathbf{N}}
\newcommand{\BO}[0]{\mathbf{O}}
\newcommand{\BP}[0]{\mathbf{P}}
\newcommand{\BQ}[0]{\mathbf{Q}}
\newcommand{\BR}[0]{\mathbf{R}}
\newcommand{\BS}[0]{\mathbf{S}}
\newcommand{\BT}[0]{\mathbf{T}}
\newcommand{\BU}[0]{\mathbf{U}}
\newcommand{\BV}[0]{\mathbf{V}}
\newcommand{\BW}[0]{\mathbf{W}}
\newcommand{\BX}[0]{\mathbf{X}}
\newcommand{\BY}[0]{\mathbf{Y}}
\newcommand{\BZ}[0]{\mathbf{Z}}

\newcommand{\Bzero}[0]{\mathbf{0}}
\newcommand{\Btheta}[0]{\boldsymbol{\theta}}
\newcommand{\Btau}[0]{\boldsymbol{\tau}}
\newcommand{\Bomega}[0]{\boldsymbol{\omega}}

%
% shorthand for unit vectors
%
\newcommand{\acap}[0]{\hat{\Ba}}
\newcommand{\bcap}[0]{\hat{\Bb}}
\newcommand{\ccap}[0]{\hat{\Bc}}
\newcommand{\dcap}[0]{\hat{\Bd}}
\newcommand{\ecap}[0]{\hat{\Be}}
\newcommand{\fcap}[0]{\hat{\Bf}}
\newcommand{\gcap}[0]{\hat{\Bg}}
\newcommand{\hcap}[0]{\hat{\Bh}}
\newcommand{\icap}[0]{\hat{\Bi}}
\newcommand{\jcap}[0]{\hat{\Bj}}
\newcommand{\kcap}[0]{\hat{\Bk}}
\newcommand{\lcap}[0]{\hat{\Bl}}
\newcommand{\mcap}[0]{\hat{\Bm}}
\newcommand{\ncap}[0]{\hat{\Bn}}
\newcommand{\ocap}[0]{\hat{\Bo}}
\newcommand{\pcap}[0]{\hat{\Bp}}
\newcommand{\qcap}[0]{\hat{\Bq}}
\newcommand{\rcap}[0]{\hat{\Br}}
\newcommand{\scap}[0]{\hat{\Bs}}
\newcommand{\tcap}[0]{\hat{\Bt}}
\newcommand{\ucap}[0]{\hat{\Bu}}
\newcommand{\vcap}[0]{\hat{\Bv}}
\newcommand{\wcap}[0]{\hat{\Bw}}
\newcommand{\xcap}[0]{\hat{\Bx}}
\newcommand{\ycap}[0]{\hat{\By}}
\newcommand{\zcap}[0]{\hat{\Bz}}
\newcommand{\thetacap}[0]{\hat{\Btheta}}

%
% to write R^n and C^n in a distinguishable fashion.  Perhaps change this
% to the double lined characters upon figuring out how to do so.
%
\newcommand{\C}[1]{$\mathbb{C}^{#1}$}
\newcommand{\R}[1]{$\mathbb{R}^{#1}$}

%
% various generally useful helpers
%

% derivative of #1 wrt. #2:
\newcommand{\D}[2] {\frac {d#2} {d#1}}

\newcommand{\inv}[1]{\frac{1}{#1}}
\newcommand{\cross}[0]{\times}

\newcommand{\abs}[1]{\lvert{#1}\rvert}
\newcommand{\norm}[1]{\lVert{#1}\rVert}
\newcommand{\innerprod}[2]{\langle{#1}, {#2}\rangle}
\newcommand{\dotprod}[2]{{#1} \cdot {#2}}
\newcommand{\bdotprod}[2]{\left({#1} \cdot {#2}\right)}
\newcommand{\crossprod}[2]{{#1} \cross {#2}}
\newcommand{\tripleprod}[3]{\dotprod{\left(\crossprod{#1}{#2}\right)}{#3}}

\DeclareMathOperator{\Proj}{Proj}
\DeclareMathOperator{\Span}{span}
\DeclareMathOperator{\Sgn}{sgn}
\DeclareMathOperator{\Area}{Area}
\DeclareMathOperator{\Volume}{Volume}

%
% A few miscellaneous things specific to this document
%
\newcommand{\crossop}[1]{\crossprod{#1}{}}

% R2 vector.
\newcommand{\VectorTwo}[2]{
\begin{bmatrix}
 {#1} \\
 {#2}
\end{bmatrix}
}

\newcommand{\VectorN}[1]{
\begin{bmatrix}
{#1}_1 \\
{#1}_2 \\
\vdots \\
{#1}_N \\
\end{bmatrix}
}

\newcommand{\DETuvij}[4]{
\begin{vmatrix}
 {#1}_{#3} & {#1}_{#4} \\
 {#2}_{#3} & {#2}_{#4}
\end{vmatrix}
}

\newcommand{\DETuvwijk}[6]{
\begin{vmatrix}
 {#1}_{#4} & {#1}_{#5} & {#1}_{#6} \\
 {#2}_{#4} & {#2}_{#5} & {#2}_{#6} \\
 {#3}_{#4} & {#3}_{#5} & {#3}_{#6}
\end{vmatrix}
}

\newcommand{\DETuvwxijkl}[8]{
\begin{vmatrix}
 {#1}_{#5} & {#1}_{#6} & {#1}_{#7} & {#1}_{#8} \\
 {#2}_{#5} & {#2}_{#6} & {#2}_{#7} & {#2}_{#8} \\
 {#3}_{#5} & {#3}_{#6} & {#3}_{#7} & {#3}_{#8} \\
 {#4}_{#5} & {#4}_{#6} & {#4}_{#7} & {#4}_{#8} \\
\end{vmatrix}
}

%\newcommand{\DETuvwxyijklm}[10]{
%\begin{vmatrix}
% {#1}_{#6} & {#1}_{#7} & {#1}_{#8} & {#1}_{#9} & {#1}_{#10} \\
% {#2}_{#6} & {#2}_{#7} & {#2}_{#8} & {#2}_{#9} & {#2}_{#10} \\
% {#3}_{#6} & {#3}_{#7} & {#3}_{#8} & {#3}_{#9} & {#3}_{#10} \\
% {#4}_{#6} & {#4}_{#7} & {#4}_{#8} & {#4}_{#9} & {#4}_{#10} \\
% {#5}_{#6} & {#5}_{#7} & {#5}_{#8} & {#5}_{#9} & {#5}_{#10}
%\end{vmatrix}
%}

% R3 vector.
\newcommand{\VectorThree}[3]{
\begin{bmatrix}
 {#1} \\
 {#2} \\
 {#3}
\end{bmatrix}
}


%<misc>
%
\newcommand{\Abs}[1]{{\left\lvert{#1}\right\rvert}}
\newcommand{\spacegrad}[0]{\boldsymbol{\nabla}}
\newcommand{\grad}[0]{\nabla}
\newcommand{\LL}[0]{\mathcal{L}}

% == \partial_{#1} {#2}
\newcommand{\PD}[2]{\frac{\partial {#2}}{\partial {#1}}}
% inline variant
\newcommand{\PDi}[2]{{\partial {#2}}/{\partial {#1}}}

\newcommand{\PDD}[3]{\frac{\partial^2 {#3}}{\partial {#1}\partial {#2}}}
%\newcommand{\PDd}[2]{\frac{\partial^2 {#2}}{{\partial{#1}}^2}}
\newcommand{\PDsq}[2]{\frac{\partial^2 {#2}}{(\partial {#1})^2}}

\newcommand{\Partial}[2]{\frac{\partial {#1}}{\partial {#2}}}
\DeclareMathOperator{\RejName}{Rej}
\newcommand{\Rej}[2]{\RejName_{#1}\left( {#2} \right)}
\newcommand{\Rm}[1]{\mathbb{R}^{#1}}
\newcommand{\Cm}[1]{\mathbb{C}^{#1}}
\newcommand{\conj}[0]{{*}}

%</misc>

% <grade selection>
%
\newcommand{\gpgrade}[2] {{\left\langle{{#1}}\right\rangle}_{#2}}

\newcommand{\gpgradezero}[1] {\gpgrade{#1}{}}
%\newcommand{\gpscalargrade}[1] {{\left\langle{{#1}}\right\rangle}}
%\newcommand{\gpgradezero}[1] {\gpgrade{#1}{0}}

%\newcommand{\gpgradeone}[1] {{\left\langle{{#1}}\right\rangle}_{1}}
\newcommand{\gpgradeone}[1] {\gpgrade{#1}{1}}

\newcommand{\gpgradetwo}[1] {\gpgrade{#1}{2}}
\newcommand{\gpgradethree}[1] {\gpgrade{#1}{3}}
\newcommand{\gpgradefour}[1] {\gpgrade{#1}{4}}
%
% </grade selection>



\newcommand{\adot}[0]{{\dot{a}}}
\newcommand{\bdot}[0]{{\dot{b}}}
% taken for centered dot:
%\newcommand{\cdot}[0]{{\dot{c}}}
%\newcommand{\ddot}[0]{{\dot{d}}}
\newcommand{\edot}[0]{{\dot{e}}}
\newcommand{\fdot}[0]{{\dot{f}}}
\newcommand{\gdot}[0]{{\dot{g}}}
\newcommand{\hdot}[0]{{\dot{h}}}
\newcommand{\idot}[0]{{\dot{i}}}
\newcommand{\jdot}[0]{{\dot{j}}}
\newcommand{\kdot}[0]{{\dot{k}}}
\newcommand{\ldot}[0]{{\dot{l}}}
\newcommand{\mdot}[0]{{\dot{m}}}
\newcommand{\ndot}[0]{{\dot{n}}}
%\newcommand{\odot}[0]{{\dot{o}}}
\newcommand{\pdot}[0]{{\dot{p}}}
\newcommand{\qdot}[0]{{\dot{q}}}
\newcommand{\rdot}[0]{{\dot{r}}}
\newcommand{\sdot}[0]{{\dot{s}}}
\newcommand{\tdot}[0]{{\dot{t}}}
\newcommand{\udot}[0]{{\dot{u}}}
\newcommand{\vdot}[0]{{\dot{v}}}
\newcommand{\wdot}[0]{{\dot{w}}}
\newcommand{\xdot}[0]{{\dot{x}}}
\newcommand{\ydot}[0]{{\dot{y}}}
\newcommand{\zdot}[0]{{\dot{z}}}
\newcommand{\addot}[0]{{\ddot{a}}}
\newcommand{\bddot}[0]{{\ddot{b}}}
\newcommand{\cddot}[0]{{\ddot{c}}}
%\newcommand{\dddot}[0]{{\ddot{d}}}
\newcommand{\eddot}[0]{{\ddot{e}}}
\newcommand{\fddot}[0]{{\ddot{f}}}
\newcommand{\gddot}[0]{{\ddot{g}}}
\newcommand{\hddot}[0]{{\ddot{h}}}
\newcommand{\iddot}[0]{{\ddot{i}}}
\newcommand{\jddot}[0]{{\ddot{j}}}
\newcommand{\kddot}[0]{{\ddot{k}}}
\newcommand{\lddot}[0]{{\ddot{l}}}
\newcommand{\mddot}[0]{{\ddot{m}}}
\newcommand{\nddot}[0]{{\ddot{n}}}
\newcommand{\oddot}[0]{{\ddot{o}}}
\newcommand{\pddot}[0]{{\ddot{p}}}
\newcommand{\qddot}[0]{{\ddot{q}}}
\newcommand{\rddot}[0]{{\ddot{r}}}
\newcommand{\sddot}[0]{{\ddot{s}}}
\newcommand{\tddot}[0]{{\ddot{t}}}
\newcommand{\uddot}[0]{{\ddot{u}}}
\newcommand{\vddot}[0]{{\ddot{v}}}
\newcommand{\wddot}[0]{{\ddot{w}}}
\newcommand{\xddot}[0]{{\ddot{x}}}
\newcommand{\yddot}[0]{{\ddot{y}}}
\newcommand{\zddot}[0]{{\ddot{z}}}

%<bold and dot greek symbols>
%

\newcommand{\Deltadot}[0]{{\dot{\Delta}}}
\newcommand{\Gammadot}[0]{{\dot{\Gamma}}}
\newcommand{\Lambdadot}[0]{{\dot{\Lambda}}}
\newcommand{\Omegadot}[0]{{\dot{\Omega}}}
\newcommand{\Phidot}[0]{{\dot{\Phi}}}
\newcommand{\Pidot}[0]{{\dot{\Pi}}}
\newcommand{\Psidot}[0]{{\dot{\Psi}}}
\newcommand{\Sigmadot}[0]{{\dot{\Sigma}}}
\newcommand{\Thetadot}[0]{{\dot{\Theta}}}
\newcommand{\Upsilondot}[0]{{\dot{\Upsilon}}}
\newcommand{\Xidot}[0]{{\dot{\Xi}}}
\newcommand{\alphadot}[0]{{\dot{\alpha}}}
\newcommand{\betadot}[0]{{\dot{\beta}}}
\newcommand{\chidot}[0]{{\dot{\chi}}}
\newcommand{\deltadot}[0]{{\dot{\delta}}}
\newcommand{\epsilondot}[0]{{\dot{\epsilon}}}
\newcommand{\etadot}[0]{{\dot{\eta}}}
\newcommand{\gammadot}[0]{{\dot{\gamma}}}
\newcommand{\kappadot}[0]{{\dot{\kappa}}}
\newcommand{\lambdadot}[0]{{\dot{\lambda}}}
\newcommand{\mudot}[0]{{\dot{\mu}}}
\newcommand{\nudot}[0]{{\dot{\nu}}}
\newcommand{\omegadot}[0]{{\dot{\omega}}}
\newcommand{\phidot}[0]{{\dot{\phi}}}
\newcommand{\pidot}[0]{{\dot{\pi}}}
\newcommand{\psidot}[0]{{\dot{\psi}}}
\newcommand{\rhodot}[0]{{\dot{\rho}}}
\newcommand{\sigmadot}[0]{{\dot{\sigma}}}
\newcommand{\taudot}[0]{{\dot{\tau}}}
\newcommand{\thetadot}[0]{{\dot{\theta}}}
\newcommand{\upsilondot}[0]{{\dot{\upsilon}}}
\newcommand{\varepsilondot}[0]{{\dot{\varepsilon}}}
\newcommand{\varphidot}[0]{{\dot{\varphi}}}
\newcommand{\varpidot}[0]{{\dot{\varpi}}}
\newcommand{\varrhodot}[0]{{\dot{\varrho}}}
\newcommand{\varsigmadot}[0]{{\dot{\varsigma}}}
\newcommand{\varthetadot}[0]{{\dot{\vartheta}}}
\newcommand{\xidot}[0]{{\dot{\xi}}}
\newcommand{\zetadot}[0]{{\dot{\zeta}}}

\newcommand{\Deltaddot}[0]{{\ddot{\Delta}}}
\newcommand{\Gammaddot}[0]{{\ddot{\Gamma}}}
\newcommand{\Lambdaddot}[0]{{\ddot{\Lambda}}}
\newcommand{\Omegaddot}[0]{{\ddot{\Omega}}}
\newcommand{\Phiddot}[0]{{\ddot{\Phi}}}
\newcommand{\Piddot}[0]{{\ddot{\Pi}}}
\newcommand{\Psiddot}[0]{{\ddot{\Psi}}}
\newcommand{\Sigmaddot}[0]{{\ddot{\Sigma}}}
\newcommand{\Thetaddot}[0]{{\ddot{\Theta}}}
\newcommand{\Upsilonddot}[0]{{\ddot{\Upsilon}}}
\newcommand{\Xiddot}[0]{{\ddot{\Xi}}}
\newcommand{\alphaddot}[0]{{\ddot{\alpha}}}
\newcommand{\betaddot}[0]{{\ddot{\beta}}}
\newcommand{\chiddot}[0]{{\ddot{\chi}}}
\newcommand{\deltaddot}[0]{{\ddot{\delta}}}
\newcommand{\epsilonddot}[0]{{\ddot{\epsilon}}}
\newcommand{\etaddot}[0]{{\ddot{\eta}}}
\newcommand{\gammaddot}[0]{{\ddot{\gamma}}}
\newcommand{\kappaddot}[0]{{\ddot{\kappa}}}
\newcommand{\lambdaddot}[0]{{\ddot{\lambda}}}
\newcommand{\muddot}[0]{{\ddot{\mu}}}
\newcommand{\nuddot}[0]{{\ddot{\nu}}}
\newcommand{\omegaddot}[0]{{\ddot{\omega}}}
\newcommand{\phiddot}[0]{{\ddot{\phi}}}
\newcommand{\piddot}[0]{{\ddot{\pi}}}
\newcommand{\psiddot}[0]{{\ddot{\psi}}}
\newcommand{\rhoddot}[0]{{\ddot{\rho}}}
\newcommand{\sigmaddot}[0]{{\ddot{\sigma}}}
\newcommand{\tauddot}[0]{{\ddot{\tau}}}
\newcommand{\thetaddot}[0]{{\ddot{\theta}}}
\newcommand{\upsilonddot}[0]{{\ddot{\upsilon}}}
\newcommand{\varepsilonddot}[0]{{\ddot{\varepsilon}}}
\newcommand{\varphiddot}[0]{{\ddot{\varphi}}}
\newcommand{\varpiddot}[0]{{\ddot{\varpi}}}
\newcommand{\varrhoddot}[0]{{\ddot{\varrho}}}
\newcommand{\varsigmaddot}[0]{{\ddot{\varsigma}}}
\newcommand{\varthetaddot}[0]{{\ddot{\vartheta}}}
\newcommand{\xiddot}[0]{{\ddot{\xi}}}
\newcommand{\zetaddot}[0]{{\ddot{\zeta}}}

\newcommand{\BDelta}[0]{\boldsymbol{\Delta}}
\newcommand{\BGamma}[0]{\boldsymbol{\Gamma}}
\newcommand{\BLambda}[0]{\boldsymbol{\Lambda}}
\newcommand{\BOmega}[0]{\boldsymbol{\Omega}}
\newcommand{\BPhi}[0]{\boldsymbol{\Phi}}
\newcommand{\BPi}[0]{\boldsymbol{\Pi}}
\newcommand{\BPsi}[0]{\boldsymbol{\Psi}}
\newcommand{\BSigma}[0]{\boldsymbol{\Sigma}}
\newcommand{\BTheta}[0]{\boldsymbol{\Theta}}
\newcommand{\BUpsilon}[0]{\boldsymbol{\Upsilon}}
\newcommand{\BXi}[0]{\boldsymbol{\Xi}}
\newcommand{\Balpha}[0]{\boldsymbol{\alpha}}
\newcommand{\Bbeta}[0]{\boldsymbol{\beta}}
\newcommand{\Bchi}[0]{\boldsymbol{\chi}}
\newcommand{\Bdelta}[0]{\boldsymbol{\delta}}
\newcommand{\Bepsilon}[0]{\boldsymbol{\epsilon}}
\newcommand{\Beta}[0]{\boldsymbol{\eta}}
\newcommand{\Bgamma}[0]{\boldsymbol{\gamma}}
\newcommand{\Bkappa}[0]{\boldsymbol{\kappa}}
\newcommand{\Blambda}[0]{\boldsymbol{\lambda}}
\newcommand{\Bmu}[0]{\boldsymbol{\mu}}
\newcommand{\Bnu}[0]{\boldsymbol{\nu}}
%\newcommand{\Bomega}[0]{\boldsymbol{\omega}}
\newcommand{\Bphi}[0]{\boldsymbol{\phi}}
\newcommand{\Bpi}[0]{\boldsymbol{\pi}}
\newcommand{\Bpsi}[0]{\boldsymbol{\psi}}
\newcommand{\Brho}[0]{\boldsymbol{\rho}}
\newcommand{\Bsigma}[0]{\boldsymbol{\sigma}}
%\newcommand{\Btau}[0]{\boldsymbol{\tau}}
%\newcommand{\Btheta}[0]{\boldsymbol{\theta}}
\newcommand{\Bupsilon}[0]{\boldsymbol{\upsilon}}
\newcommand{\Bvarepsilon}[0]{\boldsymbol{\varepsilon}}
\newcommand{\Bvarphi}[0]{\boldsymbol{\varphi}}
\newcommand{\Bvarpi}[0]{\boldsymbol{\varpi}}
\newcommand{\Bvarrho}[0]{\boldsymbol{\varrho}}
\newcommand{\Bvarsigma}[0]{\boldsymbol{\varsigma}}
\newcommand{\Bvartheta}[0]{\boldsymbol{\vartheta}}
\newcommand{\Bxi}[0]{\boldsymbol{\xi}}
\newcommand{\Bzeta}[0]{\boldsymbol{\zeta}}
%
%</bold and dot greek symbols>
%<infrequent>
%
%\newcommand{\AreaOp}[1]{\AName_{#1}}
%\newcommand{\Babs}[0]{\abs{\BB}}
%\newcommand{\Bcap}[0]{\hat{\BB}}
%\newcommand{\BrPrimeRej}[0]{\rcap(\rcap \wedge \Br')}
%\newcommand{\CA}[0]{\mathcal{A}}
%\newcommand{\Cos}[1]{\cos{\left({#1}\right)}}
%\newcommand{\Det}[1] {\abs{#1}}
%\newcommand{\Dsq}[2] {\frac {\partial^2 {#1}} {\partial {#2}^2}}
%\newcommand{\Exp}[1]{\exp{\left({#1}\right)}}
%\newcommand{\Norm}[1]{\left\lVert{#1}\right\rVert}
%\newcommand{\Sin}[1]{\sin{\left({#1}\right)}}
%\newcommand{\T}[0]{\text{T}}
%\newcommand{\VolumeOp}[1]{\VName_{#1}}
%\newcommand{\agrad}[0]{\Ba \cdot \nabla}
%\newcommand{\alphacap}[0]{\hat{\boldsymbol{\alpha}}}
%\newcommand{\Fcap}[0]{\hat{\BF}}
%\newcommand{\bithree}[0]{{\Bi}_3}
%\newcommand{\bxa}[0]{\Bx\Ba}
%\newcommand{\coordvec}[2]{
%\newcommand{\costheta}[0]{\acap \cdot \xcap}
%\newcommand{\ddt}[1]{\ddot{#1}}
%\newcommand{\ddu}[1] {\frac {d{#1}} {du}}
%\newcommand{\dsqxj}[2] {\frac {\partial^2 {#1}} {\partial {x_{#2}}^2}}
%\newcommand{\dtheta}[1]{\frac{d {#1}}{d \theta}}
%\newcommand{\dt}[1]{\dot{#1}}
%\newcommand{\dt}[1]{\frac{d {#1}}{dt}}
%\newcommand{\dxj}[2] {\frac {\partial {#1}} {\partial {x_{#2}}}}
%\newcommand{\halfPhi}[0]{\frac{\phi}{2}}
%\newcommand{\half}[0]{\inv{2}}
%\newcommand{\inv}[1]{\frac{1}{#1}}
%\newcommand{\laplacian}[0]{\nabla^2}
%\newcommand{\matrixoftx}[3]{
%\newcommand{\nrrp}[0]{\norm{\rcap \wedge \Br'}}
%\newcommand{\oiint}{\bigcirc \hspace{-1.4em} \int \hspace{-.8em} \int}
%\newcommand{\transpose}[1]{{#1}^{\text{T}}}
%\newcommand{\transpose}[1]{{{#1}^{\TextTranspose}}}
%\newcommand{\transpose}[1]{{{#1}^{\text{T}}}}
%\newcommand{\barA}[0]{\bar{A}}
%\newcommand{\qbar}[0]{\bar{q}}
%\newcommand{\qdotbar}[0]{\dot{\bar{q}}}
%
%</infrequent>





%\DeclareMathOperator{\Atan2}{atan2}
\DeclareMathOperator{\atan}{atan}

%\usepackage{listings}
%\usepackage{txfonts} % for ointctr... (also appears to make "prettier" \int and \sum's)
% makes \grad look funny though (almost like spacegrad, but narrower)
\usepackage[bookmarks=true]{hyperref}

\usepackage{color,cite,graphicx}
   % use colour in the document, put your citations as [1-4]
   % rather than [1,2,3,4] (it looks nicer, and the extended LaTeX2e
   % graphics package. 
\usepackage{latexsym,amssymb,epsf} % don't remember if these are
   % needed, but their inclusion can't do any damage


\title{ One dimensional rectanglular Quantum barrier problem. }
\author{Peeter Joot \quad peeter.joot@gmail.com }
\date{ May 11, 2009.  Last Revision: $Date: 2009/05/12 03:30:25 $ }

\begin{document}

\maketitle{}
\tableofcontents
\section{ Motivation. }

My first attempt at the probability current calculation for this
while doing the chapter 11 problems of \cite{bohm1989qt} led to trouble.
An attempt to look at this worked example in \cite{mcmahon2005qmd} didn't 
help much due to innumerable typos (at least a lot more than in my first attempt).
Here's a second bash at it, redoing the calculations from scratch.

\section{ Setup. }

The potential is taken to be zero everywhere except $x \in [0,a]$, where it is
$V$.  This divides the problem into three regions, I, II, and III, for 
before ($x<0$), in and after the barrier, and the wave functions for the 
$E >V$ case are respectively

\begin{equation}
\psi =
\left\{
\begin{array}{l l}
A e^{i k x} + B e^{-i k x} & \quad \mbox{if $x <0$} \\
C e^{ \beta(x-a)} + D e^{ -\beta(x-a)} & \quad \mbox{if $x \in [0,a]$} \\
E e^{i k(x-a)} & \quad \mbox{if $x >0$} \\
\end{array}
\right.
\end{equation}

%Where $p = \hbar \sqrt{2 m E} = \hbar k$
Where $k = \sqrt{2 m E}/\hbar$, and $\beta = \sqrt{2 m (E-V)}/\hbar$.
FIXME: check $\beta$ ... doesn't look right.

\subsection{ Equality at $x=a$ }

Equality of the wave functions and derivatives at $x=a$ gives

\begin{align*}
C + D &= E \\
C - D &= \frac{i k}{\beta}E
\end{align*}

which has solutions

\begin{align*}
C &= \frac{E}{2}( 1 + i k/\beta ) \\
D &= \frac{E}{2}( 1 - i k/\beta )
\end{align*}

Two of the free variables of the wave equation are now eliminated, and the wave function in the barrier region can now be written as
\begin{align}\label{eqn:psiInBarrier}
\psi =
\frac{E}{2}\left( \left( 1 + \frac{i k}{\beta} \right) e^{ \beta(x-a)} + \left( 1 - \frac{i k}{\beta} \right) e^{ -\beta(x-a)} \right)
\end{align}

\subsection{ Equality at $x=0$ }

Equating values and first derivatives at $x=0$ we have

\begin{align*}
A + B &=
\frac{E}{2}\left( \left( 1 + \frac{i k}{\beta} \right) e^{ -\beta a} + \left( 1 - \frac{i k}{\beta} \right) e^{ \beta a } \right) \\
A - B &=
\frac{\beta E}{2 i k}\left( \left( 1 + \frac{i k}{\beta} \right) e^{ -\beta a} - \left( 1 - \frac{i k}{\beta} \right) e^{ \beta a } \right)
\end{align*}

Taking sums and differences we have

\begin{align*}
A &= \frac{E}{4}\left( \left(1 + \frac{\beta}{ik}\right)\left( 1 + \frac{i k}{\beta} \right) e^{ -\beta a} + \left(1 - \frac{\beta}{ik}\right)\left( 1 - \frac{i k}{\beta} \right) e^{ \beta a } \right) \\
B &= \frac{E}{4}\left( \left(1 - \frac{\beta}{ik}\right)\left( 1 + \frac{i k}{\beta} \right) e^{ -\beta a} + \left(1 + \frac{\beta}{ik}\right)\left( 1 - \frac{i k}{\beta} \right) e^{ \beta a } \right) \\
\end{align*}

Expanding the products first for $B$

\begin{align*}
B 
&= \frac{E}{4}\left( \left(\frac{-\beta}{ik} + \frac{ik}{\beta} \right) e^{ -\beta a} + \left(\frac{\beta}{ik} - \frac{ik}{\beta} \right) e^{ \beta a } \right) \\
&= \frac{iE}{4}\left( \left(\frac{\beta}{k} + \frac{k}{\beta} \right) e^{ -\beta a} - \left(\frac{\beta}{k} + \frac{k}{\beta} \right) e^{ \beta a } \right) \\
&= \frac{iE}{4}\left(\frac{\beta}{k} + \frac{k}{\beta} \right) \left( e^{ -\beta a} - e^{ \beta a } \right) \\
&= \frac{-iE}{2}\left(\frac{\beta}{k} + \frac{k}{\beta} \right) \sinh\left( \beta a \right) \\
\end{align*}

Now for $A$ we have
\begin{align*}
A &= \frac{E}{4}\left( \left(2 + \frac{\beta}{ik} + \frac{ik}{\beta} \right) e^{ -\beta a} + \left(2 - \frac{\beta}{ik} - \frac{ik}{\beta} \right) e^{ \beta a } \right) \\
\end{align*}

The factor of exponentials are complex conjugates and can be put into polar form to simplify.  Writing

\begin{align*}
\gamma 
&= 2 + \frac{\beta}{ik} + \frac{ik}{\beta} \\
&= 2 + i \left( \frac{k}{\beta} -\frac{\beta}{k} \right) \\
&= \mu e^{i \theta} 
\end{align*}

Where 

\begin{align*}
\mu^2 &= 4 + \left( \frac{k}{\beta} -\frac{\beta}{k} \right)^2 \\
\theta &= \atan\left( \inv{2} \left(\frac{k}{\beta} - \frac{\beta}{k} \right) \right)
\end{align*}

We have

\begin{align*}
A 
&= \frac{\mu E}{4}\left( e^{ i\theta -\beta a} + e^{ -(i \theta - \beta a) } \right) \\
&= \frac{\mu E}{2}\cosh\left( i\theta -\beta a \right) \\
\end{align*}

For notational consistency it looks desirable to write something like

\begin{align*}
\nu &= -i \left(\frac{\beta}{k} + \frac{k}{\beta} \right) 
\end{align*}

Which leaves us with the wave function in the $x<0$ region as
\begin{align}
\psi &=
\frac{\mu E}{2}\cosh\left( i\theta -\beta a \right) e^{ i k x }
+\frac{\nu E}{2}\sinh\left( \beta a \right) e^{ -i k x }
\end{align}

\subsection{ Barrier wave function in polar form. }

Having seen how the polar form simplifies the final expression of the first region wave function, it can be seen that 
something similar can be done in the barrier region.  

In equation \ref{eqn:psiInBarrier}, we can utilize polar form for the $1+ik/\beta$ constant and its conjugate.  Writing

\begin{align*}
\tan\phi &= \frac{k}{\beta}
\end{align*}

we then have
\begin{align*}
\psi 
&=
\frac{E}{2}\sqrt{1 + \frac{k^2}{\beta^2}} \left( e^{i\phi} e^{ \beta(x-a)} + e^{-i\phi} e^{ -\beta(x-a)} \right) \\
\end{align*}

So we have in the $x \in [0,a]$ region
\begin{align}
\psi &= E \sqrt{1 + \frac{k^2}{\beta^2}} \cosh\left( i\phi + \beta(x-a) \right) 
\end{align}

Utilizing this should produce the same result.  Equality at $x=0$ then gives

\begin{align*}
A + B &= E \sqrt{1 + \frac{k^2}{\beta^2}} \cosh\left( i\phi -\beta a \right) \\
A - B &= E \frac{i\beta}{k} \sqrt{1 + \frac{k^2}{\beta^2}} \sinh\left( i\phi -\beta a \right) \\
\end{align*}

Or

\begin{align*}
A &= \frac{E}{2} \sqrt{1 + \frac{k^2}{\beta^2}} \left(
\cosh\left( i\phi -\beta a \right) + \frac{i\beta}{k} \sinh\left( i\phi -\beta a \right) 
\right) \\
B &= \frac{E}{2} \sqrt{1 + \frac{k^2}{\beta^2}} \left(
\cosh\left( i\phi -\beta a \right) - \frac{i\beta}{k} \sinh\left( i\phi -\beta a \right) 
\right) \\
\end{align*}

\bibliographystyle{plainnat}
\bibliography{myrefs}

\end{document}

\documentclass{article}

\usepackage{amsmath}
\usepackage{mathpazo}

%
% shorthand for bold symbols, convenient for vectors and matrices
%
\newcommand{\Ba}[0]{\mathbf{a}}
\newcommand{\Bb}[0]{\mathbf{b}}
\newcommand{\Bc}[0]{\mathbf{c}}
\newcommand{\Bd}[0]{\mathbf{d}}
\newcommand{\Be}[0]{\mathbf{e}}
\newcommand{\Bf}[0]{\mathbf{f}}
\newcommand{\Bg}[0]{\mathbf{g}}
\newcommand{\Bh}[0]{\mathbf{h}}
\newcommand{\Bi}[0]{\mathbf{i}}
\newcommand{\Bj}[0]{\mathbf{j}}
\newcommand{\Bk}[0]{\mathbf{k}}
\newcommand{\Bl}[0]{\mathbf{l}}
\newcommand{\Bm}[0]{\mathbf{m}}
\newcommand{\Bn}[0]{\mathbf{n}}
\newcommand{\Bo}[0]{\mathbf{o}}
\newcommand{\Bp}[0]{\mathbf{p}}
\newcommand{\Bq}[0]{\mathbf{q}}
\newcommand{\Br}[0]{\mathbf{r}}
\newcommand{\Bs}[0]{\mathbf{s}}
\newcommand{\Bt}[0]{\mathbf{t}}
\newcommand{\Bu}[0]{\mathbf{u}}
\newcommand{\Bv}[0]{\mathbf{v}}
\newcommand{\Bw}[0]{\mathbf{w}}
\newcommand{\Bx}[0]{\mathbf{x}}
\newcommand{\By}[0]{\mathbf{y}}
\newcommand{\Bz}[0]{\mathbf{z}}
\newcommand{\BA}[0]{\mathbf{A}}
\newcommand{\BB}[0]{\mathbf{B}}
\newcommand{\BC}[0]{\mathbf{C}}
\newcommand{\BD}[0]{\mathbf{D}}
\newcommand{\BE}[0]{\mathbf{E}}
\newcommand{\BF}[0]{\mathbf{F}}
\newcommand{\BG}[0]{\mathbf{G}}
\newcommand{\BH}[0]{\mathbf{H}}
\newcommand{\BI}[0]{\mathbf{I}}
\newcommand{\BJ}[0]{\mathbf{J}}
\newcommand{\BK}[0]{\mathbf{K}}
\newcommand{\BL}[0]{\mathbf{L}}
\newcommand{\BM}[0]{\mathbf{M}}
\newcommand{\BN}[0]{\mathbf{N}}
\newcommand{\BO}[0]{\mathbf{O}}
\newcommand{\BP}[0]{\mathbf{P}}
\newcommand{\BQ}[0]{\mathbf{Q}}
\newcommand{\BR}[0]{\mathbf{R}}
\newcommand{\BS}[0]{\mathbf{S}}
\newcommand{\BT}[0]{\mathbf{T}}
\newcommand{\BU}[0]{\mathbf{U}}
\newcommand{\BV}[0]{\mathbf{V}}
\newcommand{\BW}[0]{\mathbf{W}}
\newcommand{\BX}[0]{\mathbf{X}}
\newcommand{\BY}[0]{\mathbf{Y}}
\newcommand{\BZ}[0]{\mathbf{Z}}

\newcommand{\Bzero}[0]{\mathbf{0}}
\newcommand{\Btheta}[0]{\boldsymbol{\theta}}
\newcommand{\Btau}[0]{\boldsymbol{\tau}}
\newcommand{\Bomega}[0]{\boldsymbol{\omega}}

%
% shorthand for unit vectors
%
\newcommand{\acap}[0]{\hat{\Ba}}
\newcommand{\bcap}[0]{\hat{\Bb}}
\newcommand{\ccap}[0]{\hat{\Bc}}
\newcommand{\dcap}[0]{\hat{\Bd}}
\newcommand{\ecap}[0]{\hat{\Be}}
\newcommand{\fcap}[0]{\hat{\Bf}}
\newcommand{\gcap}[0]{\hat{\Bg}}
\newcommand{\hcap}[0]{\hat{\Bh}}
\newcommand{\icap}[0]{\hat{\Bi}}
\newcommand{\jcap}[0]{\hat{\Bj}}
\newcommand{\kcap}[0]{\hat{\Bk}}
\newcommand{\lcap}[0]{\hat{\Bl}}
\newcommand{\mcap}[0]{\hat{\Bm}}
\newcommand{\ncap}[0]{\hat{\Bn}}
\newcommand{\ocap}[0]{\hat{\Bo}}
\newcommand{\pcap}[0]{\hat{\Bp}}
\newcommand{\qcap}[0]{\hat{\Bq}}
\newcommand{\rcap}[0]{\hat{\Br}}
\newcommand{\scap}[0]{\hat{\Bs}}
\newcommand{\tcap}[0]{\hat{\Bt}}
\newcommand{\ucap}[0]{\hat{\Bu}}
\newcommand{\vcap}[0]{\hat{\Bv}}
\newcommand{\wcap}[0]{\hat{\Bw}}
\newcommand{\xcap}[0]{\hat{\Bx}}
\newcommand{\ycap}[0]{\hat{\By}}
\newcommand{\zcap}[0]{\hat{\Bz}}
\newcommand{\thetacap}[0]{\hat{\Btheta}}

%
% to write R^n and C^n in a distinguishable fashion.  Perhaps change this
% to the double lined characters upon figuring out how to do so.
%
\newcommand{\C}[1]{$\mathbb{C}^{#1}$}
\newcommand{\R}[1]{$\mathbb{R}^{#1}$}

%
% various generally useful helpers
%

% derivative of #1 wrt. #2:
\newcommand{\D}[2] {\frac {d#2} {d#1}}

\newcommand{\inv}[1]{\frac{1}{#1}}
\newcommand{\cross}[0]{\times}

\newcommand{\abs}[1]{\lvert{#1}\rvert}
\newcommand{\norm}[1]{\lVert{#1}\rVert}
\newcommand{\innerprod}[2]{\langle{#1}, {#2}\rangle}
\newcommand{\dotprod}[2]{{#1} \cdot {#2}}
\newcommand{\bdotprod}[2]{\left({#1} \cdot {#2}\right)}
\newcommand{\crossprod}[2]{{#1} \cross {#2}}
\newcommand{\tripleprod}[3]{\dotprod{\left(\crossprod{#1}{#2}\right)}{#3}}

\DeclareMathOperator{\Proj}{Proj}
\DeclareMathOperator{\Span}{span}
\DeclareMathOperator{\Sgn}{sgn}
\DeclareMathOperator{\Area}{Area}
\DeclareMathOperator{\Volume}{Volume}

%
% A few miscellaneous things specific to this document
%
\newcommand{\crossop}[1]{\crossprod{#1}{}}

% R2 vector.
\newcommand{\VectorTwo}[2]{
\begin{bmatrix}
 {#1} \\
 {#2}
\end{bmatrix}
}

\newcommand{\VectorN}[1]{
\begin{bmatrix}
{#1}_1 \\
{#1}_2 \\
\vdots \\
{#1}_N \\
\end{bmatrix}
}

\newcommand{\DETuvij}[4]{
\begin{vmatrix}
 {#1}_{#3} & {#1}_{#4} \\
 {#2}_{#3} & {#2}_{#4}
\end{vmatrix}
}

\newcommand{\DETuvwijk}[6]{
\begin{vmatrix}
 {#1}_{#4} & {#1}_{#5} & {#1}_{#6} \\
 {#2}_{#4} & {#2}_{#5} & {#2}_{#6} \\
 {#3}_{#4} & {#3}_{#5} & {#3}_{#6}
\end{vmatrix}
}

\newcommand{\DETuvwxijkl}[8]{
\begin{vmatrix}
 {#1}_{#5} & {#1}_{#6} & {#1}_{#7} & {#1}_{#8} \\
 {#2}_{#5} & {#2}_{#6} & {#2}_{#7} & {#2}_{#8} \\
 {#3}_{#5} & {#3}_{#6} & {#3}_{#7} & {#3}_{#8} \\
 {#4}_{#5} & {#4}_{#6} & {#4}_{#7} & {#4}_{#8} \\
\end{vmatrix}
}

%\newcommand{\DETuvwxyijklm}[10]{
%\begin{vmatrix}
% {#1}_{#6} & {#1}_{#7} & {#1}_{#8} & {#1}_{#9} & {#1}_{#10} \\
% {#2}_{#6} & {#2}_{#7} & {#2}_{#8} & {#2}_{#9} & {#2}_{#10} \\
% {#3}_{#6} & {#3}_{#7} & {#3}_{#8} & {#3}_{#9} & {#3}_{#10} \\
% {#4}_{#6} & {#4}_{#7} & {#4}_{#8} & {#4}_{#9} & {#4}_{#10} \\
% {#5}_{#6} & {#5}_{#7} & {#5}_{#8} & {#5}_{#9} & {#5}_{#10}
%\end{vmatrix}
%}

% R3 vector.
\newcommand{\VectorThree}[3]{
\begin{bmatrix}
 {#1} \\
 {#2} \\
 {#3}
\end{bmatrix}
}


%<misc>
%
\newcommand{\Abs}[1]{{\left\lvert{#1}\right\rvert}}
\newcommand{\spacegrad}[0]{\boldsymbol{\nabla}}
\newcommand{\grad}[0]{\nabla}
\newcommand{\LL}[0]{\mathcal{L}}

% == \partial_{#1} {#2}
\newcommand{\PD}[2]{\frac{\partial {#2}}{\partial {#1}}}
% inline variant
\newcommand{\PDi}[2]{{\partial {#2}}/{\partial {#1}}}

\newcommand{\PDD}[3]{\frac{\partial^2 {#3}}{\partial {#1}\partial {#2}}}
%\newcommand{\PDd}[2]{\frac{\partial^2 {#2}}{{\partial{#1}}^2}}
\newcommand{\PDsq}[2]{\frac{\partial^2 {#2}}{(\partial {#1})^2}}

\newcommand{\Partial}[2]{\frac{\partial {#1}}{\partial {#2}}}
\DeclareMathOperator{\RejName}{Rej}
\newcommand{\Rej}[2]{\RejName_{#1}\left( {#2} \right)}
\newcommand{\Rm}[1]{\mathbb{R}^{#1}}
\newcommand{\Cm}[1]{\mathbb{C}^{#1}}
\newcommand{\conj}[0]{{*}}

%</misc>

% <grade selection>
%
\newcommand{\gpgrade}[2] {{\left\langle{{#1}}\right\rangle}_{#2}}

\newcommand{\gpgradezero}[1] {\gpgrade{#1}{}}
%\newcommand{\gpscalargrade}[1] {{\left\langle{{#1}}\right\rangle}}
%\newcommand{\gpgradezero}[1] {\gpgrade{#1}{0}}

%\newcommand{\gpgradeone}[1] {{\left\langle{{#1}}\right\rangle}_{1}}
\newcommand{\gpgradeone}[1] {\gpgrade{#1}{1}}

\newcommand{\gpgradetwo}[1] {\gpgrade{#1}{2}}
\newcommand{\gpgradethree}[1] {\gpgrade{#1}{3}}
\newcommand{\gpgradefour}[1] {\gpgrade{#1}{4}}
%
% </grade selection>



\newcommand{\adot}[0]{{\dot{a}}}
\newcommand{\bdot}[0]{{\dot{b}}}
% taken for centered dot:
%\newcommand{\cdot}[0]{{\dot{c}}}
%\newcommand{\ddot}[0]{{\dot{d}}}
\newcommand{\edot}[0]{{\dot{e}}}
\newcommand{\fdot}[0]{{\dot{f}}}
\newcommand{\gdot}[0]{{\dot{g}}}
\newcommand{\hdot}[0]{{\dot{h}}}
\newcommand{\idot}[0]{{\dot{i}}}
\newcommand{\jdot}[0]{{\dot{j}}}
\newcommand{\kdot}[0]{{\dot{k}}}
\newcommand{\ldot}[0]{{\dot{l}}}
\newcommand{\mdot}[0]{{\dot{m}}}
\newcommand{\ndot}[0]{{\dot{n}}}
%\newcommand{\odot}[0]{{\dot{o}}}
\newcommand{\pdot}[0]{{\dot{p}}}
\newcommand{\qdot}[0]{{\dot{q}}}
\newcommand{\rdot}[0]{{\dot{r}}}
\newcommand{\sdot}[0]{{\dot{s}}}
\newcommand{\tdot}[0]{{\dot{t}}}
\newcommand{\udot}[0]{{\dot{u}}}
\newcommand{\vdot}[0]{{\dot{v}}}
\newcommand{\wdot}[0]{{\dot{w}}}
\newcommand{\xdot}[0]{{\dot{x}}}
\newcommand{\ydot}[0]{{\dot{y}}}
\newcommand{\zdot}[0]{{\dot{z}}}
\newcommand{\addot}[0]{{\ddot{a}}}
\newcommand{\bddot}[0]{{\ddot{b}}}
\newcommand{\cddot}[0]{{\ddot{c}}}
%\newcommand{\dddot}[0]{{\ddot{d}}}
\newcommand{\eddot}[0]{{\ddot{e}}}
\newcommand{\fddot}[0]{{\ddot{f}}}
\newcommand{\gddot}[0]{{\ddot{g}}}
\newcommand{\hddot}[0]{{\ddot{h}}}
\newcommand{\iddot}[0]{{\ddot{i}}}
\newcommand{\jddot}[0]{{\ddot{j}}}
\newcommand{\kddot}[0]{{\ddot{k}}}
\newcommand{\lddot}[0]{{\ddot{l}}}
\newcommand{\mddot}[0]{{\ddot{m}}}
\newcommand{\nddot}[0]{{\ddot{n}}}
\newcommand{\oddot}[0]{{\ddot{o}}}
\newcommand{\pddot}[0]{{\ddot{p}}}
\newcommand{\qddot}[0]{{\ddot{q}}}
\newcommand{\rddot}[0]{{\ddot{r}}}
\newcommand{\sddot}[0]{{\ddot{s}}}
\newcommand{\tddot}[0]{{\ddot{t}}}
\newcommand{\uddot}[0]{{\ddot{u}}}
\newcommand{\vddot}[0]{{\ddot{v}}}
\newcommand{\wddot}[0]{{\ddot{w}}}
\newcommand{\xddot}[0]{{\ddot{x}}}
\newcommand{\yddot}[0]{{\ddot{y}}}
\newcommand{\zddot}[0]{{\ddot{z}}}

%<bold and dot greek symbols>
%

\newcommand{\Deltadot}[0]{{\dot{\Delta}}}
\newcommand{\Gammadot}[0]{{\dot{\Gamma}}}
\newcommand{\Lambdadot}[0]{{\dot{\Lambda}}}
\newcommand{\Omegadot}[0]{{\dot{\Omega}}}
\newcommand{\Phidot}[0]{{\dot{\Phi}}}
\newcommand{\Pidot}[0]{{\dot{\Pi}}}
\newcommand{\Psidot}[0]{{\dot{\Psi}}}
\newcommand{\Sigmadot}[0]{{\dot{\Sigma}}}
\newcommand{\Thetadot}[0]{{\dot{\Theta}}}
\newcommand{\Upsilondot}[0]{{\dot{\Upsilon}}}
\newcommand{\Xidot}[0]{{\dot{\Xi}}}
\newcommand{\alphadot}[0]{{\dot{\alpha}}}
\newcommand{\betadot}[0]{{\dot{\beta}}}
\newcommand{\chidot}[0]{{\dot{\chi}}}
\newcommand{\deltadot}[0]{{\dot{\delta}}}
\newcommand{\epsilondot}[0]{{\dot{\epsilon}}}
\newcommand{\etadot}[0]{{\dot{\eta}}}
\newcommand{\gammadot}[0]{{\dot{\gamma}}}
\newcommand{\kappadot}[0]{{\dot{\kappa}}}
\newcommand{\lambdadot}[0]{{\dot{\lambda}}}
\newcommand{\mudot}[0]{{\dot{\mu}}}
\newcommand{\nudot}[0]{{\dot{\nu}}}
\newcommand{\omegadot}[0]{{\dot{\omega}}}
\newcommand{\phidot}[0]{{\dot{\phi}}}
\newcommand{\pidot}[0]{{\dot{\pi}}}
\newcommand{\psidot}[0]{{\dot{\psi}}}
\newcommand{\rhodot}[0]{{\dot{\rho}}}
\newcommand{\sigmadot}[0]{{\dot{\sigma}}}
\newcommand{\taudot}[0]{{\dot{\tau}}}
\newcommand{\thetadot}[0]{{\dot{\theta}}}
\newcommand{\upsilondot}[0]{{\dot{\upsilon}}}
\newcommand{\varepsilondot}[0]{{\dot{\varepsilon}}}
\newcommand{\varphidot}[0]{{\dot{\varphi}}}
\newcommand{\varpidot}[0]{{\dot{\varpi}}}
\newcommand{\varrhodot}[0]{{\dot{\varrho}}}
\newcommand{\varsigmadot}[0]{{\dot{\varsigma}}}
\newcommand{\varthetadot}[0]{{\dot{\vartheta}}}
\newcommand{\xidot}[0]{{\dot{\xi}}}
\newcommand{\zetadot}[0]{{\dot{\zeta}}}

\newcommand{\Deltaddot}[0]{{\ddot{\Delta}}}
\newcommand{\Gammaddot}[0]{{\ddot{\Gamma}}}
\newcommand{\Lambdaddot}[0]{{\ddot{\Lambda}}}
\newcommand{\Omegaddot}[0]{{\ddot{\Omega}}}
\newcommand{\Phiddot}[0]{{\ddot{\Phi}}}
\newcommand{\Piddot}[0]{{\ddot{\Pi}}}
\newcommand{\Psiddot}[0]{{\ddot{\Psi}}}
\newcommand{\Sigmaddot}[0]{{\ddot{\Sigma}}}
\newcommand{\Thetaddot}[0]{{\ddot{\Theta}}}
\newcommand{\Upsilonddot}[0]{{\ddot{\Upsilon}}}
\newcommand{\Xiddot}[0]{{\ddot{\Xi}}}
\newcommand{\alphaddot}[0]{{\ddot{\alpha}}}
\newcommand{\betaddot}[0]{{\ddot{\beta}}}
\newcommand{\chiddot}[0]{{\ddot{\chi}}}
\newcommand{\deltaddot}[0]{{\ddot{\delta}}}
\newcommand{\epsilonddot}[0]{{\ddot{\epsilon}}}
\newcommand{\etaddot}[0]{{\ddot{\eta}}}
\newcommand{\gammaddot}[0]{{\ddot{\gamma}}}
\newcommand{\kappaddot}[0]{{\ddot{\kappa}}}
\newcommand{\lambdaddot}[0]{{\ddot{\lambda}}}
\newcommand{\muddot}[0]{{\ddot{\mu}}}
\newcommand{\nuddot}[0]{{\ddot{\nu}}}
\newcommand{\omegaddot}[0]{{\ddot{\omega}}}
\newcommand{\phiddot}[0]{{\ddot{\phi}}}
\newcommand{\piddot}[0]{{\ddot{\pi}}}
\newcommand{\psiddot}[0]{{\ddot{\psi}}}
\newcommand{\rhoddot}[0]{{\ddot{\rho}}}
\newcommand{\sigmaddot}[0]{{\ddot{\sigma}}}
\newcommand{\tauddot}[0]{{\ddot{\tau}}}
\newcommand{\thetaddot}[0]{{\ddot{\theta}}}
\newcommand{\upsilonddot}[0]{{\ddot{\upsilon}}}
\newcommand{\varepsilonddot}[0]{{\ddot{\varepsilon}}}
\newcommand{\varphiddot}[0]{{\ddot{\varphi}}}
\newcommand{\varpiddot}[0]{{\ddot{\varpi}}}
\newcommand{\varrhoddot}[0]{{\ddot{\varrho}}}
\newcommand{\varsigmaddot}[0]{{\ddot{\varsigma}}}
\newcommand{\varthetaddot}[0]{{\ddot{\vartheta}}}
\newcommand{\xiddot}[0]{{\ddot{\xi}}}
\newcommand{\zetaddot}[0]{{\ddot{\zeta}}}

\newcommand{\BDelta}[0]{\boldsymbol{\Delta}}
\newcommand{\BGamma}[0]{\boldsymbol{\Gamma}}
\newcommand{\BLambda}[0]{\boldsymbol{\Lambda}}
\newcommand{\BOmega}[0]{\boldsymbol{\Omega}}
\newcommand{\BPhi}[0]{\boldsymbol{\Phi}}
\newcommand{\BPi}[0]{\boldsymbol{\Pi}}
\newcommand{\BPsi}[0]{\boldsymbol{\Psi}}
\newcommand{\BSigma}[0]{\boldsymbol{\Sigma}}
\newcommand{\BTheta}[0]{\boldsymbol{\Theta}}
\newcommand{\BUpsilon}[0]{\boldsymbol{\Upsilon}}
\newcommand{\BXi}[0]{\boldsymbol{\Xi}}
\newcommand{\Balpha}[0]{\boldsymbol{\alpha}}
\newcommand{\Bbeta}[0]{\boldsymbol{\beta}}
\newcommand{\Bchi}[0]{\boldsymbol{\chi}}
\newcommand{\Bdelta}[0]{\boldsymbol{\delta}}
\newcommand{\Bepsilon}[0]{\boldsymbol{\epsilon}}
\newcommand{\Beta}[0]{\boldsymbol{\eta}}
\newcommand{\Bgamma}[0]{\boldsymbol{\gamma}}
\newcommand{\Bkappa}[0]{\boldsymbol{\kappa}}
\newcommand{\Blambda}[0]{\boldsymbol{\lambda}}
\newcommand{\Bmu}[0]{\boldsymbol{\mu}}
\newcommand{\Bnu}[0]{\boldsymbol{\nu}}
%\newcommand{\Bomega}[0]{\boldsymbol{\omega}}
\newcommand{\Bphi}[0]{\boldsymbol{\phi}}
\newcommand{\Bpi}[0]{\boldsymbol{\pi}}
\newcommand{\Bpsi}[0]{\boldsymbol{\psi}}
\newcommand{\Brho}[0]{\boldsymbol{\rho}}
\newcommand{\Bsigma}[0]{\boldsymbol{\sigma}}
%\newcommand{\Btau}[0]{\boldsymbol{\tau}}
%\newcommand{\Btheta}[0]{\boldsymbol{\theta}}
\newcommand{\Bupsilon}[0]{\boldsymbol{\upsilon}}
\newcommand{\Bvarepsilon}[0]{\boldsymbol{\varepsilon}}
\newcommand{\Bvarphi}[0]{\boldsymbol{\varphi}}
\newcommand{\Bvarpi}[0]{\boldsymbol{\varpi}}
\newcommand{\Bvarrho}[0]{\boldsymbol{\varrho}}
\newcommand{\Bvarsigma}[0]{\boldsymbol{\varsigma}}
\newcommand{\Bvartheta}[0]{\boldsymbol{\vartheta}}
\newcommand{\Bxi}[0]{\boldsymbol{\xi}}
\newcommand{\Bzeta}[0]{\boldsymbol{\zeta}}
%
%</bold and dot greek symbols>
%<infrequent>
%
%\newcommand{\AreaOp}[1]{\AName_{#1}}
%\newcommand{\Babs}[0]{\abs{\BB}}
%\newcommand{\Bcap}[0]{\hat{\BB}}
%\newcommand{\BrPrimeRej}[0]{\rcap(\rcap \wedge \Br')}
%\newcommand{\CA}[0]{\mathcal{A}}
%\newcommand{\Cos}[1]{\cos{\left({#1}\right)}}
%\newcommand{\Det}[1] {\abs{#1}}
%\newcommand{\Dsq}[2] {\frac {\partial^2 {#1}} {\partial {#2}^2}}
%\newcommand{\Exp}[1]{\exp{\left({#1}\right)}}
%\newcommand{\Norm}[1]{\left\lVert{#1}\right\rVert}
%\newcommand{\Sin}[1]{\sin{\left({#1}\right)}}
%\newcommand{\T}[0]{\text{T}}
%\newcommand{\VolumeOp}[1]{\VName_{#1}}
%\newcommand{\agrad}[0]{\Ba \cdot \nabla}
%\newcommand{\alphacap}[0]{\hat{\boldsymbol{\alpha}}}
%\newcommand{\Fcap}[0]{\hat{\BF}}
%\newcommand{\bithree}[0]{{\Bi}_3}
%\newcommand{\bxa}[0]{\Bx\Ba}
%\newcommand{\coordvec}[2]{
%\newcommand{\costheta}[0]{\acap \cdot \xcap}
%\newcommand{\ddt}[1]{\ddot{#1}}
%\newcommand{\ddu}[1] {\frac {d{#1}} {du}}
%\newcommand{\dsqxj}[2] {\frac {\partial^2 {#1}} {\partial {x_{#2}}^2}}
%\newcommand{\dtheta}[1]{\frac{d {#1}}{d \theta}}
%\newcommand{\dt}[1]{\dot{#1}}
%\newcommand{\dt}[1]{\frac{d {#1}}{dt}}
%\newcommand{\dxj}[2] {\frac {\partial {#1}} {\partial {x_{#2}}}}
%\newcommand{\halfPhi}[0]{\frac{\phi}{2}}
%\newcommand{\half}[0]{\inv{2}}
%\newcommand{\inv}[1]{\frac{1}{#1}}
%\newcommand{\laplacian}[0]{\nabla^2}
%\newcommand{\matrixoftx}[3]{
%\newcommand{\nrrp}[0]{\norm{\rcap \wedge \Br'}}
%\newcommand{\oiint}{\bigcirc \hspace{-1.4em} \int \hspace{-.8em} \int}
%\newcommand{\transpose}[1]{{#1}^{\text{T}}}
%\newcommand{\transpose}[1]{{{#1}^{\TextTranspose}}}
%\newcommand{\transpose}[1]{{{#1}^{\text{T}}}}
%\newcommand{\barA}[0]{\bar{A}}
%\newcommand{\qbar}[0]{\bar{q}}
%\newcommand{\qdotbar}[0]{\dot{\bar{q}}}
%
%</infrequent>




\newcommand{\ket}[1]{\lvert {#1} \rangle}
\newcommand{\bra}[1]{\langle {#1} \rvert}
\newcommand{\braket}[2]{\langle{#1} \vert {#2}\rangle}
\newcommand{\ketbra}[2]{\ket{#1}\bra{#2}}
\newcommand{\BraOpKet}[3]{\bra{#1} \hat{#2} \ket{#3} }

\usepackage[bookmarks=true]{hyperref}

\usepackage{color,cite,graphicx}
   % use colour in the document, put your citations as [1-4]
   % rather than [1,2,3,4] (it looks nicer, and the extended LaTeX2e
   % graphics package. 
\usepackage{latexsym,amssymb,epsf} % don't remember if these are
   % needed, but their inclusion can't do any damage


\title{ Notes on Susskind's QM Lecture 3. }
\author{Peeter Joot}
\date{ Dec 23, 2008.  Last Revision: $Date: 2008/12/24 05:52:30 $ }

\begin{document}

\maketitle{}
%\tableofcontents
\section{ Bra and Ket vectors. }

An odd looking vector notation is introduced.  Instead of just using a letter, say $A$, for a vector in \C{N}, such a
vector is instead written

\begin{align*}
\ket{A}
\end{align*}

This is called a ``ket'' or ket-vector, but really just means complex vector.   The complex conjugate of this
vector is then written as a ``bra'' like so

\begin{align*}
\bra{A}
\end{align*}

The inner product of two vectors can then be written by combining this bra and ket by butting them up together, as
in

\begin{align*}
\braket{A}{B}
\end{align*}

Contrast this to the explicit complex column vector representation

\begin{align*}
{A} = 
\begin{bmatrix}
a_1 \\
a_2 \\
\vdots \\
a_n
\end{bmatrix}
\quad
{B} = 
\begin{bmatrix}
b_1 \\
b_2 \\
\vdots \\
b_n
\end{bmatrix}
\end{align*}

in a finite dimensional space.  The usual convention is to employ an inner product notation like

\begin{align*}
\innerprod{A}{B} = {A}^\text{T} \bar{B} = \sum_i {a_i} \bar{b_i}
\end{align*}

or,
\begin{align*}
\innerprod{A}{B} = {A}^\conj {B} = \sum_i \bar{a_i} {b_i}
\end{align*}

Observe that the braket notation is closer to this last form with the conjugation on the first term.  However, note that the metric associated with the braket notation has not been specified yet.  This is in fact an integral over space, where the ket vectors are complex valued functions.

\subsection{ Coordinates and basis notation. }

Ket vectors represent states, and the lables that are used for these are pretty loose.  For example, instead
of writing the n'th basis vector as 

\begin{align*}
\ket{a_n}
\end{align*}

just n was used like so

\begin{align*}
\ket{n}
\end{align*}

so if $\{\ket{n}\}$ is a basis, the coordinates of a vector $\ket{A}$ can be written

\begin{align*}
\ket{A} = \sum_n \alpha_n \ket{n}
\end{align*}

Note that here in the summation sign, and the subscript $n$ is an index, and also implicitly indexes the basis vectors, but in that context is not a number but a label for the basis vector itself.

Assuming that the $\ket{n}$ vectors are orthonormal, we can take inner products (brackets) to compute the $\alpha_n$ coordinates.

Writing $\ket{k}$ as an alternate labeling for the same basis, the congugate bra vectors when sandwiched against this bra representation denotes the inner product

\begin{align*}
\braket{n}{A} 
&= \sum_k \alpha_k \braket{n}{k} \\
&= \sum_k \alpha_k \delta_{nk} \\
&= \alpha_n \\
\end{align*}

So we have

\begin{align*}
\ket{A} = \sum_n \braket{n}{A} \ket{n}
\end{align*}

Since $\ket{n}$ is a vector and $\braket{n}{A}$ is just a complex number, this can be rearranged to butt the points
together as a mnemonic reminder that this is a projective operation.

\begin{align*}
\ket{A} = \sum_n \ket{n} \braket{n}{A} 
\end{align*}

As is the case with an othornormal split by projection matrixes, this can be observed to be more than a memory
device since the object

\begin{align*}
\ketbra{n}{n}
\end{align*}

is in fact the orthonormal projection operator onto the $\ket{n}$ direction.  ie:

\begin{align*}
\Proj_{\ket{n}}(\ket{A}) = \left( \ketbra{n}{n} \right) \ket{A}
\end{align*}

Note this is not summed over indexes $n$.

In the matrix representation, this is really nothing more than writing

\begin{align*}
\Proj_{e} (A) = e \innerprod{e}{A} = e (e^\conj A) = (e e^\conj) A
\end{align*}

So one can think of this funny looking $\ketbra{n}{n}$ as nothing more than the orthonormal projector matrix of the form $e e^\conj$.

\subsection{ Dual space. }

Susskind called the set of the conjugate vectors (the bras), the dual space.  If the basis is not orthonormal
are the conjugates really the duals (reciprocals) of the basis vectors?  Let's see with an example:

\begin{align*}
\ket{1} = 
\inv{\sqrt{2}}
\begin{bmatrix}
1 \\
i
\end{bmatrix}
, \quad
\ket{2} = 
\inv{\sqrt{5}}
\begin{bmatrix}
i \\
2
\end{bmatrix}
\end{align*}

Here we have

\begin{align*}
\braket{1}{1} = 
\inv{2}
\begin{bmatrix}
1 & -i
\end{bmatrix}
\begin{bmatrix}
1 \\
i
\end{bmatrix}
= 1
\end{align*}

and 

\begin{align*}
\braket{2}{2} = 
\inv{5}
\begin{bmatrix}
-i & 2 
\end{bmatrix}
\begin{bmatrix}
i \\
2
\end{bmatrix}
= 1
\end{align*}

\begin{align*}
\braket{1}{2} = 
\inv{\sqrt{10}}
\begin{bmatrix}
1 & -i
\end{bmatrix}
\begin{bmatrix}
i \\
2
\end{bmatrix}
= -\frac{i}{\sqrt{10}}
\end{align*}

\begin{align*}
\braket{2}{1} = 
\inv{\sqrt{10}}
\begin{bmatrix}
-i & 2
\end{bmatrix}
\begin{bmatrix}
1 \\
i
\end{bmatrix}
= \frac{i}{\sqrt{10}}
\end{align*}

Definitely not the dual space.  The conjugates are only going to be the dual basis when the primary basis is orthonormal.

Switching back to abstraction temporarily, let's calculate the coordinates with respect to a non-orthonormal
basis of dimension $k$.  That is, determine the $\alpha_i$ given a decomposition by basis vectors

\begin{align*}
\ket{x} = \sum_i \alpha_i \ket{a_i}
\end{align*}

We have 
\begin{align*}
\braket{a_j}{x} 
&= \sum_i \alpha_i \braket{a_j}{a_i} \\
&=
\begin{bmatrix}
\braket{a_j}{a_1} & \braket{a_j}{a_2} & \hdots & \braket{a_j}{a_k}
\end{bmatrix}
\begin{bmatrix}
\alpha_1 \\
\alpha_2 \\
\vdots \\
\alpha_k \\
\end{bmatrix}
\end{align*}

Assembling these into a matrix with a column for each $j$, we have
\begin{align*}
\begin{bmatrix}
\braket{a_1}{x} \\
\braket{a_2}{x} \\
\vdots \\
\braket{a_k}{x} \\
\end{bmatrix}
&=
{\begin{bmatrix}
\braket{a_i}{a_j}
\end{bmatrix}}_{ij}
\Balpha
\end{align*}

With,

\begin{align*}
A = 
\begin{bmatrix}
\ket{a_1} & \ket{a_2} & \hdots & \ket{a_k}
\end{bmatrix},
\end{align*}

this is

\begin{align*}
\Balpha = \inv{A^\conj A} A^\conj \ket{x}
\end{align*}

or 
\begin{align*}
\ket{x} = A \Balpha = A \inv{A^\conj A} A^\conj \ket{x}
\end{align*}

From this we can pick off the reciprocal frame vectors, which are the columns of 

\begin{align*}
\begin{bmatrix}
\ket{a^1} & \ket{a^2} & \hdots & \ket{a^k}
\end{bmatrix} = 
A \inv{A^\conj A}
\end{align*}

To verify we calculate 
\begin{align*}
{\begin{bmatrix}
\ket{a^1} & \ket{a^2} & \hdots & \ket{a^k}
\end{bmatrix}}^\conj
\begin{bmatrix}
\ket{a_1} & \ket{a_2} & \hdots & \ket{a_k}
\end{bmatrix}
&=
\begin{bmatrix}
\bra{a^1} \\ \bra{a^2} \\ \vdots \\ \bra{a^k}
\end{bmatrix}
\begin{bmatrix}
\ket{a_1} & \ket{a_2} & \hdots & \ket{a_k}
\end{bmatrix} \\
&= 
{
\begin{bmatrix}
\braket{a^i}{a_j}
\end{bmatrix}}_{ij}
\end{align*}

with the expectation that this is the identity matrix.  That product is

\begin{align*}
&\left(
\begin{bmatrix}
\ket{a_1} & \ket{a_2} & \hdots & \ket{a_k}
\end{bmatrix}
\inv{{\begin{bmatrix}
\braket{a_i}{a_j}
\end{bmatrix}}_{ij}} \right)^\conj
\begin{bmatrix}
\ket{a_1} & \ket{a_2} & \hdots & \ket{a_k}
\end{bmatrix} \\
&=
\inv{{\begin{bmatrix}
\braket{a_j}{a_i}^\conj
\end{bmatrix}}_{ij}} 
\begin{bmatrix}
\bra{a_1} \\ \bra{a_2} \\ \vdots \\ \bra{a_k}
\end{bmatrix}
\begin{bmatrix}
\ket{a_1} & \ket{a_2} & \hdots & \ket{a_k}
\end{bmatrix} \\
&=
\inv{{\begin{bmatrix}
\braket{a_i}{a_j}
\end{bmatrix}}_{ij}} 
{{\begin{bmatrix}
\braket{a_i}{a_j}
\end{bmatrix}}_{ij}} \\
&= I
\end{align*}

This proves the desired result, that we can calculate $\braket{a^i}{a_j} = \delta_{ij}$ where 

\begin{align*}
\begin{bmatrix}
\ket{a^1} & \ket{a^2} & \hdots & \ket{a^k}
\end{bmatrix} = 
\begin{bmatrix}
\ket{a_1} & \ket{a_2} & \hdots & \ket{a_k}
\end{bmatrix}
\inv{{\begin{bmatrix}
\braket{a_i}{a_j}
\end{bmatrix}}_{ij}}
\end{align*}

While kind of fun to see how to express this in the bra ket notation, is this useful.  Probably not since
all the operators of QM are Hermitian, and thus have orthonormal basis (the eigenvectors).  Oh well... it has
provided some comfort with the notation if nothing else.

\subsection{ Hermitian operators. }

The operators of QM are written with hats, and are applied to vectors (states).  For example for an 
operator $\hat{H}$ applied to $\ket{x}$ we write

\begin{align*}
\hat{H} \ket{x}
\end{align*}

FIXME: Express op applied to a vector in coordinates.

Additionally, the inner product is written with a sandwich bra ket format like

\begin{align*}
\braket{y}{ \left(\hat{H} \ket{x} \right) }
=
\BraOpKet{y}{H}{x}
\end{align*}

In the braket notation a Hermitian operator $\hat{H}$ is defined as one 

FIXME.

and anti-Hermitian is:

FIXME:

\subsubsection{ Tricky eigenvalue notation. } 



\section{ Postulates of QM. }



\section{ Position operator. }

\section{ Derivative (momentum) operator. }

\subsection{ Proper characterization as momentum. }

There is a discussion in the lectures mentioning that a wave function that is peaked at a position in space
corresponds to the position of a particle in an approximate, but intuitively reasonable seeming fashion.

It is mentioned however, to properly show the same sort of characterization for the 
equivalence of the QM momentum construction and momentum of classical
physics requires the consideration of a (large and massive) wave packet.  It can be shown that the equations that
govern the motion
of such an object approximate familiar newtonian dynamics in this limit.  It will be interesting to see how
this pans out.  My assumption is that if we start from the relativistic (Dirac) wave equations, we can also
get the relativistic dynamics equations, and presumably also ideas like the classical current density vector
of electromagnetism.  Very exciting to see that it is at least possible to formulate the classical results from
more fundamental underlying principles, even if I don't know what those are yet.  How will all of this relate 
to the Lagrangian formulation that we can express newtonian and relativistic dynamics and electromagnetism using.
Can one in fact produce the classical Lagrangians for (proper) Lorentz force, and Maxwell's equation (or at least the $A \cdot J$ term) directly from the QM Lagrangians?

%\bibliographystyle{plainnat}
%\bibliography{myrefs}

\end{document}

%
% Copyright � 2012 Peeter Joot.  All Rights Reserved.
% Licenced as described in the file LICENSE under the root directory of this GIT repository.
%

%
%
%\documentclass{article}

%\usepackage{amsmath}
\usepackage{mathpazo}

%
% shorthand for bold symbols, convenient for vectors and matrices
%
\newcommand{\Ba}[0]{\mathbf{a}}
\newcommand{\Bb}[0]{\mathbf{b}}
\newcommand{\Bc}[0]{\mathbf{c}}
\newcommand{\Bd}[0]{\mathbf{d}}
\newcommand{\Be}[0]{\mathbf{e}}
\newcommand{\Bf}[0]{\mathbf{f}}
\newcommand{\Bg}[0]{\mathbf{g}}
\newcommand{\Bh}[0]{\mathbf{h}}
\newcommand{\Bi}[0]{\mathbf{i}}
\newcommand{\Bj}[0]{\mathbf{j}}
\newcommand{\Bk}[0]{\mathbf{k}}
\newcommand{\Bl}[0]{\mathbf{l}}
\newcommand{\Bm}[0]{\mathbf{m}}
\newcommand{\Bn}[0]{\mathbf{n}}
\newcommand{\Bo}[0]{\mathbf{o}}
\newcommand{\Bp}[0]{\mathbf{p}}
\newcommand{\Bq}[0]{\mathbf{q}}
\newcommand{\Br}[0]{\mathbf{r}}
\newcommand{\Bs}[0]{\mathbf{s}}
\newcommand{\Bt}[0]{\mathbf{t}}
\newcommand{\Bu}[0]{\mathbf{u}}
\newcommand{\Bv}[0]{\mathbf{v}}
\newcommand{\Bw}[0]{\mathbf{w}}
\newcommand{\Bx}[0]{\mathbf{x}}
\newcommand{\By}[0]{\mathbf{y}}
\newcommand{\Bz}[0]{\mathbf{z}}
\newcommand{\BA}[0]{\mathbf{A}}
\newcommand{\BB}[0]{\mathbf{B}}
\newcommand{\BC}[0]{\mathbf{C}}
\newcommand{\BD}[0]{\mathbf{D}}
\newcommand{\BE}[0]{\mathbf{E}}
\newcommand{\BF}[0]{\mathbf{F}}
\newcommand{\BG}[0]{\mathbf{G}}
\newcommand{\BH}[0]{\mathbf{H}}
\newcommand{\BI}[0]{\mathbf{I}}
\newcommand{\BJ}[0]{\mathbf{J}}
\newcommand{\BK}[0]{\mathbf{K}}
\newcommand{\BL}[0]{\mathbf{L}}
\newcommand{\BM}[0]{\mathbf{M}}
\newcommand{\BN}[0]{\mathbf{N}}
\newcommand{\BO}[0]{\mathbf{O}}
\newcommand{\BP}[0]{\mathbf{P}}
\newcommand{\BQ}[0]{\mathbf{Q}}
\newcommand{\BR}[0]{\mathbf{R}}
\newcommand{\BS}[0]{\mathbf{S}}
\newcommand{\BT}[0]{\mathbf{T}}
\newcommand{\BU}[0]{\mathbf{U}}
\newcommand{\BV}[0]{\mathbf{V}}
\newcommand{\BW}[0]{\mathbf{W}}
\newcommand{\BX}[0]{\mathbf{X}}
\newcommand{\BY}[0]{\mathbf{Y}}
\newcommand{\BZ}[0]{\mathbf{Z}}

\newcommand{\Bzero}[0]{\mathbf{0}}
\newcommand{\Btheta}[0]{\boldsymbol{\theta}}
\newcommand{\Btau}[0]{\boldsymbol{\tau}}
\newcommand{\Bomega}[0]{\boldsymbol{\omega}}

%
% shorthand for unit vectors
%
\newcommand{\acap}[0]{\hat{\Ba}}
\newcommand{\bcap}[0]{\hat{\Bb}}
\newcommand{\ccap}[0]{\hat{\Bc}}
\newcommand{\dcap}[0]{\hat{\Bd}}
\newcommand{\ecap}[0]{\hat{\Be}}
\newcommand{\fcap}[0]{\hat{\Bf}}
\newcommand{\gcap}[0]{\hat{\Bg}}
\newcommand{\hcap}[0]{\hat{\Bh}}
\newcommand{\icap}[0]{\hat{\Bi}}
\newcommand{\jcap}[0]{\hat{\Bj}}
\newcommand{\kcap}[0]{\hat{\Bk}}
\newcommand{\lcap}[0]{\hat{\Bl}}
\newcommand{\mcap}[0]{\hat{\Bm}}
\newcommand{\ncap}[0]{\hat{\Bn}}
\newcommand{\ocap}[0]{\hat{\Bo}}
\newcommand{\pcap}[0]{\hat{\Bp}}
\newcommand{\qcap}[0]{\hat{\Bq}}
\newcommand{\rcap}[0]{\hat{\Br}}
\newcommand{\scap}[0]{\hat{\Bs}}
\newcommand{\tcap}[0]{\hat{\Bt}}
\newcommand{\ucap}[0]{\hat{\Bu}}
\newcommand{\vcap}[0]{\hat{\Bv}}
\newcommand{\wcap}[0]{\hat{\Bw}}
\newcommand{\xcap}[0]{\hat{\Bx}}
\newcommand{\ycap}[0]{\hat{\By}}
\newcommand{\zcap}[0]{\hat{\Bz}}
\newcommand{\thetacap}[0]{\hat{\Btheta}}

%
% to write R^n and C^n in a distinguishable fashion.  Perhaps change this
% to the double lined characters upon figuring out how to do so.
%
\newcommand{\C}[1]{$\mathbb{C}^{#1}$}
\newcommand{\R}[1]{$\mathbb{R}^{#1}$}

%
% various generally useful helpers
%

% derivative of #1 wrt. #2:
\newcommand{\D}[2] {\frac {d#2} {d#1}}

\newcommand{\inv}[1]{\frac{1}{#1}}
\newcommand{\cross}[0]{\times}

\newcommand{\abs}[1]{\lvert{#1}\rvert}
\newcommand{\norm}[1]{\lVert{#1}\rVert}
\newcommand{\innerprod}[2]{\langle{#1}, {#2}\rangle}
\newcommand{\dotprod}[2]{{#1} \cdot {#2}}
\newcommand{\bdotprod}[2]{\left({#1} \cdot {#2}\right)}
\newcommand{\crossprod}[2]{{#1} \cross {#2}}
\newcommand{\tripleprod}[3]{\dotprod{\left(\crossprod{#1}{#2}\right)}{#3}}

\DeclareMathOperator{\Proj}{Proj}
\DeclareMathOperator{\Span}{span}
\DeclareMathOperator{\Sgn}{sgn}
\DeclareMathOperator{\Area}{Area}
\DeclareMathOperator{\Volume}{Volume}

%
% A few miscellaneous things specific to this document
%
\newcommand{\crossop}[1]{\crossprod{#1}{}}

% R2 vector.
\newcommand{\VectorTwo}[2]{
\begin{bmatrix}
 {#1} \\
 {#2}
\end{bmatrix}
}

\newcommand{\VectorN}[1]{
\begin{bmatrix}
{#1}_1 \\
{#1}_2 \\
\vdots \\
{#1}_N \\
\end{bmatrix}
}

\newcommand{\DETuvij}[4]{
\begin{vmatrix}
 {#1}_{#3} & {#1}_{#4} \\
 {#2}_{#3} & {#2}_{#4}
\end{vmatrix}
}

\newcommand{\DETuvwijk}[6]{
\begin{vmatrix}
 {#1}_{#4} & {#1}_{#5} & {#1}_{#6} \\
 {#2}_{#4} & {#2}_{#5} & {#2}_{#6} \\
 {#3}_{#4} & {#3}_{#5} & {#3}_{#6}
\end{vmatrix}
}

\newcommand{\DETuvwxijkl}[8]{
\begin{vmatrix}
 {#1}_{#5} & {#1}_{#6} & {#1}_{#7} & {#1}_{#8} \\
 {#2}_{#5} & {#2}_{#6} & {#2}_{#7} & {#2}_{#8} \\
 {#3}_{#5} & {#3}_{#6} & {#3}_{#7} & {#3}_{#8} \\
 {#4}_{#5} & {#4}_{#6} & {#4}_{#7} & {#4}_{#8} \\
\end{vmatrix}
}

%\newcommand{\DETuvwxyijklm}[10]{
%\begin{vmatrix}
% {#1}_{#6} & {#1}_{#7} & {#1}_{#8} & {#1}_{#9} & {#1}_{#10} \\
% {#2}_{#6} & {#2}_{#7} & {#2}_{#8} & {#2}_{#9} & {#2}_{#10} \\
% {#3}_{#6} & {#3}_{#7} & {#3}_{#8} & {#3}_{#9} & {#3}_{#10} \\
% {#4}_{#6} & {#4}_{#7} & {#4}_{#8} & {#4}_{#9} & {#4}_{#10} \\
% {#5}_{#6} & {#5}_{#7} & {#5}_{#8} & {#5}_{#9} & {#5}_{#10}
%\end{vmatrix}
%}

% R3 vector.
\newcommand{\VectorThree}[3]{
\begin{bmatrix}
 {#1} \\
 {#2} \\
 {#3}
\end{bmatrix}
}


%%<misc>
%
\newcommand{\Abs}[1]{{\left\lvert{#1}\right\rvert}}
\newcommand{\spacegrad}[0]{\boldsymbol{\nabla}}
\newcommand{\grad}[0]{\nabla}
\newcommand{\LL}[0]{\mathcal{L}}

% == \partial_{#1} {#2}
\newcommand{\PD}[2]{\frac{\partial {#2}}{\partial {#1}}}
% inline variant
\newcommand{\PDi}[2]{{\partial {#2}}/{\partial {#1}}}

\newcommand{\PDD}[3]{\frac{\partial^2 {#3}}{\partial {#1}\partial {#2}}}
%\newcommand{\PDd}[2]{\frac{\partial^2 {#2}}{{\partial{#1}}^2}}
\newcommand{\PDsq}[2]{\frac{\partial^2 {#2}}{(\partial {#1})^2}}

\newcommand{\Partial}[2]{\frac{\partial {#1}}{\partial {#2}}}
\DeclareMathOperator{\RejName}{Rej}
\newcommand{\Rej}[2]{\RejName_{#1}\left( {#2} \right)}
\newcommand{\Rm}[1]{\mathbb{R}^{#1}}
\newcommand{\Cm}[1]{\mathbb{C}^{#1}}
\newcommand{\conj}[0]{{*}}

%</misc>

% <grade selection>
%
\newcommand{\gpgrade}[2] {{\left\langle{{#1}}\right\rangle}_{#2}}

\newcommand{\gpgradezero}[1] {\gpgrade{#1}{}}
%\newcommand{\gpscalargrade}[1] {{\left\langle{{#1}}\right\rangle}}
%\newcommand{\gpgradezero}[1] {\gpgrade{#1}{0}}

%\newcommand{\gpgradeone}[1] {{\left\langle{{#1}}\right\rangle}_{1}}
\newcommand{\gpgradeone}[1] {\gpgrade{#1}{1}}

\newcommand{\gpgradetwo}[1] {\gpgrade{#1}{2}}
\newcommand{\gpgradethree}[1] {\gpgrade{#1}{3}}
\newcommand{\gpgradefour}[1] {\gpgrade{#1}{4}}
%
% </grade selection>



\newcommand{\adot}[0]{{\dot{a}}}
\newcommand{\bdot}[0]{{\dot{b}}}
% taken for centered dot:
%\newcommand{\cdot}[0]{{\dot{c}}}
%\newcommand{\ddot}[0]{{\dot{d}}}
\newcommand{\edot}[0]{{\dot{e}}}
\newcommand{\fdot}[0]{{\dot{f}}}
\newcommand{\gdot}[0]{{\dot{g}}}
\newcommand{\hdot}[0]{{\dot{h}}}
\newcommand{\idot}[0]{{\dot{i}}}
\newcommand{\jdot}[0]{{\dot{j}}}
\newcommand{\kdot}[0]{{\dot{k}}}
\newcommand{\ldot}[0]{{\dot{l}}}
\newcommand{\mdot}[0]{{\dot{m}}}
\newcommand{\ndot}[0]{{\dot{n}}}
%\newcommand{\odot}[0]{{\dot{o}}}
\newcommand{\pdot}[0]{{\dot{p}}}
\newcommand{\qdot}[0]{{\dot{q}}}
\newcommand{\rdot}[0]{{\dot{r}}}
\newcommand{\sdot}[0]{{\dot{s}}}
\newcommand{\tdot}[0]{{\dot{t}}}
\newcommand{\udot}[0]{{\dot{u}}}
\newcommand{\vdot}[0]{{\dot{v}}}
\newcommand{\wdot}[0]{{\dot{w}}}
\newcommand{\xdot}[0]{{\dot{x}}}
\newcommand{\ydot}[0]{{\dot{y}}}
\newcommand{\zdot}[0]{{\dot{z}}}
\newcommand{\addot}[0]{{\ddot{a}}}
\newcommand{\bddot}[0]{{\ddot{b}}}
\newcommand{\cddot}[0]{{\ddot{c}}}
%\newcommand{\dddot}[0]{{\ddot{d}}}
\newcommand{\eddot}[0]{{\ddot{e}}}
\newcommand{\fddot}[0]{{\ddot{f}}}
\newcommand{\gddot}[0]{{\ddot{g}}}
\newcommand{\hddot}[0]{{\ddot{h}}}
\newcommand{\iddot}[0]{{\ddot{i}}}
\newcommand{\jddot}[0]{{\ddot{j}}}
\newcommand{\kddot}[0]{{\ddot{k}}}
\newcommand{\lddot}[0]{{\ddot{l}}}
\newcommand{\mddot}[0]{{\ddot{m}}}
\newcommand{\nddot}[0]{{\ddot{n}}}
\newcommand{\oddot}[0]{{\ddot{o}}}
\newcommand{\pddot}[0]{{\ddot{p}}}
\newcommand{\qddot}[0]{{\ddot{q}}}
\newcommand{\rddot}[0]{{\ddot{r}}}
\newcommand{\sddot}[0]{{\ddot{s}}}
\newcommand{\tddot}[0]{{\ddot{t}}}
\newcommand{\uddot}[0]{{\ddot{u}}}
\newcommand{\vddot}[0]{{\ddot{v}}}
\newcommand{\wddot}[0]{{\ddot{w}}}
\newcommand{\xddot}[0]{{\ddot{x}}}
\newcommand{\yddot}[0]{{\ddot{y}}}
\newcommand{\zddot}[0]{{\ddot{z}}}

%<bold and dot greek symbols>
%

\newcommand{\Deltadot}[0]{{\dot{\Delta}}}
\newcommand{\Gammadot}[0]{{\dot{\Gamma}}}
\newcommand{\Lambdadot}[0]{{\dot{\Lambda}}}
\newcommand{\Omegadot}[0]{{\dot{\Omega}}}
\newcommand{\Phidot}[0]{{\dot{\Phi}}}
\newcommand{\Pidot}[0]{{\dot{\Pi}}}
\newcommand{\Psidot}[0]{{\dot{\Psi}}}
\newcommand{\Sigmadot}[0]{{\dot{\Sigma}}}
\newcommand{\Thetadot}[0]{{\dot{\Theta}}}
\newcommand{\Upsilondot}[0]{{\dot{\Upsilon}}}
\newcommand{\Xidot}[0]{{\dot{\Xi}}}
\newcommand{\alphadot}[0]{{\dot{\alpha}}}
\newcommand{\betadot}[0]{{\dot{\beta}}}
\newcommand{\chidot}[0]{{\dot{\chi}}}
\newcommand{\deltadot}[0]{{\dot{\delta}}}
\newcommand{\epsilondot}[0]{{\dot{\epsilon}}}
\newcommand{\etadot}[0]{{\dot{\eta}}}
\newcommand{\gammadot}[0]{{\dot{\gamma}}}
\newcommand{\kappadot}[0]{{\dot{\kappa}}}
\newcommand{\lambdadot}[0]{{\dot{\lambda}}}
\newcommand{\mudot}[0]{{\dot{\mu}}}
\newcommand{\nudot}[0]{{\dot{\nu}}}
\newcommand{\omegadot}[0]{{\dot{\omega}}}
\newcommand{\phidot}[0]{{\dot{\phi}}}
\newcommand{\pidot}[0]{{\dot{\pi}}}
\newcommand{\psidot}[0]{{\dot{\psi}}}
\newcommand{\rhodot}[0]{{\dot{\rho}}}
\newcommand{\sigmadot}[0]{{\dot{\sigma}}}
\newcommand{\taudot}[0]{{\dot{\tau}}}
\newcommand{\thetadot}[0]{{\dot{\theta}}}
\newcommand{\upsilondot}[0]{{\dot{\upsilon}}}
\newcommand{\varepsilondot}[0]{{\dot{\varepsilon}}}
\newcommand{\varphidot}[0]{{\dot{\varphi}}}
\newcommand{\varpidot}[0]{{\dot{\varpi}}}
\newcommand{\varrhodot}[0]{{\dot{\varrho}}}
\newcommand{\varsigmadot}[0]{{\dot{\varsigma}}}
\newcommand{\varthetadot}[0]{{\dot{\vartheta}}}
\newcommand{\xidot}[0]{{\dot{\xi}}}
\newcommand{\zetadot}[0]{{\dot{\zeta}}}

\newcommand{\Deltaddot}[0]{{\ddot{\Delta}}}
\newcommand{\Gammaddot}[0]{{\ddot{\Gamma}}}
\newcommand{\Lambdaddot}[0]{{\ddot{\Lambda}}}
\newcommand{\Omegaddot}[0]{{\ddot{\Omega}}}
\newcommand{\Phiddot}[0]{{\ddot{\Phi}}}
\newcommand{\Piddot}[0]{{\ddot{\Pi}}}
\newcommand{\Psiddot}[0]{{\ddot{\Psi}}}
\newcommand{\Sigmaddot}[0]{{\ddot{\Sigma}}}
\newcommand{\Thetaddot}[0]{{\ddot{\Theta}}}
\newcommand{\Upsilonddot}[0]{{\ddot{\Upsilon}}}
\newcommand{\Xiddot}[0]{{\ddot{\Xi}}}
\newcommand{\alphaddot}[0]{{\ddot{\alpha}}}
\newcommand{\betaddot}[0]{{\ddot{\beta}}}
\newcommand{\chiddot}[0]{{\ddot{\chi}}}
\newcommand{\deltaddot}[0]{{\ddot{\delta}}}
\newcommand{\epsilonddot}[0]{{\ddot{\epsilon}}}
\newcommand{\etaddot}[0]{{\ddot{\eta}}}
\newcommand{\gammaddot}[0]{{\ddot{\gamma}}}
\newcommand{\kappaddot}[0]{{\ddot{\kappa}}}
\newcommand{\lambdaddot}[0]{{\ddot{\lambda}}}
\newcommand{\muddot}[0]{{\ddot{\mu}}}
\newcommand{\nuddot}[0]{{\ddot{\nu}}}
\newcommand{\omegaddot}[0]{{\ddot{\omega}}}
\newcommand{\phiddot}[0]{{\ddot{\phi}}}
\newcommand{\piddot}[0]{{\ddot{\pi}}}
\newcommand{\psiddot}[0]{{\ddot{\psi}}}
\newcommand{\rhoddot}[0]{{\ddot{\rho}}}
\newcommand{\sigmaddot}[0]{{\ddot{\sigma}}}
\newcommand{\tauddot}[0]{{\ddot{\tau}}}
\newcommand{\thetaddot}[0]{{\ddot{\theta}}}
\newcommand{\upsilonddot}[0]{{\ddot{\upsilon}}}
\newcommand{\varepsilonddot}[0]{{\ddot{\varepsilon}}}
\newcommand{\varphiddot}[0]{{\ddot{\varphi}}}
\newcommand{\varpiddot}[0]{{\ddot{\varpi}}}
\newcommand{\varrhoddot}[0]{{\ddot{\varrho}}}
\newcommand{\varsigmaddot}[0]{{\ddot{\varsigma}}}
\newcommand{\varthetaddot}[0]{{\ddot{\vartheta}}}
\newcommand{\xiddot}[0]{{\ddot{\xi}}}
\newcommand{\zetaddot}[0]{{\ddot{\zeta}}}

\newcommand{\BDelta}[0]{\boldsymbol{\Delta}}
\newcommand{\BGamma}[0]{\boldsymbol{\Gamma}}
\newcommand{\BLambda}[0]{\boldsymbol{\Lambda}}
\newcommand{\BOmega}[0]{\boldsymbol{\Omega}}
\newcommand{\BPhi}[0]{\boldsymbol{\Phi}}
\newcommand{\BPi}[0]{\boldsymbol{\Pi}}
\newcommand{\BPsi}[0]{\boldsymbol{\Psi}}
\newcommand{\BSigma}[0]{\boldsymbol{\Sigma}}
\newcommand{\BTheta}[0]{\boldsymbol{\Theta}}
\newcommand{\BUpsilon}[0]{\boldsymbol{\Upsilon}}
\newcommand{\BXi}[0]{\boldsymbol{\Xi}}
\newcommand{\Balpha}[0]{\boldsymbol{\alpha}}
\newcommand{\Bbeta}[0]{\boldsymbol{\beta}}
\newcommand{\Bchi}[0]{\boldsymbol{\chi}}
\newcommand{\Bdelta}[0]{\boldsymbol{\delta}}
\newcommand{\Bepsilon}[0]{\boldsymbol{\epsilon}}
\newcommand{\Beta}[0]{\boldsymbol{\eta}}
\newcommand{\Bgamma}[0]{\boldsymbol{\gamma}}
\newcommand{\Bkappa}[0]{\boldsymbol{\kappa}}
\newcommand{\Blambda}[0]{\boldsymbol{\lambda}}
\newcommand{\Bmu}[0]{\boldsymbol{\mu}}
\newcommand{\Bnu}[0]{\boldsymbol{\nu}}
%\newcommand{\Bomega}[0]{\boldsymbol{\omega}}
\newcommand{\Bphi}[0]{\boldsymbol{\phi}}
\newcommand{\Bpi}[0]{\boldsymbol{\pi}}
\newcommand{\Bpsi}[0]{\boldsymbol{\psi}}
\newcommand{\Brho}[0]{\boldsymbol{\rho}}
\newcommand{\Bsigma}[0]{\boldsymbol{\sigma}}
%\newcommand{\Btau}[0]{\boldsymbol{\tau}}
%\newcommand{\Btheta}[0]{\boldsymbol{\theta}}
\newcommand{\Bupsilon}[0]{\boldsymbol{\upsilon}}
\newcommand{\Bvarepsilon}[0]{\boldsymbol{\varepsilon}}
\newcommand{\Bvarphi}[0]{\boldsymbol{\varphi}}
\newcommand{\Bvarpi}[0]{\boldsymbol{\varpi}}
\newcommand{\Bvarrho}[0]{\boldsymbol{\varrho}}
\newcommand{\Bvarsigma}[0]{\boldsymbol{\varsigma}}
\newcommand{\Bvartheta}[0]{\boldsymbol{\vartheta}}
\newcommand{\Bxi}[0]{\boldsymbol{\xi}}
\newcommand{\Bzeta}[0]{\boldsymbol{\zeta}}
%
%</bold and dot greek symbols>
%<infrequent>
%
%\newcommand{\AreaOp}[1]{\AName_{#1}}
%\newcommand{\Babs}[0]{\abs{\BB}}
%\newcommand{\Bcap}[0]{\hat{\BB}}
%\newcommand{\BrPrimeRej}[0]{\rcap(\rcap \wedge \Br')}
%\newcommand{\CA}[0]{\mathcal{A}}
%\newcommand{\Cos}[1]{\cos{\left({#1}\right)}}
%\newcommand{\Det}[1] {\abs{#1}}
%\newcommand{\Dsq}[2] {\frac {\partial^2 {#1}} {\partial {#2}^2}}
%\newcommand{\Exp}[1]{\exp{\left({#1}\right)}}
%\newcommand{\Norm}[1]{\left\lVert{#1}\right\rVert}
%\newcommand{\Sin}[1]{\sin{\left({#1}\right)}}
%\newcommand{\T}[0]{\text{T}}
%\newcommand{\VolumeOp}[1]{\VName_{#1}}
%\newcommand{\agrad}[0]{\Ba \cdot \nabla}
%\newcommand{\alphacap}[0]{\hat{\boldsymbol{\alpha}}}
%\newcommand{\Fcap}[0]{\hat{\BF}}
%\newcommand{\bithree}[0]{{\Bi}_3}
%\newcommand{\bxa}[0]{\Bx\Ba}
%\newcommand{\coordvec}[2]{
%\newcommand{\costheta}[0]{\acap \cdot \xcap}
%\newcommand{\ddt}[1]{\ddot{#1}}
%\newcommand{\ddu}[1] {\frac {d{#1}} {du}}
%\newcommand{\dsqxj}[2] {\frac {\partial^2 {#1}} {\partial {x_{#2}}^2}}
%\newcommand{\dtheta}[1]{\frac{d {#1}}{d \theta}}
%\newcommand{\dt}[1]{\dot{#1}}
%\newcommand{\dt}[1]{\frac{d {#1}}{dt}}
%\newcommand{\dxj}[2] {\frac {\partial {#1}} {\partial {x_{#2}}}}
%\newcommand{\halfPhi}[0]{\frac{\phi}{2}}
%\newcommand{\half}[0]{\inv{2}}
%\newcommand{\inv}[1]{\frac{1}{#1}}
%\newcommand{\laplacian}[0]{\nabla^2}
%\newcommand{\matrixoftx}[3]{
%\newcommand{\nrrp}[0]{\norm{\rcap \wedge \Br'}}
%\newcommand{\oiint}{\bigcirc \hspace{-1.4em} \int \hspace{-.8em} \int}
%\newcommand{\transpose}[1]{{#1}^{\text{T}}}
%\newcommand{\transpose}[1]{{{#1}^{\TextTranspose}}}
%\newcommand{\transpose}[1]{{{#1}^{\text{T}}}}
%\newcommand{\barA}[0]{\bar{A}}
%\newcommand{\qbar}[0]{\bar{q}}
%\newcommand{\qdotbar}[0]{\dot{\bar{q}}}
%
%</infrequent>





%\usepackage{listings}
%\usepackage{txfonts} % for ointctr... (also appears to make "prettier" \int and \sum's)
% makes \grad look funny though (almost like spacegrad, but narrower)
%\usepackage[bookmarks=true]{hyperref}

%\usepackage{color,cite,graphicx}
   % use colour in the document, put your citations as [1-4]
   % rather than [1,2,3,4] (it looks nicer, and the extended LaTeX2e
   % graphics package.
%\usepackage{latexsym,amssymb,epsf} % do not remember if these are
   % needed, but their inclusion can not do any damage


\chapter{VERSION: First try at one dimensional rectangular Quantum barrier problem}
\label{chap:qmbFirstTry}
%\author{Peeter Joot \quad peeterjoot@protonmail.com }
\date{ May 9, 2009.  \(RCSfile: qmbFirstTry.tex,v \) Last \(Revision: 1.8 \) \(Date: 2009/06/14 23:51:45 \) }

%\begin{document}

%\maketitle{}
%\tableofcontents
\section{}

\subsection{Problem 4.  Barrier penetration}

\subsubsection{Setup}

%In the barrier \(x \in [0,a]\),
%
%\begin{align*}
%\psi
%&=
%\frac{A}{2}\left(1 + i \frac{p_1}{p_2}\right) \exp\left( (ip_1 - p_2)\frac{a}{\Hbar}
%+\frac{p_2 x}{\Hbar}\right)  \\
%&+
%\frac{A}{2}\left(1 - i \frac{p_1}{p_2}\right) \exp\left( (ip_1 + p_2)\frac{a}{\Hbar}
%-\frac{p_2 x}{\Hbar}\right) \\
%\end{align*}
%
%Or
%\begin{align*}
%\psi
%&=
%A
%\exp\left( ip_1 \frac{a}{\Hbar} \right)
%\cosh\left( p_2 \frac{(a-x)}{\Hbar} \right) \\
%&+
%i A
%\exp\left( ip_1 \frac{a}{\Hbar} \right)
%\inv{2}\left(
%\frac{p_1}{p_2}\exp\left( p_2 \frac{(x-a)}{\Hbar} \right)
%-\frac{p_2}{p_1}\exp\left( p_2 \frac{(a-x)}{\Hbar} \right)
%\right)
%\end{align*}
%
%Past the barrier, \(x >a\),
%\begin{align*}
%\psi =
%A \exp\left( \frac{ip_1 x}{\Hbar}\right)
%\end{align*}
%
%and before the barrier, \(x<0\)
%
%\begin{align*}
%\psi &=
%D \exp\left( \frac{ip_1 x}{\Hbar}\right)
%+E \exp\left( \frac{-ip_1 x}{\Hbar}\right) \\
%D
%&=
%\frac{C}{2} \left( 1 + i \frac{p_2 }{p_1}\right)
%+\frac{B}{2} \left( 1 - i \frac{p_2 }{p_1}\right) \\
%E
%&=
%\frac{C}{2} \left( 1 - i \frac{p_2 }{p_1}\right)
%+\frac{B}{2} \left( 1 + i \frac{p_2 }{p_1}\right) \\
%B &= \frac{A}{2}\left(1 + i \frac{p_1}{p_2}\right) \exp\left( (ip_1 - p_2)\frac{a}{\Hbar}\right) \\
%C &= \frac{A}{2}\left(1 - i \frac{p_1}{p_2}\right) \exp\left( (ip_1 + p_2)\frac{a}{\Hbar}\right) \\
%\end{align*}
%
%That is
%
%\begin{align*}
%D
%&=
%\frac{A}{4}\left(1 - i \frac{p_1}{p_2}\right) \exp\left( (ip_1 + p_2)\frac{a}{\Hbar}\right)
%\left( 1 + i \frac{p_2 }{p_1}\right) \\
%&+
%\frac{A}{4}\left(1 + i \frac{p_1}{p_2}\right) \exp\left( (ip_1 - p_2)\frac{a}{\Hbar}\right)
%\left( 1 - i \frac{p_2 }{p_1}\right) \\
%E
%&=
%\frac{A}{4}\left(1 - i \frac{p_1}{p_2}\right) \exp\left( (ip_1 + p_2)\frac{a}{\Hbar}\right)
%\left( 1 - i \frac{p_2 }{p_1}\right) \\
%&+
%\frac{A}{4}\left(1 + i \frac{p_1}{p_2}\right) \exp\left( (ip_1 - p_2)\frac{a}{\Hbar}\right)
%\left( 1 + i \frac{p_2 }{p_1}\right) \\
%\end{align*}
%
%Expanding the products first for \(D\) we have
%
%\begin{align*}
%D
%&=
%\frac{A}{4}
%\left(2 - i \frac{p_1^2 - p_2^2}{p_1 p_2}\right)
%\exp\left( (ip_1 + p_2)\frac{a}{\Hbar}\right)
%\\
%&+
%\frac{A}{4}
%\left(2 + i \frac{p_1^2 - p_2^2}{p_1 p_2}\right)
%\exp\left( (ip_1 - p_2)\frac{a}{\Hbar}\right)
%\\
%&=
%A
%\exp\left( ip_1 \frac{a}{\Hbar}\right)
%\cosh\left( p_2 \frac{a}{\Hbar}\right)
%%%%
%+\frac{iA}{2}\frac{p_2^2 - p_1^2}{p_1 p_2}
%\exp\left( ip_1 \frac{a}{\Hbar}\right)
%\sinh\left( p_2 \frac{a}{\Hbar}\right)
%%%%
%\end{align*}
%
%and for \(E\), we have
%\begin{align*}
%E
%&=
%\frac{-iA}{4}\frac{p_1^2 + p_2^2}{p_1 p_2}
%\exp\left( (ip_1 + p_2)\frac{a}{\Hbar}\right)
%\\
%&+
%\frac{iA}{4}\frac{p_1^2 + p_2^2}{p_1 p_2}
%\exp\left( (ip_1 - p_2)\frac{a}{\Hbar}\right)
%\\
%&=
%\frac{iA}{2}\frac{p_1^2 + p_2^2}{p_1 p_2}
%\exp\left( ip_1 \frac{a}{\Hbar}\right)
%\sinh\left( p_2 \frac{a}{\Hbar}\right)
%\end{align*}
%
%Reassembling, the sum of the incident and reflected wave functions in the \(x<0\) region is
%
%%\begin{align*}
%%D &=
%%A \exp\left( ip_1 \frac{a}{\Hbar}\right) \cosh\left( p_2 \frac{a}{\Hbar}\right)
%%+\frac{iA}{2}\frac{p_2^2 - p_1^2}{p_1 p_2} \exp\left( ip_1 \frac{a}{\Hbar}\right) \sinh\left( p_2 \frac{a}{\Hbar}\right)
%%\end{align*}
%%
%%\begin{align*}
%%E &= \frac{iA}{2}\frac{p_1^2 + p_2^2}{p_1 p_2} \exp\left( ip_1 \frac{a}{\Hbar}\right) \sinh\left( p_2 \frac{a}{\Hbar}\right)
%%\end{align*}
%
%\begin{align*}
%\psi
%&= A \exp\left( ip_1 \frac{(a+x)}{\Hbar}\right) \cosh\left( p_2 \frac{a}{\Hbar}\right) \\
%&+\frac{iA}{2}\frac{p_2^2 - p_1^2}{p_1 p_2} \exp\left( ip_1 \frac{(a+x)}{\Hbar}\right) \sinh\left( p_2 \frac{a}{\Hbar}\right) \\
%&+ \frac{iA}{2}\frac{p_1^2 + p_2^2}{p_1 p_2} \exp\left( ip_1 \frac{(a-x)}{\Hbar}\right) \sinh\left( p_2 \frac{a}{\Hbar}\right)  \\
%%%%
%&= A \exp\left( ip_1 \frac{(a+x)}{\Hbar}\right) \cosh\left( p_2 \frac{a}{\Hbar}\right) \\
%&+{iA}\frac{p_2 }{p_1 }
%\exp\left( ip_1 \frac{a}{\Hbar}\right)
%\cos\left( p_1 \frac{x}{\Hbar}\right)
%\sinh\left( p_2 \frac{a}{\Hbar}\right) \\
%&+A \frac{p_1}{p_2}
%\exp\left( ip_1 \frac{a}{\Hbar}\right)
%\sin\left( p_1 \frac{x}{\Hbar}\right)
%\sinh\left( p_2 \frac{a}{\Hbar}\right) \\
%&= A \exp\left( ip_1 \frac{(a+x)}{\Hbar}\right) \cosh\left( p_2 \frac{a}{\Hbar}\right) \\
%&
%+A \exp\left( ip_1 \frac{a}{\Hbar}\right) \sinh\left( p_2 \frac{a}{\Hbar}\right)
%\left(
%+{i}\frac{p_2 }{p_1 }
%\cos\left( p_1 \frac{x}{\Hbar}\right)
%+ \frac{p_1}{p_2}
%\sin\left( p_1 \frac{x}{\Hbar}\right)
%\right)
%\\
%\end{align*}
%
%This can be made slightly more symmetrical by expanding one of the complex exponential terms.  Doing so, also writing \(\alpha = A e^{i p_1 a/\Hbar}\),
%we can summarize the wave functions in each of the intervals.
%
%For \(x<0\)
%\begin{align*}
%\psi
%&=
%\alpha
%\cosh\left( p_2 \frac{a}{\Hbar}\right)
%%\left( \cos\left( p_1 \frac{x}{\Hbar}\right) + i \sin\left( p_1 \frac{x}{\Hbar}\right) \right) \\
%\exp\left( i p_1 \frac{x}{\Hbar}\right)
%%\\
%%&
%+\alpha
%\sinh\left( p_2 \frac{a}{\Hbar}\right)
%\left( {i}\frac{p_2 }{p_1 } \cos\left( p_1 \frac{x}{\Hbar}\right) + \frac{p_1}{p_2} \sin\left( p_1 \frac{x}{\Hbar}\right) \right)
%\\
%\end{align*}
%
%For \(x \in [0,a]\)
%\begin{align*}
%\psi
%&=
%\alpha
%\cosh\left( p_2 \frac{(x-a)}{\Hbar} \right)
%+
%\frac{i\alpha}{2}\left(
%\frac{p_1}{p_2}\exp\left( p_2 \frac{(x-a)}{\Hbar} \right)
%-\frac{p_2}{p_1}\exp\left( p_2 \frac{(a-x)}{\Hbar} \right)
%\right)
%\end{align*}
%
%and past the barrier, \(x >a\),
%\begin{align*}
%\psi =
%\alpha \exp\left( \frac{ip_1 (x-a)}{\Hbar}\right)
%\end{align*}

Potential barrier, height V, in \([0,a]\), with potential zero everywhere else.  With \(p_1 = \sqrt{2mE}\), and \(p_2 = \sqrt{2m(V-E)}\), a
solution is calculated in the text.  Attempting to substitute for the coefficients found, and reduce them to a form
that looks amenable to calculation of the currents, leads to a mess.  It does however suggest a simpler structure for the
wave functions.  Let us start over and seek solutions in the \(x>a\) region of the form

\begin{equation}\label{eqn:qmbFirstTry:20}
\begin{aligned}
\psi &= \alpha e^{ i p_1 (x-a)/\Hbar}
\end{aligned}
\end{equation}

and in the barrier region \(x \in [0,a]\)

\begin{equation}\label{eqn:qmbFirstTry:40}
\begin{aligned}
\psi &= A e^{ p_1 (x-a)/\Hbar} +B e^{ -p_1 (x-a)/\Hbar}
\end{aligned}
\end{equation}

continuity (value and derivative) at \(x=a\) gives

\begin{equation}\label{eqn:qmbFirstTry:60}
\begin{aligned}
A + B &= \alpha  \\
A - B &= i \alpha \frac{p_1}{p_2}
\end{aligned}
\end{equation}

Solving and reducing we have

\begin{equation}\label{eqn:qmbFirstTry:80}
\begin{aligned}
\psi &=
\alpha \cosh\left( p_2 (x-a)/\Hbar \right)
+ i \alpha \frac{p_1}{p_2} \sinh\left( p_2 (x-a)/\Hbar \right)
\end{aligned}
\end{equation}

For the \(x<0\) region assume again a solution of the form

\begin{equation}\label{eqn:qmbFirstTry:100}
\begin{aligned}
\psi = u e^{i p_1 x/\Hbar} +v e^{-i p_1 x/\Hbar}
\end{aligned}
\end{equation}

Matching at \(x=0\) we have

\begin{equation}\label{eqn:qmbFirstTry:120}
\begin{aligned}
u + v &= \alpha \cosh( p_2 a/\Hbar ) - i \alpha \frac{p_1}{p_2} \sinh( p_2 a /\Hbar ) \\
u - v &= i \alpha \frac{p_2}{p_1} \sinh( p_2 a/\Hbar ) + \alpha \cosh( p_2 a /\Hbar ) \\
\end{aligned}
\end{equation}

Solving
\begin{equation}\label{eqn:qmb_first_try:uAndv}
\begin{aligned}
u &= \alpha \cosh( p_2 a/\Hbar )  + \frac{i\alpha}{2} \sinh( p_2 a /\Hbar ) \left( \frac{p_2}{p_1} -  \frac{p_1}{p_2} \right) \\
v &= - \frac{i\alpha}{2} \sinh( p_2 a /\Hbar ) \left( \frac{p_2}{p_1} + \frac{p_1}{p_2} \right)
\end{aligned}
\end{equation}

\subsubsection{Currents}

Current past the barrier is simply calculated

\begin{equation}\label{eqn:qmb_first_try:JtBarrier}
\begin{aligned}
J_t
&= \frac{p_1}{m} \Abs{\alpha}^2
\end{aligned}
\end{equation}

Which we can write directly in terms of the probability density
\begin{equation}\label{eqn:qmbFirstTry:140}
\begin{aligned}
J_t
&= \frac{p_1}{m} \rho
\end{aligned}
\end{equation}

So, again, the probability current is a velocity scaling of the probably density.

In the barrier region, writing
\(C = \cosh(p_2(x-a)/\Hbar)\), and
\(S = \sinh(p_2(x-a)/\Hbar)\), we have

\begin{equation}\label{eqn:qmbFirstTry:160}
\begin{aligned}
J &=
\frac{\Abs{\alpha}^2}{ 2 m i }
\left(
\left(C - i \frac{p_1}{p_2}S\right)
\left(p_2 S + i p_1 C\right)
-
\left(C + i \frac{p_1}{p_2}S\right)
\left(p_2 S - i p_1 C\right)
\right)
\end{aligned}
\end{equation}

Reducing these products we get
\begin{equation}\label{eqn:qmbFirstTry:180}
\begin{aligned}
J
&=
\frac{p_1 \Abs{\alpha}^2}{m} (C^2 - S^2) \\
&=
\frac{p_1 \Abs{\alpha}^2}{m}
\end{aligned}
\end{equation}

So the current equals that in the region past the barrier.  This
equality makes me suspect an error in the treatment of problem 1.

Note however
that the probability density in this region is

\begin{equation}\label{eqn:qmbFirstTry:200}
\begin{aligned}
\rho
&= \Abs{\alpha}^2 \left(
C^2 + S^2 \left(\frac{p_1}{p_2}\right)^2
\right)
\end{aligned}
\end{equation}

So, we have

\begin{equation}\label{eqn:qmbFirstTry:220}
\begin{aligned}
J
&=
\frac{p_1}{m} \rho \inv{C^2 + {\left(\frac{p_1}{p_2}\right)}^2 S^2 } \\
&=
\frac{p_1}{m} \rho \inv{
\cosh^2(p_2(x-a)/\Hbar)
 +
 {\left(\frac{p_1}{p_2}\right)}^2
\sinh^2(p_2(x-a)/\Hbar)
 } \\
\end{aligned}
\end{equation}

There is a spatially dependency between the current density and the probability density that we do not have past the barrier region.  The current density
can still
be thought of as a velocity scaling of the probability density, but
we have a position dependent velocity inside the barrier

\begin{equation}\label{eqn:qmbFirstTry:240}
\begin{aligned}
v_b = \frac{p_1}{m} \inv{
\cosh^2(p_2(x-a)/\Hbar)
 +
 {\left(\frac{p_1}{p_2}\right)}^2
\sinh^2(p_2(x-a)/\Hbar)
 } \\
\end{aligned}
\end{equation}

Here the optical analogy seems appropriate, and the barrier with its sharp
boundaries ends up manifesting like a material with varying index of
refraction.  Some of that variation of index of refraction (related by Snell's
law to the velocity of the material) appears to be impacted by the shape of
the boundary, and loosely it seems like a sharp potential transition acts
as an impedance.  TODO: do some math to back this up.  Am kind of guessing
here.

SNIP.  MOVED to \(qm_barrier.ltx\)

We are now in shape to substitute \eqnref{eqn:qmb_first_try:uAndv} into these.  For short we
write

\begin{equation}\label{eqn:qmbFirstTry:260}
\begin{aligned}
u
&= \alpha \cosh( p_2 a/\Hbar )  + \frac{i\alpha}{2} \sinh( p_2 a /\Hbar ) \left( \frac{p_2}{p_1} -  \frac{p_1}{p_2} \right) \\
&= \alpha C + \frac{i\alpha}{2} S \left( p_{21} - p_{12} \right) \\
&= \alpha C + \frac{i\alpha}{2} S \delta p
\end{aligned}
\end{equation}

and
\begin{equation}\label{eqn:qmbFirstTry:280}
\begin{aligned}
v
&= - \frac{i\alpha}{2} \sinh( p_2 a /\Hbar ) \left( \frac{p_2}{p_1} + \frac{p_1}{p_2} \right) \\
&= \frac{i\alpha}{2} S \left( p_{12} - p_{21} \right) \\
&= \frac{i\alpha}{2} S \delta p
\end{aligned}
\end{equation}

Absolute squares of these are
\begin{equation}\label{eqn:qmbFirstTry:300}
\begin{aligned}
\Abs{u}^2 &= \Abs{\alpha}^2 (C^2 + S^2 (\delta p)^2/4)
\end{aligned}
\end{equation}

and
\begin{equation}\label{eqn:qmbFirstTry:320}
\begin{aligned}
\Abs{v}^2 &= \Abs{\alpha}^2 S^2 (\delta p)^2/4
\end{aligned}
\end{equation}

This gives a total incident plus reflected current in the \(x<0\) region as

\begin{equation}\label{eqn:qmb_first_try:jTotBarrier}
\begin{aligned}
J
&=
\frac{p_1}{m} \Abs{\alpha}^2 \cosh^2(p_2 a/\Hbar)
\end{aligned}
\end{equation}

For the probability density, we need

\begin{equation}\label{eqn:qmbFirstTry:340}
\begin{aligned}
\Abs{u}^2 + \Abs{v}^2 + 2 \Re(v^\conj u \epsilon^2)
&=
\Abs{\alpha}^2 \left(C^2 + S^2 (\delta p)^2/2
+ 2\Re\left(
-\frac{i}{2} S (\delta p) \left(C + \frac{i}{2} S (\delta p)\right) e^{i p_1 x/\Hbar}
\right)\right) \\
&=
\Abs{\alpha}^2 \left(C^2 + S^2 (\delta p)^2/2
+ S C (\delta p) \sin( 2 p_1 x/\Hbar)
+\frac{1}{2} S^2 (\delta p)^2 \cos( 2 p_1 x/\Hbar)
\right) \\
&=
\Abs{\alpha}^2 \left(C^2 + S^2 (\delta p)^2 \inv{2}(1 + \cos( 2 p_1 x/\Hbar))
+ S C (\delta p) \sin( 2 p_1 x/\Hbar)
\right) \\
&=
\Abs{\alpha}^2 \left(C^2 + S^2 (\delta p)^2 \cos^2( p_1 x/\Hbar)
+ S C (\delta p) \sin( 2 p_1 x/\Hbar)
\right) \\
\end{aligned}
\end{equation}

So the probability density for \(x<0\) is
\begin{equation}\label{eqn:qmbFirstTry:360}
\begin{aligned}
\rho
&=
\Abs{\alpha}^2 \left(\cosh^2(p_2 a/\Hbar) + \sinh^2(p_2 a/\Hbar) (\delta p)^2 \cos^2( p_1 x/\Hbar)
%+ \sinh(p_2 a/\Hbar)\cosh(p_2 a/\Hbar) (\delta p) \sin( 2 p_1 x/\Hbar)
+ \inv{2} \sinh(2 p_2 a/\Hbar) (\delta p) \sin( 2 p_1 x/\Hbar)
\right)
\end{aligned}
\end{equation}

The last two currents of interest to calculate in the \(x<0\) region are the incident and reflected probability currents.  The
incident current is

\begin{equation}\label{eqn:qmbFirstTry:380}
\begin{aligned}
J_i
&=
\frac{\Hbar}{2m i}\left(
u^\conj \epsilon^\conj (ip_1/\Hbar) u \epsilon
-u \epsilon (-ip_1/\Hbar) u^\conj \epsilon^\conj
\right)
\\
&=
\frac{p_1}{m} \Abs{u}^2
\end{aligned}
\end{equation}

and the reflected current is
\begin{equation}\label{eqn:qmbFirstTry:400}
\begin{aligned}
J_r
&=
\frac{\Hbar}{2m i}\left(
v^\conj \epsilon (-ip_1/\Hbar) v^\conj \epsilon^\conj
-v \epsilon^\conj (ip_1/\Hbar) v \epsilon
\right) \\
&=
\frac{-p_1}{m} \Abs{v}^2
\end{aligned}
\end{equation}

So we have

\begin{equation}\label{eqn:qmb_first_try:JiBarrier}
\begin{aligned}
J_i
&= \frac{p_1}{m} \Abs{\alpha}^2 (\cosh^2(p_2 a/\Hbar) + \sinh^2(p_2 a/\Hbar) (\delta p)^2/4)
\end{aligned}
\end{equation}

\begin{equation}\label{eqn:qmb_first_try:JrBarrier}
\begin{aligned}
J_r
&= \frac{-p_1}{m} \Abs{\alpha}^2 \sinh^2(p_2 a/\Hbar) (\delta p)^2/4
\end{aligned}
\end{equation}

\subsubsection{Transmission and reflection coefficients}

From equations \eqnref{eqn:qmb_first_try:JtBarrier} \eqnref{eqn:qmb_first_try:JiBarrier} \eqnref{eqn:qmb_first_try:JrBarrier}, the transmission and reflection coefficients can be calculated.
These are

\begin{equation}\label{eqn:qmbFirstTry:420}
\begin{aligned}
R &=
\frac{\Abs{J_r}} {\Abs{J_i}} \\
&=
\frac{\sinh^2(p_2 a/\Hbar) (\delta p)^2/4}{ \cosh^2(p_2 a/\Hbar) + \sinh^2(p_2 a/\Hbar) (\delta p)^2/4 }
\end{aligned}
\end{equation}

and

\begin{equation}\label{eqn:qmbFirstTry:440}
\begin{aligned}
T &=
\frac{\Abs{J_t}} {\Abs{J_i}} \\
&=
\frac{1}{ \cosh^2(p_2 a/\Hbar) + \sinh^2(p_2 a/\Hbar) (\delta p)^2/4 }
\end{aligned}
\end{equation}

Again, these do not sum to unity as expected.  I either misunderstand something or am making mistakes.  Review this all, and compare to the approximate treatment in the book (for the case \(p_2 a/\Hbar \gg 1\)).

A peek in \citep{mcmahon2005qmd} has got a result similar to this
but with \(1/2\) instead of a \(\cosh\) term in the denominator for \(T\).  That
said there are so many obvious typos in that treatment it is hard
to trust it.  Review
can probably focus on my own, but looking for an error in \(J_i\) or \(\psi_i\).

%\bibliographystyle{plainnat}
%\bibliography{myrefs}

%\end{document}

\documentclass{article}

\usepackage{amsmath}
\usepackage{mathpazo}

%
% shorthand for bold symbols, convenient for vectors and matrices
%
\newcommand{\Ba}[0]{\mathbf{a}}
\newcommand{\Bb}[0]{\mathbf{b}}
\newcommand{\Bc}[0]{\mathbf{c}}
\newcommand{\Bd}[0]{\mathbf{d}}
\newcommand{\Be}[0]{\mathbf{e}}
\newcommand{\Bf}[0]{\mathbf{f}}
\newcommand{\Bg}[0]{\mathbf{g}}
\newcommand{\Bh}[0]{\mathbf{h}}
\newcommand{\Bi}[0]{\mathbf{i}}
\newcommand{\Bj}[0]{\mathbf{j}}
\newcommand{\Bk}[0]{\mathbf{k}}
\newcommand{\Bl}[0]{\mathbf{l}}
\newcommand{\Bm}[0]{\mathbf{m}}
\newcommand{\Bn}[0]{\mathbf{n}}
\newcommand{\Bo}[0]{\mathbf{o}}
\newcommand{\Bp}[0]{\mathbf{p}}
\newcommand{\Bq}[0]{\mathbf{q}}
\newcommand{\Br}[0]{\mathbf{r}}
\newcommand{\Bs}[0]{\mathbf{s}}
\newcommand{\Bt}[0]{\mathbf{t}}
\newcommand{\Bu}[0]{\mathbf{u}}
\newcommand{\Bv}[0]{\mathbf{v}}
\newcommand{\Bw}[0]{\mathbf{w}}
\newcommand{\Bx}[0]{\mathbf{x}}
\newcommand{\By}[0]{\mathbf{y}}
\newcommand{\Bz}[0]{\mathbf{z}}
\newcommand{\BA}[0]{\mathbf{A}}
\newcommand{\BB}[0]{\mathbf{B}}
\newcommand{\BC}[0]{\mathbf{C}}
\newcommand{\BD}[0]{\mathbf{D}}
\newcommand{\BE}[0]{\mathbf{E}}
\newcommand{\BF}[0]{\mathbf{F}}
\newcommand{\BG}[0]{\mathbf{G}}
\newcommand{\BH}[0]{\mathbf{H}}
\newcommand{\BI}[0]{\mathbf{I}}
\newcommand{\BJ}[0]{\mathbf{J}}
\newcommand{\BK}[0]{\mathbf{K}}
\newcommand{\BL}[0]{\mathbf{L}}
\newcommand{\BM}[0]{\mathbf{M}}
\newcommand{\BN}[0]{\mathbf{N}}
\newcommand{\BO}[0]{\mathbf{O}}
\newcommand{\BP}[0]{\mathbf{P}}
\newcommand{\BQ}[0]{\mathbf{Q}}
\newcommand{\BR}[0]{\mathbf{R}}
\newcommand{\BS}[0]{\mathbf{S}}
\newcommand{\BT}[0]{\mathbf{T}}
\newcommand{\BU}[0]{\mathbf{U}}
\newcommand{\BV}[0]{\mathbf{V}}
\newcommand{\BW}[0]{\mathbf{W}}
\newcommand{\BX}[0]{\mathbf{X}}
\newcommand{\BY}[0]{\mathbf{Y}}
\newcommand{\BZ}[0]{\mathbf{Z}}

\newcommand{\Bzero}[0]{\mathbf{0}}
\newcommand{\Btheta}[0]{\boldsymbol{\theta}}
\newcommand{\Btau}[0]{\boldsymbol{\tau}}
\newcommand{\Bomega}[0]{\boldsymbol{\omega}}

%
% shorthand for unit vectors
%
\newcommand{\acap}[0]{\hat{\Ba}}
\newcommand{\bcap}[0]{\hat{\Bb}}
\newcommand{\ccap}[0]{\hat{\Bc}}
\newcommand{\dcap}[0]{\hat{\Bd}}
\newcommand{\ecap}[0]{\hat{\Be}}
\newcommand{\fcap}[0]{\hat{\Bf}}
\newcommand{\gcap}[0]{\hat{\Bg}}
\newcommand{\hcap}[0]{\hat{\Bh}}
\newcommand{\icap}[0]{\hat{\Bi}}
\newcommand{\jcap}[0]{\hat{\Bj}}
\newcommand{\kcap}[0]{\hat{\Bk}}
\newcommand{\lcap}[0]{\hat{\Bl}}
\newcommand{\mcap}[0]{\hat{\Bm}}
\newcommand{\ncap}[0]{\hat{\Bn}}
\newcommand{\ocap}[0]{\hat{\Bo}}
\newcommand{\pcap}[0]{\hat{\Bp}}
\newcommand{\qcap}[0]{\hat{\Bq}}
\newcommand{\rcap}[0]{\hat{\Br}}
\newcommand{\scap}[0]{\hat{\Bs}}
\newcommand{\tcap}[0]{\hat{\Bt}}
\newcommand{\ucap}[0]{\hat{\Bu}}
\newcommand{\vcap}[0]{\hat{\Bv}}
\newcommand{\wcap}[0]{\hat{\Bw}}
\newcommand{\xcap}[0]{\hat{\Bx}}
\newcommand{\ycap}[0]{\hat{\By}}
\newcommand{\zcap}[0]{\hat{\Bz}}
\newcommand{\thetacap}[0]{\hat{\Btheta}}

%
% to write R^n and C^n in a distinguishable fashion.  Perhaps change this
% to the double lined characters upon figuring out how to do so.
%
\newcommand{\C}[1]{$\mathbb{C}^{#1}$}
\newcommand{\R}[1]{$\mathbb{R}^{#1}$}

%
% various generally useful helpers
%

% derivative of #1 wrt. #2:
\newcommand{\D}[2] {\frac {d#2} {d#1}}

\newcommand{\inv}[1]{\frac{1}{#1}}
\newcommand{\cross}[0]{\times}

\newcommand{\abs}[1]{\lvert{#1}\rvert}
\newcommand{\norm}[1]{\lVert{#1}\rVert}
\newcommand{\innerprod}[2]{\langle{#1}, {#2}\rangle}
\newcommand{\dotprod}[2]{{#1} \cdot {#2}}
\newcommand{\bdotprod}[2]{\left({#1} \cdot {#2}\right)}
\newcommand{\crossprod}[2]{{#1} \cross {#2}}
\newcommand{\tripleprod}[3]{\dotprod{\left(\crossprod{#1}{#2}\right)}{#3}}

\DeclareMathOperator{\Proj}{Proj}
\DeclareMathOperator{\Span}{span}
\DeclareMathOperator{\Sgn}{sgn}
\DeclareMathOperator{\Area}{Area}
\DeclareMathOperator{\Volume}{Volume}

%
% A few miscellaneous things specific to this document
%
\newcommand{\crossop}[1]{\crossprod{#1}{}}

% R2 vector.
\newcommand{\VectorTwo}[2]{
\begin{bmatrix}
 {#1} \\
 {#2}
\end{bmatrix}
}

\newcommand{\VectorN}[1]{
\begin{bmatrix}
{#1}_1 \\
{#1}_2 \\
\vdots \\
{#1}_N \\
\end{bmatrix}
}

\newcommand{\DETuvij}[4]{
\begin{vmatrix}
 {#1}_{#3} & {#1}_{#4} \\
 {#2}_{#3} & {#2}_{#4}
\end{vmatrix}
}

\newcommand{\DETuvwijk}[6]{
\begin{vmatrix}
 {#1}_{#4} & {#1}_{#5} & {#1}_{#6} \\
 {#2}_{#4} & {#2}_{#5} & {#2}_{#6} \\
 {#3}_{#4} & {#3}_{#5} & {#3}_{#6}
\end{vmatrix}
}

\newcommand{\DETuvwxijkl}[8]{
\begin{vmatrix}
 {#1}_{#5} & {#1}_{#6} & {#1}_{#7} & {#1}_{#8} \\
 {#2}_{#5} & {#2}_{#6} & {#2}_{#7} & {#2}_{#8} \\
 {#3}_{#5} & {#3}_{#6} & {#3}_{#7} & {#3}_{#8} \\
 {#4}_{#5} & {#4}_{#6} & {#4}_{#7} & {#4}_{#8} \\
\end{vmatrix}
}

%\newcommand{\DETuvwxyijklm}[10]{
%\begin{vmatrix}
% {#1}_{#6} & {#1}_{#7} & {#1}_{#8} & {#1}_{#9} & {#1}_{#10} \\
% {#2}_{#6} & {#2}_{#7} & {#2}_{#8} & {#2}_{#9} & {#2}_{#10} \\
% {#3}_{#6} & {#3}_{#7} & {#3}_{#8} & {#3}_{#9} & {#3}_{#10} \\
% {#4}_{#6} & {#4}_{#7} & {#4}_{#8} & {#4}_{#9} & {#4}_{#10} \\
% {#5}_{#6} & {#5}_{#7} & {#5}_{#8} & {#5}_{#9} & {#5}_{#10}
%\end{vmatrix}
%}

% R3 vector.
\newcommand{\VectorThree}[3]{
\begin{bmatrix}
 {#1} \\
 {#2} \\
 {#3}
\end{bmatrix}
}


%<misc>
%
\newcommand{\Abs}[1]{{\left\lvert{#1}\right\rvert}}
\newcommand{\spacegrad}[0]{\boldsymbol{\nabla}}
\newcommand{\grad}[0]{\nabla}
\newcommand{\LL}[0]{\mathcal{L}}

% == \partial_{#1} {#2}
\newcommand{\PD}[2]{\frac{\partial {#2}}{\partial {#1}}}
% inline variant
\newcommand{\PDi}[2]{{\partial {#2}}/{\partial {#1}}}

\newcommand{\PDD}[3]{\frac{\partial^2 {#3}}{\partial {#1}\partial {#2}}}
%\newcommand{\PDd}[2]{\frac{\partial^2 {#2}}{{\partial{#1}}^2}}
\newcommand{\PDsq}[2]{\frac{\partial^2 {#2}}{(\partial {#1})^2}}

\newcommand{\Partial}[2]{\frac{\partial {#1}}{\partial {#2}}}
\DeclareMathOperator{\RejName}{Rej}
\newcommand{\Rej}[2]{\RejName_{#1}\left( {#2} \right)}
\newcommand{\Rm}[1]{\mathbb{R}^{#1}}
\newcommand{\Cm}[1]{\mathbb{C}^{#1}}
\newcommand{\conj}[0]{{*}}

%</misc>

% <grade selection>
%
\newcommand{\gpgrade}[2] {{\left\langle{{#1}}\right\rangle}_{#2}}

\newcommand{\gpgradezero}[1] {\gpgrade{#1}{}}
%\newcommand{\gpscalargrade}[1] {{\left\langle{{#1}}\right\rangle}}
%\newcommand{\gpgradezero}[1] {\gpgrade{#1}{0}}

%\newcommand{\gpgradeone}[1] {{\left\langle{{#1}}\right\rangle}_{1}}
\newcommand{\gpgradeone}[1] {\gpgrade{#1}{1}}

\newcommand{\gpgradetwo}[1] {\gpgrade{#1}{2}}
\newcommand{\gpgradethree}[1] {\gpgrade{#1}{3}}
\newcommand{\gpgradefour}[1] {\gpgrade{#1}{4}}
%
% </grade selection>



\newcommand{\adot}[0]{{\dot{a}}}
\newcommand{\bdot}[0]{{\dot{b}}}
% taken for centered dot:
%\newcommand{\cdot}[0]{{\dot{c}}}
%\newcommand{\ddot}[0]{{\dot{d}}}
\newcommand{\edot}[0]{{\dot{e}}}
\newcommand{\fdot}[0]{{\dot{f}}}
\newcommand{\gdot}[0]{{\dot{g}}}
\newcommand{\hdot}[0]{{\dot{h}}}
\newcommand{\idot}[0]{{\dot{i}}}
\newcommand{\jdot}[0]{{\dot{j}}}
\newcommand{\kdot}[0]{{\dot{k}}}
\newcommand{\ldot}[0]{{\dot{l}}}
\newcommand{\mdot}[0]{{\dot{m}}}
\newcommand{\ndot}[0]{{\dot{n}}}
%\newcommand{\odot}[0]{{\dot{o}}}
\newcommand{\pdot}[0]{{\dot{p}}}
\newcommand{\qdot}[0]{{\dot{q}}}
\newcommand{\rdot}[0]{{\dot{r}}}
\newcommand{\sdot}[0]{{\dot{s}}}
\newcommand{\tdot}[0]{{\dot{t}}}
\newcommand{\udot}[0]{{\dot{u}}}
\newcommand{\vdot}[0]{{\dot{v}}}
\newcommand{\wdot}[0]{{\dot{w}}}
\newcommand{\xdot}[0]{{\dot{x}}}
\newcommand{\ydot}[0]{{\dot{y}}}
\newcommand{\zdot}[0]{{\dot{z}}}
\newcommand{\addot}[0]{{\ddot{a}}}
\newcommand{\bddot}[0]{{\ddot{b}}}
\newcommand{\cddot}[0]{{\ddot{c}}}
%\newcommand{\dddot}[0]{{\ddot{d}}}
\newcommand{\eddot}[0]{{\ddot{e}}}
\newcommand{\fddot}[0]{{\ddot{f}}}
\newcommand{\gddot}[0]{{\ddot{g}}}
\newcommand{\hddot}[0]{{\ddot{h}}}
\newcommand{\iddot}[0]{{\ddot{i}}}
\newcommand{\jddot}[0]{{\ddot{j}}}
\newcommand{\kddot}[0]{{\ddot{k}}}
\newcommand{\lddot}[0]{{\ddot{l}}}
\newcommand{\mddot}[0]{{\ddot{m}}}
\newcommand{\nddot}[0]{{\ddot{n}}}
\newcommand{\oddot}[0]{{\ddot{o}}}
\newcommand{\pddot}[0]{{\ddot{p}}}
\newcommand{\qddot}[0]{{\ddot{q}}}
\newcommand{\rddot}[0]{{\ddot{r}}}
\newcommand{\sddot}[0]{{\ddot{s}}}
\newcommand{\tddot}[0]{{\ddot{t}}}
\newcommand{\uddot}[0]{{\ddot{u}}}
\newcommand{\vddot}[0]{{\ddot{v}}}
\newcommand{\wddot}[0]{{\ddot{w}}}
\newcommand{\xddot}[0]{{\ddot{x}}}
\newcommand{\yddot}[0]{{\ddot{y}}}
\newcommand{\zddot}[0]{{\ddot{z}}}

%<bold and dot greek symbols>
%

\newcommand{\Deltadot}[0]{{\dot{\Delta}}}
\newcommand{\Gammadot}[0]{{\dot{\Gamma}}}
\newcommand{\Lambdadot}[0]{{\dot{\Lambda}}}
\newcommand{\Omegadot}[0]{{\dot{\Omega}}}
\newcommand{\Phidot}[0]{{\dot{\Phi}}}
\newcommand{\Pidot}[0]{{\dot{\Pi}}}
\newcommand{\Psidot}[0]{{\dot{\Psi}}}
\newcommand{\Sigmadot}[0]{{\dot{\Sigma}}}
\newcommand{\Thetadot}[0]{{\dot{\Theta}}}
\newcommand{\Upsilondot}[0]{{\dot{\Upsilon}}}
\newcommand{\Xidot}[0]{{\dot{\Xi}}}
\newcommand{\alphadot}[0]{{\dot{\alpha}}}
\newcommand{\betadot}[0]{{\dot{\beta}}}
\newcommand{\chidot}[0]{{\dot{\chi}}}
\newcommand{\deltadot}[0]{{\dot{\delta}}}
\newcommand{\epsilondot}[0]{{\dot{\epsilon}}}
\newcommand{\etadot}[0]{{\dot{\eta}}}
\newcommand{\gammadot}[0]{{\dot{\gamma}}}
\newcommand{\kappadot}[0]{{\dot{\kappa}}}
\newcommand{\lambdadot}[0]{{\dot{\lambda}}}
\newcommand{\mudot}[0]{{\dot{\mu}}}
\newcommand{\nudot}[0]{{\dot{\nu}}}
\newcommand{\omegadot}[0]{{\dot{\omega}}}
\newcommand{\phidot}[0]{{\dot{\phi}}}
\newcommand{\pidot}[0]{{\dot{\pi}}}
\newcommand{\psidot}[0]{{\dot{\psi}}}
\newcommand{\rhodot}[0]{{\dot{\rho}}}
\newcommand{\sigmadot}[0]{{\dot{\sigma}}}
\newcommand{\taudot}[0]{{\dot{\tau}}}
\newcommand{\thetadot}[0]{{\dot{\theta}}}
\newcommand{\upsilondot}[0]{{\dot{\upsilon}}}
\newcommand{\varepsilondot}[0]{{\dot{\varepsilon}}}
\newcommand{\varphidot}[0]{{\dot{\varphi}}}
\newcommand{\varpidot}[0]{{\dot{\varpi}}}
\newcommand{\varrhodot}[0]{{\dot{\varrho}}}
\newcommand{\varsigmadot}[0]{{\dot{\varsigma}}}
\newcommand{\varthetadot}[0]{{\dot{\vartheta}}}
\newcommand{\xidot}[0]{{\dot{\xi}}}
\newcommand{\zetadot}[0]{{\dot{\zeta}}}

\newcommand{\Deltaddot}[0]{{\ddot{\Delta}}}
\newcommand{\Gammaddot}[0]{{\ddot{\Gamma}}}
\newcommand{\Lambdaddot}[0]{{\ddot{\Lambda}}}
\newcommand{\Omegaddot}[0]{{\ddot{\Omega}}}
\newcommand{\Phiddot}[0]{{\ddot{\Phi}}}
\newcommand{\Piddot}[0]{{\ddot{\Pi}}}
\newcommand{\Psiddot}[0]{{\ddot{\Psi}}}
\newcommand{\Sigmaddot}[0]{{\ddot{\Sigma}}}
\newcommand{\Thetaddot}[0]{{\ddot{\Theta}}}
\newcommand{\Upsilonddot}[0]{{\ddot{\Upsilon}}}
\newcommand{\Xiddot}[0]{{\ddot{\Xi}}}
\newcommand{\alphaddot}[0]{{\ddot{\alpha}}}
\newcommand{\betaddot}[0]{{\ddot{\beta}}}
\newcommand{\chiddot}[0]{{\ddot{\chi}}}
\newcommand{\deltaddot}[0]{{\ddot{\delta}}}
\newcommand{\epsilonddot}[0]{{\ddot{\epsilon}}}
\newcommand{\etaddot}[0]{{\ddot{\eta}}}
\newcommand{\gammaddot}[0]{{\ddot{\gamma}}}
\newcommand{\kappaddot}[0]{{\ddot{\kappa}}}
\newcommand{\lambdaddot}[0]{{\ddot{\lambda}}}
\newcommand{\muddot}[0]{{\ddot{\mu}}}
\newcommand{\nuddot}[0]{{\ddot{\nu}}}
\newcommand{\omegaddot}[0]{{\ddot{\omega}}}
\newcommand{\phiddot}[0]{{\ddot{\phi}}}
\newcommand{\piddot}[0]{{\ddot{\pi}}}
\newcommand{\psiddot}[0]{{\ddot{\psi}}}
\newcommand{\rhoddot}[0]{{\ddot{\rho}}}
\newcommand{\sigmaddot}[0]{{\ddot{\sigma}}}
\newcommand{\tauddot}[0]{{\ddot{\tau}}}
\newcommand{\thetaddot}[0]{{\ddot{\theta}}}
\newcommand{\upsilonddot}[0]{{\ddot{\upsilon}}}
\newcommand{\varepsilonddot}[0]{{\ddot{\varepsilon}}}
\newcommand{\varphiddot}[0]{{\ddot{\varphi}}}
\newcommand{\varpiddot}[0]{{\ddot{\varpi}}}
\newcommand{\varrhoddot}[0]{{\ddot{\varrho}}}
\newcommand{\varsigmaddot}[0]{{\ddot{\varsigma}}}
\newcommand{\varthetaddot}[0]{{\ddot{\vartheta}}}
\newcommand{\xiddot}[0]{{\ddot{\xi}}}
\newcommand{\zetaddot}[0]{{\ddot{\zeta}}}

\newcommand{\BDelta}[0]{\boldsymbol{\Delta}}
\newcommand{\BGamma}[0]{\boldsymbol{\Gamma}}
\newcommand{\BLambda}[0]{\boldsymbol{\Lambda}}
\newcommand{\BOmega}[0]{\boldsymbol{\Omega}}
\newcommand{\BPhi}[0]{\boldsymbol{\Phi}}
\newcommand{\BPi}[0]{\boldsymbol{\Pi}}
\newcommand{\BPsi}[0]{\boldsymbol{\Psi}}
\newcommand{\BSigma}[0]{\boldsymbol{\Sigma}}
\newcommand{\BTheta}[0]{\boldsymbol{\Theta}}
\newcommand{\BUpsilon}[0]{\boldsymbol{\Upsilon}}
\newcommand{\BXi}[0]{\boldsymbol{\Xi}}
\newcommand{\Balpha}[0]{\boldsymbol{\alpha}}
\newcommand{\Bbeta}[0]{\boldsymbol{\beta}}
\newcommand{\Bchi}[0]{\boldsymbol{\chi}}
\newcommand{\Bdelta}[0]{\boldsymbol{\delta}}
\newcommand{\Bepsilon}[0]{\boldsymbol{\epsilon}}
\newcommand{\Beta}[0]{\boldsymbol{\eta}}
\newcommand{\Bgamma}[0]{\boldsymbol{\gamma}}
\newcommand{\Bkappa}[0]{\boldsymbol{\kappa}}
\newcommand{\Blambda}[0]{\boldsymbol{\lambda}}
\newcommand{\Bmu}[0]{\boldsymbol{\mu}}
\newcommand{\Bnu}[0]{\boldsymbol{\nu}}
%\newcommand{\Bomega}[0]{\boldsymbol{\omega}}
\newcommand{\Bphi}[0]{\boldsymbol{\phi}}
\newcommand{\Bpi}[0]{\boldsymbol{\pi}}
\newcommand{\Bpsi}[0]{\boldsymbol{\psi}}
\newcommand{\Brho}[0]{\boldsymbol{\rho}}
\newcommand{\Bsigma}[0]{\boldsymbol{\sigma}}
%\newcommand{\Btau}[0]{\boldsymbol{\tau}}
%\newcommand{\Btheta}[0]{\boldsymbol{\theta}}
\newcommand{\Bupsilon}[0]{\boldsymbol{\upsilon}}
\newcommand{\Bvarepsilon}[0]{\boldsymbol{\varepsilon}}
\newcommand{\Bvarphi}[0]{\boldsymbol{\varphi}}
\newcommand{\Bvarpi}[0]{\boldsymbol{\varpi}}
\newcommand{\Bvarrho}[0]{\boldsymbol{\varrho}}
\newcommand{\Bvarsigma}[0]{\boldsymbol{\varsigma}}
\newcommand{\Bvartheta}[0]{\boldsymbol{\vartheta}}
\newcommand{\Bxi}[0]{\boldsymbol{\xi}}
\newcommand{\Bzeta}[0]{\boldsymbol{\zeta}}
%
%</bold and dot greek symbols>
%<infrequent>
%
%\newcommand{\AreaOp}[1]{\AName_{#1}}
%\newcommand{\Babs}[0]{\abs{\BB}}
%\newcommand{\Bcap}[0]{\hat{\BB}}
%\newcommand{\BrPrimeRej}[0]{\rcap(\rcap \wedge \Br')}
%\newcommand{\CA}[0]{\mathcal{A}}
%\newcommand{\Cos}[1]{\cos{\left({#1}\right)}}
%\newcommand{\Det}[1] {\abs{#1}}
%\newcommand{\Dsq}[2] {\frac {\partial^2 {#1}} {\partial {#2}^2}}
%\newcommand{\Exp}[1]{\exp{\left({#1}\right)}}
%\newcommand{\Norm}[1]{\left\lVert{#1}\right\rVert}
%\newcommand{\Sin}[1]{\sin{\left({#1}\right)}}
%\newcommand{\T}[0]{\text{T}}
%\newcommand{\VolumeOp}[1]{\VName_{#1}}
%\newcommand{\agrad}[0]{\Ba \cdot \nabla}
%\newcommand{\alphacap}[0]{\hat{\boldsymbol{\alpha}}}
%\newcommand{\Fcap}[0]{\hat{\BF}}
%\newcommand{\bithree}[0]{{\Bi}_3}
%\newcommand{\bxa}[0]{\Bx\Ba}
%\newcommand{\coordvec}[2]{
%\newcommand{\costheta}[0]{\acap \cdot \xcap}
%\newcommand{\ddt}[1]{\ddot{#1}}
%\newcommand{\ddu}[1] {\frac {d{#1}} {du}}
%\newcommand{\dsqxj}[2] {\frac {\partial^2 {#1}} {\partial {x_{#2}}^2}}
%\newcommand{\dtheta}[1]{\frac{d {#1}}{d \theta}}
%\newcommand{\dt}[1]{\dot{#1}}
%\newcommand{\dt}[1]{\frac{d {#1}}{dt}}
%\newcommand{\dxj}[2] {\frac {\partial {#1}} {\partial {x_{#2}}}}
%\newcommand{\halfPhi}[0]{\frac{\phi}{2}}
%\newcommand{\half}[0]{\inv{2}}
%\newcommand{\inv}[1]{\frac{1}{#1}}
%\newcommand{\laplacian}[0]{\nabla^2}
%\newcommand{\matrixoftx}[3]{
%\newcommand{\nrrp}[0]{\norm{\rcap \wedge \Br'}}
%\newcommand{\oiint}{\bigcirc \hspace{-1.4em} \int \hspace{-.8em} \int}
%\newcommand{\transpose}[1]{{#1}^{\text{T}}}
%\newcommand{\transpose}[1]{{{#1}^{\TextTranspose}}}
%\newcommand{\transpose}[1]{{{#1}^{\text{T}}}}
%\newcommand{\barA}[0]{\bar{A}}
%\newcommand{\qbar}[0]{\bar{q}}
%\newcommand{\qdotbar}[0]{\dot{\bar{q}}}
%
%</infrequent>





\usepackage[bookmarks=true]{hyperref}

\usepackage{color,cite,graphicx}
   % use colour in the document, put your citations as [1-4]
   % rather than [1,2,3,4] (it looks nicer, and the extended LaTeX2e
   % graphics package. 
\usepackage{latexsym,amssymb,epsf} % don't remember if these are
   % needed, but their inclusion can't do any damage


\title{ Chapter 2 Quiz solutions for QM Demystified book. }
\author{Peeter Joot}
\date{ Jan 12, 2009.  Last Revision: $Date: 2009/01/13 04:41:51 $ }

\begin{document}

\maketitle{}

%\tableofcontents
\section{ Problem 1. Separation of variables. }

This problem was to show that separation of variables leads to an exponential energy/phase
term.  Let's try this, but do it instead for three dimensions and explore a bit.

We try a test solution of the form

\begin{align*}
\phi &= X(x) Y(y) Z(z) T(t) \\
\end{align*}

and substitute into 

\begin{align*}
\left(-\frac{\hbar^2}{2m}\grad^2 + V\right) \phi &= i \hbar \partial_t \phi
\end{align*}

differentiating and dividing by $\phi$ we have
\begin{align*}
-\frac{\hbar^2}{2m}
\left(
\frac{X''}{X}
+\frac{Y''}{Y}
+\frac{Z''}{Z}
\right)
 + V &= i \hbar \frac{T'}{T}
\end{align*}

We set the right hand side equal to a constant $E$ to be determined by boundary value conditions.
According to the dimensionals of $V$ this $E$ constant can be seen to neccessarily be an energy
of some sort.  In terms of this energy, we have for the function $T$

\begin{align*}
i \hbar \frac{T'}{T} &= E
\end{align*}

With a solution of
\begin{align*}
T(t) &= e^{-i E t/\hbar}
\end{align*}

Now, the left hand side imposes some constraints on $E$, but these will be potential dependent.
The simplest case, for the wave function of a free particle, is where $V=0$.

In that case we have
\begin{align*}
\frac{X''}{X} +\frac{Y''}{Y} +\frac{Z''}{Z} &= - \frac{2 m E}{\hbar^2}
\end{align*}

The sum of each of the terms involved all identically equal a constant, which is perhaps
reasonable to assume to be negative.  If we do so and impose the usual sort of separation of
variables constrait, requiring each of the $X''/X$, $Y''/Y$, and $Z''/Z$ terms to separately
equal some negative constant (to be fixed by boundary conditions), we can write

\begin{align*}
\frac{X''}{X} &= -{k_1}^2 \\
\frac{Y''}{Y} &= -{k_2}^2 \\
\frac{Z''}{Z} &= -{k_3}^2 \\
\end{align*}

So we have for the complete equation a solution proportional to

\begin{align*}
\phi = X Y Z T &= \exp(i(\Bk \cdot \Bx - E t/\hbar))
\end{align*}

With the additonal boundary value constraint of
\begin{align*}
\Bk^2 &= \frac{2 m E}{\hbar^2}
\end{align*}

\subsection{ constant potential variation. }

Let's consider a slightly more general potential with $V$ constant in some interval.

By inspection, in the interval we now have as the solution

\begin{align*}
\phi &= \exp(i(\Bk \cdot \Bx - (E - V) t/\hbar)) \\
\Bk^2 &= \frac{2 m (E - V)}{\hbar^2}
\end{align*}

Based on just the math, we don't know that $E$ is neccessarily positive, so in addition to the trigonometric solution above is it also reasonable to
allow for possible
hyperbolic solutions

\begin{align*}
\phi &= \exp(\Bk \cdot \Bx - i(E - V) t/\hbar) \\
\Bk^2 &= \frac{2 m (V - E)}{\hbar^2}
\end{align*}

We need some physics
to augment the math in order to determine what form of solution is actually valid.  Some of that physics likely comes in the form of the boundary conditions and perhaps other constraints such as normalization.

\subsection{ General solution. }

Using superposition we should be able to form a wave packet in integral form
by allowing for any set of $\Bk$ vectors.  Suppose we assemble a test solution by summing over possible wave numbers

\begin{align*}
\phi = \int A(\Bk) \exp(\alpha(\Bk \cdot \Bx) + i(E - V) t/\hbar) dk_1 dk_2 dk_3 \\
\end{align*}

FIXME: time term sign fudged to make the $V$ cancel out.  Must have a mistake above.

Does this work?  Let's take derivatives and see what constraints we require if it does.

\begin{align*}
\grad^2 \phi &= \alpha^2 \Bk^2 \phi \\
i \hbar \partial_t \phi &= -(E - V) \phi
\end{align*}

So we have
\begin{align*}
-\frac{\hbar^2}{2m} \alpha^2 \Bk^2 \phi + V\phi = -(E-V) \phi
\end{align*}

and the constraint required for the wave number $\Bk$ is 

\begin{align*}
\alpha^2 \Bk^2 = \frac{2 m E }{\hbar^2}
\end{align*}

So, for trig solutions (plane waves) where $\alpha \propto i$ our energy integration constant $E$ takes a negative value.
For hyperbolic solutions, where $\alpha = \pm 1$ (say), our energy has a positive value.

This screams for the boundary values of a 3D particle in a box problem, which for infinite potential outside the
box will require sinusoidal solutions (to vanish on the boundaries since the particle can't get past the potential
barrier).

\subsection{ Particle in a box. }

\section{ Problem 2. Probabilities for a polynomial wavefunction. }

\subsection{ normalize it. }

This part can be done directly with a contour integration.

\subsection{ definite integral of probablility. }

While a partial fractions split of the probability density can be done
around the poles $\{\pm\sqrt{i}, \pm i \sqrt{i}\}$

\begin{align*}
\frac{1 + x^2}{1 + x^4} &=
\frac{A}{x -\sqrt{i}}
+\frac{B}{x +\sqrt{i}}
+\frac{C}{x -i\sqrt{i}}
+\frac{D}{x +i\sqrt{i}}
\end{align*}

Integrating these is straight forward (logarithms), but
the algebra gets messy to simplify the resulting expression (perhaps not so
much for the $[0,1]$ interval of the question).

A lazier way is to invoke \href{http://integrals.wolfram.com/index.jsp}{webmathematica}, which gives

\begin{align*}
\int \frac{1 + x^2}{1 + x^4} dx &=
\frac{-\tan^{-1}(1 - \sqrt{2} x) + \tan^{-1}(1 + \sqrt{2} x)}{ \sqrt{2} }
\end{align*}

\bibliographystyle{plainnat}
\bibliography{myrefs}

\end{document}

%
% Copyright � 2012 Peeter Joot.  All Rights Reserved.
% Licenced as described in the file LICENSE under the root directory of this GIT repository.
%

%
%
%\documentclass{article}

%\usepackage{amsmath}
\usepackage{mathpazo}

%
% shorthand for bold symbols, convenient for vectors and matrices
%
\newcommand{\Ba}[0]{\mathbf{a}}
\newcommand{\Bb}[0]{\mathbf{b}}
\newcommand{\Bc}[0]{\mathbf{c}}
\newcommand{\Bd}[0]{\mathbf{d}}
\newcommand{\Be}[0]{\mathbf{e}}
\newcommand{\Bf}[0]{\mathbf{f}}
\newcommand{\Bg}[0]{\mathbf{g}}
\newcommand{\Bh}[0]{\mathbf{h}}
\newcommand{\Bi}[0]{\mathbf{i}}
\newcommand{\Bj}[0]{\mathbf{j}}
\newcommand{\Bk}[0]{\mathbf{k}}
\newcommand{\Bl}[0]{\mathbf{l}}
\newcommand{\Bm}[0]{\mathbf{m}}
\newcommand{\Bn}[0]{\mathbf{n}}
\newcommand{\Bo}[0]{\mathbf{o}}
\newcommand{\Bp}[0]{\mathbf{p}}
\newcommand{\Bq}[0]{\mathbf{q}}
\newcommand{\Br}[0]{\mathbf{r}}
\newcommand{\Bs}[0]{\mathbf{s}}
\newcommand{\Bt}[0]{\mathbf{t}}
\newcommand{\Bu}[0]{\mathbf{u}}
\newcommand{\Bv}[0]{\mathbf{v}}
\newcommand{\Bw}[0]{\mathbf{w}}
\newcommand{\Bx}[0]{\mathbf{x}}
\newcommand{\By}[0]{\mathbf{y}}
\newcommand{\Bz}[0]{\mathbf{z}}
\newcommand{\BA}[0]{\mathbf{A}}
\newcommand{\BB}[0]{\mathbf{B}}
\newcommand{\BC}[0]{\mathbf{C}}
\newcommand{\BD}[0]{\mathbf{D}}
\newcommand{\BE}[0]{\mathbf{E}}
\newcommand{\BF}[0]{\mathbf{F}}
\newcommand{\BG}[0]{\mathbf{G}}
\newcommand{\BH}[0]{\mathbf{H}}
\newcommand{\BI}[0]{\mathbf{I}}
\newcommand{\BJ}[0]{\mathbf{J}}
\newcommand{\BK}[0]{\mathbf{K}}
\newcommand{\BL}[0]{\mathbf{L}}
\newcommand{\BM}[0]{\mathbf{M}}
\newcommand{\BN}[0]{\mathbf{N}}
\newcommand{\BO}[0]{\mathbf{O}}
\newcommand{\BP}[0]{\mathbf{P}}
\newcommand{\BQ}[0]{\mathbf{Q}}
\newcommand{\BR}[0]{\mathbf{R}}
\newcommand{\BS}[0]{\mathbf{S}}
\newcommand{\BT}[0]{\mathbf{T}}
\newcommand{\BU}[0]{\mathbf{U}}
\newcommand{\BV}[0]{\mathbf{V}}
\newcommand{\BW}[0]{\mathbf{W}}
\newcommand{\BX}[0]{\mathbf{X}}
\newcommand{\BY}[0]{\mathbf{Y}}
\newcommand{\BZ}[0]{\mathbf{Z}}

\newcommand{\Bzero}[0]{\mathbf{0}}
\newcommand{\Btheta}[0]{\boldsymbol{\theta}}
\newcommand{\Btau}[0]{\boldsymbol{\tau}}
\newcommand{\Bomega}[0]{\boldsymbol{\omega}}

%
% shorthand for unit vectors
%
\newcommand{\acap}[0]{\hat{\Ba}}
\newcommand{\bcap}[0]{\hat{\Bb}}
\newcommand{\ccap}[0]{\hat{\Bc}}
\newcommand{\dcap}[0]{\hat{\Bd}}
\newcommand{\ecap}[0]{\hat{\Be}}
\newcommand{\fcap}[0]{\hat{\Bf}}
\newcommand{\gcap}[0]{\hat{\Bg}}
\newcommand{\hcap}[0]{\hat{\Bh}}
\newcommand{\icap}[0]{\hat{\Bi}}
\newcommand{\jcap}[0]{\hat{\Bj}}
\newcommand{\kcap}[0]{\hat{\Bk}}
\newcommand{\lcap}[0]{\hat{\Bl}}
\newcommand{\mcap}[0]{\hat{\Bm}}
\newcommand{\ncap}[0]{\hat{\Bn}}
\newcommand{\ocap}[0]{\hat{\Bo}}
\newcommand{\pcap}[0]{\hat{\Bp}}
\newcommand{\qcap}[0]{\hat{\Bq}}
\newcommand{\rcap}[0]{\hat{\Br}}
\newcommand{\scap}[0]{\hat{\Bs}}
\newcommand{\tcap}[0]{\hat{\Bt}}
\newcommand{\ucap}[0]{\hat{\Bu}}
\newcommand{\vcap}[0]{\hat{\Bv}}
\newcommand{\wcap}[0]{\hat{\Bw}}
\newcommand{\xcap}[0]{\hat{\Bx}}
\newcommand{\ycap}[0]{\hat{\By}}
\newcommand{\zcap}[0]{\hat{\Bz}}
\newcommand{\thetacap}[0]{\hat{\Btheta}}

%
% to write R^n and C^n in a distinguishable fashion.  Perhaps change this
% to the double lined characters upon figuring out how to do so.
%
\newcommand{\C}[1]{$\mathbb{C}^{#1}$}
\newcommand{\R}[1]{$\mathbb{R}^{#1}$}

%
% various generally useful helpers
%

% derivative of #1 wrt. #2:
\newcommand{\D}[2] {\frac {d#2} {d#1}}

\newcommand{\inv}[1]{\frac{1}{#1}}
\newcommand{\cross}[0]{\times}

\newcommand{\abs}[1]{\lvert{#1}\rvert}
\newcommand{\norm}[1]{\lVert{#1}\rVert}
\newcommand{\innerprod}[2]{\langle{#1}, {#2}\rangle}
\newcommand{\dotprod}[2]{{#1} \cdot {#2}}
\newcommand{\bdotprod}[2]{\left({#1} \cdot {#2}\right)}
\newcommand{\crossprod}[2]{{#1} \cross {#2}}
\newcommand{\tripleprod}[3]{\dotprod{\left(\crossprod{#1}{#2}\right)}{#3}}

\DeclareMathOperator{\Proj}{Proj}
\DeclareMathOperator{\Span}{span}
\DeclareMathOperator{\Sgn}{sgn}
\DeclareMathOperator{\Area}{Area}
\DeclareMathOperator{\Volume}{Volume}

%
% A few miscellaneous things specific to this document
%
\newcommand{\crossop}[1]{\crossprod{#1}{}}

% R2 vector.
\newcommand{\VectorTwo}[2]{
\begin{bmatrix}
 {#1} \\
 {#2}
\end{bmatrix}
}

\newcommand{\VectorN}[1]{
\begin{bmatrix}
{#1}_1 \\
{#1}_2 \\
\vdots \\
{#1}_N \\
\end{bmatrix}
}

\newcommand{\DETuvij}[4]{
\begin{vmatrix}
 {#1}_{#3} & {#1}_{#4} \\
 {#2}_{#3} & {#2}_{#4}
\end{vmatrix}
}

\newcommand{\DETuvwijk}[6]{
\begin{vmatrix}
 {#1}_{#4} & {#1}_{#5} & {#1}_{#6} \\
 {#2}_{#4} & {#2}_{#5} & {#2}_{#6} \\
 {#3}_{#4} & {#3}_{#5} & {#3}_{#6}
\end{vmatrix}
}

\newcommand{\DETuvwxijkl}[8]{
\begin{vmatrix}
 {#1}_{#5} & {#1}_{#6} & {#1}_{#7} & {#1}_{#8} \\
 {#2}_{#5} & {#2}_{#6} & {#2}_{#7} & {#2}_{#8} \\
 {#3}_{#5} & {#3}_{#6} & {#3}_{#7} & {#3}_{#8} \\
 {#4}_{#5} & {#4}_{#6} & {#4}_{#7} & {#4}_{#8} \\
\end{vmatrix}
}

%\newcommand{\DETuvwxyijklm}[10]{
%\begin{vmatrix}
% {#1}_{#6} & {#1}_{#7} & {#1}_{#8} & {#1}_{#9} & {#1}_{#10} \\
% {#2}_{#6} & {#2}_{#7} & {#2}_{#8} & {#2}_{#9} & {#2}_{#10} \\
% {#3}_{#6} & {#3}_{#7} & {#3}_{#8} & {#3}_{#9} & {#3}_{#10} \\
% {#4}_{#6} & {#4}_{#7} & {#4}_{#8} & {#4}_{#9} & {#4}_{#10} \\
% {#5}_{#6} & {#5}_{#7} & {#5}_{#8} & {#5}_{#9} & {#5}_{#10}
%\end{vmatrix}
%}

% R3 vector.
\newcommand{\VectorThree}[3]{
\begin{bmatrix}
 {#1} \\
 {#2} \\
 {#3}
\end{bmatrix}
}


%%<misc>
%
\newcommand{\Abs}[1]{{\left\lvert{#1}\right\rvert}}
\newcommand{\spacegrad}[0]{\boldsymbol{\nabla}}
\newcommand{\grad}[0]{\nabla}
\newcommand{\LL}[0]{\mathcal{L}}

% == \partial_{#1} {#2}
\newcommand{\PD}[2]{\frac{\partial {#2}}{\partial {#1}}}
% inline variant
\newcommand{\PDi}[2]{{\partial {#2}}/{\partial {#1}}}

\newcommand{\PDD}[3]{\frac{\partial^2 {#3}}{\partial {#1}\partial {#2}}}
%\newcommand{\PDd}[2]{\frac{\partial^2 {#2}}{{\partial{#1}}^2}}
\newcommand{\PDsq}[2]{\frac{\partial^2 {#2}}{(\partial {#1})^2}}

\newcommand{\Partial}[2]{\frac{\partial {#1}}{\partial {#2}}}
\DeclareMathOperator{\RejName}{Rej}
\newcommand{\Rej}[2]{\RejName_{#1}\left( {#2} \right)}
\newcommand{\Rm}[1]{\mathbb{R}^{#1}}
\newcommand{\Cm}[1]{\mathbb{C}^{#1}}
\newcommand{\conj}[0]{{*}}

%</misc>

% <grade selection>
%
\newcommand{\gpgrade}[2] {{\left\langle{{#1}}\right\rangle}_{#2}}

\newcommand{\gpgradezero}[1] {\gpgrade{#1}{}}
%\newcommand{\gpscalargrade}[1] {{\left\langle{{#1}}\right\rangle}}
%\newcommand{\gpgradezero}[1] {\gpgrade{#1}{0}}

%\newcommand{\gpgradeone}[1] {{\left\langle{{#1}}\right\rangle}_{1}}
\newcommand{\gpgradeone}[1] {\gpgrade{#1}{1}}

\newcommand{\gpgradetwo}[1] {\gpgrade{#1}{2}}
\newcommand{\gpgradethree}[1] {\gpgrade{#1}{3}}
\newcommand{\gpgradefour}[1] {\gpgrade{#1}{4}}
%
% </grade selection>



\newcommand{\adot}[0]{{\dot{a}}}
\newcommand{\bdot}[0]{{\dot{b}}}
% taken for centered dot:
%\newcommand{\cdot}[0]{{\dot{c}}}
%\newcommand{\ddot}[0]{{\dot{d}}}
\newcommand{\edot}[0]{{\dot{e}}}
\newcommand{\fdot}[0]{{\dot{f}}}
\newcommand{\gdot}[0]{{\dot{g}}}
\newcommand{\hdot}[0]{{\dot{h}}}
\newcommand{\idot}[0]{{\dot{i}}}
\newcommand{\jdot}[0]{{\dot{j}}}
\newcommand{\kdot}[0]{{\dot{k}}}
\newcommand{\ldot}[0]{{\dot{l}}}
\newcommand{\mdot}[0]{{\dot{m}}}
\newcommand{\ndot}[0]{{\dot{n}}}
%\newcommand{\odot}[0]{{\dot{o}}}
\newcommand{\pdot}[0]{{\dot{p}}}
\newcommand{\qdot}[0]{{\dot{q}}}
\newcommand{\rdot}[0]{{\dot{r}}}
\newcommand{\sdot}[0]{{\dot{s}}}
\newcommand{\tdot}[0]{{\dot{t}}}
\newcommand{\udot}[0]{{\dot{u}}}
\newcommand{\vdot}[0]{{\dot{v}}}
\newcommand{\wdot}[0]{{\dot{w}}}
\newcommand{\xdot}[0]{{\dot{x}}}
\newcommand{\ydot}[0]{{\dot{y}}}
\newcommand{\zdot}[0]{{\dot{z}}}
\newcommand{\addot}[0]{{\ddot{a}}}
\newcommand{\bddot}[0]{{\ddot{b}}}
\newcommand{\cddot}[0]{{\ddot{c}}}
%\newcommand{\dddot}[0]{{\ddot{d}}}
\newcommand{\eddot}[0]{{\ddot{e}}}
\newcommand{\fddot}[0]{{\ddot{f}}}
\newcommand{\gddot}[0]{{\ddot{g}}}
\newcommand{\hddot}[0]{{\ddot{h}}}
\newcommand{\iddot}[0]{{\ddot{i}}}
\newcommand{\jddot}[0]{{\ddot{j}}}
\newcommand{\kddot}[0]{{\ddot{k}}}
\newcommand{\lddot}[0]{{\ddot{l}}}
\newcommand{\mddot}[0]{{\ddot{m}}}
\newcommand{\nddot}[0]{{\ddot{n}}}
\newcommand{\oddot}[0]{{\ddot{o}}}
\newcommand{\pddot}[0]{{\ddot{p}}}
\newcommand{\qddot}[0]{{\ddot{q}}}
\newcommand{\rddot}[0]{{\ddot{r}}}
\newcommand{\sddot}[0]{{\ddot{s}}}
\newcommand{\tddot}[0]{{\ddot{t}}}
\newcommand{\uddot}[0]{{\ddot{u}}}
\newcommand{\vddot}[0]{{\ddot{v}}}
\newcommand{\wddot}[0]{{\ddot{w}}}
\newcommand{\xddot}[0]{{\ddot{x}}}
\newcommand{\yddot}[0]{{\ddot{y}}}
\newcommand{\zddot}[0]{{\ddot{z}}}

%<bold and dot greek symbols>
%

\newcommand{\Deltadot}[0]{{\dot{\Delta}}}
\newcommand{\Gammadot}[0]{{\dot{\Gamma}}}
\newcommand{\Lambdadot}[0]{{\dot{\Lambda}}}
\newcommand{\Omegadot}[0]{{\dot{\Omega}}}
\newcommand{\Phidot}[0]{{\dot{\Phi}}}
\newcommand{\Pidot}[0]{{\dot{\Pi}}}
\newcommand{\Psidot}[0]{{\dot{\Psi}}}
\newcommand{\Sigmadot}[0]{{\dot{\Sigma}}}
\newcommand{\Thetadot}[0]{{\dot{\Theta}}}
\newcommand{\Upsilondot}[0]{{\dot{\Upsilon}}}
\newcommand{\Xidot}[0]{{\dot{\Xi}}}
\newcommand{\alphadot}[0]{{\dot{\alpha}}}
\newcommand{\betadot}[0]{{\dot{\beta}}}
\newcommand{\chidot}[0]{{\dot{\chi}}}
\newcommand{\deltadot}[0]{{\dot{\delta}}}
\newcommand{\epsilondot}[0]{{\dot{\epsilon}}}
\newcommand{\etadot}[0]{{\dot{\eta}}}
\newcommand{\gammadot}[0]{{\dot{\gamma}}}
\newcommand{\kappadot}[0]{{\dot{\kappa}}}
\newcommand{\lambdadot}[0]{{\dot{\lambda}}}
\newcommand{\mudot}[0]{{\dot{\mu}}}
\newcommand{\nudot}[0]{{\dot{\nu}}}
\newcommand{\omegadot}[0]{{\dot{\omega}}}
\newcommand{\phidot}[0]{{\dot{\phi}}}
\newcommand{\pidot}[0]{{\dot{\pi}}}
\newcommand{\psidot}[0]{{\dot{\psi}}}
\newcommand{\rhodot}[0]{{\dot{\rho}}}
\newcommand{\sigmadot}[0]{{\dot{\sigma}}}
\newcommand{\taudot}[0]{{\dot{\tau}}}
\newcommand{\thetadot}[0]{{\dot{\theta}}}
\newcommand{\upsilondot}[0]{{\dot{\upsilon}}}
\newcommand{\varepsilondot}[0]{{\dot{\varepsilon}}}
\newcommand{\varphidot}[0]{{\dot{\varphi}}}
\newcommand{\varpidot}[0]{{\dot{\varpi}}}
\newcommand{\varrhodot}[0]{{\dot{\varrho}}}
\newcommand{\varsigmadot}[0]{{\dot{\varsigma}}}
\newcommand{\varthetadot}[0]{{\dot{\vartheta}}}
\newcommand{\xidot}[0]{{\dot{\xi}}}
\newcommand{\zetadot}[0]{{\dot{\zeta}}}

\newcommand{\Deltaddot}[0]{{\ddot{\Delta}}}
\newcommand{\Gammaddot}[0]{{\ddot{\Gamma}}}
\newcommand{\Lambdaddot}[0]{{\ddot{\Lambda}}}
\newcommand{\Omegaddot}[0]{{\ddot{\Omega}}}
\newcommand{\Phiddot}[0]{{\ddot{\Phi}}}
\newcommand{\Piddot}[0]{{\ddot{\Pi}}}
\newcommand{\Psiddot}[0]{{\ddot{\Psi}}}
\newcommand{\Sigmaddot}[0]{{\ddot{\Sigma}}}
\newcommand{\Thetaddot}[0]{{\ddot{\Theta}}}
\newcommand{\Upsilonddot}[0]{{\ddot{\Upsilon}}}
\newcommand{\Xiddot}[0]{{\ddot{\Xi}}}
\newcommand{\alphaddot}[0]{{\ddot{\alpha}}}
\newcommand{\betaddot}[0]{{\ddot{\beta}}}
\newcommand{\chiddot}[0]{{\ddot{\chi}}}
\newcommand{\deltaddot}[0]{{\ddot{\delta}}}
\newcommand{\epsilonddot}[0]{{\ddot{\epsilon}}}
\newcommand{\etaddot}[0]{{\ddot{\eta}}}
\newcommand{\gammaddot}[0]{{\ddot{\gamma}}}
\newcommand{\kappaddot}[0]{{\ddot{\kappa}}}
\newcommand{\lambdaddot}[0]{{\ddot{\lambda}}}
\newcommand{\muddot}[0]{{\ddot{\mu}}}
\newcommand{\nuddot}[0]{{\ddot{\nu}}}
\newcommand{\omegaddot}[0]{{\ddot{\omega}}}
\newcommand{\phiddot}[0]{{\ddot{\phi}}}
\newcommand{\piddot}[0]{{\ddot{\pi}}}
\newcommand{\psiddot}[0]{{\ddot{\psi}}}
\newcommand{\rhoddot}[0]{{\ddot{\rho}}}
\newcommand{\sigmaddot}[0]{{\ddot{\sigma}}}
\newcommand{\tauddot}[0]{{\ddot{\tau}}}
\newcommand{\thetaddot}[0]{{\ddot{\theta}}}
\newcommand{\upsilonddot}[0]{{\ddot{\upsilon}}}
\newcommand{\varepsilonddot}[0]{{\ddot{\varepsilon}}}
\newcommand{\varphiddot}[0]{{\ddot{\varphi}}}
\newcommand{\varpiddot}[0]{{\ddot{\varpi}}}
\newcommand{\varrhoddot}[0]{{\ddot{\varrho}}}
\newcommand{\varsigmaddot}[0]{{\ddot{\varsigma}}}
\newcommand{\varthetaddot}[0]{{\ddot{\vartheta}}}
\newcommand{\xiddot}[0]{{\ddot{\xi}}}
\newcommand{\zetaddot}[0]{{\ddot{\zeta}}}

\newcommand{\BDelta}[0]{\boldsymbol{\Delta}}
\newcommand{\BGamma}[0]{\boldsymbol{\Gamma}}
\newcommand{\BLambda}[0]{\boldsymbol{\Lambda}}
\newcommand{\BOmega}[0]{\boldsymbol{\Omega}}
\newcommand{\BPhi}[0]{\boldsymbol{\Phi}}
\newcommand{\BPi}[0]{\boldsymbol{\Pi}}
\newcommand{\BPsi}[0]{\boldsymbol{\Psi}}
\newcommand{\BSigma}[0]{\boldsymbol{\Sigma}}
\newcommand{\BTheta}[0]{\boldsymbol{\Theta}}
\newcommand{\BUpsilon}[0]{\boldsymbol{\Upsilon}}
\newcommand{\BXi}[0]{\boldsymbol{\Xi}}
\newcommand{\Balpha}[0]{\boldsymbol{\alpha}}
\newcommand{\Bbeta}[0]{\boldsymbol{\beta}}
\newcommand{\Bchi}[0]{\boldsymbol{\chi}}
\newcommand{\Bdelta}[0]{\boldsymbol{\delta}}
\newcommand{\Bepsilon}[0]{\boldsymbol{\epsilon}}
\newcommand{\Beta}[0]{\boldsymbol{\eta}}
\newcommand{\Bgamma}[0]{\boldsymbol{\gamma}}
\newcommand{\Bkappa}[0]{\boldsymbol{\kappa}}
\newcommand{\Blambda}[0]{\boldsymbol{\lambda}}
\newcommand{\Bmu}[0]{\boldsymbol{\mu}}
\newcommand{\Bnu}[0]{\boldsymbol{\nu}}
%\newcommand{\Bomega}[0]{\boldsymbol{\omega}}
\newcommand{\Bphi}[0]{\boldsymbol{\phi}}
\newcommand{\Bpi}[0]{\boldsymbol{\pi}}
\newcommand{\Bpsi}[0]{\boldsymbol{\psi}}
\newcommand{\Brho}[0]{\boldsymbol{\rho}}
\newcommand{\Bsigma}[0]{\boldsymbol{\sigma}}
%\newcommand{\Btau}[0]{\boldsymbol{\tau}}
%\newcommand{\Btheta}[0]{\boldsymbol{\theta}}
\newcommand{\Bupsilon}[0]{\boldsymbol{\upsilon}}
\newcommand{\Bvarepsilon}[0]{\boldsymbol{\varepsilon}}
\newcommand{\Bvarphi}[0]{\boldsymbol{\varphi}}
\newcommand{\Bvarpi}[0]{\boldsymbol{\varpi}}
\newcommand{\Bvarrho}[0]{\boldsymbol{\varrho}}
\newcommand{\Bvarsigma}[0]{\boldsymbol{\varsigma}}
\newcommand{\Bvartheta}[0]{\boldsymbol{\vartheta}}
\newcommand{\Bxi}[0]{\boldsymbol{\xi}}
\newcommand{\Bzeta}[0]{\boldsymbol{\zeta}}
%
%</bold and dot greek symbols>
%<infrequent>
%
%\newcommand{\AreaOp}[1]{\AName_{#1}}
%\newcommand{\Babs}[0]{\abs{\BB}}
%\newcommand{\Bcap}[0]{\hat{\BB}}
%\newcommand{\BrPrimeRej}[0]{\rcap(\rcap \wedge \Br')}
%\newcommand{\CA}[0]{\mathcal{A}}
%\newcommand{\Cos}[1]{\cos{\left({#1}\right)}}
%\newcommand{\Det}[1] {\abs{#1}}
%\newcommand{\Dsq}[2] {\frac {\partial^2 {#1}} {\partial {#2}^2}}
%\newcommand{\Exp}[1]{\exp{\left({#1}\right)}}
%\newcommand{\Norm}[1]{\left\lVert{#1}\right\rVert}
%\newcommand{\Sin}[1]{\sin{\left({#1}\right)}}
%\newcommand{\T}[0]{\text{T}}
%\newcommand{\VolumeOp}[1]{\VName_{#1}}
%\newcommand{\agrad}[0]{\Ba \cdot \nabla}
%\newcommand{\alphacap}[0]{\hat{\boldsymbol{\alpha}}}
%\newcommand{\Fcap}[0]{\hat{\BF}}
%\newcommand{\bithree}[0]{{\Bi}_3}
%\newcommand{\bxa}[0]{\Bx\Ba}
%\newcommand{\coordvec}[2]{
%\newcommand{\costheta}[0]{\acap \cdot \xcap}
%\newcommand{\ddt}[1]{\ddot{#1}}
%\newcommand{\ddu}[1] {\frac {d{#1}} {du}}
%\newcommand{\dsqxj}[2] {\frac {\partial^2 {#1}} {\partial {x_{#2}}^2}}
%\newcommand{\dtheta}[1]{\frac{d {#1}}{d \theta}}
%\newcommand{\dt}[1]{\dot{#1}}
%\newcommand{\dt}[1]{\frac{d {#1}}{dt}}
%\newcommand{\dxj}[2] {\frac {\partial {#1}} {\partial {x_{#2}}}}
%\newcommand{\halfPhi}[0]{\frac{\phi}{2}}
%\newcommand{\half}[0]{\inv{2}}
%\newcommand{\inv}[1]{\frac{1}{#1}}
%\newcommand{\laplacian}[0]{\nabla^2}
%\newcommand{\matrixoftx}[3]{
%\newcommand{\nrrp}[0]{\norm{\rcap \wedge \Br'}}
%\newcommand{\oiint}{\bigcirc \hspace{-1.4em} \int \hspace{-.8em} \int}
%\newcommand{\transpose}[1]{{#1}^{\text{T}}}
%\newcommand{\transpose}[1]{{{#1}^{\TextTranspose}}}
%\newcommand{\transpose}[1]{{{#1}^{\text{T}}}}
%\newcommand{\barA}[0]{\bar{A}}
%\newcommand{\qbar}[0]{\bar{q}}
%\newcommand{\qdotbar}[0]{\dot{\bar{q}}}
%
%</infrequent>





%\usepackage[bookmarks=true]{hyperref}

%\usepackage{color,cite,graphicx}
   % use colour in the document, put your citations as [1-4]
   % rather than [1,2,3,4] (it looks nicer, and the extended LaTeX2e
   % graphics package.
%\usepackage{latexsym,amssymb,epsf} % do not remember if these are
   % needed, but their inclusion can not do any damage


\chapter{Simple Wave Packet Examples}
\label{chap:wavepacket}
%\author{Peeter Joot \quad peeterjoot@protonmail.com}
\date{ Feb 16, 2009.  wavepacket.tex }

%\begin{document}

%\maketitle{}
%\tableofcontents
\section{Wave packet examples}

In \citep{bohm1989qt}, chapter 3, a couple explicit wave packet examples are given.
Some of this is a little hard to follow in the small set font of the text, and some details are missing.  Here the integrals are performed
in detail.

\subsection{Unweighted example. Equation 1}

Summing plane waves over a small range of frequencies

\begin{equation}\label{eqn:wavepacket:20}
\begin{aligned}
E_z(x)
&= \int_{k_0 -\Delta k}^{k_0 -\Delta k} dk e^{ik(x-x_0)} \\
&= {\left. \frac{e^{ik(x-x_0)}}{i(x-x_0)} \right\vert}_{k_0 -\Delta k}^{k_0 -\Delta k}  \\
&= \frac{e^{i(k_0 + \Delta k)(x-x_0)}}{i(x-x_0)} -\frac{e^{i(k_0 - \Delta k)(x-x_0)}}{i(x-x_0)} \\
&= 2 {e^{i k_0 (x-x_0)}}
 \frac{{e^{i \Delta k(x-x_0)}} -{e^{-i \Delta k(x-x_0)}}}{2 i (x-x_0)} \\
\end{aligned}
\end{equation}

Which is Bohm's equation 1.
\begin{equation}\label{eqn:wavepacket:40}
\begin{aligned}
E_z(x) &= 2 {e^{i k_0 (x-x_0)}} \frac{\sin( \Delta k (x-x_0))}{ x-x_0}
\end{aligned}
\end{equation}

\subsection{Gaussian weighting example. Equation 4}

Next example, also chosen for simplicity, uses a Gaussian weighting function

\begin{equation}\label{eqn:wavepacket:60}
\begin{aligned}
\psi
&= \int_\infty^\infty
\exp\left( - \frac{(k-k_0)^2}{2(\Delta k)^2} \right)
\exp\left( i k(x-x_0) \right) dk \\
\end{aligned}
\end{equation}

Next, is the sneaky/clever step of adding and subtracting \(i k_0 (x-x_0) - (x-x_0)^2 (\Delta k)^2/2\) to the exponentials, which gives

\begin{equation}\label{eqn:wavepacket:80}
\begin{aligned}
\psi
&=
\exp\left( i k_0 (x-x_0) - \frac{(x-x_0)^2}{2} (\Delta k)^2 \right) \\
&\quad \int_\infty^\infty
\exp\left(
- \frac{(k-k_0)^2}{2(\Delta k)^2}
+i (k-k_0)(x-x_0)
+\frac{(x-x_0)^2}{2} (\Delta k)^2
\right)
dk \\
\end{aligned}
\end{equation}

Looking at just the remaining integral part, say \(I\), we have
\begin{equation}\label{eqn:wavepacket:100}
\begin{aligned}
I &=
\int_\infty^\infty
\exp\left( \inv{2} \left(
i^2 \frac{(k-k_0)^2}{(\Delta k)^2}
+2i (k-k_0)(x-x_0)
+{(x-x_0)^2}{} (\Delta k)^2
\right)
\right)
dk \\
&=
\int_\infty^\infty
\exp\left( \inv{2} \left(
i \frac{k-k_0}{\Delta k}
+(x-x_0) \Delta k
\right)^2
\right)
dk \\
&=
\int_\infty^\infty
\exp\left( -\inv{2} \left(
\frac{k-k_0}{\Delta k}
-i(x-x_0) \Delta k
\right)^2
\right)
dk \\
\end{aligned}
\end{equation}

A change of variables \(u = (k-k_0)/{\Delta k} -i(x-x_0) \Delta k\), \(du = dk/{\Delta k}\) gives

\begin{equation}\label{eqn:wavepacket:120}
\begin{aligned}
I
&= \Delta k \int_\infty^\infty e^{ -u^2/2 } du \\
&= \sqrt{2 \pi} \Delta k
\end{aligned}
\end{equation}

Which gives

\begin{equation}\label{eqn:wavepacket:140}
\begin{aligned}
\psi(x)
&= \sqrt{2 \pi} \Delta k
\exp\left( i k_0 (x-x_0) - \frac{(x-x_0)^2}{2} (\Delta k)^2 \right) \\
\end{aligned}
\end{equation}

Off by a factor of \(\sqrt{\Delta k}\) compared to the text?  Typo in the book or a mistake above?

\subsection{Gaussian weighting with angular velocity and acceleration}

Next covered is a wave packet where the angular frequency is a function of the wave number, as in equation 8

\begin{equation}\label{eqn:wavepacket:160}
\begin{aligned}
E(x,t) &= \IIinf f(k - k_0) \exp\left( i k(x-x_0) - i\omega(k) t\right) dk
\end{aligned}
\end{equation}

With a Taylor series expansion of the frequency about \(k_0\)

\begin{equation}\label{eqn:wavepacket:180}
\begin{aligned}
\omega(k)
&= \omega(k_0) + \left(\PD{k}{\omega}\right)_{k=k_0} (k-k_0) + \left(\PDSq{k}{\omega}\right)_{k=k_0} \frac{(k-k_0)^2}{2} + \cdots \\
&= \omega_0 + V_g (k-k_0) + \alpha \frac{(k-k_0)^2}{2} + \cdots \\
\end{aligned}
\end{equation}

Here \(V_g\), and \(\alpha\) are the group velocity and accelerations respectively.  Now, I had never seen the group velocity expressed
this way, which seems a particularly simple way of putting it.
The example of how \(\omega = 2 \pi c/\lambda n(\lambda)\) can vary with index of refraction and wavelength is also nice.  I
imagine a light wave going through a water oil air transition and the angular frequency in each region causing dispersion and
reflection and path alteration effects.

Back to the second order approximation of the frequency, substituting back into the wave packet integral one has

\begin{equation}\label{eqn:wavepacket:200}
\begin{aligned}
E(x,t) &\approx \IIinf f(k - k_0) \exp\left( i k(x-x_0) - i
\left(\omega_0 + V_g (k-k_0) + \alpha \frac{(k-k_0)^2}{2} \right)
t\right) dk
\end{aligned}
\end{equation}

With \(\kappa = k - k_0\), \(\Delta x = x -x_0\) and the Gaussian weighting \(f(\kappa) = e^{-\kappa^2/2(\Delta k)^2}\) this is

\begin{equation}\label{eqn:wavepacket:220}
\begin{aligned}
\exp&( i k_0 (x-x_0) )
\IIinf
\exp\left( -\frac{\kappa^2}{2(\Delta k)^2} +i \kappa (x-x_0)
- i \left(\omega_0 + V_g \kappa + \alpha \frac{\kappa^2}{2} \right) t\right) dk \\
&=
\exp( i k_0 \Delta x -i \omega_0 t)
\IIinf
\exp\left(
i \kappa ( \Delta x - V_g t )
-\frac{\kappa^2}{2} \left(\inv{(\Delta k)^2} + i \alpha t \right)
\right) d\kappa \\
\end{aligned}
\end{equation}

With \(a = \Delta x - V_g t\) and \(b = \inv{(\Delta k)^2} + i \alpha t\), the exponential in the integral takes the form
\begin{equation}\label{eqn:wavepacket:240}
\begin{aligned}
\exp\left( i \kappa a - \kappa^2 \frac{b}{2} \right)
&= \exp\left( - \frac{b}{2}\left(-2 i \kappa \frac{a}{b} + \kappa^2 \right) \right)  \\
&= \exp\left( - \frac{b}{2}\left( \kappa - i \frac{a}{b} \right)^2 + \frac{b}{2}\left(\frac{ia}{b}\right)^2 \right)  \\
&= \exp\left( - \frac{b}{2}\left( \kappa - i \frac{a}{b} \right)^2 - \frac{a^2}{2b} \right)  \\
\end{aligned}
\end{equation}

Our wave packet is now

\begin{equation}\label{eqn:wavepacket:260}
\begin{aligned}
E(x,t)
&= \exp\left( i k_0 \Delta x -i \omega_0 t - \frac{( \Delta x - V_g t)^2 (\Delta k)^2}{2(1 + i \alpha t (\Delta k)^2)} \right)
\IIinf
\exp\left( - \frac{b}{2}\left( \kappa - i \frac{a}{b} \right)^2 \right)
d\kappa \\
\end{aligned}
\end{equation}

A change of vars \(u = \sqrt{b}(\kappa - ia/b)\) gives

\begin{equation}\label{eqn:wavepacket:280}
\begin{aligned}
E(x,t)
&= \exp\left( i k_0 \Delta x -i \omega_0 t - \frac{( \Delta x - V_g t)^2 (\Delta k)^2}{2(1 + i \alpha t (\Delta k)^2)} \right)
\frac{\Delta k}{\sqrt{1 + i \alpha t (\Delta k)^2}}
\IIinf \exp\left( - \frac{u^2}{2} \right) d\kappa \\
&= \exp\left( i k_0 \Delta x -i \omega_0 t - \frac{( \Delta x - V_g t)^2 (\Delta k)^2}{2(1 + i \alpha t (\Delta k)^2)} \right)
\frac{\sqrt{2\pi}\Delta k}{\sqrt{1 + i \alpha t (\Delta k)^2}} \\
&=
\frac{\sqrt{2\pi}\Delta k}{\sqrt{1 + i \alpha t (\Delta k)^2}}
\exp\left( i \left(k_0 \Delta x - \omega_0 t
+
\alpha t (\Delta k)^2 \frac{( \Delta x - V_g t)^2 (\Delta k)^2}{2(1 + \alpha^2 t^2 (\Delta k)^4)}
\right) \right) \times \\
&\exp\left(
-
\frac{( \Delta x - V_g t)^2 (\Delta k)^2}{2(1 + \alpha^2 t^2 (\Delta k)^4)}
\right) \\
\end{aligned}
\end{equation}

This is consistent with the result in the text and with \(\alpha = 0\) confirms that equation 4 did have a typo (irrelevant to the
intensity discussion).

%gnuplot> set xlabel "x-axis"
%gnuplot> set ylabel "t-axis"
%gnuplot> splot [x=75:100] [t=-10:10] exp(-10*(x-10*t)**2/(1+10*t*t*100))
%gnuplot> splot [x=75:100] [t=-10:100] exp(-10*(x-10*t)**2/(1+10*t*t*100))
%gnuplot> splot [x=75:100] [t=-100:100] exp(-10*(x-10*t)**2/(1+10*t*t*100))
%gnuplot> splot [x=75:100] [t=-100:100] exp(-10*(x-1*t)**2/(1+10*t*t*100))
%gnuplot> splot [x=75:100] [t=-100:100] exp(-10*(x-1*t)**2/(1+10*t*t*1000))
%gnuplot> splot [x=75:100] [t=-100:100] exp(-10*(x-1*t)**2/(1+10*t*t*10000))
%gnuplot> splot [x=75:100] [t=-100:100] exp(-10*(x-1*t)**2/(1+10*t*t*2))
%gnuplot> splot [x=-75:100] [t=-100:100] exp(-10*(x-1*t)**2/(1+10*t*t*2))

%\bibliographystyle{plainnat}
%\bibliography{myrefs}

%\end{document}


\part{Relativity.}
\documentclass{article}

\usepackage{amsmath}
\usepackage{mathpazo}

%
% shorthand for bold symbols, convenient for vectors and matrices
%
\newcommand{\Ba}[0]{\mathbf{a}}
\newcommand{\Bb}[0]{\mathbf{b}}
\newcommand{\Bc}[0]{\mathbf{c}}
\newcommand{\Bd}[0]{\mathbf{d}}
\newcommand{\Be}[0]{\mathbf{e}}
\newcommand{\Bf}[0]{\mathbf{f}}
\newcommand{\Bg}[0]{\mathbf{g}}
\newcommand{\Bh}[0]{\mathbf{h}}
\newcommand{\Bi}[0]{\mathbf{i}}
\newcommand{\Bj}[0]{\mathbf{j}}
\newcommand{\Bk}[0]{\mathbf{k}}
\newcommand{\Bl}[0]{\mathbf{l}}
\newcommand{\Bm}[0]{\mathbf{m}}
\newcommand{\Bn}[0]{\mathbf{n}}
\newcommand{\Bo}[0]{\mathbf{o}}
\newcommand{\Bp}[0]{\mathbf{p}}
\newcommand{\Bq}[0]{\mathbf{q}}
\newcommand{\Br}[0]{\mathbf{r}}
\newcommand{\Bs}[0]{\mathbf{s}}
\newcommand{\Bt}[0]{\mathbf{t}}
\newcommand{\Bu}[0]{\mathbf{u}}
\newcommand{\Bv}[0]{\mathbf{v}}
\newcommand{\Bw}[0]{\mathbf{w}}
\newcommand{\Bx}[0]{\mathbf{x}}
\newcommand{\By}[0]{\mathbf{y}}
\newcommand{\Bz}[0]{\mathbf{z}}
\newcommand{\BA}[0]{\mathbf{A}}
\newcommand{\BB}[0]{\mathbf{B}}
\newcommand{\BC}[0]{\mathbf{C}}
\newcommand{\BD}[0]{\mathbf{D}}
\newcommand{\BE}[0]{\mathbf{E}}
\newcommand{\BF}[0]{\mathbf{F}}
\newcommand{\BG}[0]{\mathbf{G}}
\newcommand{\BH}[0]{\mathbf{H}}
\newcommand{\BI}[0]{\mathbf{I}}
\newcommand{\BJ}[0]{\mathbf{J}}
\newcommand{\BK}[0]{\mathbf{K}}
\newcommand{\BL}[0]{\mathbf{L}}
\newcommand{\BM}[0]{\mathbf{M}}
\newcommand{\BN}[0]{\mathbf{N}}
\newcommand{\BO}[0]{\mathbf{O}}
\newcommand{\BP}[0]{\mathbf{P}}
\newcommand{\BQ}[0]{\mathbf{Q}}
\newcommand{\BR}[0]{\mathbf{R}}
\newcommand{\BS}[0]{\mathbf{S}}
\newcommand{\BT}[0]{\mathbf{T}}
\newcommand{\BU}[0]{\mathbf{U}}
\newcommand{\BV}[0]{\mathbf{V}}
\newcommand{\BW}[0]{\mathbf{W}}
\newcommand{\BX}[0]{\mathbf{X}}
\newcommand{\BY}[0]{\mathbf{Y}}
\newcommand{\BZ}[0]{\mathbf{Z}}

\newcommand{\Bzero}[0]{\mathbf{0}}
\newcommand{\Btheta}[0]{\boldsymbol{\theta}}
\newcommand{\Btau}[0]{\boldsymbol{\tau}}
\newcommand{\Bomega}[0]{\boldsymbol{\omega}}

%
% shorthand for unit vectors
%
\newcommand{\acap}[0]{\hat{\Ba}}
\newcommand{\bcap}[0]{\hat{\Bb}}
\newcommand{\ccap}[0]{\hat{\Bc}}
\newcommand{\dcap}[0]{\hat{\Bd}}
\newcommand{\ecap}[0]{\hat{\Be}}
\newcommand{\fcap}[0]{\hat{\Bf}}
\newcommand{\gcap}[0]{\hat{\Bg}}
\newcommand{\hcap}[0]{\hat{\Bh}}
\newcommand{\icap}[0]{\hat{\Bi}}
\newcommand{\jcap}[0]{\hat{\Bj}}
\newcommand{\kcap}[0]{\hat{\Bk}}
\newcommand{\lcap}[0]{\hat{\Bl}}
\newcommand{\mcap}[0]{\hat{\Bm}}
\newcommand{\ncap}[0]{\hat{\Bn}}
\newcommand{\ocap}[0]{\hat{\Bo}}
\newcommand{\pcap}[0]{\hat{\Bp}}
\newcommand{\qcap}[0]{\hat{\Bq}}
\newcommand{\rcap}[0]{\hat{\Br}}
\newcommand{\scap}[0]{\hat{\Bs}}
\newcommand{\tcap}[0]{\hat{\Bt}}
\newcommand{\ucap}[0]{\hat{\Bu}}
\newcommand{\vcap}[0]{\hat{\Bv}}
\newcommand{\wcap}[0]{\hat{\Bw}}
\newcommand{\xcap}[0]{\hat{\Bx}}
\newcommand{\ycap}[0]{\hat{\By}}
\newcommand{\zcap}[0]{\hat{\Bz}}
\newcommand{\thetacap}[0]{\hat{\Btheta}}

%
% to write R^n and C^n in a distinguishable fashion.  Perhaps change this
% to the double lined characters upon figuring out how to do so.
%
\newcommand{\C}[1]{$\mathbb{C}^{#1}$}
\newcommand{\R}[1]{$\mathbb{R}^{#1}$}

%
% various generally useful helpers
%

% derivative of #1 wrt. #2:
\newcommand{\D}[2] {\frac {d#2} {d#1}}

\newcommand{\inv}[1]{\frac{1}{#1}}
\newcommand{\cross}[0]{\times}

\newcommand{\abs}[1]{\lvert{#1}\rvert}
\newcommand{\norm}[1]{\lVert{#1}\rVert}
\newcommand{\innerprod}[2]{\langle{#1}, {#2}\rangle}
\newcommand{\dotprod}[2]{{#1} \cdot {#2}}
\newcommand{\bdotprod}[2]{\left({#1} \cdot {#2}\right)}
\newcommand{\crossprod}[2]{{#1} \cross {#2}}
\newcommand{\tripleprod}[3]{\dotprod{\left(\crossprod{#1}{#2}\right)}{#3}}

\DeclareMathOperator{\Proj}{Proj}
\DeclareMathOperator{\Span}{span}
\DeclareMathOperator{\Sgn}{sgn}
\DeclareMathOperator{\Area}{Area}
\DeclareMathOperator{\Volume}{Volume}

%
% A few miscellaneous things specific to this document
%
\newcommand{\crossop}[1]{\crossprod{#1}{}}

% R2 vector.
\newcommand{\VectorTwo}[2]{
\begin{bmatrix}
 {#1} \\
 {#2}
\end{bmatrix}
}

\newcommand{\VectorN}[1]{
\begin{bmatrix}
{#1}_1 \\
{#1}_2 \\
\vdots \\
{#1}_N \\
\end{bmatrix}
}

\newcommand{\DETuvij}[4]{
\begin{vmatrix}
 {#1}_{#3} & {#1}_{#4} \\
 {#2}_{#3} & {#2}_{#4}
\end{vmatrix}
}

\newcommand{\DETuvwijk}[6]{
\begin{vmatrix}
 {#1}_{#4} & {#1}_{#5} & {#1}_{#6} \\
 {#2}_{#4} & {#2}_{#5} & {#2}_{#6} \\
 {#3}_{#4} & {#3}_{#5} & {#3}_{#6}
\end{vmatrix}
}

\newcommand{\DETuvwxijkl}[8]{
\begin{vmatrix}
 {#1}_{#5} & {#1}_{#6} & {#1}_{#7} & {#1}_{#8} \\
 {#2}_{#5} & {#2}_{#6} & {#2}_{#7} & {#2}_{#8} \\
 {#3}_{#5} & {#3}_{#6} & {#3}_{#7} & {#3}_{#8} \\
 {#4}_{#5} & {#4}_{#6} & {#4}_{#7} & {#4}_{#8} \\
\end{vmatrix}
}

%\newcommand{\DETuvwxyijklm}[10]{
%\begin{vmatrix}
% {#1}_{#6} & {#1}_{#7} & {#1}_{#8} & {#1}_{#9} & {#1}_{#10} \\
% {#2}_{#6} & {#2}_{#7} & {#2}_{#8} & {#2}_{#9} & {#2}_{#10} \\
% {#3}_{#6} & {#3}_{#7} & {#3}_{#8} & {#3}_{#9} & {#3}_{#10} \\
% {#4}_{#6} & {#4}_{#7} & {#4}_{#8} & {#4}_{#9} & {#4}_{#10} \\
% {#5}_{#6} & {#5}_{#7} & {#5}_{#8} & {#5}_{#9} & {#5}_{#10}
%\end{vmatrix}
%}

% R3 vector.
\newcommand{\VectorThree}[3]{
\begin{bmatrix}
 {#1} \\
 {#2} \\
 {#3}
\end{bmatrix}
}


%<misc>
%
\newcommand{\Abs}[1]{{\left\lvert{#1}\right\rvert}}
\newcommand{\spacegrad}[0]{\boldsymbol{\nabla}}
\newcommand{\grad}[0]{\nabla}
\newcommand{\LL}[0]{\mathcal{L}}

% == \partial_{#1} {#2}
\newcommand{\PD}[2]{\frac{\partial {#2}}{\partial {#1}}}
% inline variant
\newcommand{\PDi}[2]{{\partial {#2}}/{\partial {#1}}}

\newcommand{\PDD}[3]{\frac{\partial^2 {#3}}{\partial {#1}\partial {#2}}}
%\newcommand{\PDd}[2]{\frac{\partial^2 {#2}}{{\partial{#1}}^2}}
\newcommand{\PDsq}[2]{\frac{\partial^2 {#2}}{(\partial {#1})^2}}

\newcommand{\Partial}[2]{\frac{\partial {#1}}{\partial {#2}}}
\DeclareMathOperator{\RejName}{Rej}
\newcommand{\Rej}[2]{\RejName_{#1}\left( {#2} \right)}
\newcommand{\Rm}[1]{\mathbb{R}^{#1}}
\newcommand{\Cm}[1]{\mathbb{C}^{#1}}
\newcommand{\conj}[0]{{*}}

%</misc>

% <grade selection>
%
\newcommand{\gpgrade}[2] {{\left\langle{{#1}}\right\rangle}_{#2}}

\newcommand{\gpgradezero}[1] {\gpgrade{#1}{}}
%\newcommand{\gpscalargrade}[1] {{\left\langle{{#1}}\right\rangle}}
%\newcommand{\gpgradezero}[1] {\gpgrade{#1}{0}}

%\newcommand{\gpgradeone}[1] {{\left\langle{{#1}}\right\rangle}_{1}}
\newcommand{\gpgradeone}[1] {\gpgrade{#1}{1}}

\newcommand{\gpgradetwo}[1] {\gpgrade{#1}{2}}
\newcommand{\gpgradethree}[1] {\gpgrade{#1}{3}}
\newcommand{\gpgradefour}[1] {\gpgrade{#1}{4}}
%
% </grade selection>



\newcommand{\adot}[0]{{\dot{a}}}
\newcommand{\bdot}[0]{{\dot{b}}}
% taken for centered dot:
%\newcommand{\cdot}[0]{{\dot{c}}}
%\newcommand{\ddot}[0]{{\dot{d}}}
\newcommand{\edot}[0]{{\dot{e}}}
\newcommand{\fdot}[0]{{\dot{f}}}
\newcommand{\gdot}[0]{{\dot{g}}}
\newcommand{\hdot}[0]{{\dot{h}}}
\newcommand{\idot}[0]{{\dot{i}}}
\newcommand{\jdot}[0]{{\dot{j}}}
\newcommand{\kdot}[0]{{\dot{k}}}
\newcommand{\ldot}[0]{{\dot{l}}}
\newcommand{\mdot}[0]{{\dot{m}}}
\newcommand{\ndot}[0]{{\dot{n}}}
%\newcommand{\odot}[0]{{\dot{o}}}
\newcommand{\pdot}[0]{{\dot{p}}}
\newcommand{\qdot}[0]{{\dot{q}}}
\newcommand{\rdot}[0]{{\dot{r}}}
\newcommand{\sdot}[0]{{\dot{s}}}
\newcommand{\tdot}[0]{{\dot{t}}}
\newcommand{\udot}[0]{{\dot{u}}}
\newcommand{\vdot}[0]{{\dot{v}}}
\newcommand{\wdot}[0]{{\dot{w}}}
\newcommand{\xdot}[0]{{\dot{x}}}
\newcommand{\ydot}[0]{{\dot{y}}}
\newcommand{\zdot}[0]{{\dot{z}}}
\newcommand{\addot}[0]{{\ddot{a}}}
\newcommand{\bddot}[0]{{\ddot{b}}}
\newcommand{\cddot}[0]{{\ddot{c}}}
%\newcommand{\dddot}[0]{{\ddot{d}}}
\newcommand{\eddot}[0]{{\ddot{e}}}
\newcommand{\fddot}[0]{{\ddot{f}}}
\newcommand{\gddot}[0]{{\ddot{g}}}
\newcommand{\hddot}[0]{{\ddot{h}}}
\newcommand{\iddot}[0]{{\ddot{i}}}
\newcommand{\jddot}[0]{{\ddot{j}}}
\newcommand{\kddot}[0]{{\ddot{k}}}
\newcommand{\lddot}[0]{{\ddot{l}}}
\newcommand{\mddot}[0]{{\ddot{m}}}
\newcommand{\nddot}[0]{{\ddot{n}}}
\newcommand{\oddot}[0]{{\ddot{o}}}
\newcommand{\pddot}[0]{{\ddot{p}}}
\newcommand{\qddot}[0]{{\ddot{q}}}
\newcommand{\rddot}[0]{{\ddot{r}}}
\newcommand{\sddot}[0]{{\ddot{s}}}
\newcommand{\tddot}[0]{{\ddot{t}}}
\newcommand{\uddot}[0]{{\ddot{u}}}
\newcommand{\vddot}[0]{{\ddot{v}}}
\newcommand{\wddot}[0]{{\ddot{w}}}
\newcommand{\xddot}[0]{{\ddot{x}}}
\newcommand{\yddot}[0]{{\ddot{y}}}
\newcommand{\zddot}[0]{{\ddot{z}}}

%<bold and dot greek symbols>
%

\newcommand{\Deltadot}[0]{{\dot{\Delta}}}
\newcommand{\Gammadot}[0]{{\dot{\Gamma}}}
\newcommand{\Lambdadot}[0]{{\dot{\Lambda}}}
\newcommand{\Omegadot}[0]{{\dot{\Omega}}}
\newcommand{\Phidot}[0]{{\dot{\Phi}}}
\newcommand{\Pidot}[0]{{\dot{\Pi}}}
\newcommand{\Psidot}[0]{{\dot{\Psi}}}
\newcommand{\Sigmadot}[0]{{\dot{\Sigma}}}
\newcommand{\Thetadot}[0]{{\dot{\Theta}}}
\newcommand{\Upsilondot}[0]{{\dot{\Upsilon}}}
\newcommand{\Xidot}[0]{{\dot{\Xi}}}
\newcommand{\alphadot}[0]{{\dot{\alpha}}}
\newcommand{\betadot}[0]{{\dot{\beta}}}
\newcommand{\chidot}[0]{{\dot{\chi}}}
\newcommand{\deltadot}[0]{{\dot{\delta}}}
\newcommand{\epsilondot}[0]{{\dot{\epsilon}}}
\newcommand{\etadot}[0]{{\dot{\eta}}}
\newcommand{\gammadot}[0]{{\dot{\gamma}}}
\newcommand{\kappadot}[0]{{\dot{\kappa}}}
\newcommand{\lambdadot}[0]{{\dot{\lambda}}}
\newcommand{\mudot}[0]{{\dot{\mu}}}
\newcommand{\nudot}[0]{{\dot{\nu}}}
\newcommand{\omegadot}[0]{{\dot{\omega}}}
\newcommand{\phidot}[0]{{\dot{\phi}}}
\newcommand{\pidot}[0]{{\dot{\pi}}}
\newcommand{\psidot}[0]{{\dot{\psi}}}
\newcommand{\rhodot}[0]{{\dot{\rho}}}
\newcommand{\sigmadot}[0]{{\dot{\sigma}}}
\newcommand{\taudot}[0]{{\dot{\tau}}}
\newcommand{\thetadot}[0]{{\dot{\theta}}}
\newcommand{\upsilondot}[0]{{\dot{\upsilon}}}
\newcommand{\varepsilondot}[0]{{\dot{\varepsilon}}}
\newcommand{\varphidot}[0]{{\dot{\varphi}}}
\newcommand{\varpidot}[0]{{\dot{\varpi}}}
\newcommand{\varrhodot}[0]{{\dot{\varrho}}}
\newcommand{\varsigmadot}[0]{{\dot{\varsigma}}}
\newcommand{\varthetadot}[0]{{\dot{\vartheta}}}
\newcommand{\xidot}[0]{{\dot{\xi}}}
\newcommand{\zetadot}[0]{{\dot{\zeta}}}

\newcommand{\Deltaddot}[0]{{\ddot{\Delta}}}
\newcommand{\Gammaddot}[0]{{\ddot{\Gamma}}}
\newcommand{\Lambdaddot}[0]{{\ddot{\Lambda}}}
\newcommand{\Omegaddot}[0]{{\ddot{\Omega}}}
\newcommand{\Phiddot}[0]{{\ddot{\Phi}}}
\newcommand{\Piddot}[0]{{\ddot{\Pi}}}
\newcommand{\Psiddot}[0]{{\ddot{\Psi}}}
\newcommand{\Sigmaddot}[0]{{\ddot{\Sigma}}}
\newcommand{\Thetaddot}[0]{{\ddot{\Theta}}}
\newcommand{\Upsilonddot}[0]{{\ddot{\Upsilon}}}
\newcommand{\Xiddot}[0]{{\ddot{\Xi}}}
\newcommand{\alphaddot}[0]{{\ddot{\alpha}}}
\newcommand{\betaddot}[0]{{\ddot{\beta}}}
\newcommand{\chiddot}[0]{{\ddot{\chi}}}
\newcommand{\deltaddot}[0]{{\ddot{\delta}}}
\newcommand{\epsilonddot}[0]{{\ddot{\epsilon}}}
\newcommand{\etaddot}[0]{{\ddot{\eta}}}
\newcommand{\gammaddot}[0]{{\ddot{\gamma}}}
\newcommand{\kappaddot}[0]{{\ddot{\kappa}}}
\newcommand{\lambdaddot}[0]{{\ddot{\lambda}}}
\newcommand{\muddot}[0]{{\ddot{\mu}}}
\newcommand{\nuddot}[0]{{\ddot{\nu}}}
\newcommand{\omegaddot}[0]{{\ddot{\omega}}}
\newcommand{\phiddot}[0]{{\ddot{\phi}}}
\newcommand{\piddot}[0]{{\ddot{\pi}}}
\newcommand{\psiddot}[0]{{\ddot{\psi}}}
\newcommand{\rhoddot}[0]{{\ddot{\rho}}}
\newcommand{\sigmaddot}[0]{{\ddot{\sigma}}}
\newcommand{\tauddot}[0]{{\ddot{\tau}}}
\newcommand{\thetaddot}[0]{{\ddot{\theta}}}
\newcommand{\upsilonddot}[0]{{\ddot{\upsilon}}}
\newcommand{\varepsilonddot}[0]{{\ddot{\varepsilon}}}
\newcommand{\varphiddot}[0]{{\ddot{\varphi}}}
\newcommand{\varpiddot}[0]{{\ddot{\varpi}}}
\newcommand{\varrhoddot}[0]{{\ddot{\varrho}}}
\newcommand{\varsigmaddot}[0]{{\ddot{\varsigma}}}
\newcommand{\varthetaddot}[0]{{\ddot{\vartheta}}}
\newcommand{\xiddot}[0]{{\ddot{\xi}}}
\newcommand{\zetaddot}[0]{{\ddot{\zeta}}}

\newcommand{\BDelta}[0]{\boldsymbol{\Delta}}
\newcommand{\BGamma}[0]{\boldsymbol{\Gamma}}
\newcommand{\BLambda}[0]{\boldsymbol{\Lambda}}
\newcommand{\BOmega}[0]{\boldsymbol{\Omega}}
\newcommand{\BPhi}[0]{\boldsymbol{\Phi}}
\newcommand{\BPi}[0]{\boldsymbol{\Pi}}
\newcommand{\BPsi}[0]{\boldsymbol{\Psi}}
\newcommand{\BSigma}[0]{\boldsymbol{\Sigma}}
\newcommand{\BTheta}[0]{\boldsymbol{\Theta}}
\newcommand{\BUpsilon}[0]{\boldsymbol{\Upsilon}}
\newcommand{\BXi}[0]{\boldsymbol{\Xi}}
\newcommand{\Balpha}[0]{\boldsymbol{\alpha}}
\newcommand{\Bbeta}[0]{\boldsymbol{\beta}}
\newcommand{\Bchi}[0]{\boldsymbol{\chi}}
\newcommand{\Bdelta}[0]{\boldsymbol{\delta}}
\newcommand{\Bepsilon}[0]{\boldsymbol{\epsilon}}
\newcommand{\Beta}[0]{\boldsymbol{\eta}}
\newcommand{\Bgamma}[0]{\boldsymbol{\gamma}}
\newcommand{\Bkappa}[0]{\boldsymbol{\kappa}}
\newcommand{\Blambda}[0]{\boldsymbol{\lambda}}
\newcommand{\Bmu}[0]{\boldsymbol{\mu}}
\newcommand{\Bnu}[0]{\boldsymbol{\nu}}
%\newcommand{\Bomega}[0]{\boldsymbol{\omega}}
\newcommand{\Bphi}[0]{\boldsymbol{\phi}}
\newcommand{\Bpi}[0]{\boldsymbol{\pi}}
\newcommand{\Bpsi}[0]{\boldsymbol{\psi}}
\newcommand{\Brho}[0]{\boldsymbol{\rho}}
\newcommand{\Bsigma}[0]{\boldsymbol{\sigma}}
%\newcommand{\Btau}[0]{\boldsymbol{\tau}}
%\newcommand{\Btheta}[0]{\boldsymbol{\theta}}
\newcommand{\Bupsilon}[0]{\boldsymbol{\upsilon}}
\newcommand{\Bvarepsilon}[0]{\boldsymbol{\varepsilon}}
\newcommand{\Bvarphi}[0]{\boldsymbol{\varphi}}
\newcommand{\Bvarpi}[0]{\boldsymbol{\varpi}}
\newcommand{\Bvarrho}[0]{\boldsymbol{\varrho}}
\newcommand{\Bvarsigma}[0]{\boldsymbol{\varsigma}}
\newcommand{\Bvartheta}[0]{\boldsymbol{\vartheta}}
\newcommand{\Bxi}[0]{\boldsymbol{\xi}}
\newcommand{\Bzeta}[0]{\boldsymbol{\zeta}}
%
%</bold and dot greek symbols>
%<infrequent>
%
%\newcommand{\AreaOp}[1]{\AName_{#1}}
%\newcommand{\Babs}[0]{\abs{\BB}}
%\newcommand{\Bcap}[0]{\hat{\BB}}
%\newcommand{\BrPrimeRej}[0]{\rcap(\rcap \wedge \Br')}
%\newcommand{\CA}[0]{\mathcal{A}}
%\newcommand{\Cos}[1]{\cos{\left({#1}\right)}}
%\newcommand{\Det}[1] {\abs{#1}}
%\newcommand{\Dsq}[2] {\frac {\partial^2 {#1}} {\partial {#2}^2}}
%\newcommand{\Exp}[1]{\exp{\left({#1}\right)}}
%\newcommand{\Norm}[1]{\left\lVert{#1}\right\rVert}
%\newcommand{\Sin}[1]{\sin{\left({#1}\right)}}
%\newcommand{\T}[0]{\text{T}}
%\newcommand{\VolumeOp}[1]{\VName_{#1}}
%\newcommand{\agrad}[0]{\Ba \cdot \nabla}
%\newcommand{\alphacap}[0]{\hat{\boldsymbol{\alpha}}}
%\newcommand{\Fcap}[0]{\hat{\BF}}
%\newcommand{\bithree}[0]{{\Bi}_3}
%\newcommand{\bxa}[0]{\Bx\Ba}
%\newcommand{\coordvec}[2]{
%\newcommand{\costheta}[0]{\acap \cdot \xcap}
%\newcommand{\ddt}[1]{\ddot{#1}}
%\newcommand{\ddu}[1] {\frac {d{#1}} {du}}
%\newcommand{\dsqxj}[2] {\frac {\partial^2 {#1}} {\partial {x_{#2}}^2}}
%\newcommand{\dtheta}[1]{\frac{d {#1}}{d \theta}}
%\newcommand{\dt}[1]{\dot{#1}}
%\newcommand{\dt}[1]{\frac{d {#1}}{dt}}
%\newcommand{\dxj}[2] {\frac {\partial {#1}} {\partial {x_{#2}}}}
%\newcommand{\halfPhi}[0]{\frac{\phi}{2}}
%\newcommand{\half}[0]{\inv{2}}
%\newcommand{\inv}[1]{\frac{1}{#1}}
%\newcommand{\laplacian}[0]{\nabla^2}
%\newcommand{\matrixoftx}[3]{
%\newcommand{\nrrp}[0]{\norm{\rcap \wedge \Br'}}
%\newcommand{\oiint}{\bigcirc \hspace{-1.4em} \int \hspace{-.8em} \int}
%\newcommand{\transpose}[1]{{#1}^{\text{T}}}
%\newcommand{\transpose}[1]{{{#1}^{\TextTranspose}}}
%\newcommand{\transpose}[1]{{{#1}^{\text{T}}}}
%\newcommand{\barA}[0]{\bar{A}}
%\newcommand{\qbar}[0]{\bar{q}}
%\newcommand{\qdotbar}[0]{\dot{\bar{q}}}
%
%</infrequent>





%\usepackage{listings}
%\usepackage{txfonts} % for ointctr... (also appears to make "prettier" \int and \sum's)
\usepackage[bookmarks=true]{hyperref}

\usepackage{color,cite,graphicx}
   % use colour in the document, put your citations as [1-4]
   % rather than [1,2,3,4] (it looks nicer, and the extended LaTeX2e
   % graphics package. 
\usepackage{latexsym,amssymb,epsf} % don't remember if these are
   % needed, but their inclusion can't do any damage


\title{ Relativistic acceleration. }
\author{Peeter Joot \quad peeter.joot@gmail.com }
\date{ April 10, 2009.  Last Revision: $Date: 2009/04/11 01:00:47 $ }

\begin{document}

\maketitle{}
\tableofcontents
\section{ Motivation. }

Continuing on with reading of \cite{pauli1981tr}, having 
clarified aspects of the four vector velocity in \cite{PJrelativityFourVectorVelocity}, it is now
time to move on to acceleration.

Do the chain rule calculations for the acceleration four vector equation given in equation (193).

\section{ Compute it. }

Compute the spatial and timelike components of the acceleration

\begin{align*}
B^\mu 
&= \frac{d^2 x^\mu}{d\tau^2} \\
&= \frac{d }{d\tau} \left( \frac{d x^\mu }{d\tau} \right) \\
&= \frac{d }{d\tau} \left( \frac{d x^\mu }{dt} \frac{dt}{d\tau} \right) \\
&= \left( \frac{d }{d\tau} \frac{d x^\mu }{dt} \right) \frac{dt}{d\tau} + \frac{d x^\mu }{dt} \frac{d^2t}{d\tau^2} \\
&= \frac{d^2 x^\mu }{dt^2} \left( \frac{dt}{d\tau} \right)^2 + \frac{d x^\mu }{dt} \frac{d^2t}{d\tau^2} \\
\end{align*}

For $\mu \in \{1,2,3\}$, the ${d^2 x^\mu }/{dt^2}$ terms are the regular old spatial acceleration components.
, and $dx^4/dt = c$.  Writing $\Bu^2 = \sum_{k=1}^3 (dx^k/dt)^2$, and $\beta^2 = \Bu^2/c^2$, we have

\begin{align*}
B^k &= \frac{d^2 x^k }{dt^2} \inv{1-\beta^2} + \frac{d x^k }{dt} \frac{d^2t}{d\tau^2} \\
B^4 &= 0 + c \frac{d^2t}{d\tau^2} \\
\end{align*}

In both of these is the $d^2t/d\tau^2$ term.  Let's expand that.

\begin{align*}
\frac{d^2t}{d\tau^2} 
&= \frac{d}{d\tau} \left( \inv{\sqrt{ 1 - \Bu^2/c^2 }} \right) \\
&= \frac{-1}{c^2} \frac{(-1/2) }{({ 1 - \Bu^2/c^2 })^{3/2}} \frac{d\Bu^2}{d\tau} \\
&= \frac{1}{c^2} \frac{(1/2) }{({ 1 - \Bu^2/c^2 })^{3/2}} 2 \Bu \cdot \frac{d\Bu}{d\tau} \\
&= \frac{1}{c^2} \frac{1}{({ 1 - \Bu^2/c^2 })^{3/2}} \Bu \cdot \frac{d\Bu}{dt} \frac{dt}{d\tau} \\
&= \frac{1}{c^2} \frac{1}{({ 1 - \Bu^2/c^2 })^{2}} \Bu \cdot \frac{d\Bu}{dt} \\
\end{align*}

In vector form, with $\Ba = d\Bu/dt$, we now have the following 

\begin{align*}
\BB &= \Ba \inv{1-\beta^2} + \Bu (\Bu \cdot \Ba) \inv{c^2} \inv{ (1-\beta^2)^2} \\
B^4 &= \inv{c} (\Bu \cdot \Ba) \inv{ (1-\beta^2)^2}
\end{align*}

This reproduces the equation from the Pauli text (except for the imaginary factor $i$ due to the Minkowski notation).  Except for $\gamma$ factors this calculation has a similar final form to that of the decomposition of acceleration in terms of radial components.

\bibliographystyle{plainnat}
\bibliography{myrefs}

\end{document}

%\documentclass{article}

%\usepackage{amsmath}
\usepackage{mathpazo}

%
% shorthand for bold symbols, convenient for vectors and matrices
%
\newcommand{\Ba}[0]{\mathbf{a}}
\newcommand{\Bb}[0]{\mathbf{b}}
\newcommand{\Bc}[0]{\mathbf{c}}
\newcommand{\Bd}[0]{\mathbf{d}}
\newcommand{\Be}[0]{\mathbf{e}}
\newcommand{\Bf}[0]{\mathbf{f}}
\newcommand{\Bg}[0]{\mathbf{g}}
\newcommand{\Bh}[0]{\mathbf{h}}
\newcommand{\Bi}[0]{\mathbf{i}}
\newcommand{\Bj}[0]{\mathbf{j}}
\newcommand{\Bk}[0]{\mathbf{k}}
\newcommand{\Bl}[0]{\mathbf{l}}
\newcommand{\Bm}[0]{\mathbf{m}}
\newcommand{\Bn}[0]{\mathbf{n}}
\newcommand{\Bo}[0]{\mathbf{o}}
\newcommand{\Bp}[0]{\mathbf{p}}
\newcommand{\Bq}[0]{\mathbf{q}}
\newcommand{\Br}[0]{\mathbf{r}}
\newcommand{\Bs}[0]{\mathbf{s}}
\newcommand{\Bt}[0]{\mathbf{t}}
\newcommand{\Bu}[0]{\mathbf{u}}
\newcommand{\Bv}[0]{\mathbf{v}}
\newcommand{\Bw}[0]{\mathbf{w}}
\newcommand{\Bx}[0]{\mathbf{x}}
\newcommand{\By}[0]{\mathbf{y}}
\newcommand{\Bz}[0]{\mathbf{z}}
\newcommand{\BA}[0]{\mathbf{A}}
\newcommand{\BB}[0]{\mathbf{B}}
\newcommand{\BC}[0]{\mathbf{C}}
\newcommand{\BD}[0]{\mathbf{D}}
\newcommand{\BE}[0]{\mathbf{E}}
\newcommand{\BF}[0]{\mathbf{F}}
\newcommand{\BG}[0]{\mathbf{G}}
\newcommand{\BH}[0]{\mathbf{H}}
\newcommand{\BI}[0]{\mathbf{I}}
\newcommand{\BJ}[0]{\mathbf{J}}
\newcommand{\BK}[0]{\mathbf{K}}
\newcommand{\BL}[0]{\mathbf{L}}
\newcommand{\BM}[0]{\mathbf{M}}
\newcommand{\BN}[0]{\mathbf{N}}
\newcommand{\BO}[0]{\mathbf{O}}
\newcommand{\BP}[0]{\mathbf{P}}
\newcommand{\BQ}[0]{\mathbf{Q}}
\newcommand{\BR}[0]{\mathbf{R}}
\newcommand{\BS}[0]{\mathbf{S}}
\newcommand{\BT}[0]{\mathbf{T}}
\newcommand{\BU}[0]{\mathbf{U}}
\newcommand{\BV}[0]{\mathbf{V}}
\newcommand{\BW}[0]{\mathbf{W}}
\newcommand{\BX}[0]{\mathbf{X}}
\newcommand{\BY}[0]{\mathbf{Y}}
\newcommand{\BZ}[0]{\mathbf{Z}}

\newcommand{\Bzero}[0]{\mathbf{0}}
\newcommand{\Btheta}[0]{\boldsymbol{\theta}}
\newcommand{\Btau}[0]{\boldsymbol{\tau}}
\newcommand{\Bomega}[0]{\boldsymbol{\omega}}

%
% shorthand for unit vectors
%
\newcommand{\acap}[0]{\hat{\Ba}}
\newcommand{\bcap}[0]{\hat{\Bb}}
\newcommand{\ccap}[0]{\hat{\Bc}}
\newcommand{\dcap}[0]{\hat{\Bd}}
\newcommand{\ecap}[0]{\hat{\Be}}
\newcommand{\fcap}[0]{\hat{\Bf}}
\newcommand{\gcap}[0]{\hat{\Bg}}
\newcommand{\hcap}[0]{\hat{\Bh}}
\newcommand{\icap}[0]{\hat{\Bi}}
\newcommand{\jcap}[0]{\hat{\Bj}}
\newcommand{\kcap}[0]{\hat{\Bk}}
\newcommand{\lcap}[0]{\hat{\Bl}}
\newcommand{\mcap}[0]{\hat{\Bm}}
\newcommand{\ncap}[0]{\hat{\Bn}}
\newcommand{\ocap}[0]{\hat{\Bo}}
\newcommand{\pcap}[0]{\hat{\Bp}}
\newcommand{\qcap}[0]{\hat{\Bq}}
\newcommand{\rcap}[0]{\hat{\Br}}
\newcommand{\scap}[0]{\hat{\Bs}}
\newcommand{\tcap}[0]{\hat{\Bt}}
\newcommand{\ucap}[0]{\hat{\Bu}}
\newcommand{\vcap}[0]{\hat{\Bv}}
\newcommand{\wcap}[0]{\hat{\Bw}}
\newcommand{\xcap}[0]{\hat{\Bx}}
\newcommand{\ycap}[0]{\hat{\By}}
\newcommand{\zcap}[0]{\hat{\Bz}}
\newcommand{\thetacap}[0]{\hat{\Btheta}}

%
% to write R^n and C^n in a distinguishable fashion.  Perhaps change this
% to the double lined characters upon figuring out how to do so.
%
\newcommand{\C}[1]{$\mathbb{C}^{#1}$}
\newcommand{\R}[1]{$\mathbb{R}^{#1}$}

%
% various generally useful helpers
%

% derivative of #1 wrt. #2:
\newcommand{\D}[2] {\frac {d#2} {d#1}}

\newcommand{\inv}[1]{\frac{1}{#1}}
\newcommand{\cross}[0]{\times}

\newcommand{\abs}[1]{\lvert{#1}\rvert}
\newcommand{\norm}[1]{\lVert{#1}\rVert}
\newcommand{\innerprod}[2]{\langle{#1}, {#2}\rangle}
\newcommand{\dotprod}[2]{{#1} \cdot {#2}}
\newcommand{\bdotprod}[2]{\left({#1} \cdot {#2}\right)}
\newcommand{\crossprod}[2]{{#1} \cross {#2}}
\newcommand{\tripleprod}[3]{\dotprod{\left(\crossprod{#1}{#2}\right)}{#3}}

\DeclareMathOperator{\Proj}{Proj}
\DeclareMathOperator{\Span}{span}
\DeclareMathOperator{\Sgn}{sgn}
\DeclareMathOperator{\Area}{Area}
\DeclareMathOperator{\Volume}{Volume}

%
% A few miscellaneous things specific to this document
%
\newcommand{\crossop}[1]{\crossprod{#1}{}}

% R2 vector.
\newcommand{\VectorTwo}[2]{
\begin{bmatrix}
 {#1} \\
 {#2}
\end{bmatrix}
}

\newcommand{\VectorN}[1]{
\begin{bmatrix}
{#1}_1 \\
{#1}_2 \\
\vdots \\
{#1}_N \\
\end{bmatrix}
}

\newcommand{\DETuvij}[4]{
\begin{vmatrix}
 {#1}_{#3} & {#1}_{#4} \\
 {#2}_{#3} & {#2}_{#4}
\end{vmatrix}
}

\newcommand{\DETuvwijk}[6]{
\begin{vmatrix}
 {#1}_{#4} & {#1}_{#5} & {#1}_{#6} \\
 {#2}_{#4} & {#2}_{#5} & {#2}_{#6} \\
 {#3}_{#4} & {#3}_{#5} & {#3}_{#6}
\end{vmatrix}
}

\newcommand{\DETuvwxijkl}[8]{
\begin{vmatrix}
 {#1}_{#5} & {#1}_{#6} & {#1}_{#7} & {#1}_{#8} \\
 {#2}_{#5} & {#2}_{#6} & {#2}_{#7} & {#2}_{#8} \\
 {#3}_{#5} & {#3}_{#6} & {#3}_{#7} & {#3}_{#8} \\
 {#4}_{#5} & {#4}_{#6} & {#4}_{#7} & {#4}_{#8} \\
\end{vmatrix}
}

%\newcommand{\DETuvwxyijklm}[10]{
%\begin{vmatrix}
% {#1}_{#6} & {#1}_{#7} & {#1}_{#8} & {#1}_{#9} & {#1}_{#10} \\
% {#2}_{#6} & {#2}_{#7} & {#2}_{#8} & {#2}_{#9} & {#2}_{#10} \\
% {#3}_{#6} & {#3}_{#7} & {#3}_{#8} & {#3}_{#9} & {#3}_{#10} \\
% {#4}_{#6} & {#4}_{#7} & {#4}_{#8} & {#4}_{#9} & {#4}_{#10} \\
% {#5}_{#6} & {#5}_{#7} & {#5}_{#8} & {#5}_{#9} & {#5}_{#10}
%\end{vmatrix}
%}

% R3 vector.
\newcommand{\VectorThree}[3]{
\begin{bmatrix}
 {#1} \\
 {#2} \\
 {#3}
\end{bmatrix}
}


%%<misc>
%
\newcommand{\Abs}[1]{{\left\lvert{#1}\right\rvert}}
\newcommand{\spacegrad}[0]{\boldsymbol{\nabla}}
\newcommand{\grad}[0]{\nabla}
\newcommand{\LL}[0]{\mathcal{L}}

% == \partial_{#1} {#2}
\newcommand{\PD}[2]{\frac{\partial {#2}}{\partial {#1}}}
% inline variant
\newcommand{\PDi}[2]{{\partial {#2}}/{\partial {#1}}}

\newcommand{\PDD}[3]{\frac{\partial^2 {#3}}{\partial {#1}\partial {#2}}}
%\newcommand{\PDd}[2]{\frac{\partial^2 {#2}}{{\partial{#1}}^2}}
\newcommand{\PDsq}[2]{\frac{\partial^2 {#2}}{(\partial {#1})^2}}

\newcommand{\Partial}[2]{\frac{\partial {#1}}{\partial {#2}}}
\DeclareMathOperator{\RejName}{Rej}
\newcommand{\Rej}[2]{\RejName_{#1}\left( {#2} \right)}
\newcommand{\Rm}[1]{\mathbb{R}^{#1}}
\newcommand{\Cm}[1]{\mathbb{C}^{#1}}
\newcommand{\conj}[0]{{*}}

%</misc>

% <grade selection>
%
\newcommand{\gpgrade}[2] {{\left\langle{{#1}}\right\rangle}_{#2}}

\newcommand{\gpgradezero}[1] {\gpgrade{#1}{}}
%\newcommand{\gpscalargrade}[1] {{\left\langle{{#1}}\right\rangle}}
%\newcommand{\gpgradezero}[1] {\gpgrade{#1}{0}}

%\newcommand{\gpgradeone}[1] {{\left\langle{{#1}}\right\rangle}_{1}}
\newcommand{\gpgradeone}[1] {\gpgrade{#1}{1}}

\newcommand{\gpgradetwo}[1] {\gpgrade{#1}{2}}
\newcommand{\gpgradethree}[1] {\gpgrade{#1}{3}}
\newcommand{\gpgradefour}[1] {\gpgrade{#1}{4}}
%
% </grade selection>



\newcommand{\adot}[0]{{\dot{a}}}
\newcommand{\bdot}[0]{{\dot{b}}}
% taken for centered dot:
%\newcommand{\cdot}[0]{{\dot{c}}}
%\newcommand{\ddot}[0]{{\dot{d}}}
\newcommand{\edot}[0]{{\dot{e}}}
\newcommand{\fdot}[0]{{\dot{f}}}
\newcommand{\gdot}[0]{{\dot{g}}}
\newcommand{\hdot}[0]{{\dot{h}}}
\newcommand{\idot}[0]{{\dot{i}}}
\newcommand{\jdot}[0]{{\dot{j}}}
\newcommand{\kdot}[0]{{\dot{k}}}
\newcommand{\ldot}[0]{{\dot{l}}}
\newcommand{\mdot}[0]{{\dot{m}}}
\newcommand{\ndot}[0]{{\dot{n}}}
%\newcommand{\odot}[0]{{\dot{o}}}
\newcommand{\pdot}[0]{{\dot{p}}}
\newcommand{\qdot}[0]{{\dot{q}}}
\newcommand{\rdot}[0]{{\dot{r}}}
\newcommand{\sdot}[0]{{\dot{s}}}
\newcommand{\tdot}[0]{{\dot{t}}}
\newcommand{\udot}[0]{{\dot{u}}}
\newcommand{\vdot}[0]{{\dot{v}}}
\newcommand{\wdot}[0]{{\dot{w}}}
\newcommand{\xdot}[0]{{\dot{x}}}
\newcommand{\ydot}[0]{{\dot{y}}}
\newcommand{\zdot}[0]{{\dot{z}}}
\newcommand{\addot}[0]{{\ddot{a}}}
\newcommand{\bddot}[0]{{\ddot{b}}}
\newcommand{\cddot}[0]{{\ddot{c}}}
%\newcommand{\dddot}[0]{{\ddot{d}}}
\newcommand{\eddot}[0]{{\ddot{e}}}
\newcommand{\fddot}[0]{{\ddot{f}}}
\newcommand{\gddot}[0]{{\ddot{g}}}
\newcommand{\hddot}[0]{{\ddot{h}}}
\newcommand{\iddot}[0]{{\ddot{i}}}
\newcommand{\jddot}[0]{{\ddot{j}}}
\newcommand{\kddot}[0]{{\ddot{k}}}
\newcommand{\lddot}[0]{{\ddot{l}}}
\newcommand{\mddot}[0]{{\ddot{m}}}
\newcommand{\nddot}[0]{{\ddot{n}}}
\newcommand{\oddot}[0]{{\ddot{o}}}
\newcommand{\pddot}[0]{{\ddot{p}}}
\newcommand{\qddot}[0]{{\ddot{q}}}
\newcommand{\rddot}[0]{{\ddot{r}}}
\newcommand{\sddot}[0]{{\ddot{s}}}
\newcommand{\tddot}[0]{{\ddot{t}}}
\newcommand{\uddot}[0]{{\ddot{u}}}
\newcommand{\vddot}[0]{{\ddot{v}}}
\newcommand{\wddot}[0]{{\ddot{w}}}
\newcommand{\xddot}[0]{{\ddot{x}}}
\newcommand{\yddot}[0]{{\ddot{y}}}
\newcommand{\zddot}[0]{{\ddot{z}}}

%<bold and dot greek symbols>
%

\newcommand{\Deltadot}[0]{{\dot{\Delta}}}
\newcommand{\Gammadot}[0]{{\dot{\Gamma}}}
\newcommand{\Lambdadot}[0]{{\dot{\Lambda}}}
\newcommand{\Omegadot}[0]{{\dot{\Omega}}}
\newcommand{\Phidot}[0]{{\dot{\Phi}}}
\newcommand{\Pidot}[0]{{\dot{\Pi}}}
\newcommand{\Psidot}[0]{{\dot{\Psi}}}
\newcommand{\Sigmadot}[0]{{\dot{\Sigma}}}
\newcommand{\Thetadot}[0]{{\dot{\Theta}}}
\newcommand{\Upsilondot}[0]{{\dot{\Upsilon}}}
\newcommand{\Xidot}[0]{{\dot{\Xi}}}
\newcommand{\alphadot}[0]{{\dot{\alpha}}}
\newcommand{\betadot}[0]{{\dot{\beta}}}
\newcommand{\chidot}[0]{{\dot{\chi}}}
\newcommand{\deltadot}[0]{{\dot{\delta}}}
\newcommand{\epsilondot}[0]{{\dot{\epsilon}}}
\newcommand{\etadot}[0]{{\dot{\eta}}}
\newcommand{\gammadot}[0]{{\dot{\gamma}}}
\newcommand{\kappadot}[0]{{\dot{\kappa}}}
\newcommand{\lambdadot}[0]{{\dot{\lambda}}}
\newcommand{\mudot}[0]{{\dot{\mu}}}
\newcommand{\nudot}[0]{{\dot{\nu}}}
\newcommand{\omegadot}[0]{{\dot{\omega}}}
\newcommand{\phidot}[0]{{\dot{\phi}}}
\newcommand{\pidot}[0]{{\dot{\pi}}}
\newcommand{\psidot}[0]{{\dot{\psi}}}
\newcommand{\rhodot}[0]{{\dot{\rho}}}
\newcommand{\sigmadot}[0]{{\dot{\sigma}}}
\newcommand{\taudot}[0]{{\dot{\tau}}}
\newcommand{\thetadot}[0]{{\dot{\theta}}}
\newcommand{\upsilondot}[0]{{\dot{\upsilon}}}
\newcommand{\varepsilondot}[0]{{\dot{\varepsilon}}}
\newcommand{\varphidot}[0]{{\dot{\varphi}}}
\newcommand{\varpidot}[0]{{\dot{\varpi}}}
\newcommand{\varrhodot}[0]{{\dot{\varrho}}}
\newcommand{\varsigmadot}[0]{{\dot{\varsigma}}}
\newcommand{\varthetadot}[0]{{\dot{\vartheta}}}
\newcommand{\xidot}[0]{{\dot{\xi}}}
\newcommand{\zetadot}[0]{{\dot{\zeta}}}

\newcommand{\Deltaddot}[0]{{\ddot{\Delta}}}
\newcommand{\Gammaddot}[0]{{\ddot{\Gamma}}}
\newcommand{\Lambdaddot}[0]{{\ddot{\Lambda}}}
\newcommand{\Omegaddot}[0]{{\ddot{\Omega}}}
\newcommand{\Phiddot}[0]{{\ddot{\Phi}}}
\newcommand{\Piddot}[0]{{\ddot{\Pi}}}
\newcommand{\Psiddot}[0]{{\ddot{\Psi}}}
\newcommand{\Sigmaddot}[0]{{\ddot{\Sigma}}}
\newcommand{\Thetaddot}[0]{{\ddot{\Theta}}}
\newcommand{\Upsilonddot}[0]{{\ddot{\Upsilon}}}
\newcommand{\Xiddot}[0]{{\ddot{\Xi}}}
\newcommand{\alphaddot}[0]{{\ddot{\alpha}}}
\newcommand{\betaddot}[0]{{\ddot{\beta}}}
\newcommand{\chiddot}[0]{{\ddot{\chi}}}
\newcommand{\deltaddot}[0]{{\ddot{\delta}}}
\newcommand{\epsilonddot}[0]{{\ddot{\epsilon}}}
\newcommand{\etaddot}[0]{{\ddot{\eta}}}
\newcommand{\gammaddot}[0]{{\ddot{\gamma}}}
\newcommand{\kappaddot}[0]{{\ddot{\kappa}}}
\newcommand{\lambdaddot}[0]{{\ddot{\lambda}}}
\newcommand{\muddot}[0]{{\ddot{\mu}}}
\newcommand{\nuddot}[0]{{\ddot{\nu}}}
\newcommand{\omegaddot}[0]{{\ddot{\omega}}}
\newcommand{\phiddot}[0]{{\ddot{\phi}}}
\newcommand{\piddot}[0]{{\ddot{\pi}}}
\newcommand{\psiddot}[0]{{\ddot{\psi}}}
\newcommand{\rhoddot}[0]{{\ddot{\rho}}}
\newcommand{\sigmaddot}[0]{{\ddot{\sigma}}}
\newcommand{\tauddot}[0]{{\ddot{\tau}}}
\newcommand{\thetaddot}[0]{{\ddot{\theta}}}
\newcommand{\upsilonddot}[0]{{\ddot{\upsilon}}}
\newcommand{\varepsilonddot}[0]{{\ddot{\varepsilon}}}
\newcommand{\varphiddot}[0]{{\ddot{\varphi}}}
\newcommand{\varpiddot}[0]{{\ddot{\varpi}}}
\newcommand{\varrhoddot}[0]{{\ddot{\varrho}}}
\newcommand{\varsigmaddot}[0]{{\ddot{\varsigma}}}
\newcommand{\varthetaddot}[0]{{\ddot{\vartheta}}}
\newcommand{\xiddot}[0]{{\ddot{\xi}}}
\newcommand{\zetaddot}[0]{{\ddot{\zeta}}}

\newcommand{\BDelta}[0]{\boldsymbol{\Delta}}
\newcommand{\BGamma}[0]{\boldsymbol{\Gamma}}
\newcommand{\BLambda}[0]{\boldsymbol{\Lambda}}
\newcommand{\BOmega}[0]{\boldsymbol{\Omega}}
\newcommand{\BPhi}[0]{\boldsymbol{\Phi}}
\newcommand{\BPi}[0]{\boldsymbol{\Pi}}
\newcommand{\BPsi}[0]{\boldsymbol{\Psi}}
\newcommand{\BSigma}[0]{\boldsymbol{\Sigma}}
\newcommand{\BTheta}[0]{\boldsymbol{\Theta}}
\newcommand{\BUpsilon}[0]{\boldsymbol{\Upsilon}}
\newcommand{\BXi}[0]{\boldsymbol{\Xi}}
\newcommand{\Balpha}[0]{\boldsymbol{\alpha}}
\newcommand{\Bbeta}[0]{\boldsymbol{\beta}}
\newcommand{\Bchi}[0]{\boldsymbol{\chi}}
\newcommand{\Bdelta}[0]{\boldsymbol{\delta}}
\newcommand{\Bepsilon}[0]{\boldsymbol{\epsilon}}
\newcommand{\Beta}[0]{\boldsymbol{\eta}}
\newcommand{\Bgamma}[0]{\boldsymbol{\gamma}}
\newcommand{\Bkappa}[0]{\boldsymbol{\kappa}}
\newcommand{\Blambda}[0]{\boldsymbol{\lambda}}
\newcommand{\Bmu}[0]{\boldsymbol{\mu}}
\newcommand{\Bnu}[0]{\boldsymbol{\nu}}
%\newcommand{\Bomega}[0]{\boldsymbol{\omega}}
\newcommand{\Bphi}[0]{\boldsymbol{\phi}}
\newcommand{\Bpi}[0]{\boldsymbol{\pi}}
\newcommand{\Bpsi}[0]{\boldsymbol{\psi}}
\newcommand{\Brho}[0]{\boldsymbol{\rho}}
\newcommand{\Bsigma}[0]{\boldsymbol{\sigma}}
%\newcommand{\Btau}[0]{\boldsymbol{\tau}}
%\newcommand{\Btheta}[0]{\boldsymbol{\theta}}
\newcommand{\Bupsilon}[0]{\boldsymbol{\upsilon}}
\newcommand{\Bvarepsilon}[0]{\boldsymbol{\varepsilon}}
\newcommand{\Bvarphi}[0]{\boldsymbol{\varphi}}
\newcommand{\Bvarpi}[0]{\boldsymbol{\varpi}}
\newcommand{\Bvarrho}[0]{\boldsymbol{\varrho}}
\newcommand{\Bvarsigma}[0]{\boldsymbol{\varsigma}}
\newcommand{\Bvartheta}[0]{\boldsymbol{\vartheta}}
\newcommand{\Bxi}[0]{\boldsymbol{\xi}}
\newcommand{\Bzeta}[0]{\boldsymbol{\zeta}}
%
%</bold and dot greek symbols>
%<infrequent>
%
%\newcommand{\AreaOp}[1]{\AName_{#1}}
%\newcommand{\Babs}[0]{\abs{\BB}}
%\newcommand{\Bcap}[0]{\hat{\BB}}
%\newcommand{\BrPrimeRej}[0]{\rcap(\rcap \wedge \Br')}
%\newcommand{\CA}[0]{\mathcal{A}}
%\newcommand{\Cos}[1]{\cos{\left({#1}\right)}}
%\newcommand{\Det}[1] {\abs{#1}}
%\newcommand{\Dsq}[2] {\frac {\partial^2 {#1}} {\partial {#2}^2}}
%\newcommand{\Exp}[1]{\exp{\left({#1}\right)}}
%\newcommand{\Norm}[1]{\left\lVert{#1}\right\rVert}
%\newcommand{\Sin}[1]{\sin{\left({#1}\right)}}
%\newcommand{\T}[0]{\text{T}}
%\newcommand{\VolumeOp}[1]{\VName_{#1}}
%\newcommand{\agrad}[0]{\Ba \cdot \nabla}
%\newcommand{\alphacap}[0]{\hat{\boldsymbol{\alpha}}}
%\newcommand{\Fcap}[0]{\hat{\BF}}
%\newcommand{\bithree}[0]{{\Bi}_3}
%\newcommand{\bxa}[0]{\Bx\Ba}
%\newcommand{\coordvec}[2]{
%\newcommand{\costheta}[0]{\acap \cdot \xcap}
%\newcommand{\ddt}[1]{\ddot{#1}}
%\newcommand{\ddu}[1] {\frac {d{#1}} {du}}
%\newcommand{\dsqxj}[2] {\frac {\partial^2 {#1}} {\partial {x_{#2}}^2}}
%\newcommand{\dtheta}[1]{\frac{d {#1}}{d \theta}}
%\newcommand{\dt}[1]{\dot{#1}}
%\newcommand{\dt}[1]{\frac{d {#1}}{dt}}
%\newcommand{\dxj}[2] {\frac {\partial {#1}} {\partial {x_{#2}}}}
%\newcommand{\halfPhi}[0]{\frac{\phi}{2}}
%\newcommand{\half}[0]{\inv{2}}
%\newcommand{\inv}[1]{\frac{1}{#1}}
%\newcommand{\laplacian}[0]{\nabla^2}
%\newcommand{\matrixoftx}[3]{
%\newcommand{\nrrp}[0]{\norm{\rcap \wedge \Br'}}
%\newcommand{\oiint}{\bigcirc \hspace{-1.4em} \int \hspace{-.8em} \int}
%\newcommand{\transpose}[1]{{#1}^{\text{T}}}
%\newcommand{\transpose}[1]{{{#1}^{\TextTranspose}}}
%\newcommand{\transpose}[1]{{{#1}^{\text{T}}}}
%\newcommand{\barA}[0]{\bar{A}}
%\newcommand{\qbar}[0]{\bar{q}}
%\newcommand{\qdotbar}[0]{\dot{\bar{q}}}
%
%</infrequent>





%\usepackage{listings}
%\usepackage{txfonts} % for ointctr... (also appears to make "prettier" \int and \sum's)
%\usepackage[bookmarks=true]{hyperref}

%\usepackage{color,cite,graphicx}
   % use colour in the document, put your citations as [1-4]
   % rather than [1,2,3,4] (it looks nicer, and the extended LaTeX2e
   % graphics package. 
%\usepackage{latexsym,amssymb,epsf} % don't remember if these are
   % needed, but their inclusion can't do any damage


\chapter{Four vector velocity addition notes. }
\label{chap:pauliFourVectorV}
%\author{Peeter Joot \quad peeter.joot@gmail.com }
\date{ April 8, 2009.  $RCSfile: pauliFourVectorV.tex,v $ Last $Revision: 1.14 $ $Date: 2009/06/14 23:51:45 $ }

%\begin{document}

%\maketitle{}
%\tableofcontents
\section{Motivation. }

Reconcile four vector transformed velocity coordinates with non-covariant form.
Specifically, equations (10) and (191) in \citep{pauli1981tr} look considerably
different on the surface, but must have the same content.

Equations (10) were also derived in a bit more detail than in Pauli's book in
\citep{PJpauliVelocityAddition} and are

\begin{align}\label{eqn:pauli_four_vector_v:eqn10}
u_x &= \frac{ {u_x}' + v  }{ 1 + v {u_x}'/c^2} \\
u_y &= \frac{{u_y}'}{\gamma (1 + v {u_x}'/c^2)} \\
u_z &= \frac{{u_z}'}{\gamma (1 + v {u_x}'/c^2)} \\
\gamma^{-1} &= \sqrt{ 1 - v^2/c^2}
\end{align}

whereas equations (191) are given as

\begin{align}\label{eqn:pauli_four_vector_v:eqn191}
{u^1}' &= \gamma ( u^1 + i (v/c) u^4) \\
{u^2}' &= {u^2} \\
{u^3}' &= {u^3} \\
{u^4}' &= \gamma ( u^4 - i (v/c) u^1)
\end{align}

\section{Derive the transformed velocity equations. }

Pauli uses a $(+,+,+,-)$ metric, with $ct = x^4 = - x_4$.  For much of his SR treatment he also uses the Minkowski representation $x^4 = x_4 = ict$. In the first representation we have

\begin{align*}
-c^2 
&= \frac{dx^\mu}{d\tau} \frac{dx_\mu}{d\tau} \\
&= \frac{dx^k}{d\tau} \frac{dx_k}{d\tau} + \frac{dx^4}{d\tau} \frac{dx_4}{d\tau} \\
&= \left(\frac{dt}{d\tau}\right)^2 \left( \sum_{k=1}^3 \left(\frac{dx^k}{dt}\right)^2 - \left(\frac{dx^4}{dt}\right)^2 \right) \\
&= \left(\frac{dt}{d\tau}\right)^2 \left( \Bu^2 - c^2 \right) \\
\end{align*}

Shuffling and taking roots produces a $\gamma$ factor by virtue of the invariant

\begin{align*}
%\gamma 
%&= 
\frac{dt}{d\tau} \\
&= \inv{\sqrt{ 1 - \Bu^2/c^2 }}
\end{align*}

This is enough to write the proper velocity in terms of a space time split 

\begin{align*}
\dot{X} 
&= \left(\frac{dx^\mu}{d\tau}\right) \\
&= \inv{\sqrt{ 1 - \Bu^2/c^2 }} (\Bu, c)
\end{align*}

As a four vector this can be Lorentz boosted.  For an 
x-axis boost we have

\begin{align*}
{\begin{bmatrix}
u^1 \\
u^2 \\
u^3 \\
u^4 \\
\end{bmatrix}}'
&=
\begin{bmatrix}
\gamma & 0 & 0 & - \gamma \beta \\
0 & 1 & 0 & 0 \\
0 & 0 & 1 & 0 \\
- \gamma \beta & 0 & 0 & \gamma \\
\end{bmatrix}
{\begin{bmatrix}
u^1 \\
u^2 \\
u^3 \\
u^4 \\
\end{bmatrix}} 
\\
\gamma &= \inv{\sqrt{ 1 - \beta^2 }} \\
\end{align*}

Expanding this we have

\begin{align}\label{eqn:pauli_four_vector_v:realRepresentationFourVector}
{u^1}' &= \gamma ( u^1 - \beta u^4) \\
{u^2}' &= {u^2} \\
{u^3}' &= {u^3} \\
{u^4}' &= \gamma ( u^4 - \beta u^1)
\end{align}

In the imaginary representation the Lorentz transform takes the form

\begin{align*}
{\begin{bmatrix}
u^1 \\
u^2 \\
u^3 \\
u^4 \\
\end{bmatrix}}'
&=
\begin{bmatrix}
\gamma & 0 & 0 & i \gamma \beta \\
0 & 1 & 0 & 0 \\
0 & 0 & 1 & 0 \\
- i \gamma \beta & 0 & 0 & \gamma \\
\end{bmatrix}
{\begin{bmatrix}
u^1 \\
u^2 \\
u^3 \\
u^4 \\
\end{bmatrix}} 
\end{align*}

Let's verify that this produces the same result by expansion

\begin{align*}
{u^1}' &= \gamma ( u^1 + i \beta u^4) \\
{u^2}' &= {u^2} \\
{u^3}' &= {u^3} \\
{u^4}' &= \gamma ( u^4 - \beta i u^1)
\end{align*}

with $u^4 \rightarrow i u^4$ to switch to a real representation this is

\begin{align*}
{u^1}' &= \gamma ( u^1 - \beta u^4) \\
{u^2}' &= {u^2} \\
{u^3}' &= {u^3} \\
{u^4}' &= \gamma ( u^4 - \beta u^1)
\end{align*}

Good.  This matches equations \ref{eqn:pauli_four_vector_v:realRepresentationFourVector}.  Now, we want to put these in an explicit space time representation
to compare against \ref{eqn:pauli_four_vector_v:eqn10}.  Since those are in real form, work with the real representation instead of the imaginary Minkowski
representation for such a comparison.

\subsection{WRONG: Non-covariant representation of the transformed velocity. }

Expanding out the proper time derivatives (assuming that $dx'/dt' = v$ is a correct interpretation of the math), we have

\begin{align*}
\inv{\sqrt{1 - v^2/c^2}} \frac{{dx'}^1}{dt'} &= \inv{\sqrt{1 - v^2/c^2}} \inv{\sqrt{1 - \Bu^2/c^2}} \left( \frac{dx^1}{dt} - \beta c \right) \\
\inv{\sqrt{1 - v^2/c^2}} \frac{{dx'}^2}{dt'} &= \inv{\sqrt{1 - \Bu^2/c^2}} \frac{dx^2}{dt} \\
\inv{\sqrt{1 - v^2/c^2}} \frac{{dx'}^3}{dt'} &= \inv{\sqrt{1 - \Bu^2/c^2}} \frac{dx^3}{dt} \\
\inv{\sqrt{1 - v^2/c^2}} \frac{{dx'}^4}{dt'} &= \inv{\sqrt{1 - v^2/c^2}} \inv{\sqrt{1 - \Bu^2/c^2}} \left( c - \beta \frac{dx^1}{dt} \right)
\end{align*}

Hmm.  That doesn't appear to match.

\section{Try again from scratch. }

\subsection{Boost a stationary particle. }

Instead of starting with a proper velocity with a spatial component, let's cut the complexity and consider the simplest case, a particle at rest.  The worldline (in two dimensions) for a particle in its rest frame is

\begin{align*}
X = (0, ct) 
\end{align*}

The proper velocity for this particle is 

\begin{align*}
u = \frac{dX}{d\tau} = \left(0, c\frac{dt}{d\tau} \right) 
\end{align*}

But since this is a particle in its rest frame $dt/d\tau = 1$, this proper velocity is

\begin{align*}
u = \left(0, c \right) 
\end{align*}

Observe that the norm of this vector (still using the time negative metric signature) is

\begin{align*}
u \cdot u = 0^2 - c^2 = -c^2
\end{align*}

Now, what happens when we apply a Lorentz boost to this?

\begin{align*}
u' &= 
\begin{bmatrix}
\gamma & - \gamma \beta \\
- \gamma \beta & \gamma \\
\end{bmatrix}
\begin{bmatrix}
0 \\
c
\end{bmatrix} \\
\end{align*}

This is
\begin{align}\label{eqn:pauli_four_vector_v:uPrime}
u' &= 
\gamma
\begin{bmatrix}
- \beta \\
1 \\
\end{bmatrix}
c
\end{align}

What's the norm of this vector.  It should be unchanged, so let's verify.

\begin{align*}
u' \cdot u' 
&= \gamma^2 \left( (- \beta)^2 - 1^2 \right) c^2 \\
&= - \gamma^2 \left( 1 - \beta^2 \right) c^2 \\
&= - c^2 \\
\end{align*}

Good, still have the expected $-c^2$ value.  For this boosted vector, what is $dt'/d\tau'$?

Note that in general for the components of $u'$ we have

\begin{align*}
\frac{{dx'}^\mu}{d\tau'}
&=
\frac{{dx'}^\mu}{dt'} \frac{{dt'}}{d\tau'}
\end{align*}

and in particular we have ${u'}^4 = c dt'/d\tau$ %, since the proper time $\tau'$ in the primed frame measures the time for the particle at rest
%in that frame.  This gives

\begin{align*}
{u'}^4
&=
\frac{{dx'}^4}{dt'} \frac{{dt'}}{d\tau'} \\
&= c \frac{{dt'}}{d\tau'} \\
\end{align*}

Comparing to \ref{eqn:pauli_four_vector_v:uPrime} we have

\begin{align*}
{u'}^4 
&= \gamma c \\
&= c \frac{{dt'}}{d\tau'} \\
\end{align*}

and therefore can write

\begin{align*}
\frac{{dt'}}{d\tau'} 
&= \gamma 
\end{align*}

Similarly the spatial velocity of the particle in the boosted frame is

\begin{align*}
{u'}^1
&=
\frac{{dx'}^1}{dt'} \frac{{dt'}}{d\tau'} \\
&= u_x' \frac{{dt'}}{d\tau'} \\
&= - \gamma v
\end{align*}

So we have 

\begin{align*}
u_x' = -v 
\end{align*}

This seems to make sense.  We move the frame along the positive x-axis, so a particle at rest at the origin of the stationary frame has a velocity $v$ in the opposite direction from the viewpoint of something at rest in the moving frame.

\subsection{Apply a second boost transformation. }

Okay, treating the almost too simple case in detail was helpful to see where to go next.  Now that we have a view of a particle at rest
from a moving frame, let's apply another boost so we have a second frame moving with relative velocity $\beta'$ with respect to the moving
frame.  Our transformation is

\begin{align*}
L' =
\begin{bmatrix}
\gamma' & - \gamma' \beta' \\
- \gamma' \beta' & \gamma' \\
\end{bmatrix}
\end{align*}

this second transformation takes the original proper velocity to
\begin{align*}
u'' &=
\gamma \gamma'
\begin{bmatrix}
1 & - \beta' \\
- \beta' & 1 \\
\end{bmatrix}
\begin{bmatrix}
- \beta \\
1 \\
\end{bmatrix}
c \\
\end{align*}

This is
\begin{align}\label{eqn:pauli_four_vector_v:boost2}
u'' &=
\gamma \gamma'
\begin{bmatrix}
-(\beta + \beta') \\
1 + \beta\beta'
\end{bmatrix}
c
\end{align}

Let's verify that we still have our invariant norm.

\begin{align*}
u'' \cdot u'' 
&=
\gamma^2 {\gamma'}^2
\left(
(\beta + \beta')^2 
-(1 + \beta\beta')^2
\right)
c^2 \\
&=
\gamma^2 {\gamma'}^2
\left(
\beta^2
+{\beta'}^2
+2 \beta\beta'
-1
-2 \beta\beta'
-\beta^2 {\beta'}^2
\right)
c^2 \\
&=
\gamma^2 {\gamma'}^2
\left(
\beta^2 (1 - {\beta'}^2)
-(1 -{\beta'}^2)
\right)
c^2 \\
&=
-\gamma^2 {\gamma'}^2 (1 -{\beta'}^2)(1 -\beta^2) c^2 \\
&=
- c^2 \\
\end{align*}

Now, we have ${u''}^4 = c dt''/d\tau''$ as before, so from equation
\ref{eqn:pauli_four_vector_v:boost2} the new compound $\gamma$ factor can be picked off

\begin{align*}
\frac{dt''}{d\tau''} &=
\gamma \gamma'( 1 + \beta\beta' )
\end{align*}

Using this and chain rule again we have the spatial velocity in the second moving frame for the particle at rest in the original frame.  This is

\begin{align*}
u_x'' 
&=
\frac{\frac{dx''}{dt''}}{ \frac{dt''}{d\tau''} } \\
&=
\frac{-\gamma \gamma' (\beta + \beta') c}{ \gamma \gamma'( 1 + \beta\beta' ) } \\
&=
\frac{-(\beta + \beta') c}{ 1 + \beta\beta' } \\
&=
\frac{-(v + v') }{ 1 + v v'/c^2 } \\
\end{align*}

Okay, good.  From consideration of proper velocities and their transformations we have something that is of
the form of Pauli's equation 10 (here equation \ref{eqn:pauli_four_vector_v:eqn10}), which is the standard form for colinear 
relativistic velocity addition.

There is a difference though, namely that Pauli's equation 10 expresses the reverse transformation.  Shuffling
equation \ref{eqn:pauli_four_vector_v:eqn10} to solve for $u_x'$, we have

\begin{align*}
u_x ( 1 + v {u_x}') &= { {u_x}' + v  } \\
\end{align*}

which gives

\begin{align*}
u_x' &= \frac{ {u_x} + (-v)  }{ 1 + (-v) {u_x}} \\
\end{align*}

An algebraic inversion of the equation has exactly the same form, but with the velocity negated in sign.

Now with $u_x = -v'$ we have an identification between this twice boosted frame observing the particle at
rest in the original frame.

\subsection{Perpendicular directions. }

Now, the only thing left to understand is the spatial representation of the boosted velocity 
for the perpendicular to the boost direction components.

To do so, let's treat a more general case for the proper velocity of a particle as seen in some observers ``rest frame''.  Given the particle worldline

\begin{align*}
X = (x^\mu)
\end{align*}

The proper velocity is

\begin{align*}
\frac{dX}{d\tau} = \left(\frac{x^k}{dt}, c \right) \frac{dt}{d\tau}
\end{align*}

Writing

\begin{align*}
u_x &= \frac{x^1}{dt} \\
u_y &= \frac{x^2}{dt} \\
u_z &= \frac{x^3}{dt} \\
\gamma_0 &= \frac{dt}{d\tau}
\end{align*}

Application of a boost produces

\begin{align*}
u'
&=
\begin{bmatrix}
\gamma & 0 & 0 & - \gamma \beta \\
0 & 1 & 0 & 0 \\
0 & 0 & 1 & 0 \\
- \gamma \beta & 0 & 0 & \gamma \\
\end{bmatrix}
\begin{bmatrix}
u_x \\
u_y \\
u_z \\
c
\end{bmatrix}
\gamma_0 \\
&=
\begin{bmatrix}
\gamma_0 \gamma (u_x - \beta c) \\
\gamma_0 u_y \\
\gamma_0 u_z \\
\gamma_0 \gamma ( -\beta u_x + c ) \\
\end{bmatrix} \\
\end{align*}

In particular we have

\begin{align*}
\frac{dx'}{d\tau'} &= \gamma_0 \gamma ( 1 -\beta u_x/c ) \\
\end{align*}

So can write

\begin{align*}
u_x' &=
\frac{\gamma_0 \gamma (u_x - \beta c) }
{\gamma_0 \gamma ( 1 -\beta u_x/c )} \\
u_y' &=
\frac{\gamma_0 u_y }
{\gamma_0 \gamma ( 1 -\beta u_x/c )} \\
u_z' &=
\frac{\gamma_0 u_z }
{\gamma_0 \gamma ( 1 -\beta u_x/c )} \\
\end{align*}

Reversing signs in $\beta$ to invert and canceling common factors this is

\begin{align*}
u_x &=
\frac{u_x' + v }
{ 1 + v u_x'/c^2 } \\
u_y &=
\frac{u_y' }
{\gamma ( 1 + v u_x'/c^2 )} \\
u_z &=
\frac{u_z' }
{\gamma ( 1 + v u_x'/c^2 )} \\
\end{align*}

A final substitution of $\gamma^{-1} = \sqrt{1 - v^2/c^2}$ and we have
equation \ref{eqn:pauli_four_vector_v:eqn10} as desired.  Pauli says this step is easy, and that's
true enough once the simpler cases are first understood.

%\bibliographystyle{plainnat}
%\bibliography{myrefs}

%\end{document}

%\documentclass{article}

%\usepackage{amsmath}
\usepackage{mathpazo}

%
% shorthand for bold symbols, convenient for vectors and matrices
%
\newcommand{\Ba}[0]{\mathbf{a}}
\newcommand{\Bb}[0]{\mathbf{b}}
\newcommand{\Bc}[0]{\mathbf{c}}
\newcommand{\Bd}[0]{\mathbf{d}}
\newcommand{\Be}[0]{\mathbf{e}}
\newcommand{\Bf}[0]{\mathbf{f}}
\newcommand{\Bg}[0]{\mathbf{g}}
\newcommand{\Bh}[0]{\mathbf{h}}
\newcommand{\Bi}[0]{\mathbf{i}}
\newcommand{\Bj}[0]{\mathbf{j}}
\newcommand{\Bk}[0]{\mathbf{k}}
\newcommand{\Bl}[0]{\mathbf{l}}
\newcommand{\Bm}[0]{\mathbf{m}}
\newcommand{\Bn}[0]{\mathbf{n}}
\newcommand{\Bo}[0]{\mathbf{o}}
\newcommand{\Bp}[0]{\mathbf{p}}
\newcommand{\Bq}[0]{\mathbf{q}}
\newcommand{\Br}[0]{\mathbf{r}}
\newcommand{\Bs}[0]{\mathbf{s}}
\newcommand{\Bt}[0]{\mathbf{t}}
\newcommand{\Bu}[0]{\mathbf{u}}
\newcommand{\Bv}[0]{\mathbf{v}}
\newcommand{\Bw}[0]{\mathbf{w}}
\newcommand{\Bx}[0]{\mathbf{x}}
\newcommand{\By}[0]{\mathbf{y}}
\newcommand{\Bz}[0]{\mathbf{z}}
\newcommand{\BA}[0]{\mathbf{A}}
\newcommand{\BB}[0]{\mathbf{B}}
\newcommand{\BC}[0]{\mathbf{C}}
\newcommand{\BD}[0]{\mathbf{D}}
\newcommand{\BE}[0]{\mathbf{E}}
\newcommand{\BF}[0]{\mathbf{F}}
\newcommand{\BG}[0]{\mathbf{G}}
\newcommand{\BH}[0]{\mathbf{H}}
\newcommand{\BI}[0]{\mathbf{I}}
\newcommand{\BJ}[0]{\mathbf{J}}
\newcommand{\BK}[0]{\mathbf{K}}
\newcommand{\BL}[0]{\mathbf{L}}
\newcommand{\BM}[0]{\mathbf{M}}
\newcommand{\BN}[0]{\mathbf{N}}
\newcommand{\BO}[0]{\mathbf{O}}
\newcommand{\BP}[0]{\mathbf{P}}
\newcommand{\BQ}[0]{\mathbf{Q}}
\newcommand{\BR}[0]{\mathbf{R}}
\newcommand{\BS}[0]{\mathbf{S}}
\newcommand{\BT}[0]{\mathbf{T}}
\newcommand{\BU}[0]{\mathbf{U}}
\newcommand{\BV}[0]{\mathbf{V}}
\newcommand{\BW}[0]{\mathbf{W}}
\newcommand{\BX}[0]{\mathbf{X}}
\newcommand{\BY}[0]{\mathbf{Y}}
\newcommand{\BZ}[0]{\mathbf{Z}}

\newcommand{\Bzero}[0]{\mathbf{0}}
\newcommand{\Btheta}[0]{\boldsymbol{\theta}}
\newcommand{\Btau}[0]{\boldsymbol{\tau}}
\newcommand{\Bomega}[0]{\boldsymbol{\omega}}

%
% shorthand for unit vectors
%
\newcommand{\acap}[0]{\hat{\Ba}}
\newcommand{\bcap}[0]{\hat{\Bb}}
\newcommand{\ccap}[0]{\hat{\Bc}}
\newcommand{\dcap}[0]{\hat{\Bd}}
\newcommand{\ecap}[0]{\hat{\Be}}
\newcommand{\fcap}[0]{\hat{\Bf}}
\newcommand{\gcap}[0]{\hat{\Bg}}
\newcommand{\hcap}[0]{\hat{\Bh}}
\newcommand{\icap}[0]{\hat{\Bi}}
\newcommand{\jcap}[0]{\hat{\Bj}}
\newcommand{\kcap}[0]{\hat{\Bk}}
\newcommand{\lcap}[0]{\hat{\Bl}}
\newcommand{\mcap}[0]{\hat{\Bm}}
\newcommand{\ncap}[0]{\hat{\Bn}}
\newcommand{\ocap}[0]{\hat{\Bo}}
\newcommand{\pcap}[0]{\hat{\Bp}}
\newcommand{\qcap}[0]{\hat{\Bq}}
\newcommand{\rcap}[0]{\hat{\Br}}
\newcommand{\scap}[0]{\hat{\Bs}}
\newcommand{\tcap}[0]{\hat{\Bt}}
\newcommand{\ucap}[0]{\hat{\Bu}}
\newcommand{\vcap}[0]{\hat{\Bv}}
\newcommand{\wcap}[0]{\hat{\Bw}}
\newcommand{\xcap}[0]{\hat{\Bx}}
\newcommand{\ycap}[0]{\hat{\By}}
\newcommand{\zcap}[0]{\hat{\Bz}}
\newcommand{\thetacap}[0]{\hat{\Btheta}}

%
% to write R^n and C^n in a distinguishable fashion.  Perhaps change this
% to the double lined characters upon figuring out how to do so.
%
\newcommand{\C}[1]{$\mathbb{C}^{#1}$}
\newcommand{\R}[1]{$\mathbb{R}^{#1}$}

%
% various generally useful helpers
%

% derivative of #1 wrt. #2:
\newcommand{\D}[2] {\frac {d#2} {d#1}}

\newcommand{\inv}[1]{\frac{1}{#1}}
\newcommand{\cross}[0]{\times}

\newcommand{\abs}[1]{\lvert{#1}\rvert}
\newcommand{\norm}[1]{\lVert{#1}\rVert}
\newcommand{\innerprod}[2]{\langle{#1}, {#2}\rangle}
\newcommand{\dotprod}[2]{{#1} \cdot {#2}}
\newcommand{\bdotprod}[2]{\left({#1} \cdot {#2}\right)}
\newcommand{\crossprod}[2]{{#1} \cross {#2}}
\newcommand{\tripleprod}[3]{\dotprod{\left(\crossprod{#1}{#2}\right)}{#3}}

\DeclareMathOperator{\Proj}{Proj}
\DeclareMathOperator{\Span}{span}
\DeclareMathOperator{\Sgn}{sgn}
\DeclareMathOperator{\Area}{Area}
\DeclareMathOperator{\Volume}{Volume}

%
% A few miscellaneous things specific to this document
%
\newcommand{\crossop}[1]{\crossprod{#1}{}}

% R2 vector.
\newcommand{\VectorTwo}[2]{
\begin{bmatrix}
 {#1} \\
 {#2}
\end{bmatrix}
}

\newcommand{\VectorN}[1]{
\begin{bmatrix}
{#1}_1 \\
{#1}_2 \\
\vdots \\
{#1}_N \\
\end{bmatrix}
}

\newcommand{\DETuvij}[4]{
\begin{vmatrix}
 {#1}_{#3} & {#1}_{#4} \\
 {#2}_{#3} & {#2}_{#4}
\end{vmatrix}
}

\newcommand{\DETuvwijk}[6]{
\begin{vmatrix}
 {#1}_{#4} & {#1}_{#5} & {#1}_{#6} \\
 {#2}_{#4} & {#2}_{#5} & {#2}_{#6} \\
 {#3}_{#4} & {#3}_{#5} & {#3}_{#6}
\end{vmatrix}
}

\newcommand{\DETuvwxijkl}[8]{
\begin{vmatrix}
 {#1}_{#5} & {#1}_{#6} & {#1}_{#7} & {#1}_{#8} \\
 {#2}_{#5} & {#2}_{#6} & {#2}_{#7} & {#2}_{#8} \\
 {#3}_{#5} & {#3}_{#6} & {#3}_{#7} & {#3}_{#8} \\
 {#4}_{#5} & {#4}_{#6} & {#4}_{#7} & {#4}_{#8} \\
\end{vmatrix}
}

%\newcommand{\DETuvwxyijklm}[10]{
%\begin{vmatrix}
% {#1}_{#6} & {#1}_{#7} & {#1}_{#8} & {#1}_{#9} & {#1}_{#10} \\
% {#2}_{#6} & {#2}_{#7} & {#2}_{#8} & {#2}_{#9} & {#2}_{#10} \\
% {#3}_{#6} & {#3}_{#7} & {#3}_{#8} & {#3}_{#9} & {#3}_{#10} \\
% {#4}_{#6} & {#4}_{#7} & {#4}_{#8} & {#4}_{#9} & {#4}_{#10} \\
% {#5}_{#6} & {#5}_{#7} & {#5}_{#8} & {#5}_{#9} & {#5}_{#10}
%\end{vmatrix}
%}

% R3 vector.
\newcommand{\VectorThree}[3]{
\begin{bmatrix}
 {#1} \\
 {#2} \\
 {#3}
\end{bmatrix}
}


%%<misc>
%
\newcommand{\Abs}[1]{{\left\lvert{#1}\right\rvert}}
\newcommand{\spacegrad}[0]{\boldsymbol{\nabla}}
\newcommand{\grad}[0]{\nabla}
\newcommand{\LL}[0]{\mathcal{L}}

% == \partial_{#1} {#2}
\newcommand{\PD}[2]{\frac{\partial {#2}}{\partial {#1}}}
% inline variant
\newcommand{\PDi}[2]{{\partial {#2}}/{\partial {#1}}}

\newcommand{\PDD}[3]{\frac{\partial^2 {#3}}{\partial {#1}\partial {#2}}}
%\newcommand{\PDd}[2]{\frac{\partial^2 {#2}}{{\partial{#1}}^2}}
\newcommand{\PDsq}[2]{\frac{\partial^2 {#2}}{(\partial {#1})^2}}

\newcommand{\Partial}[2]{\frac{\partial {#1}}{\partial {#2}}}
\DeclareMathOperator{\RejName}{Rej}
\newcommand{\Rej}[2]{\RejName_{#1}\left( {#2} \right)}
\newcommand{\Rm}[1]{\mathbb{R}^{#1}}
\newcommand{\Cm}[1]{\mathbb{C}^{#1}}
\newcommand{\conj}[0]{{*}}

%</misc>

% <grade selection>
%
\newcommand{\gpgrade}[2] {{\left\langle{{#1}}\right\rangle}_{#2}}

\newcommand{\gpgradezero}[1] {\gpgrade{#1}{}}
%\newcommand{\gpscalargrade}[1] {{\left\langle{{#1}}\right\rangle}}
%\newcommand{\gpgradezero}[1] {\gpgrade{#1}{0}}

%\newcommand{\gpgradeone}[1] {{\left\langle{{#1}}\right\rangle}_{1}}
\newcommand{\gpgradeone}[1] {\gpgrade{#1}{1}}

\newcommand{\gpgradetwo}[1] {\gpgrade{#1}{2}}
\newcommand{\gpgradethree}[1] {\gpgrade{#1}{3}}
\newcommand{\gpgradefour}[1] {\gpgrade{#1}{4}}
%
% </grade selection>



\newcommand{\adot}[0]{{\dot{a}}}
\newcommand{\bdot}[0]{{\dot{b}}}
% taken for centered dot:
%\newcommand{\cdot}[0]{{\dot{c}}}
%\newcommand{\ddot}[0]{{\dot{d}}}
\newcommand{\edot}[0]{{\dot{e}}}
\newcommand{\fdot}[0]{{\dot{f}}}
\newcommand{\gdot}[0]{{\dot{g}}}
\newcommand{\hdot}[0]{{\dot{h}}}
\newcommand{\idot}[0]{{\dot{i}}}
\newcommand{\jdot}[0]{{\dot{j}}}
\newcommand{\kdot}[0]{{\dot{k}}}
\newcommand{\ldot}[0]{{\dot{l}}}
\newcommand{\mdot}[0]{{\dot{m}}}
\newcommand{\ndot}[0]{{\dot{n}}}
%\newcommand{\odot}[0]{{\dot{o}}}
\newcommand{\pdot}[0]{{\dot{p}}}
\newcommand{\qdot}[0]{{\dot{q}}}
\newcommand{\rdot}[0]{{\dot{r}}}
\newcommand{\sdot}[0]{{\dot{s}}}
\newcommand{\tdot}[0]{{\dot{t}}}
\newcommand{\udot}[0]{{\dot{u}}}
\newcommand{\vdot}[0]{{\dot{v}}}
\newcommand{\wdot}[0]{{\dot{w}}}
\newcommand{\xdot}[0]{{\dot{x}}}
\newcommand{\ydot}[0]{{\dot{y}}}
\newcommand{\zdot}[0]{{\dot{z}}}
\newcommand{\addot}[0]{{\ddot{a}}}
\newcommand{\bddot}[0]{{\ddot{b}}}
\newcommand{\cddot}[0]{{\ddot{c}}}
%\newcommand{\dddot}[0]{{\ddot{d}}}
\newcommand{\eddot}[0]{{\ddot{e}}}
\newcommand{\fddot}[0]{{\ddot{f}}}
\newcommand{\gddot}[0]{{\ddot{g}}}
\newcommand{\hddot}[0]{{\ddot{h}}}
\newcommand{\iddot}[0]{{\ddot{i}}}
\newcommand{\jddot}[0]{{\ddot{j}}}
\newcommand{\kddot}[0]{{\ddot{k}}}
\newcommand{\lddot}[0]{{\ddot{l}}}
\newcommand{\mddot}[0]{{\ddot{m}}}
\newcommand{\nddot}[0]{{\ddot{n}}}
\newcommand{\oddot}[0]{{\ddot{o}}}
\newcommand{\pddot}[0]{{\ddot{p}}}
\newcommand{\qddot}[0]{{\ddot{q}}}
\newcommand{\rddot}[0]{{\ddot{r}}}
\newcommand{\sddot}[0]{{\ddot{s}}}
\newcommand{\tddot}[0]{{\ddot{t}}}
\newcommand{\uddot}[0]{{\ddot{u}}}
\newcommand{\vddot}[0]{{\ddot{v}}}
\newcommand{\wddot}[0]{{\ddot{w}}}
\newcommand{\xddot}[0]{{\ddot{x}}}
\newcommand{\yddot}[0]{{\ddot{y}}}
\newcommand{\zddot}[0]{{\ddot{z}}}

%<bold and dot greek symbols>
%

\newcommand{\Deltadot}[0]{{\dot{\Delta}}}
\newcommand{\Gammadot}[0]{{\dot{\Gamma}}}
\newcommand{\Lambdadot}[0]{{\dot{\Lambda}}}
\newcommand{\Omegadot}[0]{{\dot{\Omega}}}
\newcommand{\Phidot}[0]{{\dot{\Phi}}}
\newcommand{\Pidot}[0]{{\dot{\Pi}}}
\newcommand{\Psidot}[0]{{\dot{\Psi}}}
\newcommand{\Sigmadot}[0]{{\dot{\Sigma}}}
\newcommand{\Thetadot}[0]{{\dot{\Theta}}}
\newcommand{\Upsilondot}[0]{{\dot{\Upsilon}}}
\newcommand{\Xidot}[0]{{\dot{\Xi}}}
\newcommand{\alphadot}[0]{{\dot{\alpha}}}
\newcommand{\betadot}[0]{{\dot{\beta}}}
\newcommand{\chidot}[0]{{\dot{\chi}}}
\newcommand{\deltadot}[0]{{\dot{\delta}}}
\newcommand{\epsilondot}[0]{{\dot{\epsilon}}}
\newcommand{\etadot}[0]{{\dot{\eta}}}
\newcommand{\gammadot}[0]{{\dot{\gamma}}}
\newcommand{\kappadot}[0]{{\dot{\kappa}}}
\newcommand{\lambdadot}[0]{{\dot{\lambda}}}
\newcommand{\mudot}[0]{{\dot{\mu}}}
\newcommand{\nudot}[0]{{\dot{\nu}}}
\newcommand{\omegadot}[0]{{\dot{\omega}}}
\newcommand{\phidot}[0]{{\dot{\phi}}}
\newcommand{\pidot}[0]{{\dot{\pi}}}
\newcommand{\psidot}[0]{{\dot{\psi}}}
\newcommand{\rhodot}[0]{{\dot{\rho}}}
\newcommand{\sigmadot}[0]{{\dot{\sigma}}}
\newcommand{\taudot}[0]{{\dot{\tau}}}
\newcommand{\thetadot}[0]{{\dot{\theta}}}
\newcommand{\upsilondot}[0]{{\dot{\upsilon}}}
\newcommand{\varepsilondot}[0]{{\dot{\varepsilon}}}
\newcommand{\varphidot}[0]{{\dot{\varphi}}}
\newcommand{\varpidot}[0]{{\dot{\varpi}}}
\newcommand{\varrhodot}[0]{{\dot{\varrho}}}
\newcommand{\varsigmadot}[0]{{\dot{\varsigma}}}
\newcommand{\varthetadot}[0]{{\dot{\vartheta}}}
\newcommand{\xidot}[0]{{\dot{\xi}}}
\newcommand{\zetadot}[0]{{\dot{\zeta}}}

\newcommand{\Deltaddot}[0]{{\ddot{\Delta}}}
\newcommand{\Gammaddot}[0]{{\ddot{\Gamma}}}
\newcommand{\Lambdaddot}[0]{{\ddot{\Lambda}}}
\newcommand{\Omegaddot}[0]{{\ddot{\Omega}}}
\newcommand{\Phiddot}[0]{{\ddot{\Phi}}}
\newcommand{\Piddot}[0]{{\ddot{\Pi}}}
\newcommand{\Psiddot}[0]{{\ddot{\Psi}}}
\newcommand{\Sigmaddot}[0]{{\ddot{\Sigma}}}
\newcommand{\Thetaddot}[0]{{\ddot{\Theta}}}
\newcommand{\Upsilonddot}[0]{{\ddot{\Upsilon}}}
\newcommand{\Xiddot}[0]{{\ddot{\Xi}}}
\newcommand{\alphaddot}[0]{{\ddot{\alpha}}}
\newcommand{\betaddot}[0]{{\ddot{\beta}}}
\newcommand{\chiddot}[0]{{\ddot{\chi}}}
\newcommand{\deltaddot}[0]{{\ddot{\delta}}}
\newcommand{\epsilonddot}[0]{{\ddot{\epsilon}}}
\newcommand{\etaddot}[0]{{\ddot{\eta}}}
\newcommand{\gammaddot}[0]{{\ddot{\gamma}}}
\newcommand{\kappaddot}[0]{{\ddot{\kappa}}}
\newcommand{\lambdaddot}[0]{{\ddot{\lambda}}}
\newcommand{\muddot}[0]{{\ddot{\mu}}}
\newcommand{\nuddot}[0]{{\ddot{\nu}}}
\newcommand{\omegaddot}[0]{{\ddot{\omega}}}
\newcommand{\phiddot}[0]{{\ddot{\phi}}}
\newcommand{\piddot}[0]{{\ddot{\pi}}}
\newcommand{\psiddot}[0]{{\ddot{\psi}}}
\newcommand{\rhoddot}[0]{{\ddot{\rho}}}
\newcommand{\sigmaddot}[0]{{\ddot{\sigma}}}
\newcommand{\tauddot}[0]{{\ddot{\tau}}}
\newcommand{\thetaddot}[0]{{\ddot{\theta}}}
\newcommand{\upsilonddot}[0]{{\ddot{\upsilon}}}
\newcommand{\varepsilonddot}[0]{{\ddot{\varepsilon}}}
\newcommand{\varphiddot}[0]{{\ddot{\varphi}}}
\newcommand{\varpiddot}[0]{{\ddot{\varpi}}}
\newcommand{\varrhoddot}[0]{{\ddot{\varrho}}}
\newcommand{\varsigmaddot}[0]{{\ddot{\varsigma}}}
\newcommand{\varthetaddot}[0]{{\ddot{\vartheta}}}
\newcommand{\xiddot}[0]{{\ddot{\xi}}}
\newcommand{\zetaddot}[0]{{\ddot{\zeta}}}

\newcommand{\BDelta}[0]{\boldsymbol{\Delta}}
\newcommand{\BGamma}[0]{\boldsymbol{\Gamma}}
\newcommand{\BLambda}[0]{\boldsymbol{\Lambda}}
\newcommand{\BOmega}[0]{\boldsymbol{\Omega}}
\newcommand{\BPhi}[0]{\boldsymbol{\Phi}}
\newcommand{\BPi}[0]{\boldsymbol{\Pi}}
\newcommand{\BPsi}[0]{\boldsymbol{\Psi}}
\newcommand{\BSigma}[0]{\boldsymbol{\Sigma}}
\newcommand{\BTheta}[0]{\boldsymbol{\Theta}}
\newcommand{\BUpsilon}[0]{\boldsymbol{\Upsilon}}
\newcommand{\BXi}[0]{\boldsymbol{\Xi}}
\newcommand{\Balpha}[0]{\boldsymbol{\alpha}}
\newcommand{\Bbeta}[0]{\boldsymbol{\beta}}
\newcommand{\Bchi}[0]{\boldsymbol{\chi}}
\newcommand{\Bdelta}[0]{\boldsymbol{\delta}}
\newcommand{\Bepsilon}[0]{\boldsymbol{\epsilon}}
\newcommand{\Beta}[0]{\boldsymbol{\eta}}
\newcommand{\Bgamma}[0]{\boldsymbol{\gamma}}
\newcommand{\Bkappa}[0]{\boldsymbol{\kappa}}
\newcommand{\Blambda}[0]{\boldsymbol{\lambda}}
\newcommand{\Bmu}[0]{\boldsymbol{\mu}}
\newcommand{\Bnu}[0]{\boldsymbol{\nu}}
%\newcommand{\Bomega}[0]{\boldsymbol{\omega}}
\newcommand{\Bphi}[0]{\boldsymbol{\phi}}
\newcommand{\Bpi}[0]{\boldsymbol{\pi}}
\newcommand{\Bpsi}[0]{\boldsymbol{\psi}}
\newcommand{\Brho}[0]{\boldsymbol{\rho}}
\newcommand{\Bsigma}[0]{\boldsymbol{\sigma}}
%\newcommand{\Btau}[0]{\boldsymbol{\tau}}
%\newcommand{\Btheta}[0]{\boldsymbol{\theta}}
\newcommand{\Bupsilon}[0]{\boldsymbol{\upsilon}}
\newcommand{\Bvarepsilon}[0]{\boldsymbol{\varepsilon}}
\newcommand{\Bvarphi}[0]{\boldsymbol{\varphi}}
\newcommand{\Bvarpi}[0]{\boldsymbol{\varpi}}
\newcommand{\Bvarrho}[0]{\boldsymbol{\varrho}}
\newcommand{\Bvarsigma}[0]{\boldsymbol{\varsigma}}
\newcommand{\Bvartheta}[0]{\boldsymbol{\vartheta}}
\newcommand{\Bxi}[0]{\boldsymbol{\xi}}
\newcommand{\Bzeta}[0]{\boldsymbol{\zeta}}
%
%</bold and dot greek symbols>
%<infrequent>
%
%\newcommand{\AreaOp}[1]{\AName_{#1}}
%\newcommand{\Babs}[0]{\abs{\BB}}
%\newcommand{\Bcap}[0]{\hat{\BB}}
%\newcommand{\BrPrimeRej}[0]{\rcap(\rcap \wedge \Br')}
%\newcommand{\CA}[0]{\mathcal{A}}
%\newcommand{\Cos}[1]{\cos{\left({#1}\right)}}
%\newcommand{\Det}[1] {\abs{#1}}
%\newcommand{\Dsq}[2] {\frac {\partial^2 {#1}} {\partial {#2}^2}}
%\newcommand{\Exp}[1]{\exp{\left({#1}\right)}}
%\newcommand{\Norm}[1]{\left\lVert{#1}\right\rVert}
%\newcommand{\Sin}[1]{\sin{\left({#1}\right)}}
%\newcommand{\T}[0]{\text{T}}
%\newcommand{\VolumeOp}[1]{\VName_{#1}}
%\newcommand{\agrad}[0]{\Ba \cdot \nabla}
%\newcommand{\alphacap}[0]{\hat{\boldsymbol{\alpha}}}
%\newcommand{\Fcap}[0]{\hat{\BF}}
%\newcommand{\bithree}[0]{{\Bi}_3}
%\newcommand{\bxa}[0]{\Bx\Ba}
%\newcommand{\coordvec}[2]{
%\newcommand{\costheta}[0]{\acap \cdot \xcap}
%\newcommand{\ddt}[1]{\ddot{#1}}
%\newcommand{\ddu}[1] {\frac {d{#1}} {du}}
%\newcommand{\dsqxj}[2] {\frac {\partial^2 {#1}} {\partial {x_{#2}}^2}}
%\newcommand{\dtheta}[1]{\frac{d {#1}}{d \theta}}
%\newcommand{\dt}[1]{\dot{#1}}
%\newcommand{\dt}[1]{\frac{d {#1}}{dt}}
%\newcommand{\dxj}[2] {\frac {\partial {#1}} {\partial {x_{#2}}}}
%\newcommand{\halfPhi}[0]{\frac{\phi}{2}}
%\newcommand{\half}[0]{\inv{2}}
%\newcommand{\inv}[1]{\frac{1}{#1}}
%\newcommand{\laplacian}[0]{\nabla^2}
%\newcommand{\matrixoftx}[3]{
%\newcommand{\nrrp}[0]{\norm{\rcap \wedge \Br'}}
%\newcommand{\oiint}{\bigcirc \hspace{-1.4em} \int \hspace{-.8em} \int}
%\newcommand{\transpose}[1]{{#1}^{\text{T}}}
%\newcommand{\transpose}[1]{{{#1}^{\TextTranspose}}}
%\newcommand{\transpose}[1]{{{#1}^{\text{T}}}}
%\newcommand{\barA}[0]{\bar{A}}
%\newcommand{\qbar}[0]{\bar{q}}
%\newcommand{\qdotbar}[0]{\dot{\bar{q}}}
%
%</infrequent>




%\usepackage[bookmarks=true]{hyperref}

%\usepackage{color,cite,graphicx}
   % use colour in the document, put your citations as [1-4]
   % rather than [1,2,3,4] (it looks nicer, and the extended LaTeX2e
   % graphics package. 
%\usepackage{latexsym,amssymb,epsf} % don't remember if these are
   % needed, but their inclusion can't do any damage


\chapter{Pauli's relativity background in QM intro from "Wave Mechanics". }
%\author{Peeter Joot \quad peeter.joot@gmail.com}
\date{ Jan 24, 2009.  $RCSfile: pauliQmRelativityIntro.tex,v $ Last $Revision: 1.12 $ $Date: 2009/06/11 17:00:37 $ }

%\begin{document}

%\maketitle{}

%\tableofcontents

\section{Motivation. }

In \cite{pauli2000wm} a few relativity notes are made to build up to 
a relativistic wave equation (ie: the Klein-Gordon equation),
and show one can introduce a non-relativistic approximation of this
that has close to the form of a the free particle \Sch equation.  It 
is interesting to see things use relativity as a base.  This is exactly
opposite to the Klein-Gordon treatment in a text such as
\cite{srednicki2007qft} where a way to find a 
relativistically correct form starting from the \Sch equation is searched for.

Pauli's treatment is a bit too terse for me, but has a number of
interesting and illuminating features.  Here I walk through his treatment
at my own pace.

\section{Relativistic mechanics. }

\subsection{Energy in terms of momentum. }

Equation $1.4$ is the famous energy and momentum equations

\begin{align}\label{eqn:pauli_qm_relativity_intro:p1_4}
E &= \frac{ m c^2 }{\sqrt{1 - \Bv^2/c^2}} \\
\Bp &= \frac{ m \Bv }{\sqrt{1 - \Bv^2/c^2}} \\
\end{align}

These pair of these quantities is often now expressed as a four vector in various ways

\begin{align*}
p &= \left(\frac{E}{c}, \Bp \right) \\
p &= \frac{E}{c} \gamma_0 + \Bp \gamma_0 \\
\cdots
\end{align*}

These two quantities are observably interdependent, and this dependency can be made explicit by forming the sum

\begin{align*}
\Bp^2 + m^2 c^2 
&= \frac{ m^2 \Bv^2 }{{1 - \Bv^2/c^2}} +  \frac{ m^2 c^2( 1 - \Bv^2/c^2) }{{1 - \Bv^2/c^2}}  \\
&= \inv{1 - \Bv^2/c^2} m^2 \left( \Bv^2 + c^2( 1 - \Bv^2/c^2) \right) \\
&= \inv{1 - \Bv^2/c^2} m^2 \left( \Bv^2 + c^2 - \Bv^2 \right) \\
&= \inv{1 - \Bv^2/c^2} m^2 c^2 \\
\end{align*}

This recovers Pauli's equation $1.3$ (in the square).

\begin{align}\label{eqn:pauli_qm_relativity_intro:p1_3}
\frac{E^2}{c^2} = \Bp^2 + m^2 c^2 
\end{align}

This is slightly different from how I'm used to seeing this expressed, since Energy is singled out.
Rearranging slightly recovers the scalar invariant for the energy momentum four vector:

\begin{align*}
m^2 c^2 &= \frac{E^2}{c^2} -\Bp^2 
\end{align*}

\subsection{Energy-momentum four vector from Energy }

Now, interestingly, Pauli also points out that his equation \ref{eqn:pauli_qm_relativity_intro:p1_3} can be used to derive the four vector
equations for energy and momentum, only requiring one express the relationship between Kinetic energy and momentum
as one would do in plain old non-relativistic physics.  That is, starting with

\begin{align*}
E &= \inv{2} m \Bv^2 \\
\end{align*}

differentiation with respect to some parameter we can write

\begin{align*}
\frac{dE}{d\alpha} 
&= m \Bv \cdot \frac{d\Bv}{d\alpha} \\
&= \Bv \cdot \frac{d\Bp}{d\alpha} \\
\end{align*}

If the specific parametrization of the path is implied we have

\begin{align}\label{eqn:pauli_qm_relativity_intro:EvP}
{dE} &= \Bv \cdot {d\Bp}.
\end{align}

In coordinates this gives

\begin{align*}
{dE} &= \sum_k v_k dp_k \\
\end{align*}

Pauli uses this to express the velocity coordinates in terms of energy and momentum, and writes

\begin{align}\label{eqn:pauli_qm_relativity_intro:vEp}
v_k &= \PD{p_k}{E}
\end{align}

My way of getting this seems a bit fishy, dropping the explicit parametrization to get the one form, and then switching magically to partials, but once one gets to the end result it does not appear unreasonable.

Perhaps better is to skip the one form business completely, writing

\begin{align*}
E = \inv{2} \Bv \cdot \Bp = \inv{2} \sum_k v_k p_k
\end{align*}

But taking partials from this to get \ref{eqn:pauli_qm_relativity_intro:vEp} requires care since $p_k$ and $v_k$ are dependent.
%Perhaps notable is that Pauli goes from \ref{eqn:pauli_qm_relativity_intro:EvP} to \ref{eqn:pauli_qm_relativity_intro:vEp} directly.

Assuming \ref{eqn:pauli_qm_relativity_intro:vEp} is valid and applying this to \ref{eqn:pauli_qm_relativity_intro:p1_3}, it is relatively straightforward to
recover the four-vector energy-momentum equations.  

From
\begin{align*}
E &= \sqrt{\Bp^2 c^2 + m^2 c^4 } \\
\end{align*}

we calculate
\begin{align*}
v_k 
&= \PD{p_k}{E} \\
&= (2 p_k c^2) \inv{2} \frac{1}{\sqrt{\Bp^2 c^2 + m^2 c^4 }} \\
&= c^2 \frac{p_k}{E}
\end{align*}

Summing over all components
\begin{align*}
\frac{\Bv^2}{c^2} 
&= \sum_k \frac{(v_k)^2}{c^2} \\
&= c^2 \sum_k \frac{{p_k}^2}{E^2} \\
&= c^2 \frac{\Bp^2}{E^2} \\
\end{align*}

Subtracting this from one, gives us our gamma factor (squared), which is

\begin{align*}
1 - \frac{\Bv^2}{c^2} 
&= 1 - c^2 \frac{\Bp^2}{E^2} \\
&= \inv{E^2} \left( \Bp^2 c^2 + m^2 c^4  - c^2 {\Bp^2} \right) \\
&= \frac{m^2 c^4}{E^2} \\
\end{align*}

So, we have the energy half of \ref{eqn:pauli_qm_relativity_intro:p1_4}

\begin{align*}
E^2 &= \frac{m^2 c^4}{1 - \frac{\Bv^2}{c^2} }
\end{align*}

For the momentum we then have
\begin{align*}
\Bp^2 c^2 + m^2 c^4 &= \frac{m^2 c^4}{1 - \frac{\Bv^2}{c^2} }
\end{align*}
\begin{align*}
\Bp^2 
&= \frac{m^2 c^2}{1 - \frac{\Bv^2}{c^2} } - m^2 c^2 \frac{(1 - \frac{\Bv^2}{c^2} )}{1 - \frac{\Bv^2}{c^2} } \\
&= \frac{m^2 \Bv^2}{1 - \frac{\Bv^2}{c^2} } 
\end{align*}

the second half of \ref{eqn:pauli_qm_relativity_intro:p1_4}.

Pretty cool.  Given the energy momentum invariant, $m^2 c^2 = E^2/c^2 - \Bp^2$, and a requirement that the velocity, momentum, Kinetic energy combination is related precisely as in classical mechanics, with $v_k = \PDi{p_k}{E}$, we
recover the relativistic energy momentum four vector.

This is probably not surprising to somebody who knows relativity better than I, but it was
interesting to me to see this worked ``backwards'' this way.

\subsection{Afternote. }

A timely listening to Susskind's classical mechanics lecture 6, shows that 
this surprising method used by Pauli to work backwards from the energy 
is in fact a use of the Hamiltonian formalism to relate energy, velocity
and position.  We see here that one logically just has to pick the ``right''
energy construct, then the familiar relativistic energy and momentum relations
follow directly.  This requires nothing more than using the Hamiltonian
relationships in the same way that we would get the Newtonian equations
of motion from a classical energy relationship.

My failure to study the Hamiltonian formalism now stands out.  I planned to 
get to it eventually in a QM context, but Pauli shows here that an understanding
of that tool set is well justified in a classical mechanics context as well.

%\bibliographystyle{plainnat}
%\bibliography{myrefs}

%\end{document}

%
% Copyright � 2012 Peeter Joot.  All Rights Reserved.
% Licenced as described in the file LICENSE under the root directory of this GIT repository.
%

% 
% 
%\documentclass{article}

%\usepackage{amsmath}
\usepackage{mathpazo}

%
% shorthand for bold symbols, convenient for vectors and matrices
%
\newcommand{\Ba}[0]{\mathbf{a}}
\newcommand{\Bb}[0]{\mathbf{b}}
\newcommand{\Bc}[0]{\mathbf{c}}
\newcommand{\Bd}[0]{\mathbf{d}}
\newcommand{\Be}[0]{\mathbf{e}}
\newcommand{\Bf}[0]{\mathbf{f}}
\newcommand{\Bg}[0]{\mathbf{g}}
\newcommand{\Bh}[0]{\mathbf{h}}
\newcommand{\Bi}[0]{\mathbf{i}}
\newcommand{\Bj}[0]{\mathbf{j}}
\newcommand{\Bk}[0]{\mathbf{k}}
\newcommand{\Bl}[0]{\mathbf{l}}
\newcommand{\Bm}[0]{\mathbf{m}}
\newcommand{\Bn}[0]{\mathbf{n}}
\newcommand{\Bo}[0]{\mathbf{o}}
\newcommand{\Bp}[0]{\mathbf{p}}
\newcommand{\Bq}[0]{\mathbf{q}}
\newcommand{\Br}[0]{\mathbf{r}}
\newcommand{\Bs}[0]{\mathbf{s}}
\newcommand{\Bt}[0]{\mathbf{t}}
\newcommand{\Bu}[0]{\mathbf{u}}
\newcommand{\Bv}[0]{\mathbf{v}}
\newcommand{\Bw}[0]{\mathbf{w}}
\newcommand{\Bx}[0]{\mathbf{x}}
\newcommand{\By}[0]{\mathbf{y}}
\newcommand{\Bz}[0]{\mathbf{z}}
\newcommand{\BA}[0]{\mathbf{A}}
\newcommand{\BB}[0]{\mathbf{B}}
\newcommand{\BC}[0]{\mathbf{C}}
\newcommand{\BD}[0]{\mathbf{D}}
\newcommand{\BE}[0]{\mathbf{E}}
\newcommand{\BF}[0]{\mathbf{F}}
\newcommand{\BG}[0]{\mathbf{G}}
\newcommand{\BH}[0]{\mathbf{H}}
\newcommand{\BI}[0]{\mathbf{I}}
\newcommand{\BJ}[0]{\mathbf{J}}
\newcommand{\BK}[0]{\mathbf{K}}
\newcommand{\BL}[0]{\mathbf{L}}
\newcommand{\BM}[0]{\mathbf{M}}
\newcommand{\BN}[0]{\mathbf{N}}
\newcommand{\BO}[0]{\mathbf{O}}
\newcommand{\BP}[0]{\mathbf{P}}
\newcommand{\BQ}[0]{\mathbf{Q}}
\newcommand{\BR}[0]{\mathbf{R}}
\newcommand{\BS}[0]{\mathbf{S}}
\newcommand{\BT}[0]{\mathbf{T}}
\newcommand{\BU}[0]{\mathbf{U}}
\newcommand{\BV}[0]{\mathbf{V}}
\newcommand{\BW}[0]{\mathbf{W}}
\newcommand{\BX}[0]{\mathbf{X}}
\newcommand{\BY}[0]{\mathbf{Y}}
\newcommand{\BZ}[0]{\mathbf{Z}}

\newcommand{\Bzero}[0]{\mathbf{0}}
\newcommand{\Btheta}[0]{\boldsymbol{\theta}}
\newcommand{\Btau}[0]{\boldsymbol{\tau}}
\newcommand{\Bomega}[0]{\boldsymbol{\omega}}

%
% shorthand for unit vectors
%
\newcommand{\acap}[0]{\hat{\Ba}}
\newcommand{\bcap}[0]{\hat{\Bb}}
\newcommand{\ccap}[0]{\hat{\Bc}}
\newcommand{\dcap}[0]{\hat{\Bd}}
\newcommand{\ecap}[0]{\hat{\Be}}
\newcommand{\fcap}[0]{\hat{\Bf}}
\newcommand{\gcap}[0]{\hat{\Bg}}
\newcommand{\hcap}[0]{\hat{\Bh}}
\newcommand{\icap}[0]{\hat{\Bi}}
\newcommand{\jcap}[0]{\hat{\Bj}}
\newcommand{\kcap}[0]{\hat{\Bk}}
\newcommand{\lcap}[0]{\hat{\Bl}}
\newcommand{\mcap}[0]{\hat{\Bm}}
\newcommand{\ncap}[0]{\hat{\Bn}}
\newcommand{\ocap}[0]{\hat{\Bo}}
\newcommand{\pcap}[0]{\hat{\Bp}}
\newcommand{\qcap}[0]{\hat{\Bq}}
\newcommand{\rcap}[0]{\hat{\Br}}
\newcommand{\scap}[0]{\hat{\Bs}}
\newcommand{\tcap}[0]{\hat{\Bt}}
\newcommand{\ucap}[0]{\hat{\Bu}}
\newcommand{\vcap}[0]{\hat{\Bv}}
\newcommand{\wcap}[0]{\hat{\Bw}}
\newcommand{\xcap}[0]{\hat{\Bx}}
\newcommand{\ycap}[0]{\hat{\By}}
\newcommand{\zcap}[0]{\hat{\Bz}}
\newcommand{\thetacap}[0]{\hat{\Btheta}}

%
% to write R^n and C^n in a distinguishable fashion.  Perhaps change this
% to the double lined characters upon figuring out how to do so.
%
\newcommand{\C}[1]{$\mathbb{C}^{#1}$}
\newcommand{\R}[1]{$\mathbb{R}^{#1}$}

%
% various generally useful helpers
%

% derivative of #1 wrt. #2:
\newcommand{\D}[2] {\frac {d#2} {d#1}}

\newcommand{\inv}[1]{\frac{1}{#1}}
\newcommand{\cross}[0]{\times}

\newcommand{\abs}[1]{\lvert{#1}\rvert}
\newcommand{\norm}[1]{\lVert{#1}\rVert}
\newcommand{\innerprod}[2]{\langle{#1}, {#2}\rangle}
\newcommand{\dotprod}[2]{{#1} \cdot {#2}}
\newcommand{\bdotprod}[2]{\left({#1} \cdot {#2}\right)}
\newcommand{\crossprod}[2]{{#1} \cross {#2}}
\newcommand{\tripleprod}[3]{\dotprod{\left(\crossprod{#1}{#2}\right)}{#3}}

\DeclareMathOperator{\Proj}{Proj}
\DeclareMathOperator{\Span}{span}
\DeclareMathOperator{\Sgn}{sgn}
\DeclareMathOperator{\Area}{Area}
\DeclareMathOperator{\Volume}{Volume}

%
% A few miscellaneous things specific to this document
%
\newcommand{\crossop}[1]{\crossprod{#1}{}}

% R2 vector.
\newcommand{\VectorTwo}[2]{
\begin{bmatrix}
 {#1} \\
 {#2}
\end{bmatrix}
}

\newcommand{\VectorN}[1]{
\begin{bmatrix}
{#1}_1 \\
{#1}_2 \\
\vdots \\
{#1}_N \\
\end{bmatrix}
}

\newcommand{\DETuvij}[4]{
\begin{vmatrix}
 {#1}_{#3} & {#1}_{#4} \\
 {#2}_{#3} & {#2}_{#4}
\end{vmatrix}
}

\newcommand{\DETuvwijk}[6]{
\begin{vmatrix}
 {#1}_{#4} & {#1}_{#5} & {#1}_{#6} \\
 {#2}_{#4} & {#2}_{#5} & {#2}_{#6} \\
 {#3}_{#4} & {#3}_{#5} & {#3}_{#6}
\end{vmatrix}
}

\newcommand{\DETuvwxijkl}[8]{
\begin{vmatrix}
 {#1}_{#5} & {#1}_{#6} & {#1}_{#7} & {#1}_{#8} \\
 {#2}_{#5} & {#2}_{#6} & {#2}_{#7} & {#2}_{#8} \\
 {#3}_{#5} & {#3}_{#6} & {#3}_{#7} & {#3}_{#8} \\
 {#4}_{#5} & {#4}_{#6} & {#4}_{#7} & {#4}_{#8} \\
\end{vmatrix}
}

%\newcommand{\DETuvwxyijklm}[10]{
%\begin{vmatrix}
% {#1}_{#6} & {#1}_{#7} & {#1}_{#8} & {#1}_{#9} & {#1}_{#10} \\
% {#2}_{#6} & {#2}_{#7} & {#2}_{#8} & {#2}_{#9} & {#2}_{#10} \\
% {#3}_{#6} & {#3}_{#7} & {#3}_{#8} & {#3}_{#9} & {#3}_{#10} \\
% {#4}_{#6} & {#4}_{#7} & {#4}_{#8} & {#4}_{#9} & {#4}_{#10} \\
% {#5}_{#6} & {#5}_{#7} & {#5}_{#8} & {#5}_{#9} & {#5}_{#10}
%\end{vmatrix}
%}

% R3 vector.
\newcommand{\VectorThree}[3]{
\begin{bmatrix}
 {#1} \\
 {#2} \\
 {#3}
\end{bmatrix}
}


%%<misc>
%
\newcommand{\Abs}[1]{{\left\lvert{#1}\right\rvert}}
\newcommand{\spacegrad}[0]{\boldsymbol{\nabla}}
\newcommand{\grad}[0]{\nabla}
\newcommand{\LL}[0]{\mathcal{L}}

% == \partial_{#1} {#2}
\newcommand{\PD}[2]{\frac{\partial {#2}}{\partial {#1}}}
% inline variant
\newcommand{\PDi}[2]{{\partial {#2}}/{\partial {#1}}}

\newcommand{\PDD}[3]{\frac{\partial^2 {#3}}{\partial {#1}\partial {#2}}}
%\newcommand{\PDd}[2]{\frac{\partial^2 {#2}}{{\partial{#1}}^2}}
\newcommand{\PDsq}[2]{\frac{\partial^2 {#2}}{(\partial {#1})^2}}

\newcommand{\Partial}[2]{\frac{\partial {#1}}{\partial {#2}}}
\DeclareMathOperator{\RejName}{Rej}
\newcommand{\Rej}[2]{\RejName_{#1}\left( {#2} \right)}
\newcommand{\Rm}[1]{\mathbb{R}^{#1}}
\newcommand{\Cm}[1]{\mathbb{C}^{#1}}
\newcommand{\conj}[0]{{*}}

%</misc>

% <grade selection>
%
\newcommand{\gpgrade}[2] {{\left\langle{{#1}}\right\rangle}_{#2}}

\newcommand{\gpgradezero}[1] {\gpgrade{#1}{}}
%\newcommand{\gpscalargrade}[1] {{\left\langle{{#1}}\right\rangle}}
%\newcommand{\gpgradezero}[1] {\gpgrade{#1}{0}}

%\newcommand{\gpgradeone}[1] {{\left\langle{{#1}}\right\rangle}_{1}}
\newcommand{\gpgradeone}[1] {\gpgrade{#1}{1}}

\newcommand{\gpgradetwo}[1] {\gpgrade{#1}{2}}
\newcommand{\gpgradethree}[1] {\gpgrade{#1}{3}}
\newcommand{\gpgradefour}[1] {\gpgrade{#1}{4}}
%
% </grade selection>



\newcommand{\adot}[0]{{\dot{a}}}
\newcommand{\bdot}[0]{{\dot{b}}}
% taken for centered dot:
%\newcommand{\cdot}[0]{{\dot{c}}}
%\newcommand{\ddot}[0]{{\dot{d}}}
\newcommand{\edot}[0]{{\dot{e}}}
\newcommand{\fdot}[0]{{\dot{f}}}
\newcommand{\gdot}[0]{{\dot{g}}}
\newcommand{\hdot}[0]{{\dot{h}}}
\newcommand{\idot}[0]{{\dot{i}}}
\newcommand{\jdot}[0]{{\dot{j}}}
\newcommand{\kdot}[0]{{\dot{k}}}
\newcommand{\ldot}[0]{{\dot{l}}}
\newcommand{\mdot}[0]{{\dot{m}}}
\newcommand{\ndot}[0]{{\dot{n}}}
%\newcommand{\odot}[0]{{\dot{o}}}
\newcommand{\pdot}[0]{{\dot{p}}}
\newcommand{\qdot}[0]{{\dot{q}}}
\newcommand{\rdot}[0]{{\dot{r}}}
\newcommand{\sdot}[0]{{\dot{s}}}
\newcommand{\tdot}[0]{{\dot{t}}}
\newcommand{\udot}[0]{{\dot{u}}}
\newcommand{\vdot}[0]{{\dot{v}}}
\newcommand{\wdot}[0]{{\dot{w}}}
\newcommand{\xdot}[0]{{\dot{x}}}
\newcommand{\ydot}[0]{{\dot{y}}}
\newcommand{\zdot}[0]{{\dot{z}}}
\newcommand{\addot}[0]{{\ddot{a}}}
\newcommand{\bddot}[0]{{\ddot{b}}}
\newcommand{\cddot}[0]{{\ddot{c}}}
%\newcommand{\dddot}[0]{{\ddot{d}}}
\newcommand{\eddot}[0]{{\ddot{e}}}
\newcommand{\fddot}[0]{{\ddot{f}}}
\newcommand{\gddot}[0]{{\ddot{g}}}
\newcommand{\hddot}[0]{{\ddot{h}}}
\newcommand{\iddot}[0]{{\ddot{i}}}
\newcommand{\jddot}[0]{{\ddot{j}}}
\newcommand{\kddot}[0]{{\ddot{k}}}
\newcommand{\lddot}[0]{{\ddot{l}}}
\newcommand{\mddot}[0]{{\ddot{m}}}
\newcommand{\nddot}[0]{{\ddot{n}}}
\newcommand{\oddot}[0]{{\ddot{o}}}
\newcommand{\pddot}[0]{{\ddot{p}}}
\newcommand{\qddot}[0]{{\ddot{q}}}
\newcommand{\rddot}[0]{{\ddot{r}}}
\newcommand{\sddot}[0]{{\ddot{s}}}
\newcommand{\tddot}[0]{{\ddot{t}}}
\newcommand{\uddot}[0]{{\ddot{u}}}
\newcommand{\vddot}[0]{{\ddot{v}}}
\newcommand{\wddot}[0]{{\ddot{w}}}
\newcommand{\xddot}[0]{{\ddot{x}}}
\newcommand{\yddot}[0]{{\ddot{y}}}
\newcommand{\zddot}[0]{{\ddot{z}}}

%<bold and dot greek symbols>
%

\newcommand{\Deltadot}[0]{{\dot{\Delta}}}
\newcommand{\Gammadot}[0]{{\dot{\Gamma}}}
\newcommand{\Lambdadot}[0]{{\dot{\Lambda}}}
\newcommand{\Omegadot}[0]{{\dot{\Omega}}}
\newcommand{\Phidot}[0]{{\dot{\Phi}}}
\newcommand{\Pidot}[0]{{\dot{\Pi}}}
\newcommand{\Psidot}[0]{{\dot{\Psi}}}
\newcommand{\Sigmadot}[0]{{\dot{\Sigma}}}
\newcommand{\Thetadot}[0]{{\dot{\Theta}}}
\newcommand{\Upsilondot}[0]{{\dot{\Upsilon}}}
\newcommand{\Xidot}[0]{{\dot{\Xi}}}
\newcommand{\alphadot}[0]{{\dot{\alpha}}}
\newcommand{\betadot}[0]{{\dot{\beta}}}
\newcommand{\chidot}[0]{{\dot{\chi}}}
\newcommand{\deltadot}[0]{{\dot{\delta}}}
\newcommand{\epsilondot}[0]{{\dot{\epsilon}}}
\newcommand{\etadot}[0]{{\dot{\eta}}}
\newcommand{\gammadot}[0]{{\dot{\gamma}}}
\newcommand{\kappadot}[0]{{\dot{\kappa}}}
\newcommand{\lambdadot}[0]{{\dot{\lambda}}}
\newcommand{\mudot}[0]{{\dot{\mu}}}
\newcommand{\nudot}[0]{{\dot{\nu}}}
\newcommand{\omegadot}[0]{{\dot{\omega}}}
\newcommand{\phidot}[0]{{\dot{\phi}}}
\newcommand{\pidot}[0]{{\dot{\pi}}}
\newcommand{\psidot}[0]{{\dot{\psi}}}
\newcommand{\rhodot}[0]{{\dot{\rho}}}
\newcommand{\sigmadot}[0]{{\dot{\sigma}}}
\newcommand{\taudot}[0]{{\dot{\tau}}}
\newcommand{\thetadot}[0]{{\dot{\theta}}}
\newcommand{\upsilondot}[0]{{\dot{\upsilon}}}
\newcommand{\varepsilondot}[0]{{\dot{\varepsilon}}}
\newcommand{\varphidot}[0]{{\dot{\varphi}}}
\newcommand{\varpidot}[0]{{\dot{\varpi}}}
\newcommand{\varrhodot}[0]{{\dot{\varrho}}}
\newcommand{\varsigmadot}[0]{{\dot{\varsigma}}}
\newcommand{\varthetadot}[0]{{\dot{\vartheta}}}
\newcommand{\xidot}[0]{{\dot{\xi}}}
\newcommand{\zetadot}[0]{{\dot{\zeta}}}

\newcommand{\Deltaddot}[0]{{\ddot{\Delta}}}
\newcommand{\Gammaddot}[0]{{\ddot{\Gamma}}}
\newcommand{\Lambdaddot}[0]{{\ddot{\Lambda}}}
\newcommand{\Omegaddot}[0]{{\ddot{\Omega}}}
\newcommand{\Phiddot}[0]{{\ddot{\Phi}}}
\newcommand{\Piddot}[0]{{\ddot{\Pi}}}
\newcommand{\Psiddot}[0]{{\ddot{\Psi}}}
\newcommand{\Sigmaddot}[0]{{\ddot{\Sigma}}}
\newcommand{\Thetaddot}[0]{{\ddot{\Theta}}}
\newcommand{\Upsilonddot}[0]{{\ddot{\Upsilon}}}
\newcommand{\Xiddot}[0]{{\ddot{\Xi}}}
\newcommand{\alphaddot}[0]{{\ddot{\alpha}}}
\newcommand{\betaddot}[0]{{\ddot{\beta}}}
\newcommand{\chiddot}[0]{{\ddot{\chi}}}
\newcommand{\deltaddot}[0]{{\ddot{\delta}}}
\newcommand{\epsilonddot}[0]{{\ddot{\epsilon}}}
\newcommand{\etaddot}[0]{{\ddot{\eta}}}
\newcommand{\gammaddot}[0]{{\ddot{\gamma}}}
\newcommand{\kappaddot}[0]{{\ddot{\kappa}}}
\newcommand{\lambdaddot}[0]{{\ddot{\lambda}}}
\newcommand{\muddot}[0]{{\ddot{\mu}}}
\newcommand{\nuddot}[0]{{\ddot{\nu}}}
\newcommand{\omegaddot}[0]{{\ddot{\omega}}}
\newcommand{\phiddot}[0]{{\ddot{\phi}}}
\newcommand{\piddot}[0]{{\ddot{\pi}}}
\newcommand{\psiddot}[0]{{\ddot{\psi}}}
\newcommand{\rhoddot}[0]{{\ddot{\rho}}}
\newcommand{\sigmaddot}[0]{{\ddot{\sigma}}}
\newcommand{\tauddot}[0]{{\ddot{\tau}}}
\newcommand{\thetaddot}[0]{{\ddot{\theta}}}
\newcommand{\upsilonddot}[0]{{\ddot{\upsilon}}}
\newcommand{\varepsilonddot}[0]{{\ddot{\varepsilon}}}
\newcommand{\varphiddot}[0]{{\ddot{\varphi}}}
\newcommand{\varpiddot}[0]{{\ddot{\varpi}}}
\newcommand{\varrhoddot}[0]{{\ddot{\varrho}}}
\newcommand{\varsigmaddot}[0]{{\ddot{\varsigma}}}
\newcommand{\varthetaddot}[0]{{\ddot{\vartheta}}}
\newcommand{\xiddot}[0]{{\ddot{\xi}}}
\newcommand{\zetaddot}[0]{{\ddot{\zeta}}}

\newcommand{\BDelta}[0]{\boldsymbol{\Delta}}
\newcommand{\BGamma}[0]{\boldsymbol{\Gamma}}
\newcommand{\BLambda}[0]{\boldsymbol{\Lambda}}
\newcommand{\BOmega}[0]{\boldsymbol{\Omega}}
\newcommand{\BPhi}[0]{\boldsymbol{\Phi}}
\newcommand{\BPi}[0]{\boldsymbol{\Pi}}
\newcommand{\BPsi}[0]{\boldsymbol{\Psi}}
\newcommand{\BSigma}[0]{\boldsymbol{\Sigma}}
\newcommand{\BTheta}[0]{\boldsymbol{\Theta}}
\newcommand{\BUpsilon}[0]{\boldsymbol{\Upsilon}}
\newcommand{\BXi}[0]{\boldsymbol{\Xi}}
\newcommand{\Balpha}[0]{\boldsymbol{\alpha}}
\newcommand{\Bbeta}[0]{\boldsymbol{\beta}}
\newcommand{\Bchi}[0]{\boldsymbol{\chi}}
\newcommand{\Bdelta}[0]{\boldsymbol{\delta}}
\newcommand{\Bepsilon}[0]{\boldsymbol{\epsilon}}
\newcommand{\Beta}[0]{\boldsymbol{\eta}}
\newcommand{\Bgamma}[0]{\boldsymbol{\gamma}}
\newcommand{\Bkappa}[0]{\boldsymbol{\kappa}}
\newcommand{\Blambda}[0]{\boldsymbol{\lambda}}
\newcommand{\Bmu}[0]{\boldsymbol{\mu}}
\newcommand{\Bnu}[0]{\boldsymbol{\nu}}
%\newcommand{\Bomega}[0]{\boldsymbol{\omega}}
\newcommand{\Bphi}[0]{\boldsymbol{\phi}}
\newcommand{\Bpi}[0]{\boldsymbol{\pi}}
\newcommand{\Bpsi}[0]{\boldsymbol{\psi}}
\newcommand{\Brho}[0]{\boldsymbol{\rho}}
\newcommand{\Bsigma}[0]{\boldsymbol{\sigma}}
%\newcommand{\Btau}[0]{\boldsymbol{\tau}}
%\newcommand{\Btheta}[0]{\boldsymbol{\theta}}
\newcommand{\Bupsilon}[0]{\boldsymbol{\upsilon}}
\newcommand{\Bvarepsilon}[0]{\boldsymbol{\varepsilon}}
\newcommand{\Bvarphi}[0]{\boldsymbol{\varphi}}
\newcommand{\Bvarpi}[0]{\boldsymbol{\varpi}}
\newcommand{\Bvarrho}[0]{\boldsymbol{\varrho}}
\newcommand{\Bvarsigma}[0]{\boldsymbol{\varsigma}}
\newcommand{\Bvartheta}[0]{\boldsymbol{\vartheta}}
\newcommand{\Bxi}[0]{\boldsymbol{\xi}}
\newcommand{\Bzeta}[0]{\boldsymbol{\zeta}}
%
%</bold and dot greek symbols>
%<infrequent>
%
%\newcommand{\AreaOp}[1]{\AName_{#1}}
%\newcommand{\Babs}[0]{\abs{\BB}}
%\newcommand{\Bcap}[0]{\hat{\BB}}
%\newcommand{\BrPrimeRej}[0]{\rcap(\rcap \wedge \Br')}
%\newcommand{\CA}[0]{\mathcal{A}}
%\newcommand{\Cos}[1]{\cos{\left({#1}\right)}}
%\newcommand{\Det}[1] {\abs{#1}}
%\newcommand{\Dsq}[2] {\frac {\partial^2 {#1}} {\partial {#2}^2}}
%\newcommand{\Exp}[1]{\exp{\left({#1}\right)}}
%\newcommand{\Norm}[1]{\left\lVert{#1}\right\rVert}
%\newcommand{\Sin}[1]{\sin{\left({#1}\right)}}
%\newcommand{\T}[0]{\text{T}}
%\newcommand{\VolumeOp}[1]{\VName_{#1}}
%\newcommand{\agrad}[0]{\Ba \cdot \nabla}
%\newcommand{\alphacap}[0]{\hat{\boldsymbol{\alpha}}}
%\newcommand{\Fcap}[0]{\hat{\BF}}
%\newcommand{\bithree}[0]{{\Bi}_3}
%\newcommand{\bxa}[0]{\Bx\Ba}
%\newcommand{\coordvec}[2]{
%\newcommand{\costheta}[0]{\acap \cdot \xcap}
%\newcommand{\ddt}[1]{\ddot{#1}}
%\newcommand{\ddu}[1] {\frac {d{#1}} {du}}
%\newcommand{\dsqxj}[2] {\frac {\partial^2 {#1}} {\partial {x_{#2}}^2}}
%\newcommand{\dtheta}[1]{\frac{d {#1}}{d \theta}}
%\newcommand{\dt}[1]{\dot{#1}}
%\newcommand{\dt}[1]{\frac{d {#1}}{dt}}
%\newcommand{\dxj}[2] {\frac {\partial {#1}} {\partial {x_{#2}}}}
%\newcommand{\halfPhi}[0]{\frac{\phi}{2}}
%\newcommand{\half}[0]{\inv{2}}
%\newcommand{\inv}[1]{\frac{1}{#1}}
%\newcommand{\laplacian}[0]{\nabla^2}
%\newcommand{\matrixoftx}[3]{
%\newcommand{\nrrp}[0]{\norm{\rcap \wedge \Br'}}
%\newcommand{\oiint}{\bigcirc \hspace{-1.4em} \int \hspace{-.8em} \int}
%\newcommand{\transpose}[1]{{#1}^{\text{T}}}
%\newcommand{\transpose}[1]{{{#1}^{\TextTranspose}}}
%\newcommand{\transpose}[1]{{{#1}^{\text{T}}}}
%\newcommand{\barA}[0]{\bar{A}}
%\newcommand{\qbar}[0]{\bar{q}}
%\newcommand{\qdotbar}[0]{\dot{\bar{q}}}
%
%</infrequent>





%\usepackage[bookmarks=true]{hyperref}

%\usepackage{color,cite,graphicx}
   % use colour in the document, put your citations as [1-4]
   % rather than [1,2,3,4] (it looks nicer, and the extended LaTeX2e
   % graphics package. 
%\usepackage{latexsym,amssymb,epsf} % do not remember if these are
   % needed, but their inclusion can not do any damage


\chapter{Some notes on Pauli Relativity Velocity addition}
\label{chap:PJpauliVelocityAddition}
%\author{Peeter Joot \quad peeter.joot@gmail.com}
\date{ Dec 25, 2008.  velocityAddition.tex }

%\begin{document}
%\maketitle{}
%
%\tableofcontents

\section{Motivation}

Fill out some details from part 1.6, velocity addition of \citep{pauli1981tr}.

\section{}

Given a path \(x^i(t')\) i n the moving (primed) frame \(S'\), the aim is to 
express the observed velocities for this path from the rest frame \(S\).

From the Lorentz transformation of the coordinates we have (working with \(c=1\)) for a point in the moving frame

\begin{equation}\label{eqn:velocityAddition:20}
\begin{aligned}
x' &= \gamma ( x - vt) \\
t' &= \gamma ( t - vx)
\end{aligned}
\end{equation}

Or reversed (inverting velocities)

\begin{equation}\label{eqn:velocityAddition:40}
\begin{aligned}
x &= \gamma ( x' + vt') \\
t &= \gamma ( t' + vx')
\end{aligned}
\end{equation}

taking differentials we have

\begin{equation}\label{eqn:velocityAddition:60}
\begin{aligned}
dx &= \gamma ( dx' + v dt') \\
dy &= dy' \\
dz &= dz' \\
dt &= \gamma ( dt' + v dx')
\end{aligned}
\end{equation}

Dividing by \(dt'\), this is

\begin{equation}\label{eqn:velocityAddition:80}
\begin{aligned}
\frac{dx}{dt} &= \frac{ dx' + v dt' }{ dt' + v dx'} \\
\frac{dy}{dt} &= \frac{dy'}{\gamma (dt' + v dx')} \\
\frac{dz}{dt} &= \frac{dz'}{\gamma (dt' + v dx')} \\
\end{aligned}
\end{equation}

FIXME: do not like this dividing by differentials.  Try to re-express this 
using the chain rule.

Or in terms of velocity coordinates we have Pauli's equation 10.

\begin{equation}\label{eqn:velocityAddition:100}
\begin{aligned}
u_x &= \frac{ {u_x}' + v  }{ 1 + v {u_x}'} \\
u_y &= \frac{{u_y}'}{\gamma (1 + v {u_x}')} \\
u_z &= \frac{{u_z}'}{\gamma (1 + v {u_x}')} \\
\end{aligned}
\end{equation}

Next he writes \(u^2 = \sum_i {u_i}^2\), in terms of the primed velocities

\begin{equation}\label{eqn:velocityAddition:120}
\begin{aligned}
u^2 &= {u_x}^2 +{u_y}^2 +{u_z}^2  \\
&=
\inv{( 1 + v {u_x}' )^2 } \left(
( {u_x}' + v  )^2
+(1-v^2){{u_y}'}^2
+(1-v^2){{u_z}'}^2
\right) \\
&=
\inv{( 1 + v {u_x}' )^2 } \left(
{{u_x}'}^2
+ 2 {u_x}' v
+ v^2
+(1-v^2){{u_y}'}^2
+(1-v^2){{u_z}'}^2
+v^2 {{u_x}'}^2
-v^2 {{u_x}'}^2
\right) \\
&=
\inv{( 1 + v {u_x}' )^2 } \left(
{{u}'}^2
+ 2 {u_x}' v
+ v^2
+v^2 {{u_x}'}^2
-v^2 {{u}'}^2
\right) \\
\end{aligned}
\end{equation}

This leaves the squared velocity of the path as viewed from the rest frame as
\begin{equation}\label{eqn:velocity_addition:Usquare}
\begin{aligned}
u^2 &=
\inv{( 1 + v {u_x}' )^2 } \left(
{{u}'}^2(1-v^2)
+ 2 {u_x}' v
+v^2 (1 + {{u_x}'}^2)
\right)
\end{aligned}
\end{equation}

Now direction cosines for the velocity direction vector are introduced

\begin{equation}\label{eqn:velocityAddition:140}
\begin{aligned}
\ucap' &= (u_x', u_y', u_z')/u = (\cos\alpha', \cos\sigma', \cos\delta') \\
\ucap &= (u_x, u_y, u_z)/u = (\cos\alpha, \cos\sigma, \cos\delta)
\end{aligned}
\end{equation}

and in terms of the direction cosines one has equation 11:

\begin{equation}\label{eqn:velocityAddition:160}
\begin{aligned}
u^2 &= 
\inv{( 1 + v {u_x}' )^2 } \left(
{{u}'}^2
+v^2 
+ 2 u' \cos\alpha' v
+v^2 {u'}^2 ({\cos\alpha'}^2 - 1 )
\right) \\
&= \inv{( 1 + v u' \cos\alpha' )^2 } \left(
{{u}'}^2
+v^2 
+ 2 u' \cos\alpha' v
-v^2 {u'}^2 \sin^2\alpha'
\right) \\
\end{aligned}
\end{equation}

Pauli's equation 11a is \(1-u^2\), which is then factored into a tidy form
\begin{equation}\label{eqn:velocityAddition:180}
\begin{aligned}
1- u^2
&= \inv{( 1 + v u' \cos\alpha' )^2 } \left(
1 
+ 2 v u' \cos\alpha' 
+v^2 {u'}^2 \cos^2\alpha' 
-{{u}'}^2
-v^2 
- 2 u' v \cos\alpha' 
+v^2 {u'}^2 \sin^2\alpha'
\right) \\
&= \inv{( 1 + v u' \cos\alpha' )^2 } \left(
1 
-{{u}'}^2
-v^2 
+v^2 {u'}^2 
\right) \\
&= \frac{(1 -v^2 ) (1 -{{u}'}^2)}
{( 1 + v u' \cos\alpha' )^2 }
\end{aligned}
\end{equation}

This provides the gamma factor for the effective velocity as observed from the rest frame

\begin{equation}\label{eqn:velocityAddition:200}
\begin{aligned}
\inv{\sqrt{1-u^2}} 
&= \frac{ 1 + v u' \cos\alpha' }{\sqrt{1 -v^2 }\sqrt{1 -{u'}^2}}
\end{aligned}
\end{equation}

With the factors of c's retained, and for the special case where the velocity is colinear with the frame motion, this is part of the velocity addition equation found in many intro relativity treatments.  The generalization required is that instead of the second velocity itself we have the projection of that velocity
in the direction of the frame motion.
For the special
case of when the velocity \(u'\) is directed with the path of the moving frame,
the cosine will be unity, and the projection of that velocity in the frame motion direction is exactly the velocity to be compounded:

\begin{equation}\label{eqn:velocityAddition:220}
\begin{aligned}
\inv{\sqrt{1-u^2}} 
&= \frac{ 1 + v u' }{\sqrt{1 -v^2 }\sqrt{1 -{u'}^2}}
\end{aligned}
\end{equation}

We have another special case, considering perpendicular motion, for which we have \(\cos\alpha' = 0\), and thus

\begin{equation}\label{eqn:velocityAddition:240}
\begin{aligned}
\inv{\sqrt{1-u^2}} 
&= \frac{ 1 }{\sqrt{1 -v^2 }\sqrt{1 -{u'}^2}}
\end{aligned}
\end{equation}

Next he calculates \(tan \alpha\).  That is

\begin{equation}\label{eqn:velocityAddition:260}
\begin{aligned}
\tan\alpha 
&= \frac{\sin\alpha}{\cos\alpha} \\
&= \frac{\sqrt{1-\cos^2\alpha}}{\cos\alpha} \\
&= \frac{\sqrt{1-(u_x/u)^2}}{u_x/u} \\
&= \frac{\sqrt{u^2- {u_x}^2}}{u_x} \\
\end{aligned}
\end{equation}

From \eqnref{eqn:velocity_addition:Usquare} we have
\begin{equation}\label{eqn:velocityAddition:280}
\begin{aligned}
u^2 - {u_x}^2
&= \inv{( 1 + v {u_x}' )^2 } \left(
{{u}'}^2(1-v^2)
+ 2 {u_x}' v
+v^2 (1 + {{u_x}'}^2)
-(u_x' + v)^2
\right) \\
&= \inv{( 1 + v {u_x}' )^2 } \left(
{{u}'}^2(1-v^2)
%+ 2 {u_x}' v
+v^2 (1 + {{u_x}'}^2)
- {u_x'}^2 
- v^2 
%- 2 u_x' v
\right) \\
&= \inv{( 1 + v {u_x}' )^2 } \left(
({{u}'}^2 -{u_x'}^2) (1-v^2)
\right) \\
&= \inv{( 1 + v {u_x}' )^2 } \left(
{{u}'}^2(1 -\cos^2\alpha') (1-v^2)
\right) \\
&= \inv{( 1 + v {u_x}' )^2 } \left(
{{u}'}^2 \sin^2\alpha' (1-v^2)
\right) \\
\end{aligned}
\end{equation}

So, the tangent is
\begin{equation}\label{eqn:velocityAddition:300}
\begin{aligned}
\tan\alpha 
&= \pm \frac{ u' \sin\alpha' \sqrt{1-v^2}}{ {u_x}' + v  } \\
&= \pm \frac{ u' \sin\alpha' \sqrt{1-v^2}}{ u'\cos\alpha' + v  } \\
\end{aligned}
\end{equation}

which except for the \(\pm 1\) factor is Pauli's equation twelve.

Note that in this form we see some of the relative vector structure

\begin{equation}\label{eqn:velocityAddition:320}
\begin{aligned}
\frac{v \wedge \gamma_0}{v \cdot \gamma_0}
\end{aligned}
\end{equation}

of the STA four vector formulation (ie: sine and cosine mapping to rejection and projection terms respectively onto the timelike direction).

\subsection{Perpendicular direction cosines}

What are the equivalent relations for the \(y\) and \(z\) direction cosines for the velocity between the two frames?  For the \(y\) direction, our \(\sin^2\sigma\) is

\begin{equation}\label{eqn:velocityAddition:340}
\begin{aligned}
u^2 - u_y^2
&=
\inv{( 1 + v {u_x}' )^2 } \left(
{{u}'}^2(1-v^2)
+ 2 {u_x}' v
+v^2 (1 + {{u_x}'}^2)
-{{u_y}'}^2/\gamma^2
\right) \\
&=
\inv{( 1 + v {u_x}' )^2 } \left(
({{u}'}^2 - {{u_y}'}^2 -{{u_x}'}^2)(1-v^2)
+ 2 {u_x}' v
+v^2 
+ {{u_x}'}^2
\right) \\
&=
\inv{( 1 + v {u_x}' )^2 } \left(
({{u}'}^2 - {{u_y}'}^2 -{{u_x}'}^2)(1-v^2)
+ ({u_x}' + v)^2
\right) \\
&=
\inv{( 1 + v {u_x}' )^2 } \left(
{{u_z}'}^2(1-v^2)
+ ({u_x}' + v)^2
\right) \\
\end{aligned}
\end{equation}

This leaves us with a tangent of 

\begin{equation}\label{eqn:velocityAddition:360}
\begin{aligned}
\tan^2\sigma
&=
\inv{( 1 + v {u_x}' )^2 } \left(
{{u_z}'}^2(1-v^2)
+ ({u_x}' + v)^2
\right)
/\left(\frac{{u_y}'}{\gamma (1 + v {u_x}')}\right)^2 \\
&=
\frac{ \left(
{{u_z}'}^2(1-v^2)
+ ({u_x}' + v)^2
\right)
}{(1-v^2){{u_y}'}^2}
\end{aligned}
\end{equation}

which leaves
\begin{equation}\label{eqn:velocityAddition:380}
\begin{aligned}
\tan\sigma 
&= \pm \frac{ \sqrt{ \cos^2\delta' + \inv{1-v^2}(\cos\alpha' + v)^2 } }{\cos\sigma'}
\end{aligned}
\end{equation}

This is now enough to completely assemble the 
velocity vector as observed from the rest frame

\begin{equation}\label{eqn:velocityAddition:400}
\begin{aligned}
\Bu &=
\sqrt{1 -\frac{(1 -v^2 ) (1 -{{u}'}^2)}{( 1 + v u' \cos\alpha' )^2 }}
\begin{bmatrix}
\cos\left(\tan^{-1}\left(\frac{ u' \sin\alpha' \sqrt{1-v^2}}{ u'\cos\alpha' + v  }\right)\right) \\
\cos\left(\tan^{-1}\left(\frac{ \sqrt{ \cos^2\delta' + \inv{1-v^2}(\cos\alpha' + v)^2 } }{\cos\sigma'}\right)\right) \\
\cos\left(\tan^{-1}\left(\frac{ \sqrt{ \cos^2\sigma' + \inv{1-v^2}(\cos\alpha' + v)^2 } }{\cos\delta'}\right)\right)
\end{bmatrix}
\end{aligned}
\end{equation}

Wow.  What a mess (assuming I even got the algebra right)!

%\bibliographystyle{plainnat}
%\bibliography{myrefs}

%\end{document}

\documentclass{article}      % Specifies the document class

\usepackage{amsmath}
\usepackage{mathpazo}

%
% shorthand for bold symbols, convenient for vectors and matrices
%
\newcommand{\Ba}[0]{\mathbf{a}}
\newcommand{\Bb}[0]{\mathbf{b}}
\newcommand{\Bc}[0]{\mathbf{c}}
\newcommand{\Bd}[0]{\mathbf{d}}
\newcommand{\Be}[0]{\mathbf{e}}
\newcommand{\Bf}[0]{\mathbf{f}}
\newcommand{\Bg}[0]{\mathbf{g}}
\newcommand{\Bh}[0]{\mathbf{h}}
\newcommand{\Bi}[0]{\mathbf{i}}
\newcommand{\Bj}[0]{\mathbf{j}}
\newcommand{\Bk}[0]{\mathbf{k}}
\newcommand{\Bl}[0]{\mathbf{l}}
\newcommand{\Bm}[0]{\mathbf{m}}
\newcommand{\Bn}[0]{\mathbf{n}}
\newcommand{\Bo}[0]{\mathbf{o}}
\newcommand{\Bp}[0]{\mathbf{p}}
\newcommand{\Bq}[0]{\mathbf{q}}
\newcommand{\Br}[0]{\mathbf{r}}
\newcommand{\Bs}[0]{\mathbf{s}}
\newcommand{\Bt}[0]{\mathbf{t}}
\newcommand{\Bu}[0]{\mathbf{u}}
\newcommand{\Bv}[0]{\mathbf{v}}
\newcommand{\Bw}[0]{\mathbf{w}}
\newcommand{\Bx}[0]{\mathbf{x}}
\newcommand{\By}[0]{\mathbf{y}}
\newcommand{\Bz}[0]{\mathbf{z}}
\newcommand{\BA}[0]{\mathbf{A}}
\newcommand{\BB}[0]{\mathbf{B}}
\newcommand{\BC}[0]{\mathbf{C}}
\newcommand{\BD}[0]{\mathbf{D}}
\newcommand{\BE}[0]{\mathbf{E}}
\newcommand{\BF}[0]{\mathbf{F}}
\newcommand{\BG}[0]{\mathbf{G}}
\newcommand{\BH}[0]{\mathbf{H}}
\newcommand{\BI}[0]{\mathbf{I}}
\newcommand{\BJ}[0]{\mathbf{J}}
\newcommand{\BK}[0]{\mathbf{K}}
\newcommand{\BL}[0]{\mathbf{L}}
\newcommand{\BM}[0]{\mathbf{M}}
\newcommand{\BN}[0]{\mathbf{N}}
\newcommand{\BO}[0]{\mathbf{O}}
\newcommand{\BP}[0]{\mathbf{P}}
\newcommand{\BQ}[0]{\mathbf{Q}}
\newcommand{\BR}[0]{\mathbf{R}}
\newcommand{\BS}[0]{\mathbf{S}}
\newcommand{\BT}[0]{\mathbf{T}}
\newcommand{\BU}[0]{\mathbf{U}}
\newcommand{\BV}[0]{\mathbf{V}}
\newcommand{\BW}[0]{\mathbf{W}}
\newcommand{\BX}[0]{\mathbf{X}}
\newcommand{\BY}[0]{\mathbf{Y}}
\newcommand{\BZ}[0]{\mathbf{Z}}

\newcommand{\Bzero}[0]{\mathbf{0}}
\newcommand{\Btheta}[0]{\boldsymbol{\theta}}
\newcommand{\Btau}[0]{\boldsymbol{\tau}}
\newcommand{\Bomega}[0]{\boldsymbol{\omega}}

%
% shorthand for unit vectors
%
\newcommand{\acap}[0]{\hat{\Ba}}
\newcommand{\bcap}[0]{\hat{\Bb}}
\newcommand{\ccap}[0]{\hat{\Bc}}
\newcommand{\dcap}[0]{\hat{\Bd}}
\newcommand{\ecap}[0]{\hat{\Be}}
\newcommand{\fcap}[0]{\hat{\Bf}}
\newcommand{\gcap}[0]{\hat{\Bg}}
\newcommand{\hcap}[0]{\hat{\Bh}}
\newcommand{\icap}[0]{\hat{\Bi}}
\newcommand{\jcap}[0]{\hat{\Bj}}
\newcommand{\kcap}[0]{\hat{\Bk}}
\newcommand{\lcap}[0]{\hat{\Bl}}
\newcommand{\mcap}[0]{\hat{\Bm}}
\newcommand{\ncap}[0]{\hat{\Bn}}
\newcommand{\ocap}[0]{\hat{\Bo}}
\newcommand{\pcap}[0]{\hat{\Bp}}
\newcommand{\qcap}[0]{\hat{\Bq}}
\newcommand{\rcap}[0]{\hat{\Br}}
\newcommand{\scap}[0]{\hat{\Bs}}
\newcommand{\tcap}[0]{\hat{\Bt}}
\newcommand{\ucap}[0]{\hat{\Bu}}
\newcommand{\vcap}[0]{\hat{\Bv}}
\newcommand{\wcap}[0]{\hat{\Bw}}
\newcommand{\xcap}[0]{\hat{\Bx}}
\newcommand{\ycap}[0]{\hat{\By}}
\newcommand{\zcap}[0]{\hat{\Bz}}
\newcommand{\thetacap}[0]{\hat{\Btheta}}

%
% to write R^n and C^n in a distinguishable fashion.  Perhaps change this
% to the double lined characters upon figuring out how to do so.
%
\newcommand{\C}[1]{$\mathbb{C}^{#1}$}
\newcommand{\R}[1]{$\mathbb{R}^{#1}$}

%
% various generally useful helpers
%

% derivative of #1 wrt. #2:
\newcommand{\D}[2] {\frac {d#2} {d#1}}

\newcommand{\inv}[1]{\frac{1}{#1}}
\newcommand{\cross}[0]{\times}

\newcommand{\abs}[1]{\lvert{#1}\rvert}
\newcommand{\norm}[1]{\lVert{#1}\rVert}
\newcommand{\innerprod}[2]{\langle{#1}, {#2}\rangle}
\newcommand{\dotprod}[2]{{#1} \cdot {#2}}
\newcommand{\bdotprod}[2]{\left({#1} \cdot {#2}\right)}
\newcommand{\crossprod}[2]{{#1} \cross {#2}}
\newcommand{\tripleprod}[3]{\dotprod{\left(\crossprod{#1}{#2}\right)}{#3}}

\DeclareMathOperator{\Proj}{Proj}
\DeclareMathOperator{\Span}{span}
\DeclareMathOperator{\Sgn}{sgn}
\DeclareMathOperator{\Area}{Area}
\DeclareMathOperator{\Volume}{Volume}

%
% A few miscellaneous things specific to this document
%
\newcommand{\crossop}[1]{\crossprod{#1}{}}

% R2 vector.
\newcommand{\VectorTwo}[2]{
\begin{bmatrix}
 {#1} \\
 {#2}
\end{bmatrix}
}

\newcommand{\VectorN}[1]{
\begin{bmatrix}
{#1}_1 \\
{#1}_2 \\
\vdots \\
{#1}_N \\
\end{bmatrix}
}

\newcommand{\DETuvij}[4]{
\begin{vmatrix}
 {#1}_{#3} & {#1}_{#4} \\
 {#2}_{#3} & {#2}_{#4}
\end{vmatrix}
}

\newcommand{\DETuvwijk}[6]{
\begin{vmatrix}
 {#1}_{#4} & {#1}_{#5} & {#1}_{#6} \\
 {#2}_{#4} & {#2}_{#5} & {#2}_{#6} \\
 {#3}_{#4} & {#3}_{#5} & {#3}_{#6}
\end{vmatrix}
}

\newcommand{\DETuvwxijkl}[8]{
\begin{vmatrix}
 {#1}_{#5} & {#1}_{#6} & {#1}_{#7} & {#1}_{#8} \\
 {#2}_{#5} & {#2}_{#6} & {#2}_{#7} & {#2}_{#8} \\
 {#3}_{#5} & {#3}_{#6} & {#3}_{#7} & {#3}_{#8} \\
 {#4}_{#5} & {#4}_{#6} & {#4}_{#7} & {#4}_{#8} \\
\end{vmatrix}
}

%\newcommand{\DETuvwxyijklm}[10]{
%\begin{vmatrix}
% {#1}_{#6} & {#1}_{#7} & {#1}_{#8} & {#1}_{#9} & {#1}_{#10} \\
% {#2}_{#6} & {#2}_{#7} & {#2}_{#8} & {#2}_{#9} & {#2}_{#10} \\
% {#3}_{#6} & {#3}_{#7} & {#3}_{#8} & {#3}_{#9} & {#3}_{#10} \\
% {#4}_{#6} & {#4}_{#7} & {#4}_{#8} & {#4}_{#9} & {#4}_{#10} \\
% {#5}_{#6} & {#5}_{#7} & {#5}_{#8} & {#5}_{#9} & {#5}_{#10}
%\end{vmatrix}
%}

% R3 vector.
\newcommand{\VectorThree}[3]{
\begin{bmatrix}
 {#1} \\
 {#2} \\
 {#3}
\end{bmatrix}
}



%
% The real thing:
%

                             % The preamble begins here.
\title{} % Declares the document's title.
\author{Peeter Joot}         % Declares the author's name.
%\date{}        % Deleting this command produces today's date.

\begin{document}             % End of preamble and beginning of text.

\maketitle{}

\section{}

I was summarizing for myself the various four-vectors of mechanics:

\begin{align*}
x &= ct + \mathbf{x} \\
V &= \frac{dx}{d\tau} = \gamma(c + \mathbf{v}) \\
P &= m V = E/c + \gamma\mathbf{p} \\
f &= m\frac{d^2 x}{d\tau^2} = m\frac{d V}{d\tau} \\
\end{align*}

where:

\begin{align*}
\gamma^{-2} &= 1 - {\lvert \mathbf{v}/c \rvert}^2 \\
d\tau &= {\left(\frac{dx}{d\lambda} \cdot \frac{dx}{d\lambda}\right)}^{1/2} d\lambda \\
x \cdot x = {\lvert x \rvert}^2 &= c^2t^2 - {\lvert \mathbf{x} \rvert}^2 \\
E &= \int f \cdot (c d\tau) \\
\mathbf{v} &= \frac{d\mathbf{x}}{dt} \\
\mathbf{p} &= m\mathbf{v} \\
\end{align*}

Invarients for the first three four vectors are:

\begin{align*}
{\lvert x \rvert}^2 &= c^2 t^2 - {\lvert \mathbf{x} \rvert}^2 = c^2 \tau^2 \\
{\lvert V \rvert}^2 &= \gamma^2 (c^2 - {\lvert \mathbf{v} \rvert}^2) = c^2 \\
{\lvert P \rvert}^2 &= m^2 {\lvert V \rvert}^2 = m^2 c^2 \\
\end{align*}

Is the minkowski norm of the four vector force:

\[
f = m\frac{d^2 x}{d\tau^2} 
\]

also an invarient?  I think it has to be.  Assuming that is the case, what would the value (and significance if any) of this be?

\end{document}               % End of document.

%\documentclass{article}

%\usepackage{amsmath}
\usepackage{mathpazo}

%
% shorthand for bold symbols, convenient for vectors and matrices
%
\newcommand{\Ba}[0]{\mathbf{a}}
\newcommand{\Bb}[0]{\mathbf{b}}
\newcommand{\Bc}[0]{\mathbf{c}}
\newcommand{\Bd}[0]{\mathbf{d}}
\newcommand{\Be}[0]{\mathbf{e}}
\newcommand{\Bf}[0]{\mathbf{f}}
\newcommand{\Bg}[0]{\mathbf{g}}
\newcommand{\Bh}[0]{\mathbf{h}}
\newcommand{\Bi}[0]{\mathbf{i}}
\newcommand{\Bj}[0]{\mathbf{j}}
\newcommand{\Bk}[0]{\mathbf{k}}
\newcommand{\Bl}[0]{\mathbf{l}}
\newcommand{\Bm}[0]{\mathbf{m}}
\newcommand{\Bn}[0]{\mathbf{n}}
\newcommand{\Bo}[0]{\mathbf{o}}
\newcommand{\Bp}[0]{\mathbf{p}}
\newcommand{\Bq}[0]{\mathbf{q}}
\newcommand{\Br}[0]{\mathbf{r}}
\newcommand{\Bs}[0]{\mathbf{s}}
\newcommand{\Bt}[0]{\mathbf{t}}
\newcommand{\Bu}[0]{\mathbf{u}}
\newcommand{\Bv}[0]{\mathbf{v}}
\newcommand{\Bw}[0]{\mathbf{w}}
\newcommand{\Bx}[0]{\mathbf{x}}
\newcommand{\By}[0]{\mathbf{y}}
\newcommand{\Bz}[0]{\mathbf{z}}
\newcommand{\BA}[0]{\mathbf{A}}
\newcommand{\BB}[0]{\mathbf{B}}
\newcommand{\BC}[0]{\mathbf{C}}
\newcommand{\BD}[0]{\mathbf{D}}
\newcommand{\BE}[0]{\mathbf{E}}
\newcommand{\BF}[0]{\mathbf{F}}
\newcommand{\BG}[0]{\mathbf{G}}
\newcommand{\BH}[0]{\mathbf{H}}
\newcommand{\BI}[0]{\mathbf{I}}
\newcommand{\BJ}[0]{\mathbf{J}}
\newcommand{\BK}[0]{\mathbf{K}}
\newcommand{\BL}[0]{\mathbf{L}}
\newcommand{\BM}[0]{\mathbf{M}}
\newcommand{\BN}[0]{\mathbf{N}}
\newcommand{\BO}[0]{\mathbf{O}}
\newcommand{\BP}[0]{\mathbf{P}}
\newcommand{\BQ}[0]{\mathbf{Q}}
\newcommand{\BR}[0]{\mathbf{R}}
\newcommand{\BS}[0]{\mathbf{S}}
\newcommand{\BT}[0]{\mathbf{T}}
\newcommand{\BU}[0]{\mathbf{U}}
\newcommand{\BV}[0]{\mathbf{V}}
\newcommand{\BW}[0]{\mathbf{W}}
\newcommand{\BX}[0]{\mathbf{X}}
\newcommand{\BY}[0]{\mathbf{Y}}
\newcommand{\BZ}[0]{\mathbf{Z}}

\newcommand{\Bzero}[0]{\mathbf{0}}
\newcommand{\Btheta}[0]{\boldsymbol{\theta}}
\newcommand{\Btau}[0]{\boldsymbol{\tau}}
\newcommand{\Bomega}[0]{\boldsymbol{\omega}}

%
% shorthand for unit vectors
%
\newcommand{\acap}[0]{\hat{\Ba}}
\newcommand{\bcap}[0]{\hat{\Bb}}
\newcommand{\ccap}[0]{\hat{\Bc}}
\newcommand{\dcap}[0]{\hat{\Bd}}
\newcommand{\ecap}[0]{\hat{\Be}}
\newcommand{\fcap}[0]{\hat{\Bf}}
\newcommand{\gcap}[0]{\hat{\Bg}}
\newcommand{\hcap}[0]{\hat{\Bh}}
\newcommand{\icap}[0]{\hat{\Bi}}
\newcommand{\jcap}[0]{\hat{\Bj}}
\newcommand{\kcap}[0]{\hat{\Bk}}
\newcommand{\lcap}[0]{\hat{\Bl}}
\newcommand{\mcap}[0]{\hat{\Bm}}
\newcommand{\ncap}[0]{\hat{\Bn}}
\newcommand{\ocap}[0]{\hat{\Bo}}
\newcommand{\pcap}[0]{\hat{\Bp}}
\newcommand{\qcap}[0]{\hat{\Bq}}
\newcommand{\rcap}[0]{\hat{\Br}}
\newcommand{\scap}[0]{\hat{\Bs}}
\newcommand{\tcap}[0]{\hat{\Bt}}
\newcommand{\ucap}[0]{\hat{\Bu}}
\newcommand{\vcap}[0]{\hat{\Bv}}
\newcommand{\wcap}[0]{\hat{\Bw}}
\newcommand{\xcap}[0]{\hat{\Bx}}
\newcommand{\ycap}[0]{\hat{\By}}
\newcommand{\zcap}[0]{\hat{\Bz}}
\newcommand{\thetacap}[0]{\hat{\Btheta}}

%
% to write R^n and C^n in a distinguishable fashion.  Perhaps change this
% to the double lined characters upon figuring out how to do so.
%
\newcommand{\C}[1]{$\mathbb{C}^{#1}$}
\newcommand{\R}[1]{$\mathbb{R}^{#1}$}

%
% various generally useful helpers
%

% derivative of #1 wrt. #2:
\newcommand{\D}[2] {\frac {d#2} {d#1}}

\newcommand{\inv}[1]{\frac{1}{#1}}
\newcommand{\cross}[0]{\times}

\newcommand{\abs}[1]{\lvert{#1}\rvert}
\newcommand{\norm}[1]{\lVert{#1}\rVert}
\newcommand{\innerprod}[2]{\langle{#1}, {#2}\rangle}
\newcommand{\dotprod}[2]{{#1} \cdot {#2}}
\newcommand{\bdotprod}[2]{\left({#1} \cdot {#2}\right)}
\newcommand{\crossprod}[2]{{#1} \cross {#2}}
\newcommand{\tripleprod}[3]{\dotprod{\left(\crossprod{#1}{#2}\right)}{#3}}

\DeclareMathOperator{\Proj}{Proj}
\DeclareMathOperator{\Span}{span}
\DeclareMathOperator{\Sgn}{sgn}
\DeclareMathOperator{\Area}{Area}
\DeclareMathOperator{\Volume}{Volume}

%
% A few miscellaneous things specific to this document
%
\newcommand{\crossop}[1]{\crossprod{#1}{}}

% R2 vector.
\newcommand{\VectorTwo}[2]{
\begin{bmatrix}
 {#1} \\
 {#2}
\end{bmatrix}
}

\newcommand{\VectorN}[1]{
\begin{bmatrix}
{#1}_1 \\
{#1}_2 \\
\vdots \\
{#1}_N \\
\end{bmatrix}
}

\newcommand{\DETuvij}[4]{
\begin{vmatrix}
 {#1}_{#3} & {#1}_{#4} \\
 {#2}_{#3} & {#2}_{#4}
\end{vmatrix}
}

\newcommand{\DETuvwijk}[6]{
\begin{vmatrix}
 {#1}_{#4} & {#1}_{#5} & {#1}_{#6} \\
 {#2}_{#4} & {#2}_{#5} & {#2}_{#6} \\
 {#3}_{#4} & {#3}_{#5} & {#3}_{#6}
\end{vmatrix}
}

\newcommand{\DETuvwxijkl}[8]{
\begin{vmatrix}
 {#1}_{#5} & {#1}_{#6} & {#1}_{#7} & {#1}_{#8} \\
 {#2}_{#5} & {#2}_{#6} & {#2}_{#7} & {#2}_{#8} \\
 {#3}_{#5} & {#3}_{#6} & {#3}_{#7} & {#3}_{#8} \\
 {#4}_{#5} & {#4}_{#6} & {#4}_{#7} & {#4}_{#8} \\
\end{vmatrix}
}

%\newcommand{\DETuvwxyijklm}[10]{
%\begin{vmatrix}
% {#1}_{#6} & {#1}_{#7} & {#1}_{#8} & {#1}_{#9} & {#1}_{#10} \\
% {#2}_{#6} & {#2}_{#7} & {#2}_{#8} & {#2}_{#9} & {#2}_{#10} \\
% {#3}_{#6} & {#3}_{#7} & {#3}_{#8} & {#3}_{#9} & {#3}_{#10} \\
% {#4}_{#6} & {#4}_{#7} & {#4}_{#8} & {#4}_{#9} & {#4}_{#10} \\
% {#5}_{#6} & {#5}_{#7} & {#5}_{#8} & {#5}_{#9} & {#5}_{#10}
%\end{vmatrix}
%}

% R3 vector.
\newcommand{\VectorThree}[3]{
\begin{bmatrix}
 {#1} \\
 {#2} \\
 {#3}
\end{bmatrix}
}


%%<misc>
%
\newcommand{\Abs}[1]{{\left\lvert{#1}\right\rvert}}
\newcommand{\spacegrad}[0]{\boldsymbol{\nabla}}
\newcommand{\grad}[0]{\nabla}
\newcommand{\LL}[0]{\mathcal{L}}

% == \partial_{#1} {#2}
\newcommand{\PD}[2]{\frac{\partial {#2}}{\partial {#1}}}
% inline variant
\newcommand{\PDi}[2]{{\partial {#2}}/{\partial {#1}}}

\newcommand{\PDD}[3]{\frac{\partial^2 {#3}}{\partial {#1}\partial {#2}}}
%\newcommand{\PDd}[2]{\frac{\partial^2 {#2}}{{\partial{#1}}^2}}
\newcommand{\PDsq}[2]{\frac{\partial^2 {#2}}{(\partial {#1})^2}}

\newcommand{\Partial}[2]{\frac{\partial {#1}}{\partial {#2}}}
\DeclareMathOperator{\RejName}{Rej}
\newcommand{\Rej}[2]{\RejName_{#1}\left( {#2} \right)}
\newcommand{\Rm}[1]{\mathbb{R}^{#1}}
\newcommand{\Cm}[1]{\mathbb{C}^{#1}}
\newcommand{\conj}[0]{{*}}

%</misc>

% <grade selection>
%
\newcommand{\gpgrade}[2] {{\left\langle{{#1}}\right\rangle}_{#2}}

\newcommand{\gpgradezero}[1] {\gpgrade{#1}{}}
%\newcommand{\gpscalargrade}[1] {{\left\langle{{#1}}\right\rangle}}
%\newcommand{\gpgradezero}[1] {\gpgrade{#1}{0}}

%\newcommand{\gpgradeone}[1] {{\left\langle{{#1}}\right\rangle}_{1}}
\newcommand{\gpgradeone}[1] {\gpgrade{#1}{1}}

\newcommand{\gpgradetwo}[1] {\gpgrade{#1}{2}}
\newcommand{\gpgradethree}[1] {\gpgrade{#1}{3}}
\newcommand{\gpgradefour}[1] {\gpgrade{#1}{4}}
%
% </grade selection>



\newcommand{\adot}[0]{{\dot{a}}}
\newcommand{\bdot}[0]{{\dot{b}}}
% taken for centered dot:
%\newcommand{\cdot}[0]{{\dot{c}}}
%\newcommand{\ddot}[0]{{\dot{d}}}
\newcommand{\edot}[0]{{\dot{e}}}
\newcommand{\fdot}[0]{{\dot{f}}}
\newcommand{\gdot}[0]{{\dot{g}}}
\newcommand{\hdot}[0]{{\dot{h}}}
\newcommand{\idot}[0]{{\dot{i}}}
\newcommand{\jdot}[0]{{\dot{j}}}
\newcommand{\kdot}[0]{{\dot{k}}}
\newcommand{\ldot}[0]{{\dot{l}}}
\newcommand{\mdot}[0]{{\dot{m}}}
\newcommand{\ndot}[0]{{\dot{n}}}
%\newcommand{\odot}[0]{{\dot{o}}}
\newcommand{\pdot}[0]{{\dot{p}}}
\newcommand{\qdot}[0]{{\dot{q}}}
\newcommand{\rdot}[0]{{\dot{r}}}
\newcommand{\sdot}[0]{{\dot{s}}}
\newcommand{\tdot}[0]{{\dot{t}}}
\newcommand{\udot}[0]{{\dot{u}}}
\newcommand{\vdot}[0]{{\dot{v}}}
\newcommand{\wdot}[0]{{\dot{w}}}
\newcommand{\xdot}[0]{{\dot{x}}}
\newcommand{\ydot}[0]{{\dot{y}}}
\newcommand{\zdot}[0]{{\dot{z}}}
\newcommand{\addot}[0]{{\ddot{a}}}
\newcommand{\bddot}[0]{{\ddot{b}}}
\newcommand{\cddot}[0]{{\ddot{c}}}
%\newcommand{\dddot}[0]{{\ddot{d}}}
\newcommand{\eddot}[0]{{\ddot{e}}}
\newcommand{\fddot}[0]{{\ddot{f}}}
\newcommand{\gddot}[0]{{\ddot{g}}}
\newcommand{\hddot}[0]{{\ddot{h}}}
\newcommand{\iddot}[0]{{\ddot{i}}}
\newcommand{\jddot}[0]{{\ddot{j}}}
\newcommand{\kddot}[0]{{\ddot{k}}}
\newcommand{\lddot}[0]{{\ddot{l}}}
\newcommand{\mddot}[0]{{\ddot{m}}}
\newcommand{\nddot}[0]{{\ddot{n}}}
\newcommand{\oddot}[0]{{\ddot{o}}}
\newcommand{\pddot}[0]{{\ddot{p}}}
\newcommand{\qddot}[0]{{\ddot{q}}}
\newcommand{\rddot}[0]{{\ddot{r}}}
\newcommand{\sddot}[0]{{\ddot{s}}}
\newcommand{\tddot}[0]{{\ddot{t}}}
\newcommand{\uddot}[0]{{\ddot{u}}}
\newcommand{\vddot}[0]{{\ddot{v}}}
\newcommand{\wddot}[0]{{\ddot{w}}}
\newcommand{\xddot}[0]{{\ddot{x}}}
\newcommand{\yddot}[0]{{\ddot{y}}}
\newcommand{\zddot}[0]{{\ddot{z}}}

%<bold and dot greek symbols>
%

\newcommand{\Deltadot}[0]{{\dot{\Delta}}}
\newcommand{\Gammadot}[0]{{\dot{\Gamma}}}
\newcommand{\Lambdadot}[0]{{\dot{\Lambda}}}
\newcommand{\Omegadot}[0]{{\dot{\Omega}}}
\newcommand{\Phidot}[0]{{\dot{\Phi}}}
\newcommand{\Pidot}[0]{{\dot{\Pi}}}
\newcommand{\Psidot}[0]{{\dot{\Psi}}}
\newcommand{\Sigmadot}[0]{{\dot{\Sigma}}}
\newcommand{\Thetadot}[0]{{\dot{\Theta}}}
\newcommand{\Upsilondot}[0]{{\dot{\Upsilon}}}
\newcommand{\Xidot}[0]{{\dot{\Xi}}}
\newcommand{\alphadot}[0]{{\dot{\alpha}}}
\newcommand{\betadot}[0]{{\dot{\beta}}}
\newcommand{\chidot}[0]{{\dot{\chi}}}
\newcommand{\deltadot}[0]{{\dot{\delta}}}
\newcommand{\epsilondot}[0]{{\dot{\epsilon}}}
\newcommand{\etadot}[0]{{\dot{\eta}}}
\newcommand{\gammadot}[0]{{\dot{\gamma}}}
\newcommand{\kappadot}[0]{{\dot{\kappa}}}
\newcommand{\lambdadot}[0]{{\dot{\lambda}}}
\newcommand{\mudot}[0]{{\dot{\mu}}}
\newcommand{\nudot}[0]{{\dot{\nu}}}
\newcommand{\omegadot}[0]{{\dot{\omega}}}
\newcommand{\phidot}[0]{{\dot{\phi}}}
\newcommand{\pidot}[0]{{\dot{\pi}}}
\newcommand{\psidot}[0]{{\dot{\psi}}}
\newcommand{\rhodot}[0]{{\dot{\rho}}}
\newcommand{\sigmadot}[0]{{\dot{\sigma}}}
\newcommand{\taudot}[0]{{\dot{\tau}}}
\newcommand{\thetadot}[0]{{\dot{\theta}}}
\newcommand{\upsilondot}[0]{{\dot{\upsilon}}}
\newcommand{\varepsilondot}[0]{{\dot{\varepsilon}}}
\newcommand{\varphidot}[0]{{\dot{\varphi}}}
\newcommand{\varpidot}[0]{{\dot{\varpi}}}
\newcommand{\varrhodot}[0]{{\dot{\varrho}}}
\newcommand{\varsigmadot}[0]{{\dot{\varsigma}}}
\newcommand{\varthetadot}[0]{{\dot{\vartheta}}}
\newcommand{\xidot}[0]{{\dot{\xi}}}
\newcommand{\zetadot}[0]{{\dot{\zeta}}}

\newcommand{\Deltaddot}[0]{{\ddot{\Delta}}}
\newcommand{\Gammaddot}[0]{{\ddot{\Gamma}}}
\newcommand{\Lambdaddot}[0]{{\ddot{\Lambda}}}
\newcommand{\Omegaddot}[0]{{\ddot{\Omega}}}
\newcommand{\Phiddot}[0]{{\ddot{\Phi}}}
\newcommand{\Piddot}[0]{{\ddot{\Pi}}}
\newcommand{\Psiddot}[0]{{\ddot{\Psi}}}
\newcommand{\Sigmaddot}[0]{{\ddot{\Sigma}}}
\newcommand{\Thetaddot}[0]{{\ddot{\Theta}}}
\newcommand{\Upsilonddot}[0]{{\ddot{\Upsilon}}}
\newcommand{\Xiddot}[0]{{\ddot{\Xi}}}
\newcommand{\alphaddot}[0]{{\ddot{\alpha}}}
\newcommand{\betaddot}[0]{{\ddot{\beta}}}
\newcommand{\chiddot}[0]{{\ddot{\chi}}}
\newcommand{\deltaddot}[0]{{\ddot{\delta}}}
\newcommand{\epsilonddot}[0]{{\ddot{\epsilon}}}
\newcommand{\etaddot}[0]{{\ddot{\eta}}}
\newcommand{\gammaddot}[0]{{\ddot{\gamma}}}
\newcommand{\kappaddot}[0]{{\ddot{\kappa}}}
\newcommand{\lambdaddot}[0]{{\ddot{\lambda}}}
\newcommand{\muddot}[0]{{\ddot{\mu}}}
\newcommand{\nuddot}[0]{{\ddot{\nu}}}
\newcommand{\omegaddot}[0]{{\ddot{\omega}}}
\newcommand{\phiddot}[0]{{\ddot{\phi}}}
\newcommand{\piddot}[0]{{\ddot{\pi}}}
\newcommand{\psiddot}[0]{{\ddot{\psi}}}
\newcommand{\rhoddot}[0]{{\ddot{\rho}}}
\newcommand{\sigmaddot}[0]{{\ddot{\sigma}}}
\newcommand{\tauddot}[0]{{\ddot{\tau}}}
\newcommand{\thetaddot}[0]{{\ddot{\theta}}}
\newcommand{\upsilonddot}[0]{{\ddot{\upsilon}}}
\newcommand{\varepsilonddot}[0]{{\ddot{\varepsilon}}}
\newcommand{\varphiddot}[0]{{\ddot{\varphi}}}
\newcommand{\varpiddot}[0]{{\ddot{\varpi}}}
\newcommand{\varrhoddot}[0]{{\ddot{\varrho}}}
\newcommand{\varsigmaddot}[0]{{\ddot{\varsigma}}}
\newcommand{\varthetaddot}[0]{{\ddot{\vartheta}}}
\newcommand{\xiddot}[0]{{\ddot{\xi}}}
\newcommand{\zetaddot}[0]{{\ddot{\zeta}}}

\newcommand{\BDelta}[0]{\boldsymbol{\Delta}}
\newcommand{\BGamma}[0]{\boldsymbol{\Gamma}}
\newcommand{\BLambda}[0]{\boldsymbol{\Lambda}}
\newcommand{\BOmega}[0]{\boldsymbol{\Omega}}
\newcommand{\BPhi}[0]{\boldsymbol{\Phi}}
\newcommand{\BPi}[0]{\boldsymbol{\Pi}}
\newcommand{\BPsi}[0]{\boldsymbol{\Psi}}
\newcommand{\BSigma}[0]{\boldsymbol{\Sigma}}
\newcommand{\BTheta}[0]{\boldsymbol{\Theta}}
\newcommand{\BUpsilon}[0]{\boldsymbol{\Upsilon}}
\newcommand{\BXi}[0]{\boldsymbol{\Xi}}
\newcommand{\Balpha}[0]{\boldsymbol{\alpha}}
\newcommand{\Bbeta}[0]{\boldsymbol{\beta}}
\newcommand{\Bchi}[0]{\boldsymbol{\chi}}
\newcommand{\Bdelta}[0]{\boldsymbol{\delta}}
\newcommand{\Bepsilon}[0]{\boldsymbol{\epsilon}}
\newcommand{\Beta}[0]{\boldsymbol{\eta}}
\newcommand{\Bgamma}[0]{\boldsymbol{\gamma}}
\newcommand{\Bkappa}[0]{\boldsymbol{\kappa}}
\newcommand{\Blambda}[0]{\boldsymbol{\lambda}}
\newcommand{\Bmu}[0]{\boldsymbol{\mu}}
\newcommand{\Bnu}[0]{\boldsymbol{\nu}}
%\newcommand{\Bomega}[0]{\boldsymbol{\omega}}
\newcommand{\Bphi}[0]{\boldsymbol{\phi}}
\newcommand{\Bpi}[0]{\boldsymbol{\pi}}
\newcommand{\Bpsi}[0]{\boldsymbol{\psi}}
\newcommand{\Brho}[0]{\boldsymbol{\rho}}
\newcommand{\Bsigma}[0]{\boldsymbol{\sigma}}
%\newcommand{\Btau}[0]{\boldsymbol{\tau}}
%\newcommand{\Btheta}[0]{\boldsymbol{\theta}}
\newcommand{\Bupsilon}[0]{\boldsymbol{\upsilon}}
\newcommand{\Bvarepsilon}[0]{\boldsymbol{\varepsilon}}
\newcommand{\Bvarphi}[0]{\boldsymbol{\varphi}}
\newcommand{\Bvarpi}[0]{\boldsymbol{\varpi}}
\newcommand{\Bvarrho}[0]{\boldsymbol{\varrho}}
\newcommand{\Bvarsigma}[0]{\boldsymbol{\varsigma}}
\newcommand{\Bvartheta}[0]{\boldsymbol{\vartheta}}
\newcommand{\Bxi}[0]{\boldsymbol{\xi}}
\newcommand{\Bzeta}[0]{\boldsymbol{\zeta}}
%
%</bold and dot greek symbols>
%<infrequent>
%
%\newcommand{\AreaOp}[1]{\AName_{#1}}
%\newcommand{\Babs}[0]{\abs{\BB}}
%\newcommand{\Bcap}[0]{\hat{\BB}}
%\newcommand{\BrPrimeRej}[0]{\rcap(\rcap \wedge \Br')}
%\newcommand{\CA}[0]{\mathcal{A}}
%\newcommand{\Cos}[1]{\cos{\left({#1}\right)}}
%\newcommand{\Det}[1] {\abs{#1}}
%\newcommand{\Dsq}[2] {\frac {\partial^2 {#1}} {\partial {#2}^2}}
%\newcommand{\Exp}[1]{\exp{\left({#1}\right)}}
%\newcommand{\Norm}[1]{\left\lVert{#1}\right\rVert}
%\newcommand{\Sin}[1]{\sin{\left({#1}\right)}}
%\newcommand{\T}[0]{\text{T}}
%\newcommand{\VolumeOp}[1]{\VName_{#1}}
%\newcommand{\agrad}[0]{\Ba \cdot \nabla}
%\newcommand{\alphacap}[0]{\hat{\boldsymbol{\alpha}}}
%\newcommand{\Fcap}[0]{\hat{\BF}}
%\newcommand{\bithree}[0]{{\Bi}_3}
%\newcommand{\bxa}[0]{\Bx\Ba}
%\newcommand{\coordvec}[2]{
%\newcommand{\costheta}[0]{\acap \cdot \xcap}
%\newcommand{\ddt}[1]{\ddot{#1}}
%\newcommand{\ddu}[1] {\frac {d{#1}} {du}}
%\newcommand{\dsqxj}[2] {\frac {\partial^2 {#1}} {\partial {x_{#2}}^2}}
%\newcommand{\dtheta}[1]{\frac{d {#1}}{d \theta}}
%\newcommand{\dt}[1]{\dot{#1}}
%\newcommand{\dt}[1]{\frac{d {#1}}{dt}}
%\newcommand{\dxj}[2] {\frac {\partial {#1}} {\partial {x_{#2}}}}
%\newcommand{\halfPhi}[0]{\frac{\phi}{2}}
%\newcommand{\half}[0]{\inv{2}}
%\newcommand{\inv}[1]{\frac{1}{#1}}
%\newcommand{\laplacian}[0]{\nabla^2}
%\newcommand{\matrixoftx}[3]{
%\newcommand{\nrrp}[0]{\norm{\rcap \wedge \Br'}}
%\newcommand{\oiint}{\bigcirc \hspace{-1.4em} \int \hspace{-.8em} \int}
%\newcommand{\transpose}[1]{{#1}^{\text{T}}}
%\newcommand{\transpose}[1]{{{#1}^{\TextTranspose}}}
%\newcommand{\transpose}[1]{{{#1}^{\text{T}}}}
%\newcommand{\barA}[0]{\bar{A}}
%\newcommand{\qbar}[0]{\bar{q}}
%\newcommand{\qdotbar}[0]{\dot{\bar{q}}}
%
%</infrequent>





%\usepackage[bookmarks=true]{hyperref}

%\usepackage{color,cite,graphicx}
   % use colour in the document, put your citations as [1-4]
   % rather than [1,2,3,4] (it looks nicer, and the extended LaTeX2e
   % graphics package. 
%\usepackage{latexsym,amssymb,epsf} % don't remember if these are
   % needed, but their inclusion can't do any damage


\chapter{Some rapidity angle notes. }
\label{chap:rapidity}
%\author{Peeter Joot \quad peeter.joot@gmail.com}
\date{ Dec 18, 2008.  $RCSfile: rapidity.tex,v $ Last $Revision: 1.8 $ $Date: 2009/06/14 23:51:45 $ }

%\begin{document}

%\maketitle{}
%\tableofcontents

\section{Motivation. }

Lut writes, "setting up a little calculation, I'm writing a 4-velocity as"
 
\begin{align*}
( \gamma, \gamma \beta_x, \gamma \beta_y, \gamma \beta_z ),
\end{align*}
 
which has length $-1$ if $\gamma^{-2} = 1 - (\beta_x)^2+(\beta_y)^2+(\beta_z)^2$.
 
Can you write this in terms of the 3 rapidities $a_1, a_2, a_3$ ?

I wasn't able to answer this right away so it is worth an examination
of rapidity angles to ensure that I understand the ideas.

\section{Stuff. }

Putting back in the $c$ factors, and switching to the $+---$ signature I'm used to, the position
vector is

\begin{align*}
x &= x^\mu \gamma_\mu = ct \gamma_0 + x^i \gamma_i,
\end{align*}

for which the corresponding proper velocity is
\begin{align*}
v &= \frac{dx}{d\tau} = c \frac{dt}{d\tau} \gamma_0 + \frac{dx^i}{dt} \frac{dt}{d\tau} \gamma_i
\end{align*}

Writing $\gamma = dt/d\tau$, and squaring the proper velocity we have

\begin{align*}
\frac{v^2}{c^2}
&= 1 \\
&= \gamma^2 \left(\gamma_0 + \inv{c}\frac{dx^i}{dt} \gamma_i\right)^2 \\
&= \gamma^2 \left(1 - \sum_i \inv{c^2} \left(\frac{dx^i}{dt}\right)^2 \right) \\
\end{align*}

So we have 

\begin{align*}
\gamma 
&= \inv{\sqrt{1 - \sum_i \inv{c^2} \left(\frac{dx^i}{dt}\right)^2 }} \\
\end{align*}

Observe that $\gamma$ ranges from $1$ to infinity, and can thus be described by the $[0,\infty]$ range of the hyperbolic cosine function.  With
the relative velocity $\Bv = \sum_i (dx^i/dt) \sigma_i$, this is

\begin{align*}
\frac{dt}{d\tau} &= \gamma  \\
&= \cosh\alpha \\
&= \inv{\sqrt{1 - (\Bv/c)^2}}
\end{align*}

In terms of the hyperbolic cosine for $\gamma$ our proper velocity then becomes

\begin{align*}
v/c &= \cosh\alpha \left(1 + \frac{\Bv}{c} \right) \gamma_0
\end{align*}

Taking the hint from the Lorentz transform where we have both $\sinh$ and $\cosh$ factors can one write

\begin{align*}
\gamma \frac{\Bv}{c} = \sinh\alpha
\end{align*}

This gives

\begin{align*}
\frac{\Bv}{c} = \tanh\alpha
\end{align*}

so we need $\alpha$ to be a spacetime relative vector.  With $\cosh$ being an even function $\cosh{\alpha} = \cosh{\Abs{\alpha}}$, so this
is still a 
scalar as desired.  Inverting the relationship for $\alpha$ we have

\begin{align*}
\alpha = \tanh^{-1} (\Bv/c) = \vcap \tanh^{-1} (\Abs{\Bv/c})
\end{align*}

The unit vector $\vcap$ can be factored out of the inverse hyperbolic tangent function since it is odd (consider the Taylor series expansion of $\tanh^{-1}$ to see why one can do this).

Finally, we have by 
dotting with the spatial basis vectors $\sigma_i$ three quantities in terms of spacetime vector rapidity angle

\begin{align*}
\alpha_i = (\vcap \cdot \sigma_i) \tanh^{-1} (\Abs{\Bv/c}).
\end{align*}

The $\vcap \cdot \sigma_i$ parts are direction cosines, so the three rapidities Lut was asking about all appear to be weighted direction cosines.

%\bibliographystyle{plainnat}
%\bibliography{myrefs}

%\end{document}


\part{Stubs and Fragments}
%
% Copyright � 2012 Peeter Joot.  All Rights Reserved.
% Licenced as described in the file LICENSE under the root directory of this GIT repository.
%

%
%
%\documentclass{article}

%\usepackage{amsmath}
\usepackage{mathpazo}

%
% shorthand for bold symbols, convenient for vectors and matrices
%
\newcommand{\Ba}[0]{\mathbf{a}}
\newcommand{\Bb}[0]{\mathbf{b}}
\newcommand{\Bc}[0]{\mathbf{c}}
\newcommand{\Bd}[0]{\mathbf{d}}
\newcommand{\Be}[0]{\mathbf{e}}
\newcommand{\Bf}[0]{\mathbf{f}}
\newcommand{\Bg}[0]{\mathbf{g}}
\newcommand{\Bh}[0]{\mathbf{h}}
\newcommand{\Bi}[0]{\mathbf{i}}
\newcommand{\Bj}[0]{\mathbf{j}}
\newcommand{\Bk}[0]{\mathbf{k}}
\newcommand{\Bl}[0]{\mathbf{l}}
\newcommand{\Bm}[0]{\mathbf{m}}
\newcommand{\Bn}[0]{\mathbf{n}}
\newcommand{\Bo}[0]{\mathbf{o}}
\newcommand{\Bp}[0]{\mathbf{p}}
\newcommand{\Bq}[0]{\mathbf{q}}
\newcommand{\Br}[0]{\mathbf{r}}
\newcommand{\Bs}[0]{\mathbf{s}}
\newcommand{\Bt}[0]{\mathbf{t}}
\newcommand{\Bu}[0]{\mathbf{u}}
\newcommand{\Bv}[0]{\mathbf{v}}
\newcommand{\Bw}[0]{\mathbf{w}}
\newcommand{\Bx}[0]{\mathbf{x}}
\newcommand{\By}[0]{\mathbf{y}}
\newcommand{\Bz}[0]{\mathbf{z}}
\newcommand{\BA}[0]{\mathbf{A}}
\newcommand{\BB}[0]{\mathbf{B}}
\newcommand{\BC}[0]{\mathbf{C}}
\newcommand{\BD}[0]{\mathbf{D}}
\newcommand{\BE}[0]{\mathbf{E}}
\newcommand{\BF}[0]{\mathbf{F}}
\newcommand{\BG}[0]{\mathbf{G}}
\newcommand{\BH}[0]{\mathbf{H}}
\newcommand{\BI}[0]{\mathbf{I}}
\newcommand{\BJ}[0]{\mathbf{J}}
\newcommand{\BK}[0]{\mathbf{K}}
\newcommand{\BL}[0]{\mathbf{L}}
\newcommand{\BM}[0]{\mathbf{M}}
\newcommand{\BN}[0]{\mathbf{N}}
\newcommand{\BO}[0]{\mathbf{O}}
\newcommand{\BP}[0]{\mathbf{P}}
\newcommand{\BQ}[0]{\mathbf{Q}}
\newcommand{\BR}[0]{\mathbf{R}}
\newcommand{\BS}[0]{\mathbf{S}}
\newcommand{\BT}[0]{\mathbf{T}}
\newcommand{\BU}[0]{\mathbf{U}}
\newcommand{\BV}[0]{\mathbf{V}}
\newcommand{\BW}[0]{\mathbf{W}}
\newcommand{\BX}[0]{\mathbf{X}}
\newcommand{\BY}[0]{\mathbf{Y}}
\newcommand{\BZ}[0]{\mathbf{Z}}

\newcommand{\Bzero}[0]{\mathbf{0}}
\newcommand{\Btheta}[0]{\boldsymbol{\theta}}
\newcommand{\Btau}[0]{\boldsymbol{\tau}}
\newcommand{\Bomega}[0]{\boldsymbol{\omega}}

%
% shorthand for unit vectors
%
\newcommand{\acap}[0]{\hat{\Ba}}
\newcommand{\bcap}[0]{\hat{\Bb}}
\newcommand{\ccap}[0]{\hat{\Bc}}
\newcommand{\dcap}[0]{\hat{\Bd}}
\newcommand{\ecap}[0]{\hat{\Be}}
\newcommand{\fcap}[0]{\hat{\Bf}}
\newcommand{\gcap}[0]{\hat{\Bg}}
\newcommand{\hcap}[0]{\hat{\Bh}}
\newcommand{\icap}[0]{\hat{\Bi}}
\newcommand{\jcap}[0]{\hat{\Bj}}
\newcommand{\kcap}[0]{\hat{\Bk}}
\newcommand{\lcap}[0]{\hat{\Bl}}
\newcommand{\mcap}[0]{\hat{\Bm}}
\newcommand{\ncap}[0]{\hat{\Bn}}
\newcommand{\ocap}[0]{\hat{\Bo}}
\newcommand{\pcap}[0]{\hat{\Bp}}
\newcommand{\qcap}[0]{\hat{\Bq}}
\newcommand{\rcap}[0]{\hat{\Br}}
\newcommand{\scap}[0]{\hat{\Bs}}
\newcommand{\tcap}[0]{\hat{\Bt}}
\newcommand{\ucap}[0]{\hat{\Bu}}
\newcommand{\vcap}[0]{\hat{\Bv}}
\newcommand{\wcap}[0]{\hat{\Bw}}
\newcommand{\xcap}[0]{\hat{\Bx}}
\newcommand{\ycap}[0]{\hat{\By}}
\newcommand{\zcap}[0]{\hat{\Bz}}
\newcommand{\thetacap}[0]{\hat{\Btheta}}

%
% to write R^n and C^n in a distinguishable fashion.  Perhaps change this
% to the double lined characters upon figuring out how to do so.
%
\newcommand{\C}[1]{$\mathbb{C}^{#1}$}
\newcommand{\R}[1]{$\mathbb{R}^{#1}$}

%
% various generally useful helpers
%

% derivative of #1 wrt. #2:
\newcommand{\D}[2] {\frac {d#2} {d#1}}

\newcommand{\inv}[1]{\frac{1}{#1}}
\newcommand{\cross}[0]{\times}

\newcommand{\abs}[1]{\lvert{#1}\rvert}
\newcommand{\norm}[1]{\lVert{#1}\rVert}
\newcommand{\innerprod}[2]{\langle{#1}, {#2}\rangle}
\newcommand{\dotprod}[2]{{#1} \cdot {#2}}
\newcommand{\bdotprod}[2]{\left({#1} \cdot {#2}\right)}
\newcommand{\crossprod}[2]{{#1} \cross {#2}}
\newcommand{\tripleprod}[3]{\dotprod{\left(\crossprod{#1}{#2}\right)}{#3}}

\DeclareMathOperator{\Proj}{Proj}
\DeclareMathOperator{\Span}{span}
\DeclareMathOperator{\Sgn}{sgn}
\DeclareMathOperator{\Area}{Area}
\DeclareMathOperator{\Volume}{Volume}

%
% A few miscellaneous things specific to this document
%
\newcommand{\crossop}[1]{\crossprod{#1}{}}

% R2 vector.
\newcommand{\VectorTwo}[2]{
\begin{bmatrix}
 {#1} \\
 {#2}
\end{bmatrix}
}

\newcommand{\VectorN}[1]{
\begin{bmatrix}
{#1}_1 \\
{#1}_2 \\
\vdots \\
{#1}_N \\
\end{bmatrix}
}

\newcommand{\DETuvij}[4]{
\begin{vmatrix}
 {#1}_{#3} & {#1}_{#4} \\
 {#2}_{#3} & {#2}_{#4}
\end{vmatrix}
}

\newcommand{\DETuvwijk}[6]{
\begin{vmatrix}
 {#1}_{#4} & {#1}_{#5} & {#1}_{#6} \\
 {#2}_{#4} & {#2}_{#5} & {#2}_{#6} \\
 {#3}_{#4} & {#3}_{#5} & {#3}_{#6}
\end{vmatrix}
}

\newcommand{\DETuvwxijkl}[8]{
\begin{vmatrix}
 {#1}_{#5} & {#1}_{#6} & {#1}_{#7} & {#1}_{#8} \\
 {#2}_{#5} & {#2}_{#6} & {#2}_{#7} & {#2}_{#8} \\
 {#3}_{#5} & {#3}_{#6} & {#3}_{#7} & {#3}_{#8} \\
 {#4}_{#5} & {#4}_{#6} & {#4}_{#7} & {#4}_{#8} \\
\end{vmatrix}
}

%\newcommand{\DETuvwxyijklm}[10]{
%\begin{vmatrix}
% {#1}_{#6} & {#1}_{#7} & {#1}_{#8} & {#1}_{#9} & {#1}_{#10} \\
% {#2}_{#6} & {#2}_{#7} & {#2}_{#8} & {#2}_{#9} & {#2}_{#10} \\
% {#3}_{#6} & {#3}_{#7} & {#3}_{#8} & {#3}_{#9} & {#3}_{#10} \\
% {#4}_{#6} & {#4}_{#7} & {#4}_{#8} & {#4}_{#9} & {#4}_{#10} \\
% {#5}_{#6} & {#5}_{#7} & {#5}_{#8} & {#5}_{#9} & {#5}_{#10}
%\end{vmatrix}
%}

% R3 vector.
\newcommand{\VectorThree}[3]{
\begin{bmatrix}
 {#1} \\
 {#2} \\
 {#3}
\end{bmatrix}
}


%%<misc>
%
\newcommand{\Abs}[1]{{\left\lvert{#1}\right\rvert}}
\newcommand{\spacegrad}[0]{\boldsymbol{\nabla}}
\newcommand{\grad}[0]{\nabla}
\newcommand{\LL}[0]{\mathcal{L}}

% == \partial_{#1} {#2}
\newcommand{\PD}[2]{\frac{\partial {#2}}{\partial {#1}}}
% inline variant
\newcommand{\PDi}[2]{{\partial {#2}}/{\partial {#1}}}

\newcommand{\PDD}[3]{\frac{\partial^2 {#3}}{\partial {#1}\partial {#2}}}
%\newcommand{\PDd}[2]{\frac{\partial^2 {#2}}{{\partial{#1}}^2}}
\newcommand{\PDsq}[2]{\frac{\partial^2 {#2}}{(\partial {#1})^2}}

\newcommand{\Partial}[2]{\frac{\partial {#1}}{\partial {#2}}}
\DeclareMathOperator{\RejName}{Rej}
\newcommand{\Rej}[2]{\RejName_{#1}\left( {#2} \right)}
\newcommand{\Rm}[1]{\mathbb{R}^{#1}}
\newcommand{\Cm}[1]{\mathbb{C}^{#1}}
\newcommand{\conj}[0]{{*}}

%</misc>

% <grade selection>
%
\newcommand{\gpgrade}[2] {{\left\langle{{#1}}\right\rangle}_{#2}}

\newcommand{\gpgradezero}[1] {\gpgrade{#1}{}}
%\newcommand{\gpscalargrade}[1] {{\left\langle{{#1}}\right\rangle}}
%\newcommand{\gpgradezero}[1] {\gpgrade{#1}{0}}

%\newcommand{\gpgradeone}[1] {{\left\langle{{#1}}\right\rangle}_{1}}
\newcommand{\gpgradeone}[1] {\gpgrade{#1}{1}}

\newcommand{\gpgradetwo}[1] {\gpgrade{#1}{2}}
\newcommand{\gpgradethree}[1] {\gpgrade{#1}{3}}
\newcommand{\gpgradefour}[1] {\gpgrade{#1}{4}}
%
% </grade selection>



\newcommand{\adot}[0]{{\dot{a}}}
\newcommand{\bdot}[0]{{\dot{b}}}
% taken for centered dot:
%\newcommand{\cdot}[0]{{\dot{c}}}
%\newcommand{\ddot}[0]{{\dot{d}}}
\newcommand{\edot}[0]{{\dot{e}}}
\newcommand{\fdot}[0]{{\dot{f}}}
\newcommand{\gdot}[0]{{\dot{g}}}
\newcommand{\hdot}[0]{{\dot{h}}}
\newcommand{\idot}[0]{{\dot{i}}}
\newcommand{\jdot}[0]{{\dot{j}}}
\newcommand{\kdot}[0]{{\dot{k}}}
\newcommand{\ldot}[0]{{\dot{l}}}
\newcommand{\mdot}[0]{{\dot{m}}}
\newcommand{\ndot}[0]{{\dot{n}}}
%\newcommand{\odot}[0]{{\dot{o}}}
\newcommand{\pdot}[0]{{\dot{p}}}
\newcommand{\qdot}[0]{{\dot{q}}}
\newcommand{\rdot}[0]{{\dot{r}}}
\newcommand{\sdot}[0]{{\dot{s}}}
\newcommand{\tdot}[0]{{\dot{t}}}
\newcommand{\udot}[0]{{\dot{u}}}
\newcommand{\vdot}[0]{{\dot{v}}}
\newcommand{\wdot}[0]{{\dot{w}}}
\newcommand{\xdot}[0]{{\dot{x}}}
\newcommand{\ydot}[0]{{\dot{y}}}
\newcommand{\zdot}[0]{{\dot{z}}}
\newcommand{\addot}[0]{{\ddot{a}}}
\newcommand{\bddot}[0]{{\ddot{b}}}
\newcommand{\cddot}[0]{{\ddot{c}}}
%\newcommand{\dddot}[0]{{\ddot{d}}}
\newcommand{\eddot}[0]{{\ddot{e}}}
\newcommand{\fddot}[0]{{\ddot{f}}}
\newcommand{\gddot}[0]{{\ddot{g}}}
\newcommand{\hddot}[0]{{\ddot{h}}}
\newcommand{\iddot}[0]{{\ddot{i}}}
\newcommand{\jddot}[0]{{\ddot{j}}}
\newcommand{\kddot}[0]{{\ddot{k}}}
\newcommand{\lddot}[0]{{\ddot{l}}}
\newcommand{\mddot}[0]{{\ddot{m}}}
\newcommand{\nddot}[0]{{\ddot{n}}}
\newcommand{\oddot}[0]{{\ddot{o}}}
\newcommand{\pddot}[0]{{\ddot{p}}}
\newcommand{\qddot}[0]{{\ddot{q}}}
\newcommand{\rddot}[0]{{\ddot{r}}}
\newcommand{\sddot}[0]{{\ddot{s}}}
\newcommand{\tddot}[0]{{\ddot{t}}}
\newcommand{\uddot}[0]{{\ddot{u}}}
\newcommand{\vddot}[0]{{\ddot{v}}}
\newcommand{\wddot}[0]{{\ddot{w}}}
\newcommand{\xddot}[0]{{\ddot{x}}}
\newcommand{\yddot}[0]{{\ddot{y}}}
\newcommand{\zddot}[0]{{\ddot{z}}}

%<bold and dot greek symbols>
%

\newcommand{\Deltadot}[0]{{\dot{\Delta}}}
\newcommand{\Gammadot}[0]{{\dot{\Gamma}}}
\newcommand{\Lambdadot}[0]{{\dot{\Lambda}}}
\newcommand{\Omegadot}[0]{{\dot{\Omega}}}
\newcommand{\Phidot}[0]{{\dot{\Phi}}}
\newcommand{\Pidot}[0]{{\dot{\Pi}}}
\newcommand{\Psidot}[0]{{\dot{\Psi}}}
\newcommand{\Sigmadot}[0]{{\dot{\Sigma}}}
\newcommand{\Thetadot}[0]{{\dot{\Theta}}}
\newcommand{\Upsilondot}[0]{{\dot{\Upsilon}}}
\newcommand{\Xidot}[0]{{\dot{\Xi}}}
\newcommand{\alphadot}[0]{{\dot{\alpha}}}
\newcommand{\betadot}[0]{{\dot{\beta}}}
\newcommand{\chidot}[0]{{\dot{\chi}}}
\newcommand{\deltadot}[0]{{\dot{\delta}}}
\newcommand{\epsilondot}[0]{{\dot{\epsilon}}}
\newcommand{\etadot}[0]{{\dot{\eta}}}
\newcommand{\gammadot}[0]{{\dot{\gamma}}}
\newcommand{\kappadot}[0]{{\dot{\kappa}}}
\newcommand{\lambdadot}[0]{{\dot{\lambda}}}
\newcommand{\mudot}[0]{{\dot{\mu}}}
\newcommand{\nudot}[0]{{\dot{\nu}}}
\newcommand{\omegadot}[0]{{\dot{\omega}}}
\newcommand{\phidot}[0]{{\dot{\phi}}}
\newcommand{\pidot}[0]{{\dot{\pi}}}
\newcommand{\psidot}[0]{{\dot{\psi}}}
\newcommand{\rhodot}[0]{{\dot{\rho}}}
\newcommand{\sigmadot}[0]{{\dot{\sigma}}}
\newcommand{\taudot}[0]{{\dot{\tau}}}
\newcommand{\thetadot}[0]{{\dot{\theta}}}
\newcommand{\upsilondot}[0]{{\dot{\upsilon}}}
\newcommand{\varepsilondot}[0]{{\dot{\varepsilon}}}
\newcommand{\varphidot}[0]{{\dot{\varphi}}}
\newcommand{\varpidot}[0]{{\dot{\varpi}}}
\newcommand{\varrhodot}[0]{{\dot{\varrho}}}
\newcommand{\varsigmadot}[0]{{\dot{\varsigma}}}
\newcommand{\varthetadot}[0]{{\dot{\vartheta}}}
\newcommand{\xidot}[0]{{\dot{\xi}}}
\newcommand{\zetadot}[0]{{\dot{\zeta}}}

\newcommand{\Deltaddot}[0]{{\ddot{\Delta}}}
\newcommand{\Gammaddot}[0]{{\ddot{\Gamma}}}
\newcommand{\Lambdaddot}[0]{{\ddot{\Lambda}}}
\newcommand{\Omegaddot}[0]{{\ddot{\Omega}}}
\newcommand{\Phiddot}[0]{{\ddot{\Phi}}}
\newcommand{\Piddot}[0]{{\ddot{\Pi}}}
\newcommand{\Psiddot}[0]{{\ddot{\Psi}}}
\newcommand{\Sigmaddot}[0]{{\ddot{\Sigma}}}
\newcommand{\Thetaddot}[0]{{\ddot{\Theta}}}
\newcommand{\Upsilonddot}[0]{{\ddot{\Upsilon}}}
\newcommand{\Xiddot}[0]{{\ddot{\Xi}}}
\newcommand{\alphaddot}[0]{{\ddot{\alpha}}}
\newcommand{\betaddot}[0]{{\ddot{\beta}}}
\newcommand{\chiddot}[0]{{\ddot{\chi}}}
\newcommand{\deltaddot}[0]{{\ddot{\delta}}}
\newcommand{\epsilonddot}[0]{{\ddot{\epsilon}}}
\newcommand{\etaddot}[0]{{\ddot{\eta}}}
\newcommand{\gammaddot}[0]{{\ddot{\gamma}}}
\newcommand{\kappaddot}[0]{{\ddot{\kappa}}}
\newcommand{\lambdaddot}[0]{{\ddot{\lambda}}}
\newcommand{\muddot}[0]{{\ddot{\mu}}}
\newcommand{\nuddot}[0]{{\ddot{\nu}}}
\newcommand{\omegaddot}[0]{{\ddot{\omega}}}
\newcommand{\phiddot}[0]{{\ddot{\phi}}}
\newcommand{\piddot}[0]{{\ddot{\pi}}}
\newcommand{\psiddot}[0]{{\ddot{\psi}}}
\newcommand{\rhoddot}[0]{{\ddot{\rho}}}
\newcommand{\sigmaddot}[0]{{\ddot{\sigma}}}
\newcommand{\tauddot}[0]{{\ddot{\tau}}}
\newcommand{\thetaddot}[0]{{\ddot{\theta}}}
\newcommand{\upsilonddot}[0]{{\ddot{\upsilon}}}
\newcommand{\varepsilonddot}[0]{{\ddot{\varepsilon}}}
\newcommand{\varphiddot}[0]{{\ddot{\varphi}}}
\newcommand{\varpiddot}[0]{{\ddot{\varpi}}}
\newcommand{\varrhoddot}[0]{{\ddot{\varrho}}}
\newcommand{\varsigmaddot}[0]{{\ddot{\varsigma}}}
\newcommand{\varthetaddot}[0]{{\ddot{\vartheta}}}
\newcommand{\xiddot}[0]{{\ddot{\xi}}}
\newcommand{\zetaddot}[0]{{\ddot{\zeta}}}

\newcommand{\BDelta}[0]{\boldsymbol{\Delta}}
\newcommand{\BGamma}[0]{\boldsymbol{\Gamma}}
\newcommand{\BLambda}[0]{\boldsymbol{\Lambda}}
\newcommand{\BOmega}[0]{\boldsymbol{\Omega}}
\newcommand{\BPhi}[0]{\boldsymbol{\Phi}}
\newcommand{\BPi}[0]{\boldsymbol{\Pi}}
\newcommand{\BPsi}[0]{\boldsymbol{\Psi}}
\newcommand{\BSigma}[0]{\boldsymbol{\Sigma}}
\newcommand{\BTheta}[0]{\boldsymbol{\Theta}}
\newcommand{\BUpsilon}[0]{\boldsymbol{\Upsilon}}
\newcommand{\BXi}[0]{\boldsymbol{\Xi}}
\newcommand{\Balpha}[0]{\boldsymbol{\alpha}}
\newcommand{\Bbeta}[0]{\boldsymbol{\beta}}
\newcommand{\Bchi}[0]{\boldsymbol{\chi}}
\newcommand{\Bdelta}[0]{\boldsymbol{\delta}}
\newcommand{\Bepsilon}[0]{\boldsymbol{\epsilon}}
\newcommand{\Beta}[0]{\boldsymbol{\eta}}
\newcommand{\Bgamma}[0]{\boldsymbol{\gamma}}
\newcommand{\Bkappa}[0]{\boldsymbol{\kappa}}
\newcommand{\Blambda}[0]{\boldsymbol{\lambda}}
\newcommand{\Bmu}[0]{\boldsymbol{\mu}}
\newcommand{\Bnu}[0]{\boldsymbol{\nu}}
%\newcommand{\Bomega}[0]{\boldsymbol{\omega}}
\newcommand{\Bphi}[0]{\boldsymbol{\phi}}
\newcommand{\Bpi}[0]{\boldsymbol{\pi}}
\newcommand{\Bpsi}[0]{\boldsymbol{\psi}}
\newcommand{\Brho}[0]{\boldsymbol{\rho}}
\newcommand{\Bsigma}[0]{\boldsymbol{\sigma}}
%\newcommand{\Btau}[0]{\boldsymbol{\tau}}
%\newcommand{\Btheta}[0]{\boldsymbol{\theta}}
\newcommand{\Bupsilon}[0]{\boldsymbol{\upsilon}}
\newcommand{\Bvarepsilon}[0]{\boldsymbol{\varepsilon}}
\newcommand{\Bvarphi}[0]{\boldsymbol{\varphi}}
\newcommand{\Bvarpi}[0]{\boldsymbol{\varpi}}
\newcommand{\Bvarrho}[0]{\boldsymbol{\varrho}}
\newcommand{\Bvarsigma}[0]{\boldsymbol{\varsigma}}
\newcommand{\Bvartheta}[0]{\boldsymbol{\vartheta}}
\newcommand{\Bxi}[0]{\boldsymbol{\xi}}
\newcommand{\Bzeta}[0]{\boldsymbol{\zeta}}
%
%</bold and dot greek symbols>
%<infrequent>
%
%\newcommand{\AreaOp}[1]{\AName_{#1}}
%\newcommand{\Babs}[0]{\abs{\BB}}
%\newcommand{\Bcap}[0]{\hat{\BB}}
%\newcommand{\BrPrimeRej}[0]{\rcap(\rcap \wedge \Br')}
%\newcommand{\CA}[0]{\mathcal{A}}
%\newcommand{\Cos}[1]{\cos{\left({#1}\right)}}
%\newcommand{\Det}[1] {\abs{#1}}
%\newcommand{\Dsq}[2] {\frac {\partial^2 {#1}} {\partial {#2}^2}}
%\newcommand{\Exp}[1]{\exp{\left({#1}\right)}}
%\newcommand{\Norm}[1]{\left\lVert{#1}\right\rVert}
%\newcommand{\Sin}[1]{\sin{\left({#1}\right)}}
%\newcommand{\T}[0]{\text{T}}
%\newcommand{\VolumeOp}[1]{\VName_{#1}}
%\newcommand{\agrad}[0]{\Ba \cdot \nabla}
%\newcommand{\alphacap}[0]{\hat{\boldsymbol{\alpha}}}
%\newcommand{\Fcap}[0]{\hat{\BF}}
%\newcommand{\bithree}[0]{{\Bi}_3}
%\newcommand{\bxa}[0]{\Bx\Ba}
%\newcommand{\coordvec}[2]{
%\newcommand{\costheta}[0]{\acap \cdot \xcap}
%\newcommand{\ddt}[1]{\ddot{#1}}
%\newcommand{\ddu}[1] {\frac {d{#1}} {du}}
%\newcommand{\dsqxj}[2] {\frac {\partial^2 {#1}} {\partial {x_{#2}}^2}}
%\newcommand{\dtheta}[1]{\frac{d {#1}}{d \theta}}
%\newcommand{\dt}[1]{\dot{#1}}
%\newcommand{\dt}[1]{\frac{d {#1}}{dt}}
%\newcommand{\dxj}[2] {\frac {\partial {#1}} {\partial {x_{#2}}}}
%\newcommand{\halfPhi}[0]{\frac{\phi}{2}}
%\newcommand{\half}[0]{\inv{2}}
%\newcommand{\inv}[1]{\frac{1}{#1}}
%\newcommand{\laplacian}[0]{\nabla^2}
%\newcommand{\matrixoftx}[3]{
%\newcommand{\nrrp}[0]{\norm{\rcap \wedge \Br'}}
%\newcommand{\oiint}{\bigcirc \hspace{-1.4em} \int \hspace{-.8em} \int}
%\newcommand{\transpose}[1]{{#1}^{\text{T}}}
%\newcommand{\transpose}[1]{{{#1}^{\TextTranspose}}}
%\newcommand{\transpose}[1]{{{#1}^{\text{T}}}}
%\newcommand{\barA}[0]{\bar{A}}
%\newcommand{\qbar}[0]{\bar{q}}
%\newcommand{\qdotbar}[0]{\dot{\bar{q}}}
%
%</infrequent>





%\usepackage[bookmarks=true]{hyperref}

%\usepackage{color,cite,graphicx}
   % use colour in the document, put your citations as [1-4]
   % rather than [1,2,3,4] (it looks nicer, and the extended LaTeX2e
   % graphics package.
%\usepackage{latexsym,amssymb,epsf} % do not remember if these are
   % needed, but their inclusion can not do any damage


\chapter{scratch planewave}
\label{chap:scratchPlanewave}
%\author{Peeter Joot \quad peeterjoot@protonmail.com}
\date{ Feb dd, 2009.  \(RCSfile: scratchPlanewave.tex,v \) Last \(Revision: 1.8 \) \(Date: 2009/06/14 23:51:45 \) }

%\begin{document}

%\maketitle{}

%\tableofcontents

\section{}

\section{Appendix}
\subsection{Energy density.  First attempt, scratch notes}

Let us expand out the products in pieces

\begin{equation}\label{eqn:scratchPlanewave:20}
\begin{aligned}
F \gamma_0 F \gamma_0
&=
\inv{V^2} \sum_{\Bk,\Bm}
\int e^{i \Bomega_m t + i \Bm \cdot (\Bb-\Bx) } F(\Bb, 0) d^3 b \gamma_0
e^{i \Bomega_k t + i \Bk \cdot (\Ba-\Bx) } F(\Ba, 0) d^3 a \gamma_0 \\
&=
\inv{V^2} \sum_{\Bk,\Bm}
\int e^{i \Bomega_m t + i \Bm \cdot (\Bb-\Bx) } F(\Bb, 0) d^3 b
e^{i \Bomega_k t - i \Bk \cdot (\Ba-\Bx) } \gamma_0 F(\Ba, 0) d^3 a \gamma_0 \\
&=
\inv{V^2} \sum_{\Bk,\Bm}
\int e^{i \Bomega_m t + i \Bm \cdot (\Bb-\Bx) -i \Bomega_k t - i \Bk \cdot (\Ba-\Bx) } F(\Bb, 0) \gamma_0 F(\Ba, 0) \gamma_0 d^3 a d^3 b \\
\end{aligned}
\end{equation}

For the second term we have
\begin{equation}\label{eqn:scratchPlanewave:40}
\begin{aligned}
\gamma_0 F \gamma_0 F
%&=
%\inv{V^2} \sum_{\Bk,\Bm} \int
%\gamma_0 e^{i \Bomega_k t + i \Bk \cdot (\Ba-\Bx) } F(\Ba, 0) d^3 a \gamma_0 e^{i \Bomega_m t + i \Bm \cdot (\Bb-\Bx) } F(\Bb, 0) d^3 b  \\
%&=
%\inv{V^2} \sum_{\Bk,\Bm} \int
%e^{i \Bomega_k t - i \Bk \cdot (\Ba-\Bx) } \gamma_0 F(\Ba, 0) d^3 a e^{i \Bomega_m t - i \Bm \cdot (\Bb-\Bx) } \gamma_0 F(\Bb, 0) d^3 b  \\
%&=
%\inv{V^2} \sum_{\Bk,\Bm} \int
%e^{i \Bomega_k t - i \Bk \cdot (\Ba-\Bx) } \gamma_0 d^3 a e^{-i \Bomega_m t - i \Bm \cdot (\Bb-\Bx) } F(\Ba, 0) \gamma_0 F(\Bb, 0) d^3 b  \\
%&=
%\inv{V^2} \sum_{\Bk,\Bm} \int
%e^{i \Bomega_k t - i \Bk \cdot (\Ba-\Bx) -i \Bomega_m t + i \Bm \cdot (\Bb-\Bx) } \gamma_0 F(\Ba, 0) \gamma_0 F(\Bb, 0) d^3 a d^3 b  \\
&=
\inv{V^2} \sum_{\Bk,\Bm} \int
\gamma_0 e^{i \Bomega_m t + i \Bm \cdot (\Bb-\Bx) } F(\Bb, 0) d^3 b \gamma_0 e^{i \Bomega_k t + i \Bk \cdot (\Ba-\Bx) } F(\Ba, 0) d^3 a  \\
&=
\inv{V^2} \sum_{\Bk,\Bm} \int
e^{i \Bomega_m t - i \Bm \cdot (\Bb-\Bx) } \gamma_0 F(\Bb, 0) d^3 b e^{i \Bomega_k t - i \Bk \cdot (\Ba-\Bx) } \gamma_0 F(\Ba, 0) d^3 a  \\
&=
\inv{V^2} \sum_{\Bk,\Bm} \int
e^{i \Bomega_m t - i \Bm \cdot (\Bb-\Bx) } \gamma_0 d^3 b e^{-i \Bomega_k t - i \Bk \cdot (\Ba-\Bx) } F(\Bb, 0) \gamma_0 F(\Ba, 0) d^3 a  \\
&=
\inv{V^2} \sum_{\Bk,\Bm} \int
e^{i \Bomega_m t - i \Bm \cdot (\Bb-\Bx) -i \Bomega_k t + i \Bk \cdot (\Ba-\Bx) } \gamma_0 F(\Bb, 0) \gamma_0 F(\Ba, 0) d^3 a d^3 b  \\
\end{aligned}
\end{equation}

Summing these we have
\begin{equation}\label{eqn:scratchPlanewave:60}
\begin{aligned}
F &\gamma_0 F \gamma_0 + \gamma_0 F \gamma_0 F \\
&=\inv{V^2} \sum_{\Bk,\Bm} \int d^3 a d^3 b e^{i (\Bomega_m -\Bomega_k) t } \times \\
&\quad (e^{   i \Bm \cdot (\Bb-\Bx) - i \Bk \cdot (\Ba-\Bx) } F(\Bb, 0) \gamma_0 F(\Ba, 0) \gamma_0
+e^{ - i \Bm \cdot (\Bb-\Bx) + i \Bk \cdot (\Ba-\Bx) } \gamma_0 F(\Bb, 0) \gamma_0 F(\Ba, 0))
\end{aligned}
\end{equation}

It is clear that all the \(\Bk = \Bm\) terms are considerably less complex.  Those reduce to

\begin{equation}\label{eqn:scratchPlanewave:80}
\begin{aligned}
\inv{V^2} &\sum_{\Bk} \int d^3 a d^3 b (e^{   i \Bk \cdot (\Bb-\Bx) - i \Bk \cdot (\Ba-\Bx) } F(\Bb, 0) \gamma_0 F(\Ba, 0) \gamma_0
+e^{ - i \Bk \cdot (\Bb-\Bx) + i \Bk \cdot (\Ba-\Bx) } \gamma_0 F(\Bb, 0) \gamma_0 F(\Ba, 0)) \\
&=
\inv{V^2} \sum_{\Bk} \int d^3 a d^3 b (e^{   i \Bk \cdot (\Bb - \Ba) } F(\Bb, 0) \gamma_0 F(\Ba, 0) \gamma_0
+e^{ i \Bk \cdot (\Ba -\Bb) } \gamma_0 F(\Bb, 0) \gamma_0 F(\Ba, 0)) \\
&=
\sum_{\Bk} ( \hat{F}_k \gamma_0 \hat{F}_k \gamma_0 + \gamma_0 \hat{F}_k \gamma_0 \hat{F}_k ) \\
\end{aligned}
\end{equation}


\begin{equation}\label{eqn:scratchPlanewave:100}
\begin{aligned}
&=\inv{V^2} \sum_{\{k <m\} \cup \{ k > m \}} \int d^3 a d^3 b e^{i \Bomega (m -k) t } \times \\
&\quad (e^{   i m \Bomega \cdot (\Bb-\Bx)/c - i k \Bomega \cdot (\Ba-\Bx)/c } F(\Bb, 0) \gamma_0 F(\Ba, 0) \gamma_0
+e^{ - i m \Bomega \cdot (\Bb-\Bx)/c + i k \Bomega \cdot (\Ba-\Bx)/c } \gamma_0 F(\Bb, 0) \gamma_0 F(\Ba, 0)) \\
&=\inv{V^2} \sum_{ k <m } \int d^3 a d^3 b \\
&e^{i \Bomega (m -k) t } \times \\
&\quad (e^{   i m \Bomega \cdot (\Bb-\Bx)/c - i k \Bomega \cdot (\Ba-\Bx)/c } F(\Bb, 0) \gamma_0 F(\Ba, 0) \gamma_0
+e^{ - i m \Bomega \cdot (\Bb-\Bx)/c + i k \Bomega \cdot (\Ba-\Bx)/c } \gamma_0 F(\Bb, 0) \gamma_0 F(\Ba, 0)) \\
&+e^{-i \Bomega (m -k) t } \times \\
&\quad (e^{   i k \Bomega \cdot (\Bb-\Bx)/c - i m \Bomega \cdot (\Ba-\Bx)/c } F(\Bb, 0) \gamma_0 F(\Ba, 0) \gamma_0
+e^{ - i k \Bomega \cdot (\Bb-\Bx)/c + i m \Bomega \cdot (\Ba-\Bx)/c } \gamma_0 F(\Bb, 0) \gamma_0 F(\Ba, 0)) \\
&=\inv{V^2} \sum_{ k <m } \int d^3 a d^3 b \\
&e^{i \Bomega (m -k) t } \times \\
&\quad (e^{   i m \Bomega \cdot \Bb/c - i k \Bomega \cdot \Ba/c + i (k-m) \Bomega \cdot \Bx/c } F(\Bb, 0) \gamma_0 F(\Ba, 0) \gamma_0
+e^{ i k \Bomega \cdot \Ba/c - i m \Bomega \cdot \Bb/c - i (k -m) \Bomega \cdot \Bx/c } \gamma_0 F(\Bb, 0) \gamma_0 F(\Ba, 0)) \\
&+e^{-i \Bomega (m -k) t } \times \\
&\quad (e^{   i k \Bomega \cdot \Bb/c - i m \Bomega \cdot \Ba/c -i (k-m) \Bomega \cdot \Bx/c } F(\Bb, 0) \gamma_0 F(\Ba, 0) \gamma_0
+e^{ - i k \Bomega \cdot \Bb/c + i m \Bomega \cdot \Ba/c + i (k-m) \Bomega \cdot \Bx/c } \gamma_0 F(\Bb, 0) \gamma_0 F(\Ba, 0)) \\
&=\inv{V^2} \sum_{ k <m } \int d^3 a d^3 b \times \\
&(e^{
  i \Bomega (m -k) t
+ i m \Bomega \cdot \Bb/c
- i k \Bomega \cdot \Ba/c
+ i (k-m) \Bomega \cdot \Bx/c
}
+e^{
- i \Bomega (m -k) t
- i m \Bomega \cdot \Ba/c
+ i k \Bomega \cdot \Bb/c
- i (k-m) \Bomega \cdot \Bx/c
}
)
F(\Bb, 0) \gamma_0 F(\Ba, 0) \gamma_0  \\
&+(e^{
  i \Bomega (m -k) t
+ i k \Bomega \cdot \Ba/c
- i m \Bomega \cdot \Bb/c
- i (k -m) \Bomega \cdot \Bx/c
}
+e^{
- i \Bomega (m -k) t
- i k \Bomega \cdot \Bb/c
+ i m \Bomega \cdot \Ba/c
+ i (k-m) \Bomega \cdot \Bx/c
}
) \gamma_0 F(\Bb, 0) \gamma_0 F(\Ba, 0)  \\
\end{aligned}
\end{equation}

Damn.  What an unholy mess, and does not obviously simplify.  Let us start over for this part.

\begin{equation}\label{eqn:scratchPlanewave:120}
\begin{aligned}
\sum_{\Bk \ne \Bm} &(
e^{i \Bomega_k t}
\hat{F}_{\Bk}
e^{-i \Bk \cdot \Bx}
\gamma_0
e^{i \Bomega_m t}
\hat{F}_{\Bm}
e^{-i \Bm \cdot \Bx}
\gamma_0
+ \gamma_0
e^{i \Bomega_k t}
\hat{F}_{\Bk}
e^{-i \Bk \cdot \Bx}
\gamma_0
e^{i \Bomega_m t}
\hat{F}_{\Bm}
e^{-i \Bm \cdot \Bx}
) \\
&=
\sum_{\Bk \ne \Bm}
e^{i (\Bomega_k -\Bomega_m) t}
(
e^{ i (\Bm -\Bk)\cdot \Bx}
\hat{F}_{\Bk}
\gamma_0
\hat{F}_{\Bm}
\gamma_0
+
e^{ i (\Bk -\Bm) \cdot \Bx}
\gamma_0
\hat{F}_{\Bk}
\gamma_0
\hat{F}_{\Bm}
) \\
&=
\sum_{\Bk \ne \Bm}
e^{i (k-m) \Bomega t}
(
e^{ i (m -k) \Bomega \cdot \Bx/c}
\hat{F}_{\Bk}
\gamma_0
\hat{F}_{\Bm}
\gamma_0
+
e^{ i (k -m) \Bomega \cdot \Bx/c}
\gamma_0
\hat{F}_{\Bk}
\gamma_0
\hat{F}_{\Bm}
) \\
\end{aligned}
\end{equation}

To progress it appears that we have to examine the commutation behavior of \(\gamma_0\) and \(\hat{F}_{\Bk}\), and probably also to

%\bibliographystyle{plainnat}
%\bibliography{myrefs}

%\end{document}


%
% Copyright � 2012 Peeter Joot.  All Rights Reserved.
% Licenced as described in the file LICENSE under the root directory of this GIT repository.
%

% 
% 
%\documentclass{article}

%\usepackage{amsmath}
\usepackage{mathpazo}

%
% shorthand for bold symbols, convenient for vectors and matrices
%
\newcommand{\Ba}[0]{\mathbf{a}}
\newcommand{\Bb}[0]{\mathbf{b}}
\newcommand{\Bc}[0]{\mathbf{c}}
\newcommand{\Bd}[0]{\mathbf{d}}
\newcommand{\Be}[0]{\mathbf{e}}
\newcommand{\Bf}[0]{\mathbf{f}}
\newcommand{\Bg}[0]{\mathbf{g}}
\newcommand{\Bh}[0]{\mathbf{h}}
\newcommand{\Bi}[0]{\mathbf{i}}
\newcommand{\Bj}[0]{\mathbf{j}}
\newcommand{\Bk}[0]{\mathbf{k}}
\newcommand{\Bl}[0]{\mathbf{l}}
\newcommand{\Bm}[0]{\mathbf{m}}
\newcommand{\Bn}[0]{\mathbf{n}}
\newcommand{\Bo}[0]{\mathbf{o}}
\newcommand{\Bp}[0]{\mathbf{p}}
\newcommand{\Bq}[0]{\mathbf{q}}
\newcommand{\Br}[0]{\mathbf{r}}
\newcommand{\Bs}[0]{\mathbf{s}}
\newcommand{\Bt}[0]{\mathbf{t}}
\newcommand{\Bu}[0]{\mathbf{u}}
\newcommand{\Bv}[0]{\mathbf{v}}
\newcommand{\Bw}[0]{\mathbf{w}}
\newcommand{\Bx}[0]{\mathbf{x}}
\newcommand{\By}[0]{\mathbf{y}}
\newcommand{\Bz}[0]{\mathbf{z}}
\newcommand{\BA}[0]{\mathbf{A}}
\newcommand{\BB}[0]{\mathbf{B}}
\newcommand{\BC}[0]{\mathbf{C}}
\newcommand{\BD}[0]{\mathbf{D}}
\newcommand{\BE}[0]{\mathbf{E}}
\newcommand{\BF}[0]{\mathbf{F}}
\newcommand{\BG}[0]{\mathbf{G}}
\newcommand{\BH}[0]{\mathbf{H}}
\newcommand{\BI}[0]{\mathbf{I}}
\newcommand{\BJ}[0]{\mathbf{J}}
\newcommand{\BK}[0]{\mathbf{K}}
\newcommand{\BL}[0]{\mathbf{L}}
\newcommand{\BM}[0]{\mathbf{M}}
\newcommand{\BN}[0]{\mathbf{N}}
\newcommand{\BO}[0]{\mathbf{O}}
\newcommand{\BP}[0]{\mathbf{P}}
\newcommand{\BQ}[0]{\mathbf{Q}}
\newcommand{\BR}[0]{\mathbf{R}}
\newcommand{\BS}[0]{\mathbf{S}}
\newcommand{\BT}[0]{\mathbf{T}}
\newcommand{\BU}[0]{\mathbf{U}}
\newcommand{\BV}[0]{\mathbf{V}}
\newcommand{\BW}[0]{\mathbf{W}}
\newcommand{\BX}[0]{\mathbf{X}}
\newcommand{\BY}[0]{\mathbf{Y}}
\newcommand{\BZ}[0]{\mathbf{Z}}

\newcommand{\Bzero}[0]{\mathbf{0}}
\newcommand{\Btheta}[0]{\boldsymbol{\theta}}
\newcommand{\Btau}[0]{\boldsymbol{\tau}}
\newcommand{\Bomega}[0]{\boldsymbol{\omega}}

%
% shorthand for unit vectors
%
\newcommand{\acap}[0]{\hat{\Ba}}
\newcommand{\bcap}[0]{\hat{\Bb}}
\newcommand{\ccap}[0]{\hat{\Bc}}
\newcommand{\dcap}[0]{\hat{\Bd}}
\newcommand{\ecap}[0]{\hat{\Be}}
\newcommand{\fcap}[0]{\hat{\Bf}}
\newcommand{\gcap}[0]{\hat{\Bg}}
\newcommand{\hcap}[0]{\hat{\Bh}}
\newcommand{\icap}[0]{\hat{\Bi}}
\newcommand{\jcap}[0]{\hat{\Bj}}
\newcommand{\kcap}[0]{\hat{\Bk}}
\newcommand{\lcap}[0]{\hat{\Bl}}
\newcommand{\mcap}[0]{\hat{\Bm}}
\newcommand{\ncap}[0]{\hat{\Bn}}
\newcommand{\ocap}[0]{\hat{\Bo}}
\newcommand{\pcap}[0]{\hat{\Bp}}
\newcommand{\qcap}[0]{\hat{\Bq}}
\newcommand{\rcap}[0]{\hat{\Br}}
\newcommand{\scap}[0]{\hat{\Bs}}
\newcommand{\tcap}[0]{\hat{\Bt}}
\newcommand{\ucap}[0]{\hat{\Bu}}
\newcommand{\vcap}[0]{\hat{\Bv}}
\newcommand{\wcap}[0]{\hat{\Bw}}
\newcommand{\xcap}[0]{\hat{\Bx}}
\newcommand{\ycap}[0]{\hat{\By}}
\newcommand{\zcap}[0]{\hat{\Bz}}
\newcommand{\thetacap}[0]{\hat{\Btheta}}

%
% to write R^n and C^n in a distinguishable fashion.  Perhaps change this
% to the double lined characters upon figuring out how to do so.
%
\newcommand{\C}[1]{$\mathbb{C}^{#1}$}
\newcommand{\R}[1]{$\mathbb{R}^{#1}$}

%
% various generally useful helpers
%

% derivative of #1 wrt. #2:
\newcommand{\D}[2] {\frac {d#2} {d#1}}

\newcommand{\inv}[1]{\frac{1}{#1}}
\newcommand{\cross}[0]{\times}

\newcommand{\abs}[1]{\lvert{#1}\rvert}
\newcommand{\norm}[1]{\lVert{#1}\rVert}
\newcommand{\innerprod}[2]{\langle{#1}, {#2}\rangle}
\newcommand{\dotprod}[2]{{#1} \cdot {#2}}
\newcommand{\bdotprod}[2]{\left({#1} \cdot {#2}\right)}
\newcommand{\crossprod}[2]{{#1} \cross {#2}}
\newcommand{\tripleprod}[3]{\dotprod{\left(\crossprod{#1}{#2}\right)}{#3}}

\DeclareMathOperator{\Proj}{Proj}
\DeclareMathOperator{\Span}{span}
\DeclareMathOperator{\Sgn}{sgn}
\DeclareMathOperator{\Area}{Area}
\DeclareMathOperator{\Volume}{Volume}

%
% A few miscellaneous things specific to this document
%
\newcommand{\crossop}[1]{\crossprod{#1}{}}

% R2 vector.
\newcommand{\VectorTwo}[2]{
\begin{bmatrix}
 {#1} \\
 {#2}
\end{bmatrix}
}

\newcommand{\VectorN}[1]{
\begin{bmatrix}
{#1}_1 \\
{#1}_2 \\
\vdots \\
{#1}_N \\
\end{bmatrix}
}

\newcommand{\DETuvij}[4]{
\begin{vmatrix}
 {#1}_{#3} & {#1}_{#4} \\
 {#2}_{#3} & {#2}_{#4}
\end{vmatrix}
}

\newcommand{\DETuvwijk}[6]{
\begin{vmatrix}
 {#1}_{#4} & {#1}_{#5} & {#1}_{#6} \\
 {#2}_{#4} & {#2}_{#5} & {#2}_{#6} \\
 {#3}_{#4} & {#3}_{#5} & {#3}_{#6}
\end{vmatrix}
}

\newcommand{\DETuvwxijkl}[8]{
\begin{vmatrix}
 {#1}_{#5} & {#1}_{#6} & {#1}_{#7} & {#1}_{#8} \\
 {#2}_{#5} & {#2}_{#6} & {#2}_{#7} & {#2}_{#8} \\
 {#3}_{#5} & {#3}_{#6} & {#3}_{#7} & {#3}_{#8} \\
 {#4}_{#5} & {#4}_{#6} & {#4}_{#7} & {#4}_{#8} \\
\end{vmatrix}
}

%\newcommand{\DETuvwxyijklm}[10]{
%\begin{vmatrix}
% {#1}_{#6} & {#1}_{#7} & {#1}_{#8} & {#1}_{#9} & {#1}_{#10} \\
% {#2}_{#6} & {#2}_{#7} & {#2}_{#8} & {#2}_{#9} & {#2}_{#10} \\
% {#3}_{#6} & {#3}_{#7} & {#3}_{#8} & {#3}_{#9} & {#3}_{#10} \\
% {#4}_{#6} & {#4}_{#7} & {#4}_{#8} & {#4}_{#9} & {#4}_{#10} \\
% {#5}_{#6} & {#5}_{#7} & {#5}_{#8} & {#5}_{#9} & {#5}_{#10}
%\end{vmatrix}
%}

% R3 vector.
\newcommand{\VectorThree}[3]{
\begin{bmatrix}
 {#1} \\
 {#2} \\
 {#3}
\end{bmatrix}
}


%%<misc>
%
\newcommand{\Abs}[1]{{\left\lvert{#1}\right\rvert}}
\newcommand{\spacegrad}[0]{\boldsymbol{\nabla}}
\newcommand{\grad}[0]{\nabla}
\newcommand{\LL}[0]{\mathcal{L}}

% == \partial_{#1} {#2}
\newcommand{\PD}[2]{\frac{\partial {#2}}{\partial {#1}}}
% inline variant
\newcommand{\PDi}[2]{{\partial {#2}}/{\partial {#1}}}

\newcommand{\PDD}[3]{\frac{\partial^2 {#3}}{\partial {#1}\partial {#2}}}
%\newcommand{\PDd}[2]{\frac{\partial^2 {#2}}{{\partial{#1}}^2}}
\newcommand{\PDsq}[2]{\frac{\partial^2 {#2}}{(\partial {#1})^2}}

\newcommand{\Partial}[2]{\frac{\partial {#1}}{\partial {#2}}}
\DeclareMathOperator{\RejName}{Rej}
\newcommand{\Rej}[2]{\RejName_{#1}\left( {#2} \right)}
\newcommand{\Rm}[1]{\mathbb{R}^{#1}}
\newcommand{\Cm}[1]{\mathbb{C}^{#1}}
\newcommand{\conj}[0]{{*}}

%</misc>

% <grade selection>
%
\newcommand{\gpgrade}[2] {{\left\langle{{#1}}\right\rangle}_{#2}}

\newcommand{\gpgradezero}[1] {\gpgrade{#1}{}}
%\newcommand{\gpscalargrade}[1] {{\left\langle{{#1}}\right\rangle}}
%\newcommand{\gpgradezero}[1] {\gpgrade{#1}{0}}

%\newcommand{\gpgradeone}[1] {{\left\langle{{#1}}\right\rangle}_{1}}
\newcommand{\gpgradeone}[1] {\gpgrade{#1}{1}}

\newcommand{\gpgradetwo}[1] {\gpgrade{#1}{2}}
\newcommand{\gpgradethree}[1] {\gpgrade{#1}{3}}
\newcommand{\gpgradefour}[1] {\gpgrade{#1}{4}}
%
% </grade selection>



\newcommand{\adot}[0]{{\dot{a}}}
\newcommand{\bdot}[0]{{\dot{b}}}
% taken for centered dot:
%\newcommand{\cdot}[0]{{\dot{c}}}
%\newcommand{\ddot}[0]{{\dot{d}}}
\newcommand{\edot}[0]{{\dot{e}}}
\newcommand{\fdot}[0]{{\dot{f}}}
\newcommand{\gdot}[0]{{\dot{g}}}
\newcommand{\hdot}[0]{{\dot{h}}}
\newcommand{\idot}[0]{{\dot{i}}}
\newcommand{\jdot}[0]{{\dot{j}}}
\newcommand{\kdot}[0]{{\dot{k}}}
\newcommand{\ldot}[0]{{\dot{l}}}
\newcommand{\mdot}[0]{{\dot{m}}}
\newcommand{\ndot}[0]{{\dot{n}}}
%\newcommand{\odot}[0]{{\dot{o}}}
\newcommand{\pdot}[0]{{\dot{p}}}
\newcommand{\qdot}[0]{{\dot{q}}}
\newcommand{\rdot}[0]{{\dot{r}}}
\newcommand{\sdot}[0]{{\dot{s}}}
\newcommand{\tdot}[0]{{\dot{t}}}
\newcommand{\udot}[0]{{\dot{u}}}
\newcommand{\vdot}[0]{{\dot{v}}}
\newcommand{\wdot}[0]{{\dot{w}}}
\newcommand{\xdot}[0]{{\dot{x}}}
\newcommand{\ydot}[0]{{\dot{y}}}
\newcommand{\zdot}[0]{{\dot{z}}}
\newcommand{\addot}[0]{{\ddot{a}}}
\newcommand{\bddot}[0]{{\ddot{b}}}
\newcommand{\cddot}[0]{{\ddot{c}}}
%\newcommand{\dddot}[0]{{\ddot{d}}}
\newcommand{\eddot}[0]{{\ddot{e}}}
\newcommand{\fddot}[0]{{\ddot{f}}}
\newcommand{\gddot}[0]{{\ddot{g}}}
\newcommand{\hddot}[0]{{\ddot{h}}}
\newcommand{\iddot}[0]{{\ddot{i}}}
\newcommand{\jddot}[0]{{\ddot{j}}}
\newcommand{\kddot}[0]{{\ddot{k}}}
\newcommand{\lddot}[0]{{\ddot{l}}}
\newcommand{\mddot}[0]{{\ddot{m}}}
\newcommand{\nddot}[0]{{\ddot{n}}}
\newcommand{\oddot}[0]{{\ddot{o}}}
\newcommand{\pddot}[0]{{\ddot{p}}}
\newcommand{\qddot}[0]{{\ddot{q}}}
\newcommand{\rddot}[0]{{\ddot{r}}}
\newcommand{\sddot}[0]{{\ddot{s}}}
\newcommand{\tddot}[0]{{\ddot{t}}}
\newcommand{\uddot}[0]{{\ddot{u}}}
\newcommand{\vddot}[0]{{\ddot{v}}}
\newcommand{\wddot}[0]{{\ddot{w}}}
\newcommand{\xddot}[0]{{\ddot{x}}}
\newcommand{\yddot}[0]{{\ddot{y}}}
\newcommand{\zddot}[0]{{\ddot{z}}}

%<bold and dot greek symbols>
%

\newcommand{\Deltadot}[0]{{\dot{\Delta}}}
\newcommand{\Gammadot}[0]{{\dot{\Gamma}}}
\newcommand{\Lambdadot}[0]{{\dot{\Lambda}}}
\newcommand{\Omegadot}[0]{{\dot{\Omega}}}
\newcommand{\Phidot}[0]{{\dot{\Phi}}}
\newcommand{\Pidot}[0]{{\dot{\Pi}}}
\newcommand{\Psidot}[0]{{\dot{\Psi}}}
\newcommand{\Sigmadot}[0]{{\dot{\Sigma}}}
\newcommand{\Thetadot}[0]{{\dot{\Theta}}}
\newcommand{\Upsilondot}[0]{{\dot{\Upsilon}}}
\newcommand{\Xidot}[0]{{\dot{\Xi}}}
\newcommand{\alphadot}[0]{{\dot{\alpha}}}
\newcommand{\betadot}[0]{{\dot{\beta}}}
\newcommand{\chidot}[0]{{\dot{\chi}}}
\newcommand{\deltadot}[0]{{\dot{\delta}}}
\newcommand{\epsilondot}[0]{{\dot{\epsilon}}}
\newcommand{\etadot}[0]{{\dot{\eta}}}
\newcommand{\gammadot}[0]{{\dot{\gamma}}}
\newcommand{\kappadot}[0]{{\dot{\kappa}}}
\newcommand{\lambdadot}[0]{{\dot{\lambda}}}
\newcommand{\mudot}[0]{{\dot{\mu}}}
\newcommand{\nudot}[0]{{\dot{\nu}}}
\newcommand{\omegadot}[0]{{\dot{\omega}}}
\newcommand{\phidot}[0]{{\dot{\phi}}}
\newcommand{\pidot}[0]{{\dot{\pi}}}
\newcommand{\psidot}[0]{{\dot{\psi}}}
\newcommand{\rhodot}[0]{{\dot{\rho}}}
\newcommand{\sigmadot}[0]{{\dot{\sigma}}}
\newcommand{\taudot}[0]{{\dot{\tau}}}
\newcommand{\thetadot}[0]{{\dot{\theta}}}
\newcommand{\upsilondot}[0]{{\dot{\upsilon}}}
\newcommand{\varepsilondot}[0]{{\dot{\varepsilon}}}
\newcommand{\varphidot}[0]{{\dot{\varphi}}}
\newcommand{\varpidot}[0]{{\dot{\varpi}}}
\newcommand{\varrhodot}[0]{{\dot{\varrho}}}
\newcommand{\varsigmadot}[0]{{\dot{\varsigma}}}
\newcommand{\varthetadot}[0]{{\dot{\vartheta}}}
\newcommand{\xidot}[0]{{\dot{\xi}}}
\newcommand{\zetadot}[0]{{\dot{\zeta}}}

\newcommand{\Deltaddot}[0]{{\ddot{\Delta}}}
\newcommand{\Gammaddot}[0]{{\ddot{\Gamma}}}
\newcommand{\Lambdaddot}[0]{{\ddot{\Lambda}}}
\newcommand{\Omegaddot}[0]{{\ddot{\Omega}}}
\newcommand{\Phiddot}[0]{{\ddot{\Phi}}}
\newcommand{\Piddot}[0]{{\ddot{\Pi}}}
\newcommand{\Psiddot}[0]{{\ddot{\Psi}}}
\newcommand{\Sigmaddot}[0]{{\ddot{\Sigma}}}
\newcommand{\Thetaddot}[0]{{\ddot{\Theta}}}
\newcommand{\Upsilonddot}[0]{{\ddot{\Upsilon}}}
\newcommand{\Xiddot}[0]{{\ddot{\Xi}}}
\newcommand{\alphaddot}[0]{{\ddot{\alpha}}}
\newcommand{\betaddot}[0]{{\ddot{\beta}}}
\newcommand{\chiddot}[0]{{\ddot{\chi}}}
\newcommand{\deltaddot}[0]{{\ddot{\delta}}}
\newcommand{\epsilonddot}[0]{{\ddot{\epsilon}}}
\newcommand{\etaddot}[0]{{\ddot{\eta}}}
\newcommand{\gammaddot}[0]{{\ddot{\gamma}}}
\newcommand{\kappaddot}[0]{{\ddot{\kappa}}}
\newcommand{\lambdaddot}[0]{{\ddot{\lambda}}}
\newcommand{\muddot}[0]{{\ddot{\mu}}}
\newcommand{\nuddot}[0]{{\ddot{\nu}}}
\newcommand{\omegaddot}[0]{{\ddot{\omega}}}
\newcommand{\phiddot}[0]{{\ddot{\phi}}}
\newcommand{\piddot}[0]{{\ddot{\pi}}}
\newcommand{\psiddot}[0]{{\ddot{\psi}}}
\newcommand{\rhoddot}[0]{{\ddot{\rho}}}
\newcommand{\sigmaddot}[0]{{\ddot{\sigma}}}
\newcommand{\tauddot}[0]{{\ddot{\tau}}}
\newcommand{\thetaddot}[0]{{\ddot{\theta}}}
\newcommand{\upsilonddot}[0]{{\ddot{\upsilon}}}
\newcommand{\varepsilonddot}[0]{{\ddot{\varepsilon}}}
\newcommand{\varphiddot}[0]{{\ddot{\varphi}}}
\newcommand{\varpiddot}[0]{{\ddot{\varpi}}}
\newcommand{\varrhoddot}[0]{{\ddot{\varrho}}}
\newcommand{\varsigmaddot}[0]{{\ddot{\varsigma}}}
\newcommand{\varthetaddot}[0]{{\ddot{\vartheta}}}
\newcommand{\xiddot}[0]{{\ddot{\xi}}}
\newcommand{\zetaddot}[0]{{\ddot{\zeta}}}

\newcommand{\BDelta}[0]{\boldsymbol{\Delta}}
\newcommand{\BGamma}[0]{\boldsymbol{\Gamma}}
\newcommand{\BLambda}[0]{\boldsymbol{\Lambda}}
\newcommand{\BOmega}[0]{\boldsymbol{\Omega}}
\newcommand{\BPhi}[0]{\boldsymbol{\Phi}}
\newcommand{\BPi}[0]{\boldsymbol{\Pi}}
\newcommand{\BPsi}[0]{\boldsymbol{\Psi}}
\newcommand{\BSigma}[0]{\boldsymbol{\Sigma}}
\newcommand{\BTheta}[0]{\boldsymbol{\Theta}}
\newcommand{\BUpsilon}[0]{\boldsymbol{\Upsilon}}
\newcommand{\BXi}[0]{\boldsymbol{\Xi}}
\newcommand{\Balpha}[0]{\boldsymbol{\alpha}}
\newcommand{\Bbeta}[0]{\boldsymbol{\beta}}
\newcommand{\Bchi}[0]{\boldsymbol{\chi}}
\newcommand{\Bdelta}[0]{\boldsymbol{\delta}}
\newcommand{\Bepsilon}[0]{\boldsymbol{\epsilon}}
\newcommand{\Beta}[0]{\boldsymbol{\eta}}
\newcommand{\Bgamma}[0]{\boldsymbol{\gamma}}
\newcommand{\Bkappa}[0]{\boldsymbol{\kappa}}
\newcommand{\Blambda}[0]{\boldsymbol{\lambda}}
\newcommand{\Bmu}[0]{\boldsymbol{\mu}}
\newcommand{\Bnu}[0]{\boldsymbol{\nu}}
%\newcommand{\Bomega}[0]{\boldsymbol{\omega}}
\newcommand{\Bphi}[0]{\boldsymbol{\phi}}
\newcommand{\Bpi}[0]{\boldsymbol{\pi}}
\newcommand{\Bpsi}[0]{\boldsymbol{\psi}}
\newcommand{\Brho}[0]{\boldsymbol{\rho}}
\newcommand{\Bsigma}[0]{\boldsymbol{\sigma}}
%\newcommand{\Btau}[0]{\boldsymbol{\tau}}
%\newcommand{\Btheta}[0]{\boldsymbol{\theta}}
\newcommand{\Bupsilon}[0]{\boldsymbol{\upsilon}}
\newcommand{\Bvarepsilon}[0]{\boldsymbol{\varepsilon}}
\newcommand{\Bvarphi}[0]{\boldsymbol{\varphi}}
\newcommand{\Bvarpi}[0]{\boldsymbol{\varpi}}
\newcommand{\Bvarrho}[0]{\boldsymbol{\varrho}}
\newcommand{\Bvarsigma}[0]{\boldsymbol{\varsigma}}
\newcommand{\Bvartheta}[0]{\boldsymbol{\vartheta}}
\newcommand{\Bxi}[0]{\boldsymbol{\xi}}
\newcommand{\Bzeta}[0]{\boldsymbol{\zeta}}
%
%</bold and dot greek symbols>
%<infrequent>
%
%\newcommand{\AreaOp}[1]{\AName_{#1}}
%\newcommand{\Babs}[0]{\abs{\BB}}
%\newcommand{\Bcap}[0]{\hat{\BB}}
%\newcommand{\BrPrimeRej}[0]{\rcap(\rcap \wedge \Br')}
%\newcommand{\CA}[0]{\mathcal{A}}
%\newcommand{\Cos}[1]{\cos{\left({#1}\right)}}
%\newcommand{\Det}[1] {\abs{#1}}
%\newcommand{\Dsq}[2] {\frac {\partial^2 {#1}} {\partial {#2}^2}}
%\newcommand{\Exp}[1]{\exp{\left({#1}\right)}}
%\newcommand{\Norm}[1]{\left\lVert{#1}\right\rVert}
%\newcommand{\Sin}[1]{\sin{\left({#1}\right)}}
%\newcommand{\T}[0]{\text{T}}
%\newcommand{\VolumeOp}[1]{\VName_{#1}}
%\newcommand{\agrad}[0]{\Ba \cdot \nabla}
%\newcommand{\alphacap}[0]{\hat{\boldsymbol{\alpha}}}
%\newcommand{\Fcap}[0]{\hat{\BF}}
%\newcommand{\bithree}[0]{{\Bi}_3}
%\newcommand{\bxa}[0]{\Bx\Ba}
%\newcommand{\coordvec}[2]{
%\newcommand{\costheta}[0]{\acap \cdot \xcap}
%\newcommand{\ddt}[1]{\ddot{#1}}
%\newcommand{\ddu}[1] {\frac {d{#1}} {du}}
%\newcommand{\dsqxj}[2] {\frac {\partial^2 {#1}} {\partial {x_{#2}}^2}}
%\newcommand{\dtheta}[1]{\frac{d {#1}}{d \theta}}
%\newcommand{\dt}[1]{\dot{#1}}
%\newcommand{\dt}[1]{\frac{d {#1}}{dt}}
%\newcommand{\dxj}[2] {\frac {\partial {#1}} {\partial {x_{#2}}}}
%\newcommand{\halfPhi}[0]{\frac{\phi}{2}}
%\newcommand{\half}[0]{\inv{2}}
%\newcommand{\inv}[1]{\frac{1}{#1}}
%\newcommand{\laplacian}[0]{\nabla^2}
%\newcommand{\matrixoftx}[3]{
%\newcommand{\nrrp}[0]{\norm{\rcap \wedge \Br'}}
%\newcommand{\oiint}{\bigcirc \hspace{-1.4em} \int \hspace{-.8em} \int}
%\newcommand{\transpose}[1]{{#1}^{\text{T}}}
%\newcommand{\transpose}[1]{{{#1}^{\TextTranspose}}}
%\newcommand{\transpose}[1]{{{#1}^{\text{T}}}}
%\newcommand{\barA}[0]{\bar{A}}
%\newcommand{\qbar}[0]{\bar{q}}
%\newcommand{\qdotbar}[0]{\dot{\bar{q}}}
%
%</infrequent>





%\usepackage[bookmarks=true]{hyperref}

%\usepackage{color,cite,graphicx}
   % use colour in the document, put your citations as [1-4]
   % rather than [1,2,3,4] (it looks nicer, and the extended LaTeX2e
   % graphics package. 
%\usepackage{latexsym,amssymb,epsf} % do not remember if these are
   % needed, but their inclusion can not do any damage

\chapter{stub em fields}
\label{chap:stubEmFields}
%\author{Peeter Joot \quad peeterjoot@protonmail.com}
\date{ Feb dd, 2009.  \(RCSfile: stubEmFields.tex,v \) Last \(Revision: 1.9 \) \(Date: 2009/06/14 23:51:45 \) }

%\begin{document}

%\maketitle{}

%\tableofcontents

\section{}

\section{Misc.  Does not fit yet}

Taken from \citep{gabookII:PJFourierVacuum}


\subsection{Separate electric and magnetic fields with the boundary conditions unspecified}

Fixing the boundary value in terms of the initial field at \(t=0\) is not the only option.  We see similar things in classical mechanics in constant acceleration problems where one can use an initial position and velocity, or positions at two different times, and so forth.  Bohm leaves his equivalents to the integration constants \(C_{\Bk}\) unspecified.  If we do so too we have \eqnref{eqn:stub_em_fields:undetermined}.  How does that look with separate fields?

FIXME:TODO:...  

Cut and Paste Note.  That referenced equation was:
\begin{align}\label{eqn:stub_em_fields:undetermined}
F(\Bx,t) = \sum_{\Bk} 
\exp\left(i \Bk c t \right) 
\exp\left(-i \Bk \cdot \Bx \right) 
C_{\Bk} 
\end{align}

%\bibliographystyle{plainnat}
%\bibliography{myrefs}

%\end{document}

%\documentclass{article}

%\usepackage{amsmath}
\usepackage{mathpazo}

%
% shorthand for bold symbols, convenient for vectors and matrices
%
\newcommand{\Ba}[0]{\mathbf{a}}
\newcommand{\Bb}[0]{\mathbf{b}}
\newcommand{\Bc}[0]{\mathbf{c}}
\newcommand{\Bd}[0]{\mathbf{d}}
\newcommand{\Be}[0]{\mathbf{e}}
\newcommand{\Bf}[0]{\mathbf{f}}
\newcommand{\Bg}[0]{\mathbf{g}}
\newcommand{\Bh}[0]{\mathbf{h}}
\newcommand{\Bi}[0]{\mathbf{i}}
\newcommand{\Bj}[0]{\mathbf{j}}
\newcommand{\Bk}[0]{\mathbf{k}}
\newcommand{\Bl}[0]{\mathbf{l}}
\newcommand{\Bm}[0]{\mathbf{m}}
\newcommand{\Bn}[0]{\mathbf{n}}
\newcommand{\Bo}[0]{\mathbf{o}}
\newcommand{\Bp}[0]{\mathbf{p}}
\newcommand{\Bq}[0]{\mathbf{q}}
\newcommand{\Br}[0]{\mathbf{r}}
\newcommand{\Bs}[0]{\mathbf{s}}
\newcommand{\Bt}[0]{\mathbf{t}}
\newcommand{\Bu}[0]{\mathbf{u}}
\newcommand{\Bv}[0]{\mathbf{v}}
\newcommand{\Bw}[0]{\mathbf{w}}
\newcommand{\Bx}[0]{\mathbf{x}}
\newcommand{\By}[0]{\mathbf{y}}
\newcommand{\Bz}[0]{\mathbf{z}}
\newcommand{\BA}[0]{\mathbf{A}}
\newcommand{\BB}[0]{\mathbf{B}}
\newcommand{\BC}[0]{\mathbf{C}}
\newcommand{\BD}[0]{\mathbf{D}}
\newcommand{\BE}[0]{\mathbf{E}}
\newcommand{\BF}[0]{\mathbf{F}}
\newcommand{\BG}[0]{\mathbf{G}}
\newcommand{\BH}[0]{\mathbf{H}}
\newcommand{\BI}[0]{\mathbf{I}}
\newcommand{\BJ}[0]{\mathbf{J}}
\newcommand{\BK}[0]{\mathbf{K}}
\newcommand{\BL}[0]{\mathbf{L}}
\newcommand{\BM}[0]{\mathbf{M}}
\newcommand{\BN}[0]{\mathbf{N}}
\newcommand{\BO}[0]{\mathbf{O}}
\newcommand{\BP}[0]{\mathbf{P}}
\newcommand{\BQ}[0]{\mathbf{Q}}
\newcommand{\BR}[0]{\mathbf{R}}
\newcommand{\BS}[0]{\mathbf{S}}
\newcommand{\BT}[0]{\mathbf{T}}
\newcommand{\BU}[0]{\mathbf{U}}
\newcommand{\BV}[0]{\mathbf{V}}
\newcommand{\BW}[0]{\mathbf{W}}
\newcommand{\BX}[0]{\mathbf{X}}
\newcommand{\BY}[0]{\mathbf{Y}}
\newcommand{\BZ}[0]{\mathbf{Z}}

\newcommand{\Bzero}[0]{\mathbf{0}}
\newcommand{\Btheta}[0]{\boldsymbol{\theta}}
\newcommand{\Btau}[0]{\boldsymbol{\tau}}
\newcommand{\Bomega}[0]{\boldsymbol{\omega}}

%
% shorthand for unit vectors
%
\newcommand{\acap}[0]{\hat{\Ba}}
\newcommand{\bcap}[0]{\hat{\Bb}}
\newcommand{\ccap}[0]{\hat{\Bc}}
\newcommand{\dcap}[0]{\hat{\Bd}}
\newcommand{\ecap}[0]{\hat{\Be}}
\newcommand{\fcap}[0]{\hat{\Bf}}
\newcommand{\gcap}[0]{\hat{\Bg}}
\newcommand{\hcap}[0]{\hat{\Bh}}
\newcommand{\icap}[0]{\hat{\Bi}}
\newcommand{\jcap}[0]{\hat{\Bj}}
\newcommand{\kcap}[0]{\hat{\Bk}}
\newcommand{\lcap}[0]{\hat{\Bl}}
\newcommand{\mcap}[0]{\hat{\Bm}}
\newcommand{\ncap}[0]{\hat{\Bn}}
\newcommand{\ocap}[0]{\hat{\Bo}}
\newcommand{\pcap}[0]{\hat{\Bp}}
\newcommand{\qcap}[0]{\hat{\Bq}}
\newcommand{\rcap}[0]{\hat{\Br}}
\newcommand{\scap}[0]{\hat{\Bs}}
\newcommand{\tcap}[0]{\hat{\Bt}}
\newcommand{\ucap}[0]{\hat{\Bu}}
\newcommand{\vcap}[0]{\hat{\Bv}}
\newcommand{\wcap}[0]{\hat{\Bw}}
\newcommand{\xcap}[0]{\hat{\Bx}}
\newcommand{\ycap}[0]{\hat{\By}}
\newcommand{\zcap}[0]{\hat{\Bz}}
\newcommand{\thetacap}[0]{\hat{\Btheta}}

%
% to write R^n and C^n in a distinguishable fashion.  Perhaps change this
% to the double lined characters upon figuring out how to do so.
%
\newcommand{\C}[1]{$\mathbb{C}^{#1}$}
\newcommand{\R}[1]{$\mathbb{R}^{#1}$}

%
% various generally useful helpers
%

% derivative of #1 wrt. #2:
\newcommand{\D}[2] {\frac {d#2} {d#1}}

\newcommand{\inv}[1]{\frac{1}{#1}}
\newcommand{\cross}[0]{\times}

\newcommand{\abs}[1]{\lvert{#1}\rvert}
\newcommand{\norm}[1]{\lVert{#1}\rVert}
\newcommand{\innerprod}[2]{\langle{#1}, {#2}\rangle}
\newcommand{\dotprod}[2]{{#1} \cdot {#2}}
\newcommand{\bdotprod}[2]{\left({#1} \cdot {#2}\right)}
\newcommand{\crossprod}[2]{{#1} \cross {#2}}
\newcommand{\tripleprod}[3]{\dotprod{\left(\crossprod{#1}{#2}\right)}{#3}}

\DeclareMathOperator{\Proj}{Proj}
\DeclareMathOperator{\Span}{span}
\DeclareMathOperator{\Sgn}{sgn}
\DeclareMathOperator{\Area}{Area}
\DeclareMathOperator{\Volume}{Volume}

%
% A few miscellaneous things specific to this document
%
\newcommand{\crossop}[1]{\crossprod{#1}{}}

% R2 vector.
\newcommand{\VectorTwo}[2]{
\begin{bmatrix}
 {#1} \\
 {#2}
\end{bmatrix}
}

\newcommand{\VectorN}[1]{
\begin{bmatrix}
{#1}_1 \\
{#1}_2 \\
\vdots \\
{#1}_N \\
\end{bmatrix}
}

\newcommand{\DETuvij}[4]{
\begin{vmatrix}
 {#1}_{#3} & {#1}_{#4} \\
 {#2}_{#3} & {#2}_{#4}
\end{vmatrix}
}

\newcommand{\DETuvwijk}[6]{
\begin{vmatrix}
 {#1}_{#4} & {#1}_{#5} & {#1}_{#6} \\
 {#2}_{#4} & {#2}_{#5} & {#2}_{#6} \\
 {#3}_{#4} & {#3}_{#5} & {#3}_{#6}
\end{vmatrix}
}

\newcommand{\DETuvwxijkl}[8]{
\begin{vmatrix}
 {#1}_{#5} & {#1}_{#6} & {#1}_{#7} & {#1}_{#8} \\
 {#2}_{#5} & {#2}_{#6} & {#2}_{#7} & {#2}_{#8} \\
 {#3}_{#5} & {#3}_{#6} & {#3}_{#7} & {#3}_{#8} \\
 {#4}_{#5} & {#4}_{#6} & {#4}_{#7} & {#4}_{#8} \\
\end{vmatrix}
}

%\newcommand{\DETuvwxyijklm}[10]{
%\begin{vmatrix}
% {#1}_{#6} & {#1}_{#7} & {#1}_{#8} & {#1}_{#9} & {#1}_{#10} \\
% {#2}_{#6} & {#2}_{#7} & {#2}_{#8} & {#2}_{#9} & {#2}_{#10} \\
% {#3}_{#6} & {#3}_{#7} & {#3}_{#8} & {#3}_{#9} & {#3}_{#10} \\
% {#4}_{#6} & {#4}_{#7} & {#4}_{#8} & {#4}_{#9} & {#4}_{#10} \\
% {#5}_{#6} & {#5}_{#7} & {#5}_{#8} & {#5}_{#9} & {#5}_{#10}
%\end{vmatrix}
%}

% R3 vector.
\newcommand{\VectorThree}[3]{
\begin{bmatrix}
 {#1} \\
 {#2} \\
 {#3}
\end{bmatrix}
}


%%<misc>
%
\newcommand{\Abs}[1]{{\left\lvert{#1}\right\rvert}}
\newcommand{\spacegrad}[0]{\boldsymbol{\nabla}}
\newcommand{\grad}[0]{\nabla}
\newcommand{\LL}[0]{\mathcal{L}}

% == \partial_{#1} {#2}
\newcommand{\PD}[2]{\frac{\partial {#2}}{\partial {#1}}}
% inline variant
\newcommand{\PDi}[2]{{\partial {#2}}/{\partial {#1}}}

\newcommand{\PDD}[3]{\frac{\partial^2 {#3}}{\partial {#1}\partial {#2}}}
%\newcommand{\PDd}[2]{\frac{\partial^2 {#2}}{{\partial{#1}}^2}}
\newcommand{\PDsq}[2]{\frac{\partial^2 {#2}}{(\partial {#1})^2}}

\newcommand{\Partial}[2]{\frac{\partial {#1}}{\partial {#2}}}
\DeclareMathOperator{\RejName}{Rej}
\newcommand{\Rej}[2]{\RejName_{#1}\left( {#2} \right)}
\newcommand{\Rm}[1]{\mathbb{R}^{#1}}
\newcommand{\Cm}[1]{\mathbb{C}^{#1}}
\newcommand{\conj}[0]{{*}}

%</misc>

% <grade selection>
%
\newcommand{\gpgrade}[2] {{\left\langle{{#1}}\right\rangle}_{#2}}

\newcommand{\gpgradezero}[1] {\gpgrade{#1}{}}
%\newcommand{\gpscalargrade}[1] {{\left\langle{{#1}}\right\rangle}}
%\newcommand{\gpgradezero}[1] {\gpgrade{#1}{0}}

%\newcommand{\gpgradeone}[1] {{\left\langle{{#1}}\right\rangle}_{1}}
\newcommand{\gpgradeone}[1] {\gpgrade{#1}{1}}

\newcommand{\gpgradetwo}[1] {\gpgrade{#1}{2}}
\newcommand{\gpgradethree}[1] {\gpgrade{#1}{3}}
\newcommand{\gpgradefour}[1] {\gpgrade{#1}{4}}
%
% </grade selection>



\newcommand{\adot}[0]{{\dot{a}}}
\newcommand{\bdot}[0]{{\dot{b}}}
% taken for centered dot:
%\newcommand{\cdot}[0]{{\dot{c}}}
%\newcommand{\ddot}[0]{{\dot{d}}}
\newcommand{\edot}[0]{{\dot{e}}}
\newcommand{\fdot}[0]{{\dot{f}}}
\newcommand{\gdot}[0]{{\dot{g}}}
\newcommand{\hdot}[0]{{\dot{h}}}
\newcommand{\idot}[0]{{\dot{i}}}
\newcommand{\jdot}[0]{{\dot{j}}}
\newcommand{\kdot}[0]{{\dot{k}}}
\newcommand{\ldot}[0]{{\dot{l}}}
\newcommand{\mdot}[0]{{\dot{m}}}
\newcommand{\ndot}[0]{{\dot{n}}}
%\newcommand{\odot}[0]{{\dot{o}}}
\newcommand{\pdot}[0]{{\dot{p}}}
\newcommand{\qdot}[0]{{\dot{q}}}
\newcommand{\rdot}[0]{{\dot{r}}}
\newcommand{\sdot}[0]{{\dot{s}}}
\newcommand{\tdot}[0]{{\dot{t}}}
\newcommand{\udot}[0]{{\dot{u}}}
\newcommand{\vdot}[0]{{\dot{v}}}
\newcommand{\wdot}[0]{{\dot{w}}}
\newcommand{\xdot}[0]{{\dot{x}}}
\newcommand{\ydot}[0]{{\dot{y}}}
\newcommand{\zdot}[0]{{\dot{z}}}
\newcommand{\addot}[0]{{\ddot{a}}}
\newcommand{\bddot}[0]{{\ddot{b}}}
\newcommand{\cddot}[0]{{\ddot{c}}}
%\newcommand{\dddot}[0]{{\ddot{d}}}
\newcommand{\eddot}[0]{{\ddot{e}}}
\newcommand{\fddot}[0]{{\ddot{f}}}
\newcommand{\gddot}[0]{{\ddot{g}}}
\newcommand{\hddot}[0]{{\ddot{h}}}
\newcommand{\iddot}[0]{{\ddot{i}}}
\newcommand{\jddot}[0]{{\ddot{j}}}
\newcommand{\kddot}[0]{{\ddot{k}}}
\newcommand{\lddot}[0]{{\ddot{l}}}
\newcommand{\mddot}[0]{{\ddot{m}}}
\newcommand{\nddot}[0]{{\ddot{n}}}
\newcommand{\oddot}[0]{{\ddot{o}}}
\newcommand{\pddot}[0]{{\ddot{p}}}
\newcommand{\qddot}[0]{{\ddot{q}}}
\newcommand{\rddot}[0]{{\ddot{r}}}
\newcommand{\sddot}[0]{{\ddot{s}}}
\newcommand{\tddot}[0]{{\ddot{t}}}
\newcommand{\uddot}[0]{{\ddot{u}}}
\newcommand{\vddot}[0]{{\ddot{v}}}
\newcommand{\wddot}[0]{{\ddot{w}}}
\newcommand{\xddot}[0]{{\ddot{x}}}
\newcommand{\yddot}[0]{{\ddot{y}}}
\newcommand{\zddot}[0]{{\ddot{z}}}

%<bold and dot greek symbols>
%

\newcommand{\Deltadot}[0]{{\dot{\Delta}}}
\newcommand{\Gammadot}[0]{{\dot{\Gamma}}}
\newcommand{\Lambdadot}[0]{{\dot{\Lambda}}}
\newcommand{\Omegadot}[0]{{\dot{\Omega}}}
\newcommand{\Phidot}[0]{{\dot{\Phi}}}
\newcommand{\Pidot}[0]{{\dot{\Pi}}}
\newcommand{\Psidot}[0]{{\dot{\Psi}}}
\newcommand{\Sigmadot}[0]{{\dot{\Sigma}}}
\newcommand{\Thetadot}[0]{{\dot{\Theta}}}
\newcommand{\Upsilondot}[0]{{\dot{\Upsilon}}}
\newcommand{\Xidot}[0]{{\dot{\Xi}}}
\newcommand{\alphadot}[0]{{\dot{\alpha}}}
\newcommand{\betadot}[0]{{\dot{\beta}}}
\newcommand{\chidot}[0]{{\dot{\chi}}}
\newcommand{\deltadot}[0]{{\dot{\delta}}}
\newcommand{\epsilondot}[0]{{\dot{\epsilon}}}
\newcommand{\etadot}[0]{{\dot{\eta}}}
\newcommand{\gammadot}[0]{{\dot{\gamma}}}
\newcommand{\kappadot}[0]{{\dot{\kappa}}}
\newcommand{\lambdadot}[0]{{\dot{\lambda}}}
\newcommand{\mudot}[0]{{\dot{\mu}}}
\newcommand{\nudot}[0]{{\dot{\nu}}}
\newcommand{\omegadot}[0]{{\dot{\omega}}}
\newcommand{\phidot}[0]{{\dot{\phi}}}
\newcommand{\pidot}[0]{{\dot{\pi}}}
\newcommand{\psidot}[0]{{\dot{\psi}}}
\newcommand{\rhodot}[0]{{\dot{\rho}}}
\newcommand{\sigmadot}[0]{{\dot{\sigma}}}
\newcommand{\taudot}[0]{{\dot{\tau}}}
\newcommand{\thetadot}[0]{{\dot{\theta}}}
\newcommand{\upsilondot}[0]{{\dot{\upsilon}}}
\newcommand{\varepsilondot}[0]{{\dot{\varepsilon}}}
\newcommand{\varphidot}[0]{{\dot{\varphi}}}
\newcommand{\varpidot}[0]{{\dot{\varpi}}}
\newcommand{\varrhodot}[0]{{\dot{\varrho}}}
\newcommand{\varsigmadot}[0]{{\dot{\varsigma}}}
\newcommand{\varthetadot}[0]{{\dot{\vartheta}}}
\newcommand{\xidot}[0]{{\dot{\xi}}}
\newcommand{\zetadot}[0]{{\dot{\zeta}}}

\newcommand{\Deltaddot}[0]{{\ddot{\Delta}}}
\newcommand{\Gammaddot}[0]{{\ddot{\Gamma}}}
\newcommand{\Lambdaddot}[0]{{\ddot{\Lambda}}}
\newcommand{\Omegaddot}[0]{{\ddot{\Omega}}}
\newcommand{\Phiddot}[0]{{\ddot{\Phi}}}
\newcommand{\Piddot}[0]{{\ddot{\Pi}}}
\newcommand{\Psiddot}[0]{{\ddot{\Psi}}}
\newcommand{\Sigmaddot}[0]{{\ddot{\Sigma}}}
\newcommand{\Thetaddot}[0]{{\ddot{\Theta}}}
\newcommand{\Upsilonddot}[0]{{\ddot{\Upsilon}}}
\newcommand{\Xiddot}[0]{{\ddot{\Xi}}}
\newcommand{\alphaddot}[0]{{\ddot{\alpha}}}
\newcommand{\betaddot}[0]{{\ddot{\beta}}}
\newcommand{\chiddot}[0]{{\ddot{\chi}}}
\newcommand{\deltaddot}[0]{{\ddot{\delta}}}
\newcommand{\epsilonddot}[0]{{\ddot{\epsilon}}}
\newcommand{\etaddot}[0]{{\ddot{\eta}}}
\newcommand{\gammaddot}[0]{{\ddot{\gamma}}}
\newcommand{\kappaddot}[0]{{\ddot{\kappa}}}
\newcommand{\lambdaddot}[0]{{\ddot{\lambda}}}
\newcommand{\muddot}[0]{{\ddot{\mu}}}
\newcommand{\nuddot}[0]{{\ddot{\nu}}}
\newcommand{\omegaddot}[0]{{\ddot{\omega}}}
\newcommand{\phiddot}[0]{{\ddot{\phi}}}
\newcommand{\piddot}[0]{{\ddot{\pi}}}
\newcommand{\psiddot}[0]{{\ddot{\psi}}}
\newcommand{\rhoddot}[0]{{\ddot{\rho}}}
\newcommand{\sigmaddot}[0]{{\ddot{\sigma}}}
\newcommand{\tauddot}[0]{{\ddot{\tau}}}
\newcommand{\thetaddot}[0]{{\ddot{\theta}}}
\newcommand{\upsilonddot}[0]{{\ddot{\upsilon}}}
\newcommand{\varepsilonddot}[0]{{\ddot{\varepsilon}}}
\newcommand{\varphiddot}[0]{{\ddot{\varphi}}}
\newcommand{\varpiddot}[0]{{\ddot{\varpi}}}
\newcommand{\varrhoddot}[0]{{\ddot{\varrho}}}
\newcommand{\varsigmaddot}[0]{{\ddot{\varsigma}}}
\newcommand{\varthetaddot}[0]{{\ddot{\vartheta}}}
\newcommand{\xiddot}[0]{{\ddot{\xi}}}
\newcommand{\zetaddot}[0]{{\ddot{\zeta}}}

\newcommand{\BDelta}[0]{\boldsymbol{\Delta}}
\newcommand{\BGamma}[0]{\boldsymbol{\Gamma}}
\newcommand{\BLambda}[0]{\boldsymbol{\Lambda}}
\newcommand{\BOmega}[0]{\boldsymbol{\Omega}}
\newcommand{\BPhi}[0]{\boldsymbol{\Phi}}
\newcommand{\BPi}[0]{\boldsymbol{\Pi}}
\newcommand{\BPsi}[0]{\boldsymbol{\Psi}}
\newcommand{\BSigma}[0]{\boldsymbol{\Sigma}}
\newcommand{\BTheta}[0]{\boldsymbol{\Theta}}
\newcommand{\BUpsilon}[0]{\boldsymbol{\Upsilon}}
\newcommand{\BXi}[0]{\boldsymbol{\Xi}}
\newcommand{\Balpha}[0]{\boldsymbol{\alpha}}
\newcommand{\Bbeta}[0]{\boldsymbol{\beta}}
\newcommand{\Bchi}[0]{\boldsymbol{\chi}}
\newcommand{\Bdelta}[0]{\boldsymbol{\delta}}
\newcommand{\Bepsilon}[0]{\boldsymbol{\epsilon}}
\newcommand{\Beta}[0]{\boldsymbol{\eta}}
\newcommand{\Bgamma}[0]{\boldsymbol{\gamma}}
\newcommand{\Bkappa}[0]{\boldsymbol{\kappa}}
\newcommand{\Blambda}[0]{\boldsymbol{\lambda}}
\newcommand{\Bmu}[0]{\boldsymbol{\mu}}
\newcommand{\Bnu}[0]{\boldsymbol{\nu}}
%\newcommand{\Bomega}[0]{\boldsymbol{\omega}}
\newcommand{\Bphi}[0]{\boldsymbol{\phi}}
\newcommand{\Bpi}[0]{\boldsymbol{\pi}}
\newcommand{\Bpsi}[0]{\boldsymbol{\psi}}
\newcommand{\Brho}[0]{\boldsymbol{\rho}}
\newcommand{\Bsigma}[0]{\boldsymbol{\sigma}}
%\newcommand{\Btau}[0]{\boldsymbol{\tau}}
%\newcommand{\Btheta}[0]{\boldsymbol{\theta}}
\newcommand{\Bupsilon}[0]{\boldsymbol{\upsilon}}
\newcommand{\Bvarepsilon}[0]{\boldsymbol{\varepsilon}}
\newcommand{\Bvarphi}[0]{\boldsymbol{\varphi}}
\newcommand{\Bvarpi}[0]{\boldsymbol{\varpi}}
\newcommand{\Bvarrho}[0]{\boldsymbol{\varrho}}
\newcommand{\Bvarsigma}[0]{\boldsymbol{\varsigma}}
\newcommand{\Bvartheta}[0]{\boldsymbol{\vartheta}}
\newcommand{\Bxi}[0]{\boldsymbol{\xi}}
\newcommand{\Bzeta}[0]{\boldsymbol{\zeta}}
%
%</bold and dot greek symbols>
%<infrequent>
%
%\newcommand{\AreaOp}[1]{\AName_{#1}}
%\newcommand{\Babs}[0]{\abs{\BB}}
%\newcommand{\Bcap}[0]{\hat{\BB}}
%\newcommand{\BrPrimeRej}[0]{\rcap(\rcap \wedge \Br')}
%\newcommand{\CA}[0]{\mathcal{A}}
%\newcommand{\Cos}[1]{\cos{\left({#1}\right)}}
%\newcommand{\Det}[1] {\abs{#1}}
%\newcommand{\Dsq}[2] {\frac {\partial^2 {#1}} {\partial {#2}^2}}
%\newcommand{\Exp}[1]{\exp{\left({#1}\right)}}
%\newcommand{\Norm}[1]{\left\lVert{#1}\right\rVert}
%\newcommand{\Sin}[1]{\sin{\left({#1}\right)}}
%\newcommand{\T}[0]{\text{T}}
%\newcommand{\VolumeOp}[1]{\VName_{#1}}
%\newcommand{\agrad}[0]{\Ba \cdot \nabla}
%\newcommand{\alphacap}[0]{\hat{\boldsymbol{\alpha}}}
%\newcommand{\Fcap}[0]{\hat{\BF}}
%\newcommand{\bithree}[0]{{\Bi}_3}
%\newcommand{\bxa}[0]{\Bx\Ba}
%\newcommand{\coordvec}[2]{
%\newcommand{\costheta}[0]{\acap \cdot \xcap}
%\newcommand{\ddt}[1]{\ddot{#1}}
%\newcommand{\ddu}[1] {\frac {d{#1}} {du}}
%\newcommand{\dsqxj}[2] {\frac {\partial^2 {#1}} {\partial {x_{#2}}^2}}
%\newcommand{\dtheta}[1]{\frac{d {#1}}{d \theta}}
%\newcommand{\dt}[1]{\dot{#1}}
%\newcommand{\dt}[1]{\frac{d {#1}}{dt}}
%\newcommand{\dxj}[2] {\frac {\partial {#1}} {\partial {x_{#2}}}}
%\newcommand{\halfPhi}[0]{\frac{\phi}{2}}
%\newcommand{\half}[0]{\inv{2}}
%\newcommand{\inv}[1]{\frac{1}{#1}}
%\newcommand{\laplacian}[0]{\nabla^2}
%\newcommand{\matrixoftx}[3]{
%\newcommand{\nrrp}[0]{\norm{\rcap \wedge \Br'}}
%\newcommand{\oiint}{\bigcirc \hspace{-1.4em} \int \hspace{-.8em} \int}
%\newcommand{\transpose}[1]{{#1}^{\text{T}}}
%\newcommand{\transpose}[1]{{{#1}^{\TextTranspose}}}
%\newcommand{\transpose}[1]{{{#1}^{\text{T}}}}
%\newcommand{\barA}[0]{\bar{A}}
%\newcommand{\qbar}[0]{\bar{q}}
%\newcommand{\qdotbar}[0]{\dot{\bar{q}}}
%
%</infrequent>





%\usepackage{listings}
%\usepackage{txfonts} % for ointctr... (also appears to make "prettier" \int and \sum's)

%\usepackage[bookmarks=true]{hyperref}

\chapter{Lorentz force interaction } % Declares the document's title.
%\author{Peeter Joot \quad peeter.joot@gmail.com}
\date{ October 17, 2008.  $RCSfile: lorentzTxEmPotential.tex,v $ Last $Revision: 1.11 $ $Date: 2009/06/11 17:00:37 $ }

%\begin{document}

%\maketitle{}
%\tableofcontents

\section{Motivation.}

Explore the relationships between proper velocity and current density.

\cite{goldstein1951cm} has a combined Lagrangian for the field and Lorentz force:

\begin{align*}
\LL = \int \left\{ \frac{E^2 - B^2}{8 \pi} - \sum_i q_i \delta(\Br-\Br_i)\left(\phi - \Bv_i/c \cdot \BA\right)\right\} dV + \inv{2} m_i \Bv_i^2
\end{align*}

Where alternate variation of the field, or coordinates produces the field equations or Lorentz equation respectively.

Compare to the covariant Lagrangian for the field and interaction (for metric $+---$)

\begin{align}\label{eqn:lorentz_tx_em_potential:maxlag}
\LL_{\text{field}} &= -\frac{\epsilon_0}{2} (\grad \wedge A)^2 + A \cdot J/c \\
\LL_{\text{interaction}} &= \inv{2} m v^2 + q A \cdot (v/c)
\end{align}

(SI units this time).

These can be shown to produce Maxwell's equation and the Lorentz force equation respectively
\begin{align}
\grad F &= J/c\epsilon_0 \\
\pdot &= q F \cdot (v/c)
\end{align}

Derivations of these can be found in \cite{PJFieldLagrangian}, and \cite{PJSrLorentzForce} respectively.  

The final aim is
find how to relate the $A\cdot J$, and $q A \cdot v$ interaction quantities in covariant field and
Lorentz Lagrangians.  Intuition tells me that considering the Lorentz transformation of $J$ and $A$ may help
shed some light on this.

Some four potentials discussion perhaps useful as background can be found in \cite{PJFourPotential}.

%\bibliographystyle{plainnat}
%\bibliography{myrefs}

%\end{document}

%
% Copyright � 2012 Peeter Joot.  All Rights Reserved.
% Licenced as described in the file LICENSE under the root directory of this GIT repository.
%

%
%
%\documentclass{article}

%\usepackage{amsmath}
\usepackage{mathpazo}

%
% shorthand for bold symbols, convenient for vectors and matrices
%
\newcommand{\Ba}[0]{\mathbf{a}}
\newcommand{\Bb}[0]{\mathbf{b}}
\newcommand{\Bc}[0]{\mathbf{c}}
\newcommand{\Bd}[0]{\mathbf{d}}
\newcommand{\Be}[0]{\mathbf{e}}
\newcommand{\Bf}[0]{\mathbf{f}}
\newcommand{\Bg}[0]{\mathbf{g}}
\newcommand{\Bh}[0]{\mathbf{h}}
\newcommand{\Bi}[0]{\mathbf{i}}
\newcommand{\Bj}[0]{\mathbf{j}}
\newcommand{\Bk}[0]{\mathbf{k}}
\newcommand{\Bl}[0]{\mathbf{l}}
\newcommand{\Bm}[0]{\mathbf{m}}
\newcommand{\Bn}[0]{\mathbf{n}}
\newcommand{\Bo}[0]{\mathbf{o}}
\newcommand{\Bp}[0]{\mathbf{p}}
\newcommand{\Bq}[0]{\mathbf{q}}
\newcommand{\Br}[0]{\mathbf{r}}
\newcommand{\Bs}[0]{\mathbf{s}}
\newcommand{\Bt}[0]{\mathbf{t}}
\newcommand{\Bu}[0]{\mathbf{u}}
\newcommand{\Bv}[0]{\mathbf{v}}
\newcommand{\Bw}[0]{\mathbf{w}}
\newcommand{\Bx}[0]{\mathbf{x}}
\newcommand{\By}[0]{\mathbf{y}}
\newcommand{\Bz}[0]{\mathbf{z}}
\newcommand{\BA}[0]{\mathbf{A}}
\newcommand{\BB}[0]{\mathbf{B}}
\newcommand{\BC}[0]{\mathbf{C}}
\newcommand{\BD}[0]{\mathbf{D}}
\newcommand{\BE}[0]{\mathbf{E}}
\newcommand{\BF}[0]{\mathbf{F}}
\newcommand{\BG}[0]{\mathbf{G}}
\newcommand{\BH}[0]{\mathbf{H}}
\newcommand{\BI}[0]{\mathbf{I}}
\newcommand{\BJ}[0]{\mathbf{J}}
\newcommand{\BK}[0]{\mathbf{K}}
\newcommand{\BL}[0]{\mathbf{L}}
\newcommand{\BM}[0]{\mathbf{M}}
\newcommand{\BN}[0]{\mathbf{N}}
\newcommand{\BO}[0]{\mathbf{O}}
\newcommand{\BP}[0]{\mathbf{P}}
\newcommand{\BQ}[0]{\mathbf{Q}}
\newcommand{\BR}[0]{\mathbf{R}}
\newcommand{\BS}[0]{\mathbf{S}}
\newcommand{\BT}[0]{\mathbf{T}}
\newcommand{\BU}[0]{\mathbf{U}}
\newcommand{\BV}[0]{\mathbf{V}}
\newcommand{\BW}[0]{\mathbf{W}}
\newcommand{\BX}[0]{\mathbf{X}}
\newcommand{\BY}[0]{\mathbf{Y}}
\newcommand{\BZ}[0]{\mathbf{Z}}

\newcommand{\Bzero}[0]{\mathbf{0}}
\newcommand{\Btheta}[0]{\boldsymbol{\theta}}
\newcommand{\Btau}[0]{\boldsymbol{\tau}}
\newcommand{\Bomega}[0]{\boldsymbol{\omega}}

%
% shorthand for unit vectors
%
\newcommand{\acap}[0]{\hat{\Ba}}
\newcommand{\bcap}[0]{\hat{\Bb}}
\newcommand{\ccap}[0]{\hat{\Bc}}
\newcommand{\dcap}[0]{\hat{\Bd}}
\newcommand{\ecap}[0]{\hat{\Be}}
\newcommand{\fcap}[0]{\hat{\Bf}}
\newcommand{\gcap}[0]{\hat{\Bg}}
\newcommand{\hcap}[0]{\hat{\Bh}}
\newcommand{\icap}[0]{\hat{\Bi}}
\newcommand{\jcap}[0]{\hat{\Bj}}
\newcommand{\kcap}[0]{\hat{\Bk}}
\newcommand{\lcap}[0]{\hat{\Bl}}
\newcommand{\mcap}[0]{\hat{\Bm}}
\newcommand{\ncap}[0]{\hat{\Bn}}
\newcommand{\ocap}[0]{\hat{\Bo}}
\newcommand{\pcap}[0]{\hat{\Bp}}
\newcommand{\qcap}[0]{\hat{\Bq}}
\newcommand{\rcap}[0]{\hat{\Br}}
\newcommand{\scap}[0]{\hat{\Bs}}
\newcommand{\tcap}[0]{\hat{\Bt}}
\newcommand{\ucap}[0]{\hat{\Bu}}
\newcommand{\vcap}[0]{\hat{\Bv}}
\newcommand{\wcap}[0]{\hat{\Bw}}
\newcommand{\xcap}[0]{\hat{\Bx}}
\newcommand{\ycap}[0]{\hat{\By}}
\newcommand{\zcap}[0]{\hat{\Bz}}
\newcommand{\thetacap}[0]{\hat{\Btheta}}

%
% to write R^n and C^n in a distinguishable fashion.  Perhaps change this
% to the double lined characters upon figuring out how to do so.
%
\newcommand{\C}[1]{$\mathbb{C}^{#1}$}
\newcommand{\R}[1]{$\mathbb{R}^{#1}$}

%
% various generally useful helpers
%

% derivative of #1 wrt. #2:
\newcommand{\D}[2] {\frac {d#2} {d#1}}

\newcommand{\inv}[1]{\frac{1}{#1}}
\newcommand{\cross}[0]{\times}

\newcommand{\abs}[1]{\lvert{#1}\rvert}
\newcommand{\norm}[1]{\lVert{#1}\rVert}
\newcommand{\innerprod}[2]{\langle{#1}, {#2}\rangle}
\newcommand{\dotprod}[2]{{#1} \cdot {#2}}
\newcommand{\bdotprod}[2]{\left({#1} \cdot {#2}\right)}
\newcommand{\crossprod}[2]{{#1} \cross {#2}}
\newcommand{\tripleprod}[3]{\dotprod{\left(\crossprod{#1}{#2}\right)}{#3}}

\DeclareMathOperator{\Proj}{Proj}
\DeclareMathOperator{\Span}{span}
\DeclareMathOperator{\Sgn}{sgn}
\DeclareMathOperator{\Area}{Area}
\DeclareMathOperator{\Volume}{Volume}

%
% A few miscellaneous things specific to this document
%
\newcommand{\crossop}[1]{\crossprod{#1}{}}

% R2 vector.
\newcommand{\VectorTwo}[2]{
\begin{bmatrix}
 {#1} \\
 {#2}
\end{bmatrix}
}

\newcommand{\VectorN}[1]{
\begin{bmatrix}
{#1}_1 \\
{#1}_2 \\
\vdots \\
{#1}_N \\
\end{bmatrix}
}

\newcommand{\DETuvij}[4]{
\begin{vmatrix}
 {#1}_{#3} & {#1}_{#4} \\
 {#2}_{#3} & {#2}_{#4}
\end{vmatrix}
}

\newcommand{\DETuvwijk}[6]{
\begin{vmatrix}
 {#1}_{#4} & {#1}_{#5} & {#1}_{#6} \\
 {#2}_{#4} & {#2}_{#5} & {#2}_{#6} \\
 {#3}_{#4} & {#3}_{#5} & {#3}_{#6}
\end{vmatrix}
}

\newcommand{\DETuvwxijkl}[8]{
\begin{vmatrix}
 {#1}_{#5} & {#1}_{#6} & {#1}_{#7} & {#1}_{#8} \\
 {#2}_{#5} & {#2}_{#6} & {#2}_{#7} & {#2}_{#8} \\
 {#3}_{#5} & {#3}_{#6} & {#3}_{#7} & {#3}_{#8} \\
 {#4}_{#5} & {#4}_{#6} & {#4}_{#7} & {#4}_{#8} \\
\end{vmatrix}
}

%\newcommand{\DETuvwxyijklm}[10]{
%\begin{vmatrix}
% {#1}_{#6} & {#1}_{#7} & {#1}_{#8} & {#1}_{#9} & {#1}_{#10} \\
% {#2}_{#6} & {#2}_{#7} & {#2}_{#8} & {#2}_{#9} & {#2}_{#10} \\
% {#3}_{#6} & {#3}_{#7} & {#3}_{#8} & {#3}_{#9} & {#3}_{#10} \\
% {#4}_{#6} & {#4}_{#7} & {#4}_{#8} & {#4}_{#9} & {#4}_{#10} \\
% {#5}_{#6} & {#5}_{#7} & {#5}_{#8} & {#5}_{#9} & {#5}_{#10}
%\end{vmatrix}
%}

% R3 vector.
\newcommand{\VectorThree}[3]{
\begin{bmatrix}
 {#1} \\
 {#2} \\
 {#3}
\end{bmatrix}
}


%%<misc>
%
\newcommand{\Abs}[1]{{\left\lvert{#1}\right\rvert}}
\newcommand{\spacegrad}[0]{\boldsymbol{\nabla}}
\newcommand{\grad}[0]{\nabla}
\newcommand{\LL}[0]{\mathcal{L}}

% == \partial_{#1} {#2}
\newcommand{\PD}[2]{\frac{\partial {#2}}{\partial {#1}}}
% inline variant
\newcommand{\PDi}[2]{{\partial {#2}}/{\partial {#1}}}

\newcommand{\PDD}[3]{\frac{\partial^2 {#3}}{\partial {#1}\partial {#2}}}
%\newcommand{\PDd}[2]{\frac{\partial^2 {#2}}{{\partial{#1}}^2}}
\newcommand{\PDsq}[2]{\frac{\partial^2 {#2}}{(\partial {#1})^2}}

\newcommand{\Partial}[2]{\frac{\partial {#1}}{\partial {#2}}}
\DeclareMathOperator{\RejName}{Rej}
\newcommand{\Rej}[2]{\RejName_{#1}\left( {#2} \right)}
\newcommand{\Rm}[1]{\mathbb{R}^{#1}}
\newcommand{\Cm}[1]{\mathbb{C}^{#1}}
\newcommand{\conj}[0]{{*}}

%</misc>

% <grade selection>
%
\newcommand{\gpgrade}[2] {{\left\langle{{#1}}\right\rangle}_{#2}}

\newcommand{\gpgradezero}[1] {\gpgrade{#1}{}}
%\newcommand{\gpscalargrade}[1] {{\left\langle{{#1}}\right\rangle}}
%\newcommand{\gpgradezero}[1] {\gpgrade{#1}{0}}

%\newcommand{\gpgradeone}[1] {{\left\langle{{#1}}\right\rangle}_{1}}
\newcommand{\gpgradeone}[1] {\gpgrade{#1}{1}}

\newcommand{\gpgradetwo}[1] {\gpgrade{#1}{2}}
\newcommand{\gpgradethree}[1] {\gpgrade{#1}{3}}
\newcommand{\gpgradefour}[1] {\gpgrade{#1}{4}}
%
% </grade selection>



\newcommand{\adot}[0]{{\dot{a}}}
\newcommand{\bdot}[0]{{\dot{b}}}
% taken for centered dot:
%\newcommand{\cdot}[0]{{\dot{c}}}
%\newcommand{\ddot}[0]{{\dot{d}}}
\newcommand{\edot}[0]{{\dot{e}}}
\newcommand{\fdot}[0]{{\dot{f}}}
\newcommand{\gdot}[0]{{\dot{g}}}
\newcommand{\hdot}[0]{{\dot{h}}}
\newcommand{\idot}[0]{{\dot{i}}}
\newcommand{\jdot}[0]{{\dot{j}}}
\newcommand{\kdot}[0]{{\dot{k}}}
\newcommand{\ldot}[0]{{\dot{l}}}
\newcommand{\mdot}[0]{{\dot{m}}}
\newcommand{\ndot}[0]{{\dot{n}}}
%\newcommand{\odot}[0]{{\dot{o}}}
\newcommand{\pdot}[0]{{\dot{p}}}
\newcommand{\qdot}[0]{{\dot{q}}}
\newcommand{\rdot}[0]{{\dot{r}}}
\newcommand{\sdot}[0]{{\dot{s}}}
\newcommand{\tdot}[0]{{\dot{t}}}
\newcommand{\udot}[0]{{\dot{u}}}
\newcommand{\vdot}[0]{{\dot{v}}}
\newcommand{\wdot}[0]{{\dot{w}}}
\newcommand{\xdot}[0]{{\dot{x}}}
\newcommand{\ydot}[0]{{\dot{y}}}
\newcommand{\zdot}[0]{{\dot{z}}}
\newcommand{\addot}[0]{{\ddot{a}}}
\newcommand{\bddot}[0]{{\ddot{b}}}
\newcommand{\cddot}[0]{{\ddot{c}}}
%\newcommand{\dddot}[0]{{\ddot{d}}}
\newcommand{\eddot}[0]{{\ddot{e}}}
\newcommand{\fddot}[0]{{\ddot{f}}}
\newcommand{\gddot}[0]{{\ddot{g}}}
\newcommand{\hddot}[0]{{\ddot{h}}}
\newcommand{\iddot}[0]{{\ddot{i}}}
\newcommand{\jddot}[0]{{\ddot{j}}}
\newcommand{\kddot}[0]{{\ddot{k}}}
\newcommand{\lddot}[0]{{\ddot{l}}}
\newcommand{\mddot}[0]{{\ddot{m}}}
\newcommand{\nddot}[0]{{\ddot{n}}}
\newcommand{\oddot}[0]{{\ddot{o}}}
\newcommand{\pddot}[0]{{\ddot{p}}}
\newcommand{\qddot}[0]{{\ddot{q}}}
\newcommand{\rddot}[0]{{\ddot{r}}}
\newcommand{\sddot}[0]{{\ddot{s}}}
\newcommand{\tddot}[0]{{\ddot{t}}}
\newcommand{\uddot}[0]{{\ddot{u}}}
\newcommand{\vddot}[0]{{\ddot{v}}}
\newcommand{\wddot}[0]{{\ddot{w}}}
\newcommand{\xddot}[0]{{\ddot{x}}}
\newcommand{\yddot}[0]{{\ddot{y}}}
\newcommand{\zddot}[0]{{\ddot{z}}}

%<bold and dot greek symbols>
%

\newcommand{\Deltadot}[0]{{\dot{\Delta}}}
\newcommand{\Gammadot}[0]{{\dot{\Gamma}}}
\newcommand{\Lambdadot}[0]{{\dot{\Lambda}}}
\newcommand{\Omegadot}[0]{{\dot{\Omega}}}
\newcommand{\Phidot}[0]{{\dot{\Phi}}}
\newcommand{\Pidot}[0]{{\dot{\Pi}}}
\newcommand{\Psidot}[0]{{\dot{\Psi}}}
\newcommand{\Sigmadot}[0]{{\dot{\Sigma}}}
\newcommand{\Thetadot}[0]{{\dot{\Theta}}}
\newcommand{\Upsilondot}[0]{{\dot{\Upsilon}}}
\newcommand{\Xidot}[0]{{\dot{\Xi}}}
\newcommand{\alphadot}[0]{{\dot{\alpha}}}
\newcommand{\betadot}[0]{{\dot{\beta}}}
\newcommand{\chidot}[0]{{\dot{\chi}}}
\newcommand{\deltadot}[0]{{\dot{\delta}}}
\newcommand{\epsilondot}[0]{{\dot{\epsilon}}}
\newcommand{\etadot}[0]{{\dot{\eta}}}
\newcommand{\gammadot}[0]{{\dot{\gamma}}}
\newcommand{\kappadot}[0]{{\dot{\kappa}}}
\newcommand{\lambdadot}[0]{{\dot{\lambda}}}
\newcommand{\mudot}[0]{{\dot{\mu}}}
\newcommand{\nudot}[0]{{\dot{\nu}}}
\newcommand{\omegadot}[0]{{\dot{\omega}}}
\newcommand{\phidot}[0]{{\dot{\phi}}}
\newcommand{\pidot}[0]{{\dot{\pi}}}
\newcommand{\psidot}[0]{{\dot{\psi}}}
\newcommand{\rhodot}[0]{{\dot{\rho}}}
\newcommand{\sigmadot}[0]{{\dot{\sigma}}}
\newcommand{\taudot}[0]{{\dot{\tau}}}
\newcommand{\thetadot}[0]{{\dot{\theta}}}
\newcommand{\upsilondot}[0]{{\dot{\upsilon}}}
\newcommand{\varepsilondot}[0]{{\dot{\varepsilon}}}
\newcommand{\varphidot}[0]{{\dot{\varphi}}}
\newcommand{\varpidot}[0]{{\dot{\varpi}}}
\newcommand{\varrhodot}[0]{{\dot{\varrho}}}
\newcommand{\varsigmadot}[0]{{\dot{\varsigma}}}
\newcommand{\varthetadot}[0]{{\dot{\vartheta}}}
\newcommand{\xidot}[0]{{\dot{\xi}}}
\newcommand{\zetadot}[0]{{\dot{\zeta}}}

\newcommand{\Deltaddot}[0]{{\ddot{\Delta}}}
\newcommand{\Gammaddot}[0]{{\ddot{\Gamma}}}
\newcommand{\Lambdaddot}[0]{{\ddot{\Lambda}}}
\newcommand{\Omegaddot}[0]{{\ddot{\Omega}}}
\newcommand{\Phiddot}[0]{{\ddot{\Phi}}}
\newcommand{\Piddot}[0]{{\ddot{\Pi}}}
\newcommand{\Psiddot}[0]{{\ddot{\Psi}}}
\newcommand{\Sigmaddot}[0]{{\ddot{\Sigma}}}
\newcommand{\Thetaddot}[0]{{\ddot{\Theta}}}
\newcommand{\Upsilonddot}[0]{{\ddot{\Upsilon}}}
\newcommand{\Xiddot}[0]{{\ddot{\Xi}}}
\newcommand{\alphaddot}[0]{{\ddot{\alpha}}}
\newcommand{\betaddot}[0]{{\ddot{\beta}}}
\newcommand{\chiddot}[0]{{\ddot{\chi}}}
\newcommand{\deltaddot}[0]{{\ddot{\delta}}}
\newcommand{\epsilonddot}[0]{{\ddot{\epsilon}}}
\newcommand{\etaddot}[0]{{\ddot{\eta}}}
\newcommand{\gammaddot}[0]{{\ddot{\gamma}}}
\newcommand{\kappaddot}[0]{{\ddot{\kappa}}}
\newcommand{\lambdaddot}[0]{{\ddot{\lambda}}}
\newcommand{\muddot}[0]{{\ddot{\mu}}}
\newcommand{\nuddot}[0]{{\ddot{\nu}}}
\newcommand{\omegaddot}[0]{{\ddot{\omega}}}
\newcommand{\phiddot}[0]{{\ddot{\phi}}}
\newcommand{\piddot}[0]{{\ddot{\pi}}}
\newcommand{\psiddot}[0]{{\ddot{\psi}}}
\newcommand{\rhoddot}[0]{{\ddot{\rho}}}
\newcommand{\sigmaddot}[0]{{\ddot{\sigma}}}
\newcommand{\tauddot}[0]{{\ddot{\tau}}}
\newcommand{\thetaddot}[0]{{\ddot{\theta}}}
\newcommand{\upsilonddot}[0]{{\ddot{\upsilon}}}
\newcommand{\varepsilonddot}[0]{{\ddot{\varepsilon}}}
\newcommand{\varphiddot}[0]{{\ddot{\varphi}}}
\newcommand{\varpiddot}[0]{{\ddot{\varpi}}}
\newcommand{\varrhoddot}[0]{{\ddot{\varrho}}}
\newcommand{\varsigmaddot}[0]{{\ddot{\varsigma}}}
\newcommand{\varthetaddot}[0]{{\ddot{\vartheta}}}
\newcommand{\xiddot}[0]{{\ddot{\xi}}}
\newcommand{\zetaddot}[0]{{\ddot{\zeta}}}

\newcommand{\BDelta}[0]{\boldsymbol{\Delta}}
\newcommand{\BGamma}[0]{\boldsymbol{\Gamma}}
\newcommand{\BLambda}[0]{\boldsymbol{\Lambda}}
\newcommand{\BOmega}[0]{\boldsymbol{\Omega}}
\newcommand{\BPhi}[0]{\boldsymbol{\Phi}}
\newcommand{\BPi}[0]{\boldsymbol{\Pi}}
\newcommand{\BPsi}[0]{\boldsymbol{\Psi}}
\newcommand{\BSigma}[0]{\boldsymbol{\Sigma}}
\newcommand{\BTheta}[0]{\boldsymbol{\Theta}}
\newcommand{\BUpsilon}[0]{\boldsymbol{\Upsilon}}
\newcommand{\BXi}[0]{\boldsymbol{\Xi}}
\newcommand{\Balpha}[0]{\boldsymbol{\alpha}}
\newcommand{\Bbeta}[0]{\boldsymbol{\beta}}
\newcommand{\Bchi}[0]{\boldsymbol{\chi}}
\newcommand{\Bdelta}[0]{\boldsymbol{\delta}}
\newcommand{\Bepsilon}[0]{\boldsymbol{\epsilon}}
\newcommand{\Beta}[0]{\boldsymbol{\eta}}
\newcommand{\Bgamma}[0]{\boldsymbol{\gamma}}
\newcommand{\Bkappa}[0]{\boldsymbol{\kappa}}
\newcommand{\Blambda}[0]{\boldsymbol{\lambda}}
\newcommand{\Bmu}[0]{\boldsymbol{\mu}}
\newcommand{\Bnu}[0]{\boldsymbol{\nu}}
%\newcommand{\Bomega}[0]{\boldsymbol{\omega}}
\newcommand{\Bphi}[0]{\boldsymbol{\phi}}
\newcommand{\Bpi}[0]{\boldsymbol{\pi}}
\newcommand{\Bpsi}[0]{\boldsymbol{\psi}}
\newcommand{\Brho}[0]{\boldsymbol{\rho}}
\newcommand{\Bsigma}[0]{\boldsymbol{\sigma}}
%\newcommand{\Btau}[0]{\boldsymbol{\tau}}
%\newcommand{\Btheta}[0]{\boldsymbol{\theta}}
\newcommand{\Bupsilon}[0]{\boldsymbol{\upsilon}}
\newcommand{\Bvarepsilon}[0]{\boldsymbol{\varepsilon}}
\newcommand{\Bvarphi}[0]{\boldsymbol{\varphi}}
\newcommand{\Bvarpi}[0]{\boldsymbol{\varpi}}
\newcommand{\Bvarrho}[0]{\boldsymbol{\varrho}}
\newcommand{\Bvarsigma}[0]{\boldsymbol{\varsigma}}
\newcommand{\Bvartheta}[0]{\boldsymbol{\vartheta}}
\newcommand{\Bxi}[0]{\boldsymbol{\xi}}
\newcommand{\Bzeta}[0]{\boldsymbol{\zeta}}
%
%</bold and dot greek symbols>
%<infrequent>
%
%\newcommand{\AreaOp}[1]{\AName_{#1}}
%\newcommand{\Babs}[0]{\abs{\BB}}
%\newcommand{\Bcap}[0]{\hat{\BB}}
%\newcommand{\BrPrimeRej}[0]{\rcap(\rcap \wedge \Br')}
%\newcommand{\CA}[0]{\mathcal{A}}
%\newcommand{\Cos}[1]{\cos{\left({#1}\right)}}
%\newcommand{\Det}[1] {\abs{#1}}
%\newcommand{\Dsq}[2] {\frac {\partial^2 {#1}} {\partial {#2}^2}}
%\newcommand{\Exp}[1]{\exp{\left({#1}\right)}}
%\newcommand{\Norm}[1]{\left\lVert{#1}\right\rVert}
%\newcommand{\Sin}[1]{\sin{\left({#1}\right)}}
%\newcommand{\T}[0]{\text{T}}
%\newcommand{\VolumeOp}[1]{\VName_{#1}}
%\newcommand{\agrad}[0]{\Ba \cdot \nabla}
%\newcommand{\alphacap}[0]{\hat{\boldsymbol{\alpha}}}
%\newcommand{\Fcap}[0]{\hat{\BF}}
%\newcommand{\bithree}[0]{{\Bi}_3}
%\newcommand{\bxa}[0]{\Bx\Ba}
%\newcommand{\coordvec}[2]{
%\newcommand{\costheta}[0]{\acap \cdot \xcap}
%\newcommand{\ddt}[1]{\ddot{#1}}
%\newcommand{\ddu}[1] {\frac {d{#1}} {du}}
%\newcommand{\dsqxj}[2] {\frac {\partial^2 {#1}} {\partial {x_{#2}}^2}}
%\newcommand{\dtheta}[1]{\frac{d {#1}}{d \theta}}
%\newcommand{\dt}[1]{\dot{#1}}
%\newcommand{\dt}[1]{\frac{d {#1}}{dt}}
%\newcommand{\dxj}[2] {\frac {\partial {#1}} {\partial {x_{#2}}}}
%\newcommand{\halfPhi}[0]{\frac{\phi}{2}}
%\newcommand{\half}[0]{\inv{2}}
%\newcommand{\inv}[1]{\frac{1}{#1}}
%\newcommand{\laplacian}[0]{\nabla^2}
%\newcommand{\matrixoftx}[3]{
%\newcommand{\nrrp}[0]{\norm{\rcap \wedge \Br'}}
%\newcommand{\oiint}{\bigcirc \hspace{-1.4em} \int \hspace{-.8em} \int}
%\newcommand{\transpose}[1]{{#1}^{\text{T}}}
%\newcommand{\transpose}[1]{{{#1}^{\TextTranspose}}}
%\newcommand{\transpose}[1]{{{#1}^{\text{T}}}}
%\newcommand{\barA}[0]{\bar{A}}
%\newcommand{\qbar}[0]{\bar{q}}
%\newcommand{\qdotbar}[0]{\dot{\bar{q}}}
%
%</infrequent>





%\usepackage[bookmarks=true]{hyperref}

%\usepackage{color,cite,graphicx}
   % use colour in the document, put your citations as [1-4]
   % rather than [1,2,3,4] (it looks nicer, and the extended LaTeX2e
   % graphics package.
%\usepackage{latexsym,amssymb,epsf} % do not remember if these are
   % needed, but their inclusion can not do any damage


\chapter{REMOVED FROM electric field energy}
\label{chap:faradayLagrangian}
%\author{Peeter Joot \quad peeterjoot@protonmail.com}
\date{ Feb dd, 2009.  \(RCSfile: faradayLagrangian.tex,v \) Last \(Revision: 1.8 \) \(Date: 2009/06/14 23:51:45 \) }

%\begin{document}

%\maketitle{}

%\tableofcontents

\section{}

% REMOVED FROM electric_field_energy.ltx

We have also seen in various exercises that the Lorentz force could be obtained from the action

\begin{equation}\label{eqn:faradayLagrangian:20}
\begin{aligned}
S = \int \left(\inv{2} m \left(\frac{dx}{d\tau}\right)^2 + q A \cdot \frac{dx}{c d\tau}\right) d\tau
\end{aligned}
\end{equation}

and it appeared that this plus the Maxwell action

\begin{equation}\label{eqn:faradayLagrangian:40}
\begin{aligned}
S = \int \left(- \frac{c \epsilon_0}{2} (\grad \wedge A)^2 + J \cdot A \right) d^4 x
\end{aligned}
\end{equation}

as covered in \citep{classicalmechanics:PJMaxwellLagrangian} was required.  One action for the field
equation and one for the interaction equation.  Seeing the Lorentz force show up like this out of nowhere with only
manipulation of the Maxwell equation suggests that the Lorentz force or its associated Lagrangian is not actually that
fundamental.  We have one equation at the root of both (and that equation is probably quite close to the Maxwell field Lagrangian), probably
with the proper velocity \(mv^2/2\) term added in somehow.


%\bibliographystyle{plainnat}
%\bibliography{myrefs}

%\end{document}


\part{Correspondance}
%
% Copyright � 2012 Peeter Joot.  All Rights Reserved.
% Licenced as described in the file LICENSE under the root directory of this GIT repository.
%

% 
% 
%\documentclass{article}

%\usepackage{amsmath}
\usepackage{mathpazo}

%
% shorthand for bold symbols, convenient for vectors and matrices
%
\newcommand{\Ba}[0]{\mathbf{a}}
\newcommand{\Bb}[0]{\mathbf{b}}
\newcommand{\Bc}[0]{\mathbf{c}}
\newcommand{\Bd}[0]{\mathbf{d}}
\newcommand{\Be}[0]{\mathbf{e}}
\newcommand{\Bf}[0]{\mathbf{f}}
\newcommand{\Bg}[0]{\mathbf{g}}
\newcommand{\Bh}[0]{\mathbf{h}}
\newcommand{\Bi}[0]{\mathbf{i}}
\newcommand{\Bj}[0]{\mathbf{j}}
\newcommand{\Bk}[0]{\mathbf{k}}
\newcommand{\Bl}[0]{\mathbf{l}}
\newcommand{\Bm}[0]{\mathbf{m}}
\newcommand{\Bn}[0]{\mathbf{n}}
\newcommand{\Bo}[0]{\mathbf{o}}
\newcommand{\Bp}[0]{\mathbf{p}}
\newcommand{\Bq}[0]{\mathbf{q}}
\newcommand{\Br}[0]{\mathbf{r}}
\newcommand{\Bs}[0]{\mathbf{s}}
\newcommand{\Bt}[0]{\mathbf{t}}
\newcommand{\Bu}[0]{\mathbf{u}}
\newcommand{\Bv}[0]{\mathbf{v}}
\newcommand{\Bw}[0]{\mathbf{w}}
\newcommand{\Bx}[0]{\mathbf{x}}
\newcommand{\By}[0]{\mathbf{y}}
\newcommand{\Bz}[0]{\mathbf{z}}
\newcommand{\BA}[0]{\mathbf{A}}
\newcommand{\BB}[0]{\mathbf{B}}
\newcommand{\BC}[0]{\mathbf{C}}
\newcommand{\BD}[0]{\mathbf{D}}
\newcommand{\BE}[0]{\mathbf{E}}
\newcommand{\BF}[0]{\mathbf{F}}
\newcommand{\BG}[0]{\mathbf{G}}
\newcommand{\BH}[0]{\mathbf{H}}
\newcommand{\BI}[0]{\mathbf{I}}
\newcommand{\BJ}[0]{\mathbf{J}}
\newcommand{\BK}[0]{\mathbf{K}}
\newcommand{\BL}[0]{\mathbf{L}}
\newcommand{\BM}[0]{\mathbf{M}}
\newcommand{\BN}[0]{\mathbf{N}}
\newcommand{\BO}[0]{\mathbf{O}}
\newcommand{\BP}[0]{\mathbf{P}}
\newcommand{\BQ}[0]{\mathbf{Q}}
\newcommand{\BR}[0]{\mathbf{R}}
\newcommand{\BS}[0]{\mathbf{S}}
\newcommand{\BT}[0]{\mathbf{T}}
\newcommand{\BU}[0]{\mathbf{U}}
\newcommand{\BV}[0]{\mathbf{V}}
\newcommand{\BW}[0]{\mathbf{W}}
\newcommand{\BX}[0]{\mathbf{X}}
\newcommand{\BY}[0]{\mathbf{Y}}
\newcommand{\BZ}[0]{\mathbf{Z}}

\newcommand{\Bzero}[0]{\mathbf{0}}
\newcommand{\Btheta}[0]{\boldsymbol{\theta}}
\newcommand{\Btau}[0]{\boldsymbol{\tau}}
\newcommand{\Bomega}[0]{\boldsymbol{\omega}}

%
% shorthand for unit vectors
%
\newcommand{\acap}[0]{\hat{\Ba}}
\newcommand{\bcap}[0]{\hat{\Bb}}
\newcommand{\ccap}[0]{\hat{\Bc}}
\newcommand{\dcap}[0]{\hat{\Bd}}
\newcommand{\ecap}[0]{\hat{\Be}}
\newcommand{\fcap}[0]{\hat{\Bf}}
\newcommand{\gcap}[0]{\hat{\Bg}}
\newcommand{\hcap}[0]{\hat{\Bh}}
\newcommand{\icap}[0]{\hat{\Bi}}
\newcommand{\jcap}[0]{\hat{\Bj}}
\newcommand{\kcap}[0]{\hat{\Bk}}
\newcommand{\lcap}[0]{\hat{\Bl}}
\newcommand{\mcap}[0]{\hat{\Bm}}
\newcommand{\ncap}[0]{\hat{\Bn}}
\newcommand{\ocap}[0]{\hat{\Bo}}
\newcommand{\pcap}[0]{\hat{\Bp}}
\newcommand{\qcap}[0]{\hat{\Bq}}
\newcommand{\rcap}[0]{\hat{\Br}}
\newcommand{\scap}[0]{\hat{\Bs}}
\newcommand{\tcap}[0]{\hat{\Bt}}
\newcommand{\ucap}[0]{\hat{\Bu}}
\newcommand{\vcap}[0]{\hat{\Bv}}
\newcommand{\wcap}[0]{\hat{\Bw}}
\newcommand{\xcap}[0]{\hat{\Bx}}
\newcommand{\ycap}[0]{\hat{\By}}
\newcommand{\zcap}[0]{\hat{\Bz}}
\newcommand{\thetacap}[0]{\hat{\Btheta}}

%
% to write R^n and C^n in a distinguishable fashion.  Perhaps change this
% to the double lined characters upon figuring out how to do so.
%
\newcommand{\C}[1]{$\mathbb{C}^{#1}$}
\newcommand{\R}[1]{$\mathbb{R}^{#1}$}

%
% various generally useful helpers
%

% derivative of #1 wrt. #2:
\newcommand{\D}[2] {\frac {d#2} {d#1}}

\newcommand{\inv}[1]{\frac{1}{#1}}
\newcommand{\cross}[0]{\times}

\newcommand{\abs}[1]{\lvert{#1}\rvert}
\newcommand{\norm}[1]{\lVert{#1}\rVert}
\newcommand{\innerprod}[2]{\langle{#1}, {#2}\rangle}
\newcommand{\dotprod}[2]{{#1} \cdot {#2}}
\newcommand{\bdotprod}[2]{\left({#1} \cdot {#2}\right)}
\newcommand{\crossprod}[2]{{#1} \cross {#2}}
\newcommand{\tripleprod}[3]{\dotprod{\left(\crossprod{#1}{#2}\right)}{#3}}

\DeclareMathOperator{\Proj}{Proj}
\DeclareMathOperator{\Span}{span}
\DeclareMathOperator{\Sgn}{sgn}
\DeclareMathOperator{\Area}{Area}
\DeclareMathOperator{\Volume}{Volume}

%
% A few miscellaneous things specific to this document
%
\newcommand{\crossop}[1]{\crossprod{#1}{}}

% R2 vector.
\newcommand{\VectorTwo}[2]{
\begin{bmatrix}
 {#1} \\
 {#2}
\end{bmatrix}
}

\newcommand{\VectorN}[1]{
\begin{bmatrix}
{#1}_1 \\
{#1}_2 \\
\vdots \\
{#1}_N \\
\end{bmatrix}
}

\newcommand{\DETuvij}[4]{
\begin{vmatrix}
 {#1}_{#3} & {#1}_{#4} \\
 {#2}_{#3} & {#2}_{#4}
\end{vmatrix}
}

\newcommand{\DETuvwijk}[6]{
\begin{vmatrix}
 {#1}_{#4} & {#1}_{#5} & {#1}_{#6} \\
 {#2}_{#4} & {#2}_{#5} & {#2}_{#6} \\
 {#3}_{#4} & {#3}_{#5} & {#3}_{#6}
\end{vmatrix}
}

\newcommand{\DETuvwxijkl}[8]{
\begin{vmatrix}
 {#1}_{#5} & {#1}_{#6} & {#1}_{#7} & {#1}_{#8} \\
 {#2}_{#5} & {#2}_{#6} & {#2}_{#7} & {#2}_{#8} \\
 {#3}_{#5} & {#3}_{#6} & {#3}_{#7} & {#3}_{#8} \\
 {#4}_{#5} & {#4}_{#6} & {#4}_{#7} & {#4}_{#8} \\
\end{vmatrix}
}

%\newcommand{\DETuvwxyijklm}[10]{
%\begin{vmatrix}
% {#1}_{#6} & {#1}_{#7} & {#1}_{#8} & {#1}_{#9} & {#1}_{#10} \\
% {#2}_{#6} & {#2}_{#7} & {#2}_{#8} & {#2}_{#9} & {#2}_{#10} \\
% {#3}_{#6} & {#3}_{#7} & {#3}_{#8} & {#3}_{#9} & {#3}_{#10} \\
% {#4}_{#6} & {#4}_{#7} & {#4}_{#8} & {#4}_{#9} & {#4}_{#10} \\
% {#5}_{#6} & {#5}_{#7} & {#5}_{#8} & {#5}_{#9} & {#5}_{#10}
%\end{vmatrix}
%}

% R3 vector.
\newcommand{\VectorThree}[3]{
\begin{bmatrix}
 {#1} \\
 {#2} \\
 {#3}
\end{bmatrix}
}


%%<misc>
%
\newcommand{\Abs}[1]{{\left\lvert{#1}\right\rvert}}
\newcommand{\spacegrad}[0]{\boldsymbol{\nabla}}
\newcommand{\grad}[0]{\nabla}
\newcommand{\LL}[0]{\mathcal{L}}

% == \partial_{#1} {#2}
\newcommand{\PD}[2]{\frac{\partial {#2}}{\partial {#1}}}
% inline variant
\newcommand{\PDi}[2]{{\partial {#2}}/{\partial {#1}}}

\newcommand{\PDD}[3]{\frac{\partial^2 {#3}}{\partial {#1}\partial {#2}}}
%\newcommand{\PDd}[2]{\frac{\partial^2 {#2}}{{\partial{#1}}^2}}
\newcommand{\PDsq}[2]{\frac{\partial^2 {#2}}{(\partial {#1})^2}}

\newcommand{\Partial}[2]{\frac{\partial {#1}}{\partial {#2}}}
\DeclareMathOperator{\RejName}{Rej}
\newcommand{\Rej}[2]{\RejName_{#1}\left( {#2} \right)}
\newcommand{\Rm}[1]{\mathbb{R}^{#1}}
\newcommand{\Cm}[1]{\mathbb{C}^{#1}}
\newcommand{\conj}[0]{{*}}

%</misc>

% <grade selection>
%
\newcommand{\gpgrade}[2] {{\left\langle{{#1}}\right\rangle}_{#2}}

\newcommand{\gpgradezero}[1] {\gpgrade{#1}{}}
%\newcommand{\gpscalargrade}[1] {{\left\langle{{#1}}\right\rangle}}
%\newcommand{\gpgradezero}[1] {\gpgrade{#1}{0}}

%\newcommand{\gpgradeone}[1] {{\left\langle{{#1}}\right\rangle}_{1}}
\newcommand{\gpgradeone}[1] {\gpgrade{#1}{1}}

\newcommand{\gpgradetwo}[1] {\gpgrade{#1}{2}}
\newcommand{\gpgradethree}[1] {\gpgrade{#1}{3}}
\newcommand{\gpgradefour}[1] {\gpgrade{#1}{4}}
%
% </grade selection>



\newcommand{\adot}[0]{{\dot{a}}}
\newcommand{\bdot}[0]{{\dot{b}}}
% taken for centered dot:
%\newcommand{\cdot}[0]{{\dot{c}}}
%\newcommand{\ddot}[0]{{\dot{d}}}
\newcommand{\edot}[0]{{\dot{e}}}
\newcommand{\fdot}[0]{{\dot{f}}}
\newcommand{\gdot}[0]{{\dot{g}}}
\newcommand{\hdot}[0]{{\dot{h}}}
\newcommand{\idot}[0]{{\dot{i}}}
\newcommand{\jdot}[0]{{\dot{j}}}
\newcommand{\kdot}[0]{{\dot{k}}}
\newcommand{\ldot}[0]{{\dot{l}}}
\newcommand{\mdot}[0]{{\dot{m}}}
\newcommand{\ndot}[0]{{\dot{n}}}
%\newcommand{\odot}[0]{{\dot{o}}}
\newcommand{\pdot}[0]{{\dot{p}}}
\newcommand{\qdot}[0]{{\dot{q}}}
\newcommand{\rdot}[0]{{\dot{r}}}
\newcommand{\sdot}[0]{{\dot{s}}}
\newcommand{\tdot}[0]{{\dot{t}}}
\newcommand{\udot}[0]{{\dot{u}}}
\newcommand{\vdot}[0]{{\dot{v}}}
\newcommand{\wdot}[0]{{\dot{w}}}
\newcommand{\xdot}[0]{{\dot{x}}}
\newcommand{\ydot}[0]{{\dot{y}}}
\newcommand{\zdot}[0]{{\dot{z}}}
\newcommand{\addot}[0]{{\ddot{a}}}
\newcommand{\bddot}[0]{{\ddot{b}}}
\newcommand{\cddot}[0]{{\ddot{c}}}
%\newcommand{\dddot}[0]{{\ddot{d}}}
\newcommand{\eddot}[0]{{\ddot{e}}}
\newcommand{\fddot}[0]{{\ddot{f}}}
\newcommand{\gddot}[0]{{\ddot{g}}}
\newcommand{\hddot}[0]{{\ddot{h}}}
\newcommand{\iddot}[0]{{\ddot{i}}}
\newcommand{\jddot}[0]{{\ddot{j}}}
\newcommand{\kddot}[0]{{\ddot{k}}}
\newcommand{\lddot}[0]{{\ddot{l}}}
\newcommand{\mddot}[0]{{\ddot{m}}}
\newcommand{\nddot}[0]{{\ddot{n}}}
\newcommand{\oddot}[0]{{\ddot{o}}}
\newcommand{\pddot}[0]{{\ddot{p}}}
\newcommand{\qddot}[0]{{\ddot{q}}}
\newcommand{\rddot}[0]{{\ddot{r}}}
\newcommand{\sddot}[0]{{\ddot{s}}}
\newcommand{\tddot}[0]{{\ddot{t}}}
\newcommand{\uddot}[0]{{\ddot{u}}}
\newcommand{\vddot}[0]{{\ddot{v}}}
\newcommand{\wddot}[0]{{\ddot{w}}}
\newcommand{\xddot}[0]{{\ddot{x}}}
\newcommand{\yddot}[0]{{\ddot{y}}}
\newcommand{\zddot}[0]{{\ddot{z}}}

%<bold and dot greek symbols>
%

\newcommand{\Deltadot}[0]{{\dot{\Delta}}}
\newcommand{\Gammadot}[0]{{\dot{\Gamma}}}
\newcommand{\Lambdadot}[0]{{\dot{\Lambda}}}
\newcommand{\Omegadot}[0]{{\dot{\Omega}}}
\newcommand{\Phidot}[0]{{\dot{\Phi}}}
\newcommand{\Pidot}[0]{{\dot{\Pi}}}
\newcommand{\Psidot}[0]{{\dot{\Psi}}}
\newcommand{\Sigmadot}[0]{{\dot{\Sigma}}}
\newcommand{\Thetadot}[0]{{\dot{\Theta}}}
\newcommand{\Upsilondot}[0]{{\dot{\Upsilon}}}
\newcommand{\Xidot}[0]{{\dot{\Xi}}}
\newcommand{\alphadot}[0]{{\dot{\alpha}}}
\newcommand{\betadot}[0]{{\dot{\beta}}}
\newcommand{\chidot}[0]{{\dot{\chi}}}
\newcommand{\deltadot}[0]{{\dot{\delta}}}
\newcommand{\epsilondot}[0]{{\dot{\epsilon}}}
\newcommand{\etadot}[0]{{\dot{\eta}}}
\newcommand{\gammadot}[0]{{\dot{\gamma}}}
\newcommand{\kappadot}[0]{{\dot{\kappa}}}
\newcommand{\lambdadot}[0]{{\dot{\lambda}}}
\newcommand{\mudot}[0]{{\dot{\mu}}}
\newcommand{\nudot}[0]{{\dot{\nu}}}
\newcommand{\omegadot}[0]{{\dot{\omega}}}
\newcommand{\phidot}[0]{{\dot{\phi}}}
\newcommand{\pidot}[0]{{\dot{\pi}}}
\newcommand{\psidot}[0]{{\dot{\psi}}}
\newcommand{\rhodot}[0]{{\dot{\rho}}}
\newcommand{\sigmadot}[0]{{\dot{\sigma}}}
\newcommand{\taudot}[0]{{\dot{\tau}}}
\newcommand{\thetadot}[0]{{\dot{\theta}}}
\newcommand{\upsilondot}[0]{{\dot{\upsilon}}}
\newcommand{\varepsilondot}[0]{{\dot{\varepsilon}}}
\newcommand{\varphidot}[0]{{\dot{\varphi}}}
\newcommand{\varpidot}[0]{{\dot{\varpi}}}
\newcommand{\varrhodot}[0]{{\dot{\varrho}}}
\newcommand{\varsigmadot}[0]{{\dot{\varsigma}}}
\newcommand{\varthetadot}[0]{{\dot{\vartheta}}}
\newcommand{\xidot}[0]{{\dot{\xi}}}
\newcommand{\zetadot}[0]{{\dot{\zeta}}}

\newcommand{\Deltaddot}[0]{{\ddot{\Delta}}}
\newcommand{\Gammaddot}[0]{{\ddot{\Gamma}}}
\newcommand{\Lambdaddot}[0]{{\ddot{\Lambda}}}
\newcommand{\Omegaddot}[0]{{\ddot{\Omega}}}
\newcommand{\Phiddot}[0]{{\ddot{\Phi}}}
\newcommand{\Piddot}[0]{{\ddot{\Pi}}}
\newcommand{\Psiddot}[0]{{\ddot{\Psi}}}
\newcommand{\Sigmaddot}[0]{{\ddot{\Sigma}}}
\newcommand{\Thetaddot}[0]{{\ddot{\Theta}}}
\newcommand{\Upsilonddot}[0]{{\ddot{\Upsilon}}}
\newcommand{\Xiddot}[0]{{\ddot{\Xi}}}
\newcommand{\alphaddot}[0]{{\ddot{\alpha}}}
\newcommand{\betaddot}[0]{{\ddot{\beta}}}
\newcommand{\chiddot}[0]{{\ddot{\chi}}}
\newcommand{\deltaddot}[0]{{\ddot{\delta}}}
\newcommand{\epsilonddot}[0]{{\ddot{\epsilon}}}
\newcommand{\etaddot}[0]{{\ddot{\eta}}}
\newcommand{\gammaddot}[0]{{\ddot{\gamma}}}
\newcommand{\kappaddot}[0]{{\ddot{\kappa}}}
\newcommand{\lambdaddot}[0]{{\ddot{\lambda}}}
\newcommand{\muddot}[0]{{\ddot{\mu}}}
\newcommand{\nuddot}[0]{{\ddot{\nu}}}
\newcommand{\omegaddot}[0]{{\ddot{\omega}}}
\newcommand{\phiddot}[0]{{\ddot{\phi}}}
\newcommand{\piddot}[0]{{\ddot{\pi}}}
\newcommand{\psiddot}[0]{{\ddot{\psi}}}
\newcommand{\rhoddot}[0]{{\ddot{\rho}}}
\newcommand{\sigmaddot}[0]{{\ddot{\sigma}}}
\newcommand{\tauddot}[0]{{\ddot{\tau}}}
\newcommand{\thetaddot}[0]{{\ddot{\theta}}}
\newcommand{\upsilonddot}[0]{{\ddot{\upsilon}}}
\newcommand{\varepsilonddot}[0]{{\ddot{\varepsilon}}}
\newcommand{\varphiddot}[0]{{\ddot{\varphi}}}
\newcommand{\varpiddot}[0]{{\ddot{\varpi}}}
\newcommand{\varrhoddot}[0]{{\ddot{\varrho}}}
\newcommand{\varsigmaddot}[0]{{\ddot{\varsigma}}}
\newcommand{\varthetaddot}[0]{{\ddot{\vartheta}}}
\newcommand{\xiddot}[0]{{\ddot{\xi}}}
\newcommand{\zetaddot}[0]{{\ddot{\zeta}}}

\newcommand{\BDelta}[0]{\boldsymbol{\Delta}}
\newcommand{\BGamma}[0]{\boldsymbol{\Gamma}}
\newcommand{\BLambda}[0]{\boldsymbol{\Lambda}}
\newcommand{\BOmega}[0]{\boldsymbol{\Omega}}
\newcommand{\BPhi}[0]{\boldsymbol{\Phi}}
\newcommand{\BPi}[0]{\boldsymbol{\Pi}}
\newcommand{\BPsi}[0]{\boldsymbol{\Psi}}
\newcommand{\BSigma}[0]{\boldsymbol{\Sigma}}
\newcommand{\BTheta}[0]{\boldsymbol{\Theta}}
\newcommand{\BUpsilon}[0]{\boldsymbol{\Upsilon}}
\newcommand{\BXi}[0]{\boldsymbol{\Xi}}
\newcommand{\Balpha}[0]{\boldsymbol{\alpha}}
\newcommand{\Bbeta}[0]{\boldsymbol{\beta}}
\newcommand{\Bchi}[0]{\boldsymbol{\chi}}
\newcommand{\Bdelta}[0]{\boldsymbol{\delta}}
\newcommand{\Bepsilon}[0]{\boldsymbol{\epsilon}}
\newcommand{\Beta}[0]{\boldsymbol{\eta}}
\newcommand{\Bgamma}[0]{\boldsymbol{\gamma}}
\newcommand{\Bkappa}[0]{\boldsymbol{\kappa}}
\newcommand{\Blambda}[0]{\boldsymbol{\lambda}}
\newcommand{\Bmu}[0]{\boldsymbol{\mu}}
\newcommand{\Bnu}[0]{\boldsymbol{\nu}}
%\newcommand{\Bomega}[0]{\boldsymbol{\omega}}
\newcommand{\Bphi}[0]{\boldsymbol{\phi}}
\newcommand{\Bpi}[0]{\boldsymbol{\pi}}
\newcommand{\Bpsi}[0]{\boldsymbol{\psi}}
\newcommand{\Brho}[0]{\boldsymbol{\rho}}
\newcommand{\Bsigma}[0]{\boldsymbol{\sigma}}
%\newcommand{\Btau}[0]{\boldsymbol{\tau}}
%\newcommand{\Btheta}[0]{\boldsymbol{\theta}}
\newcommand{\Bupsilon}[0]{\boldsymbol{\upsilon}}
\newcommand{\Bvarepsilon}[0]{\boldsymbol{\varepsilon}}
\newcommand{\Bvarphi}[0]{\boldsymbol{\varphi}}
\newcommand{\Bvarpi}[0]{\boldsymbol{\varpi}}
\newcommand{\Bvarrho}[0]{\boldsymbol{\varrho}}
\newcommand{\Bvarsigma}[0]{\boldsymbol{\varsigma}}
\newcommand{\Bvartheta}[0]{\boldsymbol{\vartheta}}
\newcommand{\Bxi}[0]{\boldsymbol{\xi}}
\newcommand{\Bzeta}[0]{\boldsymbol{\zeta}}
%
%</bold and dot greek symbols>
%<infrequent>
%
%\newcommand{\AreaOp}[1]{\AName_{#1}}
%\newcommand{\Babs}[0]{\abs{\BB}}
%\newcommand{\Bcap}[0]{\hat{\BB}}
%\newcommand{\BrPrimeRej}[0]{\rcap(\rcap \wedge \Br')}
%\newcommand{\CA}[0]{\mathcal{A}}
%\newcommand{\Cos}[1]{\cos{\left({#1}\right)}}
%\newcommand{\Det}[1] {\abs{#1}}
%\newcommand{\Dsq}[2] {\frac {\partial^2 {#1}} {\partial {#2}^2}}
%\newcommand{\Exp}[1]{\exp{\left({#1}\right)}}
%\newcommand{\Norm}[1]{\left\lVert{#1}\right\rVert}
%\newcommand{\Sin}[1]{\sin{\left({#1}\right)}}
%\newcommand{\T}[0]{\text{T}}
%\newcommand{\VolumeOp}[1]{\VName_{#1}}
%\newcommand{\agrad}[0]{\Ba \cdot \nabla}
%\newcommand{\alphacap}[0]{\hat{\boldsymbol{\alpha}}}
%\newcommand{\Fcap}[0]{\hat{\BF}}
%\newcommand{\bithree}[0]{{\Bi}_3}
%\newcommand{\bxa}[0]{\Bx\Ba}
%\newcommand{\coordvec}[2]{
%\newcommand{\costheta}[0]{\acap \cdot \xcap}
%\newcommand{\ddt}[1]{\ddot{#1}}
%\newcommand{\ddu}[1] {\frac {d{#1}} {du}}
%\newcommand{\dsqxj}[2] {\frac {\partial^2 {#1}} {\partial {x_{#2}}^2}}
%\newcommand{\dtheta}[1]{\frac{d {#1}}{d \theta}}
%\newcommand{\dt}[1]{\dot{#1}}
%\newcommand{\dt}[1]{\frac{d {#1}}{dt}}
%\newcommand{\dxj}[2] {\frac {\partial {#1}} {\partial {x_{#2}}}}
%\newcommand{\halfPhi}[0]{\frac{\phi}{2}}
%\newcommand{\half}[0]{\inv{2}}
%\newcommand{\inv}[1]{\frac{1}{#1}}
%\newcommand{\laplacian}[0]{\nabla^2}
%\newcommand{\matrixoftx}[3]{
%\newcommand{\nrrp}[0]{\norm{\rcap \wedge \Br'}}
%\newcommand{\oiint}{\bigcirc \hspace{-1.4em} \int \hspace{-.8em} \int}
%\newcommand{\transpose}[1]{{#1}^{\text{T}}}
%\newcommand{\transpose}[1]{{{#1}^{\TextTranspose}}}
%\newcommand{\transpose}[1]{{{#1}^{\text{T}}}}
%\newcommand{\barA}[0]{\bar{A}}
%\newcommand{\qbar}[0]{\bar{q}}
%\newcommand{\qdotbar}[0]{\dot{\bar{q}}}
%
%</infrequent>





%\usepackage{listings}
%\usepackage{txfonts} % for ointctr... (also appears to make "prettier" \int and \sum's)
% makes \grad look funny though (almost like spacegrad, but narrower)
%\usepackage[bookmarks=true]{hyperref}

%\usepackage{color,cite,graphicx}
   % use colour in the document, put your citations as [1-4]
   % rather than [1,2,3,4] (it looks nicer, and the extended LaTeX2e
   % graphics package. 
%\usepackage{latexsym,amssymb,epsf} % do not remember if these are
   % needed, but their inclusion can not do any damage


\chapter{Voltage current, and resistance}
\label{chap:voltageCurrentResistance}
%\author{Peeter Joot \quad peeterjoot@protonmail.com }
\date{ May 3, 2009.  \(RCSfile: voltageCurrentResistance.tex,v \) Last \(Revision: 1.7 \) \(Date: 2009/06/14 23:51:45 \) }

%\begin{document}

%\maketitle{}
%\tableofcontents
%\section{}

Hi Dad,

By what you have said, I am guessing that you have encountered the V I R ``triangle'' where any one can be ``defined'' in terms of the other two.
These are not definitions, but are really
just a pictorial description of the following formula, which can be written in three different ways all logically identical

\begin{equation}\label{eqn:voltageCurrentResistance:20}
\begin{aligned}
V &= I R \\
I &= \frac{V}{R} \\
R &= \frac{V}{I} \\
\end{aligned}
\end{equation}

As you have said this is very cyclic, because none of these are definitions.

To make things worse, if you go looking for descriptions of what each of these are, what you find may very well not even be true.
Here is an example that I just found on the Internet, and it is not terribly different from descriptions that I had seen in various
``hobbyist electronics explained'' books.

\begin{lstlisting}
Voltage is the electrical force, or "pressure",
that causes current to flow in a circuit.
It is measured in VOLTS (V or E). 
Take a look at the diagram.  Voltage would be
the force that is pushing the water (electrons) forward.
\end{lstlisting}

Descriptions like this make it very hard to move from electronics hobbyist mode to physics mode.  Even an introductory
physics course (like my first grade 12 physics course in high school for example) will initially be confusing with this sort of description because voltage is not a force.  You have to unlearn this sort of stuff above before things start to make sense.

Mathematically, voltage times charge (the product of the two) is a measure of energy, not force.
Similarly, energy has the mathematical description as force times distance.  You can keep going with these and
eventually you will find that you need a few fundamental definitions to define everything else related to particle motion 
and electromagnetism.  These are

\begin{enumerate}
\item mass
\item distance
\item time
\item charge
\end{enumerate}

Part of the problem with the fact that misleading descriptions like the above exist, is that to do it
properly, there is a small hierarchy of terminology that is required.  You need a bit
of prerequisite nomenclature to get to writing a sensible description of voltage and friends, as in
the \(V = I R\) relationship.  Here is what you need 
before you can get to voltage

\begin{equation}\label{eqn:voltageCurrentResistance:40}
\begin{aligned}
\text{velocity} &= \text{distance}/ \text{time} \\
\text{acceleration} &= \text{velocity} / \text{time} \\
\text{force} &= \text{mass} \times \text{acceleration} \\
\text{energy} &= \text{force} \times \text{distance}
\end{aligned}
\end{equation}

(strictly speaking all but energy above are directed quantities, and I had incur the wrath of my grade twelve teacher for
 writing the above without qualification).

With these you really only need velocity to define the current, and one can write

\begin{equation}\label{eqn:voltageCurrentResistance:60}
\begin{aligned}
\text{current} (I) = \text{velocity} \times \text{charge}
\end{aligned}
\end{equation}

However, voltage is actually a fairly more abstract quantity and you need a bit more to define it.
In particular, you can then define voltage in terms of energy (which is already an abstract quantity compared to
the concrete quantities like mass, distance, charge and time).

Energy is required 
to move towards or separate (if they the charges are attractive) any two (or more) charged objects.
Saying that energy is required means that work must be done, where a force is exerted over a distance to move repulsive
charges towards each other into some specific configuration.  If two like charges, two electrons say, are far enough away
from each other that there is not a significant or easily measurable force between them, and you move them towards each other
to where a force between them can be measured, then one can say there is an energy bill associated with the final configuration
of the charges.

You do not necessarily have to know the route that the charges got there.  If one of those ``charges'' is a
cat, and you have walked that cat around the room towards a balloon on a table seven times before getting it near enough 
to the balloon that the cat hair is sticking out towards the balloon, then the net electrical effect on the cat hair will 
be the same as if you have walked that cat straight towards the balloon.  The voltage is a measure of the energy bill associated
with a charge configuration.  Now here the cat analogy breaks down a bit because it does not take much energy to walk that cat
away from the balloon (the attractive force between the cat hair and the balloon just is not that big), but there is charge
all over that balloon, and charge all over the cat's hair, and the energy per cat required to walk that cat away from the balloon
is what one would call the voltage.

More exactly one could define this voltage indirectly in terms of energy

\begin{equation}\label{eqn:voltageCurrentResistance:80}
\begin{aligned}
\text{energy} = \text{charge} \times \text{voltage}
\end{aligned}
\end{equation}

or, equivalently
\begin{equation}\label{eqn:voltageCurrentResistance:100}
\begin{aligned}
\text{voltage} = \text{energy} / \text{charge}
\end{aligned}
\end{equation}

Here, voltage is really the voltage measured for the cat and balloon charge configuration, and the energy is 
the energy required to counteract the electrical attraction between the cat hair and the balloon when they are close
compared to when they are far enough away that there is no longer much attraction.  In the above the charge is really
the charge of the cat ... this is the charge used to 'probe' the charge configuration of the balloon.

Notice that you can not really define voltage as an absolute quantity.  There is always two spatially separated points 
involved (ie: the close distance between the cat and balloon and the far distance).  You can just as easily define a
voltage between twill not so far separated points in space, and the energy in this case is just the energy to move
that distance.

To properly define the voltage for the energy bill associated with walking the cat towards the bill, you have to have
a careful accountant.   Specifically, he has to be instructed to only count the energy required or gained that was
associated with moving the cat to or away from the balloon.  It will take more work for you to personally walk around the
room seven times, but this work is muscular work and does not have much to do with the balloon's electrical properties.
If you did not want to count that walking in the work, you can start talking about tossing cats in outer space towards or 
away from the balloons, but let us say instead that you have a careful accountant.

Now, I do not know if that really ended up being a good description of voltage or not, but if it is, you can then define
resistance in terms of current and voltage and it will not be at all cyclic since both current and voltage have been defined
in terms of the fundamental quantities (time, mass, distance, charge).

%\end{document}

%
% Copyright � 2012 Peeter Joot.  All Rights Reserved.
% Licenced as described in the file LICENSE under the root directory of this GIT repository.
%

%
%
%\documentclass{article}      % Specifies the document class

%\usepackage{amsmath}
\usepackage{mathpazo}

%
% shorthand for bold symbols, convenient for vectors and matrices
%
\newcommand{\Ba}[0]{\mathbf{a}}
\newcommand{\Bb}[0]{\mathbf{b}}
\newcommand{\Bc}[0]{\mathbf{c}}
\newcommand{\Bd}[0]{\mathbf{d}}
\newcommand{\Be}[0]{\mathbf{e}}
\newcommand{\Bf}[0]{\mathbf{f}}
\newcommand{\Bg}[0]{\mathbf{g}}
\newcommand{\Bh}[0]{\mathbf{h}}
\newcommand{\Bi}[0]{\mathbf{i}}
\newcommand{\Bj}[0]{\mathbf{j}}
\newcommand{\Bk}[0]{\mathbf{k}}
\newcommand{\Bl}[0]{\mathbf{l}}
\newcommand{\Bm}[0]{\mathbf{m}}
\newcommand{\Bn}[0]{\mathbf{n}}
\newcommand{\Bo}[0]{\mathbf{o}}
\newcommand{\Bp}[0]{\mathbf{p}}
\newcommand{\Bq}[0]{\mathbf{q}}
\newcommand{\Br}[0]{\mathbf{r}}
\newcommand{\Bs}[0]{\mathbf{s}}
\newcommand{\Bt}[0]{\mathbf{t}}
\newcommand{\Bu}[0]{\mathbf{u}}
\newcommand{\Bv}[0]{\mathbf{v}}
\newcommand{\Bw}[0]{\mathbf{w}}
\newcommand{\Bx}[0]{\mathbf{x}}
\newcommand{\By}[0]{\mathbf{y}}
\newcommand{\Bz}[0]{\mathbf{z}}
\newcommand{\BA}[0]{\mathbf{A}}
\newcommand{\BB}[0]{\mathbf{B}}
\newcommand{\BC}[0]{\mathbf{C}}
\newcommand{\BD}[0]{\mathbf{D}}
\newcommand{\BE}[0]{\mathbf{E}}
\newcommand{\BF}[0]{\mathbf{F}}
\newcommand{\BG}[0]{\mathbf{G}}
\newcommand{\BH}[0]{\mathbf{H}}
\newcommand{\BI}[0]{\mathbf{I}}
\newcommand{\BJ}[0]{\mathbf{J}}
\newcommand{\BK}[0]{\mathbf{K}}
\newcommand{\BL}[0]{\mathbf{L}}
\newcommand{\BM}[0]{\mathbf{M}}
\newcommand{\BN}[0]{\mathbf{N}}
\newcommand{\BO}[0]{\mathbf{O}}
\newcommand{\BP}[0]{\mathbf{P}}
\newcommand{\BQ}[0]{\mathbf{Q}}
\newcommand{\BR}[0]{\mathbf{R}}
\newcommand{\BS}[0]{\mathbf{S}}
\newcommand{\BT}[0]{\mathbf{T}}
\newcommand{\BU}[0]{\mathbf{U}}
\newcommand{\BV}[0]{\mathbf{V}}
\newcommand{\BW}[0]{\mathbf{W}}
\newcommand{\BX}[0]{\mathbf{X}}
\newcommand{\BY}[0]{\mathbf{Y}}
\newcommand{\BZ}[0]{\mathbf{Z}}

\newcommand{\Bzero}[0]{\mathbf{0}}
\newcommand{\Btheta}[0]{\boldsymbol{\theta}}
\newcommand{\Btau}[0]{\boldsymbol{\tau}}
\newcommand{\Bomega}[0]{\boldsymbol{\omega}}

%
% shorthand for unit vectors
%
\newcommand{\acap}[0]{\hat{\Ba}}
\newcommand{\bcap}[0]{\hat{\Bb}}
\newcommand{\ccap}[0]{\hat{\Bc}}
\newcommand{\dcap}[0]{\hat{\Bd}}
\newcommand{\ecap}[0]{\hat{\Be}}
\newcommand{\fcap}[0]{\hat{\Bf}}
\newcommand{\gcap}[0]{\hat{\Bg}}
\newcommand{\hcap}[0]{\hat{\Bh}}
\newcommand{\icap}[0]{\hat{\Bi}}
\newcommand{\jcap}[0]{\hat{\Bj}}
\newcommand{\kcap}[0]{\hat{\Bk}}
\newcommand{\lcap}[0]{\hat{\Bl}}
\newcommand{\mcap}[0]{\hat{\Bm}}
\newcommand{\ncap}[0]{\hat{\Bn}}
\newcommand{\ocap}[0]{\hat{\Bo}}
\newcommand{\pcap}[0]{\hat{\Bp}}
\newcommand{\qcap}[0]{\hat{\Bq}}
\newcommand{\rcap}[0]{\hat{\Br}}
\newcommand{\scap}[0]{\hat{\Bs}}
\newcommand{\tcap}[0]{\hat{\Bt}}
\newcommand{\ucap}[0]{\hat{\Bu}}
\newcommand{\vcap}[0]{\hat{\Bv}}
\newcommand{\wcap}[0]{\hat{\Bw}}
\newcommand{\xcap}[0]{\hat{\Bx}}
\newcommand{\ycap}[0]{\hat{\By}}
\newcommand{\zcap}[0]{\hat{\Bz}}
\newcommand{\thetacap}[0]{\hat{\Btheta}}

%
% to write R^n and C^n in a distinguishable fashion.  Perhaps change this
% to the double lined characters upon figuring out how to do so.
%
\newcommand{\C}[1]{$\mathbb{C}^{#1}$}
\newcommand{\R}[1]{$\mathbb{R}^{#1}$}

%
% various generally useful helpers
%

% derivative of #1 wrt. #2:
\newcommand{\D}[2] {\frac {d#2} {d#1}}

\newcommand{\inv}[1]{\frac{1}{#1}}
\newcommand{\cross}[0]{\times}

\newcommand{\abs}[1]{\lvert{#1}\rvert}
\newcommand{\norm}[1]{\lVert{#1}\rVert}
\newcommand{\innerprod}[2]{\langle{#1}, {#2}\rangle}
\newcommand{\dotprod}[2]{{#1} \cdot {#2}}
\newcommand{\bdotprod}[2]{\left({#1} \cdot {#2}\right)}
\newcommand{\crossprod}[2]{{#1} \cross {#2}}
\newcommand{\tripleprod}[3]{\dotprod{\left(\crossprod{#1}{#2}\right)}{#3}}

\DeclareMathOperator{\Proj}{Proj}
\DeclareMathOperator{\Span}{span}
\DeclareMathOperator{\Sgn}{sgn}
\DeclareMathOperator{\Area}{Area}
\DeclareMathOperator{\Volume}{Volume}

%
% A few miscellaneous things specific to this document
%
\newcommand{\crossop}[1]{\crossprod{#1}{}}

% R2 vector.
\newcommand{\VectorTwo}[2]{
\begin{bmatrix}
 {#1} \\
 {#2}
\end{bmatrix}
}

\newcommand{\VectorN}[1]{
\begin{bmatrix}
{#1}_1 \\
{#1}_2 \\
\vdots \\
{#1}_N \\
\end{bmatrix}
}

\newcommand{\DETuvij}[4]{
\begin{vmatrix}
 {#1}_{#3} & {#1}_{#4} \\
 {#2}_{#3} & {#2}_{#4}
\end{vmatrix}
}

\newcommand{\DETuvwijk}[6]{
\begin{vmatrix}
 {#1}_{#4} & {#1}_{#5} & {#1}_{#6} \\
 {#2}_{#4} & {#2}_{#5} & {#2}_{#6} \\
 {#3}_{#4} & {#3}_{#5} & {#3}_{#6}
\end{vmatrix}
}

\newcommand{\DETuvwxijkl}[8]{
\begin{vmatrix}
 {#1}_{#5} & {#1}_{#6} & {#1}_{#7} & {#1}_{#8} \\
 {#2}_{#5} & {#2}_{#6} & {#2}_{#7} & {#2}_{#8} \\
 {#3}_{#5} & {#3}_{#6} & {#3}_{#7} & {#3}_{#8} \\
 {#4}_{#5} & {#4}_{#6} & {#4}_{#7} & {#4}_{#8} \\
\end{vmatrix}
}

%\newcommand{\DETuvwxyijklm}[10]{
%\begin{vmatrix}
% {#1}_{#6} & {#1}_{#7} & {#1}_{#8} & {#1}_{#9} & {#1}_{#10} \\
% {#2}_{#6} & {#2}_{#7} & {#2}_{#8} & {#2}_{#9} & {#2}_{#10} \\
% {#3}_{#6} & {#3}_{#7} & {#3}_{#8} & {#3}_{#9} & {#3}_{#10} \\
% {#4}_{#6} & {#4}_{#7} & {#4}_{#8} & {#4}_{#9} & {#4}_{#10} \\
% {#5}_{#6} & {#5}_{#7} & {#5}_{#8} & {#5}_{#9} & {#5}_{#10}
%\end{vmatrix}
%}

% R3 vector.
\newcommand{\VectorThree}[3]{
\begin{bmatrix}
 {#1} \\
 {#2} \\
 {#3}
\end{bmatrix}
}



%
% The real thing:
%

%\usepackage[bookmarks=true]{hyperref}

\chapter{rindler}
\label{chap:rindler}
%\title{} % Declares the document's title.
%\author{Peeter Joot \quad peeterjoot@protonmail.com}         % Declares the author's name.
\date{ \(RCSfile: rindler.tex,v \) Last \(Revision: 1.7 \) \(Date: 2009/06/14 23:51:45 \) }
%
%\begin{document}             % End of preamble and beginning of text.

%\maketitle{}
%
%\tableofcontents
%
%\section{}

You had the following as the DE to solve with \(y = 1 + ax\)

\begin{equation}\label{eqn:rindler:20}
\begin{aligned}
y\ddot{y} &= a^2 - \dot{y}^2 \\
y\ddot{y} + \dot{y}^2 &= a^2 \\
\end{aligned}
\end{equation}

The solution appears to involve \(y\dot{y}\), which when differentiated
is

\begin{equation}\label{eqn:rindler:40}
\begin{aligned}
(y\dot{y})' &= y \ddot{y} + \dot{y}^2,
\end{aligned}
\end{equation}

as desired.  However, this can be integrated once more

\begin{equation}\label{eqn:rindler:60}
\begin{aligned}
y\dot{y} = \left(\inv{2} y^2\right)'.
\end{aligned}
\end{equation}

This puts your DE in the following convenient form

\begin{equation}\label{eqn:rindler:80}
\begin{aligned}
y\ddot{y} + \dot{y}^2 &= a^2 \\
&= \frac{d^2}{d\tau^2}\left(\inv{2} y^2\right).
\end{aligned}
\end{equation}

Integrating once, with constant of integration \(a \kappa\) gives

\begin{equation}\label{eqn:rindler:100}
\begin{aligned}
\frac{d}{d\tau}\left(\inv{2} y^2\right) &= a^2 \tau + a \kappa \\
\frac{d}{d\tau}\left(\inv{2} \left(1+ax\right)^2\right) &=  \\
\left(1+ax\right)\frac{d}{d\tau}\left(1+ax\right) &=  \\
\left(1+ax\right)a\frac{dx}{d\tau} &=  \\
\implies
\left(1+ax\right)\frac{dx}{d\tau} &= a\tau + \kappa \\
\end{aligned}
\end{equation}

Alternatively, integrating twice with \(\kappa = n/2\), and second integration constant \(a (m)/2\) we have

\begin{equation}\label{eqn:rindler:120}
\begin{aligned}
\inv{2} y^2 &= \inv{2} a^2 \tau^2 + \inv{2} a n \tau + \inv{2} a m \\
y^2 &= a^2 \tau^2 + a n\tau + a m \\
(1+ax)^2 &= \\
\implies
x
&= \sqrt{\tau^2 + n\tau + m} -1/a \\
&= \sqrt{\tau^2 + n\tau + (x_0 + 1/a)^2} -1/a \\
&= \sqrt{\tau^2 + (1 + a x_0)v_0 \tau + (x_0 + 1/a)^2} -1/a \\
\end{aligned}
\end{equation}

%\kappa = (1 + ax_0) v_0 = n/2

%\end{document}               % End of document.


% 
% 
% 
% Copyright � 2012 Peeter Joot
% All Rights Reserved
% 
% This file may be reproduced and distributed in whole or in part, without fee, subject to the following conditions:
% 
% o The copyright notice above and this permission notice must be preserved complete on all complete or partial copies.
% 
% o Any translation or derived work must be approved by the author in writing before distribution.
% 
% o If you distribute this work in part, instructions for obtaining the complete version of this file must be included, and a means for obtaining a complete version provided.
% 
% 
% Exceptions to these rules may be granted for academic purposes: Write to the author and ask.
% 
% 
% 
%\documentclass{article}      % Specifies the document class

%\usepackage{amsmath}
\usepackage{mathpazo}

%
% shorthand for bold symbols, convenient for vectors and matrices
%
\newcommand{\Ba}[0]{\mathbf{a}}
\newcommand{\Bb}[0]{\mathbf{b}}
\newcommand{\Bc}[0]{\mathbf{c}}
\newcommand{\Bd}[0]{\mathbf{d}}
\newcommand{\Be}[0]{\mathbf{e}}
\newcommand{\Bf}[0]{\mathbf{f}}
\newcommand{\Bg}[0]{\mathbf{g}}
\newcommand{\Bh}[0]{\mathbf{h}}
\newcommand{\Bi}[0]{\mathbf{i}}
\newcommand{\Bj}[0]{\mathbf{j}}
\newcommand{\Bk}[0]{\mathbf{k}}
\newcommand{\Bl}[0]{\mathbf{l}}
\newcommand{\Bm}[0]{\mathbf{m}}
\newcommand{\Bn}[0]{\mathbf{n}}
\newcommand{\Bo}[0]{\mathbf{o}}
\newcommand{\Bp}[0]{\mathbf{p}}
\newcommand{\Bq}[0]{\mathbf{q}}
\newcommand{\Br}[0]{\mathbf{r}}
\newcommand{\Bs}[0]{\mathbf{s}}
\newcommand{\Bt}[0]{\mathbf{t}}
\newcommand{\Bu}[0]{\mathbf{u}}
\newcommand{\Bv}[0]{\mathbf{v}}
\newcommand{\Bw}[0]{\mathbf{w}}
\newcommand{\Bx}[0]{\mathbf{x}}
\newcommand{\By}[0]{\mathbf{y}}
\newcommand{\Bz}[0]{\mathbf{z}}
\newcommand{\BA}[0]{\mathbf{A}}
\newcommand{\BB}[0]{\mathbf{B}}
\newcommand{\BC}[0]{\mathbf{C}}
\newcommand{\BD}[0]{\mathbf{D}}
\newcommand{\BE}[0]{\mathbf{E}}
\newcommand{\BF}[0]{\mathbf{F}}
\newcommand{\BG}[0]{\mathbf{G}}
\newcommand{\BH}[0]{\mathbf{H}}
\newcommand{\BI}[0]{\mathbf{I}}
\newcommand{\BJ}[0]{\mathbf{J}}
\newcommand{\BK}[0]{\mathbf{K}}
\newcommand{\BL}[0]{\mathbf{L}}
\newcommand{\BM}[0]{\mathbf{M}}
\newcommand{\BN}[0]{\mathbf{N}}
\newcommand{\BO}[0]{\mathbf{O}}
\newcommand{\BP}[0]{\mathbf{P}}
\newcommand{\BQ}[0]{\mathbf{Q}}
\newcommand{\BR}[0]{\mathbf{R}}
\newcommand{\BS}[0]{\mathbf{S}}
\newcommand{\BT}[0]{\mathbf{T}}
\newcommand{\BU}[0]{\mathbf{U}}
\newcommand{\BV}[0]{\mathbf{V}}
\newcommand{\BW}[0]{\mathbf{W}}
\newcommand{\BX}[0]{\mathbf{X}}
\newcommand{\BY}[0]{\mathbf{Y}}
\newcommand{\BZ}[0]{\mathbf{Z}}

\newcommand{\Bzero}[0]{\mathbf{0}}
\newcommand{\Btheta}[0]{\boldsymbol{\theta}}
\newcommand{\Btau}[0]{\boldsymbol{\tau}}
\newcommand{\Bomega}[0]{\boldsymbol{\omega}}

%
% shorthand for unit vectors
%
\newcommand{\acap}[0]{\hat{\Ba}}
\newcommand{\bcap}[0]{\hat{\Bb}}
\newcommand{\ccap}[0]{\hat{\Bc}}
\newcommand{\dcap}[0]{\hat{\Bd}}
\newcommand{\ecap}[0]{\hat{\Be}}
\newcommand{\fcap}[0]{\hat{\Bf}}
\newcommand{\gcap}[0]{\hat{\Bg}}
\newcommand{\hcap}[0]{\hat{\Bh}}
\newcommand{\icap}[0]{\hat{\Bi}}
\newcommand{\jcap}[0]{\hat{\Bj}}
\newcommand{\kcap}[0]{\hat{\Bk}}
\newcommand{\lcap}[0]{\hat{\Bl}}
\newcommand{\mcap}[0]{\hat{\Bm}}
\newcommand{\ncap}[0]{\hat{\Bn}}
\newcommand{\ocap}[0]{\hat{\Bo}}
\newcommand{\pcap}[0]{\hat{\Bp}}
\newcommand{\qcap}[0]{\hat{\Bq}}
\newcommand{\rcap}[0]{\hat{\Br}}
\newcommand{\scap}[0]{\hat{\Bs}}
\newcommand{\tcap}[0]{\hat{\Bt}}
\newcommand{\ucap}[0]{\hat{\Bu}}
\newcommand{\vcap}[0]{\hat{\Bv}}
\newcommand{\wcap}[0]{\hat{\Bw}}
\newcommand{\xcap}[0]{\hat{\Bx}}
\newcommand{\ycap}[0]{\hat{\By}}
\newcommand{\zcap}[0]{\hat{\Bz}}
\newcommand{\thetacap}[0]{\hat{\Btheta}}

%
% to write R^n and C^n in a distinguishable fashion.  Perhaps change this
% to the double lined characters upon figuring out how to do so.
%
\newcommand{\C}[1]{$\mathbb{C}^{#1}$}
\newcommand{\R}[1]{$\mathbb{R}^{#1}$}

%
% various generally useful helpers
%

% derivative of #1 wrt. #2:
\newcommand{\D}[2] {\frac {d#2} {d#1}}

\newcommand{\inv}[1]{\frac{1}{#1}}
\newcommand{\cross}[0]{\times}

\newcommand{\abs}[1]{\lvert{#1}\rvert}
\newcommand{\norm}[1]{\lVert{#1}\rVert}
\newcommand{\innerprod}[2]{\langle{#1}, {#2}\rangle}
\newcommand{\dotprod}[2]{{#1} \cdot {#2}}
\newcommand{\bdotprod}[2]{\left({#1} \cdot {#2}\right)}
\newcommand{\crossprod}[2]{{#1} \cross {#2}}
\newcommand{\tripleprod}[3]{\dotprod{\left(\crossprod{#1}{#2}\right)}{#3}}

\DeclareMathOperator{\Proj}{Proj}
\DeclareMathOperator{\Span}{span}
\DeclareMathOperator{\Sgn}{sgn}
\DeclareMathOperator{\Area}{Area}
\DeclareMathOperator{\Volume}{Volume}

%
% A few miscellaneous things specific to this document
%
\newcommand{\crossop}[1]{\crossprod{#1}{}}

% R2 vector.
\newcommand{\VectorTwo}[2]{
\begin{bmatrix}
 {#1} \\
 {#2}
\end{bmatrix}
}

\newcommand{\VectorN}[1]{
\begin{bmatrix}
{#1}_1 \\
{#1}_2 \\
\vdots \\
{#1}_N \\
\end{bmatrix}
}

\newcommand{\DETuvij}[4]{
\begin{vmatrix}
 {#1}_{#3} & {#1}_{#4} \\
 {#2}_{#3} & {#2}_{#4}
\end{vmatrix}
}

\newcommand{\DETuvwijk}[6]{
\begin{vmatrix}
 {#1}_{#4} & {#1}_{#5} & {#1}_{#6} \\
 {#2}_{#4} & {#2}_{#5} & {#2}_{#6} \\
 {#3}_{#4} & {#3}_{#5} & {#3}_{#6}
\end{vmatrix}
}

\newcommand{\DETuvwxijkl}[8]{
\begin{vmatrix}
 {#1}_{#5} & {#1}_{#6} & {#1}_{#7} & {#1}_{#8} \\
 {#2}_{#5} & {#2}_{#6} & {#2}_{#7} & {#2}_{#8} \\
 {#3}_{#5} & {#3}_{#6} & {#3}_{#7} & {#3}_{#8} \\
 {#4}_{#5} & {#4}_{#6} & {#4}_{#7} & {#4}_{#8} \\
\end{vmatrix}
}

%\newcommand{\DETuvwxyijklm}[10]{
%\begin{vmatrix}
% {#1}_{#6} & {#1}_{#7} & {#1}_{#8} & {#1}_{#9} & {#1}_{#10} \\
% {#2}_{#6} & {#2}_{#7} & {#2}_{#8} & {#2}_{#9} & {#2}_{#10} \\
% {#3}_{#6} & {#3}_{#7} & {#3}_{#8} & {#3}_{#9} & {#3}_{#10} \\
% {#4}_{#6} & {#4}_{#7} & {#4}_{#8} & {#4}_{#9} & {#4}_{#10} \\
% {#5}_{#6} & {#5}_{#7} & {#5}_{#8} & {#5}_{#9} & {#5}_{#10}
%\end{vmatrix}
%}

% R3 vector.
\newcommand{\VectorThree}[3]{
\begin{bmatrix}
 {#1} \\
 {#2} \\
 {#3}
\end{bmatrix}
}


%%<misc>
%
\newcommand{\Abs}[1]{{\left\lvert{#1}\right\rvert}}
\newcommand{\spacegrad}[0]{\boldsymbol{\nabla}}
\newcommand{\grad}[0]{\nabla}
\newcommand{\LL}[0]{\mathcal{L}}

% == \partial_{#1} {#2}
\newcommand{\PD}[2]{\frac{\partial {#2}}{\partial {#1}}}
% inline variant
\newcommand{\PDi}[2]{{\partial {#2}}/{\partial {#1}}}

\newcommand{\PDD}[3]{\frac{\partial^2 {#3}}{\partial {#1}\partial {#2}}}
%\newcommand{\PDd}[2]{\frac{\partial^2 {#2}}{{\partial{#1}}^2}}
\newcommand{\PDsq}[2]{\frac{\partial^2 {#2}}{(\partial {#1})^2}}

\newcommand{\Partial}[2]{\frac{\partial {#1}}{\partial {#2}}}
\DeclareMathOperator{\RejName}{Rej}
\newcommand{\Rej}[2]{\RejName_{#1}\left( {#2} \right)}
\newcommand{\Rm}[1]{\mathbb{R}^{#1}}
\newcommand{\Cm}[1]{\mathbb{C}^{#1}}
\newcommand{\conj}[0]{{*}}

%</misc>

% <grade selection>
%
\newcommand{\gpgrade}[2] {{\left\langle{{#1}}\right\rangle}_{#2}}

\newcommand{\gpgradezero}[1] {\gpgrade{#1}{}}
%\newcommand{\gpscalargrade}[1] {{\left\langle{{#1}}\right\rangle}}
%\newcommand{\gpgradezero}[1] {\gpgrade{#1}{0}}

%\newcommand{\gpgradeone}[1] {{\left\langle{{#1}}\right\rangle}_{1}}
\newcommand{\gpgradeone}[1] {\gpgrade{#1}{1}}

\newcommand{\gpgradetwo}[1] {\gpgrade{#1}{2}}
\newcommand{\gpgradethree}[1] {\gpgrade{#1}{3}}
\newcommand{\gpgradefour}[1] {\gpgrade{#1}{4}}
%
% </grade selection>



\newcommand{\adot}[0]{{\dot{a}}}
\newcommand{\bdot}[0]{{\dot{b}}}
% taken for centered dot:
%\newcommand{\cdot}[0]{{\dot{c}}}
%\newcommand{\ddot}[0]{{\dot{d}}}
\newcommand{\edot}[0]{{\dot{e}}}
\newcommand{\fdot}[0]{{\dot{f}}}
\newcommand{\gdot}[0]{{\dot{g}}}
\newcommand{\hdot}[0]{{\dot{h}}}
\newcommand{\idot}[0]{{\dot{i}}}
\newcommand{\jdot}[0]{{\dot{j}}}
\newcommand{\kdot}[0]{{\dot{k}}}
\newcommand{\ldot}[0]{{\dot{l}}}
\newcommand{\mdot}[0]{{\dot{m}}}
\newcommand{\ndot}[0]{{\dot{n}}}
%\newcommand{\odot}[0]{{\dot{o}}}
\newcommand{\pdot}[0]{{\dot{p}}}
\newcommand{\qdot}[0]{{\dot{q}}}
\newcommand{\rdot}[0]{{\dot{r}}}
\newcommand{\sdot}[0]{{\dot{s}}}
\newcommand{\tdot}[0]{{\dot{t}}}
\newcommand{\udot}[0]{{\dot{u}}}
\newcommand{\vdot}[0]{{\dot{v}}}
\newcommand{\wdot}[0]{{\dot{w}}}
\newcommand{\xdot}[0]{{\dot{x}}}
\newcommand{\ydot}[0]{{\dot{y}}}
\newcommand{\zdot}[0]{{\dot{z}}}
\newcommand{\addot}[0]{{\ddot{a}}}
\newcommand{\bddot}[0]{{\ddot{b}}}
\newcommand{\cddot}[0]{{\ddot{c}}}
%\newcommand{\dddot}[0]{{\ddot{d}}}
\newcommand{\eddot}[0]{{\ddot{e}}}
\newcommand{\fddot}[0]{{\ddot{f}}}
\newcommand{\gddot}[0]{{\ddot{g}}}
\newcommand{\hddot}[0]{{\ddot{h}}}
\newcommand{\iddot}[0]{{\ddot{i}}}
\newcommand{\jddot}[0]{{\ddot{j}}}
\newcommand{\kddot}[0]{{\ddot{k}}}
\newcommand{\lddot}[0]{{\ddot{l}}}
\newcommand{\mddot}[0]{{\ddot{m}}}
\newcommand{\nddot}[0]{{\ddot{n}}}
\newcommand{\oddot}[0]{{\ddot{o}}}
\newcommand{\pddot}[0]{{\ddot{p}}}
\newcommand{\qddot}[0]{{\ddot{q}}}
\newcommand{\rddot}[0]{{\ddot{r}}}
\newcommand{\sddot}[0]{{\ddot{s}}}
\newcommand{\tddot}[0]{{\ddot{t}}}
\newcommand{\uddot}[0]{{\ddot{u}}}
\newcommand{\vddot}[0]{{\ddot{v}}}
\newcommand{\wddot}[0]{{\ddot{w}}}
\newcommand{\xddot}[0]{{\ddot{x}}}
\newcommand{\yddot}[0]{{\ddot{y}}}
\newcommand{\zddot}[0]{{\ddot{z}}}

%<bold and dot greek symbols>
%

\newcommand{\Deltadot}[0]{{\dot{\Delta}}}
\newcommand{\Gammadot}[0]{{\dot{\Gamma}}}
\newcommand{\Lambdadot}[0]{{\dot{\Lambda}}}
\newcommand{\Omegadot}[0]{{\dot{\Omega}}}
\newcommand{\Phidot}[0]{{\dot{\Phi}}}
\newcommand{\Pidot}[0]{{\dot{\Pi}}}
\newcommand{\Psidot}[0]{{\dot{\Psi}}}
\newcommand{\Sigmadot}[0]{{\dot{\Sigma}}}
\newcommand{\Thetadot}[0]{{\dot{\Theta}}}
\newcommand{\Upsilondot}[0]{{\dot{\Upsilon}}}
\newcommand{\Xidot}[0]{{\dot{\Xi}}}
\newcommand{\alphadot}[0]{{\dot{\alpha}}}
\newcommand{\betadot}[0]{{\dot{\beta}}}
\newcommand{\chidot}[0]{{\dot{\chi}}}
\newcommand{\deltadot}[0]{{\dot{\delta}}}
\newcommand{\epsilondot}[0]{{\dot{\epsilon}}}
\newcommand{\etadot}[0]{{\dot{\eta}}}
\newcommand{\gammadot}[0]{{\dot{\gamma}}}
\newcommand{\kappadot}[0]{{\dot{\kappa}}}
\newcommand{\lambdadot}[0]{{\dot{\lambda}}}
\newcommand{\mudot}[0]{{\dot{\mu}}}
\newcommand{\nudot}[0]{{\dot{\nu}}}
\newcommand{\omegadot}[0]{{\dot{\omega}}}
\newcommand{\phidot}[0]{{\dot{\phi}}}
\newcommand{\pidot}[0]{{\dot{\pi}}}
\newcommand{\psidot}[0]{{\dot{\psi}}}
\newcommand{\rhodot}[0]{{\dot{\rho}}}
\newcommand{\sigmadot}[0]{{\dot{\sigma}}}
\newcommand{\taudot}[0]{{\dot{\tau}}}
\newcommand{\thetadot}[0]{{\dot{\theta}}}
\newcommand{\upsilondot}[0]{{\dot{\upsilon}}}
\newcommand{\varepsilondot}[0]{{\dot{\varepsilon}}}
\newcommand{\varphidot}[0]{{\dot{\varphi}}}
\newcommand{\varpidot}[0]{{\dot{\varpi}}}
\newcommand{\varrhodot}[0]{{\dot{\varrho}}}
\newcommand{\varsigmadot}[0]{{\dot{\varsigma}}}
\newcommand{\varthetadot}[0]{{\dot{\vartheta}}}
\newcommand{\xidot}[0]{{\dot{\xi}}}
\newcommand{\zetadot}[0]{{\dot{\zeta}}}

\newcommand{\Deltaddot}[0]{{\ddot{\Delta}}}
\newcommand{\Gammaddot}[0]{{\ddot{\Gamma}}}
\newcommand{\Lambdaddot}[0]{{\ddot{\Lambda}}}
\newcommand{\Omegaddot}[0]{{\ddot{\Omega}}}
\newcommand{\Phiddot}[0]{{\ddot{\Phi}}}
\newcommand{\Piddot}[0]{{\ddot{\Pi}}}
\newcommand{\Psiddot}[0]{{\ddot{\Psi}}}
\newcommand{\Sigmaddot}[0]{{\ddot{\Sigma}}}
\newcommand{\Thetaddot}[0]{{\ddot{\Theta}}}
\newcommand{\Upsilonddot}[0]{{\ddot{\Upsilon}}}
\newcommand{\Xiddot}[0]{{\ddot{\Xi}}}
\newcommand{\alphaddot}[0]{{\ddot{\alpha}}}
\newcommand{\betaddot}[0]{{\ddot{\beta}}}
\newcommand{\chiddot}[0]{{\ddot{\chi}}}
\newcommand{\deltaddot}[0]{{\ddot{\delta}}}
\newcommand{\epsilonddot}[0]{{\ddot{\epsilon}}}
\newcommand{\etaddot}[0]{{\ddot{\eta}}}
\newcommand{\gammaddot}[0]{{\ddot{\gamma}}}
\newcommand{\kappaddot}[0]{{\ddot{\kappa}}}
\newcommand{\lambdaddot}[0]{{\ddot{\lambda}}}
\newcommand{\muddot}[0]{{\ddot{\mu}}}
\newcommand{\nuddot}[0]{{\ddot{\nu}}}
\newcommand{\omegaddot}[0]{{\ddot{\omega}}}
\newcommand{\phiddot}[0]{{\ddot{\phi}}}
\newcommand{\piddot}[0]{{\ddot{\pi}}}
\newcommand{\psiddot}[0]{{\ddot{\psi}}}
\newcommand{\rhoddot}[0]{{\ddot{\rho}}}
\newcommand{\sigmaddot}[0]{{\ddot{\sigma}}}
\newcommand{\tauddot}[0]{{\ddot{\tau}}}
\newcommand{\thetaddot}[0]{{\ddot{\theta}}}
\newcommand{\upsilonddot}[0]{{\ddot{\upsilon}}}
\newcommand{\varepsilonddot}[0]{{\ddot{\varepsilon}}}
\newcommand{\varphiddot}[0]{{\ddot{\varphi}}}
\newcommand{\varpiddot}[0]{{\ddot{\varpi}}}
\newcommand{\varrhoddot}[0]{{\ddot{\varrho}}}
\newcommand{\varsigmaddot}[0]{{\ddot{\varsigma}}}
\newcommand{\varthetaddot}[0]{{\ddot{\vartheta}}}
\newcommand{\xiddot}[0]{{\ddot{\xi}}}
\newcommand{\zetaddot}[0]{{\ddot{\zeta}}}

\newcommand{\BDelta}[0]{\boldsymbol{\Delta}}
\newcommand{\BGamma}[0]{\boldsymbol{\Gamma}}
\newcommand{\BLambda}[0]{\boldsymbol{\Lambda}}
\newcommand{\BOmega}[0]{\boldsymbol{\Omega}}
\newcommand{\BPhi}[0]{\boldsymbol{\Phi}}
\newcommand{\BPi}[0]{\boldsymbol{\Pi}}
\newcommand{\BPsi}[0]{\boldsymbol{\Psi}}
\newcommand{\BSigma}[0]{\boldsymbol{\Sigma}}
\newcommand{\BTheta}[0]{\boldsymbol{\Theta}}
\newcommand{\BUpsilon}[0]{\boldsymbol{\Upsilon}}
\newcommand{\BXi}[0]{\boldsymbol{\Xi}}
\newcommand{\Balpha}[0]{\boldsymbol{\alpha}}
\newcommand{\Bbeta}[0]{\boldsymbol{\beta}}
\newcommand{\Bchi}[0]{\boldsymbol{\chi}}
\newcommand{\Bdelta}[0]{\boldsymbol{\delta}}
\newcommand{\Bepsilon}[0]{\boldsymbol{\epsilon}}
\newcommand{\Beta}[0]{\boldsymbol{\eta}}
\newcommand{\Bgamma}[0]{\boldsymbol{\gamma}}
\newcommand{\Bkappa}[0]{\boldsymbol{\kappa}}
\newcommand{\Blambda}[0]{\boldsymbol{\lambda}}
\newcommand{\Bmu}[0]{\boldsymbol{\mu}}
\newcommand{\Bnu}[0]{\boldsymbol{\nu}}
%\newcommand{\Bomega}[0]{\boldsymbol{\omega}}
\newcommand{\Bphi}[0]{\boldsymbol{\phi}}
\newcommand{\Bpi}[0]{\boldsymbol{\pi}}
\newcommand{\Bpsi}[0]{\boldsymbol{\psi}}
\newcommand{\Brho}[0]{\boldsymbol{\rho}}
\newcommand{\Bsigma}[0]{\boldsymbol{\sigma}}
%\newcommand{\Btau}[0]{\boldsymbol{\tau}}
%\newcommand{\Btheta}[0]{\boldsymbol{\theta}}
\newcommand{\Bupsilon}[0]{\boldsymbol{\upsilon}}
\newcommand{\Bvarepsilon}[0]{\boldsymbol{\varepsilon}}
\newcommand{\Bvarphi}[0]{\boldsymbol{\varphi}}
\newcommand{\Bvarpi}[0]{\boldsymbol{\varpi}}
\newcommand{\Bvarrho}[0]{\boldsymbol{\varrho}}
\newcommand{\Bvarsigma}[0]{\boldsymbol{\varsigma}}
\newcommand{\Bvartheta}[0]{\boldsymbol{\vartheta}}
\newcommand{\Bxi}[0]{\boldsymbol{\xi}}
\newcommand{\Bzeta}[0]{\boldsymbol{\zeta}}
%
%</bold and dot greek symbols>
%<infrequent>
%
%\newcommand{\AreaOp}[1]{\AName_{#1}}
%\newcommand{\Babs}[0]{\abs{\BB}}
%\newcommand{\Bcap}[0]{\hat{\BB}}
%\newcommand{\BrPrimeRej}[0]{\rcap(\rcap \wedge \Br')}
%\newcommand{\CA}[0]{\mathcal{A}}
%\newcommand{\Cos}[1]{\cos{\left({#1}\right)}}
%\newcommand{\Det}[1] {\abs{#1}}
%\newcommand{\Dsq}[2] {\frac {\partial^2 {#1}} {\partial {#2}^2}}
%\newcommand{\Exp}[1]{\exp{\left({#1}\right)}}
%\newcommand{\Norm}[1]{\left\lVert{#1}\right\rVert}
%\newcommand{\Sin}[1]{\sin{\left({#1}\right)}}
%\newcommand{\T}[0]{\text{T}}
%\newcommand{\VolumeOp}[1]{\VName_{#1}}
%\newcommand{\agrad}[0]{\Ba \cdot \nabla}
%\newcommand{\alphacap}[0]{\hat{\boldsymbol{\alpha}}}
%\newcommand{\Fcap}[0]{\hat{\BF}}
%\newcommand{\bithree}[0]{{\Bi}_3}
%\newcommand{\bxa}[0]{\Bx\Ba}
%\newcommand{\coordvec}[2]{
%\newcommand{\costheta}[0]{\acap \cdot \xcap}
%\newcommand{\ddt}[1]{\ddot{#1}}
%\newcommand{\ddu}[1] {\frac {d{#1}} {du}}
%\newcommand{\dsqxj}[2] {\frac {\partial^2 {#1}} {\partial {x_{#2}}^2}}
%\newcommand{\dtheta}[1]{\frac{d {#1}}{d \theta}}
%\newcommand{\dt}[1]{\dot{#1}}
%\newcommand{\dt}[1]{\frac{d {#1}}{dt}}
%\newcommand{\dxj}[2] {\frac {\partial {#1}} {\partial {x_{#2}}}}
%\newcommand{\halfPhi}[0]{\frac{\phi}{2}}
%\newcommand{\half}[0]{\inv{2}}
%\newcommand{\inv}[1]{\frac{1}{#1}}
%\newcommand{\laplacian}[0]{\nabla^2}
%\newcommand{\matrixoftx}[3]{
%\newcommand{\nrrp}[0]{\norm{\rcap \wedge \Br'}}
%\newcommand{\oiint}{\bigcirc \hspace{-1.4em} \int \hspace{-.8em} \int}
%\newcommand{\transpose}[1]{{#1}^{\text{T}}}
%\newcommand{\transpose}[1]{{{#1}^{\TextTranspose}}}
%\newcommand{\transpose}[1]{{{#1}^{\text{T}}}}
%\newcommand{\barA}[0]{\bar{A}}
%\newcommand{\qbar}[0]{\bar{q}}
%\newcommand{\qdotbar}[0]{\dot{\bar{q}}}
%
%</infrequent>





%
% The real thing:
%

%\usepackage[bookmarks=true]{hyperref}

                             % The preamble begins here.
\chapter{constant q}
\label{chap:constantQ}
%\author{Peeter Joot \quad peeter.joot@gmail.com}         % Declares the author's name.
\date{ $RCSfile: constantQ.tex,v $ Last $Revision: 1.6 $ $Date: 2009/06/14 23:51:45 $ }

%\begin{document}             % End of preamble and beginning of text.

%\maketitle{}
%
%\tableofcontents
%
%\section{}

\begin{align*}
Q &= \int \rho dV \\
  &= \int \inv{c} j^0 d^3x
\end{align*}
\begin{align*}
\frac{dQ}{dt} 
  &= \int \PD{(ct)}{j^0} d^3x \\
  &= \int \PD{x^0}{j^0} d^3x \\
  &= -\int \partial_i j^i d^3x \\
\end{align*}

Since this is a divergence you can convert to a surface integral, and then with the assumption that $J = 0$ at infinity (and thus $J \cdot \Bn$ = 0) that surface integral is also zero.

%%\bibliographystyle{plain}
%\bibliographystyle{plainnat} % supposed to allow for \url use.
%\bibliography{myrefs}      % expects file "myrefs.bib"

%\end{document}               % End of document.

\documentclass{article}

\usepackage{amsmath}
\usepackage{mathpazo}

%
% shorthand for bold symbols, convenient for vectors and matrices
%
\newcommand{\Ba}[0]{\mathbf{a}}
\newcommand{\Bb}[0]{\mathbf{b}}
\newcommand{\Bc}[0]{\mathbf{c}}
\newcommand{\Bd}[0]{\mathbf{d}}
\newcommand{\Be}[0]{\mathbf{e}}
\newcommand{\Bf}[0]{\mathbf{f}}
\newcommand{\Bg}[0]{\mathbf{g}}
\newcommand{\Bh}[0]{\mathbf{h}}
\newcommand{\Bi}[0]{\mathbf{i}}
\newcommand{\Bj}[0]{\mathbf{j}}
\newcommand{\Bk}[0]{\mathbf{k}}
\newcommand{\Bl}[0]{\mathbf{l}}
\newcommand{\Bm}[0]{\mathbf{m}}
\newcommand{\Bn}[0]{\mathbf{n}}
\newcommand{\Bo}[0]{\mathbf{o}}
\newcommand{\Bp}[0]{\mathbf{p}}
\newcommand{\Bq}[0]{\mathbf{q}}
\newcommand{\Br}[0]{\mathbf{r}}
\newcommand{\Bs}[0]{\mathbf{s}}
\newcommand{\Bt}[0]{\mathbf{t}}
\newcommand{\Bu}[0]{\mathbf{u}}
\newcommand{\Bv}[0]{\mathbf{v}}
\newcommand{\Bw}[0]{\mathbf{w}}
\newcommand{\Bx}[0]{\mathbf{x}}
\newcommand{\By}[0]{\mathbf{y}}
\newcommand{\Bz}[0]{\mathbf{z}}
\newcommand{\BA}[0]{\mathbf{A}}
\newcommand{\BB}[0]{\mathbf{B}}
\newcommand{\BC}[0]{\mathbf{C}}
\newcommand{\BD}[0]{\mathbf{D}}
\newcommand{\BE}[0]{\mathbf{E}}
\newcommand{\BF}[0]{\mathbf{F}}
\newcommand{\BG}[0]{\mathbf{G}}
\newcommand{\BH}[0]{\mathbf{H}}
\newcommand{\BI}[0]{\mathbf{I}}
\newcommand{\BJ}[0]{\mathbf{J}}
\newcommand{\BK}[0]{\mathbf{K}}
\newcommand{\BL}[0]{\mathbf{L}}
\newcommand{\BM}[0]{\mathbf{M}}
\newcommand{\BN}[0]{\mathbf{N}}
\newcommand{\BO}[0]{\mathbf{O}}
\newcommand{\BP}[0]{\mathbf{P}}
\newcommand{\BQ}[0]{\mathbf{Q}}
\newcommand{\BR}[0]{\mathbf{R}}
\newcommand{\BS}[0]{\mathbf{S}}
\newcommand{\BT}[0]{\mathbf{T}}
\newcommand{\BU}[0]{\mathbf{U}}
\newcommand{\BV}[0]{\mathbf{V}}
\newcommand{\BW}[0]{\mathbf{W}}
\newcommand{\BX}[0]{\mathbf{X}}
\newcommand{\BY}[0]{\mathbf{Y}}
\newcommand{\BZ}[0]{\mathbf{Z}}

\newcommand{\Bzero}[0]{\mathbf{0}}
\newcommand{\Btheta}[0]{\boldsymbol{\theta}}
\newcommand{\Btau}[0]{\boldsymbol{\tau}}
\newcommand{\Bomega}[0]{\boldsymbol{\omega}}

%
% shorthand for unit vectors
%
\newcommand{\acap}[0]{\hat{\Ba}}
\newcommand{\bcap}[0]{\hat{\Bb}}
\newcommand{\ccap}[0]{\hat{\Bc}}
\newcommand{\dcap}[0]{\hat{\Bd}}
\newcommand{\ecap}[0]{\hat{\Be}}
\newcommand{\fcap}[0]{\hat{\Bf}}
\newcommand{\gcap}[0]{\hat{\Bg}}
\newcommand{\hcap}[0]{\hat{\Bh}}
\newcommand{\icap}[0]{\hat{\Bi}}
\newcommand{\jcap}[0]{\hat{\Bj}}
\newcommand{\kcap}[0]{\hat{\Bk}}
\newcommand{\lcap}[0]{\hat{\Bl}}
\newcommand{\mcap}[0]{\hat{\Bm}}
\newcommand{\ncap}[0]{\hat{\Bn}}
\newcommand{\ocap}[0]{\hat{\Bo}}
\newcommand{\pcap}[0]{\hat{\Bp}}
\newcommand{\qcap}[0]{\hat{\Bq}}
\newcommand{\rcap}[0]{\hat{\Br}}
\newcommand{\scap}[0]{\hat{\Bs}}
\newcommand{\tcap}[0]{\hat{\Bt}}
\newcommand{\ucap}[0]{\hat{\Bu}}
\newcommand{\vcap}[0]{\hat{\Bv}}
\newcommand{\wcap}[0]{\hat{\Bw}}
\newcommand{\xcap}[0]{\hat{\Bx}}
\newcommand{\ycap}[0]{\hat{\By}}
\newcommand{\zcap}[0]{\hat{\Bz}}
\newcommand{\thetacap}[0]{\hat{\Btheta}}

%
% to write R^n and C^n in a distinguishable fashion.  Perhaps change this
% to the double lined characters upon figuring out how to do so.
%
\newcommand{\C}[1]{$\mathbb{C}^{#1}$}
\newcommand{\R}[1]{$\mathbb{R}^{#1}$}

%
% various generally useful helpers
%

% derivative of #1 wrt. #2:
\newcommand{\D}[2] {\frac {d#2} {d#1}}

\newcommand{\inv}[1]{\frac{1}{#1}}
\newcommand{\cross}[0]{\times}

\newcommand{\abs}[1]{\lvert{#1}\rvert}
\newcommand{\norm}[1]{\lVert{#1}\rVert}
\newcommand{\innerprod}[2]{\langle{#1}, {#2}\rangle}
\newcommand{\dotprod}[2]{{#1} \cdot {#2}}
\newcommand{\bdotprod}[2]{\left({#1} \cdot {#2}\right)}
\newcommand{\crossprod}[2]{{#1} \cross {#2}}
\newcommand{\tripleprod}[3]{\dotprod{\left(\crossprod{#1}{#2}\right)}{#3}}

\DeclareMathOperator{\Proj}{Proj}
\DeclareMathOperator{\Span}{span}
\DeclareMathOperator{\Sgn}{sgn}
\DeclareMathOperator{\Area}{Area}
\DeclareMathOperator{\Volume}{Volume}

%
% A few miscellaneous things specific to this document
%
\newcommand{\crossop}[1]{\crossprod{#1}{}}

% R2 vector.
\newcommand{\VectorTwo}[2]{
\begin{bmatrix}
 {#1} \\
 {#2}
\end{bmatrix}
}

\newcommand{\VectorN}[1]{
\begin{bmatrix}
{#1}_1 \\
{#1}_2 \\
\vdots \\
{#1}_N \\
\end{bmatrix}
}

\newcommand{\DETuvij}[4]{
\begin{vmatrix}
 {#1}_{#3} & {#1}_{#4} \\
 {#2}_{#3} & {#2}_{#4}
\end{vmatrix}
}

\newcommand{\DETuvwijk}[6]{
\begin{vmatrix}
 {#1}_{#4} & {#1}_{#5} & {#1}_{#6} \\
 {#2}_{#4} & {#2}_{#5} & {#2}_{#6} \\
 {#3}_{#4} & {#3}_{#5} & {#3}_{#6}
\end{vmatrix}
}

\newcommand{\DETuvwxijkl}[8]{
\begin{vmatrix}
 {#1}_{#5} & {#1}_{#6} & {#1}_{#7} & {#1}_{#8} \\
 {#2}_{#5} & {#2}_{#6} & {#2}_{#7} & {#2}_{#8} \\
 {#3}_{#5} & {#3}_{#6} & {#3}_{#7} & {#3}_{#8} \\
 {#4}_{#5} & {#4}_{#6} & {#4}_{#7} & {#4}_{#8} \\
\end{vmatrix}
}

%\newcommand{\DETuvwxyijklm}[10]{
%\begin{vmatrix}
% {#1}_{#6} & {#1}_{#7} & {#1}_{#8} & {#1}_{#9} & {#1}_{#10} \\
% {#2}_{#6} & {#2}_{#7} & {#2}_{#8} & {#2}_{#9} & {#2}_{#10} \\
% {#3}_{#6} & {#3}_{#7} & {#3}_{#8} & {#3}_{#9} & {#3}_{#10} \\
% {#4}_{#6} & {#4}_{#7} & {#4}_{#8} & {#4}_{#9} & {#4}_{#10} \\
% {#5}_{#6} & {#5}_{#7} & {#5}_{#8} & {#5}_{#9} & {#5}_{#10}
%\end{vmatrix}
%}

% R3 vector.
\newcommand{\VectorThree}[3]{
\begin{bmatrix}
 {#1} \\
 {#2} \\
 {#3}
\end{bmatrix}
}


%<misc>
%
\newcommand{\Abs}[1]{{\left\lvert{#1}\right\rvert}}
\newcommand{\spacegrad}[0]{\boldsymbol{\nabla}}
\newcommand{\grad}[0]{\nabla}
\newcommand{\LL}[0]{\mathcal{L}}

% == \partial_{#1} {#2}
\newcommand{\PD}[2]{\frac{\partial {#2}}{\partial {#1}}}
% inline variant
\newcommand{\PDi}[2]{{\partial {#2}}/{\partial {#1}}}

\newcommand{\PDD}[3]{\frac{\partial^2 {#3}}{\partial {#1}\partial {#2}}}
%\newcommand{\PDd}[2]{\frac{\partial^2 {#2}}{{\partial{#1}}^2}}
\newcommand{\PDsq}[2]{\frac{\partial^2 {#2}}{(\partial {#1})^2}}

\newcommand{\Partial}[2]{\frac{\partial {#1}}{\partial {#2}}}
\DeclareMathOperator{\RejName}{Rej}
\newcommand{\Rej}[2]{\RejName_{#1}\left( {#2} \right)}
\newcommand{\Rm}[1]{\mathbb{R}^{#1}}
\newcommand{\Cm}[1]{\mathbb{C}^{#1}}
\newcommand{\conj}[0]{{*}}

%</misc>

% <grade selection>
%
\newcommand{\gpgrade}[2] {{\left\langle{{#1}}\right\rangle}_{#2}}

\newcommand{\gpgradezero}[1] {\gpgrade{#1}{}}
%\newcommand{\gpscalargrade}[1] {{\left\langle{{#1}}\right\rangle}}
%\newcommand{\gpgradezero}[1] {\gpgrade{#1}{0}}

%\newcommand{\gpgradeone}[1] {{\left\langle{{#1}}\right\rangle}_{1}}
\newcommand{\gpgradeone}[1] {\gpgrade{#1}{1}}

\newcommand{\gpgradetwo}[1] {\gpgrade{#1}{2}}
\newcommand{\gpgradethree}[1] {\gpgrade{#1}{3}}
\newcommand{\gpgradefour}[1] {\gpgrade{#1}{4}}
%
% </grade selection>



\newcommand{\adot}[0]{{\dot{a}}}
\newcommand{\bdot}[0]{{\dot{b}}}
% taken for centered dot:
%\newcommand{\cdot}[0]{{\dot{c}}}
%\newcommand{\ddot}[0]{{\dot{d}}}
\newcommand{\edot}[0]{{\dot{e}}}
\newcommand{\fdot}[0]{{\dot{f}}}
\newcommand{\gdot}[0]{{\dot{g}}}
\newcommand{\hdot}[0]{{\dot{h}}}
\newcommand{\idot}[0]{{\dot{i}}}
\newcommand{\jdot}[0]{{\dot{j}}}
\newcommand{\kdot}[0]{{\dot{k}}}
\newcommand{\ldot}[0]{{\dot{l}}}
\newcommand{\mdot}[0]{{\dot{m}}}
\newcommand{\ndot}[0]{{\dot{n}}}
%\newcommand{\odot}[0]{{\dot{o}}}
\newcommand{\pdot}[0]{{\dot{p}}}
\newcommand{\qdot}[0]{{\dot{q}}}
\newcommand{\rdot}[0]{{\dot{r}}}
\newcommand{\sdot}[0]{{\dot{s}}}
\newcommand{\tdot}[0]{{\dot{t}}}
\newcommand{\udot}[0]{{\dot{u}}}
\newcommand{\vdot}[0]{{\dot{v}}}
\newcommand{\wdot}[0]{{\dot{w}}}
\newcommand{\xdot}[0]{{\dot{x}}}
\newcommand{\ydot}[0]{{\dot{y}}}
\newcommand{\zdot}[0]{{\dot{z}}}
\newcommand{\addot}[0]{{\ddot{a}}}
\newcommand{\bddot}[0]{{\ddot{b}}}
\newcommand{\cddot}[0]{{\ddot{c}}}
%\newcommand{\dddot}[0]{{\ddot{d}}}
\newcommand{\eddot}[0]{{\ddot{e}}}
\newcommand{\fddot}[0]{{\ddot{f}}}
\newcommand{\gddot}[0]{{\ddot{g}}}
\newcommand{\hddot}[0]{{\ddot{h}}}
\newcommand{\iddot}[0]{{\ddot{i}}}
\newcommand{\jddot}[0]{{\ddot{j}}}
\newcommand{\kddot}[0]{{\ddot{k}}}
\newcommand{\lddot}[0]{{\ddot{l}}}
\newcommand{\mddot}[0]{{\ddot{m}}}
\newcommand{\nddot}[0]{{\ddot{n}}}
\newcommand{\oddot}[0]{{\ddot{o}}}
\newcommand{\pddot}[0]{{\ddot{p}}}
\newcommand{\qddot}[0]{{\ddot{q}}}
\newcommand{\rddot}[0]{{\ddot{r}}}
\newcommand{\sddot}[0]{{\ddot{s}}}
\newcommand{\tddot}[0]{{\ddot{t}}}
\newcommand{\uddot}[0]{{\ddot{u}}}
\newcommand{\vddot}[0]{{\ddot{v}}}
\newcommand{\wddot}[0]{{\ddot{w}}}
\newcommand{\xddot}[0]{{\ddot{x}}}
\newcommand{\yddot}[0]{{\ddot{y}}}
\newcommand{\zddot}[0]{{\ddot{z}}}

%<bold and dot greek symbols>
%

\newcommand{\Deltadot}[0]{{\dot{\Delta}}}
\newcommand{\Gammadot}[0]{{\dot{\Gamma}}}
\newcommand{\Lambdadot}[0]{{\dot{\Lambda}}}
\newcommand{\Omegadot}[0]{{\dot{\Omega}}}
\newcommand{\Phidot}[0]{{\dot{\Phi}}}
\newcommand{\Pidot}[0]{{\dot{\Pi}}}
\newcommand{\Psidot}[0]{{\dot{\Psi}}}
\newcommand{\Sigmadot}[0]{{\dot{\Sigma}}}
\newcommand{\Thetadot}[0]{{\dot{\Theta}}}
\newcommand{\Upsilondot}[0]{{\dot{\Upsilon}}}
\newcommand{\Xidot}[0]{{\dot{\Xi}}}
\newcommand{\alphadot}[0]{{\dot{\alpha}}}
\newcommand{\betadot}[0]{{\dot{\beta}}}
\newcommand{\chidot}[0]{{\dot{\chi}}}
\newcommand{\deltadot}[0]{{\dot{\delta}}}
\newcommand{\epsilondot}[0]{{\dot{\epsilon}}}
\newcommand{\etadot}[0]{{\dot{\eta}}}
\newcommand{\gammadot}[0]{{\dot{\gamma}}}
\newcommand{\kappadot}[0]{{\dot{\kappa}}}
\newcommand{\lambdadot}[0]{{\dot{\lambda}}}
\newcommand{\mudot}[0]{{\dot{\mu}}}
\newcommand{\nudot}[0]{{\dot{\nu}}}
\newcommand{\omegadot}[0]{{\dot{\omega}}}
\newcommand{\phidot}[0]{{\dot{\phi}}}
\newcommand{\pidot}[0]{{\dot{\pi}}}
\newcommand{\psidot}[0]{{\dot{\psi}}}
\newcommand{\rhodot}[0]{{\dot{\rho}}}
\newcommand{\sigmadot}[0]{{\dot{\sigma}}}
\newcommand{\taudot}[0]{{\dot{\tau}}}
\newcommand{\thetadot}[0]{{\dot{\theta}}}
\newcommand{\upsilondot}[0]{{\dot{\upsilon}}}
\newcommand{\varepsilondot}[0]{{\dot{\varepsilon}}}
\newcommand{\varphidot}[0]{{\dot{\varphi}}}
\newcommand{\varpidot}[0]{{\dot{\varpi}}}
\newcommand{\varrhodot}[0]{{\dot{\varrho}}}
\newcommand{\varsigmadot}[0]{{\dot{\varsigma}}}
\newcommand{\varthetadot}[0]{{\dot{\vartheta}}}
\newcommand{\xidot}[0]{{\dot{\xi}}}
\newcommand{\zetadot}[0]{{\dot{\zeta}}}

\newcommand{\Deltaddot}[0]{{\ddot{\Delta}}}
\newcommand{\Gammaddot}[0]{{\ddot{\Gamma}}}
\newcommand{\Lambdaddot}[0]{{\ddot{\Lambda}}}
\newcommand{\Omegaddot}[0]{{\ddot{\Omega}}}
\newcommand{\Phiddot}[0]{{\ddot{\Phi}}}
\newcommand{\Piddot}[0]{{\ddot{\Pi}}}
\newcommand{\Psiddot}[0]{{\ddot{\Psi}}}
\newcommand{\Sigmaddot}[0]{{\ddot{\Sigma}}}
\newcommand{\Thetaddot}[0]{{\ddot{\Theta}}}
\newcommand{\Upsilonddot}[0]{{\ddot{\Upsilon}}}
\newcommand{\Xiddot}[0]{{\ddot{\Xi}}}
\newcommand{\alphaddot}[0]{{\ddot{\alpha}}}
\newcommand{\betaddot}[0]{{\ddot{\beta}}}
\newcommand{\chiddot}[0]{{\ddot{\chi}}}
\newcommand{\deltaddot}[0]{{\ddot{\delta}}}
\newcommand{\epsilonddot}[0]{{\ddot{\epsilon}}}
\newcommand{\etaddot}[0]{{\ddot{\eta}}}
\newcommand{\gammaddot}[0]{{\ddot{\gamma}}}
\newcommand{\kappaddot}[0]{{\ddot{\kappa}}}
\newcommand{\lambdaddot}[0]{{\ddot{\lambda}}}
\newcommand{\muddot}[0]{{\ddot{\mu}}}
\newcommand{\nuddot}[0]{{\ddot{\nu}}}
\newcommand{\omegaddot}[0]{{\ddot{\omega}}}
\newcommand{\phiddot}[0]{{\ddot{\phi}}}
\newcommand{\piddot}[0]{{\ddot{\pi}}}
\newcommand{\psiddot}[0]{{\ddot{\psi}}}
\newcommand{\rhoddot}[0]{{\ddot{\rho}}}
\newcommand{\sigmaddot}[0]{{\ddot{\sigma}}}
\newcommand{\tauddot}[0]{{\ddot{\tau}}}
\newcommand{\thetaddot}[0]{{\ddot{\theta}}}
\newcommand{\upsilonddot}[0]{{\ddot{\upsilon}}}
\newcommand{\varepsilonddot}[0]{{\ddot{\varepsilon}}}
\newcommand{\varphiddot}[0]{{\ddot{\varphi}}}
\newcommand{\varpiddot}[0]{{\ddot{\varpi}}}
\newcommand{\varrhoddot}[0]{{\ddot{\varrho}}}
\newcommand{\varsigmaddot}[0]{{\ddot{\varsigma}}}
\newcommand{\varthetaddot}[0]{{\ddot{\vartheta}}}
\newcommand{\xiddot}[0]{{\ddot{\xi}}}
\newcommand{\zetaddot}[0]{{\ddot{\zeta}}}

\newcommand{\BDelta}[0]{\boldsymbol{\Delta}}
\newcommand{\BGamma}[0]{\boldsymbol{\Gamma}}
\newcommand{\BLambda}[0]{\boldsymbol{\Lambda}}
\newcommand{\BOmega}[0]{\boldsymbol{\Omega}}
\newcommand{\BPhi}[0]{\boldsymbol{\Phi}}
\newcommand{\BPi}[0]{\boldsymbol{\Pi}}
\newcommand{\BPsi}[0]{\boldsymbol{\Psi}}
\newcommand{\BSigma}[0]{\boldsymbol{\Sigma}}
\newcommand{\BTheta}[0]{\boldsymbol{\Theta}}
\newcommand{\BUpsilon}[0]{\boldsymbol{\Upsilon}}
\newcommand{\BXi}[0]{\boldsymbol{\Xi}}
\newcommand{\Balpha}[0]{\boldsymbol{\alpha}}
\newcommand{\Bbeta}[0]{\boldsymbol{\beta}}
\newcommand{\Bchi}[0]{\boldsymbol{\chi}}
\newcommand{\Bdelta}[0]{\boldsymbol{\delta}}
\newcommand{\Bepsilon}[0]{\boldsymbol{\epsilon}}
\newcommand{\Beta}[0]{\boldsymbol{\eta}}
\newcommand{\Bgamma}[0]{\boldsymbol{\gamma}}
\newcommand{\Bkappa}[0]{\boldsymbol{\kappa}}
\newcommand{\Blambda}[0]{\boldsymbol{\lambda}}
\newcommand{\Bmu}[0]{\boldsymbol{\mu}}
\newcommand{\Bnu}[0]{\boldsymbol{\nu}}
%\newcommand{\Bomega}[0]{\boldsymbol{\omega}}
\newcommand{\Bphi}[0]{\boldsymbol{\phi}}
\newcommand{\Bpi}[0]{\boldsymbol{\pi}}
\newcommand{\Bpsi}[0]{\boldsymbol{\psi}}
\newcommand{\Brho}[0]{\boldsymbol{\rho}}
\newcommand{\Bsigma}[0]{\boldsymbol{\sigma}}
%\newcommand{\Btau}[0]{\boldsymbol{\tau}}
%\newcommand{\Btheta}[0]{\boldsymbol{\theta}}
\newcommand{\Bupsilon}[0]{\boldsymbol{\upsilon}}
\newcommand{\Bvarepsilon}[0]{\boldsymbol{\varepsilon}}
\newcommand{\Bvarphi}[0]{\boldsymbol{\varphi}}
\newcommand{\Bvarpi}[0]{\boldsymbol{\varpi}}
\newcommand{\Bvarrho}[0]{\boldsymbol{\varrho}}
\newcommand{\Bvarsigma}[0]{\boldsymbol{\varsigma}}
\newcommand{\Bvartheta}[0]{\boldsymbol{\vartheta}}
\newcommand{\Bxi}[0]{\boldsymbol{\xi}}
\newcommand{\Bzeta}[0]{\boldsymbol{\zeta}}
%
%</bold and dot greek symbols>
%<infrequent>
%
%\newcommand{\AreaOp}[1]{\AName_{#1}}
%\newcommand{\Babs}[0]{\abs{\BB}}
%\newcommand{\Bcap}[0]{\hat{\BB}}
%\newcommand{\BrPrimeRej}[0]{\rcap(\rcap \wedge \Br')}
%\newcommand{\CA}[0]{\mathcal{A}}
%\newcommand{\Cos}[1]{\cos{\left({#1}\right)}}
%\newcommand{\Det}[1] {\abs{#1}}
%\newcommand{\Dsq}[2] {\frac {\partial^2 {#1}} {\partial {#2}^2}}
%\newcommand{\Exp}[1]{\exp{\left({#1}\right)}}
%\newcommand{\Norm}[1]{\left\lVert{#1}\right\rVert}
%\newcommand{\Sin}[1]{\sin{\left({#1}\right)}}
%\newcommand{\T}[0]{\text{T}}
%\newcommand{\VolumeOp}[1]{\VName_{#1}}
%\newcommand{\agrad}[0]{\Ba \cdot \nabla}
%\newcommand{\alphacap}[0]{\hat{\boldsymbol{\alpha}}}
%\newcommand{\Fcap}[0]{\hat{\BF}}
%\newcommand{\bithree}[0]{{\Bi}_3}
%\newcommand{\bxa}[0]{\Bx\Ba}
%\newcommand{\coordvec}[2]{
%\newcommand{\costheta}[0]{\acap \cdot \xcap}
%\newcommand{\ddt}[1]{\ddot{#1}}
%\newcommand{\ddu}[1] {\frac {d{#1}} {du}}
%\newcommand{\dsqxj}[2] {\frac {\partial^2 {#1}} {\partial {x_{#2}}^2}}
%\newcommand{\dtheta}[1]{\frac{d {#1}}{d \theta}}
%\newcommand{\dt}[1]{\dot{#1}}
%\newcommand{\dt}[1]{\frac{d {#1}}{dt}}
%\newcommand{\dxj}[2] {\frac {\partial {#1}} {\partial {x_{#2}}}}
%\newcommand{\halfPhi}[0]{\frac{\phi}{2}}
%\newcommand{\half}[0]{\inv{2}}
%\newcommand{\inv}[1]{\frac{1}{#1}}
%\newcommand{\laplacian}[0]{\nabla^2}
%\newcommand{\matrixoftx}[3]{
%\newcommand{\nrrp}[0]{\norm{\rcap \wedge \Br'}}
%\newcommand{\oiint}{\bigcirc \hspace{-1.4em} \int \hspace{-.8em} \int}
%\newcommand{\transpose}[1]{{#1}^{\text{T}}}
%\newcommand{\transpose}[1]{{{#1}^{\TextTranspose}}}
%\newcommand{\transpose}[1]{{{#1}^{\text{T}}}}
%\newcommand{\barA}[0]{\bar{A}}
%\newcommand{\qbar}[0]{\bar{q}}
%\newcommand{\qdotbar}[0]{\dot{\bar{q}}}
%
%</infrequent>





%\usepackage{listings}
%\usepackage{txfonts} % for ointctr... (also appears to make "prettier" \int and \sum's)
\usepackage[bookmarks=true]{hyperref}

\usepackage{color,cite,graphicx}
   % use colour in the document, put your citations as [1-4]
   % rather than [1,2,3,4] (it looks nicer, and the extended LaTeX2e
   % graphics package. 
\usepackage{latexsym,amssymb,epsf} % don't remember if these are
   % needed, but their inclusion can't do any damage


%\title{}
%\author{Peeter Joot \quad peeter.joot@gmail.com }
%\date{ March dd, 2009.  Last Revision: $Date: 2009/04/10 04:43:43 $ }

\begin{document}

%\maketitle{}
%
%\tableofcontents
%
%\section{}

%\begin{figure}[htp]
%\centering
%\includegraphics[totalheight=0.4\textheight]{picturepath}
%\caption{My Caption}\label{fig:pictlabel}
%\end{figure}
%
%... see figure \ref{fig:picturepath} ...

%\bibliographystyle{plainnat}
%\bibliography{myrefs}

I don't know if you've seen it, but Feynman's lectures (volume II in what he calls the "entertainment" chapter) 
has some nice simple examples of these ideas.
They cut the complexity out of the picture and treat some simple concrete cases, like the following

\begin{align*}
S = \int_{a}^b \dot{y}^2 dx
\end{align*}

Let 

\begin{align*}
y = \bar{y} + \alpha
\end{align*}

Here $\bar{y}$ is the solution that you are looking for, and $\alpha(a) = \alpha(b) = 0$ (vanishes at the end points).

Taking derivatives
\begin{align*}
\dot{y} &= \dot{\bar{y}} + \dot{\alpha}
\end{align*}

\begin{align*}
S 
&= \int_{a}^b \left(\dot{\bar{y}} + \dot{\alpha} \right)^2 dx \\
&= \int_{a}^b \left(\dot{\bar{y}}^2 + 2 \dot{\alpha}\dot{\bar{y}} + \dot{\alpha}^2  \right) dx \\
&= \int_{a}^b \left(\bar{\dot{y}}^2 - 2 {\alpha}\ddot{\bar{y}} - {\alpha}\ddot{\alpha}  \right) dx \\
\end{align*}

and for the derivative to be zero
\begin{align*}
0 &= \left. \frac{dS }{d\alpha} \right\vert_{\alpha = 0} \\
&= -2 \int_{a}^b \ddot{\bar{y}} dx \\
\end{align*}

Dropping the overbar, you have for the desired solution to the variation

\begin{align*}
\ddot{y} = 0
\end{align*}

(particle moves with constant velocity in absence of force).

\end{document}

%\documentclass{article}

%\usepackage{amsmath}
\usepackage{mathpazo}

%
% shorthand for bold symbols, convenient for vectors and matrices
%
\newcommand{\Ba}[0]{\mathbf{a}}
\newcommand{\Bb}[0]{\mathbf{b}}
\newcommand{\Bc}[0]{\mathbf{c}}
\newcommand{\Bd}[0]{\mathbf{d}}
\newcommand{\Be}[0]{\mathbf{e}}
\newcommand{\Bf}[0]{\mathbf{f}}
\newcommand{\Bg}[0]{\mathbf{g}}
\newcommand{\Bh}[0]{\mathbf{h}}
\newcommand{\Bi}[0]{\mathbf{i}}
\newcommand{\Bj}[0]{\mathbf{j}}
\newcommand{\Bk}[0]{\mathbf{k}}
\newcommand{\Bl}[0]{\mathbf{l}}
\newcommand{\Bm}[0]{\mathbf{m}}
\newcommand{\Bn}[0]{\mathbf{n}}
\newcommand{\Bo}[0]{\mathbf{o}}
\newcommand{\Bp}[0]{\mathbf{p}}
\newcommand{\Bq}[0]{\mathbf{q}}
\newcommand{\Br}[0]{\mathbf{r}}
\newcommand{\Bs}[0]{\mathbf{s}}
\newcommand{\Bt}[0]{\mathbf{t}}
\newcommand{\Bu}[0]{\mathbf{u}}
\newcommand{\Bv}[0]{\mathbf{v}}
\newcommand{\Bw}[0]{\mathbf{w}}
\newcommand{\Bx}[0]{\mathbf{x}}
\newcommand{\By}[0]{\mathbf{y}}
\newcommand{\Bz}[0]{\mathbf{z}}
\newcommand{\BA}[0]{\mathbf{A}}
\newcommand{\BB}[0]{\mathbf{B}}
\newcommand{\BC}[0]{\mathbf{C}}
\newcommand{\BD}[0]{\mathbf{D}}
\newcommand{\BE}[0]{\mathbf{E}}
\newcommand{\BF}[0]{\mathbf{F}}
\newcommand{\BG}[0]{\mathbf{G}}
\newcommand{\BH}[0]{\mathbf{H}}
\newcommand{\BI}[0]{\mathbf{I}}
\newcommand{\BJ}[0]{\mathbf{J}}
\newcommand{\BK}[0]{\mathbf{K}}
\newcommand{\BL}[0]{\mathbf{L}}
\newcommand{\BM}[0]{\mathbf{M}}
\newcommand{\BN}[0]{\mathbf{N}}
\newcommand{\BO}[0]{\mathbf{O}}
\newcommand{\BP}[0]{\mathbf{P}}
\newcommand{\BQ}[0]{\mathbf{Q}}
\newcommand{\BR}[0]{\mathbf{R}}
\newcommand{\BS}[0]{\mathbf{S}}
\newcommand{\BT}[0]{\mathbf{T}}
\newcommand{\BU}[0]{\mathbf{U}}
\newcommand{\BV}[0]{\mathbf{V}}
\newcommand{\BW}[0]{\mathbf{W}}
\newcommand{\BX}[0]{\mathbf{X}}
\newcommand{\BY}[0]{\mathbf{Y}}
\newcommand{\BZ}[0]{\mathbf{Z}}

\newcommand{\Bzero}[0]{\mathbf{0}}
\newcommand{\Btheta}[0]{\boldsymbol{\theta}}
\newcommand{\Btau}[0]{\boldsymbol{\tau}}
\newcommand{\Bomega}[0]{\boldsymbol{\omega}}

%
% shorthand for unit vectors
%
\newcommand{\acap}[0]{\hat{\Ba}}
\newcommand{\bcap}[0]{\hat{\Bb}}
\newcommand{\ccap}[0]{\hat{\Bc}}
\newcommand{\dcap}[0]{\hat{\Bd}}
\newcommand{\ecap}[0]{\hat{\Be}}
\newcommand{\fcap}[0]{\hat{\Bf}}
\newcommand{\gcap}[0]{\hat{\Bg}}
\newcommand{\hcap}[0]{\hat{\Bh}}
\newcommand{\icap}[0]{\hat{\Bi}}
\newcommand{\jcap}[0]{\hat{\Bj}}
\newcommand{\kcap}[0]{\hat{\Bk}}
\newcommand{\lcap}[0]{\hat{\Bl}}
\newcommand{\mcap}[0]{\hat{\Bm}}
\newcommand{\ncap}[0]{\hat{\Bn}}
\newcommand{\ocap}[0]{\hat{\Bo}}
\newcommand{\pcap}[0]{\hat{\Bp}}
\newcommand{\qcap}[0]{\hat{\Bq}}
\newcommand{\rcap}[0]{\hat{\Br}}
\newcommand{\scap}[0]{\hat{\Bs}}
\newcommand{\tcap}[0]{\hat{\Bt}}
\newcommand{\ucap}[0]{\hat{\Bu}}
\newcommand{\vcap}[0]{\hat{\Bv}}
\newcommand{\wcap}[0]{\hat{\Bw}}
\newcommand{\xcap}[0]{\hat{\Bx}}
\newcommand{\ycap}[0]{\hat{\By}}
\newcommand{\zcap}[0]{\hat{\Bz}}
\newcommand{\thetacap}[0]{\hat{\Btheta}}

%
% to write R^n and C^n in a distinguishable fashion.  Perhaps change this
% to the double lined characters upon figuring out how to do so.
%
\newcommand{\C}[1]{$\mathbb{C}^{#1}$}
\newcommand{\R}[1]{$\mathbb{R}^{#1}$}

%
% various generally useful helpers
%

% derivative of #1 wrt. #2:
\newcommand{\D}[2] {\frac {d#2} {d#1}}

\newcommand{\inv}[1]{\frac{1}{#1}}
\newcommand{\cross}[0]{\times}

\newcommand{\abs}[1]{\lvert{#1}\rvert}
\newcommand{\norm}[1]{\lVert{#1}\rVert}
\newcommand{\innerprod}[2]{\langle{#1}, {#2}\rangle}
\newcommand{\dotprod}[2]{{#1} \cdot {#2}}
\newcommand{\bdotprod}[2]{\left({#1} \cdot {#2}\right)}
\newcommand{\crossprod}[2]{{#1} \cross {#2}}
\newcommand{\tripleprod}[3]{\dotprod{\left(\crossprod{#1}{#2}\right)}{#3}}

\DeclareMathOperator{\Proj}{Proj}
\DeclareMathOperator{\Span}{span}
\DeclareMathOperator{\Sgn}{sgn}
\DeclareMathOperator{\Area}{Area}
\DeclareMathOperator{\Volume}{Volume}

%
% A few miscellaneous things specific to this document
%
\newcommand{\crossop}[1]{\crossprod{#1}{}}

% R2 vector.
\newcommand{\VectorTwo}[2]{
\begin{bmatrix}
 {#1} \\
 {#2}
\end{bmatrix}
}

\newcommand{\VectorN}[1]{
\begin{bmatrix}
{#1}_1 \\
{#1}_2 \\
\vdots \\
{#1}_N \\
\end{bmatrix}
}

\newcommand{\DETuvij}[4]{
\begin{vmatrix}
 {#1}_{#3} & {#1}_{#4} \\
 {#2}_{#3} & {#2}_{#4}
\end{vmatrix}
}

\newcommand{\DETuvwijk}[6]{
\begin{vmatrix}
 {#1}_{#4} & {#1}_{#5} & {#1}_{#6} \\
 {#2}_{#4} & {#2}_{#5} & {#2}_{#6} \\
 {#3}_{#4} & {#3}_{#5} & {#3}_{#6}
\end{vmatrix}
}

\newcommand{\DETuvwxijkl}[8]{
\begin{vmatrix}
 {#1}_{#5} & {#1}_{#6} & {#1}_{#7} & {#1}_{#8} \\
 {#2}_{#5} & {#2}_{#6} & {#2}_{#7} & {#2}_{#8} \\
 {#3}_{#5} & {#3}_{#6} & {#3}_{#7} & {#3}_{#8} \\
 {#4}_{#5} & {#4}_{#6} & {#4}_{#7} & {#4}_{#8} \\
\end{vmatrix}
}

%\newcommand{\DETuvwxyijklm}[10]{
%\begin{vmatrix}
% {#1}_{#6} & {#1}_{#7} & {#1}_{#8} & {#1}_{#9} & {#1}_{#10} \\
% {#2}_{#6} & {#2}_{#7} & {#2}_{#8} & {#2}_{#9} & {#2}_{#10} \\
% {#3}_{#6} & {#3}_{#7} & {#3}_{#8} & {#3}_{#9} & {#3}_{#10} \\
% {#4}_{#6} & {#4}_{#7} & {#4}_{#8} & {#4}_{#9} & {#4}_{#10} \\
% {#5}_{#6} & {#5}_{#7} & {#5}_{#8} & {#5}_{#9} & {#5}_{#10}
%\end{vmatrix}
%}

% R3 vector.
\newcommand{\VectorThree}[3]{
\begin{bmatrix}
 {#1} \\
 {#2} \\
 {#3}
\end{bmatrix}
}


%\usepackage{amsmath}

\chapter{pf sch}
\date{ October 14, 2008.  $RCSfile: pfSch.tex,v $ Last $Revision: 1.4 $ $Date: 2009/06/11 17:00:37 $ }

%\begin{document}

\begin{align*}
L 
&= \frac{h^2}{8 \pi^2 m} \nabla \psi \cdot \nabla \psi^{*} + V \psi \psi^{*} + \frac{h}{2 \pi i} ( \psi^{*} \partial_t \psi - \psi \partial_t \psi^{*} ) \\
&= \frac{h^2}{8 \pi^2 m} \partial_k \psi \partial_k \psi^{*} + V \psi \psi^{*} + 
\frac{h}{2 \pi i} ( \psi^{*} \partial_t \psi - \psi \partial_t \psi^{*} ) \\
\end{align*}

We have
\begin{align*}
\frac{\partial L}{\partial \psi^{*} } &= V\psi + \frac{h}{2 \pi i} \partial_t \psi \\
\end{align*}

and canonical momenta
\begin{align*}
\frac{\partial L}{\partial{(\partial_k \psi^{*})}} &= \frac{h^2}{8 \pi^2 m} \partial_{k} \psi \\
\frac{\partial L}{\partial{(\partial_t \psi^{*})}} &= -\frac{h}{2 \pi i} {\psi} \\
\end{align*}

\begin{align*}
\frac{\partial L}{\partial \psi^{*}} &= \sum_k \partial_k \frac{\partial L}{\partial{(\partial_k \psi^{*})}} + \partial_t \frac{\partial L}{\partial{(\partial_t \psi^{*})}} \\
V\psi + \frac{h}{2 \pi i} \partial_t \psi &= \frac{h^2}{8 \pi^2 m} \sum_k \partial_{kk} \psi -\frac{h}{2 \pi i} \frac{\partial \psi}{\partial{t}} \\
\end{align*}

which is off by a factor of two in the time term
\begin{align*}
-\frac{h^2}{8 \pi^2 m} \nabla^2 \psi + V\psi &= \frac{h i}{\pi} \frac{\partial \psi}{\partial{t}} \\
\end{align*}

%\end{document}               % End of document.

% 
% 
% 
% Copyright � 2012 Peeter Joot
% All Rights Reserved
% 
% This file may be reproduced and distributed in whole or in part, without fee, subject to the following conditions:
% 
% o The copyright notice above and this permission notice must be preserved complete on all complete or partial copies.
% 
% o Any translation or derived work must be approved by the author in writing before distribution.
% 
% o If you distribute this work in part, instructions for obtaining the complete version of this file must be included, and a means for obtaining a complete version provided.
% 
% 
% Exceptions to these rules may be granted for academic purposes: Write to the author and ask.
% 
% 
% 
%\documentclass{article}      % Specifies the document class

%\usepackage{amsmath}
\usepackage{mathpazo}

%
% shorthand for bold symbols, convenient for vectors and matrices
%
\newcommand{\Ba}[0]{\mathbf{a}}
\newcommand{\Bb}[0]{\mathbf{b}}
\newcommand{\Bc}[0]{\mathbf{c}}
\newcommand{\Bd}[0]{\mathbf{d}}
\newcommand{\Be}[0]{\mathbf{e}}
\newcommand{\Bf}[0]{\mathbf{f}}
\newcommand{\Bg}[0]{\mathbf{g}}
\newcommand{\Bh}[0]{\mathbf{h}}
\newcommand{\Bi}[0]{\mathbf{i}}
\newcommand{\Bj}[0]{\mathbf{j}}
\newcommand{\Bk}[0]{\mathbf{k}}
\newcommand{\Bl}[0]{\mathbf{l}}
\newcommand{\Bm}[0]{\mathbf{m}}
\newcommand{\Bn}[0]{\mathbf{n}}
\newcommand{\Bo}[0]{\mathbf{o}}
\newcommand{\Bp}[0]{\mathbf{p}}
\newcommand{\Bq}[0]{\mathbf{q}}
\newcommand{\Br}[0]{\mathbf{r}}
\newcommand{\Bs}[0]{\mathbf{s}}
\newcommand{\Bt}[0]{\mathbf{t}}
\newcommand{\Bu}[0]{\mathbf{u}}
\newcommand{\Bv}[0]{\mathbf{v}}
\newcommand{\Bw}[0]{\mathbf{w}}
\newcommand{\Bx}[0]{\mathbf{x}}
\newcommand{\By}[0]{\mathbf{y}}
\newcommand{\Bz}[0]{\mathbf{z}}
\newcommand{\BA}[0]{\mathbf{A}}
\newcommand{\BB}[0]{\mathbf{B}}
\newcommand{\BC}[0]{\mathbf{C}}
\newcommand{\BD}[0]{\mathbf{D}}
\newcommand{\BE}[0]{\mathbf{E}}
\newcommand{\BF}[0]{\mathbf{F}}
\newcommand{\BG}[0]{\mathbf{G}}
\newcommand{\BH}[0]{\mathbf{H}}
\newcommand{\BI}[0]{\mathbf{I}}
\newcommand{\BJ}[0]{\mathbf{J}}
\newcommand{\BK}[0]{\mathbf{K}}
\newcommand{\BL}[0]{\mathbf{L}}
\newcommand{\BM}[0]{\mathbf{M}}
\newcommand{\BN}[0]{\mathbf{N}}
\newcommand{\BO}[0]{\mathbf{O}}
\newcommand{\BP}[0]{\mathbf{P}}
\newcommand{\BQ}[0]{\mathbf{Q}}
\newcommand{\BR}[0]{\mathbf{R}}
\newcommand{\BS}[0]{\mathbf{S}}
\newcommand{\BT}[0]{\mathbf{T}}
\newcommand{\BU}[0]{\mathbf{U}}
\newcommand{\BV}[0]{\mathbf{V}}
\newcommand{\BW}[0]{\mathbf{W}}
\newcommand{\BX}[0]{\mathbf{X}}
\newcommand{\BY}[0]{\mathbf{Y}}
\newcommand{\BZ}[0]{\mathbf{Z}}

\newcommand{\Bzero}[0]{\mathbf{0}}
\newcommand{\Btheta}[0]{\boldsymbol{\theta}}
\newcommand{\Btau}[0]{\boldsymbol{\tau}}
\newcommand{\Bomega}[0]{\boldsymbol{\omega}}

%
% shorthand for unit vectors
%
\newcommand{\acap}[0]{\hat{\Ba}}
\newcommand{\bcap}[0]{\hat{\Bb}}
\newcommand{\ccap}[0]{\hat{\Bc}}
\newcommand{\dcap}[0]{\hat{\Bd}}
\newcommand{\ecap}[0]{\hat{\Be}}
\newcommand{\fcap}[0]{\hat{\Bf}}
\newcommand{\gcap}[0]{\hat{\Bg}}
\newcommand{\hcap}[0]{\hat{\Bh}}
\newcommand{\icap}[0]{\hat{\Bi}}
\newcommand{\jcap}[0]{\hat{\Bj}}
\newcommand{\kcap}[0]{\hat{\Bk}}
\newcommand{\lcap}[0]{\hat{\Bl}}
\newcommand{\mcap}[0]{\hat{\Bm}}
\newcommand{\ncap}[0]{\hat{\Bn}}
\newcommand{\ocap}[0]{\hat{\Bo}}
\newcommand{\pcap}[0]{\hat{\Bp}}
\newcommand{\qcap}[0]{\hat{\Bq}}
\newcommand{\rcap}[0]{\hat{\Br}}
\newcommand{\scap}[0]{\hat{\Bs}}
\newcommand{\tcap}[0]{\hat{\Bt}}
\newcommand{\ucap}[0]{\hat{\Bu}}
\newcommand{\vcap}[0]{\hat{\Bv}}
\newcommand{\wcap}[0]{\hat{\Bw}}
\newcommand{\xcap}[0]{\hat{\Bx}}
\newcommand{\ycap}[0]{\hat{\By}}
\newcommand{\zcap}[0]{\hat{\Bz}}
\newcommand{\thetacap}[0]{\hat{\Btheta}}

%
% to write R^n and C^n in a distinguishable fashion.  Perhaps change this
% to the double lined characters upon figuring out how to do so.
%
\newcommand{\C}[1]{$\mathbb{C}^{#1}$}
\newcommand{\R}[1]{$\mathbb{R}^{#1}$}

%
% various generally useful helpers
%

% derivative of #1 wrt. #2:
\newcommand{\D}[2] {\frac {d#2} {d#1}}

\newcommand{\inv}[1]{\frac{1}{#1}}
\newcommand{\cross}[0]{\times}

\newcommand{\abs}[1]{\lvert{#1}\rvert}
\newcommand{\norm}[1]{\lVert{#1}\rVert}
\newcommand{\innerprod}[2]{\langle{#1}, {#2}\rangle}
\newcommand{\dotprod}[2]{{#1} \cdot {#2}}
\newcommand{\bdotprod}[2]{\left({#1} \cdot {#2}\right)}
\newcommand{\crossprod}[2]{{#1} \cross {#2}}
\newcommand{\tripleprod}[3]{\dotprod{\left(\crossprod{#1}{#2}\right)}{#3}}

\DeclareMathOperator{\Proj}{Proj}
\DeclareMathOperator{\Span}{span}
\DeclareMathOperator{\Sgn}{sgn}
\DeclareMathOperator{\Area}{Area}
\DeclareMathOperator{\Volume}{Volume}

%
% A few miscellaneous things specific to this document
%
\newcommand{\crossop}[1]{\crossprod{#1}{}}

% R2 vector.
\newcommand{\VectorTwo}[2]{
\begin{bmatrix}
 {#1} \\
 {#2}
\end{bmatrix}
}

\newcommand{\VectorN}[1]{
\begin{bmatrix}
{#1}_1 \\
{#1}_2 \\
\vdots \\
{#1}_N \\
\end{bmatrix}
}

\newcommand{\DETuvij}[4]{
\begin{vmatrix}
 {#1}_{#3} & {#1}_{#4} \\
 {#2}_{#3} & {#2}_{#4}
\end{vmatrix}
}

\newcommand{\DETuvwijk}[6]{
\begin{vmatrix}
 {#1}_{#4} & {#1}_{#5} & {#1}_{#6} \\
 {#2}_{#4} & {#2}_{#5} & {#2}_{#6} \\
 {#3}_{#4} & {#3}_{#5} & {#3}_{#6}
\end{vmatrix}
}

\newcommand{\DETuvwxijkl}[8]{
\begin{vmatrix}
 {#1}_{#5} & {#1}_{#6} & {#1}_{#7} & {#1}_{#8} \\
 {#2}_{#5} & {#2}_{#6} & {#2}_{#7} & {#2}_{#8} \\
 {#3}_{#5} & {#3}_{#6} & {#3}_{#7} & {#3}_{#8} \\
 {#4}_{#5} & {#4}_{#6} & {#4}_{#7} & {#4}_{#8} \\
\end{vmatrix}
}

%\newcommand{\DETuvwxyijklm}[10]{
%\begin{vmatrix}
% {#1}_{#6} & {#1}_{#7} & {#1}_{#8} & {#1}_{#9} & {#1}_{#10} \\
% {#2}_{#6} & {#2}_{#7} & {#2}_{#8} & {#2}_{#9} & {#2}_{#10} \\
% {#3}_{#6} & {#3}_{#7} & {#3}_{#8} & {#3}_{#9} & {#3}_{#10} \\
% {#4}_{#6} & {#4}_{#7} & {#4}_{#8} & {#4}_{#9} & {#4}_{#10} \\
% {#5}_{#6} & {#5}_{#7} & {#5}_{#8} & {#5}_{#9} & {#5}_{#10}
%\end{vmatrix}
%}

% R3 vector.
\newcommand{\VectorThree}[3]{
\begin{bmatrix}
 {#1} \\
 {#2} \\
 {#3}
\end{bmatrix}
}


%\usepackage{amsmath}
%\usepackage{mathpazo}

%
% The real thing:
%

%\usepackage[bookmarks=true]{hyperref}

                             % The preamble begins here.
\chapter{matteo m} % Declares the document's title.
\label{chap:m}
%\author{Peeter Joot}         % Declares the author's name.
\date{ $RCSfile: m.tex,v $ Last $Revision: 1.6 $ $Date: 2009/06/14 23:51:45 $ }

%\begin{document}             % End of preamble and beginning of text.

%\maketitle{}
%\tableofcontents

Let
\begin{align*}
b_i &= b_{ij} e_j \\
w_i &= w_{ij} e_j \\
w_{ij} &= \sigma_j b_{ij}
\end{align*}

Given

\begin{align*}
x = x^i b_i
\end{align*}

compute

\begin{align*}
x = u^i w_i
\end{align*}

\begin{align*}
x^i b_i &= u^j w_j \\
x^i b_{ik} e_k &= u^j \sigma_c b_{jc} e_c \\
x^i b_{ik} e_k \cdot e_d &= u^j \sigma_c b_{jc} e_c \cdot e_d \\
\end{align*}

This leaves us with $N$ equations (one for each $e_d$), for $M$ unknowns $u^j$

\begin{align*}
x^i b_{id} &= u^j \sigma_d b_{jd} \\
\end{align*}

You know there is a unique solution for this provided the matrix $\sigma_d$ is not degenerate (ie: no zeros), so I don't see any choice but to do the transpose multiplication and invert for $u^j$

%%\bibliographystyle{plain}
%\bibliographystyle{plainnat} % supposed to allow for \url use.
%\bibliography{myrefs}      % expects file "myrefs.bib"

%\end{document}               % End of document.

%
% Copyright � 2012 Peeter Joot.  All Rights Reserved.
% Licenced as described in the file LICENSE under the root directory of this GIT repository.
%

% 
% 
%\documentclass{article}

%\usepackage{amsmath}
\usepackage{mathpazo}

%
% shorthand for bold symbols, convenient for vectors and matrices
%
\newcommand{\Ba}[0]{\mathbf{a}}
\newcommand{\Bb}[0]{\mathbf{b}}
\newcommand{\Bc}[0]{\mathbf{c}}
\newcommand{\Bd}[0]{\mathbf{d}}
\newcommand{\Be}[0]{\mathbf{e}}
\newcommand{\Bf}[0]{\mathbf{f}}
\newcommand{\Bg}[0]{\mathbf{g}}
\newcommand{\Bh}[0]{\mathbf{h}}
\newcommand{\Bi}[0]{\mathbf{i}}
\newcommand{\Bj}[0]{\mathbf{j}}
\newcommand{\Bk}[0]{\mathbf{k}}
\newcommand{\Bl}[0]{\mathbf{l}}
\newcommand{\Bm}[0]{\mathbf{m}}
\newcommand{\Bn}[0]{\mathbf{n}}
\newcommand{\Bo}[0]{\mathbf{o}}
\newcommand{\Bp}[0]{\mathbf{p}}
\newcommand{\Bq}[0]{\mathbf{q}}
\newcommand{\Br}[0]{\mathbf{r}}
\newcommand{\Bs}[0]{\mathbf{s}}
\newcommand{\Bt}[0]{\mathbf{t}}
\newcommand{\Bu}[0]{\mathbf{u}}
\newcommand{\Bv}[0]{\mathbf{v}}
\newcommand{\Bw}[0]{\mathbf{w}}
\newcommand{\Bx}[0]{\mathbf{x}}
\newcommand{\By}[0]{\mathbf{y}}
\newcommand{\Bz}[0]{\mathbf{z}}
\newcommand{\BA}[0]{\mathbf{A}}
\newcommand{\BB}[0]{\mathbf{B}}
\newcommand{\BC}[0]{\mathbf{C}}
\newcommand{\BD}[0]{\mathbf{D}}
\newcommand{\BE}[0]{\mathbf{E}}
\newcommand{\BF}[0]{\mathbf{F}}
\newcommand{\BG}[0]{\mathbf{G}}
\newcommand{\BH}[0]{\mathbf{H}}
\newcommand{\BI}[0]{\mathbf{I}}
\newcommand{\BJ}[0]{\mathbf{J}}
\newcommand{\BK}[0]{\mathbf{K}}
\newcommand{\BL}[0]{\mathbf{L}}
\newcommand{\BM}[0]{\mathbf{M}}
\newcommand{\BN}[0]{\mathbf{N}}
\newcommand{\BO}[0]{\mathbf{O}}
\newcommand{\BP}[0]{\mathbf{P}}
\newcommand{\BQ}[0]{\mathbf{Q}}
\newcommand{\BR}[0]{\mathbf{R}}
\newcommand{\BS}[0]{\mathbf{S}}
\newcommand{\BT}[0]{\mathbf{T}}
\newcommand{\BU}[0]{\mathbf{U}}
\newcommand{\BV}[0]{\mathbf{V}}
\newcommand{\BW}[0]{\mathbf{W}}
\newcommand{\BX}[0]{\mathbf{X}}
\newcommand{\BY}[0]{\mathbf{Y}}
\newcommand{\BZ}[0]{\mathbf{Z}}

\newcommand{\Bzero}[0]{\mathbf{0}}
\newcommand{\Btheta}[0]{\boldsymbol{\theta}}
\newcommand{\Btau}[0]{\boldsymbol{\tau}}
\newcommand{\Bomega}[0]{\boldsymbol{\omega}}

%
% shorthand for unit vectors
%
\newcommand{\acap}[0]{\hat{\Ba}}
\newcommand{\bcap}[0]{\hat{\Bb}}
\newcommand{\ccap}[0]{\hat{\Bc}}
\newcommand{\dcap}[0]{\hat{\Bd}}
\newcommand{\ecap}[0]{\hat{\Be}}
\newcommand{\fcap}[0]{\hat{\Bf}}
\newcommand{\gcap}[0]{\hat{\Bg}}
\newcommand{\hcap}[0]{\hat{\Bh}}
\newcommand{\icap}[0]{\hat{\Bi}}
\newcommand{\jcap}[0]{\hat{\Bj}}
\newcommand{\kcap}[0]{\hat{\Bk}}
\newcommand{\lcap}[0]{\hat{\Bl}}
\newcommand{\mcap}[0]{\hat{\Bm}}
\newcommand{\ncap}[0]{\hat{\Bn}}
\newcommand{\ocap}[0]{\hat{\Bo}}
\newcommand{\pcap}[0]{\hat{\Bp}}
\newcommand{\qcap}[0]{\hat{\Bq}}
\newcommand{\rcap}[0]{\hat{\Br}}
\newcommand{\scap}[0]{\hat{\Bs}}
\newcommand{\tcap}[0]{\hat{\Bt}}
\newcommand{\ucap}[0]{\hat{\Bu}}
\newcommand{\vcap}[0]{\hat{\Bv}}
\newcommand{\wcap}[0]{\hat{\Bw}}
\newcommand{\xcap}[0]{\hat{\Bx}}
\newcommand{\ycap}[0]{\hat{\By}}
\newcommand{\zcap}[0]{\hat{\Bz}}
\newcommand{\thetacap}[0]{\hat{\Btheta}}

%
% to write R^n and C^n in a distinguishable fashion.  Perhaps change this
% to the double lined characters upon figuring out how to do so.
%
\newcommand{\C}[1]{$\mathbb{C}^{#1}$}
\newcommand{\R}[1]{$\mathbb{R}^{#1}$}

%
% various generally useful helpers
%

% derivative of #1 wrt. #2:
\newcommand{\D}[2] {\frac {d#2} {d#1}}

\newcommand{\inv}[1]{\frac{1}{#1}}
\newcommand{\cross}[0]{\times}

\newcommand{\abs}[1]{\lvert{#1}\rvert}
\newcommand{\norm}[1]{\lVert{#1}\rVert}
\newcommand{\innerprod}[2]{\langle{#1}, {#2}\rangle}
\newcommand{\dotprod}[2]{{#1} \cdot {#2}}
\newcommand{\bdotprod}[2]{\left({#1} \cdot {#2}\right)}
\newcommand{\crossprod}[2]{{#1} \cross {#2}}
\newcommand{\tripleprod}[3]{\dotprod{\left(\crossprod{#1}{#2}\right)}{#3}}

\DeclareMathOperator{\Proj}{Proj}
\DeclareMathOperator{\Span}{span}
\DeclareMathOperator{\Sgn}{sgn}
\DeclareMathOperator{\Area}{Area}
\DeclareMathOperator{\Volume}{Volume}

%
% A few miscellaneous things specific to this document
%
\newcommand{\crossop}[1]{\crossprod{#1}{}}

% R2 vector.
\newcommand{\VectorTwo}[2]{
\begin{bmatrix}
 {#1} \\
 {#2}
\end{bmatrix}
}

\newcommand{\VectorN}[1]{
\begin{bmatrix}
{#1}_1 \\
{#1}_2 \\
\vdots \\
{#1}_N \\
\end{bmatrix}
}

\newcommand{\DETuvij}[4]{
\begin{vmatrix}
 {#1}_{#3} & {#1}_{#4} \\
 {#2}_{#3} & {#2}_{#4}
\end{vmatrix}
}

\newcommand{\DETuvwijk}[6]{
\begin{vmatrix}
 {#1}_{#4} & {#1}_{#5} & {#1}_{#6} \\
 {#2}_{#4} & {#2}_{#5} & {#2}_{#6} \\
 {#3}_{#4} & {#3}_{#5} & {#3}_{#6}
\end{vmatrix}
}

\newcommand{\DETuvwxijkl}[8]{
\begin{vmatrix}
 {#1}_{#5} & {#1}_{#6} & {#1}_{#7} & {#1}_{#8} \\
 {#2}_{#5} & {#2}_{#6} & {#2}_{#7} & {#2}_{#8} \\
 {#3}_{#5} & {#3}_{#6} & {#3}_{#7} & {#3}_{#8} \\
 {#4}_{#5} & {#4}_{#6} & {#4}_{#7} & {#4}_{#8} \\
\end{vmatrix}
}

%\newcommand{\DETuvwxyijklm}[10]{
%\begin{vmatrix}
% {#1}_{#6} & {#1}_{#7} & {#1}_{#8} & {#1}_{#9} & {#1}_{#10} \\
% {#2}_{#6} & {#2}_{#7} & {#2}_{#8} & {#2}_{#9} & {#2}_{#10} \\
% {#3}_{#6} & {#3}_{#7} & {#3}_{#8} & {#3}_{#9} & {#3}_{#10} \\
% {#4}_{#6} & {#4}_{#7} & {#4}_{#8} & {#4}_{#9} & {#4}_{#10} \\
% {#5}_{#6} & {#5}_{#7} & {#5}_{#8} & {#5}_{#9} & {#5}_{#10}
%\end{vmatrix}
%}

% R3 vector.
\newcommand{\VectorThree}[3]{
\begin{bmatrix}
 {#1} \\
 {#2} \\
 {#3}
\end{bmatrix}
}


%%<misc>
%
\newcommand{\Abs}[1]{{\left\lvert{#1}\right\rvert}}
\newcommand{\spacegrad}[0]{\boldsymbol{\nabla}}
\newcommand{\grad}[0]{\nabla}
\newcommand{\LL}[0]{\mathcal{L}}

% == \partial_{#1} {#2}
\newcommand{\PD}[2]{\frac{\partial {#2}}{\partial {#1}}}
% inline variant
\newcommand{\PDi}[2]{{\partial {#2}}/{\partial {#1}}}

\newcommand{\PDD}[3]{\frac{\partial^2 {#3}}{\partial {#1}\partial {#2}}}
%\newcommand{\PDd}[2]{\frac{\partial^2 {#2}}{{\partial{#1}}^2}}
\newcommand{\PDsq}[2]{\frac{\partial^2 {#2}}{(\partial {#1})^2}}

\newcommand{\Partial}[2]{\frac{\partial {#1}}{\partial {#2}}}
\DeclareMathOperator{\RejName}{Rej}
\newcommand{\Rej}[2]{\RejName_{#1}\left( {#2} \right)}
\newcommand{\Rm}[1]{\mathbb{R}^{#1}}
\newcommand{\Cm}[1]{\mathbb{C}^{#1}}
\newcommand{\conj}[0]{{*}}

%</misc>

% <grade selection>
%
\newcommand{\gpgrade}[2] {{\left\langle{{#1}}\right\rangle}_{#2}}

\newcommand{\gpgradezero}[1] {\gpgrade{#1}{}}
%\newcommand{\gpscalargrade}[1] {{\left\langle{{#1}}\right\rangle}}
%\newcommand{\gpgradezero}[1] {\gpgrade{#1}{0}}

%\newcommand{\gpgradeone}[1] {{\left\langle{{#1}}\right\rangle}_{1}}
\newcommand{\gpgradeone}[1] {\gpgrade{#1}{1}}

\newcommand{\gpgradetwo}[1] {\gpgrade{#1}{2}}
\newcommand{\gpgradethree}[1] {\gpgrade{#1}{3}}
\newcommand{\gpgradefour}[1] {\gpgrade{#1}{4}}
%
% </grade selection>



\newcommand{\adot}[0]{{\dot{a}}}
\newcommand{\bdot}[0]{{\dot{b}}}
% taken for centered dot:
%\newcommand{\cdot}[0]{{\dot{c}}}
%\newcommand{\ddot}[0]{{\dot{d}}}
\newcommand{\edot}[0]{{\dot{e}}}
\newcommand{\fdot}[0]{{\dot{f}}}
\newcommand{\gdot}[0]{{\dot{g}}}
\newcommand{\hdot}[0]{{\dot{h}}}
\newcommand{\idot}[0]{{\dot{i}}}
\newcommand{\jdot}[0]{{\dot{j}}}
\newcommand{\kdot}[0]{{\dot{k}}}
\newcommand{\ldot}[0]{{\dot{l}}}
\newcommand{\mdot}[0]{{\dot{m}}}
\newcommand{\ndot}[0]{{\dot{n}}}
%\newcommand{\odot}[0]{{\dot{o}}}
\newcommand{\pdot}[0]{{\dot{p}}}
\newcommand{\qdot}[0]{{\dot{q}}}
\newcommand{\rdot}[0]{{\dot{r}}}
\newcommand{\sdot}[0]{{\dot{s}}}
\newcommand{\tdot}[0]{{\dot{t}}}
\newcommand{\udot}[0]{{\dot{u}}}
\newcommand{\vdot}[0]{{\dot{v}}}
\newcommand{\wdot}[0]{{\dot{w}}}
\newcommand{\xdot}[0]{{\dot{x}}}
\newcommand{\ydot}[0]{{\dot{y}}}
\newcommand{\zdot}[0]{{\dot{z}}}
\newcommand{\addot}[0]{{\ddot{a}}}
\newcommand{\bddot}[0]{{\ddot{b}}}
\newcommand{\cddot}[0]{{\ddot{c}}}
%\newcommand{\dddot}[0]{{\ddot{d}}}
\newcommand{\eddot}[0]{{\ddot{e}}}
\newcommand{\fddot}[0]{{\ddot{f}}}
\newcommand{\gddot}[0]{{\ddot{g}}}
\newcommand{\hddot}[0]{{\ddot{h}}}
\newcommand{\iddot}[0]{{\ddot{i}}}
\newcommand{\jddot}[0]{{\ddot{j}}}
\newcommand{\kddot}[0]{{\ddot{k}}}
\newcommand{\lddot}[0]{{\ddot{l}}}
\newcommand{\mddot}[0]{{\ddot{m}}}
\newcommand{\nddot}[0]{{\ddot{n}}}
\newcommand{\oddot}[0]{{\ddot{o}}}
\newcommand{\pddot}[0]{{\ddot{p}}}
\newcommand{\qddot}[0]{{\ddot{q}}}
\newcommand{\rddot}[0]{{\ddot{r}}}
\newcommand{\sddot}[0]{{\ddot{s}}}
\newcommand{\tddot}[0]{{\ddot{t}}}
\newcommand{\uddot}[0]{{\ddot{u}}}
\newcommand{\vddot}[0]{{\ddot{v}}}
\newcommand{\wddot}[0]{{\ddot{w}}}
\newcommand{\xddot}[0]{{\ddot{x}}}
\newcommand{\yddot}[0]{{\ddot{y}}}
\newcommand{\zddot}[0]{{\ddot{z}}}

%<bold and dot greek symbols>
%

\newcommand{\Deltadot}[0]{{\dot{\Delta}}}
\newcommand{\Gammadot}[0]{{\dot{\Gamma}}}
\newcommand{\Lambdadot}[0]{{\dot{\Lambda}}}
\newcommand{\Omegadot}[0]{{\dot{\Omega}}}
\newcommand{\Phidot}[0]{{\dot{\Phi}}}
\newcommand{\Pidot}[0]{{\dot{\Pi}}}
\newcommand{\Psidot}[0]{{\dot{\Psi}}}
\newcommand{\Sigmadot}[0]{{\dot{\Sigma}}}
\newcommand{\Thetadot}[0]{{\dot{\Theta}}}
\newcommand{\Upsilondot}[0]{{\dot{\Upsilon}}}
\newcommand{\Xidot}[0]{{\dot{\Xi}}}
\newcommand{\alphadot}[0]{{\dot{\alpha}}}
\newcommand{\betadot}[0]{{\dot{\beta}}}
\newcommand{\chidot}[0]{{\dot{\chi}}}
\newcommand{\deltadot}[0]{{\dot{\delta}}}
\newcommand{\epsilondot}[0]{{\dot{\epsilon}}}
\newcommand{\etadot}[0]{{\dot{\eta}}}
\newcommand{\gammadot}[0]{{\dot{\gamma}}}
\newcommand{\kappadot}[0]{{\dot{\kappa}}}
\newcommand{\lambdadot}[0]{{\dot{\lambda}}}
\newcommand{\mudot}[0]{{\dot{\mu}}}
\newcommand{\nudot}[0]{{\dot{\nu}}}
\newcommand{\omegadot}[0]{{\dot{\omega}}}
\newcommand{\phidot}[0]{{\dot{\phi}}}
\newcommand{\pidot}[0]{{\dot{\pi}}}
\newcommand{\psidot}[0]{{\dot{\psi}}}
\newcommand{\rhodot}[0]{{\dot{\rho}}}
\newcommand{\sigmadot}[0]{{\dot{\sigma}}}
\newcommand{\taudot}[0]{{\dot{\tau}}}
\newcommand{\thetadot}[0]{{\dot{\theta}}}
\newcommand{\upsilondot}[0]{{\dot{\upsilon}}}
\newcommand{\varepsilondot}[0]{{\dot{\varepsilon}}}
\newcommand{\varphidot}[0]{{\dot{\varphi}}}
\newcommand{\varpidot}[0]{{\dot{\varpi}}}
\newcommand{\varrhodot}[0]{{\dot{\varrho}}}
\newcommand{\varsigmadot}[0]{{\dot{\varsigma}}}
\newcommand{\varthetadot}[0]{{\dot{\vartheta}}}
\newcommand{\xidot}[0]{{\dot{\xi}}}
\newcommand{\zetadot}[0]{{\dot{\zeta}}}

\newcommand{\Deltaddot}[0]{{\ddot{\Delta}}}
\newcommand{\Gammaddot}[0]{{\ddot{\Gamma}}}
\newcommand{\Lambdaddot}[0]{{\ddot{\Lambda}}}
\newcommand{\Omegaddot}[0]{{\ddot{\Omega}}}
\newcommand{\Phiddot}[0]{{\ddot{\Phi}}}
\newcommand{\Piddot}[0]{{\ddot{\Pi}}}
\newcommand{\Psiddot}[0]{{\ddot{\Psi}}}
\newcommand{\Sigmaddot}[0]{{\ddot{\Sigma}}}
\newcommand{\Thetaddot}[0]{{\ddot{\Theta}}}
\newcommand{\Upsilonddot}[0]{{\ddot{\Upsilon}}}
\newcommand{\Xiddot}[0]{{\ddot{\Xi}}}
\newcommand{\alphaddot}[0]{{\ddot{\alpha}}}
\newcommand{\betaddot}[0]{{\ddot{\beta}}}
\newcommand{\chiddot}[0]{{\ddot{\chi}}}
\newcommand{\deltaddot}[0]{{\ddot{\delta}}}
\newcommand{\epsilonddot}[0]{{\ddot{\epsilon}}}
\newcommand{\etaddot}[0]{{\ddot{\eta}}}
\newcommand{\gammaddot}[0]{{\ddot{\gamma}}}
\newcommand{\kappaddot}[0]{{\ddot{\kappa}}}
\newcommand{\lambdaddot}[0]{{\ddot{\lambda}}}
\newcommand{\muddot}[0]{{\ddot{\mu}}}
\newcommand{\nuddot}[0]{{\ddot{\nu}}}
\newcommand{\omegaddot}[0]{{\ddot{\omega}}}
\newcommand{\phiddot}[0]{{\ddot{\phi}}}
\newcommand{\piddot}[0]{{\ddot{\pi}}}
\newcommand{\psiddot}[0]{{\ddot{\psi}}}
\newcommand{\rhoddot}[0]{{\ddot{\rho}}}
\newcommand{\sigmaddot}[0]{{\ddot{\sigma}}}
\newcommand{\tauddot}[0]{{\ddot{\tau}}}
\newcommand{\thetaddot}[0]{{\ddot{\theta}}}
\newcommand{\upsilonddot}[0]{{\ddot{\upsilon}}}
\newcommand{\varepsilonddot}[0]{{\ddot{\varepsilon}}}
\newcommand{\varphiddot}[0]{{\ddot{\varphi}}}
\newcommand{\varpiddot}[0]{{\ddot{\varpi}}}
\newcommand{\varrhoddot}[0]{{\ddot{\varrho}}}
\newcommand{\varsigmaddot}[0]{{\ddot{\varsigma}}}
\newcommand{\varthetaddot}[0]{{\ddot{\vartheta}}}
\newcommand{\xiddot}[0]{{\ddot{\xi}}}
\newcommand{\zetaddot}[0]{{\ddot{\zeta}}}

\newcommand{\BDelta}[0]{\boldsymbol{\Delta}}
\newcommand{\BGamma}[0]{\boldsymbol{\Gamma}}
\newcommand{\BLambda}[0]{\boldsymbol{\Lambda}}
\newcommand{\BOmega}[0]{\boldsymbol{\Omega}}
\newcommand{\BPhi}[0]{\boldsymbol{\Phi}}
\newcommand{\BPi}[0]{\boldsymbol{\Pi}}
\newcommand{\BPsi}[0]{\boldsymbol{\Psi}}
\newcommand{\BSigma}[0]{\boldsymbol{\Sigma}}
\newcommand{\BTheta}[0]{\boldsymbol{\Theta}}
\newcommand{\BUpsilon}[0]{\boldsymbol{\Upsilon}}
\newcommand{\BXi}[0]{\boldsymbol{\Xi}}
\newcommand{\Balpha}[0]{\boldsymbol{\alpha}}
\newcommand{\Bbeta}[0]{\boldsymbol{\beta}}
\newcommand{\Bchi}[0]{\boldsymbol{\chi}}
\newcommand{\Bdelta}[0]{\boldsymbol{\delta}}
\newcommand{\Bepsilon}[0]{\boldsymbol{\epsilon}}
\newcommand{\Beta}[0]{\boldsymbol{\eta}}
\newcommand{\Bgamma}[0]{\boldsymbol{\gamma}}
\newcommand{\Bkappa}[0]{\boldsymbol{\kappa}}
\newcommand{\Blambda}[0]{\boldsymbol{\lambda}}
\newcommand{\Bmu}[0]{\boldsymbol{\mu}}
\newcommand{\Bnu}[0]{\boldsymbol{\nu}}
%\newcommand{\Bomega}[0]{\boldsymbol{\omega}}
\newcommand{\Bphi}[0]{\boldsymbol{\phi}}
\newcommand{\Bpi}[0]{\boldsymbol{\pi}}
\newcommand{\Bpsi}[0]{\boldsymbol{\psi}}
\newcommand{\Brho}[0]{\boldsymbol{\rho}}
\newcommand{\Bsigma}[0]{\boldsymbol{\sigma}}
%\newcommand{\Btau}[0]{\boldsymbol{\tau}}
%\newcommand{\Btheta}[0]{\boldsymbol{\theta}}
\newcommand{\Bupsilon}[0]{\boldsymbol{\upsilon}}
\newcommand{\Bvarepsilon}[0]{\boldsymbol{\varepsilon}}
\newcommand{\Bvarphi}[0]{\boldsymbol{\varphi}}
\newcommand{\Bvarpi}[0]{\boldsymbol{\varpi}}
\newcommand{\Bvarrho}[0]{\boldsymbol{\varrho}}
\newcommand{\Bvarsigma}[0]{\boldsymbol{\varsigma}}
\newcommand{\Bvartheta}[0]{\boldsymbol{\vartheta}}
\newcommand{\Bxi}[0]{\boldsymbol{\xi}}
\newcommand{\Bzeta}[0]{\boldsymbol{\zeta}}
%
%</bold and dot greek symbols>
%<infrequent>
%
%\newcommand{\AreaOp}[1]{\AName_{#1}}
%\newcommand{\Babs}[0]{\abs{\BB}}
%\newcommand{\Bcap}[0]{\hat{\BB}}
%\newcommand{\BrPrimeRej}[0]{\rcap(\rcap \wedge \Br')}
%\newcommand{\CA}[0]{\mathcal{A}}
%\newcommand{\Cos}[1]{\cos{\left({#1}\right)}}
%\newcommand{\Det}[1] {\abs{#1}}
%\newcommand{\Dsq}[2] {\frac {\partial^2 {#1}} {\partial {#2}^2}}
%\newcommand{\Exp}[1]{\exp{\left({#1}\right)}}
%\newcommand{\Norm}[1]{\left\lVert{#1}\right\rVert}
%\newcommand{\Sin}[1]{\sin{\left({#1}\right)}}
%\newcommand{\T}[0]{\text{T}}
%\newcommand{\VolumeOp}[1]{\VName_{#1}}
%\newcommand{\agrad}[0]{\Ba \cdot \nabla}
%\newcommand{\alphacap}[0]{\hat{\boldsymbol{\alpha}}}
%\newcommand{\Fcap}[0]{\hat{\BF}}
%\newcommand{\bithree}[0]{{\Bi}_3}
%\newcommand{\bxa}[0]{\Bx\Ba}
%\newcommand{\coordvec}[2]{
%\newcommand{\costheta}[0]{\acap \cdot \xcap}
%\newcommand{\ddt}[1]{\ddot{#1}}
%\newcommand{\ddu}[1] {\frac {d{#1}} {du}}
%\newcommand{\dsqxj}[2] {\frac {\partial^2 {#1}} {\partial {x_{#2}}^2}}
%\newcommand{\dtheta}[1]{\frac{d {#1}}{d \theta}}
%\newcommand{\dt}[1]{\dot{#1}}
%\newcommand{\dt}[1]{\frac{d {#1}}{dt}}
%\newcommand{\dxj}[2] {\frac {\partial {#1}} {\partial {x_{#2}}}}
%\newcommand{\halfPhi}[0]{\frac{\phi}{2}}
%\newcommand{\half}[0]{\inv{2}}
%\newcommand{\inv}[1]{\frac{1}{#1}}
%\newcommand{\laplacian}[0]{\nabla^2}
%\newcommand{\matrixoftx}[3]{
%\newcommand{\nrrp}[0]{\norm{\rcap \wedge \Br'}}
%\newcommand{\oiint}{\bigcirc \hspace{-1.4em} \int \hspace{-.8em} \int}
%\newcommand{\transpose}[1]{{#1}^{\text{T}}}
%\newcommand{\transpose}[1]{{{#1}^{\TextTranspose}}}
%\newcommand{\transpose}[1]{{{#1}^{\text{T}}}}
%\newcommand{\barA}[0]{\bar{A}}
%\newcommand{\qbar}[0]{\bar{q}}
%\newcommand{\qdotbar}[0]{\dot{\bar{q}}}
%
%</infrequent>





%\usepackage[bookmarks=true]{hyperref}

%\usepackage{color,cite,graphicx}
   % use colour in the document, put your citations as [1-4]
   % rather than [1,2,3,4] (it looks nicer, and the extended LaTeX2e
   % graphics package. 
%\usepackage{latexsym,amssymb,epsf} % do not remember if these are
   % needed, but their inclusion can not do any damage


\chapter{box}
\label{chap:box}
%\author{Peeter Joot \quad peeterjoot@protonmail.com}
\date{ Mmm dd, 2008.  \(RCSfile: box.tex,v \) Last \(Revision: 1.7 \) \(Date: 2009/06/14 23:51:45 \) }

%\begin{document}

%\maketitle{}
%
%\tableofcontents
%\section{}


\begin{equation}\label{eqn:box:20}
\begin{aligned}
(\gamma^\mu \partial_\mu)^2 
&= 
(\gamma^\mu \partial_\mu) \cdot (\gamma^\nu \partial_\nu)
+ (\gamma^\mu \partial_\mu) \wedge (\gamma^\nu \partial_\nu) \\
&= (\gamma^\mu \partial_\mu) \cdot (\gamma_\nu \partial^\nu) \\
&= \gamma^\mu \cdot \gamma_\nu \partial_\mu \partial^\nu \\
&= {\delta^\mu}_\nu \partial_\mu \partial^\nu \\
&= \partial_\mu \partial^\mu \\
\end{aligned}
\end{equation}


\begin{equation}\label{eqn:box:40}
\begin{aligned}
\Box
&= \eta ^{\mu\nu}\partial_{\nu}  \partial_{\mu}  \\
&= \frac{1}{2}\gamma^{\mu}\gamma^{\nu} \partial_{\nu}  \partial_{\mu} +\frac{1}{2}\gamma^{\nu}\gamma^{\mu} \partial_{\nu}  \partial_{\mu} \\
&= \frac{1}{2}\left(\gamma^{\mu}\gamma_{\nu} + \gamma_{\nu}\gamma^{\mu} \right) \partial^{\nu} \partial_{\mu} \\
&= \gamma^\mu \cdot \gamma_\nu \partial^{\nu} \partial_{\mu} \\
&= \partial^{\mu} \partial_{\mu} \\
\end{aligned}
\end{equation}


Avodyne:

Here is how the Dirac square-root thing works in a more pedestrian notation.  First we write \(\gamma^\mu\gamma^\nu\) as the sum of a commutator and an anticommutator:

\begin{equation}\label{eqn:box:60}
\begin{aligned}
\gamma^\mu \gamma^\nu = \inv{2}[\gamma^\mu,\gamma^\nu] + \inv{2} \{\gamma^\mu,\gamma^\nu\}
\end{aligned}
\end{equation}

The first term is antisymmetric on exchange of \(\mu\) and \(\nu\), while the second term is symmetric.  Also, partial derivatives commute, so \(\partial_\mu\partial_\nu\) is symmetric.  Then, in general, if you contract both indices of an antisymmetric tensor \(A^{\mu\nu}\) with those of a symmetric tensor \(S_{\mu\nu}\), you get zero: \(A^{\mu\nu}S_{\mu\nu}=0\).  
So, the commutator term vanishes when contracted.  And the anticommutator is 

\begin{equation}\label{eqn:box:80}
\begin{aligned}
\inv{2}\{\gamma^\mu,\gamma^\nu\}=\eta^{\mu\nu}
\end{aligned}
\end{equation}

%\bibliographystyle{plainnat}
%\bibliography{myrefs}

%\end{document}

%
% Copyright � 2012 Peeter Joot.  All Rights Reserved.
% Licenced as described in the file LICENSE under the root directory of this GIT repository.
%

% 
% 
%\documentclass{article}

%\usepackage{amsmath}
\usepackage{mathpazo}

%
% shorthand for bold symbols, convenient for vectors and matrices
%
\newcommand{\Ba}[0]{\mathbf{a}}
\newcommand{\Bb}[0]{\mathbf{b}}
\newcommand{\Bc}[0]{\mathbf{c}}
\newcommand{\Bd}[0]{\mathbf{d}}
\newcommand{\Be}[0]{\mathbf{e}}
\newcommand{\Bf}[0]{\mathbf{f}}
\newcommand{\Bg}[0]{\mathbf{g}}
\newcommand{\Bh}[0]{\mathbf{h}}
\newcommand{\Bi}[0]{\mathbf{i}}
\newcommand{\Bj}[0]{\mathbf{j}}
\newcommand{\Bk}[0]{\mathbf{k}}
\newcommand{\Bl}[0]{\mathbf{l}}
\newcommand{\Bm}[0]{\mathbf{m}}
\newcommand{\Bn}[0]{\mathbf{n}}
\newcommand{\Bo}[0]{\mathbf{o}}
\newcommand{\Bp}[0]{\mathbf{p}}
\newcommand{\Bq}[0]{\mathbf{q}}
\newcommand{\Br}[0]{\mathbf{r}}
\newcommand{\Bs}[0]{\mathbf{s}}
\newcommand{\Bt}[0]{\mathbf{t}}
\newcommand{\Bu}[0]{\mathbf{u}}
\newcommand{\Bv}[0]{\mathbf{v}}
\newcommand{\Bw}[0]{\mathbf{w}}
\newcommand{\Bx}[0]{\mathbf{x}}
\newcommand{\By}[0]{\mathbf{y}}
\newcommand{\Bz}[0]{\mathbf{z}}
\newcommand{\BA}[0]{\mathbf{A}}
\newcommand{\BB}[0]{\mathbf{B}}
\newcommand{\BC}[0]{\mathbf{C}}
\newcommand{\BD}[0]{\mathbf{D}}
\newcommand{\BE}[0]{\mathbf{E}}
\newcommand{\BF}[0]{\mathbf{F}}
\newcommand{\BG}[0]{\mathbf{G}}
\newcommand{\BH}[0]{\mathbf{H}}
\newcommand{\BI}[0]{\mathbf{I}}
\newcommand{\BJ}[0]{\mathbf{J}}
\newcommand{\BK}[0]{\mathbf{K}}
\newcommand{\BL}[0]{\mathbf{L}}
\newcommand{\BM}[0]{\mathbf{M}}
\newcommand{\BN}[0]{\mathbf{N}}
\newcommand{\BO}[0]{\mathbf{O}}
\newcommand{\BP}[0]{\mathbf{P}}
\newcommand{\BQ}[0]{\mathbf{Q}}
\newcommand{\BR}[0]{\mathbf{R}}
\newcommand{\BS}[0]{\mathbf{S}}
\newcommand{\BT}[0]{\mathbf{T}}
\newcommand{\BU}[0]{\mathbf{U}}
\newcommand{\BV}[0]{\mathbf{V}}
\newcommand{\BW}[0]{\mathbf{W}}
\newcommand{\BX}[0]{\mathbf{X}}
\newcommand{\BY}[0]{\mathbf{Y}}
\newcommand{\BZ}[0]{\mathbf{Z}}

\newcommand{\Bzero}[0]{\mathbf{0}}
\newcommand{\Btheta}[0]{\boldsymbol{\theta}}
\newcommand{\Btau}[0]{\boldsymbol{\tau}}
\newcommand{\Bomega}[0]{\boldsymbol{\omega}}

%
% shorthand for unit vectors
%
\newcommand{\acap}[0]{\hat{\Ba}}
\newcommand{\bcap}[0]{\hat{\Bb}}
\newcommand{\ccap}[0]{\hat{\Bc}}
\newcommand{\dcap}[0]{\hat{\Bd}}
\newcommand{\ecap}[0]{\hat{\Be}}
\newcommand{\fcap}[0]{\hat{\Bf}}
\newcommand{\gcap}[0]{\hat{\Bg}}
\newcommand{\hcap}[0]{\hat{\Bh}}
\newcommand{\icap}[0]{\hat{\Bi}}
\newcommand{\jcap}[0]{\hat{\Bj}}
\newcommand{\kcap}[0]{\hat{\Bk}}
\newcommand{\lcap}[0]{\hat{\Bl}}
\newcommand{\mcap}[0]{\hat{\Bm}}
\newcommand{\ncap}[0]{\hat{\Bn}}
\newcommand{\ocap}[0]{\hat{\Bo}}
\newcommand{\pcap}[0]{\hat{\Bp}}
\newcommand{\qcap}[0]{\hat{\Bq}}
\newcommand{\rcap}[0]{\hat{\Br}}
\newcommand{\scap}[0]{\hat{\Bs}}
\newcommand{\tcap}[0]{\hat{\Bt}}
\newcommand{\ucap}[0]{\hat{\Bu}}
\newcommand{\vcap}[0]{\hat{\Bv}}
\newcommand{\wcap}[0]{\hat{\Bw}}
\newcommand{\xcap}[0]{\hat{\Bx}}
\newcommand{\ycap}[0]{\hat{\By}}
\newcommand{\zcap}[0]{\hat{\Bz}}
\newcommand{\thetacap}[0]{\hat{\Btheta}}

%
% to write R^n and C^n in a distinguishable fashion.  Perhaps change this
% to the double lined characters upon figuring out how to do so.
%
\newcommand{\C}[1]{$\mathbb{C}^{#1}$}
\newcommand{\R}[1]{$\mathbb{R}^{#1}$}

%
% various generally useful helpers
%

% derivative of #1 wrt. #2:
\newcommand{\D}[2] {\frac {d#2} {d#1}}

\newcommand{\inv}[1]{\frac{1}{#1}}
\newcommand{\cross}[0]{\times}

\newcommand{\abs}[1]{\lvert{#1}\rvert}
\newcommand{\norm}[1]{\lVert{#1}\rVert}
\newcommand{\innerprod}[2]{\langle{#1}, {#2}\rangle}
\newcommand{\dotprod}[2]{{#1} \cdot {#2}}
\newcommand{\bdotprod}[2]{\left({#1} \cdot {#2}\right)}
\newcommand{\crossprod}[2]{{#1} \cross {#2}}
\newcommand{\tripleprod}[3]{\dotprod{\left(\crossprod{#1}{#2}\right)}{#3}}

\DeclareMathOperator{\Proj}{Proj}
\DeclareMathOperator{\Span}{span}
\DeclareMathOperator{\Sgn}{sgn}
\DeclareMathOperator{\Area}{Area}
\DeclareMathOperator{\Volume}{Volume}

%
% A few miscellaneous things specific to this document
%
\newcommand{\crossop}[1]{\crossprod{#1}{}}

% R2 vector.
\newcommand{\VectorTwo}[2]{
\begin{bmatrix}
 {#1} \\
 {#2}
\end{bmatrix}
}

\newcommand{\VectorN}[1]{
\begin{bmatrix}
{#1}_1 \\
{#1}_2 \\
\vdots \\
{#1}_N \\
\end{bmatrix}
}

\newcommand{\DETuvij}[4]{
\begin{vmatrix}
 {#1}_{#3} & {#1}_{#4} \\
 {#2}_{#3} & {#2}_{#4}
\end{vmatrix}
}

\newcommand{\DETuvwijk}[6]{
\begin{vmatrix}
 {#1}_{#4} & {#1}_{#5} & {#1}_{#6} \\
 {#2}_{#4} & {#2}_{#5} & {#2}_{#6} \\
 {#3}_{#4} & {#3}_{#5} & {#3}_{#6}
\end{vmatrix}
}

\newcommand{\DETuvwxijkl}[8]{
\begin{vmatrix}
 {#1}_{#5} & {#1}_{#6} & {#1}_{#7} & {#1}_{#8} \\
 {#2}_{#5} & {#2}_{#6} & {#2}_{#7} & {#2}_{#8} \\
 {#3}_{#5} & {#3}_{#6} & {#3}_{#7} & {#3}_{#8} \\
 {#4}_{#5} & {#4}_{#6} & {#4}_{#7} & {#4}_{#8} \\
\end{vmatrix}
}

%\newcommand{\DETuvwxyijklm}[10]{
%\begin{vmatrix}
% {#1}_{#6} & {#1}_{#7} & {#1}_{#8} & {#1}_{#9} & {#1}_{#10} \\
% {#2}_{#6} & {#2}_{#7} & {#2}_{#8} & {#2}_{#9} & {#2}_{#10} \\
% {#3}_{#6} & {#3}_{#7} & {#3}_{#8} & {#3}_{#9} & {#3}_{#10} \\
% {#4}_{#6} & {#4}_{#7} & {#4}_{#8} & {#4}_{#9} & {#4}_{#10} \\
% {#5}_{#6} & {#5}_{#7} & {#5}_{#8} & {#5}_{#9} & {#5}_{#10}
%\end{vmatrix}
%}

% R3 vector.
\newcommand{\VectorThree}[3]{
\begin{bmatrix}
 {#1} \\
 {#2} \\
 {#3}
\end{bmatrix}
}


%%<misc>
%
\newcommand{\Abs}[1]{{\left\lvert{#1}\right\rvert}}
\newcommand{\spacegrad}[0]{\boldsymbol{\nabla}}
\newcommand{\grad}[0]{\nabla}
\newcommand{\LL}[0]{\mathcal{L}}

% == \partial_{#1} {#2}
\newcommand{\PD}[2]{\frac{\partial {#2}}{\partial {#1}}}
% inline variant
\newcommand{\PDi}[2]{{\partial {#2}}/{\partial {#1}}}

\newcommand{\PDD}[3]{\frac{\partial^2 {#3}}{\partial {#1}\partial {#2}}}
%\newcommand{\PDd}[2]{\frac{\partial^2 {#2}}{{\partial{#1}}^2}}
\newcommand{\PDsq}[2]{\frac{\partial^2 {#2}}{(\partial {#1})^2}}

\newcommand{\Partial}[2]{\frac{\partial {#1}}{\partial {#2}}}
\DeclareMathOperator{\RejName}{Rej}
\newcommand{\Rej}[2]{\RejName_{#1}\left( {#2} \right)}
\newcommand{\Rm}[1]{\mathbb{R}^{#1}}
\newcommand{\Cm}[1]{\mathbb{C}^{#1}}
\newcommand{\conj}[0]{{*}}

%</misc>

% <grade selection>
%
\newcommand{\gpgrade}[2] {{\left\langle{{#1}}\right\rangle}_{#2}}

\newcommand{\gpgradezero}[1] {\gpgrade{#1}{}}
%\newcommand{\gpscalargrade}[1] {{\left\langle{{#1}}\right\rangle}}
%\newcommand{\gpgradezero}[1] {\gpgrade{#1}{0}}

%\newcommand{\gpgradeone}[1] {{\left\langle{{#1}}\right\rangle}_{1}}
\newcommand{\gpgradeone}[1] {\gpgrade{#1}{1}}

\newcommand{\gpgradetwo}[1] {\gpgrade{#1}{2}}
\newcommand{\gpgradethree}[1] {\gpgrade{#1}{3}}
\newcommand{\gpgradefour}[1] {\gpgrade{#1}{4}}
%
% </grade selection>



\newcommand{\adot}[0]{{\dot{a}}}
\newcommand{\bdot}[0]{{\dot{b}}}
% taken for centered dot:
%\newcommand{\cdot}[0]{{\dot{c}}}
%\newcommand{\ddot}[0]{{\dot{d}}}
\newcommand{\edot}[0]{{\dot{e}}}
\newcommand{\fdot}[0]{{\dot{f}}}
\newcommand{\gdot}[0]{{\dot{g}}}
\newcommand{\hdot}[0]{{\dot{h}}}
\newcommand{\idot}[0]{{\dot{i}}}
\newcommand{\jdot}[0]{{\dot{j}}}
\newcommand{\kdot}[0]{{\dot{k}}}
\newcommand{\ldot}[0]{{\dot{l}}}
\newcommand{\mdot}[0]{{\dot{m}}}
\newcommand{\ndot}[0]{{\dot{n}}}
%\newcommand{\odot}[0]{{\dot{o}}}
\newcommand{\pdot}[0]{{\dot{p}}}
\newcommand{\qdot}[0]{{\dot{q}}}
\newcommand{\rdot}[0]{{\dot{r}}}
\newcommand{\sdot}[0]{{\dot{s}}}
\newcommand{\tdot}[0]{{\dot{t}}}
\newcommand{\udot}[0]{{\dot{u}}}
\newcommand{\vdot}[0]{{\dot{v}}}
\newcommand{\wdot}[0]{{\dot{w}}}
\newcommand{\xdot}[0]{{\dot{x}}}
\newcommand{\ydot}[0]{{\dot{y}}}
\newcommand{\zdot}[0]{{\dot{z}}}
\newcommand{\addot}[0]{{\ddot{a}}}
\newcommand{\bddot}[0]{{\ddot{b}}}
\newcommand{\cddot}[0]{{\ddot{c}}}
%\newcommand{\dddot}[0]{{\ddot{d}}}
\newcommand{\eddot}[0]{{\ddot{e}}}
\newcommand{\fddot}[0]{{\ddot{f}}}
\newcommand{\gddot}[0]{{\ddot{g}}}
\newcommand{\hddot}[0]{{\ddot{h}}}
\newcommand{\iddot}[0]{{\ddot{i}}}
\newcommand{\jddot}[0]{{\ddot{j}}}
\newcommand{\kddot}[0]{{\ddot{k}}}
\newcommand{\lddot}[0]{{\ddot{l}}}
\newcommand{\mddot}[0]{{\ddot{m}}}
\newcommand{\nddot}[0]{{\ddot{n}}}
\newcommand{\oddot}[0]{{\ddot{o}}}
\newcommand{\pddot}[0]{{\ddot{p}}}
\newcommand{\qddot}[0]{{\ddot{q}}}
\newcommand{\rddot}[0]{{\ddot{r}}}
\newcommand{\sddot}[0]{{\ddot{s}}}
\newcommand{\tddot}[0]{{\ddot{t}}}
\newcommand{\uddot}[0]{{\ddot{u}}}
\newcommand{\vddot}[0]{{\ddot{v}}}
\newcommand{\wddot}[0]{{\ddot{w}}}
\newcommand{\xddot}[0]{{\ddot{x}}}
\newcommand{\yddot}[0]{{\ddot{y}}}
\newcommand{\zddot}[0]{{\ddot{z}}}

%<bold and dot greek symbols>
%

\newcommand{\Deltadot}[0]{{\dot{\Delta}}}
\newcommand{\Gammadot}[0]{{\dot{\Gamma}}}
\newcommand{\Lambdadot}[0]{{\dot{\Lambda}}}
\newcommand{\Omegadot}[0]{{\dot{\Omega}}}
\newcommand{\Phidot}[0]{{\dot{\Phi}}}
\newcommand{\Pidot}[0]{{\dot{\Pi}}}
\newcommand{\Psidot}[0]{{\dot{\Psi}}}
\newcommand{\Sigmadot}[0]{{\dot{\Sigma}}}
\newcommand{\Thetadot}[0]{{\dot{\Theta}}}
\newcommand{\Upsilondot}[0]{{\dot{\Upsilon}}}
\newcommand{\Xidot}[0]{{\dot{\Xi}}}
\newcommand{\alphadot}[0]{{\dot{\alpha}}}
\newcommand{\betadot}[0]{{\dot{\beta}}}
\newcommand{\chidot}[0]{{\dot{\chi}}}
\newcommand{\deltadot}[0]{{\dot{\delta}}}
\newcommand{\epsilondot}[0]{{\dot{\epsilon}}}
\newcommand{\etadot}[0]{{\dot{\eta}}}
\newcommand{\gammadot}[0]{{\dot{\gamma}}}
\newcommand{\kappadot}[0]{{\dot{\kappa}}}
\newcommand{\lambdadot}[0]{{\dot{\lambda}}}
\newcommand{\mudot}[0]{{\dot{\mu}}}
\newcommand{\nudot}[0]{{\dot{\nu}}}
\newcommand{\omegadot}[0]{{\dot{\omega}}}
\newcommand{\phidot}[0]{{\dot{\phi}}}
\newcommand{\pidot}[0]{{\dot{\pi}}}
\newcommand{\psidot}[0]{{\dot{\psi}}}
\newcommand{\rhodot}[0]{{\dot{\rho}}}
\newcommand{\sigmadot}[0]{{\dot{\sigma}}}
\newcommand{\taudot}[0]{{\dot{\tau}}}
\newcommand{\thetadot}[0]{{\dot{\theta}}}
\newcommand{\upsilondot}[0]{{\dot{\upsilon}}}
\newcommand{\varepsilondot}[0]{{\dot{\varepsilon}}}
\newcommand{\varphidot}[0]{{\dot{\varphi}}}
\newcommand{\varpidot}[0]{{\dot{\varpi}}}
\newcommand{\varrhodot}[0]{{\dot{\varrho}}}
\newcommand{\varsigmadot}[0]{{\dot{\varsigma}}}
\newcommand{\varthetadot}[0]{{\dot{\vartheta}}}
\newcommand{\xidot}[0]{{\dot{\xi}}}
\newcommand{\zetadot}[0]{{\dot{\zeta}}}

\newcommand{\Deltaddot}[0]{{\ddot{\Delta}}}
\newcommand{\Gammaddot}[0]{{\ddot{\Gamma}}}
\newcommand{\Lambdaddot}[0]{{\ddot{\Lambda}}}
\newcommand{\Omegaddot}[0]{{\ddot{\Omega}}}
\newcommand{\Phiddot}[0]{{\ddot{\Phi}}}
\newcommand{\Piddot}[0]{{\ddot{\Pi}}}
\newcommand{\Psiddot}[0]{{\ddot{\Psi}}}
\newcommand{\Sigmaddot}[0]{{\ddot{\Sigma}}}
\newcommand{\Thetaddot}[0]{{\ddot{\Theta}}}
\newcommand{\Upsilonddot}[0]{{\ddot{\Upsilon}}}
\newcommand{\Xiddot}[0]{{\ddot{\Xi}}}
\newcommand{\alphaddot}[0]{{\ddot{\alpha}}}
\newcommand{\betaddot}[0]{{\ddot{\beta}}}
\newcommand{\chiddot}[0]{{\ddot{\chi}}}
\newcommand{\deltaddot}[0]{{\ddot{\delta}}}
\newcommand{\epsilonddot}[0]{{\ddot{\epsilon}}}
\newcommand{\etaddot}[0]{{\ddot{\eta}}}
\newcommand{\gammaddot}[0]{{\ddot{\gamma}}}
\newcommand{\kappaddot}[0]{{\ddot{\kappa}}}
\newcommand{\lambdaddot}[0]{{\ddot{\lambda}}}
\newcommand{\muddot}[0]{{\ddot{\mu}}}
\newcommand{\nuddot}[0]{{\ddot{\nu}}}
\newcommand{\omegaddot}[0]{{\ddot{\omega}}}
\newcommand{\phiddot}[0]{{\ddot{\phi}}}
\newcommand{\piddot}[0]{{\ddot{\pi}}}
\newcommand{\psiddot}[0]{{\ddot{\psi}}}
\newcommand{\rhoddot}[0]{{\ddot{\rho}}}
\newcommand{\sigmaddot}[0]{{\ddot{\sigma}}}
\newcommand{\tauddot}[0]{{\ddot{\tau}}}
\newcommand{\thetaddot}[0]{{\ddot{\theta}}}
\newcommand{\upsilonddot}[0]{{\ddot{\upsilon}}}
\newcommand{\varepsilonddot}[0]{{\ddot{\varepsilon}}}
\newcommand{\varphiddot}[0]{{\ddot{\varphi}}}
\newcommand{\varpiddot}[0]{{\ddot{\varpi}}}
\newcommand{\varrhoddot}[0]{{\ddot{\varrho}}}
\newcommand{\varsigmaddot}[0]{{\ddot{\varsigma}}}
\newcommand{\varthetaddot}[0]{{\ddot{\vartheta}}}
\newcommand{\xiddot}[0]{{\ddot{\xi}}}
\newcommand{\zetaddot}[0]{{\ddot{\zeta}}}

\newcommand{\BDelta}[0]{\boldsymbol{\Delta}}
\newcommand{\BGamma}[0]{\boldsymbol{\Gamma}}
\newcommand{\BLambda}[0]{\boldsymbol{\Lambda}}
\newcommand{\BOmega}[0]{\boldsymbol{\Omega}}
\newcommand{\BPhi}[0]{\boldsymbol{\Phi}}
\newcommand{\BPi}[0]{\boldsymbol{\Pi}}
\newcommand{\BPsi}[0]{\boldsymbol{\Psi}}
\newcommand{\BSigma}[0]{\boldsymbol{\Sigma}}
\newcommand{\BTheta}[0]{\boldsymbol{\Theta}}
\newcommand{\BUpsilon}[0]{\boldsymbol{\Upsilon}}
\newcommand{\BXi}[0]{\boldsymbol{\Xi}}
\newcommand{\Balpha}[0]{\boldsymbol{\alpha}}
\newcommand{\Bbeta}[0]{\boldsymbol{\beta}}
\newcommand{\Bchi}[0]{\boldsymbol{\chi}}
\newcommand{\Bdelta}[0]{\boldsymbol{\delta}}
\newcommand{\Bepsilon}[0]{\boldsymbol{\epsilon}}
\newcommand{\Beta}[0]{\boldsymbol{\eta}}
\newcommand{\Bgamma}[0]{\boldsymbol{\gamma}}
\newcommand{\Bkappa}[0]{\boldsymbol{\kappa}}
\newcommand{\Blambda}[0]{\boldsymbol{\lambda}}
\newcommand{\Bmu}[0]{\boldsymbol{\mu}}
\newcommand{\Bnu}[0]{\boldsymbol{\nu}}
%\newcommand{\Bomega}[0]{\boldsymbol{\omega}}
\newcommand{\Bphi}[0]{\boldsymbol{\phi}}
\newcommand{\Bpi}[0]{\boldsymbol{\pi}}
\newcommand{\Bpsi}[0]{\boldsymbol{\psi}}
\newcommand{\Brho}[0]{\boldsymbol{\rho}}
\newcommand{\Bsigma}[0]{\boldsymbol{\sigma}}
%\newcommand{\Btau}[0]{\boldsymbol{\tau}}
%\newcommand{\Btheta}[0]{\boldsymbol{\theta}}
\newcommand{\Bupsilon}[0]{\boldsymbol{\upsilon}}
\newcommand{\Bvarepsilon}[0]{\boldsymbol{\varepsilon}}
\newcommand{\Bvarphi}[0]{\boldsymbol{\varphi}}
\newcommand{\Bvarpi}[0]{\boldsymbol{\varpi}}
\newcommand{\Bvarrho}[0]{\boldsymbol{\varrho}}
\newcommand{\Bvarsigma}[0]{\boldsymbol{\varsigma}}
\newcommand{\Bvartheta}[0]{\boldsymbol{\vartheta}}
\newcommand{\Bxi}[0]{\boldsymbol{\xi}}
\newcommand{\Bzeta}[0]{\boldsymbol{\zeta}}
%
%</bold and dot greek symbols>
%<infrequent>
%
%\newcommand{\AreaOp}[1]{\AName_{#1}}
%\newcommand{\Babs}[0]{\abs{\BB}}
%\newcommand{\Bcap}[0]{\hat{\BB}}
%\newcommand{\BrPrimeRej}[0]{\rcap(\rcap \wedge \Br')}
%\newcommand{\CA}[0]{\mathcal{A}}
%\newcommand{\Cos}[1]{\cos{\left({#1}\right)}}
%\newcommand{\Det}[1] {\abs{#1}}
%\newcommand{\Dsq}[2] {\frac {\partial^2 {#1}} {\partial {#2}^2}}
%\newcommand{\Exp}[1]{\exp{\left({#1}\right)}}
%\newcommand{\Norm}[1]{\left\lVert{#1}\right\rVert}
%\newcommand{\Sin}[1]{\sin{\left({#1}\right)}}
%\newcommand{\T}[0]{\text{T}}
%\newcommand{\VolumeOp}[1]{\VName_{#1}}
%\newcommand{\agrad}[0]{\Ba \cdot \nabla}
%\newcommand{\alphacap}[0]{\hat{\boldsymbol{\alpha}}}
%\newcommand{\Fcap}[0]{\hat{\BF}}
%\newcommand{\bithree}[0]{{\Bi}_3}
%\newcommand{\bxa}[0]{\Bx\Ba}
%\newcommand{\coordvec}[2]{
%\newcommand{\costheta}[0]{\acap \cdot \xcap}
%\newcommand{\ddt}[1]{\ddot{#1}}
%\newcommand{\ddu}[1] {\frac {d{#1}} {du}}
%\newcommand{\dsqxj}[2] {\frac {\partial^2 {#1}} {\partial {x_{#2}}^2}}
%\newcommand{\dtheta}[1]{\frac{d {#1}}{d \theta}}
%\newcommand{\dt}[1]{\dot{#1}}
%\newcommand{\dt}[1]{\frac{d {#1}}{dt}}
%\newcommand{\dxj}[2] {\frac {\partial {#1}} {\partial {x_{#2}}}}
%\newcommand{\halfPhi}[0]{\frac{\phi}{2}}
%\newcommand{\half}[0]{\inv{2}}
%\newcommand{\inv}[1]{\frac{1}{#1}}
%\newcommand{\laplacian}[0]{\nabla^2}
%\newcommand{\matrixoftx}[3]{
%\newcommand{\nrrp}[0]{\norm{\rcap \wedge \Br'}}
%\newcommand{\oiint}{\bigcirc \hspace{-1.4em} \int \hspace{-.8em} \int}
%\newcommand{\transpose}[1]{{#1}^{\text{T}}}
%\newcommand{\transpose}[1]{{{#1}^{\TextTranspose}}}
%\newcommand{\transpose}[1]{{{#1}^{\text{T}}}}
%\newcommand{\barA}[0]{\bar{A}}
%\newcommand{\qbar}[0]{\bar{q}}
%\newcommand{\qdotbar}[0]{\dot{\bar{q}}}
%
%</infrequent>





%\usepackage[bookmarks=true]{hyperref}

%\usepackage{color,cite,graphicx}
   % use colour in the document, put your citations as [1-4]
   % rather than [1,2,3,4] (it looks nicer, and the extended LaTeX2e
   % graphics package. 
%\usepackage{latexsym,amssymb,epsf} % do not remember if these are
   % needed, but their inclusion can not do any damage


\chapter{lut q0304}
\label{chap:lutQ0304}
%\author{Peeter Joot \quad peeter.joot@gmail.com }
\date{ March dd, 2009.  \(RCSfile: lutQ0304.tex,v \) Last \(Revision: 1.5 \) \(Date: 2009/06/14 23:51:45 \) }

%\begin{document}

Attempt to decode a fragment of 
\href{http://en.wikipedia.org/wiki/Frame_fields_in_general_relativity}{Static observers in Schwartzchild vacuum, fro the Frame field article.}

%==Example: Static observers in Schwarzschild vacuum==
%
%It will be instructive to consider in some detail a few simple examples. Consider the famous [[Schwarzschild metric|Schwarzschild vacuum]] that models spacetime outside an isolated nonspinning spherically symmetric massive object, such as a star. In most textbooks one finds the metric tensor written in terms of a static polar spherical chart, as follows:
%:<math>ds^2 = -(1-2m/r) \, dt^2 + \frac{dr^2}{1-2m/r} + r^2 \, \left( d\theta^2 + \sin(\theta)^2 \, d\phi^2 \right)</math>
%:<math> -\infty < t < \infty, \; 2 m < r < \infty, \; 0 < \theta < \pi, \; -\pi < \phi < \pi</math>
%More formally, the metric tensor can be expanded with respect to the coordinate cobasis as
%:<math>g = -(1-2m/r) \, dt \otimes dt + \frac{1}{1-2m/r} \, dr \otimes dr + r^2 \, d\theta \otimes d\theta + r^2 \sin(\theta)^2 \, d\phi \otimes d\phi</math>
A coframe can be read off from this expression:
%:<math> 
\begin{equation}\label{eqn:lutQ0304:20}
\begin{aligned}
\sigma^0 = -\sqrt{1-2m/r} \, dt, \; 
\sigma^1 = \frac{dr}{\sqrt{1-2m/r}}, \; 
\sigma^2 = r d\theta, \; 
\sigma^3 = r \sin(\theta) d\phi
\end{aligned}
\end{equation}
%</math>
%To see that this coframe really does correspond to the Schwarzschild metric tensor, just plug this coframe into
%:<math>g = -\sigma^0 \otimes \sigma^0 + \sigma^1 \otimes \sigma^1 + \sigma^2 \otimes \sigma^2 + \sigma^3 \otimes \sigma^3</math>
%
The frame dual to the coframe is
%:<math> 
\begin{equation}\label{eqn:lutQ0304:40}
\begin{aligned}
\vec{e}_0 = \frac{1}{\sqrt{1-2m/r}} \partial_t, \; 
\vec{e}_1 = \sqrt{1-2m/r} \partial_r, \; 
\vec{e}_2 = \frac{1}{r} \partial_\theta, \; 
\vec{e}_3 = \frac{1}{r \sin(\theta)} \partial_\phi
\end{aligned}
\end{equation}
%</math>
%(The minus sign on <math>\sigma^0</math> ensures that <math>\vec{e}_0</math> is ''future pointing''.) This is the frame that models the experience of '''static observers''' who use rocket engines to ''"hover" over the massive object''. 
The thrust they require to maintain their position is given by the magnitude of the acceleration vector
%:<math> 
\begin{equation}\label{eqn:lutQ0304:60}
\begin{aligned}
\nabla_{\vec{e}_0} \vec{e}_0
&= \frac{m/r^2}{\sqrt{1-2m/r}} \, \vec{e}_1  \\
&= {m/r^2} \partial_r
\end{aligned}
\end{equation}
%</math>

What is the gradient notation mean?  Is it the gradient from Doran/Lasenby?  There for a vector in a basis \(\vec{e}_\mu\), or the dual basis \(\sigma^\mu\) is

\begin{equation}\label{eqn:lutQ0304:80}
\begin{aligned}
x 
&= x_\mu \sigma^\mu \\
&= x^\mu \vec{e}_\mu \\
\end{aligned}
\end{equation}

The gradient can be expressed in either either the basis or its dual
\begin{equation}\label{eqn:lutQ0304:100}
\begin{aligned}
\grad 
&= \sum_\mu \sigma^\mu \frac{\partial}{\partial x^\mu} \equiv \sigma^\mu \partial_\mu \\
&= \sum_\mu \vec{e}_\mu \frac{\partial}{\partial x_\mu} \equiv \vec{e}_\mu \partial_\mu
\end{aligned}
\end{equation}

Perhaps such a gradient is related to this GR frame gradient?  Let us try applying that to \(\vec{e}_0\)

\begin{equation}\label{eqn:lutQ0304:120}
\begin{aligned}
\grad \vec{e}_0 
&= \sigma^\mu \partial_\mu \vec{e}_0 \\
&=
\left( -\sqrt{1-2m/r} \partial_t
+\frac{dr}{\sqrt{1-2m/r}} \partial_r
+r d\theta \partial_\theta
+r \sin(\theta) d\phi \partial_\phi \right) \frac{1}{\sqrt{1-2m/r}} \partial_t \\
&=
\frac{dr}{\sqrt{1-2m/r}} \partial_r \frac{1}{\sqrt{1-2m/r}} \partial_t \\
&=
\frac{dr}{\sqrt{1-2m/r}} (-1/2) \frac{1}{(\sqrt{1-2m/r})^3} (-2m)(-1/r^2) \partial_t \\
&=
\frac{dr}{\sqrt{1-2m/r}} (-1) \frac{1}{(\sqrt{1-2m/r})^3} (m)(1/r^2) \partial_t \\
&=
-\frac{(m/r^2)dr}{(1-2m/r)^2} \partial_t \\
&=
-\sigma^1 \frac{(m/r^2)}{(1-2m/r)^{3/2}} \partial_t \\
&=
-\sigma^1 \frac{(m/r^2)}{(1-2m/r)} \vec{e}_0 \\
\end{aligned}
\end{equation}

This has got the \(m/r^2\), but also an extra term in the denominator 

%(which is justifiable in a GA context by vectorizing the Euler-Lagrange equations)

%This is radially outward pointing, since the observers need to accelerate ''away'' from the object to avoid falling toward it. On the other hand, the spatially projected Fermi derivatives of the spatial basis vectors (with respect to <math>\vec{e}_0</math>) vanish, so this is a nonspinning frame.
%
%The components of various tensorial quantities with respect to our frame and its dual coframe can now be computed.
%
%For example, the [[electrogravitic tensor|tidal tensor]] for our static observers is defined using tensor notation (for a coordinate basis) as
%:<math> E[X]_{ab} = R_{ambn} \, X^m \, X^n</math>
%where we write <math>\vec{X} = \vec{e}_0 </math> to avoid cluttering the notation. Its only non-zero components with respect to our coframe turn out to be
%:<math> E[X]_{11} = -2m/r^3, \; E[X]_{22} = E[X]_{33} = m/r^3</math>
%The corresponding coordinate basis components are
%:<math> E[X]_{rr} = -2m/r^2/(1-2m/r), \; E[X]_{\theta \theta} = m/r, \; E[X]_{\phi \phi} = m \sin(\theta)^2/r</math>
%
%(A quick note concerning notation: many authors put [[caret]]s over ''abstract'' indices referring to a frame. When writing down ''specific components'', it is convenient to denote frame components by 0,1,2,3 and coordinate components by <math>t,r,\theta,\phi</math>. Since an expression like <math>S_{ab} = 36 m/r</math> does not make sense as a [[tensor equation]], there should be no possibility of confusion.)
%
%Compare the [[tidal tensor]] <math>\Phi</math> of Newtonian gravity, which is the '''[[traceless]] part''' of the [[Hessian matrix|Hessian]] of the gravitational potential <math>U</math>. Using tensor notation for a tensor field defined on three-dimensional euclidean space, this can be written
%:<math>\Phi_{ij} = U_{,i j} - \frac{1}{3} {U^{,k}}_{,k} \, \eta_{ij} </math>
%The reader may wish to crank this through (notice that the trace term actually vanishes identically when U is harmonic) and compare results with the following elementary approach:
%we can compare the gravitational forces on two nearby observers lying on the same radial line:
%:<math> m/(r+h)^2 - m/r^2 = -2m/r^3 \, h + 3m/r^4 \, h^2 + O(h^3) </math>
%Because in discussing tensors we are dealing with [[multilinear algebra]], we retain only first order terms, so <math>\Phi_{11} = -2m/r^3</math>. Similarly, we can compare the gravitational force on two nearby observers lying on the same sphere <math>r = r_0</math>. Using some elementary trigonometry and the small angle approximation, we find that the force vectors differ by a vector tangent to the sphere which has magnitude
%:<math> \frac{m}{r_0^2} \, \sin(\theta) \approx \frac{m}{r_0^2} \, \frac{h}{r_0} = \frac{m}{r_0^3} \, h</math>
%By using the small angle approximation, we have ignored all terms of order <math>O(h^2)</math>, so the tangential components are <math>\Phi_{22} = \Phi_{33} = m/r^3</math>. Here, we are referring to the obvious frame obtained from the polar spherical chart for our three-dimensional euclidean space:
%:<math> \vec{\epsilon}_1 = \partial_r, \; \vec{\epsilon}_2 = \frac{1}{r} \, \partial_\theta, \; \vec{\epsilon}_3 = \frac{1}{r \sin \theta} \, \partial_\phi</math>
%
%Plainly, the coordinate components <math>E[X]_{\theta \theta}, \, E[X]_{\phi \phi}</math> computed above do not even scale the right way, so they clearly cannot correspond to what an observer will measure even approximately. (By coincidence, the Newtonian tidal tensor components agree exactly with the relativistic tidal tensor components we wrote out above.)
%

%\maketitle{}
%\tableofcontents
%\section{}

%\bibliographystyle{plainnat}
%\bibliography{myrefs}

%\end{document}

\documentclass{article}      % Specifies the document class

\usepackage{amsmath}
\usepackage{mathpazo}

%
% shorthand for bold symbols, convenient for vectors and matrices
%
\newcommand{\Ba}[0]{\mathbf{a}}
\newcommand{\Bb}[0]{\mathbf{b}}
\newcommand{\Bc}[0]{\mathbf{c}}
\newcommand{\Bd}[0]{\mathbf{d}}
\newcommand{\Be}[0]{\mathbf{e}}
\newcommand{\Bf}[0]{\mathbf{f}}
\newcommand{\Bg}[0]{\mathbf{g}}
\newcommand{\Bh}[0]{\mathbf{h}}
\newcommand{\Bi}[0]{\mathbf{i}}
\newcommand{\Bj}[0]{\mathbf{j}}
\newcommand{\Bk}[0]{\mathbf{k}}
\newcommand{\Bl}[0]{\mathbf{l}}
\newcommand{\Bm}[0]{\mathbf{m}}
\newcommand{\Bn}[0]{\mathbf{n}}
\newcommand{\Bo}[0]{\mathbf{o}}
\newcommand{\Bp}[0]{\mathbf{p}}
\newcommand{\Bq}[0]{\mathbf{q}}
\newcommand{\Br}[0]{\mathbf{r}}
\newcommand{\Bs}[0]{\mathbf{s}}
\newcommand{\Bt}[0]{\mathbf{t}}
\newcommand{\Bu}[0]{\mathbf{u}}
\newcommand{\Bv}[0]{\mathbf{v}}
\newcommand{\Bw}[0]{\mathbf{w}}
\newcommand{\Bx}[0]{\mathbf{x}}
\newcommand{\By}[0]{\mathbf{y}}
\newcommand{\Bz}[0]{\mathbf{z}}
\newcommand{\BA}[0]{\mathbf{A}}
\newcommand{\BB}[0]{\mathbf{B}}
\newcommand{\BC}[0]{\mathbf{C}}
\newcommand{\BD}[0]{\mathbf{D}}
\newcommand{\BE}[0]{\mathbf{E}}
\newcommand{\BF}[0]{\mathbf{F}}
\newcommand{\BG}[0]{\mathbf{G}}
\newcommand{\BH}[0]{\mathbf{H}}
\newcommand{\BI}[0]{\mathbf{I}}
\newcommand{\BJ}[0]{\mathbf{J}}
\newcommand{\BK}[0]{\mathbf{K}}
\newcommand{\BL}[0]{\mathbf{L}}
\newcommand{\BM}[0]{\mathbf{M}}
\newcommand{\BN}[0]{\mathbf{N}}
\newcommand{\BO}[0]{\mathbf{O}}
\newcommand{\BP}[0]{\mathbf{P}}
\newcommand{\BQ}[0]{\mathbf{Q}}
\newcommand{\BR}[0]{\mathbf{R}}
\newcommand{\BS}[0]{\mathbf{S}}
\newcommand{\BT}[0]{\mathbf{T}}
\newcommand{\BU}[0]{\mathbf{U}}
\newcommand{\BV}[0]{\mathbf{V}}
\newcommand{\BW}[0]{\mathbf{W}}
\newcommand{\BX}[0]{\mathbf{X}}
\newcommand{\BY}[0]{\mathbf{Y}}
\newcommand{\BZ}[0]{\mathbf{Z}}

\newcommand{\Bzero}[0]{\mathbf{0}}
\newcommand{\Btheta}[0]{\boldsymbol{\theta}}
\newcommand{\Btau}[0]{\boldsymbol{\tau}}
\newcommand{\Bomega}[0]{\boldsymbol{\omega}}

%
% shorthand for unit vectors
%
\newcommand{\acap}[0]{\hat{\Ba}}
\newcommand{\bcap}[0]{\hat{\Bb}}
\newcommand{\ccap}[0]{\hat{\Bc}}
\newcommand{\dcap}[0]{\hat{\Bd}}
\newcommand{\ecap}[0]{\hat{\Be}}
\newcommand{\fcap}[0]{\hat{\Bf}}
\newcommand{\gcap}[0]{\hat{\Bg}}
\newcommand{\hcap}[0]{\hat{\Bh}}
\newcommand{\icap}[0]{\hat{\Bi}}
\newcommand{\jcap}[0]{\hat{\Bj}}
\newcommand{\kcap}[0]{\hat{\Bk}}
\newcommand{\lcap}[0]{\hat{\Bl}}
\newcommand{\mcap}[0]{\hat{\Bm}}
\newcommand{\ncap}[0]{\hat{\Bn}}
\newcommand{\ocap}[0]{\hat{\Bo}}
\newcommand{\pcap}[0]{\hat{\Bp}}
\newcommand{\qcap}[0]{\hat{\Bq}}
\newcommand{\rcap}[0]{\hat{\Br}}
\newcommand{\scap}[0]{\hat{\Bs}}
\newcommand{\tcap}[0]{\hat{\Bt}}
\newcommand{\ucap}[0]{\hat{\Bu}}
\newcommand{\vcap}[0]{\hat{\Bv}}
\newcommand{\wcap}[0]{\hat{\Bw}}
\newcommand{\xcap}[0]{\hat{\Bx}}
\newcommand{\ycap}[0]{\hat{\By}}
\newcommand{\zcap}[0]{\hat{\Bz}}
\newcommand{\thetacap}[0]{\hat{\Btheta}}

%
% to write R^n and C^n in a distinguishable fashion.  Perhaps change this
% to the double lined characters upon figuring out how to do so.
%
\newcommand{\C}[1]{$\mathbb{C}^{#1}$}
\newcommand{\R}[1]{$\mathbb{R}^{#1}$}

%
% various generally useful helpers
%

% derivative of #1 wrt. #2:
\newcommand{\D}[2] {\frac {d#2} {d#1}}

\newcommand{\inv}[1]{\frac{1}{#1}}
\newcommand{\cross}[0]{\times}

\newcommand{\abs}[1]{\lvert{#1}\rvert}
\newcommand{\norm}[1]{\lVert{#1}\rVert}
\newcommand{\innerprod}[2]{\langle{#1}, {#2}\rangle}
\newcommand{\dotprod}[2]{{#1} \cdot {#2}}
\newcommand{\bdotprod}[2]{\left({#1} \cdot {#2}\right)}
\newcommand{\crossprod}[2]{{#1} \cross {#2}}
\newcommand{\tripleprod}[3]{\dotprod{\left(\crossprod{#1}{#2}\right)}{#3}}

\DeclareMathOperator{\Proj}{Proj}
\DeclareMathOperator{\Span}{span}
\DeclareMathOperator{\Sgn}{sgn}
\DeclareMathOperator{\Area}{Area}
\DeclareMathOperator{\Volume}{Volume}

%
% A few miscellaneous things specific to this document
%
\newcommand{\crossop}[1]{\crossprod{#1}{}}

% R2 vector.
\newcommand{\VectorTwo}[2]{
\begin{bmatrix}
 {#1} \\
 {#2}
\end{bmatrix}
}

\newcommand{\VectorN}[1]{
\begin{bmatrix}
{#1}_1 \\
{#1}_2 \\
\vdots \\
{#1}_N \\
\end{bmatrix}
}

\newcommand{\DETuvij}[4]{
\begin{vmatrix}
 {#1}_{#3} & {#1}_{#4} \\
 {#2}_{#3} & {#2}_{#4}
\end{vmatrix}
}

\newcommand{\DETuvwijk}[6]{
\begin{vmatrix}
 {#1}_{#4} & {#1}_{#5} & {#1}_{#6} \\
 {#2}_{#4} & {#2}_{#5} & {#2}_{#6} \\
 {#3}_{#4} & {#3}_{#5} & {#3}_{#6}
\end{vmatrix}
}

\newcommand{\DETuvwxijkl}[8]{
\begin{vmatrix}
 {#1}_{#5} & {#1}_{#6} & {#1}_{#7} & {#1}_{#8} \\
 {#2}_{#5} & {#2}_{#6} & {#2}_{#7} & {#2}_{#8} \\
 {#3}_{#5} & {#3}_{#6} & {#3}_{#7} & {#3}_{#8} \\
 {#4}_{#5} & {#4}_{#6} & {#4}_{#7} & {#4}_{#8} \\
\end{vmatrix}
}

%\newcommand{\DETuvwxyijklm}[10]{
%\begin{vmatrix}
% {#1}_{#6} & {#1}_{#7} & {#1}_{#8} & {#1}_{#9} & {#1}_{#10} \\
% {#2}_{#6} & {#2}_{#7} & {#2}_{#8} & {#2}_{#9} & {#2}_{#10} \\
% {#3}_{#6} & {#3}_{#7} & {#3}_{#8} & {#3}_{#9} & {#3}_{#10} \\
% {#4}_{#6} & {#4}_{#7} & {#4}_{#8} & {#4}_{#9} & {#4}_{#10} \\
% {#5}_{#6} & {#5}_{#7} & {#5}_{#8} & {#5}_{#9} & {#5}_{#10}
%\end{vmatrix}
%}

% R3 vector.
\newcommand{\VectorThree}[3]{
\begin{bmatrix}
 {#1} \\
 {#2} \\
 {#3}
\end{bmatrix}
}


\newcommand{\PD}[2]{\frac{\partial {#2}}{\partial {#1}}}
\newcommand{\xdot}[0]{\dot{x}}
\newcommand{\xddot}[0]{\ddot{x}}

%
% The real thing:
%

\usepackage[bookmarks=true]{hyperref}

                             % The preamble begins here.
%\title{} % Declares the document's title.
%\author{Peeter Joot \quad peeter.joot@gmail.com}         % Declares the author's name.
%\date{ Last Revision: $Date: 2009/02/22 15:11:52 $ } % Deleting this command produces today's date.

\begin{document}             % End of preamble and beginning of text.

%\maketitle{}

%\tableofcontents

%\section{}

Yes, $A$ is a vector, and the spacetime basis is $\{\gamma_\mu\}$.

Coordinates representation is done by dotting with the reciprocal frame
vectors satisifying the following:

\begin{align*}
\gamma_\mu \cdot \gamma^\nu = {\delta^\nu}_\mu
\end{align*}

For example for $A$

\begin{align*}
A &= a^\alpha \gamma_\alpha \\
A \cdot \gamma^\beta
&= a^\alpha \gamma_\alpha \cdot \gamma^\beta \\
&= a^\alpha {\delta_\alpha}^\beta \\
&= a^\beta \\
\implies \\
A &= \left(A \cdot \gamma^\alpha\right) \gamma_\alpha \equiv A^\alpha \gamma_\alpha \\
\end{align*}

Similarily, computation of the coordinates in terms of the reciprocal frame
one has

\begin{align*}
A &= \left(A \cdot \gamma_\alpha\right) \gamma^\alpha \equiv A_\alpha \gamma^\alpha \\
\end{align*}

%My view of the tensor form is generally what you get when you drop the basis, looking only at the coordinates.  I've found
%that there's often reasons to temporarily introduce tensor representations, but that they can often be temporary.  Some of the
%calculations I've attempted can probably be done with non-tensor methods, but they aren't too bad once you get used to them.
Using the above coordinate representation $A \cdot \gamma_\mu$ (yes, it's a scalar), can be observed to be:

\begin{align*}
A \cdot \gamma_\mu 
&= \left( A_\nu \gamma^\nu \right) \cdot \gamma_\mu \\
&= A_\nu {\delta^\nu}_\mu \\
&= A_\mu
\end{align*}

I see that I assumed $v$ was understood to be:

\begin{align*}
x &= x^\mu \gamma_\mu \\
v &= \frac{dx}{d\tau} = \xdot^\mu \gamma_\mu
\end{align*}

(that's one of the missing steps to get from $\PD{\xdot^\mu}{A \cdot v}$ to $A \cdot \gamma_\mu$).

Perhaps this makes more sense:

\begin{align*}
%\frac{d}{d\tau}
\PD{\xdot^\mu}{A \cdot v} 
&= \PD{\xdot^\mu}{} A \cdot \xdot^\nu \gamma_\nu \\
&= A \left( \PD{\xdot^\mu}{\xdot^\nu} \right) \cdot \gamma_\nu \\
&= {\delta^\mu}_\nu A \cdot \gamma_\nu \\
&= A \cdot \gamma_\mu \\
\end{align*}

so for the derivative

\begin{align*}
\frac{d}{d\tau} \PD{\xdot^\mu}{A \cdot v} 
&= \frac{d}{d\tau} A \cdot \gamma_\mu \\
&= \frac{d}{d\tau} (A^\nu \gamma_\nu) \cdot \gamma_\mu \\
&= \frac{dA^\nu}{d\tau} \gamma_\nu \cdot \gamma_\mu \\
\end{align*}

You included a nice little box summary of Maxwell's equation in tensor form.  That is pretty much identical to the curl
form, but to see the equivalance the product operations on a wedge are required.  Before mentioning any of that does this
part above now make sense?  I'll revise my notes to try to clarify it.

%%\bibliographystyle{plain}
%\bibliographystyle{plainnat} % supposed to allow for \url use.
%\bibliography{myrefs}      % expects file "myrefs.bib"

\end{document}               % End of document.

%
% Copyright � 2012 Peeter Joot.  All Rights Reserved.
% Licenced as described in the file LICENSE under the root directory of this GIT repository.
%

% 
% 
%\documentclass{article}

%\usepackage{amsmath}
\usepackage{mathpazo}

%
% shorthand for bold symbols, convenient for vectors and matrices
%
\newcommand{\Ba}[0]{\mathbf{a}}
\newcommand{\Bb}[0]{\mathbf{b}}
\newcommand{\Bc}[0]{\mathbf{c}}
\newcommand{\Bd}[0]{\mathbf{d}}
\newcommand{\Be}[0]{\mathbf{e}}
\newcommand{\Bf}[0]{\mathbf{f}}
\newcommand{\Bg}[0]{\mathbf{g}}
\newcommand{\Bh}[0]{\mathbf{h}}
\newcommand{\Bi}[0]{\mathbf{i}}
\newcommand{\Bj}[0]{\mathbf{j}}
\newcommand{\Bk}[0]{\mathbf{k}}
\newcommand{\Bl}[0]{\mathbf{l}}
\newcommand{\Bm}[0]{\mathbf{m}}
\newcommand{\Bn}[0]{\mathbf{n}}
\newcommand{\Bo}[0]{\mathbf{o}}
\newcommand{\Bp}[0]{\mathbf{p}}
\newcommand{\Bq}[0]{\mathbf{q}}
\newcommand{\Br}[0]{\mathbf{r}}
\newcommand{\Bs}[0]{\mathbf{s}}
\newcommand{\Bt}[0]{\mathbf{t}}
\newcommand{\Bu}[0]{\mathbf{u}}
\newcommand{\Bv}[0]{\mathbf{v}}
\newcommand{\Bw}[0]{\mathbf{w}}
\newcommand{\Bx}[0]{\mathbf{x}}
\newcommand{\By}[0]{\mathbf{y}}
\newcommand{\Bz}[0]{\mathbf{z}}
\newcommand{\BA}[0]{\mathbf{A}}
\newcommand{\BB}[0]{\mathbf{B}}
\newcommand{\BC}[0]{\mathbf{C}}
\newcommand{\BD}[0]{\mathbf{D}}
\newcommand{\BE}[0]{\mathbf{E}}
\newcommand{\BF}[0]{\mathbf{F}}
\newcommand{\BG}[0]{\mathbf{G}}
\newcommand{\BH}[0]{\mathbf{H}}
\newcommand{\BI}[0]{\mathbf{I}}
\newcommand{\BJ}[0]{\mathbf{J}}
\newcommand{\BK}[0]{\mathbf{K}}
\newcommand{\BL}[0]{\mathbf{L}}
\newcommand{\BM}[0]{\mathbf{M}}
\newcommand{\BN}[0]{\mathbf{N}}
\newcommand{\BO}[0]{\mathbf{O}}
\newcommand{\BP}[0]{\mathbf{P}}
\newcommand{\BQ}[0]{\mathbf{Q}}
\newcommand{\BR}[0]{\mathbf{R}}
\newcommand{\BS}[0]{\mathbf{S}}
\newcommand{\BT}[0]{\mathbf{T}}
\newcommand{\BU}[0]{\mathbf{U}}
\newcommand{\BV}[0]{\mathbf{V}}
\newcommand{\BW}[0]{\mathbf{W}}
\newcommand{\BX}[0]{\mathbf{X}}
\newcommand{\BY}[0]{\mathbf{Y}}
\newcommand{\BZ}[0]{\mathbf{Z}}

\newcommand{\Bzero}[0]{\mathbf{0}}
\newcommand{\Btheta}[0]{\boldsymbol{\theta}}
\newcommand{\Btau}[0]{\boldsymbol{\tau}}
\newcommand{\Bomega}[0]{\boldsymbol{\omega}}

%
% shorthand for unit vectors
%
\newcommand{\acap}[0]{\hat{\Ba}}
\newcommand{\bcap}[0]{\hat{\Bb}}
\newcommand{\ccap}[0]{\hat{\Bc}}
\newcommand{\dcap}[0]{\hat{\Bd}}
\newcommand{\ecap}[0]{\hat{\Be}}
\newcommand{\fcap}[0]{\hat{\Bf}}
\newcommand{\gcap}[0]{\hat{\Bg}}
\newcommand{\hcap}[0]{\hat{\Bh}}
\newcommand{\icap}[0]{\hat{\Bi}}
\newcommand{\jcap}[0]{\hat{\Bj}}
\newcommand{\kcap}[0]{\hat{\Bk}}
\newcommand{\lcap}[0]{\hat{\Bl}}
\newcommand{\mcap}[0]{\hat{\Bm}}
\newcommand{\ncap}[0]{\hat{\Bn}}
\newcommand{\ocap}[0]{\hat{\Bo}}
\newcommand{\pcap}[0]{\hat{\Bp}}
\newcommand{\qcap}[0]{\hat{\Bq}}
\newcommand{\rcap}[0]{\hat{\Br}}
\newcommand{\scap}[0]{\hat{\Bs}}
\newcommand{\tcap}[0]{\hat{\Bt}}
\newcommand{\ucap}[0]{\hat{\Bu}}
\newcommand{\vcap}[0]{\hat{\Bv}}
\newcommand{\wcap}[0]{\hat{\Bw}}
\newcommand{\xcap}[0]{\hat{\Bx}}
\newcommand{\ycap}[0]{\hat{\By}}
\newcommand{\zcap}[0]{\hat{\Bz}}
\newcommand{\thetacap}[0]{\hat{\Btheta}}

%
% to write R^n and C^n in a distinguishable fashion.  Perhaps change this
% to the double lined characters upon figuring out how to do so.
%
\newcommand{\C}[1]{$\mathbb{C}^{#1}$}
\newcommand{\R}[1]{$\mathbb{R}^{#1}$}

%
% various generally useful helpers
%

% derivative of #1 wrt. #2:
\newcommand{\D}[2] {\frac {d#2} {d#1}}

\newcommand{\inv}[1]{\frac{1}{#1}}
\newcommand{\cross}[0]{\times}

\newcommand{\abs}[1]{\lvert{#1}\rvert}
\newcommand{\norm}[1]{\lVert{#1}\rVert}
\newcommand{\innerprod}[2]{\langle{#1}, {#2}\rangle}
\newcommand{\dotprod}[2]{{#1} \cdot {#2}}
\newcommand{\bdotprod}[2]{\left({#1} \cdot {#2}\right)}
\newcommand{\crossprod}[2]{{#1} \cross {#2}}
\newcommand{\tripleprod}[3]{\dotprod{\left(\crossprod{#1}{#2}\right)}{#3}}

\DeclareMathOperator{\Proj}{Proj}
\DeclareMathOperator{\Span}{span}
\DeclareMathOperator{\Sgn}{sgn}
\DeclareMathOperator{\Area}{Area}
\DeclareMathOperator{\Volume}{Volume}

%
% A few miscellaneous things specific to this document
%
\newcommand{\crossop}[1]{\crossprod{#1}{}}

% R2 vector.
\newcommand{\VectorTwo}[2]{
\begin{bmatrix}
 {#1} \\
 {#2}
\end{bmatrix}
}

\newcommand{\VectorN}[1]{
\begin{bmatrix}
{#1}_1 \\
{#1}_2 \\
\vdots \\
{#1}_N \\
\end{bmatrix}
}

\newcommand{\DETuvij}[4]{
\begin{vmatrix}
 {#1}_{#3} & {#1}_{#4} \\
 {#2}_{#3} & {#2}_{#4}
\end{vmatrix}
}

\newcommand{\DETuvwijk}[6]{
\begin{vmatrix}
 {#1}_{#4} & {#1}_{#5} & {#1}_{#6} \\
 {#2}_{#4} & {#2}_{#5} & {#2}_{#6} \\
 {#3}_{#4} & {#3}_{#5} & {#3}_{#6}
\end{vmatrix}
}

\newcommand{\DETuvwxijkl}[8]{
\begin{vmatrix}
 {#1}_{#5} & {#1}_{#6} & {#1}_{#7} & {#1}_{#8} \\
 {#2}_{#5} & {#2}_{#6} & {#2}_{#7} & {#2}_{#8} \\
 {#3}_{#5} & {#3}_{#6} & {#3}_{#7} & {#3}_{#8} \\
 {#4}_{#5} & {#4}_{#6} & {#4}_{#7} & {#4}_{#8} \\
\end{vmatrix}
}

%\newcommand{\DETuvwxyijklm}[10]{
%\begin{vmatrix}
% {#1}_{#6} & {#1}_{#7} & {#1}_{#8} & {#1}_{#9} & {#1}_{#10} \\
% {#2}_{#6} & {#2}_{#7} & {#2}_{#8} & {#2}_{#9} & {#2}_{#10} \\
% {#3}_{#6} & {#3}_{#7} & {#3}_{#8} & {#3}_{#9} & {#3}_{#10} \\
% {#4}_{#6} & {#4}_{#7} & {#4}_{#8} & {#4}_{#9} & {#4}_{#10} \\
% {#5}_{#6} & {#5}_{#7} & {#5}_{#8} & {#5}_{#9} & {#5}_{#10}
%\end{vmatrix}
%}

% R3 vector.
\newcommand{\VectorThree}[3]{
\begin{bmatrix}
 {#1} \\
 {#2} \\
 {#3}
\end{bmatrix}
}


%%<misc>
%
\newcommand{\Abs}[1]{{\left\lvert{#1}\right\rvert}}
\newcommand{\spacegrad}[0]{\boldsymbol{\nabla}}
\newcommand{\grad}[0]{\nabla}
\newcommand{\LL}[0]{\mathcal{L}}

% == \partial_{#1} {#2}
\newcommand{\PD}[2]{\frac{\partial {#2}}{\partial {#1}}}
% inline variant
\newcommand{\PDi}[2]{{\partial {#2}}/{\partial {#1}}}

\newcommand{\PDD}[3]{\frac{\partial^2 {#3}}{\partial {#1}\partial {#2}}}
%\newcommand{\PDd}[2]{\frac{\partial^2 {#2}}{{\partial{#1}}^2}}
\newcommand{\PDsq}[2]{\frac{\partial^2 {#2}}{(\partial {#1})^2}}

\newcommand{\Partial}[2]{\frac{\partial {#1}}{\partial {#2}}}
\DeclareMathOperator{\RejName}{Rej}
\newcommand{\Rej}[2]{\RejName_{#1}\left( {#2} \right)}
\newcommand{\Rm}[1]{\mathbb{R}^{#1}}
\newcommand{\Cm}[1]{\mathbb{C}^{#1}}
\newcommand{\conj}[0]{{*}}

%</misc>

% <grade selection>
%
\newcommand{\gpgrade}[2] {{\left\langle{{#1}}\right\rangle}_{#2}}

\newcommand{\gpgradezero}[1] {\gpgrade{#1}{}}
%\newcommand{\gpscalargrade}[1] {{\left\langle{{#1}}\right\rangle}}
%\newcommand{\gpgradezero}[1] {\gpgrade{#1}{0}}

%\newcommand{\gpgradeone}[1] {{\left\langle{{#1}}\right\rangle}_{1}}
\newcommand{\gpgradeone}[1] {\gpgrade{#1}{1}}

\newcommand{\gpgradetwo}[1] {\gpgrade{#1}{2}}
\newcommand{\gpgradethree}[1] {\gpgrade{#1}{3}}
\newcommand{\gpgradefour}[1] {\gpgrade{#1}{4}}
%
% </grade selection>



\newcommand{\adot}[0]{{\dot{a}}}
\newcommand{\bdot}[0]{{\dot{b}}}
% taken for centered dot:
%\newcommand{\cdot}[0]{{\dot{c}}}
%\newcommand{\ddot}[0]{{\dot{d}}}
\newcommand{\edot}[0]{{\dot{e}}}
\newcommand{\fdot}[0]{{\dot{f}}}
\newcommand{\gdot}[0]{{\dot{g}}}
\newcommand{\hdot}[0]{{\dot{h}}}
\newcommand{\idot}[0]{{\dot{i}}}
\newcommand{\jdot}[0]{{\dot{j}}}
\newcommand{\kdot}[0]{{\dot{k}}}
\newcommand{\ldot}[0]{{\dot{l}}}
\newcommand{\mdot}[0]{{\dot{m}}}
\newcommand{\ndot}[0]{{\dot{n}}}
%\newcommand{\odot}[0]{{\dot{o}}}
\newcommand{\pdot}[0]{{\dot{p}}}
\newcommand{\qdot}[0]{{\dot{q}}}
\newcommand{\rdot}[0]{{\dot{r}}}
\newcommand{\sdot}[0]{{\dot{s}}}
\newcommand{\tdot}[0]{{\dot{t}}}
\newcommand{\udot}[0]{{\dot{u}}}
\newcommand{\vdot}[0]{{\dot{v}}}
\newcommand{\wdot}[0]{{\dot{w}}}
\newcommand{\xdot}[0]{{\dot{x}}}
\newcommand{\ydot}[0]{{\dot{y}}}
\newcommand{\zdot}[0]{{\dot{z}}}
\newcommand{\addot}[0]{{\ddot{a}}}
\newcommand{\bddot}[0]{{\ddot{b}}}
\newcommand{\cddot}[0]{{\ddot{c}}}
%\newcommand{\dddot}[0]{{\ddot{d}}}
\newcommand{\eddot}[0]{{\ddot{e}}}
\newcommand{\fddot}[0]{{\ddot{f}}}
\newcommand{\gddot}[0]{{\ddot{g}}}
\newcommand{\hddot}[0]{{\ddot{h}}}
\newcommand{\iddot}[0]{{\ddot{i}}}
\newcommand{\jddot}[0]{{\ddot{j}}}
\newcommand{\kddot}[0]{{\ddot{k}}}
\newcommand{\lddot}[0]{{\ddot{l}}}
\newcommand{\mddot}[0]{{\ddot{m}}}
\newcommand{\nddot}[0]{{\ddot{n}}}
\newcommand{\oddot}[0]{{\ddot{o}}}
\newcommand{\pddot}[0]{{\ddot{p}}}
\newcommand{\qddot}[0]{{\ddot{q}}}
\newcommand{\rddot}[0]{{\ddot{r}}}
\newcommand{\sddot}[0]{{\ddot{s}}}
\newcommand{\tddot}[0]{{\ddot{t}}}
\newcommand{\uddot}[0]{{\ddot{u}}}
\newcommand{\vddot}[0]{{\ddot{v}}}
\newcommand{\wddot}[0]{{\ddot{w}}}
\newcommand{\xddot}[0]{{\ddot{x}}}
\newcommand{\yddot}[0]{{\ddot{y}}}
\newcommand{\zddot}[0]{{\ddot{z}}}

%<bold and dot greek symbols>
%

\newcommand{\Deltadot}[0]{{\dot{\Delta}}}
\newcommand{\Gammadot}[0]{{\dot{\Gamma}}}
\newcommand{\Lambdadot}[0]{{\dot{\Lambda}}}
\newcommand{\Omegadot}[0]{{\dot{\Omega}}}
\newcommand{\Phidot}[0]{{\dot{\Phi}}}
\newcommand{\Pidot}[0]{{\dot{\Pi}}}
\newcommand{\Psidot}[0]{{\dot{\Psi}}}
\newcommand{\Sigmadot}[0]{{\dot{\Sigma}}}
\newcommand{\Thetadot}[0]{{\dot{\Theta}}}
\newcommand{\Upsilondot}[0]{{\dot{\Upsilon}}}
\newcommand{\Xidot}[0]{{\dot{\Xi}}}
\newcommand{\alphadot}[0]{{\dot{\alpha}}}
\newcommand{\betadot}[0]{{\dot{\beta}}}
\newcommand{\chidot}[0]{{\dot{\chi}}}
\newcommand{\deltadot}[0]{{\dot{\delta}}}
\newcommand{\epsilondot}[0]{{\dot{\epsilon}}}
\newcommand{\etadot}[0]{{\dot{\eta}}}
\newcommand{\gammadot}[0]{{\dot{\gamma}}}
\newcommand{\kappadot}[0]{{\dot{\kappa}}}
\newcommand{\lambdadot}[0]{{\dot{\lambda}}}
\newcommand{\mudot}[0]{{\dot{\mu}}}
\newcommand{\nudot}[0]{{\dot{\nu}}}
\newcommand{\omegadot}[0]{{\dot{\omega}}}
\newcommand{\phidot}[0]{{\dot{\phi}}}
\newcommand{\pidot}[0]{{\dot{\pi}}}
\newcommand{\psidot}[0]{{\dot{\psi}}}
\newcommand{\rhodot}[0]{{\dot{\rho}}}
\newcommand{\sigmadot}[0]{{\dot{\sigma}}}
\newcommand{\taudot}[0]{{\dot{\tau}}}
\newcommand{\thetadot}[0]{{\dot{\theta}}}
\newcommand{\upsilondot}[0]{{\dot{\upsilon}}}
\newcommand{\varepsilondot}[0]{{\dot{\varepsilon}}}
\newcommand{\varphidot}[0]{{\dot{\varphi}}}
\newcommand{\varpidot}[0]{{\dot{\varpi}}}
\newcommand{\varrhodot}[0]{{\dot{\varrho}}}
\newcommand{\varsigmadot}[0]{{\dot{\varsigma}}}
\newcommand{\varthetadot}[0]{{\dot{\vartheta}}}
\newcommand{\xidot}[0]{{\dot{\xi}}}
\newcommand{\zetadot}[0]{{\dot{\zeta}}}

\newcommand{\Deltaddot}[0]{{\ddot{\Delta}}}
\newcommand{\Gammaddot}[0]{{\ddot{\Gamma}}}
\newcommand{\Lambdaddot}[0]{{\ddot{\Lambda}}}
\newcommand{\Omegaddot}[0]{{\ddot{\Omega}}}
\newcommand{\Phiddot}[0]{{\ddot{\Phi}}}
\newcommand{\Piddot}[0]{{\ddot{\Pi}}}
\newcommand{\Psiddot}[0]{{\ddot{\Psi}}}
\newcommand{\Sigmaddot}[0]{{\ddot{\Sigma}}}
\newcommand{\Thetaddot}[0]{{\ddot{\Theta}}}
\newcommand{\Upsilonddot}[0]{{\ddot{\Upsilon}}}
\newcommand{\Xiddot}[0]{{\ddot{\Xi}}}
\newcommand{\alphaddot}[0]{{\ddot{\alpha}}}
\newcommand{\betaddot}[0]{{\ddot{\beta}}}
\newcommand{\chiddot}[0]{{\ddot{\chi}}}
\newcommand{\deltaddot}[0]{{\ddot{\delta}}}
\newcommand{\epsilonddot}[0]{{\ddot{\epsilon}}}
\newcommand{\etaddot}[0]{{\ddot{\eta}}}
\newcommand{\gammaddot}[0]{{\ddot{\gamma}}}
\newcommand{\kappaddot}[0]{{\ddot{\kappa}}}
\newcommand{\lambdaddot}[0]{{\ddot{\lambda}}}
\newcommand{\muddot}[0]{{\ddot{\mu}}}
\newcommand{\nuddot}[0]{{\ddot{\nu}}}
\newcommand{\omegaddot}[0]{{\ddot{\omega}}}
\newcommand{\phiddot}[0]{{\ddot{\phi}}}
\newcommand{\piddot}[0]{{\ddot{\pi}}}
\newcommand{\psiddot}[0]{{\ddot{\psi}}}
\newcommand{\rhoddot}[0]{{\ddot{\rho}}}
\newcommand{\sigmaddot}[0]{{\ddot{\sigma}}}
\newcommand{\tauddot}[0]{{\ddot{\tau}}}
\newcommand{\thetaddot}[0]{{\ddot{\theta}}}
\newcommand{\upsilonddot}[0]{{\ddot{\upsilon}}}
\newcommand{\varepsilonddot}[0]{{\ddot{\varepsilon}}}
\newcommand{\varphiddot}[0]{{\ddot{\varphi}}}
\newcommand{\varpiddot}[0]{{\ddot{\varpi}}}
\newcommand{\varrhoddot}[0]{{\ddot{\varrho}}}
\newcommand{\varsigmaddot}[0]{{\ddot{\varsigma}}}
\newcommand{\varthetaddot}[0]{{\ddot{\vartheta}}}
\newcommand{\xiddot}[0]{{\ddot{\xi}}}
\newcommand{\zetaddot}[0]{{\ddot{\zeta}}}

\newcommand{\BDelta}[0]{\boldsymbol{\Delta}}
\newcommand{\BGamma}[0]{\boldsymbol{\Gamma}}
\newcommand{\BLambda}[0]{\boldsymbol{\Lambda}}
\newcommand{\BOmega}[0]{\boldsymbol{\Omega}}
\newcommand{\BPhi}[0]{\boldsymbol{\Phi}}
\newcommand{\BPi}[0]{\boldsymbol{\Pi}}
\newcommand{\BPsi}[0]{\boldsymbol{\Psi}}
\newcommand{\BSigma}[0]{\boldsymbol{\Sigma}}
\newcommand{\BTheta}[0]{\boldsymbol{\Theta}}
\newcommand{\BUpsilon}[0]{\boldsymbol{\Upsilon}}
\newcommand{\BXi}[0]{\boldsymbol{\Xi}}
\newcommand{\Balpha}[0]{\boldsymbol{\alpha}}
\newcommand{\Bbeta}[0]{\boldsymbol{\beta}}
\newcommand{\Bchi}[0]{\boldsymbol{\chi}}
\newcommand{\Bdelta}[0]{\boldsymbol{\delta}}
\newcommand{\Bepsilon}[0]{\boldsymbol{\epsilon}}
\newcommand{\Beta}[0]{\boldsymbol{\eta}}
\newcommand{\Bgamma}[0]{\boldsymbol{\gamma}}
\newcommand{\Bkappa}[0]{\boldsymbol{\kappa}}
\newcommand{\Blambda}[0]{\boldsymbol{\lambda}}
\newcommand{\Bmu}[0]{\boldsymbol{\mu}}
\newcommand{\Bnu}[0]{\boldsymbol{\nu}}
%\newcommand{\Bomega}[0]{\boldsymbol{\omega}}
\newcommand{\Bphi}[0]{\boldsymbol{\phi}}
\newcommand{\Bpi}[0]{\boldsymbol{\pi}}
\newcommand{\Bpsi}[0]{\boldsymbol{\psi}}
\newcommand{\Brho}[0]{\boldsymbol{\rho}}
\newcommand{\Bsigma}[0]{\boldsymbol{\sigma}}
%\newcommand{\Btau}[0]{\boldsymbol{\tau}}
%\newcommand{\Btheta}[0]{\boldsymbol{\theta}}
\newcommand{\Bupsilon}[0]{\boldsymbol{\upsilon}}
\newcommand{\Bvarepsilon}[0]{\boldsymbol{\varepsilon}}
\newcommand{\Bvarphi}[0]{\boldsymbol{\varphi}}
\newcommand{\Bvarpi}[0]{\boldsymbol{\varpi}}
\newcommand{\Bvarrho}[0]{\boldsymbol{\varrho}}
\newcommand{\Bvarsigma}[0]{\boldsymbol{\varsigma}}
\newcommand{\Bvartheta}[0]{\boldsymbol{\vartheta}}
\newcommand{\Bxi}[0]{\boldsymbol{\xi}}
\newcommand{\Bzeta}[0]{\boldsymbol{\zeta}}
%
%</bold and dot greek symbols>
%<infrequent>
%
%\newcommand{\AreaOp}[1]{\AName_{#1}}
%\newcommand{\Babs}[0]{\abs{\BB}}
%\newcommand{\Bcap}[0]{\hat{\BB}}
%\newcommand{\BrPrimeRej}[0]{\rcap(\rcap \wedge \Br')}
%\newcommand{\CA}[0]{\mathcal{A}}
%\newcommand{\Cos}[1]{\cos{\left({#1}\right)}}
%\newcommand{\Det}[1] {\abs{#1}}
%\newcommand{\Dsq}[2] {\frac {\partial^2 {#1}} {\partial {#2}^2}}
%\newcommand{\Exp}[1]{\exp{\left({#1}\right)}}
%\newcommand{\Norm}[1]{\left\lVert{#1}\right\rVert}
%\newcommand{\Sin}[1]{\sin{\left({#1}\right)}}
%\newcommand{\T}[0]{\text{T}}
%\newcommand{\VolumeOp}[1]{\VName_{#1}}
%\newcommand{\agrad}[0]{\Ba \cdot \nabla}
%\newcommand{\alphacap}[0]{\hat{\boldsymbol{\alpha}}}
%\newcommand{\Fcap}[0]{\hat{\BF}}
%\newcommand{\bithree}[0]{{\Bi}_3}
%\newcommand{\bxa}[0]{\Bx\Ba}
%\newcommand{\coordvec}[2]{
%\newcommand{\costheta}[0]{\acap \cdot \xcap}
%\newcommand{\ddt}[1]{\ddot{#1}}
%\newcommand{\ddu}[1] {\frac {d{#1}} {du}}
%\newcommand{\dsqxj}[2] {\frac {\partial^2 {#1}} {\partial {x_{#2}}^2}}
%\newcommand{\dtheta}[1]{\frac{d {#1}}{d \theta}}
%\newcommand{\dt}[1]{\dot{#1}}
%\newcommand{\dt}[1]{\frac{d {#1}}{dt}}
%\newcommand{\dxj}[2] {\frac {\partial {#1}} {\partial {x_{#2}}}}
%\newcommand{\halfPhi}[0]{\frac{\phi}{2}}
%\newcommand{\half}[0]{\inv{2}}
%\newcommand{\inv}[1]{\frac{1}{#1}}
%\newcommand{\laplacian}[0]{\nabla^2}
%\newcommand{\matrixoftx}[3]{
%\newcommand{\nrrp}[0]{\norm{\rcap \wedge \Br'}}
%\newcommand{\oiint}{\bigcirc \hspace{-1.4em} \int \hspace{-.8em} \int}
%\newcommand{\transpose}[1]{{#1}^{\text{T}}}
%\newcommand{\transpose}[1]{{{#1}^{\TextTranspose}}}
%\newcommand{\transpose}[1]{{{#1}^{\text{T}}}}
%\newcommand{\barA}[0]{\bar{A}}
%\newcommand{\qbar}[0]{\bar{q}}
%\newcommand{\qdotbar}[0]{\dot{\bar{q}}}
%
%</infrequent>





%\usepackage[bookmarks=true]{hyperref}

%\usepackage{color,cite,graphicx}
%   % use colour in the document, put your citations as [1-4]
%   % rather than [1,2,3,4] (it looks nicer, and the extended LaTeX2e
%   % graphics package. 
%\usepackage{latexsym,amssymb,epsf} % do not remember if these are
%   % needed, but their inclusion can not do any damage


\chapter{Chapter I-III comments on Alan's book}
\label{chap:reviewAlanGabook}
%\author{Peeter Joot \quad peeter.joot@gmail.com}
\date{ Mmm dd, 2009.  \(RCSfile: reviewAlanGabook.tex,v \) Last \(Revision: 1.8 \) \(Date: 2009/06/14 23:51:45 \) }

%\begin{document}

%\maketitle{}
%\tableofcontents

\section{Chapter I}

\subsection{page 10. vectors}

More accurate or at least more general would be the use of
momentum instead of velocity in the 6n vector, but I am 
guessing this simplification has been made on purpose.


For \(n \approx 10^{23}\) you have not specified the size of the box.

\section{Chapter 3}

\subsection{page 35. matrices, properties and non-properties}

If I was a new learner of the subject I would have been uncomfortable with
the omission
of the associativity proof where you say "The proofs use uninteresting juggling of the indices".  While I agree that including this proof inline would detract
from the flow of things, I had be personally inclined to make this be
something more.  Perhaps a problem where you show how to do half of it:

\begin{equation}\label{eqn:reviewAlanGabook:20}
\begin{aligned}
(AB) C
&= 
\begin{bmatrix}
a_{ij} 
\end{bmatrix} 
\begin{bmatrix}
b_{ij} 
\end{bmatrix} C \\
&= 
\begin{bmatrix}
\sum_k a_{ik} b_{kj} 
\end{bmatrix} C \\
\end{aligned}
\end{equation}

Let \(\sum_k a_{ik} b_{kj} = d_{ij}\)

\begin{equation}\label{eqn:reviewAlanGabook:40}
\begin{aligned}
\implies
(AB) C
&= 
\begin{bmatrix}
d_{ij} 
\end{bmatrix} 
\begin{bmatrix}
c_{ij} 
\end{bmatrix} \\
&= 
\begin{bmatrix}
\sum_k d_{ik} c_{kj} 
\end{bmatrix} \\
&= 
\begin{bmatrix}
\sum_{m,k} a_{im} b_{mk} c_{kj} 
\end{bmatrix} \\
\end{aligned}
\end{equation}

Then make the expansion of the other grouping as the remainder of the exercise.

\subsection{Page 37.  ex 3.8, 3.9}

Inline matrices missing brackets.

\subsection{Page 38.  theorem 3.7}

Looks funny without qualification since one immediately thinks of left and
right inverses.

\begin{equation}\label{eqn:reviewAlanGabook:60}
\begin{aligned}
A = 
\begin{bmatrix}
1 & 0 & 0 \\
0 & 0 & 1 \\
\end{bmatrix}
\end{aligned}
\end{equation}

\begin{equation}\label{eqn:reviewAlanGabook:80}
\begin{aligned}
B = 
\begin{bmatrix}
1 & 0 \\
0 & 0 \\
0 & 1 \\
\end{bmatrix}
\end{aligned}
\end{equation}

\begin{equation}\label{eqn:reviewAlanGabook:100}
\begin{aligned}
A B = I_2
\end{aligned}
\end{equation}

ie: B is right inverse.

\subsection{page 39}

I think your notation for transpose will confuse people when they move on to other studies, since asterisk is much more normally used for Hermitian transpose,
while a text T is used for real numbers.  I had restrict this discussion to real
numbers for clarity and not use a symbol that implies more.

\subsection{page 41. change of basis problem}

If this were actually a readers first introduction to matrix math, then
if you punt change of basis to a problem like this, then without elaboration
I do not think there is much hope that the reader will be able to get much out
of this.  Either omit it, or elaborate (as a non-problem) IMO.

\subsection{page 42. systems of equations}

You mention that due to potential scope of the problem, systematic and automatic computed methods are required, but do not mention the other important aspect of this.  The traditional row reduction methods taught in a first year course have massive numerical stability issues since they omit pivots and do not touch on how to deal with near zero values.  Since you are systematically omitting the row reduction coverage in this text, I think it would be good to give a good example of how a naive solution would produce whacked answers.

In an exercise give a super quickly description of the row reduction method works and have the student apply it to something ill formed like

\begin{equation}\label{eqn:reviewAlanGabook:120}
\begin{aligned}
\begin{bmatrix}
1 & 1 \\
10^{-23} & 0 \\
\end{bmatrix}
\begin{bmatrix}
x \\
y \\
\end{bmatrix}
=
\begin{bmatrix}
1 \\
2 \\
\end{bmatrix}
\end{aligned}
\end{equation}

(or perhaps some other better example(s) from a numerical analysis text).

%\bibliographystyle{plainnat}
%\bibliography{myrefs}

%\end{document}

%
% Copyright � 2012 Peeter Joot.  All Rights Reserved.
% Licenced as described in the file LICENSE under the root directory of this GIT repository.
%

% 
% 
%\documentclass{article}

%\usepackage{amsmath}
\usepackage{mathpazo}

%
% shorthand for bold symbols, convenient for vectors and matrices
%
\newcommand{\Ba}[0]{\mathbf{a}}
\newcommand{\Bb}[0]{\mathbf{b}}
\newcommand{\Bc}[0]{\mathbf{c}}
\newcommand{\Bd}[0]{\mathbf{d}}
\newcommand{\Be}[0]{\mathbf{e}}
\newcommand{\Bf}[0]{\mathbf{f}}
\newcommand{\Bg}[0]{\mathbf{g}}
\newcommand{\Bh}[0]{\mathbf{h}}
\newcommand{\Bi}[0]{\mathbf{i}}
\newcommand{\Bj}[0]{\mathbf{j}}
\newcommand{\Bk}[0]{\mathbf{k}}
\newcommand{\Bl}[0]{\mathbf{l}}
\newcommand{\Bm}[0]{\mathbf{m}}
\newcommand{\Bn}[0]{\mathbf{n}}
\newcommand{\Bo}[0]{\mathbf{o}}
\newcommand{\Bp}[0]{\mathbf{p}}
\newcommand{\Bq}[0]{\mathbf{q}}
\newcommand{\Br}[0]{\mathbf{r}}
\newcommand{\Bs}[0]{\mathbf{s}}
\newcommand{\Bt}[0]{\mathbf{t}}
\newcommand{\Bu}[0]{\mathbf{u}}
\newcommand{\Bv}[0]{\mathbf{v}}
\newcommand{\Bw}[0]{\mathbf{w}}
\newcommand{\Bx}[0]{\mathbf{x}}
\newcommand{\By}[0]{\mathbf{y}}
\newcommand{\Bz}[0]{\mathbf{z}}
\newcommand{\BA}[0]{\mathbf{A}}
\newcommand{\BB}[0]{\mathbf{B}}
\newcommand{\BC}[0]{\mathbf{C}}
\newcommand{\BD}[0]{\mathbf{D}}
\newcommand{\BE}[0]{\mathbf{E}}
\newcommand{\BF}[0]{\mathbf{F}}
\newcommand{\BG}[0]{\mathbf{G}}
\newcommand{\BH}[0]{\mathbf{H}}
\newcommand{\BI}[0]{\mathbf{I}}
\newcommand{\BJ}[0]{\mathbf{J}}
\newcommand{\BK}[0]{\mathbf{K}}
\newcommand{\BL}[0]{\mathbf{L}}
\newcommand{\BM}[0]{\mathbf{M}}
\newcommand{\BN}[0]{\mathbf{N}}
\newcommand{\BO}[0]{\mathbf{O}}
\newcommand{\BP}[0]{\mathbf{P}}
\newcommand{\BQ}[0]{\mathbf{Q}}
\newcommand{\BR}[0]{\mathbf{R}}
\newcommand{\BS}[0]{\mathbf{S}}
\newcommand{\BT}[0]{\mathbf{T}}
\newcommand{\BU}[0]{\mathbf{U}}
\newcommand{\BV}[0]{\mathbf{V}}
\newcommand{\BW}[0]{\mathbf{W}}
\newcommand{\BX}[0]{\mathbf{X}}
\newcommand{\BY}[0]{\mathbf{Y}}
\newcommand{\BZ}[0]{\mathbf{Z}}

\newcommand{\Bzero}[0]{\mathbf{0}}
\newcommand{\Btheta}[0]{\boldsymbol{\theta}}
\newcommand{\Btau}[0]{\boldsymbol{\tau}}
\newcommand{\Bomega}[0]{\boldsymbol{\omega}}

%
% shorthand for unit vectors
%
\newcommand{\acap}[0]{\hat{\Ba}}
\newcommand{\bcap}[0]{\hat{\Bb}}
\newcommand{\ccap}[0]{\hat{\Bc}}
\newcommand{\dcap}[0]{\hat{\Bd}}
\newcommand{\ecap}[0]{\hat{\Be}}
\newcommand{\fcap}[0]{\hat{\Bf}}
\newcommand{\gcap}[0]{\hat{\Bg}}
\newcommand{\hcap}[0]{\hat{\Bh}}
\newcommand{\icap}[0]{\hat{\Bi}}
\newcommand{\jcap}[0]{\hat{\Bj}}
\newcommand{\kcap}[0]{\hat{\Bk}}
\newcommand{\lcap}[0]{\hat{\Bl}}
\newcommand{\mcap}[0]{\hat{\Bm}}
\newcommand{\ncap}[0]{\hat{\Bn}}
\newcommand{\ocap}[0]{\hat{\Bo}}
\newcommand{\pcap}[0]{\hat{\Bp}}
\newcommand{\qcap}[0]{\hat{\Bq}}
\newcommand{\rcap}[0]{\hat{\Br}}
\newcommand{\scap}[0]{\hat{\Bs}}
\newcommand{\tcap}[0]{\hat{\Bt}}
\newcommand{\ucap}[0]{\hat{\Bu}}
\newcommand{\vcap}[0]{\hat{\Bv}}
\newcommand{\wcap}[0]{\hat{\Bw}}
\newcommand{\xcap}[0]{\hat{\Bx}}
\newcommand{\ycap}[0]{\hat{\By}}
\newcommand{\zcap}[0]{\hat{\Bz}}
\newcommand{\thetacap}[0]{\hat{\Btheta}}

%
% to write R^n and C^n in a distinguishable fashion.  Perhaps change this
% to the double lined characters upon figuring out how to do so.
%
\newcommand{\C}[1]{$\mathbb{C}^{#1}$}
\newcommand{\R}[1]{$\mathbb{R}^{#1}$}

%
% various generally useful helpers
%

% derivative of #1 wrt. #2:
\newcommand{\D}[2] {\frac {d#2} {d#1}}

\newcommand{\inv}[1]{\frac{1}{#1}}
\newcommand{\cross}[0]{\times}

\newcommand{\abs}[1]{\lvert{#1}\rvert}
\newcommand{\norm}[1]{\lVert{#1}\rVert}
\newcommand{\innerprod}[2]{\langle{#1}, {#2}\rangle}
\newcommand{\dotprod}[2]{{#1} \cdot {#2}}
\newcommand{\bdotprod}[2]{\left({#1} \cdot {#2}\right)}
\newcommand{\crossprod}[2]{{#1} \cross {#2}}
\newcommand{\tripleprod}[3]{\dotprod{\left(\crossprod{#1}{#2}\right)}{#3}}

\DeclareMathOperator{\Proj}{Proj}
\DeclareMathOperator{\Span}{span}
\DeclareMathOperator{\Sgn}{sgn}
\DeclareMathOperator{\Area}{Area}
\DeclareMathOperator{\Volume}{Volume}

%
% A few miscellaneous things specific to this document
%
\newcommand{\crossop}[1]{\crossprod{#1}{}}

% R2 vector.
\newcommand{\VectorTwo}[2]{
\begin{bmatrix}
 {#1} \\
 {#2}
\end{bmatrix}
}

\newcommand{\VectorN}[1]{
\begin{bmatrix}
{#1}_1 \\
{#1}_2 \\
\vdots \\
{#1}_N \\
\end{bmatrix}
}

\newcommand{\DETuvij}[4]{
\begin{vmatrix}
 {#1}_{#3} & {#1}_{#4} \\
 {#2}_{#3} & {#2}_{#4}
\end{vmatrix}
}

\newcommand{\DETuvwijk}[6]{
\begin{vmatrix}
 {#1}_{#4} & {#1}_{#5} & {#1}_{#6} \\
 {#2}_{#4} & {#2}_{#5} & {#2}_{#6} \\
 {#3}_{#4} & {#3}_{#5} & {#3}_{#6}
\end{vmatrix}
}

\newcommand{\DETuvwxijkl}[8]{
\begin{vmatrix}
 {#1}_{#5} & {#1}_{#6} & {#1}_{#7} & {#1}_{#8} \\
 {#2}_{#5} & {#2}_{#6} & {#2}_{#7} & {#2}_{#8} \\
 {#3}_{#5} & {#3}_{#6} & {#3}_{#7} & {#3}_{#8} \\
 {#4}_{#5} & {#4}_{#6} & {#4}_{#7} & {#4}_{#8} \\
\end{vmatrix}
}

%\newcommand{\DETuvwxyijklm}[10]{
%\begin{vmatrix}
% {#1}_{#6} & {#1}_{#7} & {#1}_{#8} & {#1}_{#9} & {#1}_{#10} \\
% {#2}_{#6} & {#2}_{#7} & {#2}_{#8} & {#2}_{#9} & {#2}_{#10} \\
% {#3}_{#6} & {#3}_{#7} & {#3}_{#8} & {#3}_{#9} & {#3}_{#10} \\
% {#4}_{#6} & {#4}_{#7} & {#4}_{#8} & {#4}_{#9} & {#4}_{#10} \\
% {#5}_{#6} & {#5}_{#7} & {#5}_{#8} & {#5}_{#9} & {#5}_{#10}
%\end{vmatrix}
%}

% R3 vector.
\newcommand{\VectorThree}[3]{
\begin{bmatrix}
 {#1} \\
 {#2} \\
 {#3}
\end{bmatrix}
}


%%<misc>
%
\newcommand{\Abs}[1]{{\left\lvert{#1}\right\rvert}}
\newcommand{\spacegrad}[0]{\boldsymbol{\nabla}}
\newcommand{\grad}[0]{\nabla}
\newcommand{\LL}[0]{\mathcal{L}}

% == \partial_{#1} {#2}
\newcommand{\PD}[2]{\frac{\partial {#2}}{\partial {#1}}}
% inline variant
\newcommand{\PDi}[2]{{\partial {#2}}/{\partial {#1}}}

\newcommand{\PDD}[3]{\frac{\partial^2 {#3}}{\partial {#1}\partial {#2}}}
%\newcommand{\PDd}[2]{\frac{\partial^2 {#2}}{{\partial{#1}}^2}}
\newcommand{\PDsq}[2]{\frac{\partial^2 {#2}}{(\partial {#1})^2}}

\newcommand{\Partial}[2]{\frac{\partial {#1}}{\partial {#2}}}
\DeclareMathOperator{\RejName}{Rej}
\newcommand{\Rej}[2]{\RejName_{#1}\left( {#2} \right)}
\newcommand{\Rm}[1]{\mathbb{R}^{#1}}
\newcommand{\Cm}[1]{\mathbb{C}^{#1}}
\newcommand{\conj}[0]{{*}}

%</misc>

% <grade selection>
%
\newcommand{\gpgrade}[2] {{\left\langle{{#1}}\right\rangle}_{#2}}

\newcommand{\gpgradezero}[1] {\gpgrade{#1}{}}
%\newcommand{\gpscalargrade}[1] {{\left\langle{{#1}}\right\rangle}}
%\newcommand{\gpgradezero}[1] {\gpgrade{#1}{0}}

%\newcommand{\gpgradeone}[1] {{\left\langle{{#1}}\right\rangle}_{1}}
\newcommand{\gpgradeone}[1] {\gpgrade{#1}{1}}

\newcommand{\gpgradetwo}[1] {\gpgrade{#1}{2}}
\newcommand{\gpgradethree}[1] {\gpgrade{#1}{3}}
\newcommand{\gpgradefour}[1] {\gpgrade{#1}{4}}
%
% </grade selection>



\newcommand{\adot}[0]{{\dot{a}}}
\newcommand{\bdot}[0]{{\dot{b}}}
% taken for centered dot:
%\newcommand{\cdot}[0]{{\dot{c}}}
%\newcommand{\ddot}[0]{{\dot{d}}}
\newcommand{\edot}[0]{{\dot{e}}}
\newcommand{\fdot}[0]{{\dot{f}}}
\newcommand{\gdot}[0]{{\dot{g}}}
\newcommand{\hdot}[0]{{\dot{h}}}
\newcommand{\idot}[0]{{\dot{i}}}
\newcommand{\jdot}[0]{{\dot{j}}}
\newcommand{\kdot}[0]{{\dot{k}}}
\newcommand{\ldot}[0]{{\dot{l}}}
\newcommand{\mdot}[0]{{\dot{m}}}
\newcommand{\ndot}[0]{{\dot{n}}}
%\newcommand{\odot}[0]{{\dot{o}}}
\newcommand{\pdot}[0]{{\dot{p}}}
\newcommand{\qdot}[0]{{\dot{q}}}
\newcommand{\rdot}[0]{{\dot{r}}}
\newcommand{\sdot}[0]{{\dot{s}}}
\newcommand{\tdot}[0]{{\dot{t}}}
\newcommand{\udot}[0]{{\dot{u}}}
\newcommand{\vdot}[0]{{\dot{v}}}
\newcommand{\wdot}[0]{{\dot{w}}}
\newcommand{\xdot}[0]{{\dot{x}}}
\newcommand{\ydot}[0]{{\dot{y}}}
\newcommand{\zdot}[0]{{\dot{z}}}
\newcommand{\addot}[0]{{\ddot{a}}}
\newcommand{\bddot}[0]{{\ddot{b}}}
\newcommand{\cddot}[0]{{\ddot{c}}}
%\newcommand{\dddot}[0]{{\ddot{d}}}
\newcommand{\eddot}[0]{{\ddot{e}}}
\newcommand{\fddot}[0]{{\ddot{f}}}
\newcommand{\gddot}[0]{{\ddot{g}}}
\newcommand{\hddot}[0]{{\ddot{h}}}
\newcommand{\iddot}[0]{{\ddot{i}}}
\newcommand{\jddot}[0]{{\ddot{j}}}
\newcommand{\kddot}[0]{{\ddot{k}}}
\newcommand{\lddot}[0]{{\ddot{l}}}
\newcommand{\mddot}[0]{{\ddot{m}}}
\newcommand{\nddot}[0]{{\ddot{n}}}
\newcommand{\oddot}[0]{{\ddot{o}}}
\newcommand{\pddot}[0]{{\ddot{p}}}
\newcommand{\qddot}[0]{{\ddot{q}}}
\newcommand{\rddot}[0]{{\ddot{r}}}
\newcommand{\sddot}[0]{{\ddot{s}}}
\newcommand{\tddot}[0]{{\ddot{t}}}
\newcommand{\uddot}[0]{{\ddot{u}}}
\newcommand{\vddot}[0]{{\ddot{v}}}
\newcommand{\wddot}[0]{{\ddot{w}}}
\newcommand{\xddot}[0]{{\ddot{x}}}
\newcommand{\yddot}[0]{{\ddot{y}}}
\newcommand{\zddot}[0]{{\ddot{z}}}

%<bold and dot greek symbols>
%

\newcommand{\Deltadot}[0]{{\dot{\Delta}}}
\newcommand{\Gammadot}[0]{{\dot{\Gamma}}}
\newcommand{\Lambdadot}[0]{{\dot{\Lambda}}}
\newcommand{\Omegadot}[0]{{\dot{\Omega}}}
\newcommand{\Phidot}[0]{{\dot{\Phi}}}
\newcommand{\Pidot}[0]{{\dot{\Pi}}}
\newcommand{\Psidot}[0]{{\dot{\Psi}}}
\newcommand{\Sigmadot}[0]{{\dot{\Sigma}}}
\newcommand{\Thetadot}[0]{{\dot{\Theta}}}
\newcommand{\Upsilondot}[0]{{\dot{\Upsilon}}}
\newcommand{\Xidot}[0]{{\dot{\Xi}}}
\newcommand{\alphadot}[0]{{\dot{\alpha}}}
\newcommand{\betadot}[0]{{\dot{\beta}}}
\newcommand{\chidot}[0]{{\dot{\chi}}}
\newcommand{\deltadot}[0]{{\dot{\delta}}}
\newcommand{\epsilondot}[0]{{\dot{\epsilon}}}
\newcommand{\etadot}[0]{{\dot{\eta}}}
\newcommand{\gammadot}[0]{{\dot{\gamma}}}
\newcommand{\kappadot}[0]{{\dot{\kappa}}}
\newcommand{\lambdadot}[0]{{\dot{\lambda}}}
\newcommand{\mudot}[0]{{\dot{\mu}}}
\newcommand{\nudot}[0]{{\dot{\nu}}}
\newcommand{\omegadot}[0]{{\dot{\omega}}}
\newcommand{\phidot}[0]{{\dot{\phi}}}
\newcommand{\pidot}[0]{{\dot{\pi}}}
\newcommand{\psidot}[0]{{\dot{\psi}}}
\newcommand{\rhodot}[0]{{\dot{\rho}}}
\newcommand{\sigmadot}[0]{{\dot{\sigma}}}
\newcommand{\taudot}[0]{{\dot{\tau}}}
\newcommand{\thetadot}[0]{{\dot{\theta}}}
\newcommand{\upsilondot}[0]{{\dot{\upsilon}}}
\newcommand{\varepsilondot}[0]{{\dot{\varepsilon}}}
\newcommand{\varphidot}[0]{{\dot{\varphi}}}
\newcommand{\varpidot}[0]{{\dot{\varpi}}}
\newcommand{\varrhodot}[0]{{\dot{\varrho}}}
\newcommand{\varsigmadot}[0]{{\dot{\varsigma}}}
\newcommand{\varthetadot}[0]{{\dot{\vartheta}}}
\newcommand{\xidot}[0]{{\dot{\xi}}}
\newcommand{\zetadot}[0]{{\dot{\zeta}}}

\newcommand{\Deltaddot}[0]{{\ddot{\Delta}}}
\newcommand{\Gammaddot}[0]{{\ddot{\Gamma}}}
\newcommand{\Lambdaddot}[0]{{\ddot{\Lambda}}}
\newcommand{\Omegaddot}[0]{{\ddot{\Omega}}}
\newcommand{\Phiddot}[0]{{\ddot{\Phi}}}
\newcommand{\Piddot}[0]{{\ddot{\Pi}}}
\newcommand{\Psiddot}[0]{{\ddot{\Psi}}}
\newcommand{\Sigmaddot}[0]{{\ddot{\Sigma}}}
\newcommand{\Thetaddot}[0]{{\ddot{\Theta}}}
\newcommand{\Upsilonddot}[0]{{\ddot{\Upsilon}}}
\newcommand{\Xiddot}[0]{{\ddot{\Xi}}}
\newcommand{\alphaddot}[0]{{\ddot{\alpha}}}
\newcommand{\betaddot}[0]{{\ddot{\beta}}}
\newcommand{\chiddot}[0]{{\ddot{\chi}}}
\newcommand{\deltaddot}[0]{{\ddot{\delta}}}
\newcommand{\epsilonddot}[0]{{\ddot{\epsilon}}}
\newcommand{\etaddot}[0]{{\ddot{\eta}}}
\newcommand{\gammaddot}[0]{{\ddot{\gamma}}}
\newcommand{\kappaddot}[0]{{\ddot{\kappa}}}
\newcommand{\lambdaddot}[0]{{\ddot{\lambda}}}
\newcommand{\muddot}[0]{{\ddot{\mu}}}
\newcommand{\nuddot}[0]{{\ddot{\nu}}}
\newcommand{\omegaddot}[0]{{\ddot{\omega}}}
\newcommand{\phiddot}[0]{{\ddot{\phi}}}
\newcommand{\piddot}[0]{{\ddot{\pi}}}
\newcommand{\psiddot}[0]{{\ddot{\psi}}}
\newcommand{\rhoddot}[0]{{\ddot{\rho}}}
\newcommand{\sigmaddot}[0]{{\ddot{\sigma}}}
\newcommand{\tauddot}[0]{{\ddot{\tau}}}
\newcommand{\thetaddot}[0]{{\ddot{\theta}}}
\newcommand{\upsilonddot}[0]{{\ddot{\upsilon}}}
\newcommand{\varepsilonddot}[0]{{\ddot{\varepsilon}}}
\newcommand{\varphiddot}[0]{{\ddot{\varphi}}}
\newcommand{\varpiddot}[0]{{\ddot{\varpi}}}
\newcommand{\varrhoddot}[0]{{\ddot{\varrho}}}
\newcommand{\varsigmaddot}[0]{{\ddot{\varsigma}}}
\newcommand{\varthetaddot}[0]{{\ddot{\vartheta}}}
\newcommand{\xiddot}[0]{{\ddot{\xi}}}
\newcommand{\zetaddot}[0]{{\ddot{\zeta}}}

\newcommand{\BDelta}[0]{\boldsymbol{\Delta}}
\newcommand{\BGamma}[0]{\boldsymbol{\Gamma}}
\newcommand{\BLambda}[0]{\boldsymbol{\Lambda}}
\newcommand{\BOmega}[0]{\boldsymbol{\Omega}}
\newcommand{\BPhi}[0]{\boldsymbol{\Phi}}
\newcommand{\BPi}[0]{\boldsymbol{\Pi}}
\newcommand{\BPsi}[0]{\boldsymbol{\Psi}}
\newcommand{\BSigma}[0]{\boldsymbol{\Sigma}}
\newcommand{\BTheta}[0]{\boldsymbol{\Theta}}
\newcommand{\BUpsilon}[0]{\boldsymbol{\Upsilon}}
\newcommand{\BXi}[0]{\boldsymbol{\Xi}}
\newcommand{\Balpha}[0]{\boldsymbol{\alpha}}
\newcommand{\Bbeta}[0]{\boldsymbol{\beta}}
\newcommand{\Bchi}[0]{\boldsymbol{\chi}}
\newcommand{\Bdelta}[0]{\boldsymbol{\delta}}
\newcommand{\Bepsilon}[0]{\boldsymbol{\epsilon}}
\newcommand{\Beta}[0]{\boldsymbol{\eta}}
\newcommand{\Bgamma}[0]{\boldsymbol{\gamma}}
\newcommand{\Bkappa}[0]{\boldsymbol{\kappa}}
\newcommand{\Blambda}[0]{\boldsymbol{\lambda}}
\newcommand{\Bmu}[0]{\boldsymbol{\mu}}
\newcommand{\Bnu}[0]{\boldsymbol{\nu}}
%\newcommand{\Bomega}[0]{\boldsymbol{\omega}}
\newcommand{\Bphi}[0]{\boldsymbol{\phi}}
\newcommand{\Bpi}[0]{\boldsymbol{\pi}}
\newcommand{\Bpsi}[0]{\boldsymbol{\psi}}
\newcommand{\Brho}[0]{\boldsymbol{\rho}}
\newcommand{\Bsigma}[0]{\boldsymbol{\sigma}}
%\newcommand{\Btau}[0]{\boldsymbol{\tau}}
%\newcommand{\Btheta}[0]{\boldsymbol{\theta}}
\newcommand{\Bupsilon}[0]{\boldsymbol{\upsilon}}
\newcommand{\Bvarepsilon}[0]{\boldsymbol{\varepsilon}}
\newcommand{\Bvarphi}[0]{\boldsymbol{\varphi}}
\newcommand{\Bvarpi}[0]{\boldsymbol{\varpi}}
\newcommand{\Bvarrho}[0]{\boldsymbol{\varrho}}
\newcommand{\Bvarsigma}[0]{\boldsymbol{\varsigma}}
\newcommand{\Bvartheta}[0]{\boldsymbol{\vartheta}}
\newcommand{\Bxi}[0]{\boldsymbol{\xi}}
\newcommand{\Bzeta}[0]{\boldsymbol{\zeta}}
%
%</bold and dot greek symbols>
%<infrequent>
%
%\newcommand{\AreaOp}[1]{\AName_{#1}}
%\newcommand{\Babs}[0]{\abs{\BB}}
%\newcommand{\Bcap}[0]{\hat{\BB}}
%\newcommand{\BrPrimeRej}[0]{\rcap(\rcap \wedge \Br')}
%\newcommand{\CA}[0]{\mathcal{A}}
%\newcommand{\Cos}[1]{\cos{\left({#1}\right)}}
%\newcommand{\Det}[1] {\abs{#1}}
%\newcommand{\Dsq}[2] {\frac {\partial^2 {#1}} {\partial {#2}^2}}
%\newcommand{\Exp}[1]{\exp{\left({#1}\right)}}
%\newcommand{\Norm}[1]{\left\lVert{#1}\right\rVert}
%\newcommand{\Sin}[1]{\sin{\left({#1}\right)}}
%\newcommand{\T}[0]{\text{T}}
%\newcommand{\VolumeOp}[1]{\VName_{#1}}
%\newcommand{\agrad}[0]{\Ba \cdot \nabla}
%\newcommand{\alphacap}[0]{\hat{\boldsymbol{\alpha}}}
%\newcommand{\Fcap}[0]{\hat{\BF}}
%\newcommand{\bithree}[0]{{\Bi}_3}
%\newcommand{\bxa}[0]{\Bx\Ba}
%\newcommand{\coordvec}[2]{
%\newcommand{\costheta}[0]{\acap \cdot \xcap}
%\newcommand{\ddt}[1]{\ddot{#1}}
%\newcommand{\ddu}[1] {\frac {d{#1}} {du}}
%\newcommand{\dsqxj}[2] {\frac {\partial^2 {#1}} {\partial {x_{#2}}^2}}
%\newcommand{\dtheta}[1]{\frac{d {#1}}{d \theta}}
%\newcommand{\dt}[1]{\dot{#1}}
%\newcommand{\dt}[1]{\frac{d {#1}}{dt}}
%\newcommand{\dxj}[2] {\frac {\partial {#1}} {\partial {x_{#2}}}}
%\newcommand{\halfPhi}[0]{\frac{\phi}{2}}
%\newcommand{\half}[0]{\inv{2}}
%\newcommand{\inv}[1]{\frac{1}{#1}}
%\newcommand{\laplacian}[0]{\nabla^2}
%\newcommand{\matrixoftx}[3]{
%\newcommand{\nrrp}[0]{\norm{\rcap \wedge \Br'}}
%\newcommand{\oiint}{\bigcirc \hspace{-1.4em} \int \hspace{-.8em} \int}
%\newcommand{\transpose}[1]{{#1}^{\text{T}}}
%\newcommand{\transpose}[1]{{{#1}^{\TextTranspose}}}
%\newcommand{\transpose}[1]{{{#1}^{\text{T}}}}
%\newcommand{\barA}[0]{\bar{A}}
%\newcommand{\qbar}[0]{\bar{q}}
%\newcommand{\qdotbar}[0]{\dot{\bar{q}}}
%
%</infrequent>





%\usepackage[bookmarks=true]{hyperref}

\chapter{review alan ch4}
\label{chap:reviewAlanGabookCh4}
%\author{Peeter Joot \quad peeterjoot@protonmail.com}
\date{ Mmm dd, 2009.  \(RCSfile: reviewAlanGabookCh4.tex,v \) Last \(Revision: 1.7 \) \(Date: 2009/06/14 23:51:45 \) }

%\begin{document}

%\maketitle{}
%\tableofcontents

\section{Chapter 4. Inner product spaces}

\subsection{page 48}

You have:

"We often write \(\inv{\Abs{\Bu}}\Bu = \frac{\Bu}{\Abs{\Bu}}\)"

did you mean to say

"We often write \(\ucap = \frac{\Bu}{\Abs{\Bu}}\)"

?

\subsection{page 49}

In high school I recall wondering why we used \(\Be_i\) for unit vectors, instead of \(\Bu_i\) say.
I 
believe its for one-length vector (ein in German).  Perhaps silly
, but I had love to have seen this in my intro linear algebra book (take the mystery out of the symbols.)

\section{Chapter 5.  Geometric algebra}

\subsection{page 65}

last paragraph:

  "We can only compare orientations"

would make more sense to say

  "We can only compare orientations and total area"

(thinking of transported vectors as an analogy.  you can compare parallel vectors with different magnitudes).

\subsection{page 67}

typo, last line.  trivector should be bivector.

\subsection{page 68}

Figure 5.4.  I do not think the orientation displayed does not make sense as depicted.  Better would be to butt up the tail of \(\Bv\) with the head of \(\Bu\).

Theorem 5.2.  With the content of the text up to this point this seems like a
definition and not a theorem to me, and is somewhat cyclic.  ie: A quantity

\begin{equation}\label{eqn:reviewAlanGabookCh4:20}
\begin{aligned}
\Bu \wedge \Bv \equiv \Abs{\Bu} \Abs{\Bv} \sin \theta (\Be_1 \wedge \Be_2)
\end{aligned}
\end{equation}

can be said to represent the oriented area of the parallelogram spanned by the
vectors \(\Bu\) and \(\Bv\).  This however gets you into trouble conceptually
since you have not really defined what \(\Be_1 \wedge \Be_2\) means.

Conceptually speaking I think that page 68 and 69 need to be reworked, but
doing it well is tricky since you have conflicting goals of introducing some
intuition as well as being rigorous.  As a minimum I think that Theorem 5.3
O2, O3, O4 should be the fundamental definition that you work from (with O1,
and 5.2 as consequences).  From 5.2 I think that you can then motivate a
bivector norm definition based on the magnitude of the parallelogram.

\subsection{page 71}

Figure 5.8.  Like the bivector, if you put your arrows head to tail (ie: \(\Bv\), and \(\Bu\) transported parallel to the top face) the orientation arrows that
you have made would be clearer.

Also, similar to the preceding bivector page, your outer product definition is 
fuzzy, based on intuition and comparison to the described trivector.  Better
IMO would be an axiomatic definition that builds on a two vector outer product definition:

\begin{equation}\label{eqn:reviewAlanGabookCh4:40}
\begin{aligned}
\Bu \wedge \Bv \wedge \Bw &= (\Bu \wedge \Bv) \wedge \Bw = \Bu \wedge (\Bv \wedge \Bw) \\
\end{aligned}
\end{equation}

\subsection{page 74}

Your "click" does not work for me. What is it supposed to do?

\subsection{page 75}

You could connect this complex number bit this with your "Wait!" mixed bag discussion back on page 72, since this
shows that the the apples plus oranges addition is not actually that unfamiliar.

\subsection{page 81}

description of the Feynman figure looks wrong.  To me it looks like 360, 480, 720.  You have 180, 360, 720.

I tried this and I can not get my arm twisted that last 180.

\subsection{page 82}

Ex. 5.20.  Text description I think would be easier to understand if you said for the second rotation that you are to
rotate about the new x-axis (instead of x-axis).

Now, for this problem when I work it, I get a different answer, so we should
compare notes.  \href{http://sites.google.com/site/peeterjoot/math2009/two_ninety_rotations.pdf}{I have put mine on my website.}.  One of us has a mistake, or I am misinterpreting your problem.

%\subsection{}
%\subsection{}
%\subsection{}

%\bibliographystyle{plainnat}
%\bibliography{myrefs}

%\end{document}

\documentclass{article}      % Specifies the document class

\usepackage{amsmath}
\usepackage{mathpazo}

%
% shorthand for bold symbols, convenient for vectors and matrices
%
\newcommand{\Ba}[0]{\mathbf{a}}
\newcommand{\Bb}[0]{\mathbf{b}}
\newcommand{\Bc}[0]{\mathbf{c}}
\newcommand{\Bd}[0]{\mathbf{d}}
\newcommand{\Be}[0]{\mathbf{e}}
\newcommand{\Bf}[0]{\mathbf{f}}
\newcommand{\Bg}[0]{\mathbf{g}}
\newcommand{\Bh}[0]{\mathbf{h}}
\newcommand{\Bi}[0]{\mathbf{i}}
\newcommand{\Bj}[0]{\mathbf{j}}
\newcommand{\Bk}[0]{\mathbf{k}}
\newcommand{\Bl}[0]{\mathbf{l}}
\newcommand{\Bm}[0]{\mathbf{m}}
\newcommand{\Bn}[0]{\mathbf{n}}
\newcommand{\Bo}[0]{\mathbf{o}}
\newcommand{\Bp}[0]{\mathbf{p}}
\newcommand{\Bq}[0]{\mathbf{q}}
\newcommand{\Br}[0]{\mathbf{r}}
\newcommand{\Bs}[0]{\mathbf{s}}
\newcommand{\Bt}[0]{\mathbf{t}}
\newcommand{\Bu}[0]{\mathbf{u}}
\newcommand{\Bv}[0]{\mathbf{v}}
\newcommand{\Bw}[0]{\mathbf{w}}
\newcommand{\Bx}[0]{\mathbf{x}}
\newcommand{\By}[0]{\mathbf{y}}
\newcommand{\Bz}[0]{\mathbf{z}}
\newcommand{\BA}[0]{\mathbf{A}}
\newcommand{\BB}[0]{\mathbf{B}}
\newcommand{\BC}[0]{\mathbf{C}}
\newcommand{\BD}[0]{\mathbf{D}}
\newcommand{\BE}[0]{\mathbf{E}}
\newcommand{\BF}[0]{\mathbf{F}}
\newcommand{\BG}[0]{\mathbf{G}}
\newcommand{\BH}[0]{\mathbf{H}}
\newcommand{\BI}[0]{\mathbf{I}}
\newcommand{\BJ}[0]{\mathbf{J}}
\newcommand{\BK}[0]{\mathbf{K}}
\newcommand{\BL}[0]{\mathbf{L}}
\newcommand{\BM}[0]{\mathbf{M}}
\newcommand{\BN}[0]{\mathbf{N}}
\newcommand{\BO}[0]{\mathbf{O}}
\newcommand{\BP}[0]{\mathbf{P}}
\newcommand{\BQ}[0]{\mathbf{Q}}
\newcommand{\BR}[0]{\mathbf{R}}
\newcommand{\BS}[0]{\mathbf{S}}
\newcommand{\BT}[0]{\mathbf{T}}
\newcommand{\BU}[0]{\mathbf{U}}
\newcommand{\BV}[0]{\mathbf{V}}
\newcommand{\BW}[0]{\mathbf{W}}
\newcommand{\BX}[0]{\mathbf{X}}
\newcommand{\BY}[0]{\mathbf{Y}}
\newcommand{\BZ}[0]{\mathbf{Z}}

\newcommand{\Bzero}[0]{\mathbf{0}}
\newcommand{\Btheta}[0]{\boldsymbol{\theta}}
\newcommand{\Btau}[0]{\boldsymbol{\tau}}
\newcommand{\Bomega}[0]{\boldsymbol{\omega}}

%
% shorthand for unit vectors
%
\newcommand{\acap}[0]{\hat{\Ba}}
\newcommand{\bcap}[0]{\hat{\Bb}}
\newcommand{\ccap}[0]{\hat{\Bc}}
\newcommand{\dcap}[0]{\hat{\Bd}}
\newcommand{\ecap}[0]{\hat{\Be}}
\newcommand{\fcap}[0]{\hat{\Bf}}
\newcommand{\gcap}[0]{\hat{\Bg}}
\newcommand{\hcap}[0]{\hat{\Bh}}
\newcommand{\icap}[0]{\hat{\Bi}}
\newcommand{\jcap}[0]{\hat{\Bj}}
\newcommand{\kcap}[0]{\hat{\Bk}}
\newcommand{\lcap}[0]{\hat{\Bl}}
\newcommand{\mcap}[0]{\hat{\Bm}}
\newcommand{\ncap}[0]{\hat{\Bn}}
\newcommand{\ocap}[0]{\hat{\Bo}}
\newcommand{\pcap}[0]{\hat{\Bp}}
\newcommand{\qcap}[0]{\hat{\Bq}}
\newcommand{\rcap}[0]{\hat{\Br}}
\newcommand{\scap}[0]{\hat{\Bs}}
\newcommand{\tcap}[0]{\hat{\Bt}}
\newcommand{\ucap}[0]{\hat{\Bu}}
\newcommand{\vcap}[0]{\hat{\Bv}}
\newcommand{\wcap}[0]{\hat{\Bw}}
\newcommand{\xcap}[0]{\hat{\Bx}}
\newcommand{\ycap}[0]{\hat{\By}}
\newcommand{\zcap}[0]{\hat{\Bz}}
\newcommand{\thetacap}[0]{\hat{\Btheta}}

%
% to write R^n and C^n in a distinguishable fashion.  Perhaps change this
% to the double lined characters upon figuring out how to do so.
%
\newcommand{\C}[1]{$\mathbb{C}^{#1}$}
\newcommand{\R}[1]{$\mathbb{R}^{#1}$}

%
% various generally useful helpers
%

% derivative of #1 wrt. #2:
\newcommand{\D}[2] {\frac {d#2} {d#1}}

\newcommand{\inv}[1]{\frac{1}{#1}}
\newcommand{\cross}[0]{\times}

\newcommand{\abs}[1]{\lvert{#1}\rvert}
\newcommand{\norm}[1]{\lVert{#1}\rVert}
\newcommand{\innerprod}[2]{\langle{#1}, {#2}\rangle}
\newcommand{\dotprod}[2]{{#1} \cdot {#2}}
\newcommand{\bdotprod}[2]{\left({#1} \cdot {#2}\right)}
\newcommand{\crossprod}[2]{{#1} \cross {#2}}
\newcommand{\tripleprod}[3]{\dotprod{\left(\crossprod{#1}{#2}\right)}{#3}}

\DeclareMathOperator{\Proj}{Proj}
\DeclareMathOperator{\Span}{span}
\DeclareMathOperator{\Sgn}{sgn}
\DeclareMathOperator{\Area}{Area}
\DeclareMathOperator{\Volume}{Volume}

%
% A few miscellaneous things specific to this document
%
\newcommand{\crossop}[1]{\crossprod{#1}{}}

% R2 vector.
\newcommand{\VectorTwo}[2]{
\begin{bmatrix}
 {#1} \\
 {#2}
\end{bmatrix}
}

\newcommand{\VectorN}[1]{
\begin{bmatrix}
{#1}_1 \\
{#1}_2 \\
\vdots \\
{#1}_N \\
\end{bmatrix}
}

\newcommand{\DETuvij}[4]{
\begin{vmatrix}
 {#1}_{#3} & {#1}_{#4} \\
 {#2}_{#3} & {#2}_{#4}
\end{vmatrix}
}

\newcommand{\DETuvwijk}[6]{
\begin{vmatrix}
 {#1}_{#4} & {#1}_{#5} & {#1}_{#6} \\
 {#2}_{#4} & {#2}_{#5} & {#2}_{#6} \\
 {#3}_{#4} & {#3}_{#5} & {#3}_{#6}
\end{vmatrix}
}

\newcommand{\DETuvwxijkl}[8]{
\begin{vmatrix}
 {#1}_{#5} & {#1}_{#6} & {#1}_{#7} & {#1}_{#8} \\
 {#2}_{#5} & {#2}_{#6} & {#2}_{#7} & {#2}_{#8} \\
 {#3}_{#5} & {#3}_{#6} & {#3}_{#7} & {#3}_{#8} \\
 {#4}_{#5} & {#4}_{#6} & {#4}_{#7} & {#4}_{#8} \\
\end{vmatrix}
}

%\newcommand{\DETuvwxyijklm}[10]{
%\begin{vmatrix}
% {#1}_{#6} & {#1}_{#7} & {#1}_{#8} & {#1}_{#9} & {#1}_{#10} \\
% {#2}_{#6} & {#2}_{#7} & {#2}_{#8} & {#2}_{#9} & {#2}_{#10} \\
% {#3}_{#6} & {#3}_{#7} & {#3}_{#8} & {#3}_{#9} & {#3}_{#10} \\
% {#4}_{#6} & {#4}_{#7} & {#4}_{#8} & {#4}_{#9} & {#4}_{#10} \\
% {#5}_{#6} & {#5}_{#7} & {#5}_{#8} & {#5}_{#9} & {#5}_{#10}
%\end{vmatrix}
%}

% R3 vector.
\newcommand{\VectorThree}[3]{
\begin{bmatrix}
 {#1} \\
 {#2} \\
 {#3}
\end{bmatrix}
}



\usepackage[bookmarks=true]{hyperref}

                             % The preamble begins here.
%\title{} % Declares the document's title.
%\author{Peeter Joot \quad peeter.joot@gmail.com}         % Declares the author's name.
%\date{ Last Revision: $Date: 2009/06/04 00:44:05 $ } % Deleting this command produces today's date.

\begin{document}             % End of preamble and beginning of text.

%\maketitle{}
%\tableofcontents
%
%\section{}

Magnetostatics field:

\begin{equation*}
F = E + icB = icB
\end{equation*}

Or,
\begin{equation*}
iF = -cB
\end{equation*}

(here I am using Doran/Lasenby notation writing $E$ and $B$ in bivector form: $E = E^j \sigma_j = E^j \gamma_j \wedge \gamma_0$).

\begin{align*}
\grad F 
&= \grad icB \\
&= -i c \grad B \\
&= -i c \left( \grad \cdot B + \grad \wedge B \right) \\
&= J/\epsilon_0 c
\end{align*}

The multiplication by $i$ turns the grade one (dot) and three (wedge) parts into grade three and one respectively.  Equating grades on the left and right of the equation one has:

\begin{equation*}
0 = -i c \grad \cdot B = i \grad \cdot ( iF )
\end{equation*}

and
\begin{equation*}
J/\epsilon_0 c = -i c \grad \wedge B = i \grad \wedge (iF)
\end{equation*}

Assuming equation 16, integration gives

\begin{align*}
\int_S i \grad \wedge (iF) 
&= i \int_{\partial S} iF \\
&= -\int_{\partial S} F \\
&= \int_S J/\epsilon_0 c.
\end{align*}

Or
\begin{equation*}
\int_{\partial S} F = -I_S/\epsilon_0 c.
\end{equation*}

\end{document}               % End of document.


% END INCLUDES.
%-------------------------------------------------------

% from the template:

%\begin{thebibliography}{99}
%  \addcontentsline{toc}{chapter}{Bibliography}
%\bibitem{lamport} L. Lamport. {\bf \LaTeX \ A Document Preparation System}
%Addison-Wesley, California 1986.
%
%\end{thebibliography}

\bibliographystyle{plainnat}
  \addcontentsline{toc}{chapter}{Bibliography}
\bibliography{myrefs}

%Note the tag used to make an index entry. You may need to consult Lamport's
%book~\cite{lamport} for details of the procedure to make the index input
%file; \LaTeX \ will create a pre-index by listing all the tagged
%items in the file {\tt bookex.idx} then you edit this into
%a {\tt theindex} environment, as {\tt index.tex}.

%\documentclass[openany]{memoir}
\usepackage[]{makeidx}

\chapterstyle{ell}

\makeindex

\begin{document}

To solve various problems in physics, it can be advantageous
to express any arbitrary piecewise-smooth function as a Fourier Series
\index{Fourier Series}
composed of multiples of sine \index{sine} and cosine \index{cosine} functions.  These are used in \cite{acheson1990elementary}.

\printindex

\bibliography{myrefs}
\bibliographystyle{unsrturl}

\end{document}

%  \addcontentsline{toc}{chapter}{Index}

\end{document}
