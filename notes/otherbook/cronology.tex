\chapter{Cronological Index}
\label{chap:Cronology}

\begin{itemize}

\item Jan 1, 0000 \ref{chap:rindler} rindler

\item Jan 1, 0000 \ref{chap:reviewAlanGabook} Chapter I-III comments on Alan's book

\item Jan 1, 0000 \ref{chap:reviewAlanGabookCh4} review alan ch4

\item Jan 1, 0000 \ref{chap:longWireQ} long wire q

\item Jan 1, 0000 \ref{chap:constantQ} constant q

\item Jan 1, 0000 \ref{chap:box} box

\item Jan 1, 0000 \ref{chap:m} matteo m

\item January 1, 2008 \ref{chap:wedgeNormVsGaNorm} Wedge product norm and GA bivector norm comparison

\item April 7, 2008 \ref{chap:shear} Notes on shear transformation

\item May 9, 2008 \ref{chap:gacsQ88} Questioning equation 8.8 from GASC

\item May 30, 2008 \ref{chap:emTensor} Geometric Algebra: Signs of electromagnetic field tensor components

\item October 1, 2008 \ref{chap:srLagrangianQ} sr lagrangian

\item October 2, 2008 \ref{chap:schwartzchildMetric} Some GR Notes

Euler-Lagrange calculation for the Schwartzchild metric.\item October 14, 2008 \ref{chap:pfSch} pf sch

\item October 17, 2008 \ref{chap:lorentzTxEmPotential} Lorentz force interaction

\item November 25, 2008 \ref{chap:matrixToOperator} 2D matrix vs. GA wedge-dot

\item January 17, 2009 \ref{chap:twoNinetyRotations} Composition of rotations exercise

Work problem from a GA book draft to discuss difference in solution. \item February 1, 2009 \ref{chap:stubEmFields} stub em fields

\item February 1, 2009 \ref{chap:scratchPlanewave} scratch planewave

\item February 1, 2009 \ref{chap:faradayLagrangian} REMOVED FROM electric field energy

\item March 1, 2009 \ref{chap:lutQ0304} lut q0304

\item March 1, 2009 \ref{chap:euler} euler

\item March 1, 2009 \ref{chap:tangentspace} tangentspace

\item May 3, 2009 \ref{chap:voltageCurrentResistance} Voltage current, and resistance

Dad has been complaining for years that these are all cyclicly defined in terms of each other.  The point in the end is perhaps not much more than, yes it is possible to define things in a non-cyclic fashion.... but I'm not so sure that my non-cyclic description itself ended up that good. \item May 9, 2009 \ref{chap:qmbFirstTry} VERSION: First try at one dimensional rectangular Quantum barrier problem

\item August 1, 2009 \ref{chap:waveguideRetardedAdvanced} (A POSSIBLY WRONG) Superposition of transverse electromagnetic field solutions.

\item Sept 5, 2009 \ref{chap:maxwellLagRotate} Maxwell Lagrangian, rotation of coordinates.

\end{itemize}
