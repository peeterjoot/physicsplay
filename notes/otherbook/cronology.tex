\chapter{Cronological Index}
\label{chap:Cronology}

\begin{itemize}

\item Jan 1, 0000 \ref{chap:rindler} rindler

\item Jan 1, 0000 \ref{chap:reviewAlanGabook} Chapter I-III comments on Alan's book

\item Jan 1, 0000 \ref{chap:reviewAlanGabookCh4} review alan ch4

\item Jan 1, 0000 \ref{chap:longWireQ} long wire q

\item Jan 1, 0000 \ref{chap:constantQ} constant q

\item Jan 1, 0000 \ref{chap:box} box

\item Jan 1, 0000 \ref{chap:m} matteo m

\item January 1, 2008 \ref{chap:wedgeNormVsGaNorm} Wedge product norm and GA bivector norm comparison

\item April 7, 2008 \ref{chap:shear} Notes on shear transformation

\item May 9, 2008 \ref{chap:gacsQ88} Questioning equation 8.8 from GASC

\item May 30, 2008 \ref{chap:emTensor} Geometric Algebra: Signs of electromagnetic field tensor components

\item October 1, 2008 \ref{chap:srLagrangianQ} sr lagrangian

\item October 2, 2008 \ref{chap:schwartzchildMetric} Some GR Notes

Euler-Lagrange calculation for the Schwartzchild metric.\item October 14, 2008 \ref{chap:pfSch} pf sch

\item October 17, 2008 \ref{chap:lorentzTxEmPotential} Lorentz force interaction

\item November 25, 2008 \ref{chap:matrixToOperator} 2D matrix vs. GA wedge-dot

\item January 17, 2009 \ref{chap:twoNinetyRotations} Composition of rotations exercise

Work problem from a GA book draft to discuss difference in solution. \item February 1, 2009 \ref{chap:stubEmFields} stub em fields

\item February 1, 2009 \ref{chap:scratchPlanewave} scratch planewave

\item February 1, 2009 \ref{chap:faradayLagrangian} REMOVED FROM electric field energy

\item March 1, 2009 \ref{chap:lutQ0304} lut q0304

\item March 1, 2009 \ref{chap:euler} euler

\item March 1, 2009 \ref{chap:tangentspace} tangentspace

\item May 3, 2009 \ref{chap:voltageCurrentResistance} Voltage current, and resistance

Dad has been complaining for years that these are all cyclicly defined in terms of each other.  The point in the end is perhaps not much more than, yes it is possible to define things in a non-cyclic fashion.... but I'm not so sure that my non-cyclic description itself ended up that good. \item May 9, 2009 \ref{chap:qmbFirstTry} VERSION: First try at one dimensional rectangular Quantum barrier problem

\item July 1, 2009 \ref{chap:maxwellVacuum_roughNotes} blah

\item July 2, 2009 \ref{chap:lightQuantize} Light and Electron Quantization?

\item July 14, 2009 \ref{chap:epsilonMixed} epsilonMixed

\item July 16, 2009 \ref{chap:interactionEnergy} Electrodynamic interaction energy and momentum

\item August 1, 2009 \ref{chap:waveguideRetardedAdvanced} (A POSSIBLY WRONG) Superposition of transverse electromagnetic field solutions.

\item Aug 9, 2009 \ref{chap:maxwellSepVars} Separation of variables applied to homogeneous Maxwell equation

\item Aug 9, 2009 \ref{chap:maxwellHomoFirstOrder} Transverse light wave propagation with z dependence.

\item Aug 11, 2009 \ref{chap:dotBlade_rough} Dot product of vector and blade

\item Aug 11, 2009 \ref{chap:dotBlade} Dot product of vector and bivector

\item Aug 20, 2009 \ref{chap:gaBasics} Introduction to Geometric Algebra.

\item Aug 22, 2009 \ref{chap:gradientAltCoord} Gradient and Divergence in different coordinate systems.

\item Sept 5, 2009 \ref{chap:maxwellLagRotate} Maxwell Lagrangian, rotation of coordinates.

\item Sept 6, 2009 \ref{chap:bivectorSelectWrong} bivectorSelectWrong

\item Sept 10, 2009 \ref{chap:decodingMercedFlorez} Decoding the Merced Florez article.

\item Nov 11, 2009 \ref{chap:ellipticQ} question on elliptic function paper.

\item Jan 1, 2010 \ref{chap:1dpotentialIntegral} Integrating the equation of motion for a one dimensional problem.

Solve for time for an arbitary one dimensional potential.\item Feb 19, 2010 \ref{chap:1dharmonicOsc} 1D forced harmonic oscillator.  Quick solution of non-homogeneous problem.

Solve the one dimensional harmonic oscillator problem using matrix methods.\item Mar 3, 2010 \ref{chap:goldsteinRouth} Notes on Goldstein's Routh's procedure.

Puzzle through Routh's procedure as outlined in Goldstein.\item May 23, 2010 \ref{chap:liboff314} Time evolution of some wave functions

Liboff, problem 3.14.\item May 23, 2010 \ref{chap:liboff319} Effect of sinusoid operators

Liboff, problem 3.19.\item May 28, 2010 \ref{chap:feynmanQEDerrata} Errata for Feynman's Quantum Electrodynamics (Addison-Wesley)?

My collection of errata notes for some Feynman lecture notes on QED compiled by a student.\item May 29, 2010 \ref{chap:pauliFourier} Fourier transformation of the Pauli QED wave equation (Take I).

Unsuccessful attempt to find a solution to the Pauli QM Hamiltonian using Fourier transforms.  Also try to figure out the notation from the Feynman book where I saw this.\item May 30, 2010 \ref{chap:exponentialCommutation} On commutation of exponentials

Show that commutation of exponentials occurs if exponentiated terms also commute.\item May 31, 2010 \ref{chap:liboff41} Infinite square well wavefunction.

\item June 19, 2010 \ref{chap:hoopSpring} Hoop and spring oscillator problem.

A linear appromation to a hoop and spring problem.\item June 25, 2010 \ref{chap:liboff43} More problems from Liboff chapter 4

Liboff problems 4.11, 4.12, 4.14\item July 23, 2010 \ref{chap:desaiDiracNotes} Dirac Notation Ponderings.

Chapter 1 solutions and some associated notes.\item July 27, 2010 \ref{chap:rotationUnitary} Rotations using matrix exponentials

Calculating the exponential form for a unitary operator.  A unitary operator can be expressed as the exponential of a Hermitian operator.  Show how this can be calculated for the matrix representation of an operator.  Explicitly calculate this matrix for a plane rotionation yields one of the Pauli spin matrices.  While not unitary, the same procedure can be used to calculate such a rotation like angle for a Lorentz boost, and we also find that the result can be expressed in terms of one of the Pauli spin matrices.\end{itemize}
