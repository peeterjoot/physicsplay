\documentclass{article}      % Specifies the document class

\usepackage{amsmath}
\newcommand{\abs}[1]{\lvert#1\rvert}
\newcommand{\norm}[1]{\lVert#1\rVert}
\newcommand{\grad}[1]{\nabla#1}
\newcommand{\curl}[1]{\nabla \times #1}
\newcommand{\Curl}[1]{\nabla \times \mathbf{#1}}
\newcommand{\diverg}[1]{\nabla \cdot #1}
\newcommand{\Diverg}[1]{\nabla \cdot \mathbf{#1}}
\newcommand{\curlcurl}[1]{\curl\curl{#1}}
\newcommand{\curlCurl}[1]{\curl\Curl{#1}}
\newcommand{\delsquared}[1]{\nabla^2{#1}}
\newcommand{\Delsquared}[1]{{\nabla^2}{\mathbf{#1}}}
\newcommand{\ddt}[1]{ {{\partial{#1}} \over {\partial{t}}}}
\newcommand{\Ddt}[1]{ {{\partial{\mathbf{#1}}} \over {\partial{t}}}}
\newcommand{\ddts}[1]{ {{\partial^2{#1}} \over {\partial{t}^2}}}
\newcommand{\Ddts}[1]{ {{\partial^2{\mathbf{#1}}} \over {\partial{t}^2}}}
\newcommand{\Bj}[0]{\mathbf{j}}
\newcommand{\BB}[0]{\mathbf{B}}
\newcommand{\BE}[0]{\mathbf{E}}

                             % The preamble begins here.
\title{An Example Document}  % Declares the document's title.
\author{Peeter Joot}         % Declares the author's name.
\date{March 25, 2000}        % Deleting this command produces today's date.

\begin{document}             % End of preamble and beginning of text.

%\maketitle                  % Produces the title.

\section{various formulations of Maxwell's equations}

It is interesting to look at the various formulations of Maxwell's 
equations.  The standard formulation these days is the differential
form, which isn't the most intuitive form.  In cgs units, the differential
form is as follows:

\begin{equation*}
\Diverg E = 4\pi\rho
\end{equation*}
\begin{equation*}
\Diverg B = 0
\end{equation*}
\begin{equation*}
\Curl{B} = 4\pi \Bj - {1 \over c} \Ddt{E}
\end{equation*}
\begin{equation*}
\Curl{E} = {1 \over c} \Ddt{B}
\end{equation*}

It is interesting to note that these are probably not the form that Maxwell 
originally formulated his equations in.  In my 1960 version of encyclodedia 
britianica Maxwell's equations were given in a differential form.  I had 
trouble seeing how these two forms were related at first, but the relation
is via the standard integral transformations.  For example, the formula $\Diverg E = 4\pi\rho$ is an alternate formulation of Gauss' law
\begin{equation*}
\BE = \int_V{{\rho\,dV' \over \norm{\mathbf r - \mathbf r'}}}
\end{equation*}

This can be shown via application of Gauss's theorm.  Similarily the following 
integral forms of Maxwell's equations:

%[ see paper notes ].

One of the awkward things with this standard formulation is that there 
is an awkward interdependence between the electric and magnetic fields, and 
the solution of either $\BE$ or $\BB$ interdependently seems difficult.

The interdependence of the $\BE$ and $\BB$ field equations can be removed by taking the
curl of each of the curl equations, and using the fact that:

$\curlCurl{V} = \grad (\Diverg V) - \Delsquared{V}$

\begin{align*}
\curlCurl{B} &= \grad (\Diverg{B}) - \Delsquared{B} \\
\curl(4\pi\Bj - {1 \over c} \Ddt{E}) &= \\
4\pi \Curl{j} - {1 \over c^2} \Ddts{B} &= \\
      	     &= 		   - \Delsquared{B}
\end{align*}

We end up with a single differential equation for the magnetic field $\BB$.

$\delsquared{\BB} - {1 \over c^2} \Ddts{B} = - 4\pi \Curl{j}$

We can do the same calculation for the electric field.

\begin{align*}
\curlCurl{E} &= \grad (\Diverg{E}) - \Delsquared{E} \\
\curl({1 \over c} \Ddt{B}) &= \\
{1 \over c} \ddt(4\pi \Bj - {1 \over c} \Ddt{E}) &= \\
{4\pi \over c} \Ddt{j} - {1 \over c^2} \Ddts{E} &= \\
             &= \grad (4\pi\rho) - \Delsquared{E} \\
\end{align*}

which gives a single differential equation for the electric field $\BE$.

%$\Delsquared{E} - {1 \over c^2} \Ddt{E} = {4\pi \over c} \Ddt{j} - 4\pi \grad{\rho}$
$\Delsquared{E} - {1 \over c^2} \Ddt{E}$
$ = {4\pi \over c} \Ddt{j} - 4\pi \grad{\rho}$

In the absense of current density and charge these equations take the simple form of standard wave
equations.

\begin{equation*}
\Delsquared{E} - {1 \over c^2} \Ddt{E} = 0
\end{equation*}
\begin{equation*}
\Delsquared{B} - {1 \over c^2} \Ddt{B} = 0
\end{equation*}

Any solution to these is also a solution to the original form where there is current density $\Bj$, and 
charge density $\rho$ since 

\end{document}               % End of document.
