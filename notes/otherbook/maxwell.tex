\documentclass{article}      % Specifies the document class

\usepackage{amsmath}
\newcommand{\abs}[1]{\lvert#1\rvert}
\newcommand{\norm}[1]{\lVert#1\rVert}
\newcommand{\grad}[1]{\nabla#1}
\newcommand{\curl}[1]{\nabla \times #1}
\newcommand{\Curl}[1]{\nabla \times \mathbf{#1}}
\newcommand{\diverg}[1]{\nabla \cdot #1}
\newcommand{\Diverg}[1]{\nabla \cdot \mathbf{#1}}
\newcommand{\curlcurl}[1]{\curl\curl{#1}}
\newcommand{\curlCurl}[1]{\curl\Curl{#1}}
\newcommand{\delsquared}[1]{\nabla^2{#1}}
\newcommand{\Delsquared}[1]{{\nabla^2}{\mathbf{#1}}}
\newcommand{\ddt}[1]{ {{\partial{#1}} \over {\partial{t}}}}
\newcommand{\Ddt}[1]{ {{\partial{\mathbf{#1}}} \over {\partial{t}}}}
\newcommand{\ddts}[1]{ {{\partial^2{#1}} \over {\partial{t}^2}}}
\newcommand{\Ddts}[1]{ {{\partial^2{\mathbf{#1}}} \over {\partial{t}^2}}}
\newcommand{\dds}[2]{ {{\partial^2{#1}} \over {\partial{#2}^2}}}
\newcommand{\Dds}[2]{ {{\partial^2{\mathbf{#1}}} \over {\partial{#2}^2}}}
\newcommand{\Bj}[0]{\mathbf{j}}
\newcommand{\BB}[0]{\mathbf{B}}
\newcommand{\BE}[0]{\mathbf{E}}
\newcommand{\BS}[0]{\mathbf{S}}
\newcommand{\D}[2] {{{\partial #2} \over {\partial #1}}}

                             % The preamble begins here.
\title{An Example Document}  % Declares the document's title.
\author{Peeter Joot}         % Declares the author's name.
\date{March 25, 2000}        % Deleting this command produces today's date.

\begin{document}             % End of preamble and beginning of text.

%\maketitle                  % Produces the title.

\section{various formulations of Maxwell's equations}

It is interesting to look at the various formulations of Maxwell's 
equations.  The standard formulation these days is the differential
form, which isn't the most intuitive form.  In cgs units, the differential
form is as follows:

\begin{align*}
\Diverg E &= 4\pi\rho \\
\Diverg B &= 0 \\
\Curl{B} &= 4\pi \Bj - {1 \over c} \Ddt{E} \\
\Curl{E} &= {1 \over c} \Ddt{B}
\end{align*}

It is interesting to note that these are probably not the form that Maxwell 
originally formulated his equations in.  In my 1960 version of encyclodedia 
britianica Maxwell's equations were given in a integral form.  I had 
trouble seeing how these two forms were related at first, but they two 
quite different forms are a result of applying standard integral 
transformations like Gauss's theorem (where the a surface S and volume V are closed)

\begin{equation*}
\int_{V_S} \Diverg{F}\,dV =
\int_S \mathbf{F} \cdot\, d\BS 
\end{equation*}

and Stokes theorems (where S is an arbirary surface bounding curve C)
\begin{equation*}
\int_{S_C} \Curl{F} \cdot d\BS = \int_C \mathbf{F} \cdot d\mathbf{l}
\end{equation*}

Using these with maxwells equations
\begin{align*}
\int_{V_S} \Diverg E \, dV &= 4\pi \int_{V_S} \rho\, dV \\
\int_{V_S} \Diverg B \, dV &= 0 \\
\int_{S_C} \Curl{B} \, d\BS &= 4\pi \int_{S_C} \Bj \cdot \, d\BS - {1 \over c} \int_{S_C} \Ddt{E} \cdot \, d\BS \\
\int_{S_C} \Curl{E} \, d\BS &= {1 \over c} \int_{S_C} \Ddt{B} \cdot \, d\BS
\end{align*}

\begin{align*}
\int_{S} \BE \cdot \, d\BS &= 4\pi \int_{V_S} \rho\, dV \\
\int_{S} \BB \cdot \, d\BS &= 0 \\
\int_{C} \BB \cdot \, d\mathbf{l} &= 4\pi \int_{S_C} \Bj \cdot \, d\BS - {1 \over c} \int_{S_C} \Ddt{E} \cdot \, d\BS \\
\int_{C} \BE \cdot \, d\mathbf{l} &= {1 \over c} \int_{S_C} \Ddt{B} \cdot \, d\BS
\end{align*}

or in terms of normal and tangential components (as in the britanica) we have

\begin{align*}
\int_{S} E_n \, dS &= 4\pi \int_{V_S} \rho\, dV \\
\int_{S} B_n \, dS &= 0 \\
\int_{C} B_t \, dl &= 4\pi \int_{S_C} j_n \, dS - {1 \over c} {{\partial} \over {\partial t}} \int_{S_C} E_n \, dS \\
\int_{C} E_t \, dl &= {1 \over c} {{\partial} \over {\partial t}} \int_{S_C} B_n \, dS
\end{align*}

This is how 
maxwell's equations are presented in my old 
in encylopedia britianica 
except for the fact that 
the above is not in terms of SI units.  I expect that this formulation is closer to how 
maxwell actually presented his equations because they are in terms of more real 
seeming quantities.  There are no curl and divergence terms to cloud the mind or
abstract more than required to state the facts.

Other transformations to and from the differential and integral forms of these
equations are also possible.  For example, the formula $\Diverg E = 4\pi\rho$ 
can be shown by taking the divergence of the electic field due to a static
charge distribution and through use of Gauss's theorm
\begin{equation*}
\BE = \int_V{{\rho \mathbf{e}\,dV'} \over {\norm{\mathbf r - \mathbf r'}^2}}
\end{equation*}

This result is derived in my yellow electrodynamics book.  When I derived this
result myself, I found that in order to get the same results I needed to make
math workarounds and seeming hacks. ( describe later ).  These are probably 
justified and are perhaps more obvious to smarter people than myself.
Note that the general expression for the 
electric field in time varying conditions is considerably more complicated,
so perhaps some of what made the math for this derivation weird may be because
such a derivation has an artificial aspect to it to start with.

One of the awkward things with this standard formulation, whether it be the 
differential or the integral form, is that there 
is an awkward interdependence between the electric and magnetic fields. 
The solution of either $\BE$ or $\BB$ interdependently seems difficult.  The 
fact that there are four equations and two unknowns (assuming that the 
change and current distributions are known) is also slightly awkward.
There is also a great deal of asymetry in this form, which is unpleasant
to look at, and seems to suggest something missing.

In my electrodyamics book, where maxwell's equations are ``derived'',
these equations can be seen to have a more symetrical higher level form, where
these equations are written in the form of tensor relations where
the elements of the tensors are the electric and magnetic field components 
or scalar multiples of these components.  One of the things that I found 
pleasant about this formulation was that the cross product, a rather 
arbitrary sort of beast ends up occuring in a natural seeming fashion.  I 
think that this confirms the fact that in order to see the higher level 
structure or the cross product it must be expressed in a matrix or tensor form.
Another example of this can be found by looking at the three dimensional 
derivation of the torque formula, where the 
cross product can be seen to be the transformation (per unit angle)
applied to an object moving through arbirtary differential three dimensional 
rotation.

Using the vector relation

\begin{equation*}
\curlCurl{V} = \grad (\Diverg V) - \Delsquared{V}
\end{equation*}

The interdependence of the $\BE$ and $\BB$ field equations can be removed by taking the
curl of each of the curl equations.

\begin{align*}
\curlCurl{B} &= \grad (\Diverg{B}) - \Delsquared{B} \\
\curl(4\pi\Bj - {1 \over c} \Ddt{E}) &= \\
4\pi \Curl{j} - {1 \over c^2} \Ddts{B} &= \\
      	     &= 		   - \Delsquared{B}
\end{align*}

We can do the same calculation for the electric field.

\begin{align*}
\curlCurl{E} &= \grad (\Diverg{E}) - \Delsquared{E} \\
\curl({1 \over c} \Ddt{B}) &= \\
{1 \over c} \ddt(4\pi \Bj - {1 \over c} \Ddt{E}) &= \\
{4\pi \over c} \Ddt{j} - {1 \over c^2} \Ddts{E} &= \\
             &= \grad (4\pi\rho) - \Delsquared{E} \\
\end{align*}

Using these results we can express the 
electric field $\BE$ and the
magnetic field $\BB$.
as independent differential equations.

\begin{align*}
\Delsquared{B} - {1 \over c^2} \Ddts{B} &= - 4\pi \Curl{j} \\
\Delsquared{E} - {1 \over c^2} \Ddts{E} &= - {4\pi \over c} \Ddt{j} + 4\pi \grad{\rho}
\end{align*}

In the absense of current density and charge these equations take the simple form of standard wave
equations.

\begin{align*}
\Delsquared{E} - {1 \over c^2} \Ddt{E} &= 0 \\
\Delsquared{B} - {1 \over c^2} \Ddt{B} &= 0
\end{align*}

In this form, where the charge and current densities are zero, we finally have 
a symetrical description of the electric and the magnetic fields.  Also note 
that any solution to these is also a solution to the original form where 
there are non zero current density $\Bj$, and charge density $\rho$.

It can 
be noted that the general decoupled differential equations above are not 
as asymetrical as the 4 equation form of maxwell's equations.  In terms of the x, y, and z components 
of each equation, there is precisely two differential terms, with a suggestively 
similar form.
By introducing a 
single charge/current density four vector formed with the components 
$(j_x, j_y, j_z, \rho)$ and write these equations in component 
%(ie: tensor form) 
form, 
then some of the underlying symetry can be seen.

We 
can write the left hand sides of the equations in a simpler form by using the 
space time four vector $(x_\alpha)_\alpha = (x, y, z, ict)$

\begin{align*}
\sum_{\alpha = 1 .. 4}{\Dds{E} {x_\alpha}} &= - 4\pi \Curl{j} \\
\sum_{\alpha = 1 .. 4}{\Dds{B} {x_\alpha}} &= - {4\pi \over c} \Ddt{j} + 4\pi \grad{\rho}
\end{align*}

Breaking this down into components gives six equations

%curl_x F = DyFz - DzFx = {{\partial F_z} \over {\partial y}}
%curl_x F = DzFx - DxFy = {{\partial F_x} \over {\partial z}}
%curl_x F = DzFz - DzFx = {{\partial F_x} \over {\partial y}}

%\sum_{\alpha = 1 .. 4}{\dds{E_x} {x_\alpha}} &= - 4\pi \Curl{j}_x \\

%\sum_{\alpha = 1 .. 4}{\dds{E_x} {x_\alpha}} &= - 4\pi (\D {y}{j_z} - \D {z}{j_y}) \\
%\sum_{\alpha = 1 .. 4}{\dds{E_y} {x_\alpha}} &= - 4\pi (\D {z}{j_x} - \D {x}{j_z}) \\
%\sum_{\alpha = 1 .. 4}{\dds{E_z} {x_\alpha}} &= - 4\pi (\D {x}{j_y} - \D {y}{j_x}) \\

\begin{align*}
\sum_{\alpha = 1 .. 4}{\dds{E_x} {x_\alpha}} &= - 4\pi (\D {y}{j_z} - \D {z}{j_y}) \\
\sum_{\alpha = 1 .. 4}{\dds{E_y} {x_\alpha}} &= - 4\pi (\D {z}{j_x} - \D {x}{j_z}) \\
\sum_{\alpha = 1 .. 4}{\dds{E_z} {x_\alpha}} &= - 4\pi (\D {x}{j_y} - \D {y}{j_x}) \\
\sum_{\alpha = 1 .. 4}{\dds{B_x} {x_\alpha}} &= - 4\pi (\D {ct}{j_x} - \D{x}{\rho})\\
\sum_{\alpha = 1 .. 4}{\dds{B_y} {x_\alpha}} &= - 4\pi (\D {ct}{j_y} - \D{y}{\rho})\\
\sum_{\alpha = 1 .. 4}{\dds{B_z} {x_\alpha}} &= - 4\pi (\D {ct}{j_z} - \D{z}{\rho})
\end{align*}

Which can be put into the following form by replacing the $x, y, z$ indexes by numeric indexes

\begin{align*}
\sum_{\alpha = 1 .. 4}{i \dds{E_1} {x_\alpha}} &= - 4\pi (i \D {x_2}{j_z} - i \D {x_3}{j_y}) \\
\sum_{\alpha = 1 .. 4}{i \dds{E_2} {x_\alpha}} &= - 4\pi (i \D {x_3}{j_x} - i \D {x_1}{j_z}) \\
\sum_{\alpha = 1 .. 4}{i \dds{E_3} {x_\alpha}} &= - 4\pi (i \D {x_1}{j_y} - i \D {x_2}{j_x}) \\
\sum_{\alpha = 1 .. 4}{\dds{B_1} {x_\alpha}} &= - 4\pi (i \D {x_4}{j_x} - \D{x_1}{\rho})\\
\sum_{\alpha = 1 .. 4}{\dds{B_2} {x_\alpha}} &= - 4\pi (i \D {x_4}{j_y} - \D{x_2}{\rho})\\
\sum_{\alpha = 1 .. 4}{\dds{B_3} {x_\alpha}} &= - 4\pi (i \D {x_4}{j_z} - \D{x_3}{\rho})
\end{align*}

Note that a more appropriate charge density four vector includes an imaginary factor
\begin{equation*}
(j_\alpha)_\alpha = (i j_x, i j_y, i j_z, \rho)
\end{equation*}

and using this we get the following set of equations:

\begin{align*}
\sum_{\alpha = 1 .. 4}{i \dds{E_1} {x_\alpha}} &= - 4\pi (\D {x_2}{j_3} - \D {x_3}{j_2}) \\
\sum_{\alpha = 1 .. 4}{i \dds{E_2} {x_\alpha}} &= - 4\pi (\D {x_3}{j_1} - \D {x_1}{j_3}) \\
\sum_{\alpha = 1 .. 4}{i \dds{E_3} {x_\alpha}} &= - 4\pi (\D {x_1}{j_2} - \D {x_2}{j_1}) \\
\sum_{\alpha = 1 .. 4}{\dds{B_1} {x_\alpha}} &= - 4\pi (\D {x_4}{j_1} - \D{x_1}{j_4})\\
\sum_{\alpha = 1 .. 4}{\dds{B_2} {x_\alpha}} &= - 4\pi (\D {x_4}{j_2} - \D{x_2}{j_4})\\
\sum_{\alpha = 1 .. 4}{\dds{B_3} {x_\alpha}} &= - 4\pi (\D {x_4}{j_3} - \D{x_3}{j_4})
\end{align*}

This is as symetrical I can put these equations for now.  There is probably some higher level 
view that can be used to see the structure behind the particular set of indexes in the right 
hand size, but I will leave that for some other time.

\end{document}               % End of document.
