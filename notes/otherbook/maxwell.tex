\documentclass{article}      % Specifies the document class

\usepackage{amsmath}
\newcommand{\norm}[1]{\lVert#1\rVert}
\newcommand{\grad}[0]{\boldsymbol{\nabla}}
\newcommand{\curl}[0]{\grad \times}
\newcommand{\diverg}[0]{\grad \cdot}
\newcommand{\delsquared}[0]{\nabla^2}

% partial derivative of #1 wrt. #2:
\newcommand{\D}[2] {\frac {\partial #2} {\partial #1}}
% second partial derivative of #1 wrt. #2:
\newcommand{\Dsq}[2] {\frac {\partial^2 #2} {\partial {#1}^2}}

%
% shorthand for bold symbols:
%
\newcommand{\Bj}[0]{\mathbf{j}}
\newcommand{\BB}[0]{\mathbf{B}}
\newcommand{\BE}[0]{\mathbf{E}}
\newcommand{\BF}[0]{\mathbf{F}}
\newcommand{\BS}[0]{\mathbf{S}}
\newcommand{\BV}[0]{\mathbf{V}}

                             % The preamble begins here.
\title{Various formulations of Maxwell's equations} % Declares the document's title.
\author{Peeter Joot}         % Declares the author's name.
%\date{March 25, 2000}        % Deleting this command produces today's date.

\begin{document}             % End of preamble and beginning of text.

\maketitle{Notes from review and re-study of University electrodynamics}

\section{Differential and integral forms of Maxwell's equations}

The standard formulation of Maxwell's equations these days is the differential
form, which isn't the most intuitive.  In cgs units, the differential
form is as follows:

\begin{align*}
\diverg \BE &= 4\pi\rho \\
\diverg \BB &= 0 \\
\curl \BB &= 4\pi \Bj - \frac{1}{c} \D{t}{\BE} \\
\curl \BE &= \frac{1}{c} \D{t}{\BB}
\end{align*}

It is interesting to note that these are probably not the form that Maxwell 
originally formulated his equations in.  In my 1963 version of Encyclodedia 
Britianica, where most of the articles look like they are based more closely on the Author's
original works, Maxwell's equations were given in an integral form where the 
integrals were all related the tangential and normal components of the various vectors.  I had 
trouble seeing how these two forms were related at first because I hadn't expected to 
see the equations in this form.  Closer examination showed that the two forms 
were equivalent, and the transformation between the two can be made by applying 
applying standard integral transformations.

Specificially this can be done by using Gauss's theorem (where $V_S$ is the volume enclosed by surface S)

\begin{equation*}
\int_{V_S} \diverg \BF\,dV =
\int_S \mathbf{F} \cdot\, d\BS 
\end{equation*}

and Stokes theorem (where $S_C$ is an arbirary surface bounding curve C)
\begin{equation*}
\int_{S_C} \curl \BF \cdot d\BS = \oint_C \mathbf{F} \cdot d\mathbf{l}
\end{equation*}

and by integrating Maxwell's equiations over either volumes or surfaces.
\begin{align*}
\int_{V_S} \diverg \BE \, dV &= 4\pi \int_{V_S} \rho\, dV \\
\int_{V_S} \diverg \BB \, dV &= 0 \\
\int_{S_C} \curl \BB \cdot \, d\BS &= 4\pi \int_{S_C} \Bj \cdot \, d\BS - \frac{1}{c} \int_{S_C} \D{t}{\BE} \cdot \, d\BS \\
\int_{S_C} \curl \BE \cdot \, d\BS &= \frac{1}{c} \int_{S_C} \D{t}{\BB} \cdot \, d\BS
\end{align*}

The applictions of the Gauss and Stokes theorems gives
\begin{align*}
\int_{S} \BE \cdot \, d\BS &= 4\pi \int_{V_S} \rho\, dV \\
\int_{S} \BB \cdot \, d\BS &= 0 \\
\oint_{C} \BB \cdot \, d\mathbf{l} &= 4\pi \int_{S_C} \Bj \cdot \, d\BS - \frac{1}{c} \int_{S_C} \D{t}{\BE} \cdot \, d\BS \\
\oint_{C} \BE \cdot \, d\mathbf{l} &= \frac{1}{c} \int_{S_C} \D{t}{\BB} \cdot \, d\BS
\end{align*}

In terms of normal and tangential components, and after taking the time 
deriatives out of the integrals, we have

\begin{align*}
\int_{S} E_n \, dS &= 4\pi \int_{V_S} \rho\, dV \\
\int_{S} B_n \, dS &= 0 \\
\oint_{C} B_t \, dl &= 4\pi \int_{S_C} j_n \, dS - \frac{1}{c} \D{t}{} \int_{S_C} E_n \, dS \\
\oint_{C} E_t \, dl &= \frac{1}{c} \D{t}{} \int_{S_C} B_n \, dS
\end{align*}

This is how 
Maxwell's equations are presented in my old 
Encylopedia Britianica 
except for the fact that 
the above is not in terms of SI units.  I expect that this formulation is closer to how 
Maxwell actually presented his equations because they are in terms of more real 
seeming quantities.  There are no curl and divergence terms to cloud the mind or
abstract more than required to state the facts.

However, as with the differential form, it is not clear how one would find
$\BE$ and $\BB$ given an arbitrary charge and current densities.  More on 
this below.

\section{Integral and differential forms of Gauss's law}

Other transformations to and from the differential and integral forms of these
equations are also possible.  For example, the formula $\diverg \BE = 4\pi\rho$ 
can be shown by taking the divergence of the electic field due to a static
charge distribution
\begin{equation*}
\diverg \BE = \diverg \int_V{\frac{\rho (\mathbf{r - r'})\,dV'}{\norm{\mathbf r - \mathbf r'}^3}}
\end{equation*}

This result is derived in my yellow electrodynamics book, but is quite easy to 
do.  The divergence can be moved inside the integral where it operates only on 
the $\frac {\mathbf{r - r'}} {\norm{\mathbf r - r'}^3}$ terms, since $\rho$ is 
only a function of $\mathbf r'$.  
Except in a nieghbourhood 
$\norm{\mathbf{r - r'}} < \epsilon$
where the derivative cannot be taken,
this divergence works out to be zero which is pretty easy to calculate.
For this region, one can use Gauss's theorem to convert the 
integral into a surface integral, which evaluates to $4\pi$ easily by integrating in spherical polar coordinates
around a small sphere around $\mathbf r$.

Note that the general expression for the 
electric field in time varying conditions is considerably more complicated.
The electrostatics formula 
$\BE = \int_V{\frac{\rho (\mathbf{r - r'})\,dV'}{\norm{\mathbf r - \mathbf r'}^3}}$
for the electric field is not valid in time 
varing fields, but what 
about $\diverg \BE = 4\pi\rho$ (Gauss's law)?  In the statement of Maxwell's equations
there are no qualifications about validity in time varing fields.  This may
make sense since the electrostatics formula for $\BE$ implies Gauss's law, 
but Gauss's law by itself doesn't imply the former, at least so far as I can
see, leaving potential 
degrees of freedom to account for time varience.

\section{Interdepencies of the electric and magnetic field equations}

One of the awkward things with the standard formulation of 
Maxwell's equations, whether it be the 
differential or the integral form, is that there 
is an awkward interdependence between the electric and magnetic fields. 
The solution of either $\BE$ or $\BB$ interdependently seems difficult.  The 
fact that there are four equations and two unknowns (assuming that the 
change and current distributions are known) is also slightly awkward.
There is also a great deal of asymetry in this form, which is unpleasant
to look at, and seems to suggest something missing.

In my electrodyamics book, where Maxwell's equations are ``derived''
\footnote
{
Like many derivations in math and physics, knowing the answer before hand
can lead to elegant and sophisticated, but artifical, methods of showing the desired result --
methods that
nobody would naturally attempt to use if actually deriving the result from first principles.
},
these equations can be seen to have a more symetrical higher level form, where
these equations are written in the form of tensor relations where
the elements of the tensors are the electric and magnetic field components 
or scalar multiples of these components.  One of the things that I found 
pleasant about this formulation was that the cross product, a rather 
arbitrary sort of beast ends up occuring in a natural seeming fashion.  I 
think that this confirms the fact that in order to see the higher level 
structure or the cross product it must be expressed in a matrix or tensor form.
Another example of this can be found by looking at the three dimensional 
derivation of the torque formula, where the 
cross product can be seen to be the transformation (per unit angle)
applied to an object moving through arbirtary differential three dimensional 
rotation.

Using the vector relation

\begin{equation*}
\curl (\curl \BV) = \grad (\diverg \BV) - \delsquared \BV
\end{equation*}

The interdependence of the $\BE$ and $\BB$ field equations can be removed by taking the
curl of each of the curl equations.

\begin{align*}
\curl (\curl \BB) 			&= \grad (\diverg \BB) - \delsquared \BB \\
\curl \Bigl(4\pi\Bj - \frac{1}{c} \D{t}{\BE} \Bigr) 	&= \\
4\pi\curl \Bj - \frac{1}{c} \D{t}{} \curl \BE 	&= \\
4\pi\curl \Bj - \frac{1}{c^2} \Dsq{t}{\BB} &= \\
      	     				&= - \delsquared \BB
\end{align*}

We can do the same calculation for the electric field.

\begin{align*}
\curl (\curl \BE) &= \grad (\diverg \BE) - \delsquared \BE \\
\curl \Bigl(\frac{1}{c} \D{t}{\BB}\Bigr) &= \\
\frac{1}{c} \D{t}{} \curl \BB &= \\
\frac{1}{c} \D{t}{} \Bigl(4\pi \Bj - \frac{1}{c} \D{t}{\BE} \Bigr) &= \\
\frac{4\pi}{c} \D{t}{\Bj} - \frac{1}{c^2} \Dsq{t}{\BE} &= \\
             &= \grad (4\pi\rho) - \delsquared \BE \\
\end{align*}

Using these results we can express the 
electric field $\BE$ and the
magnetic field $\BB$
as independent differential equations.

\begin{align*}
\delsquared \BB - \frac{1}{c^2} \Dsq{t}{\BB} &= - 4\pi \left(\curl \Bj\right) \\
\delsquared \BE - \frac{1}{c^2} \Dsq{t}{\BE} &= - 4\pi \Bigl(\frac{1}{c} \D{t}{\Bj} + \grad{\rho}\Bigr)
\end{align*}

In the absense of current and charge densities these equations take the simple form of standard wave
equations.

\begin{align*}
\delsquared \BE - \frac{1}{c^2} \Dsq{t}{\BE} &= 0 \\
\delsquared \BB - \frac{1}{c^2} \Dsq{t}{\BB} &= 0
\end{align*}

In this form, where the charge and current densities are zero, we finally have 
a symetrical description of the electric and the magnetic fields.  Also note 
that any solution to these is also a solution to the original form where 
there are non zero current density $\Bj$, and charge density $\rho$.

It can 
be noted that the general decoupled differential equations above are not 
as asymetrical as the 4 equation form of Maxwell's equations.  In terms of the x, y, and z components 
of each equation, there is precisely two differential terms, with a suggestively 
similar form.
By introducing a 
single charge/current density four vector formed with the components 
$(j_x, j_y, j_z, \rho)$ and write these equations in component 
%(ie: tensor form) 
form, 
then some of the underlying symetry can be seen.

We 
can write the left hand sides of the equations in a simpler form by using the 
space time four vector $(x_\alpha)_\alpha = (x, y, z, ict)$

\begin{align*}
\sum_{\alpha = 1 .. 4}{\Dsq{x_\alpha}{\BE}} &= - 4\pi \curl \Bj \\
\sum_{\alpha = 1 .. 4}{\Dsq{x_\alpha}{\BB}} &= - \frac{4\pi}{c} \D{t}{\Bj} + 4\pi \grad{\rho}
\end{align*}

Breaking this down into components gives six equations

%curl_x F = DyFz - DzFx = \D{y}{F_z} - \D{z}{F_x}
%curl_x F = DzFx - DxFy = \D{z}{F_x} - \D{x}{F_y}
%curl_x F = DxFy - DyFz = \D{x}{F_y} - \D{y}{F_z}

%\sum_{\alpha = 1 .. 4}{\Dsq{x_\alpha}{E_x}} &= - 4\pi \curl \Bj_x \\

%\sum_{\alpha = 1 .. 4}{\Dsq{x_\alpha}{E_x}} &= - 4\pi (\D{y}{j_z} - \D{z}{j_y}) \\
%\sum_{\alpha = 1 .. 4}{\Dsq{x_\alpha}{E_y}} &= - 4\pi (\D{z}{j_x} - \D{x}{j_z}) \\
%\sum_{\alpha = 1 .. 4}{\Dsq{x_\alpha}{E_z}} &= - 4\pi (\D{x}{j_y} - \D{y}{j_x}) \\

\begin{align*}
\sum_{\alpha = 1 .. 4}{\Dsq{x_\alpha}{E_x}} &= - 4\pi (\D{y}{j_z} - \D{z}{j_y}) \\
\sum_{\alpha = 1 .. 4}{\Dsq{x_\alpha}{E_y}} &= - 4\pi (\D{z}{j_x} - \D{x}{j_z}) \\
\sum_{\alpha = 1 .. 4}{\Dsq{x_\alpha}{E_z}} &= - 4\pi (\D{x}{j_y} - \D{y}{j_x}) \\
\sum_{\alpha = 1 .. 4}{\Dsq{x_\alpha}{B_x}} &= - 4\pi (\D{ct}{j_x} - \D{x}{\rho})\\
\sum_{\alpha = 1 .. 4}{\Dsq{x_\alpha}{B_y}} &= - 4\pi (\D{ct}{j_y} - \D{y}{\rho})\\
\sum_{\alpha = 1 .. 4}{\Dsq{x_\alpha}{B_z}} &= - 4\pi (\D{ct}{j_z} - \D{z}{\rho})
\end{align*}

Which can be put into the following form by replacing the $x, y, z$ indexes by numeric indexes

\begin{align*}
\sum_{\alpha = 1 .. 4}{i \Dsq{x_\alpha}{E_1}} &= - 4\pi (i \D{x_2}{j_z} - i \D{x_3}{j_y}) \\
\sum_{\alpha = 1 .. 4}{i \Dsq{x_\alpha}{E_2}} &= - 4\pi (i \D{x_3}{j_x} - i \D{x_1}{j_z}) \\
\sum_{\alpha = 1 .. 4}{i \Dsq{x_\alpha}{E_3}} &= - 4\pi (i \D{x_1}{j_y} - i \D{x_2}{j_x}) \\
\sum_{\alpha = 1 .. 4}{\Dsq{x_\alpha}{B_1}} &= - 4\pi (i \D{x_4}{j_x} - \D{x_1}{\rho})\\
\sum_{\alpha = 1 .. 4}{\Dsq{x_\alpha}{B_2}} &= - 4\pi (i \D{x_4}{j_y} - \D{x_2}{\rho})\\
\sum_{\alpha = 1 .. 4}{\Dsq{x_\alpha}{B_3}} &= - 4\pi (i \D{x_4}{j_z} - \D{x_3}{\rho})
\end{align*}

Note that a more appropriate charge density four vector includes an imaginary factor
\begin{equation*}
(j_\alpha)_\alpha = (i j_x, i j_y, i j_z, \rho)
\end{equation*}

and using this we get the following set of equations:

\begin{align*}
\sum_{\alpha = 1 .. 4}{i \Dsq{x_\alpha}{E_1}} &= - 4\pi (\D{x_2}{j_3} - \D{x_3}{j_2}) \\
\sum_{\alpha = 1 .. 4}{i \Dsq{x_\alpha}{E_2}} &= - 4\pi (\D{x_3}{j_1} - \D{x_1}{j_3}) \\
\sum_{\alpha = 1 .. 4}{i \Dsq{x_\alpha}{E_3}} &= - 4\pi (\D{x_1}{j_2} - \D{x_2}{j_1}) \\
\sum_{\alpha = 1 .. 4}{\Dsq{x_\alpha}{B_1}} &= - 4\pi (\D{x_4}{j_1} - \D{x_1}{j_4})\\
\sum_{\alpha = 1 .. 4}{\Dsq{x_\alpha}{B_2}} &= - 4\pi (\D{x_4}{j_2} - \D{x_2}{j_4})\\
\sum_{\alpha = 1 .. 4}{\Dsq{x_\alpha}{B_3}} &= - 4\pi (\D{x_4}{j_3} - \D{x_3}{j_4})
\end{align*}

This is as symetrical I can put these equations for now.  There is probably some higher level 
view that can be used to see the structure behind the particular set of indexes in the right 
hand size, but I will leave that for some other time.

\end{document}               % End of document.
