\documentclass[]{eliblog}

\usepackage{amsmath}
\usepackage{mathpazo}

%
% shorthand for bold symbols, convenient for vectors and matrices
%
\newcommand{\Ba}[0]{\mathbf{a}}
\newcommand{\Bb}[0]{\mathbf{b}}
\newcommand{\Bc}[0]{\mathbf{c}}
\newcommand{\Bd}[0]{\mathbf{d}}
\newcommand{\Be}[0]{\mathbf{e}}
\newcommand{\Bf}[0]{\mathbf{f}}
\newcommand{\Bg}[0]{\mathbf{g}}
\newcommand{\Bh}[0]{\mathbf{h}}
\newcommand{\Bi}[0]{\mathbf{i}}
\newcommand{\Bj}[0]{\mathbf{j}}
\newcommand{\Bk}[0]{\mathbf{k}}
\newcommand{\Bl}[0]{\mathbf{l}}
\newcommand{\Bm}[0]{\mathbf{m}}
\newcommand{\Bn}[0]{\mathbf{n}}
\newcommand{\Bo}[0]{\mathbf{o}}
\newcommand{\Bp}[0]{\mathbf{p}}
\newcommand{\Bq}[0]{\mathbf{q}}
\newcommand{\Br}[0]{\mathbf{r}}
\newcommand{\Bs}[0]{\mathbf{s}}
\newcommand{\Bt}[0]{\mathbf{t}}
\newcommand{\Bu}[0]{\mathbf{u}}
\newcommand{\Bv}[0]{\mathbf{v}}
\newcommand{\Bw}[0]{\mathbf{w}}
\newcommand{\Bx}[0]{\mathbf{x}}
\newcommand{\By}[0]{\mathbf{y}}
\newcommand{\Bz}[0]{\mathbf{z}}
\newcommand{\BA}[0]{\mathbf{A}}
\newcommand{\BB}[0]{\mathbf{B}}
\newcommand{\BC}[0]{\mathbf{C}}
\newcommand{\BD}[0]{\mathbf{D}}
\newcommand{\BE}[0]{\mathbf{E}}
\newcommand{\BF}[0]{\mathbf{F}}
\newcommand{\BG}[0]{\mathbf{G}}
\newcommand{\BH}[0]{\mathbf{H}}
\newcommand{\BI}[0]{\mathbf{I}}
\newcommand{\BJ}[0]{\mathbf{J}}
\newcommand{\BK}[0]{\mathbf{K}}
\newcommand{\BL}[0]{\mathbf{L}}
\newcommand{\BM}[0]{\mathbf{M}}
\newcommand{\BN}[0]{\mathbf{N}}
\newcommand{\BO}[0]{\mathbf{O}}
\newcommand{\BP}[0]{\mathbf{P}}
\newcommand{\BQ}[0]{\mathbf{Q}}
\newcommand{\BR}[0]{\mathbf{R}}
\newcommand{\BS}[0]{\mathbf{S}}
\newcommand{\BT}[0]{\mathbf{T}}
\newcommand{\BU}[0]{\mathbf{U}}
\newcommand{\BV}[0]{\mathbf{V}}
\newcommand{\BW}[0]{\mathbf{W}}
\newcommand{\BX}[0]{\mathbf{X}}
\newcommand{\BY}[0]{\mathbf{Y}}
\newcommand{\BZ}[0]{\mathbf{Z}}

\newcommand{\Bzero}[0]{\mathbf{0}}
\newcommand{\Btheta}[0]{\boldsymbol{\theta}}
\newcommand{\Btau}[0]{\boldsymbol{\tau}}
\newcommand{\Bomega}[0]{\boldsymbol{\omega}}

%
% shorthand for unit vectors
%
\newcommand{\acap}[0]{\hat{\Ba}}
\newcommand{\bcap}[0]{\hat{\Bb}}
\newcommand{\ccap}[0]{\hat{\Bc}}
\newcommand{\dcap}[0]{\hat{\Bd}}
\newcommand{\ecap}[0]{\hat{\Be}}
\newcommand{\fcap}[0]{\hat{\Bf}}
\newcommand{\gcap}[0]{\hat{\Bg}}
\newcommand{\hcap}[0]{\hat{\Bh}}
\newcommand{\icap}[0]{\hat{\Bi}}
\newcommand{\jcap}[0]{\hat{\Bj}}
\newcommand{\kcap}[0]{\hat{\Bk}}
\newcommand{\lcap}[0]{\hat{\Bl}}
\newcommand{\mcap}[0]{\hat{\Bm}}
\newcommand{\ncap}[0]{\hat{\Bn}}
\newcommand{\ocap}[0]{\hat{\Bo}}
\newcommand{\pcap}[0]{\hat{\Bp}}
\newcommand{\qcap}[0]{\hat{\Bq}}
\newcommand{\rcap}[0]{\hat{\Br}}
\newcommand{\scap}[0]{\hat{\Bs}}
\newcommand{\tcap}[0]{\hat{\Bt}}
\newcommand{\ucap}[0]{\hat{\Bu}}
\newcommand{\vcap}[0]{\hat{\Bv}}
\newcommand{\wcap}[0]{\hat{\Bw}}
\newcommand{\xcap}[0]{\hat{\Bx}}
\newcommand{\ycap}[0]{\hat{\By}}
\newcommand{\zcap}[0]{\hat{\Bz}}
\newcommand{\thetacap}[0]{\hat{\Btheta}}

%
% to write R^n and C^n in a distinguishable fashion.  Perhaps change this
% to the double lined characters upon figuring out how to do so.
%
\newcommand{\C}[1]{$\mathbb{C}^{#1}$}
\newcommand{\R}[1]{$\mathbb{R}^{#1}$}

%
% various generally useful helpers
%

% derivative of #1 wrt. #2:
\newcommand{\D}[2] {\frac {d#2} {d#1}}

\newcommand{\inv}[1]{\frac{1}{#1}}
\newcommand{\cross}[0]{\times}

\newcommand{\abs}[1]{\lvert{#1}\rvert}
\newcommand{\norm}[1]{\lVert{#1}\rVert}
\newcommand{\innerprod}[2]{\langle{#1}, {#2}\rangle}
\newcommand{\dotprod}[2]{{#1} \cdot {#2}}
\newcommand{\bdotprod}[2]{\left({#1} \cdot {#2}\right)}
\newcommand{\crossprod}[2]{{#1} \cross {#2}}
\newcommand{\tripleprod}[3]{\dotprod{\left(\crossprod{#1}{#2}\right)}{#3}}

\DeclareMathOperator{\Proj}{Proj}
\DeclareMathOperator{\Span}{span}
\DeclareMathOperator{\Sgn}{sgn}
\DeclareMathOperator{\Area}{Area}
\DeclareMathOperator{\Volume}{Volume}

%
% A few miscellaneous things specific to this document
%
\newcommand{\crossop}[1]{\crossprod{#1}{}}

% R2 vector.
\newcommand{\VectorTwo}[2]{
\begin{bmatrix}
 {#1} \\
 {#2}
\end{bmatrix}
}

\newcommand{\VectorN}[1]{
\begin{bmatrix}
{#1}_1 \\
{#1}_2 \\
\vdots \\
{#1}_N \\
\end{bmatrix}
}

\newcommand{\DETuvij}[4]{
\begin{vmatrix}
 {#1}_{#3} & {#1}_{#4} \\
 {#2}_{#3} & {#2}_{#4}
\end{vmatrix}
}

\newcommand{\DETuvwijk}[6]{
\begin{vmatrix}
 {#1}_{#4} & {#1}_{#5} & {#1}_{#6} \\
 {#2}_{#4} & {#2}_{#5} & {#2}_{#6} \\
 {#3}_{#4} & {#3}_{#5} & {#3}_{#6}
\end{vmatrix}
}

\newcommand{\DETuvwxijkl}[8]{
\begin{vmatrix}
 {#1}_{#5} & {#1}_{#6} & {#1}_{#7} & {#1}_{#8} \\
 {#2}_{#5} & {#2}_{#6} & {#2}_{#7} & {#2}_{#8} \\
 {#3}_{#5} & {#3}_{#6} & {#3}_{#7} & {#3}_{#8} \\
 {#4}_{#5} & {#4}_{#6} & {#4}_{#7} & {#4}_{#8} \\
\end{vmatrix}
}

%\newcommand{\DETuvwxyijklm}[10]{
%\begin{vmatrix}
% {#1}_{#6} & {#1}_{#7} & {#1}_{#8} & {#1}_{#9} & {#1}_{#10} \\
% {#2}_{#6} & {#2}_{#7} & {#2}_{#8} & {#2}_{#9} & {#2}_{#10} \\
% {#3}_{#6} & {#3}_{#7} & {#3}_{#8} & {#3}_{#9} & {#3}_{#10} \\
% {#4}_{#6} & {#4}_{#7} & {#4}_{#8} & {#4}_{#9} & {#4}_{#10} \\
% {#5}_{#6} & {#5}_{#7} & {#5}_{#8} & {#5}_{#9} & {#5}_{#10}
%\end{vmatrix}
%}

% R3 vector.
\newcommand{\VectorThree}[3]{
\begin{bmatrix}
 {#1} \\
 {#2} \\
 {#3}
\end{bmatrix}
}



\author{Peeter Joot}
\email{peeter.joot@gmail.com}


\chapter{Dot product of vector and blade}
\label{chap:dotBlade}
%\useCCL
\blogpage{http://sites.google.com/site/peeterjoot/math2009/dotBlade.pdf}
\date{Aug 11, 2009}
\revisionInfo{$RCSfile: dotBlade.tex,v $ Last $Revision: 1.3 $ $Date: 2009/08/12 04:02:30 $}

\beginArtWithToc
%\beginArtNoToc

Scott asked about the following vector bivector product identities

\begin{align}
\Bw (\Bu \wedge \Bv) &= \Bw \cdot (\Bu \wedge \Bv) + \Bw \wedge \Bu \wedge \Bv \\
(\Bu \wedge \Bv) \Bw &= - \Bw \cdot (\Bu \wedge \Bv) + \Bw \wedge \Bu \wedge \Bv
\end{align}

Specifically, he asked why the sign in the second identity is negative, and if this was a typo (\href{http://sites.google.com/site/peeterjoot/math2009/gabook.pdf}{i.e. in my GA Notes}).

It probably would have been clearer to write

\begin{align*}
(\Bu \wedge \Bv) \Bw = (\Bu \wedge \Bv) \cdot \Bw + (\Bu \wedge \Bv) \wedge \Bw
\end{align*}

But this an equivalent statement, and not a correction.  Let's see why.  The key to this is that while the dot product of vectors is symmetric, the dot product of a vector and other objects (like this bivector) may be antisymmetric.

The fundamental definitions of the generalized dot and wedge products is really based on grade selection.  One writes for the product of a grade $r$ blade $A$ with a vector $\Bv$ is

\begin{align}\label{eqn:foo4}
A \cdot \Bv &\equiv \gpgrade{A \Bv}{r-1} \\
A \wedge \Bv &\equiv \gpgrade{A \Bv}{r+1} 
\end{align}

It is possible to show that this $A \Bv$ product into its grade $r-1$ and grade $r+1$ parts can also be expressed by the following symmetric and antisymmetric split

\begin{align}\label{eqn:foo1}
A \cdot \Bv &= \inv{2} (A \Bv -(-1)^r \Bv A) = (-1)^{r+1} \Bv \cdot A \\
A \wedge \Bv &= \inv{2} (A \Bv +(-1)^r \Bv A) = (-1)^r \Bv \wedge A
\end{align}

So for a bivector vector dot product we have 

\begin{align}\label{eqn:foo2}
A \cdot \Bv &= - \Bv \cdot A \\
A \wedge \Bv &= \Bv \wedge A
\end{align}

To remove some of the abstraction, and avoid the too general relationships above, we can illustrate this by example.  Suppose we are working in three dimensional Euclidian space with orthonormal unit vectors $\Be_1$, $\Be_2$, and $\Be_3$.  For orthonormal vectors in a Euclidean space we have a unit square, such as $\Be_1 \Be_1 = 1$.  We also have for any perpendicular vectors, such as $\Be_1$, and $\Be_2$, a change in sign on interchange $\Be_1 \Be_2 = -\Be_2 \Be_1$.  Using the fundamental definition of the generalized dot product (\ref{eqn:foo1}), as opposed to the derived antisymmetric identity of (\ref{eqn:foo1}), the dot product of a bivector for the $x-y$ plane $\Be_1 \Be_2$ with $\Be_1$, one of the unit vectors in the plane is

\begin{align*}
(\Be_1 \Be_2) \cdot \Be_1
&=
\gpgradeone{ \Be_1 \Be_2 \Be_1 } \\
&=
-\gpgradeone{ \Be_1 \Be_1 \Be_2 } \\
&=
-\gpgradeone{ \Be_2 } \\
&=
-\Be_2
\end{align*}

Similarily, dotting from the left we have

\begin{align*}
\Be_1 \cdot (\Be_1 \Be_2) 
&=
\gpgradeone{ \Be_1 \Be_1 \Be_2 } \\
&=
\gpgradeone{ \Be_2 } \\
&=
\Be_2
\end{align*}

Observe that the order of the dotting operation is in this case significant.  This interchange of sign for the dot product of a vector with a bivector is a general property and even holds in more general metrics (such as the Minkowski space where we have $\pm 1$ for the square of the basis vectors).

%\EndArticle
\EndNoBibArticle
