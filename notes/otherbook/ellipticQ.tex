%
% Copyright � 2012 Peeter Joot.  All Rights Reserved.
% Licenced as described in the file LICENSE under the root directory of this GIT repository.
%

% 
% 
%\documentclass[]{eliblog}

\usepackage{amsmath}
\usepackage{mathpazo}

%
% shorthand for bold symbols, convenient for vectors and matrices
%
\newcommand{\Ba}[0]{\mathbf{a}}
\newcommand{\Bb}[0]{\mathbf{b}}
\newcommand{\Bc}[0]{\mathbf{c}}
\newcommand{\Bd}[0]{\mathbf{d}}
\newcommand{\Be}[0]{\mathbf{e}}
\newcommand{\Bf}[0]{\mathbf{f}}
\newcommand{\Bg}[0]{\mathbf{g}}
\newcommand{\Bh}[0]{\mathbf{h}}
\newcommand{\Bi}[0]{\mathbf{i}}
\newcommand{\Bj}[0]{\mathbf{j}}
\newcommand{\Bk}[0]{\mathbf{k}}
\newcommand{\Bl}[0]{\mathbf{l}}
\newcommand{\Bm}[0]{\mathbf{m}}
\newcommand{\Bn}[0]{\mathbf{n}}
\newcommand{\Bo}[0]{\mathbf{o}}
\newcommand{\Bp}[0]{\mathbf{p}}
\newcommand{\Bq}[0]{\mathbf{q}}
\newcommand{\Br}[0]{\mathbf{r}}
\newcommand{\Bs}[0]{\mathbf{s}}
\newcommand{\Bt}[0]{\mathbf{t}}
\newcommand{\Bu}[0]{\mathbf{u}}
\newcommand{\Bv}[0]{\mathbf{v}}
\newcommand{\Bw}[0]{\mathbf{w}}
\newcommand{\Bx}[0]{\mathbf{x}}
\newcommand{\By}[0]{\mathbf{y}}
\newcommand{\Bz}[0]{\mathbf{z}}
\newcommand{\BA}[0]{\mathbf{A}}
\newcommand{\BB}[0]{\mathbf{B}}
\newcommand{\BC}[0]{\mathbf{C}}
\newcommand{\BD}[0]{\mathbf{D}}
\newcommand{\BE}[0]{\mathbf{E}}
\newcommand{\BF}[0]{\mathbf{F}}
\newcommand{\BG}[0]{\mathbf{G}}
\newcommand{\BH}[0]{\mathbf{H}}
\newcommand{\BI}[0]{\mathbf{I}}
\newcommand{\BJ}[0]{\mathbf{J}}
\newcommand{\BK}[0]{\mathbf{K}}
\newcommand{\BL}[0]{\mathbf{L}}
\newcommand{\BM}[0]{\mathbf{M}}
\newcommand{\BN}[0]{\mathbf{N}}
\newcommand{\BO}[0]{\mathbf{O}}
\newcommand{\BP}[0]{\mathbf{P}}
\newcommand{\BQ}[0]{\mathbf{Q}}
\newcommand{\BR}[0]{\mathbf{R}}
\newcommand{\BS}[0]{\mathbf{S}}
\newcommand{\BT}[0]{\mathbf{T}}
\newcommand{\BU}[0]{\mathbf{U}}
\newcommand{\BV}[0]{\mathbf{V}}
\newcommand{\BW}[0]{\mathbf{W}}
\newcommand{\BX}[0]{\mathbf{X}}
\newcommand{\BY}[0]{\mathbf{Y}}
\newcommand{\BZ}[0]{\mathbf{Z}}

\newcommand{\Bzero}[0]{\mathbf{0}}
\newcommand{\Btheta}[0]{\boldsymbol{\theta}}
\newcommand{\Btau}[0]{\boldsymbol{\tau}}
\newcommand{\Bomega}[0]{\boldsymbol{\omega}}

%
% shorthand for unit vectors
%
\newcommand{\acap}[0]{\hat{\Ba}}
\newcommand{\bcap}[0]{\hat{\Bb}}
\newcommand{\ccap}[0]{\hat{\Bc}}
\newcommand{\dcap}[0]{\hat{\Bd}}
\newcommand{\ecap}[0]{\hat{\Be}}
\newcommand{\fcap}[0]{\hat{\Bf}}
\newcommand{\gcap}[0]{\hat{\Bg}}
\newcommand{\hcap}[0]{\hat{\Bh}}
\newcommand{\icap}[0]{\hat{\Bi}}
\newcommand{\jcap}[0]{\hat{\Bj}}
\newcommand{\kcap}[0]{\hat{\Bk}}
\newcommand{\lcap}[0]{\hat{\Bl}}
\newcommand{\mcap}[0]{\hat{\Bm}}
\newcommand{\ncap}[0]{\hat{\Bn}}
\newcommand{\ocap}[0]{\hat{\Bo}}
\newcommand{\pcap}[0]{\hat{\Bp}}
\newcommand{\qcap}[0]{\hat{\Bq}}
\newcommand{\rcap}[0]{\hat{\Br}}
\newcommand{\scap}[0]{\hat{\Bs}}
\newcommand{\tcap}[0]{\hat{\Bt}}
\newcommand{\ucap}[0]{\hat{\Bu}}
\newcommand{\vcap}[0]{\hat{\Bv}}
\newcommand{\wcap}[0]{\hat{\Bw}}
\newcommand{\xcap}[0]{\hat{\Bx}}
\newcommand{\ycap}[0]{\hat{\By}}
\newcommand{\zcap}[0]{\hat{\Bz}}
\newcommand{\thetacap}[0]{\hat{\Btheta}}

%
% to write R^n and C^n in a distinguishable fashion.  Perhaps change this
% to the double lined characters upon figuring out how to do so.
%
\newcommand{\C}[1]{$\mathbb{C}^{#1}$}
\newcommand{\R}[1]{$\mathbb{R}^{#1}$}

%
% various generally useful helpers
%

% derivative of #1 wrt. #2:
\newcommand{\D}[2] {\frac {d#2} {d#1}}

\newcommand{\inv}[1]{\frac{1}{#1}}
\newcommand{\cross}[0]{\times}

\newcommand{\abs}[1]{\lvert{#1}\rvert}
\newcommand{\norm}[1]{\lVert{#1}\rVert}
\newcommand{\innerprod}[2]{\langle{#1}, {#2}\rangle}
\newcommand{\dotprod}[2]{{#1} \cdot {#2}}
\newcommand{\bdotprod}[2]{\left({#1} \cdot {#2}\right)}
\newcommand{\crossprod}[2]{{#1} \cross {#2}}
\newcommand{\tripleprod}[3]{\dotprod{\left(\crossprod{#1}{#2}\right)}{#3}}

\DeclareMathOperator{\Proj}{Proj}
\DeclareMathOperator{\Span}{span}
\DeclareMathOperator{\Sgn}{sgn}
\DeclareMathOperator{\Area}{Area}
\DeclareMathOperator{\Volume}{Volume}

%
% A few miscellaneous things specific to this document
%
\newcommand{\crossop}[1]{\crossprod{#1}{}}

% R2 vector.
\newcommand{\VectorTwo}[2]{
\begin{bmatrix}
 {#1} \\
 {#2}
\end{bmatrix}
}

\newcommand{\VectorN}[1]{
\begin{bmatrix}
{#1}_1 \\
{#1}_2 \\
\vdots \\
{#1}_N \\
\end{bmatrix}
}

\newcommand{\DETuvij}[4]{
\begin{vmatrix}
 {#1}_{#3} & {#1}_{#4} \\
 {#2}_{#3} & {#2}_{#4}
\end{vmatrix}
}

\newcommand{\DETuvwijk}[6]{
\begin{vmatrix}
 {#1}_{#4} & {#1}_{#5} & {#1}_{#6} \\
 {#2}_{#4} & {#2}_{#5} & {#2}_{#6} \\
 {#3}_{#4} & {#3}_{#5} & {#3}_{#6}
\end{vmatrix}
}

\newcommand{\DETuvwxijkl}[8]{
\begin{vmatrix}
 {#1}_{#5} & {#1}_{#6} & {#1}_{#7} & {#1}_{#8} \\
 {#2}_{#5} & {#2}_{#6} & {#2}_{#7} & {#2}_{#8} \\
 {#3}_{#5} & {#3}_{#6} & {#3}_{#7} & {#3}_{#8} \\
 {#4}_{#5} & {#4}_{#6} & {#4}_{#7} & {#4}_{#8} \\
\end{vmatrix}
}

%\newcommand{\DETuvwxyijklm}[10]{
%\begin{vmatrix}
% {#1}_{#6} & {#1}_{#7} & {#1}_{#8} & {#1}_{#9} & {#1}_{#10} \\
% {#2}_{#6} & {#2}_{#7} & {#2}_{#8} & {#2}_{#9} & {#2}_{#10} \\
% {#3}_{#6} & {#3}_{#7} & {#3}_{#8} & {#3}_{#9} & {#3}_{#10} \\
% {#4}_{#6} & {#4}_{#7} & {#4}_{#8} & {#4}_{#9} & {#4}_{#10} \\
% {#5}_{#6} & {#5}_{#7} & {#5}_{#8} & {#5}_{#9} & {#5}_{#10}
%\end{vmatrix}
%}

% R3 vector.
\newcommand{\VectorThree}[3]{
\begin{bmatrix}
 {#1} \\
 {#2} \\
 {#3}
\end{bmatrix}
}



\author{Peeter Joot}
\email{peeter.joot@gmail.com}


\title{}
%\chapter{template}
%\label{chap:template}
%\useCCL
%\blogpage{http://sites.google.com/site/peeterjoot/math2009/template.pdf}
%\date{Nov 11, 2009}
%\revisionInfo{\(RCSfile: ellipticQ.tex,v \) Last \(Revision: 1.2 \) \(Date: 2009/12/03 03:24:40 \)}

%\beginArtWithToc
\beginArtNoToc

\section{Motivation}

re: 
\href{http://arxiv.org/abs/0711.4064}{A primer on elliptic functions with applications in classical mechanics}.

\section{Q}

In your 'A primer on elliptic functions with applications in classical mechanics' is the following

"Elliptic functions \(y(x; \Ba)\) are defined as solutions of the nonlinear ordinary differential equation

\begin{equation}\label{eqn:ellipticQ:20}
\begin{aligned}
\left(\frac{dy}{dx}\right)^2 = a_4 y^4 + a_3 y^3 + a_2 y^2 + a_1 y + a_0,
\end{aligned}
\end{equation}

\begin{equation}\label{eqn:ellipticQ:40}
\begin{aligned}
x(y; \Ba) = x_0 \pm \int_{y_0(\Ba)}^y \frac{ds}{\sqrt{
a_4 s^4 + a_3 s^3 + a_2 s^2 + a_1 s + a_0
}},
\end{aligned}
\end{equation}

where \(y_0(a)\) is a root of the quartic polynomial \(a_4 y^4 + a_3 y^3 + a_2 y^2 + a_1 y + a_0\), and \(x(y_0; a) = x_0\)."

If you do not mind can you give me a hint why you would pick \(y_0\) as the lower bound in the integral when there is a zero in the denominator there?  For example if I pick a similar simpler system

\begin{equation}\label{eqn:ellipticQ:60}
\begin{aligned}
\left(\frac{dy}{dx}\right)^2 = y + 1
\end{aligned}
\end{equation}

and write for some constant \(a\), the following appears to be a solution:

\begin{equation}\label{eqn:ellipticQ:80}
\begin{aligned}
x - c = \pm \int_{a}^y \frac{ds}{\sqrt{s + 1}}
\end{aligned}
\end{equation}

differentiating both sides with respect to x, and using \(\PDi{s}{F} \equiv 1/\sqrt{1 + s}\), I get

\begin{equation}\label{eqn:ellipticQ:100}
\begin{aligned}
1 
&= \pm \frac{d}{dx} \int_{a}^y \frac{ds}{\sqrt{s + 1}} \\
&= \pm \frac{d}{dx} \int_{a}^y \PD{s}{F} ds \\
&= \pm \frac{d}{dx} \left(F(y) - F(a)\right) \\
&= \pm \PD{y}{F(y)} \frac{dy}{dx} \\
&= \pm \inv{\sqrt{y + 1}} \frac{dy}{dx} \\
\end{aligned}
\end{equation}

Squaring and rearranging, independent of \(y_0\) (unless there is a subtlety that I am missing) one recovers the original DE.  It seems to me that you would want the integration range \([y_0,y]\) to not contain any roots of the polynomial (as opposed to starting at one of the roots).

\section{A}

Look at the integral

\begin{equation}\label{eqn:ellipticQ:120}
\begin{aligned}
t(x) = \int_{-a}^x ds \inv{\sqrt{a^2 - s^2}}
\end{aligned}
\end{equation}

where the "initial" condition is \(t(-a) = 0\). This integral is easily
integrated to give

\begin{equation}\label{eqn:ellipticQ:140}
\begin{aligned}
t(x) = \sin^{-1}(x/a) + \pi/2
\end{aligned}
\end{equation}

which can be easily inverted to give

\begin{equation}\label{eqn:ellipticQ:160}
\begin{aligned}
x(t) = a \sin(t - \pi/2) = - a \cos(t)
\end{aligned}
\end{equation}

Note that this solution satisfies the initial condition \(x(0) = -a\) (as it must!).

Problems associated with motion in one-dimensional potentials \(U(x)\) can always be integrated "by quadrature" in the form of the expression

\begin{equation}\label{eqn:ellipticQ:180}
\begin{aligned}
t(x) = \int_{x_0}^x ds \inv{\sqrt{(2/m) (E - U(s))}}
\end{aligned}
\end{equation}

where \(t(x_{0}) = 0\) is the initial condition. This initial condition is often chosen as a turning point, i.e., \(E = U(x_{0})\), without jeopardizing the integrability of the integral \(t(x)\). Not all potentials can lead to an analytical expression for \(t(x)\), so that the inversion \(t(x) \rightarrow x(t)\) may be possible only numerically.

\section{my remark}

When the integral can be evaluated, despite the kernel having a pole, one can (at least in some cases) let the evaluation point tend to the zero of the polynomial in the denominator and still be well defined.  Choosing the lower limit to be the zero does not look like a requirement, but instead a convenience.  I need to think about the inversion a bit more, but have the rough idea now.  

%\EndArticle
\EndNoBibArticle
