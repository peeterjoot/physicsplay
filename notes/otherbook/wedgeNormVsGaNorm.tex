%
% Copyright � 2012 Peeter Joot.  All Rights Reserved.
% Licenced as described in the file LICENSE under the root directory of this GIT repository.
%

% 
% 
\chapter{Wedge product norm and GA bivector norm comparison}
\label{chap:wedgeNormVsGaNorm}
\date{Jan 2008 or so.  $RCSfile: wedgeNormVsGaNorm.tex,v $ Last $Revision: 1.9 $ $Date: 2009/08/20 02:24:45 $ }

\section{Motivation}

Informative to look at the bivector formed by two unit perpendicular vectors in a plane.

\[
\ucap \wedge
\frac{
\left(\Bv - \bdotprod{\ucap}{\Bv} \ucap \right)
}
{
\left(\frac{1}{\norm{\Bu}^2} \sum_{i<j} {\left(D_{ij}^{\Bu \Bv}\right)}^2\right)^{1/2}
}
\]
\[
=
\frac{\Bu \wedge \Bv}
{
{\left(\sum_{i<j} {\left(D_{ij}^{\Bu \Bv}\right)}^2\right)}^{1/2}
}
\]
\[
=
\frac{\Bu \wedge \Bv}
{
\norm{\Bu \wedge \Bv}
}
\]

Here, $\norm{ bivector }$ is taken with the most obvious definition.  For an arbitrary bivector we define:

\[
\norm{\sum_{i<j}g_{ij} \ecap_i \wedge \ecap_j}^2
=
\sum_{i<j}{g_{ij}}^2
\]

Let us see how this compares with the GA norm (square) of a bivector.
\begin{align*}
(\Bu \wedge \Bv)^2 &= (\Bu \Bv - \dotprod{\Bu}{\Bv})^2 \\
                   &= (\Bu \Bv - \dotprod{\Bu}{\Bv})(\Bu \Bv - \dotprod{\Bu}{\Bv}) \\
                   &= \Bu \Bv \Bu \Bv - 2 (\dotprod{\Bu}{\Bv}) \Bu \Bv + {(\dotprod{\Bu}{\Bv})}^2
\end{align*}

Since $\dotprod{\Bu}{\Bv} = (\Bu \Bv + \Bv \Bu)/2$, then
$\Bv \Bu = 2 \dotprod{\Bu}{\Bv} - \Bu \Bv$.

\begin{align*}
(\Bu \wedge \Bv)^2 &= \Bu ( 2 \dotprod{\Bu}{\Bv} - \Bu \Bv ) \Bv - 2 (\dotprod{\Bu}{\Bv})\Bu \Bv + {(\dotprod{\Bu}{\Bv})}^2) \\
                   &= - \Bu \Bu \Bv \Bv + {(\dotprod{\Bu}{\Bv})}^2 \\
                   &= {(\dotprod{\Bu}{\Bv})}^2 - \norm{\Bu}^2 \norm{\Bv}^2 \\
                   &= - \sum_{i<j} {\left(D_{ij}^{\Bu \Bv}\right)}^2 \\
                   &= - \norm{\Bu \wedge \Bv}^2
\end{align*}

So, we can also define the norm of a bivector in terms of its GA product:
\[
\norm{\Bu \wedge \Bv}^2 = -(\Bu \wedge \Bv)^2
\]

And, with implications to Rotors, we can see that the role of the GA product $\BI = \ecap_i \ecap_j, i \neq j$ can also be filled by the wedge product of any two perpendicular unit vectors in a plane, or equivalently any unit bivector.  So, we have $-1$ as the GA magnitude (square) of any unit bivector:

\[
\left(\frac{\Bu \wedge \Bv}
{
\norm{\Bu \wedge \Bv}
}\right)^2 = -1
\]

\subsection{Perpendicular to vector in direction of second expressed as GA product}

Starting with the component of the vector $\Bv$ that is perpendicular to $\Bu$, we have:

\[
\Bv' = \left(\Bv - \bdotprod{\ucap}{\Bv} \ucap \right)
= {\frac{1}{\norm{\Bu}^2}}\left(\norm{\Bu}^2 \Bv - \Bu \bdotprod{\Bu}{\Bv}  \right)
\]

Using the GA product $\Bu^2 = \norm{\Bu}^2$, and $\Bu \wedge \Bv = \Bu \Bv - \dotprod{\Bu}{\Bv}$,

\[
\Rightarrow
\Bv' = \frac{\Bu}{\Bu^2} \left(\Bu \Bv - \bdotprod{\Bu}{\Bv} \right)
=
\frac{1}{\Bu} \left(\Bu \wedge \Bv \right)
=
\ucap \left(\ucap \wedge \Bv \right) 
\]

Thus we can write the decomposition of the vector $\Bv$ into components parallel and perpendicular to $\Bu$ as:

\[
\Bv = \ucap \left(\dotprod{\ucap}{\Bv}\right) + \ucap \left(\ucap \wedge \Bv \right)
\]
Or,
\[
\Bv = \frac{1}{\Bu} \left(\dotprod{\Bu}{\Bv}\right) + \frac{1}{\Bu} \left(\Bu \wedge \Bv \right)
\]

Without the GA product formulation we did not have a way without messy determinant sums to formulate the perpendicular component of this vector.
