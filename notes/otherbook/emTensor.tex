%
% Copyright � 2012 Peeter Joot.  All Rights Reserved.
% Licenced as described in the file LICENSE under the root directory of this GIT repository.
%

%
%
%\documentclass{article}      % Specifies the document class

%\usepackage{amsmath}
\usepackage{mathpazo}

%
% shorthand for bold symbols, convenient for vectors and matrices
%
\newcommand{\Ba}[0]{\mathbf{a}}
\newcommand{\Bb}[0]{\mathbf{b}}
\newcommand{\Bc}[0]{\mathbf{c}}
\newcommand{\Bd}[0]{\mathbf{d}}
\newcommand{\Be}[0]{\mathbf{e}}
\newcommand{\Bf}[0]{\mathbf{f}}
\newcommand{\Bg}[0]{\mathbf{g}}
\newcommand{\Bh}[0]{\mathbf{h}}
\newcommand{\Bi}[0]{\mathbf{i}}
\newcommand{\Bj}[0]{\mathbf{j}}
\newcommand{\Bk}[0]{\mathbf{k}}
\newcommand{\Bl}[0]{\mathbf{l}}
\newcommand{\Bm}[0]{\mathbf{m}}
\newcommand{\Bn}[0]{\mathbf{n}}
\newcommand{\Bo}[0]{\mathbf{o}}
\newcommand{\Bp}[0]{\mathbf{p}}
\newcommand{\Bq}[0]{\mathbf{q}}
\newcommand{\Br}[0]{\mathbf{r}}
\newcommand{\Bs}[0]{\mathbf{s}}
\newcommand{\Bt}[0]{\mathbf{t}}
\newcommand{\Bu}[0]{\mathbf{u}}
\newcommand{\Bv}[0]{\mathbf{v}}
\newcommand{\Bw}[0]{\mathbf{w}}
\newcommand{\Bx}[0]{\mathbf{x}}
\newcommand{\By}[0]{\mathbf{y}}
\newcommand{\Bz}[0]{\mathbf{z}}
\newcommand{\BA}[0]{\mathbf{A}}
\newcommand{\BB}[0]{\mathbf{B}}
\newcommand{\BC}[0]{\mathbf{C}}
\newcommand{\BD}[0]{\mathbf{D}}
\newcommand{\BE}[0]{\mathbf{E}}
\newcommand{\BF}[0]{\mathbf{F}}
\newcommand{\BG}[0]{\mathbf{G}}
\newcommand{\BH}[0]{\mathbf{H}}
\newcommand{\BI}[0]{\mathbf{I}}
\newcommand{\BJ}[0]{\mathbf{J}}
\newcommand{\BK}[0]{\mathbf{K}}
\newcommand{\BL}[0]{\mathbf{L}}
\newcommand{\BM}[0]{\mathbf{M}}
\newcommand{\BN}[0]{\mathbf{N}}
\newcommand{\BO}[0]{\mathbf{O}}
\newcommand{\BP}[0]{\mathbf{P}}
\newcommand{\BQ}[0]{\mathbf{Q}}
\newcommand{\BR}[0]{\mathbf{R}}
\newcommand{\BS}[0]{\mathbf{S}}
\newcommand{\BT}[0]{\mathbf{T}}
\newcommand{\BU}[0]{\mathbf{U}}
\newcommand{\BV}[0]{\mathbf{V}}
\newcommand{\BW}[0]{\mathbf{W}}
\newcommand{\BX}[0]{\mathbf{X}}
\newcommand{\BY}[0]{\mathbf{Y}}
\newcommand{\BZ}[0]{\mathbf{Z}}

\newcommand{\Bzero}[0]{\mathbf{0}}
\newcommand{\Btheta}[0]{\boldsymbol{\theta}}
\newcommand{\Btau}[0]{\boldsymbol{\tau}}
\newcommand{\Bomega}[0]{\boldsymbol{\omega}}

%
% shorthand for unit vectors
%
\newcommand{\acap}[0]{\hat{\Ba}}
\newcommand{\bcap}[0]{\hat{\Bb}}
\newcommand{\ccap}[0]{\hat{\Bc}}
\newcommand{\dcap}[0]{\hat{\Bd}}
\newcommand{\ecap}[0]{\hat{\Be}}
\newcommand{\fcap}[0]{\hat{\Bf}}
\newcommand{\gcap}[0]{\hat{\Bg}}
\newcommand{\hcap}[0]{\hat{\Bh}}
\newcommand{\icap}[0]{\hat{\Bi}}
\newcommand{\jcap}[0]{\hat{\Bj}}
\newcommand{\kcap}[0]{\hat{\Bk}}
\newcommand{\lcap}[0]{\hat{\Bl}}
\newcommand{\mcap}[0]{\hat{\Bm}}
\newcommand{\ncap}[0]{\hat{\Bn}}
\newcommand{\ocap}[0]{\hat{\Bo}}
\newcommand{\pcap}[0]{\hat{\Bp}}
\newcommand{\qcap}[0]{\hat{\Bq}}
\newcommand{\rcap}[0]{\hat{\Br}}
\newcommand{\scap}[0]{\hat{\Bs}}
\newcommand{\tcap}[0]{\hat{\Bt}}
\newcommand{\ucap}[0]{\hat{\Bu}}
\newcommand{\vcap}[0]{\hat{\Bv}}
\newcommand{\wcap}[0]{\hat{\Bw}}
\newcommand{\xcap}[0]{\hat{\Bx}}
\newcommand{\ycap}[0]{\hat{\By}}
\newcommand{\zcap}[0]{\hat{\Bz}}
\newcommand{\thetacap}[0]{\hat{\Btheta}}

%
% to write R^n and C^n in a distinguishable fashion.  Perhaps change this
% to the double lined characters upon figuring out how to do so.
%
\newcommand{\C}[1]{$\mathbb{C}^{#1}$}
\newcommand{\R}[1]{$\mathbb{R}^{#1}$}

%
% various generally useful helpers
%

% derivative of #1 wrt. #2:
\newcommand{\D}[2] {\frac {d#2} {d#1}}

\newcommand{\inv}[1]{\frac{1}{#1}}
\newcommand{\cross}[0]{\times}

\newcommand{\abs}[1]{\lvert{#1}\rvert}
\newcommand{\norm}[1]{\lVert{#1}\rVert}
\newcommand{\innerprod}[2]{\langle{#1}, {#2}\rangle}
\newcommand{\dotprod}[2]{{#1} \cdot {#2}}
\newcommand{\bdotprod}[2]{\left({#1} \cdot {#2}\right)}
\newcommand{\crossprod}[2]{{#1} \cross {#2}}
\newcommand{\tripleprod}[3]{\dotprod{\left(\crossprod{#1}{#2}\right)}{#3}}

\DeclareMathOperator{\Proj}{Proj}
\DeclareMathOperator{\Span}{span}
\DeclareMathOperator{\Sgn}{sgn}
\DeclareMathOperator{\Area}{Area}
\DeclareMathOperator{\Volume}{Volume}

%
% A few miscellaneous things specific to this document
%
\newcommand{\crossop}[1]{\crossprod{#1}{}}

% R2 vector.
\newcommand{\VectorTwo}[2]{
\begin{bmatrix}
 {#1} \\
 {#2}
\end{bmatrix}
}

\newcommand{\VectorN}[1]{
\begin{bmatrix}
{#1}_1 \\
{#1}_2 \\
\vdots \\
{#1}_N \\
\end{bmatrix}
}

\newcommand{\DETuvij}[4]{
\begin{vmatrix}
 {#1}_{#3} & {#1}_{#4} \\
 {#2}_{#3} & {#2}_{#4}
\end{vmatrix}
}

\newcommand{\DETuvwijk}[6]{
\begin{vmatrix}
 {#1}_{#4} & {#1}_{#5} & {#1}_{#6} \\
 {#2}_{#4} & {#2}_{#5} & {#2}_{#6} \\
 {#3}_{#4} & {#3}_{#5} & {#3}_{#6}
\end{vmatrix}
}

\newcommand{\DETuvwxijkl}[8]{
\begin{vmatrix}
 {#1}_{#5} & {#1}_{#6} & {#1}_{#7} & {#1}_{#8} \\
 {#2}_{#5} & {#2}_{#6} & {#2}_{#7} & {#2}_{#8} \\
 {#3}_{#5} & {#3}_{#6} & {#3}_{#7} & {#3}_{#8} \\
 {#4}_{#5} & {#4}_{#6} & {#4}_{#7} & {#4}_{#8} \\
\end{vmatrix}
}

%\newcommand{\DETuvwxyijklm}[10]{
%\begin{vmatrix}
% {#1}_{#6} & {#1}_{#7} & {#1}_{#8} & {#1}_{#9} & {#1}_{#10} \\
% {#2}_{#6} & {#2}_{#7} & {#2}_{#8} & {#2}_{#9} & {#2}_{#10} \\
% {#3}_{#6} & {#3}_{#7} & {#3}_{#8} & {#3}_{#9} & {#3}_{#10} \\
% {#4}_{#6} & {#4}_{#7} & {#4}_{#8} & {#4}_{#9} & {#4}_{#10} \\
% {#5}_{#6} & {#5}_{#7} & {#5}_{#8} & {#5}_{#9} & {#5}_{#10}
%\end{vmatrix}
%}

% R3 vector.
\newcommand{\VectorThree}[3]{
\begin{bmatrix}
 {#1} \\
 {#2} \\
 {#3}
\end{bmatrix}
}



%
% The real thing:
%

\chapter{Geometric Algebra: Signs of electromagnetic field tensor components?}
\label{chap:emTensor}
%\author{Peeter Joot \quad peeterjoot@protonmail.com}         % Declares the author's name.
\date{ May 30, 2008.  \(RCSfile: emTensor.tex,v \) Last \(Revision: 1.9 \) \(Date: 2009/06/14 23:51:45 \) }

%\begin{document}             % End of preamble and beginning of text.

%\maketitle{}

\section{}

Here is a question that may look like an E \& M question, but is really just a geometric algebra question.  In particular, I have got a sign off by 1 somewhere I think and I wonder if somebody can spot it.

Doran/Lasenby (Geometric Algebra for Physicists) writes for the electromagnetic field:

\begin{equation}\label{eqn:emTensor:20}
F = \mathbf{E} + I\mathbf{B}
\end{equation}

\begin{equation}\label{eqn:emTensor:40}
\mathbf{E} = \sum \sigma_i E_i
\end{equation}

\begin{equation}\label{eqn:emTensor:60}
\mathbf{B} = \sum \sigma_i B_i
\end{equation}

\begin{equation}\label{eqn:emTensor:80}
\sigma_i = \gamma_i \gamma_0
\end{equation}

\begin{equation}\label{eqn:emTensor:100}
I = \gamma_0 \gamma_1 \gamma_2 \gamma_3
\end{equation}

Here, the \(\{\gamma_\mu\}\) vectors are an ``orthonormal'' basis, with respect to a \((+,-,-,-)\) metric dot product.

They point out that the coordinates of the bivector F can be calculated by taking dot products with the reciprocal frame vectors:

\begin{equation}\label{eqn:emTensor:120}
F^{\mu\nu} = (\gamma^\nu \wedge \gamma^\mu) \cdot F
\end{equation}

Where the reciprocal frame vectors \(\{\gamma^i\}\) are those vectors defined by:

\begin{equation}\label{eqn:emTensor:140}
\gamma^\mu \cdot \gamma_\nu = \delta_{\mu\nu}
\end{equation}

(\(\gamma^0 = \gamma_0\), and \(\gamma^i = -\gamma_i\)).

Expressed as a matrix this bivector coordinates are the tensor:

\begin{equation}\label{eqn:emTensor:160}
F^{\mu\nu} =
\begin{bmatrix}
0 & -E_1 & -E_2 & -E_3 \\
E_1 & 0 & -B_3 & B_2 \\
E_2 & B_3 & 0 & -B_1 \\
E_3 & -B_2 & B_1 & 0 \\
\end{bmatrix}
\end{equation}

They do not actually do anything with this tensor in the text, since they operate on the bivector form directly.  It is essentially written out for in matrix form for comparison to other relativistic electrodynamics texts.

If I try calculating this I get different signs for the \(B_i\) terms in the matrix above.  The
calculation of the \(E\) components is straightforward and I get the same answer.  \(\mathbf{E}\) explicitly is:

\begin{equation}\label{eqn:emTensor:180}
\mathbf{E} = E_1 \gamma_{10} + E_2 \gamma_{20} + E_3 \gamma_{30}
\end{equation}

Calculation of the \(\nu = 0\) terms is:

\begin{equation}\label{eqn:emTensor:200}
\begin{aligned}
F^{\mu 0}
&= (\gamma^0 \wedge \gamma^\mu) \cdot F \\
&= (\gamma_0 \wedge (-\gamma_\mu)) \cdot \sum E_i \gamma_{i0} \\
&= -E_\mu (\gamma_0 \wedge \gamma_\mu) \cdot (\gamma_\mu \wedge \gamma_{0}) \\
&= -E_\mu \gamma_0 (\gamma_\mu \cdot (\gamma_\mu \wedge \gamma_{0})) \\
&= -E_\mu \gamma_0 ( \mathLabelBox{\gamma_\mu \cdot \gamma_\mu}{\(=-1\)} \gamma_{0} - \mathLabelBox{\gamma_\mu \cdot \gamma_{0}}{\(=0\)} \gamma_\mu) \\
&= E_\mu \gamma_0 \cdot \gamma_0 \\
&= E_\mu \\
\end{aligned}
\end{equation}

This is consistent with column zero of their matrix of tensor components above.

For the \(B\) components, first I expanded out \(I\mathbf{B}\) explicitly:

\begin{equation}\label{eqn:emTensor:220}
\begin{aligned}
I\mathbf{B}
&= \sum \gamma_0 \gamma_1 \gamma_2 \gamma_3 \gamma_i \gamma_0 B_i \\
&= \sum \gamma_1 \gamma_2 \gamma_3 \gamma_i B_i \\
&= B_1 \gamma_{23} + B_2 \gamma_{31} + B_3 \gamma_{12}
\end{aligned}
\end{equation}

This calculation is easier for just one pair of index, and for example, calculating the \(12\) component I get:

\begin{equation}\label{eqn:emTensor:240}
\begin{aligned}
F^{12}
&= (\gamma^2 \wedge \gamma^1) \cdot ( \gamma_1 \wedge \gamma_2 ) B_3 \\
&= \gamma^2 \cdot (\gamma^1 \cdot ( \gamma_1 \wedge \gamma_2 )) B_3 \\
&= \gamma^2 \cdot ( \gamma^1 \cdot \gamma_1 \gamma_2 - \gamma^1 \cdot \gamma_2 \gamma_1 ) B_3 \\
&= \gamma^2 \cdot ( -\gamma_1 \cdot \gamma_1 \gamma_2 ) B_3 \\
&= \gamma^2 \cdot \gamma_2 B_3 \\
&= - \gamma_2 \cdot \gamma_2 B_3 \\
&= - (-1) B_3 \\
&= B_3 \\
\end{aligned}
\end{equation}

Observe that the sign is opposite for this compared to what is in the matrix above.  I do not see a mistake in my calculation, but this is not listed in the errata even after two editions
so I am assuming I have one hiding in there somewhere.

%\end{document}               % End of document.
