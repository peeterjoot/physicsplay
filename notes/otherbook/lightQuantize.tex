\documentclass[]{eliblog}

\usepackage{amsmath}
\usepackage{mathpazo}

%
% shorthand for bold symbols, convenient for vectors and matrices
%
\newcommand{\Ba}[0]{\mathbf{a}}
\newcommand{\Bb}[0]{\mathbf{b}}
\newcommand{\Bc}[0]{\mathbf{c}}
\newcommand{\Bd}[0]{\mathbf{d}}
\newcommand{\Be}[0]{\mathbf{e}}
\newcommand{\Bf}[0]{\mathbf{f}}
\newcommand{\Bg}[0]{\mathbf{g}}
\newcommand{\Bh}[0]{\mathbf{h}}
\newcommand{\Bi}[0]{\mathbf{i}}
\newcommand{\Bj}[0]{\mathbf{j}}
\newcommand{\Bk}[0]{\mathbf{k}}
\newcommand{\Bl}[0]{\mathbf{l}}
\newcommand{\Bm}[0]{\mathbf{m}}
\newcommand{\Bn}[0]{\mathbf{n}}
\newcommand{\Bo}[0]{\mathbf{o}}
\newcommand{\Bp}[0]{\mathbf{p}}
\newcommand{\Bq}[0]{\mathbf{q}}
\newcommand{\Br}[0]{\mathbf{r}}
\newcommand{\Bs}[0]{\mathbf{s}}
\newcommand{\Bt}[0]{\mathbf{t}}
\newcommand{\Bu}[0]{\mathbf{u}}
\newcommand{\Bv}[0]{\mathbf{v}}
\newcommand{\Bw}[0]{\mathbf{w}}
\newcommand{\Bx}[0]{\mathbf{x}}
\newcommand{\By}[0]{\mathbf{y}}
\newcommand{\Bz}[0]{\mathbf{z}}
\newcommand{\BA}[0]{\mathbf{A}}
\newcommand{\BB}[0]{\mathbf{B}}
\newcommand{\BC}[0]{\mathbf{C}}
\newcommand{\BD}[0]{\mathbf{D}}
\newcommand{\BE}[0]{\mathbf{E}}
\newcommand{\BF}[0]{\mathbf{F}}
\newcommand{\BG}[0]{\mathbf{G}}
\newcommand{\BH}[0]{\mathbf{H}}
\newcommand{\BI}[0]{\mathbf{I}}
\newcommand{\BJ}[0]{\mathbf{J}}
\newcommand{\BK}[0]{\mathbf{K}}
\newcommand{\BL}[0]{\mathbf{L}}
\newcommand{\BM}[0]{\mathbf{M}}
\newcommand{\BN}[0]{\mathbf{N}}
\newcommand{\BO}[0]{\mathbf{O}}
\newcommand{\BP}[0]{\mathbf{P}}
\newcommand{\BQ}[0]{\mathbf{Q}}
\newcommand{\BR}[0]{\mathbf{R}}
\newcommand{\BS}[0]{\mathbf{S}}
\newcommand{\BT}[0]{\mathbf{T}}
\newcommand{\BU}[0]{\mathbf{U}}
\newcommand{\BV}[0]{\mathbf{V}}
\newcommand{\BW}[0]{\mathbf{W}}
\newcommand{\BX}[0]{\mathbf{X}}
\newcommand{\BY}[0]{\mathbf{Y}}
\newcommand{\BZ}[0]{\mathbf{Z}}

\newcommand{\Bzero}[0]{\mathbf{0}}
\newcommand{\Btheta}[0]{\boldsymbol{\theta}}
\newcommand{\Btau}[0]{\boldsymbol{\tau}}
\newcommand{\Bomega}[0]{\boldsymbol{\omega}}

%
% shorthand for unit vectors
%
\newcommand{\acap}[0]{\hat{\Ba}}
\newcommand{\bcap}[0]{\hat{\Bb}}
\newcommand{\ccap}[0]{\hat{\Bc}}
\newcommand{\dcap}[0]{\hat{\Bd}}
\newcommand{\ecap}[0]{\hat{\Be}}
\newcommand{\fcap}[0]{\hat{\Bf}}
\newcommand{\gcap}[0]{\hat{\Bg}}
\newcommand{\hcap}[0]{\hat{\Bh}}
\newcommand{\icap}[0]{\hat{\Bi}}
\newcommand{\jcap}[0]{\hat{\Bj}}
\newcommand{\kcap}[0]{\hat{\Bk}}
\newcommand{\lcap}[0]{\hat{\Bl}}
\newcommand{\mcap}[0]{\hat{\Bm}}
\newcommand{\ncap}[0]{\hat{\Bn}}
\newcommand{\ocap}[0]{\hat{\Bo}}
\newcommand{\pcap}[0]{\hat{\Bp}}
\newcommand{\qcap}[0]{\hat{\Bq}}
\newcommand{\rcap}[0]{\hat{\Br}}
\newcommand{\scap}[0]{\hat{\Bs}}
\newcommand{\tcap}[0]{\hat{\Bt}}
\newcommand{\ucap}[0]{\hat{\Bu}}
\newcommand{\vcap}[0]{\hat{\Bv}}
\newcommand{\wcap}[0]{\hat{\Bw}}
\newcommand{\xcap}[0]{\hat{\Bx}}
\newcommand{\ycap}[0]{\hat{\By}}
\newcommand{\zcap}[0]{\hat{\Bz}}
\newcommand{\thetacap}[0]{\hat{\Btheta}}

%
% to write R^n and C^n in a distinguishable fashion.  Perhaps change this
% to the double lined characters upon figuring out how to do so.
%
\newcommand{\C}[1]{$\mathbb{C}^{#1}$}
\newcommand{\R}[1]{$\mathbb{R}^{#1}$}

%
% various generally useful helpers
%

% derivative of #1 wrt. #2:
\newcommand{\D}[2] {\frac {d#2} {d#1}}

\newcommand{\inv}[1]{\frac{1}{#1}}
\newcommand{\cross}[0]{\times}

\newcommand{\abs}[1]{\lvert{#1}\rvert}
\newcommand{\norm}[1]{\lVert{#1}\rVert}
\newcommand{\innerprod}[2]{\langle{#1}, {#2}\rangle}
\newcommand{\dotprod}[2]{{#1} \cdot {#2}}
\newcommand{\bdotprod}[2]{\left({#1} \cdot {#2}\right)}
\newcommand{\crossprod}[2]{{#1} \cross {#2}}
\newcommand{\tripleprod}[3]{\dotprod{\left(\crossprod{#1}{#2}\right)}{#3}}

\DeclareMathOperator{\Proj}{Proj}
\DeclareMathOperator{\Span}{span}
\DeclareMathOperator{\Sgn}{sgn}
\DeclareMathOperator{\Area}{Area}
\DeclareMathOperator{\Volume}{Volume}

%
% A few miscellaneous things specific to this document
%
\newcommand{\crossop}[1]{\crossprod{#1}{}}

% R2 vector.
\newcommand{\VectorTwo}[2]{
\begin{bmatrix}
 {#1} \\
 {#2}
\end{bmatrix}
}

\newcommand{\VectorN}[1]{
\begin{bmatrix}
{#1}_1 \\
{#1}_2 \\
\vdots \\
{#1}_N \\
\end{bmatrix}
}

\newcommand{\DETuvij}[4]{
\begin{vmatrix}
 {#1}_{#3} & {#1}_{#4} \\
 {#2}_{#3} & {#2}_{#4}
\end{vmatrix}
}

\newcommand{\DETuvwijk}[6]{
\begin{vmatrix}
 {#1}_{#4} & {#1}_{#5} & {#1}_{#6} \\
 {#2}_{#4} & {#2}_{#5} & {#2}_{#6} \\
 {#3}_{#4} & {#3}_{#5} & {#3}_{#6}
\end{vmatrix}
}

\newcommand{\DETuvwxijkl}[8]{
\begin{vmatrix}
 {#1}_{#5} & {#1}_{#6} & {#1}_{#7} & {#1}_{#8} \\
 {#2}_{#5} & {#2}_{#6} & {#2}_{#7} & {#2}_{#8} \\
 {#3}_{#5} & {#3}_{#6} & {#3}_{#7} & {#3}_{#8} \\
 {#4}_{#5} & {#4}_{#6} & {#4}_{#7} & {#4}_{#8} \\
\end{vmatrix}
}

%\newcommand{\DETuvwxyijklm}[10]{
%\begin{vmatrix}
% {#1}_{#6} & {#1}_{#7} & {#1}_{#8} & {#1}_{#9} & {#1}_{#10} \\
% {#2}_{#6} & {#2}_{#7} & {#2}_{#8} & {#2}_{#9} & {#2}_{#10} \\
% {#3}_{#6} & {#3}_{#7} & {#3}_{#8} & {#3}_{#9} & {#3}_{#10} \\
% {#4}_{#6} & {#4}_{#7} & {#4}_{#8} & {#4}_{#9} & {#4}_{#10} \\
% {#5}_{#6} & {#5}_{#7} & {#5}_{#8} & {#5}_{#9} & {#5}_{#10}
%\end{vmatrix}
%}

% R3 vector.
\newcommand{\VectorThree}[3]{
\begin{bmatrix}
 {#1} \\
 {#2} \\
 {#3}
\end{bmatrix}
}


%<misc>
%
\newcommand{\Abs}[1]{{\left\lvert{#1}\right\rvert}}
\newcommand{\spacegrad}[0]{\boldsymbol{\nabla}}
\newcommand{\grad}[0]{\nabla}
\newcommand{\LL}[0]{\mathcal{L}}

% == \partial_{#1} {#2}
\newcommand{\PD}[2]{\frac{\partial {#2}}{\partial {#1}}}
% inline variant
\newcommand{\PDi}[2]{{\partial {#2}}/{\partial {#1}}}

\newcommand{\PDD}[3]{\frac{\partial^2 {#3}}{\partial {#1}\partial {#2}}}
%\newcommand{\PDd}[2]{\frac{\partial^2 {#2}}{{\partial{#1}}^2}}
\newcommand{\PDsq}[2]{\frac{\partial^2 {#2}}{(\partial {#1})^2}}

\newcommand{\Partial}[2]{\frac{\partial {#1}}{\partial {#2}}}
\DeclareMathOperator{\RejName}{Rej}
\newcommand{\Rej}[2]{\RejName_{#1}\left( {#2} \right)}
\newcommand{\Rm}[1]{\mathbb{R}^{#1}}
\newcommand{\Cm}[1]{\mathbb{C}^{#1}}
\newcommand{\conj}[0]{{*}}

%</misc>

% <grade selection>
%
\newcommand{\gpgrade}[2] {{\left\langle{{#1}}\right\rangle}_{#2}}

\newcommand{\gpgradezero}[1] {\gpgrade{#1}{}}
%\newcommand{\gpscalargrade}[1] {{\left\langle{{#1}}\right\rangle}}
%\newcommand{\gpgradezero}[1] {\gpgrade{#1}{0}}

%\newcommand{\gpgradeone}[1] {{\left\langle{{#1}}\right\rangle}_{1}}
\newcommand{\gpgradeone}[1] {\gpgrade{#1}{1}}

\newcommand{\gpgradetwo}[1] {\gpgrade{#1}{2}}
\newcommand{\gpgradethree}[1] {\gpgrade{#1}{3}}
\newcommand{\gpgradefour}[1] {\gpgrade{#1}{4}}
%
% </grade selection>



\newcommand{\adot}[0]{{\dot{a}}}
\newcommand{\bdot}[0]{{\dot{b}}}
% taken for centered dot:
%\newcommand{\cdot}[0]{{\dot{c}}}
%\newcommand{\ddot}[0]{{\dot{d}}}
\newcommand{\edot}[0]{{\dot{e}}}
\newcommand{\fdot}[0]{{\dot{f}}}
\newcommand{\gdot}[0]{{\dot{g}}}
\newcommand{\hdot}[0]{{\dot{h}}}
\newcommand{\idot}[0]{{\dot{i}}}
\newcommand{\jdot}[0]{{\dot{j}}}
\newcommand{\kdot}[0]{{\dot{k}}}
\newcommand{\ldot}[0]{{\dot{l}}}
\newcommand{\mdot}[0]{{\dot{m}}}
\newcommand{\ndot}[0]{{\dot{n}}}
%\newcommand{\odot}[0]{{\dot{o}}}
\newcommand{\pdot}[0]{{\dot{p}}}
\newcommand{\qdot}[0]{{\dot{q}}}
\newcommand{\rdot}[0]{{\dot{r}}}
\newcommand{\sdot}[0]{{\dot{s}}}
\newcommand{\tdot}[0]{{\dot{t}}}
\newcommand{\udot}[0]{{\dot{u}}}
\newcommand{\vdot}[0]{{\dot{v}}}
\newcommand{\wdot}[0]{{\dot{w}}}
\newcommand{\xdot}[0]{{\dot{x}}}
\newcommand{\ydot}[0]{{\dot{y}}}
\newcommand{\zdot}[0]{{\dot{z}}}
\newcommand{\addot}[0]{{\ddot{a}}}
\newcommand{\bddot}[0]{{\ddot{b}}}
\newcommand{\cddot}[0]{{\ddot{c}}}
%\newcommand{\dddot}[0]{{\ddot{d}}}
\newcommand{\eddot}[0]{{\ddot{e}}}
\newcommand{\fddot}[0]{{\ddot{f}}}
\newcommand{\gddot}[0]{{\ddot{g}}}
\newcommand{\hddot}[0]{{\ddot{h}}}
\newcommand{\iddot}[0]{{\ddot{i}}}
\newcommand{\jddot}[0]{{\ddot{j}}}
\newcommand{\kddot}[0]{{\ddot{k}}}
\newcommand{\lddot}[0]{{\ddot{l}}}
\newcommand{\mddot}[0]{{\ddot{m}}}
\newcommand{\nddot}[0]{{\ddot{n}}}
\newcommand{\oddot}[0]{{\ddot{o}}}
\newcommand{\pddot}[0]{{\ddot{p}}}
\newcommand{\qddot}[0]{{\ddot{q}}}
\newcommand{\rddot}[0]{{\ddot{r}}}
\newcommand{\sddot}[0]{{\ddot{s}}}
\newcommand{\tddot}[0]{{\ddot{t}}}
\newcommand{\uddot}[0]{{\ddot{u}}}
\newcommand{\vddot}[0]{{\ddot{v}}}
\newcommand{\wddot}[0]{{\ddot{w}}}
\newcommand{\xddot}[0]{{\ddot{x}}}
\newcommand{\yddot}[0]{{\ddot{y}}}
\newcommand{\zddot}[0]{{\ddot{z}}}

%<bold and dot greek symbols>
%

\newcommand{\Deltadot}[0]{{\dot{\Delta}}}
\newcommand{\Gammadot}[0]{{\dot{\Gamma}}}
\newcommand{\Lambdadot}[0]{{\dot{\Lambda}}}
\newcommand{\Omegadot}[0]{{\dot{\Omega}}}
\newcommand{\Phidot}[0]{{\dot{\Phi}}}
\newcommand{\Pidot}[0]{{\dot{\Pi}}}
\newcommand{\Psidot}[0]{{\dot{\Psi}}}
\newcommand{\Sigmadot}[0]{{\dot{\Sigma}}}
\newcommand{\Thetadot}[0]{{\dot{\Theta}}}
\newcommand{\Upsilondot}[0]{{\dot{\Upsilon}}}
\newcommand{\Xidot}[0]{{\dot{\Xi}}}
\newcommand{\alphadot}[0]{{\dot{\alpha}}}
\newcommand{\betadot}[0]{{\dot{\beta}}}
\newcommand{\chidot}[0]{{\dot{\chi}}}
\newcommand{\deltadot}[0]{{\dot{\delta}}}
\newcommand{\epsilondot}[0]{{\dot{\epsilon}}}
\newcommand{\etadot}[0]{{\dot{\eta}}}
\newcommand{\gammadot}[0]{{\dot{\gamma}}}
\newcommand{\kappadot}[0]{{\dot{\kappa}}}
\newcommand{\lambdadot}[0]{{\dot{\lambda}}}
\newcommand{\mudot}[0]{{\dot{\mu}}}
\newcommand{\nudot}[0]{{\dot{\nu}}}
\newcommand{\omegadot}[0]{{\dot{\omega}}}
\newcommand{\phidot}[0]{{\dot{\phi}}}
\newcommand{\pidot}[0]{{\dot{\pi}}}
\newcommand{\psidot}[0]{{\dot{\psi}}}
\newcommand{\rhodot}[0]{{\dot{\rho}}}
\newcommand{\sigmadot}[0]{{\dot{\sigma}}}
\newcommand{\taudot}[0]{{\dot{\tau}}}
\newcommand{\thetadot}[0]{{\dot{\theta}}}
\newcommand{\upsilondot}[0]{{\dot{\upsilon}}}
\newcommand{\varepsilondot}[0]{{\dot{\varepsilon}}}
\newcommand{\varphidot}[0]{{\dot{\varphi}}}
\newcommand{\varpidot}[0]{{\dot{\varpi}}}
\newcommand{\varrhodot}[0]{{\dot{\varrho}}}
\newcommand{\varsigmadot}[0]{{\dot{\varsigma}}}
\newcommand{\varthetadot}[0]{{\dot{\vartheta}}}
\newcommand{\xidot}[0]{{\dot{\xi}}}
\newcommand{\zetadot}[0]{{\dot{\zeta}}}

\newcommand{\Deltaddot}[0]{{\ddot{\Delta}}}
\newcommand{\Gammaddot}[0]{{\ddot{\Gamma}}}
\newcommand{\Lambdaddot}[0]{{\ddot{\Lambda}}}
\newcommand{\Omegaddot}[0]{{\ddot{\Omega}}}
\newcommand{\Phiddot}[0]{{\ddot{\Phi}}}
\newcommand{\Piddot}[0]{{\ddot{\Pi}}}
\newcommand{\Psiddot}[0]{{\ddot{\Psi}}}
\newcommand{\Sigmaddot}[0]{{\ddot{\Sigma}}}
\newcommand{\Thetaddot}[0]{{\ddot{\Theta}}}
\newcommand{\Upsilonddot}[0]{{\ddot{\Upsilon}}}
\newcommand{\Xiddot}[0]{{\ddot{\Xi}}}
\newcommand{\alphaddot}[0]{{\ddot{\alpha}}}
\newcommand{\betaddot}[0]{{\ddot{\beta}}}
\newcommand{\chiddot}[0]{{\ddot{\chi}}}
\newcommand{\deltaddot}[0]{{\ddot{\delta}}}
\newcommand{\epsilonddot}[0]{{\ddot{\epsilon}}}
\newcommand{\etaddot}[0]{{\ddot{\eta}}}
\newcommand{\gammaddot}[0]{{\ddot{\gamma}}}
\newcommand{\kappaddot}[0]{{\ddot{\kappa}}}
\newcommand{\lambdaddot}[0]{{\ddot{\lambda}}}
\newcommand{\muddot}[0]{{\ddot{\mu}}}
\newcommand{\nuddot}[0]{{\ddot{\nu}}}
\newcommand{\omegaddot}[0]{{\ddot{\omega}}}
\newcommand{\phiddot}[0]{{\ddot{\phi}}}
\newcommand{\piddot}[0]{{\ddot{\pi}}}
\newcommand{\psiddot}[0]{{\ddot{\psi}}}
\newcommand{\rhoddot}[0]{{\ddot{\rho}}}
\newcommand{\sigmaddot}[0]{{\ddot{\sigma}}}
\newcommand{\tauddot}[0]{{\ddot{\tau}}}
\newcommand{\thetaddot}[0]{{\ddot{\theta}}}
\newcommand{\upsilonddot}[0]{{\ddot{\upsilon}}}
\newcommand{\varepsilonddot}[0]{{\ddot{\varepsilon}}}
\newcommand{\varphiddot}[0]{{\ddot{\varphi}}}
\newcommand{\varpiddot}[0]{{\ddot{\varpi}}}
\newcommand{\varrhoddot}[0]{{\ddot{\varrho}}}
\newcommand{\varsigmaddot}[0]{{\ddot{\varsigma}}}
\newcommand{\varthetaddot}[0]{{\ddot{\vartheta}}}
\newcommand{\xiddot}[0]{{\ddot{\xi}}}
\newcommand{\zetaddot}[0]{{\ddot{\zeta}}}

\newcommand{\BDelta}[0]{\boldsymbol{\Delta}}
\newcommand{\BGamma}[0]{\boldsymbol{\Gamma}}
\newcommand{\BLambda}[0]{\boldsymbol{\Lambda}}
\newcommand{\BOmega}[0]{\boldsymbol{\Omega}}
\newcommand{\BPhi}[0]{\boldsymbol{\Phi}}
\newcommand{\BPi}[0]{\boldsymbol{\Pi}}
\newcommand{\BPsi}[0]{\boldsymbol{\Psi}}
\newcommand{\BSigma}[0]{\boldsymbol{\Sigma}}
\newcommand{\BTheta}[0]{\boldsymbol{\Theta}}
\newcommand{\BUpsilon}[0]{\boldsymbol{\Upsilon}}
\newcommand{\BXi}[0]{\boldsymbol{\Xi}}
\newcommand{\Balpha}[0]{\boldsymbol{\alpha}}
\newcommand{\Bbeta}[0]{\boldsymbol{\beta}}
\newcommand{\Bchi}[0]{\boldsymbol{\chi}}
\newcommand{\Bdelta}[0]{\boldsymbol{\delta}}
\newcommand{\Bepsilon}[0]{\boldsymbol{\epsilon}}
\newcommand{\Beta}[0]{\boldsymbol{\eta}}
\newcommand{\Bgamma}[0]{\boldsymbol{\gamma}}
\newcommand{\Bkappa}[0]{\boldsymbol{\kappa}}
\newcommand{\Blambda}[0]{\boldsymbol{\lambda}}
\newcommand{\Bmu}[0]{\boldsymbol{\mu}}
\newcommand{\Bnu}[0]{\boldsymbol{\nu}}
%\newcommand{\Bomega}[0]{\boldsymbol{\omega}}
\newcommand{\Bphi}[0]{\boldsymbol{\phi}}
\newcommand{\Bpi}[0]{\boldsymbol{\pi}}
\newcommand{\Bpsi}[0]{\boldsymbol{\psi}}
\newcommand{\Brho}[0]{\boldsymbol{\rho}}
\newcommand{\Bsigma}[0]{\boldsymbol{\sigma}}
%\newcommand{\Btau}[0]{\boldsymbol{\tau}}
%\newcommand{\Btheta}[0]{\boldsymbol{\theta}}
\newcommand{\Bupsilon}[0]{\boldsymbol{\upsilon}}
\newcommand{\Bvarepsilon}[0]{\boldsymbol{\varepsilon}}
\newcommand{\Bvarphi}[0]{\boldsymbol{\varphi}}
\newcommand{\Bvarpi}[0]{\boldsymbol{\varpi}}
\newcommand{\Bvarrho}[0]{\boldsymbol{\varrho}}
\newcommand{\Bvarsigma}[0]{\boldsymbol{\varsigma}}
\newcommand{\Bvartheta}[0]{\boldsymbol{\vartheta}}
\newcommand{\Bxi}[0]{\boldsymbol{\xi}}
\newcommand{\Bzeta}[0]{\boldsymbol{\zeta}}
%
%</bold and dot greek symbols>
%<infrequent>
%
%\newcommand{\AreaOp}[1]{\AName_{#1}}
%\newcommand{\Babs}[0]{\abs{\BB}}
%\newcommand{\Bcap}[0]{\hat{\BB}}
%\newcommand{\BrPrimeRej}[0]{\rcap(\rcap \wedge \Br')}
%\newcommand{\CA}[0]{\mathcal{A}}
%\newcommand{\Cos}[1]{\cos{\left({#1}\right)}}
%\newcommand{\Det}[1] {\abs{#1}}
%\newcommand{\Dsq}[2] {\frac {\partial^2 {#1}} {\partial {#2}^2}}
%\newcommand{\Exp}[1]{\exp{\left({#1}\right)}}
%\newcommand{\Norm}[1]{\left\lVert{#1}\right\rVert}
%\newcommand{\Sin}[1]{\sin{\left({#1}\right)}}
%\newcommand{\T}[0]{\text{T}}
%\newcommand{\VolumeOp}[1]{\VName_{#1}}
%\newcommand{\agrad}[0]{\Ba \cdot \nabla}
%\newcommand{\alphacap}[0]{\hat{\boldsymbol{\alpha}}}
%\newcommand{\Fcap}[0]{\hat{\BF}}
%\newcommand{\bithree}[0]{{\Bi}_3}
%\newcommand{\bxa}[0]{\Bx\Ba}
%\newcommand{\coordvec}[2]{
%\newcommand{\costheta}[0]{\acap \cdot \xcap}
%\newcommand{\ddt}[1]{\ddot{#1}}
%\newcommand{\ddu}[1] {\frac {d{#1}} {du}}
%\newcommand{\dsqxj}[2] {\frac {\partial^2 {#1}} {\partial {x_{#2}}^2}}
%\newcommand{\dtheta}[1]{\frac{d {#1}}{d \theta}}
%\newcommand{\dt}[1]{\dot{#1}}
%\newcommand{\dt}[1]{\frac{d {#1}}{dt}}
%\newcommand{\dxj}[2] {\frac {\partial {#1}} {\partial {x_{#2}}}}
%\newcommand{\halfPhi}[0]{\frac{\phi}{2}}
%\newcommand{\half}[0]{\inv{2}}
%\newcommand{\inv}[1]{\frac{1}{#1}}
%\newcommand{\laplacian}[0]{\nabla^2}
%\newcommand{\matrixoftx}[3]{
%\newcommand{\nrrp}[0]{\norm{\rcap \wedge \Br'}}
%\newcommand{\oiint}{\bigcirc \hspace{-1.4em} \int \hspace{-.8em} \int}
%\newcommand{\transpose}[1]{{#1}^{\text{T}}}
%\newcommand{\transpose}[1]{{{#1}^{\TextTranspose}}}
%\newcommand{\transpose}[1]{{{#1}^{\text{T}}}}
%\newcommand{\barA}[0]{\bar{A}}
%\newcommand{\qbar}[0]{\bar{q}}
%\newcommand{\qdotbar}[0]{\dot{\bar{q}}}
%
%</infrequent>




\newcommand{\symmetric}[2]{{\left\{{#1},{#2}\right\}}}
\newcommand{\antisymmetric}[2]{\left[{#1},{#2}\right]}
\DeclareMathOperator{\sgn}{sgn}
\DeclareMathOperator{\something}{something}

\newcommand{\uDETuvij}[4]{
\begin{vmatrix}
 {#1}^{#3} & {#1}^{#4} \\
 {#2}^{#3} & {#2}^{#4}
\end{vmatrix}
}

\newcommand{\PDSq}[2]{\frac{\partial^2 {#2}}{\partial {#1}^2}}
\newcommand{\transpose}[1]{{#1}^{\mathrm{T}}}
\newcommand{\stardot}[0]{{*}}

% bivector.tex:
\newcommand{\laplacian}[0]{\nabla^2}
\newcommand{\Dsq}[2] {\frac {\partial^2 {#1}} {\partial {#2}^2}}
\newcommand{\dxj}[2] {\frac {\partial {#1}} {\partial {x_{#2}}}}
\newcommand{\dsqxj}[2] {\frac {\partial^2 {#1}} {\partial {x_{#2}}^2}}
\DeclareMathOperator{\ExpName}{e}
%\DeclareMathOperator{\Exp}{e}
%\newcommand{\Exp}[1]{\exp{\left({#1}\right)}}
%\DeclareMathOperator{\Rej}{Rej}
\DeclareMathOperator{\Rot}{R}
%\newcommand{\gpgrade}[2] {{\left\langle{{#1}}\right\rangle}_{#2}}
%\newcommand{\gpgradezero}[1] {\gpgrade{#1}{0}}
%\newcommand{\gpgradetwo}[1] {\gpgrade{#1}{2}}
%\newcommand{\gpgradefour}[1] {\gpgrade{#1}{4}}

% ga_wiki_torque.tex:
\newcommand{\Fcap}[0]{\hat{\BF}}
\newcommand{\bithree}[0]{{\Bi}_3}
\newcommand{\nrrp}[0]{\norm{\rcap \wedge \Br'}}
\newcommand{\dtheta}[1]{\frac{d {#1}}{d \theta}}

% ga_wiki_unit_derivative.tex
\newcommand{\dt}[1]{\frac{d {#1}}{dt}}
\newcommand{\BrPrimeRej}[0]{\rcap(\rcap \wedge \Br')}

% radial_vector_derivatives.tex:
%\newcommand{\BrPrimeRej}[0]{\rcap(\rcap \wedge \Br')}

% angular_velocity.tex

%\newcommand{\dt}[1]{\frac{d {#1}}{dt}}
%\newcommand{\Norm}[1]{\left\lVert{#1}\right\rVert}
%\newcommand{\dtheta}[1]{\frac{d {#1}}{d \theta}}

% reciprocal_frame.tex
\DeclareMathOperator{\AbsName}{abs}

%\DeclareMathOperator{\RejName}{Rej}
%\newcommand{\Rej}[2]{\RejName_{#1}\left( {#2} \right)}

\DeclareMathOperator{\AName}{A}
\newcommand{\AreaOp}[1]{\AName_{#1}}

\DeclareMathOperator{\VName}{V}
\newcommand{\VolumeOp}[1]{\VName_{#1}}

%\newcommand{\gpgrade}[2] {{\left\langle{{#1}}\right\rangle}_{#2}}
%\newcommand{\gpgradeone}[1] {{\left\langle{{#1}}\right\rangle}_{1}}


% projection_with_matrix_comparison.tex
%\DeclareMathOperator{\Transpose}{T}
\DeclareMathOperator{\rank}{rank}
%\newcommand{\transpose}[1]{{{#1}^{\TextTranspose}}}
%\newcommand{\transpose}[1]{{{#1}^{\text{T}}}}
\newcommand{\T}[0]{{\text{T}}}
%\newcommand{\BOmega}[0]{\boldsymbol{\Omega}}

%\newcommand{\Det}[1] {\abs{#1}}

% oblique_proj.tex
%\newcommand{\T}[0]{\text{T}}
%\newcommand{\Bbeta}[0]{\boldsymbol{\beta}}

% spherical_polar.tex
\newcommand{\phicap}[0]{\hat{\boldsymbol{\phi}}}
\newcommand{\Lor}[2]{{{\Lambda^{#1}}_{#2}}}
\newcommand{\ILor}[2]{{{ \{{\Lambda^{-1}\} }^{#1}}_{#2}}}

% slerp.tex
\DeclareMathOperator{\atan2}{atan2}

% kvector_exponential.tex
%\DeclareMathOperator{\Exp}{e}
%\DeclareMathOperator{\Rej}{Rej}
\newcommand{\Bcap}[0]{\hat{\BB}}
\newcommand{\Babs}[0]{\abs{\BB}}
%\newcommand{\gpgrade}[2] {{\left\langle{{#1}}\right\rangle}_{#2}}
%\newcommand{\gpgradezero}[1] {\gpgrade{#1}{0}}
%\newcommand{\gpgradetwo}[1] {\gpgrade{#1}{2}}
%\newcommand{\gpgradefour}[1] {\gpgrade{#1}{4}}

\newcommand{\ddu}[1] {\frac {d{#1}} {du}}

% vector_integral_relations.tex
%\newcommand{\Oiint}{\bigcirc \hspace{-1.4em} \int \hspace{-.8em} \int}

% legendre.tex
\newcommand{\agrad}[0]{\Ba \cdot \nabla}
\newcommand{\bxa}[0]{\Bx\Ba}
\newcommand{\costheta}[0]{\acap \cdot \xcap}
%\newcommand{\inv}[1]{\frac{1}{#1}}
\newcommand{\half}[0]{\inv{2}}

% ke_rotation.tex
\newcommand{\DotT}[1]{\dot{#1}}
\newcommand{\DDotT}[1]{\ddot{#1}}
%\newcommand{\transpose}[1]{{#1}^{\text{T}}}
%\newcommand{\Balpha}[0]{\boldsymbol{\alpha}}

%\newcommand{\gpgrade}[2] {{\left\langle{{#1}}\right\rangle}_{#2}}
%\newcommand{\gpgradeone}[1] {{\left\langle{{#1}}\right\rangle}_{1}}
\newcommand{\gpscalargrade}[1] {{\left\langle{{#1}}\right\rangle}}
%\newcommand{\BOmega}[0]{\boldsymbol{\Omega}}

% gaussian_surface.tex
%\newcommand{\phicap}[0]{\hat{\Bphi}}

% newtonian_lagrangian_and_gradient.tex
% PD macro that is backwards from current in macros2:
\newcommand{\PDb}[2]{ \frac{\partial{#1}}{\partial {#2}} }

% inertial_tensor.tex
\newcommand{\matrixoftx}[3]{
{
\begin{bmatrix}
{#1}
\end{bmatrix}
}_{#2}^{#3}
}

\newcommand{\coordvec}[2]{
{
\begin{bmatrix}
{#1}
\end{bmatrix}
}_{#2}
}

% bohr.tex
\newcommand{\K}[0]{\inv{4 \pi \epsilon_0}}

% euler_lagrange.tex
\newcommand{\qbar}[0]{\bar{q}}
\newcommand{\qdotbar}[0]{\dot{\bar{q}}}
\newcommand{\DD}[2]{\frac{d{#2}}{d{#1}}}
\newcommand{\Xdot}[0]{\dot{X}}

% rayleigh_jeans.tex
\newcommand{\EE}[0]{\boldsymbol{\mathcal{E}}}
\newcommand{\HH}[0]{\boldsymbol{\mathcal{H}}}

% 4d_fourier.tex

%\newcommand{\PDSq}[2]{\frac{\partial^2 {#2}}{\partial {#1}^2}}
\DeclareMathOperator{\sinc}{sinc}
\DeclareMathOperator{\PV}{PV}
\newcommand{\FF}[0]{\mathcal{F}}
\newcommand{\IIinf}[0]{ \int_{-\infty}^\infty }

% poisson.tex
%\newcommand{\PDSq}[2]{\frac{\partial^2 {#2}}{\partial {#1}^2}}
%\DeclareMathOperator{\sinc}{sinc}
%\DeclareMathOperator{\PV}{PV}
%\newcommand{\FF}[0]{\mathcal{F}}
%\newcommand{\IIinf}[0]{ \int_{-\infty}^\infty }

% fourier_maxwell.tex
%\newcommand{\PDSq}[2]{\frac{\partial^2 {#2}}{\partial {#1}^2}}
%\DeclareMathOperator{\sinc}{sinc}
%\DeclareMathOperator{\sgn}{sgn}
%\DeclareMathOperator{\PV}{PV}
%\newcommand{\FF}[0]{\mathcal{F}}
%\newcommand{\IIinf}[0]{ \int_{-\infty}^\infty }

% firstorder_fourier_maxwell.tex
%\newcommand{\PDSq}[2]{\frac{\partial^2 {#2}}{\partial {#1}^2}}
%\DeclareMathOperator{\sinc}{sinc}
%\DeclareMathOperator{\PV}{PV}
%\newcommand{\FF}[0]{\mathcal{F}}
%\newcommand{\IIinf}[0]{ \int_{-\infty}^\infty }

% wave_fourier.tex
%\newcommand{\PDSq}[2]{\frac{\partial^2 {#2}}{\partial {#1}^2}}
%\DeclareMathOperator{\sinc}{sinc}
%\DeclareMathOperator{\PV}{PV}
%\newcommand{\FF}[0]{\mathcal{F}}
%\newcommand{\IIinf}[0]{ \int_{-\infty}^\infty }

% heat_fourier.tex
%\newcommand{\PDSq}[2]{\frac{\partial^2 {#2}}{\partial {#1}^2}}
%\DeclareMathOperator{\sinc}{sinc}
%\newcommand{\FF}[0]{\mathcal{F}}
%\newcommand{\IIinf}[0]{ \int_{-\infty}^\infty }

% proj_generalized_dot_prod.tex
%\newcommand{\T}[0]{\text{T}}

% fourier_tx.tex
%\newcommand{\FF}[0]{\mathcal{F}}
\newcommand{\FM}[0]{\inv{\sqrt{2\pi\hbar}}}
\newcommand{\Iinf}[1]{ \int_{-\infty}^\infty {#1}}
%\DeclareMathOperator{\PV}{PV}

% fourier_notation.tex
%\newcommand{\FF}[0]{\mathcal{F}}
%\newcommand{\IIinf}[0]{ \int_{-\infty}^\infty }
%\DeclareMathOperator{\PV}{PV}
%\DeclareMathOperator{\sinc}{sinc}

% planewave.tex
%\newcommand{\EE}[0]{\boldsymbol{\mathcal{E}}}
%\newcommand{\HH}[0]{\boldsymbol{\mathcal{H}}}
%\newcommand{\IIinf}[0]{ \int_{-\infty}^\infty }

% dirac_lagrangian.tex
\newcommand{\Dslash}[0]{ \not\!D }

% pauli_matrix.tex
\newcommand{\Clifford}[2]{\mathcal{C}_{\{{#1},{#2}\}}}
\DeclareMathOperator{\tr}{Tr}
%\DeclareMathOperator{\Scalar}{Scalar}
\DeclareMathOperator{\Real}{Re}
\DeclareMathOperator{\Imag}{Im}
\newcommand{\trace}[1]{\tr{#1}}
\newcommand{\scalarProduct}[2]{{#1} \bullet {#2}}
\newcommand{\traceB}[1]{\tr\left({#1}\right)}
%\newcommand{\symmetric}[2]{{\left\{{#1},{#2}\right\}}}
%\newcommand{\antisymmetric}[2]{\left[{#1},{#2}\right]}
%\newcommand{\Bcap}[0]{\hat{\BB}}

\newcommand{\xhat}[0]{\hat{x}}

\newcommand{\PauliI}[0]{
\begin{bmatrix}
1 & 0 \\
0 & 1 \\
\end{bmatrix}
}

\newcommand{\PauliX}[0]{
\begin{bmatrix}
0 & 1 \\
1 & 0 \\
\end{bmatrix}
}

\newcommand{\PauliY}[0]{
\begin{bmatrix}
0 & -i \\
i & 0 \\
\end{bmatrix}
}

\newcommand{\PauliYNoI}[0]{
\begin{bmatrix}
0 & -1 \\
1 & 0 \\
\end{bmatrix}
}

\newcommand{\PauliZ}[0]{
\begin{bmatrix}
1 & 0 \\
0 & -1 \\
\end{bmatrix}
}

% gamma.tex
%\newcommand{\scalarProduct}[2]{{#1} \bullet {#2}}
%\newcommand{\symmetric}[2]{{\left\{{#1},{#2}\right\}}}
%\newcommand{\antisymmetric}[2]{\left[{#1},{#2}\right]}

%\newcommand{\PauliX}[0]{
%\begin{bmatrix}
%0 & 1 \\
%1 & 0 \\
%\end{bmatrix}
%}

%\newcommand{\PauliY}[0]{
%\begin{bmatrix}
%0 & -i \\
%i & 0 \\
%\end{bmatrix}
%}

%\newcommand{\PauliYNoI}[0]{
%\begin{bmatrix}
%0 & -1 \\
%1 & 0 \\
%\end{bmatrix}
%}

%\newcommand{\PauliZ}[0]{
%\begin{bmatrix}
%1 & 0 \\
%0 & -1 \\
%\end{bmatrix}
%}

% em_bivector_metric_dependencies.tex

%\newcommand{\LL}[0]{\mathcal{L}}
%\newcommand{\gpgrade}[2] {{\left\langle{{#1}}\right\rangle}_{#2}}
%\newcommand{\gpgradezero}[1] {\gpgrade{#1}{0}}
%\newcommand{\gpgradetwo}[1] {\gpgrade{#1}{2}}
%\newcommand{\gpgradeone}[1] {\gpgrade{#1}{1}}
%\newcommand{\gpgradefour}[1] {\gpgrade{#1}{4}}
%\newcommand{\grad}[0]{\nabla}
%\newcommand{\spacegrad}[0]{\boldsymbol{\nabla}}
% == \partial_{#1} {#2}
%\newcommand{\PD}[2]{\frac{\partial {#2}}{\partial {#1}}}
%\newcommand{\PDD}[3]{\frac{\partial^2 {#3}}{\partial {#1}\partial {#2}}}
\newcommand{\PDsQ}[2]{\frac{\partial^2 {#2}}{\partial^2 {#1}}}

% gem.tex
\newcommand{\barh}[0]{\bar{h}}

% mass_vary_lagrangian.tex
%\newcommand{\LL}[0]{\mathcal{L}}
%\newcommand{\grad}[0]{\nabla}
%\newcommand{\PD}[2]{\frac{\partial {#2}}{\partial {#1}}}
%\newcommand{\xdot}[0]{\dot{x}}
%\newcommand{\vdot}[0]{\dot{v}}
%\newcommand{\mdot}[0]{\dot{m}}
%\newcommand{\xddot}[0]{\ddot{x}}
%\newcommand{\spacegrad}[0]{\boldsymbol{\nabla}}

% fourvec_dotinvariance.tex
%\newcommand{\Balpha}[0]{\boldsymbol{\alpha}}
\newcommand{\alphacap}[0]{\hat{\boldsymbol{\alpha}}}
%\newcommand{\Bcap}[0]{\hat{\BB}}
%\newcommand{\gpgrade}[2] {{\left\langle{{#1}}\right\rangle}_{#2}}
%\newcommand{\gpgradezero}[1] {\gpgrade{#1}{0}}

% lorentz.tex
%\newcommand{\laplacian}[0]{\nabla^2}

% field_lagrangian.tex
%\newcommand{\LL}[0]{\mathcal{L}}
%\newcommand{\PD}[2]{\frac{\partial {#2}}{\partial {#1}}}
\newcommand{\barA}[0]{\bar{A}}
%\newcommand{\grad}[0]{\nabla}
%\newcommand{\conj}[0]{{*}}

%\newcommand{\spacegrad}[0]{\boldsymbol{\nabla}}

%\newcommand{\gpgrade}[2] {{\left\langle{{#1}}\right\rangle}_{#2}}
%\newcommand{\gpgradezero}[1] {\gpgrade{#1}{0}}
%\newcommand{\gpgradetwo}[1] {\gpgrade{#1}{2}}
%\newcommand{\gpgradefour}[1] {\gpgrade{#1}{4}}

% lagrangian_field_density.tex
%\newcommand{\LL}[0]{\mathcal{L}}
%\newcommand{\gpgrade}[2] {{\left\langle{{#1}}\right\rangle}_{#2}}
%\newcommand{\gpgradezero}[1] {\gpgrade{#1}{0}}
%\newcommand{\gpgradetwo}[1] {\gpgrade{#1}{2}}
%\newcommand{\gpgradefour}[1] {\gpgrade{#1}{4}}
%\newcommand{\grad}[0]{\nabla}
%\newcommand{\spacegrad}[0]{\boldsymbol{\nabla}}
%\newcommand{\PD}[2]{\frac{\partial {#2}}{\partial {#1}}}
\newcommand{\PDd}[2]{\frac{\partial^2 {#2}}{{\partial{#1}}^2}}
%\newcommand{\PDD}[3]{\frac{\partial^2 {#3}}{\partial {#1}\partial {#2}}}

%\newcommand{\barA}[0]{\bar{A}}

% lorentz_force.tex
%\newcommand{\grad}[0]{\nabla}
%\newcommand{\spacegrad}[0]{\boldsymbol{\nabla}}
%\newcommand{\LL}[0]{\mathcal{L}}
%\newcommand{\xdot}[0]{\dot{x}}
%\newcommand{\xddot}[0]{\ddot{x}}
%\newcommand{\pdot}[0]{\dot{p}}
%\newcommand{\pddot}[0]{\ddot{p}}
%\newcommand{\fdot}[0]{\dot{f}}
%\newcommand{\fddot}[0]{\ddot{f}}

%\newcommand{\gpgrade}[2] {{\left\langle{{#1}}\right\rangle}_{#2}}
%\newcommand{\gpgradeone}[1] {\gpgrade{#1}{1}}
%\newcommand{\gpgradezero}[1] {\gpgrade{#1}{}}
%\newcommand{\grad}[0] {\nabla}
%\newcommand{\spacegrad}[0]{\boldsymbol{\nabla}}

%\newcommand{\pdot}[0]{\dot{p}}
%\newcommand{\pddot}[0]{\ddot{p}}

%\newcommand{\xdot}[0]{\dot{x}}
%\newcommand{\xddot}[0]{\ddot{x}}
%\newcommand{\PD}[2]{\frac{\partial {#2}}{\partial {#1}}}

% stokes_maxwell_application.tex
%\newcommand{\grad}[0]{\nabla}
%\newcommand{\PD}[2]{\frac{\partial {#2}}{\partial {#1}}}
%\newcommand{\spacegrad}[0]{\boldsymbol{\nabla}}
%\newcommand{\gpgrade}[2] {{\left\langle{{#1}}\right\rangle}_{#2}}
%\newcommand{\gpgradezero}[1] {\gpgrade{#1}{0}}
%\newcommand{\gpgradeone}[1] {\gpgrade{#1}{1}}
%\newcommand{\gpgradetwo}[1] {\gpgrade{#1}{2}}
%\newcommand{\gpgradethree}[1] {\gpgrade{#1}{3}}

% lorentz_rotation.tex
%\DeclareMathOperator{\Transpose}{T}
%\newcommand{\T}[0]{\text{T}}

% electron_rotor.tex
\newcommand{\reverse}[1]{\tilde{{#1}}}
%\newcommand{\ILambda}[0]{{(\Lambda^{-1})}}
\newcommand{\ILambda}[0]{\Pi}

% em_potential.tex
%\newcommand{\spacegrad}[0]{\boldsymbol{\nabla}}
%\newcommand{\grad}[0]{\nabla}
\newcommand{\CA}[0]{\mathcal{A}}
 
% maxwell_to_tensor.tex
%\newcommand{\LL}[0]{\mathcal{L}}
%\newcommand{\gpgrade}[2] {{\left\langle{{#1}}\right\rangle}_{#2}}
%\newcommand{\gpgradezero}[1] {\gpgrade{#1}{0}}
%\newcommand{\gpgradetwo}[1] {\gpgrade{#1}{2}}
%\newcommand{\gpgradeone}[1] {\gpgrade{#1}{1}}
%\newcommand{\gpgradefour}[1] {\gpgrade{#1}{4}}
%\newcommand{\grad}[0]{\nabla}
%\newcommand{\spacegrad}[0]{\boldsymbol{\nabla}}
% == \partial_{#1} {#2}
%\newcommand{\PD}[2]{\frac{\partial {#2}}{\partial {#1}}}
%\newcommand{\PDD}[3]{\frac{\partial^2 {#3}}{\partial {#1}\partial {#2}}}
%\newcommand{\PDsQ}[2]{\frac{\partial^2 {#2}}{\partial^2 {#1}}}

%\newcommand{\EE}[0]{\boldsymbol{\mathcal{E}}}
%\newcommand{\HH}[0]{\boldsymbol{\mathcal{H}}}
\newcommand{\Vcap}[0]{\hat{\BV}}



\title{Light and Electron Quantization?}
\author{Peeter Joot}
\email{peeter.joot@gmail.com}
\date{July 2, 2009}
\revisionInfo{$RCSfile: lightQuantize.tex,v $ Last $Revision: 1.4 $ $Date: 2009/07/04 04:34:25 $}
\blogpage{http://sites.google.com/site/peeterjoot/math2009/lightQuantize.pdf}

%\newcommand{\Schrodinger}[0]{Schr\"{o}dinger}
\begin{document}

\maketitle{}
\tableofcontents
\section{Motivation}

In \href{http://sites.google.com/site/peeterjoot/math2009/relwave.pdf}{Relativistic origins of Schr\"odinger's equation}, an ad-hoc technique of motivating the non-relativisitic Schr\"odinger equation was detailed.  This used the DeBroglie quantization hypothesis to matter via a quantization of the Klien-Gordon equation.  While this  does result in the final desired goal some of the intermediate steps are rather unsatisfying.  In particular the wave equation for light in a vacuum was used as a starting point, before moving on to quantized matter.  Quantization of light itself was glossed over, and if that was considered in isolation, it is clear that photon quantization based on the wave equation in isolation cannot possibly be correct.  From the quantization result, whatever that is, one ought to be able to recover Maxwell's vacuum equation by considering time averages in a fashion something like the Erhenfest procedure.  This cannot be done if the coupling between the field components is neglected.

A possible alternate starting point (simpler than the coupling between six field components) that would still use the wave equation would be to use Maxwell's equations in four vector potential form, however, the Lorentz gauge constraint would also be required

\begin{align}\label{eqn:maxwellPotVacuum}
\partial_\mu \partial^\mu A^\nu &= 0 \\
\partial_\nu A^\nu &= 0
\end{align}

In one fashion or another, one must have to start with the vacuum Maxwell's equation in its entirely if attempting light quantization starting from classical arguments.  Regardless of whether one considers the previous Schr\"odinger motivation attempt handwaving, the fact that it is possible at all points to a neccessary failure of the Schr\"odinger equation if the wave function under consideration is for the electron.  Because the starting point for light quantization was wrong, the result for electron quantization must also be wrong.  With the coupling of light and electrons in classical electromagnetism, we need four coupled Schr\"odinger equations, not one.

This failure is probably what the Dirac formalism addresses.  I likely don't have the prerequisites required to learning QFT at this point to find out.  However, lets see if following the classical requirements more strictly lead anywhere interesting.

\section{Maxwell vacuum solution.}

While Fourier solutions to \ref{eqn:maxwellPotVacuum} are possible, how does one deal with the coupling constraint of the Lorentz gauge?  Let's start
over with the first order Maxwell equation, noting that (\ref{eqn:maxwellPotVacuum}) was the result of setting $\grad \cdot A$ and $J = 0$ in the
general Maxwell equation

\begin{align}\label{eqn:maxwell}
\grad F &= J/\epsilon_0 c \\
F &= \grad \wedge A
\end{align}

Picking an observer bias for the gradient by premultiplying with $\gamma_0$ the vacuum equation for light can therefore also be written as

\begin{align*}
0
&= \gamma_0 \grad F \\
&= \gamma_0 (\gamma^0 \partial_0 + \gamma^k \partial_k) F \\
&= (\partial_0 - \gamma^k \gamma_0 \partial_k) F \\
&= (\partial_0 + \sigma^k \partial_k) F \\
&= \left(\inv{c}\partial_t + \spacegrad \right) F \\
\end{align*}

A Fourier transformation of this equation produces

\begin{align*}
0 &= \inv{c} \PD{t}{\hat{F}}(\Bk,t) + \inv{\sqrt{2\pi}^3} \int \sigma^m \partial_m F(\Bx,t) e^{-i \Bk \cdot \Bx} d^3 x
\end{align*}

and with a single integration by parts one has

\begin{align*}
0
&= \inv{c} \PD{t}{\hat{F}}(\Bk,t) - \inv{\sqrt{2\pi}^3} \int \sigma^m F(\Bx,t) (-i k_m) e^{-i \Bk \cdot \Bx} d^3 x \\
&= \inv{c} \PD{t}{\hat{F}}(\Bk,t) + \inv{\sqrt{2\pi}^3} \int \Bk F(\Bx,t) i e^{-i \Bk \cdot \Bx} d^3 x \\
&= \inv{c} \PD{t}{\hat{F}}(\Bk,t) + i \Bk \hat{F}(\Bk,t)
\end{align*}

The flexibility to employ the pseudoscalare as the imaginary $i = \gamma_0 \gamma_1 \gamma_2 \gamma_3$ has been employed above, so it should
be noted that pseudoscalar commutation with Dirac bivectors was implied above, but also that we do not have the flexibility to commute $\Bk$ with $F$.

Having done this, the problem to solve is now Maxwell's vacuum equation in the frequency domain

\begin{align*}
\PD{t}{\hat{F}}(\Bk,t) = -i c \Bk \hat{F}(\Bk,t)
\end{align*}

Introducing an angular frequency (spatial bivector)

\begin{align*}
\Omega = -i \Bk c
\end{align*}

This becomes

\begin{align}\label{eqn:MaxwellFreq}
\hat{F}' = \Omega F
\end{align}

With solution

\begin{align}
\hat{F} = e^{\Omega t} \hat{F}(\Bk,0)
\end{align}

Differentiation with respect to time verifies that the ordering of the terms is correct and this does in fact solve (\ref{eqn:MaxwellFreq}).  This
is something we have to be careful of due to the possibility of non-commutating variables.

Back substituition into the inverse transform now supplies the time evolution of the field given the initial time specification

\begin{align*}
F(\Bx,t)
&= \inv{\sqrt{2\pi}^3} \int e^{\Omega t} \hat{F}(\Bk,0) e^{i \Bk \cdot \Bx} d^3 k \\
&= \inv{(2\pi)^3} \int e^{\Omega t} \left( \int {F}(\Bx',0) e^{-i \Bk \cdot \Bx'} d^3 x' \right) e^{i \Bk \cdot \Bx} d^3 k
\end{align*}

Observe that Pseudoscalar exponentials commute with the field because $i$ commutes with spatial vectors and itself

\begin{align*}
F e^{i\theta}
&= (\BE + i c \BB) (C + iS) \\
&=
C (\BE + i c \BB)
+ S (\BE + i c \BB) i  \\
&=
C (\BE + i c \BB)
+ S i (\BE + i c \BB) \\
&=
e^{i\theta} F
\end{align*}

This allows the specifics of the initial time conditions to be suppressed

\begin{align}
F(\Bx,t) &= \int d^3 k e^{\Omega t} e^{i \Bk \cdot \Bx} \int \inv{(2\pi)^3} {F}(\Bx',0) e^{-i \Bk \cdot \Bx'}  d^3 x'
\end{align}

The interior integral has the job of a weighting function over plane wave solutions, and this can be made explicit writing

\begin{align}
D(\Bk) &= \inv{(2\pi)^3} \int {F}(\Bx',0) e^{-i \Bk \cdot \Bx'}  d^3 x' \\
F(\Bx,t) &= \int e^{\Omega t} e^{i \Bk \cdot \Bx} D(\Bk) d^3 k
\end{align}

Many assumptions have been made here, not the least of which was a requirement for the fourier transform of a bivector valued function to be meaningful, and have an inverse.  It is therefore reasonable to verify that this weighted plane wave result is in fact a solution to the original Maxwell vacumm equation.  Differentiation verifies that things are okay so far

\begin{align*}
\gamma_0 \grad F(\Bx,t)
&=
\left(\inv{c}\partial_t + \spacegrad \right)\int e^{\Omega t} e^{i \Bk \cdot \Bx} D(\Bk) d^3 k \\
&=
\int \left(\inv{c}\Omega e^{\Omega t} + \sigma^m e^{\Omega t} i k_m \right) e^{i \Bk \cdot \Bx} D(\Bk) d^3 k \\
&=
\int \left(\inv{c}(-i \Bk c) + i \Bk \right) e^{\Omega t} e^{i \Bk \cdot \Bx} D(\Bk) d^3 k \\
&= 0 \quad\quad\quad\square
\end{align*}

The fact that it the integral has zero gradient does not mean that it is a bivector, so there must
also be at least also be restrictions on the grades of $D(\Bk)$.

To simplify discussion, let's discretize the integral writing

\begin{align*}
D(\Bk') = D_\Bk \delta^3 (\Bk - \Bk')
\end{align*}

So we have

\begin{align*}
F(\Bx,t) 
&= \int e^{\Omega t} e^{i \Bk' \cdot \Bx} D(\Bk') d^3 k' \\
&= \int e^{\Omega t} e^{i \Bk' \cdot \Bx} D_\Bk \delta^3(\Bk - \Bk') d^3 k' \\
\end{align*}

This produces something planewave-ish

\begin{align}\label{eqn:planewaveish}
F(\Bx,t) &= e^{\Omega t} e^{i \Bk \cdot \Bx} D_\Bk 
\end{align}

Observe that at $t=0$ we have

\begin{align*}
F(\Bx,0) 
&= e^{i \Bk \cdot \Bx} D_\Bk  \\
&= (\cos (\Bk \cdot \Bx) + i \sin(\Bk \cdot \Bx)) D_\Bk  \\
\end{align*}

There is therefore a requirement for $D_\Bk$ to be either a spatial vector or its dual, a spatial bivector.  For example taking $D_k$ to be a spatial vector we can then identify the electric and magnetic components of the field

\begin{align*}
\BE(\Bx,0) &= \cos (\Bk \cdot \Bx) D_\Bk \\
c \BB(\Bx,0) &= \sin (\Bk \cdot \Bx) D_\Bk
\end{align*}

and if $D_k$ is taken to be a spatial bivector, this pair of identifications would be inverted.

Considering (\ref{eqn:planewaveish}) at $\Bx=0$, we have

\begin{align*}
F(0, t) 
&= e^{\Omega t} D_\Bk \\
&= (\cos(\Abs{\Omega} t) + \hat{\Omega} \sin(\Abs{\Omega} t)) D_\Bk \\
&= (\cos(\Abs{\Omega} t) - i \hat{\Bk} \sin(\Abs{\Omega} t)) D_\Bk \\
\end{align*}

If $D_\Bk$ is first assumed to be a spatial vector, then $F$ would have a pseudoscalar component if $D_\Bk$ has any
component parallel to $\hat{\Bk}$.

\begin{align*}
D_\Bk \in \span\{\sigma^m\} \implies D_\Bk \cdot \hat{\Bk} = 0 
\end{align*}
\begin{align*}
D_\Bk \in \span\{\sigma^a \wedge \sigma^b\} \implies D_\Bk \cdot (i\hat{\Bk}) = 0
\end{align*}

It doesn't really appear that there is any loss of generality imposing a spatial vector restriction on $D_\Bk$, at
least in the current free case, since we can convert between the two using a duality transformation.

\section{Energy Momentum four vector.}

With the Fourier work out of the way, expressing the field energy and momentum is now required.  In the GA
formalism the energy momentum tensor is

\begin{align}
T(a) = \frac{\epsilon_0}{2} F a \tilde{F}
\end{align}

\subsection{ In terms of electric and magnetic fields. }

It is not neccessarily obvious this bivector-vector-bivector product construction is even a vector quantity.
Expansion of $T(\gamma_0)$ in terms of
the electric and magnetic fields
demonstrates this vectorial nature.

\begin{align*}
F \gamma_0 \tilde{F}
&=
-(\BE + i c \BB) \gamma_0 (\BE + i c \BB) \\
&=
-\gamma_0 (-\BE + i c \BB) (\BE + i c \BB) \\
&=
-\gamma_0 (-\BE^2 - c^2 \BB^2 + i c (\BB \BE - \BE \BB) ) \\
&=
\gamma_0 (\BE^2 + c^2 \BB^2) - 2 \gamma_0 i c (\BB \wedge \BE) ) \\
&=
\gamma_0 (\BE^2 + c^2 \BB^2) + 2 \gamma_0 c (\BB \cross \BE) \\
&=
\gamma_0 (\BE^2 + c^2 \BB^2) + 2 \gamma_0 c \gamma_k \gamma_0 (\BB \cross \BE)^k \\
&=
\gamma_0 (\BE^2 + c^2 \BB^2) + 2 \gamma_k (\BE \cross (c \BB))^k \\
\end{align*}

Therefore, $T(\gamma_0)$, the energy momentum tensor biased towards a particular observer frame $\gamma_0$
is

\begin{align}
T(\gamma_0)
&=
\gamma_0 \frac{\epsilon_0}{2} (\BE^2 + c^2 \BB^2) + \gamma_k \epsilon_0 (\BE \cross (c \BB))^k \\
\end{align}

Recognizable here in the components $T(\gamma_0)$ are the field energy density and momentum density.  In particular
the energy density can be obtained by dotting with $\gamma_0$, whereas the (spatial vector) momentum by wedging with
$\gamma_0$.

These are

\begin{align}
U \equiv T(\gamma_0) \cdot \gamma_0 &= \frac{1}{2} \left( \epsilon_0 \BE^2 + \inv{\mu_0} \BB^2 \right) \\
c \BP \equiv T(\gamma_0) \wedge \gamma_0 &= \inv{\mu_0} \BE \cross \BB
\end{align}

In terms of the combined field these are

\begin{align}
U &= \frac{-1}{2}( F \gamma_0 F \gamma_0 + \gamma_0 F \gamma_0 F) \\
c \BP &= \frac{-1}{2}( F \gamma_0 F \gamma_0 - \gamma_0 F \gamma_0 F)
\end{align}

The Hermitian conjugate

\begin{align}
A^\dagger = \gamma_0 \tilde{A} \gamma_0
\end{align}

can conviently be used to summarize these relations as follows

\begin{align}
U &= \frac{1}{2}( F F^\dagger + F^\dagger F) \\
c \BP &= \frac{1}{2}( F F^\dagger - F^\dagger F)
\end{align}

\subsection{ Divergence. }

Calculation of the divergence produces the components of the Lorentz force densities

\begin{align*}
\grad \cdot T(a)
&= \frac{\epsilon_0}{2} \gpgradezero{ \grad (F a F) } \\
&= \frac{\epsilon_0}{2} \gpgradezero{ (\grad F) a F + (F \grad) F a } \\
\end{align*}

Here the gradient is used implicitly in bidirectional form, where the direction is implied by context.
From Maxwell's equation we have

\begin{align*}
J/\epsilon_0 c
&= (\grad F)^{\tilde{}} \\
&= (\tilde{F} \tilde{\grad}) \\
&= -(F \grad)
\end{align*}

and continuing the expansion

\begin{align*}
\grad \cdot T(a)
&= \frac{1}{2c} \gpgradezero{ J a F - J F a } \\
&= \frac{1}{2c} \gpgradezero{ F J a - J F a } \\
&= \frac{1}{2c} \gpgradezero{ (F J - J F) a } \\
\end{align*}

Wrapping up, the divergence and the adjoint of the energy momentum tensor are

\begin{align}
\grad \cdot T(a) &= \frac{1}{c} (F \cdot J) \cdot a \\
\bar{T}(\grad) &= F \cdot J/c
\end{align}

When integrated over a volume, the quantities $F \cdot J/c$ are the components of the RHS of the Lorentz
force equation $\dot{p} = q F \cdot v/c$.

\bibliography{myrefs}
\bibliographystyle{unsrturl}

\end{document}
