%
% Copyright � 2012 Peeter Joot.  All Rights Reserved.
% Licenced as described in the file LICENSE under the root directory of this GIT repository.
%

%
%
%\documentclass[]{eliblog}

\usepackage{amsmath}
\usepackage{mathpazo}

%
% shorthand for bold symbols, convenient for vectors and matrices
%
\newcommand{\Ba}[0]{\mathbf{a}}
\newcommand{\Bb}[0]{\mathbf{b}}
\newcommand{\Bc}[0]{\mathbf{c}}
\newcommand{\Bd}[0]{\mathbf{d}}
\newcommand{\Be}[0]{\mathbf{e}}
\newcommand{\Bf}[0]{\mathbf{f}}
\newcommand{\Bg}[0]{\mathbf{g}}
\newcommand{\Bh}[0]{\mathbf{h}}
\newcommand{\Bi}[0]{\mathbf{i}}
\newcommand{\Bj}[0]{\mathbf{j}}
\newcommand{\Bk}[0]{\mathbf{k}}
\newcommand{\Bl}[0]{\mathbf{l}}
\newcommand{\Bm}[0]{\mathbf{m}}
\newcommand{\Bn}[0]{\mathbf{n}}
\newcommand{\Bo}[0]{\mathbf{o}}
\newcommand{\Bp}[0]{\mathbf{p}}
\newcommand{\Bq}[0]{\mathbf{q}}
\newcommand{\Br}[0]{\mathbf{r}}
\newcommand{\Bs}[0]{\mathbf{s}}
\newcommand{\Bt}[0]{\mathbf{t}}
\newcommand{\Bu}[0]{\mathbf{u}}
\newcommand{\Bv}[0]{\mathbf{v}}
\newcommand{\Bw}[0]{\mathbf{w}}
\newcommand{\Bx}[0]{\mathbf{x}}
\newcommand{\By}[0]{\mathbf{y}}
\newcommand{\Bz}[0]{\mathbf{z}}
\newcommand{\BA}[0]{\mathbf{A}}
\newcommand{\BB}[0]{\mathbf{B}}
\newcommand{\BC}[0]{\mathbf{C}}
\newcommand{\BD}[0]{\mathbf{D}}
\newcommand{\BE}[0]{\mathbf{E}}
\newcommand{\BF}[0]{\mathbf{F}}
\newcommand{\BG}[0]{\mathbf{G}}
\newcommand{\BH}[0]{\mathbf{H}}
\newcommand{\BI}[0]{\mathbf{I}}
\newcommand{\BJ}[0]{\mathbf{J}}
\newcommand{\BK}[0]{\mathbf{K}}
\newcommand{\BL}[0]{\mathbf{L}}
\newcommand{\BM}[0]{\mathbf{M}}
\newcommand{\BN}[0]{\mathbf{N}}
\newcommand{\BO}[0]{\mathbf{O}}
\newcommand{\BP}[0]{\mathbf{P}}
\newcommand{\BQ}[0]{\mathbf{Q}}
\newcommand{\BR}[0]{\mathbf{R}}
\newcommand{\BS}[0]{\mathbf{S}}
\newcommand{\BT}[0]{\mathbf{T}}
\newcommand{\BU}[0]{\mathbf{U}}
\newcommand{\BV}[0]{\mathbf{V}}
\newcommand{\BW}[0]{\mathbf{W}}
\newcommand{\BX}[0]{\mathbf{X}}
\newcommand{\BY}[0]{\mathbf{Y}}
\newcommand{\BZ}[0]{\mathbf{Z}}

\newcommand{\Bzero}[0]{\mathbf{0}}
\newcommand{\Btheta}[0]{\boldsymbol{\theta}}
\newcommand{\Btau}[0]{\boldsymbol{\tau}}
\newcommand{\Bomega}[0]{\boldsymbol{\omega}}

%
% shorthand for unit vectors
%
\newcommand{\acap}[0]{\hat{\Ba}}
\newcommand{\bcap}[0]{\hat{\Bb}}
\newcommand{\ccap}[0]{\hat{\Bc}}
\newcommand{\dcap}[0]{\hat{\Bd}}
\newcommand{\ecap}[0]{\hat{\Be}}
\newcommand{\fcap}[0]{\hat{\Bf}}
\newcommand{\gcap}[0]{\hat{\Bg}}
\newcommand{\hcap}[0]{\hat{\Bh}}
\newcommand{\icap}[0]{\hat{\Bi}}
\newcommand{\jcap}[0]{\hat{\Bj}}
\newcommand{\kcap}[0]{\hat{\Bk}}
\newcommand{\lcap}[0]{\hat{\Bl}}
\newcommand{\mcap}[0]{\hat{\Bm}}
\newcommand{\ncap}[0]{\hat{\Bn}}
\newcommand{\ocap}[0]{\hat{\Bo}}
\newcommand{\pcap}[0]{\hat{\Bp}}
\newcommand{\qcap}[0]{\hat{\Bq}}
\newcommand{\rcap}[0]{\hat{\Br}}
\newcommand{\scap}[0]{\hat{\Bs}}
\newcommand{\tcap}[0]{\hat{\Bt}}
\newcommand{\ucap}[0]{\hat{\Bu}}
\newcommand{\vcap}[0]{\hat{\Bv}}
\newcommand{\wcap}[0]{\hat{\Bw}}
\newcommand{\xcap}[0]{\hat{\Bx}}
\newcommand{\ycap}[0]{\hat{\By}}
\newcommand{\zcap}[0]{\hat{\Bz}}
\newcommand{\thetacap}[0]{\hat{\Btheta}}

%
% to write R^n and C^n in a distinguishable fashion.  Perhaps change this
% to the double lined characters upon figuring out how to do so.
%
\newcommand{\C}[1]{$\mathbb{C}^{#1}$}
\newcommand{\R}[1]{$\mathbb{R}^{#1}$}

%
% various generally useful helpers
%

% derivative of #1 wrt. #2:
\newcommand{\D}[2] {\frac {d#2} {d#1}}

\newcommand{\inv}[1]{\frac{1}{#1}}
\newcommand{\cross}[0]{\times}

\newcommand{\abs}[1]{\lvert{#1}\rvert}
\newcommand{\norm}[1]{\lVert{#1}\rVert}
\newcommand{\innerprod}[2]{\langle{#1}, {#2}\rangle}
\newcommand{\dotprod}[2]{{#1} \cdot {#2}}
\newcommand{\bdotprod}[2]{\left({#1} \cdot {#2}\right)}
\newcommand{\crossprod}[2]{{#1} \cross {#2}}
\newcommand{\tripleprod}[3]{\dotprod{\left(\crossprod{#1}{#2}\right)}{#3}}

\DeclareMathOperator{\Proj}{Proj}
\DeclareMathOperator{\Span}{span}
\DeclareMathOperator{\Sgn}{sgn}
\DeclareMathOperator{\Area}{Area}
\DeclareMathOperator{\Volume}{Volume}

%
% A few miscellaneous things specific to this document
%
\newcommand{\crossop}[1]{\crossprod{#1}{}}

% R2 vector.
\newcommand{\VectorTwo}[2]{
\begin{bmatrix}
 {#1} \\
 {#2}
\end{bmatrix}
}

\newcommand{\VectorN}[1]{
\begin{bmatrix}
{#1}_1 \\
{#1}_2 \\
\vdots \\
{#1}_N \\
\end{bmatrix}
}

\newcommand{\DETuvij}[4]{
\begin{vmatrix}
 {#1}_{#3} & {#1}_{#4} \\
 {#2}_{#3} & {#2}_{#4}
\end{vmatrix}
}

\newcommand{\DETuvwijk}[6]{
\begin{vmatrix}
 {#1}_{#4} & {#1}_{#5} & {#1}_{#6} \\
 {#2}_{#4} & {#2}_{#5} & {#2}_{#6} \\
 {#3}_{#4} & {#3}_{#5} & {#3}_{#6}
\end{vmatrix}
}

\newcommand{\DETuvwxijkl}[8]{
\begin{vmatrix}
 {#1}_{#5} & {#1}_{#6} & {#1}_{#7} & {#1}_{#8} \\
 {#2}_{#5} & {#2}_{#6} & {#2}_{#7} & {#2}_{#8} \\
 {#3}_{#5} & {#3}_{#6} & {#3}_{#7} & {#3}_{#8} \\
 {#4}_{#5} & {#4}_{#6} & {#4}_{#7} & {#4}_{#8} \\
\end{vmatrix}
}

%\newcommand{\DETuvwxyijklm}[10]{
%\begin{vmatrix}
% {#1}_{#6} & {#1}_{#7} & {#1}_{#8} & {#1}_{#9} & {#1}_{#10} \\
% {#2}_{#6} & {#2}_{#7} & {#2}_{#8} & {#2}_{#9} & {#2}_{#10} \\
% {#3}_{#6} & {#3}_{#7} & {#3}_{#8} & {#3}_{#9} & {#3}_{#10} \\
% {#4}_{#6} & {#4}_{#7} & {#4}_{#8} & {#4}_{#9} & {#4}_{#10} \\
% {#5}_{#6} & {#5}_{#7} & {#5}_{#8} & {#5}_{#9} & {#5}_{#10}
%\end{vmatrix}
%}

% R3 vector.
\newcommand{\VectorThree}[3]{
\begin{bmatrix}
 {#1} \\
 {#2} \\
 {#3}
\end{bmatrix}
}



\author{Peeter Joot}
\email{peeter.joot@gmail.com}


\chapter{blah}
\label{chap:}
%\useCCL
\date{July XX, 2009}
\revisionInfo{\(RCSfile: maxwellVacuum_roughNotes.tex,v \) Last \(Revision: 1.2 \) \(Date: 2009/12/03 03:24:40 \)}

\beginArtWithToc

\section{FIXME:blahblah}
\subsection{Electric and Magnetic field perpendicularity}

Now that we have a non-trivial (non-zero) solution for the magnetic field we should be able to verify that \(\BE \cdot \BB\) as computed in \eqnref{eqn:electricSplit}, and \eqnref{eqn:magneticSplit}.

Note as well that there was no perpendicularity requirement imposed on the parametrization vectors \(\BE_\Bk\) and \(\BB_\Bk\).

Working directly from \eqnref{eqn:phasor} is the best starting place.  Expressing the dot product as a scalar grade selection we have

\begin{equation}\label{eqn:maxwellVacuum_roughNotes:20}
\begin{aligned}
\BE \cdot (c \BB)
&=
\gpgradezero{
\inv{4i}(F + F^\dagger) (F - F^\dagger)
} \\
&=
\inv{4} \gpgradezero{
-i \left( F^2 - (F^\dagger)^2 + F^\dagger F - F F^\dagger \right) } \\
\end{aligned}
\end{equation}

Observe first that the first two terms cancel

\begin{equation}\label{eqn:maxwellVacuum_roughNotes:40}
\begin{aligned}
(F^\dagger)^2
&=
F^\dagger F^\dagger \\
&=
(F F)^\dagger \\
&=
(F^2)^\dagger \\
\end{aligned}
\end{equation}

The expansion of the antisymmetric conjugate pair is not zero, since there
are bivector terms that remain

\begin{equation}\label{eqn:maxwellVacuum_roughNotes:60}
\begin{aligned}
F^\dagger F - F F^\dagger
&=
(\BE - i c \BB)(\BE + i c \BB) - (\BE + i c \BB)(\BE - i c \BB) \\
&=
2 i c (\BE \BB - \BB \BE) \\
&=
4 i c (\BE \wedge \BB)
\end{aligned}
\end{equation}

That leaves the dot product taking the scalar subset of a pure bivector, and
is therefore zero

\begin{equation}\label{eqn:blah}
\begin{aligned}
\BE \cdot (c \BB) = {c} \gpgradezero{ \BE \wedge \BB } = 0
\end{aligned}
\end{equation}

This general dot product calculation shows that we do have \(\BE \cdot \BB = 0\), and does not require the special case result \(\BB = 0\) of the original questionable result \eqnref{eqn:questionable}.  One should be able to show that the more general results of equations \eqnref{eqn:electricSplit}, and \eqnref{eqn:magneticSplit} also have a zero dot product.

Why is it that it seemed
that this perpendicularily was required for the general case?  Revisit that calculation above and see if an error can be spotted.

\section{FIXME:blah}

Let us calculate the effects of this conjugate split on the discrete field solutions expressed by
\eqnref{eqn:planewaveish}.

\begin{equation}\label{eqn:maxwellVacuum_roughNotes:80}
\begin{aligned}
F^\dagger &= D_\Bk^\dagger (e^{i \Bk \cdot \Bx})^\dagger (e^{-i\Bomega t})^\dagger \\
\end{aligned}
\end{equation}

For the pseudoscalar exponentials the Hermitian conjugate, somewhat unsurprisingly has a conjugate effect

\begin{equation}\label{eqn:maxwellVacuum_roughNotes:100}
\begin{aligned}
(e^{i \Bk \cdot \Bx})^\dagger
&=
(\cos(\Bk \cdot \Bx) +i\sin(\Bk \cdot \Bx))^\dagger \\
&=
\cos(\Bk \cdot \Bx) -i\sin(\Bk \cdot \Bx) \\
&=
e^{-i\Bk \cdot \Bx}
\end{aligned}
\end{equation}

How about the exponentials with the spatial bivector argument?
\begin{equation}\label{eqn:maxwellVacuum_roughNotes:120}
\begin{aligned}
(e^{-i\Bomega t})^\dagger
&=
(e^{-i \kcap \omega t})^\dagger  \\
&=
(\cos(\omega t) - i \kcap \sin(\omega t))^\dagger \\
&=
\cos(\omega t) - \kcap^\dagger i^\dagger \sin(\omega t) \\
&=
\cos(\omega t) + \kcap i \sin(\omega t) \\
&=
e^{i \kcap \omega t} \\
&=
e^{i \Bomega t}
\end{aligned}
\end{equation}

Also a conjugate effect.  Assembling results we have for the field and its conjugate

\begin{equation}\label{eqn:maxwellVacuum_roughNotes:140}
\begin{aligned}
F &= e^{-i\Bomega t} e^{i \Bk \cdot \Bx} D_\Bk  \\
F^\dagger &= D_\Bk e^{-i \Bk \cdot \Bx} e^{i\Bomega t}
\end{aligned}
\end{equation}

By the constraint of \eqnref{eqn:commutationRequirementVector} we have for the product

\begin{equation}\label{eqn:maxwellVacuum_roughNotes:160}
\begin{aligned}
i \kcap D_\Bk
&=
-i D_\Bk \kcap \\
&=
- D_\Bk i \kcap \\
\end{aligned}
\end{equation}

So \(D_\Bk\) and \(e^{i\Bomega t}\) commute in a conjugate fashion (whereas \(D_\Bk\) and \(e^{i\Bk \cdot \Bx}\) just commute)

\begin{equation}\label{eqn:maxwellVacuum_roughNotes:180}
\begin{aligned}
e^{-i\Bomega t} D_\Bk &= D_\Bk e^{i\Bomega t} D_\Bk
\end{aligned}
\end{equation}

As a result we have for the conjugate split of \(F\) into \(\BE\) and \(\BB\)
\begin{equation}\label{eqn:maxwellVacuum_roughNotes:200}
\begin{aligned}
F \pm F^\dagger
&= e^{-i\Bomega t} e^{i \Bk \cdot \Bx} D_\Bk \pm D_\Bk e^{-i \Bk \cdot \Bx} e^{i\Bomega t} \\
&= D_\Bk e^{i\Bomega t} e^{i \Bk \cdot \Bx} \pm D_\Bk e^{-i \Bk \cdot \Bx} e^{i\Bomega t} \\
&= D_\Bk e^{i\Bomega t} (e^{i \Bk \cdot \Bx} \pm e^{-i \Bk \cdot \Bx} ) \\
\end{aligned}
\end{equation}

So the split into electric and magnetic fields is

\begin{equation}\label{eqn:maxwellVacuum_roughNotes:220}
\begin{aligned}
\BE &= D_\Bk e^{i\Bomega t} \cos(\Bk \cdot \Bx) \\
c\BB &= D_\Bk e^{i\Bomega t} \sin(\Bk \cdot \Bx)
\end{aligned}
\end{equation}

Again writing \(\Bomega = \kcap \omega t\), and noting that \(D_\Bk \kcap = D_\Bk \wedge \kcap\) (since \(D_\Bk \cdot \kcap = 0\)) we have for the phasor product in the fields above

\begin{equation}\label{eqn:maxwellVacuum_roughNotes:240}
\begin{aligned}
D_\Bk e^{i\Bomega t}
&=
D_\Bk (\cos(\omega t) + i\kcap \sin(\omega t)) \\
&=
D_\Bk \cos(\omega t) + i (D_\Bk \wedge \kcap) \sin(\omega t)) \\
&=
D_\Bk \cos(\omega t) + i^2 (D_\Bk \cross \kcap) \sin(\omega t)) \\
&=
D_\Bk \cos(\omega t) + (\kcap \cross D_\Bk) \sin(\omega t) \\
\end{aligned}
\end{equation}

Unlike the phasors of traditional engineering practice, utilizing \(i\kcap\), where \(i\) is the pseudoscalar, there is no requirement to take the real part to move from the phasor to the observable field representation.

This gives us for the two component fields

\begin{equation}\label{eqn:splitFields}
\begin{aligned}
\BE &= \left( D_\Bk \cos(\omega t) + (\kcap \cross D_\Bk) \sin(\omega t) \right) \cos(\Bk \cdot \Bx) \\
c \BB &= \left( D_\Bk \cos(\omega t) + (\kcap \cross D_\Bk) \sin(\omega t) \right) \sin(\Bk \cdot \Bx)
\end{aligned}
\end{equation}

\subsection{Energy density}

Squaring the phasor \(D_\Bk e^{i\Bomega t}\) allows for a power computation

\begin{equation}\label{eqn:maxwellVacuum_roughNotes:260}
\begin{aligned}
(D_\Bk e^{i\Bomega t} )^2
&=
{D_\Bk}^2 \cos^2(\omega t) + (\kcap \cross D_\Bk)^2 \sin^2(\omega t) + 2 D_\Bk \cdot (\kcap \cross D_\Bk) \sin(\omega t) \cos(\omega t)
\end{aligned}
\end{equation}

Since \(D_\Bk \cdot (\kcap \cross D_\Bk) = 0\) the last term is zero.  For the squared cross product we have

\begin{equation}\label{eqn:maxwellVacuum_roughNotes:280}
\begin{aligned}
(\kcap \cross D_\Bk)^2
&= i^2
(\kcap \wedge D_\Bk) (\kcap \wedge D_\Bk) \\
&=
((\kcap \wedge D_\Bk) \cdot D_\Bk ) \cdot \kcap  \\
&=
((\kcap {D_\Bk}^2 ) \cdot \kcap  \\
&=
\kcap^2 {D_\Bk}^2  \\
&=
{D_\Bk}^2
\end{aligned}
\end{equation}

So we have for the phasor magnitude just our vector constant squared
\begin{equation}\label{eqn:maxwellVacuum_roughNotes:300}
\begin{aligned}
(D_\Bk e^{i\Bomega t} )^2
&=
{D_\Bk}^2
\end{aligned}
\end{equation}

The energy density is thus

\begin{equation}\label{eqn:maxwellVacuum_roughNotes:320}
\begin{aligned}
U
&= \frac{\epsilon_0}{2}(\BE^2 + c^2 \BB^2) \\
&= \frac{\epsilon_0}{2} {D_\Bk}^2 ( \cos^2(\Bk \cdot \Bx) +\sin^2(\Bk \cdot \Bx) )
\end{aligned}
\end{equation}

or
\begin{equation}\label{eqn:maxwellVacuum_roughNotes:340}
\begin{aligned}
U &= \frac{\epsilon_0}{2} {D_\Bk}^2
\end{aligned}
\end{equation}

In retrospect this would have followed much more easily from \(D_\Bk e^{i\Bomega t} = e^{-i\Bomega t} D_\Bk\)

\begin{equation}\label{eqn:maxwellVacuum_roughNotes:360}
\begin{aligned}
(D_\Bk e^{i\Bomega t} )^2
&=
D_\Bk e^{i\Bomega t} e^{-i\Bomega t} D_\Bk \\
&=
D_\Bk D_\Bk \\
\end{aligned}
\end{equation}

\subsection{Is this all correct?}

I expected \(\BE \cdot \BB = 0\), the traditional notion of mutually perpendicular triplet of electric and magnetic and propagation directions.
Instead I have

\begin{equation}\label{eqn:maxwellVacuum_roughNotes:380}
\begin{aligned}
\BE \cdot \BB
&= {D_\Bk}^2 \cos(\Bk \cdot \Bx) \sin(\Bk \cdot \Bx) \\
&= \inv{2} {D_\Bk}^2 \sin( 2 \Bk \cdot \Bx)
\end{aligned}
\end{equation}

This appears to say that for any \(\Bx \perp \Bk\) (where the sine is zero), \(\BE\) is perpendicular to \(\BB\).  I do not quite see how to reconcile this with the fact that by \eqnref{eqn:splitFields} the two fields appear to be colinear

\begin{equation}\label{eqn:questionable}
\begin{aligned}
c \BB = \BE \tan(\Bk \cdot \Bx)
\end{aligned}
\end{equation}

Hmm.  Since the tangent is zero where the sine is also zero, this and \eqnref{eqn:splitFields} both imply that \(\BB\) is zero for all \(\Bx \perp \Bk\)?

Oh!  I see.  The fact that the choice to use a spatial vector \(D_\Bk\) instead of a spatial bivector appears to be the cause of the trouble.  We need both.  With only the spatial vector choice we effectively have chosen only an electric field \(D_\Bk = \BE_\Bk\).  We need a combined field!  Say,

\begin{equation}\label{eqn:maxwellVacuum_roughNotes:400}
\begin{aligned}
D_\Bk \equiv \BE_\Bk + i c \BB_\Bk
\end{aligned}
\end{equation}

\EndArticle
