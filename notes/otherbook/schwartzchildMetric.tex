\documentclass{article}      % Specifies the document class

\usepackage{amsmath}
\usepackage{mathpazo}

%
% shorthand for bold symbols, convenient for vectors and matrices
%
\newcommand{\Ba}[0]{\mathbf{a}}
\newcommand{\Bb}[0]{\mathbf{b}}
\newcommand{\Bc}[0]{\mathbf{c}}
\newcommand{\Bd}[0]{\mathbf{d}}
\newcommand{\Be}[0]{\mathbf{e}}
\newcommand{\Bf}[0]{\mathbf{f}}
\newcommand{\Bg}[0]{\mathbf{g}}
\newcommand{\Bh}[0]{\mathbf{h}}
\newcommand{\Bi}[0]{\mathbf{i}}
\newcommand{\Bj}[0]{\mathbf{j}}
\newcommand{\Bk}[0]{\mathbf{k}}
\newcommand{\Bl}[0]{\mathbf{l}}
\newcommand{\Bm}[0]{\mathbf{m}}
\newcommand{\Bn}[0]{\mathbf{n}}
\newcommand{\Bo}[0]{\mathbf{o}}
\newcommand{\Bp}[0]{\mathbf{p}}
\newcommand{\Bq}[0]{\mathbf{q}}
\newcommand{\Br}[0]{\mathbf{r}}
\newcommand{\Bs}[0]{\mathbf{s}}
\newcommand{\Bt}[0]{\mathbf{t}}
\newcommand{\Bu}[0]{\mathbf{u}}
\newcommand{\Bv}[0]{\mathbf{v}}
\newcommand{\Bw}[0]{\mathbf{w}}
\newcommand{\Bx}[0]{\mathbf{x}}
\newcommand{\By}[0]{\mathbf{y}}
\newcommand{\Bz}[0]{\mathbf{z}}
\newcommand{\BA}[0]{\mathbf{A}}
\newcommand{\BB}[0]{\mathbf{B}}
\newcommand{\BC}[0]{\mathbf{C}}
\newcommand{\BD}[0]{\mathbf{D}}
\newcommand{\BE}[0]{\mathbf{E}}
\newcommand{\BF}[0]{\mathbf{F}}
\newcommand{\BG}[0]{\mathbf{G}}
\newcommand{\BH}[0]{\mathbf{H}}
\newcommand{\BI}[0]{\mathbf{I}}
\newcommand{\BJ}[0]{\mathbf{J}}
\newcommand{\BK}[0]{\mathbf{K}}
\newcommand{\BL}[0]{\mathbf{L}}
\newcommand{\BM}[0]{\mathbf{M}}
\newcommand{\BN}[0]{\mathbf{N}}
\newcommand{\BO}[0]{\mathbf{O}}
\newcommand{\BP}[0]{\mathbf{P}}
\newcommand{\BQ}[0]{\mathbf{Q}}
\newcommand{\BR}[0]{\mathbf{R}}
\newcommand{\BS}[0]{\mathbf{S}}
\newcommand{\BT}[0]{\mathbf{T}}
\newcommand{\BU}[0]{\mathbf{U}}
\newcommand{\BV}[0]{\mathbf{V}}
\newcommand{\BW}[0]{\mathbf{W}}
\newcommand{\BX}[0]{\mathbf{X}}
\newcommand{\BY}[0]{\mathbf{Y}}
\newcommand{\BZ}[0]{\mathbf{Z}}

\newcommand{\Bzero}[0]{\mathbf{0}}
\newcommand{\Btheta}[0]{\boldsymbol{\theta}}
\newcommand{\Btau}[0]{\boldsymbol{\tau}}
\newcommand{\Bomega}[0]{\boldsymbol{\omega}}

%
% shorthand for unit vectors
%
\newcommand{\acap}[0]{\hat{\Ba}}
\newcommand{\bcap}[0]{\hat{\Bb}}
\newcommand{\ccap}[0]{\hat{\Bc}}
\newcommand{\dcap}[0]{\hat{\Bd}}
\newcommand{\ecap}[0]{\hat{\Be}}
\newcommand{\fcap}[0]{\hat{\Bf}}
\newcommand{\gcap}[0]{\hat{\Bg}}
\newcommand{\hcap}[0]{\hat{\Bh}}
\newcommand{\icap}[0]{\hat{\Bi}}
\newcommand{\jcap}[0]{\hat{\Bj}}
\newcommand{\kcap}[0]{\hat{\Bk}}
\newcommand{\lcap}[0]{\hat{\Bl}}
\newcommand{\mcap}[0]{\hat{\Bm}}
\newcommand{\ncap}[0]{\hat{\Bn}}
\newcommand{\ocap}[0]{\hat{\Bo}}
\newcommand{\pcap}[0]{\hat{\Bp}}
\newcommand{\qcap}[0]{\hat{\Bq}}
\newcommand{\rcap}[0]{\hat{\Br}}
\newcommand{\scap}[0]{\hat{\Bs}}
\newcommand{\tcap}[0]{\hat{\Bt}}
\newcommand{\ucap}[0]{\hat{\Bu}}
\newcommand{\vcap}[0]{\hat{\Bv}}
\newcommand{\wcap}[0]{\hat{\Bw}}
\newcommand{\xcap}[0]{\hat{\Bx}}
\newcommand{\ycap}[0]{\hat{\By}}
\newcommand{\zcap}[0]{\hat{\Bz}}
\newcommand{\thetacap}[0]{\hat{\Btheta}}

%
% to write R^n and C^n in a distinguishable fashion.  Perhaps change this
% to the double lined characters upon figuring out how to do so.
%
\newcommand{\C}[1]{$\mathbb{C}^{#1}$}
\newcommand{\R}[1]{$\mathbb{R}^{#1}$}

%
% various generally useful helpers
%

% derivative of #1 wrt. #2:
\newcommand{\D}[2] {\frac {d#2} {d#1}}

\newcommand{\inv}[1]{\frac{1}{#1}}
\newcommand{\cross}[0]{\times}

\newcommand{\abs}[1]{\lvert{#1}\rvert}
\newcommand{\norm}[1]{\lVert{#1}\rVert}
\newcommand{\innerprod}[2]{\langle{#1}, {#2}\rangle}
\newcommand{\dotprod}[2]{{#1} \cdot {#2}}
\newcommand{\bdotprod}[2]{\left({#1} \cdot {#2}\right)}
\newcommand{\crossprod}[2]{{#1} \cross {#2}}
\newcommand{\tripleprod}[3]{\dotprod{\left(\crossprod{#1}{#2}\right)}{#3}}

\DeclareMathOperator{\Proj}{Proj}
\DeclareMathOperator{\Span}{span}
\DeclareMathOperator{\Sgn}{sgn}
\DeclareMathOperator{\Area}{Area}
\DeclareMathOperator{\Volume}{Volume}

%
% A few miscellaneous things specific to this document
%
\newcommand{\crossop}[1]{\crossprod{#1}{}}

% R2 vector.
\newcommand{\VectorTwo}[2]{
\begin{bmatrix}
 {#1} \\
 {#2}
\end{bmatrix}
}

\newcommand{\VectorN}[1]{
\begin{bmatrix}
{#1}_1 \\
{#1}_2 \\
\vdots \\
{#1}_N \\
\end{bmatrix}
}

\newcommand{\DETuvij}[4]{
\begin{vmatrix}
 {#1}_{#3} & {#1}_{#4} \\
 {#2}_{#3} & {#2}_{#4}
\end{vmatrix}
}

\newcommand{\DETuvwijk}[6]{
\begin{vmatrix}
 {#1}_{#4} & {#1}_{#5} & {#1}_{#6} \\
 {#2}_{#4} & {#2}_{#5} & {#2}_{#6} \\
 {#3}_{#4} & {#3}_{#5} & {#3}_{#6}
\end{vmatrix}
}

\newcommand{\DETuvwxijkl}[8]{
\begin{vmatrix}
 {#1}_{#5} & {#1}_{#6} & {#1}_{#7} & {#1}_{#8} \\
 {#2}_{#5} & {#2}_{#6} & {#2}_{#7} & {#2}_{#8} \\
 {#3}_{#5} & {#3}_{#6} & {#3}_{#7} & {#3}_{#8} \\
 {#4}_{#5} & {#4}_{#6} & {#4}_{#7} & {#4}_{#8} \\
\end{vmatrix}
}

%\newcommand{\DETuvwxyijklm}[10]{
%\begin{vmatrix}
% {#1}_{#6} & {#1}_{#7} & {#1}_{#8} & {#1}_{#9} & {#1}_{#10} \\
% {#2}_{#6} & {#2}_{#7} & {#2}_{#8} & {#2}_{#9} & {#2}_{#10} \\
% {#3}_{#6} & {#3}_{#7} & {#3}_{#8} & {#3}_{#9} & {#3}_{#10} \\
% {#4}_{#6} & {#4}_{#7} & {#4}_{#8} & {#4}_{#9} & {#4}_{#10} \\
% {#5}_{#6} & {#5}_{#7} & {#5}_{#8} & {#5}_{#9} & {#5}_{#10}
%\end{vmatrix}
%}

% R3 vector.
\newcommand{\VectorThree}[3]{
\begin{bmatrix}
 {#1} \\
 {#2} \\
 {#3}
\end{bmatrix}
}


\newcommand{\grad}[0]{\nabla}
\newcommand{\spacegrad}[0]{\boldsymbol{\nabla}}
\newcommand{\LL}[0]{\mathcal{L}}
\newcommand{\PD}[2]{\frac{\partial {#2}}{\partial {#1}}}
\newcommand{\PDsq}[2]{\frac{\partial^2 {#2}}{\partial^2 {#1}}}
\newcommand{\dotalpha}[0]{\dot{\alpha}}
\newcommand{\ddotalpha}[0]{\ddot{\alpha}}

\newcommand{\dotomega}[0]{\dot{\omega}}
\newcommand{\ddotomega}[0]{\ddot{\omega}}

\newcommand{\dotOmega}[0]{\dot{\Omega}}
\newcommand{\ddotOmega}[0]{\ddot{\Omega}}

\newcommand{\CC}[0]{c^2}

\newcommand{\dottheta}[0]{\dot{\theta}}
\newcommand{\ddottheta}[0]{\ddot{\theta}}

\newcommand{\dotpsi}[0]{\dot{\psi}}
\newcommand{\ddotpsi}[0]{\ddot{\psi}}

\newcommand{\adot}[0]{\dot{a}}
\newcommand{\addot}[0]{\ddot{a}}
\newcommand{\bdot}[0]{\dot{b}}
\newcommand{\bddot}[0]{\ddot{b}}
\newcommand{\qdot}[0]{\dot{q}}
\newcommand{\qddot}[0]{\ddot{q}}
\newcommand{\tdot}[0]{\dot{t}}
\newcommand{\tddot}[0]{\ddot{t}}

\newcommand{\Rdot}[0]{\dot{R}}

\newcommand{\pdot}[0]{\dot{p}}
\newcommand{\pddot}[0]{\ddot{p}}

\newcommand{\xdot}[0]{\dot{x}}
\newcommand{\xddot}[0]{\ddot{x}}

\newcommand{\zdot}[0]{\dot{z}}
\newcommand{\zddot}[0]{\ddot{z}}

\newcommand{\rdot}[0]{\dot{r}}
\newcommand{\rddot}[0]{\ddot{r}}

%
% The real thing:
%

\usepackage[bookmarks=true]{hyperref}

                             % The preamble begins here.
\title{ Some GR Notes. } % Declares the document's title.
\author{Peeter Joot}         % Declares the author's name.
\date{ October 2, 2008.  Last Revision: $Date: 2008/10/04 17:06:40 $ } % Deleting this command produces today's date.

\begin{document}             % End of preamble and beginning of text.

\maketitle{}

\tableofcontents

\section{ Motivation. }

Some General relativity notes exploring ideas from emails with Lut Mentz.
Prior to this I was under the impression that I had zero knowledge of GR,
but it turns out that many of the ideas are really action based. 
Allowing the spacetime unit vectors to vary, something we are free to
do in SR or newtonian mechanics too, results in a more 
general metric in a purely kinetic Lagrangian.  This metric variation
can be interpretted as a mechanism for introducing more general
accererations very similar to fictious forces that one sees in a
rotating frame or other non-uniform coordinate system.

These notes contain my attempt to walk through some of these ideas, to see
if I can coherently explain them to myself.  If I can't do so then I don't
understand things sufficently.  Being able to produce such an explaination
may not mean that I truely understand the issues, but it is a required
first step.

Useful references are Lut's writeup \cite{lutSchwarzChildRadial}, 
and the schwarzchild calculation \cite{mathpagesSchwarzChildRadial}
from the online text reflections on relativity.

FIXME: Ask Lut if he put his Rindler derivation online.

FIXME: Original PF thread and email with Lut was associated with 
a Lagragian where mass was allowed to vary as described in \cite{PJMassVary}.  He was
investigating the similarities between varying mass directly and the spatial/metric
variation of GR.

\subsection{ Lagrangian for General Relativity. }

The equations of motion resulting from a purely kinetic Lagrangian

\begin{equation}\label{eqn:keLagrangian}
\LL = \inv{2} \sum  g_{b c}(q^a) \qdot^b \qdot^c,
\end{equation}

can be found to be

\begin{align}
0
&= \qddot^a + \qdot^b \qdot^c {\Gamma^a}_{b c} \\
{\Gamma^a}_{b c} &= \inv{2} g^{a d} \left( 
\PD{q^b}{g_{c d}} 
+ \PD{q^c}{g_{d b}} 
- \PD{q^d}{g_{b c}} 
\right)
\end{align}

One such derivation can be found in
the solution of problem one \cite{PJTongMf1}, associated with the 
Lagrangian problem set for
Dr. David Tong's online mechanics text \cite{TongDynamics}.

This is the Lagrangian for general relativity, once the metric tensor $g_{a b}$ is specified.

\subsection{ General kinetic Lagrangian for fixed frame basis. }

One doesn't have to go to GR to find Kinetic energy expressions of the form in equation \ref{eqn:keLagrangian}.

A simple example of a more general Kinetic energy description can be found by any use of non-orthonormal basis vectors, say $\{\Be_i\}$,
for the space.

Given such a non-orthonormal frame, the trick to calculating the coordinates is tied to an alternate set of basis vectors, called the reciprocal frame.
Provided the initial set of vectors spans the space, one can always calculate this second pair such that they meet the following relationships:

\begin{equation*}
\Be^i \cdot \Be_j = {\delta^i}_j
\end{equation*}

Calculating these reciprocal frame vectors is a linear algebra problem, essentially requiring a matrix inversion.  Here we just assume they can be calculated and use them as a convienent way to find the coordinates in this non-orthogonal frame

\begin{align*}
\Bx &= \sum \Be_j a_j \\
\Bx \cdot \Be^i &= \sum (\Be_j a_j) \cdot \Be^i = \sum {\delta^j}_i a_j = a_i \\
\implies \\
\Bx &= \sum \Be_j (\Bx \cdot \Be^j)
\end{align*}

It is customary to write $a_i = \Bx \cdot \Be^i = x^i$, in order to have
mixed upper and lower indexes for implied summation.

\begin{equation*}
\Bx = \sum \Be_j x^j = \Be_j x^j
\end{equation*}

Once one has a way of calculating coordinates for an arbitrary basis, other quantities such as velocity
can then be calculated

\begin{equation*}
\Bv^2 = \dot{\Bx} \cdot \dot{\Bx} = (\Be_j \cdot \Be_k) \xdot^j \xdot^k
\end{equation*}

This
$\Be_j \cdot \Be_k$ coefficient of the coordinates
gets a special name, the metric tensor
$g_{j k} = \Be_j \cdot \Be_k$.  It is a symmetric and invertable quantity and can be employed to 
express the Kinetic Energy term of a single particle Lagrangian in the general tensor form 

\begin{equation*}
\inv{2} m \Bv^2 = \inv{2} m g_{j k} \xdot^j \xdot^k
\end{equation*}

Now, in this case $g_{j k}$ is not a function of the coordinates (ie: of position) as in
equation \ref{eqn:keLagrangian}.

\subsection{ General kinetic Lagrangian for a basis frame. }

Now, in the GR case the metric varies (or may) with position.  What do we need to observe this same form in Newtonian
physics?  The immediate thing that comes to mind is the use of a curvalinear basis, where the basis vectors
for a space are allowed to vary direction with position along some path.  Initially that seemed reasonable to
me but for some arbitrary parameterized path, wouldn't the metric then also vary with the path parameterization?  If that
was the case, the Lagrangian in equation \ref{eqn:keLagrangian} does not have the form of a kinetic energy expression.

To resolve this I considered an example.  I can lay out two directions in my backyard, one along the vegetable garden
parallel to the house roughly pointing north, and another diagonally across to my gate.  This logically defines a coordinate
system or set of frame vectors that I can make local measurements with respect to.
Now, translation of this coordinate system
to my dad's house 40 km to the south won't be a particularily logical for measuring there.  He lives on a very 
steep hill.

I can pace out distances in my backyard without having to consider the curvature
of the Earth and my dad can do the same for a long stretch of the hill walking up the street towards his house.
The local frame vectors can be considered to lie along a flat surface if that surface area is small enough.
Alternately we can say the associated metric associated with a surface coordinate system for a point on surface of the Earth can be considered constant for small enough measurements.

Now, to express the same ideas mathematically, consider a curve expressed parametrically between two points, such that all
points along the path take the values

\begin{equation*}
\Bx(\lambda) = x^i(\lambda) \Be_i(\Bx)
\end{equation*}

The vector distance between two points on this path is

\begin{equation*}
\Bx(\lambda_2) - \Bx(\lambda_1) = x^i(\lambda_2) \Be_i(\Bx(\lambda_2)) - x^i(\lambda_1) \Be_i(\Bx(\lambda_1))
\end{equation*}

but this is the direct difference in position between these two points, not the distance along the curve.
To be a true measure of the distance the difference in position has to also be small enough that the frame vectors lie in the same direction at both points to some approximation.

Given such an approximation one can then write
\begin{align*}
\Bx(\lambda_2) - \Bx(\lambda_1) &= \left(x^i(\lambda_2) - x^i(\lambda_1)\right) \Be_i(\Bx) \\
d\Bx &= \frac{d x^i}{d\lambda} \Be_i(\Bx) d\lambda \\
\end{align*}

For such a representation to be valid, the variation of $\Be_i$ at the point $\Bx$ has to be small enough that $\Be_i$ can be 
considered constant.  This is still not a very well defined statement mathematically, and it is not too hard to imagine 
scenerios where it totally fails.  An example is a fractal like curve, something continuous but not differtiable at
any point.

Assuming a sufficently differentiable curve then the distance along the curve between two points can be obtained from the
integral

\begin{align*}
ds
&= \int_{\lambda_1}^{\lambda_2} \sqrt{\left(\frac{d\Bx}{d\lambda}\right)^2} d\lambda \\
&= \int_{\lambda_1}^{\lambda_2} \sqrt{ g_{ i j } \frac{dx^i}{d\lambda} \frac{dx^j}{d\lambda} } d\lambda \\
\end{align*}

Or
\begin{align*}
\left(\frac{ds}{d\lambda}\right)^2 = g_{ i j } \frac{dx^i}{d\lambda} \frac{dx^j}{d\lambda}
\end{align*}

Now, what is the most natural parameterization?  For physical situations time comes to mind, but if the particle stops for a
while on the path, then this derivative goes to zero for a while even if the curve is continuous and has derivatives of all
orders at all points.  Falling back to the most simple curve as a motivator, the circle, 
use of fractions $\theta$ of the total circumference of the circle $2\pi$ naturally parameterizes points on the curve.  The same thing can be done for any
curve in Euclidean space, using arc length to parameterize a path.  In terms
of the metric tensor this is

\begin{align*}
\left(\frac{ds}{ds}\right)^2 = 1 = g_{ i j } \frac{dx^i}{ds} \frac{dx^j}{ds}
\end{align*}

Introducing proper time for spacetime event path parameterization will 
be related to these spatial ideas.  Before considering proper time some 
thought about general metrics for spacetime seems to be in order.

\subsection{ General metrics for spacetime. }

It was stated above that equation \ref{eqn:keLagrangian} was the 
Lagrangian for general relativity, once the metric is specified.

It is not
at all obvious to me what physically motivates the choice of metric.
In the two examples that Lut has given me, the Rindler metric and the
Schwartzchild metric, this appears to be rather arbitrary.

\subsubsection{ Minkowski Metric. }

Picking the
Minkowski metric for special relativity is not at all an obvious 
choice.  An introductory physics book like \cite{lewis1965mbp} chooses
to introduce the idea indirectly by requiring that a coordinate transformation
used to express the equation for a spherically expanding light shell

\begin{align}\label{eqn:shell}
\sum_i (x^i)^2 - c^2 t^2 = \sum_i ({x^i}')^2 - c^2 {t'}^2
\end{align}

is identical in any spacetime coordinate system.  This expresses the idea that the speed
of light is the same in any frame of reference and that both time and distance
are to be measured only locally.  The Lorentz transformation can then be derived from
this notion of spherical shell equation invariance, and the Minkowski mixed signature
metric can be observed to fundamentally describe both the Lorentz transformation and the
shell invariance.

First reading this it was not at all obvious to me that \ref{eqn:shell} was a reasonable starting
point.  The constancy of the speed of light is easy to say, but saying it doesn't mean
that the implications are understood.  I'd bet that many of the people who make such a statement
would have trouble reconsiling this statement, if given the standard headlights on a rocket ship
travelling at $c/2$.
This non-obvious nature is likely reflected in the degree of popular disbelief of relativity in the general non-physist population.
Some of that is likely due to misunderstanding and because of the fact that scenerios like the twin paradox are so popular in fiction.
I would guess that it is likely ``obvious'' to many of the same people who would perfectly who way the speed of light is constant
in a perfectly agreeable fashion to also state that an observer who sees the spaceship
moving would see the speed of the light from the headlights going at $1.5 c$ and not detect a conflict between the two ideas.

My personal justification for the Minkowski metric comes from consideration of the wave
nature of electromagnetism \cite{PJLorentzWave}.  A bit of study of electromagnetism shows that Maxwell's equations
for electric and magnetic fields requires that

\begin{align*}
\left(-\spacegrad^2 + \inv{c^2}\PDsq{t}{}\right) \BE &= 0 \\
\left(-\spacegrad^2 + \inv{c^2}\PDsq{t}{}\right) \BB &= 0
\end{align*}

Electromagnetic radiation (light) appears to be a fundamentally wavelike phenomina.  That is an idea that we are all fairly
comfortable with.  Mathematically and mechanically introduce an arbitrary change of variables $x^{\mu} \rightarrow {x'}^{\mu}$,
then the chain rule plus a requirement that these equations still describe a wave equation logically in a second coordinate system
imposes a Lorentz transformation constraint on the coordinate transformation.  

The approach of using the wave equation as a motivator requires some calculus, and thus still isn't something that
is good for a Layman's justification of the
Lorentz transformation and the corresponding Minkowski metric.  Einstein himself actually has a nice Layman's treatment of
special relativity in his book \cite{einstein2005rsa}, which is also worth referring to.

Also related to the wave equation one can justify the Minkowski metric by factoring the scalar wave equation (Delambertian) operator

\begin{align*}
-\spacegrad^2 + \inv{c^2}\PDsq{t}{} = -\sum_i \PDsq{x^{i}}{} +\inv{c^2}\PDsq{t}{}
\end{align*}

into a vector product.  If one writes the spacetime vector (with $x^0 = ct$) as

\begin{align*}
x = \gamma_{\mu} x^{\mu}
\end{align*}

A spacetime gradient operator can be defined that squares to $\pm 1$ times the wave equation operator

\begin{align*}
\grad &= \gamma^{\mu} \PD{x^{\mu}}{} \\
\end{align*}

For the square of this operator to equal the wave equation operator we need two conditions

\begin{align*}
\gamma^{\mu} \cdot \gamma^{\nu} &= \delta^{\mu\nu} {\left(\gamma^{\mu}\right)}^2 \\
(\gamma^0)^2 (\gamma^i)^2 &= -1.
\end{align*}

With these we can then write the wave equation operator as

\begin{align*}
-\spacegrad^2 + \inv{c^2}\PDsq{t}{} = \inv{(\gamma^0)^2} \grad^2
\end{align*}

There are an infinite number of ways to factor this scalar equation into a vector product, depending on the spacetime 
coordinate systems used, but all of them vary by a Lorentz transformation!

Requiring that an orthonormal spacetime basis has a Minkowski metric signature $(\gamma^0)^2 (\gamma^i)^2 = -1$ has value
independent of any mechanics and relativistic consideration.  If nothing else the fact that this can be utilized to
summarize Maxwell's equations as the clifford algebra product \cite{doran2003gap}

\begin{align*}
\grad F &= J/c \epsilon_0 \\
\end{align*}

Here $\grad$ defined as above, and we have a bivector for the field, a vector for the current and charge density, and the 
four space pseudoscalar ties the electric and magnetic field components together into a single complex number like quanity:

\begin{align*}
F &= \BE + I c \BB \\
J &= c \rho \gamma_0 + J^i \gamma_i \\
I &= \gamma_0 \wedge \gamma_1 \wedge \gamma_2 \wedge \gamma_3 \\
\end{align*}

The spatial vectors we are used to calculating with are expressed as spacetime bivectors
\begin{align*}
\sigma_i &= \gamma_i \wedge \gamma_0 \\
\sigma_i \cdot \sigma_j &= \delta_{ij} \\
\BE &= E^i \sigma_i \\
\BB &= B^i \sigma_i \\
\BJ &= J^i \sigma_i,
\end{align*}

where the bivector basis $\{\sigma_i\}$ for the spacetime split behave in all respects like Euclidean vectors.

If nothing else the Minkowski metric idea that is at the root of all of this has got visible value as it brings together all
of electromagnetism under a single umbrella.

\subsubsection{ Schwartzchild Metric. } 

\begin{align}
\CC (d\tau)^2 &= -\CC a(r) (dt)^2 + {b(r)} (dr)^2 + r^2(d\Omega)^2 \\
a(r) &= 1 - \kappa/r \\
b(r) &= \inv{a(r)} \\
\PD{r}{a} &= \kappa/r^2 \\
\PD{r}{b} &= -\kappa/(r-k)^2 \\
\adot &= \frac{\kappa \rdot}{r^2} \\
\bdot &= \frac{-\kappa \rdot}{(r-k)^2}
\end{align}

\subsubsection{ Rindler Metric. }

Lut described the 

\section{ EOMs. }

%Work through the Euler-Lagrange equations for what Lut calls the 
%Schwartzchild metric.
%
%\subsection{ Affine Coordinates. }


%I would interpret the affine coordinates this way.  If we have two ``close'' events described by our $t,r,\Omega$ coordinates, then the path and corresponding
%element of arc length between two events can be described parameterically
%with an arbitrary parameter $\lambda$
%
%\begin{align*}
%t &= t(\lambda) \\
%r &= r(\lambda) \\
%\Omega &= \Omega(\lambda)
%\end{align*}
%
%An element of event arc length will then be
%
%\begin{align*}
%ds^2 = \left(\frac{ds}{d\lambda}\right)^2 {d\lambda}^2
%&=
%\left( -\CC a \left(\frac{dt}{d\lambda}\right)^2 + {b} \left(\frac{dr}{d\lambda}\right)^2 + r^2\left(\frac{d\Omega}{d\lambda}\right)^2 \right) {d\lambda}^2
%\end{align*}
%
%Introducing such a natural parameterization with units of time means that we need a dimensional fudge factor, so we write $ds = c d\tau$, so that $\tau$ is the unique parameterization for which $(ds/d\tau)^2 = c^2$.
%This implicitly defines the ``proper time'', and we can write our Lagrangian to minimize in terms of that variable
%
%\begin{align*}%\label{eqn:action}
%\LL
%&= \left(\frac{ds}{d\tau}\right)^2 \\
%&= c^2 \\
%&= \left( -\CC a \left(\frac{dt}{d\tau}\right)^2 + {b} \left(\frac{dr}{d\tau}\right)^2 + r^2\left(\frac{d\Omega}{d\tau}\right)^2 \right) \\
%&= -\CC a \tdot^2 + {b} \rdot^2 + r^2\dotOmega^2 \\
%\end{align*}
%
%Derivatives for $\Omega$
%
%\begin{align*}
%\PD{\Omega}{\LL} &= \left(\PD{\dotOmega}{\LL}\right)' \\
%0 &= (2 r^2 \dotOmega)' \\
%\end{align*}
%
%Observing that $\Omega$ is cyclic, we can write for some constant $A$
%
%\begin{align*}
%\dotOmega = \frac{A}{r^2}
%\end{align*}
%
%From \ref{eqn:action} this provides a relationship between $\rdot$ and $\tdot$
%
%\begin{align*}
%0 = -\CC (a \tdot^2 + 1) + \inv{a} \rdot^2 + A^2/r^2
%\end{align*}
%
%Calculating the derivatives for $t$
%
%\begin{align*}
%\PD{t}{\LL} &= \left(\PD{\tdot}{\LL}\right)' \\
%0 
%&= \left( - 2 \CC a \tdot \right)' \\
%%&= - 2 \CC \adot \tdot - 2 \CC a \tddot \\
%%\implies \\
%%\tddot &= \frac{\kappa \rdot \tdot}{a r^2} \\
%\end{align*}
%
%Again observing that $t$ is cyclic, we can write for some constant $B$
%
%\begin{align*}
%\tdot = B/a
%\end{align*}
%
%Or
%\begin{align*}
%\rdot^2 = \CC ({B^2} + a) - a A^2/r^2
%\end{align*}
%
%Last the derivatives for $r$
%
%%\LL = \CC  = -\CC a \tdot^2 + \inv{a} \rdot^2 + r^2\dotOmega^2
%\begin{align*}
%\PD{r}{\LL} &= \left(\PD{\rdot}{\LL}\right)' \\
%-\CC \PD{r}{a} \tdot^2 - \frac{\adot \rdot^2}{a^2} + 2 r \dotOmega^2 
%&= \left(\frac{2 \rdot}{a}\right)' \\
%&= -2 \frac{ \adot \rdot }{a^2} + \frac{2 \rddot}{a} \\ 
%\implies \\
%{\rddot} 
%&= - \inv{2} \CC \frac{\kappa}{r^2} a \tdot^2 -  \frac{\adot \rdot^2}{2 a} + r a \dotOmega^2 + \frac{ \adot \rdot }{a} \\
%&= - \inv{2} \CC \frac{\kappa B^2}{a r^2} -  \frac{\kappa \rdot^3}{2 a r^2} + a \frac{A^2}{r^3} + \frac{ \kappa \rdot^2 }{a r^2} \\
%\end{align*}
%
%The $\rdot^3$ term doesn't look right dimensionally.

%\bibliographystyle{plain}
\bibliographystyle{plainnat} % supposed to allow for \url use.
\bibliography{myrefs}      % expects file "myrefs.bib"

\end{document}               % End of document.
