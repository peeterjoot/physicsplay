\documentclass{article}      % Specifies the document class

\usepackage{amsmath}
\usepackage{mathpazo}

%
% shorthand for bold symbols, convenient for vectors and matrices
%
\newcommand{\Ba}[0]{\mathbf{a}}
\newcommand{\Bb}[0]{\mathbf{b}}
\newcommand{\Bc}[0]{\mathbf{c}}
\newcommand{\Bd}[0]{\mathbf{d}}
\newcommand{\Be}[0]{\mathbf{e}}
\newcommand{\Bf}[0]{\mathbf{f}}
\newcommand{\Bg}[0]{\mathbf{g}}
\newcommand{\Bh}[0]{\mathbf{h}}
\newcommand{\Bi}[0]{\mathbf{i}}
\newcommand{\Bj}[0]{\mathbf{j}}
\newcommand{\Bk}[0]{\mathbf{k}}
\newcommand{\Bl}[0]{\mathbf{l}}
\newcommand{\Bm}[0]{\mathbf{m}}
\newcommand{\Bn}[0]{\mathbf{n}}
\newcommand{\Bo}[0]{\mathbf{o}}
\newcommand{\Bp}[0]{\mathbf{p}}
\newcommand{\Bq}[0]{\mathbf{q}}
\newcommand{\Br}[0]{\mathbf{r}}
\newcommand{\Bs}[0]{\mathbf{s}}
\newcommand{\Bt}[0]{\mathbf{t}}
\newcommand{\Bu}[0]{\mathbf{u}}
\newcommand{\Bv}[0]{\mathbf{v}}
\newcommand{\Bw}[0]{\mathbf{w}}
\newcommand{\Bx}[0]{\mathbf{x}}
\newcommand{\By}[0]{\mathbf{y}}
\newcommand{\Bz}[0]{\mathbf{z}}
\newcommand{\BA}[0]{\mathbf{A}}
\newcommand{\BB}[0]{\mathbf{B}}
\newcommand{\BC}[0]{\mathbf{C}}
\newcommand{\BD}[0]{\mathbf{D}}
\newcommand{\BE}[0]{\mathbf{E}}
\newcommand{\BF}[0]{\mathbf{F}}
\newcommand{\BG}[0]{\mathbf{G}}
\newcommand{\BH}[0]{\mathbf{H}}
\newcommand{\BI}[0]{\mathbf{I}}
\newcommand{\BJ}[0]{\mathbf{J}}
\newcommand{\BK}[0]{\mathbf{K}}
\newcommand{\BL}[0]{\mathbf{L}}
\newcommand{\BM}[0]{\mathbf{M}}
\newcommand{\BN}[0]{\mathbf{N}}
\newcommand{\BO}[0]{\mathbf{O}}
\newcommand{\BP}[0]{\mathbf{P}}
\newcommand{\BQ}[0]{\mathbf{Q}}
\newcommand{\BR}[0]{\mathbf{R}}
\newcommand{\BS}[0]{\mathbf{S}}
\newcommand{\BT}[0]{\mathbf{T}}
\newcommand{\BU}[0]{\mathbf{U}}
\newcommand{\BV}[0]{\mathbf{V}}
\newcommand{\BW}[0]{\mathbf{W}}
\newcommand{\BX}[0]{\mathbf{X}}
\newcommand{\BY}[0]{\mathbf{Y}}
\newcommand{\BZ}[0]{\mathbf{Z}}

\newcommand{\Bzero}[0]{\mathbf{0}}
\newcommand{\Btheta}[0]{\boldsymbol{\theta}}
\newcommand{\Btau}[0]{\boldsymbol{\tau}}
\newcommand{\Bomega}[0]{\boldsymbol{\omega}}

%
% shorthand for unit vectors
%
\newcommand{\acap}[0]{\hat{\Ba}}
\newcommand{\bcap}[0]{\hat{\Bb}}
\newcommand{\ccap}[0]{\hat{\Bc}}
\newcommand{\dcap}[0]{\hat{\Bd}}
\newcommand{\ecap}[0]{\hat{\Be}}
\newcommand{\fcap}[0]{\hat{\Bf}}
\newcommand{\gcap}[0]{\hat{\Bg}}
\newcommand{\hcap}[0]{\hat{\Bh}}
\newcommand{\icap}[0]{\hat{\Bi}}
\newcommand{\jcap}[0]{\hat{\Bj}}
\newcommand{\kcap}[0]{\hat{\Bk}}
\newcommand{\lcap}[0]{\hat{\Bl}}
\newcommand{\mcap}[0]{\hat{\Bm}}
\newcommand{\ncap}[0]{\hat{\Bn}}
\newcommand{\ocap}[0]{\hat{\Bo}}
\newcommand{\pcap}[0]{\hat{\Bp}}
\newcommand{\qcap}[0]{\hat{\Bq}}
\newcommand{\rcap}[0]{\hat{\Br}}
\newcommand{\scap}[0]{\hat{\Bs}}
\newcommand{\tcap}[0]{\hat{\Bt}}
\newcommand{\ucap}[0]{\hat{\Bu}}
\newcommand{\vcap}[0]{\hat{\Bv}}
\newcommand{\wcap}[0]{\hat{\Bw}}
\newcommand{\xcap}[0]{\hat{\Bx}}
\newcommand{\ycap}[0]{\hat{\By}}
\newcommand{\zcap}[0]{\hat{\Bz}}
\newcommand{\thetacap}[0]{\hat{\Btheta}}

%
% to write R^n and C^n in a distinguishable fashion.  Perhaps change this
% to the double lined characters upon figuring out how to do so.
%
\newcommand{\C}[1]{$\mathbb{C}^{#1}$}
\newcommand{\R}[1]{$\mathbb{R}^{#1}$}

%
% various generally useful helpers
%

% derivative of #1 wrt. #2:
\newcommand{\D}[2] {\frac {d#2} {d#1}}

\newcommand{\inv}[1]{\frac{1}{#1}}
\newcommand{\cross}[0]{\times}

\newcommand{\abs}[1]{\lvert{#1}\rvert}
\newcommand{\norm}[1]{\lVert{#1}\rVert}
\newcommand{\innerprod}[2]{\langle{#1}, {#2}\rangle}
\newcommand{\dotprod}[2]{{#1} \cdot {#2}}
\newcommand{\bdotprod}[2]{\left({#1} \cdot {#2}\right)}
\newcommand{\crossprod}[2]{{#1} \cross {#2}}
\newcommand{\tripleprod}[3]{\dotprod{\left(\crossprod{#1}{#2}\right)}{#3}}

\DeclareMathOperator{\Proj}{Proj}
\DeclareMathOperator{\Span}{span}
\DeclareMathOperator{\Sgn}{sgn}
\DeclareMathOperator{\Area}{Area}
\DeclareMathOperator{\Volume}{Volume}

%
% A few miscellaneous things specific to this document
%
\newcommand{\crossop}[1]{\crossprod{#1}{}}

% R2 vector.
\newcommand{\VectorTwo}[2]{
\begin{bmatrix}
 {#1} \\
 {#2}
\end{bmatrix}
}

\newcommand{\VectorN}[1]{
\begin{bmatrix}
{#1}_1 \\
{#1}_2 \\
\vdots \\
{#1}_N \\
\end{bmatrix}
}

\newcommand{\DETuvij}[4]{
\begin{vmatrix}
 {#1}_{#3} & {#1}_{#4} \\
 {#2}_{#3} & {#2}_{#4}
\end{vmatrix}
}

\newcommand{\DETuvwijk}[6]{
\begin{vmatrix}
 {#1}_{#4} & {#1}_{#5} & {#1}_{#6} \\
 {#2}_{#4} & {#2}_{#5} & {#2}_{#6} \\
 {#3}_{#4} & {#3}_{#5} & {#3}_{#6}
\end{vmatrix}
}

\newcommand{\DETuvwxijkl}[8]{
\begin{vmatrix}
 {#1}_{#5} & {#1}_{#6} & {#1}_{#7} & {#1}_{#8} \\
 {#2}_{#5} & {#2}_{#6} & {#2}_{#7} & {#2}_{#8} \\
 {#3}_{#5} & {#3}_{#6} & {#3}_{#7} & {#3}_{#8} \\
 {#4}_{#5} & {#4}_{#6} & {#4}_{#7} & {#4}_{#8} \\
\end{vmatrix}
}

%\newcommand{\DETuvwxyijklm}[10]{
%\begin{vmatrix}
% {#1}_{#6} & {#1}_{#7} & {#1}_{#8} & {#1}_{#9} & {#1}_{#10} \\
% {#2}_{#6} & {#2}_{#7} & {#2}_{#8} & {#2}_{#9} & {#2}_{#10} \\
% {#3}_{#6} & {#3}_{#7} & {#3}_{#8} & {#3}_{#9} & {#3}_{#10} \\
% {#4}_{#6} & {#4}_{#7} & {#4}_{#8} & {#4}_{#9} & {#4}_{#10} \\
% {#5}_{#6} & {#5}_{#7} & {#5}_{#8} & {#5}_{#9} & {#5}_{#10}
%\end{vmatrix}
%}

% R3 vector.
\newcommand{\VectorThree}[3]{
\begin{bmatrix}
 {#1} \\
 {#2} \\
 {#3}
\end{bmatrix}
}


\newcommand{\LL}[0]{\mathcal{L}}
\newcommand{\PD}[2]{\frac{\partial {#2}}{\partial {#1}}}
\newcommand{\dotalpha}[0]{\dot{\alpha}}
\newcommand{\ddotalpha}[0]{\ddot{\alpha}}

\newcommand{\dotomega}[0]{\dot{\omega}}
\newcommand{\ddotomega}[0]{\ddot{\omega}}

\newcommand{\dotOmega}[0]{\dot{\Omega}}
\newcommand{\ddotOmega}[0]{\ddot{\Omega}}

\newcommand{\CC}[0]{c^2}

\newcommand{\dottheta}[0]{\dot{\theta}}
\newcommand{\ddottheta}[0]{\ddot{\theta}}

\newcommand{\dotpsi}[0]{\dot{\psi}}
\newcommand{\ddotpsi}[0]{\ddot{\psi}}

\newcommand{\adot}[0]{\dot{a}}
\newcommand{\addot}[0]{\ddot{a}}
\newcommand{\qdot}[0]{\dot{q}}
\newcommand{\qddot}[0]{\ddot{q}}
\newcommand{\tdot}[0]{\dot{t}}
\newcommand{\tddot}[0]{\ddot{t}}

\newcommand{\Rdot}[0]{\dot{R}}

\newcommand{\pdot}[0]{\dot{p}}
\newcommand{\pddot}[0]{\ddot{p}}

\newcommand{\xdot}[0]{\dot{x}}
\newcommand{\xddot}[0]{\ddot{x}}

\newcommand{\zdot}[0]{\dot{z}}
\newcommand{\zddot}[0]{\ddot{z}}

\newcommand{\rdot}[0]{\dot{r}}
\newcommand{\rddot}[0]{\ddot{r}}

%
% The real thing:
%

\usepackage[bookmarks=true]{hyperref}

                             % The preamble begins here.
\title{ Some GR Notes. } % Declares the document's title.
\author{Peeter Joot}         % Declares the author's name.
\date{ October 2, 2008.  Last Revision: $Date: 2008/10/03 14:38:45 $ } % Deleting this command produces today's date.

\begin{document}             % End of preamble and beginning of text.

\maketitle{}

\tableofcontents

\section{ Motivation. }

Some General relativity notes exploring ideas from emails with Lut Mentz.
Prior to this I was under the impression that I had zero knowledge of GR,
but it turns out that many of the ideas are really action based. 
Allowing the spacetime unit vectors to vary, something we are free to
do in SR or newtonian mechanics too, results in a more 
general metric in a purely kinetic Lagrangian.  This metric variation
can be interpretted as a mechanism for introducing more general
accererations very similar to fictious forces that one sees in a
rotating frame or other non-uniform coordinate system.

These notes contain my attempt to walk through some of these ideas, to see
if I can coherently explain them to myself.  If I can't do so then I don't
understand things sufficently.  Being able to produce such an explaination
may not mean that I truely understand the issues, but it is a required
first step.

Useful references are Lut's writeup \cite{lutSchwarzChildRadial}, 
and the schwarzchild calculation \cite{mathpagesSchwarzChildRadial}
from the online text reflections on relativity.

FIXME: Ask Lut if he put his Rindler derivation online.

Also associated with this email/PF thread is consideration
of Lagragian mass variation as described in \cite{PJMassVary}, and
the solution of problem one \cite{PJTongMf1}, equations of motion for
a general kinetic Lagragian, from Dr. David Tong's online mechanics text
\cite{TongDynamics}.

\section{ Rindler Metric. }

\section{ Schwartzchild Metric. } 

Work through the Euler-Lagrange equations for what Lut calls the 
Schwartzchild metric.

\subsection{ Affine Coordinates. }

%\begin{align}
%\CC (d\tau)^2 &= -\CC a(r) (dt)^2 + {b(r)} (dr)^2 + r^2(d\Omega)^2 \\
%a(r) &= 1 - \kappa/r \\
%b(r) &= \inv{a(r)} \\
%\PD{r}{a} &= \kappa/r^2 \\
%\PD{r}{b} &= -\kappa/(r-k)^2 \\
%\adot &= \frac{\kappa \rdot}{r^2}
%\bdot &= \frac{-\kappa \rdot}{(r-k)^2}
%\end{align}

I would interpret the affine coordinates this way.  If we have two ``close'' events described by our $t,r,\Omega$ coordinates, then the path and corresponding
element of arc length between two events can be described parameterically
with an arbitrary parameter $\lambda$

\begin{align*}
t &= t(\lambda) \\
r &= r(\lambda) \\
\Omega &= \Omega(\lambda)
\end{align*}

An element of event arc length will then be

\begin{align*}
ds^2 = \left(\frac{ds}{d\lambda}\right)^2 {d\lambda}^2
&=
\left( -\CC a \left(\frac{dt}{d\lambda}\right)^2 + {b} \left(\frac{dr}{d\lambda}\right)^2 + r^2\left(\frac{d\Omega}{d\lambda}\right)^2 \right) {d\lambda}^2
\end{align*}

Just as in geometry the most natural mathematic parameterization variable is often the arc length itself (as in a circle where we parameterize using fractions $\theta$ of the total arc length of the circle $2\pi$).
Introducing such a natural parameterization with units of time means that we need a dimensional fudge factor, so we write $ds = c d\tau$, so that $\tau$ is the unique parameterization for which $(ds/d\tau)^2 = c^2$.
This implicitly defines the ``proper time'', and we can write our Lagrangian to minimize in terms of that variable

\begin{align*}%\label{eqn:action}
\LL
&= \left(\frac{ds}{d\tau}\right)^2 \\
&= c^2 \\
&= \left( -\CC a \left(\frac{dt}{d\tau}\right)^2 + {b} \left(\frac{dr}{d\tau}\right)^2 + r^2\left(\frac{d\Omega}{d\tau}\right)^2 \right) \\
&= -\CC a \tdot^2 + {b} \rdot^2 + r^2\dotOmega^2 \\
\end{align*}

%Derivatives for $\Omega$
%
%\begin{align*}
%\PD{\Omega}{\LL} &= \left(\PD{\dotOmega}{\LL}\right)' \\
%0 &= (2 r^2 \dotOmega)' \\
%\end{align*}
%
%Observing that $\Omega$ is cyclic, we can write for some constant $A$
%
%\begin{align*}
%\dotOmega = \frac{A}{r^2}
%\end{align*}
%
%From \ref{eqn:action} this provides a relationship between $\rdot$ and $\tdot$
%
%\begin{align*}
%0 = -\CC (a \tdot^2 + 1) + \inv{a} \rdot^2 + A^2/r^2
%\end{align*}
%
%Calculating the derivatives for $t$
%
%\begin{align*}
%\PD{t}{\LL} &= \left(\PD{\tdot}{\LL}\right)' \\
%0 
%&= \left( - 2 \CC a \tdot \right)' \\
%%&= - 2 \CC \adot \tdot - 2 \CC a \tddot \\
%%\implies \\
%%\tddot &= \frac{\kappa \rdot \tdot}{a r^2} \\
%\end{align*}
%
%Again observing that $t$ is cyclic, we can write for some constant $B$
%
%\begin{align*}
%\tdot = B/a
%\end{align*}
%
%Or
%\begin{align*}
%\rdot^2 = \CC ({B^2} + a) - a A^2/r^2
%\end{align*}
%
%Last the derivatives for $r$
%
%%\LL = \CC  = -\CC a \tdot^2 + \inv{a} \rdot^2 + r^2\dotOmega^2
%\begin{align*}
%\PD{r}{\LL} &= \left(\PD{\rdot}{\LL}\right)' \\
%-\CC \PD{r}{a} \tdot^2 - \frac{\adot \rdot^2}{a^2} + 2 r \dotOmega^2 
%&= \left(\frac{2 \rdot}{a}\right)' \\
%&= -2 \frac{ \adot \rdot }{a^2} + \frac{2 \rddot}{a} \\ 
%\implies \\
%{\rddot} 
%&= - \inv{2} \CC \frac{\kappa}{r^2} a \tdot^2 -  \frac{\adot \rdot^2}{2 a} + r a \dotOmega^2 + \frac{ \adot \rdot }{a} \\
%&= - \inv{2} \CC \frac{\kappa B^2}{a r^2} -  \frac{\kappa \rdot^3}{2 a r^2} + a \frac{A^2}{r^3} + \frac{ \kappa \rdot^2 }{a r^2} \\
%\end{align*}
%
%The $\rdot^3$ term doesn't look right dimensionally.



%\bibliographystyle{plain}
\bibliographystyle{plainnat} % supposed to allow for \url use.
\bibliography{myrefs}      % expects file "myrefs.bib"

\end{document}               % End of document.
