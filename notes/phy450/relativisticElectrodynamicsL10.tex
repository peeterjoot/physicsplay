%%
% Copyright � 2015 Peeter Joot.  All Rights Reserved.
% Licenced as described in the file LICENSE under the root directory of this GIT repository.
%
\documentclass[]{eliblog}

\usepackage{amsmath}
\usepackage{mathpazo}

%
% shorthand for bold symbols, convenient for vectors and matrices
%
\newcommand{\Ba}[0]{\mathbf{a}}
\newcommand{\Bb}[0]{\mathbf{b}}
\newcommand{\Bc}[0]{\mathbf{c}}
\newcommand{\Bd}[0]{\mathbf{d}}
\newcommand{\Be}[0]{\mathbf{e}}
\newcommand{\Bf}[0]{\mathbf{f}}
\newcommand{\Bg}[0]{\mathbf{g}}
\newcommand{\Bh}[0]{\mathbf{h}}
\newcommand{\Bi}[0]{\mathbf{i}}
\newcommand{\Bj}[0]{\mathbf{j}}
\newcommand{\Bk}[0]{\mathbf{k}}
\newcommand{\Bl}[0]{\mathbf{l}}
\newcommand{\Bm}[0]{\mathbf{m}}
\newcommand{\Bn}[0]{\mathbf{n}}
\newcommand{\Bo}[0]{\mathbf{o}}
\newcommand{\Bp}[0]{\mathbf{p}}
\newcommand{\Bq}[0]{\mathbf{q}}
\newcommand{\Br}[0]{\mathbf{r}}
\newcommand{\Bs}[0]{\mathbf{s}}
\newcommand{\Bt}[0]{\mathbf{t}}
\newcommand{\Bu}[0]{\mathbf{u}}
\newcommand{\Bv}[0]{\mathbf{v}}
\newcommand{\Bw}[0]{\mathbf{w}}
\newcommand{\Bx}[0]{\mathbf{x}}
\newcommand{\By}[0]{\mathbf{y}}
\newcommand{\Bz}[0]{\mathbf{z}}
\newcommand{\BA}[0]{\mathbf{A}}
\newcommand{\BB}[0]{\mathbf{B}}
\newcommand{\BC}[0]{\mathbf{C}}
\newcommand{\BD}[0]{\mathbf{D}}
\newcommand{\BE}[0]{\mathbf{E}}
\newcommand{\BF}[0]{\mathbf{F}}
\newcommand{\BG}[0]{\mathbf{G}}
\newcommand{\BH}[0]{\mathbf{H}}
\newcommand{\BI}[0]{\mathbf{I}}
\newcommand{\BJ}[0]{\mathbf{J}}
\newcommand{\BK}[0]{\mathbf{K}}
\newcommand{\BL}[0]{\mathbf{L}}
\newcommand{\BM}[0]{\mathbf{M}}
\newcommand{\BN}[0]{\mathbf{N}}
\newcommand{\BO}[0]{\mathbf{O}}
\newcommand{\BP}[0]{\mathbf{P}}
\newcommand{\BQ}[0]{\mathbf{Q}}
\newcommand{\BR}[0]{\mathbf{R}}
\newcommand{\BS}[0]{\mathbf{S}}
\newcommand{\BT}[0]{\mathbf{T}}
\newcommand{\BU}[0]{\mathbf{U}}
\newcommand{\BV}[0]{\mathbf{V}}
\newcommand{\BW}[0]{\mathbf{W}}
\newcommand{\BX}[0]{\mathbf{X}}
\newcommand{\BY}[0]{\mathbf{Y}}
\newcommand{\BZ}[0]{\mathbf{Z}}

\newcommand{\Bzero}[0]{\mathbf{0}}
\newcommand{\Btheta}[0]{\boldsymbol{\theta}}
\newcommand{\Btau}[0]{\boldsymbol{\tau}}
\newcommand{\Bomega}[0]{\boldsymbol{\omega}}

%
% shorthand for unit vectors
%
\newcommand{\acap}[0]{\hat{\Ba}}
\newcommand{\bcap}[0]{\hat{\Bb}}
\newcommand{\ccap}[0]{\hat{\Bc}}
\newcommand{\dcap}[0]{\hat{\Bd}}
\newcommand{\ecap}[0]{\hat{\Be}}
\newcommand{\fcap}[0]{\hat{\Bf}}
\newcommand{\gcap}[0]{\hat{\Bg}}
\newcommand{\hcap}[0]{\hat{\Bh}}
\newcommand{\icap}[0]{\hat{\Bi}}
\newcommand{\jcap}[0]{\hat{\Bj}}
\newcommand{\kcap}[0]{\hat{\Bk}}
\newcommand{\lcap}[0]{\hat{\Bl}}
\newcommand{\mcap}[0]{\hat{\Bm}}
\newcommand{\ncap}[0]{\hat{\Bn}}
\newcommand{\ocap}[0]{\hat{\Bo}}
\newcommand{\pcap}[0]{\hat{\Bp}}
\newcommand{\qcap}[0]{\hat{\Bq}}
\newcommand{\rcap}[0]{\hat{\Br}}
\newcommand{\scap}[0]{\hat{\Bs}}
\newcommand{\tcap}[0]{\hat{\Bt}}
\newcommand{\ucap}[0]{\hat{\Bu}}
\newcommand{\vcap}[0]{\hat{\Bv}}
\newcommand{\wcap}[0]{\hat{\Bw}}
\newcommand{\xcap}[0]{\hat{\Bx}}
\newcommand{\ycap}[0]{\hat{\By}}
\newcommand{\zcap}[0]{\hat{\Bz}}
\newcommand{\thetacap}[0]{\hat{\Btheta}}

%
% to write R^n and C^n in a distinguishable fashion.  Perhaps change this
% to the double lined characters upon figuring out how to do so.
%
\newcommand{\C}[1]{$\mathbb{C}^{#1}$}
\newcommand{\R}[1]{$\mathbb{R}^{#1}$}

%
% various generally useful helpers
%

% derivative of #1 wrt. #2:
\newcommand{\D}[2] {\frac {d#2} {d#1}}

\newcommand{\inv}[1]{\frac{1}{#1}}
\newcommand{\cross}[0]{\times}

\newcommand{\abs}[1]{\lvert{#1}\rvert}
\newcommand{\norm}[1]{\lVert{#1}\rVert}
\newcommand{\innerprod}[2]{\langle{#1}, {#2}\rangle}
\newcommand{\dotprod}[2]{{#1} \cdot {#2}}
\newcommand{\bdotprod}[2]{\left({#1} \cdot {#2}\right)}
\newcommand{\crossprod}[2]{{#1} \cross {#2}}
\newcommand{\tripleprod}[3]{\dotprod{\left(\crossprod{#1}{#2}\right)}{#3}}

\DeclareMathOperator{\Proj}{Proj}
\DeclareMathOperator{\Span}{span}
\DeclareMathOperator{\Sgn}{sgn}
\DeclareMathOperator{\Area}{Area}
\DeclareMathOperator{\Volume}{Volume}

%
% A few miscellaneous things specific to this document
%
\newcommand{\crossop}[1]{\crossprod{#1}{}}

% R2 vector.
\newcommand{\VectorTwo}[2]{
\begin{bmatrix}
 {#1} \\
 {#2}
\end{bmatrix}
}

\newcommand{\VectorN}[1]{
\begin{bmatrix}
{#1}_1 \\
{#1}_2 \\
\vdots \\
{#1}_N \\
\end{bmatrix}
}

\newcommand{\DETuvij}[4]{
\begin{vmatrix}
 {#1}_{#3} & {#1}_{#4} \\
 {#2}_{#3} & {#2}_{#4}
\end{vmatrix}
}

\newcommand{\DETuvwijk}[6]{
\begin{vmatrix}
 {#1}_{#4} & {#1}_{#5} & {#1}_{#6} \\
 {#2}_{#4} & {#2}_{#5} & {#2}_{#6} \\
 {#3}_{#4} & {#3}_{#5} & {#3}_{#6}
\end{vmatrix}
}

\newcommand{\DETuvwxijkl}[8]{
\begin{vmatrix}
 {#1}_{#5} & {#1}_{#6} & {#1}_{#7} & {#1}_{#8} \\
 {#2}_{#5} & {#2}_{#6} & {#2}_{#7} & {#2}_{#8} \\
 {#3}_{#5} & {#3}_{#6} & {#3}_{#7} & {#3}_{#8} \\
 {#4}_{#5} & {#4}_{#6} & {#4}_{#7} & {#4}_{#8} \\
\end{vmatrix}
}

%\newcommand{\DETuvwxyijklm}[10]{
%\begin{vmatrix}
% {#1}_{#6} & {#1}_{#7} & {#1}_{#8} & {#1}_{#9} & {#1}_{#10} \\
% {#2}_{#6} & {#2}_{#7} & {#2}_{#8} & {#2}_{#9} & {#2}_{#10} \\
% {#3}_{#6} & {#3}_{#7} & {#3}_{#8} & {#3}_{#9} & {#3}_{#10} \\
% {#4}_{#6} & {#4}_{#7} & {#4}_{#8} & {#4}_{#9} & {#4}_{#10} \\
% {#5}_{#6} & {#5}_{#7} & {#5}_{#8} & {#5}_{#9} & {#5}_{#10}
%\end{vmatrix}
%}

% R3 vector.
\newcommand{\VectorThree}[3]{
\begin{bmatrix}
 {#1} \\
 {#2} \\
 {#3}
\end{bmatrix}
}



\author{Peeter Joot}
\email{peeter.joot@gmail.com}

%\documentclass[]{eliblogwidescreen}

\usepackage{amsmath}
\usepackage{mathpazo}

%
% shorthand for bold symbols, convenient for vectors and matrices
%
\newcommand{\Ba}[0]{\mathbf{a}}
\newcommand{\Bb}[0]{\mathbf{b}}
\newcommand{\Bc}[0]{\mathbf{c}}
\newcommand{\Bd}[0]{\mathbf{d}}
\newcommand{\Be}[0]{\mathbf{e}}
\newcommand{\Bf}[0]{\mathbf{f}}
\newcommand{\Bg}[0]{\mathbf{g}}
\newcommand{\Bh}[0]{\mathbf{h}}
\newcommand{\Bi}[0]{\mathbf{i}}
\newcommand{\Bj}[0]{\mathbf{j}}
\newcommand{\Bk}[0]{\mathbf{k}}
\newcommand{\Bl}[0]{\mathbf{l}}
\newcommand{\Bm}[0]{\mathbf{m}}
\newcommand{\Bn}[0]{\mathbf{n}}
\newcommand{\Bo}[0]{\mathbf{o}}
\newcommand{\Bp}[0]{\mathbf{p}}
\newcommand{\Bq}[0]{\mathbf{q}}
\newcommand{\Br}[0]{\mathbf{r}}
\newcommand{\Bs}[0]{\mathbf{s}}
\newcommand{\Bt}[0]{\mathbf{t}}
\newcommand{\Bu}[0]{\mathbf{u}}
\newcommand{\Bv}[0]{\mathbf{v}}
\newcommand{\Bw}[0]{\mathbf{w}}
\newcommand{\Bx}[0]{\mathbf{x}}
\newcommand{\By}[0]{\mathbf{y}}
\newcommand{\Bz}[0]{\mathbf{z}}
\newcommand{\BA}[0]{\mathbf{A}}
\newcommand{\BB}[0]{\mathbf{B}}
\newcommand{\BC}[0]{\mathbf{C}}
\newcommand{\BD}[0]{\mathbf{D}}
\newcommand{\BE}[0]{\mathbf{E}}
\newcommand{\BF}[0]{\mathbf{F}}
\newcommand{\BG}[0]{\mathbf{G}}
\newcommand{\BH}[0]{\mathbf{H}}
\newcommand{\BI}[0]{\mathbf{I}}
\newcommand{\BJ}[0]{\mathbf{J}}
\newcommand{\BK}[0]{\mathbf{K}}
\newcommand{\BL}[0]{\mathbf{L}}
\newcommand{\BM}[0]{\mathbf{M}}
\newcommand{\BN}[0]{\mathbf{N}}
\newcommand{\BO}[0]{\mathbf{O}}
\newcommand{\BP}[0]{\mathbf{P}}
\newcommand{\BQ}[0]{\mathbf{Q}}
\newcommand{\BR}[0]{\mathbf{R}}
\newcommand{\BS}[0]{\mathbf{S}}
\newcommand{\BT}[0]{\mathbf{T}}
\newcommand{\BU}[0]{\mathbf{U}}
\newcommand{\BV}[0]{\mathbf{V}}
\newcommand{\BW}[0]{\mathbf{W}}
\newcommand{\BX}[0]{\mathbf{X}}
\newcommand{\BY}[0]{\mathbf{Y}}
\newcommand{\BZ}[0]{\mathbf{Z}}

\newcommand{\Bzero}[0]{\mathbf{0}}
\newcommand{\Btheta}[0]{\boldsymbol{\theta}}
\newcommand{\Btau}[0]{\boldsymbol{\tau}}
\newcommand{\Bomega}[0]{\boldsymbol{\omega}}

%
% shorthand for unit vectors
%
\newcommand{\acap}[0]{\hat{\Ba}}
\newcommand{\bcap}[0]{\hat{\Bb}}
\newcommand{\ccap}[0]{\hat{\Bc}}
\newcommand{\dcap}[0]{\hat{\Bd}}
\newcommand{\ecap}[0]{\hat{\Be}}
\newcommand{\fcap}[0]{\hat{\Bf}}
\newcommand{\gcap}[0]{\hat{\Bg}}
\newcommand{\hcap}[0]{\hat{\Bh}}
\newcommand{\icap}[0]{\hat{\Bi}}
\newcommand{\jcap}[0]{\hat{\Bj}}
\newcommand{\kcap}[0]{\hat{\Bk}}
\newcommand{\lcap}[0]{\hat{\Bl}}
\newcommand{\mcap}[0]{\hat{\Bm}}
\newcommand{\ncap}[0]{\hat{\Bn}}
\newcommand{\ocap}[0]{\hat{\Bo}}
\newcommand{\pcap}[0]{\hat{\Bp}}
\newcommand{\qcap}[0]{\hat{\Bq}}
\newcommand{\rcap}[0]{\hat{\Br}}
\newcommand{\scap}[0]{\hat{\Bs}}
\newcommand{\tcap}[0]{\hat{\Bt}}
\newcommand{\ucap}[0]{\hat{\Bu}}
\newcommand{\vcap}[0]{\hat{\Bv}}
\newcommand{\wcap}[0]{\hat{\Bw}}
\newcommand{\xcap}[0]{\hat{\Bx}}
\newcommand{\ycap}[0]{\hat{\By}}
\newcommand{\zcap}[0]{\hat{\Bz}}
\newcommand{\thetacap}[0]{\hat{\Btheta}}

%
% to write R^n and C^n in a distinguishable fashion.  Perhaps change this
% to the double lined characters upon figuring out how to do so.
%
\newcommand{\C}[1]{$\mathbb{C}^{#1}$}
\newcommand{\R}[1]{$\mathbb{R}^{#1}$}

%
% various generally useful helpers
%

% derivative of #1 wrt. #2:
\newcommand{\D}[2] {\frac {d#2} {d#1}}

\newcommand{\inv}[1]{\frac{1}{#1}}
\newcommand{\cross}[0]{\times}

\newcommand{\abs}[1]{\lvert{#1}\rvert}
\newcommand{\norm}[1]{\lVert{#1}\rVert}
\newcommand{\innerprod}[2]{\langle{#1}, {#2}\rangle}
\newcommand{\dotprod}[2]{{#1} \cdot {#2}}
\newcommand{\bdotprod}[2]{\left({#1} \cdot {#2}\right)}
\newcommand{\crossprod}[2]{{#1} \cross {#2}}
\newcommand{\tripleprod}[3]{\dotprod{\left(\crossprod{#1}{#2}\right)}{#3}}

\DeclareMathOperator{\Proj}{Proj}
\DeclareMathOperator{\Span}{span}
\DeclareMathOperator{\Sgn}{sgn}
\DeclareMathOperator{\Area}{Area}
\DeclareMathOperator{\Volume}{Volume}

%
% A few miscellaneous things specific to this document
%
\newcommand{\crossop}[1]{\crossprod{#1}{}}

% R2 vector.
\newcommand{\VectorTwo}[2]{
\begin{bmatrix}
 {#1} \\
 {#2}
\end{bmatrix}
}

\newcommand{\VectorN}[1]{
\begin{bmatrix}
{#1}_1 \\
{#1}_2 \\
\vdots \\
{#1}_N \\
\end{bmatrix}
}

\newcommand{\DETuvij}[4]{
\begin{vmatrix}
 {#1}_{#3} & {#1}_{#4} \\
 {#2}_{#3} & {#2}_{#4}
\end{vmatrix}
}

\newcommand{\DETuvwijk}[6]{
\begin{vmatrix}
 {#1}_{#4} & {#1}_{#5} & {#1}_{#6} \\
 {#2}_{#4} & {#2}_{#5} & {#2}_{#6} \\
 {#3}_{#4} & {#3}_{#5} & {#3}_{#6}
\end{vmatrix}
}

\newcommand{\DETuvwxijkl}[8]{
\begin{vmatrix}
 {#1}_{#5} & {#1}_{#6} & {#1}_{#7} & {#1}_{#8} \\
 {#2}_{#5} & {#2}_{#6} & {#2}_{#7} & {#2}_{#8} \\
 {#3}_{#5} & {#3}_{#6} & {#3}_{#7} & {#3}_{#8} \\
 {#4}_{#5} & {#4}_{#6} & {#4}_{#7} & {#4}_{#8} \\
\end{vmatrix}
}

%\newcommand{\DETuvwxyijklm}[10]{
%\begin{vmatrix}
% {#1}_{#6} & {#1}_{#7} & {#1}_{#8} & {#1}_{#9} & {#1}_{#10} \\
% {#2}_{#6} & {#2}_{#7} & {#2}_{#8} & {#2}_{#9} & {#2}_{#10} \\
% {#3}_{#6} & {#3}_{#7} & {#3}_{#8} & {#3}_{#9} & {#3}_{#10} \\
% {#4}_{#6} & {#4}_{#7} & {#4}_{#8} & {#4}_{#9} & {#4}_{#10} \\
% {#5}_{#6} & {#5}_{#7} & {#5}_{#8} & {#5}_{#9} & {#5}_{#10}
%\end{vmatrix}
%}

% R3 vector.
\newcommand{\VectorThree}[3]{
\begin{bmatrix}
 {#1} \\
 {#2} \\
 {#3}
\end{bmatrix}
}



\author{Peeter Joot}
\email{peeter.joot@gmail.com}


\chapter{Lorentz force equation energy term, and four vector formulation of the Lorentz force equation.}
\label{chap:relativisticElectrodynamicsL10}
%\useCCL
\blogpage{http://sites.google.com/site/peeterjoot/math2011/relativisticElectrodynamicsL10.pdf}
\date{Feb 8, 2011}
\revisionInfo{relativisticElectrodynamicsL10.tex}

%\beginArtWithToc
\beginArtNoToc

\section{Reading.}

Covering chapter 3 material from the text \cite{landau1980classical}.

Covering \href{http://www.physics.utoronto.ca/~poppitz/e-poppitz/PHY450_files/RelEMpp74-83.pdf}{lecture notes pp. 74-83}: gauge transformations in 3-vector language (74); energy of a relativistic particle in EM field (75); variational principle and equation of motion in 4-vector form (76-77); the field strength tensor (78-80); the fourth equation of motion (81)

\section{What is the significance to the gauge invariance of the action?}

We had argued that under a gauge transformation

\begin{equation}\label{eqn:relativisticElectrodynamicsL10:10}
A_i \rightarrow A_i + \PD{x^i}{\chi},
\end{equation}

the action for a particle changes by a boundary term 

\begin{equation}\label{eqn:relativisticElectrodynamicsL10:30}
- \frac{e}{c} ( \chi(x_b) - \chi(x_a) ).
\end{equation}

Because $S$ changes by a boundary term only, variation problem is not affected.  The extremal trajectories are then the same, hence the EOM are the same.

\subsection{A less high brow demonstration.}

With our four potential split into space and time components
\begin{equation}\label{eqn:relativisticElectrodynamicsL10:50}
A^i = (\phi, \BA),
\end{equation}

the lower index representation of the same vector is

\begin{equation}\label{eqn:relativisticElectrodynamicsL10:70}
A_i = (\phi, -\BA).
\end{equation}

Our gauge transformation is then

\begin{align}\label{eqn:relativisticElectrodynamicsL10:90}
A_0 &\rightarrow A_0 + \PD{x^0}{\chi} \\
-\BA &\rightarrow -\BA + \PD{\Bx}{\chi}
\end{align}

or
\begin{align}\label{eqn:relativisticElectrodynamicsL10:110}
\phi &\rightarrow \phi + \inv{c}\PD{t}{\chi} \\
\BA &\rightarrow \BA - \spacegrad \chi.
\end{align}

Now observe how the electric and magnetic fields are transformed

\begin{align*}
\BE 
&= - \spacegrad \phi - \inv{c} \PD{t}{\BA} \\
&\rightarrow 
- \spacegrad \left( \phi + \inv{c}\PD{t}{\chi} \right) - \inv{c}\PD{t}{} \left( \BA - \spacegrad \chi \right) \\
\end{align*}

Sufficient continuity of $\chi$ is assumed, allowing commutation of the space and time derivatives, and we are left with just $\BE$

For the magnetic field we have

\begin{align*}
\BB 
&= \spacegrad \cross \BA  \\
&\rightarrow 
\spacegrad \cross (\BA  - \spacegrad \chi) \\
\end{align*}

Again with continuity assumptions, $\spacegrad \cross (\spacegrad \chi) = 0$, and we are left with just $\BB$.  The electromagnetic fields (as opposed to potentials) do not change under gauge transformations.

We conclude that the $\{A_i\}$ description is hugely redundant, but despite that, local $\LL$ and $H$ can only be written in terms of the potentials $A_i$.

\subsection{Energy term of the Lorentz force.  Three vector approach.}

With the Lagrangian for the particle given by

\begin{equation}\label{eqn:relativisticElectrodynamicsL10:130}
\LL = - mc^2 \InvGamma + \frac{e}{c} \BA \cdot \Bv - e \phi,
\end{equation}

we define the energy as 

\begin{equation}\label{eqn:relativisticElectrodynamicsL10:150}
\mathcal{E} = \Bv \cdot \PD{\Bv}{\LL} - \LL
\end{equation}

This is not necessarily a conserved quantity, but we define it as the energy anyways (we don't really have a Hamiltonian when the fields are time dependent).  Associated with this quantity is the general relationship

\begin{equation}\label{eqn:relativisticElectrodynamicsL10:170}
\ddt{\mathcal{E}} = -\PD{t}{\LL},
\end{equation}

and when the Lagrangian is invariant with respect to time translation the energy $\mathcal{E}$ will be a conserved quantity (and also the Hamiltonian).

Our canonical momentum is 
\begin{equation}\label{eqn:relativisticElectrodynamicsL10:190}
\PD{\Bv}{\LL} = \gamma m \Bv + \frac{e}{c} \BA
\end{equation}

So our energy is
\begin{align*}
\mathcal{E} = \gamma m \Bv^2 + \frac{e}{c} \BA \cdot \Bv - \left( - mc^2 \InvGamma + \frac{e}{c} \BA \cdot \Bv - e \phi \right).
\end{align*}

Or
\begin{equation}\label{eqn:relativisticElectrodynamicsL10:210}
\mathcal{E} = \underbrace{\frac{m c^2}{\InvGamma}}_{(\conj)} + e \phi.
\end{equation}

The contribution of $(\conj)$ to the energy $\mathcal{E}$ comes from the free (kinetic) particle portion of the Lagrangian $\LL = -m c^2 \InvGamma$, and we identify the remainder as a potential energy 

\begin{equation}\label{eqn:relativisticElectrodynamicsL10:230}
\mathcal{E} = \frac{m c^2}{\InvGamma} + \underbrace{e \phi}_{\text{"potential"}}.
\end{equation}

For the kinetic portion we can also show that we have
\begin{equation}\label{eqn:relativisticElectrodynamicsL10:250}
\frac{d}{dt} \mathcal{E}_{\text{kinetic}} 
=
\frac{m c^2}{\InvGamma} 
= e \BE \cdot \Bv.
\end{equation}

To show this observe that we have

\begin{align*}
\frac{d}{dt} \mathcal{E}_{\text{kinetic}} 
&= m c^2 \frac{d\gamma}{dt} \\
&= m c^2 \frac{d}{dt} \inv{\InvGamma} \\
&= m c^2 \frac{\frac{\Bv}{c^2} \cdot \frac{d\Bv}{dt}}{\left(1 - \frac{\Bv^2}{c^2}\right)^{3/2}} \\
&= \frac{m \gamma \Bv \cdot \frac{d\Bv}{dt}}{1 - \frac{\Bv^2}{c^2}}
\end{align*}

We also have

\begin{align*}
\Bv \cdot \ddt{\Bp} 
&= \Bv \cdot \ddt{} \frac{m \Bv}{\InvGamma} \\
&= m\Bv^2 \ddt{\gamma} + m \gamma \Bv \cdot \ddt{\Bv} \\
&= m\Bv^2 \ddt{\gamma} + m c^2 \ddt{\gamma} \left( 1 - \frac{\Bv^2}{c^2} \right) \\
&= m c^2 \ddt{\gamma}.
\end{align*}

Utilizing the Lorentz force equation, we have

\begin{equation}\label{eqn:relativisticElectrodynamicsL10:270}
\Bv \cdot \ddt{\Bp} = e \left( \BE + \frac{\Bv}{c} \cross \BB \right) \cdot \Bv = e \BE \cdot \Bv
\end{equation}

and are able to assemble the above, and find that we have
\begin{equation}\label{eqn:relativisticElectrodynamicsL10:290}
\ddt{(m c^2 \gamma)} = e \BE \cdot \Bv 
\end{equation}

\section{Four vector Lorentz force}

Using $ds = \sqrt{ dx^i dx_i } $ our action can be rewritten

\begin{align*}
S 
&= \int \left( -m c ds - \frac{e}{c} u^i A_i ds \right) \\
&= \int \left( -m c ds - \frac{e}{c} dx^i A_i \right) \\
&= \int \left( -m c \sqrt{ dx^i dx_i} - \frac{e}{c} dx^i A_i \right) \\
\end{align*}

$x^i(\tau)$ is a worldline $x^i(0) = a^i$, $x^i(1) = b^i$, 

We want $\delta S = S[ x + \delta x ] - S[ x ] = 0$ (to linear order in $\delta x$)

The variation of our proper length is
\begin{align*}
\delta ds 
&=
\delta \sqrt{ dx^i dx_i } \\
&= \inv{ 2 \sqrt{ dx^i dx_i }} \delta (dx^j dx_j)
\end{align*}

Observe that for the numerator we have
\begin{align*}
\delta (dx^j dx_j) 
&= \delta ( dx^j g_{jk} dx^k ) \\
&= \delta ( dx^j ) g_{jk} dx^k + dx^j g_{jk} \delta ( dx^k ) \\
&= \delta ( dx^j ) g_{jk} dx^k + dx^k g_{kj} \delta ( dx^j ) \\
&= 2 \delta ( dx^j ) g_{jk} dx^k \\
&= 2 \delta ( dx^j ) dx_j 
\end{align*}

\paragraph{TIP:} If this goes too quick, or there is any disbelief, write these all out explicitly as $dx^j dx_j = dx^0 dx_0 + dx^1 dx_1 + dx^2 dx_2 + dx^3 dx_3$ and compute it that way.

For the four vector potential our variation is

\begin{equation}\label{eqn:relativisticElectrodynamicsL10:310}
\delta A_i = A_i(x + \delta x) - A_i = \PD{x^j}{A_i} \delta x^j = \partial_j A_i \delta x^j
\end{equation}

(i.e. By chain rule)

Completing the proper length variations above we have

\begin{align*}
\delta \sqrt{ dx^i dx_i } 
&= \inv{ \sqrt{ dx^i dx_i }} \delta (dx^j) dx_j \\
&= \delta (dx^j) \dds{x_j}  \\
&= \delta (dx^j) u_j \\
&= d \delta x^j u_j
\end{align*}

We are now ready to assemble results and do the integration by parts

\begin{align*}
\delta S 
&= \int \left( 
-m c d (\delta x^j) u_j
- \frac{e}{c} d (\delta x^i) A_i 
- \frac{e}{c} dx^i \partial_j A_i \delta x^j
\right) \\
&= 
{\left. 
\left( -m c (\delta x^j) u_j - \frac{e}{c} (\delta x^i) A_i \right)
\right\vert}_a^b
+\int \left( 
m c \delta x^j d u_j
+ \frac{e}{c} (\delta x^i) d A_i 
- \frac{e}{c} dx^i \partial_j A_i \delta x^j
\right) \\
\end{align*}

Our variation at the endpoints is zero $\evalbar{\delta x^i}{a} = \evalbar{\delta x^i}{b} = 0$, killing the non-integral terms

\begin{align*}
\delta S 
&= 
\int 
\delta x^j
\left( 
m c d u_j
+ \frac{e}{c} d A_j 
- \frac{e}{c} dx^i \partial_j A_i 
\right).
\end{align*}

Observe that our differential can also be expanded by chain rule

\begin{equation}\label{eqn:relativisticElectrodynamicsL10:330}
d A_j = \PD{x^i}{A_j} dx^i = \partial_i A_j dx^i,
\end{equation}

which simplifies the variation further

\begin{align*}
\delta S 
&= 
\int 
\delta x^j
\left( 
m c d u_j
+ \frac{e}{c} dx^i ( \partial_i A_j - \partial_j A_i )
\right) \\
&= 
\int 
\delta x^j ds
\left( 
m c \frac{d u_j}{ds}
+ \frac{e}{c} u^i ( \partial_i A_j - \partial_j A_i )
\right) \\
\end{align*}

Since this is true for all variations $\delta x^j$, which is arbitrary, the interior part is zero everywhere in the trajectory.  The antisymmetric portion, a rank 2 4-tensor is called the electromagnetic field strength tensor, and written

\begin{equation}\label{eqn:relativisticElectrodynamicsL10:350}
\boxed{
F_{ij} = \partial_i A_j - \partial_j A_i.
}
\end{equation}

In matrix form this is

\begin{equation}\label{eqn:relativisticElectrodynamicsL10:370}
\Norm{ F_{ij} } = 
\begin{bmatrix}
0 & E_x & E_y & E_z \\
-E_x & 0 & -B_z & B_y \\
-E_y & B_z & 0 & -B_x \\
-E_z & -B_y & B_x & 0
\end{bmatrix}.
\end{equation}

In terms of the field strength tensor our Lorentz force equation takes the form

\begin{equation}\label{eqn:relativisticElectrodynamicsL10:390}
\boxed{
\dds{(m c u_i)} = \frac{e}{c} F_{ij} u^j.
}
\end{equation}

\EndArticle
