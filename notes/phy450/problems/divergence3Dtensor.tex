%
% Copyright � 2012 Peeter Joot.  All Rights Reserved.
% Licenced as described in the file LICENSE under the root directory of this GIT repository.
%

\label{chap:divergence3Dtensor}
%\blogpage{http://sites.google.com/site/peeterjoot/math2011/divergence3Dtensor.pdf}
%\date{April 10, 2011}



With the divergence of the energy momentum tensor converted from a volume to a surface integral given by

\begin{equation}\label{eqn:relativisticElectrodynamicsPS6:340}
\int_V d^3 \Bx \partial_\beta T^{\beta \alpha} = \oint_{\partial V} d^2 \sigma^\beta T^{\beta \alpha},
\end{equation}

I got to wondering what a closed form algebraic expression for this curious (and foreign seeming) quantity \(d^2 \sigma^\beta\) was.  It obviously must be related to the normal to the surface.  It seemed to me that a natural way to answer this question was to consider this divergence integral over an arbitrarily parametrized volume.  This turns out to be overkill, but a useful seeming digression.

\section{A generally parametrized parallelepiped volume element}

Suppose we parametrize a volume by specifying that all the points in that volume are covered by the position vector from the origin, given by

\begin{equation}\label{eqn:relativisticElectrodynamicsPS6:360}
\Bx = \Bx(a_1, a_2, a_3).
\end{equation}

At any point in the volume of interest, we can create a level curve, holding two of the parameters \(a_\alpha\) constant, and varying the remaining one.  In particular, we can construct three direction vectors along these level curves, one for each parameter not held constant

\begin{equation}\label{eqn:relativisticElectrodynamicsPS6:380}
\begin{aligned}
d\Bx_1 &= da_1 \PD{a_1}{\Bx} \\
d\Bx_2 &= da_2 \PD{a_2}{\Bx} \\
d\Bx_3 &= da_3 \PD{a_3}{\Bx}
\end{aligned}
\end{equation}

The span of these vectors, provided they are non-degenerate, forms a parallelepiped, the volume of which is

\begin{equation}\label{eqn:relativisticElectrodynamicsPS6:400}
d^3\Bx = d\Bx_3 \cdot (d\Bx_1 \cross d\Bx_2).
\end{equation}

This volume element can be expanded in a number of ways

\begin{equation}\label{eqn:divergence3Dtensor:660}
\begin{aligned}
d^3\Bx 
&= \PD{a_1}{\Bx} \cdot \left( \PD{a_2}{\Bx} \cross \PD{a_3}{\Bx} \right) \\
&= 
\PD{a_1}{x^\alpha} 
\PD{a_2}{x^\beta} 
\PD{a_3}{x^\gamma} 
\epsilon_{\alpha \beta \gamma} 
da_1 da_2 da_3 \\
&= 
\PD{a_\alpha}{x^1} 
\PD{a_\beta}{x^2} 
\PD{a_\gamma}{x^3} 
\epsilon_{\alpha \beta \gamma} 
da_1 da_2 da_3 \\
&= 
\PD{a_{[1}}{x^1} 
\PD{a_{2}}{x^2} 
\PD{a_{3]}}{x^3} 
da_1 da_2 da_2 \\
&= 
\Abs{ \frac{\partial(x^1, x^2, x^3)}{\partial (a_1, a_2, a_3)}}
da_1 da_2 da_3 \\
\end{aligned}
\end{equation}

where the Jacobian determinant is given by

\begin{equation}\label{eqn:relativisticElectrodynamicsPS6:420}
\Abs{ \frac{\partial(x^1, x^2, x^3)}{\partial (a_1, a_2, a_3)}}
= 
\begin{vmatrix}
 \PD{a_1}{x^1} & \PD{a_1}{x^2} & \PD{a_1}{x^3} \\
 \PD{a_2}{x^1} & \PD{a_2}{x^2} & \PD{a_2}{x^3} \\
 \PD{a_3}{x^1} & \PD{a_3}{x^2} & \PD{a_3}{x^3}
\end{vmatrix}.
\end{equation}

Provided we are interested in a volume for which the sign of this Jacobian determinant does not change sign, our task is to evaluate and reduce the integral 

\begin{equation}\label{eqn:relativisticElectrodynamicsPS6:620}
\int 
\Abs{ \frac{\partial(x^1, x^2, x^3)}{\partial (a_1, a_2, a_3)}}
da_1 da_2 da_3 
\PD{x^\beta}{T^{\beta \alpha}}
\end{equation}

to a set (and sum of) two dimensional integrals.

\section{On the geometry of the surfaces}

Suppose that we integrate over the ranges \([a_{1-}, a_{1+}]\), \([a_{2-}, a_{2+}]\), \([a_{3-}, a_{3+}]\).  Observe that the outwards normals along the \(a_1 = a_1+\) face is \(d\Bn_{1+} = da_2 da_3 \PDi{a_2}{\Bx} \cross \PDi{a_3}{\Bx}\).  This is

\begin{equation}\label{eqn:relativisticElectrodynamicsPS6:440}
d\Bn_{1+} 
= da_2 da_3 \PD{a_2}{\Bx} \cross \PD{a_3}{\Bx}
= da_2 da_3 \PD{a_2}{x^\mu} \PD{a_3}{x^\nu} \epsilon_{\mu \nu \gamma} \Be_\gamma
\end{equation}

Similarly our normal on the \(a_2 = a_{2+}\) face is

\begin{equation}\label{eqn:relativisticElectrodynamicsPS6:460}
d\Bn_{2+} 
= da_3 da_1 \PD{a_3}{\Bx} \cross \PD{a_1}{\Bx}
= da_3 da_1 \PD{a_3}{x^\mu} \PD{a_1}{x^\nu} \epsilon_{\mu \nu \gamma} \Be_\gamma,
\end{equation}

and on the \(a_3 = a_{3+}\) face the outward normal is

\begin{equation}\label{eqn:relativisticElectrodynamicsPS6:480}
d\Bn_{3+} 
= da_1 da_2 \PD{a_1}{\Bx} \cross \PD{a_2}{\Bx}
= da_1 da_2 \PD{a_1}{x^\mu} \PD{a_2}{x^\nu} \epsilon_{\mu \nu \gamma} \Be_\gamma.
\end{equation}

Along the \(a_{\alpha-}\) faces these are just negated.  We can summarize these as

\begin{equation}\label{eqn:relativisticElectrodynamicsPS6:500}
d\Bn_{\sigma\pm} 
= \pm \inv{2!} da_\alpha da_\beta \PD{a_\alpha}{\Bx} \cross \PD{a_\beta}{\Bx} \epsilon_{\alpha \beta \sigma}
= \pm \inv{2!} da_\alpha da_\beta \PD{a_\alpha}{x^\mu} \PD{a_\beta}{x^\nu} \epsilon_{\alpha \beta \sigma} \epsilon_{\mu \nu \gamma} \Be_\gamma 
\end{equation}

\section{Expansion of the Jacobian determinant}

Suppose, to start with, our divergence volume integral \eqnref{eqn:relativisticElectrodynamicsPS6:620} has just the following term

\begin{equation}\label{eqn:relativisticElectrodynamicsPS6:520}
\int d^3 \Bx \partial_3 M.
\end{equation}

The specifics of how the scalar \(M = T^{3 \alpha}\) is indexed will not matter yet, so let us suppress it.  The Jacobian determinant can be expanded along the \(\PD{a_\alpha}{x^3}\) column for

\begin{equation}\label{eqn:divergence3Dtensor:680}
\begin{aligned}
\int d^3 \Bx \partial_3 M
&=
\int da_1 da_2 da_3
\Abs{ \frac{\partial(x^1, x^2, x^3)}{\partial (a_1, a_2, a_3)}} 
\PD{x^3}{M} \\
&=
\int da_1 da_2 da_3
\left(
\PD{a_{[1}}{x^1} 
\PD{a_{2}}{x^2} 
\PD{a_{3]}}{x^3} 
\right)
\PD{x^3}{M} \\
&=
\int da_1 da_2 da_3
\left(
\PD{a_{[1}}{x^1} 
\PD{a_{2]}}{x^2} 
\PD{a_{3}}{x^3} 
+
\PD{a_{[2}}{x^1} 
\PD{a_{3]}}{x^2} 
\PD{a_{1}}{x^3} 
+
\PD{a_{[3}}{x^1} 
\PD{a_{1]}}{x^2} 
\PD{a_{2}}{x^3} 
\right)
\PD{x^3}{M} \\
&=
\int da_1 da_2 da_3
\left(
\Abs{ \frac{\partial(x^1, x^2)}{\partial (a_1, a_2)}} 
\PD{a_{3}}{x^3} 
+
\Abs{ \frac{\partial(x^1, x^2)}{\partial (a_2, a_3)}} 
\PD{a_{1}}{x^3} 
+
\Abs{ \frac{\partial(x^1, x^2)}{\partial (a_3, a_1)}} 
\PD{a_{2}}{x^3} 
\right)
\PD{x^3}{M} \\
&=
\int da_1 da_2 \Abs{ \frac{\partial(x^1, x^2)}{\partial (a_1, a_2)}} 
\int da_3 \PD{a_{3}}{x^3} \PD{x^3}{M}  \\
&\qquad +
\int da_2 da_3 \Abs{ \frac{\partial(x^1, x^2)}{\partial (a_2, a_3)}} 
\int da_1 \PD{a_{1}}{x^3} \PD{x^3}{M}  \\
&\qquad +
\int da_3 da_1 \Abs{ \frac{\partial(x^1, x^2)}{\partial (a_3, a_1)}} 
\int da_2 \PD{a_{2}}{x^3} \PD{x^3}{M}  \\
&=
\int da_1 da_2 \Abs{ \frac{\partial(x^1, x^2)}{\partial (a_1, a_2)}} 
\int da_3 \PD{a_{3}}{M}  \\
&\qquad +
\int da_2 da_3 \Abs{ \frac{\partial(x^1, x^2)}{\partial (a_2, a_3)}} 
\int da_1 \PD{a_{1}}{M}  \\
&\qquad +
\int da_3 da_1 \Abs{ \frac{\partial(x^1, x^2)}{\partial (a_3, a_1)}} 
\int da_2 \PD{a_{2}}{M}  \\
&=
\int da_1 da_2 \Abs{ \frac{\partial(x^1, x^2)}{\partial (a_1, a_2)}} 
\Bigl( M(a_{3+}) - M(a_{3+}) \Bigr) \\
&\qquad +
\int da_2 da_3 \Abs{ \frac{\partial(x^1, x^2)}{\partial (a_2, a_3)}} 
\Bigl( M(a_{1+}) - M(a_{1+}) \Bigr) \\
&\qquad +
\int da_3 da_1 \Abs{ \frac{\partial(x^1, x^2)}{\partial (a_3, a_1)}} 
\Bigl( M(a_{2+}) - M(a_{2+}) \Bigr)
\end{aligned}
\end{equation}

Performing the same task (really just performing cyclic permutation of indices) we can now construct the whole divergence integral

\begin{equation}\label{eqn:divergence3Dtensor:700}
\begin{aligned}
\int d^3 \Bx \partial_\beta T^{\beta \alpha}
&=
\int da_1 da_2 \Abs{ \frac{\partial(x^1, x^2)}{\partial (a_1, a_2)}} 
\Bigl( T^{3 \alpha}(a_{3+}) - T^{3 \alpha}(a_{3+}) \Bigr) \\
&\qquad +
\int da_2 da_3 \Abs{ \frac{\partial(x^1, x^2)}{\partial (a_2, a_3)}} 
\Bigl( T^{3 \alpha}(a_{1+}) - T^{3 \alpha}(a_{1+}) \Bigr) \\
&\qquad +
\int da_3 da_1 \Abs{ \frac{\partial(x^1, x^2)}{\partial (a_3, a_1)}} 
\Bigl( T^{3 \alpha}(a_{2+}) - T^{3 \alpha}(a_{2+}) \Bigr) \\
&+\int da_1 da_2 \Abs{ \frac{\partial(x^2, x^3)}{\partial (a_1, a_2)}} 
\Bigl( T^{1 \alpha}(a_{3+}) - T^{1 \alpha}(a_{3+}) \Bigr) \\
&\qquad +
\int da_2 da_3 \Abs{ \frac{\partial(x^2, x^3)}{\partial (a_2, a_3)}} 
\Bigl( T^{1 \alpha}(a_{1+}) - T^{1 \alpha}(a_{1+}) \Bigr) \\
&\qquad +
\int da_3 da_1 \Abs{ \frac{\partial(x^2, x^3)}{\partial (a_3, a_1)}} 
\Bigl( T^{1 \alpha}(a_{2+}) - T^{1 \alpha}(a_{2+}) \Bigr) \\
&+\int da_1 da_2 \Abs{ \frac{\partial(x^3, x^1)}{\partial (a_1, a_2)}} 
\Bigl( T^{2 \alpha}(a_{3+}) - T^{2 \alpha}(a_{3+}) \Bigr) \\
&\qquad +
\int da_2 da_3 \Abs{ \frac{\partial(x^3, x^1)}{\partial (a_2, a_3)}} 
\Bigl( T^{2 \alpha}(a_{1+}) - T^{2 \alpha}(a_{1+}) \Bigr) \\
&\qquad +
\int da_3 da_1 \Abs{ \frac{\partial(x^3, x^1)}{\partial (a_3, a_1)}} 
\Bigl( T^{2 \alpha}(a_{2+}) - T^{2 \alpha}(a_{2+}) \Bigr).
\end{aligned}
\end{equation}

Regrouping we have

\begin{equation}\label{eqn:divergence3Dtensor:720}
\begin{aligned}
\int d^3 \Bx \partial_\beta T^{\beta \alpha}
&=
\int da_1 da_2 \left( 
\Abs{ \frac{\partial(x^1, x^2)}{\partial (a_1, a_2)}} 
\evalbar{T^{3 \alpha}}{\Delta a_3}
+\Abs{ \frac{\partial(x^2, x^3)}{\partial (a_1, a_2)}} 
\evalbar{T^{1 \alpha}}{\Delta a_3}
+\Abs{ \frac{\partial(x^3, x^1)}{\partial (a_1, a_2)}} 
\evalbar{T^{2 \alpha}}{\Delta a_3}
\right) \\
&+
\int da_2 da_3 \left( 
\Abs{ \frac{\partial(x^1, x^2)}{\partial (a_2, a_3)}} 
\evalbar{T^{3 \alpha}}{\Delta a_3}
+\Abs{ \frac{\partial(x^2, x^3)}{\partial (a_2, a_3)}} 
\evalbar{T^{1 \alpha}}{\Delta a_3}
+\Abs{ \frac{\partial(x^3, x^1)}{\partial (a_2, a_3)}} 
\evalbar{T^{2 \alpha}}{\Delta a_3}
\right) \\
&+
\int da_3 da_1 \left( 
\Abs{ \frac{\partial(x^1, x^2)}{\partial (a_3, a_1)}} 
\evalbar{T^{3 \alpha}}{\Delta a_3}
+\Abs{ \frac{\partial(x^2, x^3)}{\partial (a_3, a_1)}} 
\evalbar{T^{1 \alpha}}{\Delta a_3}
+\Abs{ \frac{\partial(x^3, x^1)}{\partial (a_3, a_1)}} 
\evalbar{T^{2 \alpha}}{\Delta a_3}
\right).
\end{aligned}
\end{equation}

Observe that we can factor these sums utilizing the normals for the parallelepiped volume element

\begin{equation}\label{eqn:divergence3Dtensor:740}
\begin{aligned}
\int d^3 \Bx \partial_\beta T^{\beta \alpha}
&=
\int da_1 da_2 
\Abs{ \frac{\partial(x^\mu, x^\nu)}{\partial (a_1, a_2)}} \epsilon_{\mu \nu \gamma} \Be_\gamma \cdot \Be_\beta
\evalbar{T^{\beta \alpha}}{\Delta a_3} \\
&+
\int da_2 da_3 
\Abs{ \frac{\partial(x^\mu, x^\nu)}{\partial (a_2, a_3)}} \epsilon_{\mu \nu \gamma} \Be_\gamma \cdot \Be_\beta
\evalbar{T^{\beta \alpha}}{\Delta a_1} \\
&+
\int da_3 da_1 
\Abs{ \frac{\partial(x^\mu, x^\nu)}{\partial (a_3, a_1)}} \epsilon_{\mu \nu \gamma} \Be_\gamma \cdot \Be_\beta
\evalbar{T^{\beta \alpha}}{\Delta a_2}
\end{aligned}
\end{equation}

Let us look at the first of these integrals in more detail.  We integrate the values of the \(\Be_\beta T^{\beta \alpha}\) evaluated on the points of the surface for which \(a_3 = a_{3+}\).  To perform this integral we dot against the outward normal area element

\begin{equation}\label{eqn:relativisticElectrodynamicsPS6:539}
da_1 da_2 \PDi{a_1}{x^\mu} \PDi{a_2}{x^\nu} \epsilon_{\mu\nu\gamma} \Be_\gamma.
\end{equation}

  We do the same, but subtract the integral where \(\Be_\beta T^{\beta\alpha}\) is evaluated on the surface \(a_3 = a_{3-}\), where we dot with the area element that has the inwards normal direction on that surface.  This is then done for each of the surfaces of the parallelepiped that we are integrating over.

In terms of the outwards (area scaled) normals \(d\Bn_3, d\Bn_1, d\Bn_2\) on the \(a_{3+}, a_{1+}\) and \(a_{2+}\) surfaces respectively we can write

\begin{equation}\label{eqn:relativisticElectrodynamicsPS6:540}
\int d^3 \Bx \partial_\beta T^{\beta \alpha} = 
\int d\Bn_3 \cdot \Be_\beta \evalbar{T^{\beta}{\alpha}}{\Delta a_3}
+\int d\Bn_1 \cdot \Be_\beta \evalbar{T^{\beta}{\alpha}}{\Delta a_1}
+\int d\Bn_2 \cdot \Be_\beta \evalbar{T^{\beta}{\alpha}}{\Delta a_2}.
\end{equation}

This can be written more concisely in index form with

\begin{equation}\label{eqn:relativisticElectrodynamicsPS6:640}
d^2 \sigma^\beta = 
\epsilon_{\mu\nu\beta} \left(
\PD{a_2}{x^\mu}
\PD{a_3}{x^\nu} da_2 da_3
+\PD{a_3}{x^\mu}
\PD{a_1}{x^\nu} da_3 da_1
+\PD{a_1}{x^\mu}
\PD{a_2}{x^\nu} da_1 da_2
\right),
\end{equation}

so that the divergence integral is just

\begin{equation}\label{eqn:relativisticElectrodynamicsPS6:560}
\int d^3 \Bx = 
\int_{\text{over level surfaces \(a_{1+}\), \(a_{2+}\), \(a_{3+}\)}} d^2 \sigma^\beta T^{\beta \alpha}
-\int_{\text{over level surfaces \(a_{1-}\), \(a_{2-}\), \(a_{3-}\)}} d^2 \sigma^\beta T^{\beta \alpha}
\end{equation}

In each case, for the \(a_{\alpha-}\) surfaces, our negated inwards normal form can be redefined so that we integrate over only the outwards normal directions, and we can use the oriented integral notation

\begin{equation}\label{eqn:relativisticElectrodynamicsPS6:580}
\int d^3 \Bx = \oint d^2 \sigma^\beta T^{\beta \alpha},
\end{equation}

To encode (or imply) whether we require a positive or negative sign on the area element tensor of \eqnref{eqn:relativisticElectrodynamicsPS6:640} for the surface in question.

\section{A look back, and looking forward}

Now, having performed this long winded calculation, the meaning of \(d^2 \sigma^\beta\) becomes clear.  What is also clear is how this could have been arrived at directly utilizing the divergence theorem in its normal vector form.  We had only to re-write our equation as a vector equation in terms of the gradient

\begin{equation}\label{eqn:relativisticElectrodynamicsPS6:600}
\int_V d^3 \Bx \PD{x^\alpha}{T^{\beta \alpha}} = \int_V d^3 \Bx \spacegrad \cdot (\Be_\beta T^{\beta \alpha}) = \int_{\partial_V} dA \Bn \cdot \Be_\beta T^{\beta \alpha}
\end{equation}

From this we see directly that \(d^2 \sigma^\beta = dA \Bn \cdot \Be_\beta\).

Despite there being an easier way to find the form of \(d^2 \sigma^\beta\), I still consider this a worthwhile exercise.  It hints how one could generalize the arguments to the higher dimensional cases.  The main task would be to construct the normals to the hypersurfaces bounding the hypervolume, and how to do this algebraically utilizing determinants may not be too hard (since we want a Jacobian determinant as the hypervolume element in the ``volume'' integral).  We also got more than the normal physics text book proof of the divergence theorem for Cartesian coordinates, and did it here for a general parametrization.  This was not a complete argument since we did not consider a general surface, broken down into a triangular mesh.  We really want volume elements with triangular sides instead of parallelograms.
