%
% Copyright � 2012 Peeter Joot.  All Rights Reserved.
% Licenced as described in the file LICENSE under the root directory of this GIT repository.
%

\label{chap:relElectroDynProblemSet1}
%\blogpage{http://sites.google.com/site/peeterjoot/math2011/relElectroDynProblemSet1.pdf}
%\date{Jan 22, 2011}

%\section{Disclaimer}
%
%This problem set is as yet ungraded.  It is also incomplete since I ended up hand-writing some parts.
%
%\section{Problem 1}
%\subsection{Statement}
%
%Some Minkowski diagram exercises.
%
%\subsection{Solution}
%
%These will be hand written.
%%%I will hand write my solutions for these.  To prep for it, let us suppose that $(ct', x')$ and $(ct, x)$ are related by 
%%%
%%%\begin{equation}\label{eqn:relativisticElectrodynamicsA1:1}
%%%\begin{bmatrix}
%%%x' \\
%%%ct'
%%%\end{bmatrix}
%%%\begin{bmatrix}
%%%\cosh \alpha & \sinh\alpha \\
%%%\sinh \alpha & \cosh\alpha 
%%%\end{bmatrix}
%%%\begin{bmatrix}
%%%x \\
%%%ct
%%%\end{bmatrix}.
%%%\end{equation}
%%%
%%%Some rearrangement, after eliminating $x$ yields a hyperbolic family of curves, one for each value of $ct$
%%%
%%%\begin{equation}\label{eqn:relativisticElectrodynamicsA1:2}
%%%\frac{(ct'/ct)^2}{1 -\beta^2} - \frac{(x'/ct)^2}{(1-\beta^2)/\beta^2} = 1
%%%\end{equation}
%%%
%%%Considering the $ct = 0$ curves for $ct\ne 0$ we get real solutions for $x$ thus, fixing the orientation of the hyperbolas, so we see these hyperbolas cross the x-axis, and not the $ct$-axis.


\makeproblem{Transformation of velocities}{pr:relElectroDynProblemSet2:2}{

From the Lorentz transformations of space and time coordinates.


\makesubproblem{Derive the transformation of velocities.}{pr:relElectroDynProblemSet2:2:a}

With a particle moving with $\Bv$ in the unprimed (stationary) frame, find its velocity $\Bv'$ in the primed frame.  The primed frame is moving with some $\BV$ with respect to the unprimed one.  Make sure to finally derive the general ``addition of velocities'' equation in terms of vectors and dot products, as given in \citep{landau1971classical}.

\makesubproblem{Velocities relative to \( c \).}{pr:relElectroDynProblemSet2:2:b}

Then, use the addition of velocities rule to show that: 

\begin{enumerate}
\item 
%a) 
if $v < c$ in one frame, then $v' < c$ in any other frame.  
\item 
%b.) 
If $v = c$ in one frame, then $v' = c$ in any other frame, and 
\item
%c.) 
if $v> c$ in one frame, than $v' > c$ in any other frame.
\end{enumerate}
} % makeproblem

\makeanswer{pr:relElectroDynProblemSet2:2}{


\makeSubAnswer{}{pr:relElectroDynProblemSet2:2:a}



We need a vector form of the Lorentz transform to start with.  Let us write $\Bsigma$ for a unit vector colinear with the primed frame velocity $\BV$, so that $\BV = (\BV \cdot \Bsigma) \Bsigma$.  When our boost was in the $x$ direction, our Lorentz transformation was in terms of $x = \Bx \cdot \xcap$.  The component in the direction of the boost is now $\Bx \cdot \Bsigma$, and we have

\begin{subequations}
\begin{equation}\label{eqn:relativisticElectrodynamicsA1:200}
\begin{aligned}
c t' &= \gamma \left( ct - (\Bx \cdot \Bsigma) \frac{\BV \cdot \Bsigma}{c} \right) \\
\Bx' \cdot \Bsigma &= \gamma \left( \Bx \cdot \Bsigma - \frac{\BV \cdot \Bsigma}{c} c t \right) \\
\Bx' \wedge \Bsigma &= \Bx \wedge \Bsigma .
\end{aligned}
\end{equation}
\end{subequations}

We can add the vector components using $\Bx = (\Bx \cdot \Bsigma) \Bsigma + (\Bx \wedge \Bsigma) \Bsigma$, leaving

\begin{subequations}
\begin{equation}\label{eqn:relativisticElectrodynamicsA1:210}
\begin{aligned}
c t' &= \gamma \left( ct - (\Bx \cdot \Bsigma) \frac{\BV \cdot \Bsigma}{c} \right) \\
\Bx' &= (\Bx \wedge \Bsigma) \Bsigma + \gamma \left( (\Bx \cdot \Bsigma) \Bsigma - \frac{\BV}{c} c t \right) .
\end{aligned}
\end{equation}
\end{subequations}

Writing $(\Bx \wedge \Bsigma) \Bsigma = \Bx - (\Bx \cdot \Bsigma)\Bsigma$ we have for the spatial component transformation

\begin{equation}\label{eqn:relativisticElectrodynamicsA1:220}
\Bx' = \Bx + (\Bx \cdot \Bsigma) \Bsigma (\gamma - 1) - \gamma \frac{\BV}{c} c t.
\end{equation}

Now we are set to take derivatives to calculate the velocities.  This gives us
\begin{subequations}
\begin{equation}\label{eqn:relativisticElectrodynamicsA1:230}
\begin{aligned}
\frac{dt'}{dt} &= \gamma \left( 1 - \left( \frac{d\Bx}{dt} \cdot \Bsigma \right) \frac{\BV \cdot \Bsigma}{c^2} \right) \\
\frac{d\Bx'}{dt'} \frac{d t'}{dt} &= \frac{d\Bx}{dt} + \left(\frac{d\Bx}{dt} \cdot \Bsigma\right) \Bsigma (\gamma - 1) - \gamma \frac{\BV}{c} c .
\end{aligned}
\end{equation}
\end{subequations}

Dividing this pair of equations, and using $\Bv = d\Bx/dt$, and $\Bv' = d\Bx'/dt'$, this is

\begin{equation}\label{eqn:relativisticElectrodynamicsA1:240}
\Bv' = \frac{\gamma^{-1} \Bv + (\Bv \cdot \Bsigma) \Bsigma (1 - \gamma^{-1}) - \BV}{ 1 - \left( \Bv \cdot \Bsigma \right) (\BV \cdot \Bsigma)/c^2 }.
\end{equation}

Since $\BV$ and our direction vector $\Bsigma$ are colinear, we have $(\Bv \cdot \Bsigma) (\BV \cdot \Bsigma) = \Bv \cdot \Bsigma$, and can simplify this last expression slightly

\boxedEquation{eqn:relativisticElectrodynamicsA1:250}{
\Bv' = \frac{\gamma^{-1} \Bv + (\Bv \cdot \Bsigma) \Bsigma (1 - \gamma^{-1}) - \BV}{ 1 - \Bv \cdot \BV/c^2 }.
}

Finally, if we are to compare to the text, we note that the inverse expression requires replacement of $\BV$ with $-\BV$ and switching $\Bv$ with $\Bv'$.  That gives us

\begin{equation}\label{eqn:relativisticElectrodynamicsA1:250i}
\Bv = \frac{\gamma^{-1} \Bv' + (\Bv' \cdot \Bsigma) \Bsigma (1 - \gamma^{-1}) + \BV}{ 1 + \Bv' \cdot \BV/c^2 }.
\end{equation}

The expression in the text is also a small velocity approximation.  For $\Abs{\BV} \ll c$, we have $\gamma^{-1} \approx 1$, and $(1 + \Bv' \cdot \BV/c^2)^{-1} \approx 1 - \Bv' \cdot \BV/c^2$.  This gives us

\begin{equation}\label{eqn:relativisticElectrodynamicsA1:250a}
\Bv \approx (\Bv' + \BV)( 1 - \Bv' \cdot \BV/c^2 ) \approx \BV + \Bv' - \Bv' (\Bv' \cdot \BV)/c^2
\end{equation}

One additional approximation was made dropping the $\BV (\Bv' \cdot \BV)/c^2$ term which is quadratic in $\BV/c$, which leave us with equation $5.3$ in the text as desired.

\makeSubAnswer{}{pr:relElectroDynProblemSet2:2:b}


In \eqnref{eqn:relativisticElectrodynamicsA1:250i}, let us write $\Bv' = u \Bu$, where $\Bu$ is a unit vector, $V = \BV \cdot \Bsigma$, and $\alpha = \Bu \cdot \Bsigma$ for the direction cosine between the primed frame's direction of motion and the particle's velocity direction (also in the unprimed frame).  The stationary frame's particle velocity is then

\begin{equation}\label{eqn:relativisticElectrodynamicsA1:260}
\Bv = \frac{\gamma^{-1} u \Bu + u \alpha \Bsigma (1 - \gamma^{-1}) + V \Bsigma}{ 1 + \alpha u V/c^2 }.
\end{equation}

As a check, note that for $1 = \alpha = \Bu \cdot \Bsigma = \cos(0)$, we recover the familiar addition of velocities formula 

\begin{equation}\label{eqn:relativisticElectrodynamicsA1:260b}
\Bv = \Bu \frac{u + V}{ 1 + u V/c^2 }.
\end{equation}

We want to put \eqnref{eqn:relativisticElectrodynamicsA1:260} into a form that renders it more tractable for general angles too.  Factoring out the $\gamma^{-1}$ term appears to do the job, yielding

\begin{equation}\label{eqn:relativisticElectrodynamicsA1:260c}
\Bv = \frac{u \gamma^{-1} (\Bu -\alpha \Bsigma) + (u \alpha + V) \Bsigma}{ 1 + \alpha u V/c^2 }.
\end{equation}

After a bit of reduction and rearranging we can dot this with itself to calculate

\begin{equation}\label{eqn:relativisticElectrodynamicsA1:270}
\Bv^2 = \frac{V^2(1 - \alpha^2)(1 - u^2/c^2) + (u + \alpha V)^2}{ (1 + \alpha u V/c^2)^2 }
\end{equation}

Note that for $u = c$, we have $\Bv^2 = c^2$, regardless of the direction of $\BV$ with respect to the motion of the particle in the unprimed frame.  This should not be surprising since this light like invariance is exactly what the Lorentz transformation is designed to maintain.  It is however slightly comforting to know that the algebra appears to be still be kosher after all this.  This also answers part (b) of this question, since we have tackled the $v = c$ case in the primed frame, and seen that the speed remains $v = c$ in the unprimed frame (and thus any frame moving at constant speed relative to another).

Observe that since $1 - \alpha^2 = \sin^2\theta$, and $u \le c$, this is positive definite as expected.  If one allowed $u > c$ in some frame, our speed could go imaginary!

For the $u < c$ and $u > c$ cases, let $x = u/c$ and $y = V/c$.  This allows \eqnref{eqn:relativisticElectrodynamicsA1:270} to be casted in a simpler form

\begin{equation}\label{eqn:relativisticElectrodynamicsA1:270e}
\Bv^2 = c^2 \frac{y^2 (1 - \alpha^2)(1 - x^2) + (x + \alpha y)^2}{ (1 + \alpha x y)^2 }
\end{equation}

We wish to verify that (a) given any $x \in (-1,1)$, we have $\Bv^2 < c^2$ for all $y \in (-1,1)$, $\alpha \in (-1,1)$, and (c) given any $\Abs{x} > 1$, we have $\Bv^2 > c^2$ for all $y \in (-1,1)$, $\alpha \in (-1,1)$.

Considering (a) first, this requires a demonstration that 

\begin{equation}\label{eqn:relativisticElectrodynamicsA1:280}
y^2 (1 - \alpha^2)(1 - x^2) + (x + \alpha y)^2 < (1 + \alpha x y)^2 .
\end{equation}

Expanding out the products and canceling terms, we want to show that for (a) that if $\Abs{x},\Abs{y} < 1$ we have

\begin{equation}\label{eqn:relativisticElectrodynamicsA1:290a}
x^2 (1 - y^2) + y^2 < 1,
\end{equation}

and for (c) that if $\Abs{x} > 1$, we have for any $\Abs{y} < 1$

\begin{equation}\label{eqn:relativisticElectrodynamicsA1:290c}
x^2 (1 - y^2) + y^2 > 1.
\end{equation}

Observe that the frame velocity orientation direction cosines have completely dropped out, leaving just the (relative to $c$) velocity terms.

To get an initial feel for this function $f(x,y) = x^2 (1 - y^2) + y^2$, notice that \href{http://goo.gl/5AnNF}{when graphed} we have a bowl with a minimum (zero) at the origin, and what appears to be a uniform value of one on the boundary (case (b)).  Then provided $\Abs{y} < 1$ it appears that the function $f$ increases monotonically to a value greater than one (case (c)).  While looking at a plot is not any sort of rigorous proof, let us move on to some of the other problems for now, and return to this last loose thread later if time permits.


} % makeanswer


\makeproblem{Toy GPS model}{pr:relElectroDynProblemSet2:3}{

A toy model of a GPS system has satellites moving in a straight line with constant velocity $V_x$ and at a constant height $h$ (measured, e.g., along the y-axis) above ``ground'' (the x-axis).  The satellites broadcast the time in their rest frame as well as their location at a time of broadcast.  Imagine a person on the ground receives simultaneously broadcasts from two satellites, $A$ and $B$, reporting their locations $x_A'$ and $x_B'$ as well as times of broadcast (which happen to be equal), $t_A' = t_B'$.	

\makesubproblem{}{pr:relElectroDynProblemSet2:3:a}

Find a condition determining your position in $x$.  Evaluate it to find your deviation from the midpoint between the satellites to first order in $V_x/c$.

\makesubproblem{}{pr:relElectroDynProblemSet2:3:b}

For some real numbers, note that in reality there are 24 satellites, moving with $V ~4 \text{km}/s$, a distance $R \approx 2.7 \times 10^4 \text{km}$.  Use these numbers and the result from the previous problem (assuming a flat Earth, to be sure...) to get an idea whether (special) relativistic effects are important for the typical modern GPS accuracy of order 10 m (or less)?

} % makeproblem

\makeanswer{pr:relElectroDynProblemSet2:3}{

\makeSubAnswer{}{pr:relElectroDynProblemSet2:3:a}

We are looking for a worldpoints $(ct', x', y')$ in satellite frame on the light cone emanating from the satellite worldpoints $(ct_A', x_A', y_A')$, and $(ct_B', x_B', y_B')$.  These are

\begin{equation}\label{eqn:relElectroDynProblemSet2:1200}
\begin{aligned}
c^2 ( t_A' - t')^2 &= (x_A' - x')^2 + (y_A' - y')^2 \\
c^2 ( t_B' - t')^2 &= (x_B' - x')^2 + (y_B' - y')^2,
\end{aligned}
\end{equation}

where the worldpoints $(ct', x', y')$ are related to the stationary frame by

\begin{equation}\label{eqn:relElectroDynProblemSet2:1220}
\begin{bmatrix}
ct' \\
x' \\
y'
\end{bmatrix}
\begin{bmatrix}
\gamma & -\beta \gamma & 0 \\
-\beta \gamma & \gamma & 0 \\
0 & 0 & 1
\end{bmatrix}
\begin{bmatrix}
ct \\
x \\
y
\end{bmatrix}.
\end{equation}

The problem has been artificially simplified by stating that $t_A' = t_B'$, and we can eliminate the $y'$ terms since we want $y_A' - y' = h = y_B' - y'$ at the point where the signal is received.

Suppose that in the observer frame the light signals are received with event coordinates $(c t_0, x_0, 0)$.  In the satellites rest frame these are

\begin{equation}\label{eqn:relElectroDynProblemSet2:1240}
\begin{aligned}
ct' &= \gamma ( c t_0 - \beta x_0 ) \\
x' &= \gamma ( x_0 - \beta c t_0 )
\end{aligned}
\end{equation}

We can make these substitutions above, yielding

\begin{equation}\label{eqn:relElectroDynProblemSet2:1260}
\begin{aligned}
( c t_A' - \gamma c t_0 + \gamma \beta x_0)^2 &= (x_A' - \gamma x_0 + \gamma \beta c t_0 )^2 + h^2 \\
( c t_A' - \gamma c t_0 + \gamma \beta x_0)^2 &= (x_B' - \gamma x_0 + \gamma \beta c t_0 )^2 + h^2
\end{aligned}
\end{equation}

Observe that the $t_A' = t_B'$ condition allows us to equate the pair of RHS terms and thus have

\begin{equation}\label{eqn:relElectroDynProblemSet2:1280}
x_A' - \gamma x_0 + \gamma \beta c t_0 = \pm (x_B' - \gamma x_0 + \gamma \beta c t_0 )
\end{equation}

If we pick the positive root, then we have $x_A' = x_B'$, a perfectly valid mathematical solution, but not one that can be used for triangularization.  Taking the negative root instead and rearranging we have

\begin{equation}\label{eqn:relElectroDynProblemSet2:1301}
\gamma \beta c t_0 = \gamma x_0 - \inv{2}(x_A' + x_B')
\end{equation}

As a sanity check observe that if $\beta = 0$ we have $x_0 = \inv{2}(x_A' + x_B') = x_m'$, the midpoint in the satellite (also the observer frame for $\beta = 0$).  This is what we would expect if a simultaneous signal is received that emanated at the same time when both sources are at rest at the same height.

When $\beta \ne 0$ we have

\begin{equation}\label{eqn:relElectroDynProblemSet2:1302}
\gamma c t_0 = \inv{\beta}(\gamma x_0 - x_m'),
\end{equation}

allowing us to eliminate $\gamma c t_0$ terms from the equations we wish to solve

\begin{equation}\label{eqn:relElectroDynProblemSet2:1262}
\begin{aligned}
%( c t_A' - \inv{\beta}( \gamma x_0 - x_m') + \gamma \beta x_0)^2 &= (x_A' - \gamma x_0 + \gamma x_0 - x_m')^2 + h^2 \\
%( c t_A' - \inv{\beta}( \gamma x_0 - x_m') + \gamma \beta x_0)^2 &= (x_B' - \gamma x_0 + \gamma x_0 - x_m')^2 + h^2
\left( c t_A' - \inv{\beta}( \gamma x_0 - x_m') + \gamma \beta x_0 \right)^2 &= (x_A' - x_m')^2 + h^2 \\
\left( c t_A' - \inv{\beta}( \gamma x_0 - x_m') + \gamma \beta x_0 \right)^2 &= (x_B' - x_m')^2 + h^2.
\end{aligned}
\end{equation}

We can group the $\gamma x_0$ terms on the LHS nicely

\begin{equation}\label{eqn:relElectroDynProblemSet1:1540}
\begin{aligned}
- \inv{\beta} \gamma x_0 \gamma \beta x_0
&=
\gamma x_0 ( - \inv{\beta} + \beta ) \\
&=
\inv{\beta} \gamma x_0 ( - 1 + \beta^2 ) \\
&=
-\inv{\beta} x_0,
\end{aligned}
\end{equation}

leaving
\begin{equation}\label{eqn:relElectroDynProblemSet2:1400}
\left( c t_A' -\inv{\beta} x_0 + \inv{\beta} x_m' \right)^2 = (x_A' - x_m')^2 + h^2 = (x_B' - x_m')^2 + h^2.
\end{equation}

The value $\Abs{x_A' - x_m'} = \Abs{x_B' - x_m'} = \Abs{x_A' - x_B'}/2$ is half the separation $L'$ of the satellites in their rest frame, so we have

\begin{equation}\label{eqn:relElectroDynProblemSet2:1420}
c t_A' -\inv{\beta} x_0 + \inv{\beta} x_m' = \pm \sqrt{{L'}^2/4 + h^2},
\end{equation}

or
\begin{equation}\label{eqn:relElectroDynProblemSet2:1440}
x_0 = x_m' + \beta c t_A' \mp \beta \sqrt{{L'}^2/4 + h^2}
\end{equation}

Utilizing the inverse transformation we have for a x-axis spatial coordinate in the observer frame

\begin{equation}\label{eqn:relElectroDynProblemSet2:1460}
x = \gamma ( x' + \beta c t'),
\end{equation}

allowing the $t_A'$ term to be eliminated in favour of the position that the midpoint between the satellites would have been observed at time $t_A'$.  This gives us

\begin{equation}\label{eqn:relElectroDynProblemSet2:1480}
x_0 = \inv{\gamma} x_m \mp \beta \sqrt{{L'}^2/4 + h^2}
\end{equation}

FIXME: Which sign is correct for this problem?  I had guess the negative sign.  Fixing that is probably the toughest part of this problem!

\makeSubAnswer{}{pr:relElectroDynProblemSet2:3:b}

FIXME: Had hand written notes for this part of the problem, with how I'd attempted it first (considering the actual geometric problem in 3D.)

} % makeanswer

