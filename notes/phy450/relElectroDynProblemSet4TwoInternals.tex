\section{Problem 2.  Spherical EM waves.}

\subsection{Statement}

Suppose you are given:

\begin{equation}\label{eqn:relElectroDynProblemSet4:400}
\vec{E}(r, \theta, \phi, t) = A \frac{\sin\theta}{r} \left( \cos(k r - \omega t) - \inv{k r} \sin(k r - \omega t) \right) \phicap
\end{equation}

where $\omega = k/c$ and $\phicap$ is the unit vector in the $\phi$-direction.  This is a simple example of a spherical wave.

\begin{enumerate}
\item Show that $\vec{E}$ obeys all four Maxwell equations in vacuum and find the associated magnetic field.
\item Calculate the Poynting vector.  Average $\vec{S}$ over a full cycle to get the intensity vector $\vec{I} \equiv \expectation{\vec{S}}$.  Where does it point to?  How does it depend on $r$?
\item Integrate the intensity vector flux through a spherical surface centered at the origin to find the total power radiated.
\end{enumerate}

\subsection{Solution}

\subsubsection{Part 1.  Maxwell equation verification and magnetic field.}

Our vacuum Maxwell equations to verify are

\begin{align}\label{eqn:relElectroDynProblemSet4:420}
\grad \cdot \vec{E} &= 0 \\
\grad \cross \vec{B} -\inv{c} \PD{t}{\vec{E}} &= 0 \\
\grad \cdot \vec{B} &= 0 \\
\grad \cross \vec{E} +\inv{c} \PD{t}{\vec{B}} &= 0.
\end{align}

We'll also need the spherical polar forms of the divergence and curl operators, as found in \S 1.4 of \cite{griffith1981introduction}

\begin{align}\label{eqn:relElectroDynProblemSet4:440}
\grad \cdot \vec{v} &=
\inv{r^2} \partial_r ( r^2 v_r )
+ \inv{r\sin\theta} \partial_\theta (\sin\theta v_\theta)
+ \inv{r\sin\theta} \partial_\phi v_\phi \\
\grad \cross \vec{v} &=
\inv{r \sin\theta} \left(
\partial_\theta (\sin\theta v_\phi) - \partial_\phi v_\theta
\right) \rcap
+
\inv{r } \left(
\inv{\sin\theta} \partial_\phi v_r - \partial_r (r v_\phi)
\right) \thetacap
+
\inv{r } \left(
\partial_r (r v_\theta) - \partial_\theta v_r
\right) \phicap
\end{align}

We can start by verifying the divergence equation for the electric field.  Observe that our electric field has only an $E_\phi$ component, so our divergence is

\begin{equation}\label{eqn:relElectroDynProblemSet4:460}
\grad \cdot \vec{E}
=
\inv{r\sin\theta} \partial_\phi \left(
A \frac{\sin\theta}{r} \left( \cos(k r - \omega t) - \inv{k r} \sin(k r - \omega t) \right) \right) = 0.
\end{equation}

We have a zero divergence since the component $E_\phi$ has no $\phi$ dependence (whereas $\vec{E}$ itself does since the unit vector $\phicap = \phicap(\phi)$).

All of the rest of Maxwell's equations require $\vec{B}$ so we'll have to first calculate that before progressing further.

\subsubsection{A aside on approaches attempted to find $\vec{B}$}

I tried two approaches without success to calculate $\vec{B}$.  First I hoped that I could just integrate $-\vec{E}$ to obtain $\vec{A}$ and then take the curl.  Doing so gave me a result that had $\grad \cross \vec{B} \ne 0$.  I hunted for an algebraic error that would account for this, but could not find one.

The second approach that I tried, also without success, was to simply take the cross product $\rcap \cross \vec{E}$.  This worked in the monochromatic plane wave case where we had

\begin{align}\label{eqn:relElectroDynProblemSet4:480}
\vec{B} &= (\vec{k} \cross \vec{\beta}) \sin(\omega t - \vec{k} \cdot \vec{x}) \\
\vec{E} &= \vec{\beta} \Abs{\vec{k}} \sin(\omega t - \vec{k} \cdot \vec{x})
\end{align}

since one can easily show that $\vec{B} = \vec{k} \cross \vec{E}$.  Again, I ended up with a result for $\vec{B}$ that did not have a zero divergence.

\subsubsection{Finding $\vec{B}$ with a more systematic approach.}

Following \cite{jackson1975cew} \S 16.2, let's try a phasor approach, assuming that all the solutions, whatever they are, have all the time dependence in a $e^{-i\omega t}$ term.

Let's write our fields as

\begin{align}\label{eqn:relElectroDynProblemSet4:500}
\vec{E} &= \Real (\BE e^{-i \omega t}) \\
\vec{B} &= \Real (\BB e^{-i \omega t}).
\end{align}

Substitution back into Maxwell's equations thus requires equality in the real parts of

\begin{align}\label{eqn:relElectroDynProblemSet4:520}
\grad \cdot \BE &= 0 \\
\grad \cdot \BB &= 0 \\
\grad \cross \BB &= - i \frac{\omega}{c} \BE \\
\grad \cross \BE &= i \frac{\omega}{c} \BB
\end{align}

With $k = \omega/c$ we can now directly compute the magnetic field phasor

\begin{equation}\label{eqn:relElectroDynProblemSet4:540}
\BB = -\frac{i}{k} \grad \cross \BE.
\end{equation}

The electric field of this problem can be put into phasor form by noting

\begin{equation}\label{eqn:relElectroDynProblemSet4:560}
\vec{E} = A \frac{\sin\theta}{r} \Real \left( e^{i (k r - \omega t)} - \frac{i}{k r} e^{i(k r - \omega t)} \right) \phicap,
\end{equation}

which allows for reading off the phasor part directly

\begin{equation}\label{eqn:relElectroDynProblemSet4:580}
\BE = A \frac{\sin\theta}{r} \left( 1 - \frac{i}{k r} \right) e^{i k r} \phicap.
\end{equation}

Now we can compute the magnetic field phasor $\BB$.  Since we have only a $\phi$ component in our field, the curl will have just $\rcap$ and $\thetacap$ components.  This is reasonable since we expect it to be perpendicular to $\BE$.

\begin{equation}\label{eqn:relElectroDynProblemSet4:590}
\grad \cross (v_\phi \phicap) 
= \inv{r \sin\theta} \partial_\theta (\sin\theta v_\phi) \rcap
- \inv{r } \partial_r (r v_\phi) \thetacap.
\end{equation}

Chugging through all the algebra we have

\begin{align*}
i k \BB 
&=
\grad \cross \BE \\
&=
\frac{2 A \cos\theta}{r^2} 
\left( 1 - \frac{i}{k r} \right) e^{i k r} 
\rcap
- \frac{A\sin\theta}{r } \PD{r}{} \left( \left( 1 - \frac{i}{k r} \right) e^{i k r} \right)
\thetacap \\
&=
\frac{2 A \cos\theta}{r^2} 
\left( 1 - \frac{i}{k r} \right) e^{i k r} 
\rcap
- \frac{A\sin\theta}{r } \left( i k + \inv{r} + \frac{i}{k r^2} \right) e^{i k r} 
\thetacap,
\end{align*}

so our magnetic phasor is
\begin{equation}\label{eqn:relElectroDynProblemSet4:600}
\BB =
\frac{2 A \cos\theta}{k r^2} 
\left( -i - \frac{1}{k r} \right) e^{i k r} 
\rcap
- \frac{A\sin\theta}{r} \left( 1 - \frac{i}{k r} + \frac{1}{k^2 r^2} \right) e^{i k r} 
\thetacap
\end{equation}

Multiplying by $e^{-i\omega t}$ and taking real parts gives us the messy magnetic field expression

\begin{equation}\label{eqn:relElectroDynProblemSet4:610}
\begin{aligned}
\vec{B} 
&=
\frac{A}{r} \frac{2 \cos\theta}{k r} 
\left( \sin(k r - \omega t)
- \frac{1}{k r} \cos(k r - \omega t) \right)
\rcap \\
%&- \frac{A}{r} \sin\theta
%\left(
%\left(
%1 + \frac{1}{k^2 r^2} 
%\right) \cos(k r - \omega t)
%+ \inv{k r} \sin(k r - \omega t)
%\right)
%\thetacap
&- \frac{A}{r} \frac{\sin\theta}{k r}
\left(
\sin(k r - \omega t)
+ \frac{k^2 r^2 + 1}{k r}
\cos(k r - \omega t)
\right)
\thetacap.
\end{aligned}
\end{equation}

Since this was constructed directly from $\grad \cross \vec{E} +\inv{c} \PDi{t}{\vec{B}} = 0$, this implicitly verifies one more of Maxwell's equations, leaving only $\grad \cdot \vec{B}$, and $\grad \cross \vec{B} -\inv{c} \PDi{t}{\vec{E}} = 0$.  Neither of these looks particularly fun to verify, however, we can take a small shortcut and use the phasors to verify without the explicit time dependence.

From \ref{eqn:relElectroDynProblemSet4:600} we have for the divergence
\begin{align*}
\grad \cdot \BB 
&=
\frac{2 A \cos\theta}{k r^2 } 
\PD{r}{} \left(
\left( -i - \frac{1}{k r} \right) e^{i k r} 
\right)
- \frac{A 2 \cos\theta}{r^2} \left( 1 - \frac{i}{k r} + \frac{1}{k^2 r^2} \right) e^{i k r}  \\
&=
\frac{2 A \cos\theta}{r^2 } e^{i k r}
\left(
\inv{k}\left( \inv{k r^2} + i k \left(-i - \inv{k r}\right)
\right)
-
\left( 1 - \frac{i}{k r} + \frac{1}{k^2 r^2} \right) 
\right) \\
&= 0 \qquad \square
\end{align*}

Let's also verify the last of Maxwell's equations in phasor form.  The time dependence is knocked out, and we want to see that taking the curl of the magnetic phasor returns us (scaled) the electric phasor.  That is

\begin{equation}\label{eqn:relElectroDynProblemSet4:620}
\grad \cross \BB = - i \frac{\omega}{c} \BE
\end{equation}

With only $r$ and $\theta$ components in the magnetic phasor we have

\begin{equation}\label{eqn:relElectroDynProblemSet4:630}
\grad \cross (v_r \rcap + v_\theta \thetacap) 
=
-\inv{r \sin\theta} 
\partial_\phi v_\theta
\rcap
+
\inv{r } 
\inv{\sin\theta} \partial_\phi v_r 
\thetacap
+
\inv{r } \left(
\partial_r (r v_\theta) - \partial_\theta v_r
\right) \phicap
\end{equation}

Immediately, we see that with no explicit $\phi$ dependence in the coordinates, we have no $\rcap$ nor $\thetacap$ terms in the curl, which is good.  Our curl is now just

\begin{align*}
\grad \cross \BB 
&=
\inv{r } \left(
 A\sin\theta 
\partial_r \left( 1 - \frac{i}{k r} + \frac{1}{k^2 r^2} \right) e^{i k r} 
+\frac{2 A \sin\theta}{k r^2} 
\left( -i - \frac{1}{k r} \right) e^{i k r} 
\right) \phicap \\
&=
A \sin\theta 
\inv{r } \left(
\partial_r \left( 1 - \frac{i}{k r} + \frac{1}{k^2 r^2} \right) e^{i k r} 
+\frac{2 }{k r^2} 
\left( -i - \frac{1}{k r} \right) e^{i k r} 
\right) \phicap \\
&=
A \sin\theta e^{i k r}
\inv{r } \left(
(ik)\left( 1 - \frac{i}{k r} + \frac{1}{k^2 r^2} \right) 
+\left( \frac{i}{k r^2} - \frac{2}{k^2 r^3} \right) 
+\frac{2 }{k r^2} 
\left( -i - \frac{1}{k r} \right) 
\right) \phicap \\
&=
A \sin\theta e^{i k r}
\inv{r } \left(
i k + \inv{r} - \frac{ 4 }{k^2 r^3}
\right) \phicap \\
\end{align*}

What we expect is $\grad \cross \BB = - i k \BE$ which is

\begin{equation}\label{eqn:relElectroDynProblemSet4:640}
- i k \BE =
A \sin\theta e^{i k r}
\inv{r } \left(
- i k - \inv{r}
\right) \phicap 
\end{equation}

FIXME: Somewhere I must have made a sign error, because these aren't matching!  Have an extra $1/r^3$ term and the wrong sign on the $1/r$ term.

\subsubsection{Part 2.  Poynting and intensity.}

Our Poynting vector is

\begin{equation}\label{eqn:relElectroDynProblemSet4:650}
\vec{S} = \frac{c}{4 \pi} \vec{E} \cross \vec{B},
\end{equation}

which we could calculate from \ref{eqn:relElectroDynProblemSet4:400}, and \ref{eqn:relElectroDynProblemSet4:610}.  However, that looks like it's going to be a mess to multiply out.  Let's use instead the trick from \S 48 of the course text \cite{landau1980classical}, and work with the complex quantities directly, noting that we have

\begin{align*}
(\Real \BE e^{i \alpha}) \cross (\Real \BB e^{i \alpha}) 
&= \inv{4} 
( \BE e^{i \alpha} + \BE^\conj e^{-i \alpha}) \cross ( \BB e^{i \alpha} + \BB^\conj e^{-i \alpha}) \\
&= \inv{2} \Real \left( \BE \cross \BB^\conj + (\BE \cross \BB) e^{2 i \alpha} \right).
\end{align*}

Now we can do the Poynting calculation using the simpler relations \ref{eqn:relElectroDynProblemSet4:580}, \ref{eqn:relElectroDynProblemSet4:600}.

Let's also write
\begin{align}\label{eqn:relElectroDynProblemSet4:655}
\BE &= A e^{i k r} E_\phi \phicap \\
\BB &= A e^{i k r} ( B_r \rcap + B_\theta \thetacap )
\end{align}

where

\begin{align}\label{eqn:relElectroDynProblemSet4:660}
E_\phi &= \frac{\sin\theta}{r} \left( 1 - \frac{i}{k r} \right)  \\
B_r &= -\frac{2 \cos\theta}{k r^2} \left( i + \frac{1}{k r} \right)  \\
B_\theta &= - \frac{\sin\theta}{r} \left( 1 - \frac{i}{k r} + \frac{1}{k^2 r^2} \right) 
\end{align}

So our Poynting vector is

\begin{align*}
\vec{S} 
&= \frac{A^2 c}{2 \pi} \Real
\left(
E_\phi \phicap \cross ( B_r^\conj \rcap + B_\theta^\conj \thetacap )
+
E_\phi \phicap \cross ( B_r \rcap + B_\theta \thetacap ) e^{ 2 i ( k r - \omega t ) }
\right) \\
\end{align*}

Note that our unit vector basis $\{ \rcap, \thetacap, \phicap \}$ was rotated from $\{ \zcap, \xcap, \ycap \}$, so we have

\begin{align}\label{eqn:relElectroDynProblemSet4:670}
\phicap \cross \rcap &= \thetacap \\
\thetacap \cross \phicap &= \rcap \\
\rcap \cross \thetacap &= \phicap ,
\end{align}

and plug this into our Poynting expression

\begin{align*}
\vec{S} 
&= \frac{A^2 c}{2 \pi} \Real
\left(
E_\phi B_r^\conj \thetacap 
-E_\phi B_\theta^\conj \rcap 
+(E_\phi B_r \thetacap 
-E_\phi B_\theta \rcap )
e^{ 2 i ( k r - \omega t ) }
\right) \\
\end{align*}

Now we have to multiply out our terms.  We have

\begin{align*}
E_\phi B_r^\conj &=
- \frac{\sin\theta}{r} \frac{2 \cos\theta}{k r^2} 
\left( 1 - \frac{i}{k r} \right)
\left( -i + \frac{1}{k r} \right) \\
&=
-\frac{ \sin(2\theta)}{k r^3}
\left( -i - \frac{i}{k^2 r^2} \right),
\end{align*}

Since this has no real part, there is no average contribution to $\vec{S}$ in the $\thetacap$ direction.  What do we have for the time dependent part

\begin{align*}
E_\phi B_r &=
- \frac{\sin\theta}{r} \frac{2 \cos\theta}{k r^2} 
\left( 1 - \frac{i}{k r} \right)
\left( i + \frac{1}{k r} \right) \\
&=
-\frac{ \sin(2\theta)}{k r^3}
\left( i + \frac{2}{k r} - \frac{i}{k^2 r^2} \right) 
\end{align*}

This is non zero, so we have a time dependent $\thetacap$ contribution that averages out.  Moving on

\begin{align*}
- E_\phi B_\theta^\conj
&= 
\frac{\sin^2\theta}{r^2} 
\left( 1 - \frac{i}{k r} \right)
\left( 1 + \frac{i}{k r} + \frac{1}{k^2 r^2} \right) \\
&= 
\frac{\sin^2\theta}{r^2} 
\left( 1 + \frac{2}{k^2 r^2} - \frac{i}{k^3 r^3}\right).
\end{align*}

This is non-zero, so the steady state Poynting vector is in the outwards radial direction.  The last piece is

\begin{align*}
- E_\phi B_\theta
&= 
\frac{\sin^2\theta}{r^2} 
\left( 1 - \frac{i}{k r} \right)
\left( 1 - \frac{i}{k r} + \frac{1}{k^2 r^2} \right) \\
&= 
\frac{\sin^2\theta}{r^2} 
\left( 1 - \frac{2i}{k r} - \frac{i}{k^3 r^3}\right).
\end{align*}

Assembling all the results we have

\begin{align*}
\begin{aligned}
\vec{S} 
&= 
\frac{A^2 c}{2 \pi} 
\frac{\sin^2\theta}{r^2} 
\left( 1 + \frac{2}{k^2 r^2} \right) \rcap \\
&\quad +
\frac{A^2 c}{2 \pi} 
\Real \left(
\left(
-\frac{ \sin(2\theta)}{k r^3} \left( i + \frac{2}{k r} - \frac{i}{k^2 r^2} \right) \thetacap
+\frac{\sin^2\theta}{r^2} \left( 1 - \frac{2i}{k r} - \frac{i}{k^3 r^3}\right) \rcap 
\right) e^{ 2 i ( k r - \omega t ) }
\right) 
\end{aligned}
\end{align*}

We can read off the intensity directly

\begin{equation}\label{eqn:relElectroDynProblemSet4:680}
\vec{I} = \expectation{\vec{S}} = 
\frac{A^2 c \sin^2 \theta}{2 \pi r^2} 
\left( 1 + \frac{2}{k^2 r^2} \right) \rcap 
\end{equation}

\subsubsection{Part 3.  Find the power.}

Through a surface of radius $r$, integration of the intensity vector \ref{eqn:relElectroDynProblemSet4:680} is

\begin{align*}
\int 
r^2 \sin\theta d\theta d\phi
\vec{I} 
&= 
\int r^2 \sin\theta d\theta d\phi 
\frac{A^2 c \sin^2 \theta}{2 \pi r^2} 
\left( 1 + \frac{2}{k^2 r^2} \right) \rcap \\
&= 
A^2 c 
\left( 1 + \frac{2}{k^2 r^2} \right) \rcap 
\int_0^\pi \sin^3\theta d\theta \\
&= 
A^2 c 
\left( 1 + \frac{2}{k^2 r^2} \right) \rcap 
{\left.\inv{12}( \cos(3\theta) - 9 \cos\theta )\right\vert}_0^\pi.
\end{align*}

Our average power through the surface is therefore

\begin{equation}\label{eqn:relElectroDynProblemSet4:690}
\int d^2 \Bsigma \vec{I} =
\frac{4 A^2 c }{3}
\left( 1 + \frac{2}{k^2 r^2} \right) \rcap.
\end{equation}
