%
% Copyright © 2012 Peeter Joot.  All Rights Reserved.
% Licenced as described in the file LICENSE under the root directory of this GIT repository.
%
\section{Action for the field}

However, \(\BE\) and \(\BB\) are created by charged particles and can ``move'' or ``propagate'' on their own.  EM field is its own dynamical system.  The variables are \(A^i(\Bx, t)\).  These are the ``\(q_a(t)\)''.

The values of \(\{A^i(\Bx, t), \forall \Bx\}\) is the dynamical degrees of freedom.  This is a system with a continuum of dynamical degrees of freedom.

We need to write an action for this continuous field system \(A^i(\Bx,t)\), and need some principles to guide the construction of this action.

When we have an action with many degrees of freedom, we sum over all the particles.  The action for the electromagnetic field 

\begin{equation}\label{eqn:relativisticElectrodynamicsL12b:90}
S_{\text{EM field}} = \int dt \int d^3\Bx \LL(A^i(\Bx, t))
\end{equation}

The quantity 

\begin{equation}\label{eqn:relativisticElectrodynamicsL12b:110}
\LL(A^i(\Bx, t))
\end{equation}

is called the Lagrangian density, since the quantity

\begin{equation}\label{eqn:relativisticElectrodynamicsL12b:130}
\int d^3\Bx \LL(A^i(\Bx, t))
\end{equation}

is actually the Lagrangian.

While this may seem non-relativistic, with both \(t\) and \(\Bx\) in the integration range, because we have both, it is actually relativistic.  We are integrating over all of spacetime, or the region where the EM fields are non-zero.

We write

\begin{equation}\label{eqn:relativisticElectrodynamicsL12b:150}
\int d^4 x  = c \int dt \int d^3 \Bx,
\end{equation}

which is a Lorentz scalar.

We write our action as 

\begin{equation}\label{eqn:relativisticElectrodynamicsL12b:170}
S_{\text{EM field}} = \int d^4 x \LL(A^i(\Bx, t))
\end{equation}

and demand that the Lagrangian density \(\LL\) must also be an invariant (Lorentz) scalar in \(SO(1,3)\).

\paragraph{Analogy}: 3D rotations

\begin{equation}\label{eqn:relativisticElectrodynamicsL12b:190}
\int d^3 \Bx \phi(\Bx)
\end{equation}

Here \(\phi\) is a 3-scalar, invariant under rotations.

\paragraph{Principles for the action}

\begin{enumerate}
\item Relativity.
\item Gauge invariance.  Whatever \(\LL\) we write, it must be gauge invariant, implying that it be a function of \(F_{ij}\) only.  Recall that we can adjust \(A^i\) by a four-gradient of any scalar, but the quantities \(\BE\) and \(\BB\) were gauge invariant, and so \(F^{ij}\) must also be.

If we do not impose gauge invariance, then the resulting dynamical system will contain more than just \(\BE\) and \(\BB\).  i.e. It will not be electromagnetism.

\item Superposition principle.  The sum of two solutions is a solution.  This implies linearity of the equations for \(A^i\).

\item Locality.  Could write

\begin{equation}\label{eqn:relativisticElectrodynamicsL12b:210}
\int d^4 x \LL_1(A) \int d^4 y \LL_2(A)
\end{equation}

This would allow for fields that have aspects that effect the result from disjoint positions or times.  This would probably result in non-causal results as well as the possibility of non-local results.
\end{enumerate}

Principle 1 means we must have 

\begin{equation}\label{eqn:relativisticElectrodynamicsL12b:230}
\LL(A(\Bx, t))
\end{equation}

and principle 2

\begin{equation}\label{eqn:relativisticElectrodynamicsL12b:250}
\LL(F^{ij}(\Bx, t))
\end{equation}

and principle 1, means we must have a four scalar.

Without principle 3, we could have products of these, but we rule this out due to violation of non-linearity.

\paragraph{Example.  Lagrangian for the Harmonic oscillator}

\begin{equation}\label{eqn:relativisticElectrodynamicsL12b:270}
\LL = \inv{2} m \dot{q}^2 - \inv{2} m \omega^2 q^2 
\end{equation}

This gives 

\begin{equation}\label{eqn:relativisticElectrodynamicsL12b:290}
\ddot{q} \sim q
\end{equation}

However, if we have

\begin{equation}\label{eqn:relativisticElectrodynamicsL12b:310}
\LL = \inv{2} m \dot{q}^2 - \inv{2} m \omega^2 q^2 - \lambda q^3
\end{equation}

we get

\begin{equation}\label{eqn:relativisticElectrodynamicsL12b:330}
\ddot{q} \sim q + q^3
\end{equation}

In HW3, you will show that 

\begin{equation}\label{eqn:relativisticElectrodynamicsL12b:350}
\int dt dx \BE \cdot \BB
\end{equation}

only depends on \(A^i\) at \(\infty\) (the boundary).  Because this depends only on \(A^i\) spatial or time infinities, it can not affect the variational principle.

This is very much like in classical mechanics where we can add any total derivative to the Lagrangian.  This does not change the Euler-Lagrange equation evaluation in any way.  The \(\BE \cdot \BB\) invariant has the same effect.

The invariants possible are \(\BE^2 - \BB^2\), \((\BE \cdot \BB)^2\), ..., but we are now done, and know what is required.  Our action must depend on \(F\) squared.

Written in full with the constants in the right places we have

\begin{equation}\label{eqn:relativisticElectrodynamicsL12b:370}
\begin{aligned}
&S_{\text{``particles in field''}} \\
&= \sum_A \left( -m_A c \int_{x^A(\tau)} ds - \frac{e_A}{c} \int dx^i_A A_i(x_A(\tau))
\right)
- \inv{16 \pi c} \int d^4 x F_{ij} F^{ij}
\end{aligned}
\end{equation}

To get the equation of motion for \(A^i(\Bx, t)\) we need to vary \(S_{\text{int}} + S_{\text{EM field}}\).

\section{Current density distribution}

Before we do the variation, we want to show that

\begin{equation}\label{eqn:relativisticElectrodynamicsL12b:590}
\begin{aligned}
S_{\text{int}} 
&= -\sum_A \frac{e_A}{c} \int_{x_A(\tau)} dx^i_A A_i(x_A(\tau) \\
&= -\inv{c^2} \int d^4 x A_i(x) j^i(x) 
\end{aligned}
\end{equation}

where

\begin{equation}\label{eqn:relativisticElectrodynamicsL12b:390}
j^i(x) = 
\sum_A c e_A \int_{x(\tau)}
ds u^i_A 
\delta(x^0 - x^0_A(\tau))
\delta(x^1 - x^1_A(\tau))
\delta(x^2 - x^2_A(\tau))
\delta(x^3 - x^3_A(\tau)).
\end{equation}

We substitute in the integral

\begin{equation}\label{eqn:relativisticElectrodynamicsL12b:610}
\begin{aligned}
&\sum_A \int d^4 x A_i(x) j^i(x) \\
&= 
c e_A \sum_A \int d^4 x A_i(x) \\
&\qquad \int_{x(\tau)}
ds u^i_A 
\delta(x^0 - x^0_A(\tau))
\delta(x^1 - x^1_A(\tau))
\delta(x^2 - x^2_A(\tau))
\delta(x^3 - x^3_A(\tau)) \\
&= 
c e_A \sum_A 
\int d^4 x \\
&\qquad \int_{x(\tau)}
dx^i_A 
A_i(x) 
\delta(x^0 - x^0_A(\tau))
\delta(x^1 - x^1_A(\tau))
\delta(x^2 - x^2_A(\tau))
\delta(x^3 - x^3_A(\tau)) \\
&=
c e_A \sum_A \\
&\qquad \int_{x_A(\tau)}
dx^i_A 
A_i(x_A(\tau)) 
\end{aligned}
\end{equation}

From this we see that we have

\begin{equation}\label{eqn:relativisticElectrodynamicsL12b:430}
S_{\text{int}} = -\inv{c^2} \int d^4 x A_i(x) j^i(x) 
\end{equation}

Physical meaning of \(j^i\)

Minkowski diagram at angle \(\arctan(v/c)\), with \(x^0\) axis up and \(x^1\) axis on horizontal.

\begin{equation}\label{eqn:relativisticElectrodynamicsL12b:450}
\begin{aligned}
x^0(\lambda) &= c \lambda \\
x^1(\lambda) &= v \lambda \\
x^2(\lambda) &= 0 \\
x^3(\lambda) &= 0
\end{aligned}
\end{equation}

Note that \(\lambda\) here is just a parameter.  \(\tau\) was used in the lecture, but that makes it appear that we missing a factor of \(\gamma\) above (if one did the end result would be the same since the delta evaluation would bring down a factor of \(1/\gamma\) to cancel it out).

\begin{equation}\label{eqn:relativisticElectrodynamicsL12b:470}
j^i(x) = e c \int dx^i(\lambda) \delta^4 (x - x(\lambda))
\end{equation}

\begin{equation}\label{eqn:relativisticElectrodynamicsL12b:490}
\begin{aligned}
j^0(x) &= e c^2  \int_{-\infty}^\infty d\lambda \delta(x^0 -  c \lambda) \delta(x^1 -  v \lambda) \delta(x^2) \delta(x^3) \\
j^1(x) &= e c v  \int_{-\infty}^\infty d\lambda \delta(x^0 -  c \lambda) \delta(x^1 -  v \lambda) \delta(x^2) \delta(x^3) \\
j^2(x) &= 0 \\
j^3(x) &= 0
\end{aligned}
\end{equation}

To evaluate the \(j^0\) integral, we have only the contribution from \(\lambda = x^0/ c\).  Recall that 

\begin{equation}\label{eqn:relativisticElectrodynamicsL12b:510}
\int dx \delta( b x - a) f(x) = \inv{\Abs{b}} f\left( \frac{a}{b} \right)
\end{equation}

This \(- c\lambda\) scaling of the delta function, kills a factor of \(c \) above, and leaves us with

\begin{equation}\label{eqn:relativisticElectrodynamicsL12b:530}
\begin{aligned}
j^0(x) &= e c \delta(x^1 - v x^0/c) \delta(x^2) \delta(x^3) \\
j^1(x) &= e v \delta(x^1 - v x^0/c) \delta(x^2) \delta(x^3) \\
j^2(x) &= 0 \\
j^3(x) &= 0 
\end{aligned}
\end{equation}

The current is non-zero only on the worldline of the particle.  We identify

\begin{equation}\label{eqn:relativisticElectrodynamicsL12b:550}
\rho(ct, x^1, x^2, x^3) = e \delta(x^1 - v x^0/c) \delta(x^2) \delta(x^3) 
\end{equation}

so that our current can be interpreted as the charge and current density 

\begin{equation}\label{eqn:relativisticElectrodynamicsL12b:570}
\begin{aligned}
j^0 &= c \rho(x) \\
j^\alpha(x) &= (\Bv)^\alpha \rho(x)
\end{aligned}
\end{equation}

Except for the delta functions these are just the quantities that we are familiar with from the RHS of Maxwell's equations.
