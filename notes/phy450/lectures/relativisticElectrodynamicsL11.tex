%
% Copyright � 2012 Peeter Joot.  All Rights Reserved.
% Licenced as described in the file LICENSE under the root directory of this GIT repository.
%

\chapter{Unpacking Lorentz force equation.  Lorentz transformations of the strength tensor, Lorentz field invariants, Bianchi identity, and first half of Maxwell's}
\label{chap:relativisticElectrodynamicsL11}
%\useCCL
%\blogpage{http://sites.google.com/site/peeterjoot/math2011/relativisticElectrodynamicsL11.pdf}
%\date{Feb 9, 2011}

\section{Reading}

Covering chapter 3 material from the text \citep{landau1980classical}.

Covering \href{http://www.physics.utoronto.ca/~poppitz/epoppitz/PHY450_files/RelEMpp74-83.pdf}{lecture notes pp. 74-83}: Lorentz transformation of the strength tensor (82) [Tuesday, Feb. 8] [extra reading for the mathematically minded: gauge field, strength tensor, and gauge transformations in differential form language, not to be covered in class (83)] 

Covering \href{http://www.physics.utoronto.ca/~poppitz/epoppitz/PHY450_files/RelEMpp84-102.pdf}{lecture notes pp. 84-102}: Lorentz invariants of the electromagnetic field (84-86); Bianchi identity and the first half of Maxwell's equations (87-90)

\section{Chewing on the four vector form of the Lorentz force equation}

After much effort, we arrived at 

\begin{equation}\label{eqn:relativisticElectrodynamicsL11:10}
\dds{(m c u_l) } = \frac{e}{c} \left( \partial_l A_i - \partial_i A_l \right) u^i
\end{equation}

or
\begin{equation}\label{eqn:relativisticElectrodynamicsL11:30}
\dds{ p_l } = \frac{e}{c} F_{l i} u^i
\end{equation}

\subsection{Elements of the strength tensor}

\paragraph{Claim}: there are only 6 independent elements of this matrix (tensor)

\begin{equation}\label{eqn:relativisticElectrodynamicsL11:50}
\begin{bmatrix}
0 & . & . & . \\ 
  & 0 & . & . \\ 
  &   & 0 & . \\ 
  &   &   & 0 \\ 
\end{bmatrix}
\end{equation}

This is a no-brainer, for we just have to mechanically plug in the elements of the field strength tensor

Recall

\begin{align}\label{eqn:relativisticElectrodynamicsL11:70}
A^i &= (\phi, \BA) \\
A_i &= (\phi, -\BA)
\end{align}

\begin{align*}
F_{0\alpha} 
&= 
\partial_0 A_\alpha - \partial_\alpha A_0  \\
&= 
-\partial_0 (\BA)_\alpha - \partial_\alpha \phi  \\
\end{align*}

\begin{equation}\label{eqn:relativisticElectrodynamicsL11:71}
F_{0\alpha} = E_\alpha
\end{equation}

For the purely spatial index combinations we have

\begin{align*}
F_{\alpha\beta} 
&= \partial_\alpha A_\beta - \partial_\beta A_\alpha  \\
&= -\partial_\alpha (\BA)_\beta + \partial_\beta (\BA)_\alpha  \\
\end{align*}

Written out explicitly, these are
\begin{align}\label{eqn:relativisticElectrodynamicsL11:90}
F_{1 2} &= \partial_2 (\BA)_1 -\partial_1 (\BA)_2  \\
F_{2 3} &= \partial_3 (\BA)_2 -\partial_2 (\BA)_3  \\
F_{3 1} &= \partial_1 (\BA)_3 -\partial_3 (\BA)_1 .
\end{align}

We can compare this to the elements of $\BB$

\begin{equation}\label{eqn:relativisticElectrodynamicsL11:110}
\BB = 
\begin{vmatrix}
\xcap & \ycap & \zcap \\
\partial_1 & \partial_2 & \partial_3 \\
A_x & A_y & A_z
\end{vmatrix}
\end{equation}

We see that 
\begin{align}\label{eqn:relativisticElectrodynamicsL11:130}
(\BB)_z &= \partial_1 A_y - \partial_2 A_x \\
(\BB)_x &= \partial_2 A_z - \partial_3 A_y \\
(\BB)_y &= \partial_3 A_x - \partial_1 A_z
\end{align}

So we have 

\begin{align}\label{eqn:relativisticElectrodynamicsL11:150}
F_{1 2} &= - (\BB)_3 \\
F_{2 3} &= - (\BB)_1 \\
F_{3 1} &= - (\BB)_2.
\end{align}

These can be summarized as simply

\begin{equation}\label{eqn:relativisticElectrodynamicsL11:170}
F_{\alpha\beta} = - \epsilon_{\alpha\beta\gamma} B_\gamma.
\end{equation}

This provides all the info needed to fill in the matrix above 

\begin{equation}\label{eqn:relativisticElectrodynamicsL11:190}
\Norm{ F_{i j} } = 
\begin{bmatrix}
0 & E_x & E_y & E_z \\
-E_x & 0 & -B_z & B_y \\
-E_y & B_z & 0 & -B_x \\
-E_z & -B_y & B_x & 0.
\end{bmatrix}.
\end{equation}

\subsection{Index raising of rank 2 tensor}

To raise indexes we compute

\begin{equation}\label{eqn:relativisticElectrodynamicsL11:210}
F^{i j} = g^{i l} g^{j k} F_{l k}.
\end{equation}

\subsubsection{Justifying the raising operation}
To justify this consider raising one index at a time by applying the metric tensor to our definition of $F_{l k}$.  That is

\begin{align*}
g^{a l} F_{l k} 
&=
g^{a l} (\partial_l A_k - \partial_k A_l) \\
&=
\partial^a A_k - \partial_k A^a.
\end{align*}

Now apply the metric tensor once more

\begin{align*}
g^{b k} g^{a l} F_{l k} 
&=
g^{b k} (\partial^a A_k - \partial_k A^a) \\
&=
\partial^a A^b - \partial^b A^a.
\end{align*}

This is, by definition $F^{a b}$.  Since a rank 2 tensor has been defined as an object that transforms like the product of two pairs of coordinates, it makes sense that this particular tensor raises in the same fashion as would a product of two vector coordinates (in this case, it happens to be an antisymmetric product of two vectors, and one of which is an operator, but we have the same idea).

\subsubsection{Consider the components of the raised $F_{i j}$ tensor}

\begin{align}\label{eqn:relativisticElectrodynamicsL11:230}
F^{0\alpha} &= -F_{0\alpha} \\
F^{\alpha\beta} &= F_{\alpha\beta}.
\end{align}

\begin{equation}\label{eqn:relativisticElectrodynamicsL11:250}
\Norm{ F^{i j} } = 
\begin{bmatrix}
0 & -E_x & -E_y & -E_z \\
E_x & 0 & -B_z & B_y \\
E_y & B_z & 0 & -B_x \\
E_z & -B_y & B_x & 0
\end{bmatrix}.
\end{equation}

\subsection{Back to chewing on the Lorentz force equation}

\begin{equation}\label{eqn:relativisticElectrodynamicsL11:270}
m c \dds{ u_i } = \frac{e}{c} F_{i j} u^j
\end{equation}

\begin{align}\label{eqn:relativisticElectrodynamicsL11:290}
u^i &= \gamma \left( 1, \frac{\Bv}{c} \right) \\
u_i &= \gamma \left( 1, -\frac{\Bv}{c} \right)
\end{align}

For the spatial components of the Lorentz force equation we have

\begin{align*}
m c \dds{ u_\alpha } 
&= \frac{e}{c} F_{\alpha j} u^j \\
&= \frac{e}{c} F_{\alpha 0} u^0
+ \frac{e}{c} F_{\alpha \beta} u^\beta \\
&= \frac{e}{c} (-E_{\alpha}) \gamma
+ \frac{e}{c} (- \epsilon_{\alpha\beta\gamma} B_\gamma ) \frac{v^\beta}{c} \gamma 
\end{align*}

But
\begin{align*}
m c \dds{ u_\alpha } 
&= -m \dds{(\gamma \Bv_\alpha)} \\
&= -m \frac{d(\gamma \Bv_\alpha)}{c \InvGamma dt} \\
&= -\gamma \frac{d(m \gamma \Bv_\alpha)}{c dt}.
\end{align*}

Canceling the common $-\gamma/c$ terms, and switching to vector notation, we are left with

\begin{equation}\label{eqn:relativisticElectrodynamicsL11:310}
\frac{d( m \gamma \Bv_\alpha)}{dt} = e \left( E_\alpha + \inv{c} (\Bv \cross \BB)_\alpha \right).
\end{equation}

Now for the energy term.  We have 

\begin{align*}
m c \dds{u_0} 
&= \frac{e}{c} F_{0\alpha} u^\alpha \\
&= \frac{e}{c} E_{\alpha} \gamma \frac{v^\alpha}{c} \\
\dds{ m c \gamma } &=
\end{align*}

Putting the final two lines into vector form we have
\begin{equation}\label{eqn:relativisticElectrodynamicsL11:330}
\ddt{ (m c^2 \gamma)} = e \BE \cdot \Bv,
\end{equation}

or
\begin{equation}\label{eqn:relativisticElectrodynamicsL11:350}
\ddt{ \mathcal{E} } = e \BE \cdot \Bv
\end{equation}

\section{Transformation of rank two tensors in matrix and index form}

\subsection{Transformation of the metric tensor, and some identities}

With
\begin{equation}\label{eqn:relativisticElectrodynamicsL11:410}
\hat{G} = \Norm{ g_{i j} } = \Norm{ g^{i j} }
\end{equation}

\paragraph{We claim:}
The rank two tensor $\hat{G}$ transforms in the following sort of sandwich operation, and this leaves it invariant

\begin{equation}\label{eqn:relativisticElectrodynamicsL11:430}
\hat{G} \rightarrow \hat{O} \hat{G} \hat{O}^\T = \hat{G}.
\end{equation}

To demonstrate this let's consider a transformed vector in coordinate form as follows

\begin{align}\label{eqn:relativisticElectrodynamicsL11:450}
{x'}^i &= O^{i j} x_j = {O^i}_j x^j \\
{x'}_i &= O_{i j} x^j = {O_i}^j x_j.
\end{align}

We can thus write the equation in matrix form with

\begin{align}\label{eqn:relativisticElectrodynamicsL11:940}
X &= \Norm{x^i} \\
X' &= \Norm{{x'}^i} \\
\hat{O} &= \Norm{{O^i}_j} \\
X' &= \hat{O} X
\end{align}

Our invariant for the vector square, which is required to remain unchanged is

\begin{align*}
{x'}^i {x'}_i 
&= (O^{i j} x_j)(O_{i k} x^k) \\
&= x^k (O^{i j} O_{i k}) x_j.
\end{align*}

This shows that we have a delta function relationship for the Lorentz transform matrix, when we sum over the first index

\begin{equation}\label{eqn:relativisticElectrodynamicsL11:470}
O^{a i} O_{a j} = {\delta^i}_j.
\end{equation}

It appears we can put \ref{eqn:relativisticElectrodynamicsL11:470} into matrix form as

\begin{equation}\label{eqn:relativisticElectrodynamicsL11:471}
\hat{G} \hat{O}^\T \hat{G} \hat{O} = I
\end{equation}

Now, if one considers that the transpose of a rotation is an inverse rotation, and the transpose of a boost leaves it unchanged, the transpose of a general Lorentz transformation, a composition of an arbitrary sequence of boosts and rotations, must also be a Lorentz transformation, and must then also leave the norm unchanged.  For the transpose of our Lorentz transformation $\hat{O}$ lets write

\begin{equation}\label{eqn:relativisticElectrodynamicsL11:1000}
\hat{P} = \hat{O}^\T
\end{equation}

For the action of this on our position vector let's write

\begin{align}\label{eqn:relativisticElectrodynamicsL11:1020}
{x''}^i &= P^{i j} x_j = O^{j i} x_j \\
{x''}_i &= P_{i j} x^j = O_{j i} x^j
\end{align}

so that our norm is

\begin{align*}
{x''}^a {x''}_a 
&= (O_{k a} x^k)(O^{j a} x_j) \\
&= x^k (O_{k a} O^{j a} ) x_j \\
&= x^j x_j \\
\end{align*}

We must then also have an identity when summing over the second index

\begin{equation}\label{eqn:relativisticElectrodynamicsL11:1040}
{\delta_{k}}^j = O_{k a} O^{j a} 
\end{equation}

Armed with these facts on the products of $O_{i j}$ and $O^{i j}$ we can now consider the transformation of the metric tensor.

The rule (definition) supplied to us for the transformation of an arbitrary rank two tensor, is that this transforms as its indexes transform individually.  Very much as if it was the product of two coordinate vectors and we transform those coordinates separately.  Doing so for the metric tensor we have

\begin{align*}
g^{i j} 
&\rightarrow {O^i}_k g^{k m} {O^j}_m \\
&= ({O^i}_k g^{k m}) {O^j}_m \\
&= O^{i m} {O^j}_m \\
&= O^{i m} (O_{a m} g^{a j}) \\
&= (O^{i m} O_{a m}) g^{a j}
\end{align*}

However, by \ref{eqn:relativisticElectrodynamicsL11:1040}, we have $O_{a m} O^{i m} = {\delta_a}^i$, and we prove that 

\begin{equation}\label{eqn:relativisticElectrodynamicsL11:960}
g^{i j} \rightarrow g^{i j}.
\end{equation}

Finally, we wish to put the above transformation in matrix form, look more carefully at the very first line

\begin{align*}
g^{i j}
&\rightarrow {O^i}_k g^{k m} {O^j}_m \\
\end{align*}

which is

\begin{equation}\label{eqn:relativisticElectrodynamicsL11:1070}
\hat{G} \rightarrow \hat{O} \hat{G} \hat{O}^\T = \hat{G}
\end{equation}

We see that this particular form of transformation, a sandwich between $\hat{O}$ and $\hat{O}^\T$, leaves the metric tensor invariant.  

\subsection{Lorentz transformation of the electrodynamic tensor}

Having identified a composition of Lorentz transformation matrices, when acting on the metric tensor, leaves it invariant, it is a reasonable question to ask how this form of transformation acts on our electrodynamic tensor $F^{i j}$?

\paragraph{Claim:} A transformation of the following form is required to maintain the norm of the Lorentz force equation

\begin{equation}\label{eqn:relativisticElectrodynamicsL11:490}
\hat{F} \rightarrow \hat{O} \hat{F} \hat{O}^\T ,
\end{equation}

where $\hat{F} = \Norm{F^{i j}}$.  Observe that our Lorentz force equation can be written exclusively in upper index quantities as

\begin{equation}\label{eqn:relativisticElectrodynamicsL11:370}
m c \dds{u^i} = \frac{e}{c} F^{i j} g_{j l} u^l
\end{equation}

Because we have a vector on one side of the equation, and it transforms by multiplication with by a Lorentz matrix in SO(1,3)

\begin{equation}\label{eqn:relativisticElectrodynamicsL11:390}
\frac{du^i}{ds} \rightarrow \hat{O} \frac{du^i}{ds} 
\end{equation}

The LHS of the Lorentz force equation provides us with one invariant

\begin{equation}\label{eqn:relativisticElectrodynamicsL11:1090}
(m c)^2 \dds{u^i} \dds{u_i}
\end{equation}

so the RHS must also provide one

\begin{equation}\label{eqn:relativisticElectrodynamicsL11:1110}
\frac{e^2}{c^2} 
F^{i j} g_{j l} u^l
F_{i k} g^{k m} u_m
=
\frac{e^2}{c^2} 
F^{i j} u_j
F_{i k} u^k.
\end{equation}

Let's look at the RHS in matrix form.  Writing

\begin{equation}\label{eqn:relativisticElectrodynamicsL11:1130}
U = \Norm{u^i},
\end{equation}

we can rewrite the Lorentz force equation as

\begin{equation}\label{eqn:relativisticElectrodynamicsL11:1150}
m c \dot{U} = \frac{e}{c} \hat{F} \hat{G} U.
\end{equation}

In this matrix formalism our invariant \ref{eqn:relativisticElectrodynamicsL11:1110} is

\begin{equation}\label{eqn:relativisticElectrodynamicsL11:1200}
\frac{e^2}{c^2} (\hat{F} \hat{G} U)^\T \hat{G} \hat{F} \hat{G} U
=
\frac{e^2}{c^2} U^\T \hat{G} \hat{F}^\T \hat{G} \hat{F} \hat{G} U.
\end{equation}

If we compare this to the transformed Lorentz force equation we have
\begin{equation}\label{eqn:relativisticElectrodynamicsL11:1220}
m c \hat{O} \dot{U} = \frac{e}{c} \hat{F'} \hat{G} \hat{O} U.
\end{equation}

Our invariant for the transformed equation is

\begin{align*}
\frac{e^2}{c^2} (\hat{F'} \hat{G} \hat{O} U)^\T \hat{G} \hat{F'} \hat{G} \hat{O} U
&=
\frac{e^2}{c^2} U^\T \hat{O}^\T \hat{G} \hat{F'}^\T \hat{G} \hat{F'} \hat{G} \hat{O} U \\
\end{align*}

Thus the transformed electrodynamic tensor $\hat{F}'$ must satisfy the identity

\begin{equation}\label{eqn:relativisticElectrodynamicsL11:1240}
\hat{O}^\T \hat{G} \hat{F'}^\T \hat{G} \hat{F'} \hat{G} \hat{O} = \hat{G} \hat{F}^\T \hat{G} \hat{F} \hat{G} 
\end{equation}

With the substitution $\hat{F}' = \hat{O} \hat{F} \hat{O}^\T$ the LHS is

\begin{align*}
\hat{O}^\T \hat{G} \hat{F'}^\T \hat{G} \hat{F'} \hat{G} \hat{O} 
&=
\hat{O}^\T \hat{G} ( \hat{O} \hat{F} \hat{O}^\T)^\T
\hat{G} 
(\hat{O} \hat{F} \hat{O}^\T)
\hat{G} \hat{O}  \\
&=
(\hat{O}^\T \hat{G} \hat{O}) \hat{F}^\T (\hat{O}^\T \hat{G} \hat{O}) \hat{F} (\hat{O}^\T \hat{G} \hat{O}) \\
\end{align*}

We've argued that $\hat{P} = \hat{O}^\T$ is also a Lorentz transformation, thus 

\begin{align*}
\hat{O}^\T \hat{G} \hat{O}
&=
\hat{P} \hat{G} \hat{O}^\T \\
&=
\hat{G}
\end{align*}

This is enough to make both sides of \ref{eqn:relativisticElectrodynamicsL11:1240} match, verifying that this transformation does provide the invariant properties desired.

\subsection{Direct computation of the Lorentz transformation of the electrodynamic tensor}

We can construct the transformed field tensor more directly, by simply transforming the coordinates of the four gradient and the four potential directly.  That is

\begin{align*}
F^{i j} = 
\partial^i A^j - \partial^j A^i
&\rightarrow 
{O^i}_a 
{O^j}_b 
\left( \partial^a A^b - \partial^b A^a \right) \\
&=
{O^i}_a F^{a b} {O^j}_b 
\end{align*}

By inspection we can see that this can be represented in matrix form as

\begin{equation}\label{eqn:relativisticElectrodynamicsL11:1300}
\hat{F} \rightarrow \hat{O} \hat{F} \hat{O}^\T
\end{equation}

\section{Four vector invariants}

For three vectors $\BA$ and $\BB$ invariants are

\begin{equation}\label{eqn:relativisticElectrodynamicsL11:510}
\BA \cdot \BB = A^\alpha B_\alpha
\end{equation}

For four vectors $A^i$ and $B^i$ invariants are

\begin{equation}\label{eqn:relativisticElectrodynamicsL11:530}
A^i B_i = A^i g_{i j} B^j  
\end{equation}

For $F_{i j}$ what are the invariants?  One invariant is

\begin{equation}\label{eqn:relativisticElectrodynamicsL11:550}
g^{i j} F_{i j} = 0,
\end{equation}

but this isn't interesting since it is uniformly zero (product of symmetric and antisymmetric).

The two invariants are

\begin{equation}\label{eqn:relativisticElectrodynamicsL11:570}
F_{i j}F^{i j}
\end{equation}

and 

\begin{equation}\label{eqn:relativisticElectrodynamicsL11:590}
\epsilon^{i j k l} F_{i j}F_{k l}
\end{equation}

where
\begin{equation}\label{eqn:relativisticElectrodynamicsL11:610}
\epsilon^{i j k l} =
\left\{
\begin{array}{l l}
0 & \quad \mbox{if any two indexes coincide} \\
1 & \quad \mbox{for even permutations of $i j k l=0123$ } \\
-1 & \quad \mbox{for odd permutations of $i j k l=0123$ } \\
\end{array}
\right.
\end{equation}

We can show (homework) that

\begin{equation}\label{eqn:relativisticElectrodynamicsL11:630}
F_{i j}F^{i j} \sim \BE^2 - \BB^2
\end{equation}

\begin{equation}\label{eqn:relativisticElectrodynamicsL11:650}
\epsilon^{i j k l} F_{i j}F_{k l} \sim \BE \cdot \BB
\end{equation}

This first invariant serves as the action density for the Maxwell field equations.

There's some useful properties of these invariants.  One is that if the fields are perpendicular in one frame, then will be in any other.  

From the first, note that if $\Abs{\BE} > \Abs{\BB}$, the invariant is positive, and must be positive in all frames, or if $\Abs{\BE} < \Abs{\BB}$, the invariant is negative, and must be negative in all frames.  Because of this if $\Abs{\BE} > \Abs{\BB}$ in one frame, we can transform to a frame with only $\BE'$ component, solve that, and then transform back.  Similarly if $\Abs{\BE} < \Abs{\BB}$ in one frame, we can transform to a frame with only $\BB'$ component, solve that, and then transform back.

\section{The first half of Maxwell's equations}

\paragraph{Claim: } The source free portions of Maxwell's equations are a consequence of the definition of the field tensor alone.

Given
\begin{equation}\label{eqn:relativisticElectrodynamicsL11:670}
F_{i j} = \partial_i A_j - \partial_j A_i,
\end{equation}

where
\begin{equation}\label{eqn:relativisticElectrodynamicsL11:690}
\partial_i = \PD{x^i}{}
\end{equation}

This alone implies half of Maxwell's equations.  To show this we consider

\begin{equation}\label{eqn:relativisticElectrodynamicsL11:710}
\epsilon^{m k i j} \partial_k F_{i j} = 0.
\end{equation}

This is the Bianchi identity.  To demonstrate this identity, we'll have to swap indexes, employ derivative commutation, and then swap indexes once more

\begin{align*}
\epsilon^{m k i j} \partial_k F_{i j} 
&= \epsilon^{m k i j} \partial_k (\partial_i A_j - \partial_j A_i) \\
&= 2 \epsilon^{m k i j} \partial_k \partial_i A_j \\
&= 2 \epsilon^{m k i j} \inv{2} \left( \partial_k \partial_i A_j + \partial_i \partial_k A_j \right) \\
&= 
\epsilon^{m k i j} \partial_k \partial_i A_j 
\epsilon^{m i k j} \partial_k \partial_i A_j  \\
&= 
(\epsilon^{m k i j} - \epsilon^{m k i j}) \partial_k \partial_i A_j \\
&= 0 \qquad \square
\end{align*}

This is the 4D analogue of 

\begin{equation}\label{eqn:relativisticElectrodynamicsL11:790}
\spacegrad \cross (\spacegrad f) = 0
\end{equation}

i.e.

\begin{equation}\label{eqn:relativisticElectrodynamicsL11:810}
\epsilon^{\alpha\beta\gamma} \partial_\beta \partial_\gamma f = 0
\end{equation}

Let's do this explicitly, starting with

\begin{equation}\label{eqn:relativisticElectrodynamicsL11:370b}
\Norm{ F_{i j} } = 
\begin{bmatrix}
0 & E_x & E_y & E_z \\
-E_x & 0 & -B_z & B_y \\
-E_y & B_z & 0 & -B_x \\
-E_z & -B_y & B_x & 0.
\end{bmatrix}
\end{equation}

For the $m= 0$ case we have

\begin{align*}
\epsilon^{0 k i j} \partial_k F_{i j}
&=
\epsilon^{\alpha \beta \gamma} \partial_\alpha F_{\beta \gamma} \\
&= 
\epsilon^{\alpha \beta \gamma} \partial_\alpha (-\epsilon_{\beta \gamma \delta} B_\delta) \\
&= 
-\epsilon^{\alpha \beta \gamma} \epsilon_{\delta \beta \gamma }
\partial_\alpha B_\delta \\
&= 
- 2 {\delta^\alpha}_\delta \partial_\alpha B_\delta \\
&= 
- 2 \partial_\alpha B_\alpha 
\end{align*}

We must then have

\begin{equation}\label{eqn:relativisticElectrodynamicsL11:830}
\partial_\alpha B_\alpha = 0.
\end{equation}

This is just Gauss's law for magnetism

\begin{equation}\label{eqn:relativisticElectrodynamicsL11:850}
\spacegrad \cdot \BB = 0.
\end{equation}

Let's do the spatial portion, for which we have three equations, one for each $\alpha$ of

\begin{align*}
\epsilon^{\alpha j k l} \partial_j F_{k l}
&=
\epsilon^{\alpha 0 \beta \gamma} \partial_0 F_{\beta \gamma}
+\epsilon^{\alpha 0 \gamma \beta} \partial_0 F_{\gamma \beta}
+\epsilon^{\alpha \beta 0 \gamma} \partial_\beta F_{0 \gamma} \\
&\qquad +\epsilon^{\alpha \beta \gamma 0} \partial_\beta F_{\gamma 0}
+\epsilon^{\alpha \gamma 0 \beta} \partial_\gamma F_{0 \beta}
+\epsilon^{\alpha \gamma \beta 0} \partial_\gamma F_{\beta 0} \\
&=
2 \left( 
\epsilon^{\alpha 0 \beta \gamma} \partial_0 F_{\beta \gamma}
+\epsilon^{\alpha \beta 0 \gamma} \partial_\beta F_{0 \gamma}
+\epsilon^{\alpha \gamma 0 \beta} \partial_\gamma F_{0 \beta}
\right) \\
&=
2 \epsilon^{0 \alpha \beta \gamma} \left(
-\partial_0 F_{\beta \gamma}
+\partial_\beta F_{0 \gamma}
- \partial_\gamma F_{0 \beta}
\right)
\end{align*}

This implies

\begin{equation}\label{eqn:relativisticElectrodynamicsL11:1320}
0 =
-\partial_0 F_{\beta \gamma}
+\partial_\beta F_{0 \gamma}
- \partial_\gamma F_{0 \beta}
\end{equation}

Referring back to the previous expansions of \ref{eqn:relativisticElectrodynamicsL11:71} and \ref{eqn:relativisticElectrodynamicsL11:170}, we have

%F_{\beta\gamma} = - \epsilon_{\beta\gamma\mu} B_\mu.
%F_{0\alpha} = E_\alpha
\begin{equation}\label{eqn:relativisticElectrodynamicsL11:1340}
0 =
\partial_0 \epsilon_{\beta\gamma\mu} B_\mu
+\partial_\beta E_\gamma
- \partial_\gamma E_{\beta},
\end{equation}

or

\begin{equation}\label{eqn:relativisticElectrodynamicsL11:1360}
\inv{c} \PD{t}{B_\alpha} + (\spacegrad \cross \BE)_\alpha = 0.
\end{equation}

These are just the components of the Faraday's law

\begin{equation}\label{eqn:relativisticElectrodynamicsL11:930}
0 = \inv{c} \PD{t}{\BB} + \spacegrad \cross \BE.
\end{equation}

\section{Appendix. Some additional index gymnastics}

\subsection{Transposition of mixed index tensor}

Is the transpose of a mixed index object just a substitution of the free indexes?  This wasn't obvious to me that it would be the case, especially since I'd made an error in some index gymnastics that had me temporarily convinced differently.  However, working some examples clears the fog.  For example let's take the transpose of \ref{eqn:relativisticElectrodynamicsL11:470}.  

\begin{align*}
\Norm{ {\delta^i}_j }^\T 
&= 
\Norm{ O^{a i} O_{a j} }^\T \\
&= 
\left( \Norm{ O^{j i} } \Norm{ O_{i j} } \right)^\T \\
&=
\Norm{ O_{i j} }^\T
\Norm{ O^{j i} }^\T  \\
&=
\Norm{ O_{j i} }
\Norm{ O^{i j} } \\
&=
\Norm{ O_{a i} O^{a j} } \\
\end{align*}

If the transpose of a mixed index tensor just swapped the indexes we would have

\begin{equation}\label{eqn:relativisticElectrodynamicsL11:1060}
\Norm{ {\delta^i}_j }^\T = \Norm{ O_{a i} O^{a j} } 
\end{equation}

From this it does appear that all we have to do is switch the indexes and we will write
\begin{equation}\label{eqn:relativisticElectrodynamicsL11:1060b}
{\delta^j}_i = O_{a i} O^{a j} 
\end{equation}

%It would be easy to brush aside this bit of slight of hand here in the transposition with the positioning of the indexes?  
%With all this index gymnastics, even the transposition operation appears to be something that we have to treat carefully.  
We can consider a more general operation

\begin{align*}
\Norm{{A^i}_j}^\T
&=
\Norm{ A^{i m} g_{m j} }^\T \\
&=
\Norm{ g_{i j} }^\T
\Norm{ A^{i j} }^\T 
 \\
&=
\Norm{ g_{i j} }
\Norm{ A^{j i} }
 \\
&=
\Norm{ g_{i m} A^{j m} }
 \\
&=
\Norm{ {A^{j}}_i }
\end{align*}

So we see that we do just have to swap indexes.

%So, provided 
%
%\begin{equation}\label{eqn:relativisticElectrodynamicsL11:980}
%\Norm{A^{i j}}^\T = \Norm{A^{j i}},
%\end{equation}
%
%which we also assumed earlier as well, we have
%
%\begin{equation}\label{eqn:relativisticElectrodynamicsL11:1001}
%\Norm{{A^i}_j}^\T =
%\Norm{ {A_{i}}^j }
%\end{equation}
%
%The transposition operation on this mixed index delta function lowers and raises the indexes as opposed to just swapping them, as we are used to in plain old Euclidean \R{N} matrix operations.

\subsection{Transposition of lower index tensor}

We've saw above that we had

\begin{align}\label{eqn:relativisticElectrodynamicsL11:1022}
\Norm{ {A^{i}}_j }^\T &= \Norm{ {A_{j}}^i } \\
\Norm{ {A_{i}}^j }^\T &= \Norm{ {A^{j}}_i } 
\end{align}

which followed by careful treatment of the transposition in terms of $A^{i j}$ for which we defined a transpose operation.  We assumed as well that

\begin{equation}\label{eqn:relativisticElectrodynamicsL11:1042}
\Norm{ A_{i j} }^\T = \Norm{ A_{j i} }.
\end{equation}

However, this does not have to be assumed, provided that $g^{i j} = g_{i j}$, and $(AB)^\T = B^\T A^\T$.  We see this by expanding this transposition in products of $A^{i j}$ and $\hat{G}$

\begin{align*}
\Norm{ A_{i j} }^\T
&= \left( \Norm{g_{i j}} \Norm{ A^{i j} } \Norm{g_{i j}} \right)^\T \\
&= \left( \Norm{g^{i j}} \Norm{ A^{i j} } \Norm{g^{i j}} \right)^\T \\
&= \Norm{g^{i j}}^\T \Norm{ A^{i j}}^\T \Norm{g^{i j}}^\T \\
&= \Norm{g^{i j}} \Norm{ A^{j i}} \Norm{g^{i j}} \\
&= \Norm{g_{i j}} \Norm{ A^{i j}} \Norm{g_{i j}} \\
&= \Norm{ A_{j i}} 
\end{align*}

It would be worthwhile to go through all of this index manipulation stuff and lay it out in a structured axiomatic form.  What is the minimal set of assumptions, and how does all of this generalize to non-diagonal metric tensors (even in Euclidean spaces).

\subsection{Translating the index expression of identity from Lorentz products to matrix form}

A verification that the matrix expression \ref{eqn:relativisticElectrodynamicsL11:471}, matches the index expression \ref{eqn:relativisticElectrodynamicsL11:470} as claimed is worthwhile.  It would be easy to guess something similar like $\hat{O}^\T \hat{G} \hat{O} \hat{G}$ is instead the matrix representation.  That was in fact my first erroneous attempt to form the matrix equivalent, but is the transpose of \ref{eqn:relativisticElectrodynamicsL11:471}.  Either way you get an identity, but the indexes didn't match.

Since we have $g^{i j} = g_{i j}$ which do we pick to do this verification?  This appears to be dictated by requirements to match lower and upper indexes on the summed over index.  This is probably clearest by example, so let's expand the products on the LHS explicitly

\begin{align*}
\Norm{ g^{i j} } 
\Norm{ {O^{i}}_j } ^\T
\Norm{ g_{i j} }
\Norm{ {O^{i}}_j } 
&=
\left( \Norm{ {O^{i}}_j } 
\Norm{ g^{i j} } \right) ^\T
\Norm{ g_{i j} }
\Norm{ {O^{i}}_j }  \\
&=
\left( \Norm{ {O^{i}}_k g^{k j} } \right) ^\T
\Norm{ g_{i m} {O^{m}}_j }  \\
&=
\Norm{ O^{i j} } ^\T
\Norm{ O_{i j} }  \\
&=
\Norm{ O^{j i} } 
\Norm{ O_{i j} }  \\
&=
\Norm{ O^{k i} O_{k j} }  \\
\end{align*}

This matches the $\Norm{{\delta^i}_j}$ that we have on the RHS, and all is well.
