%
% Copyright � 2012 Peeter Joot.  All Rights Reserved.
% Licenced as described in the file LICENSE under the root directory of this GIT repository.
%

\chapter{Four vectors and a worked flux density problem}
\label{chap:relativisticElectrodynamicsT1}
%\blogpage{http://sites.google.com/site/peeterjoot/math2011/relativisticElectrodynamicsT1.pdf}
%\date{Jan 20, 2011}

\section{Worked question}

Simon (our TA) blasted through a problem from Hartle \citep{hartle2003gravity}, section 5.17 (all the while apologizing for going so slow).  It took me a while to work through my notes to come up with something that was comprehensible to me.

At one point he asked if anybody was completely lost.  Nobody said yes, but given the class title, I had the urge to say ``No, just relatively lost''.

\paragraph{Q:}
In a source's rest frame $S$ emits radiation isotropically with a frequency $\omega$ with number flux $f(\text{photons}/\text{cm}^2 s)$.  Moves along x'-axis with speed $V$ in an observer frame ($O$).  What does the energy flux in $O$ look like?

\subsection{A brief intro with four vectors}

A 3-vector: 

\begin{align}\label{eqn:relativisticElectrodynamicsT1:10}
\Ba &= (a_x, a_y, a_z) = (a^1, a^2, a^3) \\
\Bb &= (b_x, b_y, b_z) = (b^1, b^2, b^3)
\end{align}

For this we have the dot product
\begin{equation}\label{eqn:relativisticElectrodynamicsT1:20}
\Ba \cdot \Bb = \sum_{\alpha=1}^3 a^\alpha b^\alpha
\end{equation}

Greek letters in this course (opposite to everybody else in the world, because of Landau and Lifshitz) run from 1 to 3, whereas roman letters run through the set $\{0,1,2,3\}$.

We want to put space and time on an equal footing and form the composite quantity (four vector) 
\begin{equation}\label{eqn:relativisticElectrodynamicsT1:40}
x^i = (ct, \Br) = (x^0, x^1, x^2, x^3),
\end{equation}

where
\begin{align}\label{eqn:relativisticElectrodynamicsT1:80}
x^0 &= ct \\
x^1 &= x \\
x^2 &= y \\
x^3 &= z.
\end{align}

It will also be convenient to drop indexes when referring to all the components of a four vector and we will use lower or upper case non-bold letters to represent such four vectors.  For example

\begin{equation}\label{eqn:relativisticElectrodynamicsT1:81}
X = (ct, \Br),
\end{equation}

or
\begin{equation}\label{eqn:relativisticElectrodynamicsT1:82}
u = \gamma \left(1, \Bv/c \right).
\end{equation}

Three vectors will be represented as letters with over arrows $\vec{a}$ or (in text) bold face $\Ba$.

Recall that the squared spacetime interval between two events $X_1$ and $X_2$ is defined as

\begin{equation}\label{eqn:relativisticElectrodynamicsT1:60}
{S_{X_1, X_2}}^2 = (ct_1 - c t_2)^2 - (\Bx_1 - \Bx_2)^2.
\end{equation}

In particular, with one of these zero, we have an operator which takes a single four vector and spits out a scalar, measuring a ``distance'' from the origin

\begin{equation}\label{eqn:relativisticElectrodynamicsT1:30}
s^2 = (ct)^2 - \Br^2.
\end{equation}

This motivates the introduction of a dot product for our four vector space.  

\begin{equation}\label{eqn:relativisticElectrodynamicsT1:50}
X \cdot X = (ct)^2 - \Br^2 = (x^0)^2 - \sum_{\alpha=1}^3 (x^\alpha)^2
\end{equation}

Utilizing the spacetime dot product of \ref{eqn:relativisticElectrodynamicsT1:50} we have for the dot product of the difference between two events

\begin{align*}
(X - Y) \cdot (X - Y)
&=
(x^0 - y^0)^2 - \sum_{\alpha =1}^3 (x^\alpha - y^\alpha)^2 \\
&=
X \cdot X + Y \cdot Y - 2 x^0 y^0 + 2 \sum_{\alpha =1}^3 x^\alpha y^\alpha.
\end{align*}

From this, assuming our dot product \ref{eqn:relativisticElectrodynamicsT1:50} is both linear and symmetric, we have for any pair of spacetime events

\begin{equation}\label{eqn:relativisticElectrodynamicsT1:55}
X \cdot Y = x^0 y^0 - \sum_{\alpha =1}^3 x^\alpha y^\alpha.
\end{equation}

How do our four vectors transform?  This is really just a notational issue, since this has already been discussed.  In this new notation we have

\begin{align}\label{eqn:relativisticElectrodynamicsT1:90}
{x^0}' &= ct' = \gamma ( ct - \beta x) = \gamma ( x^0 - \beta x^1 ) \\
{x^1}' &= x' = \gamma ( x - \beta ct ) = \gamma ( x^1 - \beta x^0 ) \\
{x^2}' &= x^2 \\
{x^3}' &= x^3
\end{align}

where $\beta = V/c$, and $\gamma^{-2} = 1 - \beta^2$.

In order to put some structure to this, it can be helpful to express this dot product as a quadratic form.  We write

\begin{align}\label{eqn:relativisticElectrodynamicsT1:100}
A \cdot B = 
\begin{bmatrix}
a^0 & \Ba^\T 
\end{bmatrix}
\begin{bmatrix}
1 & 0 & 0 & 0 \\
0 & -1 & 0 & 0 \\
0 & 0 & -1 & 0 \\
0 & 0 & 0 & -1 
\end{bmatrix}
\begin{bmatrix}
b^0 \\
\Bb
\end{bmatrix}
= A^\T G B.
\end{align}

We can write our Lorentz boost as a matrix

\begin{equation}\label{eqn:relativisticElectrodynamicsT1:110}
\begin{bmatrix}
\gamma & -\beta \gamma & 0 & 0 \\
-\beta \gamma & \gamma & 0 & 0 \\
0 & 0 & 1 & 0 \\
0 & 0 & 0 & 1 
\end{bmatrix}
\end{equation}

so that the dot product between two transformed four vectors takes the form

\begin{equation}\label{eqn:relativisticElectrodynamicsT1:120}
A' \cdot B' = A^\T O^\T G O B
\end{equation}

\subsection{Back to the problem}

We will work in momentum space, where we have

\begin{align}\label{eqn:relativisticElectrodynamicsT1:130}
p^i &= (p^0, \Bp) = \left( \frac{E}{c}, \Bp\right) \\
p^2 &= \frac{E^2}{c^2} -\Bp^2 \\
\Bp &= \hbar \Bk \\
E &= \hbar \omega \\
p^i &= \hbar k^i \\
k^i &= \left(\frac{\omega}{c}, \Bk\right)
\end{align}

\subsubsection{Justifying this}

Now, Simon (our TA) blurted all this out.  We know some of it from the QM context, and if we have been reading ahead know a bit of this from our text \citep{landau1980classical} (the energy momentum four vector relationships).  Let us go back to the classical electromagnetism and recall what we know about the relation of frequency and wave numbers for continuous fields.  We want solutions to Maxwell's equation in vacuum and can show that such solution also implies that our fields obey a wave equation

\begin{equation}\label{eqn:relativisticElectrodynamicsT1:131}
\inv{c^2} \frac{\partial^2 \Psi}{\partial t^2} - \spacegrad^2 \Psi = 0,
\end{equation}

where $\Psi$ is one of $\BE$ or $\BB$.  We have other constraints imposed on the solutions by Maxwell's equations, but require that they at least obey \ref{eqn:relativisticElectrodynamicsT1:131} in addition to these constraints.

With application of a spatial Fourier transformation of the wave equation, we find that our solution takes the form

\begin{equation}\label{eqn:relativisticElectrodynamicsT1:132}
\Psi = (2 \pi)^{-3/2} \int \tilde{\Psi}(\Bk, 0) e^{i (\omega t \pm \Bk \cdot \Bx) } d^3 \Bk.
\end{equation}

If one takes this as a given and applies the wave equation operator to this as a test solution, one finds without doing the Fourier transform work that we also have a constraint.  That is

\begin{equation}\label{eqn:relativisticElectrodynamicsT1:133}
\inv{c^2} (i \omega)^2 \Psi - (\pm i \Bk)^2 \Psi = 0.
\end{equation}

So even in the continuous field domain, we have a relationship between frequency and wave number.  We see that this also happens to have the form of a lightlike spacetime interval

\begin{equation}\label{eqn:relativisticElectrodynamicsT1:134}
\frac{\omega^2}{c^2} - \Bk^2 = 0.
\end{equation}

Also recall that the photoelectric effect imposes an experimental constraint on photon energy, where we have

\begin{equation}\label{eqn:relativisticElectrodynamicsT1:135}
E = h \nu = \frac{h}{2\pi} 2 \pi \nu = \hbar \omega
\end{equation}

Therefore if we impose a mechanics like $P = (E/c, \Bp)$ relativistic energy-momentum relationship on light, it then makes sense to form a nilpotent (lightlike) four vector for our photon energy.  This combines our special relativistic expectations, with the constraints on the fields imposed by classical electromagnetism.  We can then write for the photon four momentum

\begin{equation}\label{eqn:relativisticElectrodynamicsT1:136}
P = \left( \frac{\hbar \omega}{c}, \hbar k \right)
\end{equation}

\subsubsection{Back to the formula blitz}

We set up the $x'$-axis to be the direction of motion, and we call $\alpha$ the angle from it, or the azimuthal angle.  The wavevector, $\Bk$, is the direction the wave travels. Therefore, if we want to find the angle the radiation makes to the direction of motion, you need the projection of the wavevector onto the $x$-axis, or $k^1/\Abs{\Bk}$. In other words, imagine a piece of radiation emitted in a certain direction, the angle it makes with the $x'$-axis is the cosine of the projection on the $x'$-axis over the magnitude.

This azimuthal angle in the unprimed frame is

\begin{equation}\label{eqn:relativisticElectrodynamicsT1:140} 
\cos \alpha = \frac{k^1}{\Abs{\Bk}} = \frac{k^1}{\omega/c},
\end{equation}

In the observer's reference frame (the primed coordinates), the source is moving in the $+x$ direction, and therefore, we must boost in the $-x$ from the source's frame, or $-\beta$.  Transforming out wave four vector in the same fashion as regular mechanical position and momentum four vectors, we have for the observer

\begin{equation}\label{eqn:relativisticElectrodynamicsT1:140b} 
\cos \alpha' = \frac{{k^1}'}{\omega'/c} = \frac{\gamma (k^1 + \beta \omega/c)}{\gamma(\omega/c + \beta k^1)}
\end{equation}

%Also note that we have the primed frame moving negatively along the x-axis, instead of the usual positive origin shift.  The question is vague enough to allow this since it only requires motion.

\paragraph{check 1}

as $\beta \rightarrow 1$ (ie: our primed frame velocity approaches the speed of light relative to the rest frame), $\cos \alpha' \rightarrow 1$, $\alpha' = 0$.  The surface gets more and more compressed.

In the original reference frame the radiation was isotropic.  In the new frame how does it change with respect to the angle?  This is really a question to find this number flux rate

\begin{equation}\label{eqn:relativisticElectrodynamicsT1:150}
f'(\alpha') = ?
\end{equation}

In our rest frame the total number of photons traveling through the surface in a given interval of time is

\begin{align}\label{eqn:relativisticElectrodynamicsT1:160}
N &= \int d\Omega dt f(\alpha) = \int d \phi \sin \alpha d\alpha = -2 \pi \int d(\cos\alpha) dt f(\alpha) \\
\end{align}

Here we utilize the spherical solid angle $d\Omega = \sin \alpha d\alpha d\phi = - d(\cos\alpha) d\phi$, and integrate $\phi$ over the $[0, 2\pi]$ interval.  We also have to assume that our number flux density is not a function of horizontal angle $\phi$ in the rest frame.

In the moving frame we similarly have
\begin{align}\label{eqn:relativisticElectrodynamicsT1:160b}
N' &= -2 \pi \int d(\cos\alpha') dt' f'(\alpha'),
\end{align}

and we again have had to assume that our transformed number flux density is not a function of the horizontal angle $\phi$.  This seems like a reasonable move since ${k^2}' = k^2$ and ${k^3}' = k^3$ as they are perpendicular to the boost direction.

\begin{equation}\label{eqn:relativisticElectrodynamicsT1:170}
f'(\alpha') = \frac{d(\cos\alpha)}{d(\cos\alpha')} \left( \frac{dt}{dt'} \right) f(\alpha)
\end{equation}


Now, utilizing a conservation of mass argument, we can argue that $N = N'$.  Regardless of the motion of the frame, the same number of particles move through the surface.  Taking ratios, and examining an infinitesimal time interval, and the associated flux through a small patch, we have

\begin{equation}\label{eqn:relativisticElectrodynamicsT1:180}
\left( \frac{d(\cos\alpha)}{d(\cos\alpha')} \right) = \left( \frac{d(\cos\alpha')}{d(\cos\alpha)} \right)^{-1} = \gamma^2 ( 1 + \beta \cos\alpha)^2
\end{equation}

Part of the statement above was a do-it-yourself.  First recall that $c t' = \gamma ( c t + \beta x )$, so $dt/dt'$ evaluated at $x=0$ is $1/\gamma$.

The rest is messier.  We can calculate the $d(\cos)$ values in the ratio above using \ref{eqn:relativisticElectrodynamicsT1:140}.  For example, for $d(\cos(\alpha))$ we have

\begin{align*}
d(\cos\alpha) 
&= d \left( \frac{k^1}{\omega/c} \right) \\
&= dk^1 \inv{\omega/c} - c \inv{\omega^2} d\omega.
\end{align*}

If one does the same thing for $d(\cos\alpha')$, after a whole whack of messy algebra one finds that the differential terms and a whole lot more mystically cancels, leaving just

\begin{equation}\label{eqn:relativisticElectrodynamicsT1:171}
\frac{d\cos\alpha'}{d\cos\alpha} = \frac{\omega^2/c^2}{(\omega/c + \beta k^1)^2} (1 - \beta^2)
\end{equation}

A bit more reduction with reference back to \ref{eqn:relativisticElectrodynamicsT1:140b} verifies \ref{eqn:relativisticElectrodynamicsT1:180}.

Also note that again from \ref{eqn:relativisticElectrodynamicsT1:140b} we have

\begin{equation}\label{eqn:relativisticElectrodynamicsT1:190a}
\cos\alpha' = \frac{\cos\alpha + \beta}{1 + \beta \cos\alpha}
\end{equation}

and rearranging this for $\cos\alpha'$ gives us
\begin{equation}\label{eqn:relativisticElectrodynamicsT1:190}
\cos\alpha = \frac{\cos\alpha' - \beta}{1 - \beta \cos\alpha'},
\end{equation}

which we can sum to find that 

\begin{equation}\label{eqn:relativisticElectrodynamicsT1:190b}
1 + \beta \cos\alpha = \inv{\gamma^2 (1 - \beta \cos \alpha') },
\end{equation}

so putting all the pieces together we have

\begin{equation}\label{eqn:relativisticElectrodynamicsT1:200}
f'(\alpha') = \inv{\gamma^3} \frac{f(\alpha)}{(1-\beta \cos\alpha')^2}
\end{equation}

The question asks for the energy flux density.  We get this by multiplying the number density by the frequency of the light in question.  This is, as a function of the polar angle, in each of the frames.

\begin{align}\label{eqn:relativisticElectrodynamicsT1:210}
L(\alpha) &= \hbar \omega(\alpha) f(\alpha) = \hbar \omega f \\
L'(\alpha') &= \hbar \omega'(\alpha') f'(\alpha') = \hbar \omega' f'
\end{align}

But we have
\begin{equation}\label{eqn:relativisticElectrodynamicsT1:220}
\omega'(\alpha')/c = \gamma( \omega/c + \beta k^1 ) = \gamma \omega/c ( 1 + \beta \cos\alpha )
\end{equation}

Aside, $\beta << 1$, 

\begin{equation}\label{eqn:relativisticElectrodynamicsT1:230}
\omega' = \omega ( 1 + \beta \cos\alpha) + O(\beta^2) = \omega + \delta \omega
\end{equation}

\begin{align}\label{eqn:relativisticElectrodynamicsT1:240}
\delta \omega &= \beta, \alpha = 0 		\qquad \text{blue shift} \\
\delta \omega &= -\beta, \alpha = \pi 		\qquad \text{red shift}
\end{align}

The energy flux density in the unprimed observer frame is now found to be

\begin{equation}\label{eqn:relativisticElectrodynamicsT1:241}
L'(\alpha') = \frac{L/\gamma}{(\gamma (1 - \beta \cos\alpha'))^3}
\end{equation}

And the forward backward ratio is

\begin{equation}\label{eqn:relativisticElectrodynamicsT1:250}
L'(0)/L'(\pi) = {\left( \frac{ 1 + \beta }{1-\beta} \right)}^3,
\end{equation}

allowing us to conclude that the forward radiation is bigger than the backwards radiation (and much bigger when the motion approaches the speed of light).
