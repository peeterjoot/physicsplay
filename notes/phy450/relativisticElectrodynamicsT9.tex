%%
% Copyright � 2015 Peeter Joot.  All Rights Reserved.
% Licenced as described in the file LICENSE under the root directory of this GIT repository.
%
\documentclass[]{eliblog}

\usepackage{amsmath}
\usepackage{mathpazo}

%
% shorthand for bold symbols, convenient for vectors and matrices
%
\newcommand{\Ba}[0]{\mathbf{a}}
\newcommand{\Bb}[0]{\mathbf{b}}
\newcommand{\Bc}[0]{\mathbf{c}}
\newcommand{\Bd}[0]{\mathbf{d}}
\newcommand{\Be}[0]{\mathbf{e}}
\newcommand{\Bf}[0]{\mathbf{f}}
\newcommand{\Bg}[0]{\mathbf{g}}
\newcommand{\Bh}[0]{\mathbf{h}}
\newcommand{\Bi}[0]{\mathbf{i}}
\newcommand{\Bj}[0]{\mathbf{j}}
\newcommand{\Bk}[0]{\mathbf{k}}
\newcommand{\Bl}[0]{\mathbf{l}}
\newcommand{\Bm}[0]{\mathbf{m}}
\newcommand{\Bn}[0]{\mathbf{n}}
\newcommand{\Bo}[0]{\mathbf{o}}
\newcommand{\Bp}[0]{\mathbf{p}}
\newcommand{\Bq}[0]{\mathbf{q}}
\newcommand{\Br}[0]{\mathbf{r}}
\newcommand{\Bs}[0]{\mathbf{s}}
\newcommand{\Bt}[0]{\mathbf{t}}
\newcommand{\Bu}[0]{\mathbf{u}}
\newcommand{\Bv}[0]{\mathbf{v}}
\newcommand{\Bw}[0]{\mathbf{w}}
\newcommand{\Bx}[0]{\mathbf{x}}
\newcommand{\By}[0]{\mathbf{y}}
\newcommand{\Bz}[0]{\mathbf{z}}
\newcommand{\BA}[0]{\mathbf{A}}
\newcommand{\BB}[0]{\mathbf{B}}
\newcommand{\BC}[0]{\mathbf{C}}
\newcommand{\BD}[0]{\mathbf{D}}
\newcommand{\BE}[0]{\mathbf{E}}
\newcommand{\BF}[0]{\mathbf{F}}
\newcommand{\BG}[0]{\mathbf{G}}
\newcommand{\BH}[0]{\mathbf{H}}
\newcommand{\BI}[0]{\mathbf{I}}
\newcommand{\BJ}[0]{\mathbf{J}}
\newcommand{\BK}[0]{\mathbf{K}}
\newcommand{\BL}[0]{\mathbf{L}}
\newcommand{\BM}[0]{\mathbf{M}}
\newcommand{\BN}[0]{\mathbf{N}}
\newcommand{\BO}[0]{\mathbf{O}}
\newcommand{\BP}[0]{\mathbf{P}}
\newcommand{\BQ}[0]{\mathbf{Q}}
\newcommand{\BR}[0]{\mathbf{R}}
\newcommand{\BS}[0]{\mathbf{S}}
\newcommand{\BT}[0]{\mathbf{T}}
\newcommand{\BU}[0]{\mathbf{U}}
\newcommand{\BV}[0]{\mathbf{V}}
\newcommand{\BW}[0]{\mathbf{W}}
\newcommand{\BX}[0]{\mathbf{X}}
\newcommand{\BY}[0]{\mathbf{Y}}
\newcommand{\BZ}[0]{\mathbf{Z}}

\newcommand{\Bzero}[0]{\mathbf{0}}
\newcommand{\Btheta}[0]{\boldsymbol{\theta}}
\newcommand{\Btau}[0]{\boldsymbol{\tau}}
\newcommand{\Bomega}[0]{\boldsymbol{\omega}}

%
% shorthand for unit vectors
%
\newcommand{\acap}[0]{\hat{\Ba}}
\newcommand{\bcap}[0]{\hat{\Bb}}
\newcommand{\ccap}[0]{\hat{\Bc}}
\newcommand{\dcap}[0]{\hat{\Bd}}
\newcommand{\ecap}[0]{\hat{\Be}}
\newcommand{\fcap}[0]{\hat{\Bf}}
\newcommand{\gcap}[0]{\hat{\Bg}}
\newcommand{\hcap}[0]{\hat{\Bh}}
\newcommand{\icap}[0]{\hat{\Bi}}
\newcommand{\jcap}[0]{\hat{\Bj}}
\newcommand{\kcap}[0]{\hat{\Bk}}
\newcommand{\lcap}[0]{\hat{\Bl}}
\newcommand{\mcap}[0]{\hat{\Bm}}
\newcommand{\ncap}[0]{\hat{\Bn}}
\newcommand{\ocap}[0]{\hat{\Bo}}
\newcommand{\pcap}[0]{\hat{\Bp}}
\newcommand{\qcap}[0]{\hat{\Bq}}
\newcommand{\rcap}[0]{\hat{\Br}}
\newcommand{\scap}[0]{\hat{\Bs}}
\newcommand{\tcap}[0]{\hat{\Bt}}
\newcommand{\ucap}[0]{\hat{\Bu}}
\newcommand{\vcap}[0]{\hat{\Bv}}
\newcommand{\wcap}[0]{\hat{\Bw}}
\newcommand{\xcap}[0]{\hat{\Bx}}
\newcommand{\ycap}[0]{\hat{\By}}
\newcommand{\zcap}[0]{\hat{\Bz}}
\newcommand{\thetacap}[0]{\hat{\Btheta}}

%
% to write R^n and C^n in a distinguishable fashion.  Perhaps change this
% to the double lined characters upon figuring out how to do so.
%
\newcommand{\C}[1]{$\mathbb{C}^{#1}$}
\newcommand{\R}[1]{$\mathbb{R}^{#1}$}

%
% various generally useful helpers
%

% derivative of #1 wrt. #2:
\newcommand{\D}[2] {\frac {d#2} {d#1}}

\newcommand{\inv}[1]{\frac{1}{#1}}
\newcommand{\cross}[0]{\times}

\newcommand{\abs}[1]{\lvert{#1}\rvert}
\newcommand{\norm}[1]{\lVert{#1}\rVert}
\newcommand{\innerprod}[2]{\langle{#1}, {#2}\rangle}
\newcommand{\dotprod}[2]{{#1} \cdot {#2}}
\newcommand{\bdotprod}[2]{\left({#1} \cdot {#2}\right)}
\newcommand{\crossprod}[2]{{#1} \cross {#2}}
\newcommand{\tripleprod}[3]{\dotprod{\left(\crossprod{#1}{#2}\right)}{#3}}

\DeclareMathOperator{\Proj}{Proj}
\DeclareMathOperator{\Span}{span}
\DeclareMathOperator{\Sgn}{sgn}
\DeclareMathOperator{\Area}{Area}
\DeclareMathOperator{\Volume}{Volume}

%
% A few miscellaneous things specific to this document
%
\newcommand{\crossop}[1]{\crossprod{#1}{}}

% R2 vector.
\newcommand{\VectorTwo}[2]{
\begin{bmatrix}
 {#1} \\
 {#2}
\end{bmatrix}
}

\newcommand{\VectorN}[1]{
\begin{bmatrix}
{#1}_1 \\
{#1}_2 \\
\vdots \\
{#1}_N \\
\end{bmatrix}
}

\newcommand{\DETuvij}[4]{
\begin{vmatrix}
 {#1}_{#3} & {#1}_{#4} \\
 {#2}_{#3} & {#2}_{#4}
\end{vmatrix}
}

\newcommand{\DETuvwijk}[6]{
\begin{vmatrix}
 {#1}_{#4} & {#1}_{#5} & {#1}_{#6} \\
 {#2}_{#4} & {#2}_{#5} & {#2}_{#6} \\
 {#3}_{#4} & {#3}_{#5} & {#3}_{#6}
\end{vmatrix}
}

\newcommand{\DETuvwxijkl}[8]{
\begin{vmatrix}
 {#1}_{#5} & {#1}_{#6} & {#1}_{#7} & {#1}_{#8} \\
 {#2}_{#5} & {#2}_{#6} & {#2}_{#7} & {#2}_{#8} \\
 {#3}_{#5} & {#3}_{#6} & {#3}_{#7} & {#3}_{#8} \\
 {#4}_{#5} & {#4}_{#6} & {#4}_{#7} & {#4}_{#8} \\
\end{vmatrix}
}

%\newcommand{\DETuvwxyijklm}[10]{
%\begin{vmatrix}
% {#1}_{#6} & {#1}_{#7} & {#1}_{#8} & {#1}_{#9} & {#1}_{#10} \\
% {#2}_{#6} & {#2}_{#7} & {#2}_{#8} & {#2}_{#9} & {#2}_{#10} \\
% {#3}_{#6} & {#3}_{#7} & {#3}_{#8} & {#3}_{#9} & {#3}_{#10} \\
% {#4}_{#6} & {#4}_{#7} & {#4}_{#8} & {#4}_{#9} & {#4}_{#10} \\
% {#5}_{#6} & {#5}_{#7} & {#5}_{#8} & {#5}_{#9} & {#5}_{#10}
%\end{vmatrix}
%}

% R3 vector.
\newcommand{\VectorThree}[3]{
\begin{bmatrix}
 {#1} \\
 {#2} \\
 {#3}
\end{bmatrix}
}



\author{Peeter Joot}
\email{peeter.joot@gmail.com}

%\documentclass[]{eliblogwidescreen}

\usepackage{amsmath}
\usepackage{mathpazo}

%
% shorthand for bold symbols, convenient for vectors and matrices
%
\newcommand{\Ba}[0]{\mathbf{a}}
\newcommand{\Bb}[0]{\mathbf{b}}
\newcommand{\Bc}[0]{\mathbf{c}}
\newcommand{\Bd}[0]{\mathbf{d}}
\newcommand{\Be}[0]{\mathbf{e}}
\newcommand{\Bf}[0]{\mathbf{f}}
\newcommand{\Bg}[0]{\mathbf{g}}
\newcommand{\Bh}[0]{\mathbf{h}}
\newcommand{\Bi}[0]{\mathbf{i}}
\newcommand{\Bj}[0]{\mathbf{j}}
\newcommand{\Bk}[0]{\mathbf{k}}
\newcommand{\Bl}[0]{\mathbf{l}}
\newcommand{\Bm}[0]{\mathbf{m}}
\newcommand{\Bn}[0]{\mathbf{n}}
\newcommand{\Bo}[0]{\mathbf{o}}
\newcommand{\Bp}[0]{\mathbf{p}}
\newcommand{\Bq}[0]{\mathbf{q}}
\newcommand{\Br}[0]{\mathbf{r}}
\newcommand{\Bs}[0]{\mathbf{s}}
\newcommand{\Bt}[0]{\mathbf{t}}
\newcommand{\Bu}[0]{\mathbf{u}}
\newcommand{\Bv}[0]{\mathbf{v}}
\newcommand{\Bw}[0]{\mathbf{w}}
\newcommand{\Bx}[0]{\mathbf{x}}
\newcommand{\By}[0]{\mathbf{y}}
\newcommand{\Bz}[0]{\mathbf{z}}
\newcommand{\BA}[0]{\mathbf{A}}
\newcommand{\BB}[0]{\mathbf{B}}
\newcommand{\BC}[0]{\mathbf{C}}
\newcommand{\BD}[0]{\mathbf{D}}
\newcommand{\BE}[0]{\mathbf{E}}
\newcommand{\BF}[0]{\mathbf{F}}
\newcommand{\BG}[0]{\mathbf{G}}
\newcommand{\BH}[0]{\mathbf{H}}
\newcommand{\BI}[0]{\mathbf{I}}
\newcommand{\BJ}[0]{\mathbf{J}}
\newcommand{\BK}[0]{\mathbf{K}}
\newcommand{\BL}[0]{\mathbf{L}}
\newcommand{\BM}[0]{\mathbf{M}}
\newcommand{\BN}[0]{\mathbf{N}}
\newcommand{\BO}[0]{\mathbf{O}}
\newcommand{\BP}[0]{\mathbf{P}}
\newcommand{\BQ}[0]{\mathbf{Q}}
\newcommand{\BR}[0]{\mathbf{R}}
\newcommand{\BS}[0]{\mathbf{S}}
\newcommand{\BT}[0]{\mathbf{T}}
\newcommand{\BU}[0]{\mathbf{U}}
\newcommand{\BV}[0]{\mathbf{V}}
\newcommand{\BW}[0]{\mathbf{W}}
\newcommand{\BX}[0]{\mathbf{X}}
\newcommand{\BY}[0]{\mathbf{Y}}
\newcommand{\BZ}[0]{\mathbf{Z}}

\newcommand{\Bzero}[0]{\mathbf{0}}
\newcommand{\Btheta}[0]{\boldsymbol{\theta}}
\newcommand{\Btau}[0]{\boldsymbol{\tau}}
\newcommand{\Bomega}[0]{\boldsymbol{\omega}}

%
% shorthand for unit vectors
%
\newcommand{\acap}[0]{\hat{\Ba}}
\newcommand{\bcap}[0]{\hat{\Bb}}
\newcommand{\ccap}[0]{\hat{\Bc}}
\newcommand{\dcap}[0]{\hat{\Bd}}
\newcommand{\ecap}[0]{\hat{\Be}}
\newcommand{\fcap}[0]{\hat{\Bf}}
\newcommand{\gcap}[0]{\hat{\Bg}}
\newcommand{\hcap}[0]{\hat{\Bh}}
\newcommand{\icap}[0]{\hat{\Bi}}
\newcommand{\jcap}[0]{\hat{\Bj}}
\newcommand{\kcap}[0]{\hat{\Bk}}
\newcommand{\lcap}[0]{\hat{\Bl}}
\newcommand{\mcap}[0]{\hat{\Bm}}
\newcommand{\ncap}[0]{\hat{\Bn}}
\newcommand{\ocap}[0]{\hat{\Bo}}
\newcommand{\pcap}[0]{\hat{\Bp}}
\newcommand{\qcap}[0]{\hat{\Bq}}
\newcommand{\rcap}[0]{\hat{\Br}}
\newcommand{\scap}[0]{\hat{\Bs}}
\newcommand{\tcap}[0]{\hat{\Bt}}
\newcommand{\ucap}[0]{\hat{\Bu}}
\newcommand{\vcap}[0]{\hat{\Bv}}
\newcommand{\wcap}[0]{\hat{\Bw}}
\newcommand{\xcap}[0]{\hat{\Bx}}
\newcommand{\ycap}[0]{\hat{\By}}
\newcommand{\zcap}[0]{\hat{\Bz}}
\newcommand{\thetacap}[0]{\hat{\Btheta}}

%
% to write R^n and C^n in a distinguishable fashion.  Perhaps change this
% to the double lined characters upon figuring out how to do so.
%
\newcommand{\C}[1]{$\mathbb{C}^{#1}$}
\newcommand{\R}[1]{$\mathbb{R}^{#1}$}

%
% various generally useful helpers
%

% derivative of #1 wrt. #2:
\newcommand{\D}[2] {\frac {d#2} {d#1}}

\newcommand{\inv}[1]{\frac{1}{#1}}
\newcommand{\cross}[0]{\times}

\newcommand{\abs}[1]{\lvert{#1}\rvert}
\newcommand{\norm}[1]{\lVert{#1}\rVert}
\newcommand{\innerprod}[2]{\langle{#1}, {#2}\rangle}
\newcommand{\dotprod}[2]{{#1} \cdot {#2}}
\newcommand{\bdotprod}[2]{\left({#1} \cdot {#2}\right)}
\newcommand{\crossprod}[2]{{#1} \cross {#2}}
\newcommand{\tripleprod}[3]{\dotprod{\left(\crossprod{#1}{#2}\right)}{#3}}

\DeclareMathOperator{\Proj}{Proj}
\DeclareMathOperator{\Span}{span}
\DeclareMathOperator{\Sgn}{sgn}
\DeclareMathOperator{\Area}{Area}
\DeclareMathOperator{\Volume}{Volume}

%
% A few miscellaneous things specific to this document
%
\newcommand{\crossop}[1]{\crossprod{#1}{}}

% R2 vector.
\newcommand{\VectorTwo}[2]{
\begin{bmatrix}
 {#1} \\
 {#2}
\end{bmatrix}
}

\newcommand{\VectorN}[1]{
\begin{bmatrix}
{#1}_1 \\
{#1}_2 \\
\vdots \\
{#1}_N \\
\end{bmatrix}
}

\newcommand{\DETuvij}[4]{
\begin{vmatrix}
 {#1}_{#3} & {#1}_{#4} \\
 {#2}_{#3} & {#2}_{#4}
\end{vmatrix}
}

\newcommand{\DETuvwijk}[6]{
\begin{vmatrix}
 {#1}_{#4} & {#1}_{#5} & {#1}_{#6} \\
 {#2}_{#4} & {#2}_{#5} & {#2}_{#6} \\
 {#3}_{#4} & {#3}_{#5} & {#3}_{#6}
\end{vmatrix}
}

\newcommand{\DETuvwxijkl}[8]{
\begin{vmatrix}
 {#1}_{#5} & {#1}_{#6} & {#1}_{#7} & {#1}_{#8} \\
 {#2}_{#5} & {#2}_{#6} & {#2}_{#7} & {#2}_{#8} \\
 {#3}_{#5} & {#3}_{#6} & {#3}_{#7} & {#3}_{#8} \\
 {#4}_{#5} & {#4}_{#6} & {#4}_{#7} & {#4}_{#8} \\
\end{vmatrix}
}

%\newcommand{\DETuvwxyijklm}[10]{
%\begin{vmatrix}
% {#1}_{#6} & {#1}_{#7} & {#1}_{#8} & {#1}_{#9} & {#1}_{#10} \\
% {#2}_{#6} & {#2}_{#7} & {#2}_{#8} & {#2}_{#9} & {#2}_{#10} \\
% {#3}_{#6} & {#3}_{#7} & {#3}_{#8} & {#3}_{#9} & {#3}_{#10} \\
% {#4}_{#6} & {#4}_{#7} & {#4}_{#8} & {#4}_{#9} & {#4}_{#10} \\
% {#5}_{#6} & {#5}_{#7} & {#5}_{#8} & {#5}_{#9} & {#5}_{#10}
%\end{vmatrix}
%}

% R3 vector.
\newcommand{\VectorThree}[3]{
\begin{bmatrix}
 {#1} \\
 {#2} \\
 {#3}
\end{bmatrix}
}



\author{Peeter Joot}
\email{peeter.joot@gmail.com}


%\chapter{PHY450H1S.  Relativistic Electrodynamics Tutorial 9 (TA: Simon Freedman).  Some worked problems.  EM reflection.  Stress energy tensor for simple configurations.}
\chapter{Some worked problems.  EM reflection.  Stress energy tensor for simple configurations.}
\label{chap:relativisticElectrodynamicsT9}
%\useCCL
\blogpage{http://sites.google.com/site/peeterjoot/math2011/relativisticElectrodynamicsT9.pdf}
\date{Mar 30, 2011}
\revisionInfo{relativisticElectrodynamicsT9.tex}

\beginArtWithToc
%\beginArtNoToc

\section{HW6. Question 3. (Non subtle hints about how important this is (i.e. for the exam)}

\subsection{Motivation.}

This is problem 1 from \S 47 of the text \citep{landau1980classical}.

Determine the force exerted on a wall from which an incident plane EM wave is reflected (w/ reflection coefficient $R$) and incident angle $\theta$.

Solution from the book

\begin{equation}\label{eqn:relativisticElectrodynamicsT9:10}
f_\alpha = - \sigma_{\alpha \beta} n_\beta - {\sigma'}_{\alpha \beta} n_\beta
\end{equation}

Here $\sigma_{\alpha \beta}$ is the Maxwell stress tensor for the incident wave, and ${\sigma'}_{\alpha \beta}$ is the Maxwell stress tensor for the reflected wave, and $n_\beta$ is normal to the wall.

\subsection{On the signs of the force per unit area}

The signs in \ref{eqn:relativisticElectrodynamicsT9:10} require a bit of thought.  We have for the rate of change of the $\alpha$ component of the field momentum

\begin{equation}\label{eqn:relativisticElectrodynamicsT9:11}
\ddt{} \int d^3 \Bx \left( \frac{S^\alpha}{c^2} \right) = - \int d^2 \sigma^\beta T^{\beta \alpha}
\end{equation}

where $d^2 \sigma^\beta = d^2 \sigma \Bn \cdot \Be_\beta$, and $\Bn$ is the outwards unit normal to the surface.  This is the rate of change of momentum for the field, the force on the field.  For the force on the wall per unit area, we wish to invert this, giving

\begin{equation}\label{eqn:relativisticElectrodynamicsT9:500}
df^\alpha_{\text{on the wall, per unit area}}
= (\Bn \cdot \Be_\beta) T^{\beta \alpha}
= -(\Bn \cdot \Be_\beta) \sigma_{\beta \alpha}
\end{equation}

\subsection{Returning to the tutorial notes}

Simon writes

\begin{align}\label{eqn:relativisticElectrodynamicsT9:30}
f_\perp &= - \sigma_{\perp \perp} - {\sigma'}_{\perp \perp} \\
f_\parallel &= - \sigma_{\parallel \perp} - {\sigma'}_{\parallel \perp} 
\end{align}

and then says stating this solution is very non-trivial, because $\sigma_{\alpha \beta}$ is non-linear in $\BE$ and $\BB$.  This non-triviality is a good point.  Without calculating it, I find the results above to be pulled out of a magic hat.  The point of the tutorial discussion was to work through this in detail.

\section{Working out the tensor.}

PICTURE: ...

The Reflection coefficient can be defined in this case as

\begin{equation}\label{eqn:relativisticElectrodynamicsT9:50}
R = \frac{ \Abs{\BE'}^2 }{ \Abs{\BE}^2 },
\end{equation}

a ratio of the powers of the reflected wave power to the incident wave power (which are proportional to ${\BE'}^2$ and ${\BE}^2$ respectively.

Suppose we pick the following orientation for the incident fields

\begin{align}\label{eqn:relativisticElectrodynamicsT9:70}
E_x &= E \sin\theta \\
E_y &= -E \cos\theta \\
B_z &= E ,
\end{align}

With the reflected assumed to be in some still perpendicular orientation (with this orientation picked for convenience)

\begin{align}\label{eqn:relativisticElectrodynamicsT9:90}
E_x' &= E' \sin\theta \\
E_y' &= E' \cos\theta \\
B_z' &= E'.
\end{align}

Here

\begin{align}\label{eqn:relativisticElectrodynamicsT9:110}
E &= E_0 \cos(\Bp \cdot \Bx - \omega t) \\
E' &= \sqrt{R} E_0 \cos(\Bp' \cdot \Bx - \omega t)
\end{align}

Observe that while the propagation directions are difference for the incident and the reflected waves, these differences in phase are incorporated into the $E$ and $E'$ variables that we will work with below.  In the very end when the forces are computed, averages will be taken, but until then we will see that these phase differences do not effect the physics explicitly.  As Simon pointed out this makes good physical sense since we can form a picture of these things as just momentum and energy fields hitting an object.  We could even incorporate an additional constant phase difference into the reflected wave (which may also make physical sense), but it wouldn't change the pressure that the radiation applies to the surface.

\begin{equation}\label{eqn:relativisticElectrodynamicsT9:130}
\sigma_{\alpha\beta} = -T^{\alpha\beta} = \inv{4\pi} \left(
\mathcal{E}^\alpha
\mathcal{E}^\beta
+\mathcal{B}^\alpha
\mathcal{B}^\beta
- \inv{2} \delta^{\alpha\beta} ( \vec{\mathcal{E}}^2 + \vec{\mathcal{B}}^2 )
\right)
\end{equation}

\subsection{Aside: On the geometry, and the angle of incidence.}

According to wikipedia \citep{wiki:angleOfIncidence} the angle of incidence is measured from the normal.

Let's use complex numbers to get the orientation of the electric and propagation direction fields right.  We have for the incident propagation direction

\begin{equation}\label{eqn:relativisticElectrodynamicsT9:520}
-\pcap \sim e^{i (\pi + \theta) }
\end{equation}

or

\begin{equation}\label{eqn:relativisticElectrodynamicsT9:540}
\pcap \sim e^{i\theta}
\end{equation}

If we pick the electric field rotated negatively from that direction, we have

\begin{align*}
\hat{\BE} 
&\sim -i e^{i \theta} \\
&= -i (\cos\theta + i \sin\theta) \\
&= -i \cos\theta + \sin\theta
\end{align*}

Or

\begin{align}\label{eqn:relativisticElectrodynamicsT9:560}
E_x &\sim \sin\theta \\
E_y &\sim -\cos\theta
\end{align}

For the reflected direction we have

\begin{equation}\label{eqn:relativisticElectrodynamicsT9:580}
\pcap' \sim e^{i (\pi - \theta)} = - e^{-i \theta}
\end{equation}

rotating negatively for the electric field direction, we have

\begin{align*}
\hat{\BE'} 
&\sim -i (- e^{-i\theta} ) \\
&= i(cos\theta - i\sin\theta) \\
&= i cos\theta + \sin\theta
\end{align*}

Or

\begin{align}\label{eqn:relativisticElectrodynamicsT9:600}
E_x' &\sim \sin\theta \\
E_y' &\sim \cos\theta 
\end{align}

\subsection{Back to the problem (again).}

Where $\vec{\mathcal{E}}$ and $\vec{\mathcal{B}}$ are the total EM fields.

\paragraph{Aside:} Why the fields are added in this fashion wasn't clear to me, but I guess this makes sense.  Even if the propagation directions differ, the total field at any point is still just a superposition.

\begin{align}\label{eqn:relativisticElectrodynamicsT9:150}
\vec{\mathcal{E}} &= \BE + \BE' \\
\vec{\mathcal{B}} &= \BB + \BB'
\end{align}

Get 

\begin{align}\label{eqn:relativisticElectrodynamicsT9:170}
\sigma_{3 3} &= \inv{4 \pi} \left( \underbrace{B_z B_z}_{=\vec{\mathcal{B}}^2} - \inv{2} (\vec{\mathcal{E}}^2 + \vec{\mathcal{B}}^2) \right) = 0 \\
\sigma_{3 1} &= 0 = \sigma_{3 2} \\
\sigma_{1 1} &= \inv{4 \pi} \left( (\mathcal{E}^1)^2 - \inv{2} (\vec{\mathcal{E}}^2 + \vec{\mathcal{B}}^2) \right) 
\end{align}

\begin{align}\label{eqn:relativisticElectrodynamicsT9:190}
\vec{\mathcal{B}}^2 &= (B_z + B_z')^2 = (E + E')^2 \\
\vec{\mathcal{E}}^2 &= (\BE + \BE')^2
\end{align}

so
\begin{align*}
\sigma_{1 1} 
&= \inv{4 \pi} \left( (\mathcal{E}^1)^2 - \inv{2} ((\mathcal{E}^1)^2 + (\mathcal{E}^2)^2 + (E + E')^2 \right) \\
&= \inv{8 \pi} \left( (\mathcal{E}^1)^2 - (\mathcal{E}^2)^2 - (E + E')^2 \right) \\
&= \inv{8 \pi} \left( 
(E + E')^2 \sin^2\theta 
-(E' - E)^2 \cos^2\theta -(E + E')^2
\right) \\
&= \inv{8 \pi} \left( 
E^2( \sin^2\theta - \cos^2\theta - 1)
+(E')^2( \sin^2\theta - \cos^2\theta - 1)
+ 2 E E' (\sin^2\theta + \cos^2\theta -1 )
\right) \\
&= \inv{8 \pi} \left( - 2 E^2 \cos^2\theta - 2 (E')^2 \cos^2\theta \right) \\
&= -\inv{4 \pi} (E^2 + (E')^2) \cos^2\theta \\
&= \sigma_\parallel + {\sigma'}_\parallel
\end{align*}

This last bit I didn't get.  What is $\sigma_\parallel$ and ${\sigma'}_\parallel$.  Are these parallel to the wall or parallel to the normal to the wall.  It turns out that this appears to mean parallel to the normal.  We can see this by direct calculation

\begin{align*}
\sigma_{x x}^{\text{incident}} 
&= \inv{4 \pi} \left( 
E_x^2 - \inv{2} (\BE^2 + \BB^2)
\right) \\
&= \inv{4 \pi} \left( 
E^2 \sin^2 \theta - \inv{2} 2 E^2 
\right) \\
&= -\inv{4 \pi} E^2 \cos^2\theta
\end{align*}

\begin{align*}
{\sigma}_{x x}^{\text{reflected}} 
&= \inv{4 \pi} \left( 
{E_x'}^2 - \inv{2} ({\BE'}^2 + {\BB'}^2)
\right) \\
&= \inv{4 \pi} \left( 
{E'}^2 \sin^2 \theta - \inv{2} 2 {E'}^2 
\right) \\
&= -\inv{4 \pi} {E'}^2 \cos^2\theta
\end{align*}

So by comparison we see that we have

\begin{equation}\label{eqn:relativisticElectrodynamicsT9:620}
\sigma_{1 1} = {\sigma}_{x x}^{\text{incident}} +{\sigma}_{x x}^{\text{reflected}} 
\end{equation}

Moving on, for our other component on the $x,y$ place $\sigma_{12}$ we have

\begin{align*}
\sigma_{12} 
&= \inv{4 \pi} \mathcal{E}^1 \mathcal{E}^2 \\
&= \inv{4 \pi} (E + E') \sin\theta (-E + E') \cos\theta \\
&= \inv{4 \pi} ((E')^2 - E^2) \sin\theta \cos\theta 
\end{align*}

Again we can compare to the sums of the reflected and incident tensors for this $x,y$ component.  Those are

\begin{align*}
\sigma_{12}^{\text{incident}} 
&= 
\inv{4 \pi} ( E^1 E^2 ) \\
&= 
-\inv{4 \pi} E^2 \sin\theta \cos\theta,
\end{align*}

and
\begin{align*}
\sigma_{12}^{\text{reflected}} 
&= 
\inv{4 \pi} ( {E'}^1 {E'}^2 ) \\
&= 
\inv{4 \pi} {E'}^2 \sin\theta \cos\theta
\end{align*}

Which demonstrates that we have

\begin{equation}\label{eqn:relativisticElectrodynamicsT9:640}
\sigma_{12} = \sigma_{12}^{\text{incident}} + \sigma_{12}^{\text{reflected}} 
\end{equation}

Summarizing, for the components in the $x,y$ plane we have found that we have

\begin{equation}\label{eqn:relativisticElectrodynamicsT9:210}
\sigma_{\alpha\beta}^{\text{total}} n_\beta = \sigma_{\alpha 1 }^{\text{total}} = \sigma_{\alpha 1} + {\sigma'}_{\alpha 1}
\end{equation}

(where $n_\beta = \delta^{\beta 1}$)

This result, assumed in the text, was non-trivial to derive.  It is also not generally true.  We have 

\begin{align*}
\sigma_{2 2} 
&= \inv{4 \pi} \left( (\mathcal{E}^y)^2 - \inv{2} ( \vec{\mathcal{E}}^2 + \vec{\mathcal{B}}^2 ) \right) \\
&= \inv{8 \pi} \left( (\mathcal{E}^y)^2 - (\mathcal{E}^x)^2 - \vec{\mathcal{B}}^2 ) \right) \\
&= \inv{8 \pi} \left( 
(E' - E)^2 \cos^2\theta - (E + E')^2 \sin^2\theta - (E + E')^2
\right) \\
&= \inv{8 \pi} \left( 
E^2 ( -1 + \cos^2 \theta - \sin^2\theta )
+{E'}^2 ( -1 + \cos^2 \theta - \sin^2\theta )
+ 2 E E' ( -\cos^2\theta - \sin^2\theta -1 ) \right) \\
&= -\inv{4 \pi} \left( E^2 \sin^2 \theta + (E')^2 \sin^2 \theta + 2 E E' \right)
\end{align*}

If we compare to the incident and reflected tensors we have

\begin{align*}
\sigma_{y y}^{\text{incident}} 
&= \inv{4 \pi} \left( (E^y)^2 -\inv{2} E^2 \right) \\
&= \inv{4 \pi} E^2 ( \cos^2\theta - 1 ) \\
&= -\inv{4 \pi} E^2 \sin^2\theta 
\end{align*}

and
\begin{align*}
\sigma_{y y}^{\text{reflected}} 
&= \inv{4 \pi} \left( ({E'}^y)^2 -\inv{2} {E'}^2 \right) \\
&= \inv{4 \pi} {E'}^2 ( \cos^2\theta - 1 ) \\
&= -\inv{4 \pi} {E'}^2 \sin^2\theta 
\end{align*}

There's a cross term that we can't have summing the two, so we have, in general

\begin{equation}\label{eqn:relativisticElectrodynamicsT9:660}
\sigma_{2 2}^{\text{total}} \ne 
\sigma_{y y}^{\text{incident}} 
+\sigma_{y y}^{\text{reflected}} 
\end{equation}

\subsection{Force per unit area?}

\begin{equation}\label{eqn:relativisticElectrodynamicsT9:230}
f_\alpha = n^x \sigma_{x \alpha} 
\end{equation}

Averaged

\begin{align}\label{eqn:relativisticElectrodynamicsT9:250}
\expectation{\sigma_{xx}} &= -\inv{8 \pi} E_0^2 ( 1 + R) \cos^2\theta \\
\expectation{\sigma_{xy}} &= -\inv{8 \pi} E_0^2 ( 1 - R) \sin\theta \cos\theta
\end{align}

\begin{align}\label{eqn:relativisticElectrodynamicsT9:270}
\expectation{\BS} &= -\frac{c}{8 \pi} E_0^2 \ncap \\
\expectation{\BS'} &= -\frac{c}{8 \pi} E_0^2 \ncap'
\end{align}

\begin{equation}\label{eqn:relativisticElectrodynamicsT9:290}
\expectation{\Abs{\BS}} = \text{Work} = W
\end{equation}

\begin{align}\label{eqn:relativisticElectrodynamicsT9:310}
f_x &= n^x \sigma_{x x} = W (1 + R) \cos^2\theta \\
f_y &= n^y \sigma_{x y} = W (1 - R) \sin\theta \cos\theta \\
f_z &= 0
\end{align}

\section{A problem from Griffiths.}

FIXME: try this.

Two charges $q+$, $q-$ reflected in a plane, separated by distance $a$.  Work out the stress energy tensor from the Coulomb fields of the charges on the plane.

Will get the Coulomb force:

\begin{equation}\label{eqn:relativisticElectrodynamicsT9:330}
\BF = k \frac{q^2}{2 a^2}.
\end{equation}

\section{Infinite parallel plate capacitor}

Write $\sigma_{\alpha\beta}$.

\begin{align}\label{eqn:relativisticElectrodynamicsT9:350}
\BB &= 0  \\
\BE &= - \frac{\sigma}{\epsilon_0} \Be_z
\end{align}

FIXME: derive this.  Observe that we have no distance dependence in the field because it is an infinite plate.

\begin{align}\label{eqn:relativisticElectrodynamicsT9:370}
\sigma_{1 1} &= \left( - \inv{2} \delta^{1 1} \left( \frac{-\sigma}{\epsilon_0} \right)^2 \right) = - \frac{ \sigma^2}{ 2 \epsilon_0^2 } = \sigma_{22} \\
\sigma_{3 3} &= \left( (E^3)^2 - \inv{2} \BE^2 \right)  = - \inv{2} \BE^2 = - \sigma_{2 2}
\end{align}

Force per unit area is then

\begin{align*}
f_\alpha 
&= n_\beta \sigma_{\alpha \beta} \\
&= n_3 \sigma_{\alpha 3}
\end{align*}

So 

\begin{align}\label{eqn:relativisticElectrodynamicsT9:390}
f_1 &= 0 = f_2 \\
f_3 &= \sigma_{3 3} = -\frac{\sigma^2}{2 \epsilon_0^2}
\end{align}

\begin{equation}\label{eqn:relativisticElectrodynamicsT9:410}
\Bf = -\frac{\sigma^2}{2 \epsilon_0^2} \Be_z
\end{equation}

REMEMBER: EXAM WEDNESDAY!
%  Simon Available on Monday.

\EndArticle
