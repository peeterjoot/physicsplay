%%
% Copyright � 2015 Peeter Joot.  All Rights Reserved.
% Licenced as described in the file LICENSE under the root directory of this GIT repository.
%
\documentclass[]{eliblog}

\usepackage{amsmath}
\usepackage{mathpazo}

%
% shorthand for bold symbols, convenient for vectors and matrices
%
\newcommand{\Ba}[0]{\mathbf{a}}
\newcommand{\Bb}[0]{\mathbf{b}}
\newcommand{\Bc}[0]{\mathbf{c}}
\newcommand{\Bd}[0]{\mathbf{d}}
\newcommand{\Be}[0]{\mathbf{e}}
\newcommand{\Bf}[0]{\mathbf{f}}
\newcommand{\Bg}[0]{\mathbf{g}}
\newcommand{\Bh}[0]{\mathbf{h}}
\newcommand{\Bi}[0]{\mathbf{i}}
\newcommand{\Bj}[0]{\mathbf{j}}
\newcommand{\Bk}[0]{\mathbf{k}}
\newcommand{\Bl}[0]{\mathbf{l}}
\newcommand{\Bm}[0]{\mathbf{m}}
\newcommand{\Bn}[0]{\mathbf{n}}
\newcommand{\Bo}[0]{\mathbf{o}}
\newcommand{\Bp}[0]{\mathbf{p}}
\newcommand{\Bq}[0]{\mathbf{q}}
\newcommand{\Br}[0]{\mathbf{r}}
\newcommand{\Bs}[0]{\mathbf{s}}
\newcommand{\Bt}[0]{\mathbf{t}}
\newcommand{\Bu}[0]{\mathbf{u}}
\newcommand{\Bv}[0]{\mathbf{v}}
\newcommand{\Bw}[0]{\mathbf{w}}
\newcommand{\Bx}[0]{\mathbf{x}}
\newcommand{\By}[0]{\mathbf{y}}
\newcommand{\Bz}[0]{\mathbf{z}}
\newcommand{\BA}[0]{\mathbf{A}}
\newcommand{\BB}[0]{\mathbf{B}}
\newcommand{\BC}[0]{\mathbf{C}}
\newcommand{\BD}[0]{\mathbf{D}}
\newcommand{\BE}[0]{\mathbf{E}}
\newcommand{\BF}[0]{\mathbf{F}}
\newcommand{\BG}[0]{\mathbf{G}}
\newcommand{\BH}[0]{\mathbf{H}}
\newcommand{\BI}[0]{\mathbf{I}}
\newcommand{\BJ}[0]{\mathbf{J}}
\newcommand{\BK}[0]{\mathbf{K}}
\newcommand{\BL}[0]{\mathbf{L}}
\newcommand{\BM}[0]{\mathbf{M}}
\newcommand{\BN}[0]{\mathbf{N}}
\newcommand{\BO}[0]{\mathbf{O}}
\newcommand{\BP}[0]{\mathbf{P}}
\newcommand{\BQ}[0]{\mathbf{Q}}
\newcommand{\BR}[0]{\mathbf{R}}
\newcommand{\BS}[0]{\mathbf{S}}
\newcommand{\BT}[0]{\mathbf{T}}
\newcommand{\BU}[0]{\mathbf{U}}
\newcommand{\BV}[0]{\mathbf{V}}
\newcommand{\BW}[0]{\mathbf{W}}
\newcommand{\BX}[0]{\mathbf{X}}
\newcommand{\BY}[0]{\mathbf{Y}}
\newcommand{\BZ}[0]{\mathbf{Z}}

\newcommand{\Bzero}[0]{\mathbf{0}}
\newcommand{\Btheta}[0]{\boldsymbol{\theta}}
\newcommand{\Btau}[0]{\boldsymbol{\tau}}
\newcommand{\Bomega}[0]{\boldsymbol{\omega}}

%
% shorthand for unit vectors
%
\newcommand{\acap}[0]{\hat{\Ba}}
\newcommand{\bcap}[0]{\hat{\Bb}}
\newcommand{\ccap}[0]{\hat{\Bc}}
\newcommand{\dcap}[0]{\hat{\Bd}}
\newcommand{\ecap}[0]{\hat{\Be}}
\newcommand{\fcap}[0]{\hat{\Bf}}
\newcommand{\gcap}[0]{\hat{\Bg}}
\newcommand{\hcap}[0]{\hat{\Bh}}
\newcommand{\icap}[0]{\hat{\Bi}}
\newcommand{\jcap}[0]{\hat{\Bj}}
\newcommand{\kcap}[0]{\hat{\Bk}}
\newcommand{\lcap}[0]{\hat{\Bl}}
\newcommand{\mcap}[0]{\hat{\Bm}}
\newcommand{\ncap}[0]{\hat{\Bn}}
\newcommand{\ocap}[0]{\hat{\Bo}}
\newcommand{\pcap}[0]{\hat{\Bp}}
\newcommand{\qcap}[0]{\hat{\Bq}}
\newcommand{\rcap}[0]{\hat{\Br}}
\newcommand{\scap}[0]{\hat{\Bs}}
\newcommand{\tcap}[0]{\hat{\Bt}}
\newcommand{\ucap}[0]{\hat{\Bu}}
\newcommand{\vcap}[0]{\hat{\Bv}}
\newcommand{\wcap}[0]{\hat{\Bw}}
\newcommand{\xcap}[0]{\hat{\Bx}}
\newcommand{\ycap}[0]{\hat{\By}}
\newcommand{\zcap}[0]{\hat{\Bz}}
\newcommand{\thetacap}[0]{\hat{\Btheta}}

%
% to write R^n and C^n in a distinguishable fashion.  Perhaps change this
% to the double lined characters upon figuring out how to do so.
%
\newcommand{\C}[1]{$\mathbb{C}^{#1}$}
\newcommand{\R}[1]{$\mathbb{R}^{#1}$}

%
% various generally useful helpers
%

% derivative of #1 wrt. #2:
\newcommand{\D}[2] {\frac {d#2} {d#1}}

\newcommand{\inv}[1]{\frac{1}{#1}}
\newcommand{\cross}[0]{\times}

\newcommand{\abs}[1]{\lvert{#1}\rvert}
\newcommand{\norm}[1]{\lVert{#1}\rVert}
\newcommand{\innerprod}[2]{\langle{#1}, {#2}\rangle}
\newcommand{\dotprod}[2]{{#1} \cdot {#2}}
\newcommand{\bdotprod}[2]{\left({#1} \cdot {#2}\right)}
\newcommand{\crossprod}[2]{{#1} \cross {#2}}
\newcommand{\tripleprod}[3]{\dotprod{\left(\crossprod{#1}{#2}\right)}{#3}}

\DeclareMathOperator{\Proj}{Proj}
\DeclareMathOperator{\Span}{span}
\DeclareMathOperator{\Sgn}{sgn}
\DeclareMathOperator{\Area}{Area}
\DeclareMathOperator{\Volume}{Volume}

%
% A few miscellaneous things specific to this document
%
\newcommand{\crossop}[1]{\crossprod{#1}{}}

% R2 vector.
\newcommand{\VectorTwo}[2]{
\begin{bmatrix}
 {#1} \\
 {#2}
\end{bmatrix}
}

\newcommand{\VectorN}[1]{
\begin{bmatrix}
{#1}_1 \\
{#1}_2 \\
\vdots \\
{#1}_N \\
\end{bmatrix}
}

\newcommand{\DETuvij}[4]{
\begin{vmatrix}
 {#1}_{#3} & {#1}_{#4} \\
 {#2}_{#3} & {#2}_{#4}
\end{vmatrix}
}

\newcommand{\DETuvwijk}[6]{
\begin{vmatrix}
 {#1}_{#4} & {#1}_{#5} & {#1}_{#6} \\
 {#2}_{#4} & {#2}_{#5} & {#2}_{#6} \\
 {#3}_{#4} & {#3}_{#5} & {#3}_{#6}
\end{vmatrix}
}

\newcommand{\DETuvwxijkl}[8]{
\begin{vmatrix}
 {#1}_{#5} & {#1}_{#6} & {#1}_{#7} & {#1}_{#8} \\
 {#2}_{#5} & {#2}_{#6} & {#2}_{#7} & {#2}_{#8} \\
 {#3}_{#5} & {#3}_{#6} & {#3}_{#7} & {#3}_{#8} \\
 {#4}_{#5} & {#4}_{#6} & {#4}_{#7} & {#4}_{#8} \\
\end{vmatrix}
}

%\newcommand{\DETuvwxyijklm}[10]{
%\begin{vmatrix}
% {#1}_{#6} & {#1}_{#7} & {#1}_{#8} & {#1}_{#9} & {#1}_{#10} \\
% {#2}_{#6} & {#2}_{#7} & {#2}_{#8} & {#2}_{#9} & {#2}_{#10} \\
% {#3}_{#6} & {#3}_{#7} & {#3}_{#8} & {#3}_{#9} & {#3}_{#10} \\
% {#4}_{#6} & {#4}_{#7} & {#4}_{#8} & {#4}_{#9} & {#4}_{#10} \\
% {#5}_{#6} & {#5}_{#7} & {#5}_{#8} & {#5}_{#9} & {#5}_{#10}
%\end{vmatrix}
%}

% R3 vector.
\newcommand{\VectorThree}[3]{
\begin{bmatrix}
 {#1} \\
 {#2} \\
 {#3}
\end{bmatrix}
}



\author{Peeter Joot}
\email{peeter.joot@gmail.com}

%\documentclass[]{eliblogwidescreen}

\usepackage{amsmath}
\usepackage{mathpazo}

%
% shorthand for bold symbols, convenient for vectors and matrices
%
\newcommand{\Ba}[0]{\mathbf{a}}
\newcommand{\Bb}[0]{\mathbf{b}}
\newcommand{\Bc}[0]{\mathbf{c}}
\newcommand{\Bd}[0]{\mathbf{d}}
\newcommand{\Be}[0]{\mathbf{e}}
\newcommand{\Bf}[0]{\mathbf{f}}
\newcommand{\Bg}[0]{\mathbf{g}}
\newcommand{\Bh}[0]{\mathbf{h}}
\newcommand{\Bi}[0]{\mathbf{i}}
\newcommand{\Bj}[0]{\mathbf{j}}
\newcommand{\Bk}[0]{\mathbf{k}}
\newcommand{\Bl}[0]{\mathbf{l}}
\newcommand{\Bm}[0]{\mathbf{m}}
\newcommand{\Bn}[0]{\mathbf{n}}
\newcommand{\Bo}[0]{\mathbf{o}}
\newcommand{\Bp}[0]{\mathbf{p}}
\newcommand{\Bq}[0]{\mathbf{q}}
\newcommand{\Br}[0]{\mathbf{r}}
\newcommand{\Bs}[0]{\mathbf{s}}
\newcommand{\Bt}[0]{\mathbf{t}}
\newcommand{\Bu}[0]{\mathbf{u}}
\newcommand{\Bv}[0]{\mathbf{v}}
\newcommand{\Bw}[0]{\mathbf{w}}
\newcommand{\Bx}[0]{\mathbf{x}}
\newcommand{\By}[0]{\mathbf{y}}
\newcommand{\Bz}[0]{\mathbf{z}}
\newcommand{\BA}[0]{\mathbf{A}}
\newcommand{\BB}[0]{\mathbf{B}}
\newcommand{\BC}[0]{\mathbf{C}}
\newcommand{\BD}[0]{\mathbf{D}}
\newcommand{\BE}[0]{\mathbf{E}}
\newcommand{\BF}[0]{\mathbf{F}}
\newcommand{\BG}[0]{\mathbf{G}}
\newcommand{\BH}[0]{\mathbf{H}}
\newcommand{\BI}[0]{\mathbf{I}}
\newcommand{\BJ}[0]{\mathbf{J}}
\newcommand{\BK}[0]{\mathbf{K}}
\newcommand{\BL}[0]{\mathbf{L}}
\newcommand{\BM}[0]{\mathbf{M}}
\newcommand{\BN}[0]{\mathbf{N}}
\newcommand{\BO}[0]{\mathbf{O}}
\newcommand{\BP}[0]{\mathbf{P}}
\newcommand{\BQ}[0]{\mathbf{Q}}
\newcommand{\BR}[0]{\mathbf{R}}
\newcommand{\BS}[0]{\mathbf{S}}
\newcommand{\BT}[0]{\mathbf{T}}
\newcommand{\BU}[0]{\mathbf{U}}
\newcommand{\BV}[0]{\mathbf{V}}
\newcommand{\BW}[0]{\mathbf{W}}
\newcommand{\BX}[0]{\mathbf{X}}
\newcommand{\BY}[0]{\mathbf{Y}}
\newcommand{\BZ}[0]{\mathbf{Z}}

\newcommand{\Bzero}[0]{\mathbf{0}}
\newcommand{\Btheta}[0]{\boldsymbol{\theta}}
\newcommand{\Btau}[0]{\boldsymbol{\tau}}
\newcommand{\Bomega}[0]{\boldsymbol{\omega}}

%
% shorthand for unit vectors
%
\newcommand{\acap}[0]{\hat{\Ba}}
\newcommand{\bcap}[0]{\hat{\Bb}}
\newcommand{\ccap}[0]{\hat{\Bc}}
\newcommand{\dcap}[0]{\hat{\Bd}}
\newcommand{\ecap}[0]{\hat{\Be}}
\newcommand{\fcap}[0]{\hat{\Bf}}
\newcommand{\gcap}[0]{\hat{\Bg}}
\newcommand{\hcap}[0]{\hat{\Bh}}
\newcommand{\icap}[0]{\hat{\Bi}}
\newcommand{\jcap}[0]{\hat{\Bj}}
\newcommand{\kcap}[0]{\hat{\Bk}}
\newcommand{\lcap}[0]{\hat{\Bl}}
\newcommand{\mcap}[0]{\hat{\Bm}}
\newcommand{\ncap}[0]{\hat{\Bn}}
\newcommand{\ocap}[0]{\hat{\Bo}}
\newcommand{\pcap}[0]{\hat{\Bp}}
\newcommand{\qcap}[0]{\hat{\Bq}}
\newcommand{\rcap}[0]{\hat{\Br}}
\newcommand{\scap}[0]{\hat{\Bs}}
\newcommand{\tcap}[0]{\hat{\Bt}}
\newcommand{\ucap}[0]{\hat{\Bu}}
\newcommand{\vcap}[0]{\hat{\Bv}}
\newcommand{\wcap}[0]{\hat{\Bw}}
\newcommand{\xcap}[0]{\hat{\Bx}}
\newcommand{\ycap}[0]{\hat{\By}}
\newcommand{\zcap}[0]{\hat{\Bz}}
\newcommand{\thetacap}[0]{\hat{\Btheta}}

%
% to write R^n and C^n in a distinguishable fashion.  Perhaps change this
% to the double lined characters upon figuring out how to do so.
%
\newcommand{\C}[1]{$\mathbb{C}^{#1}$}
\newcommand{\R}[1]{$\mathbb{R}^{#1}$}

%
% various generally useful helpers
%

% derivative of #1 wrt. #2:
\newcommand{\D}[2] {\frac {d#2} {d#1}}

\newcommand{\inv}[1]{\frac{1}{#1}}
\newcommand{\cross}[0]{\times}

\newcommand{\abs}[1]{\lvert{#1}\rvert}
\newcommand{\norm}[1]{\lVert{#1}\rVert}
\newcommand{\innerprod}[2]{\langle{#1}, {#2}\rangle}
\newcommand{\dotprod}[2]{{#1} \cdot {#2}}
\newcommand{\bdotprod}[2]{\left({#1} \cdot {#2}\right)}
\newcommand{\crossprod}[2]{{#1} \cross {#2}}
\newcommand{\tripleprod}[3]{\dotprod{\left(\crossprod{#1}{#2}\right)}{#3}}

\DeclareMathOperator{\Proj}{Proj}
\DeclareMathOperator{\Span}{span}
\DeclareMathOperator{\Sgn}{sgn}
\DeclareMathOperator{\Area}{Area}
\DeclareMathOperator{\Volume}{Volume}

%
% A few miscellaneous things specific to this document
%
\newcommand{\crossop}[1]{\crossprod{#1}{}}

% R2 vector.
\newcommand{\VectorTwo}[2]{
\begin{bmatrix}
 {#1} \\
 {#2}
\end{bmatrix}
}

\newcommand{\VectorN}[1]{
\begin{bmatrix}
{#1}_1 \\
{#1}_2 \\
\vdots \\
{#1}_N \\
\end{bmatrix}
}

\newcommand{\DETuvij}[4]{
\begin{vmatrix}
 {#1}_{#3} & {#1}_{#4} \\
 {#2}_{#3} & {#2}_{#4}
\end{vmatrix}
}

\newcommand{\DETuvwijk}[6]{
\begin{vmatrix}
 {#1}_{#4} & {#1}_{#5} & {#1}_{#6} \\
 {#2}_{#4} & {#2}_{#5} & {#2}_{#6} \\
 {#3}_{#4} & {#3}_{#5} & {#3}_{#6}
\end{vmatrix}
}

\newcommand{\DETuvwxijkl}[8]{
\begin{vmatrix}
 {#1}_{#5} & {#1}_{#6} & {#1}_{#7} & {#1}_{#8} \\
 {#2}_{#5} & {#2}_{#6} & {#2}_{#7} & {#2}_{#8} \\
 {#3}_{#5} & {#3}_{#6} & {#3}_{#7} & {#3}_{#8} \\
 {#4}_{#5} & {#4}_{#6} & {#4}_{#7} & {#4}_{#8} \\
\end{vmatrix}
}

%\newcommand{\DETuvwxyijklm}[10]{
%\begin{vmatrix}
% {#1}_{#6} & {#1}_{#7} & {#1}_{#8} & {#1}_{#9} & {#1}_{#10} \\
% {#2}_{#6} & {#2}_{#7} & {#2}_{#8} & {#2}_{#9} & {#2}_{#10} \\
% {#3}_{#6} & {#3}_{#7} & {#3}_{#8} & {#3}_{#9} & {#3}_{#10} \\
% {#4}_{#6} & {#4}_{#7} & {#4}_{#8} & {#4}_{#9} & {#4}_{#10} \\
% {#5}_{#6} & {#5}_{#7} & {#5}_{#8} & {#5}_{#9} & {#5}_{#10}
%\end{vmatrix}
%}

% R3 vector.
\newcommand{\VectorThree}[3]{
\begin{bmatrix}
 {#1} \\
 {#2} \\
 {#3}
\end{bmatrix}
}



\author{Peeter Joot}
\email{peeter.joot@gmail.com}


\chapter{Energy Momentum Tensor.}
\label{chap:relativisticElectrodynamicsL22}
%\useCCL
\blogpage{http://sites.google.com/site/peeterjoot/math2011/relativisticElectrodynamicsL22.pdf}
\date{Mar 23, 2011}
\revisionInfo{relativisticElectrodynamicsL22.tex}

%\beginArtWithToc
\beginArtNoToc

\section{Reading.}

Covering \S 32, \S 33 of chapter 4 in the text \citep{landau1980classical}.

Covering \href{http://www.physics.utoronto.ca/~poppitz/epoppitz/PHY450_files/RelEMpp166-180.pdf}{lecture notes pp. 169-172:} spacetime translation invariance of the EM field action and the conservation of the energy-momentum tensor (170-172); properties of the energy-momentum tensor (172.1); the meaning of its components: energy.

\section{Disclaimer.}

I have no class notes for this lecture, as traffic conspired against me and I missed all but the last 5 minutes (a very frustrating drive!)  Here's my own walk through of the content that we must have covered, much of which I did as part of problem set 6 preparation.

\section{Total derivative of the Lagrangian density.}

Rather cleverly, our Professor avoided the spacetime translation arguments of the text.  Inspired by an approach possible in classical mechanics to find that we have a conserved quantity derivable from a force law, he proceeds directly to taking the derivative of the Lagrangian density (see previous lecture notes for details building up to this).

I'll proceed in exactly the same fashion.

\begin{align*}
\partial_k \left( -\inv{16 \pi c} F_{i j} F^{i j} \right) 
&= -\inv{8 \pi c} (\partial_k F_{i j} )F^{i j} \\
&= -\inv{8 \pi c} (\partial_k F_{i j} )F^{i j} \\
&= -\inv{8 \pi c} (\partial_k (\partial_i A_j - \partial_j A_i) )F^{i j} \\
&= -\inv{4 \pi c} (\partial_k \partial_i A_j )F^{i j} \\
&= -\inv{4 \pi c} (\partial_i \partial_k A_j )F^{i j} \\
&= -\inv{4 \pi c} (\partial_m \partial_k A_j )F^{m j} \\
&= -\inv{4 \pi c} \left( \partial_m ((\partial_k A_j )F^{m j}) - (\partial_m F^{m j}) \partial_k A_j \right) \\
&= -\inv{4 \pi c} \left( \partial_m ((\partial_k A_j )F^{m j}) - (\partial_m F^{m a}) \partial_k A_a \right) \\
&= -\inv{4 \pi c} \left( \partial_m ((\partial_k A_j )F^{m j}) - \left(\frac{4 \pi}{c} j^a \right) \partial_k A_a \right) \\
&= -\inv{4 \pi c} \partial_m ((\partial_k A_j )F^{m j}) + \left(\frac{1}{c^2} j^a \right) \partial_k A_a 
\end{align*}

Multiplying through by $c$ and renaming our derivative index using a delta function we have

\begin{equation}\label{eqn:relativisticElectrodynamicsL22:10}
\partial_k \left( -\inv{16 \pi } F_{i j} F^{i j} \right) =
\partial_m {\delta^{m}}_k \left( -\inv{16 \pi } F_{i j} F^{i j} \right) 
= -\inv{4 \pi } \partial_m ((\partial_k A_j )F^{m j}) + \left(\frac{1}{c} j^a \right) \partial_k A_a 
\end{equation}

We can now group the $\partial_m$ terms for

\begin{equation}\label{eqn:relativisticElectrodynamicsL22:30}
\partial_m \left(
-\inv{4 \pi } (\partial_k A_j )F^{m j}
+ {\delta^{m}}_k \inv{16 \pi } F_{i j} F^{i j} 
\right)
= 
- \left(\frac{1}{c} j^a \right) \partial_k A_a 
\end{equation}

Knowing the end goal, a quantity that is expressed in terms of $F^{ij}$ let's raise the $k$ indexes, and any of the $A_i$'s that are along side of those

\begin{equation}\label{eqn:relativisticElectrodynamicsL22:50}
\partial_m \left(
-\inv{4 \pi } (\partial^k A^n )F^{m j} g_{n j}
+ g^{m k} \inv{16 \pi } F_{i j} F^{i j} 
\right)
= 
- \left(\frac{1}{c} j_a \right) \partial^k A^a.
\end{equation}

Next, we want to get rid of the explicit vector potential dependence

\begin{align*}
\partial_m \left( -\inv{4 \pi } (\partial^k A^n )F^{m j} g_{n j} \right)
&=
\partial_m \left( -\inv{4 \pi } (F^{k n} + \partial^n A^k )F^{m j} g_{n j} \right) \\
&=
\partial_m \left( -\inv{4 \pi } F^{k n} F^{m j} g_{n j} 
- \inv{4 \pi} (\partial_m (\partial^n A^k )) F^{m j} g_{n j} 
- \inv{4 \pi} (\partial^n A^k ) (\partial_m F^{m j}) g_{n j} \right) \\
&=
\partial_m \left( -\inv{4 \pi } F^{k n} F^{m j} g_{n j} \right)
- \inv{4 \pi} (\partial_m (\partial^n A^k )) F^{m j} g_{n j} 
- (\partial^n A^k ) \inv{c} j_n \\
&=
\partial_m \left( -\inv{4 \pi } F^{k n} F^{m j} g_{n j} \right)
- \inv{4 \pi} (\partial_m \partial_j A^k ) F^{m j} 
- (\partial^a A^k ) \inv{c} j_a \\
\end{align*}

Since the operator $F^{m j} \partial_m \partial_j$ is a product of symmetric and antisymmetric tensors (or operators), the middle term is zero, and we are left with

\begin{equation}\label{eqn:relativisticElectrodynamicsL22:70}
\partial_m \left(
-\inv{4 \pi } F^{k n} F^{m j} g_{n j} 
+ g^{m k} \inv{16 \pi } F_{i j} F^{i j} 
\right)
= 
- \frac{1}{c} F^{k a} j_a
\end{equation}

This provides the desired conservation relationship

\begin{equation}\label{eqn:relativisticElectrodynamicsL22:90}
\boxed{
\begin{aligned}
\partial_m T^{m k} &= - \inv{c} F^{k a} j_a \\
T^{m k} &=
\inv{4 \pi } \left(
-
F^{m j} 
F^{k n} 
g_{n j} 
+ \frac{g^{m k}}{4} F_{i j} F^{i j} \right)
\end{aligned}
}
\end{equation}

\section{Unpacking the tensor}

\subsection{Energy term of the stress energy tensor.}

\begin{align*}
T^{ 0 0 } 
&=
-\inv{4 \pi} F^{ 0 j} {F^0}_j + \inv{16 \pi} F^{i j} F_{i j} \\
&=
-\inv{4 \pi} F^{ 0 \alpha} {F^0}_\alpha + \inv{16 \pi} \left(
F^{0 j} F_{0 j} 
+F^{\alpha j} F_{\alpha j} 
\right)
\\
&=
\inv{4 \pi} F^{ 0 \alpha} F^{0 \alpha} + \inv{16 \pi} 
\left(
F^{0 \alpha} F_{0 \alpha} 
+F^{\alpha 0} F_{\alpha 0} 
+F^{\alpha \beta} F_{\alpha \beta} 
\right)
\\
&=
\inv{4 \pi} \BE^2 + \inv{16 \pi} \left(
-2 \BE^2 +F^{\alpha \beta} F^{\alpha \beta} 
\right)
\end{align*}

The spatially indexed field tensor components are
\begin{align*}
F^{\alpha \beta} 
&= \partial^\alpha A^\beta - \partial^\beta A^\alpha \\
&= -\partial_\alpha A^\beta + \partial_\beta A^\alpha \\
&= -\epsilon^{\sigma \alpha \beta} (\BB)^\sigma,
\end{align*}

so

\begin{align*}
F^{\alpha \beta} F^{\alpha \beta} 
&= 
\epsilon^{\sigma \alpha \beta} (\BB)^\sigma
\epsilon^{\mu \alpha \beta} (\BB)^\mu \\
&= (2!) \delta^{\sigma \mu} 
(\BB)^\sigma
(\BB)^\mu \\
&= 2 \BB^2
\end{align*}

A final bit of assembly gives us $T^{0 0}$

\begin{equation}\label{eqn:relativisticElectrodynamicsPS6:200}
\boxed{
T^{ 0 0 } = \inv{8 \pi} ( \BE^2 + \BB^2 ) = \mathcal{E}
}
\end{equation}

\subsection{Momentum terms of the stress energy tensor.}

For the spatial $T^{k 0}$ components we have

\begin{align*}
T^{\alpha 0} 
&= 
-\inv{4 \pi} F^{\alpha j} {F^0}_j + \inv{16 \pi} g^{\alpha 0} F^{i j} F_{i j} \\
&= 
-\inv{4 \pi} F^{\alpha j} {F^0}_j \\
&= 
-\inv{4 \pi} 
\left( 
F^{\alpha 0} {F^0}_0 
+F^{\alpha \beta} {F^0}_\beta 
\right) \\
&= 
\inv{4 \pi} F^{\alpha \beta} F^{0 \beta} \\
&= 
\inv{4 \pi} (-\epsilon^{\sigma \alpha \beta} (\BB)^\sigma) (-(\BE)^\beta) \\
&= 
\inv{4 \pi} \epsilon^{\alpha \beta \sigma} 
(\BE)^\beta 
(\BB)^\sigma
\\
\end{align*}

So we have

\begin{equation}\label{eqn:relativisticElectrodynamicsPS6:220}
\boxed{
T^{\alpha 0} = \inv{4 \pi} (\BE \cross \BB)^\alpha = \frac{\BS^\alpha}{c}.
}
\end{equation}

\subsection{Symmetry}

It is simple to show that $T^{k m}$ is symmetric

\begin{align*}
T^{m k} 
&= -\inv{4\pi} F^{m j} {F^{k}}_j + \inv{16 \pi} g^{m k} F^{i j} F_{i j} \\
&= -\inv{4\pi} {F^{m}}_j F^{k j} + \inv{16 \pi} g^{k m} F^{i j} F_{i j} \\
&= T^{k m}
\end{align*}

\subsection{Pressure and shear terms.}

Let's now expand $T^{\beta \alpha}$, starting with the diagonal terms $T^{\alpha\alpha}$.  Because this repeated index isn't summed over, things get slightly irregular, so it's easier to drop the abstraction and just pick a specific $\alpha$, say, $\alpha = 1$.  Then we have

\begin{align*}
T^{1 1} 
&= \inv{4 \pi} \left( - F^{1 k} {F^1}_k - \inv{2} (\BB^2 - \BE^2) \right) \\
&= \inv{4 \pi} \left( - F^{1 0} F^{1 0} + F^{1 \alpha} F^{1 \alpha} - \inv{2} (\BB^2 - \BE^2) \right) \\
&= \inv{4 \pi} \left( 
- E_x^2
+ F^{1 2} F^{1 2}
+ F^{1 3} F^{1 3}
 - \inv{2} (\BB^2 - \BE^2) \right) \\
\end{align*}

For the magnetic components above we have for example

\begin{align*}
F^{1 2} F^{1 2} 
&=
(\partial^1 A^2 - \partial^2 A^2) (\partial^1 A^2 - \partial^2 A^2) \\
&=
(\partial_1 A^2 - \partial_2 A^2) (\partial_1 A^2 - \partial_2 A^2) \\
&=
B_z^2
\end{align*}

So we have

\begin{equation}\label{eqn:relativisticElectrodynamicsPS6:240}
T^{1 1} 
= \inv{4 \pi} \left( 
- E_x^2 + B_y^2 + B_z^2
- \inv{2} (\BB^2 - \BE^2) \right)
\end{equation}

Or

\begin{equation}\label{eqn:relativisticElectrodynamicsPS6:260}
T^{1 1} 
= \inv{8 \pi} \left( 
- E_x^2 + E_y^2 + E_z^2
- B_x^2 + B_y^2 + B_z^2
\right).
\end{equation}

Clearly, the other diagonal terms follow the same pattern, and we can do a cyclic permutation of coordinates to find

\begin{align}\label{eqn:relativisticElectrodynamicsPS6:280}
T^{1 1} &= \inv{8 \pi} \left( - E_x^2 + E_y^2 + E_z^2 - B_x^2 + B_y^2 + B_z^2 \right) \\
T^{2 2} &= \inv{8 \pi} \left( - E_y^2 + E_z^2 + E_x^2 - B_y^2 + B_z^2 + B_x^2 \right) \\
T^{3 3} &= \inv{8 \pi} \left( - E_z^2 + E_x^2 + E_y^2 - B_z^2 + B_x^2 + B_y^2 \right) 
\end{align}

For the off diagonal terms, let's pick $T^{1 2}$ and expand that.  We have

\begin{align*}
T^{1 2} 
&= \inv{4 \pi} \left( - F^{1 k} {F^2}_k - \inv{2} g^{1 2}(\BB^2 - \BE^2) \right) \\
&= \inv{4 \pi} \left( - F^{1 0} F^{2 0} + F^{1 \alpha} F^{2 \alpha} \right) \\
&= \inv{4 \pi} \left( - E_x E_y 
+ \underbrace{F^{1 1}}_{=0} F^{2 1} 
+ F^{1 2} \underbrace{F^{2 2}}_{=0}
+ F^{1 3} F^{2 3} 
\right) \\
&= \inv{4 \pi} \left( - E_x E_y + (-B_y) B_x \right) \\
\end{align*}

Again, with cyclic permutation of the coordinates we have

\begin{align}\label{eqn:relativisticElectrodynamicsPS6:300}
T^{1 2} &= -\inv{4 \pi} \left( E_x E_y + B_x B_y \right) \\
T^{2 3} &= -\inv{4 \pi} \left( E_y E_z + B_y B_z \right) \\
T^{3 1} &= -\inv{4 \pi} \left( E_z E_x + B_z B_x \right) 
\end{align}

In class these were all written in the compact notation

\begin{equation}\label{eqn:relativisticElectrodynamicsPS6:320}
\boxed{
T^{\alpha \beta} = -\inv{4 \pi} \left( 
E_\alpha E_\beta 
+B_\alpha B_\beta 
- \inv{2} \delta_{\alpha\beta} (\BE^2 + \BB^2) \right)
}
\end{equation}

\EndArticle
