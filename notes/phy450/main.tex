\documentclass[12pt,leqno]{book}

\usepackage{amsmath,amssymb,amsfonts} % Typical maths resource packages
\usepackage{graphicx}
\usepackage{color}                   % For creating coloured text and background
\usepackage{txfonts} 
\usepackage{listings}
\usepackage[bookmarks=true,plainpages=false]{hyperref}

\parindent 1cm
\parskip 0.2cm
\topmargin 0.2cm
\oddsidemargin 1cm
\evensidemargin 0.5cm
\textwidth 15cm
\textheight 21cm

%% how do these ones work?
%\newtheorem{theorem}{Theorem}[section]
%\newtheorem{proposition}[theorem]{Proposition}
%\newtheorem{corollary}[theorem]{Corollary}
%\newtheorem{lemma}[theorem]{Lemma}
%\newtheorem{remark}[theorem]{Remark}
%\newtheorem{definition}[theorem]{Definition}

\usepackage{amsmath}
\usepackage{mathpazo}

%
% shorthand for bold symbols, convenient for vectors and matrices
%
\newcommand{\Ba}[0]{\mathbf{a}}
\newcommand{\Bb}[0]{\mathbf{b}}
\newcommand{\Bc}[0]{\mathbf{c}}
\newcommand{\Bd}[0]{\mathbf{d}}
\newcommand{\Be}[0]{\mathbf{e}}
\newcommand{\Bf}[0]{\mathbf{f}}
\newcommand{\Bg}[0]{\mathbf{g}}
\newcommand{\Bh}[0]{\mathbf{h}}
\newcommand{\Bi}[0]{\mathbf{i}}
\newcommand{\Bj}[0]{\mathbf{j}}
\newcommand{\Bk}[0]{\mathbf{k}}
\newcommand{\Bl}[0]{\mathbf{l}}
\newcommand{\Bm}[0]{\mathbf{m}}
\newcommand{\Bn}[0]{\mathbf{n}}
\newcommand{\Bo}[0]{\mathbf{o}}
\newcommand{\Bp}[0]{\mathbf{p}}
\newcommand{\Bq}[0]{\mathbf{q}}
\newcommand{\Br}[0]{\mathbf{r}}
\newcommand{\Bs}[0]{\mathbf{s}}
\newcommand{\Bt}[0]{\mathbf{t}}
\newcommand{\Bu}[0]{\mathbf{u}}
\newcommand{\Bv}[0]{\mathbf{v}}
\newcommand{\Bw}[0]{\mathbf{w}}
\newcommand{\Bx}[0]{\mathbf{x}}
\newcommand{\By}[0]{\mathbf{y}}
\newcommand{\Bz}[0]{\mathbf{z}}
\newcommand{\BA}[0]{\mathbf{A}}
\newcommand{\BB}[0]{\mathbf{B}}
\newcommand{\BC}[0]{\mathbf{C}}
\newcommand{\BD}[0]{\mathbf{D}}
\newcommand{\BE}[0]{\mathbf{E}}
\newcommand{\BF}[0]{\mathbf{F}}
\newcommand{\BG}[0]{\mathbf{G}}
\newcommand{\BH}[0]{\mathbf{H}}
\newcommand{\BI}[0]{\mathbf{I}}
\newcommand{\BJ}[0]{\mathbf{J}}
\newcommand{\BK}[0]{\mathbf{K}}
\newcommand{\BL}[0]{\mathbf{L}}
\newcommand{\BM}[0]{\mathbf{M}}
\newcommand{\BN}[0]{\mathbf{N}}
\newcommand{\BO}[0]{\mathbf{O}}
\newcommand{\BP}[0]{\mathbf{P}}
\newcommand{\BQ}[0]{\mathbf{Q}}
\newcommand{\BR}[0]{\mathbf{R}}
\newcommand{\BS}[0]{\mathbf{S}}
\newcommand{\BT}[0]{\mathbf{T}}
\newcommand{\BU}[0]{\mathbf{U}}
\newcommand{\BV}[0]{\mathbf{V}}
\newcommand{\BW}[0]{\mathbf{W}}
\newcommand{\BX}[0]{\mathbf{X}}
\newcommand{\BY}[0]{\mathbf{Y}}
\newcommand{\BZ}[0]{\mathbf{Z}}

\newcommand{\Bzero}[0]{\mathbf{0}}
\newcommand{\Btheta}[0]{\boldsymbol{\theta}}
\newcommand{\Btau}[0]{\boldsymbol{\tau}}
\newcommand{\Bomega}[0]{\boldsymbol{\omega}}

%
% shorthand for unit vectors
%
\newcommand{\acap}[0]{\hat{\Ba}}
\newcommand{\bcap}[0]{\hat{\Bb}}
\newcommand{\ccap}[0]{\hat{\Bc}}
\newcommand{\dcap}[0]{\hat{\Bd}}
\newcommand{\ecap}[0]{\hat{\Be}}
\newcommand{\fcap}[0]{\hat{\Bf}}
\newcommand{\gcap}[0]{\hat{\Bg}}
\newcommand{\hcap}[0]{\hat{\Bh}}
\newcommand{\icap}[0]{\hat{\Bi}}
\newcommand{\jcap}[0]{\hat{\Bj}}
\newcommand{\kcap}[0]{\hat{\Bk}}
\newcommand{\lcap}[0]{\hat{\Bl}}
\newcommand{\mcap}[0]{\hat{\Bm}}
\newcommand{\ncap}[0]{\hat{\Bn}}
\newcommand{\ocap}[0]{\hat{\Bo}}
\newcommand{\pcap}[0]{\hat{\Bp}}
\newcommand{\qcap}[0]{\hat{\Bq}}
\newcommand{\rcap}[0]{\hat{\Br}}
\newcommand{\scap}[0]{\hat{\Bs}}
\newcommand{\tcap}[0]{\hat{\Bt}}
\newcommand{\ucap}[0]{\hat{\Bu}}
\newcommand{\vcap}[0]{\hat{\Bv}}
\newcommand{\wcap}[0]{\hat{\Bw}}
\newcommand{\xcap}[0]{\hat{\Bx}}
\newcommand{\ycap}[0]{\hat{\By}}
\newcommand{\zcap}[0]{\hat{\Bz}}
\newcommand{\thetacap}[0]{\hat{\Btheta}}

%
% to write R^n and C^n in a distinguishable fashion.  Perhaps change this
% to the double lined characters upon figuring out how to do so.
%
\newcommand{\C}[1]{$\mathbb{C}^{#1}$}
\newcommand{\R}[1]{$\mathbb{R}^{#1}$}

%
% various generally useful helpers
%

% derivative of #1 wrt. #2:
\newcommand{\D}[2] {\frac {d#2} {d#1}}

\newcommand{\inv}[1]{\frac{1}{#1}}
\newcommand{\cross}[0]{\times}

\newcommand{\abs}[1]{\lvert{#1}\rvert}
\newcommand{\norm}[1]{\lVert{#1}\rVert}
\newcommand{\innerprod}[2]{\langle{#1}, {#2}\rangle}
\newcommand{\dotprod}[2]{{#1} \cdot {#2}}
\newcommand{\bdotprod}[2]{\left({#1} \cdot {#2}\right)}
\newcommand{\crossprod}[2]{{#1} \cross {#2}}
\newcommand{\tripleprod}[3]{\dotprod{\left(\crossprod{#1}{#2}\right)}{#3}}

\DeclareMathOperator{\Proj}{Proj}
\DeclareMathOperator{\Span}{span}
\DeclareMathOperator{\Sgn}{sgn}
\DeclareMathOperator{\Area}{Area}
\DeclareMathOperator{\Volume}{Volume}

%
% A few miscellaneous things specific to this document
%
\newcommand{\crossop}[1]{\crossprod{#1}{}}

% R2 vector.
\newcommand{\VectorTwo}[2]{
\begin{bmatrix}
 {#1} \\
 {#2}
\end{bmatrix}
}

\newcommand{\VectorN}[1]{
\begin{bmatrix}
{#1}_1 \\
{#1}_2 \\
\vdots \\
{#1}_N \\
\end{bmatrix}
}

\newcommand{\DETuvij}[4]{
\begin{vmatrix}
 {#1}_{#3} & {#1}_{#4} \\
 {#2}_{#3} & {#2}_{#4}
\end{vmatrix}
}

\newcommand{\DETuvwijk}[6]{
\begin{vmatrix}
 {#1}_{#4} & {#1}_{#5} & {#1}_{#6} \\
 {#2}_{#4} & {#2}_{#5} & {#2}_{#6} \\
 {#3}_{#4} & {#3}_{#5} & {#3}_{#6}
\end{vmatrix}
}

\newcommand{\DETuvwxijkl}[8]{
\begin{vmatrix}
 {#1}_{#5} & {#1}_{#6} & {#1}_{#7} & {#1}_{#8} \\
 {#2}_{#5} & {#2}_{#6} & {#2}_{#7} & {#2}_{#8} \\
 {#3}_{#5} & {#3}_{#6} & {#3}_{#7} & {#3}_{#8} \\
 {#4}_{#5} & {#4}_{#6} & {#4}_{#7} & {#4}_{#8} \\
\end{vmatrix}
}

%\newcommand{\DETuvwxyijklm}[10]{
%\begin{vmatrix}
% {#1}_{#6} & {#1}_{#7} & {#1}_{#8} & {#1}_{#9} & {#1}_{#10} \\
% {#2}_{#6} & {#2}_{#7} & {#2}_{#8} & {#2}_{#9} & {#2}_{#10} \\
% {#3}_{#6} & {#3}_{#7} & {#3}_{#8} & {#3}_{#9} & {#3}_{#10} \\
% {#4}_{#6} & {#4}_{#7} & {#4}_{#8} & {#4}_{#9} & {#4}_{#10} \\
% {#5}_{#6} & {#5}_{#7} & {#5}_{#8} & {#5}_{#9} & {#5}_{#10}
%\end{vmatrix}
%}

% R3 vector.
\newcommand{\VectorThree}[3]{
\begin{bmatrix}
 {#1} \\
 {#2} \\
 {#3}
\end{bmatrix}
}



%\newcommand{\Dslash}[0]{{\not{}}D}
\newcommand{\Dslash}[0]{D\!\!\!/}
\newcommand{\chapcite}[1]{\ref{chap:#1}}

%-----------------------------------------
%
% stubs for article class.
%
\newcommand{\blogpage}[1]{}
\newcommand{\email}[1]{}
\newcommand{\beginArtWithToc}[0]{}
\newcommand{\beginArtNoToc}[0]{}
\newcommand{\EndArticle}[0]{}
\newcommand{\EndNoBibArticle}[0]{}
\newcommand{\revisionInfo}[1]{}
%-----------------------------------------
\DeclareMathOperator{\Atan}{atan}

%%
% Copyright � 2012 Peeter Joot.  All Rights Reserved.
% Licenced as described in the file LICENSE under the root directory of this GIT repository.
%

% 
% 
\DeclareMathOperator{\Div}{div}
\DeclareMathOperator{\Mod}{mod}
\DeclareMathOperator{\PV}{PV}
\DeclareMathOperator{\Prob}{Prob}
\DeclareMathOperator{\rank}{rank}
\DeclareMathOperator{\sgn}{sgn}
\DeclareMathOperator{\sinc}{sinc}
%\DeclareMathOperator{\Atan2}{atan2}
\DeclareMathOperator{\atan}{atan}


\newcommand{\expectation}[1]{\langle{#1}\rangle}
%\newcommand{\gpgradefour}[1] {\gpgrade{#1}{4}}
%\newcommand{\gpgradeone}[1] {\gpgrade{#1}{1}}
%\newcommand{\gpgradethree}[1] {\gpgrade{#1}{3}}
%\newcommand{\gpgradetwo}[1] {\gpgrade{#1}{2}}
%\newcommand{\gpgradezero}[1] {\gpgrade{#1}{}}
%\newcommand{\gpgrade}[2] {{\left\langle{{#1}}\right\rangle}_{#2}}
%\newcommand{\grad}[0]{\boldsymbol{\nabla}}
%\newcommand{\grad}[0]{\nabla}


\newcommand{\ketbra}[2]{\ket{#1}\bra{#2}}
\newcommand{\ket}[1]{\lvert {#1} \rangle}
%\newcommand{\norm}[1]{\lVert#1\rVert}
\newcommand{\questionEquals}[0]{\stackrel{?}{=}}
\newcommand{\rightshift}[0]{\gg}
%\newcommand{\spacegrad}[0]{\boldsymbol{\nabla}}
\newcommand{\symmetric}[2]{{\left\{{#1},{#2}\right\}}}
\newcommand{\antisymmetric}[2]{\left[{#1},{#2}\right]}

%\newcommand{\Abs}[1]{\left\lvert{#1}\right\rvert}

%\newcommand{\BB}[0]{\mathbf{B}}
%\newcommand{\BE}[0]{\mathbf{E}}
%\newcommand{\BF}[0]{\mathbf{F}}
%\newcommand{\BS}[0]{\mathbf{S}}
%\newcommand{\BV}[0]{\mathbf{V}}
%\newcommand{\Bj}[0]{\mathbf{j}}

\newcommand{\BraOpKet}[3]{\bra{#1} \hat{#2} \ket{#3} }
%\newcommand{\Brho}[0]{\boldsymbol{\rho}}
\newcommand{\CC}[0]{c^2}
\newcommand{\Cos}[1]{\cos{\left({#1}\right)}}

% not working anymore.  think it's a conflicting macro for \not.
% compared to original usage in klien_gordon.ltx
%
%\newcommand{\Dslash}[0]{{\not}D}
%\newcommand{\Dslash}[0]{{\not{}}D}
% switched to cancel in macros.tex
%\newcommand{\Dslash}[0]{D\!\!\!/}

\newcommand{\Expectation}[1]{\left\langle {#1} \right\rangle}
\newcommand{\Exp}[1]{\exp{\left({#1}\right)}}
\newcommand{\FF}[0]{\mathcal{F}}
\newcommand{\FM}[0]{\inv{\sqrt{2\pi\hbar}}}
\newcommand{\IIinf}[0]{ \int_{-\infty}^\infty }
\newcommand{\Innerprod}[2]{\left\langle{#1}, {#2}\right\rangle}
%\newcommand{\LL}[0]{\mathcal{L}}

%\newcommand{\PD}[2] {\frac {\partial #2} {\partial #1}}

% backwards from ../peeterj_macros2:
\newcommand{\PDb}[2]{ \frac{\partial{#1}}{\partial {#2}} }

%\newcommand{\PDD}[3]{\frac{\partial^2 {#3}}{\partial {#1}\partial {#2}}}
\newcommand{\PDN}[3]{\frac{\partial^{#3} {#2}}{\partial {#1}^{#3}}}

\newcommand{\PDSq}[2]{\frac{\partial^2 {#2}}{\partial {#1}^2}}
\newcommand{\PDsQ}[2]{\frac{\partial^2 {#2}}{\partial^2 {#1}}}

\newcommand{\Sch}[0]{{Schr\"{o}dinger} }
\newcommand{\Sin}[1]{\sin{\left({#1}\right)}}
\newcommand{\Sw}[0]{\mathcal{S}}
%\newcommand{\T}[0]{\text{T}}
\newcommand{\T}[0]{{\text{T}}}

\newcommand{\braket}[2]{\langle{#1} \vert {#2}\rangle}
\newcommand{\bra}[1]{\langle {#1} \rvert}
\newcommand{\curl}[0]{\grad \times}
\newcommand{\delambert}[0]{\sum_{\alpha = 1}^4{\PDSq{x_\alpha}{}}}
\newcommand{\delsquared}[0]{\nabla^2}
\newcommand{\diverg}[0]{\grad \cdot}

\newcommand{\halfPhi}[0]{\frac{\phi}{2}}
\newcommand{\hatH}[0]{\hat{H}}
\newcommand{\hatS}[0]{\hat{S}}
\newcommand{\hatk}[0]{\hat{k}}
\newcommand{\hatp}[0]{\hat{p}}
\newcommand{\hatx}[0]{\hat{x}}


\newcommand{\Rdot}[0]{\dot{R}}
%\newcommand{\addot}[0]{\ddot{a}}
%\newcommand{\adot}[0]{\dot{a}}
%\newcommand{\fddot}[0]{\ddot{f}}
%\newcommand{\fdot}[0]{\dot{f}}
%\newcommand{\bddot}[0]{\ddot{b}}
%\newcommand{\bdot}[0]{\dot{b}}
\newcommand{\ddotOmega}[0]{\ddot{\Omega}}
\newcommand{\ddotalpha}[0]{\ddot{\alpha}}
\newcommand{\ddotomega}[0]{\ddot{\omega}}
\newcommand{\ddotphi}[0]{\ddot{\phi}}
\newcommand{\ddotpsi}[0]{\ddot{\psi}}
\newcommand{\ddottheta}[0]{\ddot{\theta}}
\newcommand{\dotOmega}[0]{\dot{\Omega}}
\newcommand{\dotalpha}[0]{\dot{\alpha}}
\newcommand{\dotomega}[0]{\dot{\omega}}
\newcommand{\dotphi}[0]{\dot{\phi}}
\newcommand{\dotpsi}[0]{\dot{\psi}}
\newcommand{\dottheta}[0]{\dot{\theta}}
%\newcommand{\pddot}[0]{\ddot{p}}
%\newcommand{\pdot}[0]{\dot{p}}
%\newcommand{\qddot}[0]{\ddot{q}}
%\newcommand{\qdot}[0]{\dot{q}}
%\newcommand{\rddot}[0]{\ddot{r}}
%\newcommand{\rdot}[0]{\dot{r}}
%\newcommand{\tddot}[0]{\ddot{t}}
%\newcommand{\tdot}[0]{\dot{t}}
%\newcommand{\uddot}[0]{\ddot{u}}
%\newcommand{\udot}[0]{\dot{u}}
%\newcommand{\xddot}[0]{\ddot{x}}
%\newcommand{\xdot}[0]{\dot{x}}
%\newcommand{\yddot}[0]{\ddot{y}}
%\newcommand{\ydot}[0]{\dot{y}}
%\newcommand{\zddot}[0]{\ddot{z}}
%\newcommand{\zdot}[0]{\dot{z}}








%-------------------------------------------------------------------
% ORIGINS:
%
% bohm11.tex

%\DeclareMathOperator{\sgn}{sgn}
%\newcommand{\PDSq}[2]{\frac{\partial^2 {#2}}{\partial {#1}^2}}
%\newcommand{\PDN}[3]{\frac{\partial^{#3} {#2}}{\partial {#1}^{#3}}}
%\DeclareMathOperator{\sinc}{sinc}
%\DeclareMathOperator{\PV}{PV}
%\newcommand{\FF}[0]{\mathcal{F}}
%\newcommand{\Sw}[0]{\mathcal{S}}
%\newcommand{\IIinf}[0]{ \int_{-\infty}^\infty }
%\newcommand{\FM}[0]{\inv{\sqrt{2\pi\hbar}}}
%\newcommand{\expectation}[1]{\langle{#1}\rangle}
%
%

% bohm_ch10.tex

%\DeclareMathOperator{\sgn}{sgn}
%\newcommand{\expectation}[1]{\langle{#1}\rangle}
%\newcommand{\IIinf}[0]{ \int_{-\infty}^\infty }
%\DeclareMathOperator{\PV}{PV}
%
%

% bohm_ch9.tex

%\newcommand{\PDSq}[2]{\frac{\partial^2 {#2}}{\partial {#1}^2}}
%\newcommand{\PDN}[3]{\frac{\partial^{#3} {#2}}{\partial {#1}^{#3}}}
%\DeclareMathOperator{\sinc}{sinc}
%\DeclareMathOperator{\PV}{PV}
%\newcommand{\FF}[0]{\mathcal{F}}
%\newcommand{\Sw}[0]{\mathcal{S}}
%\newcommand{\IIinf}[0]{ \int_{-\infty}^\infty }
%\newcommand{\FM}[0]{\inv{\sqrt{2\pi\hbar}}}
%\newcommand{\expectation}[1]{\langle{#1}\rangle}
%
%

% commutator_herm.tex

%\newcommand{\symmetric}[2]{{\left\{{#1},{#2}\right\}}}
%\newcommand{\antisymmetric}[2]{\left[{#1},{#2}\right]}
%
%%\newcommand{\ket}[1]{\lvert {#1} \rangle}
%%\newcommand{\bra}[1]{\langle {#1} \rvert}
%%\newcommand{\braket}[2]{\langle{#1} \vert {#2}\rangle}
%%\newcommand{\ketbra}[2]{\ket{#1}\bra{#2}}
%%\newcommand{\BraOpKet}[3]{\bra{#1} \hat{#2} \ket{#3} }
%%\newcommand{\Innerprod}[2]{\left\langle{#1}, {#2}\right\rangle}
%\newcommand{\Expectation}[1]{\left\langle {#1} \right\rangle}
%
%

% delta_ortho_series.tex

%\newcommand{\IIinf}[0]{ \int_{-\infty}^\infty }
%\newcommand{\ket}[1]{\lvert {#1} \rangle}
%\newcommand{\bra}[1]{\langle {#1} \rvert}
%\newcommand{\braket}[2]{\langle{#1} \vert {#2}\rangle}
%\newcommand{\ketbra}[2]{\ket{#1}\bra{#2}}
%\newcommand{\BraOpKet}[3]{\bra{#1} \hat{#2} \ket{#3} }
%\newcommand{\Innerprod}[2]{\left\langle{#1}, {#2}\right\rangle}
%
%

% distributions.tex

%\newcommand{\PDSq}[2]{\frac{\partial^2 {#2}}{\partial {#1}^2}}
%\DeclareMathOperator{\sinc}{sinc}
%\DeclareMathOperator{\PV}{PV}
%\newcommand{\FF}[0]{\mathcal{F}}
%\newcommand{\Sw}[0]{\mathcal{S}}
%\newcommand{\IIinf}[0]{ \int_{-\infty}^\infty }
%
%

% ehrenfest.tex

%\newcommand{\PDSq}[2]{\frac{\partial^2 {#2}}{\partial {#1}^2}}
%
%

% fletcher.tex

%\DeclareMathOperator{\Div}{div}
%\DeclareMathOperator{\Mod}{mod}
%\newcommand{\rightshift}[0]{\gg}
%\newcommand{\questionEquals}[0]{\stackrel{?}{=}}
%
%

% fvec.tex

%\newcommand{\grad}[0]{\nabla}
%\newcommand{\PD}[2]{ \frac{\partial{#1}}{\partial {#2}} }
%
%

% gacs_q8_8.tex

%\newcommand{\halfPhi}[0]{\frac{\phi}{2}}
%\newcommand{\Sin}[1]{\sin{\left({#1}\right)}}
%\newcommand{\Cos}[1]{\cos{\left({#1}\right)}}
%\newcommand{\Exp}[1]{\exp{\left({#1}\right)}}
%
%

% goldstein_ch1_2.tex

%\newcommand{\spacegrad}[0]{\boldsymbol{\nabla}}
%\newcommand{\Brho}[0]{\boldsymbol{\rho}}
%\newcommand{\LL}[0]{\mathcal{L}}
%\newcommand{\Abs}[1]{\left\lvert{#1}\right\rvert}
%\newcommand{\qdot}[0]{\dot{q}}
%\newcommand{\qddot}[0]{\ddot{q}}
%\newcommand{\xdot}[0]{\dot{x}}
%\newcommand{\xddot}[0]{\ddot{x}}
%\newcommand{\ydot}[0]{\dot{y}}
%\newcommand{\yddot}[0]{\ddot{y}}
%\newcommand{\dotalpha}[0]{\dot{\alpha}}
%\newcommand{\ddotalpha}[0]{\ddot{\alpha}}
%\newcommand{\dottheta}[0]{\dot{\theta}}
%\newcommand{\ddottheta}[0]{\ddot{\theta}}
%\newcommand{\dotphi}[0]{\dot{\phi}}
%\newcommand{\ddotphi}[0]{\ddot{\phi}}
%% == \partial_{#1} {#2}
%\newcommand{\PD}[2]{\frac{\partial {#2}}{\partial {#1}}}
%\newcommand{\PDD}[3]{\frac{\partial^2 {#3}}{\partial {#1}\partial {#2}}}
%
%% <grade selection>
%%
%\newcommand{\gpgrade}[2] {{\left\langle{{#1}}\right\rangle}_{#2}}
%
%\newcommand{\gpgradezero}[1] {\gpgrade{#1}{}}
%%\newcommand{\gpscalargrade}[1] {{\left\langle{{#1}}\right\rangle}}
%%\newcommand{\gpgradezero}[1] {\gpgrade{#1}{0}}
%
%%\newcommand{\gpgradeone}[1] {{\left\langle{{#1}}\right\rangle}_{1}}
%\newcommand{\gpgradeone}[1] {\gpgrade{#1}{1}}
%
%\newcommand{\gpgradetwo}[1] {\gpgrade{#1}{2}}
%\newcommand{\gpgradethree}[1] {\gpgrade{#1}{3}}
%\newcommand{\gpgradefour}[1] {\gpgrade{#1}{4}}
%%
%% </grade selection>
%
%
%

% harmonic_osc.tex

%\newcommand{\IIinf}[0]{ \int_{-\infty}^\infty }
%
%

% klein_gordon.tex

%\newcommand{\PDSq}[2]{\frac{\partial^2 {#2}}{\partial {#1}^2}}
%%\newcommand{\Dslash}[0]{D\!\!\!/}
%\newcommand{\Dslash}[0]{{\not}D}
%
%

% matrix_to_operator.tex

%\newcommand{\T}[0]{{\text{T}}}
%
%

% maxwell.tex

%\newcommand{\norm}[1]{\lVert#1\rVert}
%\newcommand{\grad}[0]{\boldsymbol{\nabla}}
%\newcommand{\curl}[0]{\grad \times}
%\newcommand{\diverg}[0]{\grad \cdot}
%\newcommand{\delsquared}[0]{\nabla^2}
%\newcommand{\delambert}[0]{\sum_{\alpha = 1}^4{\PDSq{x_\alpha}{}}}
%
%% partial derivative of #1 wrt. #2:
%\newcommand{\PD}[2] {\frac {\partial #2} {\partial #1}}
%% second partial derivative of #1 wrt. #2:
%\newcommand{\PDSq}[2] {\frac {\partial^2 #2} {\partial {#1}^2}}
%
%%
%% shorthand for bold symbols:
%%
%\newcommand{\Bj}[0]{\mathbf{j}}
%\newcommand{\BB}[0]{\mathbf{B}}
%\newcommand{\BE}[0]{\mathbf{E}}
%\newcommand{\BF}[0]{\mathbf{F}}
%\newcommand{\BS}[0]{\mathbf{S}}
%\newcommand{\BV}[0]{\mathbf{V}}
%
%

% mp_inverse_svd_rough_notes.tex

%\newcommand{\T}[0]{\text{T}}
%\DeclareMathOperator{\rank}{rank}
%
%

% outermorphism_det.tex

%\newcommand{\gpgrade}[2] {{\left\langle{{#1}}\right\rangle}_{#2}}
%\newcommand{\gpgradeone}[1] {\gpgrade{#1}{1}}
%\newcommand{\gpgradetwo}[1] {\gpgrade{#1}{2}}
%
%

% pauli_qm_relativity_intro.tex

%\newcommand{\Sch}[0]{{Schr\"{o}dinger} }
%
%

% pe.tex

%
%\newcommand{\grad}[0]{\nabla}

% qm_susskind.tex

%\newcommand{\ket}[1]{\lvert {#1} \rangle}
%\newcommand{\bra}[1]{\langle {#1} \rvert}
%\newcommand{\braket}[2]{\langle{#1} \vert {#2}\rangle}
%\newcommand{\ketbra}[2]{\ket{#1}\bra{#2}}
%\newcommand{\BraOpKet}[3]{\bra{#1} \hat{#2} \ket{#3} }
%\newcommand{\hatH}[0]{\hat{H}}
%\newcommand{\hatS}[0]{\hat{S}}
%\newcommand{\hatk}[0]{\hat{k}}
%\newcommand{\hatx}[0]{\hat{x}}
%\newcommand{\hatp}[0]{\hat{p}}
%\DeclareMathOperator{\Prob}{Prob}
%
%

% schwartzchild_metric.tex

%\newcommand{\grad}[0]{\nabla}
%\newcommand{\Abs}[1]{\left\lvert{#1}\right\rvert}
%\newcommand{\spacegrad}[0]{\boldsymbol{\nabla}}
%\newcommand{\LL}[0]{\mathcal{L}}
%\newcommand{\PD}[2]{\frac{\partial {#2}}{\partial {#1}}}
%\newcommand{\PDsQ}[2]{\frac{\partial^2 {#2}}{\partial^2 {#1}}}
%\newcommand{\dotalpha}[0]{\dot{\alpha}}
%\newcommand{\ddotalpha}[0]{\ddot{\alpha}}
%
%\newcommand{\dotomega}[0]{\dot{\omega}}
%\newcommand{\ddotomega}[0]{\ddot{\omega}}
%
%\newcommand{\dotOmega}[0]{\dot{\Omega}}
%\newcommand{\ddotOmega}[0]{\ddot{\Omega}}
%
%\newcommand{\CC}[0]{c^2}
%
%\newcommand{\dottheta}[0]{\dot{\theta}}
%\newcommand{\ddottheta}[0]{\ddot{\theta}}
%
%\newcommand{\dotpsi}[0]{\dot{\psi}}
%\newcommand{\ddotpsi}[0]{\ddot{\psi}}
%
%\newcommand{\adot}[0]{\dot{a}}
%\newcommand{\addot}[0]{\ddot{a}}
%\newcommand{\udot}[0]{\dot{u}}
%\newcommand{\uddot}[0]{\ddot{u}}
%\newcommand{\fdot}[0]{\dot{f}}
%\newcommand{\fddot}[0]{\ddot{f}}
%\newcommand{\bdot}[0]{\dot{b}}
%\newcommand{\bddot}[0]{\ddot{b}}
%\newcommand{\qdot}[0]{\dot{q}}
%\newcommand{\qddot}[0]{\ddot{q}}
%\newcommand{\tdot}[0]{\dot{t}}
%\newcommand{\tddot}[0]{\ddot{t}}
%
%\newcommand{\Rdot}[0]{\dot{R}}
%
%\newcommand{\pdot}[0]{\dot{p}}
%\newcommand{\pddot}[0]{\ddot{p}}
%
%\newcommand{\xdot}[0]{\dot{x}}
%\newcommand{\xddot}[0]{\ddot{x}}
%
%\newcommand{\zdot}[0]{\dot{z}}
%\newcommand{\zddot}[0]{\ddot{z}}
%
%\newcommand{\rdot}[0]{\dot{r}}
%\newcommand{\rddot}[0]{\ddot{r}}
%
%

% shear.tex

%\newcommand{\gpgrade}[2] {{\left\langle{{#1}}\right\rangle}_{#2}}
%
%

% tong_mf1.tex

%\newcommand{\Abs}[1]{\left\lvert{#1}\right\rvert}
%\newcommand{\grad}[0]{\nabla}
%\newcommand{\LL}[0]{\mathcal{L}}
%
%\newcommand{\dotalpha}[0]{\dot{\alpha}}
%\newcommand{\ddotalpha}[0]{\ddot{\alpha}}
%
%\newcommand{\dotomega}[0]{\dot{\omega}}
%\newcommand{\ddotomega}[0]{\ddot{\omega}}
%
%\newcommand{\dottheta}[0]{\dot{\theta}}
%\newcommand{\ddottheta}[0]{\ddot{\theta}}
%
%\newcommand{\dotpsi}[0]{\dot{\psi}}
%\newcommand{\ddotpsi}[0]{\ddot{\psi}}
%
%\newcommand{\qdot}[0]{\dot{q}}
%\newcommand{\qddot}[0]{\ddot{q}}
%
%\newcommand{\Rdot}[0]{\dot{R}}
%
%\newcommand{\pdot}[0]{\dot{p}}
%\newcommand{\pddot}[0]{\ddot{p}}
%
%\newcommand{\xdot}[0]{\dot{x}}
%\newcommand{\xddot}[0]{\ddot{x}}
%
%\newcommand{\zdot}[0]{\dot{z}}
%\newcommand{\zddot}[0]{\ddot{z}}
%
%\newcommand{\rdot}[0]{\dot{r}}
%\newcommand{\rddot}[0]{\ddot{r}}
%
%% == \partial_{#1} {#2}
%\newcommand{\PD}[2]{\frac{\partial {#2}}{\partial {#1}}}
%\newcommand{\PDD}[3]{\frac{\partial^2 {#3}}{\partial {#1}\partial {#2}}}
%
%

% wavepacket.tex

%\newcommand{\PDSq}[2]{\frac{\partial^2 {#2}}{\partial {#1}^2}}
%\newcommand{\IIinf}[0]{ \int_{-\infty}^\infty }
%
%

% wavevariation.tex

%\newcommand{\PDSq}[2]{\frac{\partial^2 {#2}}{\partial {#1}^2}}
%
%

% qm_barrier
%\DeclareMathOperator{\Atan2}{atan2}
%\DeclareMathOperator{\atan}{atan}

% twobodies.tex
%\DeclareMathOperator{\sgn}{sgn}


% sr_lagrangian_q.tex

%\newcommand{\PD}[2]{\frac{\partial {#2}}{\partial {#1}}}
%\newcommand{\xdot}[0]{\dot{x}}
%\newcommand{\xddot}[0]{\ddot{x}}

% stub_em_fields.tex

%\newcommand{\EE}[0]{\boldsymbol{\mathcal{E}}}
%\newcommand{\HH}[0]{\boldsymbol{\mathcal{H}}}
%\newcommand{\PDSq}[2]{\frac{\partial^2 {#2}}{\partial {#1}^2}}

% long_wire_q.tex

%\newcommand{\grad}[0]{\nabla}

% lorentz_tx_em_potential.tex
%\newcommand{\LL}[0]{\mathcal{L}}
%\newcommand{\grad}[0]{\nabla}
%\newcommand{\pdot}[0]{\dot{p}}
%\newcommand{\pddot}[0]{\ddot{p}}

%------------------------------------------------------
% cross_old.tex

%%
%% shorthand for bold symbols, convenient for vectors and matrices
%%
%\newcommand{\Ba}[0]{\mathbf{a}}
%\newcommand{\Bb}[0]{\mathbf{b}}
%\newcommand{\Bc}[0]{\mathbf{c}}
%\newcommand{\Bd}[0]{\mathbf{d}}
%\newcommand{\Be}[0]{\mathbf{e}}
%\newcommand{\Bf}[0]{\mathbf{f}}
%\newcommand{\Bg}[0]{\mathbf{g}}
%\newcommand{\Bh}[0]{\mathbf{h}}
%\newcommand{\Bi}[0]{\mathbf{i}}
%\newcommand{\Bj}[0]{\mathbf{j}}
%\newcommand{\Bk}[0]{\mathbf{k}}
%\newcommand{\Bl}[0]{\mathbf{l}}
%\newcommand{\Bm}[0]{\mathbf{m}}
%\newcommand{\Bn}[0]{\mathbf{n}}
%\newcommand{\Bo}[0]{\mathbf{o}}
%\newcommand{\Bp}[0]{\mathbf{p}}
%\newcommand{\Bq}[0]{\mathbf{q}}
%\newcommand{\Br}[0]{\mathbf{r}}
%\newcommand{\Bs}[0]{\mathbf{s}}
%\newcommand{\Bt}[0]{\mathbf{t}}
%\newcommand{\Bu}[0]{\mathbf{u}}
%\newcommand{\Bv}[0]{\mathbf{v}}
%\newcommand{\Bw}[0]{\mathbf{w}}
%\newcommand{\Bx}[0]{\mathbf{x}}
%\newcommand{\By}[0]{\mathbf{y}}
%\newcommand{\Bz}[0]{\mathbf{z}}
%\newcommand{\BA}[0]{\mathbf{A}}
%\newcommand{\BB}[0]{\mathbf{B}}
%\newcommand{\BC}[0]{\mathbf{C}}
%\newcommand{\BD}[0]{\mathbf{D}}
%\newcommand{\BE}[0]{\mathbf{E}}
%\newcommand{\BF}[0]{\mathbf{F}}
%\newcommand{\BG}[0]{\mathbf{G}}
%\newcommand{\BH}[0]{\mathbf{H}}
%\newcommand{\BI}[0]{\mathbf{I}}
%\newcommand{\BJ}[0]{\mathbf{J}}
%\newcommand{\BK}[0]{\mathbf{K}}
%\newcommand{\BL}[0]{\mathbf{L}}
%\newcommand{\BM}[0]{\mathbf{M}}
%\newcommand{\BN}[0]{\mathbf{N}}
%\newcommand{\BO}[0]{\mathbf{O}}
%\newcommand{\BP}[0]{\mathbf{P}}
%\newcommand{\BQ}[0]{\mathbf{Q}}
%\newcommand{\BR}[0]{\mathbf{R}}
%\newcommand{\BS}[0]{\mathbf{S}}
%\newcommand{\BT}[0]{\mathbf{T}}
%\newcommand{\BU}[0]{\mathbf{U}}
%\newcommand{\BV}[0]{\mathbf{V}}
%\newcommand{\BW}[0]{\mathbf{W}}
%\newcommand{\BX}[0]{\mathbf{X}}
%\newcommand{\BY}[0]{\mathbf{Y}}
%\newcommand{\BZ}[0]{\mathbf{Z}}
%
%\newcommand{\Bzero}[0]{\mathbf{0}}
%\newcommand{\Btheta}[0]{\boldsymbol{\theta}}
%\newcommand{\Btau}[0]{\boldsymbol{\tau}}
%\newcommand{\Bomega}[0]{\boldsymbol{\omega}}
%
%%
%% shorthand for unit vectors
%%
%\newcommand{\acap}[0]{\hat{\Ba}}
%\newcommand{\bcap}[0]{\hat{\Bb}}
%\newcommand{\ccap}[0]{\hat{\Bc}}
%\newcommand{\dcap}[0]{\hat{\Bd}}
%\newcommand{\ecap}[0]{\hat{\Be}}
%\newcommand{\fcap}[0]{\hat{\Bf}}
%\newcommand{\gcap}[0]{\hat{\Bg}}
%\newcommand{\hcap}[0]{\hat{\Bh}}
%\newcommand{\icap}[0]{\hat{\Bi}}
%\newcommand{\jCap}[0]{\hat{\Bj}}
%\newcommand{\kcap}[0]{\hat{\Bk}}
%\newcommand{\lcap}[0]{\hat{\Bl}}
%\newcommand{\mcap}[0]{\hat{\Bm}}
%\newcommand{\ncap}[0]{\hat{\Bn}}
%\newcommand{\ocap}[0]{\hat{\Bo}}
%\newcommand{\pcap}[0]{\hat{\Bp}}
%\newcommand{\qcap}[0]{\hat{\Bq}}
%\newcommand{\rcap}[0]{\hat{\Br}}
%\newcommand{\scap}[0]{\hat{\Bs}}
%\newcommand{\tcap}[0]{\hat{\Bt}}
%\newcommand{\ucap}[0]{\hat{\Bu}}
%\newcommand{\vcap}[0]{\hat{\Bv}}
%\newcommand{\wcap}[0]{\hat{\Bw}}
%\newcommand{\xcap}[0]{\hat{\Bx}}
%\newcommand{\ycap}[0]{\hat{\By}}
%\newcommand{\zcap}[0]{\hat{\Bz}}
%\newcommand{\thetacap}[0]{\hat{\Btheta}}
%
%%
%% to write R^n and C^n in a distinguishable fashion.  Perhaps change this
%% to the double lined characters upon figuring out how to do so.
%%
%\newcommand{\C}[1]{${\BC}^{#1}$}
%\newcommand{\R}[1]{${\BR}^{#1}$}
%
%%
%% various generally useful helpers
%%
%
%% derivative of #1 wrt. #2:
%\newcommand{\D}[2] {\frac {d#2} {d#1}}

%\newcommand{\inv}[1]{\frac{1}{#1}}
%\newcommand{\cross}[0]{\times}

%\newcommand{\abs}[1]{\lvert#1\rvert}
%\newcommand{\norm}[1]{\lVert#1\rVert}
%\newcommand{\innerprod}[2]{\langle{#1}, {#2}\rangle}
%\newcommand{\dotprod}[2]{#1 \cdot #2}
%\newcommand{\crossprod}[2]{#1 \cross #2}
%\newcommand{\tripleprod}[3]{\dotprod{\crossprod{#1}{#2}}{#3}}

%
% A few miscellaneous things specific to this document
%
%\newcommand{\crossop}[1]{\crossprod{#1}{}}

\newcommand{\PDP}[2]{\BP^{#1}\BD{\BP^{#2}}}
\newcommand{\PDPDP}[3]{\Bv^T\BP^{#1}\BD\BP^{#2}\BD\BP^{#3}\Bv}

\newcommand{\Mp}[0]{
\begin{bmatrix}
0 & 1 & 0 & 0 \\
0 & 0 & 1 & 0 \\
0 & 0 & 0 & 1 \\
1 & 0 & 0 & 0
\end{bmatrix}
}
\newcommand{\Mpp}[0]{
\begin{bmatrix}
0 & 0 & 1 & 0 \\
0 & 0 & 0 & 1 \\
1 & 0 & 0 & 0 \\
0 & 1 & 0 & 0
\end{bmatrix}
}
\newcommand{\Mppp}[0]{
\begin{bmatrix}
0 & 0 & 0 & 1 \\
1 & 0 & 0 & 0 \\
0 & 1 & 0 & 0 \\
0 & 0 & 1 & 0
\end{bmatrix}
}
\newcommand{\Mpu}[0]{
\begin{bmatrix}
u_1 & 0 & 0 & 0 \\
0 & u_2 & 0 & 0 \\
0 & 0 & u_3 & 0 \\
0 & 0 & 0 & u_4
\end{bmatrix}
}

%------------------------------------------------------



%\makeindex

\begin{document}
\pagenumbering{alph}

\title{Course notes and problems from\\University of Toronto PHY450H1S\\Relativistic Electrodynamics.}
\author{Peeter Joot \quad peeter.joot@gmail.com}

\maketitle

\clearpage\pagenumbering{roman}
\tableofcontents

\clearpage\pagenumbering{arabic}

\pagestyle{plain}

%
% Copyright � 2015 Peeter Joot.  All Rights Reserved.
% Licenced as described in the file LICENSE under the root directory of this GIT repository.
%

% 
%\chapter{Preface}
% this suppresses an explicit chapter number for the preface.
\chapter*{Preface}%\normalsize
  \addcontentsline{toc}{chapter}{Preface}

This document was produced while taking the Spring 2016, University of Toronto Microwave Circuits course (ECE1236H), taught by Prof.\ G. V. Eleftheriades.

\paragraph{Course Syllabus}

This course outlines the principles of designing modern microwave and RF circuits.  Signal-integrity issues in high-speed digital circuits are also examined.

\begin{itemize}
\item The wave equation.
\item Ideal transmission lines.
\item Transients on transmission-lines.
\item Planar transmission lines and introduction to MMIC's.
\item Designing with scattering parameters.
\item Planar power dividers.
\item Directional couplers.
\item Microwave filters.
\item Solid-state microwave amplifiers.
\item Noise.
\item Diode-mixers.
\item RF receiver chains.
\item Oscillators.
\end{itemize}

\withproblemsetsMessage{
\textcolor{Maroon}{
\textit{THIS DOCUMENT IS REDACTED.  THE PROBLEM SET SOLUTIONS AND ASSOCIATED MATHEMATICA CODE IS NOT VISIBLE.  PLEASE EMAIL ME FOR THE FULL VERSION IF YOU ARE NOT TAKING ECE1236.}
}
}

\paragraph{This document contains:}

\begin{itemize}
\item Lecture notes.
\item Personal notes exploring auxiliary details.
\item Worked practice problems.

\ifthenelse{\boolean{redacted}}%
{%
\item Links to Mathematica notebooks associated with the course material and problems (but not problem sets).
}%
{
\item Assigned problems.%
\item Links to Mathematica notebooks associated with problems and course material.%
}
\end{itemize}

%This set of notes is significantly different from my notes for many other classes.  With the class taught on slides (and some of those slides mirroring the text closely), I did not take live notes in class.
%These notes fill in details that I felt deserved clarification, contain problem sets solutions, as well as a number of loosely related musings on Geometric Algebra equivalents to some of the generalized concepts of electromagnetic theory encountered in this class (i.e. magnetic sources).
%
My thanks go to Professor Eleftheriades for teaching this course.

Peeter Joot  \quad peeterjoot@protonmail.com 


%-------------------------------------------------------

\part{Lecture and Tutorial Notes.}
%
% Copyright � 2012 Peeter Joot.  All Rights Reserved.
% Licenced as described in the file LICENSE under the root directory of this GIT repository.
%

%\chapter{Speed of light and simultaneity}
\label{chap:relativisticElectrodynamicsL1}
%\blogpage{http://sites.google.com/site/peeterjoot/math2011/relativisticElectrodynamicsL1.pdf}
%\date{Jan 11, 2011}

\paragraph{Reading}

No reading from \citep{landau1980classical} appears to have been assigned, but relevant stuff can be found in chapter 1.

From \href{http://www.physics.utoronto.ca/~poppitz/epoppitz/PHY450_files/RelEMp1-11.pdf}{Professor Poppitz's lecture notes}, we have reading: pp.1-11: space, time and Gallilean relativity (1-6); speed of light and Einsteins relativity principle (7-10); relativity of simultaneity (11).

\section{Distance as a clock}

The title of this course is an oxymoron since ELECTRODYNAMICS == RELATIVITY.  In classical and quantum physics (non-gravitational) we start by postulating the existence of space and time.  These are, in non-gravitational physics, the arena where everything takes place.  The space that we work with is the three dimensional Euclidean space \R{3}.  One way of describing it is using three coordinates

\begin{equation}\label{eqn:relativisticElectrodynamicsL1:10}
\text{\R{3}} = \{ x, y, z ; x,y,z \in [-\infty,\infty] \}.
\end{equation}

We define a distance between \(P\) and \(P'\) as

\begin{equation}\label{eqn:relativisticElectrodynamicsL1:20}
\Abs{P P' } = \sqrt{ (x-x')^2 + (y-y')^2 + (z-z')^2 }
\end{equation}

\textunderline{time} is a parameter with respect to which positions of \textunderline{free} particles particles change at a constant rate.

Mathematically, we describe the motion of free particles by giving \((x(t), y(t), z(t))\) : coordinates as functions of t, 

\begin{equation}\label{eqn:relativisticElectrodynamicsL1:30}
\frac{d^2 x_i(t)}{dt^2} = 0, i = 1,2,3
\end{equation}

Here \(x,y,z\) are the free particle coordinates in an ``internal frame'', the frame where \(\ddot{\Br} = 0\) holds for a free particle (\(\ddot{\Br} = d^2 \Br/dt^2 \)) for a free particle \(x = v_0 t, y = z = 0\).

\section{The principle of relativity}

\paragraph{Principle of relativity (Galileo or Einstein)}: ``Laws of nature are identical in all inertial frames''.

Equivalently, ``Identical experiments in two inertial frames yield identical results''.

\paragraph{What do we mean by laws of nature}.  Equations that describe dynamics.

Now we need to get more specific.  Identical equations means that the equations have the same form in two inertial frames provided, you express them (the equations) via the coordinates \(\Br,t\) in the given inertial frame.

FIXME: DRAW x,y,z COORDINATE SYSTEM with origin \(O\).  And another with origin \(O'\) where the origin is moving with velocity \(v\) in the y direction.

The Galilean relativity principle states that ``equations of motion are invariant under Galilean transformations''.  What do we mean by transformations?  If we have a point \(P(t)\) in space with coordinates in both frames that are related.  It is pretty clear that the coordinates \(x = x'\) and \(z = z'\).  What about the \(y'\) coordinate?  For that we have \(y' = y - v t\), so that the origins overlap (\(O = O'\)) at \(t=0\).

In Galilean relativity, time is absolute.  i.e. It is the same in all inertial frames.  It is now a no-brainer to find the velocities of the particle.  Taking derivatives we take time derivatives of

\begin{align}\label{eqn:relativisticElectrodynamicsL1:40}
x' &= x \\
y' &= y - v t \\
z' &= z,
\end{align}

for

\begin{align}\label{eqn:relativisticElectrodynamicsL1:50}
v_x' &= v_x \\
v_y' &= v_y - v \\
v_z' &= v_z.
\end{align}

In vector notation we have 

\begin{align}\label{eqn:relativisticElectrodynamicsL1:60}
\Br' &= \Br - \Bv_0 t \\
\Bv' &= \Bv - \Bv_0 
\end{align}

The principle of relativity says that the dynamical equations are invariant under such transformations.

Take Newton's law for example applied to two bodies, labeled by their masses \(M_1\) and \(M_2\).

These bodies may be interacting.  For example, with Newtonian gravitation

\begin{equation}\label{eqn:relativisticElectrodynamicsL1:70}
V(\Br_1 - \Br_2) = -G_N \frac{M_1 M_2}{\Abs{\Br_1 - \Br_2}},
\end{equation}

or the Van Der Waals, interaction

\begin{equation}\label{eqn:relativisticElectrodynamicsL1:80}
V(\Br_1 - \Br_2) = - \text{const} \inv{\Abs{\Br_1 - \Br_2}^6},
\end{equation}

Our interaction is via a gradient \(\PDi{\Br}{f(\Br)} = ( \PDi{x}{f}, \PDi{y}{f}, \PDi{z}{f} )\)

\begin{align}\label{eqn:relativisticElectrodynamicsL1:90}
M_1 \ddot{\Br}_1 &= -\PD{\Br_1}{} V(\Br_1 - \Br_2) \\
M_2 \ddot{\Br}_2 &= -\PD{\Br_2}{} V(\Br_1 - \Br_2)
\end{align}

In the unprimed frame, these are ``the laws of physics''.  Consider a primed frame \(O' : \Br_i' = \Br_i - \Bv_0 t\) (for \(i=1,2\)).  Taking derivatives we have \(\Bv_i' = \Bv_i' + \Bv_0\), and \(\dot{\Bv}_i' = \dot{\Bv}_i'\).

We note that the distance between the two particles is unchanged in the primed coordinate system

\begin{equation}\label{eqn:relativisticElectrodynamicsL1:100}
\Br_1' - \Br_2' = \Br_1 - \Bv_0 t -( \Br_2 - \Bv_0 t ) = \Br_1 - \Br_2
\end{equation}

Similarly 

\begin{equation}\label{eqn:relativisticElectrodynamicsL1:110}
\PD{\Br_i}{} = \PD{(\Br_i' + \Bv_0 t)}{} = \PD{\Br_i'}{}
\end{equation}

Observe that the interaction \eqnref{eqn:relativisticElectrodynamicsL1:90} is unchanged by this change in coordinates.

\section{Enter electromagnetism}

%This interaction invariance had no known exceptions until the advent of Maxwellian electrodynamics.  In electrodynamics we wiggle or shake a charge and generate fields, and the generation of such fields produce electodynamic waves.  
%This disturbance propaga

If the only interactions are \(1/r\) gravity and \(1/r\) Coulomb, Galilean relativity holds.  Electromagnetism came along and Maxwell's prediction that electromagnetic waves exist and propagate with speed

\begin{equation}\label{eqn:relativisticElectrodynamicsL1:200}
c \approx 3 \times 10^8 m/s
\end{equation}

(Note that in SI units \(c = 1/\sqrt{ \epsilon_0 \mu_0 }\)).

It was proposed that the speed of light was the speed in a medium (the ``aether'') through which electrodynamic waves propagate.  The idea was that the oscillations of this medium constitute electromagnetic waves.  Then ``c'' would be the speed of light with respect to that medium.  This medium would fill all space.

PICTURE: of gradient field, with aether velocity at different points.  Superimposed on this is a picture of the Earth's orbit, so that the velocity of the aether could be measured at different points of the earth's orbit by measuring the speed of light at different points in the orbit.

PICTURE: of interferometer.

We can study this effect by rotating this platform to measure at different points of the day and the year.

We note that the speed of the earth is approximately \(v_{+} = 150 \times 10^6 km/ 10^7 s \approx 15 km/s\).

Aside: It was not clear to me where these numbers came from.  \href{http://www.wolframalpha.com/input/?i=speed+of+the+earth}{Wolfram alpha says} that the Earth's orbital speed is approximately \(32 km/s\), although that is still within an order of magnitude of the number used in class.

The shift of fringes would then be \(v_{+} \approx (v_{+}/c)^2 \approx 10^{-8}\).  What Einstein did was to elevate the principle of relativity to one that applies to electromagnetism, but replacing the transformation relating frames to the Lorentz transformation, a transformation observed by Lorentz and Poincare that leave Maxwell's equations invariant.  Einstein did this by postulating that the speed of light is a constant in all frames, and we will see how this is the case.

\makequestion{Is not this true only outside of matter?}{question:relativisticElectrodynamicsL1:1}{

In matter we have electromagnetic wave propagation at speeds less than \(c\).

\paragraph{A:}  (paraphrasing)

We can consider the in-matter case to be a special case, treating collections of discreet particles as continuous approximations.  It is only as a side effect of these approximations that one produces the in-matter Maxwell's equation, and we will consider the ``vacuum'' Maxwell equation as always true, provided the points of interest do not fall exactly on any specific particle.

Yes we have speed of light different in media.  Example, speed of light in water is \(3/4\) vacuum speed due to high index of refraction.  Also note that we can have effects like an electron moving in water can constantly emit light.  This is called Cerenkov radiation.
}



%
% Copyright � 2012 Peeter Joot.  All Rights Reserved.
% Licenced as described in the file LICENSE under the root directory of this GIT repository.
%

%\chapter{Spacetime, events, worldlines, proper time, invariance}
\label{chap:relativisticElectrodynamicsL2}
%\blogpage{http://sites.google.com/site/peeterjoot/math2011/relativisticElectrodynamicsL2.pdf}
%\date{Jan 12, 2011}

\paragraph{Reading}

No reading from \citep{landau1980classical} appears to have been assigned, but relevant stuff can be found in chapter 1.

From \href{http://www.physics.utoronto.ca/~poppitz/epoppitz/PHY450_files/RelEM12-26.pdf}{Professor Poppitz's lecture notes}, we have reading: pp.12-26: spacetime, spacetime points, worldlines, interval (12-14); invariance of infinitesimal intervals (15-17).


\section{Einstein's relativity principle}

\begin{enumerate}

\item Replace Galilean transformations between coordinates in differential inertial frames with Lorentz transforms between $(\Bx, t)$.  Postulate that these constitute the symmetries of physics.  Recall that Galilean transformations are symmetries of the laws of non-relativistic physics.  

\item Speed of light $c$ is the same in all inertial frames.  Phrased in this form, relativity leads to ``relativity of simultaneity''.

PICTURE: Three people on a platform, at positions $1,3,2$, all with equidistant separation.  This stationary frame is labeled $O$.  1 and 2 flash light signals at the same time and in frame $O$ the reception of the light signal by 3 is observed as arriving at 3 simultaneously.

Now introduce a moving frame with origin $O'$ moving along the positive x axis.  To a stationary observer in $O'$ the three guys are seen to be moving in the $-x$ direction.  The middle guy (3) is eventually going to be seen to receive the light signal by this $O'$ observer, but less time is required for the light to get from 1 to 3, and more time is required for the light to get from 2 to 1 (3 is moving away from the light according to the $O'$ observer).  Because the speed of light is perceived as constant for all observers, the perception is then that the light must arrive at 3 at different times.

This is very non-intuitive since we are implicitly trained by our surroundings that Galilean transformations govern mechanical behavior.

In $O$, 1 and 2 send light signals simultaneously while in $O'$ 1 sends light later than 2.  The conclusion, rather surprisingly compared to intuition, is that simultaneity is relative.

\end{enumerate}

\makequestion{On symmetries.}{question:relativisticElectrodynamicsL2:1}{
\paragraph{Q:}
A comment made that the symmetries impose the dynamics, and the symmetries provided the form of the Lagrangian in classical physics.  My question to this comment was

``When we have transformations that leave the Lagrangian unchanged (a symmetry), we have a conserved current.  I have done various exercises to compute those currents for various types of transformations (translation, spacetime translation, rotation, boosts, ...), but can not think of a way that the Lagrangian itself is defined these sorts symmetries.  Can you elaborate on what you mean by this?''

\paragraph{A:}

Ah, you see, what I meant by that is the following. For a free particle $\LL$ should depend on $\Bx$, $\dot{\Bx}$ and $t$.  Homogeneity of space and time do not allow to have $x$ and $t$ dependence and isotropy of space only permits dependence on $\Abs{\dot{\Bx}}$. Finally, Gallilean relativity only allows $\LL = \dot{\Bx}^2$ (times a constant). (See \citep{landau1960classical} vol 1 or \href{http://www.physics.utoronto.ca/~poppitz/epoppitz/PHY354_files/CMpp13.1-27.pdf}{my notes on PHY354 website}, p. 23-27).

So what was used is:

\begin{itemize}
\item Having only dependence on $x$ and $dx/dt$.
\item Spacetime homogeneity/isotropy.
\item Gallilean relativity.
\end{itemize}

Similar story holds in relativity, as we will see. 


}



%
% Copyright � 2015 Peeter Joot.  All Rights Reserved.
% Licenced as described in the file LICENSE under the root directory of this GIT repository.
%
\documentclass[]{eliblog}

\usepackage{amsmath}
\usepackage{mathpazo}

%
% shorthand for bold symbols, convenient for vectors and matrices
%
\newcommand{\Ba}[0]{\mathbf{a}}
\newcommand{\Bb}[0]{\mathbf{b}}
\newcommand{\Bc}[0]{\mathbf{c}}
\newcommand{\Bd}[0]{\mathbf{d}}
\newcommand{\Be}[0]{\mathbf{e}}
\newcommand{\Bf}[0]{\mathbf{f}}
\newcommand{\Bg}[0]{\mathbf{g}}
\newcommand{\Bh}[0]{\mathbf{h}}
\newcommand{\Bi}[0]{\mathbf{i}}
\newcommand{\Bj}[0]{\mathbf{j}}
\newcommand{\Bk}[0]{\mathbf{k}}
\newcommand{\Bl}[0]{\mathbf{l}}
\newcommand{\Bm}[0]{\mathbf{m}}
\newcommand{\Bn}[0]{\mathbf{n}}
\newcommand{\Bo}[0]{\mathbf{o}}
\newcommand{\Bp}[0]{\mathbf{p}}
\newcommand{\Bq}[0]{\mathbf{q}}
\newcommand{\Br}[0]{\mathbf{r}}
\newcommand{\Bs}[0]{\mathbf{s}}
\newcommand{\Bt}[0]{\mathbf{t}}
\newcommand{\Bu}[0]{\mathbf{u}}
\newcommand{\Bv}[0]{\mathbf{v}}
\newcommand{\Bw}[0]{\mathbf{w}}
\newcommand{\Bx}[0]{\mathbf{x}}
\newcommand{\By}[0]{\mathbf{y}}
\newcommand{\Bz}[0]{\mathbf{z}}
\newcommand{\BA}[0]{\mathbf{A}}
\newcommand{\BB}[0]{\mathbf{B}}
\newcommand{\BC}[0]{\mathbf{C}}
\newcommand{\BD}[0]{\mathbf{D}}
\newcommand{\BE}[0]{\mathbf{E}}
\newcommand{\BF}[0]{\mathbf{F}}
\newcommand{\BG}[0]{\mathbf{G}}
\newcommand{\BH}[0]{\mathbf{H}}
\newcommand{\BI}[0]{\mathbf{I}}
\newcommand{\BJ}[0]{\mathbf{J}}
\newcommand{\BK}[0]{\mathbf{K}}
\newcommand{\BL}[0]{\mathbf{L}}
\newcommand{\BM}[0]{\mathbf{M}}
\newcommand{\BN}[0]{\mathbf{N}}
\newcommand{\BO}[0]{\mathbf{O}}
\newcommand{\BP}[0]{\mathbf{P}}
\newcommand{\BQ}[0]{\mathbf{Q}}
\newcommand{\BR}[0]{\mathbf{R}}
\newcommand{\BS}[0]{\mathbf{S}}
\newcommand{\BT}[0]{\mathbf{T}}
\newcommand{\BU}[0]{\mathbf{U}}
\newcommand{\BV}[0]{\mathbf{V}}
\newcommand{\BW}[0]{\mathbf{W}}
\newcommand{\BX}[0]{\mathbf{X}}
\newcommand{\BY}[0]{\mathbf{Y}}
\newcommand{\BZ}[0]{\mathbf{Z}}

\newcommand{\Bzero}[0]{\mathbf{0}}
\newcommand{\Btheta}[0]{\boldsymbol{\theta}}
\newcommand{\Btau}[0]{\boldsymbol{\tau}}
\newcommand{\Bomega}[0]{\boldsymbol{\omega}}

%
% shorthand for unit vectors
%
\newcommand{\acap}[0]{\hat{\Ba}}
\newcommand{\bcap}[0]{\hat{\Bb}}
\newcommand{\ccap}[0]{\hat{\Bc}}
\newcommand{\dcap}[0]{\hat{\Bd}}
\newcommand{\ecap}[0]{\hat{\Be}}
\newcommand{\fcap}[0]{\hat{\Bf}}
\newcommand{\gcap}[0]{\hat{\Bg}}
\newcommand{\hcap}[0]{\hat{\Bh}}
\newcommand{\icap}[0]{\hat{\Bi}}
\newcommand{\jcap}[0]{\hat{\Bj}}
\newcommand{\kcap}[0]{\hat{\Bk}}
\newcommand{\lcap}[0]{\hat{\Bl}}
\newcommand{\mcap}[0]{\hat{\Bm}}
\newcommand{\ncap}[0]{\hat{\Bn}}
\newcommand{\ocap}[0]{\hat{\Bo}}
\newcommand{\pcap}[0]{\hat{\Bp}}
\newcommand{\qcap}[0]{\hat{\Bq}}
\newcommand{\rcap}[0]{\hat{\Br}}
\newcommand{\scap}[0]{\hat{\Bs}}
\newcommand{\tcap}[0]{\hat{\Bt}}
\newcommand{\ucap}[0]{\hat{\Bu}}
\newcommand{\vcap}[0]{\hat{\Bv}}
\newcommand{\wcap}[0]{\hat{\Bw}}
\newcommand{\xcap}[0]{\hat{\Bx}}
\newcommand{\ycap}[0]{\hat{\By}}
\newcommand{\zcap}[0]{\hat{\Bz}}
\newcommand{\thetacap}[0]{\hat{\Btheta}}

%
% to write R^n and C^n in a distinguishable fashion.  Perhaps change this
% to the double lined characters upon figuring out how to do so.
%
\newcommand{\C}[1]{$\mathbb{C}^{#1}$}
\newcommand{\R}[1]{$\mathbb{R}^{#1}$}

%
% various generally useful helpers
%

% derivative of #1 wrt. #2:
\newcommand{\D}[2] {\frac {d#2} {d#1}}

\newcommand{\inv}[1]{\frac{1}{#1}}
\newcommand{\cross}[0]{\times}

\newcommand{\abs}[1]{\lvert{#1}\rvert}
\newcommand{\norm}[1]{\lVert{#1}\rVert}
\newcommand{\innerprod}[2]{\langle{#1}, {#2}\rangle}
\newcommand{\dotprod}[2]{{#1} \cdot {#2}}
\newcommand{\bdotprod}[2]{\left({#1} \cdot {#2}\right)}
\newcommand{\crossprod}[2]{{#1} \cross {#2}}
\newcommand{\tripleprod}[3]{\dotprod{\left(\crossprod{#1}{#2}\right)}{#3}}

\DeclareMathOperator{\Proj}{Proj}
\DeclareMathOperator{\Span}{span}
\DeclareMathOperator{\Sgn}{sgn}
\DeclareMathOperator{\Area}{Area}
\DeclareMathOperator{\Volume}{Volume}

%
% A few miscellaneous things specific to this document
%
\newcommand{\crossop}[1]{\crossprod{#1}{}}

% R2 vector.
\newcommand{\VectorTwo}[2]{
\begin{bmatrix}
 {#1} \\
 {#2}
\end{bmatrix}
}

\newcommand{\VectorN}[1]{
\begin{bmatrix}
{#1}_1 \\
{#1}_2 \\
\vdots \\
{#1}_N \\
\end{bmatrix}
}

\newcommand{\DETuvij}[4]{
\begin{vmatrix}
 {#1}_{#3} & {#1}_{#4} \\
 {#2}_{#3} & {#2}_{#4}
\end{vmatrix}
}

\newcommand{\DETuvwijk}[6]{
\begin{vmatrix}
 {#1}_{#4} & {#1}_{#5} & {#1}_{#6} \\
 {#2}_{#4} & {#2}_{#5} & {#2}_{#6} \\
 {#3}_{#4} & {#3}_{#5} & {#3}_{#6}
\end{vmatrix}
}

\newcommand{\DETuvwxijkl}[8]{
\begin{vmatrix}
 {#1}_{#5} & {#1}_{#6} & {#1}_{#7} & {#1}_{#8} \\
 {#2}_{#5} & {#2}_{#6} & {#2}_{#7} & {#2}_{#8} \\
 {#3}_{#5} & {#3}_{#6} & {#3}_{#7} & {#3}_{#8} \\
 {#4}_{#5} & {#4}_{#6} & {#4}_{#7} & {#4}_{#8} \\
\end{vmatrix}
}

%\newcommand{\DETuvwxyijklm}[10]{
%\begin{vmatrix}
% {#1}_{#6} & {#1}_{#7} & {#1}_{#8} & {#1}_{#9} & {#1}_{#10} \\
% {#2}_{#6} & {#2}_{#7} & {#2}_{#8} & {#2}_{#9} & {#2}_{#10} \\
% {#3}_{#6} & {#3}_{#7} & {#3}_{#8} & {#3}_{#9} & {#3}_{#10} \\
% {#4}_{#6} & {#4}_{#7} & {#4}_{#8} & {#4}_{#9} & {#4}_{#10} \\
% {#5}_{#6} & {#5}_{#7} & {#5}_{#8} & {#5}_{#9} & {#5}_{#10}
%\end{vmatrix}
%}

% R3 vector.
\newcommand{\VectorThree}[3]{
\begin{bmatrix}
 {#1} \\
 {#2} \\
 {#3}
\end{bmatrix}
}



\author{Peeter Joot}
\email{peeter.joot@gmail.com}

%\documentclass[]{eliblogwidescreen}

\usepackage{amsmath}
\usepackage{mathpazo}

%
% shorthand for bold symbols, convenient for vectors and matrices
%
\newcommand{\Ba}[0]{\mathbf{a}}
\newcommand{\Bb}[0]{\mathbf{b}}
\newcommand{\Bc}[0]{\mathbf{c}}
\newcommand{\Bd}[0]{\mathbf{d}}
\newcommand{\Be}[0]{\mathbf{e}}
\newcommand{\Bf}[0]{\mathbf{f}}
\newcommand{\Bg}[0]{\mathbf{g}}
\newcommand{\Bh}[0]{\mathbf{h}}
\newcommand{\Bi}[0]{\mathbf{i}}
\newcommand{\Bj}[0]{\mathbf{j}}
\newcommand{\Bk}[0]{\mathbf{k}}
\newcommand{\Bl}[0]{\mathbf{l}}
\newcommand{\Bm}[0]{\mathbf{m}}
\newcommand{\Bn}[0]{\mathbf{n}}
\newcommand{\Bo}[0]{\mathbf{o}}
\newcommand{\Bp}[0]{\mathbf{p}}
\newcommand{\Bq}[0]{\mathbf{q}}
\newcommand{\Br}[0]{\mathbf{r}}
\newcommand{\Bs}[0]{\mathbf{s}}
\newcommand{\Bt}[0]{\mathbf{t}}
\newcommand{\Bu}[0]{\mathbf{u}}
\newcommand{\Bv}[0]{\mathbf{v}}
\newcommand{\Bw}[0]{\mathbf{w}}
\newcommand{\Bx}[0]{\mathbf{x}}
\newcommand{\By}[0]{\mathbf{y}}
\newcommand{\Bz}[0]{\mathbf{z}}
\newcommand{\BA}[0]{\mathbf{A}}
\newcommand{\BB}[0]{\mathbf{B}}
\newcommand{\BC}[0]{\mathbf{C}}
\newcommand{\BD}[0]{\mathbf{D}}
\newcommand{\BE}[0]{\mathbf{E}}
\newcommand{\BF}[0]{\mathbf{F}}
\newcommand{\BG}[0]{\mathbf{G}}
\newcommand{\BH}[0]{\mathbf{H}}
\newcommand{\BI}[0]{\mathbf{I}}
\newcommand{\BJ}[0]{\mathbf{J}}
\newcommand{\BK}[0]{\mathbf{K}}
\newcommand{\BL}[0]{\mathbf{L}}
\newcommand{\BM}[0]{\mathbf{M}}
\newcommand{\BN}[0]{\mathbf{N}}
\newcommand{\BO}[0]{\mathbf{O}}
\newcommand{\BP}[0]{\mathbf{P}}
\newcommand{\BQ}[0]{\mathbf{Q}}
\newcommand{\BR}[0]{\mathbf{R}}
\newcommand{\BS}[0]{\mathbf{S}}
\newcommand{\BT}[0]{\mathbf{T}}
\newcommand{\BU}[0]{\mathbf{U}}
\newcommand{\BV}[0]{\mathbf{V}}
\newcommand{\BW}[0]{\mathbf{W}}
\newcommand{\BX}[0]{\mathbf{X}}
\newcommand{\BY}[0]{\mathbf{Y}}
\newcommand{\BZ}[0]{\mathbf{Z}}

\newcommand{\Bzero}[0]{\mathbf{0}}
\newcommand{\Btheta}[0]{\boldsymbol{\theta}}
\newcommand{\Btau}[0]{\boldsymbol{\tau}}
\newcommand{\Bomega}[0]{\boldsymbol{\omega}}

%
% shorthand for unit vectors
%
\newcommand{\acap}[0]{\hat{\Ba}}
\newcommand{\bcap}[0]{\hat{\Bb}}
\newcommand{\ccap}[0]{\hat{\Bc}}
\newcommand{\dcap}[0]{\hat{\Bd}}
\newcommand{\ecap}[0]{\hat{\Be}}
\newcommand{\fcap}[0]{\hat{\Bf}}
\newcommand{\gcap}[0]{\hat{\Bg}}
\newcommand{\hcap}[0]{\hat{\Bh}}
\newcommand{\icap}[0]{\hat{\Bi}}
\newcommand{\jcap}[0]{\hat{\Bj}}
\newcommand{\kcap}[0]{\hat{\Bk}}
\newcommand{\lcap}[0]{\hat{\Bl}}
\newcommand{\mcap}[0]{\hat{\Bm}}
\newcommand{\ncap}[0]{\hat{\Bn}}
\newcommand{\ocap}[0]{\hat{\Bo}}
\newcommand{\pcap}[0]{\hat{\Bp}}
\newcommand{\qcap}[0]{\hat{\Bq}}
\newcommand{\rcap}[0]{\hat{\Br}}
\newcommand{\scap}[0]{\hat{\Bs}}
\newcommand{\tcap}[0]{\hat{\Bt}}
\newcommand{\ucap}[0]{\hat{\Bu}}
\newcommand{\vcap}[0]{\hat{\Bv}}
\newcommand{\wcap}[0]{\hat{\Bw}}
\newcommand{\xcap}[0]{\hat{\Bx}}
\newcommand{\ycap}[0]{\hat{\By}}
\newcommand{\zcap}[0]{\hat{\Bz}}
\newcommand{\thetacap}[0]{\hat{\Btheta}}

%
% to write R^n and C^n in a distinguishable fashion.  Perhaps change this
% to the double lined characters upon figuring out how to do so.
%
\newcommand{\C}[1]{$\mathbb{C}^{#1}$}
\newcommand{\R}[1]{$\mathbb{R}^{#1}$}

%
% various generally useful helpers
%

% derivative of #1 wrt. #2:
\newcommand{\D}[2] {\frac {d#2} {d#1}}

\newcommand{\inv}[1]{\frac{1}{#1}}
\newcommand{\cross}[0]{\times}

\newcommand{\abs}[1]{\lvert{#1}\rvert}
\newcommand{\norm}[1]{\lVert{#1}\rVert}
\newcommand{\innerprod}[2]{\langle{#1}, {#2}\rangle}
\newcommand{\dotprod}[2]{{#1} \cdot {#2}}
\newcommand{\bdotprod}[2]{\left({#1} \cdot {#2}\right)}
\newcommand{\crossprod}[2]{{#1} \cross {#2}}
\newcommand{\tripleprod}[3]{\dotprod{\left(\crossprod{#1}{#2}\right)}{#3}}

\DeclareMathOperator{\Proj}{Proj}
\DeclareMathOperator{\Span}{span}
\DeclareMathOperator{\Sgn}{sgn}
\DeclareMathOperator{\Area}{Area}
\DeclareMathOperator{\Volume}{Volume}

%
% A few miscellaneous things specific to this document
%
\newcommand{\crossop}[1]{\crossprod{#1}{}}

% R2 vector.
\newcommand{\VectorTwo}[2]{
\begin{bmatrix}
 {#1} \\
 {#2}
\end{bmatrix}
}

\newcommand{\VectorN}[1]{
\begin{bmatrix}
{#1}_1 \\
{#1}_2 \\
\vdots \\
{#1}_N \\
\end{bmatrix}
}

\newcommand{\DETuvij}[4]{
\begin{vmatrix}
 {#1}_{#3} & {#1}_{#4} \\
 {#2}_{#3} & {#2}_{#4}
\end{vmatrix}
}

\newcommand{\DETuvwijk}[6]{
\begin{vmatrix}
 {#1}_{#4} & {#1}_{#5} & {#1}_{#6} \\
 {#2}_{#4} & {#2}_{#5} & {#2}_{#6} \\
 {#3}_{#4} & {#3}_{#5} & {#3}_{#6}
\end{vmatrix}
}

\newcommand{\DETuvwxijkl}[8]{
\begin{vmatrix}
 {#1}_{#5} & {#1}_{#6} & {#1}_{#7} & {#1}_{#8} \\
 {#2}_{#5} & {#2}_{#6} & {#2}_{#7} & {#2}_{#8} \\
 {#3}_{#5} & {#3}_{#6} & {#3}_{#7} & {#3}_{#8} \\
 {#4}_{#5} & {#4}_{#6} & {#4}_{#7} & {#4}_{#8} \\
\end{vmatrix}
}

%\newcommand{\DETuvwxyijklm}[10]{
%\begin{vmatrix}
% {#1}_{#6} & {#1}_{#7} & {#1}_{#8} & {#1}_{#9} & {#1}_{#10} \\
% {#2}_{#6} & {#2}_{#7} & {#2}_{#8} & {#2}_{#9} & {#2}_{#10} \\
% {#3}_{#6} & {#3}_{#7} & {#3}_{#8} & {#3}_{#9} & {#3}_{#10} \\
% {#4}_{#6} & {#4}_{#7} & {#4}_{#8} & {#4}_{#9} & {#4}_{#10} \\
% {#5}_{#6} & {#5}_{#7} & {#5}_{#8} & {#5}_{#9} & {#5}_{#10}
%\end{vmatrix}
%}

% R3 vector.
\newcommand{\VectorThree}[3]{
\begin{bmatrix}
 {#1} \\
 {#2} \\
 {#3}
\end{bmatrix}
}



\author{Peeter Joot}
\email{peeter.joot@gmail.com}


\chapter{PHY450H1S.  Relativistic Electrodynamics Lecture 3 (Taught by Prof. Erich Poppitz).  Spacetime, events, worldlines, spacetime intervals, and invariance.}
\label{chap:relativisticElectrodynamicsL3}
%\useCCL
\blogpage{http://sites.google.com/site/peeterjoot/math2011/relativisticElectrodynamicsL3.pdf}
\date{Jan 13, 2011}
\revisionInfo{relativisticElectrodynamicsL3.tex}

%\beginArtWithToc
\beginArtNoToc

\section{Reading.}

Still covering chapter 1 material from the text \cite{landau1980classical}.

Covering more from \href{http://www.physics.utoronto.ca/~poppitz/e-poppitz/PHY450_files/RelEM12-26.pdf}{Professor Poppitz's lecture notes}: geometry of spacetime, lightlike, spacelike, timelike intervals, and worldlines (18-22); proper time (23-24); invariance of finite intervals (25-26).

\section{Geometry of spacetime: lightlike, spacelike, timelike intervals.}

\section{Proper time.}

\section{Invariance of finite intervals.}

\EndArticle

%
% Copyright � 2012 Peeter Joot.  All Rights Reserved.
% Licenced as described in the file LICENSE under the root directory of this GIT repository.
%

%\chapter{Spacetime geometry, Lorentz transformations, Minkowski diagrams}
\index{Lorentz transformation}
\index{Minkowski diagram}
\label{chap:relativisticElectrodynamicsL4}
%\blogpage{http://sites.google.com/site/peeterjoot/math2011/relativisticElectrodynamicsL4.pdf}
%\date{Jan 18, 2011}

\paragraph{Reading}

Still covering chapter 1 material from the text \citep{landau1980classical},
\popcite{RelEM12-26.pdf}{lecture notes RelEM12-26.pdf}, and
%: invariance of finite intervals (25-26).
\popcite{RelEM27-44.pdf}{lecture notes RelEM27-44.pdf}.
%: analogy with rotations and derivation of Lorentz transformations (27-32); Minkowski space diagram of boosted frame (32.1); using the diagram to find length contraction (32.2) ; nonrelativistic limit of boosts (33).

\section{More spacetime geometry}

PICTURE: ct,x curvy worldline with tangent vector \(\Bv\).

In an inertial frame moving with \(\Bv\), whose origin coincides with momentary position of this moving observer \(ds^2 = c^2 {dt'}^2 = c^2 dt^2 - \Br^2\)

``proper time'' is

\begin{equation}\label{eqn:relativisticElectrodynamicsL4:10}
dt' = dt \sqrt{ 1 - \inv{c^2} \left( \frac{d\Br}{dt} \right)^2 } = dt \sqrt{ 1 - \frac{\Bv^2}{c^2}} 
\end{equation}

We see that \(dt' < dt\) if \(v > 0\), so that \(\sqrt{1-\Bv^2/c^2} < 1\).

In a manifestly invariant way we define the proper time as 

\begin{equation}\label{eqn:relativisticElectrodynamicsL4:20}
d\tau \equiv \frac{ds}{c}
\end{equation}

So that between worldpoints \(a\) and \(b\) the proper time is a line integral over the worldline

\begin{equation}\label{eqn:relativisticElectrodynamicsL4:30}
d\tau \equiv \inv{c} \int_a^b ds.
\end{equation}

PICTURE: We are splitting up the worldline into many small pieces and summing them up.

%HOLE IN LECTURE NOTES: ON PROPER TIME for ``length'' of straight vs. curved worldlines: TO BE REVISITED.  
%Prof. Poppitz promised to revisit this again next time ... his notes are confusing him, and he would like to move on.

\section{Finite interval invariance}

Tomorrow we are going to complete the proof about invariance.  We have shown that light like intervals are invariant, and that infinitesimal intervals are invariant.  We need to put these pieces together for finite intervals.

\section{Deriving the Lorentz transformation}
\index{Lorentz transformation}

Let us find the coordinate transforms that leave \(s_{12}^2\) invariant.  This generalizes Galileo's transformations.

We would like to generalize rotations, which leave spatial distance invariant.  Such a transformation also leaves the spacetime interval invariant.

In Euclidean space we can generate an arbitrary rotation by composition of rotation around any of the \(xy, yz, zx\) axis.

For 4D Euclidean space we would form any rotation by composition of any of the 6 independent rotations for the 6 available planes.  For example with \(x,y,z,w\) axis we can rotate in any of the \(xy, xz, xw, yz, yw, zw\) planes.

For spacetime we can ``rotate'' in \(x,t\), \(y,t\), \(z,t\) ``planes''.  Physically this is motion space (boosting a position).

\paragraph{Consider a \texorpdfstring{\(x,t\)}{x,t} transformation}

The trick (that is in the notes) is to rewrite the time as an analytical continuation of the time coordinate, as follows

\begin{equation}\label{eqn:relativisticElectrodynamicsL4:40}
ds^2 = c^2 dt^2 - dx^2
\end{equation}

and write

\begin{equation}\label{eqn:relativisticElectrodynamicsL4:50}
t \rightarrow i \tau,
\end{equation}

so that the interval becomes
\begin{equation}\label{eqn:relativisticElectrodynamicsL4:60}
ds^2 = - (c^2 d\tau^2 + dx^2)
\end{equation}

Now we have a structure that is familiar, and we can rotate as we normally do.  Prof does not want to go through the details of this ``trickery'' in class, but says to see the notes.  The end result is that we can transform as follows

\begin{align}\label{eqn:relativisticElectrodynamicsL4:70}
x' &= x \cosh \psi + ct \sinh \psi \\
ct' &= x \sinh \psi + ct \cosh \psi 
\end{align}

which is analogous to a spatial rotation
\begin{align}\label{eqn:relativisticElectrodynamicsL4:70b}
x' &= x \cos \alpha + y \sin \alpha \\
y' &= -x \sin \alpha + y \cos \alpha 
\end{align}


There are some differences in sign as well, but the important feature to recall is that \(\cosh^2 x - \sinh^2 x = (1/4)( e^{2x} + e^{-2x} + 2 - e^{2x} - e^{-2x} + 2 ) = 1\).  We call these hyperbolic rotations, something that is simply a mathematical transformation.  Now we want to relate this to something physical.

\paragraph{Q: What is \(\psi\)?}

The origin of \(O\) has coordinates \((t, \BO)\) in the \(O\) frame.

PICTURE (pg 32): \(O'\) frame translating along \(x\) axis with speed \(v_x\).  We have

\begin{equation}\label{eqn:relativisticElectrodynamicsL4:80}
\frac{x'}{c t'} = \frac{v_x}{c}
\end{equation}

However, using \eqnref{eqn:relativisticElectrodynamicsL4:70} we have for the (spatial) origin

\begin{align}\label{eqn:relativisticElectrodynamicsL4:90}
x' &= ct \sinh \psi \\
ct' &= ct \cosh \psi
\end{align}

so that
\begin{equation}\label{eqn:relativisticElectrodynamicsL4:100}
\frac{x'}{c t'} = \tanh \psi = \frac{v_x}{c}
\end{equation}

Using 

\begin{align}\label{eqn:relativisticElectrodynamicsL4:110}
\cosh \psi &= \inv{\sqrt{1 - \tanh^2 \psi}} \\
\sinh \psi &= \frac{\tanh \psi}{\sqrt{1 - \tanh^2 \psi}}
\end{align}

Performing all the gory substitutions one gets
\begin{align}\label{eqn:relativisticElectrodynamicsL4:120}
x' &= 
\inv{\sqrt{1 - v_x^2/c^2}} x
+
\frac{v_x/c}{\sqrt{1 - v_x^2/c^2}} c t \\
y' &= y \\
z' &= z \\
ct' &= 
\frac{v_x/c}{\sqrt{1 - v_x^2/c^2}} x
+
\inv{\sqrt{1 - v_x^2/c^2}} c t
\end{align}

PICTURE: Let us go to the more conventional case, where \(O\) is at rest and \(O'\) is moving with velocity \(v_x\).

We achieve this by simply changing the sign of \(v_x\) in \eqnref{eqn:relativisticElectrodynamicsL4:120} above.  This gives us

\begin{align}\label{eqn:relativisticElectrodynamicsL4:120b}
x' &= 
\inv{\sqrt{1 - v_x^2/c^2}} x
-
\frac{v_x/c}{\sqrt{1 - v_x^2/c^2}} c t \\
y' &= y \\
z' &= z \\
ct' &= 
-\frac{v_x/c}{\sqrt{1 - v_x^2/c^2}} x
+
\inv{\sqrt{1 - v_x^2/c^2}} c t
\end{align}

We want some shorthand to make this easier to write and introduce

\begin{equation}\label{eqn:relativisticElectrodynamicsL4:130}
\gamma = \inv{\sqrt{1 - v_x^2/c^2}},
\end{equation}

so that \eqnref{eqn:relativisticElectrodynamicsL4:120b} becomes

\begin{align}\label{eqn:relativisticElectrodynamicsL4:140}
x' &=  \gamma \left( x - \frac{v_x}{c} ct \right) \\
ct' &=  \gamma \left( ct - \frac{v_x}{c} x \right)
\end{align}

We started the class by saying these would generalize the Galilean transformations.  Observe that if we take \(c \rightarrow \infty\), we have \(\gamma \rightarrow 1\) and 

\begin{align}\label{eqn:relativisticElectrodynamicsL4:150}
x' &= x - v_x t + O((v_x/c)^2) \\
t' &= t  + O(v_x/c)
\end{align}

This is how to remember the signs.  We want things to match up with the non-relativistic limit.

\paragraph{Q: How do lines of constant \(x'\) and \(ct'\) look like on the \(x,ct\) spacetime diagram?}

Our starting point (again) is
\begin{align}\label{eqn:relativisticElectrodynamicsL4:140b}
x' &=  \gamma \left( x - \frac{v_x}{c} ct \right) \\
ct' &=  \gamma \left( ct - \frac{v_x}{c} x \right).
\end{align}

What are the points with \(x' = 0\).  Those are the points where \(x = (v_x/c) c t\).  This is the \(ct' axis\).  That is the straight worldline

PICTURE: worldline of \(O'\) origin.

What are the points with \(ct' = 0\).  Those are the points where \(c t = x v_x/c\).  This is the \(x' axis\).

Lines that are parallel to the \(x'\) axis are lines of constant \(x'\), and lines parallel to \(ct'\) axis are lines of constant \(t'\), but the light cone is the same for both.

\paragraph{What is this good for?}

We have time to pick from either length contraction or non-causality (how to kill your grandfather).  How about length contraction.  We can use the diagram to read the \(x\) or \(ct\) coordinates, or examine causality, but it is hard to read off \(t'\) or \(x'\) coordinates.

%\section{Causality}

%
% Copyright � 2012 Peeter Joot.  All Rights Reserved.
% Licenced as described in the file LICENSE under the root directory of this GIT repository.
%

%\chapter{Proper time, length contraction, time dialation, causality}
\label{chap:relativisticElectrodynamicsL5}
%\blogpage{http://sites.google.com/site/peeterjoot/math2011/relativisticElectrodynamicsL5.pdf}
%\date{Jan 19, 2011}

\paragraph{Reading}

Still covering chapter 1 material from the text \citep{landau1980classical}?

Covering \href{http://www.physics.utoronto.ca/~poppitz/epoppitz/PHY450_files/RelEM27-44.pdf}{Professor Poppitz's lecture notes}: Using Minkowski diagram to see the perils of superluminal propagation (32.3); nonrelativistic limit of boosts (33); number of parameters of Lorentz transformations (34-35); introducing four-vectors, the metric tensor, the invariant ``dot-product and SO(1,3) (36-40); the Poincare group (41); the convenience of ``upper'' and ``lower''indices (42-43); tensors (44) 

\section{More on proper time}

PICTURE:1: worldline with small interval.

Considering a small interval somewhere on the worldline trajectory, we have

\begin{equation}\label{eqn:relativisticElectrodynamicsL5:10}
ds^2 = c^2 dt^2 - dx^2 = c^2 {dt'}^2,
\end{equation}

where \(dt'\) is the proper time elapsed in a frame moving with velocity \(v\), and \(dt\) is the time elapsed in a stationary frame.

We have 

\begin{equation}\label{eqn:relativisticElectrodynamicsL5:20}
dt' = dt \sqrt{ 1 - (dx/dt)^2/c^2 } = dt \sqrt{ 1 - v^2/c^2 }.
\end{equation}

PICTURE:2: particle at rest.

For the particle at rest 

\begin{equation}\label{eqn:relativisticElectrodynamicsL5:30}
c \tau_{21}^{\text{stationary}} = c ( t_2 - t_1 ) = \int_1^2 ds = \int_1^2 c dt
\end{equation}

PICTURE:3: particle with motion.

``length'' of 1-2 ``curved'' worldline

\begin{equation}\label{eqn:relativisticElectrodynamicsL5:90}
\begin{aligned}
\int_1^2 ds' 
&= \int_1^2 c dt' \\
&= \int_1^2 c dt \sqrt{ 1 - (d\Bv/dt)^2 },
\end{aligned}
\end{equation}

where in this case \([1,2]\) denotes the range of a line integral over the worldline.  We see that the multiplier of dt for any point along the curve is smaller than \(1\), so that the length along a straight line is longest (i.e. for the particle at rest).

We have argued that if 1,2 occur at the same place, the spacetime length of a straight line between them is the longest.  This remains the time  for all 1,2 timelike separated.

LOTS OF DISCUSSION.  See \href{http://www.physics.utoronto.ca/~poppitz/epoppitz/PHY450_files/pp.24.1-24.4.pdf}{new posted notes for details}.

Back to page 18 of the notes.

We have argued that \(ds_{12} = {ds'}_{12} \implies s_{12} = {s'}_{12}\) for infinitesimal 1,2 even if not infinitesimal.

The idea is to represent the interval between twill not close 1,2 as a sum over small \(ds\)'s.

P6: \(x = x_2 t /t_2\) straight line through origin, with \(t \in [0, t_2]\).

P7: zoomed on part of this line.

\begin{equation}\label{eqn:relativisticElectrodynamicsL5:110}
\begin{aligned}
ds^2 
&= c^2 dt^2 - dx^2 \\
&= c^2 dt^2 - \left(\frac{x_2}{t_2}\right)^2 dt^2 \\
&= c^2 dt^2 \left( 1 - \inv{c^2} \left(\frac{x_2}{t_2}\right)^2 \right) \\
\end{aligned}
\end{equation}

or
\begin{equation}\label{eqn:relativisticElectrodynamicsL5:40}
\int_0^1 ds 
= c \int_0^{t_2} dt \sqrt{ 1 - \inv{c^2} \left(\frac{x_2}{t_2}\right)^2 } 
\end{equation}

In another frame just replace \(t \rightarrow t'\) and \(x_2 \rightarrow x_2'\)

\begin{equation}\label{eqn:relativisticElectrodynamicsL5:50}
\int_0^1 ds 
= c \int_0^{t_2'} dt \sqrt{1 - \inv{c^2} \left(\frac{x_2'}{t_2'}\right)^2 } \\
\end{equation}

\section{Length contraction}

Consider \(O\) and \(O'\) with \(O'\) moving in \(x\) with speed \(v_x > 0\).  Here we have

\begin{equation}\label{eqn:relativisticElectrodynamicsL5:60}
\begin{aligned}
x' &= \gamma \left( x - \frac{v_x}{c} ct \right) \\
c t' &= \gamma \left( ct - \frac{v_x}{c} x \right) 
\end{aligned}
\end{equation}

PICTURE: spacetime diagram with \(ct'\) at angle \(\alpha\), where \(\tan \alpha = v_x/c\).

Two points \((x_A,0)\), \((x_B,0)\), with rest length measured as \(L = x_B - x_A\).  From the diagram \(c(t_B - t_A) = \tan\alpha L\), and from \eqnref{eqn:relativisticElectrodynamicsL5:60} we have

\begin{equation}\label{eqn:relativisticElectrodynamicsL5:70}
\begin{aligned}
x_A' &= \gamma \left( x_A - \frac{v_x}{c} c t_A \right) \\
x_B' &= \gamma \left( x_B - \frac{v_x}{c} c t_B \right),
\end{aligned}
\end{equation}

so that 
\begin{equation}\label{eqn:relativisticElectrodynamicsL5:130}
\begin{aligned}
L' &= x_B' - x_A' \\
&= \gamma \left( (x_B - x_A) - \frac{v_x}{c} c (t_B -t_A) \right) \\
&= \gamma \left( L - \frac{v_x}{c} \tan \alpha L \right) \\
&= \gamma \left( L - \frac{v_x^2}{c^2} L \right) \\
&= \gamma L \left( 1 - \frac{v_x^2}{c^2} \right) \\
&= L \sqrt{ 1 - \frac{v_x^2}{c^2} } 
\end{aligned}
\end{equation}

\section{Superluminal speed and causality}

If Einstein's relativity holds, superliminal motion is a ``no-no''.  Imagine that some ``tachyons'' exist that can instantaneously transmit stuff between observers.

PICTURE9: two guys with resting worldlines showing.

Can send info back to \(A\) before \(A\) sends to \(B\).  Superluminal propagation allows sending information not yet available.  Can show this for finite superluminal velocities (but hard) as well as infinite velocity superluminal speeds.  We see that time ordering can not be changed for events separated by time like separation.  Events separated by spacelike separation cannot be ca usually connected.

%
% Copyright � 2012 Peeter Joot.  All Rights Reserved.
% Licenced as described in the file LICENSE under the root directory of this GIT repository.
%

\chapter{Four vectors and a worked flux density problem}
\label{chap:relativisticElectrodynamicsT1}
%\blogpage{http://sites.google.com/site/peeterjoot/math2011/relativisticElectrodynamicsT1.pdf}
%\date{Jan 20, 2011}

\section{Worked question}

Simon (our TA) blasted through a problem from Hartle \citep{hartle2003gravity}, section 5.17 (all the while apologizing for going so slow).  It took me a while to work through my notes to come up with something that was comprehensible to me.

At one point he asked if anybody was completely lost.  Nobody said yes, but given the class title, I had the urge to say ``No, just relatively lost''.

\paragraph{Q:}
In a source's rest frame $S$ emits radiation isotropically with a frequency $\omega$ with number flux $f(\text{photons}/\text{cm}^2 s)$.  Moves along x'-axis with speed $V$ in an observer frame ($O$).  What does the energy flux in $O$ look like?

\subsection{A brief intro with four vectors}

A 3-vector: 

\begin{align}\label{eqn:relativisticElectrodynamicsT1:10}
\Ba &= (a_x, a_y, a_z) = (a^1, a^2, a^3) \\
\Bb &= (b_x, b_y, b_z) = (b^1, b^2, b^3)
\end{align}

For this we have the dot product
\begin{equation}\label{eqn:relativisticElectrodynamicsT1:20}
\Ba \cdot \Bb = \sum_{\alpha=1}^3 a^\alpha b^\alpha
\end{equation}

Greek letters in this course (opposite to everybody else in the world, because of Landau and Lifshitz) run from 1 to 3, whereas roman letters run through the set $\{0,1,2,3\}$.

We want to put space and time on an equal footing and form the composite quantity (four vector) 
\begin{equation}\label{eqn:relativisticElectrodynamicsT1:40}
x^i = (ct, \Br) = (x^0, x^1, x^2, x^3),
\end{equation}

where
\begin{align}\label{eqn:relativisticElectrodynamicsT1:80}
x^0 &= ct \\
x^1 &= x \\
x^2 &= y \\
x^3 &= z.
\end{align}

It will also be convenient to drop indexes when referring to all the components of a four vector and we will use lower or upper case non-bold letters to represent such four vectors.  For example

\begin{equation}\label{eqn:relativisticElectrodynamicsT1:81}
X = (ct, \Br),
\end{equation}

or
\begin{equation}\label{eqn:relativisticElectrodynamicsT1:82}
u = \gamma \left(1, \Bv/c \right).
\end{equation}

Three vectors will be represented as letters with over arrows $\vec{a}$ or (in text) bold face $\Ba$.

Recall that the squared spacetime interval between two events $X_1$ and $X_2$ is defined as

\begin{equation}\label{eqn:relativisticElectrodynamicsT1:60}
{S_{X_1, X_2}}^2 = (ct_1 - c t_2)^2 - (\Bx_1 - \Bx_2)^2.
\end{equation}

In particular, with one of these zero, we have an operator which takes a single four vector and spits out a scalar, measuring a ``distance'' from the origin

\begin{equation}\label{eqn:relativisticElectrodynamicsT1:30}
s^2 = (ct)^2 - \Br^2.
\end{equation}

This motivates the introduction of a dot product for our four vector space.  

\begin{equation}\label{eqn:relativisticElectrodynamicsT1:50}
X \cdot X = (ct)^2 - \Br^2 = (x^0)^2 - \sum_{\alpha=1}^3 (x^\alpha)^2
\end{equation}

Utilizing the spacetime dot product of \ref{eqn:relativisticElectrodynamicsT1:50} we have for the dot product of the difference between two events

\begin{align*}
(X - Y) \cdot (X - Y)
&=
(x^0 - y^0)^2 - \sum_{\alpha =1}^3 (x^\alpha - y^\alpha)^2 \\
&=
X \cdot X + Y \cdot Y - 2 x^0 y^0 + 2 \sum_{\alpha =1}^3 x^\alpha y^\alpha.
\end{align*}

From this, assuming our dot product \ref{eqn:relativisticElectrodynamicsT1:50} is both linear and symmetric, we have for any pair of spacetime events

\begin{equation}\label{eqn:relativisticElectrodynamicsT1:55}
X \cdot Y = x^0 y^0 - \sum_{\alpha =1}^3 x^\alpha y^\alpha.
\end{equation}

How do our four vectors transform?  This is really just a notational issue, since this has already been discussed.  In this new notation we have

\begin{align}\label{eqn:relativisticElectrodynamicsT1:90}
{x^0}' &= ct' = \gamma ( ct - \beta x) = \gamma ( x^0 - \beta x^1 ) \\
{x^1}' &= x' = \gamma ( x - \beta ct ) = \gamma ( x^1 - \beta x^0 ) \\
{x^2}' &= x^2 \\
{x^3}' &= x^3
\end{align}

where $\beta = V/c$, and $\gamma^{-2} = 1 - \beta^2$.

In order to put some structure to this, it can be helpful to express this dot product as a quadratic form.  We write

\begin{align}\label{eqn:relativisticElectrodynamicsT1:100}
A \cdot B = 
\begin{bmatrix}
a^0 & \Ba^\T 
\end{bmatrix}
\begin{bmatrix}
1 & 0 & 0 & 0 \\
0 & -1 & 0 & 0 \\
0 & 0 & -1 & 0 \\
0 & 0 & 0 & -1 
\end{bmatrix}
\begin{bmatrix}
b^0 \\
\Bb
\end{bmatrix}
= A^\T G B.
\end{align}

We can write our Lorentz boost as a matrix

\begin{equation}\label{eqn:relativisticElectrodynamicsT1:110}
\begin{bmatrix}
\gamma & -\beta \gamma & 0 & 0 \\
-\beta \gamma & \gamma & 0 & 0 \\
0 & 0 & 1 & 0 \\
0 & 0 & 0 & 1 
\end{bmatrix}
\end{equation}

so that the dot product between two transformed four vectors takes the form

\begin{equation}\label{eqn:relativisticElectrodynamicsT1:120}
A' \cdot B' = A^\T O^\T G O B
\end{equation}

\subsection{Back to the problem}

We will work in momentum space, where we have

\begin{align}\label{eqn:relativisticElectrodynamicsT1:130}
p^i &= (p^0, \Bp) = \left( \frac{E}{c}, \Bp\right) \\
p^2 &= \frac{E^2}{c^2} -\Bp^2 \\
\Bp &= \hbar \Bk \\
E &= \hbar \omega \\
p^i &= \hbar k^i \\
k^i &= \left(\frac{\omega}{c}, \Bk\right)
\end{align}

\subsubsection{Justifying this}

Now, Simon (our TA) blurted all this out.  We know some of it from the QM context, and if we have been reading ahead know a bit of this from our text \citep{landau1980classical} (the energy momentum four vector relationships).  Let us go back to the classical electromagnetism and recall what we know about the relation of frequency and wave numbers for continuous fields.  We want solutions to Maxwell's equation in vacuum and can show that such solution also implies that our fields obey a wave equation

\begin{equation}\label{eqn:relativisticElectrodynamicsT1:131}
\inv{c^2} \frac{\partial^2 \Psi}{\partial t^2} - \spacegrad^2 \Psi = 0,
\end{equation}

where $\Psi$ is one of $\BE$ or $\BB$.  We have other constraints imposed on the solutions by Maxwell's equations, but require that they at least obey \ref{eqn:relativisticElectrodynamicsT1:131} in addition to these constraints.

With application of a spatial Fourier transformation of the wave equation, we find that our solution takes the form

\begin{equation}\label{eqn:relativisticElectrodynamicsT1:132}
\Psi = (2 \pi)^{-3/2} \int \tilde{\Psi}(\Bk, 0) e^{i (\omega t \pm \Bk \cdot \Bx) } d^3 \Bk.
\end{equation}

If one takes this as a given and applies the wave equation operator to this as a test solution, one finds without doing the Fourier transform work that we also have a constraint.  That is

\begin{equation}\label{eqn:relativisticElectrodynamicsT1:133}
\inv{c^2} (i \omega)^2 \Psi - (\pm i \Bk)^2 \Psi = 0.
\end{equation}

So even in the continuous field domain, we have a relationship between frequency and wave number.  We see that this also happens to have the form of a lightlike spacetime interval

\begin{equation}\label{eqn:relativisticElectrodynamicsT1:134}
\frac{\omega^2}{c^2} - \Bk^2 = 0.
\end{equation}

Also recall that the photoelectric effect imposes an experimental constraint on photon energy, where we have

\begin{equation}\label{eqn:relativisticElectrodynamicsT1:135}
E = h \nu = \frac{h}{2\pi} 2 \pi \nu = \hbar \omega
\end{equation}

Therefore if we impose a mechanics like $P = (E/c, \Bp)$ relativistic energy-momentum relationship on light, it then makes sense to form a nilpotent (lightlike) four vector for our photon energy.  This combines our special relativistic expectations, with the constraints on the fields imposed by classical electromagnetism.  We can then write for the photon four momentum

\begin{equation}\label{eqn:relativisticElectrodynamicsT1:136}
P = \left( \frac{\hbar \omega}{c}, \hbar k \right)
\end{equation}

\subsubsection{Back to the formula blitz}

We set up the $x'$-axis to be the direction of motion, and we call $\alpha$ the angle from it, or the azimuthal angle.  The wavevector, $\Bk$, is the direction the wave travels. Therefore, if we want to find the angle the radiation makes to the direction of motion, you need the projection of the wavevector onto the $x$-axis, or $k^1/\Abs{\Bk}$. In other words, imagine a piece of radiation emitted in a certain direction, the angle it makes with the $x'$-axis is the cosine of the projection on the $x'$-axis over the magnitude.

This azimuthal angle in the unprimed frame is

\begin{equation}\label{eqn:relativisticElectrodynamicsT1:140} 
\cos \alpha = \frac{k^1}{\Abs{\Bk}} = \frac{k^1}{\omega/c},
\end{equation}

In the observer's reference frame (the primed coordinates), the source is moving in the $+x$ direction, and therefore, we must boost in the $-x$ from the source's frame, or $-\beta$.  Transforming out wave four vector in the same fashion as regular mechanical position and momentum four vectors, we have for the observer

\begin{equation}\label{eqn:relativisticElectrodynamicsT1:140b} 
\cos \alpha' = \frac{{k^1}'}{\omega'/c} = \frac{\gamma (k^1 + \beta \omega/c)}{\gamma(\omega/c + \beta k^1)}
\end{equation}

%Also note that we have the primed frame moving negatively along the x-axis, instead of the usual positive origin shift.  The question is vague enough to allow this since it only requires motion.

\paragraph{check 1}

as $\beta \rightarrow 1$ (ie: our primed frame velocity approaches the speed of light relative to the rest frame), $\cos \alpha' \rightarrow 1$, $\alpha' = 0$.  The surface gets more and more compressed.

In the original reference frame the radiation was isotropic.  In the new frame how does it change with respect to the angle?  This is really a question to find this number flux rate

\begin{equation}\label{eqn:relativisticElectrodynamicsT1:150}
f'(\alpha') = ?
\end{equation}

In our rest frame the total number of photons traveling through the surface in a given interval of time is

\begin{align}\label{eqn:relativisticElectrodynamicsT1:160}
N &= \int d\Omega dt f(\alpha) = \int d \phi \sin \alpha d\alpha = -2 \pi \int d(\cos\alpha) dt f(\alpha) \\
\end{align}

Here we utilize the spherical solid angle $d\Omega = \sin \alpha d\alpha d\phi = - d(\cos\alpha) d\phi$, and integrate $\phi$ over the $[0, 2\pi]$ interval.  We also have to assume that our number flux density is not a function of horizontal angle $\phi$ in the rest frame.

In the moving frame we similarly have
\begin{align}\label{eqn:relativisticElectrodynamicsT1:160b}
N' &= -2 \pi \int d(\cos\alpha') dt' f'(\alpha'),
\end{align}

and we again have had to assume that our transformed number flux density is not a function of the horizontal angle $\phi$.  This seems like a reasonable move since ${k^2}' = k^2$ and ${k^3}' = k^3$ as they are perpendicular to the boost direction.

\begin{equation}\label{eqn:relativisticElectrodynamicsT1:170}
f'(\alpha') = \frac{d(\cos\alpha)}{d(\cos\alpha')} \left( \frac{dt}{dt'} \right) f(\alpha)
\end{equation}


Now, utilizing a conservation of mass argument, we can argue that $N = N'$.  Regardless of the motion of the frame, the same number of particles move through the surface.  Taking ratios, and examining an infinitesimal time interval, and the associated flux through a small patch, we have

\begin{equation}\label{eqn:relativisticElectrodynamicsT1:180}
\left( \frac{d(\cos\alpha)}{d(\cos\alpha')} \right) = \left( \frac{d(\cos\alpha')}{d(\cos\alpha)} \right)^{-1} = \gamma^2 ( 1 + \beta \cos\alpha)^2
\end{equation}

Part of the statement above was a do-it-yourself.  First recall that $c t' = \gamma ( c t + \beta x )$, so $dt/dt'$ evaluated at $x=0$ is $1/\gamma$.

The rest is messier.  We can calculate the $d(\cos)$ values in the ratio above using \ref{eqn:relativisticElectrodynamicsT1:140}.  For example, for $d(\cos(\alpha))$ we have

\begin{align*}
d(\cos\alpha) 
&= d \left( \frac{k^1}{\omega/c} \right) \\
&= dk^1 \inv{\omega/c} - c \inv{\omega^2} d\omega.
\end{align*}

If one does the same thing for $d(\cos\alpha')$, after a whole whack of messy algebra one finds that the differential terms and a whole lot more mystically cancels, leaving just

\begin{equation}\label{eqn:relativisticElectrodynamicsT1:171}
\frac{d\cos\alpha'}{d\cos\alpha} = \frac{\omega^2/c^2}{(\omega/c + \beta k^1)^2} (1 - \beta^2)
\end{equation}

A bit more reduction with reference back to \ref{eqn:relativisticElectrodynamicsT1:140b} verifies \ref{eqn:relativisticElectrodynamicsT1:180}.

Also note that again from \ref{eqn:relativisticElectrodynamicsT1:140b} we have

\begin{equation}\label{eqn:relativisticElectrodynamicsT1:190a}
\cos\alpha' = \frac{\cos\alpha + \beta}{1 + \beta \cos\alpha}
\end{equation}

and rearranging this for $\cos\alpha'$ gives us
\begin{equation}\label{eqn:relativisticElectrodynamicsT1:190}
\cos\alpha = \frac{\cos\alpha' - \beta}{1 - \beta \cos\alpha'},
\end{equation}

which we can sum to find that 

\begin{equation}\label{eqn:relativisticElectrodynamicsT1:190b}
1 + \beta \cos\alpha = \inv{\gamma^2 (1 - \beta \cos \alpha') },
\end{equation}

so putting all the pieces together we have

\begin{equation}\label{eqn:relativisticElectrodynamicsT1:200}
f'(\alpha') = \inv{\gamma^3} \frac{f(\alpha)}{(1-\beta \cos\alpha')^2}
\end{equation}

The question asks for the energy flux density.  We get this by multiplying the number density by the frequency of the light in question.  This is, as a function of the polar angle, in each of the frames.

\begin{align}\label{eqn:relativisticElectrodynamicsT1:210}
L(\alpha) &= \hbar \omega(\alpha) f(\alpha) = \hbar \omega f \\
L'(\alpha') &= \hbar \omega'(\alpha') f'(\alpha') = \hbar \omega' f'
\end{align}

But we have
\begin{equation}\label{eqn:relativisticElectrodynamicsT1:220}
\omega'(\alpha')/c = \gamma( \omega/c + \beta k^1 ) = \gamma \omega/c ( 1 + \beta \cos\alpha )
\end{equation}

Aside, $\beta << 1$, 

\begin{equation}\label{eqn:relativisticElectrodynamicsT1:230}
\omega' = \omega ( 1 + \beta \cos\alpha) + O(\beta^2) = \omega + \delta \omega
\end{equation}

\begin{align}\label{eqn:relativisticElectrodynamicsT1:240}
\delta \omega &= \beta, \alpha = 0 		\qquad \text{blue shift} \\
\delta \omega &= -\beta, \alpha = \pi 		\qquad \text{red shift}
\end{align}

The energy flux density in the unprimed observer frame is now found to be

\begin{equation}\label{eqn:relativisticElectrodynamicsT1:241}
L'(\alpha') = \frac{L/\gamma}{(\gamma (1 - \beta \cos\alpha'))^3}
\end{equation}

And the forward backward ratio is

\begin{equation}\label{eqn:relativisticElectrodynamicsT1:250}
L'(0)/L'(\pi) = {\left( \frac{ 1 + \beta }{1-\beta} \right)}^3,
\end{equation}

allowing us to conclude that the forward radiation is bigger than the backwards radiation (and much bigger when the motion approaches the speed of light).

%
% Copyright � 2012 Peeter Joot.  All Rights Reserved.
% Licenced as described in the file LICENSE under the root directory of this GIT repository.
%

%\chapter{Four vectors and tensors}
\label{chap:relativisticElectrodynamicsL6}
%\blogpage{http://sites.google.com/site/peeterjoot/math2011/relativisticElectrodynamicsL6.pdf}
%\date{Jan 25, 2011}

\paragraph{Reading}

Still covering chapter 1 material from the text \citep{landau1980classical}.

Covering \href{http://www.physics.utoronto.ca/~poppitz/epoppitz/PHY450_files/RelEM27-44.pdf}{Professor Poppitz's lecture notes}: nonrelativistic limit of boosts (33); number of parameters of Lorentz transformations (34-35); introducing four-vectors, the metric tensor, the invariant ``dot-product and SO(1,3) (36-40); the Poincare group (41); the convenience of ``upper'' and ``lower''indices (42-43); tensors (44) 

\section{The Special Orthogonal group (for Euclidean space)}

Lorentz transformations are like ``rotations'' for \((t, x, y, z)\) that preserve \((ct)^2 - x^2 - y^2 - z^2\).  There are 6 continuous parameters:

\begin{itemize}
\item 3 rotations in \(x,y,z\) space
\item 3 ``boosts'' in \(x\) or \(y\) or \(z\).
\end{itemize}

For rotations of space we talk about a group of transformations of 3D Euclidean space, and call this the \(S0(3)\) group.  Here \(S\) is for Special, \(O\) for Orthogonal, and \(3\) for the dimensions.

For a transformed vector in 3D space we write

\begin{equation}\label{eqn:relativisticElectrodynamicsL6:10}
\begin{bmatrix}
x \\
y \\
z
\end{bmatrix} 
\rightarrow 
\begin{bmatrix}
x \\
y \\
z
\end{bmatrix}' = O 
\begin{bmatrix}
x \\
y \\
z
\end{bmatrix}.
\end{equation}

Here \(O\) is an orthogonal \(3 \times 3\) matrix, and has the property

\begin{equation}\label{eqn:relativisticElectrodynamicsL6:11}
O^T O = \BOne.
\end{equation}

Taking determinants, we have

\begin{equation}\label{eqn:relativisticElectrodynamicsL6:12}
\det{ O^T } \det{ O} = 1,
\end{equation}

and since \(\det{O^\T} = \det{ O }\), we have

\begin{equation}\label{eqn:relativisticElectrodynamicsL6:13}
(\det{O})^2 = 1,
\end{equation}

so our determinant must be 
\begin{equation}\label{eqn:relativisticElectrodynamicsL6:14}
\det O = \pm 1.
\end{equation}

We work with the positive case only, avoiding the transformations that include reflections.

The Unitary condition \(O^\T O = 1\) is an indication that the inner product is preserved.  Observe that in matrix form we can write the inner product

\begin{equation}\label{eqn:relativisticElectrodynamicsL6:15}
\Br_1 \cdot \Br_2 = 
\begin{bmatrix}
x_1 & y_1 & z_1
\end{bmatrix}
\begin{bmatrix}
x_1 \\
y_2 \\
x_3 \\
\end{bmatrix}.
\end{equation}

For a transformed vector \(X' = O X\), we have \({X'}^\T = X^\T O^\T\), and

\begin{equation}\label{eqn:relativisticElectrodynamicsL6:16}
X' \cdot X' = (X^\T O^\T) (O X) = X^\T (O^\T O) X = X^T X = X \cdot X
\end{equation}

\section{The Special Orthogonal group (for spacetime)}

This generalizes to Lorentz boosts!  There are two differences

\begin{enumerate}
\item Lorentz transforms should be \(4 \times 4\) not \(3 \times 3\) and act in \((ct, x, y, z)\), and NOT \((x,y,z)\).
\item They should leave invariant NOT \(\Br_1 \cdot \Br_2\), but \(c2 t_2 t_1 - \Br_2 \cdot \Br_1\).
\end{enumerate}

Do not get confused that I demanded \(c^2 t_2 t_1 - \Br_2 \cdot \Br_1 = \text{invariant}\) rather than \(c^2 (t_2 - t_1)^2 - (\Br_2 - \Br_1)^2 = \text{invariant}\).  Expansion of this (squared) interval, provides just this four vector dot product and its invariance condition

\begin{equation}\label{eqn:relativisticElectrodynamicsL6:260}
\begin{aligned}
\text{invariant} 
&=
c^2 (t_2 - t_1)^2 - (\Br_2 - \Br_1)^2 \\
&=
(c^2 t_2^2 - \Br_2^2) + (c^2 t_2^2 - \Br_2^2)
- 2 c^2 t_2 t_1 + 2 \Br_1 \cdot \Br_2.
\end{aligned}
\end{equation}

Observe that we have the sum of two invariants plus our new cross term, so this cross term, (-2 times our dot product to be defined), must also be an invariant.

\paragraph{Introduce the four vector}

\begin{equation}\label{eqn:relativisticElectrodynamicsL6:280}
\begin{aligned}
x^0 &= ct \\
x^1 &= x \\
x^2 &= y \\
x^3 &= z 
\end{aligned}
\end{equation}

Or \((x^0, x^1, x^2, x^3) = \{ x^i, i = 0,1,2,3 \}\).

We will also write

\begin{equation}\label{eqn:relativisticElectrodynamicsL6:300}
\begin{aligned}
x^i &= (ct, \Br) \\
\tilde{x}^i &= (c\tilde{t}, \tilde{\Br})
\end{aligned}
\end{equation}

Our inner product is

\begin{equation}\label{eqn:relativisticElectrodynamicsL6:20}
c^2 t \tilde{t} - \Br \cdot \tilde{\Br}
\end{equation}

Introduce the \(4 \times 4\) matrix 

% used double bar abs (norm) here
\begin{equation}\label{eqn:relativisticElectrodynamicsL6:30}
\Norm{g_{ij}} = 
\begin{bmatrix}
1 & 0 & 0 & 0 \\
0 & -1 & 0 & 0 \\
0 & 0 & -1 & 0 \\
0 & 0 & 0 & -1 \\
\end{bmatrix}
\end{equation}

This is called the Minkowski spacetime metric.

Then 

\begin{equation}\label{eqn:relativisticElectrodynamicsL6:320}
\begin{aligned}
c^2 t \tilde{t} - \Br \cdot \tilde{\Br}
&\equiv \sum_{i, j = 0}^3 \tilde{x}^i g_{ij} x^j \\
&= \sum_{i, j = 0}^3 \tilde{x}^i g_{ij} x^j \\
& 
\tilde{x}^0 x^0 
-\tilde{x}^1 x^1 
-\tilde{x}^2 x^2 
-\tilde{x}^3 x^3 
\end{aligned}
\end{equation}

\paragraph{Einstein summation convention}.  Whenever indices are repeated that are assumed to be summed over.

We also write

\begin{equation}\label{eqn:relativisticElectrodynamicsL6:40}
X = 
\begin{bmatrix}
x^0 \\
x^1 \\
x^2 \\
x^3 \\
\end{bmatrix}
\end{equation}
\begin{equation}\label{eqn:relativisticElectrodynamicsL6:50}
\tilde{X} = 
\begin{bmatrix}
\tilde{x}^0 \\
\tilde{x}^1 \\
\tilde{x}^2 \\
\tilde{x}^3 \\
\end{bmatrix}
\end{equation}

\begin{equation}\label{eqn:relativisticElectrodynamicsL6:60}
G = 
\begin{bmatrix}
1 & 0 & 0 & 0 \\
0 & -1 & 0 & 0 \\
0 & 0 & -1 & 0 \\
0 & 0 & 0 & -1 \\
\end{bmatrix}
\end{equation}

Our inner product 

\begin{equation}\label{eqn:relativisticElectrodynamicsL6:340}
\begin{aligned}
c^2 t \tilde{t} - \tilde{\Br} \cdot \Br 
&= \tilde{X}^\T G X \\
&=
\begin{bmatrix}
\tilde{x}^0 & \tilde{x}^1 & \tilde{x}^2 & \tilde{x}^3 
\end{bmatrix}
\begin{bmatrix}
1 & 0 & 0 & 0 \\
0 & -1 & 0 & 0 \\
0 & 0 & -1 & 0 \\
0 & 0 & 0 & -1 \\
\end{bmatrix}
\begin{bmatrix}
\tilde{x}^0 \\
\tilde{x}^1 \\
\tilde{x}^2 \\
\tilde{x}^3 \\
\end{bmatrix}
\end{aligned}
\end{equation}

Under Lorentz boosts, we have

\begin{equation}\label{eqn:relativisticElectrodynamicsL6:70}
X = \hat{O} X',
\end{equation}

where 

\begin{equation}\label{eqn:relativisticElectrodynamicsL6:80}
\hat{O} =
\begin{bmatrix}
\gamma & - \gamma v_x/c  & 0 & 0 \\
- \gamma v_x/c & \gamma  & 0 & 0 \\
0 & 0 & 1 & 0 \\
0 & 0 & 0 & 1 
\end{bmatrix}
\end{equation}

(for x-direction boosts)

\begin{equation}\label{eqn:relativisticElectrodynamicsL6:90}
\begin{aligned}
\tilde{X} &= \hat{O} {\tilde{X}}' \\
{\tilde{X}}^\T &= {{\tilde{X'}}}^\T {\hat{O}}^\T
\end{aligned}
\end{equation}

But \(\hat{O}\) must be such that \(\tilde{X}^\T G X\) is invariant.  i.e.

\begin{equation}\label{eqn:relativisticElectrodynamicsL6:100}
\tilde{X}^\T G X = {\tilde{X'}}^\T (\hat{O}^\T G \hat{O}) X' = {X'}^\T (G) X' \qquad \mbox{\(\forall X'\) and \(\tilde{X}'\)}
\end{equation}

This implies

\boxedEquation{eqn:relativisticElectrodynamicsL6:110}{
\hat{O}^\T G \hat{O} = G
}

Such \(\hat{O}\)'s are called ``pseudo-orthogonal''.

Lorentz transformations are represented by the set of all \(4 \times 4\) pseudo-orthogonal matrices.

In symbols

\begin{equation}\label{eqn:relativisticElectrodynamicsL6:120}
\hat{O}^T G \hat{O} = G
\end{equation}

Just as before we can take the determinant of both sides.  Doing so we have

\begin{equation}\label{eqn:relativisticElectrodynamicsL6:130}
\det(\hat{O}^T G \hat{O}) = \det(\hat{O}^T) \det(G) \det(\hat{O}) = \det(G)
\end{equation}

The \(\det(G)\) terms cancel, and since \(\det(\hat{O}^T) = \det(\hat{O})\), this leaves us with \((\det(\hat{O}))^2 = 1\), or

\begin{equation}\label{eqn:relativisticElectrodynamicsL6:140}
\det(\hat{O}) = \pm 1
\end{equation}

We take the \(\det 0 = +1\) case only, so that the transformations do not change orientation (no reflection in space or time).  This set of transformation forms the group

\begin{equation*}
SO(1,3)
\end{equation*}

Special orthogonal, one time, 3 space dimensions.  Note that when the \(-1\) determinant is also allowed the group is called the \(O(1,3)\) set of transformations.

Einstein relativity can be defined as the ``laws of physics that leave four vectors invariant in the

\begin{equation*}
SO(1,3) \times T^4
\end{equation*}

symmetry group.

Here \(T^4\) is the group of translations in spacetime with 4 continuous parameters.   The complete group of transformations that form the group of relativistic physics has \(10 = 3 + 3 + 4\) continuous parameters.

This group is called the Poincare group of symmetry transforms.

\section{Lower index notation}

Our inner product is written

\begin{equation}\label{eqn:relativisticElectrodynamicsL6:150}
\tilde{x}^i g_{ij} x^j
\end{equation}

but this is very cumbersome.  The convenient way to write this is instead

\begin{equation}\label{eqn:relativisticElectrodynamicsL6:160}
\tilde{x}^i g_{ij} x^j = \tilde{x}_j x^j = \tilde{x}^i x_i
\end{equation}

where 

\begin{equation}\label{eqn:relativisticElectrodynamicsL6:170}
x_i = g_{ij} x^j = g_{ji} x^j
\end{equation}

Note: A check that we should always be able to make.  Indexes that are not summed over should be conserved.  So in the above we have a free \(i\) on the LHS, and should have a non-summed \(i\) index on the RHS too (also lower matching lower, or upper matching upper).

Non-matched indices are bad in the same sort of sense that an expression like

\begin{equation}\label{eqn:relativisticElectrodynamicsL6:180}
\Br = 1
\end{equation}

is not well defined (assuming a vector space \(\Br\) and not a multivector Clifford algebra that is;)

Expanded out explicitly (noting that all off diagonal terms of the metric tensor are zero):

\begin{equation}\label{eqn:relativisticElectrodynamicsL6:360}
\begin{aligned}
x_0 &= g_{0 0} x^0 = ct  \\
x_1 &= g_{1 j} x^j = g_{11} x^1 = -x^1 \\
x_2 &= g_{2 j} x^j = g_{22} x^2 = -x^2 \\
x_3 &= g_{3 j} x^j = g_{33} x^3 = -x^3
\end{aligned}
\end{equation}

We would not have objects of the form 

\begin{equation}\label{eqn:relativisticElectrodynamicsL6:190}
x^i x^i = (ct)^2 + \Br^2
\end{equation}

for example.  This is not a Lorentz invariant quantity.

\paragraph{Lorentz scalar example:} \(\tilde{x}^i x_i\)
\paragraph{Lorentz vector example:} \(x^i\)

This last is also called a rank-1 tensor.

Lorentz rank-2 tensors: ex: \(g_{ij}\)

or other 2-index objects.

Why in the world would we ever want to consider two index objects.  We are not just trying to be hard on ourselves.  Recall from classical mechanics that we have a two index object, the inertial tensor.

In mechanics, for a rigid body we had the energy

\begin{equation}\label{eqn:relativisticElectrodynamicsL6:200}
T = \sum_{ij = 1}^3 \Omega_i I_{ij} \Omega_j
\end{equation}

The inertial tensor was this object 

\begin{equation}\label{eqn:relativisticElectrodynamicsL6:210}
I_{ij} = \sum_{a = 1}^N m_a \left(\delta_{ij} \Br_a^2 - r_{a_i} r_{a_j} \right)
\end{equation}

or for a continuous body

\begin{equation}\label{eqn:relativisticElectrodynamicsL6:220}
I_{ij} = \int \rho(\Br) \left(\delta_{ij} \Br^2 - r_{i} r_{j} \right)
\end{equation}

In electrostatics we have the quadrupole tensor, ... and we have other such objects all over physics.

Note that the energy \(T\) of the body above cannot depend on the coordinate system in use.  This is a general property of tensors.  These are object that transform as products of vectors, as \(I_{ij}\) does.  

We call \(I_{ij}\) a rank-2 3-tensor.  rank-2 because there are two indices, and 3 because the indices range from \(1\) to \(3\).

The point is that tensors have the property that the transformed tensors transform as

\begin{equation}\label{eqn:relativisticElectrodynamicsL6:230}
I_{ij}' = \sum_{l, m = 1,2,3} O_{il} O_{jm} I_{lm}
\end{equation}

%FIXME: show this based on the definition above of \(I_{ij}\).
Another example: the completely antisymmetric rank 3, 3-tensor

\begin{equation}\label{eqn:relativisticElectrodynamicsL6:240}
\epsilon_{ijk}
\end{equation}


%
% Copyright � 2012 Peeter Joot.  All Rights Reserved.
% Licenced as described in the file LICENSE under the root directory of this GIT repository.
%

%\chapter{Action and relativistic dynamics}
\label{chap:relativisticElectrodynamicsL7}
%\blogpage{http://sites.google.com/site/peeterjoot/math2011/relativisticElectrodynamicsL7.pdf}
%\date{Jan 26, 2011}

\paragraph{Reading}

Covering chapter 2 material from the text \citep{landau1980classical}, and
\popcite{RelEMpp52-56.pdf}{lecture notes RelEMpp52-56.pdf},
%: equation of motion, symmetries, and conserved quantities (energy-momentum 4 vector) from relativistic particle action [Wednesday, Jan. 26, Tuesday, Feb. 1]
and \popcite{RelEMp53.1.pdf}{RelEMp53.1.pdf}.
%, containing some additional notes completing an argument on page 53.

\section{The relativity principle}

The relativity principle implies that the EOM should be expressed in 4-vector form, just like Newton's EOM are expressed in 3-vector form

\begin{equation}\label{eqn:relativisticElectrodynamicsL7:10}
m \ddot{\Br} = \Bf
\end{equation}

Observe that in coordinate form this is
\begin{equation}\label{eqn:relativisticElectrodynamicsL7:20}
m \ddot{r}^i = f^i, \qquad i = 1,2,3
\end{equation}

or for a rotated frame \(O'\)
\begin{equation}\label{eqn:relativisticElectrodynamicsL7:30}
m \ddot{r'}^i = {f'}^i, \qquad i = 1,2,3
\end{equation}

Need to generalize to 4 vectors, so we need 4-velocity and 4-acceleration.

Later we will study action and Lagrangian, and then relativity will require that the action be a Lorentz scalar.  The analogy for a Newtonian point particle is a scalar under rotations.

\paragraph{Four vector velocity}

\paragraph{Definition:} Velocity s the rate of change of position in \((ct, \Bx)\)-space.  Position means specifying both \(ct\) and \(\Bx\) for a point in spacetime.

PICTURE: \(x^0 = ct\) axis up, and \(x^1, x^2, x^3\) axis over, with worldline \(x = x(\tau)\).  Here \(\tau\) is a parameter for the worldline, and provides a mapping for the curve in spacetime.

PICTURE: 3D vectors, \(\Br(t)\), \(\Br(t + \Delta t)\), and the difference vector \(\Br(t + \Delta t) - \Br(t)\).

We write

\begin{equation}\label{eqn:relativisticElectrodynamicsL7:40}
\Bv(t) \equiv \lim_{\Delta t \rightarrow 0} \frac{\Br(t + \Delta t) - \Br(t)}{ \Delta t}
\end{equation}

For four vectors we will parametrize the worldline by its ``length'', with \(O\) taken from some arbitrary point on it.  We can also take \(\tau\) to be the proper time, and the only difference will be the factor of \(c\) (which becomes especially easy with the choice \(c=1\) that is avoided in this class).

\begin{equation}\label{eqn:relativisticElectrodynamicsL7:50}
\frac{x^i(\tau + \Delta \tau) - x^i(\tau)}{\Delta \tau}
\end{equation}

We will take the limit

\begin{equation}\label{eqn:relativisticElectrodynamicsL7:60}
\frac{dx^i}{d\tau} =
\lim_{\Delta \tau \rightarrow 0} 
\frac{x^i(\tau + \Delta \tau) - x^i(\tau)}{\Delta \tau}
\end{equation}

and then define a dimensionless ``proper velocity''

\begin{equation}\label{eqn:relativisticElectrodynamicsL7:61}
u^i \equiv \inv{c} \frac{dx^i}{d\tau} = \frac{dx^i}{ds}.
\end{equation}

This is a nice quantity, we are dividing a vector by a Lorentz scalar, and thus get a four vector as a result (i.e. the result transforms as a four vector).

PICTURE: small fragment of a worldline with constant slope over the infinitesimal interval.  \(dx^0\) up and \(dx^1\) over.

%\begin{equation}\label{eqn:relativisticElectrodynamicsL7:70}
%u^i \equiv \frac{dx^i}{ds}
%\end{equation}

\begin{equation}\label{eqn:relativisticElectrodynamicsL7:290}
\begin{aligned}
ds^2 
&= (dx^0)^2 - (dx^1)^2 \\
&= c^2 \left( (dt)^2 - \inv{c^2} (dx^1)^2 \right) \\
&= c^2 (dt)^2 \left( 1 - \inv{c^2} \frac{dx^1}{dt^2} \right) 
\end{aligned}
\end{equation}

Or 

\begin{equation}\label{eqn:relativisticElectrodynamicsL7:90}
\begin{aligned}
ds = c dt \sqrt{1 - \inv{c^2} \frac{dx^1}{dt^2} }
\end{aligned}
\end{equation}

NOTE: Prof admits pulling a fast one, since he has aligned the worldline along the \(x^1\) axis, however this is always possible by rotating the coordinate system.

\begin{equation}\label{eqn:relativisticElectrodynamicsL7:310}
\begin{aligned}
u^0 
&= \frac{dx^0}{ds} \\
&= \frac{c dt}{ c dt \sqrt{ 1 - \Bv^2/c^2} } \\
&= \frac{1}{ \sqrt{ 1 - \Bv^2/c^2} } \\
&= \gamma
\end{aligned}
\end{equation}

\begin{equation}\label{eqn:relativisticElectrodynamicsL7:330}
\begin{aligned}
u^1 
&= \frac{dx^1}{ds} \\
&= \frac{dx^1 }{ c dt \sqrt{ 1 - \Bv^2/c^2} } \\
&= \frac{v^1/c}{ \sqrt{ 1 - \Bv^2/c^2} } \\
&= \gamma \frac{v^1}{c}
\end{aligned}
\end{equation}

Similarly
\begin{equation}\label{eqn:relativisticElectrodynamicsL7:350}
\begin{aligned}
u^2 &= \gamma \frac{v^2}{c} \\
u^3 &= \gamma \frac{v^2}{c}
\end{aligned}
\end{equation}

We have now unpacked the four velocity, and have

\begin{equation}\label{eqn:relativisticElectrodynamicsL7:100}
u^i = \left( \gamma, \frac{\Bv}{c} \gamma \right)
\end{equation}

\paragraph{Length of the four velocity vector}

Recall that this length is

\begin{equation}\label{eqn:relativisticElectrodynamicsL7:370}
\begin{aligned}
u^i g_{ij} u^j 
&= u^i u_i  \\
&= u_i u^i  \\
&= (u^0)^2 - (u_i)^2 \\
&= \gamma^2 - \gamma^2 \frac{\Bv}{c} \cdot \frac{\Bv}{c} \\
&= \gamma^2 \left(1 - \frac{\Bv^2}{c^2} \right)
\end{aligned}
\end{equation}

The four velocity in physics is
\begin{equation}\label{eqn:relativisticElectrodynamicsL7:110}
u^i = \left( \gamma, \frac{\Bv}{c} \gamma \right)
\end{equation}

but in mathematics the meaning of \(u^i u_i = 1\) means that this quantity is the unit tangent vector to the worldline.

\paragraph{Four acceleration}

In Newtonian physics we have 

\begin{equation}\label{eqn:relativisticElectrodynamicsL7:111}
\Ba = \frac{d\Bv}{dt}
\end{equation}

Our relativistic mapping of this, with \(v \rightarrow u^i\) and \(t \rightarrow s\), gives

\begin{equation}\label{eqn:relativisticElectrodynamicsL7:120}
w^i = \frac{d u^i}{ds}
\end{equation}

Geometrically \(w^i\) is the normal to the worldline.  This follows from \(u^i g_{ij} u^j = 1\), so

\begin{equation}\label{eqn:relativisticElectrodynamicsL7:390}
\begin{aligned}
\frac{d}{ds} \left( u^i g_{ij} u^j \right) 
&=
\frac{d u^i}{ds} g_{ij} u^j 
+u^i g_{ij} \frac{d u^j}{ds} \\
&=
\frac{d u^i}{ds} g_{ij} u^j 
+u^j 
\mathLabelBox
[
   labelstyle={xshift=2cm},
   linestyle={out=270,in=90, latex-}
]
{g_{ji}}{\(= g_{ij}\)} \frac{d u^i}{ds} \\
&=
\frac{d u^i}{ds} g_{ij} u^j 
+u^j g_{ji} \frac{d u^i}{ds} \\
&=
2 \frac{d u^i}{ds} g_{ij} u^j 
\end{aligned}
\end{equation}

Note that we have utilized the fact above that the dummy summation indices can be swapped (or changed to anything else we feel inclined to use).

The conclusion is that the dot product of the acceleration and the velocity is zero

\begin{equation}\label{eqn:relativisticElectrodynamicsL7:130}
w_i u^i = 0.
\end{equation}

\section{Relativistic action}

\begin{equation}\label{eqn:relativisticElectrodynamicsL7:140}
S_{ab} = ?
\end{equation}

What is the action for a worldline from \(a \rightarrow b\).

We want something that has velocity dependence (\(u^i\) not \(\Bv\)), but that is Lorentz invariant and has only first derivatives.

The relativistic length is the simplest so we could form

\begin{equation}\label{eqn:relativisticElectrodynamicsL7:150}
\int ds u^i u_i
\end{equation}

but that is not interesting since \(u^i u_i = 1\).  We could form

\begin{equation}\label{eqn:relativisticElectrodynamicsL7:160}
\int ds u^i \frac{u_i}{ds} = \int ds w^i u_i
\end{equation}

but then this is just zero.

We could form something like

\begin{equation}\label{eqn:relativisticElectrodynamicsL7:170}
\int ds \frac{w^i}{ds} u_i
\end{equation}

This is non zero and non-constant, but evaluating the EOM for such an action would produce a result that has higher than second order derivatives.

We are left with
\begin{equation}\label{eqn:relativisticElectrodynamicsL7:180}
S_{ab} = \text{constant} \int_a^b ds 
\end{equation}

To fix this constant we note that if we want to minimize the action over the infinitesimal interval, then we need a minus sign.  Since the Lagrangian has dimensions of energy, and the dimensions of energy times time are momentum, our action must then have dimensions of momentum.  So one possible constant that fixes up our dimensions is \(mc\).  Construct an action with the following form

\begin{equation}\label{eqn:relativisticElectrodynamicsL7:190}
S_{ab} = - m c\int_a^b ds,
\end{equation}

does the job we want.  Here ``m'' is a characteristic of the particle, which \textunderline{is a Lorentz scalar}.  It also happens to have dimensions of mass.  With \(ds = c dt \sqrt{1 - \Bv^2/c^2}\), we have

\begin{equation}\label{eqn:relativisticElectrodynamicsL7:200}
S_{ab} = - m c^2 \int_{t_a}^{t_b} dt \sqrt{ 1 - \inv{c^2} \left( \frac{d \Bx(t) }{dt} \right)^2 }
\end{equation}

Now everything looks like it was in classical mechanics.

\begin{equation}\label{eqn:relativisticElectrodynamicsL7:210}
S_{ab} = \int_{t_a}^{t_b} \LL(\dot{\Bx}(t)) dt
\end{equation}
\begin{equation}\label{eqn:relativisticElectrodynamicsL7:220}
\LL(\dot{\Bx}(t)) = -m c^2 
\end{equation}

Now find the extremum of \(S\).  That problem is really to compute the variation in the action that results from varying the coordinates around the stationary point, and equate that variation to zero to find the extremum

\begin{equation}\label{eqn:relativisticElectrodynamicsL7:230}
\delta S = S[\Bx(t) + \delta \Bx(t)] - S[ \Bx(t) ] = 0
\end{equation}

The usual condition is imposed where we have zero variation of the coordinates at the boundaries of the action integral

\begin{equation}\label{eqn:relativisticElectrodynamicsL7:231}
0 = \delta \Bx(t_a) = \delta \Bx(t_b) 
\end{equation}

Returning to our action we have

\begin{equation}\label{eqn:relativisticElectrodynamicsL7:240}
\frac{d}{dt} \PD{\dot{\Bx}}{\LL} = \PD{\Bx}{\LL} = 0
\end{equation}

This last is zero because it is a free particle with no position dependence.

\begin{equation}\label{eqn:relativisticElectrodynamicsL7:410}
\begin{aligned}
0 
&= -m c^2 \frac{d}{dt} \PD{\dot{\Bx}}{} \sqrt{ 1 - \dot{\Bx}^2 } \\
&= -m c^2 \frac{d}{dt} \frac{- \dot{\Bx}}{\sqrt{ 1 - \dot{\Bx}^2 } } \\
&= m c^2 \frac{d}{dt} \gamma \dot\Bx
\end{aligned}
\end{equation}

So we have

\begin{equation}\label{eqn:relativisticElectrodynamicsL7:241}
\frac{d}{dt} (\gamma \dot{\Bx}) = 0
\end{equation}

By evaluating this, we can eventually show that we can construct a four vector equation.  Doing this we have

\begin{equation}\label{eqn:relativisticElectrodynamicsL7:430}
\begin{aligned}
\frac{d}{dt} (\gamma \Bv) 
&=
\frac{d}{dt} \left( \left(1 - \Bv^2/c^2\right)^{-1/2} \Bv \right) \\
&=
-2 (-1/2) \Bv (\Bv \cdot \dot{\Bv})/c^2 \left(1 - \Bv^2/c^2\right)^{-3/2} + \left(1 - \Bv^2/c^2\right)^{-1/2} \dot{\Bv} \\
&=
\gamma \left( \frac{\Bv (\Bv \cdot \dot{\Bv}) }{ c^2 - \Bv^2 } + \dot{\Bv} \right)
\end{aligned}
\end{equation}

Or
\begin{equation}\label{eqn:relativisticElectrodynamicsL7:242}
\frac{\Bv (\Bv \cdot \dot{\Bv}) }{ c^2 - \Bv^2 } + \dot{\Bv} = 0
\end{equation}

Clearly \(\dot{\Bv} = 0\) is a solution, but is it the only solution?

By dotting this with \(\Bv\) we have
\begin{equation}\label{eqn:relativisticElectrodynamicsL7:450}
\begin{aligned}
0 
&= \frac{\Bv^2 (\Bv \cdot \dot{\Bv}) }{ c^2 - \Bv^2 } + \dot{\Bv} \cdot \Bv  \\
&= (\Bv \cdot \dot{\Bv}) \left( 1 + \frac{\Bv^2}{c^2 - \Bv^2} \right) \\
&= (\Bv \cdot \dot{\Bv}) \frac{c^2}{c^2 - \Bv^2} 
\end{aligned}
\end{equation}

This implies that \(\dot{\Bv} = 0\) (a contraction) or that \(\Bv \cdot \dot{\Bv} = 0\).  To examine the perpendicularity question, let us take cross products.  This gives

\begin{equation}\label{eqn:relativisticElectrodynamicsL7:243}
0 =
\frac{(\Bv \cross \Bv) (\Bv \cdot \dot{\Bv}) }{ c^2 - \Bv^2 } + \dot{\Bv} \cross \Bv
\end{equation}

We have found that \(\Bv \cdot \dot{\Bv} = 0\) and \(\Bv \cross \dot{\Bv} = 0\).  This can only mean that \(\dot{\Bv} = 0\), contradicting the assumption that is non-zero.  We conclude that \(\dot{\Bv} = 0\) is the only solution to \eqnref{eqn:relativisticElectrodynamicsL7:242}.

\section{Next time}

We want to finish up and show how this results in a four velocity equation.  We have

\begin{equation}\label{eqn:relativisticElectrodynamicsL7:250}
\frac{d}{dt} ( \gamma \Bv) = 0
\end{equation}

which is

\begin{equation}\label{eqn:relativisticElectrodynamicsL7:260}
\frac{d}{dt} ( u^\alpha ) = 0, \qquad \mbox{for \(u^\alpha = u^1, u^2, u^3\)}
\end{equation}

eventually, we will show that we also have

\begin{equation}\label{eqn:relativisticElectrodynamicsL7:270}
\frac{d}{dt} ( u^i ) = 0
\end{equation}

%
% Copyright � 2015 Peeter Joot.  All Rights Reserved.
% Licenced as described in the file LICENSE under the root directory of this GIT repository.
%
\documentclass[]{eliblog}

\usepackage{amsmath}
\usepackage{mathpazo}

%
% shorthand for bold symbols, convenient for vectors and matrices
%
\newcommand{\Ba}[0]{\mathbf{a}}
\newcommand{\Bb}[0]{\mathbf{b}}
\newcommand{\Bc}[0]{\mathbf{c}}
\newcommand{\Bd}[0]{\mathbf{d}}
\newcommand{\Be}[0]{\mathbf{e}}
\newcommand{\Bf}[0]{\mathbf{f}}
\newcommand{\Bg}[0]{\mathbf{g}}
\newcommand{\Bh}[0]{\mathbf{h}}
\newcommand{\Bi}[0]{\mathbf{i}}
\newcommand{\Bj}[0]{\mathbf{j}}
\newcommand{\Bk}[0]{\mathbf{k}}
\newcommand{\Bl}[0]{\mathbf{l}}
\newcommand{\Bm}[0]{\mathbf{m}}
\newcommand{\Bn}[0]{\mathbf{n}}
\newcommand{\Bo}[0]{\mathbf{o}}
\newcommand{\Bp}[0]{\mathbf{p}}
\newcommand{\Bq}[0]{\mathbf{q}}
\newcommand{\Br}[0]{\mathbf{r}}
\newcommand{\Bs}[0]{\mathbf{s}}
\newcommand{\Bt}[0]{\mathbf{t}}
\newcommand{\Bu}[0]{\mathbf{u}}
\newcommand{\Bv}[0]{\mathbf{v}}
\newcommand{\Bw}[0]{\mathbf{w}}
\newcommand{\Bx}[0]{\mathbf{x}}
\newcommand{\By}[0]{\mathbf{y}}
\newcommand{\Bz}[0]{\mathbf{z}}
\newcommand{\BA}[0]{\mathbf{A}}
\newcommand{\BB}[0]{\mathbf{B}}
\newcommand{\BC}[0]{\mathbf{C}}
\newcommand{\BD}[0]{\mathbf{D}}
\newcommand{\BE}[0]{\mathbf{E}}
\newcommand{\BF}[0]{\mathbf{F}}
\newcommand{\BG}[0]{\mathbf{G}}
\newcommand{\BH}[0]{\mathbf{H}}
\newcommand{\BI}[0]{\mathbf{I}}
\newcommand{\BJ}[0]{\mathbf{J}}
\newcommand{\BK}[0]{\mathbf{K}}
\newcommand{\BL}[0]{\mathbf{L}}
\newcommand{\BM}[0]{\mathbf{M}}
\newcommand{\BN}[0]{\mathbf{N}}
\newcommand{\BO}[0]{\mathbf{O}}
\newcommand{\BP}[0]{\mathbf{P}}
\newcommand{\BQ}[0]{\mathbf{Q}}
\newcommand{\BR}[0]{\mathbf{R}}
\newcommand{\BS}[0]{\mathbf{S}}
\newcommand{\BT}[0]{\mathbf{T}}
\newcommand{\BU}[0]{\mathbf{U}}
\newcommand{\BV}[0]{\mathbf{V}}
\newcommand{\BW}[0]{\mathbf{W}}
\newcommand{\BX}[0]{\mathbf{X}}
\newcommand{\BY}[0]{\mathbf{Y}}
\newcommand{\BZ}[0]{\mathbf{Z}}

\newcommand{\Bzero}[0]{\mathbf{0}}
\newcommand{\Btheta}[0]{\boldsymbol{\theta}}
\newcommand{\Btau}[0]{\boldsymbol{\tau}}
\newcommand{\Bomega}[0]{\boldsymbol{\omega}}

%
% shorthand for unit vectors
%
\newcommand{\acap}[0]{\hat{\Ba}}
\newcommand{\bcap}[0]{\hat{\Bb}}
\newcommand{\ccap}[0]{\hat{\Bc}}
\newcommand{\dcap}[0]{\hat{\Bd}}
\newcommand{\ecap}[0]{\hat{\Be}}
\newcommand{\fcap}[0]{\hat{\Bf}}
\newcommand{\gcap}[0]{\hat{\Bg}}
\newcommand{\hcap}[0]{\hat{\Bh}}
\newcommand{\icap}[0]{\hat{\Bi}}
\newcommand{\jcap}[0]{\hat{\Bj}}
\newcommand{\kcap}[0]{\hat{\Bk}}
\newcommand{\lcap}[0]{\hat{\Bl}}
\newcommand{\mcap}[0]{\hat{\Bm}}
\newcommand{\ncap}[0]{\hat{\Bn}}
\newcommand{\ocap}[0]{\hat{\Bo}}
\newcommand{\pcap}[0]{\hat{\Bp}}
\newcommand{\qcap}[0]{\hat{\Bq}}
\newcommand{\rcap}[0]{\hat{\Br}}
\newcommand{\scap}[0]{\hat{\Bs}}
\newcommand{\tcap}[0]{\hat{\Bt}}
\newcommand{\ucap}[0]{\hat{\Bu}}
\newcommand{\vcap}[0]{\hat{\Bv}}
\newcommand{\wcap}[0]{\hat{\Bw}}
\newcommand{\xcap}[0]{\hat{\Bx}}
\newcommand{\ycap}[0]{\hat{\By}}
\newcommand{\zcap}[0]{\hat{\Bz}}
\newcommand{\thetacap}[0]{\hat{\Btheta}}

%
% to write R^n and C^n in a distinguishable fashion.  Perhaps change this
% to the double lined characters upon figuring out how to do so.
%
\newcommand{\C}[1]{$\mathbb{C}^{#1}$}
\newcommand{\R}[1]{$\mathbb{R}^{#1}$}

%
% various generally useful helpers
%

% derivative of #1 wrt. #2:
\newcommand{\D}[2] {\frac {d#2} {d#1}}

\newcommand{\inv}[1]{\frac{1}{#1}}
\newcommand{\cross}[0]{\times}

\newcommand{\abs}[1]{\lvert{#1}\rvert}
\newcommand{\norm}[1]{\lVert{#1}\rVert}
\newcommand{\innerprod}[2]{\langle{#1}, {#2}\rangle}
\newcommand{\dotprod}[2]{{#1} \cdot {#2}}
\newcommand{\bdotprod}[2]{\left({#1} \cdot {#2}\right)}
\newcommand{\crossprod}[2]{{#1} \cross {#2}}
\newcommand{\tripleprod}[3]{\dotprod{\left(\crossprod{#1}{#2}\right)}{#3}}

\DeclareMathOperator{\Proj}{Proj}
\DeclareMathOperator{\Span}{span}
\DeclareMathOperator{\Sgn}{sgn}
\DeclareMathOperator{\Area}{Area}
\DeclareMathOperator{\Volume}{Volume}

%
% A few miscellaneous things specific to this document
%
\newcommand{\crossop}[1]{\crossprod{#1}{}}

% R2 vector.
\newcommand{\VectorTwo}[2]{
\begin{bmatrix}
 {#1} \\
 {#2}
\end{bmatrix}
}

\newcommand{\VectorN}[1]{
\begin{bmatrix}
{#1}_1 \\
{#1}_2 \\
\vdots \\
{#1}_N \\
\end{bmatrix}
}

\newcommand{\DETuvij}[4]{
\begin{vmatrix}
 {#1}_{#3} & {#1}_{#4} \\
 {#2}_{#3} & {#2}_{#4}
\end{vmatrix}
}

\newcommand{\DETuvwijk}[6]{
\begin{vmatrix}
 {#1}_{#4} & {#1}_{#5} & {#1}_{#6} \\
 {#2}_{#4} & {#2}_{#5} & {#2}_{#6} \\
 {#3}_{#4} & {#3}_{#5} & {#3}_{#6}
\end{vmatrix}
}

\newcommand{\DETuvwxijkl}[8]{
\begin{vmatrix}
 {#1}_{#5} & {#1}_{#6} & {#1}_{#7} & {#1}_{#8} \\
 {#2}_{#5} & {#2}_{#6} & {#2}_{#7} & {#2}_{#8} \\
 {#3}_{#5} & {#3}_{#6} & {#3}_{#7} & {#3}_{#8} \\
 {#4}_{#5} & {#4}_{#6} & {#4}_{#7} & {#4}_{#8} \\
\end{vmatrix}
}

%\newcommand{\DETuvwxyijklm}[10]{
%\begin{vmatrix}
% {#1}_{#6} & {#1}_{#7} & {#1}_{#8} & {#1}_{#9} & {#1}_{#10} \\
% {#2}_{#6} & {#2}_{#7} & {#2}_{#8} & {#2}_{#9} & {#2}_{#10} \\
% {#3}_{#6} & {#3}_{#7} & {#3}_{#8} & {#3}_{#9} & {#3}_{#10} \\
% {#4}_{#6} & {#4}_{#7} & {#4}_{#8} & {#4}_{#9} & {#4}_{#10} \\
% {#5}_{#6} & {#5}_{#7} & {#5}_{#8} & {#5}_{#9} & {#5}_{#10}
%\end{vmatrix}
%}

% R3 vector.
\newcommand{\VectorThree}[3]{
\begin{bmatrix}
 {#1} \\
 {#2} \\
 {#3}
\end{bmatrix}
}



\author{Peeter Joot}
\email{peeter.joot@gmail.com}

%\documentclass[]{eliblogwidescreen}

\usepackage{amsmath}
\usepackage{mathpazo}

%
% shorthand for bold symbols, convenient for vectors and matrices
%
\newcommand{\Ba}[0]{\mathbf{a}}
\newcommand{\Bb}[0]{\mathbf{b}}
\newcommand{\Bc}[0]{\mathbf{c}}
\newcommand{\Bd}[0]{\mathbf{d}}
\newcommand{\Be}[0]{\mathbf{e}}
\newcommand{\Bf}[0]{\mathbf{f}}
\newcommand{\Bg}[0]{\mathbf{g}}
\newcommand{\Bh}[0]{\mathbf{h}}
\newcommand{\Bi}[0]{\mathbf{i}}
\newcommand{\Bj}[0]{\mathbf{j}}
\newcommand{\Bk}[0]{\mathbf{k}}
\newcommand{\Bl}[0]{\mathbf{l}}
\newcommand{\Bm}[0]{\mathbf{m}}
\newcommand{\Bn}[0]{\mathbf{n}}
\newcommand{\Bo}[0]{\mathbf{o}}
\newcommand{\Bp}[0]{\mathbf{p}}
\newcommand{\Bq}[0]{\mathbf{q}}
\newcommand{\Br}[0]{\mathbf{r}}
\newcommand{\Bs}[0]{\mathbf{s}}
\newcommand{\Bt}[0]{\mathbf{t}}
\newcommand{\Bu}[0]{\mathbf{u}}
\newcommand{\Bv}[0]{\mathbf{v}}
\newcommand{\Bw}[0]{\mathbf{w}}
\newcommand{\Bx}[0]{\mathbf{x}}
\newcommand{\By}[0]{\mathbf{y}}
\newcommand{\Bz}[0]{\mathbf{z}}
\newcommand{\BA}[0]{\mathbf{A}}
\newcommand{\BB}[0]{\mathbf{B}}
\newcommand{\BC}[0]{\mathbf{C}}
\newcommand{\BD}[0]{\mathbf{D}}
\newcommand{\BE}[0]{\mathbf{E}}
\newcommand{\BF}[0]{\mathbf{F}}
\newcommand{\BG}[0]{\mathbf{G}}
\newcommand{\BH}[0]{\mathbf{H}}
\newcommand{\BI}[0]{\mathbf{I}}
\newcommand{\BJ}[0]{\mathbf{J}}
\newcommand{\BK}[0]{\mathbf{K}}
\newcommand{\BL}[0]{\mathbf{L}}
\newcommand{\BM}[0]{\mathbf{M}}
\newcommand{\BN}[0]{\mathbf{N}}
\newcommand{\BO}[0]{\mathbf{O}}
\newcommand{\BP}[0]{\mathbf{P}}
\newcommand{\BQ}[0]{\mathbf{Q}}
\newcommand{\BR}[0]{\mathbf{R}}
\newcommand{\BS}[0]{\mathbf{S}}
\newcommand{\BT}[0]{\mathbf{T}}
\newcommand{\BU}[0]{\mathbf{U}}
\newcommand{\BV}[0]{\mathbf{V}}
\newcommand{\BW}[0]{\mathbf{W}}
\newcommand{\BX}[0]{\mathbf{X}}
\newcommand{\BY}[0]{\mathbf{Y}}
\newcommand{\BZ}[0]{\mathbf{Z}}

\newcommand{\Bzero}[0]{\mathbf{0}}
\newcommand{\Btheta}[0]{\boldsymbol{\theta}}
\newcommand{\Btau}[0]{\boldsymbol{\tau}}
\newcommand{\Bomega}[0]{\boldsymbol{\omega}}

%
% shorthand for unit vectors
%
\newcommand{\acap}[0]{\hat{\Ba}}
\newcommand{\bcap}[0]{\hat{\Bb}}
\newcommand{\ccap}[0]{\hat{\Bc}}
\newcommand{\dcap}[0]{\hat{\Bd}}
\newcommand{\ecap}[0]{\hat{\Be}}
\newcommand{\fcap}[0]{\hat{\Bf}}
\newcommand{\gcap}[0]{\hat{\Bg}}
\newcommand{\hcap}[0]{\hat{\Bh}}
\newcommand{\icap}[0]{\hat{\Bi}}
\newcommand{\jcap}[0]{\hat{\Bj}}
\newcommand{\kcap}[0]{\hat{\Bk}}
\newcommand{\lcap}[0]{\hat{\Bl}}
\newcommand{\mcap}[0]{\hat{\Bm}}
\newcommand{\ncap}[0]{\hat{\Bn}}
\newcommand{\ocap}[0]{\hat{\Bo}}
\newcommand{\pcap}[0]{\hat{\Bp}}
\newcommand{\qcap}[0]{\hat{\Bq}}
\newcommand{\rcap}[0]{\hat{\Br}}
\newcommand{\scap}[0]{\hat{\Bs}}
\newcommand{\tcap}[0]{\hat{\Bt}}
\newcommand{\ucap}[0]{\hat{\Bu}}
\newcommand{\vcap}[0]{\hat{\Bv}}
\newcommand{\wcap}[0]{\hat{\Bw}}
\newcommand{\xcap}[0]{\hat{\Bx}}
\newcommand{\ycap}[0]{\hat{\By}}
\newcommand{\zcap}[0]{\hat{\Bz}}
\newcommand{\thetacap}[0]{\hat{\Btheta}}

%
% to write R^n and C^n in a distinguishable fashion.  Perhaps change this
% to the double lined characters upon figuring out how to do so.
%
\newcommand{\C}[1]{$\mathbb{C}^{#1}$}
\newcommand{\R}[1]{$\mathbb{R}^{#1}$}

%
% various generally useful helpers
%

% derivative of #1 wrt. #2:
\newcommand{\D}[2] {\frac {d#2} {d#1}}

\newcommand{\inv}[1]{\frac{1}{#1}}
\newcommand{\cross}[0]{\times}

\newcommand{\abs}[1]{\lvert{#1}\rvert}
\newcommand{\norm}[1]{\lVert{#1}\rVert}
\newcommand{\innerprod}[2]{\langle{#1}, {#2}\rangle}
\newcommand{\dotprod}[2]{{#1} \cdot {#2}}
\newcommand{\bdotprod}[2]{\left({#1} \cdot {#2}\right)}
\newcommand{\crossprod}[2]{{#1} \cross {#2}}
\newcommand{\tripleprod}[3]{\dotprod{\left(\crossprod{#1}{#2}\right)}{#3}}

\DeclareMathOperator{\Proj}{Proj}
\DeclareMathOperator{\Span}{span}
\DeclareMathOperator{\Sgn}{sgn}
\DeclareMathOperator{\Area}{Area}
\DeclareMathOperator{\Volume}{Volume}

%
% A few miscellaneous things specific to this document
%
\newcommand{\crossop}[1]{\crossprod{#1}{}}

% R2 vector.
\newcommand{\VectorTwo}[2]{
\begin{bmatrix}
 {#1} \\
 {#2}
\end{bmatrix}
}

\newcommand{\VectorN}[1]{
\begin{bmatrix}
{#1}_1 \\
{#1}_2 \\
\vdots \\
{#1}_N \\
\end{bmatrix}
}

\newcommand{\DETuvij}[4]{
\begin{vmatrix}
 {#1}_{#3} & {#1}_{#4} \\
 {#2}_{#3} & {#2}_{#4}
\end{vmatrix}
}

\newcommand{\DETuvwijk}[6]{
\begin{vmatrix}
 {#1}_{#4} & {#1}_{#5} & {#1}_{#6} \\
 {#2}_{#4} & {#2}_{#5} & {#2}_{#6} \\
 {#3}_{#4} & {#3}_{#5} & {#3}_{#6}
\end{vmatrix}
}

\newcommand{\DETuvwxijkl}[8]{
\begin{vmatrix}
 {#1}_{#5} & {#1}_{#6} & {#1}_{#7} & {#1}_{#8} \\
 {#2}_{#5} & {#2}_{#6} & {#2}_{#7} & {#2}_{#8} \\
 {#3}_{#5} & {#3}_{#6} & {#3}_{#7} & {#3}_{#8} \\
 {#4}_{#5} & {#4}_{#6} & {#4}_{#7} & {#4}_{#8} \\
\end{vmatrix}
}

%\newcommand{\DETuvwxyijklm}[10]{
%\begin{vmatrix}
% {#1}_{#6} & {#1}_{#7} & {#1}_{#8} & {#1}_{#9} & {#1}_{#10} \\
% {#2}_{#6} & {#2}_{#7} & {#2}_{#8} & {#2}_{#9} & {#2}_{#10} \\
% {#3}_{#6} & {#3}_{#7} & {#3}_{#8} & {#3}_{#9} & {#3}_{#10} \\
% {#4}_{#6} & {#4}_{#7} & {#4}_{#8} & {#4}_{#9} & {#4}_{#10} \\
% {#5}_{#6} & {#5}_{#7} & {#5}_{#8} & {#5}_{#9} & {#5}_{#10}
%\end{vmatrix}
%}

% R3 vector.
\newcommand{\VectorThree}[3]{
\begin{bmatrix}
 {#1} \\
 {#2} \\
 {#3}
\end{bmatrix}
}



\author{Peeter Joot}
\email{peeter.joot@gmail.com}


\chapter{PHY450H1S.  Relativistic Electrodynamics Tutorial 2 (TA: Simon Freedman).  Two worked problems.}
\label{chap:relativisticElectrodynamicsT2}
%\useCCL
\blogpage{http://sites.google.com/site/peeterjoot/math2011/relativisticElectrodynamicsT2.pdf}
\date{Jan 27, 2011}
\revisionInfo{relativisticElectrodynamicsT2.tex}

\beginArtWithToc
%\beginArtNoToc

\section{What we will discuss.}

\begin{itemize}
\item 4-vectors: position, velocity, acceleration
\item non-inertial observers
\end{itemize}

\section{Problem 1.}

\subsection{Statement}
A particle moves on the x-axis along a world line described by

\begin{align}\label{eqn:relativisticElectrodynamicsT2:10}
ct(\sigma) &= \inv{a} \sinh(\sigma) \\
x(\sigma) &= \inv{a} \cosh(\sigma)
\end{align}

where the dimension of the constant $[a] = \inv{L}$, is inverse length, and our parameter takes any values $-\infty < \sigma < \infty$.

Find $x^i(\tau)$, $u^i(\tau)$, $a^i(\tau)$.

\subsection{Solution}

First note that we can re-parametrize $x = x^1$ in terms of $t$.  That is

\begin{align*}
\cosh(\sigma)
&= \sqrt{1 + \sinh^2(\sigma)}  \\
&= \sqrt{ 1 + (act)^2 }  \\
&= a \sqrt{ a^{-2} + (ct)^2 }
\end{align*}

So

\begin{equation}\label{eqn:relativisticElectrodynamicsT2:20}
x(t) = \sqrt{ a^{-2} + (ct)^2 }
\end{equation}

Squaring and rearranging, shows that our particle is moving through half of a hyperbolic arc in spacetime (two such paths are possible, one for strictly positive $x$ and one for strictly negative).

\begin{equation}\label{eqn:relativisticElectrodynamicsT2:30}
x^2 - (ct)^2 = a^{-2}
\end{equation}

Observe that the asymptotes of this curve are the lightcone boundaries.  Taking derivatives we have

\begin{equation}\label{eqn:relativisticElectrodynamicsT2:40}
2 x \frac{dx}{d(ct)} -2 (ct) = 0,
\end{equation}

and rearranging we have

\begin{align*}
\frac{dx}{d(ct)}
&= \frac{c t}{x} \\
&= \frac{ct}{\sqrt{(ct)^2 + a^{-2}}} \\
&\rightarrow \pm 1
\end{align*}

\paragraph{Is this timelike?}

Let's compute the interval between two worldpoints.  That is

\begin{align*}
s_{12}^2
&= (ct(\sigma_1) - ct(\sigma_2))^2 - (x(\sigma_1) - x(\sigma_2))^2  \\
&= a^{-2} (\sinh \sigma_1 - \sinh \sigma_2)^2 - a^{-2} (\cosh\sigma_1 - \cosh\sigma_2)^2 \\
&= 2 a^{-2} \left( -1 - \sinh\sigma_1 \sinh \sigma_2 + \cosh\sigma_1 \cosh\sigma_2 \right) \\
&= 2 a^{-2} \left( \cosh( \sigma_2 - \sigma_1) -1 \right) \ge 0
\end{align*}

Yes, this is timelike.  That's what we want for a particle that is realistic moving along a worldline in spacetime.  If the spacetime interval between any two points were to be negative, we would be talking about something of tachyon like hypothetical nature.

Our first task is to compute $x^i(\tau)$.  We have $x^i(\sigma)$ so we need the relation between our proper length $\tau$ and the worldline parameter $\sigma$.  Such a relation is implicitly provided by the differential spacetime interval

\begin{align*}
\left(\frac{d\tau}{d\sigma}\right)^2
&= \inv{c^2} \left(\frac{ds}{d\sigma}\right)^2 \\
&= \inv{c^2} \left(
\left( \frac{d(x^0)}{d\sigma}\right)^2
-\left( \frac{d(x^1)}{d\sigma}\right)^2
\right) \\
&= \inv{c^2} \left( a^{-2} \cosh^2 \sigma - a^{-2} \sinh^2 \sigma \right) \\
&= \inv{a^2 c^2}.
\end{align*}

Taking roots we have
\begin{equation}\label{eqn:relativisticElectrodynamicsT2:50}
\frac{d\tau}{d\sigma} = \pm \inv{a c},
\end{equation}

We take the positive root, so that the worldline is traversed in a strictly increasing fashion, and then integrate once

\begin{equation}\label{eqn:relativisticElectrodynamicsT2:60}
\tau = \inv{ac} \sigma + \tau_s.
\end{equation}

We are free to let $\tau_s = 0$, effectively starting our proper time at $t=0$.

\begin{equation}\label{eqn:relativisticElectrodynamicsT2:70}
x^i(\tau) = ( a^{-1} \sinh( a c \tau), a^{-1} \cosh( a c \tau ), 0, 0 )
\end{equation}

As noted already this is a hyperbola (or degenerate hyperboloid) in spacetime, with asymptote 1 (ie: approaching the speed of light).

The next computational task is now simple.
\begin{equation}\label{eqn:relativisticElectrodynamicsT2:77}
u^i
= \frac{dx^i}{d\tau} 
= c ( \cosh( a c \tau ), \sinh( a c \tau ), 0, 0) \\
\end{equation}

Is this light like or time like?  We can answer this by considering the four vector square

\begin{equation}\label{eqn:relativisticElectrodynamicsT2:80}
u \cdot u 
\end{equation}

What is a light like or a time like vector?

Recall that we have defined lightlike, spacelike, and timelike intervals.  A lightlike interval between two world points had $(ct - c\tilde{t})^2 - (\Bx -\tilde{\Bx})^2 = 0$, whereas a timelike interval had $(ct - c\tilde{t})^2 - (\Bx -\tilde{\Bx})^2 > 0$.  Taking the vector $(c \tilde{t}, \tilde{\Bx})$ as the origin, the distance to any single four vector from the origin is then just that vector's square, so it logically makes sense to call a vector light like if it has a zero square, and time like if it has a positive square.

Consider the very simplest example of a time like trajectory, that of a particle at rest at a fixed position $\Bx_0$.  Such a particle's worldline is

\begin{equation}\label{eqn:relativisticElectrodynamicsT2:71}
X = ( c t, \Bx_0 )
\end{equation}

While we interpret $t$ here as time, it functions as a parametrization of the curve, just as $\sigma$ does in this question.  If we want to compute the proper time interval between two points on this worldline we have

\begin{align*}
\tau_b - \tau_0 
&=
\inv{c} \int_{\lambda = t_a}^{t_b} \sqrt{ \frac{dX(\lambda)}{d\lambda} \cdot \frac{dX(\lambda)}{d\lambda} } d\lambda \\
&=
\inv{c} \int_{\lambda = t_a}^{t_b} \sqrt{ (c, 0)^2 } d\lambda \\
&=
\inv{c} \int_{\lambda = t_a}^{t_b} c d\lambda \\
&= t_b - t_a
\end{align*}

The conclusion (arrived at the hard way, but methodologically) is that proper time on this worldline is just the parameter $t$ itself.

Now examine the proper velocity for this trajectory.  This is

\begin{equation}\label{eqn:relativisticElectrodynamicsT2:72}
u = \frac{dX(\tau)}{d\tau} = (c, 0, 0, 0)
\end{equation}

We can compute the dot product $u \cdot u = c^2 > 0$ easily enough, and in this case for the particle at rest (but moving in time) we see that this four-vector velocity does have a time like separation from the origin, and it therefore makes sense to label the four-velocity vector itself as time like.

Now, let's return to our non-inertial system.  Is our four velocity vector time like?  Let's compute it's square to check

\begin{equation}\label{eqn:relativisticElectrodynamicsT2:90}
u \cdot u = c^2 ( \cosh^2 - \sinh^2 ) = c^2 > 0
\end{equation}

Yes, it is timelike.

Now, let's calculate our spatial velocity

\begin{equation}\label{eqn:relativisticElectrodynamicsT2:100}
v^\alpha
= \frac{dx^\alpha}{dt}
=
\frac{dx^\alpha}{d\tau} \frac{d\tau}{dt}
\end{equation}

Since $ct = \sinh( a c \tau )/a$ we have

\begin{equation}\label{eqn:relativisticElectrodynamicsT2:110}
c = \inv{a} a c \cosh( a c \tau ) \frac{d\tau}{dt},
\end{equation}

or
\begin{equation}\label{eqn:relativisticElectrodynamicsT2:110b}
\frac{d\tau}{dt} = \inv{\cosh( a c \tau) }
\end{equation}

Similarily from \ref{eqn:relativisticElectrodynamicsT2:70}, we have

\begin{equation}\label{eqn:relativisticElectrodynamicsT2:120}
\frac{dx^1}{d\tau} = c \sinh( a c \tau )
\end{equation}

So our spatial velocity is $\sinh/\cosh = \tanh$, and we have

\begin{equation}\label{eqn:relativisticElectrodynamicsT2:130}
v^\alpha = (c \tanh( a c \tau), 0, 0)
\end{equation}

Note how tricky this index notation is.  Four our four vector velocity we use $u^i = dx^i/d\tau$, whereas our spatial velocity is distinguished by a change of letter as well as the indexes, so when we write $v^\alpha$ we are taking our derivatives with respect to time and not proper time (i.e. $v^\alpha = dx^\alpha/dt$).

\subsection{ What is our four-acceleration? }

From \ref{eqn:relativisticElectrodynamicsT2:77}, we have

\begin{align*}
w^i (\tau) = \frac{ du^i }{d\tau} = a c^2 x^i(\tau)
\end{align*}

Observe that our four-velocity square is

\begin{equation}\label{eqn:relativisticElectrodynamicsT2:78}
w \cdot w = a^2 c^2 a^{-1} (-1)
\end{equation}

What does this really signify?  Think on this.  A check to verify that things are okay is to see if this four-acceleration is orthogonal to our four-velocity as expected

\begin{align*}
w \cdot u 
&= 
a c^2 ( a^{-1} \sinh( a c \tau), a^{-1} \cosh( a c \tau ), 0, 0 ) \cdot c ( \cosh( a c \tau ), \sinh( a c \tau ), 0, 0) \\
&=
c^3 ( \sinh(a c \tau)\cosh(a c \tau) - \cosh(a c \tau) \sinh(a c \tau) ) \\
&=
0
\end{align*}

A last beastie that we can compute is the spatial acceleration.

\begin{align*}
a^\alpha 
&= \frac{du^\alpha}{dt} \\
&= \frac{dv^\alpha}{d\tau} \frac{d \tau}{dt} \\
&= \frac{a c^2}{\cosh^2(a c \tau)} \inv{\cosh(a c \tau) } \\
&= \frac{a c^2}{\cosh^3(a c \tau)} \\
%&= \frac{a c^2}{(a x)^3} \\
%&= \frac{c^2}{a^2 x^3}
\end{align*}

Summarizing

\begin{align}\label{eqn:relativisticElectrodynamicsT2:150}
x^i(\tau) &= \left( a^{-1} \sinh( a c \tau), a^{-1} \cosh( a c \tau ), 0, 0 \right) \\
u^i(\tau) &= c \left( \cosh( a c \tau ), \sinh( a c \tau ), 0, 0\right) \\
v^\alpha(\tau) &= \left( c \tanh(a c \tau), 0, 0 \right) \\
w^i(\tau) &= a c^2 x^i(\tau) \\
a^\alpha(\tau) &= \left( \frac{a c^2}{\cosh^3 (a c \tau)}, 0, 0 \right) \\
\end{align}

\section{Problem 2.  Local observers.}

Observations are made of either the three-vector, or the time like components of four-vectors, since these are the quantities that we can measure from our local observer frame.  This is something that can be viewed in an approximate sense as being inertial, provided that we ignore the earth's rotation, the rotation around the solar system, the rotation of the solar system in the galaxy, the rotation of the galaxy in the local cluster, and so forth.  Provided none of these are changing too fast relative to our measurements, we can make the inertial approximation.

Example.  If we want to measure energy, it is the timelike component of the momentum.

\begin{equation}\label{eqn:relativisticElectrodynamicsT2:500}
E = c p^0
\end{equation}

PICTURE:  Let's imagine a moving worldline in three dimensions.  We can setup a frame and associated basis along the worldline of the particle, as well as a frame and basis for the stationary observer.

In class Simon used notation like $\{ e_{\hat{o}}^i \}$, and $\{ e_{\hat{a}}^i \}$, but also used $e_{\hat{0}}^i$, $e_{\hat{1}}^i$, $e_{\hat{2}}^i$, $e_{\hat{3}}^i$.  It was fairly clear by the context what was meant, but lets avoid any more than one index at a time, and write $\{ e^i \}$ for the frame moving along the worldline, and $\{ \gamma^i \}$ for the stationary frame.  The use of $\gamma^i$ as a spacetime basis is borrowed from \cite{doran2003gap}.

For any timelike four-vector worldline we have a four-vector velocity of magnitude $c$, so we are free to define a timelike basis vector for our moving frame as

\begin{equation}\label{eqn:relativisticElectrodynamicsT2:510}
e^0 = u / c
\end{equation}

going back to the first problem for $u^i$ we have

\begin{equation*}
e^0 = ( \cosh( a c t ), \sinh( a c t), 0, 0 ) 
\end{equation*}

We are free to pick spatial unit vectors perpendicular to this, so for the $y$ and $z$ components it is natural to use
\begin{align*}
e^2 &= ( 0, 0, 1, 0 ) \\
e^3 &= ( 0, 0, 0, 1 )
\end{align*}

We need one more, that's perpendicular to each of the above.  By inspection one can pick

\begin{equation*}
e^1 = ( \sinh( a c t ), \cosh( a c t), 0, 0) 
\end{equation*}

Did Simon use any other principle to define this last beastie?  I missed it if he did.  I see that this happens to be the unit vector proportional to $x^i$.

\subsection{Consider the stationary observer.}

For a stationary observer, our worldline and four velocity respectively, for some constant $\Bx_0$ is

\begin{align}\label{eqn:relativisticElectrodynamicsT2:600}
X &= ( ct, \Bx_0 ) \\
\frac{dX}{d\tau} &= c ( 1, \Bzero) 
\end{align}

Our time like unit vector is very simple

\begin{equation}\label{eqn:relativisticElectrodynamicsT2:610}
\gamma^0 = \inv{c} \frac{dX}{d\tau} = ( 1, \Bzero ) 
\end{equation}

For the spatial unit vectors we have many choices.  One would be aligned from the origin to the position vector

\begin{equation}\label{eqn:relativisticElectrodynamicsT2:620}
\gamma^1 = \left( 0, \frac{\Bx}{\Abs{\Bx}} \right),
\end{equation}

with $\gamma^2$ and $\gamma^3$ oriented in any pair of mutually perpendicular spatial directions.  Another option would be simply pick a $\gamma$ for each of the normal Euclidean basis directions

\begin{align}\label{eqn:relativisticElectrodynamicsT2:630}
\gamma^1 &= ( 0, 1, 0, 0 ) \\
\gamma^2 &= ( 0, 0, 1, 0 ) \\
\gamma^3 &= ( 0, 0, 0, 1 )
\end{align}

Observe, that just as in Geometric Algebra, we have $\gamma^\alpha \cdot \gamma^\alpha = -1$ (and $\gamma^0 \cdot \gamma^0 = 1$).

\subsection{Consider an inertial observer.}

Now lets consider a slightly more complex case, where an observer is moving with some constant velocity $\BV = c \Bbeta$.  Our worldline is

\begin{equation}\label{eqn:relativisticElectrodynamicsT2:700}
X = ( ct, \Bx_0 + \Bbeta c t) .
\end{equation}

Let's calculate the four velocity.  We have

\begin{equation}\label{eqn:relativisticElectrodynamicsT2:710}
\frac{dX}{dt} = c ( 1, \Bbeta ).
\end{equation}

From this our proper time is

\begin{equation}\label{eqn:relativisticElectrodynamicsT2:720}
\tau = \inv{c} \int_0^t c \sqrt{ (1, \Bbeta)^2 } dt = \sqrt{1 - \Beta^2} t.
\end{equation}

Our worldline and four-velocity, parameterized in terms of proper time, with $\gamma = (1 - \Bbeta^2)^{-1/2}$, are then

\begin{align}\label{eqn:relativisticElectrodynamicsT2:730}
X &= ( \gamma c\tau, \Bx_0 + \gamma \Bbeta c \tau) \\
u &= \gamma c ( 1, \Bbeta )
\end{align}

From this our time like unit vector is

\begin{equation}\label{eqn:relativisticElectrodynamicsT2:740}
e^0 = \gamma ( 1, \Bbeta )
\end{equation}

We observe that this has the desired time like property, $(e^0)^2 = 1 > 0$.

%%%
%%%
%%%
%%%
%%%\begin{equation}\label{eqn:relativisticElectrodynamicsT2:n}
%%%p^0 = p \cdot x^0
%%%\end{equation}
%%%
%%%\begin{equation}\label{eqn:relativisticElectrodynamicsT2:n}
%%%E_obs = c p \cdot e_ohat \equiv c p^ohat
%%%\end{equation}
%%%
%%%In the observers reference frame
%%%
%%%\begin{equation}\label{eqn:relativisticElectrodynamicsT2:n}
%%%{p'}^i = (mc,0, 0, 0)
%%%\end{equation}
%%%
%%%\begin{equation}\label{eqn:relativisticElectrodynamicsT2:n}
%%%{u'}^i_obs = c \gamma (( 1, v/c , 0, 0)
%%%\end{equation}
%%%
%%%\begin{equation}\label{eqn:relativisticElectrodynamicsT2:n}
%%%u^i_obj c (1, 0, 0, 0)
%%%\end{equation}
%%%
%%%\begin{equation}\label{eqn:relativisticElectrodynamicsT2:n}
%%%{u'}_obs^i =
%%%\begin{bmatrix}
%%%\gamma & \gamma v/c  & 0 & 0 \\
%%%\gamma v/c  & \gamma  & 0 & 0 \\
%%%0 & 0 & 0 & 0 \\
%%%0 & 0 & 0 & 0
%%%\end{bmatrix}
%%%\end{equation}
%%%
%%%\begin{equation}\label{eqn:relativisticElectrodynamicsT2:n}
%%%p^ohat = \gamma m c
%%%\end{equation}
%%%
%%%Suppose we have a star far away.  What is the frequency of the light emitted
%%%
%%%\begin{equation}\label{eqn:relativisticElectrodynamicsT2:n}
%%%\hat{\omega} = \omega e^{- a c \tau }
%%%\end{equation}
%%%
%%%FIXME: derive.
%%%
%%%where $\omega$ is the emitted frequency.
%%%
%%%FIXME: This implied an elapsed time before the star would no longer be visible?

\EndArticle

%
% Copyright � 2015 Peeter Joot.  All Rights Reserved.
% Licenced as described in the file LICENSE under the root directory of this GIT repository.
%
\documentclass[]{eliblog}

\usepackage{amsmath}
\usepackage{mathpazo}

%
% shorthand for bold symbols, convenient for vectors and matrices
%
\newcommand{\Ba}[0]{\mathbf{a}}
\newcommand{\Bb}[0]{\mathbf{b}}
\newcommand{\Bc}[0]{\mathbf{c}}
\newcommand{\Bd}[0]{\mathbf{d}}
\newcommand{\Be}[0]{\mathbf{e}}
\newcommand{\Bf}[0]{\mathbf{f}}
\newcommand{\Bg}[0]{\mathbf{g}}
\newcommand{\Bh}[0]{\mathbf{h}}
\newcommand{\Bi}[0]{\mathbf{i}}
\newcommand{\Bj}[0]{\mathbf{j}}
\newcommand{\Bk}[0]{\mathbf{k}}
\newcommand{\Bl}[0]{\mathbf{l}}
\newcommand{\Bm}[0]{\mathbf{m}}
\newcommand{\Bn}[0]{\mathbf{n}}
\newcommand{\Bo}[0]{\mathbf{o}}
\newcommand{\Bp}[0]{\mathbf{p}}
\newcommand{\Bq}[0]{\mathbf{q}}
\newcommand{\Br}[0]{\mathbf{r}}
\newcommand{\Bs}[0]{\mathbf{s}}
\newcommand{\Bt}[0]{\mathbf{t}}
\newcommand{\Bu}[0]{\mathbf{u}}
\newcommand{\Bv}[0]{\mathbf{v}}
\newcommand{\Bw}[0]{\mathbf{w}}
\newcommand{\Bx}[0]{\mathbf{x}}
\newcommand{\By}[0]{\mathbf{y}}
\newcommand{\Bz}[0]{\mathbf{z}}
\newcommand{\BA}[0]{\mathbf{A}}
\newcommand{\BB}[0]{\mathbf{B}}
\newcommand{\BC}[0]{\mathbf{C}}
\newcommand{\BD}[0]{\mathbf{D}}
\newcommand{\BE}[0]{\mathbf{E}}
\newcommand{\BF}[0]{\mathbf{F}}
\newcommand{\BG}[0]{\mathbf{G}}
\newcommand{\BH}[0]{\mathbf{H}}
\newcommand{\BI}[0]{\mathbf{I}}
\newcommand{\BJ}[0]{\mathbf{J}}
\newcommand{\BK}[0]{\mathbf{K}}
\newcommand{\BL}[0]{\mathbf{L}}
\newcommand{\BM}[0]{\mathbf{M}}
\newcommand{\BN}[0]{\mathbf{N}}
\newcommand{\BO}[0]{\mathbf{O}}
\newcommand{\BP}[0]{\mathbf{P}}
\newcommand{\BQ}[0]{\mathbf{Q}}
\newcommand{\BR}[0]{\mathbf{R}}
\newcommand{\BS}[0]{\mathbf{S}}
\newcommand{\BT}[0]{\mathbf{T}}
\newcommand{\BU}[0]{\mathbf{U}}
\newcommand{\BV}[0]{\mathbf{V}}
\newcommand{\BW}[0]{\mathbf{W}}
\newcommand{\BX}[0]{\mathbf{X}}
\newcommand{\BY}[0]{\mathbf{Y}}
\newcommand{\BZ}[0]{\mathbf{Z}}

\newcommand{\Bzero}[0]{\mathbf{0}}
\newcommand{\Btheta}[0]{\boldsymbol{\theta}}
\newcommand{\Btau}[0]{\boldsymbol{\tau}}
\newcommand{\Bomega}[0]{\boldsymbol{\omega}}

%
% shorthand for unit vectors
%
\newcommand{\acap}[0]{\hat{\Ba}}
\newcommand{\bcap}[0]{\hat{\Bb}}
\newcommand{\ccap}[0]{\hat{\Bc}}
\newcommand{\dcap}[0]{\hat{\Bd}}
\newcommand{\ecap}[0]{\hat{\Be}}
\newcommand{\fcap}[0]{\hat{\Bf}}
\newcommand{\gcap}[0]{\hat{\Bg}}
\newcommand{\hcap}[0]{\hat{\Bh}}
\newcommand{\icap}[0]{\hat{\Bi}}
\newcommand{\jcap}[0]{\hat{\Bj}}
\newcommand{\kcap}[0]{\hat{\Bk}}
\newcommand{\lcap}[0]{\hat{\Bl}}
\newcommand{\mcap}[0]{\hat{\Bm}}
\newcommand{\ncap}[0]{\hat{\Bn}}
\newcommand{\ocap}[0]{\hat{\Bo}}
\newcommand{\pcap}[0]{\hat{\Bp}}
\newcommand{\qcap}[0]{\hat{\Bq}}
\newcommand{\rcap}[0]{\hat{\Br}}
\newcommand{\scap}[0]{\hat{\Bs}}
\newcommand{\tcap}[0]{\hat{\Bt}}
\newcommand{\ucap}[0]{\hat{\Bu}}
\newcommand{\vcap}[0]{\hat{\Bv}}
\newcommand{\wcap}[0]{\hat{\Bw}}
\newcommand{\xcap}[0]{\hat{\Bx}}
\newcommand{\ycap}[0]{\hat{\By}}
\newcommand{\zcap}[0]{\hat{\Bz}}
\newcommand{\thetacap}[0]{\hat{\Btheta}}

%
% to write R^n and C^n in a distinguishable fashion.  Perhaps change this
% to the double lined characters upon figuring out how to do so.
%
\newcommand{\C}[1]{$\mathbb{C}^{#1}$}
\newcommand{\R}[1]{$\mathbb{R}^{#1}$}

%
% various generally useful helpers
%

% derivative of #1 wrt. #2:
\newcommand{\D}[2] {\frac {d#2} {d#1}}

\newcommand{\inv}[1]{\frac{1}{#1}}
\newcommand{\cross}[0]{\times}

\newcommand{\abs}[1]{\lvert{#1}\rvert}
\newcommand{\norm}[1]{\lVert{#1}\rVert}
\newcommand{\innerprod}[2]{\langle{#1}, {#2}\rangle}
\newcommand{\dotprod}[2]{{#1} \cdot {#2}}
\newcommand{\bdotprod}[2]{\left({#1} \cdot {#2}\right)}
\newcommand{\crossprod}[2]{{#1} \cross {#2}}
\newcommand{\tripleprod}[3]{\dotprod{\left(\crossprod{#1}{#2}\right)}{#3}}

\DeclareMathOperator{\Proj}{Proj}
\DeclareMathOperator{\Span}{span}
\DeclareMathOperator{\Sgn}{sgn}
\DeclareMathOperator{\Area}{Area}
\DeclareMathOperator{\Volume}{Volume}

%
% A few miscellaneous things specific to this document
%
\newcommand{\crossop}[1]{\crossprod{#1}{}}

% R2 vector.
\newcommand{\VectorTwo}[2]{
\begin{bmatrix}
 {#1} \\
 {#2}
\end{bmatrix}
}

\newcommand{\VectorN}[1]{
\begin{bmatrix}
{#1}_1 \\
{#1}_2 \\
\vdots \\
{#1}_N \\
\end{bmatrix}
}

\newcommand{\DETuvij}[4]{
\begin{vmatrix}
 {#1}_{#3} & {#1}_{#4} \\
 {#2}_{#3} & {#2}_{#4}
\end{vmatrix}
}

\newcommand{\DETuvwijk}[6]{
\begin{vmatrix}
 {#1}_{#4} & {#1}_{#5} & {#1}_{#6} \\
 {#2}_{#4} & {#2}_{#5} & {#2}_{#6} \\
 {#3}_{#4} & {#3}_{#5} & {#3}_{#6}
\end{vmatrix}
}

\newcommand{\DETuvwxijkl}[8]{
\begin{vmatrix}
 {#1}_{#5} & {#1}_{#6} & {#1}_{#7} & {#1}_{#8} \\
 {#2}_{#5} & {#2}_{#6} & {#2}_{#7} & {#2}_{#8} \\
 {#3}_{#5} & {#3}_{#6} & {#3}_{#7} & {#3}_{#8} \\
 {#4}_{#5} & {#4}_{#6} & {#4}_{#7} & {#4}_{#8} \\
\end{vmatrix}
}

%\newcommand{\DETuvwxyijklm}[10]{
%\begin{vmatrix}
% {#1}_{#6} & {#1}_{#7} & {#1}_{#8} & {#1}_{#9} & {#1}_{#10} \\
% {#2}_{#6} & {#2}_{#7} & {#2}_{#8} & {#2}_{#9} & {#2}_{#10} \\
% {#3}_{#6} & {#3}_{#7} & {#3}_{#8} & {#3}_{#9} & {#3}_{#10} \\
% {#4}_{#6} & {#4}_{#7} & {#4}_{#8} & {#4}_{#9} & {#4}_{#10} \\
% {#5}_{#6} & {#5}_{#7} & {#5}_{#8} & {#5}_{#9} & {#5}_{#10}
%\end{vmatrix}
%}

% R3 vector.
\newcommand{\VectorThree}[3]{
\begin{bmatrix}
 {#1} \\
 {#2} \\
 {#3}
\end{bmatrix}
}



\author{Peeter Joot}
\email{peeter.joot@gmail.com}

%\documentclass[]{eliblogwidescreen}

\usepackage{amsmath}
\usepackage{mathpazo}

%
% shorthand for bold symbols, convenient for vectors and matrices
%
\newcommand{\Ba}[0]{\mathbf{a}}
\newcommand{\Bb}[0]{\mathbf{b}}
\newcommand{\Bc}[0]{\mathbf{c}}
\newcommand{\Bd}[0]{\mathbf{d}}
\newcommand{\Be}[0]{\mathbf{e}}
\newcommand{\Bf}[0]{\mathbf{f}}
\newcommand{\Bg}[0]{\mathbf{g}}
\newcommand{\Bh}[0]{\mathbf{h}}
\newcommand{\Bi}[0]{\mathbf{i}}
\newcommand{\Bj}[0]{\mathbf{j}}
\newcommand{\Bk}[0]{\mathbf{k}}
\newcommand{\Bl}[0]{\mathbf{l}}
\newcommand{\Bm}[0]{\mathbf{m}}
\newcommand{\Bn}[0]{\mathbf{n}}
\newcommand{\Bo}[0]{\mathbf{o}}
\newcommand{\Bp}[0]{\mathbf{p}}
\newcommand{\Bq}[0]{\mathbf{q}}
\newcommand{\Br}[0]{\mathbf{r}}
\newcommand{\Bs}[0]{\mathbf{s}}
\newcommand{\Bt}[0]{\mathbf{t}}
\newcommand{\Bu}[0]{\mathbf{u}}
\newcommand{\Bv}[0]{\mathbf{v}}
\newcommand{\Bw}[0]{\mathbf{w}}
\newcommand{\Bx}[0]{\mathbf{x}}
\newcommand{\By}[0]{\mathbf{y}}
\newcommand{\Bz}[0]{\mathbf{z}}
\newcommand{\BA}[0]{\mathbf{A}}
\newcommand{\BB}[0]{\mathbf{B}}
\newcommand{\BC}[0]{\mathbf{C}}
\newcommand{\BD}[0]{\mathbf{D}}
\newcommand{\BE}[0]{\mathbf{E}}
\newcommand{\BF}[0]{\mathbf{F}}
\newcommand{\BG}[0]{\mathbf{G}}
\newcommand{\BH}[0]{\mathbf{H}}
\newcommand{\BI}[0]{\mathbf{I}}
\newcommand{\BJ}[0]{\mathbf{J}}
\newcommand{\BK}[0]{\mathbf{K}}
\newcommand{\BL}[0]{\mathbf{L}}
\newcommand{\BM}[0]{\mathbf{M}}
\newcommand{\BN}[0]{\mathbf{N}}
\newcommand{\BO}[0]{\mathbf{O}}
\newcommand{\BP}[0]{\mathbf{P}}
\newcommand{\BQ}[0]{\mathbf{Q}}
\newcommand{\BR}[0]{\mathbf{R}}
\newcommand{\BS}[0]{\mathbf{S}}
\newcommand{\BT}[0]{\mathbf{T}}
\newcommand{\BU}[0]{\mathbf{U}}
\newcommand{\BV}[0]{\mathbf{V}}
\newcommand{\BW}[0]{\mathbf{W}}
\newcommand{\BX}[0]{\mathbf{X}}
\newcommand{\BY}[0]{\mathbf{Y}}
\newcommand{\BZ}[0]{\mathbf{Z}}

\newcommand{\Bzero}[0]{\mathbf{0}}
\newcommand{\Btheta}[0]{\boldsymbol{\theta}}
\newcommand{\Btau}[0]{\boldsymbol{\tau}}
\newcommand{\Bomega}[0]{\boldsymbol{\omega}}

%
% shorthand for unit vectors
%
\newcommand{\acap}[0]{\hat{\Ba}}
\newcommand{\bcap}[0]{\hat{\Bb}}
\newcommand{\ccap}[0]{\hat{\Bc}}
\newcommand{\dcap}[0]{\hat{\Bd}}
\newcommand{\ecap}[0]{\hat{\Be}}
\newcommand{\fcap}[0]{\hat{\Bf}}
\newcommand{\gcap}[0]{\hat{\Bg}}
\newcommand{\hcap}[0]{\hat{\Bh}}
\newcommand{\icap}[0]{\hat{\Bi}}
\newcommand{\jcap}[0]{\hat{\Bj}}
\newcommand{\kcap}[0]{\hat{\Bk}}
\newcommand{\lcap}[0]{\hat{\Bl}}
\newcommand{\mcap}[0]{\hat{\Bm}}
\newcommand{\ncap}[0]{\hat{\Bn}}
\newcommand{\ocap}[0]{\hat{\Bo}}
\newcommand{\pcap}[0]{\hat{\Bp}}
\newcommand{\qcap}[0]{\hat{\Bq}}
\newcommand{\rcap}[0]{\hat{\Br}}
\newcommand{\scap}[0]{\hat{\Bs}}
\newcommand{\tcap}[0]{\hat{\Bt}}
\newcommand{\ucap}[0]{\hat{\Bu}}
\newcommand{\vcap}[0]{\hat{\Bv}}
\newcommand{\wcap}[0]{\hat{\Bw}}
\newcommand{\xcap}[0]{\hat{\Bx}}
\newcommand{\ycap}[0]{\hat{\By}}
\newcommand{\zcap}[0]{\hat{\Bz}}
\newcommand{\thetacap}[0]{\hat{\Btheta}}

%
% to write R^n and C^n in a distinguishable fashion.  Perhaps change this
% to the double lined characters upon figuring out how to do so.
%
\newcommand{\C}[1]{$\mathbb{C}^{#1}$}
\newcommand{\R}[1]{$\mathbb{R}^{#1}$}

%
% various generally useful helpers
%

% derivative of #1 wrt. #2:
\newcommand{\D}[2] {\frac {d#2} {d#1}}

\newcommand{\inv}[1]{\frac{1}{#1}}
\newcommand{\cross}[0]{\times}

\newcommand{\abs}[1]{\lvert{#1}\rvert}
\newcommand{\norm}[1]{\lVert{#1}\rVert}
\newcommand{\innerprod}[2]{\langle{#1}, {#2}\rangle}
\newcommand{\dotprod}[2]{{#1} \cdot {#2}}
\newcommand{\bdotprod}[2]{\left({#1} \cdot {#2}\right)}
\newcommand{\crossprod}[2]{{#1} \cross {#2}}
\newcommand{\tripleprod}[3]{\dotprod{\left(\crossprod{#1}{#2}\right)}{#3}}

\DeclareMathOperator{\Proj}{Proj}
\DeclareMathOperator{\Span}{span}
\DeclareMathOperator{\Sgn}{sgn}
\DeclareMathOperator{\Area}{Area}
\DeclareMathOperator{\Volume}{Volume}

%
% A few miscellaneous things specific to this document
%
\newcommand{\crossop}[1]{\crossprod{#1}{}}

% R2 vector.
\newcommand{\VectorTwo}[2]{
\begin{bmatrix}
 {#1} \\
 {#2}
\end{bmatrix}
}

\newcommand{\VectorN}[1]{
\begin{bmatrix}
{#1}_1 \\
{#1}_2 \\
\vdots \\
{#1}_N \\
\end{bmatrix}
}

\newcommand{\DETuvij}[4]{
\begin{vmatrix}
 {#1}_{#3} & {#1}_{#4} \\
 {#2}_{#3} & {#2}_{#4}
\end{vmatrix}
}

\newcommand{\DETuvwijk}[6]{
\begin{vmatrix}
 {#1}_{#4} & {#1}_{#5} & {#1}_{#6} \\
 {#2}_{#4} & {#2}_{#5} & {#2}_{#6} \\
 {#3}_{#4} & {#3}_{#5} & {#3}_{#6}
\end{vmatrix}
}

\newcommand{\DETuvwxijkl}[8]{
\begin{vmatrix}
 {#1}_{#5} & {#1}_{#6} & {#1}_{#7} & {#1}_{#8} \\
 {#2}_{#5} & {#2}_{#6} & {#2}_{#7} & {#2}_{#8} \\
 {#3}_{#5} & {#3}_{#6} & {#3}_{#7} & {#3}_{#8} \\
 {#4}_{#5} & {#4}_{#6} & {#4}_{#7} & {#4}_{#8} \\
\end{vmatrix}
}

%\newcommand{\DETuvwxyijklm}[10]{
%\begin{vmatrix}
% {#1}_{#6} & {#1}_{#7} & {#1}_{#8} & {#1}_{#9} & {#1}_{#10} \\
% {#2}_{#6} & {#2}_{#7} & {#2}_{#8} & {#2}_{#9} & {#2}_{#10} \\
% {#3}_{#6} & {#3}_{#7} & {#3}_{#8} & {#3}_{#9} & {#3}_{#10} \\
% {#4}_{#6} & {#4}_{#7} & {#4}_{#8} & {#4}_{#9} & {#4}_{#10} \\
% {#5}_{#6} & {#5}_{#7} & {#5}_{#8} & {#5}_{#9} & {#5}_{#10}
%\end{vmatrix}
%}

% R3 vector.
\newcommand{\VectorThree}[3]{
\begin{bmatrix}
 {#1} \\
 {#2} \\
 {#3}
\end{bmatrix}
}



\author{Peeter Joot}
\email{peeter.joot@gmail.com}


\chapter{PHY450H1S.  Relativistic Electrodynamics Lecture 8 (Taught by Prof. Erich Poppitz).  Relativistic dynamics.}
\label{chap:relativisticElectrodynamicsL8}
%\useCCL
\blogpage{http://sites.google.com/site/peeterjoot/math2011/relativisticElectrodynamicsL8.pdf}
\date{Feb 1, 2011}
\revisionInfo{relativisticElectrodynamicsL8.tex}

%\beginArtWithToc
\beginArtNoToc

\section{Reading.}

Covering chapter 2 material from the text \cite{landau1980classical}.

Covering a bit more of \href{http://www.physics.utoronto.ca/~poppitz/e-poppitz/PHY450_files/RelEMpp52-56.pdf}{Professor Poppitz's lecture notes}: equation of motion, symmetries, and conserved quantities (energy-momentum 4 vector) from relativistic particle action.

\EndArticle

%
% Copyright � 2012 Peeter Joot.  All Rights Reserved.
% Licenced as described in the file LICENSE under the root directory of this GIT repository.
%

%\chapter{Dynamics in a vector field}
\label{chap:relativisticElectrodynamicsL9}
%\blogpage{http://sites.google.com/site/peeterjoot/math2011/relativisticElectrodynamicsL9.pdf}
%\date{Feb 2, 2011}

\paragraph{Reading}

Covering chapter 2 material from the text \citep{landau1980classical}, and
\popcite{RelEMpp56.1-73.pdf}{lecture notes RelEMpp56.1-73.pdf}.
%: comments on mass, energy, momentum, and massless particles (56.1-58); particles in external fields: Lorentz scalar field (59-62); reminder of a vector field under spatial rotations (63) and a Lorentz vector field (64-65) [Tuesday, Feb. 1]; the action for a relativistic particle in an external 4-vector field (65-66); the equation of motion of a relativistic particle in an external electromagnetic (4-vector) field (67,68,73) [Wednesday, Feb. 2]; mathematical interlude: (69-72): on 3x3 antisymmetric matrices, 3-vectors, and totally antisymmetric 3-index tensor - please read by yourselves, preferably by Wed., Feb. 2 class! (this is important, we will also soon need the 4-dimensional generalization)

\section{More on the action}

Action for a relativistic particle in an external 4-scalar field

\begin{equation}\label{eqn:relativisticElectrodynamicsL9:700}
S = -m c \int ds - g \int ds \phi(x)
\end{equation}

Unfortunately we have no 4-vector scalar fields (at least for particles that are long lived and stable).

PICTURE: 3-vector field, some arrows in various directions.

PICTURE: A vector \(\BA\) in an \(x,y\) frame, and a rotated (counterclockwise by angle \(\alpha\)) \(x', y'\) frame with the components in each shown pictorially.

We have

\begin{equation}\label{eqn:relativisticElectrodynamicsL9:710}
\begin{aligned}
A_x'(x', y') &= \cos\alpha A_x(x,y) + \sin\alpha A_y(x,y) \\
A_y'(x', y') &= -\sin\alpha A_x(x,y) + \cos\alpha A_y(x,y) 
\end{aligned}
\end{equation}

\begin{equation}\label{eqn:relativisticElectrodynamicsL9:20}
\begin{bmatrix}
A_x'(x', y') \\
A_y'(x', y')
\end{bmatrix}
=
\begin{bmatrix}
\cos\alpha A_x(x,y) & \sin\alpha A_y(x,y) \\
-\sin\alpha A_x(x,y) & \cos\alpha A_y(x,y) 
\end{bmatrix}
\begin{bmatrix}
A_x(x, y) \\
A_y(x, y)
\end{bmatrix}
\end{equation}

More generally we have

\begin{equation}\label{eqn:relativisticElectrodynamicsL9:40}
\begin{bmatrix}
A_x'(x', y', z') \\
A_y'(x', y', z') \\
A_z'(x', y', z')
\end{bmatrix}
=
\hat{O}
\begin{bmatrix}
A_x(x, y, z) \\
A_y(x, y, z) \\
A_z(x, y, z)
\end{bmatrix}
\end{equation}

Here \(\hat{O}\) is an \(SO(3)\) matrix rotating \(x \rightarrow x'\)

\begin{equation}\label{eqn:relativisticElectrodynamicsL9:60}
\BA(\Bx) \cdot \By = \BA'(\Bx') \cdot \By'
\end{equation}

\begin{equation}\label{eqn:relativisticElectrodynamicsL9:80}
\BA \cdot \BB = \text{invariant}
\end{equation}

A four vector field is \(A^i(x)\), with \(x = x^i, i = 0,1,2,3\) and we would write

\begin{equation}\label{eqn:relativisticElectrodynamicsL9:100}
\begin{bmatrix}
(x^0)' \\
(x^1)' \\
(x^2)' \\
(x^3)'
\end{bmatrix}
=
\hat{O}
\begin{bmatrix}
x^0 \\
x^1 \\
x^2 \\
x^3
\end{bmatrix}
\end{equation}

Now \(\hat{O}\) is an \(SO(1,3)\) matrix.    Our four vector field is then

\begin{equation}\label{eqn:relativisticElectrodynamicsL9:120}
\begin{bmatrix}
(A^0)' \\
(A^1)' \\
(A^2)' \\
(A^3)'
\end{bmatrix}
=
\hat{O}
\begin{bmatrix}
A^0 \\
A^1 \\
A^2 \\
A^3
\end{bmatrix}
\end{equation}

We have 

\begin{equation}\label{eqn:relativisticElectrodynamicsL9:140}
A^i g_{ij} x^i = \text{invariant} = {A'}^i g_{ij} {x'}^i 
\end{equation}

From electrodynamics we know that we have a scalar field, the electrostatic potential, and a vector field 

What is a plausible action?

How about

\begin{equation}\label{eqn:relativisticElectrodynamicsL9:160}
\int ds x^i g_{ij} A^j
\end{equation}

This is not translation invariant.

\begin{equation}\label{eqn:relativisticElectrodynamicsL9:180}
\int ds x^i g_{ij} A^j
\end{equation}

Next simplest is

\begin{equation}\label{eqn:relativisticElectrodynamicsL9:200}
\int ds u^i g_{ij} A^j
\end{equation}

Could also do

\begin{equation}\label{eqn:relativisticElectrodynamicsL9:220}
\int ds A^i g_{ij} A^j
\end{equation}

but it turns out that this is not gauge invariant (to be defined and discussed in detail).

\paragraph{An aside.  Dimensions of proper velocity.}

Note that the convention for this course is to write

\begin{equation}\label{eqn:relativisticElectrodynamicsL9:240}
u^i = \left( \gamma, \gamma \frac{\Bv}{c} \right) = \frac{dx^i}{ds}
\end{equation}

Where \(u^i\) is dimensionless (\(u^i u_i = 1\)).  Some authors use 

\begin{equation}\label{eqn:relativisticElectrodynamicsL9:260}
u^i = \left( \gamma c, \gamma \Bv \right) = \frac{dx^i}{d\tau},
\end{equation}

where \(u^i u_i = c^2\), and \(u^i\) has dimensions of velocity.

\paragraph{Return to the problem}

The simplest action for a four vector field \(A^i\) is then

\begin{equation}\label{eqn:relativisticElectrodynamicsL9:280}
S = - m c \int ds - \frac{e}{c} \int ds u^i A_i
\end{equation}

(Recall that \(u^i A_i = u^i g_{ij} A^j\)).

In this action \(e\) is nothing but a Lorentz scalar, a property of the particle that describes how it ``couples'' (or ``feels'') the electrodynamics field.

Similarly \(mc\) is a Lorentz scalar which is a property of the particle (inertia).

It turns out that all the electric charges in nature are quantized, and there are some deep reasons (in magnetic monopoles exist) for this.

Another reason for charge quantization apparently has to do with gauge invariance and associated compact groups.  Poppitz is amusing himself a bit here, hinting at some stuff that we can eventually learn.

Returning to our discussion, we have

\begin{equation}\label{eqn:relativisticElectrodynamicsL9:300}
S = - m c \int ds - \frac{e}{c} \int ds u^i g_{ij} A^j
\end{equation}

with the electrodynamics four vector potential

\begin{equation}\label{eqn:relativisticElectrodynamicsL9:320}
\begin{aligned}
A^i &= (\phi, \BA) \\
u^i &= \left(\gamma, \gamma \frac{\Bv}{c} \right) \\
u^i g_{ij} A^j &= \gamma \phi - \gamma \frac{\Bv \cdot \BA}{c}
\end{aligned}
\end{equation}

\begin{equation}\label{eqn:relativisticElectrodynamicsL9:730}
\begin{aligned}
S 
&= - m c^2 \int dt \sqrt{1 - \frac{\Bv^2}{c^2}} - \frac{e}{c} \int c dt \sqrt{1 - \frac{\Bv^2}{c^2}} \left( \gamma \phi - \gamma \frac{\Bv}{c} \cdot \BA \right) \\
&= \int dt \left(
- m c^2 \sqrt{1 - \frac{\Bv^2}{c^2}} - e \phi(\Bx, t) + \frac{e}{c} \Bv \cdot \BA(\Bx, t)
\right) \\
\end{aligned}
\end{equation}

\begin{equation}\label{eqn:relativisticElectrodynamicsL9:340}
\PD{\Bv}{\LL} = \frac{m c^2}{\sqrt{1 - \frac{\Bv^2}{c^2}}} \frac{\Bv}{c^2} + \frac{e}{c} \BA(\Bx, t)
\end{equation}

\begin{equation}\label{eqn:relativisticElectrodynamicsL9:360}
\frac{d}{dt} \PD{\Bv}{\LL} = m \frac{d}{dt} (\gamma \Bv) + \frac{e}{c} \PD{t}{\BA} + \frac{e}{c} \PD{x^\alpha}{\BA} v^\alpha
\end{equation}

Here \(\alpha,\beta = 1,2,3\) and are summed over.

For the other half of the Euler-Lagrange equations we have

\begin{equation}\label{eqn:relativisticElectrodynamicsL9:380}
\PD{x^\alpha}{\LL} = - e \PD{x^\alpha}{\phi} + \frac{e}{c} v^\beta \PD{x^\alpha}{A^\beta}
\end{equation}

Equating these, and switching to coordinates for \eqnref{eqn:relativisticElectrodynamicsL9:360}, we have

\begin{equation}\label{eqn:relativisticElectrodynamicsL9:381}
m \frac{d}{dt} (\gamma v^\alpha) + \frac{e}{c} \PD{t}{A^\alpha} + \frac{e}{c} \PD{x^\beta}{A^\alpha} v^\beta
= - e \PD{x^\alpha}{\phi} + \frac{e}{c} v^\beta \PD{x^\alpha}{A^\beta}
\end{equation}

A final rearrangement yields

\begin{equation}\label{eqn:relativisticElectrodynamicsL9:400}
\frac{d}{dt} m \gamma v^\alpha = e \mathLabelBox{\left( - \inv{c} \PD{t}{A^\alpha} - \PD{x^\alpha}{\phi} \right)}{\(E^\alpha\)} + \frac{e}{c} v^\beta \left( \PD{x^\alpha}{A^\beta} - \PD{x^\beta}{A^\alpha} \right)
\end{equation}

We can identity the second term with the magnetic field but first have to introduce antisymmetric matrices.

\section{antisymmetric matrices}

\begin{equation}\label{eqn:relativisticElectrodynamicsL9:750}
\begin{aligned}
M_{\mu\nu} 
&= \PD{x^\mu}{A^\nu} - \PD{x^\nu}{A^\mu} \\
&= \epsilon_{\mu\nu\lambda} B_\lambda,
\end{aligned}
\end{equation}

where

\begin{equation}\label{eqn:relativisticElectrodynamicsL9:440}
\epsilon_{\mu\nu\lambda} =
\begin{array}{l l}
0 & \quad \mbox{if any two indices coincide} \\
1 & \quad \mbox{for even permutations of \(\mu\nu\lambda\)} \\
-1 & \quad \mbox{for odd permutations of \(\mu\nu\lambda\)}
\end{array}
\end{equation}

Example:

\begin{equation}\label{eqn:relativisticElectrodynamicsL9:770}
\begin{aligned}
\epsilon_{123} &= 1 \\
\epsilon_{213} &= -1 \\
\epsilon_{231} &= 1.
\end{aligned}
\end{equation}

We can show that 

\begin{equation}\label{eqn:relativisticElectrodynamicsL9:420}
B_\lambda = \inv{2} \epsilon_{\lambda\mu\nu} M_{\mu\nu}
\end{equation}

\begin{equation}\label{eqn:relativisticElectrodynamicsL9:790}
\begin{aligned}
B_1 
&= \inv{2} ( \epsilon_{123} M_{23} + \epsilon_{132} M_{32})  \\
&= \inv{2} ( M_{23} - M_{32})  \\
&= \partial_2 A_3 - \partial_3 A_2.
\end{aligned}
\end{equation}

Using 

\begin{equation}\label{eqn:relativisticElectrodynamicsL9:430}
\epsilon_{\mu\nu\alpha} \epsilon_{\sigma\kappa\alpha} = 
\delta_{\mu\sigma} \delta_{\nu\kappa} - \delta_{\nu\sigma} \delta_{\mu\kappa},
\end{equation}

we can verify the identity \eqnref{eqn:relativisticElectrodynamicsL9:420} by expanding

\begin{equation}\label{eqn:relativisticElectrodynamicsL9:810}
\begin{aligned}
\epsilon_{\mu\nu\lambda} B_\lambda
&=
\inv{2} \epsilon_{\mu\nu\lambda} \epsilon_{\lambda\alpha\beta} M_{\alpha\beta} \\
&=
\inv{2} (
\delta_{\mu\alpha} \delta_{\nu\beta} - \delta_{\nu\alpha} \delta_{\mu\beta}
)
M_{\alpha\beta} \\
&=
\inv{2} (M_{\mu\nu} - M_{\nu\mu}) \\
&=
M_{\mu\nu}
\end{aligned}
\end{equation}

Returning to the action evaluation we have

\begin{equation}\label{eqn:relativisticElectrodynamicsL9:460}
\frac{d}{dt} ( m \gamma v^\alpha ) = e E^\alpha + \frac{e}{c} \epsilon_{\alpha\beta\gamma} v^\beta B_\gamma,
\end{equation}

but
\begin{equation}\label{eqn:relativisticElectrodynamicsL9:480}
\epsilon_{\alpha\beta\gamma} B_\gamma = (\Bv \cross \BB)_\alpha.
\end{equation}

So

\begin{equation}\label{eqn:relativisticElectrodynamicsL9:500}
\frac{d}{dt} ( m \gamma \Bv ) = e \BE + \frac{e}{c} \Bv \cross \BB
\end{equation}

or

\begin{equation}\label{eqn:relativisticElectrodynamicsL9:520}
\frac{d}{dt} ( \Bp ) = e \left( \BE + \frac{\Bv}{c} \cross \BB \right).
\end{equation}

\paragraph{What is the energy component of the Lorentz force equation}

I asked this, not because I do not know (I could answer this myself from \(dp/d\tau = F \cdot v/c\), in the geometric algebra formalism, but I was curious if he had a way of determining this from what we have derived so far (intuitively I had expect this to be possible).  Answer was:

Observe that this is almost a relativistic equation, but we are not going to get to the full equation yet.  The energy component can be obtained from

\begin{equation}\label{eqn:relativisticElectrodynamicsL9:540}
\frac{du^0}{ds} = e F^{0j} u_j
\end{equation}

Since the full equation is

\begin{equation}\label{eqn:relativisticElectrodynamicsL9:560}
\frac{du^i}{ds} = e F^{ij} u_j
\end{equation}

``take with a grain of salt, may be off by sign, or factors of \(c\)''.

Also curious is that he claimed the energy component of this equation was not very important.  Why would that be?

\section{Gauge transformations}

Claim

\begin{equation}\label{eqn:relativisticElectrodynamicsL9:580}
S_{\text{interaction}} = - \frac{e}{c} \int ds u^i A_i
\end{equation}

changes by boundary terms only under 

``gauge transformation'' :

\begin{equation}\label{eqn:relativisticElectrodynamicsL9:600}
A_i = A_i' + \PD{x^i}{\chi}
\end{equation}

where \(\chi\) is a Lorentz scalar.  This \(\PDi{x^i}{}\) is the four gradient.  Let us see this

Therefore the equations of motion are the same in an external \(A^i\) and \({A'}^i\).

Recall that the \(\BE\) and \(\BB\) fields do not change under such transformations.  Let us see how the action transforms

\begin{equation}\label{eqn:relativisticElectrodynamicsL9:830}
\begin{aligned}
S 
&= - \frac{e}{c} \int ds u^i A_i  \\
&= - \frac{e}{c} \int ds u^i \left( {A'}_i + \PD{x^i}{\chi} \right) \\
&= - \frac{e}{c} \int ds u^i {A'}_i  - \frac{e}{c} \int ds \frac{dx^i}{ds} \PD{x^i}{\chi} \\
\end{aligned}
\end{equation}

Observe that this last bit is just a chain rule expansion

\begin{equation}\label{eqn:relativisticElectrodynamicsL9:850}
\begin{aligned}
\frac{d}{ds} \chi(x^0, x^1, x^2, x^3) 
&= 
\PD{x^0}{\chi}\frac{dx^0}{ds} + 
\PD{x^1}{\chi}\frac{dx^1}{ds} + 
\PD{x^2}{\chi}\frac{dx^2}{ds} + 
\PD{x^3}{\chi}\frac{dx^3}{ds} \\
&= 
\PD{x^i}{\chi} \frac{dx^i}{ds},
\end{aligned}
\end{equation}

so we have
\begin{equation}\label{eqn:relativisticElectrodynamicsL9:610}
S 
= - \frac{e}{c} \int ds u^i {A'}_i - \frac{e}{c} \int ds \frac{d \chi}{ds}.
\end{equation}

This allows the line integral to be evaluated, and we find that it only depends on the end points of the interval

\begin{equation}\label{eqn:relativisticElectrodynamicsL9:620}
S = - \frac{e}{c} \int ds u^i {A'}_i - \frac{e}{c} ( \chi(x_b) - \chi(x_a) ),
\end{equation}

which completes the proof of the claim that this gauge transformation results in an action difference that only depends on the end points of the interval.  

\paragraph{Gauge invariance of \texorpdfstring{\(A \cdot A\)}{A squared} action}

Now that we know what gauge invariance means, let us look at the portion of the potential action \eqnref{eqn:relativisticElectrodynamicsL9:220} discarded because it was not gauge invariant.  Under gauge transformation this becomes

\begin{equation}\label{eqn:relativisticElectrodynamicsL9:870}
\begin{aligned}
\int ds {A'}^i {A'}_i
&=
\int ds \left({A}_i + \PD{x^i}{\chi}\right) \left(A^i + \PD{x_i}{\chi}\right) \\
&=
\int ds {A}^i A_i 
+ A^i \PD{x^i}{\chi}
+ A_i \PD{x_i}{\chi}
+ \PD{x^i}{\chi} \PD{x_i}{\chi} \\
&=
\int ds {A}^i A_i 
+ 2 A^i \PD{x^i}{\chi}
+ \PD{x^i}{\chi} \PD{x_i}{\chi}  \\
\end{aligned}
\end{equation}

Without the proper velocity term we do not have a way to simply re-pack the chain rule expansion and eliminate the last two terms as we did with the Lorentz force action.

%
% Copyright � 2015 Peeter Joot.  All Rights Reserved.
% Licenced as described in the file LICENSE under the root directory of this GIT repository.
%
\documentclass[]{eliblog}

\usepackage{amsmath}
\usepackage{mathpazo}

%
% shorthand for bold symbols, convenient for vectors and matrices
%
\newcommand{\Ba}[0]{\mathbf{a}}
\newcommand{\Bb}[0]{\mathbf{b}}
\newcommand{\Bc}[0]{\mathbf{c}}
\newcommand{\Bd}[0]{\mathbf{d}}
\newcommand{\Be}[0]{\mathbf{e}}
\newcommand{\Bf}[0]{\mathbf{f}}
\newcommand{\Bg}[0]{\mathbf{g}}
\newcommand{\Bh}[0]{\mathbf{h}}
\newcommand{\Bi}[0]{\mathbf{i}}
\newcommand{\Bj}[0]{\mathbf{j}}
\newcommand{\Bk}[0]{\mathbf{k}}
\newcommand{\Bl}[0]{\mathbf{l}}
\newcommand{\Bm}[0]{\mathbf{m}}
\newcommand{\Bn}[0]{\mathbf{n}}
\newcommand{\Bo}[0]{\mathbf{o}}
\newcommand{\Bp}[0]{\mathbf{p}}
\newcommand{\Bq}[0]{\mathbf{q}}
\newcommand{\Br}[0]{\mathbf{r}}
\newcommand{\Bs}[0]{\mathbf{s}}
\newcommand{\Bt}[0]{\mathbf{t}}
\newcommand{\Bu}[0]{\mathbf{u}}
\newcommand{\Bv}[0]{\mathbf{v}}
\newcommand{\Bw}[0]{\mathbf{w}}
\newcommand{\Bx}[0]{\mathbf{x}}
\newcommand{\By}[0]{\mathbf{y}}
\newcommand{\Bz}[0]{\mathbf{z}}
\newcommand{\BA}[0]{\mathbf{A}}
\newcommand{\BB}[0]{\mathbf{B}}
\newcommand{\BC}[0]{\mathbf{C}}
\newcommand{\BD}[0]{\mathbf{D}}
\newcommand{\BE}[0]{\mathbf{E}}
\newcommand{\BF}[0]{\mathbf{F}}
\newcommand{\BG}[0]{\mathbf{G}}
\newcommand{\BH}[0]{\mathbf{H}}
\newcommand{\BI}[0]{\mathbf{I}}
\newcommand{\BJ}[0]{\mathbf{J}}
\newcommand{\BK}[0]{\mathbf{K}}
\newcommand{\BL}[0]{\mathbf{L}}
\newcommand{\BM}[0]{\mathbf{M}}
\newcommand{\BN}[0]{\mathbf{N}}
\newcommand{\BO}[0]{\mathbf{O}}
\newcommand{\BP}[0]{\mathbf{P}}
\newcommand{\BQ}[0]{\mathbf{Q}}
\newcommand{\BR}[0]{\mathbf{R}}
\newcommand{\BS}[0]{\mathbf{S}}
\newcommand{\BT}[0]{\mathbf{T}}
\newcommand{\BU}[0]{\mathbf{U}}
\newcommand{\BV}[0]{\mathbf{V}}
\newcommand{\BW}[0]{\mathbf{W}}
\newcommand{\BX}[0]{\mathbf{X}}
\newcommand{\BY}[0]{\mathbf{Y}}
\newcommand{\BZ}[0]{\mathbf{Z}}

\newcommand{\Bzero}[0]{\mathbf{0}}
\newcommand{\Btheta}[0]{\boldsymbol{\theta}}
\newcommand{\Btau}[0]{\boldsymbol{\tau}}
\newcommand{\Bomega}[0]{\boldsymbol{\omega}}

%
% shorthand for unit vectors
%
\newcommand{\acap}[0]{\hat{\Ba}}
\newcommand{\bcap}[0]{\hat{\Bb}}
\newcommand{\ccap}[0]{\hat{\Bc}}
\newcommand{\dcap}[0]{\hat{\Bd}}
\newcommand{\ecap}[0]{\hat{\Be}}
\newcommand{\fcap}[0]{\hat{\Bf}}
\newcommand{\gcap}[0]{\hat{\Bg}}
\newcommand{\hcap}[0]{\hat{\Bh}}
\newcommand{\icap}[0]{\hat{\Bi}}
\newcommand{\jcap}[0]{\hat{\Bj}}
\newcommand{\kcap}[0]{\hat{\Bk}}
\newcommand{\lcap}[0]{\hat{\Bl}}
\newcommand{\mcap}[0]{\hat{\Bm}}
\newcommand{\ncap}[0]{\hat{\Bn}}
\newcommand{\ocap}[0]{\hat{\Bo}}
\newcommand{\pcap}[0]{\hat{\Bp}}
\newcommand{\qcap}[0]{\hat{\Bq}}
\newcommand{\rcap}[0]{\hat{\Br}}
\newcommand{\scap}[0]{\hat{\Bs}}
\newcommand{\tcap}[0]{\hat{\Bt}}
\newcommand{\ucap}[0]{\hat{\Bu}}
\newcommand{\vcap}[0]{\hat{\Bv}}
\newcommand{\wcap}[0]{\hat{\Bw}}
\newcommand{\xcap}[0]{\hat{\Bx}}
\newcommand{\ycap}[0]{\hat{\By}}
\newcommand{\zcap}[0]{\hat{\Bz}}
\newcommand{\thetacap}[0]{\hat{\Btheta}}

%
% to write R^n and C^n in a distinguishable fashion.  Perhaps change this
% to the double lined characters upon figuring out how to do so.
%
\newcommand{\C}[1]{$\mathbb{C}^{#1}$}
\newcommand{\R}[1]{$\mathbb{R}^{#1}$}

%
% various generally useful helpers
%

% derivative of #1 wrt. #2:
\newcommand{\D}[2] {\frac {d#2} {d#1}}

\newcommand{\inv}[1]{\frac{1}{#1}}
\newcommand{\cross}[0]{\times}

\newcommand{\abs}[1]{\lvert{#1}\rvert}
\newcommand{\norm}[1]{\lVert{#1}\rVert}
\newcommand{\innerprod}[2]{\langle{#1}, {#2}\rangle}
\newcommand{\dotprod}[2]{{#1} \cdot {#2}}
\newcommand{\bdotprod}[2]{\left({#1} \cdot {#2}\right)}
\newcommand{\crossprod}[2]{{#1} \cross {#2}}
\newcommand{\tripleprod}[3]{\dotprod{\left(\crossprod{#1}{#2}\right)}{#3}}

\DeclareMathOperator{\Proj}{Proj}
\DeclareMathOperator{\Span}{span}
\DeclareMathOperator{\Sgn}{sgn}
\DeclareMathOperator{\Area}{Area}
\DeclareMathOperator{\Volume}{Volume}

%
% A few miscellaneous things specific to this document
%
\newcommand{\crossop}[1]{\crossprod{#1}{}}

% R2 vector.
\newcommand{\VectorTwo}[2]{
\begin{bmatrix}
 {#1} \\
 {#2}
\end{bmatrix}
}

\newcommand{\VectorN}[1]{
\begin{bmatrix}
{#1}_1 \\
{#1}_2 \\
\vdots \\
{#1}_N \\
\end{bmatrix}
}

\newcommand{\DETuvij}[4]{
\begin{vmatrix}
 {#1}_{#3} & {#1}_{#4} \\
 {#2}_{#3} & {#2}_{#4}
\end{vmatrix}
}

\newcommand{\DETuvwijk}[6]{
\begin{vmatrix}
 {#1}_{#4} & {#1}_{#5} & {#1}_{#6} \\
 {#2}_{#4} & {#2}_{#5} & {#2}_{#6} \\
 {#3}_{#4} & {#3}_{#5} & {#3}_{#6}
\end{vmatrix}
}

\newcommand{\DETuvwxijkl}[8]{
\begin{vmatrix}
 {#1}_{#5} & {#1}_{#6} & {#1}_{#7} & {#1}_{#8} \\
 {#2}_{#5} & {#2}_{#6} & {#2}_{#7} & {#2}_{#8} \\
 {#3}_{#5} & {#3}_{#6} & {#3}_{#7} & {#3}_{#8} \\
 {#4}_{#5} & {#4}_{#6} & {#4}_{#7} & {#4}_{#8} \\
\end{vmatrix}
}

%\newcommand{\DETuvwxyijklm}[10]{
%\begin{vmatrix}
% {#1}_{#6} & {#1}_{#7} & {#1}_{#8} & {#1}_{#9} & {#1}_{#10} \\
% {#2}_{#6} & {#2}_{#7} & {#2}_{#8} & {#2}_{#9} & {#2}_{#10} \\
% {#3}_{#6} & {#3}_{#7} & {#3}_{#8} & {#3}_{#9} & {#3}_{#10} \\
% {#4}_{#6} & {#4}_{#7} & {#4}_{#8} & {#4}_{#9} & {#4}_{#10} \\
% {#5}_{#6} & {#5}_{#7} & {#5}_{#8} & {#5}_{#9} & {#5}_{#10}
%\end{vmatrix}
%}

% R3 vector.
\newcommand{\VectorThree}[3]{
\begin{bmatrix}
 {#1} \\
 {#2} \\
 {#3}
\end{bmatrix}
}



\author{Peeter Joot}
\email{peeter.joot@gmail.com}

%\documentclass[]{eliblogwidescreen}

\usepackage{amsmath}
\usepackage{mathpazo}

%
% shorthand for bold symbols, convenient for vectors and matrices
%
\newcommand{\Ba}[0]{\mathbf{a}}
\newcommand{\Bb}[0]{\mathbf{b}}
\newcommand{\Bc}[0]{\mathbf{c}}
\newcommand{\Bd}[0]{\mathbf{d}}
\newcommand{\Be}[0]{\mathbf{e}}
\newcommand{\Bf}[0]{\mathbf{f}}
\newcommand{\Bg}[0]{\mathbf{g}}
\newcommand{\Bh}[0]{\mathbf{h}}
\newcommand{\Bi}[0]{\mathbf{i}}
\newcommand{\Bj}[0]{\mathbf{j}}
\newcommand{\Bk}[0]{\mathbf{k}}
\newcommand{\Bl}[0]{\mathbf{l}}
\newcommand{\Bm}[0]{\mathbf{m}}
\newcommand{\Bn}[0]{\mathbf{n}}
\newcommand{\Bo}[0]{\mathbf{o}}
\newcommand{\Bp}[0]{\mathbf{p}}
\newcommand{\Bq}[0]{\mathbf{q}}
\newcommand{\Br}[0]{\mathbf{r}}
\newcommand{\Bs}[0]{\mathbf{s}}
\newcommand{\Bt}[0]{\mathbf{t}}
\newcommand{\Bu}[0]{\mathbf{u}}
\newcommand{\Bv}[0]{\mathbf{v}}
\newcommand{\Bw}[0]{\mathbf{w}}
\newcommand{\Bx}[0]{\mathbf{x}}
\newcommand{\By}[0]{\mathbf{y}}
\newcommand{\Bz}[0]{\mathbf{z}}
\newcommand{\BA}[0]{\mathbf{A}}
\newcommand{\BB}[0]{\mathbf{B}}
\newcommand{\BC}[0]{\mathbf{C}}
\newcommand{\BD}[0]{\mathbf{D}}
\newcommand{\BE}[0]{\mathbf{E}}
\newcommand{\BF}[0]{\mathbf{F}}
\newcommand{\BG}[0]{\mathbf{G}}
\newcommand{\BH}[0]{\mathbf{H}}
\newcommand{\BI}[0]{\mathbf{I}}
\newcommand{\BJ}[0]{\mathbf{J}}
\newcommand{\BK}[0]{\mathbf{K}}
\newcommand{\BL}[0]{\mathbf{L}}
\newcommand{\BM}[0]{\mathbf{M}}
\newcommand{\BN}[0]{\mathbf{N}}
\newcommand{\BO}[0]{\mathbf{O}}
\newcommand{\BP}[0]{\mathbf{P}}
\newcommand{\BQ}[0]{\mathbf{Q}}
\newcommand{\BR}[0]{\mathbf{R}}
\newcommand{\BS}[0]{\mathbf{S}}
\newcommand{\BT}[0]{\mathbf{T}}
\newcommand{\BU}[0]{\mathbf{U}}
\newcommand{\BV}[0]{\mathbf{V}}
\newcommand{\BW}[0]{\mathbf{W}}
\newcommand{\BX}[0]{\mathbf{X}}
\newcommand{\BY}[0]{\mathbf{Y}}
\newcommand{\BZ}[0]{\mathbf{Z}}

\newcommand{\Bzero}[0]{\mathbf{0}}
\newcommand{\Btheta}[0]{\boldsymbol{\theta}}
\newcommand{\Btau}[0]{\boldsymbol{\tau}}
\newcommand{\Bomega}[0]{\boldsymbol{\omega}}

%
% shorthand for unit vectors
%
\newcommand{\acap}[0]{\hat{\Ba}}
\newcommand{\bcap}[0]{\hat{\Bb}}
\newcommand{\ccap}[0]{\hat{\Bc}}
\newcommand{\dcap}[0]{\hat{\Bd}}
\newcommand{\ecap}[0]{\hat{\Be}}
\newcommand{\fcap}[0]{\hat{\Bf}}
\newcommand{\gcap}[0]{\hat{\Bg}}
\newcommand{\hcap}[0]{\hat{\Bh}}
\newcommand{\icap}[0]{\hat{\Bi}}
\newcommand{\jcap}[0]{\hat{\Bj}}
\newcommand{\kcap}[0]{\hat{\Bk}}
\newcommand{\lcap}[0]{\hat{\Bl}}
\newcommand{\mcap}[0]{\hat{\Bm}}
\newcommand{\ncap}[0]{\hat{\Bn}}
\newcommand{\ocap}[0]{\hat{\Bo}}
\newcommand{\pcap}[0]{\hat{\Bp}}
\newcommand{\qcap}[0]{\hat{\Bq}}
\newcommand{\rcap}[0]{\hat{\Br}}
\newcommand{\scap}[0]{\hat{\Bs}}
\newcommand{\tcap}[0]{\hat{\Bt}}
\newcommand{\ucap}[0]{\hat{\Bu}}
\newcommand{\vcap}[0]{\hat{\Bv}}
\newcommand{\wcap}[0]{\hat{\Bw}}
\newcommand{\xcap}[0]{\hat{\Bx}}
\newcommand{\ycap}[0]{\hat{\By}}
\newcommand{\zcap}[0]{\hat{\Bz}}
\newcommand{\thetacap}[0]{\hat{\Btheta}}

%
% to write R^n and C^n in a distinguishable fashion.  Perhaps change this
% to the double lined characters upon figuring out how to do so.
%
\newcommand{\C}[1]{$\mathbb{C}^{#1}$}
\newcommand{\R}[1]{$\mathbb{R}^{#1}$}

%
% various generally useful helpers
%

% derivative of #1 wrt. #2:
\newcommand{\D}[2] {\frac {d#2} {d#1}}

\newcommand{\inv}[1]{\frac{1}{#1}}
\newcommand{\cross}[0]{\times}

\newcommand{\abs}[1]{\lvert{#1}\rvert}
\newcommand{\norm}[1]{\lVert{#1}\rVert}
\newcommand{\innerprod}[2]{\langle{#1}, {#2}\rangle}
\newcommand{\dotprod}[2]{{#1} \cdot {#2}}
\newcommand{\bdotprod}[2]{\left({#1} \cdot {#2}\right)}
\newcommand{\crossprod}[2]{{#1} \cross {#2}}
\newcommand{\tripleprod}[3]{\dotprod{\left(\crossprod{#1}{#2}\right)}{#3}}

\DeclareMathOperator{\Proj}{Proj}
\DeclareMathOperator{\Span}{span}
\DeclareMathOperator{\Sgn}{sgn}
\DeclareMathOperator{\Area}{Area}
\DeclareMathOperator{\Volume}{Volume}

%
% A few miscellaneous things specific to this document
%
\newcommand{\crossop}[1]{\crossprod{#1}{}}

% R2 vector.
\newcommand{\VectorTwo}[2]{
\begin{bmatrix}
 {#1} \\
 {#2}
\end{bmatrix}
}

\newcommand{\VectorN}[1]{
\begin{bmatrix}
{#1}_1 \\
{#1}_2 \\
\vdots \\
{#1}_N \\
\end{bmatrix}
}

\newcommand{\DETuvij}[4]{
\begin{vmatrix}
 {#1}_{#3} & {#1}_{#4} \\
 {#2}_{#3} & {#2}_{#4}
\end{vmatrix}
}

\newcommand{\DETuvwijk}[6]{
\begin{vmatrix}
 {#1}_{#4} & {#1}_{#5} & {#1}_{#6} \\
 {#2}_{#4} & {#2}_{#5} & {#2}_{#6} \\
 {#3}_{#4} & {#3}_{#5} & {#3}_{#6}
\end{vmatrix}
}

\newcommand{\DETuvwxijkl}[8]{
\begin{vmatrix}
 {#1}_{#5} & {#1}_{#6} & {#1}_{#7} & {#1}_{#8} \\
 {#2}_{#5} & {#2}_{#6} & {#2}_{#7} & {#2}_{#8} \\
 {#3}_{#5} & {#3}_{#6} & {#3}_{#7} & {#3}_{#8} \\
 {#4}_{#5} & {#4}_{#6} & {#4}_{#7} & {#4}_{#8} \\
\end{vmatrix}
}

%\newcommand{\DETuvwxyijklm}[10]{
%\begin{vmatrix}
% {#1}_{#6} & {#1}_{#7} & {#1}_{#8} & {#1}_{#9} & {#1}_{#10} \\
% {#2}_{#6} & {#2}_{#7} & {#2}_{#8} & {#2}_{#9} & {#2}_{#10} \\
% {#3}_{#6} & {#3}_{#7} & {#3}_{#8} & {#3}_{#9} & {#3}_{#10} \\
% {#4}_{#6} & {#4}_{#7} & {#4}_{#8} & {#4}_{#9} & {#4}_{#10} \\
% {#5}_{#6} & {#5}_{#7} & {#5}_{#8} & {#5}_{#9} & {#5}_{#10}
%\end{vmatrix}
%}

% R3 vector.
\newcommand{\VectorThree}[3]{
\begin{bmatrix}
 {#1} \\
 {#2} \\
 {#3}
\end{bmatrix}
}



\author{Peeter Joot}
\email{peeter.joot@gmail.com}


\chapter{PHY450H1S.  Relativistic Electrodynamics Tutorial 3 (TA: Simon Freedman).  XXX FIXME TITLE XXX.}
\label{chap:relativisticElectrodynamicsT3}
%\useCCL
\blogpage{http://sites.google.com/site/peeterjoot/math2011/relativisticElectrodynamicsT3.pdf}
\date{Feb 3, 2011}
\revisionInfo{relativisticElectrodynamicsT3.tex}

\beginArtWithToc
%\beginArtNoToc

\section{Motion in an constant uniform Electric field.}

Given
\begin{equation}\label{eqn:relativisticElectrodynamicsT3:n}
\BE = E \xcap,
\end{equation}

We want to solve the problem
\begin{equation}\label{eqn:relativisticElectrodynamicsT3:n}
\BF = \frac{d\Bp}{dt} =
e ( \BE + \frac{\Bv}{c} \cross \BH ) = e ( \BE ).
\end{equation}

Unlike second year classical physics, we will use relativistic momentum, so for only a constant electric field, our Lorentz force equation to solve becomes

\begin{equation}\label{eqn:relativisticElectrodynamicsT3:n}
\frac{d\Bp}{dt} = \frac{d m \gamma \Bv}{dt} = e \BE.
\end{equation}

In components this is

\begin{align}\label{eqn:relativisticElectrodynamicsT3:n}
\pdot_x &= e E \\
\pdot_y &= \text{constant}
\end{align}

\begin{align*}
e E t + p_x(0)
&=
\frac{m \xdot}{\sqrt{1 - (\xdot^2 + \ydot^2)/c^2}} = e E t \\
&= 
\frac{m^2}{(e E t)^2} \xdot^2 = 1 - \frac{\xdot^2 - \ydot^2}{c^2}
\end{align*}

where we let $p_x(0) = 0$

\begin{equation}\label{eqn:relativisticElectrodynamicsT3:n}
\xdot^2 = \frac{c^2 - \ydot^2}{1 + (\frac{mc}{eEt})^2}
\end{equation}

With $p_y(0) = p_0$

\begin{align*}
\frac{m \ydot}{\sqrt{1 - (\xdot^2 + \ydot^2)/c^2}} &= p_0 \\
\implies \\
\frac{c^2 m^2}{p_0^2} \ydot^2 &= c^2 - \xdot^2 - \ydot^2 \\
\implies \\
\ydot^2 &= \frac{ p_0^2 (c^2 - \xdot^2)}{1 + \frac{m^2 c^2}{p_0^2}}
\end{align*}

Observe that 

\begin{equation}\label{eqn:relativisticElectrodynamicsT3:n}
\mathcal{E}^2 = p_0^2 c^2 + m^2 c^4
\end{equation}

so we can write

\begin{equation}\label{eqn:relativisticElectrodynamicsT3:n}
\ydot^2 = \frac{ c^2 p_0^2 (c^2 - \xdot^2)}{ \mathcal{E}_0^2 }
\end{equation}

Substitution yields

\begin{align}\label{eqn:relativisticElectrodynamicsT3:n}
\xdot &= \frac{c^2 e \mathcal{E} t}{\sqrt{ \mathcal{E}_0^2 + (e c \mathcal{E} t)^2 }} \\
\ydot &= \frac{c p_0^2 }{\sqrt{ \mathcal{E}_0^2 + (e c \mathcal{E} t)^2 }}
\end{align}

There's also a tricky way (as in the text), with 

\begin{align}\label{eqn:relativisticElectrodynamicsT3:n}
\Bp &= m \gamma \Bv  \\
\mathcal{E} &= \gamma m c^2 
\end{align}

Solve for $\Bp$ and get

\begin{equation}\label{eqn:relativisticElectrodynamicsT3:n}
\Bp = \frac{\mathcal{E} \Bv}{c^2},
\end{equation}

which implies

\begin{equation}\label{eqn:relativisticElectrodynamicsT3:n}
\xdot = \frac{c^2 p_x}{\mathcal{E}}.
\end{equation}

Solving for $x$ we have

\begin{equation}\label{eqn:relativisticElectrodynamicsT3:n}
x(t) = c^2 e E \int \frac{dt' t'}{\sqrt{ \mathcal{E}_0^2 + (e c E t)^2 }}
\end{equation}

Can solve with hyperbolic substitution or

\begin{equation}\label{eqn:relativisticElectrodynamicsT3:n}
x(t) = c^2 e E \int \frac{dt' t'}{\sqrt{ \mathcal{E}_0^2 + (e c E t)^2 }}
\end{equation}

\begin{equation}\label{eqn:relativisticElectrodynamicsT3:n}
du^2 = 2 u du \implies u du = \inv{2} d(u^2)
\end{equation}

\begin{equation}\label{eqn:relativisticElectrodynamicsT3:n}
x(t) = \frac{c^2 e E}{2 \mathcal{E}_0} \int \frac{d (u^2)}{\sqrt{ 1 + \left(\frac{e c E}{\mathcal{E}_0}\right)^2 u^2 }}
\end{equation}

Solving this we get

\begin{equation}\label{eqn:relativisticElectrodynamicsT3:n}
x(t) = \frac{e E}{\sqrt{ \mathcal{E}_0^2 + (e c E t)^2 }}
\end{equation}

or

\begin{equation}\label{eqn:relativisticElectrodynamicsT3:n}
x^2 - c^2 t^2 = \frac{E_0^2}{ e^2 E^2 } = a^{-2}
\end{equation}

\begin{equation}\label{eqn:relativisticElectrodynamicsT3:n}
y(t) = c^2 p_0^2 \int \frac{dt}{ \sqrt{\mathcal{E}_0^2 + (e c E t)^2 }}
\end{equation}

\begin{equation}\label{eqn:relativisticElectrodynamicsT3:n}
t = \frac{E_0}{ e c E} \sinh(u) 
\end{equation}

\begin{equation}\label{eqn:relativisticElectrodynamicsT3:n}
dt = \frac{E_0}{ e c E} \cosh(u) du
\end{equation}

\begin{align*}
y(t) 
&= \frac{c^2 p_0^2}{\mathcal{E}_0} \int \frac{dt}{\sqrt{1 + (\frac{e c E}{\mathcal{E}_0)})^2 t^2 }} \\
&= \frac{c^2 p_0^2}{\mathcal{E}_0} 
\frac{E_0}{ e c E} \cosh(u) du
\int \frac{ du \cosh u }{\sqrt{1 + (\sinh u)^2 }} \\
&= \frac{c p_0}{ e E} u 
\end{align*}

for

\begin{equation}\label{eqn:relativisticElectrodynamicsT3:n}
\boxed{
\begin{aligned}
y(t) &= \frac{c p_0}{ e E} \sinh^{-1} \left( \frac{e c E t}{\mathcal{E}_0} \right) \\
x(y) &= \frac{\mathcal{E}_0}{c E} \cosh( \frac{y e E }{ c p_0} )
\end{aligned}
}
\end{equation}

Checks:
\begin{equation}\label{eqn:relativisticElectrodynamicsT3:n}
v \rightarrow c, t \rightarrow \infty
\end{equation}

\begin{equation}\label{eqn:relativisticElectrodynamicsT3:n}
v << c, t \rightarrow 0
\end{equation}

\begin{align*}
m v_x &= e E t + ...
x &\sim t^2 
\end{align*}

\begin{equation}\label{eqn:relativisticElectrodynamicsT3:n}
m v_y = p_0 \rightarrow y \sim t
\end{equation}

\begin{equation}\label{eqn:relativisticElectrodynamicsT3:n}
x(y) \sim y^2
\end{equation}

(a parabola)

\section{Motion in an constant uniform Magnetic field.}

Note that the magnetic field does no work

\begin{equation}\label{eqn:relativisticElectrodynamicsT3:n}
\BF = \frac{e}{c} \Bv \cross \BH
\end{equation}

\begin{align*}
dW = 
\BF \cdot d\Bl
&=
\frac{e}{c} (\Bv \cross \BH) \cdot d\Bl
&= 0
\end{align*}

(perpendicular)

So our energy is only the initial time value

\begin{equation}\label{eqn:relativisticElectrodynamicsT3:n}
\mathcal{E} = .... + e A^0
\end{equation}

\begin{equation}\label{eqn:relativisticElectrodynamicsT3:n}
\mathcal{E}(t) = \mathcal{E}_0
\end{equation}

\begin{equation}\label{eqn:relativisticElectrodynamicsT3:n}
\BH = H \zcap
\end{equation}

\begin{equation}\label{eqn:relativisticElectrodynamicsT3:n}
\delta_{\alpha 3} H = H_\alpha
\end{equation}

\begin{equation}\label{eqn:relativisticElectrodynamicsT3:n}
\Bp = \mathcal{E} \frac{\Bv}{c^2} = \mathcal{E}_0 \frac{\Bv}{c^2}
\end{equation}

implies

\begin{equation}\label{eqn:relativisticElectrodynamicsT3:n}
\Bv = \Bp \frac{c^2}{\mathcal{E}_0}
\end{equation}

\begin{equation}\label{eqn:relativisticElectrodynamicsT3:n}
\dot{\Bv} = \dot{\Bp} \frac{c^2}{\mathcal{E}_0}
\end{equation}

\begin{equation}\label{eqn:relativisticElectrodynamicsT3:n}
\vdot_\alpha = \frac{e c}{\mathcal{E}_0} \epsilon_{\alpha \beta \gamma} v^\beta H_\gamma
\end{equation}

write

\begin{equation}\label{eqn:relativisticElectrodynamicsT3:n}
\omega = \frac{e c H}{\mathcal{E}_0}
\end{equation}

Evaluating the delta 
\begin{equation}\label{eqn:relativisticElectrodynamicsT3:n}
\vdot_\alpha = \omega \epsilon_{\alpha \beta 3} v_\beta 
\end{equation}
FIXME: SHOW.

\begin{align}\label{eqn:relativisticElectrodynamicsT3:n}
\vdot_1 &= \omega \epsilon_{1 \beta 3} v_\beta = \omega v_2 \\
\vdot_2 &= \omega \epsilon_{2 \beta 3} v_\beta = - \omega v_1 \\
\vdot_3 &= \omega \epsilon_{3 \beta 3} v_\beta = 0
\end{align}

Looks like circular motion, so it's natural to use complex variables.  With

\begin{equation}\label{eqn:relativisticElectrodynamicsT3:n}
z = v_1 + i v_2 
\end{equation}

We can show
FIXME: do this:

\begin{equation}\label{eqn:relativisticElectrodynamicsT3:n}
ddt ( v_1 + i v_2 ) = -i \omega ( v_1 + i v_2 ),
\end{equation}

for
\begin{equation}\label{eqn:relativisticElectrodynamicsT3:n}
z = V_0 e^{-i \omega z t + i \alpha}
\end{equation}

Real and imaginary parts

\begin{align}\label{eqn:relativisticElectrodynamicsT3:n}
v_1(t) &= V_0 \cos( \omega z t + \alpha) \\
v_2(t) &= -V_0 \sin( \omega z t + \alpha)
\end{align}

Integrating
\begin{align}\label{eqn:relativisticElectrodynamicsT3:n}
x_1(t) &= x_1(0) + V_0 \sin( \omega z t + \alpha) \\
x_2(t) &= x_2(0) + V_0 \cos( \omega z t + \alpha)
\end{align}

Which is a helix.
FIXME: PICTURE.

%\EndArticle
\EndNoBibArticle

%
% Copyright � 2015 Peeter Joot.  All Rights Reserved.
% Licenced as described in the file LICENSE under the root directory of this GIT repository.
%
\documentclass[]{eliblog}

\usepackage{amsmath}
\usepackage{mathpazo}

%
% shorthand for bold symbols, convenient for vectors and matrices
%
\newcommand{\Ba}[0]{\mathbf{a}}
\newcommand{\Bb}[0]{\mathbf{b}}
\newcommand{\Bc}[0]{\mathbf{c}}
\newcommand{\Bd}[0]{\mathbf{d}}
\newcommand{\Be}[0]{\mathbf{e}}
\newcommand{\Bf}[0]{\mathbf{f}}
\newcommand{\Bg}[0]{\mathbf{g}}
\newcommand{\Bh}[0]{\mathbf{h}}
\newcommand{\Bi}[0]{\mathbf{i}}
\newcommand{\Bj}[0]{\mathbf{j}}
\newcommand{\Bk}[0]{\mathbf{k}}
\newcommand{\Bl}[0]{\mathbf{l}}
\newcommand{\Bm}[0]{\mathbf{m}}
\newcommand{\Bn}[0]{\mathbf{n}}
\newcommand{\Bo}[0]{\mathbf{o}}
\newcommand{\Bp}[0]{\mathbf{p}}
\newcommand{\Bq}[0]{\mathbf{q}}
\newcommand{\Br}[0]{\mathbf{r}}
\newcommand{\Bs}[0]{\mathbf{s}}
\newcommand{\Bt}[0]{\mathbf{t}}
\newcommand{\Bu}[0]{\mathbf{u}}
\newcommand{\Bv}[0]{\mathbf{v}}
\newcommand{\Bw}[0]{\mathbf{w}}
\newcommand{\Bx}[0]{\mathbf{x}}
\newcommand{\By}[0]{\mathbf{y}}
\newcommand{\Bz}[0]{\mathbf{z}}
\newcommand{\BA}[0]{\mathbf{A}}
\newcommand{\BB}[0]{\mathbf{B}}
\newcommand{\BC}[0]{\mathbf{C}}
\newcommand{\BD}[0]{\mathbf{D}}
\newcommand{\BE}[0]{\mathbf{E}}
\newcommand{\BF}[0]{\mathbf{F}}
\newcommand{\BG}[0]{\mathbf{G}}
\newcommand{\BH}[0]{\mathbf{H}}
\newcommand{\BI}[0]{\mathbf{I}}
\newcommand{\BJ}[0]{\mathbf{J}}
\newcommand{\BK}[0]{\mathbf{K}}
\newcommand{\BL}[0]{\mathbf{L}}
\newcommand{\BM}[0]{\mathbf{M}}
\newcommand{\BN}[0]{\mathbf{N}}
\newcommand{\BO}[0]{\mathbf{O}}
\newcommand{\BP}[0]{\mathbf{P}}
\newcommand{\BQ}[0]{\mathbf{Q}}
\newcommand{\BR}[0]{\mathbf{R}}
\newcommand{\BS}[0]{\mathbf{S}}
\newcommand{\BT}[0]{\mathbf{T}}
\newcommand{\BU}[0]{\mathbf{U}}
\newcommand{\BV}[0]{\mathbf{V}}
\newcommand{\BW}[0]{\mathbf{W}}
\newcommand{\BX}[0]{\mathbf{X}}
\newcommand{\BY}[0]{\mathbf{Y}}
\newcommand{\BZ}[0]{\mathbf{Z}}

\newcommand{\Bzero}[0]{\mathbf{0}}
\newcommand{\Btheta}[0]{\boldsymbol{\theta}}
\newcommand{\Btau}[0]{\boldsymbol{\tau}}
\newcommand{\Bomega}[0]{\boldsymbol{\omega}}

%
% shorthand for unit vectors
%
\newcommand{\acap}[0]{\hat{\Ba}}
\newcommand{\bcap}[0]{\hat{\Bb}}
\newcommand{\ccap}[0]{\hat{\Bc}}
\newcommand{\dcap}[0]{\hat{\Bd}}
\newcommand{\ecap}[0]{\hat{\Be}}
\newcommand{\fcap}[0]{\hat{\Bf}}
\newcommand{\gcap}[0]{\hat{\Bg}}
\newcommand{\hcap}[0]{\hat{\Bh}}
\newcommand{\icap}[0]{\hat{\Bi}}
\newcommand{\jcap}[0]{\hat{\Bj}}
\newcommand{\kcap}[0]{\hat{\Bk}}
\newcommand{\lcap}[0]{\hat{\Bl}}
\newcommand{\mcap}[0]{\hat{\Bm}}
\newcommand{\ncap}[0]{\hat{\Bn}}
\newcommand{\ocap}[0]{\hat{\Bo}}
\newcommand{\pcap}[0]{\hat{\Bp}}
\newcommand{\qcap}[0]{\hat{\Bq}}
\newcommand{\rcap}[0]{\hat{\Br}}
\newcommand{\scap}[0]{\hat{\Bs}}
\newcommand{\tcap}[0]{\hat{\Bt}}
\newcommand{\ucap}[0]{\hat{\Bu}}
\newcommand{\vcap}[0]{\hat{\Bv}}
\newcommand{\wcap}[0]{\hat{\Bw}}
\newcommand{\xcap}[0]{\hat{\Bx}}
\newcommand{\ycap}[0]{\hat{\By}}
\newcommand{\zcap}[0]{\hat{\Bz}}
\newcommand{\thetacap}[0]{\hat{\Btheta}}

%
% to write R^n and C^n in a distinguishable fashion.  Perhaps change this
% to the double lined characters upon figuring out how to do so.
%
\newcommand{\C}[1]{$\mathbb{C}^{#1}$}
\newcommand{\R}[1]{$\mathbb{R}^{#1}$}

%
% various generally useful helpers
%

% derivative of #1 wrt. #2:
\newcommand{\D}[2] {\frac {d#2} {d#1}}

\newcommand{\inv}[1]{\frac{1}{#1}}
\newcommand{\cross}[0]{\times}

\newcommand{\abs}[1]{\lvert{#1}\rvert}
\newcommand{\norm}[1]{\lVert{#1}\rVert}
\newcommand{\innerprod}[2]{\langle{#1}, {#2}\rangle}
\newcommand{\dotprod}[2]{{#1} \cdot {#2}}
\newcommand{\bdotprod}[2]{\left({#1} \cdot {#2}\right)}
\newcommand{\crossprod}[2]{{#1} \cross {#2}}
\newcommand{\tripleprod}[3]{\dotprod{\left(\crossprod{#1}{#2}\right)}{#3}}

\DeclareMathOperator{\Proj}{Proj}
\DeclareMathOperator{\Span}{span}
\DeclareMathOperator{\Sgn}{sgn}
\DeclareMathOperator{\Area}{Area}
\DeclareMathOperator{\Volume}{Volume}

%
% A few miscellaneous things specific to this document
%
\newcommand{\crossop}[1]{\crossprod{#1}{}}

% R2 vector.
\newcommand{\VectorTwo}[2]{
\begin{bmatrix}
 {#1} \\
 {#2}
\end{bmatrix}
}

\newcommand{\VectorN}[1]{
\begin{bmatrix}
{#1}_1 \\
{#1}_2 \\
\vdots \\
{#1}_N \\
\end{bmatrix}
}

\newcommand{\DETuvij}[4]{
\begin{vmatrix}
 {#1}_{#3} & {#1}_{#4} \\
 {#2}_{#3} & {#2}_{#4}
\end{vmatrix}
}

\newcommand{\DETuvwijk}[6]{
\begin{vmatrix}
 {#1}_{#4} & {#1}_{#5} & {#1}_{#6} \\
 {#2}_{#4} & {#2}_{#5} & {#2}_{#6} \\
 {#3}_{#4} & {#3}_{#5} & {#3}_{#6}
\end{vmatrix}
}

\newcommand{\DETuvwxijkl}[8]{
\begin{vmatrix}
 {#1}_{#5} & {#1}_{#6} & {#1}_{#7} & {#1}_{#8} \\
 {#2}_{#5} & {#2}_{#6} & {#2}_{#7} & {#2}_{#8} \\
 {#3}_{#5} & {#3}_{#6} & {#3}_{#7} & {#3}_{#8} \\
 {#4}_{#5} & {#4}_{#6} & {#4}_{#7} & {#4}_{#8} \\
\end{vmatrix}
}

%\newcommand{\DETuvwxyijklm}[10]{
%\begin{vmatrix}
% {#1}_{#6} & {#1}_{#7} & {#1}_{#8} & {#1}_{#9} & {#1}_{#10} \\
% {#2}_{#6} & {#2}_{#7} & {#2}_{#8} & {#2}_{#9} & {#2}_{#10} \\
% {#3}_{#6} & {#3}_{#7} & {#3}_{#8} & {#3}_{#9} & {#3}_{#10} \\
% {#4}_{#6} & {#4}_{#7} & {#4}_{#8} & {#4}_{#9} & {#4}_{#10} \\
% {#5}_{#6} & {#5}_{#7} & {#5}_{#8} & {#5}_{#9} & {#5}_{#10}
%\end{vmatrix}
%}

% R3 vector.
\newcommand{\VectorThree}[3]{
\begin{bmatrix}
 {#1} \\
 {#2} \\
 {#3}
\end{bmatrix}
}



\author{Peeter Joot}
\email{peeter.joot@gmail.com}

%\documentclass[]{eliblogwidescreen}

\usepackage{amsmath}
\usepackage{mathpazo}

%
% shorthand for bold symbols, convenient for vectors and matrices
%
\newcommand{\Ba}[0]{\mathbf{a}}
\newcommand{\Bb}[0]{\mathbf{b}}
\newcommand{\Bc}[0]{\mathbf{c}}
\newcommand{\Bd}[0]{\mathbf{d}}
\newcommand{\Be}[0]{\mathbf{e}}
\newcommand{\Bf}[0]{\mathbf{f}}
\newcommand{\Bg}[0]{\mathbf{g}}
\newcommand{\Bh}[0]{\mathbf{h}}
\newcommand{\Bi}[0]{\mathbf{i}}
\newcommand{\Bj}[0]{\mathbf{j}}
\newcommand{\Bk}[0]{\mathbf{k}}
\newcommand{\Bl}[0]{\mathbf{l}}
\newcommand{\Bm}[0]{\mathbf{m}}
\newcommand{\Bn}[0]{\mathbf{n}}
\newcommand{\Bo}[0]{\mathbf{o}}
\newcommand{\Bp}[0]{\mathbf{p}}
\newcommand{\Bq}[0]{\mathbf{q}}
\newcommand{\Br}[0]{\mathbf{r}}
\newcommand{\Bs}[0]{\mathbf{s}}
\newcommand{\Bt}[0]{\mathbf{t}}
\newcommand{\Bu}[0]{\mathbf{u}}
\newcommand{\Bv}[0]{\mathbf{v}}
\newcommand{\Bw}[0]{\mathbf{w}}
\newcommand{\Bx}[0]{\mathbf{x}}
\newcommand{\By}[0]{\mathbf{y}}
\newcommand{\Bz}[0]{\mathbf{z}}
\newcommand{\BA}[0]{\mathbf{A}}
\newcommand{\BB}[0]{\mathbf{B}}
\newcommand{\BC}[0]{\mathbf{C}}
\newcommand{\BD}[0]{\mathbf{D}}
\newcommand{\BE}[0]{\mathbf{E}}
\newcommand{\BF}[0]{\mathbf{F}}
\newcommand{\BG}[0]{\mathbf{G}}
\newcommand{\BH}[0]{\mathbf{H}}
\newcommand{\BI}[0]{\mathbf{I}}
\newcommand{\BJ}[0]{\mathbf{J}}
\newcommand{\BK}[0]{\mathbf{K}}
\newcommand{\BL}[0]{\mathbf{L}}
\newcommand{\BM}[0]{\mathbf{M}}
\newcommand{\BN}[0]{\mathbf{N}}
\newcommand{\BO}[0]{\mathbf{O}}
\newcommand{\BP}[0]{\mathbf{P}}
\newcommand{\BQ}[0]{\mathbf{Q}}
\newcommand{\BR}[0]{\mathbf{R}}
\newcommand{\BS}[0]{\mathbf{S}}
\newcommand{\BT}[0]{\mathbf{T}}
\newcommand{\BU}[0]{\mathbf{U}}
\newcommand{\BV}[0]{\mathbf{V}}
\newcommand{\BW}[0]{\mathbf{W}}
\newcommand{\BX}[0]{\mathbf{X}}
\newcommand{\BY}[0]{\mathbf{Y}}
\newcommand{\BZ}[0]{\mathbf{Z}}

\newcommand{\Bzero}[0]{\mathbf{0}}
\newcommand{\Btheta}[0]{\boldsymbol{\theta}}
\newcommand{\Btau}[0]{\boldsymbol{\tau}}
\newcommand{\Bomega}[0]{\boldsymbol{\omega}}

%
% shorthand for unit vectors
%
\newcommand{\acap}[0]{\hat{\Ba}}
\newcommand{\bcap}[0]{\hat{\Bb}}
\newcommand{\ccap}[0]{\hat{\Bc}}
\newcommand{\dcap}[0]{\hat{\Bd}}
\newcommand{\ecap}[0]{\hat{\Be}}
\newcommand{\fcap}[0]{\hat{\Bf}}
\newcommand{\gcap}[0]{\hat{\Bg}}
\newcommand{\hcap}[0]{\hat{\Bh}}
\newcommand{\icap}[0]{\hat{\Bi}}
\newcommand{\jcap}[0]{\hat{\Bj}}
\newcommand{\kcap}[0]{\hat{\Bk}}
\newcommand{\lcap}[0]{\hat{\Bl}}
\newcommand{\mcap}[0]{\hat{\Bm}}
\newcommand{\ncap}[0]{\hat{\Bn}}
\newcommand{\ocap}[0]{\hat{\Bo}}
\newcommand{\pcap}[0]{\hat{\Bp}}
\newcommand{\qcap}[0]{\hat{\Bq}}
\newcommand{\rcap}[0]{\hat{\Br}}
\newcommand{\scap}[0]{\hat{\Bs}}
\newcommand{\tcap}[0]{\hat{\Bt}}
\newcommand{\ucap}[0]{\hat{\Bu}}
\newcommand{\vcap}[0]{\hat{\Bv}}
\newcommand{\wcap}[0]{\hat{\Bw}}
\newcommand{\xcap}[0]{\hat{\Bx}}
\newcommand{\ycap}[0]{\hat{\By}}
\newcommand{\zcap}[0]{\hat{\Bz}}
\newcommand{\thetacap}[0]{\hat{\Btheta}}

%
% to write R^n and C^n in a distinguishable fashion.  Perhaps change this
% to the double lined characters upon figuring out how to do so.
%
\newcommand{\C}[1]{$\mathbb{C}^{#1}$}
\newcommand{\R}[1]{$\mathbb{R}^{#1}$}

%
% various generally useful helpers
%

% derivative of #1 wrt. #2:
\newcommand{\D}[2] {\frac {d#2} {d#1}}

\newcommand{\inv}[1]{\frac{1}{#1}}
\newcommand{\cross}[0]{\times}

\newcommand{\abs}[1]{\lvert{#1}\rvert}
\newcommand{\norm}[1]{\lVert{#1}\rVert}
\newcommand{\innerprod}[2]{\langle{#1}, {#2}\rangle}
\newcommand{\dotprod}[2]{{#1} \cdot {#2}}
\newcommand{\bdotprod}[2]{\left({#1} \cdot {#2}\right)}
\newcommand{\crossprod}[2]{{#1} \cross {#2}}
\newcommand{\tripleprod}[3]{\dotprod{\left(\crossprod{#1}{#2}\right)}{#3}}

\DeclareMathOperator{\Proj}{Proj}
\DeclareMathOperator{\Span}{span}
\DeclareMathOperator{\Sgn}{sgn}
\DeclareMathOperator{\Area}{Area}
\DeclareMathOperator{\Volume}{Volume}

%
% A few miscellaneous things specific to this document
%
\newcommand{\crossop}[1]{\crossprod{#1}{}}

% R2 vector.
\newcommand{\VectorTwo}[2]{
\begin{bmatrix}
 {#1} \\
 {#2}
\end{bmatrix}
}

\newcommand{\VectorN}[1]{
\begin{bmatrix}
{#1}_1 \\
{#1}_2 \\
\vdots \\
{#1}_N \\
\end{bmatrix}
}

\newcommand{\DETuvij}[4]{
\begin{vmatrix}
 {#1}_{#3} & {#1}_{#4} \\
 {#2}_{#3} & {#2}_{#4}
\end{vmatrix}
}

\newcommand{\DETuvwijk}[6]{
\begin{vmatrix}
 {#1}_{#4} & {#1}_{#5} & {#1}_{#6} \\
 {#2}_{#4} & {#2}_{#5} & {#2}_{#6} \\
 {#3}_{#4} & {#3}_{#5} & {#3}_{#6}
\end{vmatrix}
}

\newcommand{\DETuvwxijkl}[8]{
\begin{vmatrix}
 {#1}_{#5} & {#1}_{#6} & {#1}_{#7} & {#1}_{#8} \\
 {#2}_{#5} & {#2}_{#6} & {#2}_{#7} & {#2}_{#8} \\
 {#3}_{#5} & {#3}_{#6} & {#3}_{#7} & {#3}_{#8} \\
 {#4}_{#5} & {#4}_{#6} & {#4}_{#7} & {#4}_{#8} \\
\end{vmatrix}
}

%\newcommand{\DETuvwxyijklm}[10]{
%\begin{vmatrix}
% {#1}_{#6} & {#1}_{#7} & {#1}_{#8} & {#1}_{#9} & {#1}_{#10} \\
% {#2}_{#6} & {#2}_{#7} & {#2}_{#8} & {#2}_{#9} & {#2}_{#10} \\
% {#3}_{#6} & {#3}_{#7} & {#3}_{#8} & {#3}_{#9} & {#3}_{#10} \\
% {#4}_{#6} & {#4}_{#7} & {#4}_{#8} & {#4}_{#9} & {#4}_{#10} \\
% {#5}_{#6} & {#5}_{#7} & {#5}_{#8} & {#5}_{#9} & {#5}_{#10}
%\end{vmatrix}
%}

% R3 vector.
\newcommand{\VectorThree}[3]{
\begin{bmatrix}
 {#1} \\
 {#2} \\
 {#3}
\end{bmatrix}
}



\author{Peeter Joot}
\email{peeter.joot@gmail.com}


\chapter{PHY450H1S.  Relativistic Electrodynamics Lecture 10 (Taught by Prof. Erich Poppitz).  Lorentz force equation energy term, and four vector formulation of the Lorentz force equation.}
\label{chap:relativisticElectrodynamicsL10}
%\useCCL
\blogpage{http://sites.google.com/site/peeterjoot/math2011/relativisticElectrodynamicsL10.pdf}
\date{Feb 8, 2011}
\revisionInfo{relativisticElectrodynamicsL10.tex}

%\beginArtWithToc
\beginArtNoToc

\section{Reading.}

Covering chapter 3 material from the text \cite{landau1980classical}.

Covering \href{http://www.physics.utoronto.ca/~poppitz/e-poppitz/PHY450_files/RelEMpp74-83.pdf}{lecture notes pp. 74-83}: gauge transformations in 3-vector language (74); energy of a relativistic particle in EM field (75); variational principle and equation of motion in 4-vector form (76-77); the field strength tensor (78-80); the fourth equation of motion (81)

\section{What is the significance to the gauge invariance of the action?}

We had argued that under a gauge transformation

\begin{equation}\label{eqn:relativisticElectrodynamicsL10:n}
A_i \rightarrow A_i + \PD{x^i}{\chi},
\end{equation}

the action for a particle changes by a boundary term 

\begin{equation}\label{eqn:relativisticElectrodynamicsL10:n}
- \frac{e}{c} ( \chi(x_b) - \chi(x_a) ).
\end{equation}

Because $S$ changes by a boundary term only, variation problem is not affected.  The extremal trajectories are then the same, hence the EOM are the same.

\subsection{A less high brow demonstration.}

With our four potential split into space and time components
\begin{equation}\label{eqn:relativisticElectrodynamicsL10:n}
A^i = (\phi, \BA),
\end{equation}

the lower index representation of the same vector is

\begin{equation}\label{eqn:relativisticElectrodynamicsL10:n}
A_i = (\phi, -\BA).
\end{equation}

Our gauge transformation is then

\begin{align}\label{eqn:relativisticElectrodynamicsL10:n}
A_0 &\rightarrow A_0 + \PD{x^0}{\chi} \\
-\BA &\rightarrow -\BA + \PD{\Bx}{\chi}
\end{align}

or
\begin{align}\label{eqn:relativisticElectrodynamicsL10:n}
\phi &\rightarrow \phi + \inv{c}\PD{t}{\chi} \\
\BA &\rightarrow \BA - \spacegrad \chi.
\end{align}

Now observe how the electric and magnetic fields are transformed

\begin{align*}
\BE 
&= - \spacegrad \phi - \inv{c} \PD{t}{\BA} \\
&\rightarrow 
- \spacegrad \left( \phi + \inv{c}\PD{t}{\chi} \right) - \inv{c}\PD{t}{} \left( \BA - \spacegrad \chi \right) \\
\end{align*}

Sufficent continuity of $\chi$ is assumed, allowing commutation of the space and time derivatives, and we are left with just $\BE$

For the magnetic field we have

\begin{align*}
\BB 
&= \spacegrad \cross \BA  \\
&\rightarrow 
\spacegrad \cross (\BA  - \spacegrad \chi) \\
\end{align*}

Again with continuity assumptions, $\spacegrad \cross (\spacegrad \chi) = 0$, and we are left with just $\BB$.  The electromagnetic fields (as opposed to potentials) do not change under grauge transformations.

We conclude that the $\{A_i\}$ description is hugely redundant, but despite that, local $\LL$ and $H$ can only be written in terms of the potentials $A_i$.

\subsection{Energy term of the Lorentz force.  Three vector approach.}

With the Lagrangian for the particle given by

\begin{equation}\label{eqn:relativisticElectrodynamicsL10:n}
\LL = - mc^2 \InvGamma + \frac{e}{c} \BA \cdot \Bv - e \phi,
\end{equation}

we define the energy as 

\begin{equation}\label{eqn:relativisticElectrodynamicsL10:n}
\mathcal{E} = \Bv \cdot \PD{\Bv}{\LL} - \LL
\end{equation}

This is not neccessarily a conserved quantity, but we define it as the energy anyways (we don't really have a Hamiltonian when the fields are time dependent).  Associated with this quantity is the general relationship

\begin{equation}\label{eqn:relativisticElectrodynamicsL10:n}
\ddt{\mathcal{E}} = -\PD{t}{\LL},
\end{equation}

and when the Lagrangian is invariant with respect to time translation the energy $\mathcal{E}$ will be a conserved quantity (and also the Hamiltonian).

Our canonical momentum is 
\begin{equation}\label{eqn:relativisticElectrodynamicsL10:n}
\PD{\Bv}{\LL} = \gamma m \Bv + \frac{e}{c} \BA
\end{equation}

So our energy is
\begin{align*}
\mathcal{E} = \gamma m \Bv^2 + \frac{e}{c} \BA \cdot \Bv - \left( - mc^2 \InvGamma + \frac{e}{c} \BA \cdot \Bv - e \phi \right).
\end{align*}

Or
\begin{equation}\label{eqn:relativisticElectrodynamicsL10:n}
\mathcal{E} = \underbrace{\frac{m c^2}{\InvGamma}}_{(\conj)} + e \phi.
\end{equation}

The contribution of $(\conj)$ to the energy $\mathcal{E}$ comes from the free (kinetic) particle portion of the Lagrangian $\LL = -m c^2 \InvGamma$, and we identify the remainder as a potential energy 

\begin{equation}\label{eqn:relativisticElectrodynamicsL10:n}
\mathcal{E} = \frac{m c^2}{\InvGamma} + \underbrace{e \phi}_{\text{"potential"}}.
\end{equation}

For the kinetic portion we can also show that we have
\begin{equation}\label{eqn:relativisticElectrodynamicsL10:n}
\frac{d}{dt} \mathcal{E}_{\text{kinetic}} 
=
\frac{m c^2}{\InvGamma} 
= e \BE \cdot \Bv.
\end{equation}

To show this observe that we have

\begin{align*}
\frac{d}{dt} \mathcal{E}_{\text{kinetic}} 
&= m c^2 \frac{d\gamma}{dt} \\
&= m c^2 \frac{d}{dt} \inv{\InvGamma} \\
&= m c^2 \frac{\frac{\Bv}{c^2} \cdot \frac{d\Bv}{dt}}{\left(1 - \frac{\Bv^2}{c^2}\right)^{3/2}} \\
&= \frac{m \gamma \Bv \cdot \frac{d\Bv}{dt}}{1 - \frac{\Bv^2}{c^2}}
\end{align*}

We also have

\begin{align*}
\Bv \cdot \ddt{\Bp} 
&= \Bv \cdot \ddt{} \frac{m \Bv}{\InvGamma} \\
&= m\Bv^2 \ddt{\gamma} + m \gamma \Bv \cdot \ddt{\Bv} \\
&= m\Bv^2 \ddt{\gamma} + m c^2 \ddt{\gamma} \left( 1 - \frac{\Bv^2}{c^2} \right) \\
&= m c^2 \ddt{\gamma}.
\end{align*}

Utilizing the Lorentz force equation, we have

\begin{equation}\label{eqn:relativisticElectrodynamicsL10:n}
\Bv \cdot \ddt{\Bp} = e \left( \BE + \frac{\Bv}{c} \cross \BB \right) \cdot \Bv = e \BE \cdot \Bv
\end{equation}

and are able to assemble the above, and find that we have
\begin{equation}\label{eqn:relativisticElectrodynamicsL10:n}
\ddt{(m c^2 \gamma)} = e \BE \cdot \Bv 
\end{equation}

\section{Four vector Lorentz force}

Using $ds = \sqrt{ dx^i dx_i } $ our action can be rewritten

\begin{align*}
S 
&= \int \left( -m c ds - \frac{e}{c} u^i A_i ds \right) \\
&= \int \left( -m c ds - \frac{e}{c} dx^i A_i \right) \\
&= \int \left( -m c \sqrt{ dx^i dx_i} - \frac{e}{c} dx^i A_i \right) \\
\end{align*}

$x^i(\tau)$ is a worldline $x^i(0) = a^i$, $x^i(1) = b^i$, 

We want $\delta S = S[ x + \delta x ] - S[ x ] = 0$ (to linear order in $\delta x$)

The variation of our proper length is
\begin{align*}
\delta ds 
&=
\delta \sqrt{ dx^i dx_i } \\
&= \inv{ 2 \sqrt{ dx^i dx_i }} \delta (dx^j dx_j)
\end{align*}

Observe that for the numerator we have
\begin{align*}
\delta (dx^j dx_j) 
&= \delta ( dx^j g_{jk} dx^k ) \\
&= \delta ( dx^j ) g_{jk} dx^k + dx^j g_{jk} \delta ( dx^k ) \\
&= \delta ( dx^j ) g_{jk} dx^k + dx^k g_{kj} \delta ( dx^j ) \\
&= 2 \delta ( dx^j ) g_{jk} dx^k \\
&= 2 \delta ( dx^j ) dx_j 
\end{align*}

\paragraph{TIP:} If this goes to quick, or there is any disbelief, write these all out explicitly as $dx^j dx_j = dx^0 dx_0 + dx^1 dx_1 + dx^2 dx_2 + dx^3 dx_3$ and compute it that way.

For the four vector potential our variation is

\begin{equation}\label{eqn:relativisticElectrodynamicsL10:n}
\delta A_i = A_i(x + \delta x) - A_i = \PD{x^j}{A_i} \delta x^j = \partial_j A_i \delta x^j
\end{equation}

(i.e. By chain rule)

Completing the proper length variations above we have

\begin{align*}
\delta \sqrt{ dx^i dx_i } 
&= \inv{ \sqrt{ dx^i dx_i }} \delta (dx^j) dx_j \\
&= \delta (dx^j) \dds{x_j}  \\
&= \delta (dx^j) u_j \\
&= d \delta x^j u_j
\end{align*}

We are now ready to assemble results and do the integration by parts

\begin{align*}
\delta S 
&= \int \left( 
-m c d (\delta x^j) u_j
- \frac{e}{c} d (\delta x^i) A_i 
- \frac{e}{c} dx^i \partial_j A_i \delta x^j
\right) \\
&= 
{\left. 
\left( -m c (\delta x^j) u_j - \frac{e}{c} (\delta x^i) A_i \right)
\right\vert}_a^b
+\int \left( 
m c \delta x^j d u_j
+ \frac{e}{c} (\delta x^i) d A_i 
- \frac{e}{c} dx^i \partial_j A_i \delta x^j
\right) \\
\end{align*}

Our variation at the endpoints is zero $\evalbar{\delta x^i}{a} = \evalbar{\delta x^i}{b} = 0$, killing the non-integral terms

\begin{align*}
\delta S 
&= 
\int 
\delta x^j
\left( 
m c d u_j
+ \frac{e}{c} d A_j 
- \frac{e}{c} dx^i \partial_j A_i 
\right).
\end{align*}

Observe that our differential can also be expanded by chain rule

\begin{equation}\label{eqn:relativisticElectrodynamicsL10:n}
d A_j = \PD{x^i}{A_j} dx^i = \partial_i A_j dx^i,
\end{equation}

which simplifies the variation further

\begin{equation}\label{eqn:relativisticElectrodynamicsL10:n}
\delta S 
= 
\int 
\delta x^j
\left( 
m c d u_j
+ \frac{e}{c} dx^i ( \partial_i A_j - \partial_j A_i )
\right).
\end{equation}

%%%and our action variance is now
%%%
%%%ROUGH CLASS NOTES:
%%%\begin{align*}
%%%\delta S 
%%%&= \int ( -m c \delta \sqrt{ dx^i dx_i} - \frac{e}{c} d \delta x^i A_i -\frac{e}{c} dx^i \delta A_i ) \\
%%%&= \int ( -m c \frac{ dx^j g_{jk} d \delta x^k }{\sqrt{ dx^i dx_i}} - \frac{e}{c} d \delta x^i A_i - \frac{e}{c} dx^i \PD{x^j}{A_i} \delta x^j ) \\
%%%&= \int ( -m c u^j g_{jk} d \delta x^k - \frac{e}{c} d \delta x^i A_i - \frac{e}{c} dx^i \PD{x^j}{A_i} \delta x^j ) \\
%%%&= \int ( -m c u^j g_{jk} \delta x^k + d u^j g_{jk} \delta x^k m c + d ( -\frac{e}{c} \delta x^i A_i + \frac{e}{c} \delta x^i d A_i ) - \frac{e}{c} dx^i \PD{x^j}{A_i} \delta x^j ) \\
%%%&= \int ( -m c u^j g_{jk} d \delta x^k + d u^j g_{jk} + d ( -\frac{e}{c} \delta x^i A_i + \frac{e}{c} \delta x^i d A_i ) - \frac{e}{c} dx^i \PD{x^j}{A_i} \delta x^j ) \\
%%%\end{align*}

%%%\delta S = 
%%%\int d ( - m c u^j g_{jk} \delta x^k - \frac{e}{c} A_i \delta x^i
%%%+ \int m c d u^i g_{ij} \delta x^j + \frac{e}{c} \delta x^l d A_l - \frac{e}{c} \delta x^i \PD x^i A_j d x^j
%%%
%%%The first term is
%%%
%%% - mc u^i \delta x_i - \frac{e}{c} A_i \delta x^i \vert^b_a = 0
%%%
%%%
%%%We are left with
%%%
%%%\int ds m c \dds u^j g_{jk} \delta x^k + \frac{e}{c} \delta x^l \PD x^k A_l \dds x^k ds - \frac{e}{c} \PD x^l A_i \dds x^i ds
%%%\int ds \delta x^l  ...
%%%
%%%\delta S = \int ds \delta x^l ( m c \dds u_l + \frac{e}{c} u^k ( \partial_k A_l - \partial_l A_k )
%%%
%%%Since this is true for all variations $\delta x^l$, which is arbitrary, the interior part is zero everywhere in the tragectory
%%%
%%%m c \dds u_l 
%%%= - \frac{e}{c} ( \PD x^l A_k - \PD x_k A_l ) u^k
%%%= \frac{e}{c} ( \PD x^k A_l - \PD x_l A_k ) u^k
%%%
%%%m c \dds u_l = \frac{e}{c} ( \underbrace{ \PD x^k A_l - \PD x_l A_k }_{F_{lk} }) u^k
%%%
%%%The quantity $F_{lk}$ is called the electromagnetic field strength tensor ( 4-tensor, rank-2 antisymmetric ).
%%%
%%%We write
%%%
%%%\dds m c u_l = \frac{e}{c} F_{lk} u^k 
%%%
%%%\Abs{ F_{lk} } = 
%%%\begin{bmatrix}
%%%0 & E_x & E_y & E_z \\
%%%-E_x & 0 & -B_z & B_y \\
%%%-E_y & B_z & 0 & -B_x \\
%%%-E_z & -B_y & B_x & 0
%%%\end{bmatrix}

\EndArticle

%
% Copyright � 2015 Peeter Joot.  All Rights Reserved.
% Licenced as described in the file LICENSE under the root directory of this GIT repository.
%
\documentclass[]{eliblog}

\usepackage{amsmath}
\usepackage{mathpazo}

%
% shorthand for bold symbols, convenient for vectors and matrices
%
\newcommand{\Ba}[0]{\mathbf{a}}
\newcommand{\Bb}[0]{\mathbf{b}}
\newcommand{\Bc}[0]{\mathbf{c}}
\newcommand{\Bd}[0]{\mathbf{d}}
\newcommand{\Be}[0]{\mathbf{e}}
\newcommand{\Bf}[0]{\mathbf{f}}
\newcommand{\Bg}[0]{\mathbf{g}}
\newcommand{\Bh}[0]{\mathbf{h}}
\newcommand{\Bi}[0]{\mathbf{i}}
\newcommand{\Bj}[0]{\mathbf{j}}
\newcommand{\Bk}[0]{\mathbf{k}}
\newcommand{\Bl}[0]{\mathbf{l}}
\newcommand{\Bm}[0]{\mathbf{m}}
\newcommand{\Bn}[0]{\mathbf{n}}
\newcommand{\Bo}[0]{\mathbf{o}}
\newcommand{\Bp}[0]{\mathbf{p}}
\newcommand{\Bq}[0]{\mathbf{q}}
\newcommand{\Br}[0]{\mathbf{r}}
\newcommand{\Bs}[0]{\mathbf{s}}
\newcommand{\Bt}[0]{\mathbf{t}}
\newcommand{\Bu}[0]{\mathbf{u}}
\newcommand{\Bv}[0]{\mathbf{v}}
\newcommand{\Bw}[0]{\mathbf{w}}
\newcommand{\Bx}[0]{\mathbf{x}}
\newcommand{\By}[0]{\mathbf{y}}
\newcommand{\Bz}[0]{\mathbf{z}}
\newcommand{\BA}[0]{\mathbf{A}}
\newcommand{\BB}[0]{\mathbf{B}}
\newcommand{\BC}[0]{\mathbf{C}}
\newcommand{\BD}[0]{\mathbf{D}}
\newcommand{\BE}[0]{\mathbf{E}}
\newcommand{\BF}[0]{\mathbf{F}}
\newcommand{\BG}[0]{\mathbf{G}}
\newcommand{\BH}[0]{\mathbf{H}}
\newcommand{\BI}[0]{\mathbf{I}}
\newcommand{\BJ}[0]{\mathbf{J}}
\newcommand{\BK}[0]{\mathbf{K}}
\newcommand{\BL}[0]{\mathbf{L}}
\newcommand{\BM}[0]{\mathbf{M}}
\newcommand{\BN}[0]{\mathbf{N}}
\newcommand{\BO}[0]{\mathbf{O}}
\newcommand{\BP}[0]{\mathbf{P}}
\newcommand{\BQ}[0]{\mathbf{Q}}
\newcommand{\BR}[0]{\mathbf{R}}
\newcommand{\BS}[0]{\mathbf{S}}
\newcommand{\BT}[0]{\mathbf{T}}
\newcommand{\BU}[0]{\mathbf{U}}
\newcommand{\BV}[0]{\mathbf{V}}
\newcommand{\BW}[0]{\mathbf{W}}
\newcommand{\BX}[0]{\mathbf{X}}
\newcommand{\BY}[0]{\mathbf{Y}}
\newcommand{\BZ}[0]{\mathbf{Z}}

\newcommand{\Bzero}[0]{\mathbf{0}}
\newcommand{\Btheta}[0]{\boldsymbol{\theta}}
\newcommand{\Btau}[0]{\boldsymbol{\tau}}
\newcommand{\Bomega}[0]{\boldsymbol{\omega}}

%
% shorthand for unit vectors
%
\newcommand{\acap}[0]{\hat{\Ba}}
\newcommand{\bcap}[0]{\hat{\Bb}}
\newcommand{\ccap}[0]{\hat{\Bc}}
\newcommand{\dcap}[0]{\hat{\Bd}}
\newcommand{\ecap}[0]{\hat{\Be}}
\newcommand{\fcap}[0]{\hat{\Bf}}
\newcommand{\gcap}[0]{\hat{\Bg}}
\newcommand{\hcap}[0]{\hat{\Bh}}
\newcommand{\icap}[0]{\hat{\Bi}}
\newcommand{\jcap}[0]{\hat{\Bj}}
\newcommand{\kcap}[0]{\hat{\Bk}}
\newcommand{\lcap}[0]{\hat{\Bl}}
\newcommand{\mcap}[0]{\hat{\Bm}}
\newcommand{\ncap}[0]{\hat{\Bn}}
\newcommand{\ocap}[0]{\hat{\Bo}}
\newcommand{\pcap}[0]{\hat{\Bp}}
\newcommand{\qcap}[0]{\hat{\Bq}}
\newcommand{\rcap}[0]{\hat{\Br}}
\newcommand{\scap}[0]{\hat{\Bs}}
\newcommand{\tcap}[0]{\hat{\Bt}}
\newcommand{\ucap}[0]{\hat{\Bu}}
\newcommand{\vcap}[0]{\hat{\Bv}}
\newcommand{\wcap}[0]{\hat{\Bw}}
\newcommand{\xcap}[0]{\hat{\Bx}}
\newcommand{\ycap}[0]{\hat{\By}}
\newcommand{\zcap}[0]{\hat{\Bz}}
\newcommand{\thetacap}[0]{\hat{\Btheta}}

%
% to write R^n and C^n in a distinguishable fashion.  Perhaps change this
% to the double lined characters upon figuring out how to do so.
%
\newcommand{\C}[1]{$\mathbb{C}^{#1}$}
\newcommand{\R}[1]{$\mathbb{R}^{#1}$}

%
% various generally useful helpers
%

% derivative of #1 wrt. #2:
\newcommand{\D}[2] {\frac {d#2} {d#1}}

\newcommand{\inv}[1]{\frac{1}{#1}}
\newcommand{\cross}[0]{\times}

\newcommand{\abs}[1]{\lvert{#1}\rvert}
\newcommand{\norm}[1]{\lVert{#1}\rVert}
\newcommand{\innerprod}[2]{\langle{#1}, {#2}\rangle}
\newcommand{\dotprod}[2]{{#1} \cdot {#2}}
\newcommand{\bdotprod}[2]{\left({#1} \cdot {#2}\right)}
\newcommand{\crossprod}[2]{{#1} \cross {#2}}
\newcommand{\tripleprod}[3]{\dotprod{\left(\crossprod{#1}{#2}\right)}{#3}}

\DeclareMathOperator{\Proj}{Proj}
\DeclareMathOperator{\Span}{span}
\DeclareMathOperator{\Sgn}{sgn}
\DeclareMathOperator{\Area}{Area}
\DeclareMathOperator{\Volume}{Volume}

%
% A few miscellaneous things specific to this document
%
\newcommand{\crossop}[1]{\crossprod{#1}{}}

% R2 vector.
\newcommand{\VectorTwo}[2]{
\begin{bmatrix}
 {#1} \\
 {#2}
\end{bmatrix}
}

\newcommand{\VectorN}[1]{
\begin{bmatrix}
{#1}_1 \\
{#1}_2 \\
\vdots \\
{#1}_N \\
\end{bmatrix}
}

\newcommand{\DETuvij}[4]{
\begin{vmatrix}
 {#1}_{#3} & {#1}_{#4} \\
 {#2}_{#3} & {#2}_{#4}
\end{vmatrix}
}

\newcommand{\DETuvwijk}[6]{
\begin{vmatrix}
 {#1}_{#4} & {#1}_{#5} & {#1}_{#6} \\
 {#2}_{#4} & {#2}_{#5} & {#2}_{#6} \\
 {#3}_{#4} & {#3}_{#5} & {#3}_{#6}
\end{vmatrix}
}

\newcommand{\DETuvwxijkl}[8]{
\begin{vmatrix}
 {#1}_{#5} & {#1}_{#6} & {#1}_{#7} & {#1}_{#8} \\
 {#2}_{#5} & {#2}_{#6} & {#2}_{#7} & {#2}_{#8} \\
 {#3}_{#5} & {#3}_{#6} & {#3}_{#7} & {#3}_{#8} \\
 {#4}_{#5} & {#4}_{#6} & {#4}_{#7} & {#4}_{#8} \\
\end{vmatrix}
}

%\newcommand{\DETuvwxyijklm}[10]{
%\begin{vmatrix}
% {#1}_{#6} & {#1}_{#7} & {#1}_{#8} & {#1}_{#9} & {#1}_{#10} \\
% {#2}_{#6} & {#2}_{#7} & {#2}_{#8} & {#2}_{#9} & {#2}_{#10} \\
% {#3}_{#6} & {#3}_{#7} & {#3}_{#8} & {#3}_{#9} & {#3}_{#10} \\
% {#4}_{#6} & {#4}_{#7} & {#4}_{#8} & {#4}_{#9} & {#4}_{#10} \\
% {#5}_{#6} & {#5}_{#7} & {#5}_{#8} & {#5}_{#9} & {#5}_{#10}
%\end{vmatrix}
%}

% R3 vector.
\newcommand{\VectorThree}[3]{
\begin{bmatrix}
 {#1} \\
 {#2} \\
 {#3}
\end{bmatrix}
}



\author{Peeter Joot}
\email{peeter.joot@gmail.com}

%\documentclass[]{eliblogwidescreen}

\usepackage{amsmath}
\usepackage{mathpazo}

%
% shorthand for bold symbols, convenient for vectors and matrices
%
\newcommand{\Ba}[0]{\mathbf{a}}
\newcommand{\Bb}[0]{\mathbf{b}}
\newcommand{\Bc}[0]{\mathbf{c}}
\newcommand{\Bd}[0]{\mathbf{d}}
\newcommand{\Be}[0]{\mathbf{e}}
\newcommand{\Bf}[0]{\mathbf{f}}
\newcommand{\Bg}[0]{\mathbf{g}}
\newcommand{\Bh}[0]{\mathbf{h}}
\newcommand{\Bi}[0]{\mathbf{i}}
\newcommand{\Bj}[0]{\mathbf{j}}
\newcommand{\Bk}[0]{\mathbf{k}}
\newcommand{\Bl}[0]{\mathbf{l}}
\newcommand{\Bm}[0]{\mathbf{m}}
\newcommand{\Bn}[0]{\mathbf{n}}
\newcommand{\Bo}[0]{\mathbf{o}}
\newcommand{\Bp}[0]{\mathbf{p}}
\newcommand{\Bq}[0]{\mathbf{q}}
\newcommand{\Br}[0]{\mathbf{r}}
\newcommand{\Bs}[0]{\mathbf{s}}
\newcommand{\Bt}[0]{\mathbf{t}}
\newcommand{\Bu}[0]{\mathbf{u}}
\newcommand{\Bv}[0]{\mathbf{v}}
\newcommand{\Bw}[0]{\mathbf{w}}
\newcommand{\Bx}[0]{\mathbf{x}}
\newcommand{\By}[0]{\mathbf{y}}
\newcommand{\Bz}[0]{\mathbf{z}}
\newcommand{\BA}[0]{\mathbf{A}}
\newcommand{\BB}[0]{\mathbf{B}}
\newcommand{\BC}[0]{\mathbf{C}}
\newcommand{\BD}[0]{\mathbf{D}}
\newcommand{\BE}[0]{\mathbf{E}}
\newcommand{\BF}[0]{\mathbf{F}}
\newcommand{\BG}[0]{\mathbf{G}}
\newcommand{\BH}[0]{\mathbf{H}}
\newcommand{\BI}[0]{\mathbf{I}}
\newcommand{\BJ}[0]{\mathbf{J}}
\newcommand{\BK}[0]{\mathbf{K}}
\newcommand{\BL}[0]{\mathbf{L}}
\newcommand{\BM}[0]{\mathbf{M}}
\newcommand{\BN}[0]{\mathbf{N}}
\newcommand{\BO}[0]{\mathbf{O}}
\newcommand{\BP}[0]{\mathbf{P}}
\newcommand{\BQ}[0]{\mathbf{Q}}
\newcommand{\BR}[0]{\mathbf{R}}
\newcommand{\BS}[0]{\mathbf{S}}
\newcommand{\BT}[0]{\mathbf{T}}
\newcommand{\BU}[0]{\mathbf{U}}
\newcommand{\BV}[0]{\mathbf{V}}
\newcommand{\BW}[0]{\mathbf{W}}
\newcommand{\BX}[0]{\mathbf{X}}
\newcommand{\BY}[0]{\mathbf{Y}}
\newcommand{\BZ}[0]{\mathbf{Z}}

\newcommand{\Bzero}[0]{\mathbf{0}}
\newcommand{\Btheta}[0]{\boldsymbol{\theta}}
\newcommand{\Btau}[0]{\boldsymbol{\tau}}
\newcommand{\Bomega}[0]{\boldsymbol{\omega}}

%
% shorthand for unit vectors
%
\newcommand{\acap}[0]{\hat{\Ba}}
\newcommand{\bcap}[0]{\hat{\Bb}}
\newcommand{\ccap}[0]{\hat{\Bc}}
\newcommand{\dcap}[0]{\hat{\Bd}}
\newcommand{\ecap}[0]{\hat{\Be}}
\newcommand{\fcap}[0]{\hat{\Bf}}
\newcommand{\gcap}[0]{\hat{\Bg}}
\newcommand{\hcap}[0]{\hat{\Bh}}
\newcommand{\icap}[0]{\hat{\Bi}}
\newcommand{\jcap}[0]{\hat{\Bj}}
\newcommand{\kcap}[0]{\hat{\Bk}}
\newcommand{\lcap}[0]{\hat{\Bl}}
\newcommand{\mcap}[0]{\hat{\Bm}}
\newcommand{\ncap}[0]{\hat{\Bn}}
\newcommand{\ocap}[0]{\hat{\Bo}}
\newcommand{\pcap}[0]{\hat{\Bp}}
\newcommand{\qcap}[0]{\hat{\Bq}}
\newcommand{\rcap}[0]{\hat{\Br}}
\newcommand{\scap}[0]{\hat{\Bs}}
\newcommand{\tcap}[0]{\hat{\Bt}}
\newcommand{\ucap}[0]{\hat{\Bu}}
\newcommand{\vcap}[0]{\hat{\Bv}}
\newcommand{\wcap}[0]{\hat{\Bw}}
\newcommand{\xcap}[0]{\hat{\Bx}}
\newcommand{\ycap}[0]{\hat{\By}}
\newcommand{\zcap}[0]{\hat{\Bz}}
\newcommand{\thetacap}[0]{\hat{\Btheta}}

%
% to write R^n and C^n in a distinguishable fashion.  Perhaps change this
% to the double lined characters upon figuring out how to do so.
%
\newcommand{\C}[1]{$\mathbb{C}^{#1}$}
\newcommand{\R}[1]{$\mathbb{R}^{#1}$}

%
% various generally useful helpers
%

% derivative of #1 wrt. #2:
\newcommand{\D}[2] {\frac {d#2} {d#1}}

\newcommand{\inv}[1]{\frac{1}{#1}}
\newcommand{\cross}[0]{\times}

\newcommand{\abs}[1]{\lvert{#1}\rvert}
\newcommand{\norm}[1]{\lVert{#1}\rVert}
\newcommand{\innerprod}[2]{\langle{#1}, {#2}\rangle}
\newcommand{\dotprod}[2]{{#1} \cdot {#2}}
\newcommand{\bdotprod}[2]{\left({#1} \cdot {#2}\right)}
\newcommand{\crossprod}[2]{{#1} \cross {#2}}
\newcommand{\tripleprod}[3]{\dotprod{\left(\crossprod{#1}{#2}\right)}{#3}}

\DeclareMathOperator{\Proj}{Proj}
\DeclareMathOperator{\Span}{span}
\DeclareMathOperator{\Sgn}{sgn}
\DeclareMathOperator{\Area}{Area}
\DeclareMathOperator{\Volume}{Volume}

%
% A few miscellaneous things specific to this document
%
\newcommand{\crossop}[1]{\crossprod{#1}{}}

% R2 vector.
\newcommand{\VectorTwo}[2]{
\begin{bmatrix}
 {#1} \\
 {#2}
\end{bmatrix}
}

\newcommand{\VectorN}[1]{
\begin{bmatrix}
{#1}_1 \\
{#1}_2 \\
\vdots \\
{#1}_N \\
\end{bmatrix}
}

\newcommand{\DETuvij}[4]{
\begin{vmatrix}
 {#1}_{#3} & {#1}_{#4} \\
 {#2}_{#3} & {#2}_{#4}
\end{vmatrix}
}

\newcommand{\DETuvwijk}[6]{
\begin{vmatrix}
 {#1}_{#4} & {#1}_{#5} & {#1}_{#6} \\
 {#2}_{#4} & {#2}_{#5} & {#2}_{#6} \\
 {#3}_{#4} & {#3}_{#5} & {#3}_{#6}
\end{vmatrix}
}

\newcommand{\DETuvwxijkl}[8]{
\begin{vmatrix}
 {#1}_{#5} & {#1}_{#6} & {#1}_{#7} & {#1}_{#8} \\
 {#2}_{#5} & {#2}_{#6} & {#2}_{#7} & {#2}_{#8} \\
 {#3}_{#5} & {#3}_{#6} & {#3}_{#7} & {#3}_{#8} \\
 {#4}_{#5} & {#4}_{#6} & {#4}_{#7} & {#4}_{#8} \\
\end{vmatrix}
}

%\newcommand{\DETuvwxyijklm}[10]{
%\begin{vmatrix}
% {#1}_{#6} & {#1}_{#7} & {#1}_{#8} & {#1}_{#9} & {#1}_{#10} \\
% {#2}_{#6} & {#2}_{#7} & {#2}_{#8} & {#2}_{#9} & {#2}_{#10} \\
% {#3}_{#6} & {#3}_{#7} & {#3}_{#8} & {#3}_{#9} & {#3}_{#10} \\
% {#4}_{#6} & {#4}_{#7} & {#4}_{#8} & {#4}_{#9} & {#4}_{#10} \\
% {#5}_{#6} & {#5}_{#7} & {#5}_{#8} & {#5}_{#9} & {#5}_{#10}
%\end{vmatrix}
%}

% R3 vector.
\newcommand{\VectorThree}[3]{
\begin{bmatrix}
 {#1} \\
 {#2} \\
 {#3}
\end{bmatrix}
}



\author{Peeter Joot}
\email{peeter.joot@gmail.com}


\chapter{PHY450H1S.  Relativistic Electrodynamics Lecture 11 (Taught by Prof. Erich Poppitz).  Unpacking Lorentz force equation.  Lorentz transformations of the strength tensor, Lorentz field invariants, Bianchi identity, and first half of Maxwell's.}
\label{chap:relativisticElectrodynamicsL11}
%\useCCL
\blogpage{http://sites.google.com/site/peeterjoot/math2011/relativisticElectrodynamicsL11.pdf}
\date{Feb 9, 2011}
\revisionInfo{relativisticElectrodynamicsL11.tex}

\beginArtWithToc
%\beginArtNoToc

\section{Reading.}

Covering chapter 3 material from the text \cite{landau1980classical}.

Covering \href{http://www.physics.utoronto.ca/~poppitz/e-poppitz/PHY450_files/RelEMpp74-83.pdf}{lecture notes pp. 74-83}: Lorentz transformation of the strength tensor (82) [Tuesday, Feb. 8] [extra reading for the mathematically minded: gauge field, strength tensor, and gauge transformations in differential form language, not to be covered in class (83)] 

Covering \href{http://www.physics.utoronto.ca/~poppitz/e-poppitz/PHY450_files/RelEMpp84-102.pdf}{lecture notes pp. 84-102}: Lorentz invariants of the electromagnetic field (84-86); Bianchi identity and the first half of Maxwell's equations (87-90)

\section{Chewing on the four vector form of the Lorentz force equation.}

After much effort, we arrived at 

\begin{equation}\label{eqn:relativisticElectrodynamicsL11:10}
\dds{(m c u_l) } = \frac{e}{c} \left( \partial_l A_i - \partial_i A_l \right) u^i
\end{equation}

or
\begin{equation}\label{eqn:relativisticElectrodynamicsL11:30}
\dds{ p_l } = \frac{e}{c} F_{li} u^i
\end{equation}

\subsection{Elements of the strength tensor}

\paragraph{Claim}: there are only 6 independent elements of this matrix (tensor)

\begin{equation}\label{eqn:relativisticElectrodynamicsL11:50}
\begin{bmatrix}
0 & . & . & . \\ 
  & 0 & . & . \\ 
  &   & 0 & . \\ 
  &   &   & 0 \\ 
\end{bmatrix}
\end{equation}

This is a no-brainer, for we just have to mechanically plug in the elements of the field strength tensor

Recall

\begin{align}\label{eqn:relativisticElectrodynamicsL11:70}
A^i &= (\phi, \BA) \\
A_i &= (\phi, -\BA)
\end{align}

\begin{align*}
F_{0\alpha} 
&= 
\partial_0 A_\alpha - \partial_\alpha A_0  \\
&= 
-\partial_0 (\BA)_\alpha - \partial_\alpha \phi  \\
&= E_\alpha
\end{align*}

For the purely spatial index combinations we have

\begin{align*}
F_{\alpha\beta} 
&= \partial_\alpha A_\beta - \partial_\beta A_\alpha  \\
&= -\partial_\alpha (\BA)_\beta + \partial_\beta (\BA)_\alpha  \\
\end{align*}

Written out explicitly, these are
\begin{align}\label{eqn:relativisticElectrodynamicsL11:90}
F_{12} &= \partial_2 (\BA)_1 -\partial_1 (\BA)_2  \\
F_{23} &= \partial_3 (\BA)_2 -\partial_2 (\BA)_3  \\
F_{31} &= \partial_1 (\BA)_3 -\partial_3 (\BA)_1 .
\end{align}

We can compare this to the elements of $\BB$

\begin{equation}\label{eqn:relativisticElectrodynamicsL11:110}
\BB = 
\begin{vmatrix}
\xcap & \ycap & \zcap \\
\partial_1 & \partial_2 & \partial_3 \\
A_x & A_y & A_z
\end{vmatrix}
\end{equation}

We see that 
\begin{align}\label{eqn:relativisticElectrodynamicsL11:130}
(\BB)_z &= \partial_1 A_y - \partial_2 A_x \\
(\BB)_x &= \partial_2 A_z - \partial_3 A_y \\
(\BB)_y &= \partial_3 A_x - \partial_1 A_z
\end{align}

So we have 

\begin{align}\label{eqn:relativisticElectrodynamicsL11:150}
F_{12} &= - (\BB)_3 \\
F_{23} &= - (\BB)_1 \\
F_{31} &= - (\BB)_2.
\end{align}

These can be summarized as simply

\begin{equation}\label{eqn:relativisticElectrodynamicsL11:170}
F_{\alpha\beta} = - \epsilon_{\alpha\beta\gamma} B_\gamma.
\end{equation}

This provides all the info needed to fill in the matrix above 

\begin{equation}\label{eqn:relativisticElectrodynamicsL11:190}
\Norm{ F_{ij} } = 
\begin{bmatrix}
0 & E_x & E_y & E_z \\
-E_x & 0 & -B_z & B_y \\
-E_y & B_z & 0 & -B_x \\
-E_z & -B_y & B_x & 0.
\end{bmatrix}.
\end{equation}

\subsection{Index raising of rank 2 tensor}

To raise indexes we compute

\begin{equation}\label{eqn:relativisticElectrodynamicsL11:210}
F^{ij} = g^{il} g^{jk} F_{lk}.
\end{equation}

\subsubsection{Justifying the raising operation.}
To justify this consider raising one index at a time by applying the metric tensor to our definition of $F_{lk}$.  That is

\begin{align*}
g^{al} F_{lk} 
&=
g^{al} (\partial_l A_k - \partial_k A_l) \\
&=
\partial^a A_k - \partial_k A^a.
\end{align*}

Now apply the metric tensor once more

\begin{align*}
g^{bk} g^{al} F_{lk} 
&=
g^{bk} (\partial^a A_k - \partial_k A^a) \\
&=
\partial^a A^b - \partial^b A^a.
\end{align*}

This is, by definition $F^{ab}$.  Since a rank 2 tensor has been defined as an object that transforms like the product of two pairs of coordinates, it makes sense that this particular tensor raises in the same fashion as would a product of two vector coordinates (in this case, it happens to be an antisymmetric product of two vectors, and one of which is an operator, but we have the same idea).

\subsubsection{Consider the components of the raised $F_{ij}$ tensor.}

\begin{align}\label{eqn:relativisticElectrodynamicsL11:230}
F^{0\alpha} &= -F_{0\alpha} \\
F^{\alpha\beta} &= F_{\alpha\beta}.
\end{align}

\begin{equation}\label{eqn:relativisticElectrodynamicsL11:250}
\Norm{ F^{ij} } = 
\begin{bmatrix}
0 & -E_x & -E_y & -E_z \\
E_x & 0 & -B_z & B_y \\
E_y & B_z & 0 & -B_x \\
E_z & -B_y & B_x & 0
\end{bmatrix}.
\end{equation}

\subsection{Back to chewing on the Lorentz force equation.}

\begin{equation}\label{eqn:relativisticElectrodynamicsL11:270}
m c \dds{ u_i } = \frac{e}{c} F_{ij} u^j
\end{equation}

\begin{align}\label{eqn:relativisticElectrodynamicsL11:290}
u^i &= \gamma \left( 1, \frac{\Bv}{c} \right) \\
u_i &= \gamma \left( 1, -\frac{\Bv}{c} \right)
\end{align}

For the spatial components of the Lorentz force equation we have

\begin{align*}
m c \dds{ u_\alpha } 
&= \frac{e}{c} F_{\alpha j} u^j \\
&= \frac{e}{c} F_{\alpha 0} u^0
+ \frac{e}{c} F_{\alpha \beta} u^\beta \\
&= \frac{e}{c} (-E_{\alpha}) \gamma
+ \frac{e}{c} (- \epsilon_{\alpha\beta\gamma} B_\gamma ) \frac{v^\beta}{c} \gamma 
\end{align*}

But
\begin{align*}
m c \dds{ u_\alpha } 
&= -m \dds{(\gamma \Bv_\alpha)} \\
&= -m \frac{d(\gamma \Bv_\alpha)}{c \InvGamma dt} \\
&= -\gamma \frac{d(m \gamma \Bv_\alpha)}{c dt}.
\end{align*}

Canceling the common $-\gamma/c$ terms, and switching to vector notation, we are left with

\begin{equation}\label{eqn:relativisticElectrodynamicsL11:310}
\frac{d( m \gamma \Bv_\alpha)}{dt} = e \left( E_\alpha + \inv{c} (\Bv \cross \BB)_\alpha \right).
\end{equation}

Now for the energy term.  We have 

\begin{align*}
m c \dds{u_0} 
&= \frac{e}{c} F_{0\alpha} u^\alpha \\
&= \frac{e}{c} E_{\alpha} \gamma \frac{v^\alpha}{c} \\
\dds{ m c \gamma } &=
\end{align*}

Putting the final two lines into vector form we have
\begin{equation}\label{eqn:relativisticElectrodynamicsL11:330}
\ddt{ (m c^2 \gamma)} = e \BE \cdot \Bv,
\end{equation}

or
\begin{equation}\label{eqn:relativisticElectrodynamicsL11:350}
\ddt{ \mathcal{E} } = e \BE \cdot \Bv
\end{equation}

\section{Transformation of rank two tensors in matrix and index form.}

Observe that our Lorentz force equation can be written exclusively in upper index quantities as

\begin{equation}\label{eqn:relativisticElectrodynamicsL11:370}
m c \dds{u^i} = \frac{e}{c} F^{ij} g_{jl} u^l
\end{equation}

Because we have a vector on one side of the equation, and it transforms by multiplication with by a Lorentz matrix in SO(1,3)

\begin{equation}\label{eqn:relativisticElectrodynamicsL11:390}
\frac{du^i}{ds} \rightarrow \hat{O} \frac{du^i}{ds} 
\end{equation}

\begin{equation}\label{eqn:relativisticElectrodynamicsL11:410}
\hat{G} = \Norm{ g_{ij} } = \Norm{ g^{ij} }
\end{equation}

\subsection{Transformation of the metric tensor, and some identities.}

\paragraph{Claim:}
We must have the two tensors transforming in the following sort of sandwich form

\begin{equation}\label{eqn:relativisticElectrodynamicsL11:430}
\hat{G} \rightarrow \hat{O} \hat{G} \hat{O}^\T = \hat{G}
\end{equation}

To demonstrate this let's consider a transformed vector in coordinate form as follows

\begin{align}\label{eqn:relativisticElectrodynamicsL11:450}
{x'}^i &= O^{i j} x_j = {O^i}_j x^j \\
{x'}_i &= O_{i j} x^j = {O_i}^j x_j
\end{align}

Our invariant for the vector square, which is required to remain unchanged is

\begin{align*}
{x'}^i {x'}_i 
&= (O^{i j} x_j)(O_{i k} x^k) \\
&= x^k (O^{i j} O_{i k}) x_j.
\end{align*}

This shows that we have a delta function relationship for the Lorentz transform matrix, when we sum over the first index

\begin{equation}\label{eqn:relativisticElectrodynamicsL11:470}
O^{a i} O_{a j} = {\delta^i}_j.
\end{equation}

If we write 

\begin{equation}\label{eqn:relativisticElectrodynamicsL11:n}
X' = \hat{O} X,
\end{equation}

then it appears we can put \ref{eqn:relativisticElectrodynamicsL11:470} into matrix form as

\begin{equation}\label{eqn:relativisticElectrodynamicsL11:n}
\hat{O}^\T G \hat{O} G = I
\end{equation}

Let's verify this, by expanding the products on the LHS explicitly

FIXME: WRONG: want $\hat{O} = {O^i}_j$.

\begin{align*}
\Norm{O^{ji}} \Norm{ g_{ij} } \Norm{ O^{ij} } \Norm{ g_{ij} }
&=
\Norm{O^{a i} g_{aj} } \Norm{ O^{ij} } \Norm{ g_{ij} } \\
&=
\Norm{O^{a i} g_{a b} O^{b j} } \Norm{ g_{ij} } \\
&=
\Norm{ O^{a i} g_{a b} O^{b c} g_{c j} } \\
&=
\Norm{ O^{a i} O_{a j} }
\end{align*}

This matches the $\Norm{{\delta^i}_j}$ that we have on the RHS, and all is well.

It is also worthwhile to point out that we have a delta function relationship summing over the second index too.  We can see this by taking the transpose of \ref{eqn:relativisticElectrodynamicsL11:470}, which is

\begin{equation}\label{eqn:relativisticElectrodynamicsL11:480}
{\delta_i}^j 
=
O_{i a} 
O^{j a}
\end{equation}

Did you notice the slight of hand here in the transposition with the positioning of the indexes?  With all this index gymnastics, even the transposition operation appears to be something that we have to treat carefully.  To justify the magic step above required to make the indexes match up properly consider

\begin{align*}
\Norm{{\delta^i}_j}^\T
&=
\Norm{ \delta^{im} g_{mj} }^\T \\
&=
\Norm{ g_{ij} }^\T
\Norm{ \delta^{ij} }^\T 
 \\
&=
\Norm{ g_{ij} }
\Norm{ \delta^{ij} }
 \\
&=
\Norm{ g_{im} \delta^{mj} }
 \\
&=
\Norm{ {\delta_{i}}^j }
\end{align*}

So, provided 

\begin{equation}\label{eqn:relativisticElectrodynamicsL11:n}
\Norm{A^{ij}}^\T = \Norm{A^{ji}},
\end{equation}

which we also assumed earlier as well, we have

\begin{equation}\label{eqn:relativisticElectrodynamicsL11:n}
\Norm{{\delta^i}_j}^\T =
\Norm{ {\delta_{i}}^j }
\end{equation}

The transposition operation on this mixed index delta function lowers and raises the indexes as opposed to just swapping them, as we are used to in plain old Euclidean \R{N} matrix operations.

Now, if we take it as a rule that a rank two matrix transforms as its indexes transform individually (as if it was the product of two coordinate vectors and we transform those coordinates separately), then we have

\begin{align*}
g^{ij} 
&\rightarrow {O^i}_k g^{km} {O^j}_m \\
&= ({O^i}_k g^{km}) {O^j}_m \\
&= O^{i m} {O^j}_m \\
&= O^{i m} (O_{a m} g^{a j}) \\
&= (O^{i m} O_{a m}) g^{a j}
\end{align*}

However, by \ref{eqn:relativisticElectrodynamicsL11:480}, we have $O_{a m} O^{i m} = {\delta_a}^i$, and we prove that 

\begin{equation}\label{eqn:relativisticElectrodynamicsL11:n}
g^{ij} \rightarrow g^{ij}.
\end{equation}

Finally, we wish to put the above transformation in matrix form, look more carefully at the very first line

FIXME: was working at this point, but see that things are hosed at the FIXME above.
\begin{align*}
g^{ij}
&\rightarrow {O^i}_k g^{km} {O^j}_m \\
\end{align*}

FIXME: REVIEWED TO HERE.
(this transformation must leave this matrix unchanged by definition) 

FIXME: PROVE TO SELF.

\begin{equation}\label{eqn:relativisticElectrodynamicsL11:490}
\hat{F} \rightarrow \hat{O} \hat{F} \hat{O}^\T 
\end{equation}

FIXME: PROVE TO SELF.

\section{Four vector invariants}

For three vectors $\BA$ and $\BB$ invariants are

\begin{equation}\label{eqn:relativisticElectrodynamicsL11:510}
\BA \cdot \BB = A^\alpha B_\alpha
\end{equation}

For four vectors $A^i$ and $B^i$ invariants are

\begin{equation}\label{eqn:relativisticElectrodynamicsL11:530}
A^i B_i = A^i g_{ij} B^j  
\end{equation}

For $F_{ij}$ what are the invariants

One invariant is

\begin{equation}\label{eqn:relativisticElectrodynamicsL11:550}
g^{ij} F_{ij} = 0
\end{equation}

but this isn't interesting since it is uniformly zero (product of symmetric and antisymmetric)

The two invariants are

\begin{equation}\label{eqn:relativisticElectrodynamicsL11:570}
F_{ij}F^{ij}
\end{equation}

and 

\begin{equation}\label{eqn:relativisticElectrodynamicsL11:590}
\epsilon^{ijkl} F_{ij}F_{kl}
\end{equation}

where
\begin{equation}\label{eqn:relativisticElectrodynamicsL11:610}
\epsilon^{ijkl} =
\left\{
\begin{array}{l l}
0 & \quad \mbox{if any two indexes coincide} \\
1 & \quad \mbox{for even permutations of $ijkl=0123$ } \\
-1 & \quad \mbox{for odd permutations of $ijkl=0123$ } \\
\end{array}
\right.
\end{equation}

We can show (homework) that

\begin{equation}\label{eqn:relativisticElectrodynamicsL11:630}
F_{ij}F^{ij} \propto \BE^2 - \BB^2
\end{equation}

\begin{equation}\label{eqn:relativisticElectrodynamicsL11:650}
\epsilon^{ijkl} F_{ij}F_{kl} \propto \BE \cdot \BB
\end{equation}

This first invariant serves as the action density for the Maxwell field equations.

There's some useful properties of these invariants.  One is that if the fields are perpendicular in one frame, then will be in any other.  

From the first, note that if $\Abs{\BE} > \Abs{\BB}$, the invariant is positive, and must be positive in all frames, or if $\Abs{\BE} < \Abs{\BB}$, the invariant is negative, and must be negative in all frames.  Because of this if $\Abs{\BE} > \Abs{\BB}$ in one frame, we can transform to a frame with only $\BE'$ component, solve that, and then transform back.  Similarly if $\Abs{\BE} < \Abs{\BB}$ in one frame, we can transform to a frame with only $\BB'$ component, solve that, and then transform back.

\section{}

Claim: 

\begin{equation}\label{eqn:relativisticElectrodynamicsL11:670}
F_{ij} = \partial_i A_j - \partial_j A_i
\end{equation}

where
\begin{equation}\label{eqn:relativisticElectrodynamicsL11:690}
\partial_i = \PD{x^i}{}
\end{equation}

This alone implies half of Maxwell's equations.

Consider

\begin{equation}\label{eqn:relativisticElectrodynamicsL11:710}
e^{m k i j} \partial_k F_{ij} = 0
\end{equation}

This is the Bianchi identity.

To show this consider

\begin{equation}\label{eqn:relativisticElectrodynamicsL11:730}
e^{m k i j} \partial_k (\partial_i A_j - \partial_j A_i)
\end{equation}

The first term of this is
\begin{equation}\label{eqn:relativisticElectrodynamicsL11:750}
\sum_j=0^3
\sum_{k,i=0}^3
e^{m k i j} \partial_k \partial_i A_j 
\end{equation}

but 
\begin{equation}\label{eqn:relativisticElectrodynamicsL11:770}
\partial_k \partial_i A_j 
= \inv{2} \left( \partial_k \partial_i A_j + \partial_i \partial_k A_j \right)
\end{equation}

FIXME: DIY: Continue working through this, swapping indexes, and other tricks like that to show that this is zero.  Similarly do this for the second term and also find it to be zero.

This is the 4D analogue of 

\begin{equation}\label{eqn:relativisticElectrodynamicsL11:790}
\spacegrad \cross (\spacegrad f) = 0
\end{equation}

i.e.

\begin{equation}\label{eqn:relativisticElectrodynamicsL11:810}
e^{\alpha\beta\gamma} \partial_\beta \partial_\gamma f = 0
\end{equation}

Let's do this explicitly, starting with

\begin{equation}\label{eqn:relativisticElectrodynamicsL10:370}
\Norm{ F_{ij} } = 
\begin{bmatrix}
0 & E_x & E_y & E_z \\
-E_x & 0 & -B_z & B_y \\
-E_y & B_z & 0 & -B_x \\
-E_z & -B_y & B_x & 0.
\end{bmatrix}
\end{equation}

For the $m= 0$ case we have

\begin{align*}
\epsilon^{0 k i j} \partial_k F_{ij}
&=
\epsilon^{\alpha \beta \gamma} \partial_\alpha F_{\beta \gamma}
&= 
\epsilon^{\alpha \beta \gamma} \partial_\alpha -\epsilon_{\beta \gamma \delta} B_\delta
\end{align*}

implies

\begin{equation}\label{eqn:relativisticElectrodynamicsL11:830}
\partial_\delta B_\delta = 0
\end{equation}

which is just Gauss's law for magnetism

\begin{equation}\label{eqn:relativisticElectrodynamicsL11:850}
\spacegrad \cdot \BB = 0
\end{equation}

Let's do the spatial portion

\begin{align*}
\epsilon^{\mu 0 \alpha \beta} \partial_0 F_{\alpha \beta}
+\epsilon^{\mu \alpha 0 \beta} \partial_\alpha F_{0 \beta}
+\epsilon^{\mu \alpha \beta 0} \partial_\alpha F_{\beta 0}
&= -
\epsilon^{0 \mu i j} \partial_0 F_{ij}
...
\end{align*}

\begin{equation}\label{eqn:relativisticElectrodynamicsL11:870}
0 = -\epsilon^{\mu \alpha \beta} \partial_0 (-) \epsilon_{\alpha \beta \gamma} B_\gamma + 2 (\spacegrad \cross \BE)^\mu
\end{equation}

We'll use

\begin{equation}\label{eqn:relativisticElectrodynamicsL11:890}
\epsilon^{\mu \alpha \beta} \epsilon_{\gamma \alpha \beta} = 2 {\delta^\mu}_\gamma
...
\end{equation}

\begin{equation}\label{eqn:relativisticElectrodynamicsL11:910}
0 = \partial_0 (\BB)_\mu + (\spacegrad \cross \BE)^\mu.
\end{equation}

Which is the Maxwell-Faraday equation

\begin{equation}\label{eqn:relativisticElectrodynamicsL11:930}
0 = \partial_0 \BB + \spacegrad \cross \BE.
\end{equation}

\EndArticle

%
% Copyright � 2012 Peeter Joot.  All Rights Reserved.
% Licenced as described in the file LICENSE under the root directory of this GIT repository.
%

%\chapter{Action for the field}
\label{chap:relativisticElectrodynamicsL12}
%\useCCL
%\blogpage{http://sites.google.com/site/peeterjoot/math2011/relativisticElectrodynamicsL12.pdf}
%\date{Feb 10, 2011}

\paragraph{Reading}

Covering chapter 3 material from the text \citep{landau1980classical}.

Covering \href{http://www.physics.utoronto.ca/~poppitz/epoppitz/PHY450_files/RelEMpp84-102.pdf}{lecture notes pp. 84-102}: relativity, gauge invariance, and superposition principles and the action for the electromagnetic field coupled to charged particles (91-95); the 4-current and its physical interpretation (96-102), including a needed mathematical interlude on delta-functions of functions (98-100) [Wednesday, Feb. 8; Thursday, Feb. 10]

\section{Where we are}

\begin{equation}\label{eqn:relativisticElectrodynamicsL12:10}
F_{ij} = \partial_i A_j - \partial_j A_i
\end{equation}

We learned that one half of Maxwell's equations comes from the Bianchi identity

\begin{equation}\label{eqn:relativisticElectrodynamicsL12:30}
\epsilon^{ijkl} \partial_j F_{kl} = 0
\end{equation}

the other half (for vacuum) is

\begin{equation}\label{eqn:relativisticElectrodynamicsL12:50}
\partial_j F_{ji} = 0
\end{equation}

To get here we have to consider the action for the field.

\section{Generalizing the action to multiple particles}

We have learned that the action for a single particle is

\begin{equation}\label{eqn:relativisticElectrodynamicsL12:90}
\begin{aligned}
S 
&= S_{\text{matter}} + S_{\text{interaction}} \\
&= -m c \int ds - \frac{e}{c} \int ds^i A_i
\end{aligned}
\end{equation}

This generalizes to more particles 

\begin{equation}\label{eqn:relativisticElectrodynamicsL12:70}
S_{\text{``particles in field''}}
= 
-
\sum_A 
m_A c \int_{x^A(\tau)} ds 
- 
\sum_A 
\frac{e_A}{c} \int dx^i_A A_i(x_A(\tau))
\end{equation}

\(A\) labels the particles, and \(x^A(\tau)\), \(\{x^A(\tau), A= 1 \cdots N\}\) is the worldline of particle \(A\).


%
% Copyright � 2015 Peeter Joot.  All Rights Reserved.
% Licenced as described in the file LICENSE under the root directory of this GIT repository.
%
\documentclass[]{eliblog}

\usepackage{amsmath}
\usepackage{mathpazo}

%
% shorthand for bold symbols, convenient for vectors and matrices
%
\newcommand{\Ba}[0]{\mathbf{a}}
\newcommand{\Bb}[0]{\mathbf{b}}
\newcommand{\Bc}[0]{\mathbf{c}}
\newcommand{\Bd}[0]{\mathbf{d}}
\newcommand{\Be}[0]{\mathbf{e}}
\newcommand{\Bf}[0]{\mathbf{f}}
\newcommand{\Bg}[0]{\mathbf{g}}
\newcommand{\Bh}[0]{\mathbf{h}}
\newcommand{\Bi}[0]{\mathbf{i}}
\newcommand{\Bj}[0]{\mathbf{j}}
\newcommand{\Bk}[0]{\mathbf{k}}
\newcommand{\Bl}[0]{\mathbf{l}}
\newcommand{\Bm}[0]{\mathbf{m}}
\newcommand{\Bn}[0]{\mathbf{n}}
\newcommand{\Bo}[0]{\mathbf{o}}
\newcommand{\Bp}[0]{\mathbf{p}}
\newcommand{\Bq}[0]{\mathbf{q}}
\newcommand{\Br}[0]{\mathbf{r}}
\newcommand{\Bs}[0]{\mathbf{s}}
\newcommand{\Bt}[0]{\mathbf{t}}
\newcommand{\Bu}[0]{\mathbf{u}}
\newcommand{\Bv}[0]{\mathbf{v}}
\newcommand{\Bw}[0]{\mathbf{w}}
\newcommand{\Bx}[0]{\mathbf{x}}
\newcommand{\By}[0]{\mathbf{y}}
\newcommand{\Bz}[0]{\mathbf{z}}
\newcommand{\BA}[0]{\mathbf{A}}
\newcommand{\BB}[0]{\mathbf{B}}
\newcommand{\BC}[0]{\mathbf{C}}
\newcommand{\BD}[0]{\mathbf{D}}
\newcommand{\BE}[0]{\mathbf{E}}
\newcommand{\BF}[0]{\mathbf{F}}
\newcommand{\BG}[0]{\mathbf{G}}
\newcommand{\BH}[0]{\mathbf{H}}
\newcommand{\BI}[0]{\mathbf{I}}
\newcommand{\BJ}[0]{\mathbf{J}}
\newcommand{\BK}[0]{\mathbf{K}}
\newcommand{\BL}[0]{\mathbf{L}}
\newcommand{\BM}[0]{\mathbf{M}}
\newcommand{\BN}[0]{\mathbf{N}}
\newcommand{\BO}[0]{\mathbf{O}}
\newcommand{\BP}[0]{\mathbf{P}}
\newcommand{\BQ}[0]{\mathbf{Q}}
\newcommand{\BR}[0]{\mathbf{R}}
\newcommand{\BS}[0]{\mathbf{S}}
\newcommand{\BT}[0]{\mathbf{T}}
\newcommand{\BU}[0]{\mathbf{U}}
\newcommand{\BV}[0]{\mathbf{V}}
\newcommand{\BW}[0]{\mathbf{W}}
\newcommand{\BX}[0]{\mathbf{X}}
\newcommand{\BY}[0]{\mathbf{Y}}
\newcommand{\BZ}[0]{\mathbf{Z}}

\newcommand{\Bzero}[0]{\mathbf{0}}
\newcommand{\Btheta}[0]{\boldsymbol{\theta}}
\newcommand{\Btau}[0]{\boldsymbol{\tau}}
\newcommand{\Bomega}[0]{\boldsymbol{\omega}}

%
% shorthand for unit vectors
%
\newcommand{\acap}[0]{\hat{\Ba}}
\newcommand{\bcap}[0]{\hat{\Bb}}
\newcommand{\ccap}[0]{\hat{\Bc}}
\newcommand{\dcap}[0]{\hat{\Bd}}
\newcommand{\ecap}[0]{\hat{\Be}}
\newcommand{\fcap}[0]{\hat{\Bf}}
\newcommand{\gcap}[0]{\hat{\Bg}}
\newcommand{\hcap}[0]{\hat{\Bh}}
\newcommand{\icap}[0]{\hat{\Bi}}
\newcommand{\jcap}[0]{\hat{\Bj}}
\newcommand{\kcap}[0]{\hat{\Bk}}
\newcommand{\lcap}[0]{\hat{\Bl}}
\newcommand{\mcap}[0]{\hat{\Bm}}
\newcommand{\ncap}[0]{\hat{\Bn}}
\newcommand{\ocap}[0]{\hat{\Bo}}
\newcommand{\pcap}[0]{\hat{\Bp}}
\newcommand{\qcap}[0]{\hat{\Bq}}
\newcommand{\rcap}[0]{\hat{\Br}}
\newcommand{\scap}[0]{\hat{\Bs}}
\newcommand{\tcap}[0]{\hat{\Bt}}
\newcommand{\ucap}[0]{\hat{\Bu}}
\newcommand{\vcap}[0]{\hat{\Bv}}
\newcommand{\wcap}[0]{\hat{\Bw}}
\newcommand{\xcap}[0]{\hat{\Bx}}
\newcommand{\ycap}[0]{\hat{\By}}
\newcommand{\zcap}[0]{\hat{\Bz}}
\newcommand{\thetacap}[0]{\hat{\Btheta}}

%
% to write R^n and C^n in a distinguishable fashion.  Perhaps change this
% to the double lined characters upon figuring out how to do so.
%
\newcommand{\C}[1]{$\mathbb{C}^{#1}$}
\newcommand{\R}[1]{$\mathbb{R}^{#1}$}

%
% various generally useful helpers
%

% derivative of #1 wrt. #2:
\newcommand{\D}[2] {\frac {d#2} {d#1}}

\newcommand{\inv}[1]{\frac{1}{#1}}
\newcommand{\cross}[0]{\times}

\newcommand{\abs}[1]{\lvert{#1}\rvert}
\newcommand{\norm}[1]{\lVert{#1}\rVert}
\newcommand{\innerprod}[2]{\langle{#1}, {#2}\rangle}
\newcommand{\dotprod}[2]{{#1} \cdot {#2}}
\newcommand{\bdotprod}[2]{\left({#1} \cdot {#2}\right)}
\newcommand{\crossprod}[2]{{#1} \cross {#2}}
\newcommand{\tripleprod}[3]{\dotprod{\left(\crossprod{#1}{#2}\right)}{#3}}

\DeclareMathOperator{\Proj}{Proj}
\DeclareMathOperator{\Span}{span}
\DeclareMathOperator{\Sgn}{sgn}
\DeclareMathOperator{\Area}{Area}
\DeclareMathOperator{\Volume}{Volume}

%
% A few miscellaneous things specific to this document
%
\newcommand{\crossop}[1]{\crossprod{#1}{}}

% R2 vector.
\newcommand{\VectorTwo}[2]{
\begin{bmatrix}
 {#1} \\
 {#2}
\end{bmatrix}
}

\newcommand{\VectorN}[1]{
\begin{bmatrix}
{#1}_1 \\
{#1}_2 \\
\vdots \\
{#1}_N \\
\end{bmatrix}
}

\newcommand{\DETuvij}[4]{
\begin{vmatrix}
 {#1}_{#3} & {#1}_{#4} \\
 {#2}_{#3} & {#2}_{#4}
\end{vmatrix}
}

\newcommand{\DETuvwijk}[6]{
\begin{vmatrix}
 {#1}_{#4} & {#1}_{#5} & {#1}_{#6} \\
 {#2}_{#4} & {#2}_{#5} & {#2}_{#6} \\
 {#3}_{#4} & {#3}_{#5} & {#3}_{#6}
\end{vmatrix}
}

\newcommand{\DETuvwxijkl}[8]{
\begin{vmatrix}
 {#1}_{#5} & {#1}_{#6} & {#1}_{#7} & {#1}_{#8} \\
 {#2}_{#5} & {#2}_{#6} & {#2}_{#7} & {#2}_{#8} \\
 {#3}_{#5} & {#3}_{#6} & {#3}_{#7} & {#3}_{#8} \\
 {#4}_{#5} & {#4}_{#6} & {#4}_{#7} & {#4}_{#8} \\
\end{vmatrix}
}

%\newcommand{\DETuvwxyijklm}[10]{
%\begin{vmatrix}
% {#1}_{#6} & {#1}_{#7} & {#1}_{#8} & {#1}_{#9} & {#1}_{#10} \\
% {#2}_{#6} & {#2}_{#7} & {#2}_{#8} & {#2}_{#9} & {#2}_{#10} \\
% {#3}_{#6} & {#3}_{#7} & {#3}_{#8} & {#3}_{#9} & {#3}_{#10} \\
% {#4}_{#6} & {#4}_{#7} & {#4}_{#8} & {#4}_{#9} & {#4}_{#10} \\
% {#5}_{#6} & {#5}_{#7} & {#5}_{#8} & {#5}_{#9} & {#5}_{#10}
%\end{vmatrix}
%}

% R3 vector.
\newcommand{\VectorThree}[3]{
\begin{bmatrix}
 {#1} \\
 {#2} \\
 {#3}
\end{bmatrix}
}



\author{Peeter Joot}
\email{peeter.joot@gmail.com}

%\documentclass[]{eliblogwidescreen}

\usepackage{amsmath}
\usepackage{mathpazo}

%
% shorthand for bold symbols, convenient for vectors and matrices
%
\newcommand{\Ba}[0]{\mathbf{a}}
\newcommand{\Bb}[0]{\mathbf{b}}
\newcommand{\Bc}[0]{\mathbf{c}}
\newcommand{\Bd}[0]{\mathbf{d}}
\newcommand{\Be}[0]{\mathbf{e}}
\newcommand{\Bf}[0]{\mathbf{f}}
\newcommand{\Bg}[0]{\mathbf{g}}
\newcommand{\Bh}[0]{\mathbf{h}}
\newcommand{\Bi}[0]{\mathbf{i}}
\newcommand{\Bj}[0]{\mathbf{j}}
\newcommand{\Bk}[0]{\mathbf{k}}
\newcommand{\Bl}[0]{\mathbf{l}}
\newcommand{\Bm}[0]{\mathbf{m}}
\newcommand{\Bn}[0]{\mathbf{n}}
\newcommand{\Bo}[0]{\mathbf{o}}
\newcommand{\Bp}[0]{\mathbf{p}}
\newcommand{\Bq}[0]{\mathbf{q}}
\newcommand{\Br}[0]{\mathbf{r}}
\newcommand{\Bs}[0]{\mathbf{s}}
\newcommand{\Bt}[0]{\mathbf{t}}
\newcommand{\Bu}[0]{\mathbf{u}}
\newcommand{\Bv}[0]{\mathbf{v}}
\newcommand{\Bw}[0]{\mathbf{w}}
\newcommand{\Bx}[0]{\mathbf{x}}
\newcommand{\By}[0]{\mathbf{y}}
\newcommand{\Bz}[0]{\mathbf{z}}
\newcommand{\BA}[0]{\mathbf{A}}
\newcommand{\BB}[0]{\mathbf{B}}
\newcommand{\BC}[0]{\mathbf{C}}
\newcommand{\BD}[0]{\mathbf{D}}
\newcommand{\BE}[0]{\mathbf{E}}
\newcommand{\BF}[0]{\mathbf{F}}
\newcommand{\BG}[0]{\mathbf{G}}
\newcommand{\BH}[0]{\mathbf{H}}
\newcommand{\BI}[0]{\mathbf{I}}
\newcommand{\BJ}[0]{\mathbf{J}}
\newcommand{\BK}[0]{\mathbf{K}}
\newcommand{\BL}[0]{\mathbf{L}}
\newcommand{\BM}[0]{\mathbf{M}}
\newcommand{\BN}[0]{\mathbf{N}}
\newcommand{\BO}[0]{\mathbf{O}}
\newcommand{\BP}[0]{\mathbf{P}}
\newcommand{\BQ}[0]{\mathbf{Q}}
\newcommand{\BR}[0]{\mathbf{R}}
\newcommand{\BS}[0]{\mathbf{S}}
\newcommand{\BT}[0]{\mathbf{T}}
\newcommand{\BU}[0]{\mathbf{U}}
\newcommand{\BV}[0]{\mathbf{V}}
\newcommand{\BW}[0]{\mathbf{W}}
\newcommand{\BX}[0]{\mathbf{X}}
\newcommand{\BY}[0]{\mathbf{Y}}
\newcommand{\BZ}[0]{\mathbf{Z}}

\newcommand{\Bzero}[0]{\mathbf{0}}
\newcommand{\Btheta}[0]{\boldsymbol{\theta}}
\newcommand{\Btau}[0]{\boldsymbol{\tau}}
\newcommand{\Bomega}[0]{\boldsymbol{\omega}}

%
% shorthand for unit vectors
%
\newcommand{\acap}[0]{\hat{\Ba}}
\newcommand{\bcap}[0]{\hat{\Bb}}
\newcommand{\ccap}[0]{\hat{\Bc}}
\newcommand{\dcap}[0]{\hat{\Bd}}
\newcommand{\ecap}[0]{\hat{\Be}}
\newcommand{\fcap}[0]{\hat{\Bf}}
\newcommand{\gcap}[0]{\hat{\Bg}}
\newcommand{\hcap}[0]{\hat{\Bh}}
\newcommand{\icap}[0]{\hat{\Bi}}
\newcommand{\jcap}[0]{\hat{\Bj}}
\newcommand{\kcap}[0]{\hat{\Bk}}
\newcommand{\lcap}[0]{\hat{\Bl}}
\newcommand{\mcap}[0]{\hat{\Bm}}
\newcommand{\ncap}[0]{\hat{\Bn}}
\newcommand{\ocap}[0]{\hat{\Bo}}
\newcommand{\pcap}[0]{\hat{\Bp}}
\newcommand{\qcap}[0]{\hat{\Bq}}
\newcommand{\rcap}[0]{\hat{\Br}}
\newcommand{\scap}[0]{\hat{\Bs}}
\newcommand{\tcap}[0]{\hat{\Bt}}
\newcommand{\ucap}[0]{\hat{\Bu}}
\newcommand{\vcap}[0]{\hat{\Bv}}
\newcommand{\wcap}[0]{\hat{\Bw}}
\newcommand{\xcap}[0]{\hat{\Bx}}
\newcommand{\ycap}[0]{\hat{\By}}
\newcommand{\zcap}[0]{\hat{\Bz}}
\newcommand{\thetacap}[0]{\hat{\Btheta}}

%
% to write R^n and C^n in a distinguishable fashion.  Perhaps change this
% to the double lined characters upon figuring out how to do so.
%
\newcommand{\C}[1]{$\mathbb{C}^{#1}$}
\newcommand{\R}[1]{$\mathbb{R}^{#1}$}

%
% various generally useful helpers
%

% derivative of #1 wrt. #2:
\newcommand{\D}[2] {\frac {d#2} {d#1}}

\newcommand{\inv}[1]{\frac{1}{#1}}
\newcommand{\cross}[0]{\times}

\newcommand{\abs}[1]{\lvert{#1}\rvert}
\newcommand{\norm}[1]{\lVert{#1}\rVert}
\newcommand{\innerprod}[2]{\langle{#1}, {#2}\rangle}
\newcommand{\dotprod}[2]{{#1} \cdot {#2}}
\newcommand{\bdotprod}[2]{\left({#1} \cdot {#2}\right)}
\newcommand{\crossprod}[2]{{#1} \cross {#2}}
\newcommand{\tripleprod}[3]{\dotprod{\left(\crossprod{#1}{#2}\right)}{#3}}

\DeclareMathOperator{\Proj}{Proj}
\DeclareMathOperator{\Span}{span}
\DeclareMathOperator{\Sgn}{sgn}
\DeclareMathOperator{\Area}{Area}
\DeclareMathOperator{\Volume}{Volume}

%
% A few miscellaneous things specific to this document
%
\newcommand{\crossop}[1]{\crossprod{#1}{}}

% R2 vector.
\newcommand{\VectorTwo}[2]{
\begin{bmatrix}
 {#1} \\
 {#2}
\end{bmatrix}
}

\newcommand{\VectorN}[1]{
\begin{bmatrix}
{#1}_1 \\
{#1}_2 \\
\vdots \\
{#1}_N \\
\end{bmatrix}
}

\newcommand{\DETuvij}[4]{
\begin{vmatrix}
 {#1}_{#3} & {#1}_{#4} \\
 {#2}_{#3} & {#2}_{#4}
\end{vmatrix}
}

\newcommand{\DETuvwijk}[6]{
\begin{vmatrix}
 {#1}_{#4} & {#1}_{#5} & {#1}_{#6} \\
 {#2}_{#4} & {#2}_{#5} & {#2}_{#6} \\
 {#3}_{#4} & {#3}_{#5} & {#3}_{#6}
\end{vmatrix}
}

\newcommand{\DETuvwxijkl}[8]{
\begin{vmatrix}
 {#1}_{#5} & {#1}_{#6} & {#1}_{#7} & {#1}_{#8} \\
 {#2}_{#5} & {#2}_{#6} & {#2}_{#7} & {#2}_{#8} \\
 {#3}_{#5} & {#3}_{#6} & {#3}_{#7} & {#3}_{#8} \\
 {#4}_{#5} & {#4}_{#6} & {#4}_{#7} & {#4}_{#8} \\
\end{vmatrix}
}

%\newcommand{\DETuvwxyijklm}[10]{
%\begin{vmatrix}
% {#1}_{#6} & {#1}_{#7} & {#1}_{#8} & {#1}_{#9} & {#1}_{#10} \\
% {#2}_{#6} & {#2}_{#7} & {#2}_{#8} & {#2}_{#9} & {#2}_{#10} \\
% {#3}_{#6} & {#3}_{#7} & {#3}_{#8} & {#3}_{#9} & {#3}_{#10} \\
% {#4}_{#6} & {#4}_{#7} & {#4}_{#8} & {#4}_{#9} & {#4}_{#10} \\
% {#5}_{#6} & {#5}_{#7} & {#5}_{#8} & {#5}_{#9} & {#5}_{#10}
%\end{vmatrix}
%}

% R3 vector.
\newcommand{\VectorThree}[3]{
\begin{bmatrix}
 {#1} \\
 {#2} \\
 {#3}
\end{bmatrix}
}



\author{Peeter Joot}
\email{peeter.joot@gmail.com}


\chapter{PHY450H1S.  Relativistic Electrodynamics Lecture 11 (Taught by Prof. Erich Poppitz).  FIXME.}
\label{chap:relativisticElectrodynamicsL13}
%\useCCL
\blogpage{http://sites.google.com/site/peeterjoot/math2011/relativisticElectrodynamicsL13.pdf}
\date{Feb 15, 2011}
\revisionInfo{relativisticElectrodynamicsL13.tex}

%\beginArtWithToc
\beginArtNoToc

\section{Reading.}

Covering chapter 4 material from the text \cite{landau1980classical}.

Covering \href{http://www.physics.utoronto.ca/~poppitz/e-poppitz/PHY450_files/RelEMpp103-113.pdf}{lecture notes pp.103-113}: variational principle for the electromagnetic field and the relevant boundary conditions (103-105); the second set of Maxwell�s equations from the variational principle (106-108); Maxwell�s equations in vacuum and the wave equation in the nonrelativistic Coulomb gauge (109-111)

\section{Review.  Our action.}

\begin{align*}\label{eqn:relativisticElectrodynamicsL13:10}
S 
&= S_{\text{particles}} + S_{\text{interaction}} + S_{\text{EM field}}
&= \sum_A \int_{x_A^i(\tau)} ds ( -m_A c )
- \sum_A 
\frac{e_A}{c}
\int dx_A^i A_i(x_A) 
- \inv{16 \pi c} \int d^4 x F^{ij } F_{ij}
\end{align*}

Our dynamics variables are 

\begin{equation}\label{eqn:relativisticElectrodynamicsL13:30}
\left\{
\begin{array}{l l}
x_A^i(\tau) & \quad \mbox{$A = 1, \cdots, N$} \\
A^i(x) & \quad \mbox{$A = 1, \cdots, N$}
\end{array}
\right.
\end{equation}

Also
\begin{equation}\label{eqn:relativisticElectrodynamicsL13:50}
S_{\text{interaction}} 
= -\inv{c^2} \int d^4x j^i(x) A_i(x)
\end{equation}

\begin{equation}\label{eqn:relativisticElectrodynamicsL13:70}
j^i(x) = \sum_A c e_A \int dx_A^i \delta^4( x - x_A(\tau))
\end{equation}

Variation with respect to $x_A^i(\tau)$ we have

\begin{equation}\label{eqn:relativisticElectrodynamicsL13:90}
m c \dds{u^i_A} = \frac{e}{c} u_A^j F_{ij}
\end{equation}

Note that it's easy to get the sign mixed up here.  With our $(+,-,-,-)$ metric tensor, if the second index is the summation index, we have a positive sign.

Only the $S_{\text{particles}}$ and $S_{\text{interaction}}$ depend on $x_A^i(\tau)$.

\section{The field action variation.}

\paragraph{Today}: We'll find the EOM for $A^i(x)$.  The dynamical degrees of freedom are $A^i(\Bx,t)$

\begin{equation}\label{eqn:relativisticElectrodynamicsL13:110}
S[A^i(\Bx, t)] = -\inv{16 \pi c} \int d^4x F_{ij}F^{ij} - \inv{c^2} \int d^4 x A^i j_i
\end{equation}

Here $j^i$ are treated as ``sources''.

We demand that 

\begin{equation}\label{eqn:relativisticElectrodynamicsL13:130}
\delta S = S[ A^i(\Bx, t) + \delta A^i(\Bx, t)] - S[ A^i(\Bx, t) ] = 0 + O(\delta A)^2
\end{equation}

We need to impose two conditions.
\begin{itemize}
\item At spatial $\infty$, i.e. at $\Abs{\Bx} \rightarrow \infty, \forall t$, we'll impose the condition 

\begin{equation}\label{eqn:relativisticElectrodynamicsL13:150}
\evalbar{A^i(\Bx, t)}{\Abs{\Bx} \rightarrow \infty} \rightarrow 0
\end{equation}

This is sensible, because fields are created by charges, and charges are assumed to be localized in a bounded region.  The field outside charges will $\rightarrow 0$ at $\Abs{\Bx} \rightarrow \infty$.  Later we will treat the integration range as finite, and bounded, then later allow the boundary to go to infitiy.

\item at $t = -T$ and $t = T$ we'll imagine that the values of $A^i(\Bx, \pm T)$ are fixed.

This is analogous to $x(t_i) = x_1$ and $x(t_f) = x_2$ in particle mechanics.

Since $A^i(\Bx, \pm T)$ is given, and equivalent to the initial and final field configurations, our extremes at the boundary is zero

\begin{equation}\label{eqn:relativisticElectrodynamicsL13:170}
\delta A^i(\Bx, \pm T) = 0
\end{equation}

\end{itemize}

PICTURE: a cylinder in spacetime, with an attempt to depict the boundary.

\section{Computing the variation.}

\begin{equation}\label{eqn:relativisticElectrodynamicsL13:190}
\delta S[A^i(\Bx, t)] 
= -\inv{16 \pi c} \int d^4 x \delta (F_{ij}F^{ij}) - \inv{c^2} \int d^4 x \delta(A^i) j_i 
\end{equation}

But 

\begin{align*}
\delta (F_{ij}F^{ij}) 
&= 
\delta(F_{ij}) F^{ij} + F_{ij} \delta(F^{ij}) \\
&= 
2 \delta(F^{ij}) F_{ij} \\
&= 
2 \delta(\partial^i A^j - \partial^j A^i) F_{ij} \\
&= 
2 \delta(\partial^i A^j) F_{ij} - 2 \delta(\partial^j A^i) F_{ij} \\
&= 
2 \delta(\partial^j A^i) F_{ji} - 2 \delta(\partial^j A^i) F_{ij} \\
&= 
4 \delta(\partial^i A^j) F_{ij} \\
&= 
4 F_{ij} \partial^i \delta(A^j) \\
\end{align*}

So
\begin{align*}
\delta S[A^i(\Bx, t)] 
&= -\inv{4 \pi c} \int d^4 x F_{ij} \partial^i \delta(A^j) - \inv{c^2} \int d^4 x j^i \delta(A_i) \\
\end{align*}

\begin{align*}
\int d^4 x F_{ij} \partial^i \delta(A^j) 
=
\int d^4 x F^{ij} \partial_i \delta(A_j) 
=
\int d^4 x F^{ij} \PD{x^i}{} \delta(A_j) 
=
\int d^4 x \PD{x^i}{} \delta A_j F^{ij}
\delta(A^j) 
\end{align*}

DIY.  Integrate by parts.

= ... 0 (zero over the boundary)

Noting that 

\begin{align*}
\PD{x^\alpha}{}\left( \delta A_j F^{\alpha j} \right) \\
\spacegrad \cdot \BC, \BC^\alpha = \delta A_j F^{\alpha j} \\
...
\end{align*}

We are left with

\begin{align*}
\delta S[A^i(\Bx, t)] 
&= -\inv{4 \pi c} \int d^4 x \delta (A_j) \partial_i F^{ij} - \inv{c^2} \int d^4 x j^i \delta(A_i) \\
&= 
\int d^4 x \delta A_j(x)
\left(
-\inv{4 \pi c} \partial_i F^{ij}(x) - \inv{c^2} j^i 
\right)  \\
&= 0
\end{align*}

This gives us

\begin{equation}\label{eqn:relativisticElectrodynamicsL13:210}
\boxed{
\partial_i F^{ij} = \frac{4 \pi}{c} j^j
}
\end{equation}

\section{Unpacking these.}

Recall that 

\begin{equation}\label{eqn:relativisticElectrodynamicsL13:230}
e^{ijkl} \partial_j F_{kl} = 0
\end{equation}

gave us

\begin{align}\label{eqn:relativisticElectrodynamicsL13:250}
\spacegrad \cdot \BB &= 0 \\
\spacegrad \cross \BE &= -\inv{c} \PD{t}{\BB}
\end{align}

\begin{equation}\label{eqn:relativisticElectrodynamicsL13:270}
\partial_\alpha F^{\alpha 0} = \frac{4 \pi}{c} j^0 = 4 \pi \rho
\end{equation}

(since $j^0 = c \rho$).

Because 

\begin{equation}\label{eqn:relativisticElectrodynamicsL13:290}
F^{\alpha 0} = (\BE)^\alpha
\end{equation}

or
\begin{equation}\label{eqn:relativisticElectrodynamicsL13:310}
\spacegrad \cdot \BE = 4 \pi \rho
\end{equation}

The messier one to deal with is

\begin{equation}\label{eqn:relativisticElectrodynamicsL13:330}
\partial_i F^{i\alpha} = \frac{4 \pi}{c} j^\alpha
\end{equation}

...

\begin{align*}
\partial_i F^{i\alpha} 
&= \partial_\beta F^{\beta \alpha} + \partial_0 F^{0 \alpha} \\
&= \partial_\beta F^{\beta \alpha} - \inv{c} \PD{t}{(\BE)^\alpha} \\
\end{align*}

Details: DIY.  We get

\begin{equation}\label{eqn:relativisticElectrodynamicsL13:350}
\frac{4 \pi}{c} j^\alpha
= (\spacegrad \cross \BB)^\alpha - \inv{c} \PD{t}{(\BE)^\alpha} \\
\end{equation}

or

\begin{equation}\label{eqn:relativisticElectrodynamicsL13:370}
\spacegrad \cross \BB - \inv{c} \PD{t}{\BE} = \frac{4 \pi}{c} \Bj
\end{equation}

Summarizing what we know so far, we have

\begin{equation}\label{eqn:relativisticElectrodynamicsL13:390}
\boxed{
\begin{aligned}
\partial_i F^{ij} &= \frac{4 \pi}{c} j^j \\
\epsilon^{ijkl} \partial_j F_{kl} &= 0
\end{aligned}
}
\end{equation}

or in vector form

\begin{equation}\label{eqn:relativisticElectrodynamicsL13:410}
\boxed{
\begin{aligned}
\spacegrad \cdot \BE &= 4 \pi \rho \\
\spacegrad \cross \BB -\inv{c} \PD{t}{\BE} &= \frac{4 \pi}{c} \Bj \\
\spacegrad \cdot \BB &= 0 \\
\spacegrad \cross \BE +\inv{c} \PD{t}{\BB} &= 0
\end{aligned}
}
\end{equation}

\section{Speed of light}

\paragraph{Claim}: ``$c$'' is the speed of EM waves in vacuum.

Study equations in vacuum (no sources, so $j^i = 0$) for $A^i = (\phi, \BA)$.

\begin{align}\label{eqn:relativisticElectrodynamicsL13:430}
\spacegrad \cdot \BE &= 0 \\
\spacegrad \cross \BB &= \inv{c} \PD{t}{\BE} 
\end{align}

where

\begin{align}\label{eqn:relativisticElectrodynamicsL13:450}
\BE &= - \spacegrad \phi - \inv{c} \PD{t}{\BA} \\
\BB &= \spacegrad \cross \BA
\end{align}

In terms of potentials

\begin{align}\label{eqn:relativisticElectrodynamicsL13:470}
\spacegrad \cross (\spacegrad \spacegrad \BA) &= - \inv{c} \spacegrad \PD{t}{\phi} - \inv{c} \frac{\partial^2}{\partial t^2} \BA
-\spacegrad^2 \phi - \inv{c} \spacegrad \cdot \BA &= 0
\end{align}

Can make a gauge transformation

\begin{align}\label{eqn:relativisticElectrodynamicsL13:490}
(\phi, \BA) \rightarrow (\phi', \BA')
\end{align}

with 

\begin{align}\label{eqn:relativisticElectrodynamicsL13:510}
\phi &= \phi' + \inv{c} \PD{t}{\chi} \\
\BA &= \BA' - \spacegrad \chi
\end{align}

Can choose $\chi(\Bx, t)$ to make $\phi' = 0$ ($\forall \phi \exists \chi, \phi' = 0$)

\begin{align}\label{eqn:relativisticElectrodynamicsL13:530}
\inv{c} \PD{t}{} \chi(\Bx, t) = \phi(\Bx, t)
\end{align}

\begin{align}\label{eqn:relativisticElectrodynamicsL13:550}
\chi(\Bx, t) = c \int_{-\infty}^t dt' \phi(\Bx, t')
\end{align}

Can also find a transformation that also allows $\spacegrad \cdot \BA = 0$

This is the Coulomb gauge 

\begin{align}\label{eqn:relativisticElectrodynamicsL13:570}
\phi &= 0 \\
\spacegrad \cdot \BA &= 0
\end{align}

DIY.  Can use this above to kill a bunch of the terms, leaving 

\begin{equation}\label{eqn:relativisticElectrodynamicsL13:590}
\spacegrad \cross (\spacegrad \cross \BA') = -\inv{c^2} \frac{\partial^2 \BA'}{\partial t^2}
\end{equation}

Using 

\begin{equation}\label{eqn:relativisticElectrodynamicsL13:600}
\spacegrad \cross (\spacegrad \cross \BA') = \spacegrad (\spacegrad \cdot \BA') - \spacegrad^2 \BA',
\end{equation}

we have

\begin{equation}\label{eqn:relativisticElectrodynamicsL13:610}
\inv{c^2} \frac{\partial^2 \BA'}{\partial t^2} -\spacegrad^2 \BA' = 0
\end{equation}

which is the wave equation for the propagation of the vector potential $\BA'(\Bx, t)$ through space at velocity $c$.

\EndArticle

%
% Copyright � 2012 Peeter Joot.  All Rights Reserved.
% Licenced as described in the file LICENSE under the root directory of this GIT repository.
%

%\chapter{Wave equation in Coulomb and Lorentz gauges}
\index{wave equation!Coulomb gauge}
\index{wave equation!Lorentz gauge}
\index{Coulomb gauge!wave equation}
\index{Lorentz gauge!wave equation}
\label{chap:relativisticElectrodynamicsL14}
%\blogpage{http://sites.google.com/site/peeterjoot/math2011/relativisticElectrodynamicsL14.pdf}
%\date{Feb 16, 2011}

\paragraph{Reading}

Covering chapter 4 material from the text \citep{landau1980classical}, and
%Covering \popcite{RelEMpp103-113.pdf}{lecture notes RelEMpp103-113.pdf}: the wave equation in the relativistic Lorentz gauge (114-114) [Tuesday, Feb. 15; Wednesday, Feb.16]...
\popcite{RelEMpp114-127.pdf}{lecture notes RelEMpp114-127.pdf}.
%: reminder on wave equations (114); reminder on Fourier series and integral (115-117); Fourier expansion of the EM potential in Coulomb gauge and equation of motion for the spatial Fourier components (118-119); the general solution of Maxwell's equations in vacuum (120-121) [Tuesday, Mar. 1]; properties of monochromatic plane EM waves (122-124); energy and energy flux of the EM field and energy conservation from the equations of motion (125-127)  [Wednesday, Mar. 2]

\section{Trying to understand ``c''}

\begin{equation}\label{eqn:relativisticElectrodynamicsL14:10}
\begin{aligned}
\spacegrad \cdot \BE &= 0 \\
\spacegrad \cross \BB &= \inv{c} \PD{t}{\BE}
\end{aligned}
\end{equation}

Maxwell's equations in a vacuum were

\begin{equation}\label{eqn:relativisticElectrodynamicsL14:30}
\begin{aligned}
\spacegrad (\spacegrad \cdot \BA) &= \spacegrad^2 \BA  -\inv{c} \PD{t}{} \spacegrad \phi - \inv{c^2} \frac{\partial^2 \BA}{\partial t^2} \\
\spacegrad \cdot \BE &= - \spacegrad^2 \phi - \inv{c} \PD{t}{\spacegrad \cdot \BA} 
\end{aligned}
\end{equation}

There is a redundancy here since we can change \(\phi\) and \(\BA\) without changing the EOM

\begin{equation}\label{eqn:relativisticElectrodynamicsL14:50}
(\phi, \BA) \rightarrow (\phi', \BA')
\end{equation}

with

\begin{equation}\label{eqn:relativisticElectrodynamicsL14:70}
\begin{aligned}
\phi &= \phi' + \inv{c} \PD{t}{\chi} \\
\BA &= \BA' - \spacegrad \chi
\end{aligned}
\end{equation}

\begin{equation}\label{eqn:relativisticElectrodynamicsL14:90}
\chi(\Bx, t) = c \int dt \phi(\Bx, t)
\end{equation}

which gives 

\begin{equation}\label{eqn:relativisticElectrodynamicsL14:110}
\phi' = 0
\end{equation}

\begin{equation}\label{eqn:relativisticElectrodynamicsL14:130}
\begin{aligned}
(\phi, \BA) \sim (\phi = 0, \BA')
\end{aligned}
\end{equation}

Maxwell's equations are now

\begin{equation}\label{eqn:relativisticElectrodynamicsL14:820}
\begin{aligned}
\spacegrad (\spacegrad \cdot \BA') &= \spacegrad^2 \BA'  - \inv{c^2} \frac{\partial^2 \BA'}{\partial t^2} \\
\PD{t}{\spacegrad \cdot \BA'}  &= 0
\end{aligned}
\end{equation}

Can we make \(\spacegrad \cdot \BA'' = 0\), while \(\phi'' = 0\).

\begin{equation}\label{eqn:relativisticElectrodynamicsL14:150}
\begin{aligned}
\mathLabelBox{\phi}{\(=0\)} &= \mathLabelBox
[
   labelstyle={below of=m\themathLableNode, below of=m\themathLableNode}
]
{\phi'}{\(=0\)} + \inv{c} \PD{t}{\chi'} \\
\end{aligned}
\end{equation}

We need 
\begin{equation}\label{eqn:relativisticElectrodynamicsL14:170}
\PD{t}{\chi'} = 0
\end{equation}

How about \(\BA'\)

\begin{equation}\label{eqn:relativisticElectrodynamicsL14:190}
\BA' = \BA'' - \spacegrad \chi'
\end{equation}

We want the divergence of \(\BA'\) to be zero, which means

\begin{equation}\label{eqn:relativisticElectrodynamicsL14:210}
\spacegrad \cdot \BA' = \mathLabelBox{\spacegrad \cdot \BA''}{\(=0\)} - \spacegrad^2 \chi'
\end{equation}

So we want

\begin{equation}\label{eqn:relativisticElectrodynamicsL14:230}
\spacegrad^2 \chi' = \spacegrad \cdot \BA'
\end{equation}

This has the solution

\begin{equation}\label{eqn:relativisticElectrodynamicsL14:240}
\chi'(\Bx) = -\inv{4\pi} \int d^3 \Bx' \frac{\spacegrad' \cdot \BA'(\Bx')}{\Abs{\Bx - \Bx'}}.
\end{equation}

\paragraph{Green's function for the Laplacian}
\paragraph{Laplacian!Green's function}

Recall that in electrostatics we have

\begin{equation}\label{eqn:relativisticElectrodynamicsL14:250}
\spacegrad \cdot \BE = 4 \pi \rho
\end{equation}

and 

\begin{equation}\label{eqn:relativisticElectrodynamicsL14:270}
\BE = -\spacegrad \phi
\end{equation}

which meant that we had 

\begin{equation}\label{eqn:relativisticElectrodynamicsL14:290}
\spacegrad^2 \phi = -4 \pi \rho
\end{equation}

This has the identical form to the equation in \(\chi'\) that we wanted to solve (with \(\phi \sim \chi\), and \(4 \pi \rho \sim \spacegrad \cdot \BA'\)).

Without resorting to electrostatics another way to look at this problem is that it is just a Laplace equation, and we could utilize a Green's function solution if desired.  This would generate the same result for \(\chi'\) above, and also works for the electrostatics case.

Recall that the Green's function for the Laplacian was 

\begin{equation}\label{eqn:relativisticElectrodynamicsL14:292}
G(\Bx, \Bx') = -\inv{4 \pi \Abs{\Bx - \Bx'}}
\end{equation}

with the property 

\begin{equation}\label{eqn:relativisticElectrodynamicsL14:294}
\spacegrad^2 G(\Bx, \Bx') = \delta(\Bx - \Bx')
\end{equation}

Our LDE to solve by Green's method is

\begin{equation}\label{eqn:relativisticElectrodynamicsL14:296}
\spacegrad^2 \phi = 4 \pi \rho,
\end{equation}

We let this equation (after switching to primed coordinates) operate on the Green's function

\begin{equation}\label{eqn:relativisticElectrodynamicsL14:298}
\int d^3 \Bx' {\spacegrad'}^2 \phi(\Bx') G(\Bx, \Bx') 
=
-\int d^3 \Bx' 4 \pi \rho(\Bx') G(\Bx, \Bx').
\end{equation}

Assuming that the left action of the Green's function on the test function \(\phi(\Bx')\) is the same as the right action (i.e. \(\phi(\Bx')\) and \(G(\Bx, \Bx')\) commute), we have for the LHS

\begin{equation}\label{eqn:relativisticElectrodynamicsL14:840}
\begin{aligned}
\int d^3 \Bx' {\spacegrad'}^2 \phi(\Bx') G(\Bx, \Bx') 
&=
\int d^3 \Bx' {\spacegrad'}^2 G(\Bx, \Bx') \phi(\Bx') \\
&=
\int d^3 \Bx' \delta(\Bx - \Bx') \phi(\Bx') \\
&=
\phi(\Bx).
\end{aligned}
\end{equation}

Substitution of \(G(\Bx, \Bx') = -1/4 \pi \Abs{\Bx - \Bx'}\) on the RHS then gives us the general solution

\begin{equation}\label{eqn:relativisticElectrodynamicsL14:300}
\phi(\Bx) = \int d^3 \Bx' \frac{\rho(\Bx') }{\Abs{\Bx - \Bx'}}
\end{equation}

\paragraph{Back to Maxwell's equations in vacuum}
\index{Maxwell's equations!vacuum}
What are the Maxwell's vacuum equations now?

With the second gauge substitution we have

\begin{equation}\label{eqn:relativisticElectrodynamicsL14:860}
\begin{aligned}
\spacegrad (\spacegrad \cdot \BA'') &= \spacegrad^2 \BA''  - \inv{c^2} \frac{\partial^2 \BA''}{\partial t^2} \\
\PD{t}{\spacegrad \cdot \BA''}  &= 0
\end{aligned}
\end{equation}

but we can utilize

\begin{equation}\label{eqn:relativisticElectrodynamicsL14:310}
\spacegrad \cross (\spacegrad \cross \BA) = \spacegrad (\spacegrad \cdot \BA) - \spacegrad^2 \BA,
\end{equation}

to reduce Maxwell's equations (after dropping primes) to just

\begin{equation}\label{eqn:relativisticElectrodynamicsL14:330}
\inv{c^2} \frac{\partial^2 \BA''}{\partial t^2} - \Delta \BA = 0
\end{equation}

where 
\begin{equation}\label{eqn:relativisticElectrodynamicsL14:350}
\Delta = \spacegrad^2 = \spacegrad \cdot \spacegrad = 
\frac{\partial^2}{\partial x^2}
+\frac{\partial^2}{\partial y^2}
+\frac{\partial^2}{\partial y^2}
\end{equation}

Note that for this to be correct we have to also explicitly include the gauge condition used.  This particular gauge is called the \textunderline{Coulomb gauge}.

\begin{equation}\label{eqn:relativisticElectrodynamicsL14:370}
\begin{aligned}
\phi &= 0 \\
\spacegrad \cdot \BA'' &= 0 
\end{aligned}
\end{equation}

\section{Claim: EM waves propagate with speed \texorpdfstring{\(c\)}{c} and are transverse}

\paragraph{Note:} Is the Coulomb gauge Lorentz invariant?
\index{Coulomb gauge}
\index{Lorentz invariant}
\paragraph{No.} We can boost which will introduce a non-zero \(\phi\).
\index{boost}

The gauge that is Lorentz Invariant is the ``Lorentz gauge''.  This one uses
\index{Lorentz gauge}

\begin{equation}\label{eqn:relativisticElectrodynamicsL14:390}
\partial_i A^i = 0
\end{equation}

Recall that Maxwell's equations are

\begin{equation}\label{eqn:relativisticElectrodynamicsL14:410}
\partial_i F^{ij} = j^j = 0
\end{equation}

where 

\begin{equation}\label{eqn:relativisticElectrodynamicsL14:430}
\begin{aligned}
\partial_i &= \PD{x^i}{} \\
\partial^i &= \PD{x_i}{}
\end{aligned}
\end{equation}

Writing out the equations in terms of potentials we have
\begin{equation}\label{eqn:relativisticElectrodynamicsL14:880}
\begin{aligned}
0 &= \partial_i (\partial^i A^j - \partial^j A^i)  \\
&= \partial_i \partial^i A^j - \partial_i \partial^j A^i \\
&= \partial_i \partial^i A^j - \partial^j \partial_i A^i \\
\end{aligned}
\end{equation}

So, if we pick the gauge condition \(\partial_i A^i = 0\), we are left with just 

\begin{equation}\label{eqn:relativisticElectrodynamicsL14:450}
0 = \partial_i \partial^i A^j
\end{equation}

Can we choose \({A'}^i\) such that \(\partial_i A^i = 0\)?

Our gauge condition is 

\begin{equation}\label{eqn:relativisticElectrodynamicsL14:470}
A^i = {A'}^i + \partial^i \chi
\end{equation}

Hit it with a derivative for

\begin{equation}\label{eqn:relativisticElectrodynamicsL14:490}
\partial_i A^i = \partial_i {A'}^i + \partial_i \partial^i \chi
\end{equation}

If we want \(\partial_i A^i = 0\), then we have

\begin{equation}\label{eqn:relativisticElectrodynamicsL14:510}
-\partial_i {A'}^i = \partial_i \partial^i \chi = \left( \inv{c^2} \frac{\partial^2}{\partial t^2} - \Delta \right) \chi
\end{equation}

This is the physicist proof.  Yes, it can be solved.  To really solve this, we would want to use Green's functions.  I seem to recall the Green's function is a retarded time version of the Laplacian Green's function, and we can figure that exact form out by switching to a Fourier frequency domain representation.

Anyways.  Returning to Maxwell's equations we have

\begin{equation}\label{eqn:relativisticElectrodynamicsL14:530}
\begin{aligned}
0 &= \partial_i \partial^i A^j \\
0 &= \partial_i A^i ,
\end{aligned}
\end{equation}

where the first is Maxwell's equation, and the second is our gauge condition.

Observe that the gauge condition is now a Lorentz scalar.

\begin{equation}\label{eqn:relativisticElectrodynamicsL14:550}
\partial^i A_i \rightarrow \partial^j {O_j}^i {O_i}^k A_k
\end{equation}

But the Lorentz transform matrices multiply out to identity, in the same way that they do for the transformation of a plain old four vector dot product \(x^i y_i\).

\section{What happens with a Massive vector field?}

\begin{equation}\label{eqn:relativisticElectrodynamicsL14:570}
S = \int d^4 x \left( \inv{4} F^{ij} F_{ij} + \frac{m^2}{2} A^i A_i \right)
\end{equation}

\paragraph{An aside on units}

``Note that this action is expressed in dimensions where \(\Hbar = c = 1\), making the action is unit-less (energy and time are inverse units of each other).  The \(d^4x\) has units of \(m^{-4}\) (since \([x] = \Hbar/mc\)), so \(F\) has units of \(m^2\), and then \(A\) has units of mass. Therefore \(d^4x A A\) has units of \(m^{-2}\) and therefore you need something that has units of \(m^2\) to make the action unit-less. When you do not take \(c=1\), then you have got to worry about those factors, but I think you will see it works out fine.''

For what it is worth, I can adjust the units of this action to those that we have used in class with,

\begin{equation}\label{eqn:relativisticElectrodynamicsL14:800}
S = \int d^4 x \left( -\inv{16 \pi c} F^{ij} F_{ij} - \frac{m^2 c^2}{8 \Hbar^2} A^i A_i \right)
\end{equation}

\paragraph{Back to the problem}

%\begin{align*}
%\delta S 
%&= 0  \\
%&= \int d^4 x \left( \partial^i \partial_i A_j - \partial_j \partial_i A^i \delta A^i + m^2 A_i \delta A^i \right)
%\end{align*}
%
The variation of the field invariant is

\begin{equation}\label{eqn:relativisticElectrodynamicsL14:900}
\begin{aligned}
\delta (F_{ij} F^{ij})
&=
2 (\delta F_{ij}) F^{ij}) \\
&=
2 (\delta(\partial_i A_j -\partial_j A_i)) F^{ij}) \\
&=
2 (\partial_i \delta(A_j) -\partial_j \delta(A_i)) F^{ij}) \\
&=
4 F^{ij} \partial_i \delta(A_j) \\
&=
4 \partial_i (F^{ij} \delta(A_j)) - 4 (\partial_i F^{ij}) \delta(A_j).
\end{aligned}
\end{equation}

Variation of the \(A^2\) term gives us

\begin{equation}\label{eqn:relativisticElectrodynamicsL14:700}
\delta (A^j A_j) = 2 A^j \delta(A_j),
\end{equation}

so we have

\begin{equation}\label{eqn:relativisticElectrodynamicsL14:920}
\begin{aligned}
0 &= \delta S \\
&= \int d^4 x \delta(A_j) \left( -\partial_i F^{ij} + m^2 A^j \right)
+ \int d^4 x \partial_i (F^{ij} \delta(A_j))
\end{aligned}
\end{equation}

The last integral vanishes on the boundary with the assumption that \(\delta(A_j) = 0\) on that boundary.

Since this must be true for all variations, this leaves us with

\begin{equation}\label{eqn:relativisticElectrodynamicsL14:720}
\partial_i F^{ij} = m^2 A^j
\end{equation}

The RHS can be expanded into wave equation and divergence parts

\begin{equation}\label{eqn:relativisticElectrodynamicsL14:940}
\begin{aligned}
\partial_i F^{ij}
&=
\partial_i (\partial^i A^j - \partial^j A^i) \\
&=
(\partial_i \partial^i) A^j - \partial^j (\partial_i A^i) \\
\end{aligned}
\end{equation}

With \(\square\) for the wave equation operator

\begin{equation}\label{eqn:relativisticElectrodynamicsL14:750}
\square = \partial_i \partial^i = \inv{c^2} \PDSq{t}{} - \Delta,
\end{equation}

we can manipulate the EOM to pull out an \(A_i\) factor

\begin{equation}\label{eqn:relativisticElectrodynamicsL14:960}
\begin{aligned}
0 
&= \left( \square -m^2 \right) A^j - \partial^j (\partial_i A^i) \\
&= \left( \square -m^2 \right) g^{ij} A_i - \partial^j (\partial^i A_i) \\
&= \left( \left( \square -m^2 \right) g^{ij} - \partial^j \partial^i \right) A_i.
\end{aligned}
\end{equation}

If we hit this with a derivative we get	

\begin{equation}\label{eqn:relativisticElectrodynamicsL14:980}
\begin{aligned}
0 
&= \partial_j \left( \left( \square -m^2 \right) g^{ij} - \partial^j \partial^i \right) A_i \\
&= \left( \left( \square -m^2 \right) \partial^i - \partial_j \partial^j \partial^i \right) A_i \\
&= \left( \left( \square -m^2 \right) \partial^i - \square \partial^i \right) A_i \\
&= \left( \square -m^2 - \square \right) \partial^i A_i \\
&= -m^2 \partial^i A_i \\
\end{aligned}
\end{equation}

Since \(m\) is presumed to be non-zero here, this means that the Lorentz gauge is already chosen for us by the equations of motion.

%
% Copyright � 2012 Peeter Joot.  All Rights Reserved.
% Licenced as described in the file LICENSE under the root directory of this GIT repository.
%

%\chapter{Fourier solution of Maxwell's vacuum wave equation in the Coulomb gauge}
\index{wave equation!Coulomb gauge}
\label{chap:relativisticElectrodynamicsL15}
%\blogpage{http://sites.google.com/site/peeterjoot/math2011/relativisticElectrodynamicsL15.pdf}
%\date{Mar 1, 2011}

\paragraph{Reading}

Covering chapter 6 material from the text \citep{landau1980classical}, and
\popcite{RelEMpp114-127.pdf}{lecture notes RelEMpp114-127.pdf}.
%: reminder on wave equations (115); reminder on Fourier series and integral (115-117); Fourier expansion of the EM potential in Coulomb gauge and equation of motion for the spatial Fourier components (118-119); the general solution of Maxwell's equations in vacuum (120-121) [Tuesday, Mar. 1]

\section{Review of wave equation results obtained}
\index{wave equation}

Maxwell's equations in vacuum lead to Coulomb gauge and the Lorentz gauge.

\paragraph{Coulomb gauge}
\index{Coulomb gauge}

\begin{equation}\label{eqn:relativisticElectrodynamicsL15:10}
\begin{aligned}
A^0 &= 0 \\
\spacegrad \cdot \BA &= 0 \\
\left( \inv{c^2} \PDSq{t}{} - \Delta \right) \BA &= 0
\end{aligned}
\end{equation}

\paragraph{Lorentz gauge}
\index{Lorentz gauge}

\begin{equation}\label{eqn:relativisticElectrodynamicsL15:30}
\begin{aligned}
\partial_i A^i &= 0 \\
\left( \inv{c^2} \PDSq{t}{} - \Delta \right) A^i &= 0
\end{aligned}
\end{equation}

Note that \(\partial_i A^i = 0\) is invariant under gauge transformations

\begin{equation}\label{eqn:relativisticElectrodynamicsL15:50}
A^i \rightarrow A^i + \partial^i \chi
\end{equation}

where 

\begin{equation}\label{eqn:relativisticElectrodynamicsL15:70}
\partial_i \partial^i \chi = 0,
\end{equation}

So if one uses the Lorentz gauge, this has to be fixed.

However, in both cases we have 

\begin{equation}\label{eqn:relativisticElectrodynamicsL15:90}
\left( \inv{c^2} \PDSq{t}{} - \Delta \right) f = 0
\end{equation}

where 

\begin{equation}\label{eqn:relativisticElectrodynamicsL15:110}
\inv{c^2} \PDSq{t}{} - \Delta 
\end{equation}

is the wave operator.

Consider 

\begin{equation}\label{eqn:relativisticElectrodynamicsL15:130}
\Delta = \PDSq{x}{}
\end{equation}

where we are looking for a solution that is independent of \(y, z\).  Recall that the general solution for this equation has the form

\begin{equation}\label{eqn:relativisticElectrodynamicsL15:150}
f(t, x) = 
F_1 \left(t - \frac{x}{c}\right)
+F_2 \left(t + \frac{x}{c}\right)
\end{equation}

PICTURE: superposition of two waves with \(F_1\) moving along the x-axis in the positive direction, and \(F_2\) in the negative x direction.

It is notable that the text derives \eqnref{eqn:relativisticElectrodynamicsL15:150} in a particularly slick way.  It is still black magic, since one has to know the solution to find it, but very very cool.

\section{Review of Fourier methods}

It is often convenient to impose periodic boundary conditions

\begin{equation}\label{eqn:relativisticElectrodynamicsL15:170}
\BA(\Bx + \Be_i L) = \BA(\Bx), i = 1,2,3
\end{equation}

\paragraph{In one dimension}

\begin{equation}\label{eqn:relativisticElectrodynamicsL15:190}
f(x + L) = f(x)
\end{equation}

\begin{equation}\label{eqn:relativisticElectrodynamicsL15:210}
f(x) = \sum_{n=-\infty}^\infty e^{i \frac{2 \pi n}{L} x} \tilde{f}_n
\end{equation}

When \(f(x)\) is real we also have

\begin{equation}\label{eqn:relativisticElectrodynamicsL15:230}
f^\conj(x) = \sum_{n = -\infty}^\infty e^{-i \frac{2 \pi n}{L} x} (\tilde{f}_n)^\conj
\end{equation}

which implies

\begin{equation}\label{eqn:relativisticElectrodynamicsL15:250}
{\tilde{f}^\conj}_{n} = \tilde{f}_{-n}.
\end{equation}

We introduce a wave number

\begin{equation}\label{eqn:relativisticElectrodynamicsL15:270}
k_n = \frac{2 \pi n}{L},
\end{equation}

allowing a slightly simpler expression of the Fourier decomposition

\begin{equation}\label{eqn:relativisticElectrodynamicsL15:290}
f(x) = \sum_{n=-\infty}^\infty e^{i k_n x} \tilde{f}_{k_n}.
\end{equation}

The inverse transform is obtained by integration over some length \(L\) interval

\begin{equation}\label{eqn:relativisticElectrodynamicsL15:310}
\tilde{f}_{k_n} = \inv{L} \int_{-L/2}^{L/2} dx e^{-i k_n x} f(x)
\end{equation}

\paragraph{Verify:}

We should be able to recover the Fourier coefficient by utilizing the above

\begin{equation}\label{eqn:relativisticElectrodynamicsL15:880}
\begin{aligned}
\inv{L} \int_{-L/2}^{L/2} dx e^{-i k_n x} \sum_{m=-\infty}^\infty e^{i k_m x} \tilde{f}_{k_m} \\
&= \sum_{m = -\infty}^\infty \tilde{f}_{k_m} \delta_{mn} = \tilde{f}_{k_n},
\end{aligned}
\end{equation}

where we use the easily verifiable fact that 

\begin{equation}\label{eqn:relativisticElectrodynamicsL15:800}
\inv{L} \int_{-L/2}^{L/2} dx e^{i (k_m - k_n) x} = 
\begin{array}{l l}
0 & \quad \mbox{if \(m \ne n\)} \\
1 & \quad \mbox{if \(m = n\)} \\
\end{array}.
\end{equation}

It is conventional to absorb \(\tilde{f}_{k_n} = \tilde{f}(k_n)\) for

\begin{equation}\label{eqn:relativisticElectrodynamicsL15:330}
\begin{aligned}
f(x) &= \inv{L} \sum_n \tilde{f}(k_n) e^{i k_n x} \\
\tilde{f}(k_n) &= \int_{-L/2}^{L/2} dx f(x) e^{-i k_n x}
\end{aligned}
\end{equation}

To take \(L \rightarrow \infty\) notice

\begin{equation}\label{eqn:relativisticElectrodynamicsL15:350}
k_n = \frac{2 \pi}{L} n
\end{equation}

when \(n\) changes by \(\Delta n = 1\), \(k_n\) changes by \(\Delta k_n = \frac{2 \pi}{L} \Delta n\)

Using this 

\begin{equation}\label{eqn:relativisticElectrodynamicsL15:370}
f(x) = \inv{2\pi} \sum_n \left( \frac{2\pi}{L} \Delta n \right) \tilde{f}(k_n) e^{i k_n x}
\end{equation}

With \(L \rightarrow \infty\), and \(\Delta k_n \rightarrow 0\)

\begin{equation}\label{eqn:relativisticElectrodynamicsL15:390}
\begin{aligned}
f(x) &= \int_{-\infty}^\infty \frac{dk}{2\pi} \tilde{f}(k) e^{i k x} \\
\tilde{f}(k) &= \int_{-\infty}^\infty dx f(x) e^{-i k x}
\end{aligned}
\end{equation}

\paragraph{Verify:}

A loose verification of the inversion relationship (the most important bit) is possible by substitution

\begin{equation}\label{eqn:relativisticElectrodynamicsL15:900}
\begin{aligned}
\int \frac{dk}{2\pi} e^{i k x} \tilde{f}(k) 
&= 
\iint \frac{dk}{2\pi} e^{i k x} dx' f(x') e^{-i k x'} \\
&= 
\int dx' f(x') \inv{2\pi} \int dk e^{i k (x - x')}
\end{aligned}
\end{equation}

Now we employ the old physics ploy where we identify

\begin{equation}\label{eqn:relativisticElectrodynamicsL15:820}
\inv{2\pi} \int dk e^{i k (x - x')} = \delta(x - x').
\end{equation}

With that we see that we recover the function \(f(x)\) above as desired.

\paragraph{In three dimensions}

\begin{equation}\label{eqn:relativisticElectrodynamicsL15:470}
\begin{aligned}
\BA(\Bx, t) &= \int 
\frac{d^3 \Bk}{(2\pi)^3} 
\tilde{\BA}(\Bk, t) e^{i \Bk \cdot \Bx} \\
\tilde{\BA}(\Bx, t) &= \int d^3 \Bx \BA(\Bx, t) e^{-i \Bk \cdot \Bx}
\end{aligned}
\end{equation}

\paragraph{Application to the wave equation}
\index{wave equation}

\begin{equation}\label{eqn:relativisticElectrodynamicsL15:920}
\begin{aligned}
0 &= 
\left( \inv{c^2} \PDSq{t}{} - \Delta \right) \BA(\Bx, t) \\
&=
\left( \inv{c^2} \PDSq{t}{} - \Delta \right) 
\int 
\frac{d^3 \Bk}{(2\pi)^3} 
\tilde{\BA}(\Bk, t) e^{i \Bk \cdot \Bx} \\
&=
\int 
\frac{d^3 \Bk}{(2\pi)^3} 
\left( 
\inv{c^2} \partial_{tt} \tilde{\BA}(\Bk, t) + \Bk^2 \BA(\Bk, t)
\right)
e^{i \Bk \cdot \Bx} 
\end{aligned}
\end{equation}

Now operate with \(\int d^3 \Bx e^{-i \Bp \cdot \Bx }\)

\begin{equation}\label{eqn:relativisticElectrodynamicsL15:940}
\begin{aligned}
0 &=
\int d^3 \Bx e^{-i \Bp \cdot \Bx }
\int 
\frac{d^3 \Bk}{(2\pi)^3} 
\left( 
\inv{c^2} \partial_{tt} \tilde{\BA}(\Bk, t) + \Bk^2 \BA(\Bk, t)
\right)
e^{i \Bk \cdot \Bx}  \\
&=
\int 
d^3 \Bk
\delta^3(\Bp -\Bk) 
\left( 
\inv{c^2} \partial_{tt} \tilde{\BA}(\Bk, t) + \Bk^2 \BA(\Bk, t)
\right)
\end{aligned}
\end{equation}

Since this is true for all \(\Bp\) we have

\begin{equation}\label{eqn:relativisticElectrodynamicsL15:490}
\partial_{tt} \tilde{\BA}(\Bp, t) = -c^2 \Bp^2 \tilde{\BA}(\Bp, t) 
\end{equation}

For every value of momentum we have a harmonic oscillator!

\begin{equation}\label{eqn:relativisticElectrodynamicsL15:510}
\ddot{x} = -\omega^2 x
\end{equation}

Fourier modes of EM potential in vacuum obey

\begin{equation}\label{eqn:relativisticElectrodynamicsL15:530}
\partial_{tt} \tilde{\BA}(\Bk, t) = -c^2 \Bk^2 \tilde{\BA}(\Bk, t)
\end{equation}

Because we are operating in the Coulomb gauge we must also have zero divergence.  Let us see how that translates to our Fourier representation


implies

\begin{equation}\label{eqn:relativisticElectrodynamicsL15:960}
\begin{aligned}
0 &= \spacegrad \cdot \BA(\Bx, t) \\
&= \int \frac{d^3 \Bk }{(2 \pi)^3} \spacegrad \cdot \left( e^{i \Bk \cdot \Bx} \cdot \tilde{\BA}(\Bk, t) \right)
\end{aligned}
\end{equation}

The chain rule for the divergence in this case takes the form

\begin{equation}\label{eqn:relativisticElectrodynamicsL15:840}
\spacegrad \cdot (\phi \BB) = (\spacegrad \phi) \cdot \BB + \phi \spacegrad \cdot \BB.
\end{equation}

But since our vector function \(\tilde{\BA}\) is not a function of spatial coordinates we have
\begin{equation}\label{eqn:relativisticElectrodynamicsL15:570}
0 = \int \frac{d^3 \Bk }{(2 \pi)^3} e^{i \Bk \cdot \Bx} (i \Bk \cdot \tilde{\BA}(\Bk, t)).
\end{equation}

This has two immediate consequences.  The first is that our momentum space potential is perpendicular to the wave number vector at all points in momentum space, and the second gives us a conjugate relation (substitute \(\Bk \rightarrow -\Bk'\) after taking conjugates for that one)

\begin{equation}\label{eqn:relativisticElectrodynamicsL15:590}
\begin{aligned}
\Bk \cdot \tilde{\BA}(\Bk, t) &= 0 \\
\tilde{\BA}(-\Bk, t) &= \tilde{\BA}^\conj(\Bk, t).
\end{aligned}
\end{equation}

\begin{equation}\label{eqn:relativisticElectrodynamicsL15:610}
\BA(\Bx, t) = \int 
\frac{d^3 \Bk}{(2\pi)^3} 
e^{i \Bk \cdot \Bx} \left( \inv{2} \tilde{\BA}(\Bk, t) + \inv{2} \tilde{\BA}^\conj(- \Bk, t) \right)
\end{equation}

Since out system is essentially a harmonic oscillator at each point in momentum space

\begin{equation}\label{eqn:relativisticElectrodynamicsL15:630}
\begin{aligned}
\partial_{tt} \tilde{\BA}(\Bk, t) &= - \omega_k^2 \tilde{\BA}(\Bk, t) \\
\omega_k^2 &= c^2 \Bk^2
\end{aligned}
\end{equation}

our general solution is of the form

\begin{equation}\label{eqn:relativisticElectrodynamicsL15:650}
\begin{aligned}
\tilde{\BA}(\Bk, t) &= e^{i \omega_k t} \Ba_{+}(\Bk) +e^{-i \omega_k t} \Ba_{-}(\Bk) \\
\tilde{\BA}^\conj(\Bk, t) &= e^{-i \omega_k t} \Ba_{+}^\conj(\Bk) +e^{i \omega_k t} \Ba_{-}^\conj(\Bk)
\end{aligned}
\end{equation}

\begin{equation}\label{eqn:relativisticElectrodynamicsL15:670}
\BA(\Bx, t) 
= \int \frac{d^3 \Bk}{(2 \pi)^3} e^{i \Bk \cdot \Bx} 
\inv{2} \left( 
e^{i \omega_k t} (\Ba_{+}(\Bk) + \Ba_{-}^\conj(-\Bk)) 
+e^{-i \omega_k t} (\Ba_{-}(\Bk) + \Ba_{+}^\conj(-\Bk)) 
\right)
\end{equation}

Define

\begin{equation}\label{eqn:relativisticElectrodynamicsL15:690}
\Bbeta(\Bk) \equiv \inv{2} (\Ba_{-}(\Bk) + \Ba_{+}^\conj(-\Bk) )
\end{equation}

so that

\begin{equation}\label{eqn:relativisticElectrodynamicsL15:710}
\Bbeta(-\Bk) = \inv{2} (\Ba_{+}^\conj(\Bk) + \Ba_{-}(-\Bk))
\end{equation}

Our solution now takes the form

\begin{equation}\label{eqn:relativisticElectrodynamicsL15:730}
\BA(\Bx, t) = \int \frac{d^3\Bk}{(2 \pi)^3} \left( 
e^{i (\Bk \cdot \Bx + \omega_k t)} \Bbeta^\conj(-\Bk)
+e^{i (\Bk \cdot \Bx - \omega_k t)} \Bbeta(\Bk)
\right)
\end{equation}

\paragraph{Claim:}

This is now manifestly real.  To see this, consider the first term with \(\Bk = -\Bk'\), noting that \(\int_{-\infty}^\infty dk = \int_{\infty}^\infty -dk' = \int_{-\infty}^\infty dk' \) with \(dk = -dk'\)

\begin{equation}\label{eqn:relativisticElectrodynamicsL15:860}
\int \frac{d^3\Bk'}{(2 \pi)^3} e^{i (-\Bk' \cdot \Bx + \omega_k t)} \Bbeta^\conj(\Bk')
\end{equation}

Dropping primes this is the conjugate of the second term.

\paragraph{Claim:}

We have \(\Bk \cdot \Bbeta(\Bk)  = 0\).

Since we have \(\Bk \cdot \tilde{\BA}(\Bk, t) = 0\), \eqnref{eqn:relativisticElectrodynamicsL15:650} implies that we have \(\Bk \cdot \Ba_{\pm}(\Bk) = 0\).  With each of these vector integration constants being perpendicular to \(\Bk\) at that point in momentum space, so must be the linear combination of these constants \(\Bbeta(\Bk)\).

%
% what I've marked as T4 was really T5.  T4 was the fun with tensors
% stuff that was part of the ungraded portion of the problem set.
%
%
% Copyright � 2012 Peeter Joot.  All Rights Reserved.
% Licenced as described in the file LICENSE under the root directory of this GIT repository.
%

%\chapter{Waveguides: confined EM waves}
\label{chap:relativisticElectrodynamicsT4}
%\blogpage{http://sites.google.com/site/peeterjoot/math2011/relativisticElectrodynamicsT4.pdf}
%\date{Mar 3, 2011}

\section{Motivation}

While this is not part of the course, the topic of waveguides is one of so many applications that it is worth a mention, and that will be done in this tutorial.

We will setup our system with a waveguide (conducting surface that confines the radiation) oriented in the \(\zcap\) direction.  The shape can be arbitrary

PICTURE: cross section of wacky shape.

\paragraph{At the surface of a conductor}

At the surface of the conductor (I presume this means the interior surface where there is no charge or current enclosed) we have

\begin{equation}\label{eqn:relativisticElectrodynamicsT4:10}
\begin{aligned}
\spacegrad \cross \BE &= - \inv{c} \PD{t}{\BB} \\
\spacegrad \cross \BB &= \inv{c} \PD{t}{\BE} \\
\spacegrad \cdot \BB &= 0 \\
\spacegrad \cdot \BE &= 0
\end{aligned}
\end{equation}

If we are talking about the exterior surface, do we need to make any other assumptions (perfect conductors, or constant potentials)?

\paragraph{Wave equations}

For electric and magnetic fields in vacuum, we can show easily that these, like the potentials, separately satisfy the wave equation

Taking curls of the Maxwell curl equations above we have

\begin{equation}\label{eqn:relativisticElectrodynamicsT4:30}
\begin{aligned}
\spacegrad \cross (\spacegrad \cross \BE) &= - \inv{c^2} \PDSq{t}{\BE} \\
\spacegrad \cross (\spacegrad \cross \BB) &= - \inv{c^2} \PDSq{t}{\BB},
\end{aligned}
\end{equation}

but we have for vector \(\BM\)

\begin{equation}\label{eqn:relativisticElectrodynamicsT4:50}
\spacegrad \cross (\spacegrad \cross \BM)
=
\spacegrad (\spacegrad \cdot \BM) - \Delta \BM,
\end{equation}

which gives us a pair of wave equations

\begin{equation}\label{eqn:relativisticElectrodynamicsT4:70}
\begin{aligned}
\square \BE &= 0 \\
\square \BB &= 0.
\end{aligned}
\end{equation}

We still have the original constraints of Maxwell's equations to deal with, but we are free now to pick the complex exponentials as fundamental solutions, as our starting point

\begin{equation}\label{eqn:relativisticElectrodynamicsT4:90}
\begin{aligned}
\BE &= \BE_0 e^{i k^a x_a} = \BE_0 e^{ i (k^0 x_0 - \Bk \cdot \Bx) } \\
\BB &= \BB_0 e^{i k^a x_a} = \BB_0 e^{ i (k^0 x_0 - \Bk \cdot \Bx) },
\end{aligned}
\end{equation}

With \(k_0 = \omega/c\) and \(x_0 = c t\) this is

\begin{equation}\label{eqn:relativisticElectrodynamicsT4:110}
\begin{aligned}
\BE &= \BE_0 e^{ i (\omega t - \Bk \cdot \Bx) } \\
\BB &= \BB_0 e^{ i (\omega t - \Bk \cdot \Bx) }.
\end{aligned}
\end{equation}

For the vacuum case, with monochromatic light, we treated the amplitudes as constants.  Let us see what happens if we relax this assumption, and allow for spatial dependence (but no time dependence) of \(\BE_0\) and \(\BB_0\).  For the LHS of the electric field curl equation we have

\begin{equation}\label{eqn:relativisticElectrodynamicsT4:580}
\begin{aligned}
0 
&= \spacegrad \cross \BE_0 e^{i k_a x^a} \\
&= (\spacegrad \cross \BE_0 - \BE_0 \cross \spacegrad) e^{i k_a x^a} \\
&= (\spacegrad \cross \BE_0 - \BE_0 \cross \Be^\alpha i k_a \partial_\alpha x^a) e^{i k_a x^a} \\
&= (\spacegrad \cross \BE_0 + \BE_0 \cross \Be^\alpha i k^a {\delta_\alpha}^a ) e^{i k_a x^a} \\
&= (\spacegrad \cross \BE_0 + i \BE_0 \cross \Bk ) e^{i k_a x^a}.
\end{aligned}
\end{equation}

Similarly for the divergence we have

\begin{equation}\label{eqn:relativisticElectrodynamicsT4:600}
\begin{aligned}
0 
&= \spacegrad \cdot \BE_0 e^{i k_a x^a} \\
&= (\spacegrad \cdot \BE_0 + \BE_0 \cdot \spacegrad) e^{i k_a x^a} \\
&= (\spacegrad \cdot \BE_0 + \BE_0 \cdot \Be^\alpha i k_a \partial_\alpha x^a) e^{i k_a x^a} \\
&= (\spacegrad \cdot \BE_0 - \BE_0 \cdot \Be^\alpha i k^a {\delta_\alpha}^a ) e^{i k_a x^a} \\
&= (\spacegrad \cdot \BE_0 - i \Bk \cdot \BE_0 ) e^{i k_a x^a}.
\end{aligned}
\end{equation}

This provides constraints on the amplitudes

\begin{equation}\label{eqn:relativisticElectrodynamicsT4:130}
\begin{aligned}
\spacegrad \cross \BE_0 - i \Bk \cross \BE_0 &= -i \frac{\omega}{c} \BB_0 \\
\spacegrad \cross \BB_0 - i \Bk \cross \BB_0 &= i \frac{\omega}{c} \BE_0 \\
\spacegrad \cdot \BE_0 - i \Bk \cdot \BE_0 &= 0 \\
\spacegrad \cdot \BB_0 - i \Bk \cdot \BB_0 &= 0
\end{aligned}
\end{equation}

%Having seen that taking curls of the curl equations gave us the wave equation, let us see what happens when we do so in momentum space.  It is worth recalling that
%
%\begin{equation}\label{eqn:relativisticElectrodynamicsT4:150}
%\spacegrad \cross (\Bk \cross \BM) = \Bk (\spacegrad \cdot \BM) - (\Bk \cdot \spacegrad) \BM,
%\end{equation}
%
%where \(\Bk\) is a constant, and \(\BM\) is not.  We have
%
%\begin{align*}
%\spacegrad (\spacegrad \cdot \BE_0) - \Delta \BE_0 - i (
%\Bk (\spacegrad \cdot \BE_0) - (\Bk \cdot \spacegrad) \BE_0
%) &= -i \frac{\omega}{c} \spacegrad \cross \BB_0.
%\end{align*}
%
%This does not look like it will be terribly helpful.  Backing up, and 
Applying the wave equation operator to our phasor we get

\begin{equation}\label{eqn:relativisticElectrodynamicsT4:620}
\begin{aligned}
0 &=
\left(\inv{c^2} \partial_{tt} - \spacegrad^2 \right) \BE_0 e^{i (\omega t - \Bk \cdot \Bx)} \\
&=
\left(-\frac{\omega^2}{c^2} - \spacegrad^2 + \Bk^2 \right) \BE_0 e^{i (\omega t - \Bk \cdot \Bx)}
\end{aligned}
\end{equation}

So the momentum space equivalents of the wave equations are

\begin{equation}\label{eqn:relativisticElectrodynamicsT4:170}
\begin{aligned}
\left( \spacegrad^2 +\frac{\omega^2}{c^2} - \Bk^2 \right) \BE_0 &= 0 \\
\left( \spacegrad^2 +\frac{\omega^2}{c^2} - \Bk^2 \right) \BB_0 &= 0.
\end{aligned}
\end{equation}

Observe that if \(c^2 \Bk^2 = \omega^2\), then these amplitudes are harmonic functions (solutions to the Laplacian equation).  However, it does not appear that we require such a light like relation for the four vector \(k^a = (\omega/c, \Bk)\).

\section{Back to the tutorial notes}

In class we went straight to an assumed solution of the form

\begin{equation}\label{eqn:relativisticElectrodynamicsT4:190}
\begin{aligned}
\BE &= \BE_0(x, y) e^{ i(\omega t - k z) } \\
\BB &= \BB_0(x, y) e^{ i(\omega t - k z) },
\end{aligned}
\end{equation}

where \(\Bk = k \zcap\).  Our Laplacian was also written as the sum of components in the propagation and perpendicular directions

\begin{equation}\label{eqn:relativisticElectrodynamicsT4:210}
\spacegrad^2 = \PDSq{x_\perp}{} + \PDSq{z}{}.
\end{equation}

With no \(z\) dependence in the amplitudes we have

\begin{equation}\label{eqn:relativisticElectrodynamicsT4:170b}
\begin{aligned}
\left( \PDSq{x_\perp}{} +\frac{\omega^2}{c^2} - \Bk^2 \right) \BE_0 &= 0 \\
\left( \PDSq{x_\perp}{} +\frac{\omega^2}{c^2} - \Bk^2 \right) \BB_0 &= 0.
\end{aligned}
\end{equation}

\section{Separation into components}

It was left as an exercise to separate out our Maxwell equations, so that our field components \(\BE_0 = \BE_\perp + \BE_z\) and \(\BB_0 = \BB_\perp + \BB_z\) in the propagation direction, and components in the perpendicular direction are separated

\begin{equation}\label{eqn:relativisticElectrodynamicsT4:640}
\begin{aligned}
\spacegrad \cross \BE_0 
&=
(\spacegrad_\perp + \zcap\partial_z) \cross \BE_0 \\
&=
\spacegrad_\perp \cross \BE_0 \\
&=
\spacegrad_\perp \cross (\BE_\perp + \BE_z) \\
&=
\spacegrad_\perp \cross \BE_\perp 
+\spacegrad_\perp \cross \BE_z \\
&=
( \xcap \partial_x +\ycap \partial_y ) \cross ( \xcap E_x +\ycap E_y ) 
+\spacegrad_\perp \cross \BE_z \\
&=
\zcap (\partial_x E_y - \partial_z E_z) 
+\spacegrad_\perp \cross \BE_z.
\end{aligned}
\end{equation}

We can do something similar for \(\BB_0\).  This allows for a split of \eqnref{eqn:relativisticElectrodynamicsT4:130} into \(\zcap\) and perpendicular components

\begin{equation}\label{eqn:relativisticElectrodynamicsT4:130b}
\begin{aligned}
% zcap terms
\spacegrad_\perp \cross \BE_\perp &= -i \frac{\omega}{c} \BB_z \\
\spacegrad_\perp \cross \BB_\perp &= i \frac{\omega}{c} \BE_z \\
% perp terms
\spacegrad_\perp \cross \BE_z - i \Bk \cross \BE_\perp &= -i \frac{\omega}{c} \BB_\perp \\
\spacegrad_\perp \cross \BB_z - i \Bk \cross \BB_\perp &= i \frac{\omega}{c} \BE_\perp \\
\spacegrad_\perp \cdot \BE_\perp &= i k E_z - \partial_z E_z \\
\spacegrad_\perp \cdot \BB_\perp &= i k B_z - \partial_z B_z.
\end{aligned}
\end{equation}

So we see that once we have a solution for \(\BE_z\) and \(\BB_z\) (by solving the wave equation above for those components), the components for the fields in terms of those components can be found.  Alternately, if one solves for the perpendicular components of the fields, these propagation components are available immediately with only differentiation.

In the case where the perpendicular components are taken as given

\begin{equation}\label{eqn:relativisticElectrodynamicsT4:300}
\begin{aligned}
\BB_z &= i \frac{ c  }{\omega} \spacegrad_\perp \cross \BE_\perp \\
\BE_z &= -i \frac{ c  }{\omega} \spacegrad_\perp \cross \BB_\perp,
\end{aligned}
\end{equation}

we can express the remaining ones strictly in terms of the perpendicular fields

\begin{equation}\label{eqn:relativisticElectrodynamicsT4:320}
\begin{aligned}
\frac{\omega}{c} \BB_\perp &= \frac{c}{\omega} \spacegrad_\perp \cross (\spacegrad_\perp \cross \BB_\perp) + \Bk \cross \BE_\perp \\
\frac{\omega}{c} \BE_\perp &= \frac{c}{\omega} \spacegrad_\perp \cross (\spacegrad_\perp \cross \BE_\perp) - \Bk \cross \BB_\perp \\
\spacegrad_\perp \cdot \BE_\perp &= -i \frac{c}{\omega} (i k - \partial_z) \zcap \cdot (\spacegrad_\perp \cross \BB_\perp) \\
\spacegrad_\perp \cdot \BB_\perp &= i \frac{c}{\omega} (i k - \partial_z) \zcap \cdot (\spacegrad_\perp \cross \BE_\perp).
\end{aligned}
\end{equation}

Is it at all helpful to expand the double cross products?

\begin{equation}\label{eqn:relativisticElectrodynamicsT4:660}
\begin{aligned}
\frac{\omega^2}{c^2} \BB_\perp 
&= 
\spacegrad_\perp (\spacegrad_\perp \cdot \BB_\perp) -{\spacegrad_\perp}^2 \BB_\perp + \frac{\omega}{c} \Bk \cross \BE_\perp \\
&= 
i \frac{c}{\omega}
(i k - \partial_z)
\spacegrad_\perp \zcap \cdot (\spacegrad_\perp \cross \BE_\perp)
-{\spacegrad_\perp}^2 \BB_\perp + \frac{\omega}{c} \Bk \cross \BE_\perp 
\end{aligned}
\end{equation}
%\frac{\omega}{c} \BE_\perp &= 
%\frac{c}{\omega} (
%\spacegrad_\perp (\spacegrad_\perp \cdot \BE_\perp)
%-{\spacegrad_\perp}^2 \BE_\perp
%)
%- \Bk \cross \BB_\perp  \\

This gives us
\begin{equation}\label{eqn:relativisticElectrodynamicsT4:340}
\begin{aligned}
\left( {\spacegrad_\perp}^2 + \frac{\omega^2}{c^2} \right) \BB_\perp 
&= - \frac{c}{\omega} (k + i\partial_z) \spacegrad_\perp \zcap \cdot (\spacegrad_\perp \cross \BE_\perp) + \frac{\omega}{c} \Bk \cross \BE_\perp \\
\left( {\spacegrad_\perp}^2 + \frac{\omega^2}{c^2} \right) \BE_\perp 
&= -\frac{c}{\omega} (k + i\partial_z) \spacegrad_\perp \zcap \cdot (\spacegrad_\perp \cross \BB_\perp) - \frac{\omega}{c} \Bk \cross \BB_\perp,
\end{aligned}
\end{equation}

but that does not seem particularly useful for completely solving the system?  It appears fairly messy to try to solve for \(\BE_\perp\) and \(\BB_\perp\) given the propagation direction fields.  I wonder if there is a simplification available that I am missing?

\section{Solving the momentum space wave equations}

Back to the class notes.  We proceeded to solve for \(\BE_z\) and \(\BB_z\) from the wave equations by separation of variables.  We wish to solve equations of the form

\begin{equation}\label{eqn:relativisticElectrodynamicsT4:360}
\left( \PDSq{x}{} + \PDSq{y}{} + \frac{\omega^2}{c^2} - \Bk^2 \right) \phi(x,y) = 0
\end{equation}

Write \(\phi(x,y) = X(x) Y(y)\), so that we have

\begin{equation}\label{eqn:relativisticElectrodynamicsT4:380}
\frac{X''}{X} + \frac{Y''}{Y} = \Bk^2 - \frac{\omega^2}{c^2}
\end{equation}

One solution is sinusoidal
\begin{equation}\label{eqn:relativisticElectrodynamicsT4:400}
\begin{aligned}
\frac{X''}{X} &= -k_1^2 \\
\frac{Y''}{Y} &= -k_2^2 \\
-k_1^2 - k_2^2
&= \Bk^2 - \frac{\omega^2}{c^2}.
\end{aligned}
\end{equation}

The example in the tutorial now switched to a rectangular waveguide, still oriented with the propagation direction down the z-axis, but with lengths \(a\) and \(b\) along the \(x\) and \(y\) axis respectively.

Writing \(k_1 = 2\pi m/a\), and \(k_2 = 2 \pi n/ b\), we have

\begin{equation}\label{eqn:relativisticElectrodynamicsT4:420}
\phi(x, y) = \sum_{m n} a_{m n} 
\exp\left( \frac{2 \pi i m}{a} x \right)
\exp\left( \frac{2 \pi i n}{b} y \right)
\end{equation}

We were also provided with some definitions

\begin{definition}TE (Transverse Electric)

\(\BE_3 = 0\).
\end{definition}
\begin{definition}
TM (Transverse Magnetic)

\(\BB_3 = 0\).
\end{definition}
\begin{definition}
TM (Transverse Electromagnetic)

\(\BE_3 = \BB_3 = 0\).
\end{definition}

\paragraph{claim:} TEM do not existing in a hollow waveguide.

Why: I had in my notes
\begin{equation}\label{eqn:relativisticElectrodynamicsT4:680}
\begin{aligned}
\spacegrad \cross \BE = 0 & \implies \PD{x^1}{E_2} -\PD{x^2}{E_1} = 0 \\
\spacegrad \cdot \BE = 0 & \implies \PD{x^1}{E_1} +\PD{x^2}{E_2} = 0
\end{aligned}
\end{equation}

and then

\begin{equation}\label{eqn:relativisticElectrodynamicsT4:700}
\begin{aligned}
\spacegrad^2 \phi &= 0 \\
\phi &= \text{const}
\end{aligned}
\end{equation}

In retrospect I fail to see how these are connected?  What happened to the \(\partial_t \BB\) term in the curl equation above?

It was argued that we have \(\BE_\parallel = \BB_\perp = 0\) on the boundary.

So for the TE case, where \(\BE_3 = 0\), we have from the separation of variables argument

\begin{equation}\label{eqn:relativisticElectrodynamicsT4:440}
\zcap \cdot \BB_0(x, y) =
\sum_{m n} a_{m n} 
\cos\left( \frac{2 \pi i m}{a} x \right)
\cos\left( \frac{2 \pi i n}{b} y \right).
\end{equation}

No sines because 

\begin{equation}\label{eqn:relativisticElectrodynamicsT4:460}
B_1 \sim \PD{x_a}{B_3} \rightarrow \cos(k_1 x^1).
\end{equation}

The quantity

\begin{equation}\label{eqn:relativisticElectrodynamicsT4:480}
a_{m n}
\cos\left( \frac{2 \pi i m}{a} x \right)
\cos\left( \frac{2 \pi i n}{b} y \right).
\end{equation}

is called the \(TE_{m n}\) mode.  Note that since \(B = \text{const}\) an ampere loop requires \(\BB = 0\) since there is no current.

Writing 

\begin{equation}\label{eqn:relativisticElectrodynamicsT4:500}
\begin{aligned}
k &= \frac{\omega}{c} \sqrt{ 1 - \left(\frac{\omega_{m n}}{\omega}\right)^2 } \\
\omega_{m n} &= 2 \pi c \sqrt{ \left(\frac{m}{a} \right)^2 + \left(\frac{n}{b} \right)^2 }
\end{aligned}
\end{equation}

Since \(\omega < \omega_{m n}\) we have \(k\) purely imaginary, and the term

\begin{equation}\label{eqn:relativisticElectrodynamicsT4:520}
e^{-i k z} = e^{- \Abs{k} z}
\end{equation}

represents the die off.

\(\omega_{10}\) is the smallest.

Note that the convention is that the \(m\) in \(TE_{m n}\) is the bigger of the two indices, so \(\omega > \omega_{10}\).

The phase velocity 

\begin{equation}\label{eqn:relativisticElectrodynamicsT4:530}
V_\phi = \frac{\omega}{k} = \frac{c}{\sqrt{ 1 - \left(\frac{\omega_{m n}}{\omega}\right)^2 }} \ge c
\end{equation}

However, energy is transmitted with the group velocity, the ratio of the Poynting vector and energy density

\begin{equation}\label{eqn:relativisticElectrodynamicsT4:540}
\frac{\expectation{\BS}}{\expectation{U}} = V_g = \PD{k}{\omega} = 1/\PD{\omega}{k}
\end{equation}

(This can be shown).

Since

\begin{equation}\label{eqn:relativisticElectrodynamicsT4:560}
\left(\PD{\omega}{k}\right)^{-1} = 
\left(
\PD{\omega}{}
\sqrt{ (\omega/c)^2 - (\omega_{m n}/c)^2 }
\right)^{-1} = c \sqrt{ 1 - (\omega_{m n}/\omega)^2 } \le c
\end{equation}

We see that the energy is transmitted at less than the speed of light as expected.

\section{Final remarks}

I had started converting my handwritten scrawl for this tutorial into an attempt at working through these ideas with enough detail that they self contained, but gave up part way.  This appears to me to be too big of a sub-discipline to give it justice in one hours class.  As is, it is enough to at least get an concept of some of the ideas involved.  I think were I to learn this for real, I had need a good text as a reference (or the time to attempt to blunder through the ideas in much much more detail).

%
% Copyright � 2015 Peeter Joot.  All Rights Reserved.
% Licenced as described in the file LICENSE under the root directory of this GIT repository.
%
\documentclass[]{eliblog}

\usepackage{amsmath}
\usepackage{mathpazo}

%
% shorthand for bold symbols, convenient for vectors and matrices
%
\newcommand{\Ba}[0]{\mathbf{a}}
\newcommand{\Bb}[0]{\mathbf{b}}
\newcommand{\Bc}[0]{\mathbf{c}}
\newcommand{\Bd}[0]{\mathbf{d}}
\newcommand{\Be}[0]{\mathbf{e}}
\newcommand{\Bf}[0]{\mathbf{f}}
\newcommand{\Bg}[0]{\mathbf{g}}
\newcommand{\Bh}[0]{\mathbf{h}}
\newcommand{\Bi}[0]{\mathbf{i}}
\newcommand{\Bj}[0]{\mathbf{j}}
\newcommand{\Bk}[0]{\mathbf{k}}
\newcommand{\Bl}[0]{\mathbf{l}}
\newcommand{\Bm}[0]{\mathbf{m}}
\newcommand{\Bn}[0]{\mathbf{n}}
\newcommand{\Bo}[0]{\mathbf{o}}
\newcommand{\Bp}[0]{\mathbf{p}}
\newcommand{\Bq}[0]{\mathbf{q}}
\newcommand{\Br}[0]{\mathbf{r}}
\newcommand{\Bs}[0]{\mathbf{s}}
\newcommand{\Bt}[0]{\mathbf{t}}
\newcommand{\Bu}[0]{\mathbf{u}}
\newcommand{\Bv}[0]{\mathbf{v}}
\newcommand{\Bw}[0]{\mathbf{w}}
\newcommand{\Bx}[0]{\mathbf{x}}
\newcommand{\By}[0]{\mathbf{y}}
\newcommand{\Bz}[0]{\mathbf{z}}
\newcommand{\BA}[0]{\mathbf{A}}
\newcommand{\BB}[0]{\mathbf{B}}
\newcommand{\BC}[0]{\mathbf{C}}
\newcommand{\BD}[0]{\mathbf{D}}
\newcommand{\BE}[0]{\mathbf{E}}
\newcommand{\BF}[0]{\mathbf{F}}
\newcommand{\BG}[0]{\mathbf{G}}
\newcommand{\BH}[0]{\mathbf{H}}
\newcommand{\BI}[0]{\mathbf{I}}
\newcommand{\BJ}[0]{\mathbf{J}}
\newcommand{\BK}[0]{\mathbf{K}}
\newcommand{\BL}[0]{\mathbf{L}}
\newcommand{\BM}[0]{\mathbf{M}}
\newcommand{\BN}[0]{\mathbf{N}}
\newcommand{\BO}[0]{\mathbf{O}}
\newcommand{\BP}[0]{\mathbf{P}}
\newcommand{\BQ}[0]{\mathbf{Q}}
\newcommand{\BR}[0]{\mathbf{R}}
\newcommand{\BS}[0]{\mathbf{S}}
\newcommand{\BT}[0]{\mathbf{T}}
\newcommand{\BU}[0]{\mathbf{U}}
\newcommand{\BV}[0]{\mathbf{V}}
\newcommand{\BW}[0]{\mathbf{W}}
\newcommand{\BX}[0]{\mathbf{X}}
\newcommand{\BY}[0]{\mathbf{Y}}
\newcommand{\BZ}[0]{\mathbf{Z}}

\newcommand{\Bzero}[0]{\mathbf{0}}
\newcommand{\Btheta}[0]{\boldsymbol{\theta}}
\newcommand{\Btau}[0]{\boldsymbol{\tau}}
\newcommand{\Bomega}[0]{\boldsymbol{\omega}}

%
% shorthand for unit vectors
%
\newcommand{\acap}[0]{\hat{\Ba}}
\newcommand{\bcap}[0]{\hat{\Bb}}
\newcommand{\ccap}[0]{\hat{\Bc}}
\newcommand{\dcap}[0]{\hat{\Bd}}
\newcommand{\ecap}[0]{\hat{\Be}}
\newcommand{\fcap}[0]{\hat{\Bf}}
\newcommand{\gcap}[0]{\hat{\Bg}}
\newcommand{\hcap}[0]{\hat{\Bh}}
\newcommand{\icap}[0]{\hat{\Bi}}
\newcommand{\jcap}[0]{\hat{\Bj}}
\newcommand{\kcap}[0]{\hat{\Bk}}
\newcommand{\lcap}[0]{\hat{\Bl}}
\newcommand{\mcap}[0]{\hat{\Bm}}
\newcommand{\ncap}[0]{\hat{\Bn}}
\newcommand{\ocap}[0]{\hat{\Bo}}
\newcommand{\pcap}[0]{\hat{\Bp}}
\newcommand{\qcap}[0]{\hat{\Bq}}
\newcommand{\rcap}[0]{\hat{\Br}}
\newcommand{\scap}[0]{\hat{\Bs}}
\newcommand{\tcap}[0]{\hat{\Bt}}
\newcommand{\ucap}[0]{\hat{\Bu}}
\newcommand{\vcap}[0]{\hat{\Bv}}
\newcommand{\wcap}[0]{\hat{\Bw}}
\newcommand{\xcap}[0]{\hat{\Bx}}
\newcommand{\ycap}[0]{\hat{\By}}
\newcommand{\zcap}[0]{\hat{\Bz}}
\newcommand{\thetacap}[0]{\hat{\Btheta}}

%
% to write R^n and C^n in a distinguishable fashion.  Perhaps change this
% to the double lined characters upon figuring out how to do so.
%
\newcommand{\C}[1]{$\mathbb{C}^{#1}$}
\newcommand{\R}[1]{$\mathbb{R}^{#1}$}

%
% various generally useful helpers
%

% derivative of #1 wrt. #2:
\newcommand{\D}[2] {\frac {d#2} {d#1}}

\newcommand{\inv}[1]{\frac{1}{#1}}
\newcommand{\cross}[0]{\times}

\newcommand{\abs}[1]{\lvert{#1}\rvert}
\newcommand{\norm}[1]{\lVert{#1}\rVert}
\newcommand{\innerprod}[2]{\langle{#1}, {#2}\rangle}
\newcommand{\dotprod}[2]{{#1} \cdot {#2}}
\newcommand{\bdotprod}[2]{\left({#1} \cdot {#2}\right)}
\newcommand{\crossprod}[2]{{#1} \cross {#2}}
\newcommand{\tripleprod}[3]{\dotprod{\left(\crossprod{#1}{#2}\right)}{#3}}

\DeclareMathOperator{\Proj}{Proj}
\DeclareMathOperator{\Span}{span}
\DeclareMathOperator{\Sgn}{sgn}
\DeclareMathOperator{\Area}{Area}
\DeclareMathOperator{\Volume}{Volume}

%
% A few miscellaneous things specific to this document
%
\newcommand{\crossop}[1]{\crossprod{#1}{}}

% R2 vector.
\newcommand{\VectorTwo}[2]{
\begin{bmatrix}
 {#1} \\
 {#2}
\end{bmatrix}
}

\newcommand{\VectorN}[1]{
\begin{bmatrix}
{#1}_1 \\
{#1}_2 \\
\vdots \\
{#1}_N \\
\end{bmatrix}
}

\newcommand{\DETuvij}[4]{
\begin{vmatrix}
 {#1}_{#3} & {#1}_{#4} \\
 {#2}_{#3} & {#2}_{#4}
\end{vmatrix}
}

\newcommand{\DETuvwijk}[6]{
\begin{vmatrix}
 {#1}_{#4} & {#1}_{#5} & {#1}_{#6} \\
 {#2}_{#4} & {#2}_{#5} & {#2}_{#6} \\
 {#3}_{#4} & {#3}_{#5} & {#3}_{#6}
\end{vmatrix}
}

\newcommand{\DETuvwxijkl}[8]{
\begin{vmatrix}
 {#1}_{#5} & {#1}_{#6} & {#1}_{#7} & {#1}_{#8} \\
 {#2}_{#5} & {#2}_{#6} & {#2}_{#7} & {#2}_{#8} \\
 {#3}_{#5} & {#3}_{#6} & {#3}_{#7} & {#3}_{#8} \\
 {#4}_{#5} & {#4}_{#6} & {#4}_{#7} & {#4}_{#8} \\
\end{vmatrix}
}

%\newcommand{\DETuvwxyijklm}[10]{
%\begin{vmatrix}
% {#1}_{#6} & {#1}_{#7} & {#1}_{#8} & {#1}_{#9} & {#1}_{#10} \\
% {#2}_{#6} & {#2}_{#7} & {#2}_{#8} & {#2}_{#9} & {#2}_{#10} \\
% {#3}_{#6} & {#3}_{#7} & {#3}_{#8} & {#3}_{#9} & {#3}_{#10} \\
% {#4}_{#6} & {#4}_{#7} & {#4}_{#8} & {#4}_{#9} & {#4}_{#10} \\
% {#5}_{#6} & {#5}_{#7} & {#5}_{#8} & {#5}_{#9} & {#5}_{#10}
%\end{vmatrix}
%}

% R3 vector.
\newcommand{\VectorThree}[3]{
\begin{bmatrix}
 {#1} \\
 {#2} \\
 {#3}
\end{bmatrix}
}



\author{Peeter Joot}
\email{peeter.joot@gmail.com}

%\documentclass[]{eliblogwidescreen}

\usepackage{amsmath}
\usepackage{mathpazo}

%
% shorthand for bold symbols, convenient for vectors and matrices
%
\newcommand{\Ba}[0]{\mathbf{a}}
\newcommand{\Bb}[0]{\mathbf{b}}
\newcommand{\Bc}[0]{\mathbf{c}}
\newcommand{\Bd}[0]{\mathbf{d}}
\newcommand{\Be}[0]{\mathbf{e}}
\newcommand{\Bf}[0]{\mathbf{f}}
\newcommand{\Bg}[0]{\mathbf{g}}
\newcommand{\Bh}[0]{\mathbf{h}}
\newcommand{\Bi}[0]{\mathbf{i}}
\newcommand{\Bj}[0]{\mathbf{j}}
\newcommand{\Bk}[0]{\mathbf{k}}
\newcommand{\Bl}[0]{\mathbf{l}}
\newcommand{\Bm}[0]{\mathbf{m}}
\newcommand{\Bn}[0]{\mathbf{n}}
\newcommand{\Bo}[0]{\mathbf{o}}
\newcommand{\Bp}[0]{\mathbf{p}}
\newcommand{\Bq}[0]{\mathbf{q}}
\newcommand{\Br}[0]{\mathbf{r}}
\newcommand{\Bs}[0]{\mathbf{s}}
\newcommand{\Bt}[0]{\mathbf{t}}
\newcommand{\Bu}[0]{\mathbf{u}}
\newcommand{\Bv}[0]{\mathbf{v}}
\newcommand{\Bw}[0]{\mathbf{w}}
\newcommand{\Bx}[0]{\mathbf{x}}
\newcommand{\By}[0]{\mathbf{y}}
\newcommand{\Bz}[0]{\mathbf{z}}
\newcommand{\BA}[0]{\mathbf{A}}
\newcommand{\BB}[0]{\mathbf{B}}
\newcommand{\BC}[0]{\mathbf{C}}
\newcommand{\BD}[0]{\mathbf{D}}
\newcommand{\BE}[0]{\mathbf{E}}
\newcommand{\BF}[0]{\mathbf{F}}
\newcommand{\BG}[0]{\mathbf{G}}
\newcommand{\BH}[0]{\mathbf{H}}
\newcommand{\BI}[0]{\mathbf{I}}
\newcommand{\BJ}[0]{\mathbf{J}}
\newcommand{\BK}[0]{\mathbf{K}}
\newcommand{\BL}[0]{\mathbf{L}}
\newcommand{\BM}[0]{\mathbf{M}}
\newcommand{\BN}[0]{\mathbf{N}}
\newcommand{\BO}[0]{\mathbf{O}}
\newcommand{\BP}[0]{\mathbf{P}}
\newcommand{\BQ}[0]{\mathbf{Q}}
\newcommand{\BR}[0]{\mathbf{R}}
\newcommand{\BS}[0]{\mathbf{S}}
\newcommand{\BT}[0]{\mathbf{T}}
\newcommand{\BU}[0]{\mathbf{U}}
\newcommand{\BV}[0]{\mathbf{V}}
\newcommand{\BW}[0]{\mathbf{W}}
\newcommand{\BX}[0]{\mathbf{X}}
\newcommand{\BY}[0]{\mathbf{Y}}
\newcommand{\BZ}[0]{\mathbf{Z}}

\newcommand{\Bzero}[0]{\mathbf{0}}
\newcommand{\Btheta}[0]{\boldsymbol{\theta}}
\newcommand{\Btau}[0]{\boldsymbol{\tau}}
\newcommand{\Bomega}[0]{\boldsymbol{\omega}}

%
% shorthand for unit vectors
%
\newcommand{\acap}[0]{\hat{\Ba}}
\newcommand{\bcap}[0]{\hat{\Bb}}
\newcommand{\ccap}[0]{\hat{\Bc}}
\newcommand{\dcap}[0]{\hat{\Bd}}
\newcommand{\ecap}[0]{\hat{\Be}}
\newcommand{\fcap}[0]{\hat{\Bf}}
\newcommand{\gcap}[0]{\hat{\Bg}}
\newcommand{\hcap}[0]{\hat{\Bh}}
\newcommand{\icap}[0]{\hat{\Bi}}
\newcommand{\jcap}[0]{\hat{\Bj}}
\newcommand{\kcap}[0]{\hat{\Bk}}
\newcommand{\lcap}[0]{\hat{\Bl}}
\newcommand{\mcap}[0]{\hat{\Bm}}
\newcommand{\ncap}[0]{\hat{\Bn}}
\newcommand{\ocap}[0]{\hat{\Bo}}
\newcommand{\pcap}[0]{\hat{\Bp}}
\newcommand{\qcap}[0]{\hat{\Bq}}
\newcommand{\rcap}[0]{\hat{\Br}}
\newcommand{\scap}[0]{\hat{\Bs}}
\newcommand{\tcap}[0]{\hat{\Bt}}
\newcommand{\ucap}[0]{\hat{\Bu}}
\newcommand{\vcap}[0]{\hat{\Bv}}
\newcommand{\wcap}[0]{\hat{\Bw}}
\newcommand{\xcap}[0]{\hat{\Bx}}
\newcommand{\ycap}[0]{\hat{\By}}
\newcommand{\zcap}[0]{\hat{\Bz}}
\newcommand{\thetacap}[0]{\hat{\Btheta}}

%
% to write R^n and C^n in a distinguishable fashion.  Perhaps change this
% to the double lined characters upon figuring out how to do so.
%
\newcommand{\C}[1]{$\mathbb{C}^{#1}$}
\newcommand{\R}[1]{$\mathbb{R}^{#1}$}

%
% various generally useful helpers
%

% derivative of #1 wrt. #2:
\newcommand{\D}[2] {\frac {d#2} {d#1}}

\newcommand{\inv}[1]{\frac{1}{#1}}
\newcommand{\cross}[0]{\times}

\newcommand{\abs}[1]{\lvert{#1}\rvert}
\newcommand{\norm}[1]{\lVert{#1}\rVert}
\newcommand{\innerprod}[2]{\langle{#1}, {#2}\rangle}
\newcommand{\dotprod}[2]{{#1} \cdot {#2}}
\newcommand{\bdotprod}[2]{\left({#1} \cdot {#2}\right)}
\newcommand{\crossprod}[2]{{#1} \cross {#2}}
\newcommand{\tripleprod}[3]{\dotprod{\left(\crossprod{#1}{#2}\right)}{#3}}

\DeclareMathOperator{\Proj}{Proj}
\DeclareMathOperator{\Span}{span}
\DeclareMathOperator{\Sgn}{sgn}
\DeclareMathOperator{\Area}{Area}
\DeclareMathOperator{\Volume}{Volume}

%
% A few miscellaneous things specific to this document
%
\newcommand{\crossop}[1]{\crossprod{#1}{}}

% R2 vector.
\newcommand{\VectorTwo}[2]{
\begin{bmatrix}
 {#1} \\
 {#2}
\end{bmatrix}
}

\newcommand{\VectorN}[1]{
\begin{bmatrix}
{#1}_1 \\
{#1}_2 \\
\vdots \\
{#1}_N \\
\end{bmatrix}
}

\newcommand{\DETuvij}[4]{
\begin{vmatrix}
 {#1}_{#3} & {#1}_{#4} \\
 {#2}_{#3} & {#2}_{#4}
\end{vmatrix}
}

\newcommand{\DETuvwijk}[6]{
\begin{vmatrix}
 {#1}_{#4} & {#1}_{#5} & {#1}_{#6} \\
 {#2}_{#4} & {#2}_{#5} & {#2}_{#6} \\
 {#3}_{#4} & {#3}_{#5} & {#3}_{#6}
\end{vmatrix}
}

\newcommand{\DETuvwxijkl}[8]{
\begin{vmatrix}
 {#1}_{#5} & {#1}_{#6} & {#1}_{#7} & {#1}_{#8} \\
 {#2}_{#5} & {#2}_{#6} & {#2}_{#7} & {#2}_{#8} \\
 {#3}_{#5} & {#3}_{#6} & {#3}_{#7} & {#3}_{#8} \\
 {#4}_{#5} & {#4}_{#6} & {#4}_{#7} & {#4}_{#8} \\
\end{vmatrix}
}

%\newcommand{\DETuvwxyijklm}[10]{
%\begin{vmatrix}
% {#1}_{#6} & {#1}_{#7} & {#1}_{#8} & {#1}_{#9} & {#1}_{#10} \\
% {#2}_{#6} & {#2}_{#7} & {#2}_{#8} & {#2}_{#9} & {#2}_{#10} \\
% {#3}_{#6} & {#3}_{#7} & {#3}_{#8} & {#3}_{#9} & {#3}_{#10} \\
% {#4}_{#6} & {#4}_{#7} & {#4}_{#8} & {#4}_{#9} & {#4}_{#10} \\
% {#5}_{#6} & {#5}_{#7} & {#5}_{#8} & {#5}_{#9} & {#5}_{#10}
%\end{vmatrix}
%}

% R3 vector.
\newcommand{\VectorThree}[3]{
\begin{bmatrix}
 {#1} \\
 {#2} \\
 {#3}
\end{bmatrix}
}



\author{Peeter Joot}
\email{peeter.joot@gmail.com}


\chapter{PHY450H1S.  Relativistic Electrodynamics Lecture 16 (Taught by Prof. Erich Poppitz).  Monochromatic EM fields.  Poynting vector and energy density conservation.}
\label{chap:relativisticElectrodynamicsL16}
%\useCCL
\blogpage{http://sites.google.com/site/peeterjoot/math2011/relativisticElectrodynamicsL16.pdf}
\date{Mar 2, 2011}
\revisionInfo{relativisticElectrodynamicsL16.tex}

%\beginArtWithToc
\beginArtNoToc

\section{Reading.}

Covering chapter 6 material from the text \cite{landau1980classical}.

Covering \href{http://www.physics.utoronto.ca/~poppitz/e-poppitz/PHY450_files/RelEMpp115-127.pdf}{lecture notes pp. 115-127}: properties of monochromatic plane EM waves (122-124); energy and energy flux of the EM field and energy conservation from the equations of motion (125-127)  [Wednesday, Mar. 2]

\section{Review.  Solution to the wave equation.}

Recall that in the Coulomb gauge

\begin{align}\label{eqn:relativisticElectrodynamicsL16:10}
A^0 &= 0 \\
\spacegrad \cdot \BA &= 0
\end{align}

our equation to solve is

\begin{equation}\label{eqn:relativisticElectrodynamicsL16:30}
\left( \inv{c^2} \PDSq{t}{} - \Delta \right) \BA = 0.
\end{equation}

We found that the general solution was

\begin{equation}\label{eqn:relativisticElectrodynamicsL16:50}
\BA(\Bx, t) = \int \frac{d^3\Bk}{(2 \pi)^3} \left(
e^{i (\Bk \cdot \Bx + \omega_k t)} \Bbeta^\conj(-\Bk)
+e^{i (\Bk \cdot \Bx - \omega_k t)} \Bbeta(\Bk)
\right)
\end{equation}

where

\begin{equation}\label{eqn:relativisticElectrodynamicsL16:70}
\Bk \cdot \Bbeta(\Bk) = 0
\end{equation}

It is clear that this is a solution since

\begin{equation}\label{eqn:relativisticElectrodynamicsL16:90}
\left( \inv{c^2} \PDSq{t}{} - \Delta \right) e^{i (\Bk \cdot \Bx \pm \omega_k t)} = 0
\end{equation}

\section{Moving to physically relevant results.}

Since the most general solution is a sum over $\Bk$, it is enough to consider only a single $\Bk$, or equivalently, take

\begin{align}\label{eqn:relativisticElectrodynamicsL16:110}
\Bbeta(\Bk) &= \Bbeta ( 2\pi)^3 \delta^3(\Bk - \Bp) \\
\Bbeta^\conj(-\Bk) &= \Bbeta^\conj ( 2\pi)^3 \delta^3(-\Bk - \Bp)
\end{align}

but we have the freedom to pick a real and constant $\Bbeta$.  Now our solution is

\begin{equation}\label{eqn:relativisticElectrodynamicsL16:130}
\BA(\Bx, t) = \Bbeta \left(
e^{-i (\Bp \cdot \Bx + \omega_k t)}
+e^{i (\Bp \cdot \Bx - \omega_k t)}
\right)
= \Bbeta \cos( \omega t - \Bp \cdot \Bx)
\end{equation}

where

\begin{equation}\label{eqn:relativisticElectrodynamicsL16:150}
\Bbeta \cdot \Bp = 0
\end{equation}

FIXME:DIY: show that also using $\Bbeta$ complex also works.

Let's choose

\begin{equation}\label{eqn:relativisticElectrodynamicsL16:170}
\Bp = (p, 0, 0)
\end{equation}

Since
\begin{equation}\label{eqn:relativisticElectrodynamicsL16:190}
\Bp \cdot \Bbeta = p_x \beta_x
\end{equation}

we must have

\begin{equation}\label{eqn:relativisticElectrodynamicsL16:210}
\Bbeta_x = 0
\end{equation}

so

\begin{equation}\label{eqn:relativisticElectrodynamicsL16:230}
\Bbeta = (0, \beta_y, \beta_z)
\end{equation}

\paragraph{Claim:} The Coulomb gauge $0 = \spacegrad \cdot \BA = (\Bbeta \cdot \Bp)\sin(\omega t - \Bp \cdot \Bx)$ implies that there are two linearly independent choices of $\Bbeta$ and $\Bp$.

FIXME: missing exactly how this is?

PICTURE:

$\Bbeta_1$, $\Bbeta_2$, $\Bp$ all mutually perpendicular.

\begin{align*}
\BE
&= -\PD{ct}{\BA}  \\
&= -\frac{\Bbeta}{c} \PD{t}{} \cos(\omega t - \Bp \cdot \Bx) \\
&= -\inv{c} \Bbeta \omega_p
\sin(\omega t - \Bp \cdot \Bx)
\end{align*}

(recall: $\omega_p = c\Abs{\Bp}$)

\begin{equation}\label{eqn:relativisticElectrodynamicsL16:250}
\boxed{
\BE = \Bbeta \Abs{\Bp} \sin(\omega t - \Bp \cdot \Bx)
}
\end{equation}

\begin{align*}
\BB
&= \spacegrad \cross \BA \\
&= \spacegrad \cross ( \Bbeta \cos(\omega t - \Bp \cdot \Bx) \\
&= (\spacegrad \cos(\omega t - \Bp \cdot \Bx)) \cross \Bbeta \\
&= \sin(\omega t - \Bp \cdot \Bx) \Bp \cross \Bbeta
\end{align*}

\begin{equation}\label{eqn:relativisticElectrodynamicsL16:270}
\boxed{
\BB = (\Bp \cross \Bbeta) \sin(\omega t - \Bp \cdot \Bx)
}
\end{equation}

\paragraph{Example:} $\Bp \parallel \Be_x$, $\BB \parallel \Be_y$ or $\Be_z$

(since we have two linearly independent choices)

\paragraph{Example:} take $\Bbeta \parallel \Be_y$

\begin{align}\label{eqn:relativisticElectrodynamicsL16:290}
\BE &= \Bbeta p \sin(c p t - p x)  \\
\BB &= (\Bp \cross \Bbeta) \sin(c p t - p x)
\end{align}

At $t = 0$

\begin{align}\label{eqn:relativisticElectrodynamicsL16:310}
\BE &= -\Bbeta p \sin( p x)  \\
%\BB &= -(\Bp \cross \Bbeta) \sin(p x)
B_z &= - \Abs{\Bbeta} \Be_z c p \sin(p x)
\end{align}

PICTURE: two oscillating mutually perpendicular sinusoids.

So physically, we see that $\Bp$ is the direction of propagation.  We have always

\begin{equation}\label{eqn:relativisticElectrodynamicsL16:330}
\Bp \perp \BE
\end{equation}

and we have two possible polarizations.

Convention is usually to take the direction of oscillation of $\BE$ the polarization of the wave.

This is the starting point for the field of optics, because the polarization of the incident wave, is strongly tied to how much of the wave will reflect off of a surface with a given index of refraction $n$.

\section{EM waves carrying energy and momentum}

Maxwell field in vacuum is the sum of plane monochromatic waves, two per wave vector.

PICTURE:

\begin{align*}\label{eqn:relativisticElectrodynamicsL16:350}
\BE &\parallel \Be_3 \\
\BB &\parallel \Be_1 \\
\Bk &\parallel \Be_2
\end{align*}

PICTURE:

\begin{align*}
\BB &\parallel -\Be_3 \\
\BE &\parallel \Be_1 \\
\Bk &\parallel \Be_2
\end{align*}

(two linearly independent polarizations)

Our wave frequency is

\begin{equation}\label{eqn:relativisticElectrodynamicsL16:370}
\omega_{\Bk} = c \Abs{\Bk}
\end{equation}

The wavelength, the value such that $x \rightarrow x + \frac{2 \pi}{k}$

FIXME:DIY: see:
\begin{equation}\label{eqn:relativisticElectrodynamicsL16:390}
\sin(k c t - k x)
\end{equation}

\begin{equation}\label{eqn:relativisticElectrodynamicsL16:410}
\lambda_{\Bk} = \frac{2 \pi}{k}
\end{equation}

period

\begin{equation}\label{eqn:relativisticElectrodynamicsL16:430}
T = \frac{ 2 \pi} {k c} = \frac{\lambda_\Bk}{c}
\end{equation}

\section{Energy and momentum of EM waves.}

\subsection{Classical mechanics motivation.}

To motivate our approach, let's recall one route from our equations of motion in classical mechanics, to the energy conservation relation.  Our EOM in one dimension is

\begin{equation}\label{eqn:relativisticElectrodynamicsL16:450}
m \frac{d}{dt} \dot{x} = - \mathcal{U}'(x).
\end{equation}

We can multiply both sides by what we take the time derivative of

\begin{equation}\label{eqn:relativisticElectrodynamicsL16:470}
m \dot{x} \ddt{\dot{x}} = - \dot{x} \mathcal{U}'(x),
\end{equation}

and then manipulate it a bit so that we have time derivatives on both sides

\begin{equation}\label{eqn:relativisticElectrodynamicsL16:490}
\ddt{} \frac{m \dot{x}^2}{2} = - \ddt{ \mathcal{U}(x) }.
\end{equation}

Taking differences, we have

\begin{equation}\label{eqn:relativisticElectrodynamicsL16:510}
\ddt{} \left( \frac{m \xdot^2}{2} + \mathcal{U}(x) \right) = 0,
\end{equation}

which allows us to find a conservation relationship that we label energy conservation ($\mathcal{E} = K + \mathcal{U}$).

\subsection{Doing the same thing for Maxwell's equations.}

Poppitz claims we have very little tricks in physics, and we really just do the same thing for our EM case.  Our equations are a bit messier to start with, and for the vacuum, our non-divergence equations are

%label{eqn:relativisticElectrodynamicsL13:410}
\begin{align}\label{eqn:relativisticElectrodynamicsL16:530}
\spacegrad \cross \BB -\inv{c} \PD{t}{\BE} &= \frac{4 \pi}{c} \Bj \\
\spacegrad \cross \BE +\inv{c} \PD{t}{\BB} &= 0
\end{align}

We can dot these with $\BE$ and $\BB$ respectively, repeating the trick of ``multiplying'' by what we take the time derivative of

\begin{align}\label{eqn:relativisticElectrodynamicsL16:550}
\BE \cdot (\spacegrad \cross \BB) -\inv{c} \BE \cdot \PD{t}{\BE} &= \frac{4 \pi}{c} \BE \cdot \Bj \\
\BB \cdot (\spacegrad \cross \BE) +\inv{c} \BB \cdot \PD{t}{\BB} &= 0,
\end{align}

and then take differences

\begin{equation}\label{eqn:relativisticElectrodynamicsL16:570}
\inv{c} \left(
\BB \cdot \PD{t}{\BB}
+ \BE \cdot \PD{t}{\BE} \right) + \BB \cdot (\spacegrad \cross \BE) -\BE \cdot (\spacegrad \cross \BB) =
-\frac{4 \pi}{c} \BE \cdot \Bj.
\end{equation}

\paragraph{Claim:}

\begin{equation}\label{eqn:relativisticElectrodynamicsL16:590}
-\BB \cdot (\spacegrad \cross \BE) +\BE \cdot (\spacegrad \cross \BB) = \spacegrad \cdot ( \BB \cross \BE ).
\end{equation}

This is almost trivial with an expansion of the RHS in tensor notation

\begin{align*}
\spacegrad \cdot ( \BB \cross \BE )
&=
\partial_\alpha e^{\alpha \beta \sigma} B^\beta E^\sigma \\
&=
e^{\alpha \beta \sigma} (\partial_\alpha B^\beta) E^\sigma
+
e^{\alpha \beta \sigma} B^\beta (\partial_\alpha E^\sigma) \\
&=
\BE \cdot (\spacegrad \cross \BB)
-\BB \cdot (\spacegrad \cross \BE)\qquad \square
\end{align*}

Regrouping we have

\begin{equation}\label{eqn:relativisticElectrodynamicsL16:610}
\inv{2 c} \PD{t}{} \left(
\BB^2 + \BE^2 \right) - \spacegrad \cdot ( \BB \cross \BE )
=
-\frac{4 \pi}{c} \BE \cdot \Bj.
\end{equation}

A final rescaling makes the units natural

\begin{equation}\label{eqn:relativisticElectrodynamicsL16:630}
\PD{t}{} \frac{ \BE^2 + \BB^2 }{8 \pi} - \spacegrad \cdot \left( \frac{c}{4 \pi} \BB \cross \BE \right) = - \BE \cdot \Bj.
\end{equation}

We define the cross product term as the Poynting vector

\begin{align}\label{eqn:relativisticElectrodynamicsL16:650}
\BS &= \frac{c}{4 \pi} \BB \cross \BE.
\end{align}

Suppose we integrate over a spatial volume.  This gives us

\begin{equation}\label{eqn:relativisticElectrodynamicsL16:670b}
\PD{t}{}\int_V d^3 \Bx \frac{ \BE^2 + \BB^2 }{8 \pi} - \int_V d^3 \Bx \spacegrad \cdot \BS = - \int_V d^3 \Bx \BE \cdot \Bj.
\end{equation}

Our Poynting integral can be converted to a surface integral utilizing Stokes theorem

\begin{equation}\label{eqn:relativisticElectrodynamicsL16:800}
\int_V d^3 \Bx \spacegrad \cdot \BS = \int_{\partial V} d^2 \sigma \Bn \cdot \BS =
\int_{\partial V} d^2 \Bsigma \cdot \BS
\end{equation}

We make the interpretations

\begin{align*}
\int_V d^3 \Bx \frac{ \BE^2 + \BB^2 }{8 \pi} &= \mbox{energy} \\
\int_V d^3 \Bx \spacegrad \cdot \BS &= \mbox{momentum change through surface per unit time} \\
- \int_V d^3 \Bx \BE \cdot \Bj &= \mbox{work done}
\end{align*}

\paragraph{Justifying the sign, and clarifying work done by what, above.}

Recall that the energy term of the Lorentz force equation was

\begin{equation}\label{eqn:relativisticElectrodynamicsL16:820}
\ddt{\mathcal{E}_{\text{kinetic}}} = e \BE \cdot \Bv
\end{equation}

and

\begin{equation}\label{eqn:relativisticElectrodynamicsL16:840}
\Bj = e \rho \Bv
\end{equation}

so
\begin{equation}\label{eqn:relativisticElectrodynamicsL16:860}
\int_V d^3 \Bx \BE \cdot \Bj
\end{equation}

represents the rate of change of kinetic energy of the charged particles as they move through through a field.  If this is positive, then the charge distribution has gained energy.  The negation of this quantity would represent energy transfer to the field from the charge distribution, the work done \underline{on the field} by the charge distribution.

\subsection{Aside: As a four vector relationship.}

In tutorial today (after this lecture, but before typing up these lecture notes in full), we used $\mathcal{U}$ for the energy density term above

\begin{equation}\label{eqn:relativisticElectrodynamicsL16:650b}
\mathcal{U} = \frac{ \BE^2 + \BB^2 }{8 \pi} .
\end{equation}

This allows us to group the quantities in our conservation relationship above nicely

\begin{equation}\label{eqn:relativisticElectrodynamicsL16:670}
\PD{t}{\mathcal{U}} - \spacegrad \cdot \BS = - \BE \cdot \Bj.
\end{equation}

It appears natural to write \ref{eqn:relativisticElectrodynamicsL16:670} in the form of a four divergence.  Suppose we define

\begin{equation}\label{eqn:relativisticElectrodynamicsL16:710}
P^i = (\mathcal{U}, \BS/c^2)
\end{equation}

then we have

\begin{equation}\label{eqn:relativisticElectrodynamicsL16:730}
\partial_i P^i = - c \BE \cdot \Bj.
\end{equation}

Since the LHS has the appearance of a four scalar, this seems to imply that $\BE \cdot \Bj$ is a Lorentz invariant.  It is curious that we have only the four scalar that comes from the energy term of the Lorentz force on the RHS of the conservation relationship.  Peeking ahead at the text, this appears to be why a rank two energy tensor $T^{ij}$ is introduced.  For a relativistically natural quantity, we ought to have a conservation relationship also associated with each of the momentum change components of the four vector Lorentz force equation too.

\EndArticle

%
% Copyright � 2015 Peeter Joot.  All Rights Reserved.
% Licenced as described in the file LICENSE under the root directory of this GIT repository.
%
\documentclass[]{eliblog}

\usepackage{amsmath}
\usepackage{mathpazo}

%
% shorthand for bold symbols, convenient for vectors and matrices
%
\newcommand{\Ba}[0]{\mathbf{a}}
\newcommand{\Bb}[0]{\mathbf{b}}
\newcommand{\Bc}[0]{\mathbf{c}}
\newcommand{\Bd}[0]{\mathbf{d}}
\newcommand{\Be}[0]{\mathbf{e}}
\newcommand{\Bf}[0]{\mathbf{f}}
\newcommand{\Bg}[0]{\mathbf{g}}
\newcommand{\Bh}[0]{\mathbf{h}}
\newcommand{\Bi}[0]{\mathbf{i}}
\newcommand{\Bj}[0]{\mathbf{j}}
\newcommand{\Bk}[0]{\mathbf{k}}
\newcommand{\Bl}[0]{\mathbf{l}}
\newcommand{\Bm}[0]{\mathbf{m}}
\newcommand{\Bn}[0]{\mathbf{n}}
\newcommand{\Bo}[0]{\mathbf{o}}
\newcommand{\Bp}[0]{\mathbf{p}}
\newcommand{\Bq}[0]{\mathbf{q}}
\newcommand{\Br}[0]{\mathbf{r}}
\newcommand{\Bs}[0]{\mathbf{s}}
\newcommand{\Bt}[0]{\mathbf{t}}
\newcommand{\Bu}[0]{\mathbf{u}}
\newcommand{\Bv}[0]{\mathbf{v}}
\newcommand{\Bw}[0]{\mathbf{w}}
\newcommand{\Bx}[0]{\mathbf{x}}
\newcommand{\By}[0]{\mathbf{y}}
\newcommand{\Bz}[0]{\mathbf{z}}
\newcommand{\BA}[0]{\mathbf{A}}
\newcommand{\BB}[0]{\mathbf{B}}
\newcommand{\BC}[0]{\mathbf{C}}
\newcommand{\BD}[0]{\mathbf{D}}
\newcommand{\BE}[0]{\mathbf{E}}
\newcommand{\BF}[0]{\mathbf{F}}
\newcommand{\BG}[0]{\mathbf{G}}
\newcommand{\BH}[0]{\mathbf{H}}
\newcommand{\BI}[0]{\mathbf{I}}
\newcommand{\BJ}[0]{\mathbf{J}}
\newcommand{\BK}[0]{\mathbf{K}}
\newcommand{\BL}[0]{\mathbf{L}}
\newcommand{\BM}[0]{\mathbf{M}}
\newcommand{\BN}[0]{\mathbf{N}}
\newcommand{\BO}[0]{\mathbf{O}}
\newcommand{\BP}[0]{\mathbf{P}}
\newcommand{\BQ}[0]{\mathbf{Q}}
\newcommand{\BR}[0]{\mathbf{R}}
\newcommand{\BS}[0]{\mathbf{S}}
\newcommand{\BT}[0]{\mathbf{T}}
\newcommand{\BU}[0]{\mathbf{U}}
\newcommand{\BV}[0]{\mathbf{V}}
\newcommand{\BW}[0]{\mathbf{W}}
\newcommand{\BX}[0]{\mathbf{X}}
\newcommand{\BY}[0]{\mathbf{Y}}
\newcommand{\BZ}[0]{\mathbf{Z}}

\newcommand{\Bzero}[0]{\mathbf{0}}
\newcommand{\Btheta}[0]{\boldsymbol{\theta}}
\newcommand{\Btau}[0]{\boldsymbol{\tau}}
\newcommand{\Bomega}[0]{\boldsymbol{\omega}}

%
% shorthand for unit vectors
%
\newcommand{\acap}[0]{\hat{\Ba}}
\newcommand{\bcap}[0]{\hat{\Bb}}
\newcommand{\ccap}[0]{\hat{\Bc}}
\newcommand{\dcap}[0]{\hat{\Bd}}
\newcommand{\ecap}[0]{\hat{\Be}}
\newcommand{\fcap}[0]{\hat{\Bf}}
\newcommand{\gcap}[0]{\hat{\Bg}}
\newcommand{\hcap}[0]{\hat{\Bh}}
\newcommand{\icap}[0]{\hat{\Bi}}
\newcommand{\jcap}[0]{\hat{\Bj}}
\newcommand{\kcap}[0]{\hat{\Bk}}
\newcommand{\lcap}[0]{\hat{\Bl}}
\newcommand{\mcap}[0]{\hat{\Bm}}
\newcommand{\ncap}[0]{\hat{\Bn}}
\newcommand{\ocap}[0]{\hat{\Bo}}
\newcommand{\pcap}[0]{\hat{\Bp}}
\newcommand{\qcap}[0]{\hat{\Bq}}
\newcommand{\rcap}[0]{\hat{\Br}}
\newcommand{\scap}[0]{\hat{\Bs}}
\newcommand{\tcap}[0]{\hat{\Bt}}
\newcommand{\ucap}[0]{\hat{\Bu}}
\newcommand{\vcap}[0]{\hat{\Bv}}
\newcommand{\wcap}[0]{\hat{\Bw}}
\newcommand{\xcap}[0]{\hat{\Bx}}
\newcommand{\ycap}[0]{\hat{\By}}
\newcommand{\zcap}[0]{\hat{\Bz}}
\newcommand{\thetacap}[0]{\hat{\Btheta}}

%
% to write R^n and C^n in a distinguishable fashion.  Perhaps change this
% to the double lined characters upon figuring out how to do so.
%
\newcommand{\C}[1]{$\mathbb{C}^{#1}$}
\newcommand{\R}[1]{$\mathbb{R}^{#1}$}

%
% various generally useful helpers
%

% derivative of #1 wrt. #2:
\newcommand{\D}[2] {\frac {d#2} {d#1}}

\newcommand{\inv}[1]{\frac{1}{#1}}
\newcommand{\cross}[0]{\times}

\newcommand{\abs}[1]{\lvert{#1}\rvert}
\newcommand{\norm}[1]{\lVert{#1}\rVert}
\newcommand{\innerprod}[2]{\langle{#1}, {#2}\rangle}
\newcommand{\dotprod}[2]{{#1} \cdot {#2}}
\newcommand{\bdotprod}[2]{\left({#1} \cdot {#2}\right)}
\newcommand{\crossprod}[2]{{#1} \cross {#2}}
\newcommand{\tripleprod}[3]{\dotprod{\left(\crossprod{#1}{#2}\right)}{#3}}

\DeclareMathOperator{\Proj}{Proj}
\DeclareMathOperator{\Span}{span}
\DeclareMathOperator{\Sgn}{sgn}
\DeclareMathOperator{\Area}{Area}
\DeclareMathOperator{\Volume}{Volume}

%
% A few miscellaneous things specific to this document
%
\newcommand{\crossop}[1]{\crossprod{#1}{}}

% R2 vector.
\newcommand{\VectorTwo}[2]{
\begin{bmatrix}
 {#1} \\
 {#2}
\end{bmatrix}
}

\newcommand{\VectorN}[1]{
\begin{bmatrix}
{#1}_1 \\
{#1}_2 \\
\vdots \\
{#1}_N \\
\end{bmatrix}
}

\newcommand{\DETuvij}[4]{
\begin{vmatrix}
 {#1}_{#3} & {#1}_{#4} \\
 {#2}_{#3} & {#2}_{#4}
\end{vmatrix}
}

\newcommand{\DETuvwijk}[6]{
\begin{vmatrix}
 {#1}_{#4} & {#1}_{#5} & {#1}_{#6} \\
 {#2}_{#4} & {#2}_{#5} & {#2}_{#6} \\
 {#3}_{#4} & {#3}_{#5} & {#3}_{#6}
\end{vmatrix}
}

\newcommand{\DETuvwxijkl}[8]{
\begin{vmatrix}
 {#1}_{#5} & {#1}_{#6} & {#1}_{#7} & {#1}_{#8} \\
 {#2}_{#5} & {#2}_{#6} & {#2}_{#7} & {#2}_{#8} \\
 {#3}_{#5} & {#3}_{#6} & {#3}_{#7} & {#3}_{#8} \\
 {#4}_{#5} & {#4}_{#6} & {#4}_{#7} & {#4}_{#8} \\
\end{vmatrix}
}

%\newcommand{\DETuvwxyijklm}[10]{
%\begin{vmatrix}
% {#1}_{#6} & {#1}_{#7} & {#1}_{#8} & {#1}_{#9} & {#1}_{#10} \\
% {#2}_{#6} & {#2}_{#7} & {#2}_{#8} & {#2}_{#9} & {#2}_{#10} \\
% {#3}_{#6} & {#3}_{#7} & {#3}_{#8} & {#3}_{#9} & {#3}_{#10} \\
% {#4}_{#6} & {#4}_{#7} & {#4}_{#8} & {#4}_{#9} & {#4}_{#10} \\
% {#5}_{#6} & {#5}_{#7} & {#5}_{#8} & {#5}_{#9} & {#5}_{#10}
%\end{vmatrix}
%}

% R3 vector.
\newcommand{\VectorThree}[3]{
\begin{bmatrix}
 {#1} \\
 {#2} \\
 {#3}
\end{bmatrix}
}



\author{Peeter Joot}
\email{peeter.joot@gmail.com}

%\documentclass[]{eliblogwidescreen}

\usepackage{amsmath}
\usepackage{mathpazo}

%
% shorthand for bold symbols, convenient for vectors and matrices
%
\newcommand{\Ba}[0]{\mathbf{a}}
\newcommand{\Bb}[0]{\mathbf{b}}
\newcommand{\Bc}[0]{\mathbf{c}}
\newcommand{\Bd}[0]{\mathbf{d}}
\newcommand{\Be}[0]{\mathbf{e}}
\newcommand{\Bf}[0]{\mathbf{f}}
\newcommand{\Bg}[0]{\mathbf{g}}
\newcommand{\Bh}[0]{\mathbf{h}}
\newcommand{\Bi}[0]{\mathbf{i}}
\newcommand{\Bj}[0]{\mathbf{j}}
\newcommand{\Bk}[0]{\mathbf{k}}
\newcommand{\Bl}[0]{\mathbf{l}}
\newcommand{\Bm}[0]{\mathbf{m}}
\newcommand{\Bn}[0]{\mathbf{n}}
\newcommand{\Bo}[0]{\mathbf{o}}
\newcommand{\Bp}[0]{\mathbf{p}}
\newcommand{\Bq}[0]{\mathbf{q}}
\newcommand{\Br}[0]{\mathbf{r}}
\newcommand{\Bs}[0]{\mathbf{s}}
\newcommand{\Bt}[0]{\mathbf{t}}
\newcommand{\Bu}[0]{\mathbf{u}}
\newcommand{\Bv}[0]{\mathbf{v}}
\newcommand{\Bw}[0]{\mathbf{w}}
\newcommand{\Bx}[0]{\mathbf{x}}
\newcommand{\By}[0]{\mathbf{y}}
\newcommand{\Bz}[0]{\mathbf{z}}
\newcommand{\BA}[0]{\mathbf{A}}
\newcommand{\BB}[0]{\mathbf{B}}
\newcommand{\BC}[0]{\mathbf{C}}
\newcommand{\BD}[0]{\mathbf{D}}
\newcommand{\BE}[0]{\mathbf{E}}
\newcommand{\BF}[0]{\mathbf{F}}
\newcommand{\BG}[0]{\mathbf{G}}
\newcommand{\BH}[0]{\mathbf{H}}
\newcommand{\BI}[0]{\mathbf{I}}
\newcommand{\BJ}[0]{\mathbf{J}}
\newcommand{\BK}[0]{\mathbf{K}}
\newcommand{\BL}[0]{\mathbf{L}}
\newcommand{\BM}[0]{\mathbf{M}}
\newcommand{\BN}[0]{\mathbf{N}}
\newcommand{\BO}[0]{\mathbf{O}}
\newcommand{\BP}[0]{\mathbf{P}}
\newcommand{\BQ}[0]{\mathbf{Q}}
\newcommand{\BR}[0]{\mathbf{R}}
\newcommand{\BS}[0]{\mathbf{S}}
\newcommand{\BT}[0]{\mathbf{T}}
\newcommand{\BU}[0]{\mathbf{U}}
\newcommand{\BV}[0]{\mathbf{V}}
\newcommand{\BW}[0]{\mathbf{W}}
\newcommand{\BX}[0]{\mathbf{X}}
\newcommand{\BY}[0]{\mathbf{Y}}
\newcommand{\BZ}[0]{\mathbf{Z}}

\newcommand{\Bzero}[0]{\mathbf{0}}
\newcommand{\Btheta}[0]{\boldsymbol{\theta}}
\newcommand{\Btau}[0]{\boldsymbol{\tau}}
\newcommand{\Bomega}[0]{\boldsymbol{\omega}}

%
% shorthand for unit vectors
%
\newcommand{\acap}[0]{\hat{\Ba}}
\newcommand{\bcap}[0]{\hat{\Bb}}
\newcommand{\ccap}[0]{\hat{\Bc}}
\newcommand{\dcap}[0]{\hat{\Bd}}
\newcommand{\ecap}[0]{\hat{\Be}}
\newcommand{\fcap}[0]{\hat{\Bf}}
\newcommand{\gcap}[0]{\hat{\Bg}}
\newcommand{\hcap}[0]{\hat{\Bh}}
\newcommand{\icap}[0]{\hat{\Bi}}
\newcommand{\jcap}[0]{\hat{\Bj}}
\newcommand{\kcap}[0]{\hat{\Bk}}
\newcommand{\lcap}[0]{\hat{\Bl}}
\newcommand{\mcap}[0]{\hat{\Bm}}
\newcommand{\ncap}[0]{\hat{\Bn}}
\newcommand{\ocap}[0]{\hat{\Bo}}
\newcommand{\pcap}[0]{\hat{\Bp}}
\newcommand{\qcap}[0]{\hat{\Bq}}
\newcommand{\rcap}[0]{\hat{\Br}}
\newcommand{\scap}[0]{\hat{\Bs}}
\newcommand{\tcap}[0]{\hat{\Bt}}
\newcommand{\ucap}[0]{\hat{\Bu}}
\newcommand{\vcap}[0]{\hat{\Bv}}
\newcommand{\wcap}[0]{\hat{\Bw}}
\newcommand{\xcap}[0]{\hat{\Bx}}
\newcommand{\ycap}[0]{\hat{\By}}
\newcommand{\zcap}[0]{\hat{\Bz}}
\newcommand{\thetacap}[0]{\hat{\Btheta}}

%
% to write R^n and C^n in a distinguishable fashion.  Perhaps change this
% to the double lined characters upon figuring out how to do so.
%
\newcommand{\C}[1]{$\mathbb{C}^{#1}$}
\newcommand{\R}[1]{$\mathbb{R}^{#1}$}

%
% various generally useful helpers
%

% derivative of #1 wrt. #2:
\newcommand{\D}[2] {\frac {d#2} {d#1}}

\newcommand{\inv}[1]{\frac{1}{#1}}
\newcommand{\cross}[0]{\times}

\newcommand{\abs}[1]{\lvert{#1}\rvert}
\newcommand{\norm}[1]{\lVert{#1}\rVert}
\newcommand{\innerprod}[2]{\langle{#1}, {#2}\rangle}
\newcommand{\dotprod}[2]{{#1} \cdot {#2}}
\newcommand{\bdotprod}[2]{\left({#1} \cdot {#2}\right)}
\newcommand{\crossprod}[2]{{#1} \cross {#2}}
\newcommand{\tripleprod}[3]{\dotprod{\left(\crossprod{#1}{#2}\right)}{#3}}

\DeclareMathOperator{\Proj}{Proj}
\DeclareMathOperator{\Span}{span}
\DeclareMathOperator{\Sgn}{sgn}
\DeclareMathOperator{\Area}{Area}
\DeclareMathOperator{\Volume}{Volume}

%
% A few miscellaneous things specific to this document
%
\newcommand{\crossop}[1]{\crossprod{#1}{}}

% R2 vector.
\newcommand{\VectorTwo}[2]{
\begin{bmatrix}
 {#1} \\
 {#2}
\end{bmatrix}
}

\newcommand{\VectorN}[1]{
\begin{bmatrix}
{#1}_1 \\
{#1}_2 \\
\vdots \\
{#1}_N \\
\end{bmatrix}
}

\newcommand{\DETuvij}[4]{
\begin{vmatrix}
 {#1}_{#3} & {#1}_{#4} \\
 {#2}_{#3} & {#2}_{#4}
\end{vmatrix}
}

\newcommand{\DETuvwijk}[6]{
\begin{vmatrix}
 {#1}_{#4} & {#1}_{#5} & {#1}_{#6} \\
 {#2}_{#4} & {#2}_{#5} & {#2}_{#6} \\
 {#3}_{#4} & {#3}_{#5} & {#3}_{#6}
\end{vmatrix}
}

\newcommand{\DETuvwxijkl}[8]{
\begin{vmatrix}
 {#1}_{#5} & {#1}_{#6} & {#1}_{#7} & {#1}_{#8} \\
 {#2}_{#5} & {#2}_{#6} & {#2}_{#7} & {#2}_{#8} \\
 {#3}_{#5} & {#3}_{#6} & {#3}_{#7} & {#3}_{#8} \\
 {#4}_{#5} & {#4}_{#6} & {#4}_{#7} & {#4}_{#8} \\
\end{vmatrix}
}

%\newcommand{\DETuvwxyijklm}[10]{
%\begin{vmatrix}
% {#1}_{#6} & {#1}_{#7} & {#1}_{#8} & {#1}_{#9} & {#1}_{#10} \\
% {#2}_{#6} & {#2}_{#7} & {#2}_{#8} & {#2}_{#9} & {#2}_{#10} \\
% {#3}_{#6} & {#3}_{#7} & {#3}_{#8} & {#3}_{#9} & {#3}_{#10} \\
% {#4}_{#6} & {#4}_{#7} & {#4}_{#8} & {#4}_{#9} & {#4}_{#10} \\
% {#5}_{#6} & {#5}_{#7} & {#5}_{#8} & {#5}_{#9} & {#5}_{#10}
%\end{vmatrix}
%}

% R3 vector.
\newcommand{\VectorThree}[3]{
\begin{bmatrix}
 {#1} \\
 {#2} \\
 {#3}
\end{bmatrix}
}



\author{Peeter Joot}
\email{peeter.joot@gmail.com}


\chapter{PHY450H1S.  Relativistic Electrodynamics Lecture 17 (Taught by Prof. Erich Poppitz).  FIXME: summary.}
\label{chap:relativisticElectrodynamicsL17}
%\useCCL
\blogpage{http://sites.google.com/site/peeterjoot/math2011/relativisticElectrodynamicsL17.pdf}
\date{Mar 8, 2011}
\revisionInfo{relativisticElectrodynamicsL17.tex}

%\beginArtWithToc
\beginArtNoToc

\section{Reading.}

Covering chapter 6 material from the text \cite{landau1980classical}.

%Covering \href{}{lecture notes pp. xx-yy}:

\section{.}

\EndArticle

%
% Copyright � 2015 Peeter Joot.  All Rights Reserved.
% Licenced as described in the file LICENSE under the root directory of this GIT repository.
%
\documentclass[]{eliblog}

\usepackage{amsmath}
\usepackage{mathpazo}

%
% shorthand for bold symbols, convenient for vectors and matrices
%
\newcommand{\Ba}[0]{\mathbf{a}}
\newcommand{\Bb}[0]{\mathbf{b}}
\newcommand{\Bc}[0]{\mathbf{c}}
\newcommand{\Bd}[0]{\mathbf{d}}
\newcommand{\Be}[0]{\mathbf{e}}
\newcommand{\Bf}[0]{\mathbf{f}}
\newcommand{\Bg}[0]{\mathbf{g}}
\newcommand{\Bh}[0]{\mathbf{h}}
\newcommand{\Bi}[0]{\mathbf{i}}
\newcommand{\Bj}[0]{\mathbf{j}}
\newcommand{\Bk}[0]{\mathbf{k}}
\newcommand{\Bl}[0]{\mathbf{l}}
\newcommand{\Bm}[0]{\mathbf{m}}
\newcommand{\Bn}[0]{\mathbf{n}}
\newcommand{\Bo}[0]{\mathbf{o}}
\newcommand{\Bp}[0]{\mathbf{p}}
\newcommand{\Bq}[0]{\mathbf{q}}
\newcommand{\Br}[0]{\mathbf{r}}
\newcommand{\Bs}[0]{\mathbf{s}}
\newcommand{\Bt}[0]{\mathbf{t}}
\newcommand{\Bu}[0]{\mathbf{u}}
\newcommand{\Bv}[0]{\mathbf{v}}
\newcommand{\Bw}[0]{\mathbf{w}}
\newcommand{\Bx}[0]{\mathbf{x}}
\newcommand{\By}[0]{\mathbf{y}}
\newcommand{\Bz}[0]{\mathbf{z}}
\newcommand{\BA}[0]{\mathbf{A}}
\newcommand{\BB}[0]{\mathbf{B}}
\newcommand{\BC}[0]{\mathbf{C}}
\newcommand{\BD}[0]{\mathbf{D}}
\newcommand{\BE}[0]{\mathbf{E}}
\newcommand{\BF}[0]{\mathbf{F}}
\newcommand{\BG}[0]{\mathbf{G}}
\newcommand{\BH}[0]{\mathbf{H}}
\newcommand{\BI}[0]{\mathbf{I}}
\newcommand{\BJ}[0]{\mathbf{J}}
\newcommand{\BK}[0]{\mathbf{K}}
\newcommand{\BL}[0]{\mathbf{L}}
\newcommand{\BM}[0]{\mathbf{M}}
\newcommand{\BN}[0]{\mathbf{N}}
\newcommand{\BO}[0]{\mathbf{O}}
\newcommand{\BP}[0]{\mathbf{P}}
\newcommand{\BQ}[0]{\mathbf{Q}}
\newcommand{\BR}[0]{\mathbf{R}}
\newcommand{\BS}[0]{\mathbf{S}}
\newcommand{\BT}[0]{\mathbf{T}}
\newcommand{\BU}[0]{\mathbf{U}}
\newcommand{\BV}[0]{\mathbf{V}}
\newcommand{\BW}[0]{\mathbf{W}}
\newcommand{\BX}[0]{\mathbf{X}}
\newcommand{\BY}[0]{\mathbf{Y}}
\newcommand{\BZ}[0]{\mathbf{Z}}

\newcommand{\Bzero}[0]{\mathbf{0}}
\newcommand{\Btheta}[0]{\boldsymbol{\theta}}
\newcommand{\Btau}[0]{\boldsymbol{\tau}}
\newcommand{\Bomega}[0]{\boldsymbol{\omega}}

%
% shorthand for unit vectors
%
\newcommand{\acap}[0]{\hat{\Ba}}
\newcommand{\bcap}[0]{\hat{\Bb}}
\newcommand{\ccap}[0]{\hat{\Bc}}
\newcommand{\dcap}[0]{\hat{\Bd}}
\newcommand{\ecap}[0]{\hat{\Be}}
\newcommand{\fcap}[0]{\hat{\Bf}}
\newcommand{\gcap}[0]{\hat{\Bg}}
\newcommand{\hcap}[0]{\hat{\Bh}}
\newcommand{\icap}[0]{\hat{\Bi}}
\newcommand{\jcap}[0]{\hat{\Bj}}
\newcommand{\kcap}[0]{\hat{\Bk}}
\newcommand{\lcap}[0]{\hat{\Bl}}
\newcommand{\mcap}[0]{\hat{\Bm}}
\newcommand{\ncap}[0]{\hat{\Bn}}
\newcommand{\ocap}[0]{\hat{\Bo}}
\newcommand{\pcap}[0]{\hat{\Bp}}
\newcommand{\qcap}[0]{\hat{\Bq}}
\newcommand{\rcap}[0]{\hat{\Br}}
\newcommand{\scap}[0]{\hat{\Bs}}
\newcommand{\tcap}[0]{\hat{\Bt}}
\newcommand{\ucap}[0]{\hat{\Bu}}
\newcommand{\vcap}[0]{\hat{\Bv}}
\newcommand{\wcap}[0]{\hat{\Bw}}
\newcommand{\xcap}[0]{\hat{\Bx}}
\newcommand{\ycap}[0]{\hat{\By}}
\newcommand{\zcap}[0]{\hat{\Bz}}
\newcommand{\thetacap}[0]{\hat{\Btheta}}

%
% to write R^n and C^n in a distinguishable fashion.  Perhaps change this
% to the double lined characters upon figuring out how to do so.
%
\newcommand{\C}[1]{$\mathbb{C}^{#1}$}
\newcommand{\R}[1]{$\mathbb{R}^{#1}$}

%
% various generally useful helpers
%

% derivative of #1 wrt. #2:
\newcommand{\D}[2] {\frac {d#2} {d#1}}

\newcommand{\inv}[1]{\frac{1}{#1}}
\newcommand{\cross}[0]{\times}

\newcommand{\abs}[1]{\lvert{#1}\rvert}
\newcommand{\norm}[1]{\lVert{#1}\rVert}
\newcommand{\innerprod}[2]{\langle{#1}, {#2}\rangle}
\newcommand{\dotprod}[2]{{#1} \cdot {#2}}
\newcommand{\bdotprod}[2]{\left({#1} \cdot {#2}\right)}
\newcommand{\crossprod}[2]{{#1} \cross {#2}}
\newcommand{\tripleprod}[3]{\dotprod{\left(\crossprod{#1}{#2}\right)}{#3}}

\DeclareMathOperator{\Proj}{Proj}
\DeclareMathOperator{\Span}{span}
\DeclareMathOperator{\Sgn}{sgn}
\DeclareMathOperator{\Area}{Area}
\DeclareMathOperator{\Volume}{Volume}

%
% A few miscellaneous things specific to this document
%
\newcommand{\crossop}[1]{\crossprod{#1}{}}

% R2 vector.
\newcommand{\VectorTwo}[2]{
\begin{bmatrix}
 {#1} \\
 {#2}
\end{bmatrix}
}

\newcommand{\VectorN}[1]{
\begin{bmatrix}
{#1}_1 \\
{#1}_2 \\
\vdots \\
{#1}_N \\
\end{bmatrix}
}

\newcommand{\DETuvij}[4]{
\begin{vmatrix}
 {#1}_{#3} & {#1}_{#4} \\
 {#2}_{#3} & {#2}_{#4}
\end{vmatrix}
}

\newcommand{\DETuvwijk}[6]{
\begin{vmatrix}
 {#1}_{#4} & {#1}_{#5} & {#1}_{#6} \\
 {#2}_{#4} & {#2}_{#5} & {#2}_{#6} \\
 {#3}_{#4} & {#3}_{#5} & {#3}_{#6}
\end{vmatrix}
}

\newcommand{\DETuvwxijkl}[8]{
\begin{vmatrix}
 {#1}_{#5} & {#1}_{#6} & {#1}_{#7} & {#1}_{#8} \\
 {#2}_{#5} & {#2}_{#6} & {#2}_{#7} & {#2}_{#8} \\
 {#3}_{#5} & {#3}_{#6} & {#3}_{#7} & {#3}_{#8} \\
 {#4}_{#5} & {#4}_{#6} & {#4}_{#7} & {#4}_{#8} \\
\end{vmatrix}
}

%\newcommand{\DETuvwxyijklm}[10]{
%\begin{vmatrix}
% {#1}_{#6} & {#1}_{#7} & {#1}_{#8} & {#1}_{#9} & {#1}_{#10} \\
% {#2}_{#6} & {#2}_{#7} & {#2}_{#8} & {#2}_{#9} & {#2}_{#10} \\
% {#3}_{#6} & {#3}_{#7} & {#3}_{#8} & {#3}_{#9} & {#3}_{#10} \\
% {#4}_{#6} & {#4}_{#7} & {#4}_{#8} & {#4}_{#9} & {#4}_{#10} \\
% {#5}_{#6} & {#5}_{#7} & {#5}_{#8} & {#5}_{#9} & {#5}_{#10}
%\end{vmatrix}
%}

% R3 vector.
\newcommand{\VectorThree}[3]{
\begin{bmatrix}
 {#1} \\
 {#2} \\
 {#3}
\end{bmatrix}
}



\author{Peeter Joot}
\email{peeter.joot@gmail.com}

%\documentclass[]{eliblogwidescreen}

\usepackage{amsmath}
\usepackage{mathpazo}

%
% shorthand for bold symbols, convenient for vectors and matrices
%
\newcommand{\Ba}[0]{\mathbf{a}}
\newcommand{\Bb}[0]{\mathbf{b}}
\newcommand{\Bc}[0]{\mathbf{c}}
\newcommand{\Bd}[0]{\mathbf{d}}
\newcommand{\Be}[0]{\mathbf{e}}
\newcommand{\Bf}[0]{\mathbf{f}}
\newcommand{\Bg}[0]{\mathbf{g}}
\newcommand{\Bh}[0]{\mathbf{h}}
\newcommand{\Bi}[0]{\mathbf{i}}
\newcommand{\Bj}[0]{\mathbf{j}}
\newcommand{\Bk}[0]{\mathbf{k}}
\newcommand{\Bl}[0]{\mathbf{l}}
\newcommand{\Bm}[0]{\mathbf{m}}
\newcommand{\Bn}[0]{\mathbf{n}}
\newcommand{\Bo}[0]{\mathbf{o}}
\newcommand{\Bp}[0]{\mathbf{p}}
\newcommand{\Bq}[0]{\mathbf{q}}
\newcommand{\Br}[0]{\mathbf{r}}
\newcommand{\Bs}[0]{\mathbf{s}}
\newcommand{\Bt}[0]{\mathbf{t}}
\newcommand{\Bu}[0]{\mathbf{u}}
\newcommand{\Bv}[0]{\mathbf{v}}
\newcommand{\Bw}[0]{\mathbf{w}}
\newcommand{\Bx}[0]{\mathbf{x}}
\newcommand{\By}[0]{\mathbf{y}}
\newcommand{\Bz}[0]{\mathbf{z}}
\newcommand{\BA}[0]{\mathbf{A}}
\newcommand{\BB}[0]{\mathbf{B}}
\newcommand{\BC}[0]{\mathbf{C}}
\newcommand{\BD}[0]{\mathbf{D}}
\newcommand{\BE}[0]{\mathbf{E}}
\newcommand{\BF}[0]{\mathbf{F}}
\newcommand{\BG}[0]{\mathbf{G}}
\newcommand{\BH}[0]{\mathbf{H}}
\newcommand{\BI}[0]{\mathbf{I}}
\newcommand{\BJ}[0]{\mathbf{J}}
\newcommand{\BK}[0]{\mathbf{K}}
\newcommand{\BL}[0]{\mathbf{L}}
\newcommand{\BM}[0]{\mathbf{M}}
\newcommand{\BN}[0]{\mathbf{N}}
\newcommand{\BO}[0]{\mathbf{O}}
\newcommand{\BP}[0]{\mathbf{P}}
\newcommand{\BQ}[0]{\mathbf{Q}}
\newcommand{\BR}[0]{\mathbf{R}}
\newcommand{\BS}[0]{\mathbf{S}}
\newcommand{\BT}[0]{\mathbf{T}}
\newcommand{\BU}[0]{\mathbf{U}}
\newcommand{\BV}[0]{\mathbf{V}}
\newcommand{\BW}[0]{\mathbf{W}}
\newcommand{\BX}[0]{\mathbf{X}}
\newcommand{\BY}[0]{\mathbf{Y}}
\newcommand{\BZ}[0]{\mathbf{Z}}

\newcommand{\Bzero}[0]{\mathbf{0}}
\newcommand{\Btheta}[0]{\boldsymbol{\theta}}
\newcommand{\Btau}[0]{\boldsymbol{\tau}}
\newcommand{\Bomega}[0]{\boldsymbol{\omega}}

%
% shorthand for unit vectors
%
\newcommand{\acap}[0]{\hat{\Ba}}
\newcommand{\bcap}[0]{\hat{\Bb}}
\newcommand{\ccap}[0]{\hat{\Bc}}
\newcommand{\dcap}[0]{\hat{\Bd}}
\newcommand{\ecap}[0]{\hat{\Be}}
\newcommand{\fcap}[0]{\hat{\Bf}}
\newcommand{\gcap}[0]{\hat{\Bg}}
\newcommand{\hcap}[0]{\hat{\Bh}}
\newcommand{\icap}[0]{\hat{\Bi}}
\newcommand{\jcap}[0]{\hat{\Bj}}
\newcommand{\kcap}[0]{\hat{\Bk}}
\newcommand{\lcap}[0]{\hat{\Bl}}
\newcommand{\mcap}[0]{\hat{\Bm}}
\newcommand{\ncap}[0]{\hat{\Bn}}
\newcommand{\ocap}[0]{\hat{\Bo}}
\newcommand{\pcap}[0]{\hat{\Bp}}
\newcommand{\qcap}[0]{\hat{\Bq}}
\newcommand{\rcap}[0]{\hat{\Br}}
\newcommand{\scap}[0]{\hat{\Bs}}
\newcommand{\tcap}[0]{\hat{\Bt}}
\newcommand{\ucap}[0]{\hat{\Bu}}
\newcommand{\vcap}[0]{\hat{\Bv}}
\newcommand{\wcap}[0]{\hat{\Bw}}
\newcommand{\xcap}[0]{\hat{\Bx}}
\newcommand{\ycap}[0]{\hat{\By}}
\newcommand{\zcap}[0]{\hat{\Bz}}
\newcommand{\thetacap}[0]{\hat{\Btheta}}

%
% to write R^n and C^n in a distinguishable fashion.  Perhaps change this
% to the double lined characters upon figuring out how to do so.
%
\newcommand{\C}[1]{$\mathbb{C}^{#1}$}
\newcommand{\R}[1]{$\mathbb{R}^{#1}$}

%
% various generally useful helpers
%

% derivative of #1 wrt. #2:
\newcommand{\D}[2] {\frac {d#2} {d#1}}

\newcommand{\inv}[1]{\frac{1}{#1}}
\newcommand{\cross}[0]{\times}

\newcommand{\abs}[1]{\lvert{#1}\rvert}
\newcommand{\norm}[1]{\lVert{#1}\rVert}
\newcommand{\innerprod}[2]{\langle{#1}, {#2}\rangle}
\newcommand{\dotprod}[2]{{#1} \cdot {#2}}
\newcommand{\bdotprod}[2]{\left({#1} \cdot {#2}\right)}
\newcommand{\crossprod}[2]{{#1} \cross {#2}}
\newcommand{\tripleprod}[3]{\dotprod{\left(\crossprod{#1}{#2}\right)}{#3}}

\DeclareMathOperator{\Proj}{Proj}
\DeclareMathOperator{\Span}{span}
\DeclareMathOperator{\Sgn}{sgn}
\DeclareMathOperator{\Area}{Area}
\DeclareMathOperator{\Volume}{Volume}

%
% A few miscellaneous things specific to this document
%
\newcommand{\crossop}[1]{\crossprod{#1}{}}

% R2 vector.
\newcommand{\VectorTwo}[2]{
\begin{bmatrix}
 {#1} \\
 {#2}
\end{bmatrix}
}

\newcommand{\VectorN}[1]{
\begin{bmatrix}
{#1}_1 \\
{#1}_2 \\
\vdots \\
{#1}_N \\
\end{bmatrix}
}

\newcommand{\DETuvij}[4]{
\begin{vmatrix}
 {#1}_{#3} & {#1}_{#4} \\
 {#2}_{#3} & {#2}_{#4}
\end{vmatrix}
}

\newcommand{\DETuvwijk}[6]{
\begin{vmatrix}
 {#1}_{#4} & {#1}_{#5} & {#1}_{#6} \\
 {#2}_{#4} & {#2}_{#5} & {#2}_{#6} \\
 {#3}_{#4} & {#3}_{#5} & {#3}_{#6}
\end{vmatrix}
}

\newcommand{\DETuvwxijkl}[8]{
\begin{vmatrix}
 {#1}_{#5} & {#1}_{#6} & {#1}_{#7} & {#1}_{#8} \\
 {#2}_{#5} & {#2}_{#6} & {#2}_{#7} & {#2}_{#8} \\
 {#3}_{#5} & {#3}_{#6} & {#3}_{#7} & {#3}_{#8} \\
 {#4}_{#5} & {#4}_{#6} & {#4}_{#7} & {#4}_{#8} \\
\end{vmatrix}
}

%\newcommand{\DETuvwxyijklm}[10]{
%\begin{vmatrix}
% {#1}_{#6} & {#1}_{#7} & {#1}_{#8} & {#1}_{#9} & {#1}_{#10} \\
% {#2}_{#6} & {#2}_{#7} & {#2}_{#8} & {#2}_{#9} & {#2}_{#10} \\
% {#3}_{#6} & {#3}_{#7} & {#3}_{#8} & {#3}_{#9} & {#3}_{#10} \\
% {#4}_{#6} & {#4}_{#7} & {#4}_{#8} & {#4}_{#9} & {#4}_{#10} \\
% {#5}_{#6} & {#5}_{#7} & {#5}_{#8} & {#5}_{#9} & {#5}_{#10}
%\end{vmatrix}
%}

% R3 vector.
\newcommand{\VectorThree}[3]{
\begin{bmatrix}
 {#1} \\
 {#2} \\
 {#3}
\end{bmatrix}
}



\author{Peeter Joot}
\email{peeter.joot@gmail.com}


\chapter{PHY450H1S.  Relativistic Electrodynamics Tutorial 5 (TA: Simon Freedman).  FIXME}
\label{chap:relativisticElectrodynamicsT5}
%\useCCL
\blogpage{http://sites.google.com/site/peeterjoot/math2011/relativisticElectrodynamicsT5.pdf}
\date{Feb 17, 2011}
\revisionInfo{relativisticElectrodynamicsT5.tex}

\beginArtWithToc
%\beginArtNoToc

\section{.}

%\EndArticle
\EndNoBibArticle

%
% Copyright � 2015 Peeter Joot.  All Rights Reserved.
% Licenced as described in the file LICENSE under the root directory of this GIT repository.
%
\documentclass[]{eliblog}

\usepackage{amsmath}
\usepackage{mathpazo}

%
% shorthand for bold symbols, convenient for vectors and matrices
%
\newcommand{\Ba}[0]{\mathbf{a}}
\newcommand{\Bb}[0]{\mathbf{b}}
\newcommand{\Bc}[0]{\mathbf{c}}
\newcommand{\Bd}[0]{\mathbf{d}}
\newcommand{\Be}[0]{\mathbf{e}}
\newcommand{\Bf}[0]{\mathbf{f}}
\newcommand{\Bg}[0]{\mathbf{g}}
\newcommand{\Bh}[0]{\mathbf{h}}
\newcommand{\Bi}[0]{\mathbf{i}}
\newcommand{\Bj}[0]{\mathbf{j}}
\newcommand{\Bk}[0]{\mathbf{k}}
\newcommand{\Bl}[0]{\mathbf{l}}
\newcommand{\Bm}[0]{\mathbf{m}}
\newcommand{\Bn}[0]{\mathbf{n}}
\newcommand{\Bo}[0]{\mathbf{o}}
\newcommand{\Bp}[0]{\mathbf{p}}
\newcommand{\Bq}[0]{\mathbf{q}}
\newcommand{\Br}[0]{\mathbf{r}}
\newcommand{\Bs}[0]{\mathbf{s}}
\newcommand{\Bt}[0]{\mathbf{t}}
\newcommand{\Bu}[0]{\mathbf{u}}
\newcommand{\Bv}[0]{\mathbf{v}}
\newcommand{\Bw}[0]{\mathbf{w}}
\newcommand{\Bx}[0]{\mathbf{x}}
\newcommand{\By}[0]{\mathbf{y}}
\newcommand{\Bz}[0]{\mathbf{z}}
\newcommand{\BA}[0]{\mathbf{A}}
\newcommand{\BB}[0]{\mathbf{B}}
\newcommand{\BC}[0]{\mathbf{C}}
\newcommand{\BD}[0]{\mathbf{D}}
\newcommand{\BE}[0]{\mathbf{E}}
\newcommand{\BF}[0]{\mathbf{F}}
\newcommand{\BG}[0]{\mathbf{G}}
\newcommand{\BH}[0]{\mathbf{H}}
\newcommand{\BI}[0]{\mathbf{I}}
\newcommand{\BJ}[0]{\mathbf{J}}
\newcommand{\BK}[0]{\mathbf{K}}
\newcommand{\BL}[0]{\mathbf{L}}
\newcommand{\BM}[0]{\mathbf{M}}
\newcommand{\BN}[0]{\mathbf{N}}
\newcommand{\BO}[0]{\mathbf{O}}
\newcommand{\BP}[0]{\mathbf{P}}
\newcommand{\BQ}[0]{\mathbf{Q}}
\newcommand{\BR}[0]{\mathbf{R}}
\newcommand{\BS}[0]{\mathbf{S}}
\newcommand{\BT}[0]{\mathbf{T}}
\newcommand{\BU}[0]{\mathbf{U}}
\newcommand{\BV}[0]{\mathbf{V}}
\newcommand{\BW}[0]{\mathbf{W}}
\newcommand{\BX}[0]{\mathbf{X}}
\newcommand{\BY}[0]{\mathbf{Y}}
\newcommand{\BZ}[0]{\mathbf{Z}}

\newcommand{\Bzero}[0]{\mathbf{0}}
\newcommand{\Btheta}[0]{\boldsymbol{\theta}}
\newcommand{\Btau}[0]{\boldsymbol{\tau}}
\newcommand{\Bomega}[0]{\boldsymbol{\omega}}

%
% shorthand for unit vectors
%
\newcommand{\acap}[0]{\hat{\Ba}}
\newcommand{\bcap}[0]{\hat{\Bb}}
\newcommand{\ccap}[0]{\hat{\Bc}}
\newcommand{\dcap}[0]{\hat{\Bd}}
\newcommand{\ecap}[0]{\hat{\Be}}
\newcommand{\fcap}[0]{\hat{\Bf}}
\newcommand{\gcap}[0]{\hat{\Bg}}
\newcommand{\hcap}[0]{\hat{\Bh}}
\newcommand{\icap}[0]{\hat{\Bi}}
\newcommand{\jcap}[0]{\hat{\Bj}}
\newcommand{\kcap}[0]{\hat{\Bk}}
\newcommand{\lcap}[0]{\hat{\Bl}}
\newcommand{\mcap}[0]{\hat{\Bm}}
\newcommand{\ncap}[0]{\hat{\Bn}}
\newcommand{\ocap}[0]{\hat{\Bo}}
\newcommand{\pcap}[0]{\hat{\Bp}}
\newcommand{\qcap}[0]{\hat{\Bq}}
\newcommand{\rcap}[0]{\hat{\Br}}
\newcommand{\scap}[0]{\hat{\Bs}}
\newcommand{\tcap}[0]{\hat{\Bt}}
\newcommand{\ucap}[0]{\hat{\Bu}}
\newcommand{\vcap}[0]{\hat{\Bv}}
\newcommand{\wcap}[0]{\hat{\Bw}}
\newcommand{\xcap}[0]{\hat{\Bx}}
\newcommand{\ycap}[0]{\hat{\By}}
\newcommand{\zcap}[0]{\hat{\Bz}}
\newcommand{\thetacap}[0]{\hat{\Btheta}}

%
% to write R^n and C^n in a distinguishable fashion.  Perhaps change this
% to the double lined characters upon figuring out how to do so.
%
\newcommand{\C}[1]{$\mathbb{C}^{#1}$}
\newcommand{\R}[1]{$\mathbb{R}^{#1}$}

%
% various generally useful helpers
%

% derivative of #1 wrt. #2:
\newcommand{\D}[2] {\frac {d#2} {d#1}}

\newcommand{\inv}[1]{\frac{1}{#1}}
\newcommand{\cross}[0]{\times}

\newcommand{\abs}[1]{\lvert{#1}\rvert}
\newcommand{\norm}[1]{\lVert{#1}\rVert}
\newcommand{\innerprod}[2]{\langle{#1}, {#2}\rangle}
\newcommand{\dotprod}[2]{{#1} \cdot {#2}}
\newcommand{\bdotprod}[2]{\left({#1} \cdot {#2}\right)}
\newcommand{\crossprod}[2]{{#1} \cross {#2}}
\newcommand{\tripleprod}[3]{\dotprod{\left(\crossprod{#1}{#2}\right)}{#3}}

\DeclareMathOperator{\Proj}{Proj}
\DeclareMathOperator{\Span}{span}
\DeclareMathOperator{\Sgn}{sgn}
\DeclareMathOperator{\Area}{Area}
\DeclareMathOperator{\Volume}{Volume}

%
% A few miscellaneous things specific to this document
%
\newcommand{\crossop}[1]{\crossprod{#1}{}}

% R2 vector.
\newcommand{\VectorTwo}[2]{
\begin{bmatrix}
 {#1} \\
 {#2}
\end{bmatrix}
}

\newcommand{\VectorN}[1]{
\begin{bmatrix}
{#1}_1 \\
{#1}_2 \\
\vdots \\
{#1}_N \\
\end{bmatrix}
}

\newcommand{\DETuvij}[4]{
\begin{vmatrix}
 {#1}_{#3} & {#1}_{#4} \\
 {#2}_{#3} & {#2}_{#4}
\end{vmatrix}
}

\newcommand{\DETuvwijk}[6]{
\begin{vmatrix}
 {#1}_{#4} & {#1}_{#5} & {#1}_{#6} \\
 {#2}_{#4} & {#2}_{#5} & {#2}_{#6} \\
 {#3}_{#4} & {#3}_{#5} & {#3}_{#6}
\end{vmatrix}
}

\newcommand{\DETuvwxijkl}[8]{
\begin{vmatrix}
 {#1}_{#5} & {#1}_{#6} & {#1}_{#7} & {#1}_{#8} \\
 {#2}_{#5} & {#2}_{#6} & {#2}_{#7} & {#2}_{#8} \\
 {#3}_{#5} & {#3}_{#6} & {#3}_{#7} & {#3}_{#8} \\
 {#4}_{#5} & {#4}_{#6} & {#4}_{#7} & {#4}_{#8} \\
\end{vmatrix}
}

%\newcommand{\DETuvwxyijklm}[10]{
%\begin{vmatrix}
% {#1}_{#6} & {#1}_{#7} & {#1}_{#8} & {#1}_{#9} & {#1}_{#10} \\
% {#2}_{#6} & {#2}_{#7} & {#2}_{#8} & {#2}_{#9} & {#2}_{#10} \\
% {#3}_{#6} & {#3}_{#7} & {#3}_{#8} & {#3}_{#9} & {#3}_{#10} \\
% {#4}_{#6} & {#4}_{#7} & {#4}_{#8} & {#4}_{#9} & {#4}_{#10} \\
% {#5}_{#6} & {#5}_{#7} & {#5}_{#8} & {#5}_{#9} & {#5}_{#10}
%\end{vmatrix}
%}

% R3 vector.
\newcommand{\VectorThree}[3]{
\begin{bmatrix}
 {#1} \\
 {#2} \\
 {#3}
\end{bmatrix}
}



\author{Peeter Joot}
\email{peeter.joot@gmail.com}

%\documentclass[]{eliblogwidescreen}

\usepackage{amsmath}
\usepackage{mathpazo}

%
% shorthand for bold symbols, convenient for vectors and matrices
%
\newcommand{\Ba}[0]{\mathbf{a}}
\newcommand{\Bb}[0]{\mathbf{b}}
\newcommand{\Bc}[0]{\mathbf{c}}
\newcommand{\Bd}[0]{\mathbf{d}}
\newcommand{\Be}[0]{\mathbf{e}}
\newcommand{\Bf}[0]{\mathbf{f}}
\newcommand{\Bg}[0]{\mathbf{g}}
\newcommand{\Bh}[0]{\mathbf{h}}
\newcommand{\Bi}[0]{\mathbf{i}}
\newcommand{\Bj}[0]{\mathbf{j}}
\newcommand{\Bk}[0]{\mathbf{k}}
\newcommand{\Bl}[0]{\mathbf{l}}
\newcommand{\Bm}[0]{\mathbf{m}}
\newcommand{\Bn}[0]{\mathbf{n}}
\newcommand{\Bo}[0]{\mathbf{o}}
\newcommand{\Bp}[0]{\mathbf{p}}
\newcommand{\Bq}[0]{\mathbf{q}}
\newcommand{\Br}[0]{\mathbf{r}}
\newcommand{\Bs}[0]{\mathbf{s}}
\newcommand{\Bt}[0]{\mathbf{t}}
\newcommand{\Bu}[0]{\mathbf{u}}
\newcommand{\Bv}[0]{\mathbf{v}}
\newcommand{\Bw}[0]{\mathbf{w}}
\newcommand{\Bx}[0]{\mathbf{x}}
\newcommand{\By}[0]{\mathbf{y}}
\newcommand{\Bz}[0]{\mathbf{z}}
\newcommand{\BA}[0]{\mathbf{A}}
\newcommand{\BB}[0]{\mathbf{B}}
\newcommand{\BC}[0]{\mathbf{C}}
\newcommand{\BD}[0]{\mathbf{D}}
\newcommand{\BE}[0]{\mathbf{E}}
\newcommand{\BF}[0]{\mathbf{F}}
\newcommand{\BG}[0]{\mathbf{G}}
\newcommand{\BH}[0]{\mathbf{H}}
\newcommand{\BI}[0]{\mathbf{I}}
\newcommand{\BJ}[0]{\mathbf{J}}
\newcommand{\BK}[0]{\mathbf{K}}
\newcommand{\BL}[0]{\mathbf{L}}
\newcommand{\BM}[0]{\mathbf{M}}
\newcommand{\BN}[0]{\mathbf{N}}
\newcommand{\BO}[0]{\mathbf{O}}
\newcommand{\BP}[0]{\mathbf{P}}
\newcommand{\BQ}[0]{\mathbf{Q}}
\newcommand{\BR}[0]{\mathbf{R}}
\newcommand{\BS}[0]{\mathbf{S}}
\newcommand{\BT}[0]{\mathbf{T}}
\newcommand{\BU}[0]{\mathbf{U}}
\newcommand{\BV}[0]{\mathbf{V}}
\newcommand{\BW}[0]{\mathbf{W}}
\newcommand{\BX}[0]{\mathbf{X}}
\newcommand{\BY}[0]{\mathbf{Y}}
\newcommand{\BZ}[0]{\mathbf{Z}}

\newcommand{\Bzero}[0]{\mathbf{0}}
\newcommand{\Btheta}[0]{\boldsymbol{\theta}}
\newcommand{\Btau}[0]{\boldsymbol{\tau}}
\newcommand{\Bomega}[0]{\boldsymbol{\omega}}

%
% shorthand for unit vectors
%
\newcommand{\acap}[0]{\hat{\Ba}}
\newcommand{\bcap}[0]{\hat{\Bb}}
\newcommand{\ccap}[0]{\hat{\Bc}}
\newcommand{\dcap}[0]{\hat{\Bd}}
\newcommand{\ecap}[0]{\hat{\Be}}
\newcommand{\fcap}[0]{\hat{\Bf}}
\newcommand{\gcap}[0]{\hat{\Bg}}
\newcommand{\hcap}[0]{\hat{\Bh}}
\newcommand{\icap}[0]{\hat{\Bi}}
\newcommand{\jcap}[0]{\hat{\Bj}}
\newcommand{\kcap}[0]{\hat{\Bk}}
\newcommand{\lcap}[0]{\hat{\Bl}}
\newcommand{\mcap}[0]{\hat{\Bm}}
\newcommand{\ncap}[0]{\hat{\Bn}}
\newcommand{\ocap}[0]{\hat{\Bo}}
\newcommand{\pcap}[0]{\hat{\Bp}}
\newcommand{\qcap}[0]{\hat{\Bq}}
\newcommand{\rcap}[0]{\hat{\Br}}
\newcommand{\scap}[0]{\hat{\Bs}}
\newcommand{\tcap}[0]{\hat{\Bt}}
\newcommand{\ucap}[0]{\hat{\Bu}}
\newcommand{\vcap}[0]{\hat{\Bv}}
\newcommand{\wcap}[0]{\hat{\Bw}}
\newcommand{\xcap}[0]{\hat{\Bx}}
\newcommand{\ycap}[0]{\hat{\By}}
\newcommand{\zcap}[0]{\hat{\Bz}}
\newcommand{\thetacap}[0]{\hat{\Btheta}}

%
% to write R^n and C^n in a distinguishable fashion.  Perhaps change this
% to the double lined characters upon figuring out how to do so.
%
\newcommand{\C}[1]{$\mathbb{C}^{#1}$}
\newcommand{\R}[1]{$\mathbb{R}^{#1}$}

%
% various generally useful helpers
%

% derivative of #1 wrt. #2:
\newcommand{\D}[2] {\frac {d#2} {d#1}}

\newcommand{\inv}[1]{\frac{1}{#1}}
\newcommand{\cross}[0]{\times}

\newcommand{\abs}[1]{\lvert{#1}\rvert}
\newcommand{\norm}[1]{\lVert{#1}\rVert}
\newcommand{\innerprod}[2]{\langle{#1}, {#2}\rangle}
\newcommand{\dotprod}[2]{{#1} \cdot {#2}}
\newcommand{\bdotprod}[2]{\left({#1} \cdot {#2}\right)}
\newcommand{\crossprod}[2]{{#1} \cross {#2}}
\newcommand{\tripleprod}[3]{\dotprod{\left(\crossprod{#1}{#2}\right)}{#3}}

\DeclareMathOperator{\Proj}{Proj}
\DeclareMathOperator{\Span}{span}
\DeclareMathOperator{\Sgn}{sgn}
\DeclareMathOperator{\Area}{Area}
\DeclareMathOperator{\Volume}{Volume}

%
% A few miscellaneous things specific to this document
%
\newcommand{\crossop}[1]{\crossprod{#1}{}}

% R2 vector.
\newcommand{\VectorTwo}[2]{
\begin{bmatrix}
 {#1} \\
 {#2}
\end{bmatrix}
}

\newcommand{\VectorN}[1]{
\begin{bmatrix}
{#1}_1 \\
{#1}_2 \\
\vdots \\
{#1}_N \\
\end{bmatrix}
}

\newcommand{\DETuvij}[4]{
\begin{vmatrix}
 {#1}_{#3} & {#1}_{#4} \\
 {#2}_{#3} & {#2}_{#4}
\end{vmatrix}
}

\newcommand{\DETuvwijk}[6]{
\begin{vmatrix}
 {#1}_{#4} & {#1}_{#5} & {#1}_{#6} \\
 {#2}_{#4} & {#2}_{#5} & {#2}_{#6} \\
 {#3}_{#4} & {#3}_{#5} & {#3}_{#6}
\end{vmatrix}
}

\newcommand{\DETuvwxijkl}[8]{
\begin{vmatrix}
 {#1}_{#5} & {#1}_{#6} & {#1}_{#7} & {#1}_{#8} \\
 {#2}_{#5} & {#2}_{#6} & {#2}_{#7} & {#2}_{#8} \\
 {#3}_{#5} & {#3}_{#6} & {#3}_{#7} & {#3}_{#8} \\
 {#4}_{#5} & {#4}_{#6} & {#4}_{#7} & {#4}_{#8} \\
\end{vmatrix}
}

%\newcommand{\DETuvwxyijklm}[10]{
%\begin{vmatrix}
% {#1}_{#6} & {#1}_{#7} & {#1}_{#8} & {#1}_{#9} & {#1}_{#10} \\
% {#2}_{#6} & {#2}_{#7} & {#2}_{#8} & {#2}_{#9} & {#2}_{#10} \\
% {#3}_{#6} & {#3}_{#7} & {#3}_{#8} & {#3}_{#9} & {#3}_{#10} \\
% {#4}_{#6} & {#4}_{#7} & {#4}_{#8} & {#4}_{#9} & {#4}_{#10} \\
% {#5}_{#6} & {#5}_{#7} & {#5}_{#8} & {#5}_{#9} & {#5}_{#10}
%\end{vmatrix}
%}

% R3 vector.
\newcommand{\VectorThree}[3]{
\begin{bmatrix}
 {#1} \\
 {#2} \\
 {#3}
\end{bmatrix}
}



\author{Peeter Joot}
\email{peeter.joot@gmail.com}


\chapter{PHY450H1S.  Relativistic Electrodynamics Lecture 18 (Taught by Prof. Erich Poppitz).  Green's function solution to Maxwell's equation.}
\label{chap:relativisticElectrodynamicsL18}
%\useCCL
\blogpage{http://sites.google.com/site/peeterjoot/math2011/relativisticElectrodynamicsL18.pdf}
\date{Mar 9, 2011}
\revisionInfo{relativisticElectrodynamicsL18.tex}

%\beginArtWithToc
\beginArtNoToc

\section{Reading.}

Covering chapter 8 material from the text \cite{landau1980classical}.

Covering \href{http://www.physics.utoronto.ca/~poppitz/epoppitz/PHY450_files/RelEMpp136-146.pdf}{lecture notes pp. 136-146}: continued reminder of electrostatic Green�s function (136); the retarded Green�s function of the d�Alembert operator: derivation and properties (137-140); the solution of the d�Alembert equation with a source: retarded potentials (141-142); retarded time: meaning and the Lienard-Wiechert potentials (143-146) [Wednesday, Mar. 9...]

\section{.}

\EndArticle


\part{Notes and Problems.}
%
% Copyright � 2012 Peeter Joot.  All Rights Reserved.
% Licenced as described in the file LICENSE under the root directory of this GIT repository.
%

\label{chap:antisymmetricTensorTx}
%\blogpage{http://sites.google.com/site/peeterjoot/math2011/antisymmetricTensorTx.pdf}
%\date{Jan 14, 2011}

\section{Motivation}

I have an ancient copy of the course text \citep{landau1951classical} from the library right now (mine is on order still) for my PHY450H1S course (relativistic electrodynamics).  Given the transformation rule for a first rank tensor

\begin{equation}\label{eqn:antisymmetricTensorTx:5}
A_{i} = \alpha_{im} A'_{m},
\end{equation}

they list the transformation rule for a second rank tensor as

\begin{equation}\label{eqn:antisymmetricTensorTx:10}
A_{ik} = \alpha_{im} \alpha_{kl} A'_{ml}.
\end{equation}

This is not motivated in any way.  Let us compare to transformation of a bivector expressed in the Dirac basis, transformed by outermorphism.  That is specifically a transformation of a antisymmetric tensor (once expressed in components anyways), but should provide some intuition.

It is also worthwhile to note that there are some old fashioned notational quirks in this text (at least the old version that I have currently borrowed).  Specifically, they uses Latin indices four vectors with Greek indices for three vectors, completely opposite to what appears to be the current conventions.  They also do not use upper and lower indices to keep track of bookkeeping.  I will use the conventions I am used to for now.

\section{Notation and use of Geometric Algebra herein}

I will use conventions from \citep{doran2003gap} using the Dirac basis, with a preference for index upper coordinates, and express a vector as

\begin{equation}\label{eqn:antisymmetricTensorTx:20}
x = x^\alpha \gamma_\alpha = x_\alpha \gamma^\alpha,
\end{equation}

Here the basis pairs \(\{\gamma_\mu\}\) and \(\{\gamma^\mu\}\) are reciprocal frames with \(\gamma^\mu \cdot \gamma_\nu = {\delta^\mu}_\nu\).  I will have no need for any specific metric convention here.

The dot and wedge products used will be defined in terms of their Clifford Algebra formulation

\begin{equation}\label{eqn:antisymmetricTensorTx:25}
\begin{aligned}
a \cdot b &= \inv{2} (a b + b a) \\
a \wedge b &= \inv{2} (a b - b a).
\end{aligned}
\end{equation}

The dot product between two bivectors \(A\), \(B\) will also be used, defined as the scalar part of the product \(AB\).  In particular the identity for extraction of that scalar component from the dot product of two wedge products will be required

\begin{equation}\label{eqn:antisymmetricTensorTx:26}
(a \wedge b ) \cdot (c \wedge d)
= ( a (b \cdot c) - b ( a \cdot c) ) \cdot d
= (a \cdot d) (b \cdot c) - (b \cdot d) ( a \cdot c)
\end{equation}

\section{Transformation of the coordinates}

Let us assume our transformation is linear, and we will denote its action on vectors as follows

\begin{equation}\label{eqn:antisymmetricTensorTx:30}
x' = L(x) = x^\alpha L( \gamma_\alpha).
\end{equation}

Extracting coordinates for the transformed coordinates (assuming a non-moving frame where the unit vectors on both sides are the same), we have after dotting with \(\gamma^\mu\)

\begin{equation}\label{eqn:antisymmetricTensorTx:40}
{x'}^\mu = \left( {x'}^\alpha \gamma_\alpha \right) \cdot \gamma^\mu
= x^\alpha \left( L( \gamma_\alpha) \cdot \gamma^\mu \right)
\end{equation}

Now introduce a coordinate representation for the transformation \(L\)

\begin{equation}\label{eqn:antisymmetricTensorTx:50}
L( \gamma_\alpha) \cdot \gamma^\mu  = {L_\alpha}^\mu,
\end{equation}

so our transformation rule for the four vector coordinates becomes

\begin{equation}\label{eqn:antisymmetricTensorTx:55}
{x'}^\mu = x^\alpha {L_\alpha}^\mu.
\end{equation}

We are now ready to look at the transformation of a bivector (a quantity having a rank two antisymmetric tensor representation in coordinates), and see how the coordinates transform.

Let us transform by outermorphism of the transformed vector factors the bivector

\begin{equation}\label{eqn:antisymmetricTensorTx:60}
c = a \wedge b \rightarrow a' \wedge b'.
\end{equation}

First we will need the coordinate representation of the bivector before transformation.  We dot with \(\gamma^\nu \wedge \gamma^\mu\) to pick up the desired term

\begin{equation}\label{eqn:antisymmetricTensorTx:670}
\begin{aligned}
(a \wedge b) \cdot (\gamma_\nu \wedge \gamma_\mu)
&=
a^\alpha b^\beta (\gamma_\alpha \wedge \gamma_\beta) \cdot (\gamma^\nu \wedge \gamma^\mu) \\
&=
a^\alpha b^\beta ( \gamma_\alpha {\delta_\beta}^\nu -\gamma_\beta {\delta_\alpha}^\nu ) \cdot \gamma^\mu \\
&=
a^\alpha b^\beta ( {\delta_\alpha}^\mu {\delta_\beta}^\nu -{\delta_\beta}^\mu {\delta_\alpha}^\nu ) \\
&=
a^\mu b^\nu
-a^\nu b^\mu \\
\end{aligned}
\end{equation}

If we introduce a rank two tensor now, say

\begin{equation}\label{eqn:antisymmetricTensorTx:70}
T^{\mu\nu} = a^\mu b^\nu -a^\nu b^\mu,
\end{equation}

we recover our bivector with
\begin{equation}\label{eqn:antisymmetricTensorTx:80}
a \wedge b = \inv{2} T^{\alpha \beta} \gamma_\alpha \wedge \gamma_\beta.
\end{equation}

Now let us look at the coordinate representation of the transformed bivector.  It will also be helpful to make use of the identity that can be observed above from the initial coordinate extraction

\begin{equation}\label{eqn:antisymmetricTensorTx:90}
(\gamma_\alpha \wedge \gamma_\beta) \cdot (\gamma^\nu \wedge \gamma^\mu) = {\delta_\alpha}^\mu {\delta_\beta}^\nu -{\delta_\beta}^\mu {\delta_\alpha}^\nu
\end{equation}

In coordinates our transformed bivector is

\begin{equation}\label{eqn:antisymmetricTensorTx:100}
a' \wedge b' =
a^\sigma {L_\sigma}^\alpha
b^\pi {L_\pi}^\beta
\gamma_\alpha \wedge \gamma_\beta,
\end{equation}

and we can proceed with the coordinate extraction by taking dot products with \(\gamma^\nu \wedge \gamma^\mu\) as before.  This gives us

\begin{equation}\label{eqn:antisymmetricTensorTx:690}
\begin{aligned}
( a' \wedge b' ) \cdot (\gamma^\nu \wedge \gamma^\mu)
&=
a^\sigma {L_\sigma}^\alpha
b^\pi {L_\pi}^\beta
\gamma_\alpha \wedge \gamma_\beta \\
&=
a^\sigma {L_\sigma}^\alpha
b^\pi {L_\pi}^\beta
( {\delta_\alpha}^\mu {\delta_\beta}^\nu -{\delta_\beta}^\mu {\delta_\alpha}^\nu  ) \\
&=
a^\sigma {L_\sigma}^\mu
b^\pi {L_\pi}^\nu
-
a^\sigma {L_\sigma}^\nu
b^\pi {L_\pi}^\mu \\
&=
a^\sigma {L_\sigma}^\mu
b^\pi {L_\pi}^\nu
-
a^\pi {L_\pi}^\nu
b^\sigma {L_\sigma}^\mu \\
&=
(a^\sigma b^\pi - a^\pi b^\sigma) {L_\sigma}^\mu {L_\pi}^\nu
\\
&=
T^{\sigma \pi} {L_\sigma}^\mu {L_\pi}^\nu
\\
\end{aligned}
\end{equation}

We are able to conclude that the bivector coordinates transform as

\begin{equation}\label{eqn:antisymmetricTensorTx:200}
T^{\mu \nu} \rightarrow T^{\sigma \pi} {L_\sigma}^\mu {L_\pi}^\nu.
\end{equation}

Except for the lowering index differences this verifies the rule \eqnref{eqn:antisymmetricTensorTx:10} from the text.

It would be reasonable seeming to impose such a tensor transformation rule on any antisymmetric rank 2 tensor, and in the text this is also imposed as the rule for transformation of symmetric rank 2 tensors.  Do we have a simple example of a rank 2 symmetric tensor that can be expressed geometrically?  The only one that comes to mind off the top of my head is the electrodynamic stress tensor, which is not exactly simple to work with.

\section{Lorentz transformation of the metric tensors}

Following up on the previous thought, it is not hard to come up with an example of a symmetric tensor a whole lot simpler than the electrodynamic stress tensor.  The metric tensor is probably the simplest symmetric tensor, and we get that by considering the dot product of two vectors.  Taking the dot product of vectors \(a\) and \(b\) for example we have

\begin{equation}\label{eqn:antisymmetricTensorTx:300}
a \cdot b
= a^\mu b^\nu \gamma_\mu \cdot \gamma_\nu
\end{equation}

From this, the metric tensors are defined as

\begin{equation}\label{eqn:antisymmetricTensorTx:310}
\begin{aligned}
g_{\mu\nu} &= \gamma_\mu \cdot \gamma_\nu \\
g^{\mu\nu} &= \gamma^\mu \cdot \gamma^\nu
\end{aligned}
\end{equation}

These are both symmetric and diagonal, and in fact equal (regardless of whether one picks a \(+,-,-,-\) or \(-,+,+,+\) signature for the space).

Let us look at the transformation of the dot product, utilizing the transformation of the four vectors being dotted to do so.  By definition, when both vectors are equal, we have the (squared) spacetime interval, which based on the speed of light being constant, has been found to be an invariant under transformation.

\begin{equation}\label{eqn:antisymmetricTensorTx:320}
a' \cdot b'
=
a^\mu b^\nu L(\gamma_\mu) \cdot L(\gamma_\nu)
\end{equation}

We note that, like any other vector, the image \(L(\gamma_\mu)\) of the Lorentz transform of the vector \(\gamma_\mu\) can be written as

\begin{equation}\label{eqn:antisymmetricTensorTx:330}
L(\gamma_\mu) = \left( L(\gamma_\mu) \cdot \gamma^\nu \right) \gamma_\nu
\end{equation}

Similarly we can write any vector in terms of the reciprocal frame
\begin{equation}\label{eqn:antisymmetricTensorTx:340}
\gamma_\nu = (\gamma_\nu \cdot \gamma_\mu) \gamma^\mu.
\end{equation}

The dot product factor is a component of the metric tensor

\begin{equation}\label{eqn:antisymmetricTensorTx:350}
g_{\nu \mu} = \gamma_\nu \cdot \gamma_\mu,
\end{equation}

so we see that the dot product transforms as

\begin{equation}\label{eqn:antisymmetricTensorTx:320b}
a' \cdot b'
= a^\mu b^\nu
( L(\gamma_\mu) \cdot \gamma^\alpha )
( L(\gamma_\nu) \cdot \gamma^\beta )
\gamma_\alpha
\cdot
\gamma_\beta
= a^\mu b^\nu
{L_\mu}^\alpha
{L_\nu}^\beta
g_{\alpha \beta}
\end{equation}

In particular, for \(a = b\) where we have the invariant interval defined by the condition \(a^2 = {a'}^2\), we must have

\begin{equation}\label{eqn:antisymmetricTensorTx:360}
a^\mu a^\nu g_{\mu \nu}
= a^\mu a^\nu
{L_\mu}^\alpha
{L_\nu}^\beta
g_{\alpha \beta}
\end{equation}

This implies that the symmetric metric tensor transforms as

\begin{equation}\label{eqn:antisymmetricTensorTx:370}
g_{\mu\nu}
=
{L_\mu}^\alpha
{L_\nu}^\beta
g_{\alpha \beta}
\end{equation}

Recall from \eqnref{eqn:antisymmetricTensorTx:200} that the coordinates representation of a bivector, an antisymmetric quantity transformed as

\begin{equation}\label{eqn:antisymmetricTensorTx:200b}
T^{\mu \nu} \rightarrow T^{\sigma \pi} {L_\sigma}^\mu {L_\pi}^\nu.
\end{equation}

This is a very similar transformation, but differs from the bivector case where our free indices were upper indices.  Suppose that we define an alternate set of coordinates for the Lorentz transformation.  Let

\begin{equation}\label{eqn:antisymmetricTensorTx:380}
{L^\mu}_\nu = L(\gamma^\mu) \cdot \gamma_\nu.
\end{equation}

This can be related to the previous coordinate matrix by
\begin{equation}\label{eqn:antisymmetricTensorTx:390}
{L^\mu}_\nu = g^{\mu \alpha } g_{\nu \beta } {L_\alpha}^\beta.
\end{equation}

If we examine how the coordinates of \(x^2\) transform in their lower index representation we find

\begin{equation}\label{eqn:antisymmetricTensorTx:400}
{x'}^2 = x_\mu x_\nu {L^\mu}_\alpha {L^\nu}_\beta g^{\alpha \beta} = x^2 = x_\mu x_\nu g^{\mu \nu},
\end{equation}

and therefore find that the (upper index) metric tensor transforms as

\begin{equation}\label{eqn:antisymmetricTensorTx:410}
g^{\mu \nu} \rightarrow
g^{\alpha \beta}
{L^\mu}_\alpha {L^\nu}_\beta .
\end{equation}

Compared to \(\eqnref{eqn:antisymmetricTensorTx:200b}\) we have almost the same structure of transformation.  Are these the same?  Does the notation I picked here introduce an apparent difference that does not actually exist?  We really want to know if we have the identity

\begin{equation}\label{eqn:antisymmetricTensorTx:420}
L(\gamma_\mu) \cdot \gamma^\nu
\questionEquals
L(\gamma^\nu) \cdot \gamma_\mu,
\end{equation}

If that were to be the case, then given the notation selected it would mean that \({L_\mu}^\nu = {L^\nu}_\mu\).  If that were true it would justify a notational simplification \({L_\mu}^\nu = {L^\nu}_\mu = L^\nu_\mu\).

\section{The inverse Lorentz transformation}

To answer this question, let us consider a specific example, an x-axis boost of rapidity \(\alpha\).  For that our Lorentz transformation takes the following form

\begin{equation}\label{eqn:antisymmetricTensorTx:430}
L(x) = e^{-\sigma_1 \alpha/2} x e^{\sigma_1 \alpha/2},
\end{equation}

where \(\sigma_k = \gamma_k \gamma_0\).  Since \(\sigma_1\) anticommutes with \(\gamma_0\) and \(\gamma_1\), but commutes with \(\gamma_2\) and \(\gamma_3\), we have

\begin{equation}\label{eqn:antisymmetricTensorTx:440}
L(x) = (x^0 \gamma_0 + x^1 \gamma_1) e^{\sigma_1 \alpha} + x^2 \gamma_2 + x^3 \gamma_3,
\end{equation}

and after expansion this is
\begin{equation}\label{eqn:antisymmetricTensorTx:441}
L(x) =
\gamma_0 ( x^0 \cosh \alpha - x^1 \sinh \alpha )
+\gamma_1 ( x^1 \cosh \alpha - x^0 \sinh \alpha )
+\gamma_2
+\gamma_3.
\end{equation}

Note that this is the first time a specific metric preference has been imposed, and \(+,-,-,-\) has been used.

Observe that for the basis vectors themselves we have

\begin{equation}\label{eqn:antisymmetricTensorTx:450}
\begin{bmatrix}
L(\gamma_0) \\
L(\gamma_1) \\
L(\gamma_2) \\
L(\gamma_3)
\end{bmatrix}
=
\begin{bmatrix}
\gamma_0 \cosh \alpha - \gamma_1 \sinh \alpha \\
-\gamma_0 \sinh \alpha + \gamma_1 \cosh \alpha \\
\gamma_2 \\
\gamma_3
\end{bmatrix}
\end{equation}

Forming a matrix with \(\mu\) indexing over rows and \(\nu\) indexing over columns we have

\begin{equation}\label{eqn:antisymmetricTensorTx:460}
{L_\mu}^\nu =
\begin{bmatrix}
\cosh \alpha &- \sinh \alpha & 0 & 0 \\
-\sinh \alpha & \cosh \alpha & 0 & 0 \\
0 & 0 & 1 & 0 \\
0 & 0 & 0 & 1
\end{bmatrix}
\end{equation}

Performing the same expansion for \({L^\nu}_\mu\), again with \(\mu\) indexing over rows, we have

\begin{equation}\label{eqn:antisymmetricTensorTx:470}
{L^\nu}_\mu =
\begin{bmatrix}
\cosh \alpha & \sinh \alpha & 0 & 0 \\
\sinh \alpha & \cosh \alpha & 0 & 0 \\
0 & 0 & 1 & 0 \\
0 & 0 & 0 & 1
\end{bmatrix}.
\end{equation}

This answers the question.  We cannot assume that \({L_\mu}^\nu = {L^\nu}_\mu\).  In fact, in this particular case, we have \({L^\nu}_\mu = ({L_\mu}^\nu)^{-1}\).  Is that a general condition?  Note that for the general case, we have to consider compounded transformations, where each can be a boost or rotation.

With my text still not here I have obtained a newer version of the course text from a different UofT library.  In this newer version \citep{landau1971classical} (still not the 4th edition) it is at least updated with the ``modern'' upper and lower index formalism.

In this version they define a four-dimensional second rank tensor as the set of sixteen quantities

\begin{equation}\label{eqn:antisymmetricTensorTx:471}
A^{\mu\nu},
\end{equation}

provided these transform under coordinate transformations like the products of components of two four vectors.  They also provide raising and lowering rules that distinguish the quantities \({A^{\mu}}_\nu\), and \({A_{\mu}}^\nu\) by relating these to the raising and lowering operations so that, for example, \({A_0}^1 = A^{01}\), \({A^0}_1 = -A^{01}\).  This is consistent with the notation I have used fairly blunderingly that seemed natural.  This also highlights the difference between \({L_\mu}^\nu\), and \({L^\nu}_\mu\).  We can relate both of these back to the index upper tensor representation

\begin{equation}\label{eqn:antisymmetricTensorTx:472}
\begin{aligned}
{L_\alpha}^\nu &= g_{\mu \alpha} L^{\mu \nu} \\
{L^\mu}_\alpha &= g_{\nu \alpha} L^{\mu \nu}
\end{aligned}
\end{equation}

This shows precisely how the two objects relate back to the original tensor \(L^{\mu \nu}\), and why we cannot just write \(L_\alpha^\nu\) or \(L^\mu_\alpha\) respectively.

Note that in the third edition they still (somewhat surprisingly to me) continue to latin indices for \(0,1,2,3\) and greek for \(1,2,3\) as in the original 1951 version.

\section{Duality in tensor form}

Let us consider the subject of duality to antisymmetric forms.  Within a geometric algebra context our duality is provided by multiplication by the pseudoscalar for the space.

For instance in \R{3} the dual to a bivector is the familiar cross product

\begin{equation}\label{eqn:antisymmetricTensorTx:500}
\Ba \cross \Bb = -I (\Ba \wedge \Bb),
\end{equation}

where \(I = \Be_1 \Be_2 \Be_3\).  In our spacetime context we use the pseudoscalar \(I = \gamma_0 \gamma_1 \gamma_2 \gamma_3\).  Let us compute the coordinate representation of our vector, bivector, and trivector duals, which should compare with the tensor representation of the text.

In the text we have a statement that given an antisymmetric tensor \(T^{\mu \nu}\), its dual is

\begin{equation}\label{eqn:antisymmetricTensorTx:510}
\inv{2} {e^{\mu \nu}}_{\alpha \beta} T^{\alpha \beta}
\end{equation}

(I have adjusted the notation for the antisymmetric pseudotensor \(\epsilon\) to retain free upper indices).

How does this compare the to Geometric Algebra bivector dual in spacetime?  Let

\begin{equation}\label{eqn:antisymmetricTensorTx:520}
T = \inv{2} T^{\mu \nu} \gamma_\mu \wedge \gamma_\nu
%= \inv{2} \sum_{\mu < \nu} (T^{\mu \nu} - T^{\nu\mu} ) \gamma_\mu \wedge \gamma_\nu
= \sum_{\mu < \nu} T^{\mu \nu} \gamma_\mu \wedge \gamma_\nu.
\end{equation}

We dot with \(\gamma^\nu \wedge \gamma^\mu\) to extract the (tensor) coordinate representation

\begin{equation}\label{eqn:antisymmetricTensorTx:710}
\begin{aligned}
T \cdot (\gamma^\nu \wedge \gamma^\mu)
&=
\inv{2} T^{\alpha \beta} (\gamma_\alpha \wedge \gamma_\beta ) \cdot (\gamma^\nu \wedge \gamma^\mu) \\
&=
\inv{2} T^{\alpha \beta} ( {\delta_\beta}^\nu {\delta_\alpha}^\mu -{\delta_\alpha}^\nu {\delta_\beta}^\mu ) \\
&=
\inv{2} (T^{\mu \nu} - T^{\nu \mu}) \\
&=
T^{\mu \nu}.
\end{aligned}
\end{equation}

The index manipulation gets a little hairy, but one can expand the dot products \((I T) \cdot (\gamma^\nu \wedge \gamma^\mu)\) to find that this dual has coordinates have the value,

\begin{equation}\label{eqn:antisymmetricTensorTx:530}
(I T) \cdot (\gamma^\nu \wedge \gamma^\mu) = C {e^{\mu \nu}}_{\alpha \beta} T^{\alpha \beta},
\end{equation}

where \(C\) is a constant multiplier that I messed up computing the actual value for.

It is also possible to verify that \((IT) \cdot T = 0\).  Thus we can describe the duality of \(T^{\mu \nu}\) and \({e^{\mu \nu}}_{\alpha \beta} T^{\alpha \beta}\) as the geometrical condition \(T = a b\), \(IT = c d\), where \(a, b, c, d\) are all mutually perpendicular.

Given a vector \(x = x^\mu \gamma_\mu = x_\mu \gamma^\mu\) it is also possible to confirm that the coordinate representation of the Geometric Algebra vector dual has the form

\begin{equation}\label{eqn:antisymmetricTensorTx:540}
I x \sim e^{\sigma \pi \nu \mu} \gamma_\sigma \gamma_\sigma \gamma_\pi x_\nu
\end{equation}

The coordinates of this product are a multiple of \(\epsilon^{\sigma \pi \nu \mu } x_\mu\), which has the form specified in the text.

\section{Stokes Theorem}

%In \citep{gabook:stokesNoTensor} I worked through the Geometric Algebra expression for Stokes Theorem.  For a \(k-1\) grade blade, the final result of that work was
I once worked through the Geometric Algebra expression for Stokes Theorem.  For a \(k-1\) grade blade, the final result of that work was

\begin{equation}\label{eqn:antisymmetricTensorTx:600}
\int
( \grad \wedge F ) \cdot d^k x
=
\inv{(k-1)!} \epsilon^{ r s \cdots t u } \int da_u \PD{a_{u}}{F} \cdot
(dx_r \wedge dx_s \wedge \cdots \wedge dx_t)
\end{equation}

Let us expand this in coordinates to attempt to get the equivalent expression for an antisymmetric tensor of rank \(k-1\).

Starting with the RHS of \eqnref{eqn:antisymmetricTensorTx:600} we have

\begin{equation}\label{eqn:antisymmetricTensorTx:610}
\begin{aligned}
F &= \inv{(k-1)!}
F_{\mu_1 \mu_2 \cdots \mu_{k-1} }
\gamma^{\mu_1} \wedge \gamma^{ \mu_2 } \wedge \cdots \wedge \gamma^{\mu_{k-1}}
\\
dx_r \wedge dx_s \wedge \cdots \wedge dx_t &=
\PD{a_r}{x^{\nu_1}}
\PD{a_s}{x^{\nu_2}}
\cdots
\PD{a_t}{x^{\nu_{k-1}}}
\gamma_{\nu_1} \wedge \gamma_{ \nu_2 } \wedge \cdots \wedge \gamma_{\nu_{k-1}}
da_r da_s \cdots da_t
\end{aligned}
\end{equation}

We need to expand the dot product of the wedges, for which we have
\begin{equation}\label{eqn:antisymmetricTensorTx:620}
\begin{aligned}
&\left(
\gamma^{\mu_1} \wedge \gamma^{ \mu_2 } \wedge \cdots \wedge \gamma^{\mu_{k-1}}
\right)
\cdot
\left(
\gamma_{\nu_1} \wedge \gamma_{ \nu_2 } \wedge \cdots \wedge \gamma_{\nu_{k-1}}
\right) \\
&=
{\delta^{\mu_{k-1}}}_{\nu_1}  {\delta^{ \mu_{k-2} }}_{\nu_2}  \cdots  {\delta^{\mu_{1}} }_{\nu_{k-1}}
\epsilon^{\nu_1 \nu_2 \cdots \nu_{k-1}}
\end{aligned}
\end{equation}

Putting all the LHS bits together we have
\begin{equation}\label{eqn:antisymmetricTensorTx:730}
\begin{aligned}
&\inv{((k-1)!)^2} \epsilon^{ r s \cdots t u } \int da_u \PD{a_{u}}{} F_{\mu_1 \mu_2 \cdots \mu_{k-1} } \\
&\qquad {\delta^{\mu_{k-1}}}_{\nu_1}  {\delta^{ \mu_{k-2} }}_{\nu_2}  \cdots  {\delta^{\mu_{1}} }_{\nu_{k-1}}
\epsilon^{\nu_1 \nu_2 \cdots \nu_{k-1}}
\PD{a_r}{x^{\nu_1}}
\PD{a_s}{x^{\nu_2}}
\cdots
\PD{a_t}{x^{\nu_{k-1}}}
da_r da_s \cdots da_t \\
&=\inv{((k-1)!)^2} \epsilon^{ r s \cdots t u } \int da_u \PD{a_{u}}{} F_{\mu_1 \mu_2 \cdots \mu_{k-1} } \\
&\qquad \epsilon^{\mu_{k-1} \mu_{k-2} \cdots \mu_{1}}
\PD{a_r}{x^{\mu_{k-1}}}
\PD{a_s}{x^{\mu_{k-2}}}
\cdots
\PD{a_t}{x^{\mu_1}}
da_r da_s \cdots da_t \\
&=
\inv{((k-1)!)^2} \epsilon^{ r s \cdots t u } \int da_u \PD{a_{u}}{} F_{\mu_1 \mu_2 \cdots \mu_{k-1} }
\Abs{\frac{\partial(
x^{\mu_{k-1}},x^{\mu_{k-2}},\cdots,x^{\mu_1}
)}{\partial(a_r, a_s, \cdots, a_t)}}
da_r da_s \cdots da_t \\
\end{aligned}
\end{equation}

Now, for the LHS of \eqnref{eqn:antisymmetricTensorTx:600} we have

\begin{equation}\label{eqn:antisymmetricTensorTx:750}
\begin{aligned}
\grad \wedge F
&=
\gamma^\mu \wedge \partial_\mu F \\
&=
\inv{(k-1)!}\PD{x^{\mu_k}}{} F_{\mu_1 \mu_2 \cdots \mu_{k-1}}
\gamma^{\mu_k} \wedge
\gamma^{\mu_1} \wedge \gamma^{ \mu_2 } \wedge \cdots \wedge \gamma^{\mu_{k-1}}
\end{aligned}
\end{equation}

and the volume element of
\begin{equation}\label{eqn:antisymmetricTensorTx:770}
\begin{aligned}
d^k x
&=
\PD{a_1}{x^{\nu_1}}
\PD{a_2}{x^{\nu_2}}
\cdots
\PD{a_k}{x^{\nu_{k}}}
\gamma_{\nu_1} \wedge \gamma_{ \nu_2 } \wedge \cdots \wedge \gamma_{\nu_k}
da_1 da_2 \cdots da_k
\end{aligned}
\end{equation}

Our dot product is
\begin{equation}\label{eqn:antisymmetricTensorTx:630}
\begin{aligned}
\left(\gamma^{\mu_k} \wedge
\gamma^{\mu_1} \wedge \gamma^{ \mu_2 } \wedge \cdots \wedge \gamma^{\mu_{k-1}} \right) & \cdot
\left( \gamma_{\nu_1} \wedge \gamma_{ \nu_2 } \wedge \cdots \wedge \gamma_{\nu_k} \right) \\
&=
{\delta^{\mu_{k-1}}}_{\nu_1}  {\delta^{ \mu_{k-2} }}_{\nu_2}  \cdots
{\delta^{\mu_{1}} }_{\nu_{k-1}}
{\delta^{\mu_{k}} }_{\nu_{k}}
\epsilon^{\nu_1 \nu_2 \cdots \nu_{k}}
\end{aligned}
\end{equation}

The LHS of our k-form now evaluates to

\begin{equation}\label{eqn:antisymmetricTensorTx:790}
\begin{aligned}
(\gamma^\mu \wedge \partial_\mu F) \cdot d^k x
&=
\inv{(k-1)!}\PD{x^{\mu_k}}{} F_{\mu_1 \mu_2 \cdots \mu_{k-1}} \\
&{\delta^{\mu_{k-1}}}_{\nu_1}  {\delta^{ \mu_{k-2} }}_{\nu_2}  \cdots
{\delta^{\mu_{1}} }_{\nu_{k-1}}
{\delta^{\mu_{k}} }_{\nu_{k}}
\epsilon^{\nu_1 \nu_2 \cdots \nu_{k}}
\PD{a_1}{x^{\nu_1}}
\PD{a_2}{x^{\nu_2}}
\cdots
\PD{a_k}{x^{\nu_{k}}}
da_1 da_2 \cdots da_k \\
&=
\inv{(k-1)!}\PD{x^{\mu_k}}{} F_{\mu_1 \mu_2 \cdots \mu_{k-1}} \\
&\epsilon^{\mu_{k-1} \mu_{k-2} \cdots \mu_1 \mu_{k}}
\PD{a_1}{x^{\mu_{k-1}}}
\PD{a_2}{x^{\mu_{k-2}}}
\cdots
\PD{a_{k-1}}{x^{\mu_{1}}}
\PD{a_k}{x^{\mu_{k}}}
da_1 da_2 \cdots da_k \\
&=
\inv{(k-1)!}\PD{x^{\mu_k}}{} F_{\mu_1 \mu_2 \cdots \mu_{k-1}}
\Abs{\frac{\partial(
x^{\mu_{k-1}},
x^{\mu_{k-2}},
\cdots
x^{\mu_{1}},
x^{\mu_{k}}
)}{\partial(a_1, a_2, \cdots, a_{k-1}, a_k)}
}
da_1 da_2 \cdots da_k \\
\end{aligned}
\end{equation}

Presuming no mistakes were made anywhere along the way (including in the original Geometric Algebra expression), we have arrived at Stokes Theorem for rank \(k-1\) antisymmetric tensors \(F\)

\boxedEquation{eqn:antisymmetricTensorTx:650}{
\begin{aligned}
&\int
\PD{x^{\mu_k}}{} F_{\mu_1 \mu_2 \cdots \mu_{k-1}}
\Abs{\frac{\partial(
x^{\mu_{k-1}},
x^{\mu_{k-2}},
\cdots
x^{\mu_{1}},
x^{\mu_{k}}
)}{\partial(a_1, a_2, \cdots, a_{k-1}, a_k)}
}
da_1 da_2 \cdots da_k \\
&=
\inv{(k-1)!} \epsilon^{ r s \cdots t u } \int da_u \PD{a_{u}}{} F_{\nu_1 \nu_2 \cdots \nu_{k-1} }
\Abs{\frac{\partial(
x^{\nu_{k-1}},x^{\nu_{k-2}},\cdots,x^{\nu_1}
)}{\partial(a_r, a_s, \cdots, a_t)}}
da_r da_s \cdots da_t
\end{aligned}
}

The next task is to validate this, expanding it out for some specific ranks and hypervolume element types, and to compare the results with the familiar 3d expressions.

%\documentclass[]{eliblog}
%\usepackage{color}
%\usepackage{txfonts} % for xi
%\usepackage{amsmath}
\usepackage{mathpazo}

%
% shorthand for bold symbols, convenient for vectors and matrices
%
\newcommand{\Ba}[0]{\mathbf{a}}
\newcommand{\Bb}[0]{\mathbf{b}}
\newcommand{\Bc}[0]{\mathbf{c}}
\newcommand{\Bd}[0]{\mathbf{d}}
\newcommand{\Be}[0]{\mathbf{e}}
\newcommand{\Bf}[0]{\mathbf{f}}
\newcommand{\Bg}[0]{\mathbf{g}}
\newcommand{\Bh}[0]{\mathbf{h}}
\newcommand{\Bi}[0]{\mathbf{i}}
\newcommand{\Bj}[0]{\mathbf{j}}
\newcommand{\Bk}[0]{\mathbf{k}}
\newcommand{\Bl}[0]{\mathbf{l}}
\newcommand{\Bm}[0]{\mathbf{m}}
\newcommand{\Bn}[0]{\mathbf{n}}
\newcommand{\Bo}[0]{\mathbf{o}}
\newcommand{\Bp}[0]{\mathbf{p}}
\newcommand{\Bq}[0]{\mathbf{q}}
\newcommand{\Br}[0]{\mathbf{r}}
\newcommand{\Bs}[0]{\mathbf{s}}
\newcommand{\Bt}[0]{\mathbf{t}}
\newcommand{\Bu}[0]{\mathbf{u}}
\newcommand{\Bv}[0]{\mathbf{v}}
\newcommand{\Bw}[0]{\mathbf{w}}
\newcommand{\Bx}[0]{\mathbf{x}}
\newcommand{\By}[0]{\mathbf{y}}
\newcommand{\Bz}[0]{\mathbf{z}}
\newcommand{\BA}[0]{\mathbf{A}}
\newcommand{\BB}[0]{\mathbf{B}}
\newcommand{\BC}[0]{\mathbf{C}}
\newcommand{\BD}[0]{\mathbf{D}}
\newcommand{\BE}[0]{\mathbf{E}}
\newcommand{\BF}[0]{\mathbf{F}}
\newcommand{\BG}[0]{\mathbf{G}}
\newcommand{\BH}[0]{\mathbf{H}}
\newcommand{\BI}[0]{\mathbf{I}}
\newcommand{\BJ}[0]{\mathbf{J}}
\newcommand{\BK}[0]{\mathbf{K}}
\newcommand{\BL}[0]{\mathbf{L}}
\newcommand{\BM}[0]{\mathbf{M}}
\newcommand{\BN}[0]{\mathbf{N}}
\newcommand{\BO}[0]{\mathbf{O}}
\newcommand{\BP}[0]{\mathbf{P}}
\newcommand{\BQ}[0]{\mathbf{Q}}
\newcommand{\BR}[0]{\mathbf{R}}
\newcommand{\BS}[0]{\mathbf{S}}
\newcommand{\BT}[0]{\mathbf{T}}
\newcommand{\BU}[0]{\mathbf{U}}
\newcommand{\BV}[0]{\mathbf{V}}
\newcommand{\BW}[0]{\mathbf{W}}
\newcommand{\BX}[0]{\mathbf{X}}
\newcommand{\BY}[0]{\mathbf{Y}}
\newcommand{\BZ}[0]{\mathbf{Z}}

\newcommand{\Bzero}[0]{\mathbf{0}}
\newcommand{\Btheta}[0]{\boldsymbol{\theta}}
\newcommand{\Btau}[0]{\boldsymbol{\tau}}
\newcommand{\Bomega}[0]{\boldsymbol{\omega}}

%
% shorthand for unit vectors
%
\newcommand{\acap}[0]{\hat{\Ba}}
\newcommand{\bcap}[0]{\hat{\Bb}}
\newcommand{\ccap}[0]{\hat{\Bc}}
\newcommand{\dcap}[0]{\hat{\Bd}}
\newcommand{\ecap}[0]{\hat{\Be}}
\newcommand{\fcap}[0]{\hat{\Bf}}
\newcommand{\gcap}[0]{\hat{\Bg}}
\newcommand{\hcap}[0]{\hat{\Bh}}
\newcommand{\icap}[0]{\hat{\Bi}}
\newcommand{\jcap}[0]{\hat{\Bj}}
\newcommand{\kcap}[0]{\hat{\Bk}}
\newcommand{\lcap}[0]{\hat{\Bl}}
\newcommand{\mcap}[0]{\hat{\Bm}}
\newcommand{\ncap}[0]{\hat{\Bn}}
\newcommand{\ocap}[0]{\hat{\Bo}}
\newcommand{\pcap}[0]{\hat{\Bp}}
\newcommand{\qcap}[0]{\hat{\Bq}}
\newcommand{\rcap}[0]{\hat{\Br}}
\newcommand{\scap}[0]{\hat{\Bs}}
\newcommand{\tcap}[0]{\hat{\Bt}}
\newcommand{\ucap}[0]{\hat{\Bu}}
\newcommand{\vcap}[0]{\hat{\Bv}}
\newcommand{\wcap}[0]{\hat{\Bw}}
\newcommand{\xcap}[0]{\hat{\Bx}}
\newcommand{\ycap}[0]{\hat{\By}}
\newcommand{\zcap}[0]{\hat{\Bz}}
\newcommand{\thetacap}[0]{\hat{\Btheta}}

%
% to write R^n and C^n in a distinguishable fashion.  Perhaps change this
% to the double lined characters upon figuring out how to do so.
%
\newcommand{\C}[1]{$\mathbb{C}^{#1}$}
\newcommand{\R}[1]{$\mathbb{R}^{#1}$}

%
% various generally useful helpers
%

% derivative of #1 wrt. #2:
\newcommand{\D}[2] {\frac {d#2} {d#1}}

\newcommand{\inv}[1]{\frac{1}{#1}}
\newcommand{\cross}[0]{\times}

\newcommand{\abs}[1]{\lvert{#1}\rvert}
\newcommand{\norm}[1]{\lVert{#1}\rVert}
\newcommand{\innerprod}[2]{\langle{#1}, {#2}\rangle}
\newcommand{\dotprod}[2]{{#1} \cdot {#2}}
\newcommand{\bdotprod}[2]{\left({#1} \cdot {#2}\right)}
\newcommand{\crossprod}[2]{{#1} \cross {#2}}
\newcommand{\tripleprod}[3]{\dotprod{\left(\crossprod{#1}{#2}\right)}{#3}}

\DeclareMathOperator{\Proj}{Proj}
\DeclareMathOperator{\Span}{span}
\DeclareMathOperator{\Sgn}{sgn}
\DeclareMathOperator{\Area}{Area}
\DeclareMathOperator{\Volume}{Volume}

%
% A few miscellaneous things specific to this document
%
\newcommand{\crossop}[1]{\crossprod{#1}{}}

% R2 vector.
\newcommand{\VectorTwo}[2]{
\begin{bmatrix}
 {#1} \\
 {#2}
\end{bmatrix}
}

\newcommand{\VectorN}[1]{
\begin{bmatrix}
{#1}_1 \\
{#1}_2 \\
\vdots \\
{#1}_N \\
\end{bmatrix}
}

\newcommand{\DETuvij}[4]{
\begin{vmatrix}
 {#1}_{#3} & {#1}_{#4} \\
 {#2}_{#3} & {#2}_{#4}
\end{vmatrix}
}

\newcommand{\DETuvwijk}[6]{
\begin{vmatrix}
 {#1}_{#4} & {#1}_{#5} & {#1}_{#6} \\
 {#2}_{#4} & {#2}_{#5} & {#2}_{#6} \\
 {#3}_{#4} & {#3}_{#5} & {#3}_{#6}
\end{vmatrix}
}

\newcommand{\DETuvwxijkl}[8]{
\begin{vmatrix}
 {#1}_{#5} & {#1}_{#6} & {#1}_{#7} & {#1}_{#8} \\
 {#2}_{#5} & {#2}_{#6} & {#2}_{#7} & {#2}_{#8} \\
 {#3}_{#5} & {#3}_{#6} & {#3}_{#7} & {#3}_{#8} \\
 {#4}_{#5} & {#4}_{#6} & {#4}_{#7} & {#4}_{#8} \\
\end{vmatrix}
}

%\newcommand{\DETuvwxyijklm}[10]{
%\begin{vmatrix}
% {#1}_{#6} & {#1}_{#7} & {#1}_{#8} & {#1}_{#9} & {#1}_{#10} \\
% {#2}_{#6} & {#2}_{#7} & {#2}_{#8} & {#2}_{#9} & {#2}_{#10} \\
% {#3}_{#6} & {#3}_{#7} & {#3}_{#8} & {#3}_{#9} & {#3}_{#10} \\
% {#4}_{#6} & {#4}_{#7} & {#4}_{#8} & {#4}_{#9} & {#4}_{#10} \\
% {#5}_{#6} & {#5}_{#7} & {#5}_{#8} & {#5}_{#9} & {#5}_{#10}
%\end{vmatrix}
%}

% R3 vector.
\newcommand{\VectorThree}[3]{
\begin{bmatrix}
 {#1} \\
 {#2} \\
 {#3}
\end{bmatrix}
}



%
% Copyright � 2015 Peeter Joot.  All Rights Reserved.
% Licenced as described in the file LICENSE under the root directory of this GIT repository.
%
\documentclass[]{eliblog}

\usepackage{amsmath}
\usepackage{mathpazo}

%
% shorthand for bold symbols, convenient for vectors and matrices
%
\newcommand{\Ba}[0]{\mathbf{a}}
\newcommand{\Bb}[0]{\mathbf{b}}
\newcommand{\Bc}[0]{\mathbf{c}}
\newcommand{\Bd}[0]{\mathbf{d}}
\newcommand{\Be}[0]{\mathbf{e}}
\newcommand{\Bf}[0]{\mathbf{f}}
\newcommand{\Bg}[0]{\mathbf{g}}
\newcommand{\Bh}[0]{\mathbf{h}}
\newcommand{\Bi}[0]{\mathbf{i}}
\newcommand{\Bj}[0]{\mathbf{j}}
\newcommand{\Bk}[0]{\mathbf{k}}
\newcommand{\Bl}[0]{\mathbf{l}}
\newcommand{\Bm}[0]{\mathbf{m}}
\newcommand{\Bn}[0]{\mathbf{n}}
\newcommand{\Bo}[0]{\mathbf{o}}
\newcommand{\Bp}[0]{\mathbf{p}}
\newcommand{\Bq}[0]{\mathbf{q}}
\newcommand{\Br}[0]{\mathbf{r}}
\newcommand{\Bs}[0]{\mathbf{s}}
\newcommand{\Bt}[0]{\mathbf{t}}
\newcommand{\Bu}[0]{\mathbf{u}}
\newcommand{\Bv}[0]{\mathbf{v}}
\newcommand{\Bw}[0]{\mathbf{w}}
\newcommand{\Bx}[0]{\mathbf{x}}
\newcommand{\By}[0]{\mathbf{y}}
\newcommand{\Bz}[0]{\mathbf{z}}
\newcommand{\BA}[0]{\mathbf{A}}
\newcommand{\BB}[0]{\mathbf{B}}
\newcommand{\BC}[0]{\mathbf{C}}
\newcommand{\BD}[0]{\mathbf{D}}
\newcommand{\BE}[0]{\mathbf{E}}
\newcommand{\BF}[0]{\mathbf{F}}
\newcommand{\BG}[0]{\mathbf{G}}
\newcommand{\BH}[0]{\mathbf{H}}
\newcommand{\BI}[0]{\mathbf{I}}
\newcommand{\BJ}[0]{\mathbf{J}}
\newcommand{\BK}[0]{\mathbf{K}}
\newcommand{\BL}[0]{\mathbf{L}}
\newcommand{\BM}[0]{\mathbf{M}}
\newcommand{\BN}[0]{\mathbf{N}}
\newcommand{\BO}[0]{\mathbf{O}}
\newcommand{\BP}[0]{\mathbf{P}}
\newcommand{\BQ}[0]{\mathbf{Q}}
\newcommand{\BR}[0]{\mathbf{R}}
\newcommand{\BS}[0]{\mathbf{S}}
\newcommand{\BT}[0]{\mathbf{T}}
\newcommand{\BU}[0]{\mathbf{U}}
\newcommand{\BV}[0]{\mathbf{V}}
\newcommand{\BW}[0]{\mathbf{W}}
\newcommand{\BX}[0]{\mathbf{X}}
\newcommand{\BY}[0]{\mathbf{Y}}
\newcommand{\BZ}[0]{\mathbf{Z}}

\newcommand{\Bzero}[0]{\mathbf{0}}
\newcommand{\Btheta}[0]{\boldsymbol{\theta}}
\newcommand{\Btau}[0]{\boldsymbol{\tau}}
\newcommand{\Bomega}[0]{\boldsymbol{\omega}}

%
% shorthand for unit vectors
%
\newcommand{\acap}[0]{\hat{\Ba}}
\newcommand{\bcap}[0]{\hat{\Bb}}
\newcommand{\ccap}[0]{\hat{\Bc}}
\newcommand{\dcap}[0]{\hat{\Bd}}
\newcommand{\ecap}[0]{\hat{\Be}}
\newcommand{\fcap}[0]{\hat{\Bf}}
\newcommand{\gcap}[0]{\hat{\Bg}}
\newcommand{\hcap}[0]{\hat{\Bh}}
\newcommand{\icap}[0]{\hat{\Bi}}
\newcommand{\jcap}[0]{\hat{\Bj}}
\newcommand{\kcap}[0]{\hat{\Bk}}
\newcommand{\lcap}[0]{\hat{\Bl}}
\newcommand{\mcap}[0]{\hat{\Bm}}
\newcommand{\ncap}[0]{\hat{\Bn}}
\newcommand{\ocap}[0]{\hat{\Bo}}
\newcommand{\pcap}[0]{\hat{\Bp}}
\newcommand{\qcap}[0]{\hat{\Bq}}
\newcommand{\rcap}[0]{\hat{\Br}}
\newcommand{\scap}[0]{\hat{\Bs}}
\newcommand{\tcap}[0]{\hat{\Bt}}
\newcommand{\ucap}[0]{\hat{\Bu}}
\newcommand{\vcap}[0]{\hat{\Bv}}
\newcommand{\wcap}[0]{\hat{\Bw}}
\newcommand{\xcap}[0]{\hat{\Bx}}
\newcommand{\ycap}[0]{\hat{\By}}
\newcommand{\zcap}[0]{\hat{\Bz}}
\newcommand{\thetacap}[0]{\hat{\Btheta}}

%
% to write R^n and C^n in a distinguishable fashion.  Perhaps change this
% to the double lined characters upon figuring out how to do so.
%
\newcommand{\C}[1]{$\mathbb{C}^{#1}$}
\newcommand{\R}[1]{$\mathbb{R}^{#1}$}

%
% various generally useful helpers
%

% derivative of #1 wrt. #2:
\newcommand{\D}[2] {\frac {d#2} {d#1}}

\newcommand{\inv}[1]{\frac{1}{#1}}
\newcommand{\cross}[0]{\times}

\newcommand{\abs}[1]{\lvert{#1}\rvert}
\newcommand{\norm}[1]{\lVert{#1}\rVert}
\newcommand{\innerprod}[2]{\langle{#1}, {#2}\rangle}
\newcommand{\dotprod}[2]{{#1} \cdot {#2}}
\newcommand{\bdotprod}[2]{\left({#1} \cdot {#2}\right)}
\newcommand{\crossprod}[2]{{#1} \cross {#2}}
\newcommand{\tripleprod}[3]{\dotprod{\left(\crossprod{#1}{#2}\right)}{#3}}

\DeclareMathOperator{\Proj}{Proj}
\DeclareMathOperator{\Span}{span}
\DeclareMathOperator{\Sgn}{sgn}
\DeclareMathOperator{\Area}{Area}
\DeclareMathOperator{\Volume}{Volume}

%
% A few miscellaneous things specific to this document
%
\newcommand{\crossop}[1]{\crossprod{#1}{}}

% R2 vector.
\newcommand{\VectorTwo}[2]{
\begin{bmatrix}
 {#1} \\
 {#2}
\end{bmatrix}
}

\newcommand{\VectorN}[1]{
\begin{bmatrix}
{#1}_1 \\
{#1}_2 \\
\vdots \\
{#1}_N \\
\end{bmatrix}
}

\newcommand{\DETuvij}[4]{
\begin{vmatrix}
 {#1}_{#3} & {#1}_{#4} \\
 {#2}_{#3} & {#2}_{#4}
\end{vmatrix}
}

\newcommand{\DETuvwijk}[6]{
\begin{vmatrix}
 {#1}_{#4} & {#1}_{#5} & {#1}_{#6} \\
 {#2}_{#4} & {#2}_{#5} & {#2}_{#6} \\
 {#3}_{#4} & {#3}_{#5} & {#3}_{#6}
\end{vmatrix}
}

\newcommand{\DETuvwxijkl}[8]{
\begin{vmatrix}
 {#1}_{#5} & {#1}_{#6} & {#1}_{#7} & {#1}_{#8} \\
 {#2}_{#5} & {#2}_{#6} & {#2}_{#7} & {#2}_{#8} \\
 {#3}_{#5} & {#3}_{#6} & {#3}_{#7} & {#3}_{#8} \\
 {#4}_{#5} & {#4}_{#6} & {#4}_{#7} & {#4}_{#8} \\
\end{vmatrix}
}

%\newcommand{\DETuvwxyijklm}[10]{
%\begin{vmatrix}
% {#1}_{#6} & {#1}_{#7} & {#1}_{#8} & {#1}_{#9} & {#1}_{#10} \\
% {#2}_{#6} & {#2}_{#7} & {#2}_{#8} & {#2}_{#9} & {#2}_{#10} \\
% {#3}_{#6} & {#3}_{#7} & {#3}_{#8} & {#3}_{#9} & {#3}_{#10} \\
% {#4}_{#6} & {#4}_{#7} & {#4}_{#8} & {#4}_{#9} & {#4}_{#10} \\
% {#5}_{#6} & {#5}_{#7} & {#5}_{#8} & {#5}_{#9} & {#5}_{#10}
%\end{vmatrix}
%}

% R3 vector.
\newcommand{\VectorThree}[3]{
\begin{bmatrix}
 {#1} \\
 {#2} \\
 {#3}
\end{bmatrix}
}



\author{Peeter Joot}
\email{peeter.joot@gmail.com}

\author{Peeter Joot}
\email{peeter.joot@utoronto.ca, 920798560}

\chapter{PHY450H1S Problem Set 1.}
\label{chap:relElectroDynProblemSet1}
%\blogpage{http://sites.google.com/site/peeterjoot/math2011/relElectroDynProblemSet1.pdf}
\date{Jan 22, 2011}
\revisionInfo{relElectroDynProblemSet1.tex}

\beginArtNoToc
%\section{Disclaimer.}
%
%This problem set is as yet ungraded.

\section{Problem 1.}
\subsection{Statement}

Some Minkowski diagram exersizes.

\subsection{Solution}

These will be hand written.
%%%I'll hand write my solutions for these.  To prep for it, let's suppose that $(ct', x')$ and $(ct, x)$ are related by 
%%%
%%%\begin{equation}\label{eqn:relativisticElectrodynamicsA1:1}
%%%\begin{bmatrix}
%%%x' \\
%%%ct'
%%%\end{bmatrix}
%%%\begin{bmatrix}
%%%\cosh \alpha & \sinh\alpha \\
%%%\sinh \alpha & \cosh\alpha 
%%%\end{bmatrix}
%%%\begin{bmatrix}
%%%x \\
%%%ct
%%%\end{bmatrix}.
%%%\end{equation}
%%%
%%%Some rearrangement, after eliminating $x$ yields a hyperbolic family of curves, one for each value of $ct$
%%%
%%%\begin{equation}\label{eqn:relativisticElectrodynamicsA1:2}
%%%\frac{(ct'/ct)^2}{1 -\beta^2} - \frac{(x'/ct)^2}{(1-\beta^2)/\beta^2} = 1
%%%\end{equation}
%%%
%%%Considering the $ct = 0$ curves for $ct\ne 0$ we get real solutions for $x$ thus, fixing the orientation of the hyperbolas, so we see these hyperbolas cross the x-axis, and not the $ct$-axis.

\section{Problem 2.}
\subsection{Statement}

From the Lorentz transformations of space and time coordinates.

\begin{enumerate}
\item Derive the transformation of velocities.  With a particle moving with $\Bv$ in the unprimed (stationary) frame, find its velocity $\Bv'$ in the primed frame.  The primed frame is moving with some $\BV$ with respect to the unprimed one.  Make sure to finally derive the general ``addition of velocities'' equation in terms of vectors and dot products, as given in \cite{landau1971classical}.
\item Then, use the addition of velocities rule to show that: a) if $v < c$ in one frame, then $v' < c$ in any other frame.  b.) If $v = c$ in one frame, then $v' = c$ in any other frame, and c.) if $v> c$ in one frame, than $v' > c$ in any other frame.
\end{enumerate}

\subsection{Solution}
\subsubsection{Part 1.}

We need a vector form of the Lorentz transform to start with.  Let's write $\Bsigma$ for a unit vector colinear with the primed frame velocity $\BV$, so that $\BV = (\BV \cdot \Bsigma) \Bsigma$.  When our boost was in the $x$ direction, our Lorentz transformation was in terms of $x = \Bx \cdot \xcap$.  The component in the direction of the boost is now $\Bx \cdot \Bsigma$, and we have

\begin{subequations}
\begin{align}\label{eqn:relativisticElectrodynamicsA1:200}
c t' &= \gamma \left( ct - (\Bx \cdot \Bsigma) \frac{\BV \cdot \Bsigma}{c} \right) \\
\Bx' \cdot \Bsigma &= \gamma \left( \Bx \cdot \Bsigma - \frac{\BV \cdot \Bsigma}{c} c t \right) \\
\Bx' \wedge \Bsigma &= \Bx \wedge \Bsigma .
\end{align}
\end{subequations}

We can add the vector components using $\Bx = (\Bx \cdot \Bsigma) \Bsigma + (\Bx \wedge \Bsigma) \Bsigma$, leaving

\begin{subequations}
\begin{align}\label{eqn:relativisticElectrodynamicsA1:210}
c t' &= \gamma \left( ct - (\Bx \cdot \Bsigma) \frac{\BV \cdot \Bsigma}{c} \right) \\
\Bx' &= (\Bx \wedge \Bsigma) \Bsigma + \gamma \left( (\Bx \cdot \Bsigma) \Bsigma - \frac{\BV}{c} c t \right) .
\end{align}
\end{subequations}

Writing $(\Bx \wedge \Bsigma) \Bsigma = \Bx - (\Bx \cdot \Bsigma)\Bsigma$ we have for the spatial component transformation

\begin{equation}\label{eqn:relativisticElectrodynamicsA1:220}
\Bx' = \Bx + (\Bx \cdot \Bsigma) \Bsigma (\gamma - 1) - \gamma \frac{\BV}{c} c t.
\end{equation}

Now we are set to take derivatives to calculate the velocities.  This gives us
\begin{subequations}
\begin{align}\label{eqn:relativisticElectrodynamicsA1:230}
\frac{dt'}{dt} &= \gamma \left( 1 - \left( \frac{d\Bx}{dt} \cdot \Bsigma \right) \frac{\BV \cdot \Bsigma}{c^2} \right) \\
\frac{d\Bx'}{dt'} \frac{d t'}{dt} &= \frac{d\Bx}{dt} + \left(\frac{d\Bx}{dt} \cdot \Bsigma\right) \Bsigma (\gamma - 1) - \gamma \frac{\BV}{c} c .
\end{align}
\end{subequations}

Dividing this pair of equations, and using $\Bv = d\Bx/dt$, and $\Bv' = d\Bx'/dt'$, this is

\begin{equation}\label{eqn:relativisticElectrodynamicsA1:240}
\Bv' = \frac{\gamma^{-1} \Bv + (\Bv \cdot \Bsigma) \Bsigma (1 - \gamma^{-1}) - \BV}{ 1 - \left( \Bv \cdot \Bsigma \right) (\BV \cdot \Bsigma)/c^2 }.
\end{equation}

Since $\BV$ and our direction vector $\Bsigma$ are colinear, we have $(\Bv \cdot \Bsigma) (\BV \cdot \Bsigma) = \Bv \cdot \Bsigma$, and can simplify this last expression slightly

\begin{equation}\label{eqn:relativisticElectrodynamicsA1:250}
\Bv' = \frac{\gamma^{-1} \Bv + (\Bv \cdot \Bsigma) \Bsigma (1 - \gamma^{-1}) - \BV}{ 1 - \Bv \cdot \BV/c^2 }.
\end{equation}

Finally, if we are to compare to the text, we note that the inverse expression requires replacement of $\BV$ with $-\BV$ and switching $\Bv$ with $\Bv'$.  That gives us

\begin{equation}\label{eqn:relativisticElectrodynamicsA1:250i}
\Bv = \frac{\gamma^{-1} \Bv' + (\Bv' \cdot \Bsigma) \Bsigma (1 - \gamma^{-1}) + \BV}{ 1 + \Bv' \cdot \BV/c^2 }.
\end{equation}

The expression in the text is also a small velocity approximation.  For $\Abs{\BV} \ll c$, we have $\gamma^{-1} \approx 1$, and $(1 + \Bv' \cdot \BV/c^2)^{-1} \approx 1 - \Bv' \cdot \BV/c^2$.  This gives us

\begin{equation}\label{eqn:relativisticElectrodynamicsA1:250a}
\Bv \approx (\Bv' + \BV)( 1 - \Bv' \cdot \BV/c^2 ) \approx \BV + \Bv' - \Bv' (\Bv' \cdot \BV)/c^2
\end{equation}

One additional approximation was made dropping the $\BV (\Bv' \cdot \BV)/c^2$ term which is quadratic in $\BV/c$, which leave us with equation $5.3$ in the text as desired.

\subsubsection{Part 2.}

In \ref{eqn:relativisticElectrodynamicsA1:250i}, let's write $\Bv' = u \Bu$, where $\Bu$ is a unit vector, $V = \BV \cdot \Bsigma$, and $\alpha = \Bu \cdot \Bsigma$ for the direction cosine between the primed frame's direction of motion and the particle's velocity direction (also in the unprimed frame).  The stationary frame's particle velocity is then

\begin{equation}\label{eqn:relativisticElectrodynamicsA1:260}
\Bv = \frac{\gamma^{-1} u \Bu + u \alpha \Bsigma (1 - \gamma^{-1}) + V \Bsigma}{ 1 + \alpha u V/c^2 }.
\end{equation}

As a check, note that for $1 = \alpha = \Bu \cdot \Bsigma = \cos(0)$, we recover the familiar addition of velocities formula 

\begin{equation}\label{eqn:relativisticElectrodynamicsA1:260b}
\Bv = \Bu \frac{u + V}{ 1 + u V/c^2 }.
\end{equation}

We want to put \ref{eqn:relativisticElectrodynamicsA1:260} into a form that renders it more tractable for general angles too.  Factoring out the $\gamma^{-1}$ term appears to do the job, yielding

\begin{equation}\label{eqn:relativisticElectrodynamicsA1:260c}
\Bv = \frac{u \gamma^{-1} (\Bu -\alpha \Bsigma) + (u \alpha + V) \Bsigma}{ 1 + \alpha u V/c^2 }.
\end{equation}

After a bit of reduction and rearranging we can dot this with itself to calculate

\begin{equation}\label{eqn:relativisticElectrodynamicsA1:270}
\Bv^2 = \frac{V^2(1 - \alpha^2)(1 - u^2/c^2) + (u + \alpha V)^2}{ (1 + \alpha u V/c^2)^2 }
\end{equation}

Note that for $u = c$, we have $\Bv^2 = c^2$, regardless of the direction of $\BV$ with respect to the motion of the particle in the unprimed frame.  This shouldn't be surprising since this light like invariance is exactly what the Lorentz transformation is designed to maintain.  It is however slightly comforting to know that the algebra appears to be still be kosher after all this.  This also answers part (b) of this question, since we have tackled the $v = c$ case in the primed frame, and seen that the speed remains $v = c$ in the unprimed frame (and thus any frame moving at constant speed relative to another).

Observe that since $1 - \alpha^2 = \sin^2\theta$, and $u \le c$, this is positive definite as expected.  If one allowed $u > c$ in some frame, our speed could go imaginary!

For the $u < c$ and $u > c$ cases, let $x = u/c$ and $y = V/c$.  This allows \ref{eqn:relativisticElectrodynamicsA1:270} to be casted in a simpler form

\begin{equation}\label{eqn:relativisticElectrodynamicsA1:270e}
\Bv^2 = c^2 \frac{y^2 (1 - \alpha^2)(1 - x^2) + (x + \alpha y)^2}{ (1 + \alpha x y)^2 }
\end{equation}

We wish to verify that (a) given any $x \in (-1,1)$, we have $\Bv^2 < c^2$ for all $y \in (-1,1)$, $\alpha \in (-1,1)$, and (c) given any $\Abs{x} > 1$, we have $\Bv^2 > c^2$ for all $y \in (-1,1)$, $\alpha \in (-1,1)$.

Considering (a) first, this requires a demonstration that 

\begin{equation}\label{eqn:relativisticElectrodynamicsA1:280}
y^2 (1 - \alpha^2)(1 - x^2) + (x + \alpha y)^2 < (1 + \alpha x y)^2 .
\end{equation}

Expanding out the products and cancelling terms, we want to show that for (a) that if $\Abs{x},\Abs{y} < 1$ we have

\begin{equation}\label{eqn:relativisticElectrodynamicsA1:290a}
x^2 (1 - y^2) + y^2 < 1,
\end{equation}

and for (c) that if $\Abs{x} > 1$, we have for any $\Abs{y} < 1$

\begin{equation}\label{eqn:relativisticElectrodynamicsA1:290c}
x^2 (1 - y^2) + y^2 > 1.
\end{equation}

Observe that the frame velocity orientation direction cosines have completely dropped out, leaving just the (relative to $c$) velocity terms.

To get an initial feel for this function $f(x,y) = x^2 (1 - y^2) + y^2$, notice that \href{http://goo.gl/5AnNF}{when graphed} we have a bowl with a minimum (zero) at the origin, and what appears to be a uniform value of one on the boundary (case (b)).  Then provided $\Abs{y} < 1$ it appears that the function $f$ increases monotonically to a value greater than one (case (c)).  While looking at a plot is not any sort of rigorous proof, let's move on to some of the other problems for now, and return to this last loose thread later if time permits.

\section{Problem 3.}
\subsection{Statement}

A toy model of a GPS system has satellites moving in a straight line with constant velocity $V_x$ and at a constant height $h$ (measured, e.g., along the y-axis) above ``ground'' (the x-axis).  The satellites broadcast the time in their rest frame as well as their location at a time of broadcast.  Imagine a person on the ground recieves simultaneously broadcasts from two satellites, $A$ and $B$, reporting their locations $x_A'$ and $x_B'$ as well as times of broadcast (which happen to be equal), $t_A' = t_B'$.	

\begin{enumerate}
\item  Find a condition determining your position in $x$.  Evaluate it to find your deviation from the midpoint between the satellites to first order in $V_x/c$.
\item For some real numbers, note that in reality there are 24 satellites, moving with $V ~4 \text{km}/s$, a distance $R \approx 2.7 \times 10^4 \text{km}$.  Use these numbers and the result from the previous problem (assuming a flat Earth, to be sure...) to get an idea whether (special) relativisitic effects are important for the typical modern GPS accuracy of order 10 m (or less)?
\end{enumerate}

\subsection{Discussion of the non-toy model.}

It is fairly easy to find interesting info about the mechanisms that real GPS works using.  NASA has a nice \href{http://www.nasm.si.edu/gps/work.html}{How does GPS work} page \cite{nasaGPS}, and How stuff works has a nice \href{http://electronics.howstuffworks.com/gadgets/travel/gps.htm}{How GPS Receivers Work} article \cite{howStuffWorksGPS}.  Reading these one finds that the GPS clocks are actually kept synchronized.  The typical GPS reciever obviously has a clock, since we have countdown timers for time until arrival, is that clock accurate enough compared to the satellite atomic clocks to be used for the GPS location algorithm?  What is done in fact is to use the local reciever time to seed the iterative algorithms, allowing the local time to be calculated eventually with an accuracy that actually approaches that of the satellite's atomic clocks.  Some of the sources of error, like reflection of the signals, delaying them, and interference by atmosphere are also discussed in these articles.  Also interesting is that there is a table lookup of the satellites position implemented in the GPS recievers.  This table lookup is used to seed the iterative algorithms, and can be used to reduce calculation error.

Our basic GPS problem is to calcuate the intersection of a number of ``spherical'' hypersurfaces.  This is made more interesting by the fact that this is both a non-linear and an over-specified problem.  Let's consider the geometric problem to get an idea of how to set up this problem.  Suppose that we have a set of $k$ satellites, located at position $\Bp_i, i \in [1,k]$, and we know that these are located with distance $d_i$ from our position $\Bx$.

Our problem is then to find the simultaneous solution to the following set of equations

\begin{equation}\label{eqn:relativisticElectrodynamicsA1:300}
\begin{aligned}
(\Bx - \Bp_1)^2 &= d_1^2 \\
(\Bx - \Bp_2)^2 &= d_2^2 \\
\vdots \\
(\Bx - \Bk_2)^2 &= d_k^2.
\end{aligned}
\end{equation}

Observe that even if we reduce this to a one dimensional problem in a single variable $x$, we still have a non-linear system

\begin{equation}\label{eqn:relativisticElectrodynamicsA1:300b}
\begin{aligned}
0 &= (x - p_1)^2 - d_1^2 \\
0 &= (x - p_2)^2 - d_2^2 \\
\vdots \\
0 &= (x - k_2)^2 - d_k^2.
\end{aligned}
\end{equation}

We also need to be aware of the fact that each of the positions $p_i$, and the respective distances $d_i$ will in reality both have associated errors, so there is not likely any specific single value of $x$ that ``solves'' this problem, unless it is setup in a contrived and perfect fashion.  This intrinsic error, and the $k$ equations, one unknown nature of the problem (or three unknowns for spatial, or four for spatial and time position) suggests a least squares approach, but it will have to be one that also incorporates iteration.

We can setup our problem in matrix form, where we are looking for a solution to 

\begin{equation}\label{eqn:relativisticElectrodynamicsA1:310}
F(x) = 
\begin{bmatrix}
F_i(x)
\end{bmatrix}
=
\begin{bmatrix}
\Abs{\Bx - \Bp_1} - c \Abs{ t - t_1 } \\
\Abs{\Bx - \Bp_2} - c \Abs{ t - t_2 } \\
\vdots \\
\Abs{\Bx - \Bp_k} - c \Abs{ t - t_k } \\
\end{bmatrix}
= 0.
\end{equation}

We seek the spacetime event vector $x = (c t, \Bx)$ for the spatial location and the exact local time at the location of the GPS reciever.  Given any approximation of the solution, we can refine the solution using Newton's root finding method by taking partials, forming the Jacobian matrix for our function $F(x)$.  That is

\begin{equation}\label{eqn:relativisticElectrodynamicsA1:320}
F(x_0 + \Delta x) \approx F(x_0) + \evalbar{\PD{x^j}{F_i(x)}}{x_0} \Delta x^j = F(x_0) + J(x_0) \Delta x = 0
\end{equation}

This leaves us with the our least squares problem, requiring the generalized inverse to the matrix equation

\begin{equation}\label{eqn:relativisticElectrodynamicsA1:330}
x_1 = x_0 - J^\dagger (x_0) F(x_0),
\end{equation}

where 

\begin{equation}\label{eqn:relativisticElectrodynamicsA1:340}
J^\dagger = (J^\T J)^{-1} J^\T.
\end{equation}

This is a solution in the least squares sense that given $b = J x$, the norm $\Abs{ J \bar{x} - b}$ is minimized by $\bar{x} = J^\dagger b$.

This iterative method of solution, in the context of finding fitting circles and ellipses can be found discussed in deteail in \cite{gander1994least}.

\subsection{Solution. Part 1.}

For our toy model we have two satellites $A$ and $B$ both moving in the positive x-axis direction at velocity $V_x$ at height $h$.  As seen above, we do not require the velocity of the satellites to setup the problem, and could express the problem to solve as the numerical solution of the set of equations

\begin{equation}\label{eqn:relativisticElectrodynamicsA1:350}
F(x) = 
\begin{bmatrix}
\sqrt{ h^2 + (x - x_A')^2 } - c \Abs{t - t_A'} \\
\sqrt{ h^2 + (x - x_B')^2 } - c \Abs{t - t_B'} \\
\end{bmatrix} = 0.
\end{equation}

If we assume that our GPS reciever's clock is synchronized sufficiently with satellites $A$ and $B$, this single variable problem admits a closed form for one iteration of the least squares process.  However, since we are asked for a result that includes a $V_x/c$ term, we can augment our matrix equation by two additional rows, with a secondary set of data points introduced at an offset time interval.  That is

\begin{equation}\label{eqn:relativisticElectrodynamicsA1:360}
F(x) = 
\begin{bmatrix}
\sqrt{ h^2 + (x - x_A')^2 } - \Abs{c t - c t_A'} \\
\sqrt{ h^2 + (x - x_B')^2 } - \Abs{c t - c t_B'} \\
\sqrt{ h^2 + (x - x_A' - (V_x/c) c \delta t)^2 } - \Abs{c t - c t_A' - c \delta t} \\
\sqrt{ h^2 + (x - x_B' - (V_x/c) c \delta t)^2 } - \Abs{c t - c t_B' - c \delta t} \\
\end{bmatrix} = 0.
\end{equation}

With $t$, $\delta t$, $x_A'$, and $x_B'$ given, and an initial seed value for the iterative procedure assumed to be the midpoint $x_0 = (x_A' + x_B')/2$, we can calculate a first approximation to the reciever position $x_1 = x_0 + \Delta x$ using the Newton's procedure outlined above.

For this system our Jacobian elements are all differentials of the following form

\begin{equation}\label{eqn:relativisticElectrodynamicsA1:370}
\PD{x}{} \sqrt{ h^2 + (x - p)^2 } - \Delta = \frac{x-p}{\sqrt{h^2 + (x-p)^2}},
\end{equation}

so, our Jacobian is

\begin{equation}\label{eqn:relativisticElectrodynamicsA1:380}
J = 
\begin{bmatrix}
\frac{x - x_A'}{\sqrt{ h^2 + (x - x_A')^2 } } \\
\frac{x - x_B'}{\sqrt{ h^2 + (x - x_B')^2 } } \\
\frac{x - x_A' - (V_x/c) c \delta t}{\sqrt{ h^2 + (x - x_A' - (V_x/c) c \delta t)^2 } } \\
\frac{x - x_B' - (V_x/c) c \delta t}{\sqrt{ h^2 + (x - x_B' - (V_x/c) c \delta t)^2 } } \\
\end{bmatrix}
\end{equation}

Our deviation from the midpoint to first order in $V_x/c$ is

\begin{equation}\label{eqn:relativisticElectrodynamicsA1:390}
\Delta x = - \evalbar{ \frac{J^\T F}{J^T J} }{x = (x_A' + x_B')/2}
\end{equation}

To tidy this up, let

\begin{equation}\label{eqn:relativisticElectrodynamicsA1:391}
s = \inv{2} (x_B' - x_A')
\end{equation}

\begin{equation}\label{eqn:relativisticElectrodynamicsA1:392}
D = (V_x/c) c \delta t
\end{equation}

\begin{align*}
\evalbar{J^\T J}{(x_A' + x_B')/2} 
&= 
\frac{s^2}{h^2 + s^2}
+\frac{(-s)^2}{h^2 + s^2}
+\frac{(s - D)^2}{h^2 + (s -D)^2}
+\frac{(-s - D)^2}{h^2 + (-s -D)^2} \\
&= 
\frac{2 s^2}{h^2 + s^2}
+\frac{(s - D)^2}{h^2 + (s -D)^2}
+\frac{(s + D)^2}{h^2 + (s +D)^2}
\end{align*}

and

\begin{align*}
\evalbar{J^\T F}{(x_A' + x_B')/2} 
&=
\begin{bmatrix}
\frac{s}{\sqrt{ h^2 + (s)^2 } } &
\frac{-s}{\sqrt{ h^2 + (-s)^2 } } &
\frac{s - D}{\sqrt{ h^2 + (s - D)^2 } } &
\frac{-s - D}{\sqrt{ h^2 + (-s - D)^2 } } 
\end{bmatrix}
\begin{bmatrix}
\sqrt{ h^2 + (s)^2 } - \Abs{c t - c t_A'} \\
\sqrt{ h^2 + (-s)^2 } - \Abs{c t - c t_B'} \\
\sqrt{ h^2 + (s - D)^2 } - \Abs{c t - c t_A' - c \delta t} \\
\sqrt{ h^2 + (-s - D)^2 } - \Abs{c t - c t_B' - c \delta t} \\
\end{bmatrix} \\
&=
\frac{s}{\sqrt{s^2 + D^2}} \left( \Abs{ ct - c t_B' } - \Abs{ c t - c t_A'} \right) -2 D
- \frac{(s - D)\Abs{c t - c t_A' - c \delta t} }{ \sqrt{ h^2 + (s - D)^2 } } 
+ \frac{(s + D)\Abs{c t - c t_B' - c \delta t} }{ \sqrt{ h^2 + (s + D)^2 } } 
\end{align*}

The final beastly ugly result, utilizing the helper variables of \ref{eqn:relativisticElectrodynamicsA1:392}, and \ref{eqn:relativisticElectrodynamicsA1:391}, we have for the deviation from the midpoint (after one iteration of this least squares Newton's method):

\begin{equation}\label{eqn:relativisticElectrodynamicsA1:400}
\Delta x =
\frac{
   -\frac{s}{\sqrt{s^2 + D^2}} \left( \Abs{ ct - c t_B' } - \Abs{ c t - c t_A'} \right) +2 D
   + \frac{(s - D)\Abs{c t - c t_A' - c \delta t} }{ \sqrt{ h^2 + (s - D)^2 } } 
   - \frac{(s + D)\Abs{c t - c t_B' - c \delta t} }{ \sqrt{ h^2 + (s + D)^2 } } }
{
   \frac{2 s^2}{h^2 + s^2}
   +\frac{(s - D)^2}{h^2 + (s -D)^2}
   +\frac{(s + D)^2}{h^2 + (s +D)^2}
}
\end{equation}

In practice it doesn't make much sense to compute this.  You'll want to use a computer, and assuming the availability of a pre-canned SVD routine to compute the generalized inverse, the toy model wouldn't be any easier to solve than the real thing.


%@inproceedings{ ion:gnss04,
%%http://www.gpstk.org/pub/Documentation/UsersGuide/gpstk-user-reference.pdf

\subsection{Solution. Part 2.}

Now, let's bring some special relativity into the mix.  Specifically, lets look at how much the satellites have to correct their clocks to match Earth time, given their velocity relative to any point on the surface.

See hand written notes for this part of the problem.  Given all the numbers, this would be cumbersome to try to type up.

%\paragraph{Disclaimer.  Was this really what was asked for?} While the calculation of part I was a fun, it was a strictly geometrical problem, and does not involve relativity.  It does not have to since GPS clocks are kept synchronized.  It seems very plausible to imagine that a completely different approach was to have been tried.

\EndArticle

%
% Copyright � 2015 Peeter Joot.  All Rights Reserved.
% Licenced as described in the file LICENSE under the root directory of this GIT repository.
%
\documentclass[]{eliblog}

\usepackage{amsmath}
\usepackage{mathpazo}

%
% shorthand for bold symbols, convenient for vectors and matrices
%
\newcommand{\Ba}[0]{\mathbf{a}}
\newcommand{\Bb}[0]{\mathbf{b}}
\newcommand{\Bc}[0]{\mathbf{c}}
\newcommand{\Bd}[0]{\mathbf{d}}
\newcommand{\Be}[0]{\mathbf{e}}
\newcommand{\Bf}[0]{\mathbf{f}}
\newcommand{\Bg}[0]{\mathbf{g}}
\newcommand{\Bh}[0]{\mathbf{h}}
\newcommand{\Bi}[0]{\mathbf{i}}
\newcommand{\Bj}[0]{\mathbf{j}}
\newcommand{\Bk}[0]{\mathbf{k}}
\newcommand{\Bl}[0]{\mathbf{l}}
\newcommand{\Bm}[0]{\mathbf{m}}
\newcommand{\Bn}[0]{\mathbf{n}}
\newcommand{\Bo}[0]{\mathbf{o}}
\newcommand{\Bp}[0]{\mathbf{p}}
\newcommand{\Bq}[0]{\mathbf{q}}
\newcommand{\Br}[0]{\mathbf{r}}
\newcommand{\Bs}[0]{\mathbf{s}}
\newcommand{\Bt}[0]{\mathbf{t}}
\newcommand{\Bu}[0]{\mathbf{u}}
\newcommand{\Bv}[0]{\mathbf{v}}
\newcommand{\Bw}[0]{\mathbf{w}}
\newcommand{\Bx}[0]{\mathbf{x}}
\newcommand{\By}[0]{\mathbf{y}}
\newcommand{\Bz}[0]{\mathbf{z}}
\newcommand{\BA}[0]{\mathbf{A}}
\newcommand{\BB}[0]{\mathbf{B}}
\newcommand{\BC}[0]{\mathbf{C}}
\newcommand{\BD}[0]{\mathbf{D}}
\newcommand{\BE}[0]{\mathbf{E}}
\newcommand{\BF}[0]{\mathbf{F}}
\newcommand{\BG}[0]{\mathbf{G}}
\newcommand{\BH}[0]{\mathbf{H}}
\newcommand{\BI}[0]{\mathbf{I}}
\newcommand{\BJ}[0]{\mathbf{J}}
\newcommand{\BK}[0]{\mathbf{K}}
\newcommand{\BL}[0]{\mathbf{L}}
\newcommand{\BM}[0]{\mathbf{M}}
\newcommand{\BN}[0]{\mathbf{N}}
\newcommand{\BO}[0]{\mathbf{O}}
\newcommand{\BP}[0]{\mathbf{P}}
\newcommand{\BQ}[0]{\mathbf{Q}}
\newcommand{\BR}[0]{\mathbf{R}}
\newcommand{\BS}[0]{\mathbf{S}}
\newcommand{\BT}[0]{\mathbf{T}}
\newcommand{\BU}[0]{\mathbf{U}}
\newcommand{\BV}[0]{\mathbf{V}}
\newcommand{\BW}[0]{\mathbf{W}}
\newcommand{\BX}[0]{\mathbf{X}}
\newcommand{\BY}[0]{\mathbf{Y}}
\newcommand{\BZ}[0]{\mathbf{Z}}

\newcommand{\Bzero}[0]{\mathbf{0}}
\newcommand{\Btheta}[0]{\boldsymbol{\theta}}
\newcommand{\Btau}[0]{\boldsymbol{\tau}}
\newcommand{\Bomega}[0]{\boldsymbol{\omega}}

%
% shorthand for unit vectors
%
\newcommand{\acap}[0]{\hat{\Ba}}
\newcommand{\bcap}[0]{\hat{\Bb}}
\newcommand{\ccap}[0]{\hat{\Bc}}
\newcommand{\dcap}[0]{\hat{\Bd}}
\newcommand{\ecap}[0]{\hat{\Be}}
\newcommand{\fcap}[0]{\hat{\Bf}}
\newcommand{\gcap}[0]{\hat{\Bg}}
\newcommand{\hcap}[0]{\hat{\Bh}}
\newcommand{\icap}[0]{\hat{\Bi}}
\newcommand{\jcap}[0]{\hat{\Bj}}
\newcommand{\kcap}[0]{\hat{\Bk}}
\newcommand{\lcap}[0]{\hat{\Bl}}
\newcommand{\mcap}[0]{\hat{\Bm}}
\newcommand{\ncap}[0]{\hat{\Bn}}
\newcommand{\ocap}[0]{\hat{\Bo}}
\newcommand{\pcap}[0]{\hat{\Bp}}
\newcommand{\qcap}[0]{\hat{\Bq}}
\newcommand{\rcap}[0]{\hat{\Br}}
\newcommand{\scap}[0]{\hat{\Bs}}
\newcommand{\tcap}[0]{\hat{\Bt}}
\newcommand{\ucap}[0]{\hat{\Bu}}
\newcommand{\vcap}[0]{\hat{\Bv}}
\newcommand{\wcap}[0]{\hat{\Bw}}
\newcommand{\xcap}[0]{\hat{\Bx}}
\newcommand{\ycap}[0]{\hat{\By}}
\newcommand{\zcap}[0]{\hat{\Bz}}
\newcommand{\thetacap}[0]{\hat{\Btheta}}

%
% to write R^n and C^n in a distinguishable fashion.  Perhaps change this
% to the double lined characters upon figuring out how to do so.
%
\newcommand{\C}[1]{$\mathbb{C}^{#1}$}
\newcommand{\R}[1]{$\mathbb{R}^{#1}$}

%
% various generally useful helpers
%

% derivative of #1 wrt. #2:
\newcommand{\D}[2] {\frac {d#2} {d#1}}

\newcommand{\inv}[1]{\frac{1}{#1}}
\newcommand{\cross}[0]{\times}

\newcommand{\abs}[1]{\lvert{#1}\rvert}
\newcommand{\norm}[1]{\lVert{#1}\rVert}
\newcommand{\innerprod}[2]{\langle{#1}, {#2}\rangle}
\newcommand{\dotprod}[2]{{#1} \cdot {#2}}
\newcommand{\bdotprod}[2]{\left({#1} \cdot {#2}\right)}
\newcommand{\crossprod}[2]{{#1} \cross {#2}}
\newcommand{\tripleprod}[3]{\dotprod{\left(\crossprod{#1}{#2}\right)}{#3}}

\DeclareMathOperator{\Proj}{Proj}
\DeclareMathOperator{\Span}{span}
\DeclareMathOperator{\Sgn}{sgn}
\DeclareMathOperator{\Area}{Area}
\DeclareMathOperator{\Volume}{Volume}

%
% A few miscellaneous things specific to this document
%
\newcommand{\crossop}[1]{\crossprod{#1}{}}

% R2 vector.
\newcommand{\VectorTwo}[2]{
\begin{bmatrix}
 {#1} \\
 {#2}
\end{bmatrix}
}

\newcommand{\VectorN}[1]{
\begin{bmatrix}
{#1}_1 \\
{#1}_2 \\
\vdots \\
{#1}_N \\
\end{bmatrix}
}

\newcommand{\DETuvij}[4]{
\begin{vmatrix}
 {#1}_{#3} & {#1}_{#4} \\
 {#2}_{#3} & {#2}_{#4}
\end{vmatrix}
}

\newcommand{\DETuvwijk}[6]{
\begin{vmatrix}
 {#1}_{#4} & {#1}_{#5} & {#1}_{#6} \\
 {#2}_{#4} & {#2}_{#5} & {#2}_{#6} \\
 {#3}_{#4} & {#3}_{#5} & {#3}_{#6}
\end{vmatrix}
}

\newcommand{\DETuvwxijkl}[8]{
\begin{vmatrix}
 {#1}_{#5} & {#1}_{#6} & {#1}_{#7} & {#1}_{#8} \\
 {#2}_{#5} & {#2}_{#6} & {#2}_{#7} & {#2}_{#8} \\
 {#3}_{#5} & {#3}_{#6} & {#3}_{#7} & {#3}_{#8} \\
 {#4}_{#5} & {#4}_{#6} & {#4}_{#7} & {#4}_{#8} \\
\end{vmatrix}
}

%\newcommand{\DETuvwxyijklm}[10]{
%\begin{vmatrix}
% {#1}_{#6} & {#1}_{#7} & {#1}_{#8} & {#1}_{#9} & {#1}_{#10} \\
% {#2}_{#6} & {#2}_{#7} & {#2}_{#8} & {#2}_{#9} & {#2}_{#10} \\
% {#3}_{#6} & {#3}_{#7} & {#3}_{#8} & {#3}_{#9} & {#3}_{#10} \\
% {#4}_{#6} & {#4}_{#7} & {#4}_{#8} & {#4}_{#9} & {#4}_{#10} \\
% {#5}_{#6} & {#5}_{#7} & {#5}_{#8} & {#5}_{#9} & {#5}_{#10}
%\end{vmatrix}
%}

% R3 vector.
\newcommand{\VectorThree}[3]{
\begin{bmatrix}
 {#1} \\
 {#2} \\
 {#3}
\end{bmatrix}
}



\author{Peeter Joot}
\email{peeter.joot@gmail.com}

%\documentclass[]{eliblogwidescreen}

\usepackage{amsmath}
\usepackage{mathpazo}

%
% shorthand for bold symbols, convenient for vectors and matrices
%
\newcommand{\Ba}[0]{\mathbf{a}}
\newcommand{\Bb}[0]{\mathbf{b}}
\newcommand{\Bc}[0]{\mathbf{c}}
\newcommand{\Bd}[0]{\mathbf{d}}
\newcommand{\Be}[0]{\mathbf{e}}
\newcommand{\Bf}[0]{\mathbf{f}}
\newcommand{\Bg}[0]{\mathbf{g}}
\newcommand{\Bh}[0]{\mathbf{h}}
\newcommand{\Bi}[0]{\mathbf{i}}
\newcommand{\Bj}[0]{\mathbf{j}}
\newcommand{\Bk}[0]{\mathbf{k}}
\newcommand{\Bl}[0]{\mathbf{l}}
\newcommand{\Bm}[0]{\mathbf{m}}
\newcommand{\Bn}[0]{\mathbf{n}}
\newcommand{\Bo}[0]{\mathbf{o}}
\newcommand{\Bp}[0]{\mathbf{p}}
\newcommand{\Bq}[0]{\mathbf{q}}
\newcommand{\Br}[0]{\mathbf{r}}
\newcommand{\Bs}[0]{\mathbf{s}}
\newcommand{\Bt}[0]{\mathbf{t}}
\newcommand{\Bu}[0]{\mathbf{u}}
\newcommand{\Bv}[0]{\mathbf{v}}
\newcommand{\Bw}[0]{\mathbf{w}}
\newcommand{\Bx}[0]{\mathbf{x}}
\newcommand{\By}[0]{\mathbf{y}}
\newcommand{\Bz}[0]{\mathbf{z}}
\newcommand{\BA}[0]{\mathbf{A}}
\newcommand{\BB}[0]{\mathbf{B}}
\newcommand{\BC}[0]{\mathbf{C}}
\newcommand{\BD}[0]{\mathbf{D}}
\newcommand{\BE}[0]{\mathbf{E}}
\newcommand{\BF}[0]{\mathbf{F}}
\newcommand{\BG}[0]{\mathbf{G}}
\newcommand{\BH}[0]{\mathbf{H}}
\newcommand{\BI}[0]{\mathbf{I}}
\newcommand{\BJ}[0]{\mathbf{J}}
\newcommand{\BK}[0]{\mathbf{K}}
\newcommand{\BL}[0]{\mathbf{L}}
\newcommand{\BM}[0]{\mathbf{M}}
\newcommand{\BN}[0]{\mathbf{N}}
\newcommand{\BO}[0]{\mathbf{O}}
\newcommand{\BP}[0]{\mathbf{P}}
\newcommand{\BQ}[0]{\mathbf{Q}}
\newcommand{\BR}[0]{\mathbf{R}}
\newcommand{\BS}[0]{\mathbf{S}}
\newcommand{\BT}[0]{\mathbf{T}}
\newcommand{\BU}[0]{\mathbf{U}}
\newcommand{\BV}[0]{\mathbf{V}}
\newcommand{\BW}[0]{\mathbf{W}}
\newcommand{\BX}[0]{\mathbf{X}}
\newcommand{\BY}[0]{\mathbf{Y}}
\newcommand{\BZ}[0]{\mathbf{Z}}

\newcommand{\Bzero}[0]{\mathbf{0}}
\newcommand{\Btheta}[0]{\boldsymbol{\theta}}
\newcommand{\Btau}[0]{\boldsymbol{\tau}}
\newcommand{\Bomega}[0]{\boldsymbol{\omega}}

%
% shorthand for unit vectors
%
\newcommand{\acap}[0]{\hat{\Ba}}
\newcommand{\bcap}[0]{\hat{\Bb}}
\newcommand{\ccap}[0]{\hat{\Bc}}
\newcommand{\dcap}[0]{\hat{\Bd}}
\newcommand{\ecap}[0]{\hat{\Be}}
\newcommand{\fcap}[0]{\hat{\Bf}}
\newcommand{\gcap}[0]{\hat{\Bg}}
\newcommand{\hcap}[0]{\hat{\Bh}}
\newcommand{\icap}[0]{\hat{\Bi}}
\newcommand{\jcap}[0]{\hat{\Bj}}
\newcommand{\kcap}[0]{\hat{\Bk}}
\newcommand{\lcap}[0]{\hat{\Bl}}
\newcommand{\mcap}[0]{\hat{\Bm}}
\newcommand{\ncap}[0]{\hat{\Bn}}
\newcommand{\ocap}[0]{\hat{\Bo}}
\newcommand{\pcap}[0]{\hat{\Bp}}
\newcommand{\qcap}[0]{\hat{\Bq}}
\newcommand{\rcap}[0]{\hat{\Br}}
\newcommand{\scap}[0]{\hat{\Bs}}
\newcommand{\tcap}[0]{\hat{\Bt}}
\newcommand{\ucap}[0]{\hat{\Bu}}
\newcommand{\vcap}[0]{\hat{\Bv}}
\newcommand{\wcap}[0]{\hat{\Bw}}
\newcommand{\xcap}[0]{\hat{\Bx}}
\newcommand{\ycap}[0]{\hat{\By}}
\newcommand{\zcap}[0]{\hat{\Bz}}
\newcommand{\thetacap}[0]{\hat{\Btheta}}

%
% to write R^n and C^n in a distinguishable fashion.  Perhaps change this
% to the double lined characters upon figuring out how to do so.
%
\newcommand{\C}[1]{$\mathbb{C}^{#1}$}
\newcommand{\R}[1]{$\mathbb{R}^{#1}$}

%
% various generally useful helpers
%

% derivative of #1 wrt. #2:
\newcommand{\D}[2] {\frac {d#2} {d#1}}

\newcommand{\inv}[1]{\frac{1}{#1}}
\newcommand{\cross}[0]{\times}

\newcommand{\abs}[1]{\lvert{#1}\rvert}
\newcommand{\norm}[1]{\lVert{#1}\rVert}
\newcommand{\innerprod}[2]{\langle{#1}, {#2}\rangle}
\newcommand{\dotprod}[2]{{#1} \cdot {#2}}
\newcommand{\bdotprod}[2]{\left({#1} \cdot {#2}\right)}
\newcommand{\crossprod}[2]{{#1} \cross {#2}}
\newcommand{\tripleprod}[3]{\dotprod{\left(\crossprod{#1}{#2}\right)}{#3}}

\DeclareMathOperator{\Proj}{Proj}
\DeclareMathOperator{\Span}{span}
\DeclareMathOperator{\Sgn}{sgn}
\DeclareMathOperator{\Area}{Area}
\DeclareMathOperator{\Volume}{Volume}

%
% A few miscellaneous things specific to this document
%
\newcommand{\crossop}[1]{\crossprod{#1}{}}

% R2 vector.
\newcommand{\VectorTwo}[2]{
\begin{bmatrix}
 {#1} \\
 {#2}
\end{bmatrix}
}

\newcommand{\VectorN}[1]{
\begin{bmatrix}
{#1}_1 \\
{#1}_2 \\
\vdots \\
{#1}_N \\
\end{bmatrix}
}

\newcommand{\DETuvij}[4]{
\begin{vmatrix}
 {#1}_{#3} & {#1}_{#4} \\
 {#2}_{#3} & {#2}_{#4}
\end{vmatrix}
}

\newcommand{\DETuvwijk}[6]{
\begin{vmatrix}
 {#1}_{#4} & {#1}_{#5} & {#1}_{#6} \\
 {#2}_{#4} & {#2}_{#5} & {#2}_{#6} \\
 {#3}_{#4} & {#3}_{#5} & {#3}_{#6}
\end{vmatrix}
}

\newcommand{\DETuvwxijkl}[8]{
\begin{vmatrix}
 {#1}_{#5} & {#1}_{#6} & {#1}_{#7} & {#1}_{#8} \\
 {#2}_{#5} & {#2}_{#6} & {#2}_{#7} & {#2}_{#8} \\
 {#3}_{#5} & {#3}_{#6} & {#3}_{#7} & {#3}_{#8} \\
 {#4}_{#5} & {#4}_{#6} & {#4}_{#7} & {#4}_{#8} \\
\end{vmatrix}
}

%\newcommand{\DETuvwxyijklm}[10]{
%\begin{vmatrix}
% {#1}_{#6} & {#1}_{#7} & {#1}_{#8} & {#1}_{#9} & {#1}_{#10} \\
% {#2}_{#6} & {#2}_{#7} & {#2}_{#8} & {#2}_{#9} & {#2}_{#10} \\
% {#3}_{#6} & {#3}_{#7} & {#3}_{#8} & {#3}_{#9} & {#3}_{#10} \\
% {#4}_{#6} & {#4}_{#7} & {#4}_{#8} & {#4}_{#9} & {#4}_{#10} \\
% {#5}_{#6} & {#5}_{#7} & {#5}_{#8} & {#5}_{#9} & {#5}_{#10}
%\end{vmatrix}
%}

% R3 vector.
\newcommand{\VectorThree}[3]{
\begin{bmatrix}
 {#1} \\
 {#2} \\
 {#3}
\end{bmatrix}
}



\author{Peeter Joot}
\email{peeter.joot@gmail.com}


\chapter{Energy term of the Lorentz force equation.}
\label{chap:lorentzForceHamiltonian}
%\useCCL
\blogpage{http://sites.google.com/site/peeterjoot/math2011/lorentzForceHamiltonian.pdf}
\date{Feb 6, 2011}
\revisionInfo{lorentzForceHamiltonian.tex}

\beginArtWithToc
%\beginArtNoToc

\section{Motivation.}

In class this week, the Lorentz force was derived from an action (the simplest Lorentz invariant, gauge invariant, action that could be constructed)

\begin{equation}\label{eqn:lorentzForceHamiltonian:10}
S = - m c \int ds - \frac{e}{c} \int ds A^i u_i.
\end{equation}

We end up with the familiar equation, with the exception that the momentum includes the relativistically required gamma factor

\begin{equation}\label{eqn:lorentzForceHamiltonian:20}
\frac{d (\gamma m \Bv)}{dt} = e \left( \BE + \frac{\Bv}{c} \cross \BB \right).
\end{equation}

I asked what the energy term of this equation would be and was answered that we would get to it, and it could be obtained by a four vector minimization of the action which produces the Lorentz force equation of the following form

\begin{equation}\label{eqn:lorentzForceHamiltonian:30}
\frac{du^i}{d\tau} \propto e F^{ij} u_j.
\end{equation}

Let's see if we can work this out without the four-vector approach, using the action expressed with an explicit space time split, then also work it out in the four vector form and compare as a consistency check.

\section{Three vector approach.}
\subsection{The Lorentz force derivation.}

For completeness, let's work out the Lorentz force equation from the action \ref{eqn:lorentzForceHamiltonian:10}.  Parameterizing by time we have

\begin{align*}
S 
&= -m c^2 \int dt \InvGamma - e \int dt \InvGamma \gamma \left( 1, \inv{c} \Bv\right) \cdot (\phi, \BA) \\
&= -m c^2 \int dt \InvGamma - e \int dt \left( \phi - \inv{c} \BA \cdot \Bv \right)
\end{align*}

Our Lagrangian is therefore

\begin{equation}\label{eqn:lorentzForceHamiltonian:n}
\LL(\Bx, \Bv, t) = 
-m c^2 \InvGamma - e \phi(\Bx, t) + \frac{e}{c} \BA(\Bx, t) \cdot \Bv
\end{equation}

We can calculate our conjugate momentum easily enough

\begin{equation}\label{eqn:lorentzForceHamiltonian:n}
\PD{\Bv}{\LL} = \gamma m \Bv + \frac{e}{c} \BA,
\end{equation}

and for the gradient portion of the Euler-Lagrange equations we have

\begin{equation}\label{eqn:lorentzForceHamiltonian:n}
\PD{\Bx}{\LL} = -e \spacegrad \phi + e \spacegrad \left( \frac{\Bv}{c} \cdot \BA \right).
\end{equation}

Utilizing the convective derivative (i.e. chain rule in fancy clothes) 

\begin{equation}\label{eqn:lorentzForceHamiltonian:n}
\frac{d}{dt} = \Bv \cdot \spacegrad + \PD{t}{}.
\end{equation}

This gives us

\begin{equation}\label{eqn:lorentzForceHamiltonian:n}
-e \spacegrad \phi + e \spacegrad \left( \frac{\Bv}{c} \cdot \BA \right) = 
\frac{d(\gamma m \Bv)}{dt} 
+ \frac{e}{c} (\Bv \cdot \spacegrad) \BA
+ \frac{e}{c} \PD{t}{\BA},
\end{equation}

and a final bit of rearranging gives us

\begin{equation}\label{eqn:lorentzForceHamiltonian:n}
\frac{d(\gamma m \Bv)}{dt} =
e \left( -\spacegrad \phi - \inv{c} \PD{t}{\BA}
\right)
+ \frac{e}{c} \left( 
\spacegrad \left( \Bv \cdot \BA \right) - (\Bv \cdot \spacegrad) \BA
\right).
\end{equation}

The first set of derivatives we identify with the electric field $\BE$.  For the second, utilizing the \href{http://en.wikipedia.org/wiki/Triple_product#Vector_triple_product}{vector triple product identity} \cite{wiki:tripleProduct}

\begin{equation}\label{eqn:lorentzForceHamiltonian:n}
\Ba \cross (\Bb \cross \Bc) = \Bb (\Ba \cdot \Bc) - (\Ba \cdot \Bb) \Bc,
\end{equation}

we recognize as related to the magnetic field $\Bv \cross \BB = \Bv \cross (\spacegrad \cross \BA)$.

\subsection{The power (energy) term.}

When we start with an action explicity constructed with Lorentz invariance as a requirement, it is somewhat odd to end up with a result that has only the spatial vector portion of what should logically be a four vector result.  We have an equation for the particle momentum, but not one for the energy.  In tutorial Simon provided the hint of how to approach this, and asked if we had calcuated the Hamiltonian for the Lorentz force.   We had only calculated the Hamiltonian for the free particle.

Considering this, we can only actually calculate a Hamiltonian for the case where $\phi(\Bx, t) = \phi(\Bx)$ and $\BA(\Bx, t) = \BA(\Bx)$, because when the potentials have any sort of time dependence we do not have a Lagrangian that is invariant under time translation.  Returning to the derivation of the Hamiltonian conservation equation, we see that we must modify the argument slightly when there is a time dependence and get instead

\begin{equation}\label{eqn:lorentzForceHamiltonian:n}
\frac{d}{dt} \left( \PD{\Bv}{\LL} \cdot \LL - \LL \right) + \PD{t}{\LL} = 0.
\end{equation}

Only when there is no time dependence in the Lagrangian, do we have our conserved quantity, what we label as energy, or Hamiltonian.

From

\section{Four vector approach.}
\subsection{The Lorentz force derivation from invariant action.}
\subsection{Expressed explicitly in terms of the three vector fields.}
\subsubsection{The power term.}
\subsubsection{The Lorentz force terms.}

\EndArticle

%
% Copyright � 2012 Peeter Joot.  All Rights Reserved.
% Licenced as described in the file LICENSE under the root directory of this GIT repository.
%

% 
% 
%\chapter{Problem Set 2}
\label{chap:relElectroDynProblemSet2}
%\blogpage{http://sites.google.com/site/peeterjoot/math2011/relElectroDynProblemSet2.pdf}
%\date{Feb 1, 2011}

\makeproblem{Particle collision}{pr:relElectroDynProblemSet2:1}{ 

A particle of rest mass $m$ whose energy is three times its rest energy collides with an identical particle at rest.  Suppose they stick together.  Use conservation laws to find the mass of the resulting particle and its velocity.  Is its mass greater or smaller than $2m$?  Comment.

} % makeproblem

\makeanswer{pr:relElectroDynProblemSet2:1}{ 


The energy of the initially moving particle before collision is

\begin{equation}\label{eqn:relElectroDynProblemSet2:10}
\mathcal{E} = \frac{m c^2 }{\InvGamma} = 3 m c^2.
\end{equation}

Solving for the velocity we have

\begin{equation}\label{eqn:relElectroDynProblemSet2:30}
\Abs{\frac{\Bv}{c}} = \frac{2 \sqrt{2}}{3}.
\end{equation}

Our four velocity is

\begin{equation}\label{eqn:relElectroDynProblemSet2:50}
u^i
= \gamma \left( 1, \frac{\Bv}{c} \right) = ( 3, 2 \sqrt{2} ).
\end{equation}

Designate the four momentum for this particle as

\begin{equation}\label{eqn:relElectroDynProblemSet2:70}
p_{(1)}^i = m c ( 3, 2 \sqrt{2} ).
\end{equation}

For the second particle we have

\begin{equation}\label{eqn:relElectroDynProblemSet2:90}
p_{(2)}^i = m c ( 1, 0 ).
\end{equation}

Our initial and final four momentum will be equal, and our resulting velocity can only be in the direction of the initial particle.  This leaves us with

\begin{equation}\label{eqn:relElectroDynProblemSet2:1110}
\begin{aligned}
p_{(f)}^i
&= M c \inv{\sqrt{1 - \frac{\Bv_f^2}{c^2}}} \left( 1, \frac{\Bv_f}{c} \right) \\
&= m c ( 1, 0 ) + m c ( 3, 2 \sqrt{2} )  \\
&= m c ( 4, 2 \sqrt{2} ) \\
&= 4 m c \left( 1, \inv{\sqrt{2}} \right)
\end{aligned}
\end{equation}

Our final velocity is $v_f = c/\sqrt{2}$.

We have $M \gamma = 4$ for the final particle, but we have

\begin{equation}\label{eqn:relElectroDynProblemSet2:110}
\gamma = \frac{1}{\sqrt{1 - 1/2}} = \sqrt{2},
\end{equation}

so our final mass is

\begin{equation}\label{eqn:relElectroDynProblemSet2:130}
M = \frac{4}{\sqrt{2}} = 2 \sqrt{2} > 2.
\end{equation}

Relativistically, we have conservation of four-momentum, not conservation of mass, so a composite body will not necessarily have a mass measurement that is the sum of the parts.  One possible way to reconcile this statement with intuition is to define mass in terms of the four momentum

\begin{equation}\label{eqn:relElectroDynProblemSet2:130b}
m^2 = \frac{p^i p_i}{c^2},
\end{equation}

and think of it as a derived quantity, not fundamental.

}

\makeproblem{Particle in an electromagnetic field}{pr:relElectroDynProblemSet2:2}{ 

This problem has three parts


\makesubproblem{}{pr:relElectroDynProblemSet2:2a}

Express the ``normal'' (i.e. not 4-, but 3-) acceleration, equal to $\dot{\Bv}$, or a particle in terms of its velocity, $\BE$, and $\BB$, using the equation of motion of a relativistic particle in an external electromagnetic field.

\makesubproblem{}{pr:relElectroDynProblemSet2:2b}

Consider now a beam of electrons, moving along the $x$ direction with a known energy $\mathcal{E}$, entering a region with constant homogeneous $\BE$ and $\BB$ fields.  The fields are perpendicular, $\BE$ is along the $y$ direction while $\BB$ is along the $z$ direction.

\begin{enumerate}
\item
Show that by tuning the values of $\BE$ and $\BB$ it is possible to balance electric and magnetic forces so that the beam does not deviate from its original direction (and, say, hits a screen directly ahead).
\item Find a relation determining the mass of the electron using $\mathcal{E}$ and the measured values of the fields for which no deviation occurs.  Do not assume a non-relativistic limit and elucidate which part of this problem (a way to measure the mass of the electron) is affected by relativity.
\end{enumerate}

\makesubproblem{}{pr:relElectroDynProblemSet2:2c}

Solve for the motion (i.e. find the trajectories) of a relativistic charged particle in perpendicular constant and homogeneous electric and magnetic fields; do not assume $\BE = \BB$.

} % makeproblem

\makeanswer{pr:relElectroDynProblemSet2:2}{ 


\makeSubAnswer{Finding \( \dot{\Bv} \).}{pr:relElectroDynProblemSet2:2a}

With the particle's energy given by

\begin{equation}\label{eqn:relElectroDynProblemSet2:150}
\mathcal{E} = \gamma m c^2,
\end{equation}

we note that

\begin{equation}\label{eqn:relElectroDynProblemSet2:170}
\mathcal{E}\Bv = (\gamma m \Bv) c^2 = \Bp c^2.
\end{equation}

Taking derivatives we have

\begin{equation}\label{eqn:relElectroDynProblemSet2:1130}
\begin{aligned}
c^2 \ddt{\Bp} 
&= \Bv \ddt{\mathcal{E}} + \ddt{\Bv} \mathcal{E} \\
&= \Bv (e \BE \cdot \Bv) + \ddt{\Bv} \mathcal{E} \\
\end{aligned}
\end{equation}

Rearranging we have

\begin{equation}\label{eqn:relElectroDynProblemSet2:190}
\ddt{\Bv}
=
\frac{c^2 e \left( \BE + \frac{\Bv}{c} \cross \BB \right) - \Bv (e \BE \cdot \Bv) }{ \mathcal{E} }
\end{equation}

which leaves us with the desired result
\boxedEquation{eqn:relElectroDynProblemSet2:210}{
\dot{\Bv} =
\frac{e}{m} \InvGamma \left( \BE + \frac{\Bv}{c} \cross \BB - \frac{\Bv}{c} \left(\BE \cdot \frac{\Bv}{c} \right) \right)
}




\makeSubAnswer{On the energy change rate.}{pr:relElectroDynProblemSet2:2b}

Note that when the problem set was assigned, the relation

\begin{equation}\label{eqn:relElectroDynProblemSet2:230}
\ddt{\mathcal{E}} = e \BE \cdot \Bv
\end{equation}

had not been demonstrated.  To show this observe that we have

\begin{equation}\label{eqn:relElectroDynProblemSet2:1150}
\begin{aligned}
\frac{d}{dt} \mathcal{E}
&= m c^2 \frac{d\gamma}{dt} \\
&= m c^2 \frac{d}{dt} \inv{\InvGamma} \\
&= m c^2 \frac{\frac{\Bv}{c^2} \cdot \frac{d\Bv}{dt}}{\left(1 - \frac{\Bv^2}{c^2}\right)^{3/2}} \\
&= \frac{m \gamma \Bv \cdot \frac{d\Bv}{dt}}{1 - \frac{\Bv^2}{c^2}}
\end{aligned}
\end{equation}

We also have

\begin{equation}\label{eqn:relElectroDynProblemSet2:1170}
\begin{aligned}
\Bv \cdot \ddt{\Bp} 
&= \Bv \cdot \ddt{} \frac{m \Bv}{\InvGamma} \\
&= m\Bv^2 \ddt{\gamma} + m \gamma \Bv \cdot \ddt{\Bv} \\
&= m\Bv^2 \ddt{\gamma} + m c^2 \ddt{\gamma} \left( 1 - \frac{\Bv^2}{c^2} \right) \\
&= m c^2 \ddt{\gamma}.
\end{aligned}
\end{equation}

Utilizing the Lorentz force equation, we have

\begin{equation}\label{eqn:relElectroDynProblemSet2:250}
\Bv \cdot \ddt{\Bp} = e \left( \BE + \frac{\Bv}{c} \cross \BB \right) \cdot \Bv = e \BE \cdot \Bv
\end{equation}

and are able to assemble the above, and find that we have
\begin{equation}\label{eqn:relElectroDynProblemSet2:270}
\ddt{(m c^2 \gamma)} = e \BE \cdot \Bv
\end{equation}

\paragraph{2. (a). Tuning \texorpdfstring{$\BE$ and $\BB$}{E and B}}

Using our previous result with $\BE = E \ycap$ and $\BB = B \zcap$, our system of equations takes the form

\begin{equation}\label{eqn:relElectroDynProblemSet2:290}
\dot{\Bv} = \frac{e}{m} \InvGamma \left( E \ycap + \xcap \frac{v_y}{c} B - \ycap \frac{y_x} B - \frac{\Bv}{c} E \frac{v_y}{c} \right)
\end{equation}

This is really three equations, but they are coupled with the nasty $\InvGamma$ term.  However, since it is specified that the particles have a known energy $\mathcal{E}$, and that energy is

\begin{equation}\label{eqn:relElectroDynProblemSet2:310}
\mathcal{E} = \frac{ m c^2 }{\InvGamma},
\end{equation}

we can write

\begin{equation}\label{eqn:relElectroDynProblemSet2:330}
\InvGamma = \frac{ m c^2 }{\mathcal{E}}
\end{equation}

This eliminates the worst of the coupling, leaving three less hairy equations to solve

\begin{equation}\label{eqn:relElectroDynProblemSet2:350}
\begin{aligned}
\dot{v}_x &= \frac{e c^2}{\mathcal{E}} \left( \frac{v_y}{c} B - \frac{v_x v_y}{c^2} E \right) \\
\dot{v}_y &= \frac{e c^2}{\mathcal{E}} \left( E - \frac{v_x}{c} B - \frac{v_y^2}{c^2} E \right) \\
\dot{v}_z &= \frac{e c^2}{\mathcal{E}} \left( - \frac{v_y v_z}{c^2} E \right)
\end{aligned}
\end{equation}

We do not actually want to compute general solutions for these equations.  Instead we just wish to examine the constraints on $E$ and $B$ that will keep $v_y = v_z = 0$.

First off we see from the $\dot{v}_z$ equation above that if $v_y = 0$ or $v_z = 0$ initially, then $\dot{v}_z = 0$, and $v_z(t) = \text{constant} = v_z(0) = 0$.  So, if the beam is initially aligned with the x direction, it will not deviate towards the $z$ axis (in the direction of the magnetic field) at all.

Next, if we initially have $v_y = 0$, then at that point of time, our equation for $\dot{v}_x$ and $\dot{v}_y$ are respectively

\begin{equation}\label{eqn:relElectroDynProblemSet2:370}
\begin{aligned}
\dot{v}_x &= 0 \\
\dot{v}_y &= \frac{e c^2}{\mathcal{E}} \left( E - \frac{v_x}{c} B \right) 
\end{aligned}
\end{equation}

We are able to solve for the time evolution of the velocities directly

\begin{equation}\label{eqn:relElectroDynProblemSet2:390}
\begin{aligned}
v_x(t) &= \text{constant} = v_x(0) \\
v_y(t) &= \frac{e c^2}{\mathcal{E}} \left( E - \frac{v_x(0)}{c} B \right) t
\end{aligned}
\end{equation}

We can maintain zero deviation in the $y$ direction ($v_y(t) = 0$) provided we pick

\begin{equation}\label{eqn:relElectroDynProblemSet2:410}
E = \frac{v_x(0)}{c} B
\end{equation}

\subsubsection{2. (b). Finding the mass of the electron}

After measuring the fields that once adjusted produce no deviation in the $y$ and $z$ directions, our particles velocity must then be

\begin{equation}\label{eqn:relElectroDynProblemSet2:430}
\frac{v_x}{c} = \frac{E}{B}
\end{equation}

If the energy has also been measured, we have a relation between the mass from

\begin{equation}\label{eqn:relElectroDynProblemSet2:450}
\mathcal{E} = \frac{m c^2}{\sqrt{1 - v_x^2/c^2}} = \frac{ m c^2 }{ \sqrt{ 1 - E^2/B^2 }}
\end{equation}

With a slight rearrangement, our mass can then be calculated from the energy $\mathcal{E}$, and field measurements

\begin{equation}\label{eqn:relElectroDynProblemSet2:470}
m = \frac{ \mathcal{E} }{c^2} \sqrt{ 1 - E^2/B^2 }.
\end{equation}


\makeSubAnswer{Solve for the relativistic trajectory of a particle in perpendicular fields.}{pr:relElectroDynProblemSet2:2c}

Our equation to solve is

\begin{equation}\label{eqn:relElectroDynProblemSet2:490}
\dds{u^i} = \frac{e}{m c^2} F^{ij} g_{jk} u^k,
\end{equation}

where

\begin{equation}\label{eqn:relElectroDynProblemSet2:510}
\begin{aligned}
\Norm{ F^{ij} g_{jk} } 
&= 
\begin{bmatrix}
0 & -E_x & -E_y & -E_z \\
E_x & 0 & -B_z & B_y \\
E_y & B_z & 0 & -B_x \\
E_z & -B_y & B_x & 0
\end{bmatrix}
\begin{bmatrix}
1 & 0 & 0 & 0 \\
0 & -1 & 0 & 0 \\
0 & 0 & -1 & 0 \\
0 & 0 & 0 & -1 \\
\end{bmatrix}
&=
\begin{bmatrix}
0 & E_x & E_y & E_z \\
E_x & 0 & B_z & -B_y \\
E_y & -B_z & 0 & B_x \\
E_z & B_y & -B_x & 0
\end{bmatrix}.
\end{aligned}
\end{equation}

However, with the fields being perpendicular, we are free to align them with our choice of axis.  As above, let us use $\BE = E \ycap$, and $\BB = B \zcap$.  Writing $u$ for the column vector with components $u^i$ we have a matrix equation to solve

\begin{equation}\label{eqn:relElectroDynProblemSet2:530}
\dds{u} = 
\frac{ e }{m c^2}
\begin{bmatrix}
0 & 0 & E & 0 \\
0 & 0 & B & 0 \\
E & -B & 0 & 0 \\
0 & 0 & 0 & 0
\end{bmatrix} u = F u.
\end{equation}

It is simple to verify that our characteristic equation is

\begin{equation}\label{eqn:relElectroDynProblemSet2:1190}
\begin{aligned}
0 
&= \Abs{ F - \lambda I } \\
&= \begin{vmatrix}
-\lambda & 0 & E & 0 \\
0 & -\lambda & B & 0 \\
E & -B & -\lambda & 0 \\
0 & 0 & 0 & -\lambda
\end{vmatrix} \\
&= -\lambda^2 ( -\lambda^2 - B^2 + E^2 )
\end{aligned}
\end{equation}

so that our eigenvalues are

\begin{equation}\label{eqn:relElectroDynProblemSet2:550}
\lambda = 0, 0, \pm \sqrt{E^2 - B^2}.
\end{equation}

Since the fields are constant, we can diagonalize this, and solve by exponentiation.

Let 

\begin{equation}\label{eqn:relElectroDynProblemSet2:570}
D = \sqrt{E^2 - B^2}.
\end{equation}

To solve for the eigenvector $e_D$ for $\lambda = D$ we need solutions to

\begin{equation}\label{eqn:relElectroDynProblemSet2:590}
\begin{bmatrix}
-D & 0 & E & 0 \\
0 & -D & B & 0 \\
E & -B & -D & 0 \\
0 & 0 & 0 & -D
\end{bmatrix} 
\begin{bmatrix} 
a \\
b \\
c \\
d
\end{bmatrix} 
 = 0,
\end{equation}

and it is straightforward to compute

\begin{equation}\label{eqn:relElectroDynProblemSet2:610}
e_D = 
\inv{\sqrt{2}E}
\begin{bmatrix} 
E \\
B \\
D \\
0
\end{bmatrix}.
\end{equation}

Similarly for the $\lambda = -D$ eigenvector $e_{-D}$ we wish to solve

\begin{equation}\label{eqn:relElectroDynProblemSet2:630}
\begin{bmatrix}
D & 0 & E & 0 \\
0 & D & B & 0 \\
E & -B & D & 0 \\
0 & 0 & 0 & D
\end{bmatrix} 
\begin{bmatrix} 
a \\
b \\
c \\
d
\end{bmatrix} 
 = 0,
\end{equation}

and find that

\begin{equation}\label{eqn:relElectroDynProblemSet2:650}
e_{-D} = 
\inv{\sqrt{2}E}
\begin{bmatrix} 
E \\
B \\
-D \\
0
\end{bmatrix}.
\end{equation}

We can also pick orthonormal eigenvectors for the degenerate zero eigenvalues from the null space of the matrix

\begin{equation}\label{eqn:relElectroDynProblemSet2:670}
\begin{bmatrix}
0 & 0 & E & 0 \\
0 & 0 & B & 0 \\
E & -B & 0 & 0 \\
0 & 0 & 0 & 0
\end{bmatrix}
\end{equation}

By inspection, two such eigenvectors are 
\begin{equation}\label{eqn:relElectroDynProblemSet2:690}
\inv{\sqrt{E^2 + B^2}}
\begin{bmatrix} 
B \\
E \\
0 \\
0 
\end{bmatrix},
\begin{bmatrix} 
0 \\
0 \\
0 \\
1 
\end{bmatrix}.
\end{equation}

Unfortunately, the first is not generally orthonormal to either of $e_{\pm D}$, so our similarity transformation matrix is not invertible by Hermitian transposition.  Regardless, we are now well on track to putting the matrix equation we wish to solve into a much simpler form.  With

\begin{equation}\label{eqn:relElectroDynProblemSet2:710}
S =
\begin{bmatrix}
\inv{\sqrt{2}E}
\begin{bmatrix} 
E \\
B \\
D \\
0
\end{bmatrix} 
&
\inv{\sqrt{2}E}
\begin{bmatrix} 
E \\
B \\
-D \\
0
\end{bmatrix} &
\inv{\sqrt{E^2 + B^2}}
\begin{bmatrix} 
B \\
E \\
0 \\
0 
\end{bmatrix} &
\begin{bmatrix} 
0 \\
0 \\
0 \\
1 
\end{bmatrix}
\end{bmatrix},
\end{equation}

and 

\begin{equation}\label{eqn:relElectroDynProblemSet2:730}
\Sigma = 
\begin{bmatrix}
D & 0 & 0 & 0 \\
0 & -D & 0 & 0 \\
0 & 0 & 0 & 0 \\
0 & 0 & 0 & 0 \\
\end{bmatrix}
\end{equation}

observe that our Lorentz force equation can now be written

\begin{equation}\label{eqn:relElectroDynProblemSet2:750}
\dds{u} = \frac{e}{m c^2} S \Sigma S^{-1} u.
\end{equation}

This we can rearrange, leaving us with a diagonal system that has a trivial solution

\begin{equation}\label{eqn:relElectroDynProblemSet2:770}
\dds{} (S^{-1} u) = \frac{e}{m c^2} \Sigma (S^{-1} u).
\end{equation}

Let us write

\begin{equation}\label{eqn:relElectroDynProblemSet2:790}
v = S^{-1} u,
\end{equation}

and introduce a sort of proper distance wave number

\begin{equation}\label{eqn:relElectroDynProblemSet2:810}
k = \frac{e \sqrt{E^2 - B^2}}{m c^2}.
\end{equation}

With this the Lorentz force equation is left in the form

\begin{equation}\label{eqn:relElectroDynProblemSet2:830}
\dds{v} = 
\begin{bmatrix}
k & 0 & 0 & 0 \\
0 & -k & 0 & 0 \\
0 & 0 & 0 & 0 \\
0 & 0 & 0 & 0 \\
\end{bmatrix} v.
\end{equation}

Integrating once, our solution is

\begin{equation}\label{eqn:relElectroDynProblemSet2:850}
v(s) = 
\begin{bmatrix}
e^{ks} & 0 & 0 & 0 \\
0 & e^{-ks} & 0 & 0 \\
0 & 0 & 1 & 0 \\
0 & 0 & 0 & 1 \\
\end{bmatrix} v(s=0)
\end{equation}

Our proper velocity is thus given by

\begin{equation}\label{eqn:relElectroDynProblemSet2:870}
u = \dds{X} = S 
\begin{bmatrix}
e^{ks} & 0 & 0 & 0 \\
0 & e^{-ks} & 0 & 0 \\
0 & 0 & 1 & 0 \\
0 & 0 & 0 & 1 \\
\end{bmatrix} S^{-1} u(s=0).
\end{equation}

We can integrate once more for our trajectory, parametrized by proper distance on the worldline of the particle.  That is

\begin{equation}\label{eqn:relElectroDynProblemSet2:890}
X(s) - X(0) 
= S \left( \int_{s'=0}^s 
ds'
\begin{bmatrix}
e^{ks'} & 0 & 0 & 0 \\
0 & e^{-ks'} & 0 & 0 \\
0 & 0 & 1 & 0 \\
0 & 0 & 0 & 1 \\
\end{bmatrix} \right) S^{-1} u(s=0).
\end{equation}

With $u(0) = \gamma_0 (1, \Bv_0/c)$, and $X = (c t_0, \Bx_0)$, plus the defining relations \eqnref{eqn:relElectroDynProblemSet2:710}, and \eqnref{eqn:relElectroDynProblemSet2:810} our parametric equation for the trajectory is fully specified

\begin{equation}\label{eqn:relElectroDynProblemSet2:910}
\begin{aligned}
&\begin{bmatrix}
c t(s) \\
\Bx^\T(s)
\end{bmatrix}
- 
\begin{bmatrix}
c t_0 \\
\Bx_0^\T
\end{bmatrix} \\
&= S 
\begin{bmatrix}
\inv{k}(e^{ks} -1) & 0 & 0 & 0 \\
0 & -\inv{k}(e^{-ks} -1) & 0 & 0 \\
0 & 0 & s & 0 \\
0 & 0 & 0 & s \\
\end{bmatrix} S^{-1} \inv{\sqrt{1 - (\Bv_0)^2/c^2}}
\begin{bmatrix}
1 \\
\Bv_0^\T/c
\end{bmatrix}.
\end{aligned}
\end{equation}

Observe that for the case $E^2 > B^2$, our value $k$ is real, so the solution is entirely composed of linear combinations of the hyperbolic functions $\cosh(k s)$ and $\sinh(ks)$.  However, for the $E^2 < B^2$ case where our eigenvalues are purely imaginary, the constant $k$ is also purely imaginary (and our eigenvectors $e_{\pm D}$ are complex).  In that case, we can take the real part of this equation, and will be left with a solution that is formed of linear combinations of $\sin(ks)$ and $\cos(ks)$ terms.  The $E = B$ case would have to be handled separately, and this is done in depth in the text, so there is little value repeating it here.

}

\makeproblem{Transformation of fields.}{pr:relElectroDynProblemSet2:3}{ 

In class, we introduced the 4-vector potential $A^i$ and its transformation law under Lorentz transformations.  While we have not yet discussed how $\BE$ and $\BB$ transform, knowing how $A^i$ transforms is enough to solve some concrete problems.  Suppose in one (unprimed) frame there is a charge at rest, which creates an electrostatic field: $A^0 = \phi = \frac{q}{r}, \BA = 0$.


\makesubproblem{}{pr:relElectroDynProblemSet2:3a}

Find the values of $\BE$ and $\BB$ in this frame.

\makesubproblem{}{pr:relElectroDynProblemSet2:3b}

Consider now the same field in a (primed) frame moving in the $x$-direction with velocity $v$.  Using the transformation law of the vector potential, find ${A^i}'$ in the primed frame.

\makesubproblem{}{pr:relElectroDynProblemSet2:3c}

Use the relations between electric and magnetic field strengths and vector potential (valid in every frame) to find the electric and magnetic fields in the primed frame (i.e. find the electromagnetic field of a moving charge).  Sketch the lines of constant electric and magnetic field and comment on the result.

} % makeproblem

\makeanswer{pr:relElectroDynProblemSet2:3}{ 
\makeSubAnswer{}{pr:relElectroDynProblemSet2:3a}

In the unprimed frame we have

\begin{equation}\label{eqn:relElectroDynProblemSet2:1210}
\begin{aligned}
\BE 
&= - \spacegrad \phi - \inv{c} \PD{t}{\BA} \\
&= -\spacegrad \phi \\
&= - \rcap q \partial_r (1/r) \\
&= \rcap \frac{q}{r^2},
\end{aligned}
\end{equation}

and
\begin{equation}\label{eqn:relElectroDynProblemSet2:1230}
\begin{aligned}
\BB = \spacegrad \cross \BA = 0
\end{aligned}
\end{equation}

\makeSubAnswer{}{pr:relElectroDynProblemSet2:3b}

The coordinates in the moving frame, assuming the frames are overlapping at $t=0$, are related to the unprimed coordinates by

\begin{equation}\label{eqn:relElectroDynProblemSet2:930}
\begin{bmatrix}
ct' \\
x' \\
y' \\
z'
\end{bmatrix}
=
\begin{bmatrix}
\gamma & -\gamma \beta & 0 & 0 \\
-\gamma \beta & \gamma & 0 & 0 \\
0 & 0 & 1 & 0 \\
0 & 0 & 0 & 1 
\end{bmatrix}
\begin{bmatrix}
ct \\
x \\
y \\
z
\end{bmatrix}
\end{equation}

Our four vector potential also transforms in the same fashion, and we have

\begin{equation}\label{eqn:relElectroDynProblemSet2:950}
\begin{bmatrix}
\phi' \\
A_x' \\
A_y' \\
A_z' \\
\end{bmatrix}
=
\begin{bmatrix}
\gamma & -\gamma \beta & 0 & 0 \\
-\gamma \beta & \gamma & 0 & 0 \\
0 & 0 & 1 & 0 \\
0 & 0 & 0 & 1 
\end{bmatrix}
\begin{bmatrix}
\phi \\
0 \\
0 \\
0 \\
\end{bmatrix}
= \gamma \phi ( 1, -\beta, 0, 0 )
\end{equation}

So in the primed frame we have
\begin{equation}\label{eqn:relElectroDynProblemSet2:970}
\begin{aligned}
\phi' &= \gamma \frac{q}{r} \\
A_x' &= -\gamma \beta \frac{q}{r} \\
A_y' &= 0 \\
A_z' &= 0 
\end{aligned}
\end{equation}


\makeSubAnswer{}{pr:relElectroDynProblemSet2:3c}
In the primed frame our electric and magnetic fields are

\begin{equation}\label{eqn:relElectroDynProblemSet2:990}
\begin{aligned}
\BE' &= - \spacegrad' \phi' - \inv{c} \PD{t'}{\BA'} \\
\BB' &= \spacegrad' \cross \BA'
\end{aligned}
\end{equation}

We have $\phi'$ and $\BA'$ expressed in terms of the unprimed coordinates, so need to calculate the transformation of the gradient and time partial too.  These partials transform as

\begin{equation}\label{eqn:relElectroDynProblemSet2:1010}
\begin{aligned}
\PD{c t'}{} &= \PD{ct'}{ct} \PD{ct}{} + \PD{ct'}{x} \PD{x}{} \\
\PD{x'}{} &= \PD{x'}{ct} \PD{ct}{} + \PD{x'}{x} \PD{x}{} \\
\PD{y'}{} &= \PD{y}{} \\
\PD{z'}{} &= \PD{z}{}
\end{aligned}
\end{equation}

Utilizing the inverse transformation 

\begin{equation}\label{eqn:relElectroDynProblemSet2:1030}
\begin{bmatrix}
ct \\
x \\
y \\
z
\end{bmatrix}
=
\begin{bmatrix}
\gamma & \gamma \beta & 0 & 0 \\
\gamma \beta & \gamma & 0 & 0 \\
0 & 0 & 1 & 0 \\
0 & 0 & 0 & 1 
\end{bmatrix}
\begin{bmatrix}
ct' \\
x' \\
y' \\
z'
\end{bmatrix}
\end{equation}

we have
\begin{equation}\label{eqn:relElectroDynProblemSet2:1050}
\begin{aligned}
\PD{c t'}{} &= \gamma \PD{ct}{} + \gamma \beta \PD{x}{} \\
\PD{x'}{} &= \gamma \beta \PD{ct}{} + \gamma \PD{x}{} \\
\PD{y'}{} &= \PD{y}{} \\
\PD{z'}{} &= \PD{z}{}
\end{aligned}
\end{equation}

Since neither $\phi'$ nor $\BA'$ have time dependence, we have for electric field in the primed frame

\begin{equation}\label{eqn:relElectroDynProblemSet2:1250}
\begin{aligned}
\BE' 
&= -\spacegrad' \phi' - \inv{c} \PD{t'}{\BA'} \\
&= 
-\left( \gamma \PD{x}{}, \PD{y}{}, \PD{z}{} \right) \phi'
- \gamma \beta \PD{x}{\BA'} \\
&= 
-\left( \gamma \PD{x}{}, \PD{y}{}, \PD{z}{} \right) \gamma \frac{q}{r}
- \gamma \beta \PD{x}{} \left( -\gamma \beta \frac{q}{r}, 0, 0 \right) \\
&= -q \left( \gamma^2 ( 1 - \beta^2 ) \PD{x}{}, \gamma \PD{y}{}, \gamma \PD{z}{} \right) \inv{r} \\
&= -q \left( \PD{x}{}, \gamma \PD{y}{}, \gamma \PD{z}{} \right) \inv{r}
\end{aligned}
\end{equation}

Our electric field in the primed frame is thus
\begin{equation}\label{eqn:relElectroDynProblemSet2:1070}
\BE' = \frac{q}{r^3} \left( x, \gamma y, \gamma z \right)
\end{equation}

Now for the magnetic field.  We want

\begin{equation}\label{eqn:relElectroDynProblemSet2:1270}
\begin{aligned}
\BB' 
&= 
\begin{vmatrix}
\xcap & \ycap & \zcap \\
\partial_{x'} & \partial_{y'} & \partial_{z'} \\
-\gamma \beta q/r & 0 & 0
\end{vmatrix} \\
&=
\left( 0, \partial_{z'}, -\partial_{y'} \right) \frac{-\gamma \beta q}{r} \\
\end{aligned}
\end{equation}

\begin{equation}\label{eqn:relElectroDynProblemSet2:1090}
\BB'
=
\frac{q \gamma \beta}{r^3} \left( 0, -z, y \right)
\end{equation}

FIXME: sketch and comment.

\paragraph{Notes on grading of my solution}

I lost two marks for not reducing my solution for the trajectory in \eqnref{eqn:relElectroDynProblemSet2:910} to $x(t), y(t)$ or $x(y)$ form.  That is difficult in the form that I solved this for arbitrary initial conditions (this is easy for $u^i = (1, 0, 0, 0)$ when $\BB = 0$).  I will be curious to see the Professor's approach later.  

FIXME: I had expanded out the trajectory in the way that appears to have been desired on paper for the special case above.  Re-do this and include it here (at least as a check of my final result since I switched the orientation of the fields when I typed it up).  Also include a similar special case expansion for the case where the invariant $E^2 - B^2$ is negative.

}




%\documentclass[]{eliblog}
%\usepackage{color}
%\usepackage{txfonts} % for xi
%\usepackage{amsmath}
\usepackage{mathpazo}

%
% shorthand for bold symbols, convenient for vectors and matrices
%
\newcommand{\Ba}[0]{\mathbf{a}}
\newcommand{\Bb}[0]{\mathbf{b}}
\newcommand{\Bc}[0]{\mathbf{c}}
\newcommand{\Bd}[0]{\mathbf{d}}
\newcommand{\Be}[0]{\mathbf{e}}
\newcommand{\Bf}[0]{\mathbf{f}}
\newcommand{\Bg}[0]{\mathbf{g}}
\newcommand{\Bh}[0]{\mathbf{h}}
\newcommand{\Bi}[0]{\mathbf{i}}
\newcommand{\Bj}[0]{\mathbf{j}}
\newcommand{\Bk}[0]{\mathbf{k}}
\newcommand{\Bl}[0]{\mathbf{l}}
\newcommand{\Bm}[0]{\mathbf{m}}
\newcommand{\Bn}[0]{\mathbf{n}}
\newcommand{\Bo}[0]{\mathbf{o}}
\newcommand{\Bp}[0]{\mathbf{p}}
\newcommand{\Bq}[0]{\mathbf{q}}
\newcommand{\Br}[0]{\mathbf{r}}
\newcommand{\Bs}[0]{\mathbf{s}}
\newcommand{\Bt}[0]{\mathbf{t}}
\newcommand{\Bu}[0]{\mathbf{u}}
\newcommand{\Bv}[0]{\mathbf{v}}
\newcommand{\Bw}[0]{\mathbf{w}}
\newcommand{\Bx}[0]{\mathbf{x}}
\newcommand{\By}[0]{\mathbf{y}}
\newcommand{\Bz}[0]{\mathbf{z}}
\newcommand{\BA}[0]{\mathbf{A}}
\newcommand{\BB}[0]{\mathbf{B}}
\newcommand{\BC}[0]{\mathbf{C}}
\newcommand{\BD}[0]{\mathbf{D}}
\newcommand{\BE}[0]{\mathbf{E}}
\newcommand{\BF}[0]{\mathbf{F}}
\newcommand{\BG}[0]{\mathbf{G}}
\newcommand{\BH}[0]{\mathbf{H}}
\newcommand{\BI}[0]{\mathbf{I}}
\newcommand{\BJ}[0]{\mathbf{J}}
\newcommand{\BK}[0]{\mathbf{K}}
\newcommand{\BL}[0]{\mathbf{L}}
\newcommand{\BM}[0]{\mathbf{M}}
\newcommand{\BN}[0]{\mathbf{N}}
\newcommand{\BO}[0]{\mathbf{O}}
\newcommand{\BP}[0]{\mathbf{P}}
\newcommand{\BQ}[0]{\mathbf{Q}}
\newcommand{\BR}[0]{\mathbf{R}}
\newcommand{\BS}[0]{\mathbf{S}}
\newcommand{\BT}[0]{\mathbf{T}}
\newcommand{\BU}[0]{\mathbf{U}}
\newcommand{\BV}[0]{\mathbf{V}}
\newcommand{\BW}[0]{\mathbf{W}}
\newcommand{\BX}[0]{\mathbf{X}}
\newcommand{\BY}[0]{\mathbf{Y}}
\newcommand{\BZ}[0]{\mathbf{Z}}

\newcommand{\Bzero}[0]{\mathbf{0}}
\newcommand{\Btheta}[0]{\boldsymbol{\theta}}
\newcommand{\Btau}[0]{\boldsymbol{\tau}}
\newcommand{\Bomega}[0]{\boldsymbol{\omega}}

%
% shorthand for unit vectors
%
\newcommand{\acap}[0]{\hat{\Ba}}
\newcommand{\bcap}[0]{\hat{\Bb}}
\newcommand{\ccap}[0]{\hat{\Bc}}
\newcommand{\dcap}[0]{\hat{\Bd}}
\newcommand{\ecap}[0]{\hat{\Be}}
\newcommand{\fcap}[0]{\hat{\Bf}}
\newcommand{\gcap}[0]{\hat{\Bg}}
\newcommand{\hcap}[0]{\hat{\Bh}}
\newcommand{\icap}[0]{\hat{\Bi}}
\newcommand{\jcap}[0]{\hat{\Bj}}
\newcommand{\kcap}[0]{\hat{\Bk}}
\newcommand{\lcap}[0]{\hat{\Bl}}
\newcommand{\mcap}[0]{\hat{\Bm}}
\newcommand{\ncap}[0]{\hat{\Bn}}
\newcommand{\ocap}[0]{\hat{\Bo}}
\newcommand{\pcap}[0]{\hat{\Bp}}
\newcommand{\qcap}[0]{\hat{\Bq}}
\newcommand{\rcap}[0]{\hat{\Br}}
\newcommand{\scap}[0]{\hat{\Bs}}
\newcommand{\tcap}[0]{\hat{\Bt}}
\newcommand{\ucap}[0]{\hat{\Bu}}
\newcommand{\vcap}[0]{\hat{\Bv}}
\newcommand{\wcap}[0]{\hat{\Bw}}
\newcommand{\xcap}[0]{\hat{\Bx}}
\newcommand{\ycap}[0]{\hat{\By}}
\newcommand{\zcap}[0]{\hat{\Bz}}
\newcommand{\thetacap}[0]{\hat{\Btheta}}

%
% to write R^n and C^n in a distinguishable fashion.  Perhaps change this
% to the double lined characters upon figuring out how to do so.
%
\newcommand{\C}[1]{$\mathbb{C}^{#1}$}
\newcommand{\R}[1]{$\mathbb{R}^{#1}$}

%
% various generally useful helpers
%

% derivative of #1 wrt. #2:
\newcommand{\D}[2] {\frac {d#2} {d#1}}

\newcommand{\inv}[1]{\frac{1}{#1}}
\newcommand{\cross}[0]{\times}

\newcommand{\abs}[1]{\lvert{#1}\rvert}
\newcommand{\norm}[1]{\lVert{#1}\rVert}
\newcommand{\innerprod}[2]{\langle{#1}, {#2}\rangle}
\newcommand{\dotprod}[2]{{#1} \cdot {#2}}
\newcommand{\bdotprod}[2]{\left({#1} \cdot {#2}\right)}
\newcommand{\crossprod}[2]{{#1} \cross {#2}}
\newcommand{\tripleprod}[3]{\dotprod{\left(\crossprod{#1}{#2}\right)}{#3}}

\DeclareMathOperator{\Proj}{Proj}
\DeclareMathOperator{\Span}{span}
\DeclareMathOperator{\Sgn}{sgn}
\DeclareMathOperator{\Area}{Area}
\DeclareMathOperator{\Volume}{Volume}

%
% A few miscellaneous things specific to this document
%
\newcommand{\crossop}[1]{\crossprod{#1}{}}

% R2 vector.
\newcommand{\VectorTwo}[2]{
\begin{bmatrix}
 {#1} \\
 {#2}
\end{bmatrix}
}

\newcommand{\VectorN}[1]{
\begin{bmatrix}
{#1}_1 \\
{#1}_2 \\
\vdots \\
{#1}_N \\
\end{bmatrix}
}

\newcommand{\DETuvij}[4]{
\begin{vmatrix}
 {#1}_{#3} & {#1}_{#4} \\
 {#2}_{#3} & {#2}_{#4}
\end{vmatrix}
}

\newcommand{\DETuvwijk}[6]{
\begin{vmatrix}
 {#1}_{#4} & {#1}_{#5} & {#1}_{#6} \\
 {#2}_{#4} & {#2}_{#5} & {#2}_{#6} \\
 {#3}_{#4} & {#3}_{#5} & {#3}_{#6}
\end{vmatrix}
}

\newcommand{\DETuvwxijkl}[8]{
\begin{vmatrix}
 {#1}_{#5} & {#1}_{#6} & {#1}_{#7} & {#1}_{#8} \\
 {#2}_{#5} & {#2}_{#6} & {#2}_{#7} & {#2}_{#8} \\
 {#3}_{#5} & {#3}_{#6} & {#3}_{#7} & {#3}_{#8} \\
 {#4}_{#5} & {#4}_{#6} & {#4}_{#7} & {#4}_{#8} \\
\end{vmatrix}
}

%\newcommand{\DETuvwxyijklm}[10]{
%\begin{vmatrix}
% {#1}_{#6} & {#1}_{#7} & {#1}_{#8} & {#1}_{#9} & {#1}_{#10} \\
% {#2}_{#6} & {#2}_{#7} & {#2}_{#8} & {#2}_{#9} & {#2}_{#10} \\
% {#3}_{#6} & {#3}_{#7} & {#3}_{#8} & {#3}_{#9} & {#3}_{#10} \\
% {#4}_{#6} & {#4}_{#7} & {#4}_{#8} & {#4}_{#9} & {#4}_{#10} \\
% {#5}_{#6} & {#5}_{#7} & {#5}_{#8} & {#5}_{#9} & {#5}_{#10}
%\end{vmatrix}
%}

% R3 vector.
\newcommand{\VectorThree}[3]{
\begin{bmatrix}
 {#1} \\
 {#2} \\
 {#3}
\end{bmatrix}
}



%
% Copyright � 2015 Peeter Joot.  All Rights Reserved.
% Licenced as described in the file LICENSE under the root directory of this GIT repository.
%
\documentclass[]{eliblog}

\usepackage{amsmath}
\usepackage{mathpazo}

%
% shorthand for bold symbols, convenient for vectors and matrices
%
\newcommand{\Ba}[0]{\mathbf{a}}
\newcommand{\Bb}[0]{\mathbf{b}}
\newcommand{\Bc}[0]{\mathbf{c}}
\newcommand{\Bd}[0]{\mathbf{d}}
\newcommand{\Be}[0]{\mathbf{e}}
\newcommand{\Bf}[0]{\mathbf{f}}
\newcommand{\Bg}[0]{\mathbf{g}}
\newcommand{\Bh}[0]{\mathbf{h}}
\newcommand{\Bi}[0]{\mathbf{i}}
\newcommand{\Bj}[0]{\mathbf{j}}
\newcommand{\Bk}[0]{\mathbf{k}}
\newcommand{\Bl}[0]{\mathbf{l}}
\newcommand{\Bm}[0]{\mathbf{m}}
\newcommand{\Bn}[0]{\mathbf{n}}
\newcommand{\Bo}[0]{\mathbf{o}}
\newcommand{\Bp}[0]{\mathbf{p}}
\newcommand{\Bq}[0]{\mathbf{q}}
\newcommand{\Br}[0]{\mathbf{r}}
\newcommand{\Bs}[0]{\mathbf{s}}
\newcommand{\Bt}[0]{\mathbf{t}}
\newcommand{\Bu}[0]{\mathbf{u}}
\newcommand{\Bv}[0]{\mathbf{v}}
\newcommand{\Bw}[0]{\mathbf{w}}
\newcommand{\Bx}[0]{\mathbf{x}}
\newcommand{\By}[0]{\mathbf{y}}
\newcommand{\Bz}[0]{\mathbf{z}}
\newcommand{\BA}[0]{\mathbf{A}}
\newcommand{\BB}[0]{\mathbf{B}}
\newcommand{\BC}[0]{\mathbf{C}}
\newcommand{\BD}[0]{\mathbf{D}}
\newcommand{\BE}[0]{\mathbf{E}}
\newcommand{\BF}[0]{\mathbf{F}}
\newcommand{\BG}[0]{\mathbf{G}}
\newcommand{\BH}[0]{\mathbf{H}}
\newcommand{\BI}[0]{\mathbf{I}}
\newcommand{\BJ}[0]{\mathbf{J}}
\newcommand{\BK}[0]{\mathbf{K}}
\newcommand{\BL}[0]{\mathbf{L}}
\newcommand{\BM}[0]{\mathbf{M}}
\newcommand{\BN}[0]{\mathbf{N}}
\newcommand{\BO}[0]{\mathbf{O}}
\newcommand{\BP}[0]{\mathbf{P}}
\newcommand{\BQ}[0]{\mathbf{Q}}
\newcommand{\BR}[0]{\mathbf{R}}
\newcommand{\BS}[0]{\mathbf{S}}
\newcommand{\BT}[0]{\mathbf{T}}
\newcommand{\BU}[0]{\mathbf{U}}
\newcommand{\BV}[0]{\mathbf{V}}
\newcommand{\BW}[0]{\mathbf{W}}
\newcommand{\BX}[0]{\mathbf{X}}
\newcommand{\BY}[0]{\mathbf{Y}}
\newcommand{\BZ}[0]{\mathbf{Z}}

\newcommand{\Bzero}[0]{\mathbf{0}}
\newcommand{\Btheta}[0]{\boldsymbol{\theta}}
\newcommand{\Btau}[0]{\boldsymbol{\tau}}
\newcommand{\Bomega}[0]{\boldsymbol{\omega}}

%
% shorthand for unit vectors
%
\newcommand{\acap}[0]{\hat{\Ba}}
\newcommand{\bcap}[0]{\hat{\Bb}}
\newcommand{\ccap}[0]{\hat{\Bc}}
\newcommand{\dcap}[0]{\hat{\Bd}}
\newcommand{\ecap}[0]{\hat{\Be}}
\newcommand{\fcap}[0]{\hat{\Bf}}
\newcommand{\gcap}[0]{\hat{\Bg}}
\newcommand{\hcap}[0]{\hat{\Bh}}
\newcommand{\icap}[0]{\hat{\Bi}}
\newcommand{\jcap}[0]{\hat{\Bj}}
\newcommand{\kcap}[0]{\hat{\Bk}}
\newcommand{\lcap}[0]{\hat{\Bl}}
\newcommand{\mcap}[0]{\hat{\Bm}}
\newcommand{\ncap}[0]{\hat{\Bn}}
\newcommand{\ocap}[0]{\hat{\Bo}}
\newcommand{\pcap}[0]{\hat{\Bp}}
\newcommand{\qcap}[0]{\hat{\Bq}}
\newcommand{\rcap}[0]{\hat{\Br}}
\newcommand{\scap}[0]{\hat{\Bs}}
\newcommand{\tcap}[0]{\hat{\Bt}}
\newcommand{\ucap}[0]{\hat{\Bu}}
\newcommand{\vcap}[0]{\hat{\Bv}}
\newcommand{\wcap}[0]{\hat{\Bw}}
\newcommand{\xcap}[0]{\hat{\Bx}}
\newcommand{\ycap}[0]{\hat{\By}}
\newcommand{\zcap}[0]{\hat{\Bz}}
\newcommand{\thetacap}[0]{\hat{\Btheta}}

%
% to write R^n and C^n in a distinguishable fashion.  Perhaps change this
% to the double lined characters upon figuring out how to do so.
%
\newcommand{\C}[1]{$\mathbb{C}^{#1}$}
\newcommand{\R}[1]{$\mathbb{R}^{#1}$}

%
% various generally useful helpers
%

% derivative of #1 wrt. #2:
\newcommand{\D}[2] {\frac {d#2} {d#1}}

\newcommand{\inv}[1]{\frac{1}{#1}}
\newcommand{\cross}[0]{\times}

\newcommand{\abs}[1]{\lvert{#1}\rvert}
\newcommand{\norm}[1]{\lVert{#1}\rVert}
\newcommand{\innerprod}[2]{\langle{#1}, {#2}\rangle}
\newcommand{\dotprod}[2]{{#1} \cdot {#2}}
\newcommand{\bdotprod}[2]{\left({#1} \cdot {#2}\right)}
\newcommand{\crossprod}[2]{{#1} \cross {#2}}
\newcommand{\tripleprod}[3]{\dotprod{\left(\crossprod{#1}{#2}\right)}{#3}}

\DeclareMathOperator{\Proj}{Proj}
\DeclareMathOperator{\Span}{span}
\DeclareMathOperator{\Sgn}{sgn}
\DeclareMathOperator{\Area}{Area}
\DeclareMathOperator{\Volume}{Volume}

%
% A few miscellaneous things specific to this document
%
\newcommand{\crossop}[1]{\crossprod{#1}{}}

% R2 vector.
\newcommand{\VectorTwo}[2]{
\begin{bmatrix}
 {#1} \\
 {#2}
\end{bmatrix}
}

\newcommand{\VectorN}[1]{
\begin{bmatrix}
{#1}_1 \\
{#1}_2 \\
\vdots \\
{#1}_N \\
\end{bmatrix}
}

\newcommand{\DETuvij}[4]{
\begin{vmatrix}
 {#1}_{#3} & {#1}_{#4} \\
 {#2}_{#3} & {#2}_{#4}
\end{vmatrix}
}

\newcommand{\DETuvwijk}[6]{
\begin{vmatrix}
 {#1}_{#4} & {#1}_{#5} & {#1}_{#6} \\
 {#2}_{#4} & {#2}_{#5} & {#2}_{#6} \\
 {#3}_{#4} & {#3}_{#5} & {#3}_{#6}
\end{vmatrix}
}

\newcommand{\DETuvwxijkl}[8]{
\begin{vmatrix}
 {#1}_{#5} & {#1}_{#6} & {#1}_{#7} & {#1}_{#8} \\
 {#2}_{#5} & {#2}_{#6} & {#2}_{#7} & {#2}_{#8} \\
 {#3}_{#5} & {#3}_{#6} & {#3}_{#7} & {#3}_{#8} \\
 {#4}_{#5} & {#4}_{#6} & {#4}_{#7} & {#4}_{#8} \\
\end{vmatrix}
}

%\newcommand{\DETuvwxyijklm}[10]{
%\begin{vmatrix}
% {#1}_{#6} & {#1}_{#7} & {#1}_{#8} & {#1}_{#9} & {#1}_{#10} \\
% {#2}_{#6} & {#2}_{#7} & {#2}_{#8} & {#2}_{#9} & {#2}_{#10} \\
% {#3}_{#6} & {#3}_{#7} & {#3}_{#8} & {#3}_{#9} & {#3}_{#10} \\
% {#4}_{#6} & {#4}_{#7} & {#4}_{#8} & {#4}_{#9} & {#4}_{#10} \\
% {#5}_{#6} & {#5}_{#7} & {#5}_{#8} & {#5}_{#9} & {#5}_{#10}
%\end{vmatrix}
%}

% R3 vector.
\newcommand{\VectorThree}[3]{
\begin{bmatrix}
 {#1} \\
 {#2} \\
 {#3}
\end{bmatrix}
}



\author{Peeter Joot}
\email{peeter.joot@gmail.com}

\author{Peeter Joot}
\email{peeter.joot@utoronto.ca, 920798560}

\chapter{PHY450H1S Problem Set 3.}
\label{chap:relElectroDynProblemSet3}
%\blogpage{http://sites.google.com/site/peeterjoot/math2011/relElectroDynProblemSet3.pdf}
\date{Feb 15, 2011}
\revisionInfo{relElectroDynProblemSet3.tex}

\beginArtNoToc
%\beginArtWithToc
%\section{Disclaimer.}
%
%This problem set is as yet ungraded.

\section{Problem 1.  Fun with $\epsilon_{\alpha\beta\gamma}$, $\epsilon^{ijkl}$, $F_{ij}$, and the duality of Maxwell's equations in vacuum.}

\subsection{1. Statement}

Prove that

\begin{equation}\label{eqn:relElectroDynProblemSet3:10}
\epsilon_{\alpha \beta \gamma}
\epsilon_{\mu \nu \gamma}
=
\delta_{\alpha\mu} \delta_{\beta\nu}
-\delta_{\alpha\nu} \delta_{\beta\mu}
\end{equation}

and use it to find the familar relation for

\begin{equation}\label{eqn:relElectroDynProblemSet3:30}
(\BA \cross \BB) \cdot (\BC \cross \BD)
\end{equation}

Also show that

\begin{equation}\label{eqn:relElectroDynProblemSet3:50}
\epsilon_{\alpha \beta \gamma}
\epsilon_{\mu \beta \gamma}
=
2 \delta_{\alpha\mu}.
\end{equation}

(Einstein summation implied all throughout this problem).

\subsection{1. Solution}

We can explicitly expand the (implied) sum over indexes $\gamma$.  This is

\begin{equation}\label{eqn:relElectroDynProblemSet3:70}
\epsilon_{\alpha \beta \gamma}
\epsilon_{\mu \nu \gamma}
=
\epsilon_{\alpha \beta 1} \epsilon_{\mu \nu 1}
+\epsilon_{\alpha \beta 2} \epsilon_{\mu \nu 2}
+\epsilon_{\alpha \beta 3} \epsilon_{\mu \nu 3}
\end{equation}

For any $\alpha \ne \beta$ only one term is non-zero.  For example with $\alpha,\beta = 2,3$, we have just a contribution from the $\gamma = 1$ part of the sum

\begin{equation}\label{eqn:relElectroDynProblemSet3:90}
\epsilon_{2 3 1} \epsilon_{\mu \nu 1}.
\end{equation}

The value of this for $(\mu,\nu) = (\alpha,\beta)$ is

\begin{equation}\label{eqn:relElectroDynProblemSet3:110}
(\epsilon_{2 3 1})^2
\end{equation}

whereas for $(\mu,\nu) = (\beta,\alpha)$ we have

\begin{equation}\label{eqn:relElectroDynProblemSet3:130}
-(\epsilon_{2 3 1})^2
\end{equation}

Our sum has value one when $(\alpha, \beta)$ matches $(\mu, \nu)$, and value minus one for when $(\mu, \nu)$ are permuted.  We can summarize this, by saying that when $\alpha \ne \beta$ we have

\begin{equation}\label{eqn:relElectroDynProblemSet3:150}
\epsilon_{\alpha \beta \gamma}
\epsilon_{\mu \nu \gamma}
=
\delta_{\alpha\mu} \delta_{\beta\nu}
-\delta_{\alpha\nu} \delta_{\beta\mu} .
\end{equation}

However, observe that when $\alpha = \beta$ the RHS is

\begin{equation}\label{eqn:relElectroDynProblemSet3:170}
\delta_{\alpha\mu} \delta_{\alpha\nu}
-\delta_{\alpha\nu} \delta_{\alpha\mu} = 0,
\end{equation}

as desired, so this form works in general without any $\alpha \ne \beta$ qualifier, completing this part of the problem.

\begin{align*}
(\BA \cross \BB) \cdot (\BC \cross \BD)
&=
(\epsilon_{\alpha \beta \gamma} \Be^\alpha A^\beta B^\gamma ) \cdot
(\epsilon_{\mu \nu \sigma} \Be^\mu C^\nu D^\sigma ) \\
&=
\epsilon_{\alpha \beta \gamma} A^\beta B^\gamma
\epsilon_{\alpha \nu \sigma} C^\nu D^\sigma \\
&=
(
\delta_{\beta \nu} \delta_{\gamma\sigma}
-\delta_{\beta \sigma} \delta_{\gamma\nu} )
A^\beta B^\gamma
C^\nu D^\sigma \\
&=
A^\nu B^\sigma
C^\nu D^\sigma
-A^\sigma B^\nu
C^\nu D^\sigma.
\end{align*}

This gives us
\begin{equation}\label{eqn:relElectroDynProblemSet3:190}
(\BA \cross \BB) \cdot (\BC \cross \BD)
=
(\BA \cdot \BC)
(\BB \cdot \BD)
-
(\BA \cdot \BD)
(\BB \cdot \BC).
\end{equation}

We have one more identity to deal with.

\begin{equation}\label{eqn:relElectroDynProblemSet3:210}
\epsilon_{\alpha \beta \gamma}
\epsilon_{\mu \beta \gamma}
\end{equation}

We can expand out this (implied) sum slow and dumb as well

\begin{align*}
\epsilon_{\alpha \beta \gamma}
\epsilon_{\mu \beta \gamma}
&=
\epsilon_{\alpha 1 2} \epsilon_{\mu 1 2}
+\epsilon_{\alpha 2 1} \epsilon_{\mu 2 1} \\
&+\epsilon_{\alpha 1 3} \epsilon_{\mu 1 3}
+\epsilon_{\alpha 3 1} \epsilon_{\mu 3 1} \\
&+\epsilon_{\alpha 2 3} \epsilon_{\mu 2 3}
+\epsilon_{\alpha 3 2} \epsilon_{\mu 3 2} \\
&=
2 \epsilon_{\alpha 1 2} \epsilon_{\mu 1 2}
+ 2 \epsilon_{\alpha 1 3} \epsilon_{\mu 1 3}
+ 2 \epsilon_{\alpha 2 3} \epsilon_{\mu 2 3}
\end{align*}

Now, observe that for any $\alpha \in (1,2,3)$ only one term of this sum is picked up.  For example, with no loss of generality, pick $\alpha = 1$.  We are left with only

\begin{equation}\label{eqn:relElectroDynProblemSet3:230}
2 \epsilon_{1 2 3} \epsilon_{\mu 2 3}
\end{equation}

This has the value
\begin{equation}\label{eqn:relElectroDynProblemSet3:250}
2 (\epsilon_{1 2 3})^2 = 2
\end{equation}

when $\mu = \alpha$ and is zero otherwise.  We can therefore summarize the evaluation of this sum as 

\begin{equation}\label{eqn:relElectroDynProblemSet3:270}
\epsilon_{\alpha \beta \gamma}
\epsilon_{\mu \beta \gamma}
=  2\delta_{\alpha\mu},
\end{equation}

completing this problem.

\subsection{2. Statement}

Prove that for any $3 \times 3$ matrix $\Norm{A_{\alpha\beta}}$: $\epsilon_{\mu\nu\lambda} A_{\alpha \mu} A_{\beta\nu} A_{\gamma\lambda} = \epsilon_{\alpha \beta \gamma} \Det A$ and that $\epsilon_{\alpha\beta\gamma} \epsilon_{\mu\nu\lambda} A_{\alpha \mu} A_{\beta\nu} A_{\gamma\lambda} = 6 \Det A$.  

\subsection{2. Solution}

In class Simon showed us how the first identity can be arrived at using the triple product $\Ba \cdot (\Bb \cross \Bc) = \Det(\Ba \Bb \Bc)$.  It occured to me later that I'd seen the identity to be proven in the context of Geometric Algebra, but hadn't recognized it in this tensor form.  Basically, a wedge product can be expanded in sums of determinants, and when the dimension of the space is the same as the vector, we have a pseudoscalar times the determinant of the components.

For example, in \R{2}, let's take the wedge product of a pair of vectors.  As preparation for the relativistic \R{4} case We won't require an orthonormal basis, but express the vector in terms of a reciprocal frame and the associated components

\begin{equation}\label{eqn:relElectroDynProblemSet3:290}
a = a^i e_i = a_j e^j
\end{equation}

where 
\begin{equation}\label{eqn:relElectroDynProblemSet3:310}
e^i \cdot e_j = {\delta^i}_j.
\end{equation}

When we get to the relativistic case, we can pick (but don't have to) the standard basis

\begin{align}\label{eqn:relElectroDynProblemSet3:330}
e_0 &= (1, 0, 0, 0) \\
e_1 &= (0, 1, 0, 0) \\
e_2 &= (0, 0, 1, 0) \\
e_3 &= (0, 0, 0, 1),
\end{align}

for which our reciprocal frame is implicitly defined by the metric
\begin{align}\label{eqn:relElectroDynProblemSet3:350}
e^0 &= (1, 0, 0, 0) \\
e^1 &= (0, -1, 0, 0) \\
e^2 &= (0, 0, -1, 0) \\
e^3 &= (0, 0, 0, -1).
\end{align}

Anyways.  Back to the problem.  Let's examine the \R{2} case.  Our wedge product in coordinates is

\begin{equation}\label{eqn:relElectroDynProblemSet3:370}
a \wedge b
=
a^i b^j (e_i \wedge e_j)
\end{equation}

Since there are only two basis vectors we have

\begin{equation}\label{eqn:relElectroDynProblemSet3:390}
a \wedge b
=
(a^1 b^2 - a^2 b^1) e_1 \wedge e_2 = \Det \Norm{a^i b^j} (e_1 \wedge e_2).
\end{equation}

Our wedge product is a product of the determinant of the vector coordinates, times the \R{2} pseudoscalar $e_1 \wedge e_2$.

This doesn't look quite like the \R{3} relation that we want to prove, which had an antisymmetric tensor factor for the determinant.  Observe that we get the determinant by picking off the $e_1 \wedge e_2$ component of the bivector result (the only component in this case), and we can do that by dotting with $e^2 \cdot e^1$.  To get an antisymmetric tensor times the determinant, we have only to dot with a different pseudoscalar (one that differs by a possible sign due to permutation of the indexes).  That is

\begin{align*}
(e^t \wedge e^s) \cdot (a \wedge b)
&=
a^i b^j (e^t \wedge e^s) \cdot (e_i \wedge e_j) \\
&=
a^i b^j 
\left( {\delta^{s}}_i {\delta^{t}}_j 
-{\delta^{t}}_i {\delta^{s}}_j  \right) \\
&=
a^i b^j 
{\delta^{[t}}_j {\delta^{s]}}_i \\
&=
a^i b^j 
{\delta^{t}}_{[j} {\delta^{s}}_{i]} \\
&=
a^{[i} b^{j]}
{\delta^{t}}_{j} {\delta^{s}}_{i} \\
&=
a^{[s} b^{t]}
\end{align*}

Now, if we write $a^i = A^{1 i}$ and $b^j = A^{2 j}$ we have

\begin{equation}\label{eqn:relElectroDynProblemSet3:410}
(e^t \wedge e^s) \cdot (a \wedge b)
= 
A^{1 s} A^{2 t} -A^{1 t} A^{2 s}
\end{equation}

We can write this in two different ways.  One of which is

\begin{equation}\label{eqn:relElectroDynProblemSet3:430}
A^{1 s} A^{2 t} -A^{1 t} A^{2 s} =
\epsilon^{s t} \Det A
\end{equation}

and the other of which is by introducing free indexes for $1$ and $2$, and summing antisymmetrically over these.  That is

\begin{equation}\label{eqn:relElectroDynProblemSet3:450}
A^{1 s} A^{2 t} -A^{1 t} A^{2 s} 
=
A^{a s} A^{b t} \epsilon_{a b}
\end{equation}

So, we have

\begin{equation}\label{eqn:relElectroDynProblemSet3:470}
A^{a s} A^{b t} \epsilon_{a b} = \epsilon^{s t} \Det A,
\end{equation}

This result hold regardless of the metric for the space, and does not require that we were using an orthonormal basis.  When the metric is Euclidean and we have an orthonormal basis, then all the indexes can be dropped.

The \R{3} and \R{4} cases follow in exactly the same way, we just need more vectors in the wedge products.

For the \R{3} case we have

\begin{align*}
(e^u \wedge e^t \wedge e^s) \cdot ( a \wedge b \wedge c) 
&=
a^i b^j c^k 
(e^u \wedge e^t \wedge e^s) \cdot (e_i \wedge e_j \wedge e_k) \\
&=
a^i b^j c^k 
{\delta^{[u}}_k
{\delta^{t}}_j
{\delta^{s]}}_i \\
&=
a^{[s} b^t c^{u]} 
\end{align*}

Again, with $a^i = A^{1 i}$ and $b^j = A^{2 j}$, and $c^k = A^{3 k}$ we have

\begin{equation}\label{eqn:relElectroDynProblemSet3:490}
(e^u \wedge e^t \wedge e^s) \cdot ( a \wedge b \wedge c) 
=
A^{1 i} A^{2 j} A^{3 k}
{\delta^{[u}}_k
{\delta^{t}}_j
{\delta^{s]}}_i 
\end{equation}

and we can choose to write this in either form, resulting in the identity

\begin{equation}\label{eqn:relElectroDynProblemSet3:510}
\epsilon^{s t u} \Det A
=
A^{1 i} A^{2 j} A^{3 k}
{\delta^{[u}}_k
{\delta^{t}}_j
{\delta^{s]}}_i 
=
\epsilon_{a b c} A^{a s} A^{b t} A^{c u}.
\end{equation}

The \R{4} case follows exactly the same way, and we have

\begin{align*}
(e^v \wedge e^u \wedge e^t \wedge e^s) \cdot ( a \wedge b \wedge c \wedge d) 
&=
a^i b^j c^k d^l
(e^v \wedge e^u \wedge e^t \wedge e^s) \cdot (e_i \wedge e_j \wedge e_k \wedge e_l) \\
&=
a^i b^j c^k d^l
{\delta^{[v}}_l
{\delta^{u}}_k
{\delta^{t}}_j
{\delta^{s]}}_i \\
&=
a^{[s} b^t c^{u} d^{v]} 
\end{align*}

and we have again
\begin{equation}\label{eqn:relElectroDynProblemSet3:530}
\epsilon^{s t u v} \Det A
=
A^{1 i} A^{2 j} A^{3 k} A^{4 l}
{\delta^{[v}}_k
{\delta^{u}}_k
{\delta^{t}}_j
{\delta^{s]}}_i 
=
\epsilon_{a b c d} A^{a s} A^{b t} A^{c u} A^{d u}.
\end{equation}

This one is almost the identity to be established later in problem 1.4.  We have only to raise and lower some indexes to get that one.  Note that in the Minkowski standard basis above, because $s, t, u, v$ must be a permutation of $0,1,2,3$ for a non-zero result, we must have

\begin{equation}\label{eqn:relElectroDynProblemSet3:550}
\epsilon^{s t u v} = (-1)^3 (+1) \epsilon_{s t u v}.
\end{equation}

So raising and lowering the identity above gives us

\begin{equation}\label{eqn:relElectroDynProblemSet3:570}
-\epsilon_{s t u v} \Det A
=
\epsilon^{a b c d} A_{a s} A_{b t} A_{c u} A_{d u}.
\end{equation}

No sign changes were required for the indexes $a, b, c, d$, since they are paired.

Until we did the raising and lowering operations here, there was no specific metric required, so our first result \ref{eqn:relElectroDynProblemSet3:530} is the more general one.

\subsection{3. Statement}
\subsection{3. Solution}
\subsection{4. Statement}
\subsection{4. Solution}
\subsection{5. Statement}
\subsection{5. Solution}
\subsection{6. Statement}
\subsection{6. Solution}
\subsection{7. Statement}
\subsection{7. Solution}
\subsection{8. Statement}
\subsection{8. Solution}

\section{Problem 2.}
\subsection{1. Statement}
\subsection{1. Solution}
\subsection{2. Statement}
\subsection{2. Solution}

\section{Problem 3.}
\subsection{Statement}
\subsection{Solution}

%\EndArticle
\EndNoBibArticle

%
% Copyright � 2015 Peeter Joot.  All Rights Reserved.
% Licenced as described in the file LICENSE under the root directory of this GIT repository.
%
\documentclass[]{eliblog}

\usepackage{amsmath}
\usepackage{mathpazo}

%
% shorthand for bold symbols, convenient for vectors and matrices
%
\newcommand{\Ba}[0]{\mathbf{a}}
\newcommand{\Bb}[0]{\mathbf{b}}
\newcommand{\Bc}[0]{\mathbf{c}}
\newcommand{\Bd}[0]{\mathbf{d}}
\newcommand{\Be}[0]{\mathbf{e}}
\newcommand{\Bf}[0]{\mathbf{f}}
\newcommand{\Bg}[0]{\mathbf{g}}
\newcommand{\Bh}[0]{\mathbf{h}}
\newcommand{\Bi}[0]{\mathbf{i}}
\newcommand{\Bj}[0]{\mathbf{j}}
\newcommand{\Bk}[0]{\mathbf{k}}
\newcommand{\Bl}[0]{\mathbf{l}}
\newcommand{\Bm}[0]{\mathbf{m}}
\newcommand{\Bn}[0]{\mathbf{n}}
\newcommand{\Bo}[0]{\mathbf{o}}
\newcommand{\Bp}[0]{\mathbf{p}}
\newcommand{\Bq}[0]{\mathbf{q}}
\newcommand{\Br}[0]{\mathbf{r}}
\newcommand{\Bs}[0]{\mathbf{s}}
\newcommand{\Bt}[0]{\mathbf{t}}
\newcommand{\Bu}[0]{\mathbf{u}}
\newcommand{\Bv}[0]{\mathbf{v}}
\newcommand{\Bw}[0]{\mathbf{w}}
\newcommand{\Bx}[0]{\mathbf{x}}
\newcommand{\By}[0]{\mathbf{y}}
\newcommand{\Bz}[0]{\mathbf{z}}
\newcommand{\BA}[0]{\mathbf{A}}
\newcommand{\BB}[0]{\mathbf{B}}
\newcommand{\BC}[0]{\mathbf{C}}
\newcommand{\BD}[0]{\mathbf{D}}
\newcommand{\BE}[0]{\mathbf{E}}
\newcommand{\BF}[0]{\mathbf{F}}
\newcommand{\BG}[0]{\mathbf{G}}
\newcommand{\BH}[0]{\mathbf{H}}
\newcommand{\BI}[0]{\mathbf{I}}
\newcommand{\BJ}[0]{\mathbf{J}}
\newcommand{\BK}[0]{\mathbf{K}}
\newcommand{\BL}[0]{\mathbf{L}}
\newcommand{\BM}[0]{\mathbf{M}}
\newcommand{\BN}[0]{\mathbf{N}}
\newcommand{\BO}[0]{\mathbf{O}}
\newcommand{\BP}[0]{\mathbf{P}}
\newcommand{\BQ}[0]{\mathbf{Q}}
\newcommand{\BR}[0]{\mathbf{R}}
\newcommand{\BS}[0]{\mathbf{S}}
\newcommand{\BT}[0]{\mathbf{T}}
\newcommand{\BU}[0]{\mathbf{U}}
\newcommand{\BV}[0]{\mathbf{V}}
\newcommand{\BW}[0]{\mathbf{W}}
\newcommand{\BX}[0]{\mathbf{X}}
\newcommand{\BY}[0]{\mathbf{Y}}
\newcommand{\BZ}[0]{\mathbf{Z}}

\newcommand{\Bzero}[0]{\mathbf{0}}
\newcommand{\Btheta}[0]{\boldsymbol{\theta}}
\newcommand{\Btau}[0]{\boldsymbol{\tau}}
\newcommand{\Bomega}[0]{\boldsymbol{\omega}}

%
% shorthand for unit vectors
%
\newcommand{\acap}[0]{\hat{\Ba}}
\newcommand{\bcap}[0]{\hat{\Bb}}
\newcommand{\ccap}[0]{\hat{\Bc}}
\newcommand{\dcap}[0]{\hat{\Bd}}
\newcommand{\ecap}[0]{\hat{\Be}}
\newcommand{\fcap}[0]{\hat{\Bf}}
\newcommand{\gcap}[0]{\hat{\Bg}}
\newcommand{\hcap}[0]{\hat{\Bh}}
\newcommand{\icap}[0]{\hat{\Bi}}
\newcommand{\jcap}[0]{\hat{\Bj}}
\newcommand{\kcap}[0]{\hat{\Bk}}
\newcommand{\lcap}[0]{\hat{\Bl}}
\newcommand{\mcap}[0]{\hat{\Bm}}
\newcommand{\ncap}[0]{\hat{\Bn}}
\newcommand{\ocap}[0]{\hat{\Bo}}
\newcommand{\pcap}[0]{\hat{\Bp}}
\newcommand{\qcap}[0]{\hat{\Bq}}
\newcommand{\rcap}[0]{\hat{\Br}}
\newcommand{\scap}[0]{\hat{\Bs}}
\newcommand{\tcap}[0]{\hat{\Bt}}
\newcommand{\ucap}[0]{\hat{\Bu}}
\newcommand{\vcap}[0]{\hat{\Bv}}
\newcommand{\wcap}[0]{\hat{\Bw}}
\newcommand{\xcap}[0]{\hat{\Bx}}
\newcommand{\ycap}[0]{\hat{\By}}
\newcommand{\zcap}[0]{\hat{\Bz}}
\newcommand{\thetacap}[0]{\hat{\Btheta}}

%
% to write R^n and C^n in a distinguishable fashion.  Perhaps change this
% to the double lined characters upon figuring out how to do so.
%
\newcommand{\C}[1]{$\mathbb{C}^{#1}$}
\newcommand{\R}[1]{$\mathbb{R}^{#1}$}

%
% various generally useful helpers
%

% derivative of #1 wrt. #2:
\newcommand{\D}[2] {\frac {d#2} {d#1}}

\newcommand{\inv}[1]{\frac{1}{#1}}
\newcommand{\cross}[0]{\times}

\newcommand{\abs}[1]{\lvert{#1}\rvert}
\newcommand{\norm}[1]{\lVert{#1}\rVert}
\newcommand{\innerprod}[2]{\langle{#1}, {#2}\rangle}
\newcommand{\dotprod}[2]{{#1} \cdot {#2}}
\newcommand{\bdotprod}[2]{\left({#1} \cdot {#2}\right)}
\newcommand{\crossprod}[2]{{#1} \cross {#2}}
\newcommand{\tripleprod}[3]{\dotprod{\left(\crossprod{#1}{#2}\right)}{#3}}

\DeclareMathOperator{\Proj}{Proj}
\DeclareMathOperator{\Span}{span}
\DeclareMathOperator{\Sgn}{sgn}
\DeclareMathOperator{\Area}{Area}
\DeclareMathOperator{\Volume}{Volume}

%
% A few miscellaneous things specific to this document
%
\newcommand{\crossop}[1]{\crossprod{#1}{}}

% R2 vector.
\newcommand{\VectorTwo}[2]{
\begin{bmatrix}
 {#1} \\
 {#2}
\end{bmatrix}
}

\newcommand{\VectorN}[1]{
\begin{bmatrix}
{#1}_1 \\
{#1}_2 \\
\vdots \\
{#1}_N \\
\end{bmatrix}
}

\newcommand{\DETuvij}[4]{
\begin{vmatrix}
 {#1}_{#3} & {#1}_{#4} \\
 {#2}_{#3} & {#2}_{#4}
\end{vmatrix}
}

\newcommand{\DETuvwijk}[6]{
\begin{vmatrix}
 {#1}_{#4} & {#1}_{#5} & {#1}_{#6} \\
 {#2}_{#4} & {#2}_{#5} & {#2}_{#6} \\
 {#3}_{#4} & {#3}_{#5} & {#3}_{#6}
\end{vmatrix}
}

\newcommand{\DETuvwxijkl}[8]{
\begin{vmatrix}
 {#1}_{#5} & {#1}_{#6} & {#1}_{#7} & {#1}_{#8} \\
 {#2}_{#5} & {#2}_{#6} & {#2}_{#7} & {#2}_{#8} \\
 {#3}_{#5} & {#3}_{#6} & {#3}_{#7} & {#3}_{#8} \\
 {#4}_{#5} & {#4}_{#6} & {#4}_{#7} & {#4}_{#8} \\
\end{vmatrix}
}

%\newcommand{\DETuvwxyijklm}[10]{
%\begin{vmatrix}
% {#1}_{#6} & {#1}_{#7} & {#1}_{#8} & {#1}_{#9} & {#1}_{#10} \\
% {#2}_{#6} & {#2}_{#7} & {#2}_{#8} & {#2}_{#9} & {#2}_{#10} \\
% {#3}_{#6} & {#3}_{#7} & {#3}_{#8} & {#3}_{#9} & {#3}_{#10} \\
% {#4}_{#6} & {#4}_{#7} & {#4}_{#8} & {#4}_{#9} & {#4}_{#10} \\
% {#5}_{#6} & {#5}_{#7} & {#5}_{#8} & {#5}_{#9} & {#5}_{#10}
%\end{vmatrix}
%}

% R3 vector.
\newcommand{\VectorThree}[3]{
\begin{bmatrix}
 {#1} \\
 {#2} \\
 {#3}
\end{bmatrix}
}



\author{Peeter Joot}
\email{peeter.joot@gmail.com}

%\documentclass[]{eliblogwidescreen}

\usepackage{amsmath}
\usepackage{mathpazo}

%
% shorthand for bold symbols, convenient for vectors and matrices
%
\newcommand{\Ba}[0]{\mathbf{a}}
\newcommand{\Bb}[0]{\mathbf{b}}
\newcommand{\Bc}[0]{\mathbf{c}}
\newcommand{\Bd}[0]{\mathbf{d}}
\newcommand{\Be}[0]{\mathbf{e}}
\newcommand{\Bf}[0]{\mathbf{f}}
\newcommand{\Bg}[0]{\mathbf{g}}
\newcommand{\Bh}[0]{\mathbf{h}}
\newcommand{\Bi}[0]{\mathbf{i}}
\newcommand{\Bj}[0]{\mathbf{j}}
\newcommand{\Bk}[0]{\mathbf{k}}
\newcommand{\Bl}[0]{\mathbf{l}}
\newcommand{\Bm}[0]{\mathbf{m}}
\newcommand{\Bn}[0]{\mathbf{n}}
\newcommand{\Bo}[0]{\mathbf{o}}
\newcommand{\Bp}[0]{\mathbf{p}}
\newcommand{\Bq}[0]{\mathbf{q}}
\newcommand{\Br}[0]{\mathbf{r}}
\newcommand{\Bs}[0]{\mathbf{s}}
\newcommand{\Bt}[0]{\mathbf{t}}
\newcommand{\Bu}[0]{\mathbf{u}}
\newcommand{\Bv}[0]{\mathbf{v}}
\newcommand{\Bw}[0]{\mathbf{w}}
\newcommand{\Bx}[0]{\mathbf{x}}
\newcommand{\By}[0]{\mathbf{y}}
\newcommand{\Bz}[0]{\mathbf{z}}
\newcommand{\BA}[0]{\mathbf{A}}
\newcommand{\BB}[0]{\mathbf{B}}
\newcommand{\BC}[0]{\mathbf{C}}
\newcommand{\BD}[0]{\mathbf{D}}
\newcommand{\BE}[0]{\mathbf{E}}
\newcommand{\BF}[0]{\mathbf{F}}
\newcommand{\BG}[0]{\mathbf{G}}
\newcommand{\BH}[0]{\mathbf{H}}
\newcommand{\BI}[0]{\mathbf{I}}
\newcommand{\BJ}[0]{\mathbf{J}}
\newcommand{\BK}[0]{\mathbf{K}}
\newcommand{\BL}[0]{\mathbf{L}}
\newcommand{\BM}[0]{\mathbf{M}}
\newcommand{\BN}[0]{\mathbf{N}}
\newcommand{\BO}[0]{\mathbf{O}}
\newcommand{\BP}[0]{\mathbf{P}}
\newcommand{\BQ}[0]{\mathbf{Q}}
\newcommand{\BR}[0]{\mathbf{R}}
\newcommand{\BS}[0]{\mathbf{S}}
\newcommand{\BT}[0]{\mathbf{T}}
\newcommand{\BU}[0]{\mathbf{U}}
\newcommand{\BV}[0]{\mathbf{V}}
\newcommand{\BW}[0]{\mathbf{W}}
\newcommand{\BX}[0]{\mathbf{X}}
\newcommand{\BY}[0]{\mathbf{Y}}
\newcommand{\BZ}[0]{\mathbf{Z}}

\newcommand{\Bzero}[0]{\mathbf{0}}
\newcommand{\Btheta}[0]{\boldsymbol{\theta}}
\newcommand{\Btau}[0]{\boldsymbol{\tau}}
\newcommand{\Bomega}[0]{\boldsymbol{\omega}}

%
% shorthand for unit vectors
%
\newcommand{\acap}[0]{\hat{\Ba}}
\newcommand{\bcap}[0]{\hat{\Bb}}
\newcommand{\ccap}[0]{\hat{\Bc}}
\newcommand{\dcap}[0]{\hat{\Bd}}
\newcommand{\ecap}[0]{\hat{\Be}}
\newcommand{\fcap}[0]{\hat{\Bf}}
\newcommand{\gcap}[0]{\hat{\Bg}}
\newcommand{\hcap}[0]{\hat{\Bh}}
\newcommand{\icap}[0]{\hat{\Bi}}
\newcommand{\jcap}[0]{\hat{\Bj}}
\newcommand{\kcap}[0]{\hat{\Bk}}
\newcommand{\lcap}[0]{\hat{\Bl}}
\newcommand{\mcap}[0]{\hat{\Bm}}
\newcommand{\ncap}[0]{\hat{\Bn}}
\newcommand{\ocap}[0]{\hat{\Bo}}
\newcommand{\pcap}[0]{\hat{\Bp}}
\newcommand{\qcap}[0]{\hat{\Bq}}
\newcommand{\rcap}[0]{\hat{\Br}}
\newcommand{\scap}[0]{\hat{\Bs}}
\newcommand{\tcap}[0]{\hat{\Bt}}
\newcommand{\ucap}[0]{\hat{\Bu}}
\newcommand{\vcap}[0]{\hat{\Bv}}
\newcommand{\wcap}[0]{\hat{\Bw}}
\newcommand{\xcap}[0]{\hat{\Bx}}
\newcommand{\ycap}[0]{\hat{\By}}
\newcommand{\zcap}[0]{\hat{\Bz}}
\newcommand{\thetacap}[0]{\hat{\Btheta}}

%
% to write R^n and C^n in a distinguishable fashion.  Perhaps change this
% to the double lined characters upon figuring out how to do so.
%
\newcommand{\C}[1]{$\mathbb{C}^{#1}$}
\newcommand{\R}[1]{$\mathbb{R}^{#1}$}

%
% various generally useful helpers
%

% derivative of #1 wrt. #2:
\newcommand{\D}[2] {\frac {d#2} {d#1}}

\newcommand{\inv}[1]{\frac{1}{#1}}
\newcommand{\cross}[0]{\times}

\newcommand{\abs}[1]{\lvert{#1}\rvert}
\newcommand{\norm}[1]{\lVert{#1}\rVert}
\newcommand{\innerprod}[2]{\langle{#1}, {#2}\rangle}
\newcommand{\dotprod}[2]{{#1} \cdot {#2}}
\newcommand{\bdotprod}[2]{\left({#1} \cdot {#2}\right)}
\newcommand{\crossprod}[2]{{#1} \cross {#2}}
\newcommand{\tripleprod}[3]{\dotprod{\left(\crossprod{#1}{#2}\right)}{#3}}

\DeclareMathOperator{\Proj}{Proj}
\DeclareMathOperator{\Span}{span}
\DeclareMathOperator{\Sgn}{sgn}
\DeclareMathOperator{\Area}{Area}
\DeclareMathOperator{\Volume}{Volume}

%
% A few miscellaneous things specific to this document
%
\newcommand{\crossop}[1]{\crossprod{#1}{}}

% R2 vector.
\newcommand{\VectorTwo}[2]{
\begin{bmatrix}
 {#1} \\
 {#2}
\end{bmatrix}
}

\newcommand{\VectorN}[1]{
\begin{bmatrix}
{#1}_1 \\
{#1}_2 \\
\vdots \\
{#1}_N \\
\end{bmatrix}
}

\newcommand{\DETuvij}[4]{
\begin{vmatrix}
 {#1}_{#3} & {#1}_{#4} \\
 {#2}_{#3} & {#2}_{#4}
\end{vmatrix}
}

\newcommand{\DETuvwijk}[6]{
\begin{vmatrix}
 {#1}_{#4} & {#1}_{#5} & {#1}_{#6} \\
 {#2}_{#4} & {#2}_{#5} & {#2}_{#6} \\
 {#3}_{#4} & {#3}_{#5} & {#3}_{#6}
\end{vmatrix}
}

\newcommand{\DETuvwxijkl}[8]{
\begin{vmatrix}
 {#1}_{#5} & {#1}_{#6} & {#1}_{#7} & {#1}_{#8} \\
 {#2}_{#5} & {#2}_{#6} & {#2}_{#7} & {#2}_{#8} \\
 {#3}_{#5} & {#3}_{#6} & {#3}_{#7} & {#3}_{#8} \\
 {#4}_{#5} & {#4}_{#6} & {#4}_{#7} & {#4}_{#8} \\
\end{vmatrix}
}

%\newcommand{\DETuvwxyijklm}[10]{
%\begin{vmatrix}
% {#1}_{#6} & {#1}_{#7} & {#1}_{#8} & {#1}_{#9} & {#1}_{#10} \\
% {#2}_{#6} & {#2}_{#7} & {#2}_{#8} & {#2}_{#9} & {#2}_{#10} \\
% {#3}_{#6} & {#3}_{#7} & {#3}_{#8} & {#3}_{#9} & {#3}_{#10} \\
% {#4}_{#6} & {#4}_{#7} & {#4}_{#8} & {#4}_{#9} & {#4}_{#10} \\
% {#5}_{#6} & {#5}_{#7} & {#5}_{#8} & {#5}_{#9} & {#5}_{#10}
%\end{vmatrix}
%}

% R3 vector.
\newcommand{\VectorThree}[3]{
\begin{bmatrix}
 {#1} \\
 {#2} \\
 {#3}
\end{bmatrix}
}



\author{Peeter Joot}
\email{peeter.joot@gmail.com}


\chapter{PHY450H1S Problem Set 4.}
\label{chap:relElectroDynProblemSet4}
\blogpage{http://sites.google.com/site/peeterjoot/math2011/relElectroDynProblemSet4.pdf}
\date{Mar 3, 2011}
\revisionInfo{relElectroDynProblemSet4.tex}

\beginArtWithToc
%\beginArtNoToc

\section{Disclaimer.}

This problem set is as yet ungraded (although only the second question will be graded).

\section{Problem 1.  Energy, momentum, etc., of EM waves.}

\subsection{Statement}

\begin{enumerate}
\item Calculate the energy density, energy flux, and momentum density of a plane monochromatic linearly polarized electromagnetic wave.
\item Calculate the values of these quantities averaged over a period.
\item Imagine that a plane monochromatic linearly polarized wave incident on a surface (let the angle between the wave vector and the normal to the surface be $\theta$) is completely reflected.  Find the pressure that the EM wave exerts on the surface.
\item To plug in some numbrers, note that the intensity of sunlight hitting the Earth is about $1300 W/m^2$ ( the intensity is the average power per unit area transported by the wave).  If sunlight strikes a perfect absorber, what is the pressure exerted?  What if it strikes a perfect reflector?  What fraction of the atmoshperic pressure does this amount to?
\end{enumerate}
\subsection{Solution}
\subsubsection{Part 1.  Energy and momentum density.}

Because it doesn't add too much complexity, I'm going to calculate these using the more general eliptically polarized wave solutions.  Our vector potential (in the Coulomb gauge $\phi = 0$, $\spacegrad \cdot \BA = 0$) has the form

\begin{equation}\label{eqn:relElectroDynProblemSet4:10}
\BA = \Real \Bbeta e^{i (\omega t - \Bk \cdot \Bx) }.
\end{equation}

The eliptical polarization case only differs from the linear by allowing $\Bbeta$ to be complex, rather than purely real or purely imaginary.  Observe that the Coulomb gauge condition $\spacegrad \cdot \BA$ implies

\begin{equation}\label{eqn:relElectroDynProblemSet4:30}
\Bbeta \cdot \Bk = 0,
\end{equation}

a fact that will kill of terms in a number of places in the following manipulations.

Also observe that for this to be a solution to the wave equation operator

\begin{equation}\label{eqn:relElectroDynProblemSet4:50}
\inv{c^2} \PDSq{t}{} - \Delta,
\end{equation}

the frequency and wave vector must be related by the condition

\begin{equation}\label{eqn:relElectroDynProblemSet4:70}
\frac{\omega}{c} = \Abs{\Bk} = k.
\end{equation}

For the time and spatial phase let's write

\begin{equation}\label{eqn:relElectroDynProblemSet4:90}
\theta = \omega t - \Bk \cdot \Bx.
\end{equation}

In the Coulomb gauge, our electric and magnetic fields are

\begin{align}\label{eqn:relElectroDynProblemSet4:110}
\BE &= -\inv{c}\PD{t}{\BA} = \Real \frac{-i\omega}{c} \Bbeta e^{i\theta} \\
\BB &= \spacegrad \cross \BA = \Real i \Bbeta \cross \Bk e^{i\theta}
\end{align}

Similar to \S 48 of the text \cite{landau1980classical}, let's split $\Bbeta$ into a phase and perpendicular vector components so that

\begin{equation}\label{eqn:relElectroDynProblemSet4:130}
\Bbeta = \Bb e^{-i\alpha}
\end{equation}

where $\Bb$ has a real square

\begin{equation}\label{eqn:relElectroDynProblemSet4:150}
\Bb^2 = \Abs{\Bbeta}^2.
\end{equation}

This allows a split into two perpendicular real vectors

\begin{equation}\label{eqn:relElectroDynProblemSet4:170}
\Bb = \Bb_1 + i \Bb_2,
\end{equation}

where $\Bb_1 \cdot \Bb_2 = 0$ since $\Bb^2 = \Bb_1^2 - \Bb_2^2 + 2 \Bb_1 \cdot \Bb_2$ is real.

Our electric and magnetic fields are now reduced to
\begin{align}\label{eqn:relElectroDynProblemSet4:190}
\BE &= \Real \left( \frac{-i\omega}{c} \Bb e^{i(\theta - \alpha)} \right) \\
\BB &= \Real \left( i \Bb \cross \Bk e^{i(\theta - \alpha)} \right) 
\end{align}

or explicitly in terms of $\Bb_1$ and $\Bb_2$ 

\begin{align}\label{eqn:relElectroDynProblemSet4:210}
\BE &= \frac{\omega}{c} ( \Bb_1 \sin(\theta-\alpha) + \Bb_2 \cos(\theta-\alpha)) \\
\BB &= ( \Bk \cross \Bb_1 ) \sin(\theta-\alpha) + (\Bk \cross \Bb_2) \cos(\theta-\alpha) 
\end{align}

The special case of interest for this problem, since it only strictly asked for linear polarization, is where $\alpha = 0$ and one of $\Bb_1$ or $\Bb_2$ is zero (i.e. $\Bbeta$ is strictly real or strictly imaginary).  The case with $\Beta$ strictly real, as done in class, is

\begin{align}\label{eqn:relElectroDynProblemSet4:230}
\BE &= \frac{\omega}{c} \Bb_1 \sin(\theta-\alpha) \\
\BB &= ( \Bk \cross \Bb_1 ) \sin(\theta-\alpha) 
\end{align}

Now lets calculate the energy density and Poynting vectors.  We'll need a few intermediate results.

\begin{align*}
(\Real \Bd e^{i\phi})^2 
&= \inv{4} ( \Bd e^{i\phi} + \Bd^\conj e^{-i\phi})^2 \\
&= \inv{4} ( \Bd^2 e^{2 i \phi} + (\Bd^\conj)^2 e^{-2 i \phi} + 2 \Abs{\Bd}^2 ) \\
&= \inv{2} \left( \Abs{\Bd}^2 + \Real ( \Bd e^{i \phi} )^2 \right),
\end{align*}

and

\begin{align*}
(\Real \Bd e^{i\phi}) \cross (\Real \Be e^{i\phi}) 
&= \inv{4} 
( \Bd e^{i\phi} + \Bd^\conj e^{-i\phi}) \cross ( \Be e^{i\phi} + \Be^\conj e^{-i\phi}) \\
&= \inv{2} \Real \left( \Bd \cross \Be^\conj + (\Bd \cross \Be) e^{2 i \phi} \right).
\end{align*}

Let's use arrowed vectors for the phasor parts

\begin{align}\label{eqn:relElectroDynProblemSet4:250}
\vec{E} &= \frac{-i\omega}{c} \Bb e^{i(\theta - \alpha)} \\
\vec{B} &= i \Bb \cross \Bk e^{i(\theta - \alpha)},
\end{align}

where we can recover our vector quantities by taking real parts $\BE = \Real \vec{E}$, $\BB = \Real \vec{B}$.  Our energy density in terms of these phasors is then

\begin{equation}\label{eqn:relElectroDynProblemSet4:270}
\mathcal{E} 
= \inv{8\pi} (\BE^2 + \BB^2)
= \inv{16\pi} \left( \Abs{\vec{E}}^2 + \Abs{\vec{B}}^2 + \Real ({\vec{E}}^2 + {\vec{B}}^2) \right).
\end{equation}

This is
\begin{align*}
\mathcal{E} 
&=
\inv{16\pi}
\left(
\frac{\omega^2}{c^2} \Abs{\Bb}^2 + \Abs{\Bb \cross \Bk}^2
-\Real \left(
\frac{\omega^2}{c^2} \Bb^2 + (\Bb \cross \Bk)^2
\right)
e^{2 i(\theta - \alpha)} 
\right).
\end{align*}

Note that $\omega^2/c^2 = \Bk^2$, and $\Abs{\Bb \cross \Bk} = \Abs{\Bb}^2 \Bk^2$ (since $\Bb \cdot \Bk = 0$).  Also $(\Bb \cross \Bk)^2 = \Bb^2 \Bk^2$, so we have

\begin{equation}\label{eqn:relElectroDynProblemSet4:290}
\boxed{
\mathcal{E} 
=
\frac{ \Bk^2 }{8\pi}
\left(
\Abs{\Bb}^2 
-\Real \Bb^2 e^{2 i(\theta - \alpha)} 
\right).
}
\end{equation}

Now, for the Poynting vector.  We have

\begin{equation}\label{eqn:relElectroDynProblemSet4:310}
S = \frac{c}{4 \pi} \BE \cross \BB = \frac{c}{8 \pi} \Real \left( \vec{E} \cross \vec{B}^\conj + \vec{E} \cross \vec{B} \right).
\end{equation}

This is
\begin{align*}
S 
&= \frac{c}{8 \pi} \Real \left( -k \Bb \cross (\Bb^\conj \cross \Bk) + k \Bb \cross (\Bb \cross \Bk ) e^{2 i(\theta - \alpha)} \right) \\
\end{align*}

Reducing the terms we get $\Bb \cross (\Bb^\conj \cross \Bk) = -\Bk \Abs{\Bb}^2$, and $\Bb \cross (\Bb \cross \Bk) = -\Bk \Bb^2$, leaving

\begin{equation}\label{eqn:relElectroDynProblemSet4:330}
\boxed{
S 
= \frac{c \hat{\Bk} \Bk^2 }{8 \pi} \left( \Abs{\Bb}^2 - \Real \Bb^2 e^{2 i(\theta - \alpha)} \right) = c \hat{\Bk} \mathcal{E}
}
\end{equation}

Now, the text in \S 47 defines the energy flux as the Poynting vector, and the momentum density as $\BS/c^2$, so we just divide \ref{eqn:relElectroDynProblemSet4:330} by $c^2$ for the momentum density and we are done.  For the linearly polarized case (all that was actually asked for, but less cool to calculate), where $\Bb$ is real, we have

\begin{align}\label{eqn:relElectroDynProblemSet4:350}
\mbox{Energy density} &= \mathcal{E} = \frac{ \Bk^2 \Bb^2 }{8\pi} ( 1 - \cos( 2 (\omega t - \Bk \cdot \Bx)) ) \\
\mbox{Energy flux} &= \BS = c \hat{\Bk} \mathcal{E} \\
\mbox{Momentum density} &= \inv{c^2} \BS = \frac{\hat{\Bk}}{c} \mathcal{E}
\end{align}

\subsubsection{Part 2.  Averaged.}
\subsubsection{Part 3.  Pressure.}
\subsubsection{Part 4.  Sunlight.}

\section{Problem 2.}

\subsection{Statement}
\subsection{Solution}


\EndArticle
%\EndNoBibArticle


%\part{Possible Text Typos.}

%\part{More worked problems.}

%\part{Cronology}
%\chapter{Cronological Index}
\begin{itemize}

\item October 13, 2007 \ref{chap:gaWiki} Comparison of many traditional vector and GA identities

\item October 13, 2007 \ref{chap:gaWikiTorque} Torque

\item October 16, 2007 \ref{chap:PJUnitDer} Derivatives of a unit vector

\item October 16, 2007 \ref{chap:gaWikiCramers} Cramer's rule

\item October 22, 2007 \ref{chap:PJRadialDer} Radial components of vector derivatives

\item January 1, 2008 \ref{chap:plane} More details on NFCM plane formulation

\item January 29, 2008 \ref{chap:PJAngVel} Rotational dynamics

\item January 29, 2008 \ref{chap:maxwellsGa} Maxwell's equations expressed with Geometric Algebra

\item February 2, 2008 \ref{chap:quaternion} Quaternions

\item February 4, 2008 \ref{chap:legendre} Legendre Polynomials

\item February 15, 2008 \ref{chap:inertialTensor} Inertia Tensor

\item February 19, 2008 \ref{chap:rotor} Rotor Notes

\item February 28, 2008 \ref{chap:laplace} Exponential Solutions to Laplace Equation in \R{N}

\item March 9, 2008 \ref{chap:bivector} Bivector Geometry

\item March 9, 2008 \ref{chap:trivector} Trivector geometry

\item March 12, 2008 \ref{chap:kvectorExponential} Exponential of a blade

\item March 16, 2008 \ref{chap:scalarCommutes} Multivector product grade zero terms

\item March 17, 2008 \ref{chap:angleBetweenLineAndPlane} Angle between geometric elements

\item March 17, 2008 \ref{chap:gaGradeDotWedge} An earlier attempt to intuitively introduce the dot, wedge, cross, and geometric products

\item March 25, 2008 \ref{chap:bladegradereduction} Blade grade reduction

\item March 29, 2008 \ref{chap:reciprocalFrame} Reciprocal Frame Vectors

\item March 31, 2008 \ref{chap:gradientAndForms} Exterior derivative and chain rule components of the gradient

\item April 1, 2008 \ref{chap:orthodecomp} Orthogonal decomposition take II

\item April 11, 2008 \ref{chap:matrixReview} Matrix review

\item April 13, 2008 \ref{chap:locateSatellite} Satellite triangulation over sphere

\item April 30, 2008 \ref{chap:PJKeRot} Kinetic Energy in rotational frame

\item May 7, 2008 \ref{chap:lorentzRotation} Lorentz Force Trajectory

\item May 16, 2008 \ref{chap:obliqueProj} Oblique projection and reciprocal frame vectors

\item May 16, 2008 \ref{chap:matrixOfLinearTx} Matrix of grade k multivector linear transformations

\item May 16, 2008 \ref{chap:projectionAndMoorePenroseVectorInverse} Projection and Moore-Penrose vector inverse

\item May 17, 2008 \ref{chap:PJprojGen} Projection with generalized dot product

\item June 6, 2008 \ref{chap:tensor} Gradient and tensor notes

\item June 10, 2008 \ref{chap:PJAngAcc} Angular Velocity and Acceleration.  Again

\item June 25, 2008 \ref{chap:lorentz} Wave equation based Lorentz transformation derivation

\item July 8, 2008 \ref{chap:PJAngAccCross} Cross product Radial decomposition

\item July 12, 2008 \ref{chap:PJMaxwell2} Back to Maxwell's equations

\item July 16, 2008 \ref{chap:spacetimegrad} Lorentz transformation of spacetime gradient

\item July 20, 2008 \ref{chap:sgMx41} Magnetic field between two parallel wires

\item August 1, 2008 \ref{chap:fourvecDotinvariance} Four vector dot product invariance and Lorentz rotors

\item August 9, 2008 \ref{chap:newtonianLagrangianAndGradient} Newton's Law from Lagrangian

\item August 13, 2008 \ref{chap:cauchyGradient} Cauchy Equations expressed as a gradient

\item August 13, 2008 \ref{chap:velocityTx} Understanding four velocity transform from rest frame

\item August 15, 2008 \ref{chap:emPotential} Four vector potential

\item August 16, 2008 \ref{chap:PJSrGAFPLorentzForce} Lorentz force Law

\item August 21, 2008 \ref{chap:PJSrLagrangian} Covariant Lagrangian, and electrodynamic potential

\item August 25, 2008 \ref{chap:PJTongMf1} Solutions to David Tong's mf1 Lagrangian problems

\item August 28, 2008 \ref{chap:massVaryLagrangian} Equations of motion given mass variation with spacetime position

\item September 1, 2008 \ref{chap:PJCanMomentum} Vector canonical momentum

\item September 2, 2008 \ref{chap:outermorphismDet} OuterMorphism Question 

\item September 5, 2008 \ref{chap:emBivectorMetricDependencies} Metric signature dependencies

\item September 7, 2008 \ref{chap:PJMaxwellTensor} Tensor relations from bivector field equation

\item September 8, 2008 \ref{chap:PJMaxwellLagrangian} Direct variation of Maxwell equations

\item September 9, 2008 \ref{chap:PJMaxwellProj} Vector forms of Maxwell's equations as projection and rejection operations

\item September 18, 2008 \ref{chap:PJStokes1} Stokes law in wedge product form

\item September 26, 2008 \ref{chap:stokesMaxwellApplication} Application of Stokes Integrals to Maxwell's Equation

\item September 27, 2008 \ref{chap:PJStokes2} Stokes Law revisited with algebraic enumeration of boundary

\item October 8, 2008 \ref{chap:PJSrLorentzForce} Revisit Lorentz force from Lagrangian

\item October 10, 2008 \ref{chap:PJFieldLagrangian} Derivation of Euler-Lagrange field equations

\item October 12, 2008 \ref{chap:maxwellTensorLagrangian} Tensor Derivation of Covariant Lorentz Force from Lagrangian

\item October 13, 2008 \ref{chap:PJEulerLagrange} Euler Lagrange Equations

\item October 19, 2008 \ref{chap:PJBoostMaxwell} Lorentz Invariance of Maxwell Lagrangian

\item October 22, 2008 \ref{chap:PJLorentzTxInteraction} Lorentz transform Noether current for interaction Lagrangian

\item October 26, 2008 \ref{chap:gem} GravitoElectroMagnetism

\item October 29, 2008 \ref{chap:PJNoethersField} Field form of Noether's Law

\item November 1, 2008 \ref{chap:eulerangle} Euler Angle Notes

\item November 8, 2008 \ref{chap:complex} Hyper complex numbers and symplectic structure

\item November 13, 2008 \ref{chap:sphericalPolar} Spherical polar coordinates

\item November 22, 2008 \ref{chap:gaussianSurface} Gaussian Surface invariance for radial field

\item November 23, 2008 \ref{chap:chargeArcElement} Field due to line charge in arc

\item November 23, 2008 \ref{chap:chargeLineElement} Charge line element

\item November 27, 2008 \ref{chap:nfcmCh2} Some NFCM exercise solutions and notes

\item November 30, 2008 \ref{chap:PJwaveFourVector} Expressing wave equation exponential solutions using four vectors

\item November 30, 2008 \ref{chap:slerp} Rotor interpolation calculation

\item December 6, 2008 \ref{chap:pauliMatrix} Pauli Matrixes in Clifford Algebra

\item December 11, 2008 \ref{chap:bohr} Bohr Model

\item December 13, 2008 \ref{chap:PJDiracGamma} Gamma Matrices

\item December 21, 2008 \ref{chap:diracLagrangian} Dirac Lagrangian

\item December 27, 2008 \ref{chap:PJrayleighJeans} Rayleigh-Jeans Law Notes

\item December 29, 2008 \ref{chap:PJpoynting} Poynting vector and Electromagnetic Energy conservation

\item January 1, 2009 \ref{chap:PJemstresstensor} Energy momentum tensor

\item January 3, 2009 \ref{chap:PJelectricFieldEnergy} Field and wave energy and momentum

\item January 5, 2009 \ref{chap:vectorDifferentialIdentities} Vector Differential Identities

\item January 6, 2009 \ref{chap:dcPower} DC Power consumption formula for resistive load

\item January 9, 2009 \ref{chap:PJqmFourier} Some Fourier transform notes

\item January 11, 2009 \ref{chap:schCurrent} Schr\"{o}dinger equation probability conservation

\item January 13, 2009 \ref{chap:radial} Polar velocity and acceleration

\item January 18, 2009 \ref{chap:PJpoyntingRate} Time rate of change of the Poynting vector, and its conservation law

\item January 19, 2009 \ref{chap:PJheatFourier} Fourier Solutions to Heat and Wave equations

\item January 21, 2009 \ref{chap:fourierNotation} A cheatsheet for Fourier transform conventions

\item January 25, 2009 \ref{chap:PJemWave} Electrodynamic wave equation solutions

\item January 26, 2009 \ref{chap:PJwaveFourier} Fourier transform solutions to the wave equation

\item January 29, 2009 \ref{chap:PJfourierMaxwellSecondOrder} Fourier transform solutions to Maxwell's equation

\item January 31, 2009 \ref{chap:PJfirstOrderMaxwell} First order Fourier transform solution of Maxwell's equation

\item February 1, 2009 \ref{chap:PJ4dFourier} 4D Fourier transforms applied to Maxwell's equation

\item February 3, 2009 \ref{chap:PJFourierVacuum} Fourier series Vacuum Maxwell's equations

\item February 7, 2009 \ref{chap:potentialFourier} Lorentz Gauge Fourier Vacuum potential solutions

\item February 8, 2009 \ref{chap:PJplaneWave} Plane wave Fourier series solutions to the Maxwell vacuum equation

\item February 13, 2009 \ref{chap:PJstressEnergyLorentz} Lorentz force relation to the energy momentum tensor

\item February 17, 2009 \ref{chap:en_m_tensor} Energy momentum tensor relation to Lorentz force

\item February 18, 2009 \ref{chap:PJpoisson} Poisson and retarded Potential Green's functions from Fourier kernels

\item February 26, 2009 \ref{chap:nvolume} Spherical and hyperspherical parametrization

\item March 13, 2009 \ref{chap:levi} Levi-Civitica summation identity

\item March 18, 2009 \ref{chap:electronRotor} Lorentz force rotor formulation

\item April 15, 2009 \ref{chap:lorentzForcePQA} Lorentz force Lagrangian with conjugate momentum

\item April 18, 2009 \ref{chap:biotSavart} Biot Savart Derivation

\item April 20, 2009 \ref{chap:maxwellTensorFromLagrangian} Tensor derivation of non-dual Maxwell equation from Lagrangian

\item April 28, 2009 \ref{chap:PJmultiTaylors} Developing some intuition for Multivariable and Multivector Taylor Series

\item May 23, 2009 \ref{chap:lorentzForceTx} Lorentz boost of Lorentz force equations

\item May 28, 2009 \ref{chap:macroscopicMaxwell} Macroscopic Maxwell's equation

\item June 1, 2009 \ref{chap:poincareTx} Poincare transformations

\item June 5, 2009 \ref{chap:stressEnergyNoethers} Canonical energy momentum tensor and Lagrangian translation

\item June 17, 2009 \ref{chap:lForceLag2} Comparison of two covariant Lorentz force Lagrangians

\item June 21, 2009 \ref{chap:emVacWave} Wave equation form of Maxwell's equations

\item June 27, 2009 \ref{chap:frequencyTx} Relativistic Doppler formula

\end{itemize}


% END INCLUDES.
%-------------------------------------------------------

\bibliography{myrefs}
\bibliographystyle{unsrturl}
  \addcontentsline{toc}{chapter}{Bibliography}

\end{document}
%%%
% Copyright � 2015 Peeter Joot.  All Rights Reserved.
% Licenced as described in the file LICENSE under the root directory of this GIT repository.
%
\documentclass[]{eliblog}

\usepackage{amsmath}
\usepackage{mathpazo}

%
% shorthand for bold symbols, convenient for vectors and matrices
%
\newcommand{\Ba}[0]{\mathbf{a}}
\newcommand{\Bb}[0]{\mathbf{b}}
\newcommand{\Bc}[0]{\mathbf{c}}
\newcommand{\Bd}[0]{\mathbf{d}}
\newcommand{\Be}[0]{\mathbf{e}}
\newcommand{\Bf}[0]{\mathbf{f}}
\newcommand{\Bg}[0]{\mathbf{g}}
\newcommand{\Bh}[0]{\mathbf{h}}
\newcommand{\Bi}[0]{\mathbf{i}}
\newcommand{\Bj}[0]{\mathbf{j}}
\newcommand{\Bk}[0]{\mathbf{k}}
\newcommand{\Bl}[0]{\mathbf{l}}
\newcommand{\Bm}[0]{\mathbf{m}}
\newcommand{\Bn}[0]{\mathbf{n}}
\newcommand{\Bo}[0]{\mathbf{o}}
\newcommand{\Bp}[0]{\mathbf{p}}
\newcommand{\Bq}[0]{\mathbf{q}}
\newcommand{\Br}[0]{\mathbf{r}}
\newcommand{\Bs}[0]{\mathbf{s}}
\newcommand{\Bt}[0]{\mathbf{t}}
\newcommand{\Bu}[0]{\mathbf{u}}
\newcommand{\Bv}[0]{\mathbf{v}}
\newcommand{\Bw}[0]{\mathbf{w}}
\newcommand{\Bx}[0]{\mathbf{x}}
\newcommand{\By}[0]{\mathbf{y}}
\newcommand{\Bz}[0]{\mathbf{z}}
\newcommand{\BA}[0]{\mathbf{A}}
\newcommand{\BB}[0]{\mathbf{B}}
\newcommand{\BC}[0]{\mathbf{C}}
\newcommand{\BD}[0]{\mathbf{D}}
\newcommand{\BE}[0]{\mathbf{E}}
\newcommand{\BF}[0]{\mathbf{F}}
\newcommand{\BG}[0]{\mathbf{G}}
\newcommand{\BH}[0]{\mathbf{H}}
\newcommand{\BI}[0]{\mathbf{I}}
\newcommand{\BJ}[0]{\mathbf{J}}
\newcommand{\BK}[0]{\mathbf{K}}
\newcommand{\BL}[0]{\mathbf{L}}
\newcommand{\BM}[0]{\mathbf{M}}
\newcommand{\BN}[0]{\mathbf{N}}
\newcommand{\BO}[0]{\mathbf{O}}
\newcommand{\BP}[0]{\mathbf{P}}
\newcommand{\BQ}[0]{\mathbf{Q}}
\newcommand{\BR}[0]{\mathbf{R}}
\newcommand{\BS}[0]{\mathbf{S}}
\newcommand{\BT}[0]{\mathbf{T}}
\newcommand{\BU}[0]{\mathbf{U}}
\newcommand{\BV}[0]{\mathbf{V}}
\newcommand{\BW}[0]{\mathbf{W}}
\newcommand{\BX}[0]{\mathbf{X}}
\newcommand{\BY}[0]{\mathbf{Y}}
\newcommand{\BZ}[0]{\mathbf{Z}}

\newcommand{\Bzero}[0]{\mathbf{0}}
\newcommand{\Btheta}[0]{\boldsymbol{\theta}}
\newcommand{\Btau}[0]{\boldsymbol{\tau}}
\newcommand{\Bomega}[0]{\boldsymbol{\omega}}

%
% shorthand for unit vectors
%
\newcommand{\acap}[0]{\hat{\Ba}}
\newcommand{\bcap}[0]{\hat{\Bb}}
\newcommand{\ccap}[0]{\hat{\Bc}}
\newcommand{\dcap}[0]{\hat{\Bd}}
\newcommand{\ecap}[0]{\hat{\Be}}
\newcommand{\fcap}[0]{\hat{\Bf}}
\newcommand{\gcap}[0]{\hat{\Bg}}
\newcommand{\hcap}[0]{\hat{\Bh}}
\newcommand{\icap}[0]{\hat{\Bi}}
\newcommand{\jcap}[0]{\hat{\Bj}}
\newcommand{\kcap}[0]{\hat{\Bk}}
\newcommand{\lcap}[0]{\hat{\Bl}}
\newcommand{\mcap}[0]{\hat{\Bm}}
\newcommand{\ncap}[0]{\hat{\Bn}}
\newcommand{\ocap}[0]{\hat{\Bo}}
\newcommand{\pcap}[0]{\hat{\Bp}}
\newcommand{\qcap}[0]{\hat{\Bq}}
\newcommand{\rcap}[0]{\hat{\Br}}
\newcommand{\scap}[0]{\hat{\Bs}}
\newcommand{\tcap}[0]{\hat{\Bt}}
\newcommand{\ucap}[0]{\hat{\Bu}}
\newcommand{\vcap}[0]{\hat{\Bv}}
\newcommand{\wcap}[0]{\hat{\Bw}}
\newcommand{\xcap}[0]{\hat{\Bx}}
\newcommand{\ycap}[0]{\hat{\By}}
\newcommand{\zcap}[0]{\hat{\Bz}}
\newcommand{\thetacap}[0]{\hat{\Btheta}}

%
% to write R^n and C^n in a distinguishable fashion.  Perhaps change this
% to the double lined characters upon figuring out how to do so.
%
\newcommand{\C}[1]{$\mathbb{C}^{#1}$}
\newcommand{\R}[1]{$\mathbb{R}^{#1}$}

%
% various generally useful helpers
%

% derivative of #1 wrt. #2:
\newcommand{\D}[2] {\frac {d#2} {d#1}}

\newcommand{\inv}[1]{\frac{1}{#1}}
\newcommand{\cross}[0]{\times}

\newcommand{\abs}[1]{\lvert{#1}\rvert}
\newcommand{\norm}[1]{\lVert{#1}\rVert}
\newcommand{\innerprod}[2]{\langle{#1}, {#2}\rangle}
\newcommand{\dotprod}[2]{{#1} \cdot {#2}}
\newcommand{\bdotprod}[2]{\left({#1} \cdot {#2}\right)}
\newcommand{\crossprod}[2]{{#1} \cross {#2}}
\newcommand{\tripleprod}[3]{\dotprod{\left(\crossprod{#1}{#2}\right)}{#3}}

\DeclareMathOperator{\Proj}{Proj}
\DeclareMathOperator{\Span}{span}
\DeclareMathOperator{\Sgn}{sgn}
\DeclareMathOperator{\Area}{Area}
\DeclareMathOperator{\Volume}{Volume}

%
% A few miscellaneous things specific to this document
%
\newcommand{\crossop}[1]{\crossprod{#1}{}}

% R2 vector.
\newcommand{\VectorTwo}[2]{
\begin{bmatrix}
 {#1} \\
 {#2}
\end{bmatrix}
}

\newcommand{\VectorN}[1]{
\begin{bmatrix}
{#1}_1 \\
{#1}_2 \\
\vdots \\
{#1}_N \\
\end{bmatrix}
}

\newcommand{\DETuvij}[4]{
\begin{vmatrix}
 {#1}_{#3} & {#1}_{#4} \\
 {#2}_{#3} & {#2}_{#4}
\end{vmatrix}
}

\newcommand{\DETuvwijk}[6]{
\begin{vmatrix}
 {#1}_{#4} & {#1}_{#5} & {#1}_{#6} \\
 {#2}_{#4} & {#2}_{#5} & {#2}_{#6} \\
 {#3}_{#4} & {#3}_{#5} & {#3}_{#6}
\end{vmatrix}
}

\newcommand{\DETuvwxijkl}[8]{
\begin{vmatrix}
 {#1}_{#5} & {#1}_{#6} & {#1}_{#7} & {#1}_{#8} \\
 {#2}_{#5} & {#2}_{#6} & {#2}_{#7} & {#2}_{#8} \\
 {#3}_{#5} & {#3}_{#6} & {#3}_{#7} & {#3}_{#8} \\
 {#4}_{#5} & {#4}_{#6} & {#4}_{#7} & {#4}_{#8} \\
\end{vmatrix}
}

%\newcommand{\DETuvwxyijklm}[10]{
%\begin{vmatrix}
% {#1}_{#6} & {#1}_{#7} & {#1}_{#8} & {#1}_{#9} & {#1}_{#10} \\
% {#2}_{#6} & {#2}_{#7} & {#2}_{#8} & {#2}_{#9} & {#2}_{#10} \\
% {#3}_{#6} & {#3}_{#7} & {#3}_{#8} & {#3}_{#9} & {#3}_{#10} \\
% {#4}_{#6} & {#4}_{#7} & {#4}_{#8} & {#4}_{#9} & {#4}_{#10} \\
% {#5}_{#6} & {#5}_{#7} & {#5}_{#8} & {#5}_{#9} & {#5}_{#10}
%\end{vmatrix}
%}

% R3 vector.
\newcommand{\VectorThree}[3]{
\begin{bmatrix}
 {#1} \\
 {#2} \\
 {#3}
\end{bmatrix}
}



\author{Peeter Joot}
\email{peeter.joot@gmail.com}

%\documentclass[]{eliblogwidescreen}

\usepackage{amsmath}
\usepackage{mathpazo}

%
% shorthand for bold symbols, convenient for vectors and matrices
%
\newcommand{\Ba}[0]{\mathbf{a}}
\newcommand{\Bb}[0]{\mathbf{b}}
\newcommand{\Bc}[0]{\mathbf{c}}
\newcommand{\Bd}[0]{\mathbf{d}}
\newcommand{\Be}[0]{\mathbf{e}}
\newcommand{\Bf}[0]{\mathbf{f}}
\newcommand{\Bg}[0]{\mathbf{g}}
\newcommand{\Bh}[0]{\mathbf{h}}
\newcommand{\Bi}[0]{\mathbf{i}}
\newcommand{\Bj}[0]{\mathbf{j}}
\newcommand{\Bk}[0]{\mathbf{k}}
\newcommand{\Bl}[0]{\mathbf{l}}
\newcommand{\Bm}[0]{\mathbf{m}}
\newcommand{\Bn}[0]{\mathbf{n}}
\newcommand{\Bo}[0]{\mathbf{o}}
\newcommand{\Bp}[0]{\mathbf{p}}
\newcommand{\Bq}[0]{\mathbf{q}}
\newcommand{\Br}[0]{\mathbf{r}}
\newcommand{\Bs}[0]{\mathbf{s}}
\newcommand{\Bt}[0]{\mathbf{t}}
\newcommand{\Bu}[0]{\mathbf{u}}
\newcommand{\Bv}[0]{\mathbf{v}}
\newcommand{\Bw}[0]{\mathbf{w}}
\newcommand{\Bx}[0]{\mathbf{x}}
\newcommand{\By}[0]{\mathbf{y}}
\newcommand{\Bz}[0]{\mathbf{z}}
\newcommand{\BA}[0]{\mathbf{A}}
\newcommand{\BB}[0]{\mathbf{B}}
\newcommand{\BC}[0]{\mathbf{C}}
\newcommand{\BD}[0]{\mathbf{D}}
\newcommand{\BE}[0]{\mathbf{E}}
\newcommand{\BF}[0]{\mathbf{F}}
\newcommand{\BG}[0]{\mathbf{G}}
\newcommand{\BH}[0]{\mathbf{H}}
\newcommand{\BI}[0]{\mathbf{I}}
\newcommand{\BJ}[0]{\mathbf{J}}
\newcommand{\BK}[0]{\mathbf{K}}
\newcommand{\BL}[0]{\mathbf{L}}
\newcommand{\BM}[0]{\mathbf{M}}
\newcommand{\BN}[0]{\mathbf{N}}
\newcommand{\BO}[0]{\mathbf{O}}
\newcommand{\BP}[0]{\mathbf{P}}
\newcommand{\BQ}[0]{\mathbf{Q}}
\newcommand{\BR}[0]{\mathbf{R}}
\newcommand{\BS}[0]{\mathbf{S}}
\newcommand{\BT}[0]{\mathbf{T}}
\newcommand{\BU}[0]{\mathbf{U}}
\newcommand{\BV}[0]{\mathbf{V}}
\newcommand{\BW}[0]{\mathbf{W}}
\newcommand{\BX}[0]{\mathbf{X}}
\newcommand{\BY}[0]{\mathbf{Y}}
\newcommand{\BZ}[0]{\mathbf{Z}}

\newcommand{\Bzero}[0]{\mathbf{0}}
\newcommand{\Btheta}[0]{\boldsymbol{\theta}}
\newcommand{\Btau}[0]{\boldsymbol{\tau}}
\newcommand{\Bomega}[0]{\boldsymbol{\omega}}

%
% shorthand for unit vectors
%
\newcommand{\acap}[0]{\hat{\Ba}}
\newcommand{\bcap}[0]{\hat{\Bb}}
\newcommand{\ccap}[0]{\hat{\Bc}}
\newcommand{\dcap}[0]{\hat{\Bd}}
\newcommand{\ecap}[0]{\hat{\Be}}
\newcommand{\fcap}[0]{\hat{\Bf}}
\newcommand{\gcap}[0]{\hat{\Bg}}
\newcommand{\hcap}[0]{\hat{\Bh}}
\newcommand{\icap}[0]{\hat{\Bi}}
\newcommand{\jcap}[0]{\hat{\Bj}}
\newcommand{\kcap}[0]{\hat{\Bk}}
\newcommand{\lcap}[0]{\hat{\Bl}}
\newcommand{\mcap}[0]{\hat{\Bm}}
\newcommand{\ncap}[0]{\hat{\Bn}}
\newcommand{\ocap}[0]{\hat{\Bo}}
\newcommand{\pcap}[0]{\hat{\Bp}}
\newcommand{\qcap}[0]{\hat{\Bq}}
\newcommand{\rcap}[0]{\hat{\Br}}
\newcommand{\scap}[0]{\hat{\Bs}}
\newcommand{\tcap}[0]{\hat{\Bt}}
\newcommand{\ucap}[0]{\hat{\Bu}}
\newcommand{\vcap}[0]{\hat{\Bv}}
\newcommand{\wcap}[0]{\hat{\Bw}}
\newcommand{\xcap}[0]{\hat{\Bx}}
\newcommand{\ycap}[0]{\hat{\By}}
\newcommand{\zcap}[0]{\hat{\Bz}}
\newcommand{\thetacap}[0]{\hat{\Btheta}}

%
% to write R^n and C^n in a distinguishable fashion.  Perhaps change this
% to the double lined characters upon figuring out how to do so.
%
\newcommand{\C}[1]{$\mathbb{C}^{#1}$}
\newcommand{\R}[1]{$\mathbb{R}^{#1}$}

%
% various generally useful helpers
%

% derivative of #1 wrt. #2:
\newcommand{\D}[2] {\frac {d#2} {d#1}}

\newcommand{\inv}[1]{\frac{1}{#1}}
\newcommand{\cross}[0]{\times}

\newcommand{\abs}[1]{\lvert{#1}\rvert}
\newcommand{\norm}[1]{\lVert{#1}\rVert}
\newcommand{\innerprod}[2]{\langle{#1}, {#2}\rangle}
\newcommand{\dotprod}[2]{{#1} \cdot {#2}}
\newcommand{\bdotprod}[2]{\left({#1} \cdot {#2}\right)}
\newcommand{\crossprod}[2]{{#1} \cross {#2}}
\newcommand{\tripleprod}[3]{\dotprod{\left(\crossprod{#1}{#2}\right)}{#3}}

\DeclareMathOperator{\Proj}{Proj}
\DeclareMathOperator{\Span}{span}
\DeclareMathOperator{\Sgn}{sgn}
\DeclareMathOperator{\Area}{Area}
\DeclareMathOperator{\Volume}{Volume}

%
% A few miscellaneous things specific to this document
%
\newcommand{\crossop}[1]{\crossprod{#1}{}}

% R2 vector.
\newcommand{\VectorTwo}[2]{
\begin{bmatrix}
 {#1} \\
 {#2}
\end{bmatrix}
}

\newcommand{\VectorN}[1]{
\begin{bmatrix}
{#1}_1 \\
{#1}_2 \\
\vdots \\
{#1}_N \\
\end{bmatrix}
}

\newcommand{\DETuvij}[4]{
\begin{vmatrix}
 {#1}_{#3} & {#1}_{#4} \\
 {#2}_{#3} & {#2}_{#4}
\end{vmatrix}
}

\newcommand{\DETuvwijk}[6]{
\begin{vmatrix}
 {#1}_{#4} & {#1}_{#5} & {#1}_{#6} \\
 {#2}_{#4} & {#2}_{#5} & {#2}_{#6} \\
 {#3}_{#4} & {#3}_{#5} & {#3}_{#6}
\end{vmatrix}
}

\newcommand{\DETuvwxijkl}[8]{
\begin{vmatrix}
 {#1}_{#5} & {#1}_{#6} & {#1}_{#7} & {#1}_{#8} \\
 {#2}_{#5} & {#2}_{#6} & {#2}_{#7} & {#2}_{#8} \\
 {#3}_{#5} & {#3}_{#6} & {#3}_{#7} & {#3}_{#8} \\
 {#4}_{#5} & {#4}_{#6} & {#4}_{#7} & {#4}_{#8} \\
\end{vmatrix}
}

%\newcommand{\DETuvwxyijklm}[10]{
%\begin{vmatrix}
% {#1}_{#6} & {#1}_{#7} & {#1}_{#8} & {#1}_{#9} & {#1}_{#10} \\
% {#2}_{#6} & {#2}_{#7} & {#2}_{#8} & {#2}_{#9} & {#2}_{#10} \\
% {#3}_{#6} & {#3}_{#7} & {#3}_{#8} & {#3}_{#9} & {#3}_{#10} \\
% {#4}_{#6} & {#4}_{#7} & {#4}_{#8} & {#4}_{#9} & {#4}_{#10} \\
% {#5}_{#6} & {#5}_{#7} & {#5}_{#8} & {#5}_{#9} & {#5}_{#10}
%\end{vmatrix}
%}

% R3 vector.
\newcommand{\VectorThree}[3]{
\begin{bmatrix}
 {#1} \\
 {#2} \\
 {#3}
\end{bmatrix}
}



\author{Peeter Joot}
\email{peeter.joot@gmail.com}


\chapter{Energy Momentum Tensor.}
\label{chap:relativisticElectrodynamicsL22}
%\useCCL
\blogpage{http://sites.google.com/site/peeterjoot/math2011/relativisticElectrodynamicsL22.pdf}
\date{Mar 23, 2011}
\revisionInfo{relativisticElectrodynamicsL22.tex}

%\beginArtWithToc
\beginArtNoToc

\section{Reading.}

Covering \S 32, \S 33 of chapter 4 in the text \citep{landau1980classical}.

Covering \href{http://www.physics.utoronto.ca/~poppitz/epoppitz/PHY450_files/RelEMpp166-180.pdf}{lecture notes pp. 169-172:} spacetime translation invariance of the EM field action and the conservation of the energy-momentum tensor (170-172); properties of the energy-momentum tensor (172.1); the meaning of its components: energy.

\section{Disclaimer.}

I have no class notes for this lecture, as traffic conspired against me and I missed all but the last 5 minutes (a very frustrating drive!)  Here's my own walk through of the content that we must have covered, much of which I did as part of problem set 6 preparation.

\section{Total derivative of the Lagrangian density.}

Rather cleverly, our Professor avoided the spacetime translation arguments of the text.  Inspired by an approach possible in classical mechanics to find that we have a conserved quantity derivable from a force law, he proceeds directly to taking the derivative of the Lagrangian density (see previous lecture notes for details building up to this).

I'll proceed in exactly the same fashion.

\begin{align*}
\partial_k \left( -\inv{16 \pi c} F_{i j} F^{i j} \right) 
&= -\inv{8 \pi c} (\partial_k F_{i j} )F^{i j} \\
&= -\inv{8 \pi c} (\partial_k F_{i j} )F^{i j} \\
&= -\inv{8 \pi c} (\partial_k (\partial_i A_j - \partial_j A_i) )F^{i j} \\
&= -\inv{4 \pi c} (\partial_k \partial_i A_j )F^{i j} \\
&= -\inv{4 \pi c} (\partial_i \partial_k A_j )F^{i j} \\
&= -\inv{4 \pi c} (\partial_m \partial_k A_j )F^{m j} \\
&= -\inv{4 \pi c} \left( \partial_m ((\partial_k A_j )F^{m j}) - (\partial_m F^{m j}) \partial_k A_j \right) \\
&= -\inv{4 \pi c} \left( \partial_m ((\partial_k A_j )F^{m j}) - (\partial_m F^{m a}) \partial_k A_a \right) \\
&= -\inv{4 \pi c} \left( \partial_m ((\partial_k A_j )F^{m j}) - \left(\frac{4 \pi}{c} j^a \right) \partial_k A_a \right) \\
&= -\inv{4 \pi c} \partial_m ((\partial_k A_j )F^{m j}) + \left(\frac{1}{c^2} j^a \right) \partial_k A_a 
\end{align*}

Multiplying through by $c$ and renaming our derivative index using a delta function we have

\begin{equation}\label{eqn:relativisticElectrodynamicsL22:10}
\partial_k \left( -\inv{16 \pi } F_{i j} F^{i j} \right) =
\partial_m {\delta^{m}}_k \left( -\inv{16 \pi } F_{i j} F^{i j} \right) 
= -\inv{4 \pi } \partial_m ((\partial_k A_j )F^{m j}) + \left(\frac{1}{c} j^a \right) \partial_k A_a 
\end{equation}

We can now group the $\partial_m$ terms for

\begin{equation}\label{eqn:relativisticElectrodynamicsL22:30}
\partial_m \left(
-\inv{4 \pi } (\partial_k A_j )F^{m j}
+ {\delta^{m}}_k \inv{16 \pi } F_{i j} F^{i j} 
\right)
= 
- \left(\frac{1}{c} j^a \right) \partial_k A_a 
\end{equation}

Knowing the end goal, a quantity that is expressed in terms of $F^{ij}$ let's raise the $k$ indexes, and any of the $A_i$'s that are along side of those

\begin{equation}\label{eqn:relativisticElectrodynamicsL22:50}
\partial_m \left(
-\inv{4 \pi } (\partial^k A^n )F^{m j} g_{n j}
+ g^{m k} \inv{16 \pi } F_{i j} F^{i j} 
\right)
= 
- \left(\frac{1}{c} j_a \right) \partial^k A^a.
\end{equation}

Next, we want to get rid of the explicit vector potential dependence

\begin{align*}
\partial_m \left( -\inv{4 \pi } (\partial^k A^n )F^{m j} g_{n j} \right)
&=
\partial_m \left( -\inv{4 \pi } (F^{k n} + \partial^n A^k )F^{m j} g_{n j} \right) \\
&=
\partial_m \left( -\inv{4 \pi } F^{k n} F^{m j} g_{n j} 
- \inv{4 \pi} (\partial_m (\partial^n A^k )) F^{m j} g_{n j} 
- \inv{4 \pi} (\partial^n A^k ) (\partial_m F^{m j}) g_{n j} \right) \\
&=
\partial_m \left( -\inv{4 \pi } F^{k n} F^{m j} g_{n j} \right)
- \inv{4 \pi} (\partial_m (\partial^n A^k )) F^{m j} g_{n j} 
- (\partial^n A^k ) \inv{c} j_n \\
&=
\partial_m \left( -\inv{4 \pi } F^{k n} F^{m j} g_{n j} \right)
- \inv{4 \pi} (\partial_m \partial_j A^k ) F^{m j} 
- (\partial^a A^k ) \inv{c} j_a \\
\end{align*}

Since the operator $F^{m j} \partial_m \partial_j$ is a product of symmetric and antisymmetric tensors (or operators), the middle term is zero, and we are left with

\begin{equation}\label{eqn:relativisticElectrodynamicsL22:70}
\partial_m \left(
-\inv{4 \pi } F^{k n} F^{m j} g_{n j} 
+ g^{m k} \inv{16 \pi } F_{i j} F^{i j} 
\right)
= 
- \frac{1}{c} F^{k a} j_a
\end{equation}

This provides the desired conservation relationship

\begin{equation}\label{eqn:relativisticElectrodynamicsL22:90}
\boxed{
\begin{aligned}
\partial_m T^{m k} &= - \inv{c} F^{k a} j_a \\
T^{m k} &=
\inv{4 \pi } \left(
-
F^{m j} 
F^{k n} 
g_{n j} 
+ \frac{g^{m k}}{4} F_{i j} F^{i j} \right)
\end{aligned}
}
\end{equation}

\section{Unpacking the tensor}

\subsection{Energy term of the stress energy tensor.}

\begin{align*}
T^{ 0 0 } 
&=
-\inv{4 \pi} F^{ 0 j} {F^0}_j + \inv{16 \pi} F^{i j} F_{i j} \\
&=
-\inv{4 \pi} F^{ 0 \alpha} {F^0}_\alpha + \inv{16 \pi} \left(
F^{0 j} F_{0 j} 
+F^{\alpha j} F_{\alpha j} 
\right)
\\
&=
\inv{4 \pi} F^{ 0 \alpha} F^{0 \alpha} + \inv{16 \pi} 
\left(
F^{0 \alpha} F_{0 \alpha} 
+F^{\alpha 0} F_{\alpha 0} 
+F^{\alpha \beta} F_{\alpha \beta} 
\right)
\\
&=
\inv{4 \pi} \BE^2 + \inv{16 \pi} \left(
-2 \BE^2 +F^{\alpha \beta} F^{\alpha \beta} 
\right)
\end{align*}

The spatially indexed field tensor components are
\begin{align*}
F^{\alpha \beta} 
&= \partial^\alpha A^\beta - \partial^\beta A^\alpha \\
&= -\partial_\alpha A^\beta + \partial_\beta A^\alpha \\
&= -\epsilon^{\sigma \alpha \beta} (\BB)^\sigma,
\end{align*}

so

\begin{align*}
F^{\alpha \beta} F^{\alpha \beta} 
&= 
\epsilon^{\sigma \alpha \beta} (\BB)^\sigma
\epsilon^{\mu \alpha \beta} (\BB)^\mu \\
&= (2!) \delta^{\sigma \mu} 
(\BB)^\sigma
(\BB)^\mu \\
&= 2 \BB^2
\end{align*}

A final bit of assembly gives us $T^{0 0}$

\begin{equation}\label{eqn:relativisticElectrodynamicsPS6:200}
\boxed{
T^{ 0 0 } = \inv{8 \pi} ( \BE^2 + \BB^2 ) = \mathcal{E}
}
\end{equation}

\subsection{Momentum terms of the stress energy tensor.}

For the spatial $T^{k 0}$ components we have

\begin{align*}
T^{\alpha 0} 
&= 
-\inv{4 \pi} F^{\alpha j} {F^0}_j + \inv{16 \pi} g^{\alpha 0} F^{i j} F_{i j} \\
&= 
-\inv{4 \pi} F^{\alpha j} {F^0}_j \\
&= 
-\inv{4 \pi} 
\left( 
F^{\alpha 0} {F^0}_0 
+F^{\alpha \beta} {F^0}_\beta 
\right) \\
&= 
\inv{4 \pi} F^{\alpha \beta} F^{0 \beta} \\
&= 
\inv{4 \pi} (-\epsilon^{\sigma \alpha \beta} (\BB)^\sigma) (-(\BE)^\beta) \\
&= 
\inv{4 \pi} \epsilon^{\alpha \beta \sigma} 
(\BE)^\beta 
(\BB)^\sigma
\\
\end{align*}

So we have

\begin{equation}\label{eqn:relativisticElectrodynamicsPS6:220}
\boxed{
T^{\alpha 0} = \inv{4 \pi} (\BE \cross \BB)^\alpha = \frac{\BS^\alpha}{c}.
}
\end{equation}

\subsection{Symmetry}

It is simple to show that $T^{k m}$ is symmetric

\begin{align*}
T^{m k} 
&= -\inv{4\pi} F^{m j} {F^{k}}_j + \inv{16 \pi} g^{m k} F^{i j} F_{i j} \\
&= -\inv{4\pi} {F^{m}}_j F^{k j} + \inv{16 \pi} g^{k m} F^{i j} F_{i j} \\
&= T^{k m}
\end{align*}

\subsection{Pressure and shear terms.}

Let's now expand $T^{\beta \alpha}$, starting with the diagonal terms $T^{\alpha\alpha}$.  Because this repeated index isn't summed over, things get slightly irregular, so it's easier to drop the abstraction and just pick a specific $\alpha$, say, $\alpha = 1$.  Then we have

\begin{align*}
T^{1 1} 
&= \inv{4 \pi} \left( - F^{1 k} {F^1}_k - \inv{2} (\BB^2 - \BE^2) \right) \\
&= \inv{4 \pi} \left( - F^{1 0} F^{1 0} + F^{1 \alpha} F^{1 \alpha} - \inv{2} (\BB^2 - \BE^2) \right) \\
&= \inv{4 \pi} \left( 
- E_x^2
+ F^{1 2} F^{1 2}
+ F^{1 3} F^{1 3}
 - \inv{2} (\BB^2 - \BE^2) \right) \\
\end{align*}

For the magnetic components above we have for example

\begin{align*}
F^{1 2} F^{1 2} 
&=
(\partial^1 A^2 - \partial^2 A^2) (\partial^1 A^2 - \partial^2 A^2) \\
&=
(\partial_1 A^2 - \partial_2 A^2) (\partial_1 A^2 - \partial_2 A^2) \\
&=
B_z^2
\end{align*}

So we have

\begin{equation}\label{eqn:relativisticElectrodynamicsPS6:240}
T^{1 1} 
= \inv{4 \pi} \left( 
- E_x^2 + B_y^2 + B_z^2
- \inv{2} (\BB^2 - \BE^2) \right)
\end{equation}

Or

\begin{equation}\label{eqn:relativisticElectrodynamicsPS6:260}
T^{1 1} 
= \inv{8 \pi} \left( 
- E_x^2 + E_y^2 + E_z^2
- B_x^2 + B_y^2 + B_z^2
\right).
\end{equation}

Clearly, the other diagonal terms follow the same pattern, and we can do a cyclic permutation of coordinates to find

\begin{align}\label{eqn:relativisticElectrodynamicsPS6:280}
T^{1 1} &= \inv{8 \pi} \left( - E_x^2 + E_y^2 + E_z^2 - B_x^2 + B_y^2 + B_z^2 \right) \\
T^{2 2} &= \inv{8 \pi} \left( - E_y^2 + E_z^2 + E_x^2 - B_y^2 + B_z^2 + B_x^2 \right) \\
T^{3 3} &= \inv{8 \pi} \left( - E_z^2 + E_x^2 + E_y^2 - B_z^2 + B_x^2 + B_y^2 \right) 
\end{align}

For the off diagonal terms, let's pick $T^{1 2}$ and expand that.  We have

\begin{align*}
T^{1 2} 
&= \inv{4 \pi} \left( - F^{1 k} {F^2}_k - \inv{2} g^{1 2}(\BB^2 - \BE^2) \right) \\
&= \inv{4 \pi} \left( - F^{1 0} F^{2 0} + F^{1 \alpha} F^{2 \alpha} \right) \\
&= \inv{4 \pi} \left( - E_x E_y 
+ \underbrace{F^{1 1}}_{=0} F^{2 1} 
+ F^{1 2} \underbrace{F^{2 2}}_{=0}
+ F^{1 3} F^{2 3} 
\right) \\
&= \inv{4 \pi} \left( - E_x E_y + (-B_y) B_x \right) \\
\end{align*}

Again, with cyclic permutation of the coordinates we have

\begin{align}\label{eqn:relativisticElectrodynamicsPS6:300}
T^{1 2} &= -\inv{4 \pi} \left( E_x E_y + B_x B_y \right) \\
T^{2 3} &= -\inv{4 \pi} \left( E_y E_z + B_y B_z \right) \\
T^{3 1} &= -\inv{4 \pi} \left( E_z E_x + B_z B_x \right) 
\end{align}

In class these were all written in the compact notation

\begin{equation}\label{eqn:relativisticElectrodynamicsPS6:320}
\boxed{
T^{\alpha \beta} = -\inv{4 \pi} \left( 
E_\alpha E_\beta 
+B_\alpha B_\beta 
- \inv{2} \delta_{\alpha\beta} (\BE^2 + \BB^2) \right)
}
\end{equation}

\EndArticle

