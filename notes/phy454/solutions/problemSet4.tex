\chapter{Problem Set 4.  Thermal stability.}
\label{chap:continuumProblemSet4}

\begin{Exercise}
What is the physical meaning of the Rayleigh number?
\end{Exercise}
\begin{Exercise}
If $R_c = \frac{27 \pi^4}{4}$ is the critical Rayleigh number for the onset of convection for water, what is the corresponding critical temperature difference between the top and bottom plates in a $10 \text{cm}$ layer of fluid?
\end{Exercise}
\begin{Exercise}
What is the dimensional value of the critical wavelength of the convection cells?
\end{Exercise}
\begin{Exercise}
If instead of taking stress free boundary conditions at the top and bottom plates, if we consider both the plates 'rigid' (no-slip) how does the solution of \ref{eqn:continuumL22:970} change?
\end{Exercise}
\begin{Exercise}
Consider the problem

\begin{equation}\label{eqn:problemSet4:10}
\PD{t}{u} - \sin u = \inv{R} \PDSq{y}{u}
\end{equation}

where $R$ is a real parameter and the boundary conditions are given by

\begin{equation}\label{eqn:problemSet4:20}
u(y = 0, t) = u(y = \pi, t) = 0
\end{equation}

for all time $t$. Examine the trivial base state $u = u_B(y) = 0$ by seeking normal mode solutions to the linearized perturbed equations. Find the eigenfunctions and eigenvalues and show that the base state is linearly stable only if $R \le 1$.
\end{Exercise}
