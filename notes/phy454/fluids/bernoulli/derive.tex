%
% Copyright � 2012 Peeter Joot.  All Rights Reserved.
% Licenced as described in the file LICENSE under the root directory of this GIT repository.
%
\section{Derivation}

We start with Navier-Stokes in vector form

\begin{equation}\label{eqn:continuumL18:290}
\PD{t}{\Bu} + (\Bu \cdot \spacegrad) \Bu = -\spacegrad \left( \frac{p}{\rho} \right) + \nu \spacegrad^2 \Bu + \Bg.
\end{equation}

Writing the body force as a potential

\begin{equation}\label{eqn:continuumL18:310}
\Bg = -\spacegrad \chi,
\end{equation}

so that we have

\begin{equation}\label{eqn:continuumL18:330}
\PD{t}{\Bu} + (\Bu \cdot \spacegrad) \Bu = -\spacegrad \left( \frac{p}{\rho} + \chi \right) + \nu \spacegrad^2 \Bu .
\end{equation}

Using the vector identity

\begin{equation}\label{eqn:continuumL18:350}
(\Bu \cdot \spacegrad) \Bu = \spacegrad \cross (\spacegrad \cross \Bu) + \spacegrad \left( \inv{2} \Bu^2 \right),
\end{equation}

we can write

\begin{equation}\label{eqn:continuumL18:370}
\PD{t}{\Bu} + 
\spacegrad \cross (\spacegrad \cross \Bu) + \spacegrad \left( \inv{2} \Bu^2 \right)
= -\spacegrad \left( \frac{p}{\rho} + \chi \right) + \nu \spacegrad^2 \Bu .
\end{equation}

If we consider the non-viscous region of the flow (far from the boundary layer), we can kill the Laplacian term.  Again, considering only the steady state, and assuming that we have irrotational flow (\(\spacegrad \cross \Bu = 0\)) in the non-viscous region, we have

\begin{equation}\label{eqn:continuumL18:390}
\spacegrad \left( \frac{p}{\rho} + \chi + \inv{2} \Bu^2 \right) = 0.
\end{equation}

or

\boxedEquation{eqn:continuumL18:410}{
\frac{p}{\rho} + \chi + \inv{2} \Bu^2 = \text{constant}
}

This is the Bernoulli equation.
