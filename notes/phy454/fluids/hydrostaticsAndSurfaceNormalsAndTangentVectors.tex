% 
% 
% 
% Copyright � 2012 Peeter Joot
% All Rights Reserved
% 
% This file may be reproduced and distributed in whole or in part, without fee, subject to the following conditions:
% 
% o The copyright notice above and this permission notice must be preserved complete on all complete or partial copies.
% 
% o Any translation or derived work must be approved by the author in writing before distribution.
% 
% o If you distribute this work in part, instructions for obtaining the complete version of this file must be included, and a means for obtaining a complete version provided.
% 
% 
% Exceptions to these rules may be granted for academic purposes: Write to the author and ask.
% 
% 
% 
%%
% Copyright � 2015 Peeter Joot.  All Rights Reserved.
% Licenced as described in the file LICENSE under the root directory of this GIT repository.
%
\documentclass[]{eliblog}

\usepackage{amsmath}
\usepackage{mathpazo}

%
% shorthand for bold symbols, convenient for vectors and matrices
%
\newcommand{\Ba}[0]{\mathbf{a}}
\newcommand{\Bb}[0]{\mathbf{b}}
\newcommand{\Bc}[0]{\mathbf{c}}
\newcommand{\Bd}[0]{\mathbf{d}}
\newcommand{\Be}[0]{\mathbf{e}}
\newcommand{\Bf}[0]{\mathbf{f}}
\newcommand{\Bg}[0]{\mathbf{g}}
\newcommand{\Bh}[0]{\mathbf{h}}
\newcommand{\Bi}[0]{\mathbf{i}}
\newcommand{\Bj}[0]{\mathbf{j}}
\newcommand{\Bk}[0]{\mathbf{k}}
\newcommand{\Bl}[0]{\mathbf{l}}
\newcommand{\Bm}[0]{\mathbf{m}}
\newcommand{\Bn}[0]{\mathbf{n}}
\newcommand{\Bo}[0]{\mathbf{o}}
\newcommand{\Bp}[0]{\mathbf{p}}
\newcommand{\Bq}[0]{\mathbf{q}}
\newcommand{\Br}[0]{\mathbf{r}}
\newcommand{\Bs}[0]{\mathbf{s}}
\newcommand{\Bt}[0]{\mathbf{t}}
\newcommand{\Bu}[0]{\mathbf{u}}
\newcommand{\Bv}[0]{\mathbf{v}}
\newcommand{\Bw}[0]{\mathbf{w}}
\newcommand{\Bx}[0]{\mathbf{x}}
\newcommand{\By}[0]{\mathbf{y}}
\newcommand{\Bz}[0]{\mathbf{z}}
\newcommand{\BA}[0]{\mathbf{A}}
\newcommand{\BB}[0]{\mathbf{B}}
\newcommand{\BC}[0]{\mathbf{C}}
\newcommand{\BD}[0]{\mathbf{D}}
\newcommand{\BE}[0]{\mathbf{E}}
\newcommand{\BF}[0]{\mathbf{F}}
\newcommand{\BG}[0]{\mathbf{G}}
\newcommand{\BH}[0]{\mathbf{H}}
\newcommand{\BI}[0]{\mathbf{I}}
\newcommand{\BJ}[0]{\mathbf{J}}
\newcommand{\BK}[0]{\mathbf{K}}
\newcommand{\BL}[0]{\mathbf{L}}
\newcommand{\BM}[0]{\mathbf{M}}
\newcommand{\BN}[0]{\mathbf{N}}
\newcommand{\BO}[0]{\mathbf{O}}
\newcommand{\BP}[0]{\mathbf{P}}
\newcommand{\BQ}[0]{\mathbf{Q}}
\newcommand{\BR}[0]{\mathbf{R}}
\newcommand{\BS}[0]{\mathbf{S}}
\newcommand{\BT}[0]{\mathbf{T}}
\newcommand{\BU}[0]{\mathbf{U}}
\newcommand{\BV}[0]{\mathbf{V}}
\newcommand{\BW}[0]{\mathbf{W}}
\newcommand{\BX}[0]{\mathbf{X}}
\newcommand{\BY}[0]{\mathbf{Y}}
\newcommand{\BZ}[0]{\mathbf{Z}}

\newcommand{\Bzero}[0]{\mathbf{0}}
\newcommand{\Btheta}[0]{\boldsymbol{\theta}}
\newcommand{\Btau}[0]{\boldsymbol{\tau}}
\newcommand{\Bomega}[0]{\boldsymbol{\omega}}

%
% shorthand for unit vectors
%
\newcommand{\acap}[0]{\hat{\Ba}}
\newcommand{\bcap}[0]{\hat{\Bb}}
\newcommand{\ccap}[0]{\hat{\Bc}}
\newcommand{\dcap}[0]{\hat{\Bd}}
\newcommand{\ecap}[0]{\hat{\Be}}
\newcommand{\fcap}[0]{\hat{\Bf}}
\newcommand{\gcap}[0]{\hat{\Bg}}
\newcommand{\hcap}[0]{\hat{\Bh}}
\newcommand{\icap}[0]{\hat{\Bi}}
\newcommand{\jcap}[0]{\hat{\Bj}}
\newcommand{\kcap}[0]{\hat{\Bk}}
\newcommand{\lcap}[0]{\hat{\Bl}}
\newcommand{\mcap}[0]{\hat{\Bm}}
\newcommand{\ncap}[0]{\hat{\Bn}}
\newcommand{\ocap}[0]{\hat{\Bo}}
\newcommand{\pcap}[0]{\hat{\Bp}}
\newcommand{\qcap}[0]{\hat{\Bq}}
\newcommand{\rcap}[0]{\hat{\Br}}
\newcommand{\scap}[0]{\hat{\Bs}}
\newcommand{\tcap}[0]{\hat{\Bt}}
\newcommand{\ucap}[0]{\hat{\Bu}}
\newcommand{\vcap}[0]{\hat{\Bv}}
\newcommand{\wcap}[0]{\hat{\Bw}}
\newcommand{\xcap}[0]{\hat{\Bx}}
\newcommand{\ycap}[0]{\hat{\By}}
\newcommand{\zcap}[0]{\hat{\Bz}}
\newcommand{\thetacap}[0]{\hat{\Btheta}}

%
% to write R^n and C^n in a distinguishable fashion.  Perhaps change this
% to the double lined characters upon figuring out how to do so.
%
\newcommand{\C}[1]{$\mathbb{C}^{#1}$}
\newcommand{\R}[1]{$\mathbb{R}^{#1}$}

%
% various generally useful helpers
%

% derivative of #1 wrt. #2:
\newcommand{\D}[2] {\frac {d#2} {d#1}}

\newcommand{\inv}[1]{\frac{1}{#1}}
\newcommand{\cross}[0]{\times}

\newcommand{\abs}[1]{\lvert{#1}\rvert}
\newcommand{\norm}[1]{\lVert{#1}\rVert}
\newcommand{\innerprod}[2]{\langle{#1}, {#2}\rangle}
\newcommand{\dotprod}[2]{{#1} \cdot {#2}}
\newcommand{\bdotprod}[2]{\left({#1} \cdot {#2}\right)}
\newcommand{\crossprod}[2]{{#1} \cross {#2}}
\newcommand{\tripleprod}[3]{\dotprod{\left(\crossprod{#1}{#2}\right)}{#3}}

\DeclareMathOperator{\Proj}{Proj}
\DeclareMathOperator{\Span}{span}
\DeclareMathOperator{\Sgn}{sgn}
\DeclareMathOperator{\Area}{Area}
\DeclareMathOperator{\Volume}{Volume}

%
% A few miscellaneous things specific to this document
%
\newcommand{\crossop}[1]{\crossprod{#1}{}}

% R2 vector.
\newcommand{\VectorTwo}[2]{
\begin{bmatrix}
 {#1} \\
 {#2}
\end{bmatrix}
}

\newcommand{\VectorN}[1]{
\begin{bmatrix}
{#1}_1 \\
{#1}_2 \\
\vdots \\
{#1}_N \\
\end{bmatrix}
}

\newcommand{\DETuvij}[4]{
\begin{vmatrix}
 {#1}_{#3} & {#1}_{#4} \\
 {#2}_{#3} & {#2}_{#4}
\end{vmatrix}
}

\newcommand{\DETuvwijk}[6]{
\begin{vmatrix}
 {#1}_{#4} & {#1}_{#5} & {#1}_{#6} \\
 {#2}_{#4} & {#2}_{#5} & {#2}_{#6} \\
 {#3}_{#4} & {#3}_{#5} & {#3}_{#6}
\end{vmatrix}
}

\newcommand{\DETuvwxijkl}[8]{
\begin{vmatrix}
 {#1}_{#5} & {#1}_{#6} & {#1}_{#7} & {#1}_{#8} \\
 {#2}_{#5} & {#2}_{#6} & {#2}_{#7} & {#2}_{#8} \\
 {#3}_{#5} & {#3}_{#6} & {#3}_{#7} & {#3}_{#8} \\
 {#4}_{#5} & {#4}_{#6} & {#4}_{#7} & {#4}_{#8} \\
\end{vmatrix}
}

%\newcommand{\DETuvwxyijklm}[10]{
%\begin{vmatrix}
% {#1}_{#6} & {#1}_{#7} & {#1}_{#8} & {#1}_{#9} & {#1}_{#10} \\
% {#2}_{#6} & {#2}_{#7} & {#2}_{#8} & {#2}_{#9} & {#2}_{#10} \\
% {#3}_{#6} & {#3}_{#7} & {#3}_{#8} & {#3}_{#9} & {#3}_{#10} \\
% {#4}_{#6} & {#4}_{#7} & {#4}_{#8} & {#4}_{#9} & {#4}_{#10} \\
% {#5}_{#6} & {#5}_{#7} & {#5}_{#8} & {#5}_{#9} & {#5}_{#10}
%\end{vmatrix}
%}

% R3 vector.
\newcommand{\VectorThree}[3]{
\begin{bmatrix}
 {#1} \\
 {#2} \\
 {#3}
\end{bmatrix}
}



\author{Peeter Joot}
\email{peeter.joot@gmail.com}

%\documentclass[]{eliblogwidescreen}

\usepackage{amsmath}
\usepackage{mathpazo}

%
% shorthand for bold symbols, convenient for vectors and matrices
%
\newcommand{\Ba}[0]{\mathbf{a}}
\newcommand{\Bb}[0]{\mathbf{b}}
\newcommand{\Bc}[0]{\mathbf{c}}
\newcommand{\Bd}[0]{\mathbf{d}}
\newcommand{\Be}[0]{\mathbf{e}}
\newcommand{\Bf}[0]{\mathbf{f}}
\newcommand{\Bg}[0]{\mathbf{g}}
\newcommand{\Bh}[0]{\mathbf{h}}
\newcommand{\Bi}[0]{\mathbf{i}}
\newcommand{\Bj}[0]{\mathbf{j}}
\newcommand{\Bk}[0]{\mathbf{k}}
\newcommand{\Bl}[0]{\mathbf{l}}
\newcommand{\Bm}[0]{\mathbf{m}}
\newcommand{\Bn}[0]{\mathbf{n}}
\newcommand{\Bo}[0]{\mathbf{o}}
\newcommand{\Bp}[0]{\mathbf{p}}
\newcommand{\Bq}[0]{\mathbf{q}}
\newcommand{\Br}[0]{\mathbf{r}}
\newcommand{\Bs}[0]{\mathbf{s}}
\newcommand{\Bt}[0]{\mathbf{t}}
\newcommand{\Bu}[0]{\mathbf{u}}
\newcommand{\Bv}[0]{\mathbf{v}}
\newcommand{\Bw}[0]{\mathbf{w}}
\newcommand{\Bx}[0]{\mathbf{x}}
\newcommand{\By}[0]{\mathbf{y}}
\newcommand{\Bz}[0]{\mathbf{z}}
\newcommand{\BA}[0]{\mathbf{A}}
\newcommand{\BB}[0]{\mathbf{B}}
\newcommand{\BC}[0]{\mathbf{C}}
\newcommand{\BD}[0]{\mathbf{D}}
\newcommand{\BE}[0]{\mathbf{E}}
\newcommand{\BF}[0]{\mathbf{F}}
\newcommand{\BG}[0]{\mathbf{G}}
\newcommand{\BH}[0]{\mathbf{H}}
\newcommand{\BI}[0]{\mathbf{I}}
\newcommand{\BJ}[0]{\mathbf{J}}
\newcommand{\BK}[0]{\mathbf{K}}
\newcommand{\BL}[0]{\mathbf{L}}
\newcommand{\BM}[0]{\mathbf{M}}
\newcommand{\BN}[0]{\mathbf{N}}
\newcommand{\BO}[0]{\mathbf{O}}
\newcommand{\BP}[0]{\mathbf{P}}
\newcommand{\BQ}[0]{\mathbf{Q}}
\newcommand{\BR}[0]{\mathbf{R}}
\newcommand{\BS}[0]{\mathbf{S}}
\newcommand{\BT}[0]{\mathbf{T}}
\newcommand{\BU}[0]{\mathbf{U}}
\newcommand{\BV}[0]{\mathbf{V}}
\newcommand{\BW}[0]{\mathbf{W}}
\newcommand{\BX}[0]{\mathbf{X}}
\newcommand{\BY}[0]{\mathbf{Y}}
\newcommand{\BZ}[0]{\mathbf{Z}}

\newcommand{\Bzero}[0]{\mathbf{0}}
\newcommand{\Btheta}[0]{\boldsymbol{\theta}}
\newcommand{\Btau}[0]{\boldsymbol{\tau}}
\newcommand{\Bomega}[0]{\boldsymbol{\omega}}

%
% shorthand for unit vectors
%
\newcommand{\acap}[0]{\hat{\Ba}}
\newcommand{\bcap}[0]{\hat{\Bb}}
\newcommand{\ccap}[0]{\hat{\Bc}}
\newcommand{\dcap}[0]{\hat{\Bd}}
\newcommand{\ecap}[0]{\hat{\Be}}
\newcommand{\fcap}[0]{\hat{\Bf}}
\newcommand{\gcap}[0]{\hat{\Bg}}
\newcommand{\hcap}[0]{\hat{\Bh}}
\newcommand{\icap}[0]{\hat{\Bi}}
\newcommand{\jcap}[0]{\hat{\Bj}}
\newcommand{\kcap}[0]{\hat{\Bk}}
\newcommand{\lcap}[0]{\hat{\Bl}}
\newcommand{\mcap}[0]{\hat{\Bm}}
\newcommand{\ncap}[0]{\hat{\Bn}}
\newcommand{\ocap}[0]{\hat{\Bo}}
\newcommand{\pcap}[0]{\hat{\Bp}}
\newcommand{\qcap}[0]{\hat{\Bq}}
\newcommand{\rcap}[0]{\hat{\Br}}
\newcommand{\scap}[0]{\hat{\Bs}}
\newcommand{\tcap}[0]{\hat{\Bt}}
\newcommand{\ucap}[0]{\hat{\Bu}}
\newcommand{\vcap}[0]{\hat{\Bv}}
\newcommand{\wcap}[0]{\hat{\Bw}}
\newcommand{\xcap}[0]{\hat{\Bx}}
\newcommand{\ycap}[0]{\hat{\By}}
\newcommand{\zcap}[0]{\hat{\Bz}}
\newcommand{\thetacap}[0]{\hat{\Btheta}}

%
% to write R^n and C^n in a distinguishable fashion.  Perhaps change this
% to the double lined characters upon figuring out how to do so.
%
\newcommand{\C}[1]{$\mathbb{C}^{#1}$}
\newcommand{\R}[1]{$\mathbb{R}^{#1}$}

%
% various generally useful helpers
%

% derivative of #1 wrt. #2:
\newcommand{\D}[2] {\frac {d#2} {d#1}}

\newcommand{\inv}[1]{\frac{1}{#1}}
\newcommand{\cross}[0]{\times}

\newcommand{\abs}[1]{\lvert{#1}\rvert}
\newcommand{\norm}[1]{\lVert{#1}\rVert}
\newcommand{\innerprod}[2]{\langle{#1}, {#2}\rangle}
\newcommand{\dotprod}[2]{{#1} \cdot {#2}}
\newcommand{\bdotprod}[2]{\left({#1} \cdot {#2}\right)}
\newcommand{\crossprod}[2]{{#1} \cross {#2}}
\newcommand{\tripleprod}[3]{\dotprod{\left(\crossprod{#1}{#2}\right)}{#3}}

\DeclareMathOperator{\Proj}{Proj}
\DeclareMathOperator{\Span}{span}
\DeclareMathOperator{\Sgn}{sgn}
\DeclareMathOperator{\Area}{Area}
\DeclareMathOperator{\Volume}{Volume}

%
% A few miscellaneous things specific to this document
%
\newcommand{\crossop}[1]{\crossprod{#1}{}}

% R2 vector.
\newcommand{\VectorTwo}[2]{
\begin{bmatrix}
 {#1} \\
 {#2}
\end{bmatrix}
}

\newcommand{\VectorN}[1]{
\begin{bmatrix}
{#1}_1 \\
{#1}_2 \\
\vdots \\
{#1}_N \\
\end{bmatrix}
}

\newcommand{\DETuvij}[4]{
\begin{vmatrix}
 {#1}_{#3} & {#1}_{#4} \\
 {#2}_{#3} & {#2}_{#4}
\end{vmatrix}
}

\newcommand{\DETuvwijk}[6]{
\begin{vmatrix}
 {#1}_{#4} & {#1}_{#5} & {#1}_{#6} \\
 {#2}_{#4} & {#2}_{#5} & {#2}_{#6} \\
 {#3}_{#4} & {#3}_{#5} & {#3}_{#6}
\end{vmatrix}
}

\newcommand{\DETuvwxijkl}[8]{
\begin{vmatrix}
 {#1}_{#5} & {#1}_{#6} & {#1}_{#7} & {#1}_{#8} \\
 {#2}_{#5} & {#2}_{#6} & {#2}_{#7} & {#2}_{#8} \\
 {#3}_{#5} & {#3}_{#6} & {#3}_{#7} & {#3}_{#8} \\
 {#4}_{#5} & {#4}_{#6} & {#4}_{#7} & {#4}_{#8} \\
\end{vmatrix}
}

%\newcommand{\DETuvwxyijklm}[10]{
%\begin{vmatrix}
% {#1}_{#6} & {#1}_{#7} & {#1}_{#8} & {#1}_{#9} & {#1}_{#10} \\
% {#2}_{#6} & {#2}_{#7} & {#2}_{#8} & {#2}_{#9} & {#2}_{#10} \\
% {#3}_{#6} & {#3}_{#7} & {#3}_{#8} & {#3}_{#9} & {#3}_{#10} \\
% {#4}_{#6} & {#4}_{#7} & {#4}_{#8} & {#4}_{#9} & {#4}_{#10} \\
% {#5}_{#6} & {#5}_{#7} & {#5}_{#8} & {#5}_{#9} & {#5}_{#10}
%\end{vmatrix}
%}

% R3 vector.
\newcommand{\VectorThree}[3]{
\begin{bmatrix}
 {#1} \\
 {#2} \\
 {#3}
\end{bmatrix}
}



\author{Peeter Joot}
\email{peeter.joot@gmail.com}


%\chapter{PHY454H1S Continuum Mechanics.  Lecture 13: Hydrostatics.  Surface normals and tangent vectors.  Taught by Prof. K. Das.}
\chapter{Hydrostatics.  Surface normals and tangent vectors.}
\label{chap:continuumL13}
%\useCCL
\blogpage{http://sites.google.com/site/peeterjoot2/math2012/continuumL13.pdf}
%\date{Mar 2, 2012}
\gitRevisionInfo{continuumL13}

\section{Midterm discussion.}

We had been asked to solve a 1D flow problem with one fixed surface, and one moving surface, as illustrated in figure (\ref{fig:continuumL13:continuumL13Fig1})
\begin{figure}[htp]
   \centering
   \includegraphics[totalheight=0.2\textheight]{continuumL13Fig1}
   \caption{Return flow configuration.  One fixed surface, one moving surface.}\label{fig:continuumL13:continuumL13Fig1}
\end{figure}

we end up showing that our solutions are of the forms

\begin{itemize}
\item 
Shear flow (\ref{fig:continuumL13:continuumL13Fig2a}).  $U \ne 0, G = 0$.
\begin{figure}[htp]
   \centering
   \includegraphics[totalheight=0.2\textheight]{continuumL13Fig2a}
   \caption{Shear Flow.}\label{fig:continuumL13:continuumL13Fig2a}
\end{figure}
\item Channel flow (\ref{fig:continuumL13:continuumL13Fig2b}).  $U = 0, G \ne 0$.
\begin{figure}[htp]
   \centering
   \includegraphics[totalheight=0.2\textheight]{continuumL13Fig2b}
   \caption{Channel flow}\label{fig:continuumL13:continuumL13Fig2b}
\end{figure}
\item Return flow (\ref{fig:continuumL13:continuumL13Fig2c}).  $U \ne 0, G \ne 0$.
\begin{figure}[htp]
   \centering
   \includegraphics[totalheight=0.2\textheight]{continuumL13Fig2c}
   \caption{Return flow.}\label{fig:continuumL13:continuumL13Fig2c}
\end{figure}
\end{itemize}

Note that the last sort of solution, that of return flow, was discussed in the context of surface tension.

\section{Hydrostatics}

Consider a sample volume of water, not moving with respect to the rest of the surrounding water.  If it is not moving the forces must be in balance.  What are the forces acting on this bit of fluid, considering a cylinder of the fluid above it as in figure (\ref{fig:continuumL13:continuumL13Fig3a})
\begin{figure}[htp]
   \centering
   \includegraphics[totalheight=0.2\textheight]{continuumL13Fig3a}
   \caption{A control volume of fluid in a fluid.}\label{fig:continuumL13:continuumL13Fig3a}
\end{figure}

In the column of fluid above the control volume (\ref{fig:continuumL13:continuumL13Fig3b}) we have
\begin{figure}[htp]
   \centering
   \includegraphics[totalheight=0.2\textheight]{continuumL13Fig3b}
   \caption{column of fluid above a control volume.}\label{fig:continuumL13:continuumL13Fig3b}
\end{figure}

\begin{equation}\label{eqn:continuumL13:20}
h A_w \rho g + p_A A_w = p_w A_w
\end{equation}

so
\begin{equation}\label{eqn:continuumL13:40}
p_w = h \rho g + p_A
\end{equation}

If we were to replace this blob of water with something of equal density, it should not change the dynamics (or statics) of the situations and that would not move.

We call this the

\begin{definition}
\emph{(Buoyancy force)}
\label{dfn:continuumL13:60}
Buoyancy force = 
weight of the equivalent volume of water - weight of the foreign body.
\end{definition}

If the densities are not equal, then we'd have motion of the new bit of mass as depicted in figure (\ref{fig:continuumL13:continuumL13Fig4})
\begin{figure}[htp]
   \centering
   \includegraphics[totalheight=0.2\textheight]{continuumL13Fig4}
   \caption{A mass of different density in a fluid.}\label{fig:continuumL13:continuumL13Fig4}
\end{figure}

Consider a volume of ice floating on the surface of water, one with solid ice and one with partially frozen ice (with water or air or dirt or an anchor or anything else in it) as in figure (\ref{fig:continuumL13:continuumL13Fig5})

\begin{figure}[htp]
   \centering
   \includegraphics[totalheight=0.2\textheight]{continuumL13Fig5}
   \caption{Various floating ice configurations on water.}\label{fig:continuumL13:continuumL13Fig5}
\end{figure}

No matter the situation, the water level will not change if the ice melts, because the total weight of the displaced water must have been matched by the weight of the unmelted ice plus additives.

Now what happens when we have fluid flows?  Consider figure (\ref{fig:continuumL13:continuumL13Fig6})
\begin{figure}[htp]
   \centering
   \includegraphics[totalheight=0.2\textheight]{continuumL13Fig6}
   \caption{flow through channel with different apertures.}\label{fig:continuumL13:continuumL13Fig6}
\end{figure}

Conservation of mass is going to mean that the masses of fluid flowing through any pair of cross sections will have to be equal

\begin{equation}\label{eqn:continuumL13:80}
\rho_1 A_1 v_1 = \rho_2 A_2 v_2,
\end{equation}

With incompressible fluids ($\rho = \rho_1 = \rho_2$) we have

\begin{equation}\label{eqn:continuumL13:100}
A_1 v_1 = A_2 v_2,
\end{equation}

so that if 

\begin{equation}\label{eqn:continuumL13:120}
A_1 > A_2,
\end{equation}

we must have
\begin{equation}\label{eqn:continuumL13:140}
v_1 < v_2,
\end{equation}

to balance this.

In class this was illustrated with a pair of computer animations, one showing the deformation of patches of the fluid, and another showing how the velocities vary through the channel.  This is crudely depicted in figure (\ref{fig:continuumL13:continuumL13Fig7})

\begin{figure}[htp]
   \centering
   \includegraphics[totalheight=0.2\textheight]{continuumL13Fig7}
   \caption{area and velocity flows in unequal aperture channel configuration.}\label{fig:continuumL13:continuumL13Fig7}
\end{figure}

We see the same behavior for channels that return to the original diameter after widening as in figure (\ref{fig:continuumL13:continuumL13Fig8})

\begin{figure}[htp]
   \centering
   \includegraphics[totalheight=0.2\textheight]{continuumL13Fig8}
   \caption{velocity variation in channel with bulge}\label{fig:continuumL13:continuumL13Fig8}
\end{figure}

If we consider half of such a channel as in figure (\ref{fig:continuumL13:continuumL13Fig9a})
\begin{figure}[htp]
   \centering
   \includegraphics[totalheight=0.2\textheight]{continuumL13Fig9a}
   \caption{vorticity induction due to pressure gradients in unequal aperture channel.}\label{fig:continuumL13:continuumL13Fig9a}
\end{figure}

considering the flow around a small triangular section we must have a pressure gradient, which induces a vorticity flow.  We'd see something similar in a rectangular channel where there is a block in the channel, as depicted in figure (\ref{fig:continuumL13:continuumL13Fig9b})

\begin{figure}[htp]
   \centering
   \includegraphics[totalheight=0.2\textheight]{continuumL13Fig9b}
   \caption{vorticity due to rectangular blockage.}\label{fig:continuumL13:continuumL13Fig9b}
\end{figure}

\subsection{Back to hydrostatics.  Height matching in odd geometries.}

Now let's consider the hydrostatics case again, with an arbitrarily weird channel as in figure (\ref{fig:continuumL13:continuumL13Fig10})

\begin{figure}[htp]
   \centering
   \includegraphics[totalheight=0.2\textheight]{continuumL13Fig10}
   \caption{height matching in odd geometries.}\label{fig:continuumL13:continuumL13Fig10}
\end{figure}

This was also illustrated with a glass blown container in class as in figure (\ref{fig:continuumL13:continuumL13Fig11})

\begin{figure}[htp]
   \centering
   \includegraphics[totalheight=0.2\textheight]{continuumL13Fig11}
   \caption{a physical demonstration with glass blown apparatus.}\label{fig:continuumL13:continuumL13Fig11}
\end{figure}

In this real apparatus, we didn't have exactly the same height (because of bubbles and capillary effects (surface tension induced meniscus curves), but we see first hand what we are talking about.

To account for this, we need to consider the situation in pieces as in figure (\ref{fig:continuumL13:continuumL13Fig12})

\begin{figure}[htp]
   \centering
   \includegraphics[totalheight=0.2\textheight]{continuumL13Fig12}
   \caption{column volume element decomposition for odd geometries}\label{fig:continuumL13:continuumL13Fig12}
\end{figure}

Breaking down the total pressure effects into individual bits, any column of fluid contributes to the pressure below it, even if that column of fluid is not directly on top of a continuous column of fluid all the way to the ``bottom''.

\section{Surface tension: normals and tangents.}

Consider a surface with some variation as in figure (\ref{fig:continuumL13:continuumL13Fig13})

\begin{figure}[htp]
   \centering
   \includegraphics[totalheight=0.2\textheight]{continuumL13Fig13}
   \caption{variable surface geometries}\label{fig:continuumL13:continuumL13Fig13}
\end{figure}

We can construct an equation for the surface

\begin{equation}\label{eqn:continuumL13:160}
z = h(x, t),
\end{equation}

or equivalently

\begin{equation}\label{eqn:continuumL13:180}
\phi = z - h(x, t) = 0.
\end{equation}

If $d$ is the average height, with the $\eta(x,t)$ the variation of the height from this average, we can also write

% FIXME: had z = d + \eta(x, t) on board?
\begin{equation}\label{eqn:continuumL13:200}
h = d + \eta(x, t)
\end{equation}

and for the surface

% FIXME: had: z - \eta(x, t) = \text{constant} on the board?
\begin{equation}\label{eqn:continuumL13:220}
\phi = d - \eta(x, t) = 0
\end{equation}

We can generalize this and define a surface function as one that satisfies

\begin{equation}\label{eqn:continuumL13:220b}
\phi = d - \eta(x, t) = \text{constant}.
\end{equation}

Consider a small section of a 2D surface as in figure (\ref{fig:continuumL13:continuumL13Fig14})

\begin{figure}[htp]
   \centering
   \includegraphics[totalheight=0.2\textheight]{continuumL13Fig14}
   \caption{a vector differential element}\label{fig:continuumL13:continuumL13Fig14}
\end{figure}

With $\phi = \text{constant}$ on the surface, we have for $\phi = \phi(x, y, z)$

\begin{equation}\label{eqn:continuumL13:240}
d\phi = 0 
\end{equation}

or, in coordinates

\begin{align*}
d\phi &= 
\PD{x}{\phi} dx
+\PD{y}{\phi} dy
+\PD{z}{\phi} dz \\
&= \spacegrad \phi \cdot d\Br
\end{align*}

Pictorially we see that $d\Br$ is tangential to the surface, but since we also have

\begin{equation}\label{eqn:continuumL13:360}
\spacegrad \phi \cdot d\Br = 0,
\end{equation}

the implication is that the gradient is normal to the surface

\begin{equation}\label{eqn:continuumL13:380}
d\Br \perp \spacegrad \phi.
\end{equation}

We can therefore construct the unit normal by scaling the gradient

\begin{equation}\label{eqn:continuumL13:260}
\ncap = \frac{\spacegrad \phi}{\Abs{\spacegrad \phi}},
\end{equation}

since the direction of $\ncap$ is $\spacegrad \phi$.

For example, in our case where $\phi = y - h(x, t)$, we have

\begin{equation}\label{eqn:continuumL13:280}
\spacegrad \phi = \xcap \left( -\PD{x}{h} \right) + \ycap
\end{equation}

This has norm

\begin{equation}\label{eqn:continuumL13:300}
\Abs{\spacegrad \phi} = \sqrt{ 1 + \left( \PD{x}{h} \right)^2 }
\end{equation}

and our unit normal is

\begin{equation}\label{eqn:continuumL13:320}
\ncap = 
\frac{\xcap \left( -\PD{x}{h} \right) + \ycap}{
\sqrt{ 1 + \left( \PD{x}{h} \right)^2 }
}
\end{equation}

By inspection we can also express the unit tangent, and we have for both

\begin{align}\label{eqn:continuumL13:340}
\ncap &= 
\inv{
\sqrt{ 1 + \left( \PD{x}{h} \right)^2 }
}
\left( -\PD{x}{h}, 1 \right)  \\
\taucap &= 
\inv{
\sqrt{ 1 + \left( \PD{x}{h} \right)^2 }
}
\left( 1, \PD{x}{h} \right)
\end{align}

%\EndArticle
\EndNoBibArticle
