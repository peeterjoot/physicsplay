% 
% 
% 
% Copyright � 2012 Peeter Joot
% All Rights Reserved
% 
% This file may be reproduced and distributed in whole or in part, without fee, subject to the following conditions:
% 
% o The copyright notice above and this permission notice must be preserved complete on all complete or partial copies.
% 
% o Any translation or derived work must be approved by the author in writing before distribution.
% 
% o If you distribute this work in part, instructions for obtaining the complete version of this file must be included, and a means for obtaining a complete version provided.
% 
% 
% Exceptions to these rules may be granted for academic purposes: Write to the author and ask.
% 
% 
% 
\label{chap:continuumL21}
\section{Stability.  Rayleigh-Benard problem}
\subsection{Stability.  Some graphical illustrations}

What do we mean by stability?  A configuration is stable if after a small disturbance it returns to it's original position.  A couple systems to consider are shown in figures (\ref{fig:continuumL21:continuumL21Fig1a}), (\ref{fig:continuumL21:continuumL21Fig1b}) and (\ref{fig:continuumL21:continuumL21Fig1c}).

\imageFigure{figures/continuumL21Fig1a}{Stable well configuration}{fig:continuumL21:continuumL21Fig1a}{0.2}
\imageFigure{figures/continuumL21Fig1b}{Instable peak configuration}{fig:continuumL21:continuumL21Fig1b}{0.2}
\imageFigure{figures/continuumL21Fig1c}{Stable tabletop configuration}{fig:continuumL21:continuumL21Fig1c}{0.2}

We can examine how a displacement $\delta x$ changes with time after making it.  In a stable configuration without friction we will induce an oscillation as plotted in figure (\ref{fig:continuumL21:continuumL21Fig2a}) for the parabolic configuration.  With friction we'll have a damping effect.  This is plotted for the parabolic well in figure (\ref{fig:continuumL21:continuumL21Fig2b}).

\imageFigure{figures/continuumL21Fig2a}{Displacement time evolution in undamped well system}{fig:continuumL21:continuumL21Fig2a}{0.2}
\imageFigure{figures/continuumL21Fig2b}{Displacement time evolution in damped well system}{fig:continuumL21:continuumL21Fig2b}{0.2}

For the inverted parabola our displacement takes the form of figure (\ref{fig:continuumL21:continuumL21Fig3})
\imageFigure{figures/continuumL21Fig3}{Time evolution of displacement in instable parabolic configuration}{fig:continuumL21:continuumL21Fig3}{0.2}

For the ball on the table, assuming some friction that stops the ball, fairly quickly, we'll have a displacement as illustrated in figure (\ref{fig:continuumL21:continuumL21Fig3b})
\imageFigure{figures/continuumL21Fig3b}{Time evolution of displacement in tabletop configuration}{fig:continuumL21:continuumL21Fig3b}{0.2}

\subsection{Characterizing stability}

Let's suppose that our displacement can be described in exponential form

\begin{equation}\label{eqn:continuumL21:10}
\delta x \sim e^{\sigma t}
\end{equation}

where $\sigma$ is the \textit{growth rate of perturbation}, and is in general a complex number of the form

\begin{equation}\label{eqn:continuumL21:30}
\sigma = \sigma_\text{R} + i \sigma_\text{I}
\end{equation}

\subsubsection{Case I.  Oscillatory unstability}

A system of the form

\begin{align*}
\sigma_{\text{R}} &= 0 \\
\sigma_{\text{I}} &> 0
\end{align*}

\textit{oscillatory unstable}.  An example of this is the undamped parabolic system illustrated above.

\subsubsection{Case II.  Marginal unstability}

\begin{align*}
\delta x 
&\sim e^{\sigma_{\text{R}} t} e^{i \sigma_{\text{I}} t} \\
&\sim e^{\sigma_{\text{R}} t} \left( \cos \sigma_{\text{I}} t + i \sin \sigma_{\text{I}} t \right)
\end{align*}

We'll call systems of the form

\begin{align*}
\sigma_{\text{I}} &= 0 \\
\sigma_{\text{R}} &> 0
\end{align*}

\textit{marginally unstable}.  We can have unstable systems with $\sigma_{\text{I}} \ne 0$ but still $\sigma_{\text{R}} > 0$, but these are less common.

\subsubsection{Case III.  Neutral stability}

\begin{align*}
\sigma &= 0 \\
\sigma_{\text{R}} &= \sigma_{\text{I}} = 0
\end{align*}

An example of this was the billiard table example where the ball moved to a new location on the table after being bumped slightly.

\subsection{A mathematical description}

For a discussion of stability in fluids we'll not only have to incorporate the Navier-Stokes equation as we've done, but will also have to bring in the heat equation.  Unfortunately that isn't in the scope of this course to derive.  Let's consider as system heated on a bottom plate, and consider the fluid and convection due to heating.  This system is illustrated in figure (\ref{fig:continuumL21:continuumL21Fig4})

\imageFigure{figures/continuumL21Fig4}{Fluid in cavity heated on the bottom plate}{fig:continuumL21:continuumL21Fig4}{0.2}

We start with Navier-Stokes as normal

\begin{equation}\label{eqn:continuumL21:50}
\rho \PD{t}{\Bu} + \rho (\Bu \cdot \spacegrad) \Bu = - \spacegrad p + \mu \spacegrad^2 \Bu - \rho \zcap g.
\end{equation}

For steady state with $\Bu = 0$ initially (our base state), we'll call the following the equation of the base state

\begin{equation}\label{eqn:continuumL21:70}
\boxed{
\spacegrad p_s = -\rho_s \zcap g
}
\end{equation}

We'll allow perturbations of each of our variables

\begin{align*}
\Bu &= \Bu_\text{base} + \delta \Bu = 0 + \delta \Bu \\
p &= p_s + \delta p \\
\rho &= \rho_s + \delta \rho
\end{align*}

After perturbation Navier-Stokes takes the form

\begin{equation}\label{eqn:continuumL21:90}
\begin{aligned}
(\rho_s &+ \delta \rho )\PD{t}{(0 + \delta \Bu)} + (\rho_s + \delta \rho) ((0 + \delta \Bu) \cdot \spacegrad) (0 + \delta \Bu) = \\
&- \spacegrad (p_s + \delta p) + \mu \spacegrad^2 (0 + \delta \Bu) - (\rho_s + \delta \rho) \zcap g
\end{aligned}
\end{equation}

Retaining only terms that are of first order of smallness.

\begin{equation}\label{eqn:continuumL21:110}
\rho_s \PD{t}{\delta \Bu} = - \spacegrad p_s - \spacegrad \delta p + \mu \spacegrad^2 \delta \Bu - \rho_s \zcap g - \delta \rho \zcap g
\end{equation}

applying our equation of base state \ref{eqn:continuumL21:70}, we have
%\spacegrad p_s = -\rho_s \zcap g
\begin{equation}\label{eqn:continuumL21:110b}
\rho_s \PD{t}{\delta \Bu} = \cancel{\rho_s \zcap g} - \spacegrad \delta p + \mu \spacegrad^2 \delta \Bu - \cancel{\rho_s \zcap g} - \delta \rho \zcap g,
\end{equation}

or

\begin{equation}\label{eqn:continuumL21:110c}
\rho_s \PD{t}{\delta \Bu} = - \spacegrad \delta p + \mu \spacegrad^2 \delta \Bu - \delta \rho \zcap g.
\end{equation}

we can write

\begin{equation}\label{eqn:continuumL21:130}
\left( \PD{t}{} - \nu \spacegrad^2 \right) \delta \Bu = -\inv{\rho_s} \spacegrad \delta p - \frac{\delta \rho}{\rho_s} \zcap g
\end{equation}

Applying the divergence operation on both sides, and using $\spacegrad \cdot \Bu = 0$ so that $\spacegrad \cdot \delta \Bu = 0$ we have

\begin{equation}\label{eqn:continuumL21:150}
\left( \PD{t}{} - \nu \spacegrad^2 \right) \cancel{\spacegrad \cdot \delta \Bu} = -\inv{\rho_s} \spacegrad^2 \delta p - (\zcap \cdot \spacegrad ) \frac{\delta \rho}{\rho_s} g,
\end{equation}

or

\begin{equation}\label{eqn:continuumL21:170}
\inv{\rho_s} \spacegrad^2 \delta p = - (\zcap \cdot \spacegrad ) \frac{\delta \rho}{\rho_s} g.
\end{equation}

Assuming that $\rho_s$ is constant (actually that's already been done above), we can cancel it, leaving

\begin{equation}\label{eqn:continuumL21:190}
\spacegrad^2 \delta p = - (\zcap \cdot \spacegrad ) g \delta \rho = -g \PD{z}{} \delta \rho.
\end{equation}

operating once more with $\PDi{z}{}$ we have

\begin{equation}\label{eqn:continuumL21:210}
\spacegrad^2 \PD{z}{\delta p} = -g \PDSq{z}{\delta \rho}.
\end{equation}

Going back to \ref{eqn:continuumL21:130} and taking only the $z$ component we have

\begin{equation}\label{eqn:continuumL21:230}
\left( \PD{t}{} - \nu \spacegrad^2 \right) \delta w = -\inv{\rho_s} \PD{z}{\delta p} - \frac{\delta \rho}{\rho_s} g
\end{equation}

\begin{align*}
\left( \PD{t}{} - \nu \spacegrad^2 \right) \spacegrad^2 \delta w 
&= -\inv{\rho_s} \PD{z}{ \spacegrad^2 \delta p} - \frac{g}{\rho_s} \spacegrad^2 \delta \rho \\
&= -\frac{g}{\rho_s} \PDSq{z}{\delta \rho} - \frac{g}{\rho_s} \spacegrad^2 \delta \rho \\
&= 
-\frac{g}{\rho_s} \left( 
\PDSq{x}{}
+\PDSq{y}{}
\right)
\delta \rho \\
&=
g \alpha \left( 
\PDSq{x}{}
+\PDSq{y}{}
\right)
\delta T
\end{align*}

in the last step we use the following assumed relation for temperature

\begin{equation}\label{eqn:continuumL21:250}
\delta \rho = - \rho_s \alpha \delta T.
\end{equation}

Here $\alpha$ is the coefficient of thermal expansion.  This is just a statement that expansion and temperature are related (as we heat something, it expands), with the ratio of the density change relative to the original being linearly related to the change in temperature.

We have finally

\begin{equation}\label{eqn:continuumL21:290}
\left( \PD{t}{} - \nu \spacegrad^2 \right) \spacegrad^2 \delta w 
= 
g \alpha \left( 
\PDSq{x}{}
+\PDSq{y}{}
\right)
\delta T.
\end{equation}

Solving this is the Rayleigh-Benard instability problem.

While this is a fourth order differential equation, it's still the same sort of problem logically as we've been working on.  Our boundary value conditions at $z = 0$ are

\begin{equation}\label{eqn:continuumL21:n}
u, v, w, \delta u, \delta v, \delta w = 0.
\end{equation}

Also relevant will be a similar equation relating temperature and fluid flow rate

\begin{equation}\label{eqn:continuumL21:270}
\left( \PD{t}{} - \kappa \spacegrad^2 \right) \delta T = \Delta T \frac{\delta w}{d},
\end{equation}

which we will cover in the next (and final) lecture of the course.
