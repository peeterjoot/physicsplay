% 
% 
% 
% Copyright � 2012 Peeter Joot
% All Rights Reserved
% 
% This file may be reproduced and distributed in whole or in part, without fee, subject to the following conditions:
% 
% o The copyright notice above and this permission notice must be preserved complete on all complete or partial copies.
% 
% o Any translation or derived work must be approved by the author in writing before distribution.
% 
% o If you distribute this work in part, instructions for obtaining the complete version of this file must be included, and a means for obtaining a complete version provided.
% 
% 
% Exceptions to these rules may be granted for academic purposes: Write to the author and ask.
% 
% 
% 
%%
% Copyright � 2015 Peeter Joot.  All Rights Reserved.
% Licenced as described in the file LICENSE under the root directory of this GIT repository.
%
\documentclass[]{eliblog}

\usepackage{amsmath}
\usepackage{mathpazo}

%
% shorthand for bold symbols, convenient for vectors and matrices
%
\newcommand{\Ba}[0]{\mathbf{a}}
\newcommand{\Bb}[0]{\mathbf{b}}
\newcommand{\Bc}[0]{\mathbf{c}}
\newcommand{\Bd}[0]{\mathbf{d}}
\newcommand{\Be}[0]{\mathbf{e}}
\newcommand{\Bf}[0]{\mathbf{f}}
\newcommand{\Bg}[0]{\mathbf{g}}
\newcommand{\Bh}[0]{\mathbf{h}}
\newcommand{\Bi}[0]{\mathbf{i}}
\newcommand{\Bj}[0]{\mathbf{j}}
\newcommand{\Bk}[0]{\mathbf{k}}
\newcommand{\Bl}[0]{\mathbf{l}}
\newcommand{\Bm}[0]{\mathbf{m}}
\newcommand{\Bn}[0]{\mathbf{n}}
\newcommand{\Bo}[0]{\mathbf{o}}
\newcommand{\Bp}[0]{\mathbf{p}}
\newcommand{\Bq}[0]{\mathbf{q}}
\newcommand{\Br}[0]{\mathbf{r}}
\newcommand{\Bs}[0]{\mathbf{s}}
\newcommand{\Bt}[0]{\mathbf{t}}
\newcommand{\Bu}[0]{\mathbf{u}}
\newcommand{\Bv}[0]{\mathbf{v}}
\newcommand{\Bw}[0]{\mathbf{w}}
\newcommand{\Bx}[0]{\mathbf{x}}
\newcommand{\By}[0]{\mathbf{y}}
\newcommand{\Bz}[0]{\mathbf{z}}
\newcommand{\BA}[0]{\mathbf{A}}
\newcommand{\BB}[0]{\mathbf{B}}
\newcommand{\BC}[0]{\mathbf{C}}
\newcommand{\BD}[0]{\mathbf{D}}
\newcommand{\BE}[0]{\mathbf{E}}
\newcommand{\BF}[0]{\mathbf{F}}
\newcommand{\BG}[0]{\mathbf{G}}
\newcommand{\BH}[0]{\mathbf{H}}
\newcommand{\BI}[0]{\mathbf{I}}
\newcommand{\BJ}[0]{\mathbf{J}}
\newcommand{\BK}[0]{\mathbf{K}}
\newcommand{\BL}[0]{\mathbf{L}}
\newcommand{\BM}[0]{\mathbf{M}}
\newcommand{\BN}[0]{\mathbf{N}}
\newcommand{\BO}[0]{\mathbf{O}}
\newcommand{\BP}[0]{\mathbf{P}}
\newcommand{\BQ}[0]{\mathbf{Q}}
\newcommand{\BR}[0]{\mathbf{R}}
\newcommand{\BS}[0]{\mathbf{S}}
\newcommand{\BT}[0]{\mathbf{T}}
\newcommand{\BU}[0]{\mathbf{U}}
\newcommand{\BV}[0]{\mathbf{V}}
\newcommand{\BW}[0]{\mathbf{W}}
\newcommand{\BX}[0]{\mathbf{X}}
\newcommand{\BY}[0]{\mathbf{Y}}
\newcommand{\BZ}[0]{\mathbf{Z}}

\newcommand{\Bzero}[0]{\mathbf{0}}
\newcommand{\Btheta}[0]{\boldsymbol{\theta}}
\newcommand{\Btau}[0]{\boldsymbol{\tau}}
\newcommand{\Bomega}[0]{\boldsymbol{\omega}}

%
% shorthand for unit vectors
%
\newcommand{\acap}[0]{\hat{\Ba}}
\newcommand{\bcap}[0]{\hat{\Bb}}
\newcommand{\ccap}[0]{\hat{\Bc}}
\newcommand{\dcap}[0]{\hat{\Bd}}
\newcommand{\ecap}[0]{\hat{\Be}}
\newcommand{\fcap}[0]{\hat{\Bf}}
\newcommand{\gcap}[0]{\hat{\Bg}}
\newcommand{\hcap}[0]{\hat{\Bh}}
\newcommand{\icap}[0]{\hat{\Bi}}
\newcommand{\jcap}[0]{\hat{\Bj}}
\newcommand{\kcap}[0]{\hat{\Bk}}
\newcommand{\lcap}[0]{\hat{\Bl}}
\newcommand{\mcap}[0]{\hat{\Bm}}
\newcommand{\ncap}[0]{\hat{\Bn}}
\newcommand{\ocap}[0]{\hat{\Bo}}
\newcommand{\pcap}[0]{\hat{\Bp}}
\newcommand{\qcap}[0]{\hat{\Bq}}
\newcommand{\rcap}[0]{\hat{\Br}}
\newcommand{\scap}[0]{\hat{\Bs}}
\newcommand{\tcap}[0]{\hat{\Bt}}
\newcommand{\ucap}[0]{\hat{\Bu}}
\newcommand{\vcap}[0]{\hat{\Bv}}
\newcommand{\wcap}[0]{\hat{\Bw}}
\newcommand{\xcap}[0]{\hat{\Bx}}
\newcommand{\ycap}[0]{\hat{\By}}
\newcommand{\zcap}[0]{\hat{\Bz}}
\newcommand{\thetacap}[0]{\hat{\Btheta}}

%
% to write R^n and C^n in a distinguishable fashion.  Perhaps change this
% to the double lined characters upon figuring out how to do so.
%
\newcommand{\C}[1]{$\mathbb{C}^{#1}$}
\newcommand{\R}[1]{$\mathbb{R}^{#1}$}

%
% various generally useful helpers
%

% derivative of #1 wrt. #2:
\newcommand{\D}[2] {\frac {d#2} {d#1}}

\newcommand{\inv}[1]{\frac{1}{#1}}
\newcommand{\cross}[0]{\times}

\newcommand{\abs}[1]{\lvert{#1}\rvert}
\newcommand{\norm}[1]{\lVert{#1}\rVert}
\newcommand{\innerprod}[2]{\langle{#1}, {#2}\rangle}
\newcommand{\dotprod}[2]{{#1} \cdot {#2}}
\newcommand{\bdotprod}[2]{\left({#1} \cdot {#2}\right)}
\newcommand{\crossprod}[2]{{#1} \cross {#2}}
\newcommand{\tripleprod}[3]{\dotprod{\left(\crossprod{#1}{#2}\right)}{#3}}

\DeclareMathOperator{\Proj}{Proj}
\DeclareMathOperator{\Span}{span}
\DeclareMathOperator{\Sgn}{sgn}
\DeclareMathOperator{\Area}{Area}
\DeclareMathOperator{\Volume}{Volume}

%
% A few miscellaneous things specific to this document
%
\newcommand{\crossop}[1]{\crossprod{#1}{}}

% R2 vector.
\newcommand{\VectorTwo}[2]{
\begin{bmatrix}
 {#1} \\
 {#2}
\end{bmatrix}
}

\newcommand{\VectorN}[1]{
\begin{bmatrix}
{#1}_1 \\
{#1}_2 \\
\vdots \\
{#1}_N \\
\end{bmatrix}
}

\newcommand{\DETuvij}[4]{
\begin{vmatrix}
 {#1}_{#3} & {#1}_{#4} \\
 {#2}_{#3} & {#2}_{#4}
\end{vmatrix}
}

\newcommand{\DETuvwijk}[6]{
\begin{vmatrix}
 {#1}_{#4} & {#1}_{#5} & {#1}_{#6} \\
 {#2}_{#4} & {#2}_{#5} & {#2}_{#6} \\
 {#3}_{#4} & {#3}_{#5} & {#3}_{#6}
\end{vmatrix}
}

\newcommand{\DETuvwxijkl}[8]{
\begin{vmatrix}
 {#1}_{#5} & {#1}_{#6} & {#1}_{#7} & {#1}_{#8} \\
 {#2}_{#5} & {#2}_{#6} & {#2}_{#7} & {#2}_{#8} \\
 {#3}_{#5} & {#3}_{#6} & {#3}_{#7} & {#3}_{#8} \\
 {#4}_{#5} & {#4}_{#6} & {#4}_{#7} & {#4}_{#8} \\
\end{vmatrix}
}

%\newcommand{\DETuvwxyijklm}[10]{
%\begin{vmatrix}
% {#1}_{#6} & {#1}_{#7} & {#1}_{#8} & {#1}_{#9} & {#1}_{#10} \\
% {#2}_{#6} & {#2}_{#7} & {#2}_{#8} & {#2}_{#9} & {#2}_{#10} \\
% {#3}_{#6} & {#3}_{#7} & {#3}_{#8} & {#3}_{#9} & {#3}_{#10} \\
% {#4}_{#6} & {#4}_{#7} & {#4}_{#8} & {#4}_{#9} & {#4}_{#10} \\
% {#5}_{#6} & {#5}_{#7} & {#5}_{#8} & {#5}_{#9} & {#5}_{#10}
%\end{vmatrix}
%}

% R3 vector.
\newcommand{\VectorThree}[3]{
\begin{bmatrix}
 {#1} \\
 {#2} \\
 {#3}
\end{bmatrix}
}



\author{Peeter Joot}
\email{peeter.joot@gmail.com}

%\documentclass[]{eliblogwidescreen}

\usepackage{amsmath}
\usepackage{mathpazo}

%
% shorthand for bold symbols, convenient for vectors and matrices
%
\newcommand{\Ba}[0]{\mathbf{a}}
\newcommand{\Bb}[0]{\mathbf{b}}
\newcommand{\Bc}[0]{\mathbf{c}}
\newcommand{\Bd}[0]{\mathbf{d}}
\newcommand{\Be}[0]{\mathbf{e}}
\newcommand{\Bf}[0]{\mathbf{f}}
\newcommand{\Bg}[0]{\mathbf{g}}
\newcommand{\Bh}[0]{\mathbf{h}}
\newcommand{\Bi}[0]{\mathbf{i}}
\newcommand{\Bj}[0]{\mathbf{j}}
\newcommand{\Bk}[0]{\mathbf{k}}
\newcommand{\Bl}[0]{\mathbf{l}}
\newcommand{\Bm}[0]{\mathbf{m}}
\newcommand{\Bn}[0]{\mathbf{n}}
\newcommand{\Bo}[0]{\mathbf{o}}
\newcommand{\Bp}[0]{\mathbf{p}}
\newcommand{\Bq}[0]{\mathbf{q}}
\newcommand{\Br}[0]{\mathbf{r}}
\newcommand{\Bs}[0]{\mathbf{s}}
\newcommand{\Bt}[0]{\mathbf{t}}
\newcommand{\Bu}[0]{\mathbf{u}}
\newcommand{\Bv}[0]{\mathbf{v}}
\newcommand{\Bw}[0]{\mathbf{w}}
\newcommand{\Bx}[0]{\mathbf{x}}
\newcommand{\By}[0]{\mathbf{y}}
\newcommand{\Bz}[0]{\mathbf{z}}
\newcommand{\BA}[0]{\mathbf{A}}
\newcommand{\BB}[0]{\mathbf{B}}
\newcommand{\BC}[0]{\mathbf{C}}
\newcommand{\BD}[0]{\mathbf{D}}
\newcommand{\BE}[0]{\mathbf{E}}
\newcommand{\BF}[0]{\mathbf{F}}
\newcommand{\BG}[0]{\mathbf{G}}
\newcommand{\BH}[0]{\mathbf{H}}
\newcommand{\BI}[0]{\mathbf{I}}
\newcommand{\BJ}[0]{\mathbf{J}}
\newcommand{\BK}[0]{\mathbf{K}}
\newcommand{\BL}[0]{\mathbf{L}}
\newcommand{\BM}[0]{\mathbf{M}}
\newcommand{\BN}[0]{\mathbf{N}}
\newcommand{\BO}[0]{\mathbf{O}}
\newcommand{\BP}[0]{\mathbf{P}}
\newcommand{\BQ}[0]{\mathbf{Q}}
\newcommand{\BR}[0]{\mathbf{R}}
\newcommand{\BS}[0]{\mathbf{S}}
\newcommand{\BT}[0]{\mathbf{T}}
\newcommand{\BU}[0]{\mathbf{U}}
\newcommand{\BV}[0]{\mathbf{V}}
\newcommand{\BW}[0]{\mathbf{W}}
\newcommand{\BX}[0]{\mathbf{X}}
\newcommand{\BY}[0]{\mathbf{Y}}
\newcommand{\BZ}[0]{\mathbf{Z}}

\newcommand{\Bzero}[0]{\mathbf{0}}
\newcommand{\Btheta}[0]{\boldsymbol{\theta}}
\newcommand{\Btau}[0]{\boldsymbol{\tau}}
\newcommand{\Bomega}[0]{\boldsymbol{\omega}}

%
% shorthand for unit vectors
%
\newcommand{\acap}[0]{\hat{\Ba}}
\newcommand{\bcap}[0]{\hat{\Bb}}
\newcommand{\ccap}[0]{\hat{\Bc}}
\newcommand{\dcap}[0]{\hat{\Bd}}
\newcommand{\ecap}[0]{\hat{\Be}}
\newcommand{\fcap}[0]{\hat{\Bf}}
\newcommand{\gcap}[0]{\hat{\Bg}}
\newcommand{\hcap}[0]{\hat{\Bh}}
\newcommand{\icap}[0]{\hat{\Bi}}
\newcommand{\jcap}[0]{\hat{\Bj}}
\newcommand{\kcap}[0]{\hat{\Bk}}
\newcommand{\lcap}[0]{\hat{\Bl}}
\newcommand{\mcap}[0]{\hat{\Bm}}
\newcommand{\ncap}[0]{\hat{\Bn}}
\newcommand{\ocap}[0]{\hat{\Bo}}
\newcommand{\pcap}[0]{\hat{\Bp}}
\newcommand{\qcap}[0]{\hat{\Bq}}
\newcommand{\rcap}[0]{\hat{\Br}}
\newcommand{\scap}[0]{\hat{\Bs}}
\newcommand{\tcap}[0]{\hat{\Bt}}
\newcommand{\ucap}[0]{\hat{\Bu}}
\newcommand{\vcap}[0]{\hat{\Bv}}
\newcommand{\wcap}[0]{\hat{\Bw}}
\newcommand{\xcap}[0]{\hat{\Bx}}
\newcommand{\ycap}[0]{\hat{\By}}
\newcommand{\zcap}[0]{\hat{\Bz}}
\newcommand{\thetacap}[0]{\hat{\Btheta}}

%
% to write R^n and C^n in a distinguishable fashion.  Perhaps change this
% to the double lined characters upon figuring out how to do so.
%
\newcommand{\C}[1]{$\mathbb{C}^{#1}$}
\newcommand{\R}[1]{$\mathbb{R}^{#1}$}

%
% various generally useful helpers
%

% derivative of #1 wrt. #2:
\newcommand{\D}[2] {\frac {d#2} {d#1}}

\newcommand{\inv}[1]{\frac{1}{#1}}
\newcommand{\cross}[0]{\times}

\newcommand{\abs}[1]{\lvert{#1}\rvert}
\newcommand{\norm}[1]{\lVert{#1}\rVert}
\newcommand{\innerprod}[2]{\langle{#1}, {#2}\rangle}
\newcommand{\dotprod}[2]{{#1} \cdot {#2}}
\newcommand{\bdotprod}[2]{\left({#1} \cdot {#2}\right)}
\newcommand{\crossprod}[2]{{#1} \cross {#2}}
\newcommand{\tripleprod}[3]{\dotprod{\left(\crossprod{#1}{#2}\right)}{#3}}

\DeclareMathOperator{\Proj}{Proj}
\DeclareMathOperator{\Span}{span}
\DeclareMathOperator{\Sgn}{sgn}
\DeclareMathOperator{\Area}{Area}
\DeclareMathOperator{\Volume}{Volume}

%
% A few miscellaneous things specific to this document
%
\newcommand{\crossop}[1]{\crossprod{#1}{}}

% R2 vector.
\newcommand{\VectorTwo}[2]{
\begin{bmatrix}
 {#1} \\
 {#2}
\end{bmatrix}
}

\newcommand{\VectorN}[1]{
\begin{bmatrix}
{#1}_1 \\
{#1}_2 \\
\vdots \\
{#1}_N \\
\end{bmatrix}
}

\newcommand{\DETuvij}[4]{
\begin{vmatrix}
 {#1}_{#3} & {#1}_{#4} \\
 {#2}_{#3} & {#2}_{#4}
\end{vmatrix}
}

\newcommand{\DETuvwijk}[6]{
\begin{vmatrix}
 {#1}_{#4} & {#1}_{#5} & {#1}_{#6} \\
 {#2}_{#4} & {#2}_{#5} & {#2}_{#6} \\
 {#3}_{#4} & {#3}_{#5} & {#3}_{#6}
\end{vmatrix}
}

\newcommand{\DETuvwxijkl}[8]{
\begin{vmatrix}
 {#1}_{#5} & {#1}_{#6} & {#1}_{#7} & {#1}_{#8} \\
 {#2}_{#5} & {#2}_{#6} & {#2}_{#7} & {#2}_{#8} \\
 {#3}_{#5} & {#3}_{#6} & {#3}_{#7} & {#3}_{#8} \\
 {#4}_{#5} & {#4}_{#6} & {#4}_{#7} & {#4}_{#8} \\
\end{vmatrix}
}

%\newcommand{\DETuvwxyijklm}[10]{
%\begin{vmatrix}
% {#1}_{#6} & {#1}_{#7} & {#1}_{#8} & {#1}_{#9} & {#1}_{#10} \\
% {#2}_{#6} & {#2}_{#7} & {#2}_{#8} & {#2}_{#9} & {#2}_{#10} \\
% {#3}_{#6} & {#3}_{#7} & {#3}_{#8} & {#3}_{#9} & {#3}_{#10} \\
% {#4}_{#6} & {#4}_{#7} & {#4}_{#8} & {#4}_{#9} & {#4}_{#10} \\
% {#5}_{#6} & {#5}_{#7} & {#5}_{#8} & {#5}_{#9} & {#5}_{#10}
%\end{vmatrix}
%}

% R3 vector.
\newcommand{\VectorThree}[3]{
\begin{bmatrix}
 {#1} \\
 {#2} \\
 {#3}
\end{bmatrix}
}



\author{Peeter Joot}
\email{peeter.joot@gmail.com}


%\chapter{PHY454H1S Continuum Mechanics.  Lecture 14: Non-dimensionality and scaling.  Taught by Prof. K. Das.}
\chapter{Non-dimensionality and scaling.}
\label{chap:continuumL14}
\blogpage{http://sites.google.com/site/peeterjoot2/math2012/continuumL14.pdf}
%\date{Mar 7, 2012}
\gitRevisionInfo{continuumL14}

\section{Review.  Surfaces}

We are considering a surface as depicted in (\ref{fig:continuumL14:continuumL14Fig13})

\imageFigure{continuumL13Fig13}{Variable surface geometries}{fig:continuumL14:continuumL14Fig13}{0.2}

With the surface height given by

\begin{equation}\label{eqn:continuumL14:10}
z = h(x, t),
\end{equation}

where this describes the interface.  Taking the difference

\begin{equation}\label{eqn:continuumL14:30}
\phi = z - h(x, t) = 0,
\end{equation}

we define a surface.  We considered a small displacement as in (\ref{fig:continuumL14:continuumL14Fig14}).

\imageFigure{continuumL13Fig14}{A vector differential element}{fig:continuumL14:continuumL14Fig14}{0.2}

Recall that if $\phi$ is a constant, then $\spacegrad \phi$ is a normal to the surface.  We showed this by considering the differential

\begin{align*}
0 
&= d\phi \\
&= 
\PD{x}{\phi} dx
+\PD{y}{\phi} dy
+\PD{z}{\phi} dz \\
&=
(\spacegrad \phi) \cdot d\Br.
\end{align*}

We can construct the unit normal by scaling.  For our 1D example we have

\begin{align*}
\ncap 
&= \frac{\spacegrad \phi}{\Abs{\spacegrad \phi}} \\
&= \inv{\Abs{\spacegrad \phi}} 
\left(
\PD{x}{\phi},
\PD{y}{\phi}
\right) 
\end{align*}

so that our unit normal is
\begin{equation}\label{eqn:continuumL14:50}
\ncap 
= \inv{ \sqrt{1 + (h')^2}}
\left( -\PD{x}{h}, 1 \right)
\end{equation}

A unit tangent can also be constructed by inspection

\begin{equation}\label{eqn:continuumL14:70}
\taucap 
= \inv{ \sqrt{1 + (h')^2}}
\left( 1, \PD{x}{h} \right).
\end{equation}

\section{Traction vector at the interface.}

Recall that our stress tensor has the form

\begin{equation}\label{eqn:continuumL14:90}
T_{ij} = 
- p \delta_{ij} + \rho \nu \left( 
\PD{x_j}{u_i}
+\PD{x_i}{u_j}
\right)
\end{equation}

(here we are switching notations for the stress since we will be using $\sigma$ for surface tension in this section)

The traction vector components are

\begin{equation}\label{eqn:continuumL14:110}
t_i = T_{ij} n_j =
- p n_i + \rho \nu \left( 
\PD{x_j}{u_i}
+\PD{x_i}{u_j}
\right) n_j
\end{equation}

Considering a control volume as illustrated in we can arrive at what we call the jump stress balance equation

figure (\ref{fig:continuumL14:continuumL14fig3})
\imageFigure{continuumL14fig3}{Control volume for liquid air interface}{fig:continuumL14:continuumL14fig3}{0.2}

\begin{equation}\label{eqn:continuumL14:130}
[\BT \ncap]^2_1 = \frac{2 \sigma}{R} \ncap - \spacegrad_I \sigma
\end{equation}

where

\begin{align}\label{eqn:continuumL14:150}
\sigma &= \text{surface tension} \\
R &= \text{radius of curvature} \\
\spacegrad_I &= \text{gradient along the interface}
\end{align}

and the suffix $2$ and prefix $1$ indicates that we are considering the interface between fluids labeled $1$ and $2$ (liquid and air respectively in the diagram).

For a derivation see Prof after class?

Force balance along the normal direction gives

\begin{equation}\label{eqn:continuumL14:170}
\ncap [\BT \ncap]^2_1 = \ncap \cdot \frac{2 \sigma}{R} \ncap - \cancel{\ncap \cdot (\spacegrad_I \sigma)}
\end{equation}

If you do this calculation, you will get 

\begin{equation}\label{eqn:continuumL14:190}
[-p]^2_1 = \frac{ 2 \sigma}{R}
\end{equation}

I think this was called the Laplace equation?

Question: How was $\sigma$ defined?  A: Energy per unit area.  

Figure (\ref{fig:continuumL14:continuumL14fig4}) was given as part of an explanation of surface tension and curvature, but I missed part of that discussion.  Perhaps this is elaborated on in the class notes?

\imageFigure{continuumL14fig4}{Molecular gas and liquid interactions at a surface.}{fig:continuumL14:continuumL14fig4}{0.2}

Reading: An treatment of this topic that looks complete enough to understand looks like it can be found in \S 7 of \cite{landau1987course}.

\section{Non dimensionalization and scaling}

\subsection{Motivation.}

By scaling we mean how much detail do you want to look at in the analysis.  Consider the figure (\ref{fig:continuumL14:continuumL14fig5a}) where we imagine that we zoom in on something that appears smooth from a distance.  However, we are free to perform a change of variables on our coordinates and rescale in any arbitrary fashion.  For example

\imageFigure{continuumL14fig5a}{Coarse scaling example.}{fig:continuumL14:continuumL14fig5a}{0.2}

\begin{align}\label{eqn:continuumL14:210}
x &\rightarrow A u^\alpha \\
y &\rightarrow B v^\beta
\end{align}

For a linear zoom scaling ($\alpha = \beta = 1$) we could perhaps find that we have something very granular close up as in figure (\ref{fig:continuumL14:continuumL14fig5b}).  Picking the length scale to be used in this case can be very important.

\imageFigure{continuumL14fig5b}{Fine grain scaling example (a zoom).}{fig:continuumL14:continuumL14fig5b}{0.2}

The flexiblity to rescale with non unity values for $\alpha$ and $\beta$ can, for example, come in handy, should we choose to rescale time and position differently.

\subsection{Rescaling by characteristic length and velocity.}

Suppose that a fluid is flowing with

\begin{itemize}
\item a characteristic velocity $U$, with dimensions $[U] \sim L T^{-1}$
\item a characteristic length scale $L$
\end{itemize}

Considering the dimensions of the terms in the Navier-Stokes equation

\begin{equation}\label{eqn:continuumL14:230}
[\rho] = M L^{-3}
\end{equation}
\begin{equation}\label{eqn:continuumL14:250}
[p] = M L T^{-2} L^{-2} = M L^{-1} T^{-2}
\end{equation}
\begin{equation}\label{eqn:continuumL14:270}
[t] = T = \frac{L}{U}
\end{equation}

so 
\begin{equation}\label{eqn:continuumL14:290}
[p] = [\rho U^2] = M L^{-3} L^2 T^{-2}  = M L^{-1} T^{-2}
\end{equation}

Now let's alter the Navier-Stokes equation using some scaling to put it into a dimensionless form

\begin{equation}\label{eqn:continuumL14:310}
\PD{t}{\Bu} + (\Bu \cdot \spacegrad) \Bu = - \inv{\rho} \spacegrad + \nu \spacegrad^2 \Bu
\end{equation}

\begin{equation}\label{eqn:continuumL14:330}
\PD{t}{\Bu} \rightarrow  \PD{\left(\frac{L}{U} t'\right)}{(U \Bu')} = \frac{U^2}{L} \PD{t'}{\Bu'}
\end{equation}

\begin{equation}\label{eqn:continuumL14:350}
\spacegrad = 
\xcap \PD{x}{}
+\ycap \PD{y}{}
+\zcap \PD{z}{}
\rightarrow 
\xcap \PD{L x'}{}
\ycap \PD{L y'}{}
\zcap \PD{L z'}{}
\end{equation}

so that 

\begin{equation}\label{eqn:continuumL14:370}
\spacegrad \rightarrow \inv{L} \spacegrad'
\end{equation}

\begin{equation}\label{eqn:continuumL14:390}
(\Bu \cdot \spacegrad ) \Bu \rightarrow 
\left( U \Bu' \cdot \inv{L} \spacegrad' \right) U \Bu' = \frac{U^2}{L} (\Bu' \cdot \spacegrad') \Bu'
\end{equation}

\begin{equation}\label{eqn:continuumL14:410}
\inv{\rho} \spacegrad p \rightarrow \inv{L} \frac{\spacegrad' (\cancel{\rho} U^2) }{\cancel{\rho}} p' = \frac{U^2}{L} \spacegrad' p'
\end{equation}

\begin{equation}\label{eqn:continuumL14:430}
\nu \spacegrad^2 \Bu \rightarrow \frac{\nu}{L^2} \spacegrad' U \Bu' = \frac{\nu U}{L^2} \spacegrad' \Bu'
\end{equation}

Putting everything together, Navier-Stokes takes the form

\begin{equation}\label{eqn:continuumL14:450}
\frac{U^2}{L} \PD{t'}{\Bu'} + \frac{U^2}{L} (\Bu' \cdot \spacegrad') \Bu' = \frac{U^2}{L} \spacegrad' p' + \frac{\nu U}{L^2} \spacegrad' \Bu'
\end{equation}

or
\begin{equation}\label{eqn:continuumL14:470}
\PD{t'}{\Bu'} + (\Bu' \cdot \spacegrad') \Bu' = \spacegrad' p' + \frac{\nu}{U L} \spacegrad' \Bu'
\end{equation}

Introducing the Reynold's number

\begin{equation}\label{eqn:continuumL14:490}
R = \frac{L U}{\nu}
\end{equation}

We have Navier-Stokes in dimensionless form

\begin{equation}\label{eqn:continuumL14:510}
\PD{t'}{\Bu'} + (\Bu' \cdot \spacegrad') \Bu' = \spacegrad' p' + \inv{R} \spacegrad' \Bu'
\end{equation}

The implications of this will be discussed further in the next lecture.

%FIXME: Reading: \S XX from \cite{acheson1990elementary}
Reading: Coverage of this topic (with some problems) can be found in \S 7.6, \S 7.7 of \cite{granger1995fluid}.

\EndArticle
