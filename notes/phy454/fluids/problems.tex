\begin{Exercise}[
title={Rectilinear flow problem with a pressure gradient and shearing surface},
label={problem:fluids:review:q1}
]
Solve for the velocity and discuss.
\end{Exercise}

\begin{Answer}[ref={problem:fluids:review:q1}]
Lets specify that we have fluid flowing between surfaces at $z = \pm h$, the lower surface moving at velocity $v$ and pressure gradient $dp/dx = -G$ we find that Navier-Stokes for an assumed flow of $\Bu = u(z) \xcap$ takes the form

\begin{align}\label{eqn:continuumFluidsReview:1270}
0 &= \partial_x u + \partial_y (0) + \partial_z (0) \\
u \cancel{\partial_x u} &= - \partial_x p + \mu \partial_{zz} u \\
0 &= -\partial_y p \\
0 &= -\partial_z p
\end{align}

We find that this reduces to 

\begin{equation}\label{eqn:continuumFluidsReview:1290}
\frac{d^2 u}{dz^2} = -\frac{G}{\mu}
\end{equation}

with solution

\begin{equation}\label{eqn:continuumFluidsReview:1310}
u(z) = \frac{G}{2\mu}(h^2 - z^2) + A (z + h) + B.
\end{equation}

Application of the no-slip velocity matching constraint gives us in short order

\begin{equation}\label{eqn:continuumFluidsReview:1330}
u(z) = \frac{G}{2\mu}(h^2 - z^2) + v \left( 1 - \inv{2h} (z + h) \right).
\end{equation}

With $v = 0$ this is the channel flow solution, and with $G = 0$ this is the shearing flow solution.

Having solved for the velocity at any height, we can also solve for the mass or volume flux through a slice of the channel.  For the mass flux $\rho Q$ per unit time (given volume flux $Q$)

\begin{equation}\label{eqn:continuumFluidsReview:1350}
\int \frac{dm}{dt} 
=
\int \rho \frac{dV}{dt} 
=
\rho (\Delta A) \int \Bu \cdot \taucap,
\end{equation}

we find 

\begin{equation}\label{eqn:continuumFluidsReview:1370}
\rho Q =
\rho (\Delta y) \left( 
\frac{2 G h^3}{3 \mu} + h v
\right).
\end{equation}

We can also calculate the force of the boundaries on the fluid.  For example, the force per unit volume of the boundary at $z = \pm h$ on the fluid is found by calculating the tangential component of the traction vector taken with normal $\ncap = \mp \zcap$.  That tangent vector is found to be

\begin{equation}\label{eqn:continuumFluidsReview:1390}
\Bsigma \cdot (\pm \ncap) = -p \zcap \pm 2 \mu \Be_i e_{ij} \delta_{j 3} = - p \zcap \pm \xcap \mu \PD{z}{u}.
\end{equation}

The tangential component is the $\xcap$ component evaluated at $z = \pm h$, so for the lower and upper interfaces we have

\begin{align}\label{eqn:continuumFluidsReview:1410}
\evalbar{(\Bsigma \cdot \ncap) \cdot \xcap}{z = -h} &= -G (-h) - \frac{v \mu}{2 h} \\
\evalbar{(\Bsigma \cdot -\ncap) \cdot \xcap}{z = +h} &= -G (+h) + \frac{v \mu}{2 h},
\end{align}

so the force per unit area that the boundary applies to the fluid is

\begin{align}\label{eqn:continuumFluidsReview:1430}
\text{force per unit length of lower interface on fluid} &= L \left( G h - \frac{v \mu}{2 h} \right) \\
\text{force per unit length of upper interface on fluid} 
&= L \left( -G h + \frac{v \mu}{2 h} \right).
\end{align}

Does the sign of the velocity term make sense?  Let's consider the case where we have a zero pressure gradient and look at the lower interface.  This is the force of the interface on the fluid, so the force of the fluid on the interface would have the opposite sign

\begin{equation}\label{eqn:continuumFluidsReview:1470}
\frac{v \mu}{2 h}.
\end{equation}

This does seem reasonable.  Our fluid flowing along with a positive velocity is imparting a force on what it is flowing over in the same direction.
\end{Answer}

\begin{Exercise}[
title={Curve for tap discharge.},
label={problem:fluids:review:q2}
]
\end{Exercise}

\begin{Answer}[ref={problem:fluids:review:q2}]
We can use Bernoulli's theorem to get a rough idea what the curve for water coming out a tap would be.  Suppose we measure the volume flux, putting a measuring cup under the tap, and timing how long it takes to fill up.  We then measure the radii at different points.  This can be done from a photo as in figure (\ref{fig:tapFlowCropped6cm:tapFlowCropped6cmFig3}).

\imageFigure{figures/tapFlowCropped6cmFig3}{Tap flow measurement.}{fig:tapFlowCropped6cm:tapFlowCropped6cmFig3}{0.3}

After making the measurement, we can get an idea of the velocity between two points given a velocity estimate at a point higher in the discharge.  For a plain old falling mass, our final velocity at a point measured from where the velocity was originally measured can be found from Newton's law

\begin{equation}\label{eqn:continuumFluidsReview:1910}
\Delta v = g t
\end{equation}
\begin{equation}\label{eqn:continuumFluidsReview:1930}
\Delta z = \inv{2} g t^2 + v_0 t
\end{equation}

Solving for $v_f = v_0 + \Delta v$, we find

\begin{equation}\label{eqn:continuumFluidsReview:1950}
v_f = v_0 \sqrt{ 1 + \frac{2 g \Delta z}{v_0^2} }.
\end{equation}

Mass conservation gives us

\begin{equation}\label{eqn:continuumFluidsReview:1970}
v_0 \pi R^2 = v_f \pi r^2
\end{equation}

or

\begin{equation}\label{eqn:continuumFluidsReview:1990}
r(\Delta z) = R \sqrt{ \frac{v_0}{v_f} } = R \left( 1 + \frac{2 g \Delta z}{v_0^2} \right)^{-1/4}.
\end{equation}

For the image above I measured a flow rate of about 250 ml in 10 seconds.  With that, plus the measured radii at 0 and $6 \text{cm}$, I calculated that the average fluid velocity was $0.9 \text{m}/\text{s}$, vs a free fall rate increase of $1.3 \text{m}/\text{s}$.  Not the best match in the world, but that's to be expected since the velocity has been considered uniform throughout the stream profile, which would not actually be the case.  A proper treatment would also have to treat viscosity and surface tension.

In figure (\ref{fig:tapFlowCropped6cm:tapFlowCropped6cmFig4}) is a plot of the measured radial distance compared to what was computed with \ref{eqn:continuumFluidsReview:1990}.  The blue line is the measured width of the stream as measured, the red is a polynomial curve fitted to the raw data, and the green is the computed curve above.

\imageFigure{figures/tapFlowCropped6cmFig4}{Comparison of measured stream radii and calculated.}{fig:tapFlowCropped6cm:tapFlowCropped6cmFig4}{0.3}
\end{Answer}

\begin{Exercise}[
title={The meniscus curve against one wall.},
label={problem:fluids:review:q3}
]
As an application of our surface tension results, solve for the shape of a meniscus of water against a wall.  Work from the brief solution found in \cite{landau1987course} and add sufficient details that the solution can be understood more easily.
\end{Exercise}

\begin{Answer}[ref={problem:fluids:review:q3}]
As in the text we'll work with $z$ axis up, and the fluid up against a wall at $x = 0$ as illustrated in figure (\ref{fig:continuumFluidsReview:continuumFluidsReviewFig2}).
To get some idea a better feeling for , let's look to a worked problem.  

\imageFigure{figures/continuumFluidsReviewFig2}{Curvature of fluid against a wall.}{fig:continuumFluidsReview:continuumFluidsReviewFig2}{0.3}

The starting point is a variation of what we have in class

\begin{equation}\label{eqn:continuumFluidsReview:1530}
p_1 - p_2 = \sigma \left( \inv{R_1} + \inv{R_2} \right),
\end{equation}

where $p_2$ is the atmospheric pressure, $p_1$ is the fluid pressure, and the (signed!) radius of curvatures positive if pointing into medium 1 (the fluid).

For fluid at rest, Navier-Stokes takes the form

\begin{equation}\label{eqn:continuumFluidsReview:1630}
0 = -\spacegrad p_1 + \rho \Bg.
\end{equation}

With $\Bg = -g \zcap$ we have

\begin{equation}\label{eqn:continuumFluidsReview:1650}
0 = -\PD{z}{p_1} - \rho g,
\end{equation}

or

\begin{equation}\label{eqn:continuumFluidsReview:1670}
p_1 = \text{constant} - \rho g z.
\end{equation}

%At a height $z$ from the base of the surface (i.e. the bottom of the meniscous), our pressure is 
%
%\begin{equation}\label{eqn:continuumFluidsReview:1550}
%p_1 = p_a + \rho g z.
%\end{equation}

We have $p_2 = p_a$, the atmospheric pressure, so our pressure difference is

\begin{equation}\label{eqn:continuumFluidsReview:1570}
p_1 - p_2 = \text{constant} - \rho g z.
\end{equation}

We have then

\begin{equation}\label{eqn:continuumFluidsReview:1590}
\text{constant} -\frac{\rho g z}{\sigma} = \inv{R_1} + \inv{R_2}.
\end{equation}

One of our axis of curvature directions is directly along the $y$ axis so that curvature is zero $1/R_1 = 0$.  We can fix the constant by noting that at $x = \infty$, $z = 0$, we have no curvature $1/R_2 = 0$.  This gives

\begin{equation}\label{eqn:continuumFluidsReview:1690}
\text{constant} -0 = 0 + 0.
\end{equation}

That leaves just the second curvature to determine.  For a curve $z = z(x)$ our absolute curvature, according to \cite{wiki:curvature} is

\begin{equation}\label{eqn:continuumFluidsReview:1610}
\Abs{\inv{R_2}} = \frac{\Abs{z''}}{(1 + (z')^2)^{3/2}}.
\end{equation}

Now we have to fix the sign.  I didn't recall any sort of notion of a signed radius of curvature, but there's a blurb about it on the curvature article above, including a nice illustration of signed radius of curvatures can be found in this \href{http://goo.gl/Wqzz2}{wikipedia radius of curvature figure for a Lemniscate}.  Following that definition for a curve such as $z(x) = (1-x)^2$ we'd have a positive curvature, but the text explicitly points out that the curvatures are will be set positive if pointing into the medium.  For us to point the normal into the medium as in the figure, we have to invert the sign, so our equation to solve for $z$ is given by
% shorten to get rid of _ that latex doesn't like.
%http://upload.wikimedia.org/wikipedia/commons/b/b2/Lemniscate_nebeneinander_animated.gif

\begin{equation}\label{eqn:continuumFluidsReview:1710}
-\frac{\rho g z}{\sigma} = -\frac{z''}{(1 + (z')^2)^{3/2}}.
\end{equation}

The text introduces the capillary constant

\begin{equation}\label{eqn:continuumFluidsReview:1730}
a = \sqrt{2 \sigma/ g \rho}.
\end{equation}

Using that capillary constant $a$ to tidy up a bit and multiplying by a $z'$ integrating factor we have

\begin{equation}\label{eqn:continuumFluidsReview:1750}
-\frac{2 z z'}{a^2} = -\frac{z'' z'}{(1 + (z')^2)^{3/2}},
\end{equation}

we can integrate to find

\begin{equation}\label{eqn:continuumFluidsReview:1770}
A - \frac{z^2}{a^2} = \frac{1}{(1 + (z')^2)^{1/2}}.
\end{equation}

Again for $x = \infty$ we have $z = 0$, $z' = 0$, so $A = 1$.  Rearranging we have

\begin{equation}\label{eqn:continuumFluidsReview:1790}
\int dx = \int dz \left( \inv{(1 - z^2/a^2)^2} - 1 \right)^{-1/2}.
\end{equation}

Integrating this with Mathematica I get

\begin{equation}\label{eqn:continuumFluidsReview:1810}
x - x_0 =
\sqrt{2 a^2-z^2} \sgn(a-z)+ \frac{a}{\sqrt{2}} \ln \left(\frac{a \left(2 a-\sqrt{4 a^2-2 z^2} \sgn(a-z)\right)}{z}\right).
\end{equation}

It looks like the constant would have to be fixed numerically.  We require at $x = 0$

\begin{equation}\label{eqn:continuumFluidsReview:1830}
z'(0) = \frac{-\cos\theta}{\sin\theta} = -\cot \theta,
\end{equation}

but we don't have an explicit function for $z$.
\end{Answer}
