\begin{Exercise}[
title={Rectilinear flow problem with a pressure gradient and shearing surface},
label={problem:fluids:review:q1}
]
Solve for the velocity and discuss.
\end{Exercise}

\begin{Answer}[ref={problem:fluids:review:q1}]
Lets specify that we have fluid flowing between surfaces at $z = \pm h$, the lower surface moving at velocity $v$ and pressure gradient $dp/dx = -G$ we find that Navier-Stokes for an assumed flow of $\Bu = u(z) \xcap$ takes the form

\begin{align}\label{eqn:continuumFluidsReview:1270}
0 &= \partial_x u + \partial_y (0) + \partial_z (0) \\
u \cancel{\partial_x u} &= - \partial_x p + \mu \partial_{zz} u \\
0 &= -\partial_y p \\
0 &= -\partial_z p
\end{align}

We find that this reduces to 

\begin{equation}\label{eqn:continuumFluidsReview:1290}
\frac{d^2 u}{dz^2} = -\frac{G}{\mu}
\end{equation}

with solution

\begin{equation}\label{eqn:continuumFluidsReview:1310}
u(z) = \frac{G}{2\mu}(h^2 - z^2) + A (z + h) + B.
\end{equation}

Application of the no-slip velocity matching constraint gives us in short order

\begin{equation}\label{eqn:continuumFluidsReview:1330}
u(z) = \frac{G}{2\mu}(h^2 - z^2) + v \left( 1 - \inv{2h} (z + h) \right).
\end{equation}

With $v = 0$ this is the channel flow solution, and with $G = 0$ this is the shearing flow solution.

Having solved for the velocity at any height, we can also solve for the mass or volume flux through a slice of the channel.  For the mass flux $\rho Q$ per unit time (given volume flux $Q$)

\begin{equation}\label{eqn:continuumFluidsReview:1350}
\int \frac{dm}{dt} 
=
\int \rho \frac{dV}{dt} 
=
\rho (\Delta A) \int \Bu \cdot \taucap,
\end{equation}

we find 

\begin{equation}\label{eqn:continuumFluidsReview:1370}
\rho Q =
\rho (\Delta y) \left( 
\frac{2 G h^3}{3 \mu} + h v
\right).
\end{equation}

We can also calculate the force of the boundaries on the fluid.  For example, the force per unit volume of the boundary at $z = \pm h$ on the fluid is found by calculating the tangential component of the traction vector taken with normal $\ncap = \mp \zcap$.  That tangent vector is found to be

\begin{equation}\label{eqn:continuumFluidsReview:1390}
\Bsigma \cdot (\pm \ncap) = -p \zcap \pm 2 \mu \Be_i e_{ij} \delta_{j 3} = - p \zcap \pm \xcap \mu \PD{z}{u}.
\end{equation}

The tangential component is the $\xcap$ component evaluated at $z = \pm h$, so for the lower and upper interfaces we have

\begin{align}\label{eqn:continuumFluidsReview:1410}
\evalbar{(\Bsigma \cdot \ncap) \cdot \xcap}{z = -h} &= -G (-h) - \frac{v \mu}{2 h} \\
\evalbar{(\Bsigma \cdot -\ncap) \cdot \xcap}{z = +h} &= -G (+h) + \frac{v \mu}{2 h},
\end{align}

so the force per unit area that the boundary applies to the fluid is

\begin{align}\label{eqn:continuumFluidsReview:1430}
\text{force per unit length of lower interface on fluid} &= L \left( G h - \frac{v \mu}{2 h} \right) \\
\text{force per unit length of upper interface on fluid} 
&= L \left( -G h + \frac{v \mu}{2 h} \right).
\end{align}

Does the sign of the velocity term make sense?  Let's consider the case where we have a zero pressure gradient and look at the lower interface.  This is the force of the interface on the fluid, so the force of the fluid on the interface would have the opposite sign

\begin{equation}\label{eqn:continuumFluidsReview:1470}
\frac{v \mu}{2 h}.
\end{equation}

This does seem reasonable.  Our fluid flowing along with a positive velocity is imparting a force on what it is flowing over in the same direction.
\end{Answer}

\begin{Exercise}[
title={Curve for tap discharge.},
label={problem:fluids:review:q2}
]
\end{Exercise}

\begin{Answer}[ref={problem:fluids:review:q2}]
We can use Bernoulli's theorem to get a rough idea what the curve for water coming out a tap would be.  Suppose we measure the volume flux, putting a measuring cup under the tap, and timing how long it takes to fill up.  We then measure the radii at different points.  This can be done from a photo as in figure (\ref{fig:tapFlowCropped6cm:tapFlowCropped6cmFig3}).

\imageFigure{figures/tapFlowCropped6cmFig3}{Tap flow measurement.}{fig:tapFlowCropped6cm:tapFlowCropped6cmFig3}{0.3}

After making the measurement, we can get an idea of the velocity between two points given a velocity estimate at a point higher in the discharge.  For a plain old falling mass, our final velocity at a point measured from where the velocity was originally measured can be found from Newton's law

\begin{equation}\label{eqn:continuumFluidsReview:1910}
\Delta v = g t
\end{equation}
\begin{equation}\label{eqn:continuumFluidsReview:1930}
\Delta z = \inv{2} g t^2 + v_0 t
\end{equation}

Solving for $v_f = v_0 + \Delta v$, we find

\begin{equation}\label{eqn:continuumFluidsReview:1950}
v_f = v_0 \sqrt{ 1 + \frac{2 g \Delta z}{v_0^2} }.
\end{equation}

Mass conservation gives us

\begin{equation}\label{eqn:continuumFluidsReview:1970}
v_0 \pi R^2 = v_f \pi r^2
\end{equation}

or

\begin{equation}\label{eqn:continuumFluidsReview:1990}
r(\Delta z) = R \sqrt{ \frac{v_0}{v_f} } = R \left( 1 + \frac{2 g \Delta z}{v_0^2} \right)^{-1/4}.
\end{equation}

For the image above I measured a flow rate of about 250 ml in 10 seconds.  With that, plus the measured radii at 0 and $6 \text{cm}$, I calculated that the average fluid velocity was $0.9 \text{m}/\text{s}$, vs a free fall rate increase of $1.3 \text{m}/\text{s}$.  Not the best match in the world, but that's to be expected since the velocity has been considered uniform throughout the stream profile, which would not actually be the case.  A proper treatment would also have to treat viscosity and surface tension.

In figure (\ref{fig:tapFlowCropped6cm:tapFlowCropped6cmFig4}) is a plot of the measured radial distance compared to what was computed with \ref{eqn:continuumFluidsReview:1990}.  The blue line is the measured width of the stream as measured, the red is a polynomial curve fitted to the raw data, and the green is the computed curve above.

\imageFigure{figures/tapFlowCropped6cmFig4}{Comparison of measured stream radii and calculated.}{fig:tapFlowCropped6cm:tapFlowCropped6cmFig4}{0.3}
\end{Answer}

