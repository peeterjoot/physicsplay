\section{Review.  Surface tension clarifications}

For a surface like figure (\ref{fig:continuumL15:continuumL15Fig1})
\imageFigure{figures/lec15_Vapor_liquid_interfaceFig1}{Vapor liquid interface}{fig:continuumL15:continuumL15Fig1}{0.2}

we have a discontinuous jump in density.  We will have to consider three boundary value constraints

\begin{enumerate}
\item Mass balance.  This is the continuity equation.
\item Momentum balance.  This is the Navier-Stokes equation.
\item Energy balance.  This is the heat equation.
\end{enumerate}

We have not yet discussed the heat equation, but this is required for non-isothermal problems.

The boundary condition at the interface is given by the stress balance.  Denoting the difference in the traction vector at the interface by

\begin{equation}\label{eqn:continuumL15:130}
[\Bt]_1^2 = \Bt_2 - \Bt_1 = -\frac{\sigma}{2 R} \ncap - \spacegrad_I \sigma
\end{equation}

Here the gradient is in the tangential direction of the surface as in figure (\ref{fig:continuumL15:continuumL15Fig2})
\imageFigure{figures/lec15_normal_and_tangent_vectors_on_a_curveFig2}{normal and tangent vectors on a curve}{fig:continuumL15:continuumL15Fig2}{0.3}

In the normal direction

\begin{align*}
[\Bt]_1^2 \cdot \ncap
&= (\Bt_2 - \Bt_1) \cdot \ncap \\
&= -\frac{\sigma}{2 R} 
\end{align*}

With the traction vector having the value

\begin{align*}
\Bt 
&= \Be_i T_{ij} n_j \\
&= 
\Be_i \left( 
-p \delta_{ij} + \mu \left( 
\PD{x_j}{u_i}
+\PD{x_i}{u_j}
\right)
\right)
n_j
\end{align*}

We have in the normal direction

\begin{equation}\label{eqn:continuumL15:10}
\Bt \cdot \Bn 
=
n_i \left( 
-p \delta_{ij} + \mu \left( 
\PD{x_j}{u_i}
+\PD{x_i}{u_j}
\right)
\right) n_j
\end{equation}

With $\Bu = 0$ on the surface, and $n_i \delta_{ij} n_j = n_j n_j = 1$ we have

\begin{equation}\label{eqn:continuumL15:30}
\Bt \cdot \Bn = -p
\end{equation}

Returning to $(\Bt_2 - \Bt_1) \cdot \ncap$ we have

\begin{equation}\label{eqn:continuumL15:50}
\boxed{
-p_2 + p_1 = -\frac{\sigma}{2 R} 
}
\end{equation}

This is the Laplace pressure.  Note that the sign of the difference is significant, since it effects the direction of the curvature.  This is depicted pictorially in figure (\ref{fig:continuumL15:continuumL15Fig3})
\imageFigure{figures/lec15_pressure_and_curvature_relationshipsFig3}{pressure and curvature relationships}{fig:continuumL15:continuumL15Fig3}{0.2}

\unnumberedSubsection{Question} In \citep{landau1987course} the curvature term is written

\begin{equation}\label{eqn:continuumL15:70}
\inv{2 R} \rightarrow \inv{R_1} + \inv{R_2}.
\end{equation}

Why the difference?  Answer: This is to account for non-spherical surfaces, with curvature in two directions.  Illustrating by example, imagine a surface like as in figure (\ref{fig:continuumL15:continuumL15Figq})
\imageFigure{figures/lec15_Example_of_non-spherical_curvatureFigq}{Example of non-spherical curvature}{fig:continuumL15:continuumL15Figq}{0.2}

\section{Followup required to truly understand things}

While, this review clarifies things, we still don't really know how the surface tension term $\sigma$ is defined.  Nor have we been given any sort of derivation of \ref{eqn:continuumL15:130}, from which the end result follows.

I'm assuming that $\sigma$ is a property of the two fluids at the interface, so if you have, for example, oil and vinegar in a bottle, we have surface tension and curvature that's probably related to how the two of these interact.  If there is still a mixing or settling process occurring, I'd imagine that this could even vary from point to point on the surface (imagine adding soap to a surface where stuff can float until the soap mixes in enough that things start sinking in the radius of influence of the soap).

\section{Surface tension gradients}

Now consider the tangential component of the traction vector

\begin{equation}\label{eqn:continuumL15:90}
\Bt_2 \cdot \taucap - \Bt_1 \cdot \taucap = - \cancel{ \frac{\sigma}{2 R} \ncap \cdot \taucap} - \taucap \cdot \spacegrad_I \sigma
\end{equation}

So we see that for a static fluid, we must have

\begin{equation}\label{eqn:continuumL15:110}
\spacegrad_I \sigma = 0
\end{equation}

For a static interface there cannot be any surface tension gradient.  This becomes very important when considering stability issues.  We can have surface tension induced flow called capillary, or mandarin (?) flow.
