%
%
%
% Copyright � 2012 Peeter Joot
% All Rights Reserved
%
% This file may be reproduced and distributed in whole or in part, without fee, subject to the following conditions:
%
% o The copyright notice above and this permission notice must be preserved complete on all complete or partial copies.
%
% o Any translation or derived work must be approved by the author in writing before distribution.
%
% o If you distribute this work in part, instructions for obtaining the complete version of this file must be included, and a means for obtaining a complete version provided.
%
%
% Exceptions to these rules may be granted for academic purposes: Write to the author and ask.
%
%
%
\begin{Exercise}[
title={Flow between infinite cylinders, one moving},
label={problem:fluids:twoCylinders}
]

A problem from the 2009 phy1530 final.

An infinite cylinder of radius $R_1$ is moving with velocity $v$ parallel to its axis.  It is placed inside another cylinder of radius $R_2$.  The axes of the two cylinders coincide.  The fluid is incompressible, with viscosity $\mu$ and density $\rho$, the flow is assumed to be stationary, and no external pressure gradient is applied.

\label{problem:fluids:twoCylinders:1}
\Question{Find and sketch the velocity field of the fluid between the cylinders.}
\label{problem:fluids:twoCylinders:2}
\Question{Find the friction force per unit length acting on each cylinder.}
\label{problem:fluids:twoCylinders:3}
\Question{Find and sketch the pressure field of the liquid.}
\label{problem:fluids:twoCylinders:4}
\Question{If an external pressure gradient is present, how do you think your answer will change?  Sketch your expectation for the velocity and pressure in this case.}
\end{Exercise}

\begin{Answer}[ref={problem:fluids:twoCylinders}]
\ref{problem:fluids:twoCylinders:1}
\ref{problem:fluids:twoCylinders:3}
\paragraph{Velocity and pressure}

Let's start with the illustration of figure (\ref{fig:twoCylinder:twoCylinderFig1}) to fix coordinates.

\imageFigure{figures/twoCylinderFig1}{Coordinates for flow between two cylinders.}{fig:twoCylinder:twoCylinderFig1}{0.2}

We'll assume that we can find a solution of the following form

\begin{subequations}
\begin{equation}\label{eqn:twoCylinder:10}
\Bu = w(r) \zcap
\end{equation}
\begin{equation}\label{eqn:twoCylinder:30}
p = p(r).
\end{equation}
\end{subequations}

We'll also work in cylindrical coordinates where our gradient is

\begin{equation}\label{eqn:twoCylinder:50}
\spacegrad = \rcap \partial_r + \frac{\phicap}{r} \partial_\phi + \zcap \partial_z.
\end{equation}

Let's look at the various terms of the Navier-Stokes equation.  Our non-linear term is

\begin{equation}\label{eqn:twoCylinder:70}
\Bu \cdot \spacegrad \Bu = w \partial_z ( w(r) \zcap ) = 0,
\end{equation}

Our Laplacian term is

\begin{align*}
\mu \spacegrad^2 \Bu
&= \mu \left( \inv{r} \partial_r ( r \partial_r ) + \inv{r^2} \partial_{\phi\phi} + \partial_{z z} \right) w(r) \zcap \\
&=
\frac{\mu}{r} ( r w' )' \zcap.
\end{align*}

Putting the pieces together we have

\begin{equation}\label{eqn:twoCylinder:90}
0 = - \rcap p' + \frac{\mu}{r} (r w')' \zcap.
\end{equation}

Decomposing these into one equation for each component we have

\begin{equation}\label{eqn:twoCylinder:110}
p' = 0,
\end{equation}

and

\begin{equation}\label{eqn:twoCylinder:130}
(r w')' = 0.
\end{equation}

The pressure can be trivially solved

\begin{equation}\label{eqn:twoCylinder:150}
p(r) = \text{constant},
\end{equation}

and for our velocity equation we get

\begin{equation}\label{eqn:twoCylinder:170}
r w' = A,
\end{equation}

Short of satisfying our boundary value constraints our velocity is

\begin{equation}\label{eqn:twoCylinder:190}
w = A \ln r + B.
\end{equation}

Our boundary value conditions are given by

\begin{align}\label{eqn:twoCylinder:210}
w(R_2) &= 0 \\
w(R_1) &= v,
\end{align}

so our integration constants are given by

\begin{align}\label{eqn:twoCylinder:230}
0 &= A \ln R_2 + B \\
v &= A \ln R_1 + B.
\end{align}

Taking differences we've got

\begin{equation}\label{eqn:twoCylinder:250}
v = A \ln( R_1/R_2 ).
\end{equation}

So our constants are

\begin{subequations}
\begin{equation}\label{eqn:twoCylinder:270}
A = \frac{v}{ \ln( R_1/R_2 ) }
\end{equation}
\begin{equation}\label{eqn:twoCylinder:290}
B = -\frac{v \ln R_2}{ \ln( R_1/R_2 ) },
\end{equation}
\end{subequations}

and

\begin{equation}\label{eqn:twoCylinder:310}
\boxed{
w(r) = \frac{ v \ln (r/R_2) }{ \ln( R_1/R_2 ) }.
}
\end{equation}

A plot of this function can be found in figure (\ref{fig:twoCylinder:twoCylinderFig2}), and the Mathematica notebook that generated this plot can be found in \href{https://raw.github.com/peeterjoot/physicsplay/master/notes/phy454/mathematica/twoCylinders.cdf}{twoCylinders.cdf}.  That notebook has some slider controls that can be used interactively.  A sample animation of that interactive capability is available at
\FIXME{upload to youtube and link}
%included in figure (\ref{fig:twoCylinders:twoCylindersFig3}).

\imageFigure{figures/twoCylinderFig2}{Velocity plot due to inner cylinder dragging fluid along with it.}{fig:twoCylinder:twoCylinderFig2}{0.2}

%\movieFigure{twoCylindersFig3.mp4}{Animating the two cylinder velocity field for a set of parameter values.}{fig:twoCylinders:twoCylindersFig3}{width=298pt,height=320pt}

\ref{problem:fluids:twoCylinders:2}
\paragraph{Friction force per unit length.}

For the frictional force per unit area on the fluid by the inner cylinder we have

\begin{align*}
(\Bsigma \cdot \rcap ) \cdot \zcap
&=
-p \rcap \cdot \zcap +
2 \mu \inv{2} \left(
\PD{r}{u_z}
+\cancel{\PD{z}{u_r}}
\right) \\
&=
\mu v \frac{\ln r }{\ln (R_1/R_2) }
\end{align*}

So the forces on the inner and outer cylinders for a strip of width $\Delta z$ is

\begin{subequations}
\begin{equation}\label{eqn:twoCylinder:330}
\text{frictional force on inner cylinder}
= -2 \pi R_1 \Delta z \mu v \zcap \frac{\ln R_1 }{\ln (R_1/R_2) }
\end{equation}
\begin{equation}\label{eqn:twoCylinder:350}
\text{frictional force on inner cylinder}
=
2 \pi R_2 \Delta z \mu v \zcap \frac{\ln R_2 }{\ln (R_1/R_2) }
\end{equation}
\end{subequations}

\ref{problem:fluids:twoCylinders:4}
\paragraph{With external pressure gradient.}

With an external pressure gradient imposed we expect a superposition of a parabolic flow profile with what we've calculated above.  With

\begin{equation}\label{eqn:twoCylinder:370}
G = - \frac{dp}{dz},
\end{equation}

our Navier-Stokes equation will now take the form

\begin{equation}\label{eqn:twoCylinder:390}
0 = - \rcap p' - (-G \zcap) + \frac{\mu}{r} (r w')' \zcap.
\end{equation}

We want to solve the LDE

\begin{equation}\label{eqn:twoCylinder:410}
- \frac{ G r}{\mu} =
(r w')' = r w'' + w'
\end{equation}

The homogeneous portion of this equation

\begin{equation}\label{eqn:twoCylinder:430}
(r w')' = 0,
\end{equation}

we have already solved finding $w = C \ln r + D$.  It looks reasonable to try a polynomial solution for the specific solution.  Let's try a second order polynomial

\begin{subequations}
\begin{equation}\label{eqn:twoCylinder:450}
w = A r^2 + B r
\end{equation}
\begin{equation}\label{eqn:twoCylinder:470}
w' = 2 A r + B
\end{equation}
\begin{equation}\label{eqn:twoCylinder:490}
w'' = 2 A.
\end{equation}
\end{subequations}

We need

\begin{equation}\label{eqn:twoCylinder:510}
-\frac{G r}{\mu} = 2 A r + 2 A r + B,
\end{equation}

So $B = 0$ and $4 A = -G/\mu$, and our general solution has the form

\begin{equation}\label{eqn:twoCylinder:530}
w = -\frac{G}{4 \mu} r^2 + C \ln r + D.
\end{equation}

requiring just the boundary condition fitting.  Let's tweak the constants slightly, writing

\begin{equation}\label{eqn:twoCylinder:550}
w = \frac{G}{4 \mu} (R_2^2 - r^2) + C \ln r/R_2 + D,
\end{equation}

so that $D = 0$ falls out of the $w(R_2) = 0$ constraint.  Our last integration constant is then determined by the solution of

\begin{equation}\label{eqn:twoCylinder:570}
v = \frac{G}{4 \mu} (R_2^2 - R_1^2) + C \ln R_1/R_2.
\end{equation}

Or

\begin{equation}\label{eqn:twoCylinder:590}
\boxed{
w = \frac{G}{4 \mu} (R_2^2 - r^2) +
\left(v - \frac{G}{4 \mu} (R_2^2 - R_1^2)\right)
\frac{\ln r/R_2}{\ln R_1/R_2}.
}
\end{equation}

A plot of this, with a pressure gradient small enough that we still see the logarithmic profile is shown in figure (\ref{fig:twoCylinder:twoCylinderFig4}).  An animation of this with different values for $R_1$, $v$, and $G/4\mu$ is available on \youtubehref{BNgpnYeRpLo}, but the Mathematica notebook above can also be used.

\imageFigure{figures/twoCylinderFig4}{Pressure gradient added.}{fig:twoCylinder:twoCylinderFig4}{0.2}

% Too big and two slow to embed:
%figure (\ref{fig:twoCylinders:twoCylindersFig5}).
%\movieFigure{twoCylindersFig5.flv}{5: FIXME: CAPTION}{fig:twoCylinders:twoCylindersFig5}{width=320pt,height=240pt}

Even cooler is to look at some plots of the velocity profiles in 3D

%figure (\ref{fig:twoCylinders:twoCylindersFig6}).
\imageFigure{figures/twoCylindersFig6}{3D plot 1}{fig:twoCylinders:twoCylindersFig6}{0.2}
%figure (\ref{fig:twoCylinders:twoCylindersFig7}).
\imageFigure{figures/twoCylindersFig7}{3D plot 2}{fig:twoCylinders:twoCylindersFig7}{0.2}
%figure (\ref{fig:twoCylinders:twoCylindersFig8}).
\imageFigure{figures/twoCylindersFig8}{3D plot 3}{fig:twoCylinders:twoCylindersFig8}{0.2}
%figure (\ref{fig:twoCylinders:twoCylindersFig9}).
\imageFigure{figures/twoCylindersFig9}{3D plot 4}{fig:twoCylinders:twoCylindersFig9}{0.2}

An animation of this can be found at \youtubehref{OiJTopWx7L8}, and the Mathematica notebook that generated it at \href{https://raw.github.com/peeterjoot/physicsplay/master/notes/phy454/mathematica/twoCylinders3D.cdf}{twoCylinders3D.cdf}.  That notebook is now also available online on the Wolfram demonstrations project \citep{wolfram3DcylinderFlow}.

%\movieFigure{twoCylindersFig11ve.mp4}{FIXME: CAPTION}{fig:twoCylindersFig11ve:twoCylindersFig11ve}{width=746pt,height=608pt}
%\movieFigure{twoCylindersFig11m.mp4}{FIXME: CAPTION}{fig:twoCylindersFig11m:twoCylindersFig11m}{width=546pt,height=408pt}
%\movieFigure{twoCylindersFig11f.mp4}{FIXME: CAPTION}{fig:twoCylindersFig11f:twoCylindersFig11f}{width=546pt,height=408pt}
%\movieFigure{twoCylindersFig11a.mp4}{FIXME: CAPTION}{fig:twoCylindersFig11a:twoCylindersFig11a}{width=546pt,height=412pt}

\end{Answer}
