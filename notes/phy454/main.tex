%
%
%
% Copyright � 2012 Peeter Joot
% All Rights Reserved
%
% This file may be reproduced and distributed in whole or in part, without fee, subject to the following conditions:
%
% o The copyright notice above and this permission notice must be preserved complete on all complete or partial copies.
%
% o Any translation or derived work must be approved by the author in writing before distribution.
%
% o If you distribute this work in part, instructions for obtaining the complete version of this file must be included, and a means for obtaining a complete version provided.
%
%
% Exceptions to these rules may be granted for academic purposes: Write to the author and ask.
%
%
%
%\documentclass[12pt,leqno]{book}
\documentclass[12pt,openany]{memoir}

%% http://tex.stackexchange.com/questions/26467/why-did-they-put-pdf-titling-in-the-hyperref-package (Stefan's answer) :
%\pdfinfo{
%  /Title (Course notes and problems from University of Toronto PHY454H1S Continuum Mechanics.)%
%%  /Creator ()
%  /Producer (pdflatex)
%  /Author (Peeter Joot)%
%  /Subject (Fluid Mechanics)
%  /Keywords (Elasticity, Fluid Mechanics, Navier-Stokes)
%}
% didn't work.  try hyperref instead.

\usepackage{amsmath,amssymb,amsfonts} % Typical maths resource packages
\usepackage{graphicx}
\usepackage{txfonts}
\usepackage{listings}
\usepackage[bookmarks=true,plainpages=false]{hyperref}
%\usepackage{easybmat}

%\usepackage[english]{babel}
%\usepackage{media9}
%\usepackage{movie15}

% for captionof
%\usepackage[]{caption}
%\usepackage[hypcap=false]{caption}

% also using caption for color captions:
% http://www.latex-community.org/forum/viewtopic.php?f=45&p=42828
\usepackage[svgnames]{xcolor}
\usepackage[labelfont={color=Maroon,bf}]{caption}
%\usepackage{caption}

\usepackage[answerdelayed]{exercise}
\usepackage[]{makeidx}

%------------------------
% colorize section headers:
%------------------------
% http://stackoverflow.com/questions/3275770/modifying-section-to-make-it-colorful-with-latex
%\titleformat{\section}
%{\normalfont\Large\bfseries}
%{\color{blue}\thesection}{1em}{}
%\titlelabel{\color{blue}{\thetitle}.\quad}
% better:
%------------------------
%
% http://tex.stackexchange.com/questions/56004/customizing-section-formatting-using-memoir-class-color-and-numbering/56006#56006
%
% To control the level up to which sectional units will be numbered, you can use \setsecnumdepth{<sectional unit>}; to color the sectional unit numbers, you can use \setsecnumformat:
%
\setsecnumformat{\color{blue}\csname the#1\endcsname\quad} % csname is for macro expansion: http://www.tex.ac.uk/cgi-bin/texfaq2html?label=csname
\setsecnumdepth{subsubsection}
%------------------------

%% MMM: these were from my book class usage, but work fine with memoir too.
\parindent 1cm
\parskip 0.2cm
\topmargin 0.2cm
\oddsidemargin 1cm
\evensidemargin 0.5cm
\textwidth 15cm
\textheight 21cm

\usepackage{amsmath}
\usepackage{mathpazo}

%
% shorthand for bold symbols, convenient for vectors and matrices
%
\newcommand{\Ba}[0]{\mathbf{a}}
\newcommand{\Bb}[0]{\mathbf{b}}
\newcommand{\Bc}[0]{\mathbf{c}}
\newcommand{\Bd}[0]{\mathbf{d}}
\newcommand{\Be}[0]{\mathbf{e}}
\newcommand{\Bf}[0]{\mathbf{f}}
\newcommand{\Bg}[0]{\mathbf{g}}
\newcommand{\Bh}[0]{\mathbf{h}}
\newcommand{\Bi}[0]{\mathbf{i}}
\newcommand{\Bj}[0]{\mathbf{j}}
\newcommand{\Bk}[0]{\mathbf{k}}
\newcommand{\Bl}[0]{\mathbf{l}}
\newcommand{\Bm}[0]{\mathbf{m}}
\newcommand{\Bn}[0]{\mathbf{n}}
\newcommand{\Bo}[0]{\mathbf{o}}
\newcommand{\Bp}[0]{\mathbf{p}}
\newcommand{\Bq}[0]{\mathbf{q}}
\newcommand{\Br}[0]{\mathbf{r}}
\newcommand{\Bs}[0]{\mathbf{s}}
\newcommand{\Bt}[0]{\mathbf{t}}
\newcommand{\Bu}[0]{\mathbf{u}}
\newcommand{\Bv}[0]{\mathbf{v}}
\newcommand{\Bw}[0]{\mathbf{w}}
\newcommand{\Bx}[0]{\mathbf{x}}
\newcommand{\By}[0]{\mathbf{y}}
\newcommand{\Bz}[0]{\mathbf{z}}
\newcommand{\BA}[0]{\mathbf{A}}
\newcommand{\BB}[0]{\mathbf{B}}
\newcommand{\BC}[0]{\mathbf{C}}
\newcommand{\BD}[0]{\mathbf{D}}
\newcommand{\BE}[0]{\mathbf{E}}
\newcommand{\BF}[0]{\mathbf{F}}
\newcommand{\BG}[0]{\mathbf{G}}
\newcommand{\BH}[0]{\mathbf{H}}
\newcommand{\BI}[0]{\mathbf{I}}
\newcommand{\BJ}[0]{\mathbf{J}}
\newcommand{\BK}[0]{\mathbf{K}}
\newcommand{\BL}[0]{\mathbf{L}}
\newcommand{\BM}[0]{\mathbf{M}}
\newcommand{\BN}[0]{\mathbf{N}}
\newcommand{\BO}[0]{\mathbf{O}}
\newcommand{\BP}[0]{\mathbf{P}}
\newcommand{\BQ}[0]{\mathbf{Q}}
\newcommand{\BR}[0]{\mathbf{R}}
\newcommand{\BS}[0]{\mathbf{S}}
\newcommand{\BT}[0]{\mathbf{T}}
\newcommand{\BU}[0]{\mathbf{U}}
\newcommand{\BV}[0]{\mathbf{V}}
\newcommand{\BW}[0]{\mathbf{W}}
\newcommand{\BX}[0]{\mathbf{X}}
\newcommand{\BY}[0]{\mathbf{Y}}
\newcommand{\BZ}[0]{\mathbf{Z}}

\newcommand{\Bzero}[0]{\mathbf{0}}
\newcommand{\Btheta}[0]{\boldsymbol{\theta}}
\newcommand{\Btau}[0]{\boldsymbol{\tau}}
\newcommand{\Bomega}[0]{\boldsymbol{\omega}}

%
% shorthand for unit vectors
%
\newcommand{\acap}[0]{\hat{\Ba}}
\newcommand{\bcap}[0]{\hat{\Bb}}
\newcommand{\ccap}[0]{\hat{\Bc}}
\newcommand{\dcap}[0]{\hat{\Bd}}
\newcommand{\ecap}[0]{\hat{\Be}}
\newcommand{\fcap}[0]{\hat{\Bf}}
\newcommand{\gcap}[0]{\hat{\Bg}}
\newcommand{\hcap}[0]{\hat{\Bh}}
\newcommand{\icap}[0]{\hat{\Bi}}
\newcommand{\jcap}[0]{\hat{\Bj}}
\newcommand{\kcap}[0]{\hat{\Bk}}
\newcommand{\lcap}[0]{\hat{\Bl}}
\newcommand{\mcap}[0]{\hat{\Bm}}
\newcommand{\ncap}[0]{\hat{\Bn}}
\newcommand{\ocap}[0]{\hat{\Bo}}
\newcommand{\pcap}[0]{\hat{\Bp}}
\newcommand{\qcap}[0]{\hat{\Bq}}
\newcommand{\rcap}[0]{\hat{\Br}}
\newcommand{\scap}[0]{\hat{\Bs}}
\newcommand{\tcap}[0]{\hat{\Bt}}
\newcommand{\ucap}[0]{\hat{\Bu}}
\newcommand{\vcap}[0]{\hat{\Bv}}
\newcommand{\wcap}[0]{\hat{\Bw}}
\newcommand{\xcap}[0]{\hat{\Bx}}
\newcommand{\ycap}[0]{\hat{\By}}
\newcommand{\zcap}[0]{\hat{\Bz}}
\newcommand{\thetacap}[0]{\hat{\Btheta}}

%
% to write R^n and C^n in a distinguishable fashion.  Perhaps change this
% to the double lined characters upon figuring out how to do so.
%
\newcommand{\C}[1]{$\mathbb{C}^{#1}$}
\newcommand{\R}[1]{$\mathbb{R}^{#1}$}

%
% various generally useful helpers
%

% derivative of #1 wrt. #2:
\newcommand{\D}[2] {\frac {d#2} {d#1}}

\newcommand{\inv}[1]{\frac{1}{#1}}
\newcommand{\cross}[0]{\times}

\newcommand{\abs}[1]{\lvert{#1}\rvert}
\newcommand{\norm}[1]{\lVert{#1}\rVert}
\newcommand{\innerprod}[2]{\langle{#1}, {#2}\rangle}
\newcommand{\dotprod}[2]{{#1} \cdot {#2}}
\newcommand{\bdotprod}[2]{\left({#1} \cdot {#2}\right)}
\newcommand{\crossprod}[2]{{#1} \cross {#2}}
\newcommand{\tripleprod}[3]{\dotprod{\left(\crossprod{#1}{#2}\right)}{#3}}

\DeclareMathOperator{\Proj}{Proj}
\DeclareMathOperator{\Span}{span}
\DeclareMathOperator{\Sgn}{sgn}
\DeclareMathOperator{\Area}{Area}
\DeclareMathOperator{\Volume}{Volume}

%
% A few miscellaneous things specific to this document
%
\newcommand{\crossop}[1]{\crossprod{#1}{}}

% R2 vector.
\newcommand{\VectorTwo}[2]{
\begin{bmatrix}
 {#1} \\
 {#2}
\end{bmatrix}
}

\newcommand{\VectorN}[1]{
\begin{bmatrix}
{#1}_1 \\
{#1}_2 \\
\vdots \\
{#1}_N \\
\end{bmatrix}
}

\newcommand{\DETuvij}[4]{
\begin{vmatrix}
 {#1}_{#3} & {#1}_{#4} \\
 {#2}_{#3} & {#2}_{#4}
\end{vmatrix}
}

\newcommand{\DETuvwijk}[6]{
\begin{vmatrix}
 {#1}_{#4} & {#1}_{#5} & {#1}_{#6} \\
 {#2}_{#4} & {#2}_{#5} & {#2}_{#6} \\
 {#3}_{#4} & {#3}_{#5} & {#3}_{#6}
\end{vmatrix}
}

\newcommand{\DETuvwxijkl}[8]{
\begin{vmatrix}
 {#1}_{#5} & {#1}_{#6} & {#1}_{#7} & {#1}_{#8} \\
 {#2}_{#5} & {#2}_{#6} & {#2}_{#7} & {#2}_{#8} \\
 {#3}_{#5} & {#3}_{#6} & {#3}_{#7} & {#3}_{#8} \\
 {#4}_{#5} & {#4}_{#6} & {#4}_{#7} & {#4}_{#8} \\
\end{vmatrix}
}

%\newcommand{\DETuvwxyijklm}[10]{
%\begin{vmatrix}
% {#1}_{#6} & {#1}_{#7} & {#1}_{#8} & {#1}_{#9} & {#1}_{#10} \\
% {#2}_{#6} & {#2}_{#7} & {#2}_{#8} & {#2}_{#9} & {#2}_{#10} \\
% {#3}_{#6} & {#3}_{#7} & {#3}_{#8} & {#3}_{#9} & {#3}_{#10} \\
% {#4}_{#6} & {#4}_{#7} & {#4}_{#8} & {#4}_{#9} & {#4}_{#10} \\
% {#5}_{#6} & {#5}_{#7} & {#5}_{#8} & {#5}_{#9} & {#5}_{#10}
%\end{vmatrix}
%}

% R3 vector.
\newcommand{\VectorThree}[3]{
\begin{bmatrix}
 {#1} \\
 {#2} \\
 {#3}
\end{bmatrix}
}



\newcommand{\indexpair}[1]{#1 \index{#1}}

\newcommand{\citep}[1]{\cite{#1}}
\newcommand{\citet}[1]{\cite{#1}}
\newcommand{\chapcite}[1]{\ref{chap:#1}}
% make this one a citation to an external notes compilation, but an actual formula reference
% when that 'chapter' is eventually included in the notes compilation.
\newcommand{\inbookref}[2]{\ref{#2}}

%-----------------------------------------
%
% stubs for article class.
%
\newcommand{\blogpage}[1]{}
\newcommand{\email}[1]{}
\newcommand{\beginArtWithToc}[0]{}
\newcommand{\beginArtNoToc}[0]{}
\newcommand{\EndArticle}[0]{}
\newcommand{\EndNoBibArticle}[0]{}
\newcommand{\revisionInfo}[1]{}
\newcommand{\gitRevisionInfo}[1]{}
\newcommand{\keywords}[1]{}
%-----------------------------------------
\DeclareMathOperator{\Atan}{atan}

% change \part formatting so that it matches \chapterstyle{ell}
%-----------------------------------------------------------------------------------
% http://tex.stackexchange.com/questions/49512/part-style-in-memoir-class?answertab=active#tab-top
% numeric instead of roman numeral for the part num:
\renewcommand*{\thepart}{\arabic{part}}
% ell like Fonts:
%
% ragged left: from http://tex.stackexchange.com/questions/42489/a-modification-of-part-style-in-memoir
% (moves the text to the right)
\renewcommand*{\parttitlefont}{\color{MidnightBlue}\normalfont\huge\sffamily\raggedleft}
\renewcommand*{\partnumfont}{\normalfont\HUGE\sffamily\raggedleft}
\renewcommand*{\partnamefont}{\normalfont\HUGE\sffamily\raggedleft}
%-----------------------------------------------------------------------------------
% http://tex.stackexchange.com/questions/55790/modifying-part-style-in-memoir-class-to-match-ell-chapterstyle/55795#55795
%\renewcommand{\midpartskip}{\begingroup
% \vspace*{-.4\beforechapskip}
% \begin{adjustwidth}{}{-.5\chapindent}
% \hrulefill
% \smash{\rule{0.4pt}{15mm}}
% \end{adjustwidth}\endgroup}
%-----------------------------------------------------------------------------------
% better:
% http://tex.stackexchange.com/questions/55790/modifying-part-style-in-memoir-class-to-match-ell-chapterstyle/55797#55797
\renewcommand{\beforepartskip}{%
  \null
  \vspace*{\beforechapskip}
  \vspace*{\onelineskip}
  }
\renewcommand{\midpartskip}{\begingroup
  % \vspace*{\beforechapskip}%
  \begin{adjustwidth}{}{-\chapindent}%
  \hrulefill
  \smash{\rule{0.4pt}{15mm}}
  \end{adjustwidth}\endgroup}

% this (without the \beforepartskip of the same form) force the part text higher onto the page.
%\renewcommand{\afterpartskip}{\vspace*{\fill}}
% with \beforepartskip redefined as above, don't need this.

%-----------------------------------------------------------------------------------
% http://tex.stackexchange.com/questions/55832/in-memoir-environment-the-chapter-heading-placement-gets-messed-up-if-a-too-big/55845#55845
%\makechapterstyle{boxedell}{%
%  \chapterstyle{ell}%
%  \renewcommand*{\chapterheadstart}{\begingroup%
%    \vspace*{\beforechapskip}%
%    \vbox{\begin{adjustwidth}{}{-\chapindent}%
%    \hrulefill%
%    \smash{\rule{0.4pt}{15mm}}%
%    \end{adjustwidth}}\endgroup}%
%  \renewcommand*{\printchapternum}{%
%    \vbox{\begin{adjustwidth}{}{-\chapindent}%
%    \hfill%
%    \raisebox{10mm}[0pt][0pt]{\chapnumfont \thechapter}%
%                              \hspace*{1em}%
%    \end{adjustwidth}}\vspace*{-3.0\onelineskip}}%
%}
%
% While the above is probably strictly more correct, it's not really required with
% the use of the problematic \begin{center} ... \end{center} for figures avoided.
%
% See the new \imagefigure command to see how that was done.
%
%-----------------------------------------------------------------------------------
%% http://tex.stackexchange.com/questions/10183/change-font-colour-of-bringhurst-chapterstyle-in-memoir
%\renewcommand{\printchaptertitle}[1]{%
%  \memRTLraggedright\Large\scshape\MakeLowercase{\textcolor{red}{#1}}%
%}
%-----------------------------------------------------------------------------------
\makechapterstyle{myell}{%
  \chapterstyle{ell}%
%  \renewcommand*{\chapnumfont}{\normalfont\HUGE\sffamily}
%  \renewcommand*{\chaptitlefont}{\normalfont\huge\sffamily}
%  \settowidth{\chapindent}{\chapnumfont 111}
%  \renewcommand*{\chapterheadstart}{\begingroup
%    \vspace*{\beforechapskip}%
%    \begin{adjustwidth}{}{-\chapindent}%
%    \hrulefill
%    \smash{\rule{0.4pt}{15mm}}
%    \end{adjustwidth}\endgroup}
%  \renewcommand*{\printchaptername}{}
%  \renewcommand*{\chapternamenum}{}
%  \renewcommand*{\printchapternum}{%
%    \begin{adjustwidth}{}{-\chapindent}
%    \hfill
%    \raisebox{10mm}[0pt][0pt]{\chapnumfont \thechapter}%
%                              \hspace*{1em}
%    \end{adjustwidth}\vspace*{-3.0\onelineskip}}
  \renewcommand*{\printchaptertitle}[1]{%
    \vskip\onelineskip
    \raggedleft \color{red}{\chaptitlefont ##1}\par\nobreak}}
%-----------------------------------------------------------------------------------

%%
% Copyright � 2012 Peeter Joot.  All Rights Reserved.
% Licenced as described in the file LICENSE under the root directory of this GIT repository.
%

% 
% 
\DeclareMathOperator{\Div}{div}
\DeclareMathOperator{\Mod}{mod}
\DeclareMathOperator{\PV}{PV}
\DeclareMathOperator{\Prob}{Prob}
\DeclareMathOperator{\rank}{rank}
\DeclareMathOperator{\sgn}{sgn}
\DeclareMathOperator{\sinc}{sinc}
%\DeclareMathOperator{\Atan2}{atan2}
\DeclareMathOperator{\atan}{atan}


\newcommand{\expectation}[1]{\langle{#1}\rangle}
%\newcommand{\gpgradefour}[1] {\gpgrade{#1}{4}}
%\newcommand{\gpgradeone}[1] {\gpgrade{#1}{1}}
%\newcommand{\gpgradethree}[1] {\gpgrade{#1}{3}}
%\newcommand{\gpgradetwo}[1] {\gpgrade{#1}{2}}
%\newcommand{\gpgradezero}[1] {\gpgrade{#1}{}}
%\newcommand{\gpgrade}[2] {{\left\langle{{#1}}\right\rangle}_{#2}}
%\newcommand{\grad}[0]{\boldsymbol{\nabla}}
%\newcommand{\grad}[0]{\nabla}


\newcommand{\ketbra}[2]{\ket{#1}\bra{#2}}
\newcommand{\ket}[1]{\lvert {#1} \rangle}
%\newcommand{\norm}[1]{\lVert#1\rVert}
\newcommand{\questionEquals}[0]{\stackrel{?}{=}}
\newcommand{\rightshift}[0]{\gg}
%\newcommand{\spacegrad}[0]{\boldsymbol{\nabla}}
\newcommand{\symmetric}[2]{{\left\{{#1},{#2}\right\}}}
\newcommand{\antisymmetric}[2]{\left[{#1},{#2}\right]}

%\newcommand{\Abs}[1]{\left\lvert{#1}\right\rvert}

%\newcommand{\BB}[0]{\mathbf{B}}
%\newcommand{\BE}[0]{\mathbf{E}}
%\newcommand{\BF}[0]{\mathbf{F}}
%\newcommand{\BS}[0]{\mathbf{S}}
%\newcommand{\BV}[0]{\mathbf{V}}
%\newcommand{\Bj}[0]{\mathbf{j}}

\newcommand{\BraOpKet}[3]{\bra{#1} \hat{#2} \ket{#3} }
%\newcommand{\Brho}[0]{\boldsymbol{\rho}}
\newcommand{\CC}[0]{c^2}
\newcommand{\Cos}[1]{\cos{\left({#1}\right)}}

% not working anymore.  think it's a conflicting macro for \not.
% compared to original usage in klien_gordon.ltx
%
%\newcommand{\Dslash}[0]{{\not}D}
%\newcommand{\Dslash}[0]{{\not{}}D}
% switched to cancel in macros.tex
%\newcommand{\Dslash}[0]{D\!\!\!/}

\newcommand{\Expectation}[1]{\left\langle {#1} \right\rangle}
\newcommand{\Exp}[1]{\exp{\left({#1}\right)}}
\newcommand{\FF}[0]{\mathcal{F}}
\newcommand{\FM}[0]{\inv{\sqrt{2\pi\hbar}}}
\newcommand{\IIinf}[0]{ \int_{-\infty}^\infty }
\newcommand{\Innerprod}[2]{\left\langle{#1}, {#2}\right\rangle}
%\newcommand{\LL}[0]{\mathcal{L}}

%\newcommand{\PD}[2] {\frac {\partial #2} {\partial #1}}

% backwards from ../peeterj_macros2:
\newcommand{\PDb}[2]{ \frac{\partial{#1}}{\partial {#2}} }

%\newcommand{\PDD}[3]{\frac{\partial^2 {#3}}{\partial {#1}\partial {#2}}}
\newcommand{\PDN}[3]{\frac{\partial^{#3} {#2}}{\partial {#1}^{#3}}}

\newcommand{\PDSq}[2]{\frac{\partial^2 {#2}}{\partial {#1}^2}}
\newcommand{\PDsQ}[2]{\frac{\partial^2 {#2}}{\partial^2 {#1}}}

\newcommand{\Sch}[0]{{Schr\"{o}dinger} }
\newcommand{\Sin}[1]{\sin{\left({#1}\right)}}
\newcommand{\Sw}[0]{\mathcal{S}}
%\newcommand{\T}[0]{\text{T}}
\newcommand{\T}[0]{{\text{T}}}

\newcommand{\braket}[2]{\langle{#1} \vert {#2}\rangle}
\newcommand{\bra}[1]{\langle {#1} \rvert}
\newcommand{\curl}[0]{\grad \times}
\newcommand{\delambert}[0]{\sum_{\alpha = 1}^4{\PDSq{x_\alpha}{}}}
\newcommand{\delsquared}[0]{\nabla^2}
\newcommand{\diverg}[0]{\grad \cdot}

\newcommand{\halfPhi}[0]{\frac{\phi}{2}}
\newcommand{\hatH}[0]{\hat{H}}
\newcommand{\hatS}[0]{\hat{S}}
\newcommand{\hatk}[0]{\hat{k}}
\newcommand{\hatp}[0]{\hat{p}}
\newcommand{\hatx}[0]{\hat{x}}


\newcommand{\Rdot}[0]{\dot{R}}
%\newcommand{\addot}[0]{\ddot{a}}
%\newcommand{\adot}[0]{\dot{a}}
%\newcommand{\fddot}[0]{\ddot{f}}
%\newcommand{\fdot}[0]{\dot{f}}
%\newcommand{\bddot}[0]{\ddot{b}}
%\newcommand{\bdot}[0]{\dot{b}}
\newcommand{\ddotOmega}[0]{\ddot{\Omega}}
\newcommand{\ddotalpha}[0]{\ddot{\alpha}}
\newcommand{\ddotomega}[0]{\ddot{\omega}}
\newcommand{\ddotphi}[0]{\ddot{\phi}}
\newcommand{\ddotpsi}[0]{\ddot{\psi}}
\newcommand{\ddottheta}[0]{\ddot{\theta}}
\newcommand{\dotOmega}[0]{\dot{\Omega}}
\newcommand{\dotalpha}[0]{\dot{\alpha}}
\newcommand{\dotomega}[0]{\dot{\omega}}
\newcommand{\dotphi}[0]{\dot{\phi}}
\newcommand{\dotpsi}[0]{\dot{\psi}}
\newcommand{\dottheta}[0]{\dot{\theta}}
%\newcommand{\pddot}[0]{\ddot{p}}
%\newcommand{\pdot}[0]{\dot{p}}
%\newcommand{\qddot}[0]{\ddot{q}}
%\newcommand{\qdot}[0]{\dot{q}}
%\newcommand{\rddot}[0]{\ddot{r}}
%\newcommand{\rdot}[0]{\dot{r}}
%\newcommand{\tddot}[0]{\ddot{t}}
%\newcommand{\tdot}[0]{\dot{t}}
%\newcommand{\uddot}[0]{\ddot{u}}
%\newcommand{\udot}[0]{\dot{u}}
%\newcommand{\xddot}[0]{\ddot{x}}
%\newcommand{\xdot}[0]{\dot{x}}
%\newcommand{\yddot}[0]{\ddot{y}}
%\newcommand{\ydot}[0]{\dot{y}}
%\newcommand{\zddot}[0]{\ddot{z}}
%\newcommand{\zdot}[0]{\dot{z}}








%-------------------------------------------------------------------
% ORIGINS:
%
% bohm11.tex

%\DeclareMathOperator{\sgn}{sgn}
%\newcommand{\PDSq}[2]{\frac{\partial^2 {#2}}{\partial {#1}^2}}
%\newcommand{\PDN}[3]{\frac{\partial^{#3} {#2}}{\partial {#1}^{#3}}}
%\DeclareMathOperator{\sinc}{sinc}
%\DeclareMathOperator{\PV}{PV}
%\newcommand{\FF}[0]{\mathcal{F}}
%\newcommand{\Sw}[0]{\mathcal{S}}
%\newcommand{\IIinf}[0]{ \int_{-\infty}^\infty }
%\newcommand{\FM}[0]{\inv{\sqrt{2\pi\hbar}}}
%\newcommand{\expectation}[1]{\langle{#1}\rangle}
%
%

% bohm_ch10.tex

%\DeclareMathOperator{\sgn}{sgn}
%\newcommand{\expectation}[1]{\langle{#1}\rangle}
%\newcommand{\IIinf}[0]{ \int_{-\infty}^\infty }
%\DeclareMathOperator{\PV}{PV}
%
%

% bohm_ch9.tex

%\newcommand{\PDSq}[2]{\frac{\partial^2 {#2}}{\partial {#1}^2}}
%\newcommand{\PDN}[3]{\frac{\partial^{#3} {#2}}{\partial {#1}^{#3}}}
%\DeclareMathOperator{\sinc}{sinc}
%\DeclareMathOperator{\PV}{PV}
%\newcommand{\FF}[0]{\mathcal{F}}
%\newcommand{\Sw}[0]{\mathcal{S}}
%\newcommand{\IIinf}[0]{ \int_{-\infty}^\infty }
%\newcommand{\FM}[0]{\inv{\sqrt{2\pi\hbar}}}
%\newcommand{\expectation}[1]{\langle{#1}\rangle}
%
%

% commutator_herm.tex

%\newcommand{\symmetric}[2]{{\left\{{#1},{#2}\right\}}}
%\newcommand{\antisymmetric}[2]{\left[{#1},{#2}\right]}
%
%%\newcommand{\ket}[1]{\lvert {#1} \rangle}
%%\newcommand{\bra}[1]{\langle {#1} \rvert}
%%\newcommand{\braket}[2]{\langle{#1} \vert {#2}\rangle}
%%\newcommand{\ketbra}[2]{\ket{#1}\bra{#2}}
%%\newcommand{\BraOpKet}[3]{\bra{#1} \hat{#2} \ket{#3} }
%%\newcommand{\Innerprod}[2]{\left\langle{#1}, {#2}\right\rangle}
%\newcommand{\Expectation}[1]{\left\langle {#1} \right\rangle}
%
%

% delta_ortho_series.tex

%\newcommand{\IIinf}[0]{ \int_{-\infty}^\infty }
%\newcommand{\ket}[1]{\lvert {#1} \rangle}
%\newcommand{\bra}[1]{\langle {#1} \rvert}
%\newcommand{\braket}[2]{\langle{#1} \vert {#2}\rangle}
%\newcommand{\ketbra}[2]{\ket{#1}\bra{#2}}
%\newcommand{\BraOpKet}[3]{\bra{#1} \hat{#2} \ket{#3} }
%\newcommand{\Innerprod}[2]{\left\langle{#1}, {#2}\right\rangle}
%
%

% distributions.tex

%\newcommand{\PDSq}[2]{\frac{\partial^2 {#2}}{\partial {#1}^2}}
%\DeclareMathOperator{\sinc}{sinc}
%\DeclareMathOperator{\PV}{PV}
%\newcommand{\FF}[0]{\mathcal{F}}
%\newcommand{\Sw}[0]{\mathcal{S}}
%\newcommand{\IIinf}[0]{ \int_{-\infty}^\infty }
%
%

% ehrenfest.tex

%\newcommand{\PDSq}[2]{\frac{\partial^2 {#2}}{\partial {#1}^2}}
%
%

% fletcher.tex

%\DeclareMathOperator{\Div}{div}
%\DeclareMathOperator{\Mod}{mod}
%\newcommand{\rightshift}[0]{\gg}
%\newcommand{\questionEquals}[0]{\stackrel{?}{=}}
%
%

% fvec.tex

%\newcommand{\grad}[0]{\nabla}
%\newcommand{\PD}[2]{ \frac{\partial{#1}}{\partial {#2}} }
%
%

% gacs_q8_8.tex

%\newcommand{\halfPhi}[0]{\frac{\phi}{2}}
%\newcommand{\Sin}[1]{\sin{\left({#1}\right)}}
%\newcommand{\Cos}[1]{\cos{\left({#1}\right)}}
%\newcommand{\Exp}[1]{\exp{\left({#1}\right)}}
%
%

% goldstein_ch1_2.tex

%\newcommand{\spacegrad}[0]{\boldsymbol{\nabla}}
%\newcommand{\Brho}[0]{\boldsymbol{\rho}}
%\newcommand{\LL}[0]{\mathcal{L}}
%\newcommand{\Abs}[1]{\left\lvert{#1}\right\rvert}
%\newcommand{\qdot}[0]{\dot{q}}
%\newcommand{\qddot}[0]{\ddot{q}}
%\newcommand{\xdot}[0]{\dot{x}}
%\newcommand{\xddot}[0]{\ddot{x}}
%\newcommand{\ydot}[0]{\dot{y}}
%\newcommand{\yddot}[0]{\ddot{y}}
%\newcommand{\dotalpha}[0]{\dot{\alpha}}
%\newcommand{\ddotalpha}[0]{\ddot{\alpha}}
%\newcommand{\dottheta}[0]{\dot{\theta}}
%\newcommand{\ddottheta}[0]{\ddot{\theta}}
%\newcommand{\dotphi}[0]{\dot{\phi}}
%\newcommand{\ddotphi}[0]{\ddot{\phi}}
%% == \partial_{#1} {#2}
%\newcommand{\PD}[2]{\frac{\partial {#2}}{\partial {#1}}}
%\newcommand{\PDD}[3]{\frac{\partial^2 {#3}}{\partial {#1}\partial {#2}}}
%
%% <grade selection>
%%
%\newcommand{\gpgrade}[2] {{\left\langle{{#1}}\right\rangle}_{#2}}
%
%\newcommand{\gpgradezero}[1] {\gpgrade{#1}{}}
%%\newcommand{\gpscalargrade}[1] {{\left\langle{{#1}}\right\rangle}}
%%\newcommand{\gpgradezero}[1] {\gpgrade{#1}{0}}
%
%%\newcommand{\gpgradeone}[1] {{\left\langle{{#1}}\right\rangle}_{1}}
%\newcommand{\gpgradeone}[1] {\gpgrade{#1}{1}}
%
%\newcommand{\gpgradetwo}[1] {\gpgrade{#1}{2}}
%\newcommand{\gpgradethree}[1] {\gpgrade{#1}{3}}
%\newcommand{\gpgradefour}[1] {\gpgrade{#1}{4}}
%%
%% </grade selection>
%
%
%

% harmonic_osc.tex

%\newcommand{\IIinf}[0]{ \int_{-\infty}^\infty }
%
%

% klein_gordon.tex

%\newcommand{\PDSq}[2]{\frac{\partial^2 {#2}}{\partial {#1}^2}}
%%\newcommand{\Dslash}[0]{D\!\!\!/}
%\newcommand{\Dslash}[0]{{\not}D}
%
%

% matrix_to_operator.tex

%\newcommand{\T}[0]{{\text{T}}}
%
%

% maxwell.tex

%\newcommand{\norm}[1]{\lVert#1\rVert}
%\newcommand{\grad}[0]{\boldsymbol{\nabla}}
%\newcommand{\curl}[0]{\grad \times}
%\newcommand{\diverg}[0]{\grad \cdot}
%\newcommand{\delsquared}[0]{\nabla^2}
%\newcommand{\delambert}[0]{\sum_{\alpha = 1}^4{\PDSq{x_\alpha}{}}}
%
%% partial derivative of #1 wrt. #2:
%\newcommand{\PD}[2] {\frac {\partial #2} {\partial #1}}
%% second partial derivative of #1 wrt. #2:
%\newcommand{\PDSq}[2] {\frac {\partial^2 #2} {\partial {#1}^2}}
%
%%
%% shorthand for bold symbols:
%%
%\newcommand{\Bj}[0]{\mathbf{j}}
%\newcommand{\BB}[0]{\mathbf{B}}
%\newcommand{\BE}[0]{\mathbf{E}}
%\newcommand{\BF}[0]{\mathbf{F}}
%\newcommand{\BS}[0]{\mathbf{S}}
%\newcommand{\BV}[0]{\mathbf{V}}
%
%

% mp_inverse_svd_rough_notes.tex

%\newcommand{\T}[0]{\text{T}}
%\DeclareMathOperator{\rank}{rank}
%
%

% outermorphism_det.tex

%\newcommand{\gpgrade}[2] {{\left\langle{{#1}}\right\rangle}_{#2}}
%\newcommand{\gpgradeone}[1] {\gpgrade{#1}{1}}
%\newcommand{\gpgradetwo}[1] {\gpgrade{#1}{2}}
%
%

% pauli_qm_relativity_intro.tex

%\newcommand{\Sch}[0]{{Schr\"{o}dinger} }
%
%

% pe.tex

%
%\newcommand{\grad}[0]{\nabla}

% qm_susskind.tex

%\newcommand{\ket}[1]{\lvert {#1} \rangle}
%\newcommand{\bra}[1]{\langle {#1} \rvert}
%\newcommand{\braket}[2]{\langle{#1} \vert {#2}\rangle}
%\newcommand{\ketbra}[2]{\ket{#1}\bra{#2}}
%\newcommand{\BraOpKet}[3]{\bra{#1} \hat{#2} \ket{#3} }
%\newcommand{\hatH}[0]{\hat{H}}
%\newcommand{\hatS}[0]{\hat{S}}
%\newcommand{\hatk}[0]{\hat{k}}
%\newcommand{\hatx}[0]{\hat{x}}
%\newcommand{\hatp}[0]{\hat{p}}
%\DeclareMathOperator{\Prob}{Prob}
%
%

% schwartzchild_metric.tex

%\newcommand{\grad}[0]{\nabla}
%\newcommand{\Abs}[1]{\left\lvert{#1}\right\rvert}
%\newcommand{\spacegrad}[0]{\boldsymbol{\nabla}}
%\newcommand{\LL}[0]{\mathcal{L}}
%\newcommand{\PD}[2]{\frac{\partial {#2}}{\partial {#1}}}
%\newcommand{\PDsQ}[2]{\frac{\partial^2 {#2}}{\partial^2 {#1}}}
%\newcommand{\dotalpha}[0]{\dot{\alpha}}
%\newcommand{\ddotalpha}[0]{\ddot{\alpha}}
%
%\newcommand{\dotomega}[0]{\dot{\omega}}
%\newcommand{\ddotomega}[0]{\ddot{\omega}}
%
%\newcommand{\dotOmega}[0]{\dot{\Omega}}
%\newcommand{\ddotOmega}[0]{\ddot{\Omega}}
%
%\newcommand{\CC}[0]{c^2}
%
%\newcommand{\dottheta}[0]{\dot{\theta}}
%\newcommand{\ddottheta}[0]{\ddot{\theta}}
%
%\newcommand{\dotpsi}[0]{\dot{\psi}}
%\newcommand{\ddotpsi}[0]{\ddot{\psi}}
%
%\newcommand{\adot}[0]{\dot{a}}
%\newcommand{\addot}[0]{\ddot{a}}
%\newcommand{\udot}[0]{\dot{u}}
%\newcommand{\uddot}[0]{\ddot{u}}
%\newcommand{\fdot}[0]{\dot{f}}
%\newcommand{\fddot}[0]{\ddot{f}}
%\newcommand{\bdot}[0]{\dot{b}}
%\newcommand{\bddot}[0]{\ddot{b}}
%\newcommand{\qdot}[0]{\dot{q}}
%\newcommand{\qddot}[0]{\ddot{q}}
%\newcommand{\tdot}[0]{\dot{t}}
%\newcommand{\tddot}[0]{\ddot{t}}
%
%\newcommand{\Rdot}[0]{\dot{R}}
%
%\newcommand{\pdot}[0]{\dot{p}}
%\newcommand{\pddot}[0]{\ddot{p}}
%
%\newcommand{\xdot}[0]{\dot{x}}
%\newcommand{\xddot}[0]{\ddot{x}}
%
%\newcommand{\zdot}[0]{\dot{z}}
%\newcommand{\zddot}[0]{\ddot{z}}
%
%\newcommand{\rdot}[0]{\dot{r}}
%\newcommand{\rddot}[0]{\ddot{r}}
%
%

% shear.tex

%\newcommand{\gpgrade}[2] {{\left\langle{{#1}}\right\rangle}_{#2}}
%
%

% tong_mf1.tex

%\newcommand{\Abs}[1]{\left\lvert{#1}\right\rvert}
%\newcommand{\grad}[0]{\nabla}
%\newcommand{\LL}[0]{\mathcal{L}}
%
%\newcommand{\dotalpha}[0]{\dot{\alpha}}
%\newcommand{\ddotalpha}[0]{\ddot{\alpha}}
%
%\newcommand{\dotomega}[0]{\dot{\omega}}
%\newcommand{\ddotomega}[0]{\ddot{\omega}}
%
%\newcommand{\dottheta}[0]{\dot{\theta}}
%\newcommand{\ddottheta}[0]{\ddot{\theta}}
%
%\newcommand{\dotpsi}[0]{\dot{\psi}}
%\newcommand{\ddotpsi}[0]{\ddot{\psi}}
%
%\newcommand{\qdot}[0]{\dot{q}}
%\newcommand{\qddot}[0]{\ddot{q}}
%
%\newcommand{\Rdot}[0]{\dot{R}}
%
%\newcommand{\pdot}[0]{\dot{p}}
%\newcommand{\pddot}[0]{\ddot{p}}
%
%\newcommand{\xdot}[0]{\dot{x}}
%\newcommand{\xddot}[0]{\ddot{x}}
%
%\newcommand{\zdot}[0]{\dot{z}}
%\newcommand{\zddot}[0]{\ddot{z}}
%
%\newcommand{\rdot}[0]{\dot{r}}
%\newcommand{\rddot}[0]{\ddot{r}}
%
%% == \partial_{#1} {#2}
%\newcommand{\PD}[2]{\frac{\partial {#2}}{\partial {#1}}}
%\newcommand{\PDD}[3]{\frac{\partial^2 {#3}}{\partial {#1}\partial {#2}}}
%
%

% wavepacket.tex

%\newcommand{\PDSq}[2]{\frac{\partial^2 {#2}}{\partial {#1}^2}}
%\newcommand{\IIinf}[0]{ \int_{-\infty}^\infty }
%
%

% wavevariation.tex

%\newcommand{\PDSq}[2]{\frac{\partial^2 {#2}}{\partial {#1}^2}}
%
%

% qm_barrier
%\DeclareMathOperator{\Atan2}{atan2}
%\DeclareMathOperator{\atan}{atan}

% twobodies.tex
%\DeclareMathOperator{\sgn}{sgn}


% sr_lagrangian_q.tex

%\newcommand{\PD}[2]{\frac{\partial {#2}}{\partial {#1}}}
%\newcommand{\xdot}[0]{\dot{x}}
%\newcommand{\xddot}[0]{\ddot{x}}

% stub_em_fields.tex

%\newcommand{\EE}[0]{\boldsymbol{\mathcal{E}}}
%\newcommand{\HH}[0]{\boldsymbol{\mathcal{H}}}
%\newcommand{\PDSq}[2]{\frac{\partial^2 {#2}}{\partial {#1}^2}}

% long_wire_q.tex

%\newcommand{\grad}[0]{\nabla}

% lorentz_tx_em_potential.tex
%\newcommand{\LL}[0]{\mathcal{L}}
%\newcommand{\grad}[0]{\nabla}
%\newcommand{\pdot}[0]{\dot{p}}
%\newcommand{\pddot}[0]{\ddot{p}}

%------------------------------------------------------
% cross_old.tex

%%
%% shorthand for bold symbols, convenient for vectors and matrices
%%
%\newcommand{\Ba}[0]{\mathbf{a}}
%\newcommand{\Bb}[0]{\mathbf{b}}
%\newcommand{\Bc}[0]{\mathbf{c}}
%\newcommand{\Bd}[0]{\mathbf{d}}
%\newcommand{\Be}[0]{\mathbf{e}}
%\newcommand{\Bf}[0]{\mathbf{f}}
%\newcommand{\Bg}[0]{\mathbf{g}}
%\newcommand{\Bh}[0]{\mathbf{h}}
%\newcommand{\Bi}[0]{\mathbf{i}}
%\newcommand{\Bj}[0]{\mathbf{j}}
%\newcommand{\Bk}[0]{\mathbf{k}}
%\newcommand{\Bl}[0]{\mathbf{l}}
%\newcommand{\Bm}[0]{\mathbf{m}}
%\newcommand{\Bn}[0]{\mathbf{n}}
%\newcommand{\Bo}[0]{\mathbf{o}}
%\newcommand{\Bp}[0]{\mathbf{p}}
%\newcommand{\Bq}[0]{\mathbf{q}}
%\newcommand{\Br}[0]{\mathbf{r}}
%\newcommand{\Bs}[0]{\mathbf{s}}
%\newcommand{\Bt}[0]{\mathbf{t}}
%\newcommand{\Bu}[0]{\mathbf{u}}
%\newcommand{\Bv}[0]{\mathbf{v}}
%\newcommand{\Bw}[0]{\mathbf{w}}
%\newcommand{\Bx}[0]{\mathbf{x}}
%\newcommand{\By}[0]{\mathbf{y}}
%\newcommand{\Bz}[0]{\mathbf{z}}
%\newcommand{\BA}[0]{\mathbf{A}}
%\newcommand{\BB}[0]{\mathbf{B}}
%\newcommand{\BC}[0]{\mathbf{C}}
%\newcommand{\BD}[0]{\mathbf{D}}
%\newcommand{\BE}[0]{\mathbf{E}}
%\newcommand{\BF}[0]{\mathbf{F}}
%\newcommand{\BG}[0]{\mathbf{G}}
%\newcommand{\BH}[0]{\mathbf{H}}
%\newcommand{\BI}[0]{\mathbf{I}}
%\newcommand{\BJ}[0]{\mathbf{J}}
%\newcommand{\BK}[0]{\mathbf{K}}
%\newcommand{\BL}[0]{\mathbf{L}}
%\newcommand{\BM}[0]{\mathbf{M}}
%\newcommand{\BN}[0]{\mathbf{N}}
%\newcommand{\BO}[0]{\mathbf{O}}
%\newcommand{\BP}[0]{\mathbf{P}}
%\newcommand{\BQ}[0]{\mathbf{Q}}
%\newcommand{\BR}[0]{\mathbf{R}}
%\newcommand{\BS}[0]{\mathbf{S}}
%\newcommand{\BT}[0]{\mathbf{T}}
%\newcommand{\BU}[0]{\mathbf{U}}
%\newcommand{\BV}[0]{\mathbf{V}}
%\newcommand{\BW}[0]{\mathbf{W}}
%\newcommand{\BX}[0]{\mathbf{X}}
%\newcommand{\BY}[0]{\mathbf{Y}}
%\newcommand{\BZ}[0]{\mathbf{Z}}
%
%\newcommand{\Bzero}[0]{\mathbf{0}}
%\newcommand{\Btheta}[0]{\boldsymbol{\theta}}
%\newcommand{\Btau}[0]{\boldsymbol{\tau}}
%\newcommand{\Bomega}[0]{\boldsymbol{\omega}}
%
%%
%% shorthand for unit vectors
%%
%\newcommand{\acap}[0]{\hat{\Ba}}
%\newcommand{\bcap}[0]{\hat{\Bb}}
%\newcommand{\ccap}[0]{\hat{\Bc}}
%\newcommand{\dcap}[0]{\hat{\Bd}}
%\newcommand{\ecap}[0]{\hat{\Be}}
%\newcommand{\fcap}[0]{\hat{\Bf}}
%\newcommand{\gcap}[0]{\hat{\Bg}}
%\newcommand{\hcap}[0]{\hat{\Bh}}
%\newcommand{\icap}[0]{\hat{\Bi}}
%\newcommand{\jCap}[0]{\hat{\Bj}}
%\newcommand{\kcap}[0]{\hat{\Bk}}
%\newcommand{\lcap}[0]{\hat{\Bl}}
%\newcommand{\mcap}[0]{\hat{\Bm}}
%\newcommand{\ncap}[0]{\hat{\Bn}}
%\newcommand{\ocap}[0]{\hat{\Bo}}
%\newcommand{\pcap}[0]{\hat{\Bp}}
%\newcommand{\qcap}[0]{\hat{\Bq}}
%\newcommand{\rcap}[0]{\hat{\Br}}
%\newcommand{\scap}[0]{\hat{\Bs}}
%\newcommand{\tcap}[0]{\hat{\Bt}}
%\newcommand{\ucap}[0]{\hat{\Bu}}
%\newcommand{\vcap}[0]{\hat{\Bv}}
%\newcommand{\wcap}[0]{\hat{\Bw}}
%\newcommand{\xcap}[0]{\hat{\Bx}}
%\newcommand{\ycap}[0]{\hat{\By}}
%\newcommand{\zcap}[0]{\hat{\Bz}}
%\newcommand{\thetacap}[0]{\hat{\Btheta}}
%
%%
%% to write R^n and C^n in a distinguishable fashion.  Perhaps change this
%% to the double lined characters upon figuring out how to do so.
%%
%\newcommand{\C}[1]{${\BC}^{#1}$}
%\newcommand{\R}[1]{${\BR}^{#1}$}
%
%%
%% various generally useful helpers
%%
%
%% derivative of #1 wrt. #2:
%\newcommand{\D}[2] {\frac {d#2} {d#1}}

%\newcommand{\inv}[1]{\frac{1}{#1}}
%\newcommand{\cross}[0]{\times}

%\newcommand{\abs}[1]{\lvert#1\rvert}
%\newcommand{\norm}[1]{\lVert#1\rVert}
%\newcommand{\innerprod}[2]{\langle{#1}, {#2}\rangle}
%\newcommand{\dotprod}[2]{#1 \cdot #2}
%\newcommand{\crossprod}[2]{#1 \cross #2}
%\newcommand{\tripleprod}[3]{\dotprod{\crossprod{#1}{#2}}{#3}}

%
% A few miscellaneous things specific to this document
%
%\newcommand{\crossop}[1]{\crossprod{#1}{}}

\newcommand{\PDP}[2]{\BP^{#1}\BD{\BP^{#2}}}
\newcommand{\PDPDP}[3]{\Bv^T\BP^{#1}\BD\BP^{#2}\BD\BP^{#3}\Bv}

\newcommand{\Mp}[0]{
\begin{bmatrix}
0 & 1 & 0 & 0 \\
0 & 0 & 1 & 0 \\
0 & 0 & 0 & 1 \\
1 & 0 & 0 & 0
\end{bmatrix}
}
\newcommand{\Mpp}[0]{
\begin{bmatrix}
0 & 0 & 1 & 0 \\
0 & 0 & 0 & 1 \\
1 & 0 & 0 & 0 \\
0 & 1 & 0 & 0
\end{bmatrix}
}
\newcommand{\Mppp}[0]{
\begin{bmatrix}
0 & 0 & 0 & 1 \\
1 & 0 & 0 & 0 \\
0 & 1 & 0 & 0 \\
0 & 0 & 1 & 0
\end{bmatrix}
}
\newcommand{\Mpu}[0]{
\begin{bmatrix}
u_1 & 0 & 0 & 0 \\
0 & u_2 & 0 & 0 \\
0 & 0 & u_3 & 0 \\
0 & 0 & 0 & u_4
\end{bmatrix}
}

%------------------------------------------------------



% from memoir package:
%\newcommand{\titleJT}{\begingroup% Jan Tschichold: typographer
%\tdrop = 0.08\txtheight
%\vspace*{\tdrop}
%\hspace*{0.3\txtwidth}
%{\Large Peeter Joot \quad peeter.joot@gmail.com}\\[2\tdrop]
%\hspace*{0.3\txtwidth}{\Huge\itshape University of Toronto PHY454H1S}\par
%{\raggedleft\Huge\itshape Continuum Mechanics\par}
%\vfill
%\vspace*{\tdrop}
%\endgroup}
%%\FSfont{5gm}% Garamond
%%\hspace*{0.3\txtwidth}{\Large \plogo} \\[0.5\baselineskip]
%%\hspace*{0.3\txtwidth}{\Large The Publisher}

%\newcommand*{\titleJT}{\begingroup% Jan Tschichold: typographer
%\FSfont{5gm}% Garamond
%\tdrop = 0.08\txtheight
%\vspace*{\tdrop}
%\hspace*{0.3\txtwidth}
%{\Large The Author}\\[2\tdrop]
%\hspace*{0.3\txtwidth}{\Huge\itshape The Big Book of}\par
%{\raggedleft\Huge\itshape Conundrums\par}
%\vfill
%\hspace*{0.3\txtwidth}{\Large \plogo} \\[0.5\baselineskip]
%\hspace*{0.3\txtwidth}{\Large The Publisher}
%\vspace*{\tdrop}
%\endgroup}

% remove the centerline that we have in exercise.sty
%\renewcommand{\ExerciseHeader}{\textbf{\large\ExerciseName\ExerciseHeaderNB\ExerciseHeaderTitle\ExerciseHeaderOrigin\medskip}\par}
%\renewcommand{\AnswerHeader}{\medskip\textbf{Answer for \ExerciseName\ \ExerciseHeaderNB}\smallskip\par}

\renewcommand{\ExerciseHeader}{\leftline{\textbf{\large\ExerciseName\ExerciseHeaderNB\ExerciseHeaderTitle\ExerciseHeaderOrigin\medskip}}}
\renewcommand{\AnswerHeader}{\medskip\leftline{\textbf{Answer for \ExerciseName\ \ExerciseHeaderNB}\smallskip}}

%-----------------------------------------------------------------------------
% http://www.latextemplates.com/template/multi-purpose-large-font-title-page
%
%\usepackage[a4paper,pdftex]{geometry}	% Use A4 paper margins
\usepackage[english]{babel}
%\usepackage{xcolor} % Required for specifying custom colors
\usepackage{fix-cm} % Allows increasing the font size of specific fonts beyond LaTeX default specifications

% doesn't work (probably incompatible with memoir), so may as well omit it.
%\setlength{\oddsidemargin}{0mm} % Adjust margins to center the colored title box
%\setlength{\evensidemargin}{0mm} % Margins on even pages - only necessary if adding more content to this template

\newcommand{\HRule}[1]{\hfill \rule{0.2\linewidth}{#1}} % Horizontal rule at the bottom of the page, adjust width here

\definecolor{grey}{rgb}{0.9,0.9,0.9} % Color of the box surrounding the title - these values can be changed to give the box a different color	
%
%-----------------------------------------------------------------------------




\makeindex

\begin{document}
%\chapterstyle{boxedell}
\chapterstyle{myell}
%\pagenumbering{alph}

% with the MMM block above commented out, this is required to get even left and right margins:
%
% http://tex.stackexchange.com/questions/25516/same-margins-when-specifying-twoside-in-memoir
%
%\setlrmargins{*}{*}{1}
%\checkandfixthelayout

%--------------------------------------------------------------------------------------------
%
% original plain old title page
%
%\title{Course notes and problems from\\University of Toronto PHY454H1S\\Continuum Mechanics.}
%
%\author{Peeter Joot \quad peeter.joot@gmail.com}
%\maketitle
%\newpage
%--------------------------------------------------------------------------------------------
%
% new title page (still fairly plain, but better) from
%
% http://www.latextemplates.com/template/multi-purpose-large-font-title-page
%
\input{titlePage.tex}
%--------------------------------------------------------------------------------------------

%\clearpage\pagenumbering{roman}
\tableofcontents
\newpage
%\listoftables
\listoffigures

%\clearpage\pagenumbering{arabic}

%\pagestyle{plain}

\chapter*{Copyright, Document Version, and Source access.}
  \addcontentsline{toc}{chapter}{Copyright, Document Version, and Source access.}

Copyright \copyright 2016 Peeter Joot
All Rights Reserved

This work is licenced under Creative Commons Attribution-NonCommercial-NoDerivs 3.0 (CC BY-NC-ND 3.0 \ccbyncnd) as described in \href{http://creativecommons.org/licenses/by-nc-nd/3.0/}{http://creativecommons.org/licenses/by-nc-nd/3.0/}

You are free \textbf{to Share}, to copy, distribute and transmit the work

\paragraph{Under the following conditions}:

\begin{itemize}
\item \ccAttribution Attribution.  You must attribute the work in the manner specified by the author or licensor (but not in any way that suggests that they endorse you or your use of the work).
\item \ccNonCommercial Noncommercial. You may not use this work for commercial purposes.
\item \ccNoDerivatives No Derivative Works. You may not alter, transform, or build upon this work.
\end{itemize}

With the understanding that:

\begin{itemize}
\item \textbf{Waiver}. Any of the above conditions can be waived if you get permission from the copyright holder.
\item \textbf{Public Domain}. Where the work or any of its elements is in the public domain under applicable law, that status is in no way affected by the license.
\item \textbf{Other Rights}. In no way are any of the following rights affected by the license:

\begin{itemize}
\item Your fair dealing or fair use rights, or other applicable copyright exceptions and limitations;
\item The author's moral rights;
\item Rights other persons may have either in the work itself or in how the work is used, such as publicity or privacy rights.
\end{itemize}

\item \textbf{Notice}. For any reuse or distribution, you must make clear to others the license terms of this work. The best way to do this is with a link to this web page (\href{http://creativecommons.org/licenses/by-nc-nd/3.0/}{http://creativecommons.org/licenses/by-nc-nd/3.0/})
\end{itemize}

\input{./.revinfo/lastCommitBook.tex}
%%
% Copyright � 2014 Peeter Joot.  All Rights Reserved.
% Licenced as described in the file LICENSE under the root directory of this GIT repository.
%
Here is an outline of how to get access to the sources for this document

\begin{itemize}
\item Set up a github userid and ssh key for yourself.  Instructions for that here can be found on \href{http://help.github.com/win-set-up-git/}{github}.

Even if you do not want to use this for my source repository, if you are not using version control software for your own document text files I \textit{highly} recommend you do so.

\item Install a git client on your machine.  I use either linux or a \href{http://www.cygwin.com/}{Windows cygwin environment} (free implementation of gnu Unix command line utilities.  Any other Unix would work provided you have both gnu-make, perl and a recent latex distribution.  I use MikTex on Windows, and texlive on Linux.

\item Get your self a copy of the source repository that contains the sources.  Here is an example of how to do so

\lstinputlisting[language=bash]{gitclone.ksh}

If there are mathematica notebooks associated with this document, they can be found with the sources.  You can execute any of those with the \href{http://www.wolfram.com/cdf-player/}{free Wolfram CDF} player once you install it.

\item Build yourself a copy of the pdf.  For example, after cloning the repository as above

\lstinputlisting[language=bash]{sampleBuildMake.ksh}

The first time you do this with MikTex, you will probably take a hit to install a number of packages.  If running on a non-Linux Unix platform, use gnu-make explicitly.  The document structure can be explored starting from main.tex.

\item After running the git clone command, an invokation of:

\lstinputlisting[language=bash]{gitpull.ksh}

from anywhere in the git created phyLatex directory, will get you a more recent version of the source if there have been updates.
\end{itemize}


%\part{Preface}
   %
% Copyright � 2015 Peeter Joot.  All Rights Reserved.
% Licenced as described in the file LICENSE under the root directory of this GIT repository.
%

% 
%\chapter{Preface}
% this suppresses an explicit chapter number for the preface.
\chapter*{Preface}%\normalsize
  \addcontentsline{toc}{chapter}{Preface}

This document was produced while taking the Spring 2016, University of Toronto Microwave Circuits course (ECE1236H), taught by Prof.\ G. V. Eleftheriades.

\paragraph{Course Syllabus}

This course outlines the principles of designing modern microwave and RF circuits.  Signal-integrity issues in high-speed digital circuits are also examined.

\begin{itemize}
\item The wave equation.
\item Ideal transmission lines.
\item Transients on transmission-lines.
\item Planar transmission lines and introduction to MMIC's.
\item Designing with scattering parameters.
\item Planar power dividers.
\item Directional couplers.
\item Microwave filters.
\item Solid-state microwave amplifiers.
\item Noise.
\item Diode-mixers.
\item RF receiver chains.
\item Oscillators.
\end{itemize}

\withproblemsetsMessage{
\textcolor{Maroon}{
\textit{THIS DOCUMENT IS REDACTED.  THE PROBLEM SET SOLUTIONS AND ASSOCIATED MATHEMATICA CODE IS NOT VISIBLE.  PLEASE EMAIL ME FOR THE FULL VERSION IF YOU ARE NOT TAKING ECE1236.}
}
}

\paragraph{This document contains:}

\begin{itemize}
\item Lecture notes.
\item Personal notes exploring auxiliary details.
\item Worked practice problems.

\ifthenelse{\boolean{redacted}}%
{%
\item Links to Mathematica notebooks associated with the course material and problems (but not problem sets).
}%
{
\item Assigned problems.%
\item Links to Mathematica notebooks associated with problems and course material.%
}
\end{itemize}

%This set of notes is significantly different from my notes for many other classes.  With the class taught on slides (and some of those slides mirroring the text closely), I did not take live notes in class.
%These notes fill in details that I felt deserved clarification, contain problem sets solutions, as well as a number of loosely related musings on Geometric Algebra equivalents to some of the generalized concepts of electromagnetic theory encountered in this class (i.e. magnetic sources).
%
My thanks go to Professor Eleftheriades for teaching this course.

Peeter Joot  \quad peeterjoot@protonmail.com 

%\thispagestyle{empty}

\part{Lecture Notes}
   % 
% 
% 
% Copyright © 2012 Peeter Joot
% All Rights Reserved
% 
% This file may be reproduced and distributed in whole or in part, without fee, subject to the following conditions:
% 
% o The copyright notice above and this permission notice must be preserved complete on all complete or partial copies.
% 
% o Any translation or derived work must be approved by the author in writing before distribution.
% 
% o If you distribute this work in part, instructions for obtaining the complete version of this file must be included, and a means for obtaining a complete version provided.
% 
% 
% Exceptions to these rules may be granted for academic purposes: Write to the author and ask.
% 
% 
% 

\section{Continuum Mechanics.}

Mechanics could be defined as the study of effects of forces and displacements on a physical body

%figure (\ref{fig:continuumL2:continuumL2fig1})
\begin{figure}[htp]
   \centering
   \includegraphics[totalheight=0.2\textheight]{continuumL2fig1}
   \caption{Physical body.}\label{fig:continuumL2:continuumL2fig1}
\end{figure}

In continuum mechanics we have a physical body and we are interested in the internal motions in the object.

%figure (\ref{fig:continuumL2:continuumL2fig2})
\begin{figure}[htp]
   \centering
   \includegraphics[totalheight=0.2\textheight]{continuumL2fig2}
   \caption{Control volume elements.}\label{fig:continuumL2:continuumL2fig2}
\end{figure}

For the first time considering mechanics we have to introduce the concepts of fields to make progress tackling these problems.

We will have use of the following types of fields

\begin{itemize}
\item Scalar fields.  $3^0$ components.  Examples: density, Temperature, ...
\item Vector fields.  $3^1$ components.  Examples: Force, velocity.
\item Tensor fields.  $3^2$ components.  Examples: stress, strain.
\end{itemize}

We have to consider objects (a control volume) that is small enough that we can consider that we have a point in space limit for the quantities of density and velocity.  At the same time we cannot take this limiting process to the extreme, since if we use a control volume that is sufficiently small, quantum and inter-atomic effects would have to be considered.

%figure (\ref{fig:continuumL2:continuumL2fig3})
\begin{figure}[htp]
   \centering
   \includegraphics[totalheight=0.2\textheight]{continuumL2fig3}
   \caption{Mass and volume ratios at different scales.}\label{fig:continuumL2:continuumL2fig3}
\end{figure}

\subsection{Stress and Strain definitions.}

\begin{definition}
\emph{(Stress)}
\label{dfn:continuumL2:10}
Measure of the Internal force on the surfaces.
\end{definition}

\begin{definition}
\emph{(Strain)}
\label{dfn:continuumL2:30}
Measure of the deformation of the body.
\end{definition}


%
% clutter.  Don't want too many part sections.
%\part{Readings}
%%
% Copyright � 2012 Peeter Joot.  All Rights Reserved.
% Licenced as described in the file LICENSE under the root directory of this GIT repository.
%

%
%
\chapter{Some relevant reading material}

Some noteworthy text references.

\begin{itemize}
\item \citep{landau1960theory} \S 1.  Strain tensor. (lecture 2).
\item \citep{landau1960theory} \S 2.  Strain tensor (lectures 3,4).
\item \citep{landau1960theory} \S 2, \S 4, \S 5.  Constitutive relation (lecture 5).
\item \citep{landau1960theory} Chapter I \S 7, chapter III (\S 22 - \S 26).  Compatibility condition and elastostatics (lecture 6).
\item \citep{landau1960theory} \S 22 .  P-waves and S-waves (lecture 7).
\item \citep{landau1960theory} \S 24 .  Rayleigh wave (lecture 8).
\item \citep{acheson1990elementary} \S 1.4 .  Rayleigh wave (lecture 8).
\item \citep{acheson1990elementary} \S 1 \S 6.  Navier stokes, mass conservation, ...(lecture 9, 10).
\item \citep{acheson1990elementary} \S 2.  Viscous flows (lecture 12).  Impulsively started flow (Lecture 16).
\item \citep{landau1987course} \S 7.  Surface tension (lecture 14).
\item \citep{granger1995fluid}.  \S 7.6, \S 7.7 Non-dimensionalisation and scaling.
\item \citep{acheson1990elementary} \S 8.3.  Boundary layer flows (lecture 19).
\item \citep{acheson1990elementary} \S 9.3.  Thermal stability (lecture 22).
\end{itemize}

%-------------------------------------------------------

%
% Removed because they were too unreliable.  FIXME: remove references to these videos
%\part{Multimedia figures.}
%%
% Copyright � 2012 Peeter Joot.  All Rights Reserved.
% Licenced as described in the file LICENSE under the root directory of this GIT repository.
%

%
%
\chapter{Multimedia figures}

A table of animated figures in this document.  These require Adobe Reader (google's online pdf reader will not display these for example).  Some patience is required to view these (waiting for the underlying javascript to run before clicking the play button ... on my machine I can hear acrobat churning away on the CPU at these points.)

\begin{itemize}
\item Fluids with densities of mercury and water in the lower (1) and upper (2) layers respectively.  \cref{fig:continuumProblemSet2:continuumProblemSet2Animation1}
\item Fluids with densities of water and mercury in the lower (1) and upper (2) layers respectively.  \cref{fig:continuumProblemSet2:continuumProblemSet2Animation2}
\item Time evolution of channel flow velocity profile after turning on a constant pressure gradient.  \cref{fig:channelFlowWithStepPressureGradient:channelFlowWithStepPressureGradientFig3}
\item Early time evolution of channel flow velocity profile after turning on a constant pressure gradient.  \cref{fig:channelFlowWithStepPressureGradient:channelFlowWithStepPressureGradientFig4}
\item Animation of Couette flow, with continuous variation of outer angular velocity. \cref{fig:couetteFlow:couetteFlowFig6}.
\item Animating the two cylinder velocity field for a set of parameter values.  \cref{fig:twoCylinders:twoCylindersFig3}.
\end{itemize}

%-------------------------------------------------------

%
%\part{Elasticity}
   % now removed:
   %%
%
%
% Copyright � 2012 Peeter Joot
% All Rights Reserved
%
% This file may be reproduced and distributed in whole or in part, without fee, subject to the following conditions:
%
% o The copyright notice above and this permission notice must be preserved complete on all complete or partial copies.
%
% o Any translation or derived work must be approved by the author in writing before distribution.
%
% o If you distribute this work in part, instructions for obtaining the complete version of this file must be included, and a means for obtaining a complete version provided.
%
%
% Exceptions to these rules may be granted for academic purposes: Write to the author and ask.
%
%
%
%%
% Copyright � 2015 Peeter Joot.  All Rights Reserved.
% Licenced as described in the file LICENSE under the root directory of this GIT repository.
%
\documentclass[]{eliblog}

\usepackage{amsmath}
\usepackage{mathpazo}

%
% shorthand for bold symbols, convenient for vectors and matrices
%
\newcommand{\Ba}[0]{\mathbf{a}}
\newcommand{\Bb}[0]{\mathbf{b}}
\newcommand{\Bc}[0]{\mathbf{c}}
\newcommand{\Bd}[0]{\mathbf{d}}
\newcommand{\Be}[0]{\mathbf{e}}
\newcommand{\Bf}[0]{\mathbf{f}}
\newcommand{\Bg}[0]{\mathbf{g}}
\newcommand{\Bh}[0]{\mathbf{h}}
\newcommand{\Bi}[0]{\mathbf{i}}
\newcommand{\Bj}[0]{\mathbf{j}}
\newcommand{\Bk}[0]{\mathbf{k}}
\newcommand{\Bl}[0]{\mathbf{l}}
\newcommand{\Bm}[0]{\mathbf{m}}
\newcommand{\Bn}[0]{\mathbf{n}}
\newcommand{\Bo}[0]{\mathbf{o}}
\newcommand{\Bp}[0]{\mathbf{p}}
\newcommand{\Bq}[0]{\mathbf{q}}
\newcommand{\Br}[0]{\mathbf{r}}
\newcommand{\Bs}[0]{\mathbf{s}}
\newcommand{\Bt}[0]{\mathbf{t}}
\newcommand{\Bu}[0]{\mathbf{u}}
\newcommand{\Bv}[0]{\mathbf{v}}
\newcommand{\Bw}[0]{\mathbf{w}}
\newcommand{\Bx}[0]{\mathbf{x}}
\newcommand{\By}[0]{\mathbf{y}}
\newcommand{\Bz}[0]{\mathbf{z}}
\newcommand{\BA}[0]{\mathbf{A}}
\newcommand{\BB}[0]{\mathbf{B}}
\newcommand{\BC}[0]{\mathbf{C}}
\newcommand{\BD}[0]{\mathbf{D}}
\newcommand{\BE}[0]{\mathbf{E}}
\newcommand{\BF}[0]{\mathbf{F}}
\newcommand{\BG}[0]{\mathbf{G}}
\newcommand{\BH}[0]{\mathbf{H}}
\newcommand{\BI}[0]{\mathbf{I}}
\newcommand{\BJ}[0]{\mathbf{J}}
\newcommand{\BK}[0]{\mathbf{K}}
\newcommand{\BL}[0]{\mathbf{L}}
\newcommand{\BM}[0]{\mathbf{M}}
\newcommand{\BN}[0]{\mathbf{N}}
\newcommand{\BO}[0]{\mathbf{O}}
\newcommand{\BP}[0]{\mathbf{P}}
\newcommand{\BQ}[0]{\mathbf{Q}}
\newcommand{\BR}[0]{\mathbf{R}}
\newcommand{\BS}[0]{\mathbf{S}}
\newcommand{\BT}[0]{\mathbf{T}}
\newcommand{\BU}[0]{\mathbf{U}}
\newcommand{\BV}[0]{\mathbf{V}}
\newcommand{\BW}[0]{\mathbf{W}}
\newcommand{\BX}[0]{\mathbf{X}}
\newcommand{\BY}[0]{\mathbf{Y}}
\newcommand{\BZ}[0]{\mathbf{Z}}

\newcommand{\Bzero}[0]{\mathbf{0}}
\newcommand{\Btheta}[0]{\boldsymbol{\theta}}
\newcommand{\Btau}[0]{\boldsymbol{\tau}}
\newcommand{\Bomega}[0]{\boldsymbol{\omega}}

%
% shorthand for unit vectors
%
\newcommand{\acap}[0]{\hat{\Ba}}
\newcommand{\bcap}[0]{\hat{\Bb}}
\newcommand{\ccap}[0]{\hat{\Bc}}
\newcommand{\dcap}[0]{\hat{\Bd}}
\newcommand{\ecap}[0]{\hat{\Be}}
\newcommand{\fcap}[0]{\hat{\Bf}}
\newcommand{\gcap}[0]{\hat{\Bg}}
\newcommand{\hcap}[0]{\hat{\Bh}}
\newcommand{\icap}[0]{\hat{\Bi}}
\newcommand{\jcap}[0]{\hat{\Bj}}
\newcommand{\kcap}[0]{\hat{\Bk}}
\newcommand{\lcap}[0]{\hat{\Bl}}
\newcommand{\mcap}[0]{\hat{\Bm}}
\newcommand{\ncap}[0]{\hat{\Bn}}
\newcommand{\ocap}[0]{\hat{\Bo}}
\newcommand{\pcap}[0]{\hat{\Bp}}
\newcommand{\qcap}[0]{\hat{\Bq}}
\newcommand{\rcap}[0]{\hat{\Br}}
\newcommand{\scap}[0]{\hat{\Bs}}
\newcommand{\tcap}[0]{\hat{\Bt}}
\newcommand{\ucap}[0]{\hat{\Bu}}
\newcommand{\vcap}[0]{\hat{\Bv}}
\newcommand{\wcap}[0]{\hat{\Bw}}
\newcommand{\xcap}[0]{\hat{\Bx}}
\newcommand{\ycap}[0]{\hat{\By}}
\newcommand{\zcap}[0]{\hat{\Bz}}
\newcommand{\thetacap}[0]{\hat{\Btheta}}

%
% to write R^n and C^n in a distinguishable fashion.  Perhaps change this
% to the double lined characters upon figuring out how to do so.
%
\newcommand{\C}[1]{$\mathbb{C}^{#1}$}
\newcommand{\R}[1]{$\mathbb{R}^{#1}$}

%
% various generally useful helpers
%

% derivative of #1 wrt. #2:
\newcommand{\D}[2] {\frac {d#2} {d#1}}

\newcommand{\inv}[1]{\frac{1}{#1}}
\newcommand{\cross}[0]{\times}

\newcommand{\abs}[1]{\lvert{#1}\rvert}
\newcommand{\norm}[1]{\lVert{#1}\rVert}
\newcommand{\innerprod}[2]{\langle{#1}, {#2}\rangle}
\newcommand{\dotprod}[2]{{#1} \cdot {#2}}
\newcommand{\bdotprod}[2]{\left({#1} \cdot {#2}\right)}
\newcommand{\crossprod}[2]{{#1} \cross {#2}}
\newcommand{\tripleprod}[3]{\dotprod{\left(\crossprod{#1}{#2}\right)}{#3}}

\DeclareMathOperator{\Proj}{Proj}
\DeclareMathOperator{\Span}{span}
\DeclareMathOperator{\Sgn}{sgn}
\DeclareMathOperator{\Area}{Area}
\DeclareMathOperator{\Volume}{Volume}

%
% A few miscellaneous things specific to this document
%
\newcommand{\crossop}[1]{\crossprod{#1}{}}

% R2 vector.
\newcommand{\VectorTwo}[2]{
\begin{bmatrix}
 {#1} \\
 {#2}
\end{bmatrix}
}

\newcommand{\VectorN}[1]{
\begin{bmatrix}
{#1}_1 \\
{#1}_2 \\
\vdots \\
{#1}_N \\
\end{bmatrix}
}

\newcommand{\DETuvij}[4]{
\begin{vmatrix}
 {#1}_{#3} & {#1}_{#4} \\
 {#2}_{#3} & {#2}_{#4}
\end{vmatrix}
}

\newcommand{\DETuvwijk}[6]{
\begin{vmatrix}
 {#1}_{#4} & {#1}_{#5} & {#1}_{#6} \\
 {#2}_{#4} & {#2}_{#5} & {#2}_{#6} \\
 {#3}_{#4} & {#3}_{#5} & {#3}_{#6}
\end{vmatrix}
}

\newcommand{\DETuvwxijkl}[8]{
\begin{vmatrix}
 {#1}_{#5} & {#1}_{#6} & {#1}_{#7} & {#1}_{#8} \\
 {#2}_{#5} & {#2}_{#6} & {#2}_{#7} & {#2}_{#8} \\
 {#3}_{#5} & {#3}_{#6} & {#3}_{#7} & {#3}_{#8} \\
 {#4}_{#5} & {#4}_{#6} & {#4}_{#7} & {#4}_{#8} \\
\end{vmatrix}
}

%\newcommand{\DETuvwxyijklm}[10]{
%\begin{vmatrix}
% {#1}_{#6} & {#1}_{#7} & {#1}_{#8} & {#1}_{#9} & {#1}_{#10} \\
% {#2}_{#6} & {#2}_{#7} & {#2}_{#8} & {#2}_{#9} & {#2}_{#10} \\
% {#3}_{#6} & {#3}_{#7} & {#3}_{#8} & {#3}_{#9} & {#3}_{#10} \\
% {#4}_{#6} & {#4}_{#7} & {#4}_{#8} & {#4}_{#9} & {#4}_{#10} \\
% {#5}_{#6} & {#5}_{#7} & {#5}_{#8} & {#5}_{#9} & {#5}_{#10}
%\end{vmatrix}
%}

% R3 vector.
\newcommand{\VectorThree}[3]{
\begin{bmatrix}
 {#1} \\
 {#2} \\
 {#3}
\end{bmatrix}
}



\author{Peeter Joot}
\email{peeter.joot@gmail.com}

%\documentclass[]{eliblogwidescreen}

\usepackage{amsmath}
\usepackage{mathpazo}

%
% shorthand for bold symbols, convenient for vectors and matrices
%
\newcommand{\Ba}[0]{\mathbf{a}}
\newcommand{\Bb}[0]{\mathbf{b}}
\newcommand{\Bc}[0]{\mathbf{c}}
\newcommand{\Bd}[0]{\mathbf{d}}
\newcommand{\Be}[0]{\mathbf{e}}
\newcommand{\Bf}[0]{\mathbf{f}}
\newcommand{\Bg}[0]{\mathbf{g}}
\newcommand{\Bh}[0]{\mathbf{h}}
\newcommand{\Bi}[0]{\mathbf{i}}
\newcommand{\Bj}[0]{\mathbf{j}}
\newcommand{\Bk}[0]{\mathbf{k}}
\newcommand{\Bl}[0]{\mathbf{l}}
\newcommand{\Bm}[0]{\mathbf{m}}
\newcommand{\Bn}[0]{\mathbf{n}}
\newcommand{\Bo}[0]{\mathbf{o}}
\newcommand{\Bp}[0]{\mathbf{p}}
\newcommand{\Bq}[0]{\mathbf{q}}
\newcommand{\Br}[0]{\mathbf{r}}
\newcommand{\Bs}[0]{\mathbf{s}}
\newcommand{\Bt}[0]{\mathbf{t}}
\newcommand{\Bu}[0]{\mathbf{u}}
\newcommand{\Bv}[0]{\mathbf{v}}
\newcommand{\Bw}[0]{\mathbf{w}}
\newcommand{\Bx}[0]{\mathbf{x}}
\newcommand{\By}[0]{\mathbf{y}}
\newcommand{\Bz}[0]{\mathbf{z}}
\newcommand{\BA}[0]{\mathbf{A}}
\newcommand{\BB}[0]{\mathbf{B}}
\newcommand{\BC}[0]{\mathbf{C}}
\newcommand{\BD}[0]{\mathbf{D}}
\newcommand{\BE}[0]{\mathbf{E}}
\newcommand{\BF}[0]{\mathbf{F}}
\newcommand{\BG}[0]{\mathbf{G}}
\newcommand{\BH}[0]{\mathbf{H}}
\newcommand{\BI}[0]{\mathbf{I}}
\newcommand{\BJ}[0]{\mathbf{J}}
\newcommand{\BK}[0]{\mathbf{K}}
\newcommand{\BL}[0]{\mathbf{L}}
\newcommand{\BM}[0]{\mathbf{M}}
\newcommand{\BN}[0]{\mathbf{N}}
\newcommand{\BO}[0]{\mathbf{O}}
\newcommand{\BP}[0]{\mathbf{P}}
\newcommand{\BQ}[0]{\mathbf{Q}}
\newcommand{\BR}[0]{\mathbf{R}}
\newcommand{\BS}[0]{\mathbf{S}}
\newcommand{\BT}[0]{\mathbf{T}}
\newcommand{\BU}[0]{\mathbf{U}}
\newcommand{\BV}[0]{\mathbf{V}}
\newcommand{\BW}[0]{\mathbf{W}}
\newcommand{\BX}[0]{\mathbf{X}}
\newcommand{\BY}[0]{\mathbf{Y}}
\newcommand{\BZ}[0]{\mathbf{Z}}

\newcommand{\Bzero}[0]{\mathbf{0}}
\newcommand{\Btheta}[0]{\boldsymbol{\theta}}
\newcommand{\Btau}[0]{\boldsymbol{\tau}}
\newcommand{\Bomega}[0]{\boldsymbol{\omega}}

%
% shorthand for unit vectors
%
\newcommand{\acap}[0]{\hat{\Ba}}
\newcommand{\bcap}[0]{\hat{\Bb}}
\newcommand{\ccap}[0]{\hat{\Bc}}
\newcommand{\dcap}[0]{\hat{\Bd}}
\newcommand{\ecap}[0]{\hat{\Be}}
\newcommand{\fcap}[0]{\hat{\Bf}}
\newcommand{\gcap}[0]{\hat{\Bg}}
\newcommand{\hcap}[0]{\hat{\Bh}}
\newcommand{\icap}[0]{\hat{\Bi}}
\newcommand{\jcap}[0]{\hat{\Bj}}
\newcommand{\kcap}[0]{\hat{\Bk}}
\newcommand{\lcap}[0]{\hat{\Bl}}
\newcommand{\mcap}[0]{\hat{\Bm}}
\newcommand{\ncap}[0]{\hat{\Bn}}
\newcommand{\ocap}[0]{\hat{\Bo}}
\newcommand{\pcap}[0]{\hat{\Bp}}
\newcommand{\qcap}[0]{\hat{\Bq}}
\newcommand{\rcap}[0]{\hat{\Br}}
\newcommand{\scap}[0]{\hat{\Bs}}
\newcommand{\tcap}[0]{\hat{\Bt}}
\newcommand{\ucap}[0]{\hat{\Bu}}
\newcommand{\vcap}[0]{\hat{\Bv}}
\newcommand{\wcap}[0]{\hat{\Bw}}
\newcommand{\xcap}[0]{\hat{\Bx}}
\newcommand{\ycap}[0]{\hat{\By}}
\newcommand{\zcap}[0]{\hat{\Bz}}
\newcommand{\thetacap}[0]{\hat{\Btheta}}

%
% to write R^n and C^n in a distinguishable fashion.  Perhaps change this
% to the double lined characters upon figuring out how to do so.
%
\newcommand{\C}[1]{$\mathbb{C}^{#1}$}
\newcommand{\R}[1]{$\mathbb{R}^{#1}$}

%
% various generally useful helpers
%

% derivative of #1 wrt. #2:
\newcommand{\D}[2] {\frac {d#2} {d#1}}

\newcommand{\inv}[1]{\frac{1}{#1}}
\newcommand{\cross}[0]{\times}

\newcommand{\abs}[1]{\lvert{#1}\rvert}
\newcommand{\norm}[1]{\lVert{#1}\rVert}
\newcommand{\innerprod}[2]{\langle{#1}, {#2}\rangle}
\newcommand{\dotprod}[2]{{#1} \cdot {#2}}
\newcommand{\bdotprod}[2]{\left({#1} \cdot {#2}\right)}
\newcommand{\crossprod}[2]{{#1} \cross {#2}}
\newcommand{\tripleprod}[3]{\dotprod{\left(\crossprod{#1}{#2}\right)}{#3}}

\DeclareMathOperator{\Proj}{Proj}
\DeclareMathOperator{\Span}{span}
\DeclareMathOperator{\Sgn}{sgn}
\DeclareMathOperator{\Area}{Area}
\DeclareMathOperator{\Volume}{Volume}

%
% A few miscellaneous things specific to this document
%
\newcommand{\crossop}[1]{\crossprod{#1}{}}

% R2 vector.
\newcommand{\VectorTwo}[2]{
\begin{bmatrix}
 {#1} \\
 {#2}
\end{bmatrix}
}

\newcommand{\VectorN}[1]{
\begin{bmatrix}
{#1}_1 \\
{#1}_2 \\
\vdots \\
{#1}_N \\
\end{bmatrix}
}

\newcommand{\DETuvij}[4]{
\begin{vmatrix}
 {#1}_{#3} & {#1}_{#4} \\
 {#2}_{#3} & {#2}_{#4}
\end{vmatrix}
}

\newcommand{\DETuvwijk}[6]{
\begin{vmatrix}
 {#1}_{#4} & {#1}_{#5} & {#1}_{#6} \\
 {#2}_{#4} & {#2}_{#5} & {#2}_{#6} \\
 {#3}_{#4} & {#3}_{#5} & {#3}_{#6}
\end{vmatrix}
}

\newcommand{\DETuvwxijkl}[8]{
\begin{vmatrix}
 {#1}_{#5} & {#1}_{#6} & {#1}_{#7} & {#1}_{#8} \\
 {#2}_{#5} & {#2}_{#6} & {#2}_{#7} & {#2}_{#8} \\
 {#3}_{#5} & {#3}_{#6} & {#3}_{#7} & {#3}_{#8} \\
 {#4}_{#5} & {#4}_{#6} & {#4}_{#7} & {#4}_{#8} \\
\end{vmatrix}
}

%\newcommand{\DETuvwxyijklm}[10]{
%\begin{vmatrix}
% {#1}_{#6} & {#1}_{#7} & {#1}_{#8} & {#1}_{#9} & {#1}_{#10} \\
% {#2}_{#6} & {#2}_{#7} & {#2}_{#8} & {#2}_{#9} & {#2}_{#10} \\
% {#3}_{#6} & {#3}_{#7} & {#3}_{#8} & {#3}_{#9} & {#3}_{#10} \\
% {#4}_{#6} & {#4}_{#7} & {#4}_{#8} & {#4}_{#9} & {#4}_{#10} \\
% {#5}_{#6} & {#5}_{#7} & {#5}_{#8} & {#5}_{#9} & {#5}_{#10}
%\end{vmatrix}
%}

% R3 vector.
\newcommand{\VectorThree}[3]{
\begin{bmatrix}
 {#1} \\
 {#2} \\
 {#3}
\end{bmatrix}
}



\author{Peeter Joot}
\email{peeter.joot@gmail.com}


%\usepackage{media9}
\chapter{Continuum mechanics elasticity review.}

\label{chap:continuumElasticityReview}
%\useCCL
\blogpage{http://sites.google.com/site/peeterjoot2/math2012/continuumElasticityReview.pdf}
\date{Apr 21, 2012}
\gitRevisionInfo{continuumElasticityReview}
\keywords{PHY454H1S, PHY454H1, strain, displacement vector, stress, constitutive relation, Lame parameters, shear modulus, Cauchy tetrahedron, Uniform hydrostatic compression, Bulk modulus, Uniaxial stress, Young's modulus, Poisson's ratio, Compatibility condition,P-waves, S-waves, wave equation, displacement potentials, phasor, Love wave, Rayleigh wave}

\beginArtWithToc
%\beginArtNoToc
%\wordpresscategory{}

\section{Motivation.}

Review of key ideas and equations from the theory of elasticity portion of the class.

\section{Strain Tensor}

Identifying a point in a solid with coordinates $x_i$ and the coordinates of that portion of the solid after displacement, we formed the difference as a measure of the displacement

\begin{equation}\label{eqn:continuumElasticityReview:10}
u_i = x_i' - x_i.
\end{equation}

With $du_i = \PDi{x_j}{u_i} dx_j$, we computed the difference in length (squared) for an element of the displaced solid and found 

\begin{equation}\label{eqn:continuumElasticityReview:30}
dx_k' dx_k' - dx_k dx_k = 
\left( 
\PD{x_i}{u_j} + 
\PD{x_j}{u_i} + 
\PD{x_i}{u_k} 
\PD{x_j}{u_k} 
\right) 
dx_i dx_j,
\end{equation}

or defining the \textit{strain tensor} $e_{ij}$, we have

\begin{subequations}
\begin{equation}\label{eqn:continuumElasticityReview:50}
(d\Bx')^2 - (d\Bx)^2
= 2 e_{ij} dx_i dx_j
\end{equation}
\begin{equation}\label{eqn:continuumElasticityReview:70}
e_{ij}
=
\inv{2}
\left( 
\PD{x_i}{u_j} + 
\PD{x_j}{u_i} + 
\PD{x_i}{u_k} 
\PD{x_j}{u_k} 
\right).
\end{equation}
\end{subequations}

In this course we use only the linear terms and write

\begin{equation}\label{eqn:continuumElasticityReview:90}
e_{ij}
=
\inv{2}
\left( 
\PD{x_i}{u_j} + 
\PD{x_j}{u_i} 
\right).
\end{equation}

\subsection{Unresolved: Relating displacement and position by strain}

In \cite{feynman1963flp:elasticMaterials} it is pointed out that this strain tensor simply relates the displacement vector coordinates $u_i$ to the coordinates at the point at which it is measured

\begin{equation}\label{eqn:continuumElasticityReview:110}
u_i = e_{ij} x_j.
\end{equation}

When we get to fluid dynamics we perform a linear expansion of $du_i$ and find something similar

\begin{equation}\label{eqn:continuumElasticityReview:530}
dx_i' - dx_i = du_i = \PD{x_k}{u_i} dx_k = e_{ij} dx_k + \omega_{ij} dx_k
\end{equation}

where

\begin{equation}\label{eqn:continuumElasticityReview:550}
\omega_{ij} = \inv{2} \left( \PD{x_i}{u_j} +\PD{x_j}{u_i} \right).
\end{equation}

Except for the antisymmetric term, note the structural similarity of \ref{eqn:continuumElasticityReview:110} and \ref{eqn:continuumElasticityReview:530}.  Why is it that we neglect the vorticity tensor in statics?
%  If we are approximating the displacement, it appears to have a natural place in things, as we can see
%
%\begin{align*}
%u_i 
%&\approx \PD{x_j}{u_i} x_j \\
%&=
%e_{ij} x_j + \omega_{ik} x_k.
%\end{align*}
%
%This is easily seen to be the case, recovering \ref{eqn:continuumElasticityReview:90} by taking derivatives of \ref{eqn:continuumElasticityReview:110}, plus an assumption that $e_{ij}$ is symmetric.

\subsection{Diagonal strain representation.}

In a basis for which the strain tensor is diagonal, it was pointed out that we can write our difference in squared displacement as (for $k = 1, 2, 3$, no summation convention)

\begin{equation}\label{eqn:continuumElasticityReview:130}
(dx_k')^2 - (dx_k)^2 = 2 e_{kk} dx_k dx_k
\end{equation}

from which we can rearrange, take roots, and apply a first order Taylor expansion to find (again no summation convention)

\begin{equation}\label{eqn:continuumElasticityReview:150}
dx_k' \approx (1 + e_{kk}) dx_k.
\end{equation}

An approximation of the displaced volume was then found in terms of the strain tensor trace (summation convention back again)

\begin{equation}\label{eqn:continuumElasticityReview:170}
dV' \approx (1 + e_{kk}) dV,
\end{equation}

allowing us to identify this trace as a relative difference in displaced volume

\begin{equation}\label{eqn:continuumElasticityReview:190}
e_{kk} \approx \frac{dV' - dV}{dV}.
\end{equation}

\subsection{Strain in cylindrical coordinates.}

Useful in many practice problems are the cylindrical coordinate representation of the strain tensor 

\begin{align}\label{eqn:continuumElasticityReview:210}
2 e_{rr} &= \PD{r}{u_r}  \\
2 e_{\phi\phi} &= \inv{r} \PD{\phi}{u_\phi} +\inv{r} u_r  \\
2 e_{zz} &= \PD{z}{u_z}  \\
2 e_{zr} &= \PD{z}{u_r} + \PD{r}{u_z} \\
2 e_{r\phi} &= \PD{r}{u_\phi} - \inv{r} u_\phi + \inv{r} \PD{\phi}{u_r} \\
2 e_{\phi z} &= \PD{z}{u_\phi} +\inv{r} \PD{\phi}{u_z}.
\end{align}

This can be found in \cite{landau1960theory}.  It was not derived there or in class, but is not too hard, even using the second order methods we used for the Cartesian form of the tensor.

An easier way to do this derivation (and understand what the coordinates represent) follows from the relation found in \S 6 of \cite{acheson1990elementary}

\begin{equation}\label{eqn:continuumElasticityReview:230}
2 \Be_i e_{ij} n_j = 2 (\ncap \cdot \spacegrad) \Bu + \ncap \cross (\spacegrad \cross \Bu),
\end{equation}

where $\ncap$ is the normal to the surface at which we are measuring a force applied to the solid (our Cauchy tetrahedron).

The cylindrical tensor coordinates of \ref{eqn:continuumElasticityReview:210} follow from 
\ref{eqn:continuumElasticityReview:230} nicely taking $\ncap = \rcap, \phicap, \zcap$ in turn.

\subsection{Compatibility condition.}

For a 2D strain tensor we found an interrelationship between the components of the strain tensor

\begin{equation}\label{eqn:continuumElasticityReview:510}
2 \frac{\partial^2 e_{12}}{\partial x_1 \partial x_2} 
=
\PDSq{x_1}{e_{22}} 
+\PDSq{x_2}{e_{11}},
\end{equation}

and called this the compatibility condition.  It was claimed, but not demonstrated that this is what is required to ensure a deformation maintained a coherent solid geometry.

I wasn't able to find any references to this compatibility condition in any of the texts I have, but found \cite{wiki:compatibilityMechanics}, \cite{wiki:infinitesimalStrainTheory}, and \cite{wiki:saintVenantCompat}.  It's not terribly surprising to see Christoffel symbol and differential forms references on those pages, since one can imagine that we'd wish to look at the mappings of all the points in the object as it undergoes the transformation from the original to the deformed state.

Even with just three points in a plane, say $\Ba$, $\Bb$, $\Bc$, the general deformation of an object doesn't seem like it's the easiest thing to describe.  We can imagine that these have trajectories in the deformation process $\Ba = \Ba(\alpha$, $\Bb = \Bb(\beta)$, $\Bc = \Bc(\gamma)$, with $\Ba', \Bb', \Bc'$ at the end points of the trajectories.  We'd want to look at displacement vectors $\Bu_a, \Bu_b, \Bu_c$ along each of these trajectories, and then see how they must be related.  Doing that carefully must result in this compatibility condition.

\section{Stress tensor.}

By sought and found a representation of the force per unit area acting on a body by expressing the components of that force as a set of divergence relations

\begin{equation}\label{eqn:continuumElasticityReview:250}
f_i = \partial_k \sigma_{i k},
\end{equation}

and call the associated tensor $\sigma_{ij}$ the \textit{stress}.

Unlike the strain, we don't have any expectation that this tensor is symmetric, and identify the diagonal components (no sum) $\sigma_{i i}$ as quantifying the amount of compressive or contractive force per unit area, whereas the cross terms of the stress tensor introduce shearing deformations in the solid.

With force balance arguments (the Cauchy tetrahedron) we found that the force per unit area on the solid, for a surface with unit normal pointing into the solid, was

\begin{equation}\label{eqn:continuumElasticityReview:270}
\Bt = \Be_i t_i = \Be_i \sigma_{ij} n_j.
\end{equation}

\subsection{Constitutive relation.}

In the scope of this course we considered only Newtonian materials, those for which the stress and strain tensors are linearly related

\begin{equation}\label{eqn:continuumElasticityReview:290}
\sigma_{ij} = c_{ijkl} e_{kl},
\end{equation}

and further restricted our attention to isotropic materials, which can be shown to have the form

\begin{equation}\label{eqn:continuumElasticityReview:310}
\sigma_{ij} = \lambda e_{kk} \delta_{ij} + 2 \mu e_{ij},
\end{equation}

where $\lambda$ and $\mu$ are the Lame parameters and $\mu$ is called the shear modulus (and viscosity in the context of fluids).

By computing the trace of the stress $\sigma_{ii}$ we can invert this to find

\begin{equation}\label{eqn:continuumElasticityReview:330}
2 \mu e_{ij} = \sigma_{ij} - \frac{\lambda}{3 \lambda + 2 \mu} \sigma_{kk} \delta_{ij}.
\end{equation}

\subsection{Uniform hydrostatic compression.}

With only normal components of the stress (no shear), and the stress having the same value in all directions, we find

\begin{equation}\label{eqn:continuumElasticityReview:350}
\sigma_{ij} = ( 3 \lambda + 2 \mu ) e_{ij},
\end{equation}

and identify this combination $-3 \lambda - 2 \mu$ as the pressure, linearly relating the stress and strain tensors

\begin{equation}\label{eqn:continuumElasticityReview:370}
\sigma_{ij} = -p e_{ij}.
\end{equation}

With $e_{ii} = (dV' - dV)/dV = \Delta V/V$, we formed the Bulk modulus $K$ with the value

\begin{equation}\label{eqn:continuumElasticityReview:390}
K = \left( \lambda + \frac{2 \mu}{3} \right) = -\frac{p V}{\Delta V}.
\end{equation}

\subsection{Uniaxial stress.  Young's modulus.  Poisson's ratio.}

For the special case with only one non-zero stress component (we used $\sigma_{11}$) we were able to compute Young's modulus $E$, the ratio between stress and strain in that direction

\begin{equation}\label{eqn:continuumElasticityReview:410}
E = \frac{\sigma_{11}}{e_{11}} = \frac{\mu(3 \lambda + 2 \mu)}{\lambda + \mu }  = \frac{3 K \mu}{K + \mu/3}.
\end{equation}

Just because only one component of the stress is non-zero, does not mean that we have no deformation in any other directions.  Introducing Poisson's ratio $\nu$ in terms of the ratio of the strains relative to the strain in the direction of the force we write and then subsequently found

\begin{equation}\label{eqn:continuumElasticityReview:430}
\nu = -\frac{e_{22}}{e_{11}} = -\frac{e_{33}}{e_{11}} = \frac{\lambda}{2(\lambda + \mu)}.
\end{equation}

We were also able to find

We can also relate the Poisson's ratio $\nu$ to the shear modulus $\mu$

\begin{equation}\label{eqn:continuumElasticityReview:450}
\mu = \frac{E}{2(1 + \nu)}
\end{equation}

\begin{equation}\label{eqn:continuumElasticityReview:470}
\lambda = \frac{E \nu}{(1 - 2 \nu)(1 + \nu)}
\end{equation}

\begin{align}\label{eqn:continuumElasticityReview:490}
e_{11} &= \inv{E}\left( \sigma_{11} - \nu(\sigma_{22} + \sigma_{33}) \right) \\
e_{22} &= \inv{E}\left( \sigma_{22} - \nu(\sigma_{11} + \sigma_{33}) \right) \\
e_{33} &= \inv{E}\left( \sigma_{33} - \nu(\sigma_{11} + \sigma_{22}) \right)
\end{align}

\section{Displacement propagation}

It was argued that the equation relating the time evolution of a one of the vector displacement coordinates was given by

\begin{equation}\label{eqn:continuumElasticityReview:570}
\rho \PDSq{t}{u_i} = \PD{x_j}{\sigma_{ij}} + f_i,
\end{equation}

where the divergence term $\PDi{x_j}{\sigma_{ij}}$ is the internal force per unit volume on the object and $f_i$ is the external force.  Employing the constitutive relation we showed that this can be expanded as

\begin{equation}\label{eqn:continuumElasticityReview:590}
\rho \PDSq{t}{u_i} = (\lambda + \mu) \frac{\partial^2 u_k}{\partial x_i \partial x_k}
+ \mu
\frac{\partial^2 u_i}
{\partial x_j^2
},
\end{equation}

or in vector form

\begin{equation}\label{eqn:continuumElasticityReview:610}
\rho \PDSq{t}{\Bu} = (\lambda + \mu) \spacegrad (\spacegrad \cdot \Bu) + \mu \spacegrad^2 \Bu.
\end{equation}

\subsection{P-waves}

Operating on \ref{eqn:continuumElasticityReview:610} with the divergence operator, and writing $\Theta = \spacegrad \cdot \Bu$, a quantity that was our relative change in volume in the diagonal strain basis, we were able to find this divergence obeys a wave equation

\begin{equation}\label{eqn:continuumElasticityReview:630}
\PDSq{t}{\Theta} = \frac{\lambda + 2 \mu}{\rho} \spacegrad^2 \Theta.
\end{equation}

We called these P-waves.

\subsection{S-waves}

Similarly, operating on \ref{eqn:continuumElasticityReview:610} with the curl operator, and writing $\Bomega = \spacegrad \cross \Bu$, we were able to find this curl also obeys a wave equation

\begin{equation}\label{eqn:continuumElasticityReview:650}
\rho \PDSq{t}{\Bomega} = \mu \spacegrad^2 \Bomega.
\end{equation}

These we called S-waves.  We also noted that the (transverse) compression waves (P-waves) with speed $C_T = \sqrt{\mu/\rho}$, traveled faster than the (longitudinal) vorticity (S) waves with speed $C_L = \sqrt{(\lambda + 2 \mu)/\rho}$ since $\lambda > 0$ and $\mu > 0$, and 

\begin{equation}\label{eqn:continuumElasticityReview:670}
\frac{C_L}{C_T} = \sqrt{\frac{ \lambda + 2 \mu}{\mu}} = \sqrt{ \frac{\lambda}{\mu} + 2}.
\end{equation}

\subsection{Scalar and vector potential representation.}

Assuming a vector displacement representation with gradient and curl components

\begin{equation}\label{eqn:continuumElasticityReview:690}
\Bu = \spacegrad \phi + \spacegrad \cross \BH,
\end{equation}

We found that the displacement time evolution equation split nicely into curl free and divergence free terms

\begin{equation}\label{eqn:continuumElasticityReview:710}
\spacegrad
\left(
\rho \PDSq{t}{\phi} - (\lambda + 2\mu) \spacegrad^2 \phi
\right)
+
\spacegrad \cross
\left(
\rho \PDSq{t}{\BH} - \mu \spacegrad^2 \BH
\right)
= 0.
\end{equation}

When neglecting boundary value effects this could be written as a pair of independent equations

\begin{subequations}
\begin{equation}\label{eqn:continuumElasticityReview:730}
\rho \PDSq{t}{\phi} - (\lambda + 2\mu) \spacegrad^2 \phi = 0
\end{equation}
\begin{equation}\label{eqn:continuumElasticityReview:750}
\rho \PDSq{t}{\BH} - \mu \spacegrad^2 \BH
= 0.
\end{equation}
\end{subequations}

This are the irrotational (curl free) P-wave and solenoidal (divergence free) S-wave equations respectively.

%This theory led to no actual calculation work, just a few videos that illustrated what we'd presumably be able to calculate if we were to attempt to apply these concepts.

\subsection{Phasor description.}

It was mentioned that we could assume a phasor representation for our potentials, writing

\begin{subequations}
\begin{equation}\label{eqn:continuumElasticityReview:770}
\phi = A \exp\left( i ( \Bk \cdot \Bx - \omega t) \right) 
\end{equation}
\begin{equation}\label{eqn:continuumElasticityReview:790}
\BH = \BB \exp\left( i ( \Bk \cdot \Bx - \omega t) \right)
\end{equation}
\end{subequations}

finding

\begin{equation}\label{eqn:continuumElasticityReview:810}
\Bu = i \Bk \phi + i \Bk \cross \BH.
\end{equation}

We did nothing with neither the potential nor the phasor theory for solid displacement time evolution, and presumably won't on the exam either.

\section{Some wave types}

Some time was spent on non-qualitative descriptions and review of descriptions for solutions to the time evolution equations we did not attempt

\begin{itemize}
\item P-waves \cite{wiki:pwave}.  Irrotational, non volume preserving body wave.
\item S-waves \cite{wiki:swave}.  Divergence free body wave.  Shearing forces are present and volume is preserved (slower than S-waves)
\item Rayleigh wave \cite{wiki:rayleighwave}.  A surface wave that propagates near the surface of a body without penetrating into it.
\item Love wave \cite{wiki:lovewave}.  A polarized shear surface wave with the shear displacements moving perpendicular to the direction of propagation.
\end{itemize}

For reasons that aren't clear both the midterm and last years final ask us to spew this sort of stuff (instead of actually trying to do something analytic associated with them).

\EndArticle

   % (keep this temporariliy for the keywords, and use those eventually for building an index)
%
%   \chapter{Strain Tensor}
      %
% Copyright � 2012 Peeter Joot.  All Rights Reserved.
% Licenced as described in the file LICENSE under the root directory of this GIT repository.
%

%
%
%\section{Introduction and strain tensor}
%\chapter{PHY454H1S\\Continuum Mechanics.  Lecture 2.  Introduction and strain tensor.  Taught by Prof. K. Das}
%\label{chap:continuumL2}
\section{Deformations}

We have defined strain \ref{dfn:continuumL2:30} as the measure of deformation of a body.  This is a purely geometric definition, and by itself has no requirement to understand the forces that put the object into the deformed configuration.  A mathematical statement of this definition needs to be made.

A solid deformation of an object with vertexes located at \(\Ba\), \(\Bb\), and \(\Bc\) is illustrated in \cref{fig:strain:strainFig1}, where the deformed vertexes are located at \(\Ba'\), \(\Bb'\), and \(\Bc'\).

\imageFigure{../../figures/phy454/strainDeformation_of_a_planar_objectFig1}{Deformation of a planar object}{fig:strain:strainFig1}{0.5}

Identifying a specific point in the object with an undeformed position \(\Bx\), we can consider the deformation of the object in the vicinity of this point.  If this point has deformed position \(\Bx'\), we define the \textit{displacement vector}, the vectoral difference between the displaced and original point in the object, as

\begin{equation}\label{eqn:strainTensor:210}
\Bu = \Bx' - \Bx,
\end{equation}

or in coordinates

\begin{equation}\label{eqn:revTextcontinuumL2:10}
u_i = x_i' - x_i.
\end{equation}

In general each of the displaced coordinate locations, and therefore also the displacement vector coordinates, is some function of position

\begin{equation}\label{eqn:strainTensor:230}
\Bx' = \Bf(\Bx),
\end{equation}

or in coordinates

\begin{equation}\label{eqn:strainTensor:250}
x_i' = f_i(\Bx).
\end{equation}

Now we will consider how a vector difference between two infinitesimally close points in the object change under deformation.  Imagine that we are looking at points along some parameterized trajectory within the object as illustrated in \cref{fig:strain:strainFig2}.

\imageFigure{../../figures/phy454/strainTransformation_under_deformation_of_an_infinitesimal_line_element_along_a_trajectoryFig2}{Transformation under deformation of an infinitesimal line element along a trajectory}{fig:strain:strainFig2}{0.4}

In the original object, we can locate a point \(\By = \Bx + d\Bx\) a little bit further along the parameterized path.  In the deformed object we find this point at location \(\By' = \Bx' + d\Bx'\).  We wish to consider how this line element differs in the original and deformed configurations, indirectly calculating the magnitude of the difference

\begin{equation}\label{eqn:strainTensor:270}
d\Bu = d\Bx' - d\Bx.
\end{equation}

There are two ways we can perform this calculation.  The first, following \citep{landau1960theory} \S 1, is to take a difference of the lengths of the displacement vector in the deformed and the original object.  The second, an approach we will use later in our treatment of fluids is to consider a linear expansion of the change in displacement between the deformed and original objects.

%Utilizing summation convention consider a set of small internal displacements \(u_1, u_2, u_3\) to the \(x, y, z\) coordinates so that the transformation \(x_i \rightarrow x_i'\) is related by
%
%\cref{fig:continuumL2:continuumL2fig4}
%\imageFigure{../../figures/phy454/lec2_Differential_change_to_the_objectFig4}{Differential change to the object}{fig:continuumL2:continuumL2fig4}{0.3}

Rearranging for the displacement line element in the deformed object, and working in coordinates we write

\begin{equation}\label{eqn:continuumL2:70}
dx_i' = dx_i + du_i
\end{equation}

Employing summation convention with implied summation over repeated indices the lengths of the pairs of line elements are

\begin{equation}\label{eqn:continuumL2:90}
\begin{aligned}
dl &= \Abs{d\Bx} = \sqrt{dx_k dx_k} \\
dl' &= \Abs{d\Bx'} = \sqrt{d{x'}_k d{x'}_k},
\end{aligned}
\end{equation}

or

\begin{equation}\label{eqn:continuumL2:110}
{dl'}^2 =
(dx_k + du_k)
(dx_k + du_k)
=
dl^2 + 2 dx_k du_k + du_k du_k.
\end{equation}

Taylor expanding

\begin{equation}\label{eqn:continuumL2:130}
du_i = \PD{x_k}{u_i} dx_k,
\end{equation}

so that

\begin{equation}\label{eqn:continuumL2:150}
du_i^2 =
\PD{x_k}{u_i} dx_k
\PD{x_l}{u_i} dx_l
\end{equation}

\begin{equation}\label{eqn:lec1StrainTensor:290}
\begin{aligned}
{dl'}^2
&=
dl^2
+ 2 \PD{x_k}{u_i} dx_k dx_i
+ \PD{x_i}{u_l}
\PD{x_k}{u_l}
dx_i dx_k \\
&=
dl^2
+
\left(
\PD{x_k}{u_i}
+
\PD{x_i}{u_k}
\right)
dx_k dx_i
+ \PD{x_i}{u_l}
\PD{x_k}{u_l}
dx_i dx_k \\
&=
dl^2
+
2 e_{ik} dx_i dx_k
\end{aligned}
\end{equation}

We write

\begin{equation}\label{eqn:continuumL2:170}
{dl'}^2 - dl^2 = 2 e_{ik} dx_i dx_k
\end{equation}

where we define the \emph{strain tensor} as

\boxedEquation{eqn:continuumL2:190}{
e_{ik} = \inv{2} \left(
\left(
\PD{x_k}{u_i}
+
\PD{x_i}{u_k}
\right)
+ \PD{x_i}{u_l}
\PD{x_k}{u_l}
\right).
}

In this course we will make use of only the linear terms, essentially defining the strain tensor as

\boxedEquation{eqn:revTextcontinuumL2:90}{
e_{ij}
=
\inv{2}
\left(
\PD{x_i}{u_j} +
\PD{x_j}{u_i}
\right).
}


      % 
% 
% 
% Copyright © 2012 Peeter Joot
% All Rights Reserved
% 
% This file may be reproduced and distributed in whole or in part, without fee, subject to the following conditions:
% 
% o The copyright notice above and this permission notice must be preserved complete on all complete or partial copies.
% 
% o Any translation or derived work must be approved by the author in writing before distribution.
% 
% o If you distribute this work in part, instructions for obtaining the complete version of this file must be included, and a means for obtaining a complete version provided.
% 
% 
% Exceptions to these rules may be granted for academic purposes: Write to the author and ask.
% 
% 
% 


\section{Matrix representation, diagonalization, and deformed volume element}

The strain tensor $e_{ik}$ can be worked with in coordinates, but we will often us a matrix representation when working in Cartesian coordinates 

\begin{equation}\label{eqn:continuumL2:210}
\Be = 
\begin{bmatrix}
e_{11} & e_{12} & e_{13} \\
e_{21} & e_{22} & e_{23} \\
e_{31} & e_{32} & e_{33} \\
\end{bmatrix}
\end{equation}

We see from \ref{eqn:continuumL2:190} that $e_{ik}$ is symmetric, so we have

\begin{align}\label{eqn:continuumL2:230}
e_{21} &= e_{12} \\
e_{31} &= e_{13} \\
e_{32} &= e_{23}
\end{align}

Given this real symmetric matrix there must exist an orthonormal basis at each point that allows the strain tensor to be written in diagonal form 

\begin{equation}\label{eqn:continuumL2:250}
\bar{e}_{ik} =
\begin{bmatrix}
\bar{e}_{11} & 0 & 0 \\
0 & \bar{e}_{22} & 0 \\
0 & 0 & \bar{e}_{33} \\
\end{bmatrix}.
\end{equation}

In that basis the difference between two close points in the deformed object, in terms of the difference between the original positions of those points in the original object, can be expressed as

\begin{align}\label{eqn:continuumL2:270}
{dx_1'}^2 &= (1 + 2 \bar{e}_{11}) dx_1^2 \\
{dx_2'}^2 &= (1 + 2 \bar{e}_{22}) dx_2^2 \\
{dx_3'}^2 &= (1 + 2 \bar{e}_{33}) dx_3^2,
\end{align}

or

\begin{align}\label{eqn:continuumL2:280}
dx_1' &= \sqrt{1 + 2 \bar{e}_{11}} dx_1 \\
dx_2' &= \sqrt{1 + 2 \bar{e}_{22}} dx_2 \\
dx_3' &= \sqrt{1 + 2 \bar{e}_{33}} dx_3.
\end{align}

If these points are close enough, we can employ a first order Taylor expansion of the square root, yielding

\begin{align}\label{eqn:continuumL2:290}
dx_1' &\approx (1 + \bar{e}_{11}) dx_1 \\
dx_2' &\approx (1 + \bar{e}_{22}) dx_2 \\
dx_3' &\approx (1 + \bar{e}_{33}) dx_3
\end{align}

Our deformed volume element in the neighborhood of the point of interest can then be seen to be

\begin{equation}\label{eqn:continuumL2:310}
dV' =
dx_1'
dx_2'
dx_3'
\approx
(1 + e_{11})
(1 + e_{22})
(1 + e_{33})
dx_1 dx_2 dx_3
\end{equation}

\begin{equation}\label{eqn:continuumL2:330}
dV' \approx (1 + e_{11} +e_{22} +e_{33} ) dV.
\end{equation}

Reverting again to summation convention, this is

\begin{equation}\label{eqn:continuumL2:350}
dV' \approx ( 1 + e_{ii} ) dV.
\end{equation}

This allows us to give a physical interpretation to the trace of the strain tensor, so that in a small enough neighborhood we have

\begin{equation}\label{eqn:revTextcontinuumL2:190}
e_{kk} = \frac{dV' - dV}{dV}.
\end{equation}

The trace of the strain tensor quantifies the relative difference between the deformed volume element and the original volume element.

      % 
% 
% 
% Copyright © 2012 Peeter Joot
% All Rights Reserved
% 
% This file may be reproduced and distributed in whole or in part, without fee, subject to the following conditions:
% 
% o The copyright notice above and this permission notice must be preserved complete on all complete or partial copies.
% 
% o Any translation or derived work must be approved by the author in writing before distribution.
% 
% o If you distribute this work in part, instructions for obtaining the complete version of this file must be included, and a means for obtaining a complete version provided.
% 
% 
% Exceptions to these rules may be granted for academic purposes: Write to the author and ask.
% 
% 
% 

\section{Strain in cylindrical coordinates}

Useful in many practice problems are the cylindrical coordinate representation of the strain tensor 

\begin{align}\label{eqn:continuumElasticityReview:210}
2 e_{rr} &= \PD{r}{u_r}  \\
2 e_{\phi\phi} &= \inv{r} \PD{\phi}{u_\phi} +\inv{r} u_r  \\
2 e_{zz} &= \PD{z}{u_z}  \\
2 e_{zr} &= \PD{z}{u_r} + \PD{r}{u_z} \\
2 e_{r\phi} &= \PD{r}{u_\phi} - \inv{r} u_\phi + \inv{r} \PD{\phi}{u_r} \\
2 e_{\phi z} &= \PD{z}{u_\phi} +\inv{r} \PD{\phi}{u_z}.
\end{align}

This result can be found in \citep{landau1960theory}, and is derived in appendix \ref{chap:appendix:strainCoordinates} using the second order methods found above for the Cartesian tensor.

An easier way to do this derivation (and understand what the coordinates represent) follows from the relation found in \S 6 of \citep{acheson1990elementary}

\begin{equation}\label{eqn:continuumElasticityReview:230}
2 \Be_i e_{ij} \Be_j \cdot \ncap = 2 (\ncap \cdot \spacegrad) \Bu + \ncap \cross (\spacegrad \cross \Bu),
\end{equation}

where $\ncap$ is the normal to the surface at which we are measuring a force applied to the solid (our Cauchy tetrahedron).

The cylindrical tensor coordinates of \ref{eqn:continuumElasticityReview:210} follow from 
\ref{eqn:continuumElasticityReview:230} nicely taking $\ncap = \rcap, \phicap, \zcap$ in turn.  This derivation can be found in appendix \ref{chap:continuumstressTensorVectorForm}.

      %
% Copyright � 2012 Peeter Joot.  All Rights Reserved.
% Licenced as described in the file LICENSE under the root directory of this GIT repository.
%

%
%

\section{Compatibility condition compatibility condition for 2D strain} \index{compatibility condition}

\begin{equation}\label{eqn:continuumL6:50}
e_{ij} =
\begin{bmatrix}
e_{11} & e_{12} \\
e_{21} & e_{22}
\end{bmatrix}
\end{equation}

From \eqnref{eqn:revTextcontinuumL2:90} we see that we have

\begin{equation}\label{eqn:continuumL6:70}
\begin{aligned}
e_{11} &= \PD{x_1}{e_1} \\
e_{22} &= \PD{x_2}{e_2} \\
e_{12} = e_{21} &=
\inv{2} \left(
\PD{x_1}{e_2}
+ \PD{x_2}{e_1}
\right).
\end{aligned}
\end{equation}

We have a relationship between these displacements (called the compatibility relationship), which is

\boxedEquation{eqn:continuumL6:110}{
\PDSq{x_2}{e_{11}} +
\PDSq{x_1}{e_{22}} =
2
\frac{\partial^2 e_{12}}{\partial x_1 \partial x_2}.
}

We find this by straight computation

\begin{equation}\label{eqn:4compatibility:149}
\begin{aligned}
\PDSq{x_2}{e_{11}}
&=
\PDSq{x_2}{}\left(
\PD{x_1}{e_1}
\right) \\
&=
\frac{\partial^3 e_1}{\partial x_1 \partial x_2^2},
\end{aligned}
\end{equation}

and

\begin{equation}\label{eqn:4compatibility:169}
\begin{aligned}
\PDSq{x_1}{e_{22}}
&=
\PDSq{x_1}{}\left(
\PD{x_2}{e_2}
\right) \\
&=
\frac{\partial^3 e_2}{\partial x_2 \partial x_1^2},
\end{aligned}
\end{equation}

Now, looking at the cross term we find

\begin{equation}\label{eqn:4compatibility:189}
\begin{aligned}
2 \frac{\partial^2 e_{12}}{\partial x_1 \partial x_2}
&=
\frac{\partial^2 e_{12}}{\partial x_1 \partial x_2}
\left(
\PD{x_1}{e_2}
+ \PD{x_2}{e_1}
\right) \\
&=
\left(
\frac{\partial^3 e_1}{\partial x_1 \partial x_2^2}
+
\frac{\partial^3 e_2}{\partial x_2 \partial x_1^2}
\right)
\end{aligned}
\end{equation}

We have found an interrelationship between the components of the strain

\boxedEquation{eqn:continuumL6:129}{
2 \frac{\partial^2 e_{12}}{\partial x_1 \partial x_2}
=
\PDSq{x_1}{e_{22}}
+\PDSq{x_2}{e_{11}}.
}

This relationship is called the \textit{compatibility condition}, and ensures that we do not have a disjoint deformation of the form in \cref{fig:continuumL6:continuumL6fig1}.

\imageFigure{../../figures/phy454/lec6_disjoint_deformation_illustratedFig1}{disjoint deformation illustrated}{fig:continuumL6:continuumL6fig1}{0.3}

I went looking for something to substantiate the claim that the compatibility condition \eqnref{eqn:continuumL6:129} is what is required to ensure a deformation maintained a coherent solid geometry.  I was not able to find any references to this compatibility condition in any of the texts I have, but found \citep{wiki:compatibilityMechanics}, \citep{wiki:infinitesimalStrainTheory}, and \citep{wiki:saintVenantCompat}.  It is not terribly surprising to see Christoffel symbol and differential forms references on those pages, since one can imagine that we would wish to look at the mappings of all the points in the object as it undergoes the transformation from the original to the deformed state.

Even with just three points in a plane, say \(\Ba\), \(\Bb\), \(\Bc\), the general deformation of an object does not seem like it is the easiest thing to describe.  We can imagine that these have trajectories in the deformation process \(\Ba = \Ba(\alpha\), \(\Bb = \Bb(\beta)\), \(\Bc = \Bc(\gamma)\), with \(\Ba', \Bb', \Bc'\) at the end points of the trajectories.  We would want to look at displacement vectors \(\Bu_a, \Bu_b, \Bu_c\) along each of these trajectories, and then see how they must be related.  Doing that carefully must result in this compatibility condition.

\section{Compatibility condition for 3D strain}

While we have 9 components in the tensor, not all of these are independent.  The sets above and below the diagonal can be related.  It can be shown that there are 6 relationships between the components of the general three dimensional strain tensor \(e_{ij}\).
%, as illustrated in \cref{fig:continuumL6:continuumL6fig2}.
%
%\imageFigure{../../figures/phy454/continuumL6fig2}{continuumL6fig2}{fig:continuumL6:continuumL6fig2}{0.2}



      \label{chap:continuumL3}
%\useCCL
\blogpage{http://sites.google.com/site/peeterjoot2/math2012/continuumL3.pdf}
%\date{Jan 18, 2012}
\revisionInfo{continuumL3.tex}

\section{Review.  Strain.}

Strain is the measure of stretching.  This is illustrated pictorially in figure (\ref{fig:continuumL3:continuumL3fig1})
\imageFigure{continuumL3fig1}{Stretched line elements.}{fig:continuumL3:continuumL3fig1}{0.2}

\begin{equation}\label{eqn:continuumL3:10}
{ds'}^2 - ds^2 = 2 e_{ik} dx_i dx_k,
\end{equation}

where $e_{ik}$ is the strain tensor.  We found

\begin{equation}\label{eqn:continuumL3:30}
e_{ik} = \inv{2} \left( 
\PD{x_k}{e_i} 
+\PD{x_i}{e_k} 
+
\PD{x_i}{e_l} 
\PD{x_k}{e_l} 
\right)
\end{equation}

Why do we have a factor two?  Observe that if the deformation is small we can write

\begin{align*}
{ds'}^2 - ds^2 
&= (ds' - ds)(ds' + ds) \\
&\approx
 (ds' - ds) 2 ds
\end{align*}

so that we find 

\begin{equation}\label{eqn:continuumL3:50}
\frac{{ds'}^2 - ds^2 }{ds^2}
\approx
\frac{ds' - ds }{ds}
\end{equation}

Suppose for example, that we have a diagonalized strain tensor, then we find

\begin{equation}\label{eqn:continuumL3:70}
{ds'}^2 - ds^2 
= 2 e_{ii} \left(\frac{dx_i}{ds}\right)^2
\end{equation}

so that

\begin{equation}\label{eqn:continuumL3:90}
\frac{
{ds'}^2 - ds^2 
}{ds^2}
= 2 e_{ii} dx_i^2
\end{equation}

Observe that here again we see this factor of two.

If we have a diagonalized strain tensor, the tensor is of the form

\begin{equation}\label{eqn:continuumL3:110}
\begin{bmatrix}
e_{11} & 0 & 0 \\
0 & e_{22} & 0 \\
0 & 0 & e_{33} 
\end{bmatrix}
\end{equation}

we have

\begin{equation}\label{eqn:continuumL3:130}
{dx_i'}^2 - dx_i^2 = 2 e_{ii} dx_i^2
\end{equation}

\begin{equation}\label{eqn:continuumL3:150}
{ds'}^2 = 
(1 + 2 e_{11}) dx_1^2
+(1 + 2 e_{22}) dx_2^2
+(1 + 2 e_{33}) dx_3^2
\end{equation}

\begin{equation}\label{eqn:continuumL3:170}
ds^2 = 
dx_1^2
+dx_2^2
+dx_3^2
\end{equation}

so 

\begin{align}\label{eqn:continuumL3:190}
dx_1' &= \sqrt{1 + 2 e_{11}} dx_1 \sim ( 1 + e_{11}) dx_1 \\
dx_2' &= \sqrt{1 + 2 e_{22}} dx_2 \sim ( 1 + e_{22}) dx_2 \\
dx_3' &= \sqrt{1 + 2 e_{33}} dx_3 \sim ( 1 + e_{33}) dx_3
\end{align}

Observe that the change in the volume element becomes the trace

\begin{equation}\label{eqn:continuumL3:210}
dV' = 
dx_1'
dx_2'
dx_3'
= dV(1 + e_{ii})
\end{equation}

How do we use this?  Suppose that you are given a strain tensor.  This should allow you to compute the stretch in any given direction.

FIXME: find problem and try this.

      %
% Copyright � 2012 Peeter Joot.  All Rights Reserved.
% Licenced as described in the file LICENSE under the root directory of this GIT repository.
%

%
%
\makeoproblem{\(\BP\)-waves, \(\BS\)-waves, and Love-waves}{problem:elastic:displacements:midtermQ1a}
{2012 midterm, question 1a}
{
Show that in \(\BP\)-waves the divergence of the displacement vector represents a measure of the relative change in the volume of the body.
} % makeoproblem

\makeanswer{problem:elastic:displacements:midtermQ1a}{
The \(\BP\)-wave equation was a result of operating on the displacement equation with the divergence operator

\begin{equation}\label{eqn:continuumMidTermReflection:10}
\spacegrad \cdot \left(
\rho \PDSq{t}{\Be} = (\lambda + \mu) \spacegrad (\spacegrad \cdot \Be) + \mu \spacegrad^2 \Be
\right)
\end{equation}

we obtain

\begin{equation}\label{eqn:continuumMidTermReflection:30}
\PDSq{t}{} \left( \spacegrad \cdot \Be \right) = \frac{\lambda + 2 \mu}{\rho} \spacegrad^2 (\spacegrad \cdot \Be).
\end{equation}

We have a wave equation where the ``waving'' quantity is \(\Theta = \spacegrad \cdot \Be\).  Explicitly

\begin{equation}\label{eqn:problems:3230}
\begin{aligned}
\Theta
&= \spacegrad \cdot \Be \\
&=
\PD{x}{e_1}
+\PD{y}{e_2}
+\PD{z}{e_3}
\end{aligned}
\end{equation}

Recall that, in a coordinate basis for which the strain \(e_{ij}\) is diagonal we have

\begin{equation}\label{eqn:continuumMidTermReflection:50}
\begin{aligned}
dx' &= \sqrt{1 + 2 e_{11}} dx \\
dy' &= \sqrt{1 + 2 e_{22}} dy \\
dz' &= \sqrt{1 + 2 e_{33}} dz.
\end{aligned}
\end{equation}

Expanding in Taylor series to \(O(1)\) we have for \(i = 1, 2, 3\) (no sum)

\begin{equation}\label{eqn:continuumMidTermReflection:70}
dx_i' \approx (1 + e_{ii}) dx_i.
\end{equation}

so the displaced volume is

\begin{equation}\label{eqn:problems:3250}
\begin{aligned}
dV' &=
dx_1
dx_2
dx_3
(1 + e_{11})
(1 + e_{22})
(1 + e_{33}) \\
&=
dx_1
dx_2
dx_3
( 1  + e_{11} + e_{22} + e_{33} + O(e_{kk}^2) )
\end{aligned}
\end{equation}

Since

\begin{equation}\label{eqn:continuumMidTermReflection:90}
\begin{aligned}
e_{11} &= \inv{2} \left( \PD{x}{e_1} +\PD{x}{e_1} \right) = \PD{x}{e_1} \\
e_{22} &= \inv{2} \left( \PD{y}{e_2} +\PD{y}{e_2} \right) = \PD{y}{e_2} \\
e_{33} &= \inv{2} \left( \PD{z}{e_3} +\PD{z}{e_3} \right) = \PD{z}{e_3}
\end{aligned}
\end{equation}

We have

\begin{equation}\label{eqn:continuumMidTermReflection:110}
dV' = (1 + \spacegrad \cdot \Be) dV,
\end{equation}

or

\begin{equation}\label{eqn:continuumMidTermReflection:130}
\frac{dV' - dV}{dV} = \spacegrad \cdot \Be
\end{equation}

The relative change in volume can therefore be expressed as the divergence of \(\Be\), the displacement vector, and it is this relative volume change that is ``waving'' in the \(\BP\)-wave equation as illustrated in the following \cref{fig:continuumMidtermReflection:continuumMidtermReflectionFig1} sample 1D compression wave

\imageFigure{../../figures/phy454/midtermReflectionA_1D_compression_waveFig1}{A 1D compression wave}{fig:continuumMidtermReflection:continuumMidtermReflectionFig1}{0.2}
} % end answer

\makeoproblem{Classify \(\BP\)-waves and \(\BS\)-waves as longitudinal or transverse}{problem:elastic:displacements:midtermQ1b}
{2012 midterm, question 1b}
{
Between a \(\BP\)-wave and an \(\BS\)-wave which one is longitudinal and which one is transverse?
} % makeoproblem

\makeanswer{problem:elastic:displacements:midtermQ1b}{
\(\BP\)-waves are longitudinal.

\(\BS\)-waves are transverse.
} % end answer

\makeoproblem{Speed of \(\BP\)-waves and \(\BS\)-waves}{problem:elastic:displacements:midtermQ1c}
{2012 midterm, question 1c}
{
Whose speed is higher?
} % makeoproblem

\makeanswer{problem:elastic:displacements:midtermQ1c}{
From the (midterm) formula sheet we have

\begin{equation}\label{eqn:problems:3270}
\begin{aligned}
\left( \frac{c_L}{c_T} \right)^2
&= \frac{ \lambda + 2 \mu}{\rho} \frac{\rho}{\mu}  \\
&= \frac{\lambda}{\mu} + 2  \\
&> 1
\end{aligned}
\end{equation}

so \(\BP\)-waves travel faster than \(\BS\)-waves.
} % end answer

\makeoproblem{Love waves}{problem:elastic:displacements:midtermQ1d}
{2012 midterm, question 1d}
{
Is Love wave a body wave or a surface wave?
} % makeoproblem

\makeanswer{problem:elastic:displacements:midtermQ1d}{
Love waves are surface waves, traveling in a medium that can slide on top of another surface.  They are characterized by shear displacements perpendicular to the direction of propagation.

Reviewing for the final I see that I had answered this wrong, and have corrected it.  I had described a Rayleigh wave (also a surface wave).  A Rayleigh wave is characterized by vorticity rotating backwards compared to the direction of propagation as shown in \cref{fig:continuumMidtermReflection:continuumMidtermReflectionFig2}


\pdfTexFigure{../../figures/phy454/continuumMidtermReflectionFig2.pdf_tex}{Rayleigh wave illustrated}{fig:continuumMidtermReflection:continuumMidtermReflectionFig2}{0.6}
} % end answer

\makeproblem{Equilibrium}{problem:elastic:displacements:exampractiseEquilibrium}{
Suppose that the state of a body is given by

\begin{dmath}\label{eqn:continuumFluidsReviewXX:3130}
\sigma_{11} = A x^4 y^3
\sigma_{22} = 3 B x^2 y^5
\sigma_{12} = -C x^3 y^4
\end{dmath}

Determine the constants \(A\), \(B\) and \(C\) so that the body is in equilibrium (2011 Final Exam question II).
} % makeproblem

\makeanswer{problem:elastic:displacements:exampractiseEquilibrium}{
We have

\begin{dmath}\label{eqn:continuumFluidsReviewXX:3150}
0
= \PD{x_j}{\sigma_{1j}}
=
\PD{x}{\sigma_{11}} + \PD{y}{\sigma_{12}}
= 4 A x^3 y^3 - 4 C x^3 y^3,
\end{dmath}

and


\begin{dmath}\label{eqn:continuumFluidsReviewXX:3170}
0
= \PD{x_j}{\sigma_{2j}}
=
\PD{x}{\sigma_{21}} + \PD{y}{\sigma_{22}}
= -3 C x^2 y^4 + 15 B x^2 y^4
\end{dmath}

We must then have

\begin{dmath}\label{eqn:continuumFluidsReviewXX:3190}
0 = A - C
0 = -C + 5 B
\end{dmath}

Or


\begin{dmath}\label{eqn:continuumFluidsReviewXX:3210}
A = C
B = \frac{C}{5}.
\end{dmath}
} % end answer

\makeoproblem{Tsunami}{problem:elastic:displacements:exampractiseTsunami}
{2011 final exam}
{
Explain how the strain energy of tectonic plates causes Tsunami.
} % makeoproblem

\makeanswer{problem:elastic:displacements:exampractiseTsunami}{
The root cause of the Tsunami is the earthquake under the body of water.  Once that earthquake occurs we will have a body wave in the mantle, which will trigger a much more destructive (higher amplitude) surface wave (probably of the Rayleigh type).  Looking back to the connection with strain energy, we see that once we have a change in the strain divergence, we will have to have a restoring force to put things back in equilibrium.  That restoring force can come either from the surrounding mantle or the fluid above it, and it is that fluid restoring force that induces the wave as a side effect.
} % end answer

\FIXME{This is from the 2012 midterm.  We never got any real problems on elastic waves.  My preference would have been for actual problems that require solutions to the wave equations under various conditions.  If we then examined those solutions and characterized them (Love, Rayleigh, ...) we would not just have a requirement to restate memorized descriptive stuff, a task of little value.}

\makeproblem{Wave equation solutions}{problem:elastic:displacements:placeholderWaveEquation}{

PLACEHOLDER.

\FIXME{We never did get any homework assignments with actual problems where we find Rayleigh or Love wave solutions, so that we could get a feel for how to apply the formalism.  This would be a good place to put some.}
} % makeproblem

   \chapter{Stress tensor.}
      \section{Stress tensor.}

Reading for this section is \S 2 from \cite{landau1960theory}.

We'd like to consider a macroscopic model that contains the net effects of all the internal forces in the object as depicted in figure (\ref{fig:continuumL3:continuumL3fig2})

\imageFigure{figures/continuumL3fig2}{Internal forces.}{fig:continuumL3:continuumL3fig2}{0.2}

We will consider a volume big enough that we won't have to consider the individual atomic interactions, only the average effects of those interactions.  Will will look at the force per unit volume on a differential volume element

The total force on the body is 

\begin{equation}\label{eqn:continuumL3:230}
\iiint \BF dV,
\end{equation}

where $\BF$ is the force per unit volume.  We will evaluate this by utilizing the divergence theorem.  Recall that this was

\begin{equation}\label{eqn:continuumL3:250}
\iiint (\spacegrad \cdot \BA) dV
= \iint \BA \cdot d\Bs
\end{equation}

We have a small problem, since we have a non-divergence expression of the force here, and it is not immediately obvious that we can apply the divergence theorem.  We can deal with this by assuming that we can find a vector valued tensor, so that if we take the divergence of this tensor, we end up with the force.  We introduce the vector valued quantity

\begin{equation}\label{eqn:continuumL3:270}
\BF = \Be_i \PD{x_k}{\sigma_{ik}},
\end{equation}

and then apply the divergence theorem

\begin{equation}\label{eqn:continuumL3:290}
\iiint \BF dV 
= \iiint \Be_i \PD{x_k}{\sigma_{ik}} d\Bx^3 
=
\iint \Be_i \sigma_{ik} ds_k,
\end{equation}

where $ds_k$ is a surface element.  We identify this tensor

\begin{equation}\label{eqn:continuumL3:310}
\sigma_{ik} = \frac{\text{Force} \cdot \Be_i}{\text{Unit Area}},
\end{equation}

often writing it in matrix form

\begin{equation}\label{eqn:continuumL3:350}
\begin{bmatrix}
\sigma_{11} & \sigma_{12} & \sigma_{13} \\
\sigma_{21} & \sigma_{22} & \sigma_{23} \\
\sigma_{31} & \sigma_{32} & \sigma_{33}
\end{bmatrix}
\end{equation}

Our total force acting on the surface is given by the matrix product of the stress with the triplet of surface area elements

\begin{equation}\label{eqn:continuumL3:330}
f_i = \sigma_{ik} ds_k,
\end{equation}

as the force on the surface element $ds_k$.  

\subsection{Example: stretch in two opposing directions.}

\imageFigure{figures/continuumL3fig4}{Opposing stresses in one direction.}{fig:continuumL3:continuumL3fig4}{0.2}

Here, as illustrated in figure (\ref{fig:continuumL3:continuumL3fig4}), the associated (2D) stress tensor takes the simple form

\begin{equation}\label{eqn:continuumL3:370}
\begin{bmatrix}
\sigma_{11} & 0 \\
0 & 0
\end{bmatrix}
\end{equation}

This is called uniaxial stress \index{uniaxial stress}.

\subsection{Example: stretch in a pair of mutually perpendicular directions}

For a pair of perpendicular forces applied in two dimensions, as illustrated in figure (\ref{fig:continuumL3:continuumL3fig5})
\imageFigure{figures/continuumL3fig5}{Mutually perpendicular forces}{fig:continuumL3:continuumL3fig5}{0.2}

our stress tensor now just takes the form

\begin{equation}\label{eqn:continuumL3:390}
\begin{bmatrix}
\sigma_{11} & 0 \\
0 & \sigma_{22}
\end{bmatrix}
\end{equation}

This is called biaxial stress \index{biaxial stress}.

It's easy to imagine now how to get some more general stress tensors, should we make a change of basis that rotates our frame.

\subsection{Example: radial stretch}

Suppose we have a fire fighter's safety net, used to catch somebody jumping from a burning building (do they ever do that outside of movies?), as in figure (\ref{fig:continuumL3:continuumL3fig6}).  Each of the firefighters contributes to the stretch.  

\imageFigure{figures/continuumL3fig6}{Radial forces.}{fig:continuumL3:continuumL3fig6}{0.2}

      %
% Copyright � 2012 Peeter Joot.  All Rights Reserved.
% Licenced as described in the file LICENSE under the root directory of this GIT repository.
%
\label{chap:continuumL4}
\section{Stress tensor in 2D}

In two dimensions this is illustrated in \cref{fig:continuumL3:continuumL3fig3}

\imageFigure{../../figures/phy454/lec3_2D_stress_tensorFig3}{2D stress tensor}{fig:continuumL3:continuumL3fig3}{0.2}

Observe that we use the index \(i\) above as the direction of the force, and index \(k\) as the direction normal to the surface.

We will show later that this tensor is in fact symmetric.
\FIXME{was this done?}

For the stress tensor

\begin{equation}\label{eqn:continuumL4:10}
\sigma_{ij},
\end{equation}

a second rank tensor, the first index \(i\) defines the direction of the force, and the second index \(j\) defines the surface.

Observe that the dimensions of \(\sigma_{ij}\) is force per unit area, just like pressure.  We will in fact show that this tensor is akin to the pressure, and the diagonalized components of this tensor represent the pressure.

We have illustrated the stress tensor in a couple of 2D examples.  The first we call \textAndIndex{uniaxial stress}, having just the \(1,1\) element of the matrix as illustrated in \cref{fig:continuumL3:continuumL3fig4}.
%\cref{fig:continuumL4:continuumL4fig1}
%\imageFigure{../../figures/phy454/lec4_Uniaxial_stressFig1}{Uniaxial stress}{fig:continuumL4:continuumL4fig1}{0.3}

\begin{equation}\label{eqn:continuumL4:30}
\sigma =
\begin{bmatrix}
\sigma_{11} & 0 \\
0 & 0
\end{bmatrix}.
\end{equation}

A \textAndIndex{biaxial stress} is illustrated in \cref{fig:continuumL3:continuumL3fig5}.
%\cref{fig:continuumL4:continuumL4fig2}
%\imageFigure{../../figures/phy454/lec4_Biaxial_stressFig2}{Biaxial stress}{fig:continuumL4:continuumL4fig2}{0.3}

where for \(\sigma_{11} \ne \sigma_{22}\) our tensor takes the form

\begin{equation}\label{eqn:continuumL4:50}
\sigma =
\begin{bmatrix}
\sigma_{11} & 0 \\
0 & \sigma_{22}
\end{bmatrix}.
\end{equation}

In the general case we have

\begin{equation}\label{eqn:continuumL4:70}
\sigma =
\begin{bmatrix}
\sigma_{11} & \sigma_{12} \\
\sigma_{21} & \sigma_{22}
\end{bmatrix}.
\end{equation}

We can attempt to illustrate this, but it becomes much harder to visualize as shown in \cref{fig:continuumL4:continuumL4fig3}
\imageFigure{../../figures/phy454/lec4_General_stressFig3}{General stress}{fig:continuumL4:continuumL4fig3}{0.3}

In equilibrium we must have

\begin{equation}\label{eqn:continuumL4:90}
\sigma_{12} = \sigma_{21}.
\end{equation}

We can use similar arguments to show that the stress tensor is symmetric.
\FIXME{perhaps there was a verbal argument in class here.  This is not a sensible explanation of the symmetry requirement as is}

%\section{Diagonalization}
We will look at the two dimensional case in some detail, as in \cref{fig:continuumL4:continuumL4fig6}

\imageFigure{../../figures/phy454/lec4_Area_element_under_stress_with_and_without_rotationFig6}{Area element under stress with and without rotation}{fig:continuumL4:continuumL4fig6}{0.4}

Under this coordinate transformation, a rotation, the diagonal stress tensor is taken to a non-diagonal form \index{diagonalization}

\begin{equation}\label{eqn:continuumL4:130}
\begin{bmatrix}
\sigma_{11} & 0 \\
0 & \sigma_{22}
\end{bmatrix}
\leftrightarrow
\begin{bmatrix}
\sigma_{11}' & \sigma_{12}' \\
\sigma_{21}' & \sigma_{22}'
\end{bmatrix}
\end{equation}

      %
% Copyright � 2012 Peeter Joot.  All Rights Reserved.
% Licenced as described in the file LICENSE under the root directory of this GIT repository.
%
\section{Stress tensor in 3D}

In 3D we have three components of the stress tensor acting on each surface, as illustrated in \cref{fig:continuumL4:continuumL4fig5}
\imageFigure{../../figures/phy454/lec4_Strain_components_on_a_3D_volumeFig5}{Strain components on a 3D volume}{fig:continuumL4:continuumL4fig5}{0.4}

We have three unique surface orientations and three components of the force for each of these, resulting in nine components, but these are not all independent.  For an object in equilibrium we must have \(\sigma_{ij} = \sigma_{ji}\).
\FIXME{justify?}
Explicitly, that is

\begin{align}\label{eqn:continuumL4:110}
\sigma_{12} &= \sigma_{21} \\
\sigma_{23} &= \sigma_{32} \\
\sigma_{31} &= \sigma_{13}
\end{align}


      %
% Copyright � 2012 Peeter Joot.  All Rights Reserved.
% Licenced as described in the file LICENSE under the root directory of this GIT repository.
%
\section{Cauchy tetrahedron}

To examine the question of how the stress tensor and the force relate, we project the force onto a planar surface.  This is called the \textAndIndex{Cauchy tetrahedron} as in \cref{fig:continuumL4:continuumL4fig7}

\imageFigure{../../figures/phy454/lec4_Cauchy_tetrahedronFig7}{Cauchy tetrahedron}{fig:continuumL4:continuumL4fig7}{0.4}

\begin{equation}\label{eqn:continuumL4:150}
\Bf = \frac{\text{external force}}{\text{unit area}} = f_j \Be_j
\end{equation}

\begin{equation}\label{eqn:continuumL4:170}
\text{internal stress} = \text{external force}
\end{equation}

We write \(\ncap\) in terms of the direction cosines

\begin{equation}\label{eqn:continuumL4:190}
\ncap =
n_1 \Be_1 +
n_2 \Be_2 +
n_3 \Be_3
\end{equation}

Here

\begin{align}\label{eqn:continuumL4:210}
n_1 &= \ncap \cdot \Be_1 \\
n_2 &= \ncap \cdot \Be_2 \\
n_3 &= \ncap \cdot \Be_3,
\end{align}

or

\begin{equation}\label{eqn:continuumL4:230}
n_j = \ncap \cdot \Be_j = \cos\phi_j
\end{equation}

This is illustrated in \cref{fig:continuumL5:continuumL5fig1}.

\imageFigure{../../figures/phy454/lec5_Cauchy_tetrahedron_direction_cosinesFig1}{Cauchy tetrahedron direction cosines}{fig:continuumL5:continuumL5fig1}{0.3}

Performing a force balance on \(x_1\) direction, where we match total external force in each direction to the total internal force (\(\sigma_{ij}'s\)) as follows

\begin{equation}\label{eqn:continuumL4:250}
\begin{aligned}
f_1 \times \text{area ABC}
&=
\sigma_{11} \times \text{area BOC} \\
&+\sigma_{12} \times \text{area AOC} \\
&+\sigma_{13} \times \text{area AOB}
\end{aligned}
\end{equation}

Similarly

\begin{equation}\label{eqn:continuumL4:270}
\begin{aligned}
f_2 \times \text{area ABC}
&=
\sigma_{21} \times \text{area BOC} \\
&+\sigma_{22} \times \text{area AOC} \\
&+\sigma_{23} \times \text{area AOB},
\end{aligned}
\end{equation}

and

\begin{equation}\label{eqn:continuumL4:290}
\begin{aligned}
f_3 \times \text{area ABC}
&=
\sigma_{31} \times \text{area BOC} \\
&+\sigma_{32} \times \text{area AOC} \\
&+\sigma_{33} \times \text{area AOB},
\end{aligned}
\end{equation}

We can therefore write these force components like

\begin{equation}\label{eqn:continuumL4:310}
f_1 =
\sigma_{11} \frac{BOC}{ABC} +
\sigma_{12} \frac{AOC}{ABC} +
\sigma_{13} \frac{AOB}{ABC}
\end{equation}

but these ratios are really just the projections of the areas as illustrated in \cref{fig:continuumL4:continuumL4fig8}

\imageFigure{../../figures/phy454/lec4_Area_projectionFig8}{Area projection}{fig:continuumL4:continuumL4fig8}{0.3}

where an arbitrary surface with area \(\Delta S\) can be decomposed into projections

\begin{equation}\label{eqn:continuumL4:330}
\Delta S \cos\phi_j,
\end{equation}

utilizing the direction cosines.  We can therefore write

\begin{align}\label{eqn:continuumL4:350}
f_1 &= \sigma_{11} n_1 + \sigma_{12} n_2 + \sigma_{13} n_3 \\
f_2 &= \sigma_{21} n_1 + \sigma_{22} n_2 + \sigma_{23} n_3 \\
f_3 &= \sigma_{31} n_1 + \sigma_{32} n_2 + \sigma_{33} n_3,
\end{align}

or in matrix notation

\begin{equation}\label{eqn:continuumL4:370}
\begin{bmatrix}
f_1  \\
f_2  \\
f_3
\end{bmatrix}
=
\begin{bmatrix}
\sigma_{11} & \sigma_{12} & \sigma_{13} \\
\sigma_{21} & \sigma_{22} & \sigma_{23} \\
\sigma_{31} & \sigma_{32} & \sigma_{33}
\end{bmatrix}
\begin{bmatrix}
n_1 \\
n_2 \\
n_3 \\
\end{bmatrix}.
\end{equation}

This is just

\boxedEquation{eqn:continuumL4:390}{
f_i = \sigma_{ij} n_j.
}

This force with components \(f_i\) is also called the \textAndIndex{traction vector}

\begin{equation}\label{eqn:continuumL4:410}
\tau_i = \sigma_{ij} n_j.
\end{equation}

In matrix form the traction vector is

\begin{equation}\label{eqn:continuumL5:50}
\Btau = \Bsigma \cdot \ncap
=
\begin{bmatrix}
\sigma_{11} & \sigma_{12} & \sigma_{13} \\
\sigma_{21} & \sigma_{22} & \sigma_{23} \\
\sigma_{31} & \sigma_{32} & \sigma_{33}
\end{bmatrix}
\begin{bmatrix}
n_1 \\
n_2 \\
n_3
\end{bmatrix}
\end{equation}

      %
% Copyright � 2012 Peeter Joot.  All Rights Reserved.
% Licenced as described in the file LICENSE under the root directory of this GIT repository.
%
\section{Constitutive relation}

Reading: \S 2, \S 4 and \S 5 from the text \citep{landau1960theory}.

We can find the relationship between stress and strain, both analytically and experimentally, and call this the Constitutive relation.  We prefer to deal with ranges of distortion that are small enough that we can make a linear approximation for this relation.  In general such a linear relationship takes the form

\begin{equation}\label{eqn:continuumL5:70}
\sigma_{ij} = c_{ijkl} e_{kl}.
\end{equation}

Materials for which the stress and strain tensors are linearly related are called Newtonian \index{Newtonian material}.  We will not consider any non-Newtonian materials in this course.

Consider the number of components that we are talking about for various rank tensors

\begin{equation}\label{eqn:continuumL5:90}
\begin{array}{l l}
\mbox{\(0^{\text{th}}\) rank tensor} & \mbox{\(3^0 = 1\) components} \\
\mbox{\(1^{\text{st}}\) rank tensor} & \mbox{\(3^1 = 3\) components} \\
\mbox{\(2^{\text{nd}}\) rank tensor} & \mbox{\(3^2 = 9\) components} \\
\mbox{\(3^{\text{rd}}\) rank tensor} & \mbox{\(3^3 = 81\) components}
\end{array}
\end{equation}

We have a lot of components, even for a linear relation between stress and strain.  For isotropic materials we model the constitutive relation instead as

\boxedEquation{eqn:continuumL5:110}{
\sigma_{ij} = \lambda e_{kk} \delta_{ij} + 2 \mu e_{ij}.
}

It can be shown \citep{feynman1963flp:elasticMaterials} that a relationship between stress and strain of this form is actually required by isotropy.

For such a modeling of the material the (measured) values \(\lambda\) and \(\mu\) (shear modulus or modulus of rigidity) are called the Lam\'e parameters.

It will be useful to compute the trace of the stress tensor in the form of the constitutive relation for the isotropic model.  We find

\begin{equation}\label{eqn:14constituative:170}
\begin{aligned}
\sigma_{ii}
&= \lambda e_{kk} \delta_{ii} + 2 \mu e_{ii} \\
&= 3 \lambda e_{kk} + 2 \mu e_{jj},
\end{aligned}
\end{equation}

or

\begin{equation}\label{eqn:continuumL5:150}
\sigma_{ii} = (3 \lambda + 2 \mu) e_{kk}.
\end{equation}

We can now also invert this, to find the trace of the strain tensor in terms of the stress tensor

\begin{equation}\label{eqn:continuumL5:130}
e_{ii} = \frac{\sigma_{kk}}{3 \lambda + 2 \mu}
\end{equation}

Substituting back into our original relationship \eqnref{eqn:continuumL5:110}, and find

\begin{equation}\label{eqn:continuumL5:110b}
\sigma_{ij} = \lambda \frac{\sigma_{kk}}{3 \lambda + 2 \mu} \delta_{ij} + 2 \mu e_{ij},
\end{equation}

which finally provides an inverted expression with the strain tensor expressed in terms of the stress tensor

\boxedEquation{eqn:continuumL5:110c}{
2 \mu e_{ij} =
\sigma_{ij} - \lambda \frac{\sigma_{kk}}{3 \lambda + 2 \mu} \delta_{ij}.
}


      %
% Copyright � 2012 Peeter Joot.  All Rights Reserved.
% Licenced as described in the file LICENSE under the root directory of this GIT repository.
%
\section{Constitutive relation for Hydrostatic compression}

Hydrostatic compression is when we have no shear stress, only normal components of the stress matrix \(\sigma_{ij}\) is nonzero.  Strictly speaking we define Hydrostatic compression as

\begin{equation}\label{eqn:continuumL5:170}
\sigma_{ij} = -p \delta_{ij},
\end{equation}

i.e. not only diagonal, but with all the components of the stress tensor equal.

We can write the trace of the stress tensor as

\begin{equation}\label{eqn:continuumL5:190}
\sigma_{ii} = - 3 p = (3 \lambda + 2 \mu) e_{kk}.
\end{equation}

Now, from our discussion of the strain tensor \(e_{ij}\) recall that we found in the limit

\begin{equation}\label{eqn:continuumL5:210}
dV' = (1 + e_{ii}) dV,
\end{equation}

allowing us to express the change in volume relative to the original volume in terms of the strain trace

\begin{equation}\label{eqn:continuumL5:230}
e_{ii} = \frac{dV' - dV}{dV}.
\end{equation}

Writing that relative volume difference as \(\Delta V/V\) we find

\begin{equation}\label{eqn:continuumL5:250}
- 3 p = (3 \lambda + 2 \mu) \frac{\Delta V}{V},
\end{equation}

or

\begin{equation}\label{eqn:continuumL5:270}
- \frac{ p V}{\Delta V} = \left( \lambda + \frac{2}{3} \mu \right) = K,
\end{equation}

where \(K\) is called the Bulk modulus.


      %
% Copyright � 2012 Peeter Joot.  All Rights Reserved.
% Licenced as described in the file LICENSE under the root directory of this GIT repository.
%
\section{Constitutive relation for uniaxial stress} \index{constitutive relation} \index{uniaxial stress}

Referring to \cref{fig:continuumL3:continuumL3fig4} and expanding out \eqnref{eqn:continuumL5:110c} we have for the \(1,1\) element of the strain tensor
%Again illustrated in the plane as in \cref{fig:continuumL5:continuumL5fig2}
%\imageFigure{../../figures/phy454/lec5_Uniaxial_stressFig2}{Uniaxial stress}{fig:continuumL5:continuumL5fig2}{0.2}

%2 \mu e_{ij} = \sigma_{ij} - \lambda \frac{\sigma_{kk}}{3 \lambda + 2 \mu} \delta_{ij}.

\begin{equation}\label{eqn:continuumL5:290}
\Bsigma =
\begin{bmatrix}
\sigma_{11} & 0 & 0\\
0 & 0 & 0 \\
0 & 0 & 0
\end{bmatrix}
\end{equation}

\begin{equation}\label{eqn:16constituativeUniaxial:490}
\begin{aligned}
2 \mu e_{11}
&= \sigma_{11} - \frac{\lambda ( \sigma_{11} + \cancel{\sigma_{22}} ) }{3 \lambda + 2 \mu} \\
&= \sigma_{11} \frac{3 \lambda + 2 \mu - \lambda }{3 \lambda + 2 \mu} \\
&= 2 \sigma_{11} \frac{\lambda + \mu }{3 \lambda + 2 \mu}
\end{aligned}
\end{equation}

or

\begin{equation}\label{eqn:continuumL5:310}
%\frac{e_{11}}{\sigma_{11}} = \frac{\lambda + \mu }{\mu(3 \lambda + 2 \mu)}
\frac{\sigma_{11}}{e_{11}} = \frac{\mu(3 \lambda + 2 \mu)}{\lambda + \mu } = E
\end{equation}

where \(E\) is Young's modulus.  Young's modulus in the text (5.3) is given in terms of the bulk modulus \(K\).  Using \(\lambda = K - 2\mu/3\) we find

\begin{equation}\label{eqn:16constituativeUniaxial:510}
\begin{aligned}
E
&=
\frac{\mu(3 \lambda + 2 \mu)}{\lambda + \mu } \\
&=
\frac{\mu(3 (K - 2\mu/3)+ 2 \mu)}{K - 2\mu/3 + \mu } \\
&=
\frac{3 K \mu}{ K + \mu/3 }
\end{aligned}
\end{equation}

\boxedEquation{eqn:continuumL5:330}{
E =
\frac{\mu(3 \lambda + 2 \mu)}{\lambda + \mu } =
\frac{9 K \mu}{ 3 K + \mu }
}

\FIXME{random: there is no \cref{fig:continuumL5:continuumL5fig3} reference that I can find?

\imageFigure{../../figures/phy454/lec5_stress_associated_with_Youngs_modulusFig3}{stress associated with Young's modulus}{fig:continuumL5:continuumL5fig3}{0.2}
}

We define Poisson's ratio \(\nu\) as the quantity

\begin{equation}\label{eqn:continuumL5:350}
\frac{e_{22}}{e_{11}} = \frac{e_{33}}{e_{11}} = - \nu.
\end{equation}

Note that we are still talking about uniaxial stress here.  Referring back to \eqnref{eqn:continuumL5:110c} we have

\begin{equation}\label{eqn:16constituativeUniaxial:530}
\begin{aligned}
%2 \mu e_{i j} = \sigma_{i j} - \lambda \frac{\sigma_{k k}}{3 \lambda + 2 \mu} \delta_{i j}
2 \mu e_{2 2}
&= \sigma_{2 2} - \lambda \frac{\sigma_{k k}}{3 \lambda + 2 \mu} \delta_{2 2} \\
&= \sigma_{2 2} - \lambda \frac{\sigma_{k k}}{3 \lambda + 2 \mu} \\
&= - \frac{\lambda \sigma_{11}}{3 \lambda + 2 \mu}
\end{aligned}
\end{equation}

Recall \eqnref{eqn:continuumL5:310} that we had

\begin{equation}\label{eqn:continuumL5:370}
\sigma_{11} = \frac{\mu (3 \lambda + 2 \mu)}{\lambda + \mu} e_{11}.
\end{equation}

Inserting this gives us
\begin{equation}\label{eqn:16constituativeUniaxial:550}
\begin{aligned}
2 \mu e_{22}
%&= - \frac{\lambda}{3 \lambda + 2 \mu} \frac{ \sigma_{11}}{e_{11}} e_{11} \\
= - \frac{\lambda}{\cancel{3 \lambda + 2 \mu}} \frac{ \mu (\cancel{3 \lambda + 2\mu})}{\lambda + \mu} e_{11}
\end{aligned}
\end{equation}

so
%\begin{equation}\label{eqn:continuumL5:390}
%\frac{e_{22}}{e_{11}} = -\frac{ \lambda \mu }{2 \mu (\lambda + \mu)}
%\end{equation}

\boxedEquation{eqn:continuumL5:410}{
\nu = -\frac{e_{22}}{e_{11}} = \frac{\lambda}{2 (\lambda + \mu)}.
}

We can also relate the Poisson's ratio \(\nu\) to the shear modulus \(\mu\) (see the appendix: \ref{chap:appendix:poissonAndShearModulus})

\begin{equation}\label{eqn:continuumL5:430}
% prof had:
%\mu = \frac{E}{1 + 4 \nu}
\mu = \frac{E}{2(1 + \nu)}
\end{equation}

\begin{equation}\label{eqn:continuumL5:450}
% prof had:
%\lambda = \frac{2 E \nu}{(1 - 2 \nu)(1 + 4 \mu)}
\lambda = \frac{E \nu}{(1 - 2 \nu)(1 + \nu)}
\end{equation}

\begin{equation}\label{eqn:continuumL5:470}
\begin{aligned}
e_{11} &= \inv{E}\left( \sigma_{11} - \nu(\sigma_{22} + \sigma_{33}) \right) \\
e_{22} &= \inv{E}\left( \sigma_{22} - \nu(\sigma_{11} + \sigma_{33}) \right) \\
e_{33} &= \inv{E}\left( \sigma_{33} - \nu(\sigma_{11} + \sigma_{22}) \right)
\end{aligned}
\end{equation}

These ones are (5.14) in the text, and are easy enough to verify (not done here).

      %
% Copyright � 2012 Peeter Joot.  All Rights Reserved.
% Licenced as described in the file LICENSE under the root directory of this GIT repository.
%

%
%
\makeoproblem{\(\BP\)-waves, \(\BS\)-waves, and Love-waves}{problem:elastic:displacements:midtermQ1a}
{2012 midterm, question 1a}
{
Show that in \(\BP\)-waves the divergence of the displacement vector represents a measure of the relative change in the volume of the body.
} % makeoproblem

\makeanswer{problem:elastic:displacements:midtermQ1a}{
The \(\BP\)-wave equation was a result of operating on the displacement equation with the divergence operator

\begin{equation}\label{eqn:continuumMidTermReflection:10}
\spacegrad \cdot \left(
\rho \PDSq{t}{\Be} = (\lambda + \mu) \spacegrad (\spacegrad \cdot \Be) + \mu \spacegrad^2 \Be
\right)
\end{equation}

we obtain

\begin{equation}\label{eqn:continuumMidTermReflection:30}
\PDSq{t}{} \left( \spacegrad \cdot \Be \right) = \frac{\lambda + 2 \mu}{\rho} \spacegrad^2 (\spacegrad \cdot \Be).
\end{equation}

We have a wave equation where the ``waving'' quantity is \(\Theta = \spacegrad \cdot \Be\).  Explicitly

\begin{equation}\label{eqn:problems:3230}
\begin{aligned}
\Theta
&= \spacegrad \cdot \Be \\
&=
\PD{x}{e_1}
+\PD{y}{e_2}
+\PD{z}{e_3}
\end{aligned}
\end{equation}

Recall that, in a coordinate basis for which the strain \(e_{ij}\) is diagonal we have

\begin{equation}\label{eqn:continuumMidTermReflection:50}
\begin{aligned}
dx' &= \sqrt{1 + 2 e_{11}} dx \\
dy' &= \sqrt{1 + 2 e_{22}} dy \\
dz' &= \sqrt{1 + 2 e_{33}} dz.
\end{aligned}
\end{equation}

Expanding in Taylor series to \(O(1)\) we have for \(i = 1, 2, 3\) (no sum)

\begin{equation}\label{eqn:continuumMidTermReflection:70}
dx_i' \approx (1 + e_{ii}) dx_i.
\end{equation}

so the displaced volume is

\begin{equation}\label{eqn:problems:3250}
\begin{aligned}
dV' &=
dx_1
dx_2
dx_3
(1 + e_{11})
(1 + e_{22})
(1 + e_{33}) \\
&=
dx_1
dx_2
dx_3
( 1  + e_{11} + e_{22} + e_{33} + O(e_{kk}^2) )
\end{aligned}
\end{equation}

Since

\begin{equation}\label{eqn:continuumMidTermReflection:90}
\begin{aligned}
e_{11} &= \inv{2} \left( \PD{x}{e_1} +\PD{x}{e_1} \right) = \PD{x}{e_1} \\
e_{22} &= \inv{2} \left( \PD{y}{e_2} +\PD{y}{e_2} \right) = \PD{y}{e_2} \\
e_{33} &= \inv{2} \left( \PD{z}{e_3} +\PD{z}{e_3} \right) = \PD{z}{e_3}
\end{aligned}
\end{equation}

We have

\begin{equation}\label{eqn:continuumMidTermReflection:110}
dV' = (1 + \spacegrad \cdot \Be) dV,
\end{equation}

or

\begin{equation}\label{eqn:continuumMidTermReflection:130}
\frac{dV' - dV}{dV} = \spacegrad \cdot \Be
\end{equation}

The relative change in volume can therefore be expressed as the divergence of \(\Be\), the displacement vector, and it is this relative volume change that is ``waving'' in the \(\BP\)-wave equation as illustrated in the following \cref{fig:continuumMidtermReflection:continuumMidtermReflectionFig1} sample 1D compression wave

\imageFigure{../../figures/phy454/midtermReflectionA_1D_compression_waveFig1}{A 1D compression wave}{fig:continuumMidtermReflection:continuumMidtermReflectionFig1}{0.2}
} % end answer

\makeoproblem{Classify \(\BP\)-waves and \(\BS\)-waves as longitudinal or transverse}{problem:elastic:displacements:midtermQ1b}
{2012 midterm, question 1b}
{
Between a \(\BP\)-wave and an \(\BS\)-wave which one is longitudinal and which one is transverse?
} % makeoproblem

\makeanswer{problem:elastic:displacements:midtermQ1b}{
\(\BP\)-waves are longitudinal.

\(\BS\)-waves are transverse.
} % end answer

\makeoproblem{Speed of \(\BP\)-waves and \(\BS\)-waves}{problem:elastic:displacements:midtermQ1c}
{2012 midterm, question 1c}
{
Whose speed is higher?
} % makeoproblem

\makeanswer{problem:elastic:displacements:midtermQ1c}{
From the (midterm) formula sheet we have

\begin{equation}\label{eqn:problems:3270}
\begin{aligned}
\left( \frac{c_L}{c_T} \right)^2
&= \frac{ \lambda + 2 \mu}{\rho} \frac{\rho}{\mu}  \\
&= \frac{\lambda}{\mu} + 2  \\
&> 1
\end{aligned}
\end{equation}

so \(\BP\)-waves travel faster than \(\BS\)-waves.
} % end answer

\makeoproblem{Love waves}{problem:elastic:displacements:midtermQ1d}
{2012 midterm, question 1d}
{
Is Love wave a body wave or a surface wave?
} % makeoproblem

\makeanswer{problem:elastic:displacements:midtermQ1d}{
Love waves are surface waves, traveling in a medium that can slide on top of another surface.  They are characterized by shear displacements perpendicular to the direction of propagation.

Reviewing for the final I see that I had answered this wrong, and have corrected it.  I had described a Rayleigh wave (also a surface wave).  A Rayleigh wave is characterized by vorticity rotating backwards compared to the direction of propagation as shown in \cref{fig:continuumMidtermReflection:continuumMidtermReflectionFig2}


\pdfTexFigure{../../figures/phy454/continuumMidtermReflectionFig2.pdf_tex}{Rayleigh wave illustrated}{fig:continuumMidtermReflection:continuumMidtermReflectionFig2}{0.6}
} % end answer

\makeproblem{Equilibrium}{problem:elastic:displacements:exampractiseEquilibrium}{
Suppose that the state of a body is given by

\begin{dmath}\label{eqn:continuumFluidsReviewXX:3130}
\sigma_{11} = A x^4 y^3
\sigma_{22} = 3 B x^2 y^5
\sigma_{12} = -C x^3 y^4
\end{dmath}

Determine the constants \(A\), \(B\) and \(C\) so that the body is in equilibrium (2011 Final Exam question II).
} % makeproblem

\makeanswer{problem:elastic:displacements:exampractiseEquilibrium}{
We have

\begin{dmath}\label{eqn:continuumFluidsReviewXX:3150}
0
= \PD{x_j}{\sigma_{1j}}
=
\PD{x}{\sigma_{11}} + \PD{y}{\sigma_{12}}
= 4 A x^3 y^3 - 4 C x^3 y^3,
\end{dmath}

and


\begin{dmath}\label{eqn:continuumFluidsReviewXX:3170}
0
= \PD{x_j}{\sigma_{2j}}
=
\PD{x}{\sigma_{21}} + \PD{y}{\sigma_{22}}
= -3 C x^2 y^4 + 15 B x^2 y^4
\end{dmath}

We must then have

\begin{dmath}\label{eqn:continuumFluidsReviewXX:3190}
0 = A - C
0 = -C + 5 B
\end{dmath}

Or


\begin{dmath}\label{eqn:continuumFluidsReviewXX:3210}
A = C
B = \frac{C}{5}.
\end{dmath}
} % end answer

\makeoproblem{Tsunami}{problem:elastic:displacements:exampractiseTsunami}
{2011 final exam}
{
Explain how the strain energy of tectonic plates causes Tsunami.
} % makeoproblem

\makeanswer{problem:elastic:displacements:exampractiseTsunami}{
The root cause of the Tsunami is the earthquake under the body of water.  Once that earthquake occurs we will have a body wave in the mantle, which will trigger a much more destructive (higher amplitude) surface wave (probably of the Rayleigh type).  Looking back to the connection with strain energy, we see that once we have a change in the strain divergence, we will have to have a restoring force to put things back in equilibrium.  That restoring force can come either from the surrounding mantle or the fluid above it, and it is that fluid restoring force that induces the wave as a side effect.
} % end answer

\FIXME{This is from the 2012 midterm.  We never got any real problems on elastic waves.  My preference would have been for actual problems that require solutions to the wave equations under various conditions.  If we then examined those solutions and characterized them (Love, Rayleigh, ...) we would not just have a requirement to restate memorized descriptive stuff, a task of little value.}

\makeproblem{Wave equation solutions}{problem:elastic:displacements:placeholderWaveEquation}{

PLACEHOLDER.

\FIXME{We never did get any homework assignments with actual problems where we find Rayleigh or Love wave solutions, so that we could get a feel for how to apply the formalism.  This would be a good place to put some.}
} % makeproblem


   \chapter{Displacement propagation}
      %
% Copyright � 2012 Peeter Joot.  All Rights Reserved.
% Licenced as described in the file LICENSE under the root directory of this GIT repository.
%

%
%
%\chapter{PHY454H1S\\Continuum Mechanics.  Lecture 7: P-waves and S-waves.  Taught by Prof. K. Das}
%\section{P-waves and S-waves}
\label{chap:continuumL7}


%\section{Setup}
%
%
%We got as far as expressing the vector displacement \(\Be\) for an isotropic material at a given point in terms of the Lam\'e parameters
%
%\begin{equation}\label{eqn:continuumL7:10}
%\rho \PDSq{t}{\Be} = (\lambda + \mu) \spacegrad (\spacegrad \cdot \Be) + \mu \spacegrad^2 \Be.
%\end{equation}
%
\section{P-waves} \index{p-wave}

Reading: \S 22 from \citep{landau1960theory}.

Operating on this with the divergence operator, and writing \(\theta = \spacegrad \cdot \Be\), we have

\begin{equation}\label{eqn:continuumL7:30}
\rho \PDSq{t}{\spacegrad \cdot \Be} = (\lambda + \mu) \spacegrad \cdot \spacegrad (\spacegrad \cdot \Be) + \mu \spacegrad^2 (\spacegrad \cdot \Be)
\end{equation}

or

\begin{equation}\label{eqn:continuumL7:50}
\PDSq{t}{\theta} = \frac{\lambda + 2 \mu}{\rho} \spacegrad^2 \theta.
\end{equation}

We see that our divergence is governed by a wave equation where the speed of the wave \(C_L\) is specified by

\begin{equation}\label{eqn:continuumL7:70}
C_L^2 = \frac{\lambda + 2 \mu}{\rho},
\end{equation}

so the displacement wave equation is given by

\begin{equation}\label{eqn:continuumL7:90}
\PDSq{t}{\theta} = C_L^2 \spacegrad^2 \theta.
\end{equation}

Let us look at the divergence of the displacement vector in some more detail.  By definition this is just

\begin{equation}\label{eqn:continuumL7:110}
\spacegrad \cdot \Be =
\PD{x_1}{e_1}
+\PD{x_2}{e_2}
+\PD{x_3}{e_3}.
\end{equation}

Recall that the strain tensor \(e_{ij}\) was defined as

\begin{equation}\label{eqn:continuumL7:130}
e_{ij} = \inv{2} \left(
\PD{x_j}{e_i}
+
\PD{x_i}{e_j}
\right),
\end{equation}

so we have

\begin{equation}\label{eqn:continuumL7:150}
\begin{aligned}
\PD{x_1}{e_1} &= e_{11} \\
\PD{x_2}{e_2} &= e_{22} \\
\PD{x_3}{e_3} &= e_{33}.
\end{aligned}
\end{equation}

So the divergence in question can be written in terms of the strain tensor

\begin{equation}\label{eqn:continuumL7:170}
\spacegrad \cdot \Be =
e_{11}
+e_{22}
+e_{33} = e_{ii}.
\end{equation}

We also found that the trace of the strain tensor was the relative change in volume.  We call this the dilatation.  A measure of change in volume as illustrated (badly) in \cref{fig:continuumL7:continuumL7fig1}

\imageFigure{../../figures/phy454/lec7_Illustrating_changes_in_a_control_volumeFig1}{Illustrating changes in a control volume}{fig:continuumL7:continuumL7fig1}{0.2}

This idea can be found nicely animated in the wikipedia page \citep{wiki:pwave}.

\section{S-waves} \index{s-wave}

Now let us operate on our \eqnref{eqn:continuumL6:290} with the curl operator

\begin{equation}\label{eqn:continuumL7:190}
\rho \PDSq{t}{\spacegrad \cross \Be} = (\lambda + \mu) \spacegrad \cross (\spacegrad (\spacegrad \cdot \Be)) + \mu \spacegrad^2 (\spacegrad \cross \Be).
\end{equation}

Writing

\begin{equation}\label{eqn:continuumL7:210}
\Bomega = \spacegrad \cross \Be,
\end{equation}

and observing that \(\spacegrad \cross \spacegrad f = 0\) (with \(f = \spacegrad \cdot \Be\)), we find

\begin{equation}\label{eqn:continuumL7:230}
\rho \PDSq{t}{\Bomega} = \mu \spacegrad^2 \Bomega.
\end{equation}

We call this the S-wave equation, and write \(C_T\) for the speed of this wave

\begin{equation}\label{eqn:continuumL7:250}
C_T^2 = \frac{\mu}{\rho},
\end{equation}

so that we have

\begin{equation}\label{eqn:continuumL7:270}
\PDSq{t}{\Bomega} = C_T^2 \spacegrad^2 \Bomega.
\end{equation}

Again, we can find nice animations of this on wikipedia \citep{wiki:swave}.

\section{Relative speeds of the p-waves and s-waves}

Taking ratios of the wave speeds we find

\begin{equation}\label{eqn:continuumL7:290}
\frac{C_L}{C_T} = \sqrt{\frac{ \lambda + 2 \mu}{\mu}} = \sqrt{ \frac{\lambda}{\mu} + 2}.
\end{equation}

Since both \(\lambda > 0\) and \(\mu > 0\), we have

\begin{equation}\label{eqn:continuumL7:310}
C_L > C_T.
\end{equation}

Divergence (p-waves) are faster than rotational (s-waves) waves.

In terms of the Poisson ratio \(\nu = \lambda/(2(\lambda + \mu))\), we find

\begin{equation}\label{eqn:continuumL7:330}
\frac{\mu}{\lambda} = \inv{2 \nu} - 1.
\end{equation}

we see that Poisson's ratio characterizes the speeds of the waves for the medium

\begin{equation}\label{eqn:continuumL7:350}
\frac{C_L}{C_T} = \sqrt{\frac{2(1-\nu)}{1 - 2\nu}}
\end{equation}

\section{Assuming a gradient plus curl representation} \index{gradient} \index{curl}

Let us assume that our displacement can be written in terms of a gradient and curl as we do for the electric field

\begin{equation}\label{eqn:continuumL7:370}
\Be = \spacegrad \phi + \spacegrad \cross \BH,
\end{equation}

Inserting this into \eqnref{eqn:continuumL6:290} we find

\begin{equation}\label{eqn:continuumL7:390}
\rho \PDSq{t}{(\spacegrad \phi + \spacegrad \cross \BH)} = (\lambda + \mu) \spacegrad (\spacegrad \cdot (\spacegrad \phi + \spacegrad \cross \BH)) + \mu \spacegrad^2 (\spacegrad \phi + \spacegrad \cross \BH).
\end{equation}

The first term on the RHS can be simplified.  First note that the divergence of the gradient is just a Laplacian

\begin{equation}\label{eqn:continuumL7:410}
\spacegrad \cdot \spacegrad \phi = \spacegrad^2 \phi,
\end{equation}

and then note that the divergence of a curl is zero
\begin{equation}\label{eqn:elasticWaves:510}
\begin{aligned}
\spacegrad \cdot (\spacegrad \cross \BH)
=
\partial_k (\partial_a H_b \epsilon_{abk}) =
0.
\end{aligned}
\end{equation}

The zero follows from the fact that the antisymmetric sum of symmetric partials is zero (assuming sufficient continuity).  Grouping terms we have

\begin{equation}\label{eqn:continuumL7:450}
\spacegrad
\left(
\rho \PDSq{t}{\phi} - (\lambda + 2\mu) \spacegrad^2 \phi
\right)
+
\spacegrad \cross
\left(
\rho \PDSq{t}{\BH} - \mu \spacegrad^2 \BH
\right)
= 0.
\end{equation}

When the material is infinite in scope, so that boundary value coupling is not a factor, we can write this as a set of independent P-wave and S-wave equations

\begin{equation}\label{eqn:continuumL7:470}
\rho \PDSq{t}{\phi} - (\lambda + 2\mu) \spacegrad^2 \phi = 0
\end{equation}

The P-wave is irrotational (curl free).

\begin{equation}\label{eqn:continuumL7:490}
\rho \PDSq{t}{\BH} - \mu \spacegrad^2 \BH = 0
\end{equation}

The S-wave is solenoidal (divergence free).

\section{A couple summarizing statements}

\begin{itemize}
\item
P-waves: irrotational.  Volume not preserved.
\item
S-waves: divergence free.  Shearing forces are present and volume is preserved.
\item
P-waves are faster than S-waves.
\end{itemize}

      % 
% 
% 
% Copyright � 2012 Peeter Joot
% All Rights Reserved
% 
% This file may be reproduced and distributed in whole or in part, without fee, subject to the following conditions:
% 
% o The copyright notice above and this permission notice must be preserved complete on all complete or partial copies.
% 
% o Any translation or derived work must be approved by the author in writing before distribution.
% 
% o If you distribute this work in part, instructions for obtaining the complete version of this file must be included, and a means for obtaining a complete version provided.
% 
% 
% Exceptions to these rules may be granted for academic purposes: Write to the author and ask.
% 
% 
% 
%%
% Copyright � 2015 Peeter Joot.  All Rights Reserved.
% Licenced as described in the file LICENSE under the root directory of this GIT repository.
%
\documentclass[]{eliblog}

\usepackage{amsmath}
\usepackage{mathpazo}

%
% shorthand for bold symbols, convenient for vectors and matrices
%
\newcommand{\Ba}[0]{\mathbf{a}}
\newcommand{\Bb}[0]{\mathbf{b}}
\newcommand{\Bc}[0]{\mathbf{c}}
\newcommand{\Bd}[0]{\mathbf{d}}
\newcommand{\Be}[0]{\mathbf{e}}
\newcommand{\Bf}[0]{\mathbf{f}}
\newcommand{\Bg}[0]{\mathbf{g}}
\newcommand{\Bh}[0]{\mathbf{h}}
\newcommand{\Bi}[0]{\mathbf{i}}
\newcommand{\Bj}[0]{\mathbf{j}}
\newcommand{\Bk}[0]{\mathbf{k}}
\newcommand{\Bl}[0]{\mathbf{l}}
\newcommand{\Bm}[0]{\mathbf{m}}
\newcommand{\Bn}[0]{\mathbf{n}}
\newcommand{\Bo}[0]{\mathbf{o}}
\newcommand{\Bp}[0]{\mathbf{p}}
\newcommand{\Bq}[0]{\mathbf{q}}
\newcommand{\Br}[0]{\mathbf{r}}
\newcommand{\Bs}[0]{\mathbf{s}}
\newcommand{\Bt}[0]{\mathbf{t}}
\newcommand{\Bu}[0]{\mathbf{u}}
\newcommand{\Bv}[0]{\mathbf{v}}
\newcommand{\Bw}[0]{\mathbf{w}}
\newcommand{\Bx}[0]{\mathbf{x}}
\newcommand{\By}[0]{\mathbf{y}}
\newcommand{\Bz}[0]{\mathbf{z}}
\newcommand{\BA}[0]{\mathbf{A}}
\newcommand{\BB}[0]{\mathbf{B}}
\newcommand{\BC}[0]{\mathbf{C}}
\newcommand{\BD}[0]{\mathbf{D}}
\newcommand{\BE}[0]{\mathbf{E}}
\newcommand{\BF}[0]{\mathbf{F}}
\newcommand{\BG}[0]{\mathbf{G}}
\newcommand{\BH}[0]{\mathbf{H}}
\newcommand{\BI}[0]{\mathbf{I}}
\newcommand{\BJ}[0]{\mathbf{J}}
\newcommand{\BK}[0]{\mathbf{K}}
\newcommand{\BL}[0]{\mathbf{L}}
\newcommand{\BM}[0]{\mathbf{M}}
\newcommand{\BN}[0]{\mathbf{N}}
\newcommand{\BO}[0]{\mathbf{O}}
\newcommand{\BP}[0]{\mathbf{P}}
\newcommand{\BQ}[0]{\mathbf{Q}}
\newcommand{\BR}[0]{\mathbf{R}}
\newcommand{\BS}[0]{\mathbf{S}}
\newcommand{\BT}[0]{\mathbf{T}}
\newcommand{\BU}[0]{\mathbf{U}}
\newcommand{\BV}[0]{\mathbf{V}}
\newcommand{\BW}[0]{\mathbf{W}}
\newcommand{\BX}[0]{\mathbf{X}}
\newcommand{\BY}[0]{\mathbf{Y}}
\newcommand{\BZ}[0]{\mathbf{Z}}

\newcommand{\Bzero}[0]{\mathbf{0}}
\newcommand{\Btheta}[0]{\boldsymbol{\theta}}
\newcommand{\Btau}[0]{\boldsymbol{\tau}}
\newcommand{\Bomega}[0]{\boldsymbol{\omega}}

%
% shorthand for unit vectors
%
\newcommand{\acap}[0]{\hat{\Ba}}
\newcommand{\bcap}[0]{\hat{\Bb}}
\newcommand{\ccap}[0]{\hat{\Bc}}
\newcommand{\dcap}[0]{\hat{\Bd}}
\newcommand{\ecap}[0]{\hat{\Be}}
\newcommand{\fcap}[0]{\hat{\Bf}}
\newcommand{\gcap}[0]{\hat{\Bg}}
\newcommand{\hcap}[0]{\hat{\Bh}}
\newcommand{\icap}[0]{\hat{\Bi}}
\newcommand{\jcap}[0]{\hat{\Bj}}
\newcommand{\kcap}[0]{\hat{\Bk}}
\newcommand{\lcap}[0]{\hat{\Bl}}
\newcommand{\mcap}[0]{\hat{\Bm}}
\newcommand{\ncap}[0]{\hat{\Bn}}
\newcommand{\ocap}[0]{\hat{\Bo}}
\newcommand{\pcap}[0]{\hat{\Bp}}
\newcommand{\qcap}[0]{\hat{\Bq}}
\newcommand{\rcap}[0]{\hat{\Br}}
\newcommand{\scap}[0]{\hat{\Bs}}
\newcommand{\tcap}[0]{\hat{\Bt}}
\newcommand{\ucap}[0]{\hat{\Bu}}
\newcommand{\vcap}[0]{\hat{\Bv}}
\newcommand{\wcap}[0]{\hat{\Bw}}
\newcommand{\xcap}[0]{\hat{\Bx}}
\newcommand{\ycap}[0]{\hat{\By}}
\newcommand{\zcap}[0]{\hat{\Bz}}
\newcommand{\thetacap}[0]{\hat{\Btheta}}

%
% to write R^n and C^n in a distinguishable fashion.  Perhaps change this
% to the double lined characters upon figuring out how to do so.
%
\newcommand{\C}[1]{$\mathbb{C}^{#1}$}
\newcommand{\R}[1]{$\mathbb{R}^{#1}$}

%
% various generally useful helpers
%

% derivative of #1 wrt. #2:
\newcommand{\D}[2] {\frac {d#2} {d#1}}

\newcommand{\inv}[1]{\frac{1}{#1}}
\newcommand{\cross}[0]{\times}

\newcommand{\abs}[1]{\lvert{#1}\rvert}
\newcommand{\norm}[1]{\lVert{#1}\rVert}
\newcommand{\innerprod}[2]{\langle{#1}, {#2}\rangle}
\newcommand{\dotprod}[2]{{#1} \cdot {#2}}
\newcommand{\bdotprod}[2]{\left({#1} \cdot {#2}\right)}
\newcommand{\crossprod}[2]{{#1} \cross {#2}}
\newcommand{\tripleprod}[3]{\dotprod{\left(\crossprod{#1}{#2}\right)}{#3}}

\DeclareMathOperator{\Proj}{Proj}
\DeclareMathOperator{\Span}{span}
\DeclareMathOperator{\Sgn}{sgn}
\DeclareMathOperator{\Area}{Area}
\DeclareMathOperator{\Volume}{Volume}

%
% A few miscellaneous things specific to this document
%
\newcommand{\crossop}[1]{\crossprod{#1}{}}

% R2 vector.
\newcommand{\VectorTwo}[2]{
\begin{bmatrix}
 {#1} \\
 {#2}
\end{bmatrix}
}

\newcommand{\VectorN}[1]{
\begin{bmatrix}
{#1}_1 \\
{#1}_2 \\
\vdots \\
{#1}_N \\
\end{bmatrix}
}

\newcommand{\DETuvij}[4]{
\begin{vmatrix}
 {#1}_{#3} & {#1}_{#4} \\
 {#2}_{#3} & {#2}_{#4}
\end{vmatrix}
}

\newcommand{\DETuvwijk}[6]{
\begin{vmatrix}
 {#1}_{#4} & {#1}_{#5} & {#1}_{#6} \\
 {#2}_{#4} & {#2}_{#5} & {#2}_{#6} \\
 {#3}_{#4} & {#3}_{#5} & {#3}_{#6}
\end{vmatrix}
}

\newcommand{\DETuvwxijkl}[8]{
\begin{vmatrix}
 {#1}_{#5} & {#1}_{#6} & {#1}_{#7} & {#1}_{#8} \\
 {#2}_{#5} & {#2}_{#6} & {#2}_{#7} & {#2}_{#8} \\
 {#3}_{#5} & {#3}_{#6} & {#3}_{#7} & {#3}_{#8} \\
 {#4}_{#5} & {#4}_{#6} & {#4}_{#7} & {#4}_{#8} \\
\end{vmatrix}
}

%\newcommand{\DETuvwxyijklm}[10]{
%\begin{vmatrix}
% {#1}_{#6} & {#1}_{#7} & {#1}_{#8} & {#1}_{#9} & {#1}_{#10} \\
% {#2}_{#6} & {#2}_{#7} & {#2}_{#8} & {#2}_{#9} & {#2}_{#10} \\
% {#3}_{#6} & {#3}_{#7} & {#3}_{#8} & {#3}_{#9} & {#3}_{#10} \\
% {#4}_{#6} & {#4}_{#7} & {#4}_{#8} & {#4}_{#9} & {#4}_{#10} \\
% {#5}_{#6} & {#5}_{#7} & {#5}_{#8} & {#5}_{#9} & {#5}_{#10}
%\end{vmatrix}
%}

% R3 vector.
\newcommand{\VectorThree}[3]{
\begin{bmatrix}
 {#1} \\
 {#2} \\
 {#3}
\end{bmatrix}
}



\author{Peeter Joot}
\email{peeter.joot@gmail.com}

%\documentclass[]{eliblogwidescreen}

\usepackage{amsmath}
\usepackage{mathpazo}

%
% shorthand for bold symbols, convenient for vectors and matrices
%
\newcommand{\Ba}[0]{\mathbf{a}}
\newcommand{\Bb}[0]{\mathbf{b}}
\newcommand{\Bc}[0]{\mathbf{c}}
\newcommand{\Bd}[0]{\mathbf{d}}
\newcommand{\Be}[0]{\mathbf{e}}
\newcommand{\Bf}[0]{\mathbf{f}}
\newcommand{\Bg}[0]{\mathbf{g}}
\newcommand{\Bh}[0]{\mathbf{h}}
\newcommand{\Bi}[0]{\mathbf{i}}
\newcommand{\Bj}[0]{\mathbf{j}}
\newcommand{\Bk}[0]{\mathbf{k}}
\newcommand{\Bl}[0]{\mathbf{l}}
\newcommand{\Bm}[0]{\mathbf{m}}
\newcommand{\Bn}[0]{\mathbf{n}}
\newcommand{\Bo}[0]{\mathbf{o}}
\newcommand{\Bp}[0]{\mathbf{p}}
\newcommand{\Bq}[0]{\mathbf{q}}
\newcommand{\Br}[0]{\mathbf{r}}
\newcommand{\Bs}[0]{\mathbf{s}}
\newcommand{\Bt}[0]{\mathbf{t}}
\newcommand{\Bu}[0]{\mathbf{u}}
\newcommand{\Bv}[0]{\mathbf{v}}
\newcommand{\Bw}[0]{\mathbf{w}}
\newcommand{\Bx}[0]{\mathbf{x}}
\newcommand{\By}[0]{\mathbf{y}}
\newcommand{\Bz}[0]{\mathbf{z}}
\newcommand{\BA}[0]{\mathbf{A}}
\newcommand{\BB}[0]{\mathbf{B}}
\newcommand{\BC}[0]{\mathbf{C}}
\newcommand{\BD}[0]{\mathbf{D}}
\newcommand{\BE}[0]{\mathbf{E}}
\newcommand{\BF}[0]{\mathbf{F}}
\newcommand{\BG}[0]{\mathbf{G}}
\newcommand{\BH}[0]{\mathbf{H}}
\newcommand{\BI}[0]{\mathbf{I}}
\newcommand{\BJ}[0]{\mathbf{J}}
\newcommand{\BK}[0]{\mathbf{K}}
\newcommand{\BL}[0]{\mathbf{L}}
\newcommand{\BM}[0]{\mathbf{M}}
\newcommand{\BN}[0]{\mathbf{N}}
\newcommand{\BO}[0]{\mathbf{O}}
\newcommand{\BP}[0]{\mathbf{P}}
\newcommand{\BQ}[0]{\mathbf{Q}}
\newcommand{\BR}[0]{\mathbf{R}}
\newcommand{\BS}[0]{\mathbf{S}}
\newcommand{\BT}[0]{\mathbf{T}}
\newcommand{\BU}[0]{\mathbf{U}}
\newcommand{\BV}[0]{\mathbf{V}}
\newcommand{\BW}[0]{\mathbf{W}}
\newcommand{\BX}[0]{\mathbf{X}}
\newcommand{\BY}[0]{\mathbf{Y}}
\newcommand{\BZ}[0]{\mathbf{Z}}

\newcommand{\Bzero}[0]{\mathbf{0}}
\newcommand{\Btheta}[0]{\boldsymbol{\theta}}
\newcommand{\Btau}[0]{\boldsymbol{\tau}}
\newcommand{\Bomega}[0]{\boldsymbol{\omega}}

%
% shorthand for unit vectors
%
\newcommand{\acap}[0]{\hat{\Ba}}
\newcommand{\bcap}[0]{\hat{\Bb}}
\newcommand{\ccap}[0]{\hat{\Bc}}
\newcommand{\dcap}[0]{\hat{\Bd}}
\newcommand{\ecap}[0]{\hat{\Be}}
\newcommand{\fcap}[0]{\hat{\Bf}}
\newcommand{\gcap}[0]{\hat{\Bg}}
\newcommand{\hcap}[0]{\hat{\Bh}}
\newcommand{\icap}[0]{\hat{\Bi}}
\newcommand{\jcap}[0]{\hat{\Bj}}
\newcommand{\kcap}[0]{\hat{\Bk}}
\newcommand{\lcap}[0]{\hat{\Bl}}
\newcommand{\mcap}[0]{\hat{\Bm}}
\newcommand{\ncap}[0]{\hat{\Bn}}
\newcommand{\ocap}[0]{\hat{\Bo}}
\newcommand{\pcap}[0]{\hat{\Bp}}
\newcommand{\qcap}[0]{\hat{\Bq}}
\newcommand{\rcap}[0]{\hat{\Br}}
\newcommand{\scap}[0]{\hat{\Bs}}
\newcommand{\tcap}[0]{\hat{\Bt}}
\newcommand{\ucap}[0]{\hat{\Bu}}
\newcommand{\vcap}[0]{\hat{\Bv}}
\newcommand{\wcap}[0]{\hat{\Bw}}
\newcommand{\xcap}[0]{\hat{\Bx}}
\newcommand{\ycap}[0]{\hat{\By}}
\newcommand{\zcap}[0]{\hat{\Bz}}
\newcommand{\thetacap}[0]{\hat{\Btheta}}

%
% to write R^n and C^n in a distinguishable fashion.  Perhaps change this
% to the double lined characters upon figuring out how to do so.
%
\newcommand{\C}[1]{$\mathbb{C}^{#1}$}
\newcommand{\R}[1]{$\mathbb{R}^{#1}$}

%
% various generally useful helpers
%

% derivative of #1 wrt. #2:
\newcommand{\D}[2] {\frac {d#2} {d#1}}

\newcommand{\inv}[1]{\frac{1}{#1}}
\newcommand{\cross}[0]{\times}

\newcommand{\abs}[1]{\lvert{#1}\rvert}
\newcommand{\norm}[1]{\lVert{#1}\rVert}
\newcommand{\innerprod}[2]{\langle{#1}, {#2}\rangle}
\newcommand{\dotprod}[2]{{#1} \cdot {#2}}
\newcommand{\bdotprod}[2]{\left({#1} \cdot {#2}\right)}
\newcommand{\crossprod}[2]{{#1} \cross {#2}}
\newcommand{\tripleprod}[3]{\dotprod{\left(\crossprod{#1}{#2}\right)}{#3}}

\DeclareMathOperator{\Proj}{Proj}
\DeclareMathOperator{\Span}{span}
\DeclareMathOperator{\Sgn}{sgn}
\DeclareMathOperator{\Area}{Area}
\DeclareMathOperator{\Volume}{Volume}

%
% A few miscellaneous things specific to this document
%
\newcommand{\crossop}[1]{\crossprod{#1}{}}

% R2 vector.
\newcommand{\VectorTwo}[2]{
\begin{bmatrix}
 {#1} \\
 {#2}
\end{bmatrix}
}

\newcommand{\VectorN}[1]{
\begin{bmatrix}
{#1}_1 \\
{#1}_2 \\
\vdots \\
{#1}_N \\
\end{bmatrix}
}

\newcommand{\DETuvij}[4]{
\begin{vmatrix}
 {#1}_{#3} & {#1}_{#4} \\
 {#2}_{#3} & {#2}_{#4}
\end{vmatrix}
}

\newcommand{\DETuvwijk}[6]{
\begin{vmatrix}
 {#1}_{#4} & {#1}_{#5} & {#1}_{#6} \\
 {#2}_{#4} & {#2}_{#5} & {#2}_{#6} \\
 {#3}_{#4} & {#3}_{#5} & {#3}_{#6}
\end{vmatrix}
}

\newcommand{\DETuvwxijkl}[8]{
\begin{vmatrix}
 {#1}_{#5} & {#1}_{#6} & {#1}_{#7} & {#1}_{#8} \\
 {#2}_{#5} & {#2}_{#6} & {#2}_{#7} & {#2}_{#8} \\
 {#3}_{#5} & {#3}_{#6} & {#3}_{#7} & {#3}_{#8} \\
 {#4}_{#5} & {#4}_{#6} & {#4}_{#7} & {#4}_{#8} \\
\end{vmatrix}
}

%\newcommand{\DETuvwxyijklm}[10]{
%\begin{vmatrix}
% {#1}_{#6} & {#1}_{#7} & {#1}_{#8} & {#1}_{#9} & {#1}_{#10} \\
% {#2}_{#6} & {#2}_{#7} & {#2}_{#8} & {#2}_{#9} & {#2}_{#10} \\
% {#3}_{#6} & {#3}_{#7} & {#3}_{#8} & {#3}_{#9} & {#3}_{#10} \\
% {#4}_{#6} & {#4}_{#7} & {#4}_{#8} & {#4}_{#9} & {#4}_{#10} \\
% {#5}_{#6} & {#5}_{#7} & {#5}_{#8} & {#5}_{#9} & {#5}_{#10}
%\end{vmatrix}
%}

% R3 vector.
\newcommand{\VectorThree}[3]{
\begin{bmatrix}
 {#1} \\
 {#2} \\
 {#3}
\end{bmatrix}
}



\author{Peeter Joot}
\email{peeter.joot@gmail.com}


%\chapter{PHY454H1S\\Continuum Mechanics.  Lecture 6: Compatibility condition and elastostatics.  Taught by Prof. K. Das.}
\section{Compatibility condition and elastostatics.}
\label{chap:continuumL6}
%\useCCL
\blogpage{http://sites.google.com/site/peeterjoot2/math2012/continuumL6.pdf}
%\date{Jan 27, 2012}
\revisionInfo{continuumL6.tex}

\beginArtWithToc
%\beginArtNoToc

%\section{Disclaimer.}
%
%Peeter's lecture notes from class.  May not be entirely coherent.

\section{Review: Elastostatics}

We've defined the strain tensor, where assuming the second order terms are ignored, was

\begin{equation}\label{eqn:continuumL6:10}
e_{ij} = 
\inv{2} \left( 
\PD{x_j}{e_i}
+ \PD{x_i}{e_j} \right).
\end{equation}

We've also defined a stress tensor defined implicitly as a divergence relationship using the force per unit volume $F_i$ in direction $i$

\begin{equation}\label{eqn:continuumL6:30}
\sigma_{ij} \leftrightarrow F_i = \PD{x_j}{\sigma_{ij}}.
\end{equation}

We've also discussed the constitutive relation, relating stress $\sigma_{ij}$ and strain $e_{ij}$.

We've also discussed linear constitutive relationships (Hooke's law).  

\section{2D strain.}

\begin{equation}\label{eqn:continuumL6:50}
e_{ij} = 
\begin{bmatrix}
e_{11} & e_{12} \\
e_{21} & e_{22}
\end{bmatrix}
\end{equation}

From \ref{eqn:continuumL6:10} we see that we have

\begin{align}\label{eqn:continuumL6:70}
e_{11} &= \PD{x_1}{e_1} \\
e_{22} &= \PD{x_2}{e_2} \\
e_{12} = e_{21} &= 
\inv{2} \left( 
\PD{x_1}{e_2}
+ \PD{x_2}{e_1} 
\right).
\end{align}

We have a relationship between these displacements (called the compatibility relationship), which is

\begin{equation}\label{eqn:continuumL6:110}
\boxed{
\PDSq{x_2}{e_{11}} +
\PDSq{x_1}{e_{22}} = 
2
\frac{\partial^2 e_{12}}{\partial x_1 \partial x_2}.
}
\end{equation}

We find this by straight computation

\begin{align*}
\PDSq{x_2}{e_{11}} 
&= 
\PDSq{x_2}{}\left( 
\PD{x_1}{e_1}
\right) \\
&=
\frac{\partial^3 e_1}{\partial x_1 \partial x_2^2},
\end{align*}

and

\begin{align*}
\PDSq{x_1}{e_{22}} 
&= 
\PDSq{x_1}{}\left( 
\PD{x_2}{e_2}
\right) \\
&= 
\frac{\partial^3 e_2}{\partial x_2 \partial x_1^2},
\end{align*}

Now, looking at the cross term we find

\begin{align*}
2 \frac{\partial^2 e_{12}}{\partial x_1 \partial x_2} 
&= 
\frac{\partial^2 e_{12}}{\partial x_1 \partial x_2} 
\left(
\PD{x_1}{e_2}
+ \PD{x_2}{e_1} 
\right) \\
&=
\left(
\frac{\partial^3 e_1}{\partial x_1 \partial x_2^2} 
+
\frac{\partial^3 e_2}{\partial x_2 \partial x_1^2} 
\right)
\end{align*}

We've found an interrelationship between the components of the strain

\begin{equation}\label{eqn:continuumL6:129}
2 \frac{\partial^2 e_{12}}{\partial x_1 \partial x_2} 
=
\PDSq{x_1}{e_{22}} 
+\PDSq{x_2}{e_{11}}.
\end{equation}

This relationship is called the compatibility condition, and ensures that we don't have a disjoint deformation of the form in figure (\ref{fig:continuumL6:continuumL6fig1}).

\begin{figure}[htp]
   \centering
   \includegraphics[totalheight=0.2\textheight]{continuumL6fig1}
   \caption{disjoint deformation illustrated.}\label{fig:continuumL6:continuumL6fig1}
\end{figure}

\section{3D strain.}

While we have 9 components in the tensor, not all of these are independent.  The sets above and below the diagonal can be related, as illustrated in figure (\ref{fig:continuumL6:continuumL6fig2}).

\begin{figure}[htp]
   \centering
   \includegraphics[totalheight=0.2\textheight]{continuumL6fig2}
   \caption{continuumL6fig2}\label{fig:continuumL6:continuumL6fig2}
\end{figure}

Here we have 6 relationships between the components of the strain tensor $e_{ij}$.  Deriving these will be assigned in the homework.

\section{Elastodynamics.  Elastic waves.}

Reading: Chapter I \S 7, chapter III (\S 22 - \S 26) of the text \cite{landau1960theory}.

Example: sound or water waves (i.e. waves in a solid or liquid material that comes back to its original position.)

\begin{definition}
\emph{(Elastic Wave)}
\label{dfn:continuumL6:10}
An elastic wave is a type of mechanical wave that propagates through or on the surface of a medium.  The elasticity of the material provides the restoring force (that returns the material to its original state).  The displacement and the restoring force are assumed to be linearly related.
\end{definition}

In symbols we say

\begin{equation}\label{eqn:continuumL6:130}
e_i(x_j, t) \quad \mbox{related to force},
\end{equation}

and specifically

\begin{equation}\label{eqn:continuumL6:150}
\rho \PDSq{t}{e_i} = F_i = \PD{x_j}{\sigma_{ij}}.
\end{equation}

This is just Newton's second law, $F = ma$, but expressed in terms of a unit volume.

Should we have an external body force (per unit volume) $f_i$ acting on the body then we must modify this, writing

\begin{equation}\label{eqn:continuumL6:170}
\boxed{
\rho \PDSq{t}{e_i} = \PD{x_j}{\sigma_{ij}} + f_i
}
\end{equation}

Note that we are separating out the ``original'' forces that produced the stress and strain on the object from any constant external forces that act on the body (i.e. a gravitational field).

With 

\begin{equation}\label{eqn:continuumL6:190}
e_{ij} = 
\inv{2} \left( 
\PD{x_j}{e_i}
+ \PD{x_i}{e_j} \right),
\end{equation}

we can expand the stress divergence, for the case of homogeneous deformation, in terms of the Lam\'e parameters

\begin{equation}\label{eqn:continuumL6:210}
\sigma_{ij} = \lambda e_{kk} \delta_{ij} + 2 \mu e_{ij}.
\end{equation}

We compute

\begin{align*}
\PD{x_j}{\sigma_{ij}}
&=
\lambda 
\PD{x_j}{
e_{kk}
}
\delta_{ij} + 2 \mu 
\PD{x_j}{
}
\inv{2} \left( 
\PD{x_j}{e_i}
+ \PD{x_i}{e_j} \right),
 \\
&=
\lambda 
\PD{x_i}{
e_{kk}
}
+ \mu 
\left(
\PDSq{x_j}{
e_{i}
}
+
\frac{\partial^2 e_{j} }{ \partial x_j \partial x_i}
\right) \\
&=
%\sum_k 
\lambda 
\PD{x_i}{
}
\PD{x_k}{e_k}
+ \mu 
\left(
\PDSq{x_j}{
e_{i}
}
+
\frac{\partial^2 e_{k} }{ \partial x_k \partial x_i}
\right) \\
&=
(\lambda + \mu)
\PD{x_i}{
}
\PD{x_k}{e_k}
+ \mu 
\PDSq{x_j}{
e_{i}
}
\end{align*}

%With 
%
%\begin{equation}\label{eqn:continuumL6:230}
%e_{kk} = e_{11} +e_{22} +e_{33}
%= 
%\PD{x_1}{e_1}
%+\PD{x_2}{e_2}
%+\PD{x_3}{e_3}
%\end{equation}
%
We find, for homogeneous deformations, that the force per unit volume on our element of mass, in the absence of external forces (the body forces), takes the form
%
%\begin{equation}\label{eqn:continuumL6:250}
%\PD{x_j}{\sigma_{ij}} = (\lambda + \mu) \frac{\partial^2 e_j}{\partial x_i \partial x_j}
%+ \mu
%\frac{\partial^2 e_i}
%{\partial x_j^2
%}
%\end{equation}

\begin{equation}\label{eqn:continuumL6:270}
\rho \PDSq{t}{e_i} = (\lambda + \mu) \frac{\partial^2 e_k}{\partial x_i \partial x_k}
+ \mu
\frac{\partial^2 e_i}
{\partial x_j^2
}.
\end{equation}

This can be seen to be equivalent to the vector relationship

\begin{equation}\label{eqn:continuumL6:290}
\boxed{
\rho \PDSq{t}{\Be} = (\lambda + \mu) \spacegrad (\spacegrad \cdot \Be) + \mu \spacegrad^2 \Be.
}
\end{equation}

TODO: What form do the stress and strain tensors take in vector form?

\EndArticle

      %
% Copyright � 2012 Peeter Joot.  All Rights Reserved.
% Licenced as described in the file LICENSE under the root directory of this GIT repository.
%

%
%
\section{Phasor description of elastic waves} \index{phasor} \index{elastic wave}
%\chapter{PHY454H1S\\Continuum Mechanics.  Lecture 8: Phasor description of elastic waves.  Fluid dynamics.  Taught by Prof. K. Das}
\label{chap:continuumL8}

%\section{Review.  Elastic wave equation}
%
%Starting with
%
%\begin{equation}\label{eqn:continuumL8:10}
%\rho \PDSq{t}{\Be} = (\lambda + \mu) \spacegrad (\spacegrad \cdot \Be) + \mu \spacegrad^2 \Be
%\end{equation}
%
%and applying a divergence operation we find
%
%\begin{align}\label{eqn:continuumL8:30}
%\rho \PDSq{t}{\theta} &= C_L^2 \spacegrad^2 \theta \\
%\theta &= \spacegrad \cdot \Be \\
%C_L^2 &= \frac{\lambda + 2\mu}{\rho}.
%\end{align}
%
%This is the P-wave equation.  Applying a curl operation we find
%
%\begin{align}\label{eqn:continuumL8:50}
%\rho \PDSq{t}{\Bomega} &= C_T^2 \spacegrad^2 \Bomega \\
%\Bomega &= \spacegrad \cross \Be \\
%C_T^2 &= \frac{\lambda + 2\mu}{\rho}.
%\end{align}
%
%This is the S-wave equation.  We also found that
%
%\begin{equation}\label{eqn:continuumL8:70}
%\frac{C_L}{C_T} > 1,
%\end{equation}
%
%and concluded that P waves are faster than S waves.  What we have not shown is that the P waves are longitudinal, and that the S waves are transverse.
%
%Assuming a gradient and curl description of our displacement
%
%\begin{equation}\label{eqn:continuumL8:90}
%\Be = \spacegrad \phi + \spacegrad \cross \BH = \BP + \BS,
%\end{equation}
%
%we found
%
%\begin{align}\label{eqn:continuumL8:110}
%(\lambda + 2 \mu) \spacegrad^2 \phi - \rho \PDSq{t}{\phi} &= 0 \\
%\mu \spacegrad^2 \BH - \rho \PDSq{t}{\BH} &= 0,
%\end{align}
%
%allowing us to separately solve for the P and the S wave solutions respectively.
%
Let us introduce a phasor representation (again following \S 22 of the text \citep{landau1960theory})

\begin{equation}\label{eqn:continuumL8:130}
\begin{aligned}
\phi &= A \exp\left( i ( \Bk \cdot \Bx - \omega t) \right) \\
\BH &= \BB \exp\left( i ( \Bk \cdot \Bx - \omega t) \right)
\end{aligned}
\end{equation}

Operating with the gradient we find

\begin{equation}\label{eqn:phasorWaveSolutions:230}
\begin{aligned}
\BP
&= \spacegrad \phi \\
&= \Be_k \partial_k A \exp\left( i ( \Bk \cdot \Bx - \omega t) \right) \\
&= \Be_k \partial_k A \exp\left( i ( k_m x_m - \omega t) \right) \\
&= \Be_k i k_k A \exp\left( i ( k_m x_m - \omega t) \right) \\
&= i \Bk A \exp\left( i ( \Bk \cdot \Bx - \omega t) \right) \\
&= i \Bk \phi
\end{aligned}
\end{equation}

We can also write

\begin{equation}\label{eqn:continuumL8:150}
\BP = \Bk \phi'
\end{equation}

where \(\phi'\) is the derivative of \(\phi\) ``with respect to its argument''.   Here argument must mean the entire phase \(\Bk \cdot \Bx - \omega t\).

\begin{equation}\label{eqn:continuumL8:170}
\phi' = \frac{ d\phi( \Bk \cdot \Bx - \omega t )}{ d(\Bk \cdot \Bx - \omega t) } = i \phi
\end{equation}

Actually, argument is a good label here, since we can use the word in the complex number sense.

For the curl term we find

\begin{equation}\label{eqn:phasorWaveSolutions:250}
\begin{aligned}
\BS
&= \spacegrad \cross \BH \\
&= \Be_a \partial_b H_c \epsilon_{a b c} \\
&= \Be_a \partial_b \epsilon_{a b c} B_c \exp\left( i ( \Bk \cdot \Bx - \omega t) \right) \\
&= \Be_a \partial_b \epsilon_{a b c} B_c \exp\left( i ( k_m x_m - \omega t) \right) \\
&= \Be_a i k_b \epsilon_{a b c} B_c \exp\left( i ( \Bk \cdot \Bx - \omega t) \right) \\
&= i \Bk \cross \BH
\end{aligned}
\end{equation}

Again writing
\begin{equation}\label{eqn:continuumL8:190}
\BH' = \frac{ d\BH( \Bk \cdot \Bx - \omega t )}{ d(\Bk \cdot \Bx - \omega t) } = i \BH
\end{equation}

we can write the S wave as

\begin{equation}\label{eqn:continuumL8:210}
\BS = \Bk \cross \BH'
\end{equation}

\section{Some wave types described}

The following wave types were noted, but not defined:

\begin{itemize}
\item Rayleigh wave.  This is discussed in \S 24 of the text (a wave that propagates near the surface of a body without penetrating into it).  Wikipedia has an illustration of one possible mode of propagation \citep{wiki:rayleighwave}.
\item Love wave.  These are not discussed in the text, but wikipedia \citep{wiki:lovewave} describes them as polarized shear waves (where the figure indicates that the shear displacements are perpendicular to the direction of propagation).
\end{itemize}

Some illustrations from the class notes were also shown.

% review this review text more carefully.  Is there anything actually worth keeping here.  Have tossed the strain and stress
% review text.  Should perhaps do that here too (or get them back and add them in as summaries as here.)
      % 
% 
% 
% Copyright © 2012 Peeter Joot
% All Rights Reserved
% 
% This file may be reproduced and distributed in whole or in part, without fee, subject to the following conditions:
% 
% o The copyright notice above and this permission notice must be preserved complete on all complete or partial copies.
% 
% o Any translation or derived work must be approved by the author in writing before distribution.
% 
% o If you distribute this work in part, instructions for obtaining the complete version of this file must be included, and a means for obtaining a complete version provided.
% 
% 
% Exceptions to these rules may be granted for academic purposes: Write to the author and ask.
% 
% 
% 
%\chapter{Displacement propagation}

It was argued that the equation relating the time evolution of a one of the vector displacement coordinates was given by

\begin{equation}\label{eqn:continuumElasticityReview:570}
\rho \PDSq{t}{u_i} = \PD{x_j}{\sigma_{ij}} + f_i,
\end{equation}

where the divergence term $\PDi{x_j}{\sigma_{ij}}$ is the internal force per unit volume on the object and $f_i$ is the external force.  Employing the constitutive relation we showed that this can be expanded as

\begin{equation}\label{eqn:continuumElasticityReview:590}
\rho \PDSq{t}{u_i} = (\lambda + \mu) \frac{\partial^2 u_k}{\partial x_i \partial x_k}
+ \mu
\frac{\partial^2 u_i}
{\partial x_j^2
},
\end{equation}

or in vector form

\begin{equation}\label{eqn:continuumElasticityReview:610}
\rho \PDSq{t}{\Bu} = (\lambda + \mu) \spacegrad (\spacegrad \cdot \Bu) + \mu \spacegrad^2 \Bu.
\end{equation}

\section{Equilibrium}

When a body is in static equilibrium \ref{eqn:continuumElasticityReview:570} reduces to just a simple force balance

\begin{equation}\label{eqn:continuumFluidsReviewXX:1090b}
f_i = - \PD{x_j}{\sigma_{ij}}.
\end{equation}

In particular, if there are no external forces then all of these divergences must be zero.  As an example, suppose that the state of a body is given by

\begin{equation}\label{eqn:continuumFluidsReviewXX:3130}
\begin{aligned}
\sigma_{11} &= A x^4 y^3 \\
\sigma_{22} &= 3 B x^2 y^5 \\
\sigma_{12} &= -C x^3 y^4
\end{aligned}
\end{equation}

We can determine the constants $A$, $B$ and $C$ so that the body is in equilibrium (2011 Final Exam question II).  We have

\begin{equation}\label{eqn:continuumFluidsReviewXX:3150}
\begin{aligned}
0 
&= \PD{x_j}{\sigma_{1j}} \\
&= 
\PD{x}{\sigma_{11}} + \PD{y}{\sigma_{12}} \\
&= 4 A x^3 y^3 - 4 C x^3 y^3,
\end{aligned}
\end{equation}

and

\begin{equation}\label{eqn:continuumFluidsReviewXX:3170}
\begin{aligned}
0 
&= \PD{x_j}{\sigma_{2j}} \\
&= 
\PD{x}{\sigma_{21}} + \PD{y}{\sigma_{22}} \\
&= -3 C x^2 y^4 + 15 B x^2 y^4
\end{aligned}
\end{equation}

We must then have
\begin{equation}\label{eqn:continuumFluidsReviewXX:3190}
\begin{aligned}
0 &= A - C \\
0 &= -C + 5 B
\end{aligned}
\end{equation} 

Or

\begin{equation}\label{eqn:continuumFluidsReviewXX:3210}
\begin{aligned}
A &= C \\
B &= \frac{C}{5}.
\end{aligned}
\end{equation}

Also asked on last years final was an explaination of how the strain energy of tectonic plates causes Tsunami.  The root cause of the Tsunami is the earthquake under the body of water.  Once that earthquake occurs we'll have a body wave in the mantle, which will trigger a much more destructive (higher amplitude) surface wave (probably of the Rayleigh type).  Looking back to the connection with strain energy, we see that once we have a change in the strain divergence, we'll have to have a restoring force to put things back in equilibrium.  That restoring force can come either from the surrounding mantle or the fluid above it, and it's that fluid restoring force that induces the wave as a side effect.

\section{P-waves}

Operating on \ref{eqn:continuumElasticityReview:610} with the divergence operator, and writing $\Theta = \spacegrad \cdot \Bu$, a quantity that was our relative change in volume in the diagonal strain basis, we were able to find this divergence obeys a wave equation

\begin{equation}\label{eqn:continuumElasticityReview:630}
\PDSq{t}{\Theta} = \frac{\lambda + 2 \mu}{\rho} \spacegrad^2 \Theta.
\end{equation}

We called these P-waves.

\section{S-waves}

Similarly, operating on \ref{eqn:continuumElasticityReview:610} with the curl operator, and writing $\Bomega = \spacegrad \cross \Bu$, we were able to find this curl also obeys a wave equation

\begin{equation}\label{eqn:continuumElasticityReview:650}
\rho \PDSq{t}{\Bomega} = \mu \spacegrad^2 \Bomega.
\end{equation}

These we called S-waves.  We also noted that the (transverse) compression waves (P-waves) with speed $C_T = \sqrt{\mu/\rho}$, traveled faster than the (longitudinal) vorticity (S) waves with speed $C_L = \sqrt{(\lambda + 2 \mu)/\rho}$ since $\lambda > 0$ and $\mu > 0$, and 

\begin{equation}\label{eqn:continuumElasticityReview:670}
\frac{C_L}{C_T} = \sqrt{\frac{ \lambda + 2 \mu}{\mu}} = \sqrt{ \frac{\lambda}{\mu} + 2}.
\end{equation}

\section{Scalar and vector potential representation.}

Assuming a vector displacement representation with gradient and curl components

\begin{equation}\label{eqn:continuumElasticityReview:690}
\Bu = \spacegrad \phi + \spacegrad \cross \BH,
\end{equation}

We found that the displacement time evolution equation split nicely into curl free and divergence free terms

\begin{equation}\label{eqn:continuumElasticityReview:710}
\spacegrad
\left(
\rho \PDSq{t}{\phi} - (\lambda + 2\mu) \spacegrad^2 \phi
\right)
+
\spacegrad \cross
\left(
\rho \PDSq{t}{\BH} - \mu \spacegrad^2 \BH
\right)
= 0.
\end{equation}

When neglecting boundary value effects this could be written as a pair of independent equations

\begin{subequations}
\begin{equation}\label{eqn:continuumElasticityReview:730}
\rho \PDSq{t}{\phi} - (\lambda + 2\mu) \spacegrad^2 \phi = 0
\end{equation}
\begin{equation}\label{eqn:continuumElasticityReview:750}
\rho \PDSq{t}{\BH} - \mu \spacegrad^2 \BH
= 0.
\end{equation}
\end{subequations}

This are the irrotational (curl free) P-wave and solenoidal (divergence free) S-wave equations respectively.

%This theory led to no actual calculation work, just a few videos that illustrated what we'd presumably be able to calculate if we were to attempt to apply these concepts.

\section{Phasor description.}

It was mentioned that we could assume a phasor representation for our potentials, writing

\begin{subequations}
\begin{equation}\label{eqn:continuumElasticityReview:770}
\phi = A \exp\left( i ( \Bk \cdot \Bx - \omega t) \right) 
\end{equation}
\begin{equation}\label{eqn:continuumElasticityReview:790}
\BH = \BB \exp\left( i ( \Bk \cdot \Bx - \omega t) \right)
\end{equation}
\end{subequations}

finding

\begin{equation}\label{eqn:continuumElasticityReview:810}
\Bu = i \Bk \phi + i \Bk \cross \BH.
\end{equation}

We did nothing with neither the potential nor the phasor theory for solid displacement time evolution, and presumably won't on the exam either.

\section{Some wave types}

Some time was spent on qualitative descriptions and review of descriptions for solutions of the time evolution elasticity equations, despite the fact that we didn't actually attempt to find or analyze any of these solutions

\begin{itemize}
\item P-waves \cite{wiki:pwave}.  Irrotational, non volume preserving body wave.
\item S-waves \cite{wiki:swave}.  Divergence free body wave.  Shearing forces are present and volume is preserved (slower than S-waves)
\item Rayleigh wave \cite{wiki:rayleighwave}.  A surface wave that propagates near the surface of a body without penetrating into it.  It's pointed out in the class notes in the seismogram figure that these, while moving slower than the P (primary) or S (secondary) waves, have larger amplitude and are therefore the most destructive.
\item Love wave \cite{wiki:lovewave}.  A polarized shear surface wave with the shear displacements moving perpendicular to the direction of propagation.
\end{itemize}

For reasons that aren't clear both the midterm and last years final ask us to spew this sort of stuff (instead of actually trying to do something analytic associated with them).

      %
% Copyright � 2012 Peeter Joot.  All Rights Reserved.
% Licenced as described in the file LICENSE under the root directory of this GIT repository.
%

%
%
\makeoproblem{\(\BP\)-waves, \(\BS\)-waves, and Love-waves}{problem:elastic:displacements:midtermQ1a}
{2012 midterm, question 1a}
{
Show that in \(\BP\)-waves the divergence of the displacement vector represents a measure of the relative change in the volume of the body.
} % makeoproblem

\makeanswer{problem:elastic:displacements:midtermQ1a}{
The \(\BP\)-wave equation was a result of operating on the displacement equation with the divergence operator

\begin{equation}\label{eqn:continuumMidTermReflection:10}
\spacegrad \cdot \left(
\rho \PDSq{t}{\Be} = (\lambda + \mu) \spacegrad (\spacegrad \cdot \Be) + \mu \spacegrad^2 \Be
\right)
\end{equation}

we obtain

\begin{equation}\label{eqn:continuumMidTermReflection:30}
\PDSq{t}{} \left( \spacegrad \cdot \Be \right) = \frac{\lambda + 2 \mu}{\rho} \spacegrad^2 (\spacegrad \cdot \Be).
\end{equation}

We have a wave equation where the ``waving'' quantity is \(\Theta = \spacegrad \cdot \Be\).  Explicitly

\begin{equation}\label{eqn:problems:3230}
\begin{aligned}
\Theta
&= \spacegrad \cdot \Be \\
&=
\PD{x}{e_1}
+\PD{y}{e_2}
+\PD{z}{e_3}
\end{aligned}
\end{equation}

Recall that, in a coordinate basis for which the strain \(e_{ij}\) is diagonal we have

\begin{equation}\label{eqn:continuumMidTermReflection:50}
\begin{aligned}
dx' &= \sqrt{1 + 2 e_{11}} dx \\
dy' &= \sqrt{1 + 2 e_{22}} dy \\
dz' &= \sqrt{1 + 2 e_{33}} dz.
\end{aligned}
\end{equation}

Expanding in Taylor series to \(O(1)\) we have for \(i = 1, 2, 3\) (no sum)

\begin{equation}\label{eqn:continuumMidTermReflection:70}
dx_i' \approx (1 + e_{ii}) dx_i.
\end{equation}

so the displaced volume is

\begin{equation}\label{eqn:problems:3250}
\begin{aligned}
dV' &=
dx_1
dx_2
dx_3
(1 + e_{11})
(1 + e_{22})
(1 + e_{33}) \\
&=
dx_1
dx_2
dx_3
( 1  + e_{11} + e_{22} + e_{33} + O(e_{kk}^2) )
\end{aligned}
\end{equation}

Since

\begin{equation}\label{eqn:continuumMidTermReflection:90}
\begin{aligned}
e_{11} &= \inv{2} \left( \PD{x}{e_1} +\PD{x}{e_1} \right) = \PD{x}{e_1} \\
e_{22} &= \inv{2} \left( \PD{y}{e_2} +\PD{y}{e_2} \right) = \PD{y}{e_2} \\
e_{33} &= \inv{2} \left( \PD{z}{e_3} +\PD{z}{e_3} \right) = \PD{z}{e_3}
\end{aligned}
\end{equation}

We have

\begin{equation}\label{eqn:continuumMidTermReflection:110}
dV' = (1 + \spacegrad \cdot \Be) dV,
\end{equation}

or

\begin{equation}\label{eqn:continuumMidTermReflection:130}
\frac{dV' - dV}{dV} = \spacegrad \cdot \Be
\end{equation}

The relative change in volume can therefore be expressed as the divergence of \(\Be\), the displacement vector, and it is this relative volume change that is ``waving'' in the \(\BP\)-wave equation as illustrated in the following \cref{fig:continuumMidtermReflection:continuumMidtermReflectionFig1} sample 1D compression wave

\imageFigure{../../figures/phy454/midtermReflectionA_1D_compression_waveFig1}{A 1D compression wave}{fig:continuumMidtermReflection:continuumMidtermReflectionFig1}{0.2}
} % end answer

\makeoproblem{Classify \(\BP\)-waves and \(\BS\)-waves as longitudinal or transverse}{problem:elastic:displacements:midtermQ1b}
{2012 midterm, question 1b}
{
Between a \(\BP\)-wave and an \(\BS\)-wave which one is longitudinal and which one is transverse?
} % makeoproblem

\makeanswer{problem:elastic:displacements:midtermQ1b}{
\(\BP\)-waves are longitudinal.

\(\BS\)-waves are transverse.
} % end answer

\makeoproblem{Speed of \(\BP\)-waves and \(\BS\)-waves}{problem:elastic:displacements:midtermQ1c}
{2012 midterm, question 1c}
{
Whose speed is higher?
} % makeoproblem

\makeanswer{problem:elastic:displacements:midtermQ1c}{
From the (midterm) formula sheet we have

\begin{equation}\label{eqn:problems:3270}
\begin{aligned}
\left( \frac{c_L}{c_T} \right)^2
&= \frac{ \lambda + 2 \mu}{\rho} \frac{\rho}{\mu}  \\
&= \frac{\lambda}{\mu} + 2  \\
&> 1
\end{aligned}
\end{equation}

so \(\BP\)-waves travel faster than \(\BS\)-waves.
} % end answer

\makeoproblem{Love waves}{problem:elastic:displacements:midtermQ1d}
{2012 midterm, question 1d}
{
Is Love wave a body wave or a surface wave?
} % makeoproblem

\makeanswer{problem:elastic:displacements:midtermQ1d}{
Love waves are surface waves, traveling in a medium that can slide on top of another surface.  They are characterized by shear displacements perpendicular to the direction of propagation.

Reviewing for the final I see that I had answered this wrong, and have corrected it.  I had described a Rayleigh wave (also a surface wave).  A Rayleigh wave is characterized by vorticity rotating backwards compared to the direction of propagation as shown in \cref{fig:continuumMidtermReflection:continuumMidtermReflectionFig2}


\pdfTexFigure{../../figures/phy454/continuumMidtermReflectionFig2.pdf_tex}{Rayleigh wave illustrated}{fig:continuumMidtermReflection:continuumMidtermReflectionFig2}{0.6}
} % end answer

\makeproblem{Equilibrium}{problem:elastic:displacements:exampractiseEquilibrium}{
Suppose that the state of a body is given by

\begin{dmath}\label{eqn:continuumFluidsReviewXX:3130}
\sigma_{11} = A x^4 y^3
\sigma_{22} = 3 B x^2 y^5
\sigma_{12} = -C x^3 y^4
\end{dmath}

Determine the constants \(A\), \(B\) and \(C\) so that the body is in equilibrium (2011 Final Exam question II).
} % makeproblem

\makeanswer{problem:elastic:displacements:exampractiseEquilibrium}{
We have

\begin{dmath}\label{eqn:continuumFluidsReviewXX:3150}
0
= \PD{x_j}{\sigma_{1j}}
=
\PD{x}{\sigma_{11}} + \PD{y}{\sigma_{12}}
= 4 A x^3 y^3 - 4 C x^3 y^3,
\end{dmath}

and


\begin{dmath}\label{eqn:continuumFluidsReviewXX:3170}
0
= \PD{x_j}{\sigma_{2j}}
=
\PD{x}{\sigma_{21}} + \PD{y}{\sigma_{22}}
= -3 C x^2 y^4 + 15 B x^2 y^4
\end{dmath}

We must then have

\begin{dmath}\label{eqn:continuumFluidsReviewXX:3190}
0 = A - C
0 = -C + 5 B
\end{dmath}

Or


\begin{dmath}\label{eqn:continuumFluidsReviewXX:3210}
A = C
B = \frac{C}{5}.
\end{dmath}
} % end answer

\makeoproblem{Tsunami}{problem:elastic:displacements:exampractiseTsunami}
{2011 final exam}
{
Explain how the strain energy of tectonic plates causes Tsunami.
} % makeoproblem

\makeanswer{problem:elastic:displacements:exampractiseTsunami}{
The root cause of the Tsunami is the earthquake under the body of water.  Once that earthquake occurs we will have a body wave in the mantle, which will trigger a much more destructive (higher amplitude) surface wave (probably of the Rayleigh type).  Looking back to the connection with strain energy, we see that once we have a change in the strain divergence, we will have to have a restoring force to put things back in equilibrium.  That restoring force can come either from the surrounding mantle or the fluid above it, and it is that fluid restoring force that induces the wave as a side effect.
} % end answer

\FIXME{This is from the 2012 midterm.  We never got any real problems on elastic waves.  My preference would have been for actual problems that require solutions to the wave equations under various conditions.  If we then examined those solutions and characterized them (Love, Rayleigh, ...) we would not just have a requirement to restate memorized descriptive stuff, a task of little value.}

\makeproblem{Wave equation solutions}{problem:elastic:displacements:placeholderWaveEquation}{

PLACEHOLDER.

\FIXME{We never did get any homework assignments with actual problems where we find Rayleigh or Love wave solutions, so that we could get a feel for how to apply the formalism.  This would be a good place to put some.}
} % makeproblem


%\part{Fluid dynamics}
   %
%
%
% Copyright � 2012 Peeter Joot
% All Rights Reserved
%
% This file may be reproduced and distributed in whole or in part, without fee, subject to the following conditions:
%
% o The copyright notice above and this permission notice must be preserved complete on all complete or partial copies.
%
% o Any translation or derived work must be approved by the author in writing before distribution.
%
% o If you distribute this work in part, instructions for obtaining the complete version of this file must be included, and a means for obtaining a complete version provided.
%
%
% Exceptions to these rules may be granted for academic purposes: Write to the author and ask.
%
%
%
%
% Copyright � 2015 Peeter Joot.  All Rights Reserved.
% Licenced as described in the file LICENSE under the root directory of this GIT repository.
%
\documentclass[]{eliblog}

\usepackage{amsmath}
\usepackage{mathpazo}

%
% shorthand for bold symbols, convenient for vectors and matrices
%
\newcommand{\Ba}[0]{\mathbf{a}}
\newcommand{\Bb}[0]{\mathbf{b}}
\newcommand{\Bc}[0]{\mathbf{c}}
\newcommand{\Bd}[0]{\mathbf{d}}
\newcommand{\Be}[0]{\mathbf{e}}
\newcommand{\Bf}[0]{\mathbf{f}}
\newcommand{\Bg}[0]{\mathbf{g}}
\newcommand{\Bh}[0]{\mathbf{h}}
\newcommand{\Bi}[0]{\mathbf{i}}
\newcommand{\Bj}[0]{\mathbf{j}}
\newcommand{\Bk}[0]{\mathbf{k}}
\newcommand{\Bl}[0]{\mathbf{l}}
\newcommand{\Bm}[0]{\mathbf{m}}
\newcommand{\Bn}[0]{\mathbf{n}}
\newcommand{\Bo}[0]{\mathbf{o}}
\newcommand{\Bp}[0]{\mathbf{p}}
\newcommand{\Bq}[0]{\mathbf{q}}
\newcommand{\Br}[0]{\mathbf{r}}
\newcommand{\Bs}[0]{\mathbf{s}}
\newcommand{\Bt}[0]{\mathbf{t}}
\newcommand{\Bu}[0]{\mathbf{u}}
\newcommand{\Bv}[0]{\mathbf{v}}
\newcommand{\Bw}[0]{\mathbf{w}}
\newcommand{\Bx}[0]{\mathbf{x}}
\newcommand{\By}[0]{\mathbf{y}}
\newcommand{\Bz}[0]{\mathbf{z}}
\newcommand{\BA}[0]{\mathbf{A}}
\newcommand{\BB}[0]{\mathbf{B}}
\newcommand{\BC}[0]{\mathbf{C}}
\newcommand{\BD}[0]{\mathbf{D}}
\newcommand{\BE}[0]{\mathbf{E}}
\newcommand{\BF}[0]{\mathbf{F}}
\newcommand{\BG}[0]{\mathbf{G}}
\newcommand{\BH}[0]{\mathbf{H}}
\newcommand{\BI}[0]{\mathbf{I}}
\newcommand{\BJ}[0]{\mathbf{J}}
\newcommand{\BK}[0]{\mathbf{K}}
\newcommand{\BL}[0]{\mathbf{L}}
\newcommand{\BM}[0]{\mathbf{M}}
\newcommand{\BN}[0]{\mathbf{N}}
\newcommand{\BO}[0]{\mathbf{O}}
\newcommand{\BP}[0]{\mathbf{P}}
\newcommand{\BQ}[0]{\mathbf{Q}}
\newcommand{\BR}[0]{\mathbf{R}}
\newcommand{\BS}[0]{\mathbf{S}}
\newcommand{\BT}[0]{\mathbf{T}}
\newcommand{\BU}[0]{\mathbf{U}}
\newcommand{\BV}[0]{\mathbf{V}}
\newcommand{\BW}[0]{\mathbf{W}}
\newcommand{\BX}[0]{\mathbf{X}}
\newcommand{\BY}[0]{\mathbf{Y}}
\newcommand{\BZ}[0]{\mathbf{Z}}

\newcommand{\Bzero}[0]{\mathbf{0}}
\newcommand{\Btheta}[0]{\boldsymbol{\theta}}
\newcommand{\Btau}[0]{\boldsymbol{\tau}}
\newcommand{\Bomega}[0]{\boldsymbol{\omega}}

%
% shorthand for unit vectors
%
\newcommand{\acap}[0]{\hat{\Ba}}
\newcommand{\bcap}[0]{\hat{\Bb}}
\newcommand{\ccap}[0]{\hat{\Bc}}
\newcommand{\dcap}[0]{\hat{\Bd}}
\newcommand{\ecap}[0]{\hat{\Be}}
\newcommand{\fcap}[0]{\hat{\Bf}}
\newcommand{\gcap}[0]{\hat{\Bg}}
\newcommand{\hcap}[0]{\hat{\Bh}}
\newcommand{\icap}[0]{\hat{\Bi}}
\newcommand{\jcap}[0]{\hat{\Bj}}
\newcommand{\kcap}[0]{\hat{\Bk}}
\newcommand{\lcap}[0]{\hat{\Bl}}
\newcommand{\mcap}[0]{\hat{\Bm}}
\newcommand{\ncap}[0]{\hat{\Bn}}
\newcommand{\ocap}[0]{\hat{\Bo}}
\newcommand{\pcap}[0]{\hat{\Bp}}
\newcommand{\qcap}[0]{\hat{\Bq}}
\newcommand{\rcap}[0]{\hat{\Br}}
\newcommand{\scap}[0]{\hat{\Bs}}
\newcommand{\tcap}[0]{\hat{\Bt}}
\newcommand{\ucap}[0]{\hat{\Bu}}
\newcommand{\vcap}[0]{\hat{\Bv}}
\newcommand{\wcap}[0]{\hat{\Bw}}
\newcommand{\xcap}[0]{\hat{\Bx}}
\newcommand{\ycap}[0]{\hat{\By}}
\newcommand{\zcap}[0]{\hat{\Bz}}
\newcommand{\thetacap}[0]{\hat{\Btheta}}

%
% to write R^n and C^n in a distinguishable fashion.  Perhaps change this
% to the double lined characters upon figuring out how to do so.
%
\newcommand{\C}[1]{$\mathbb{C}^{#1}$}
\newcommand{\R}[1]{$\mathbb{R}^{#1}$}

%
% various generally useful helpers
%

% derivative of #1 wrt. #2:
\newcommand{\D}[2] {\frac {d#2} {d#1}}

\newcommand{\inv}[1]{\frac{1}{#1}}
\newcommand{\cross}[0]{\times}

\newcommand{\abs}[1]{\lvert{#1}\rvert}
\newcommand{\norm}[1]{\lVert{#1}\rVert}
\newcommand{\innerprod}[2]{\langle{#1}, {#2}\rangle}
\newcommand{\dotprod}[2]{{#1} \cdot {#2}}
\newcommand{\bdotprod}[2]{\left({#1} \cdot {#2}\right)}
\newcommand{\crossprod}[2]{{#1} \cross {#2}}
\newcommand{\tripleprod}[3]{\dotprod{\left(\crossprod{#1}{#2}\right)}{#3}}

\DeclareMathOperator{\Proj}{Proj}
\DeclareMathOperator{\Span}{span}
\DeclareMathOperator{\Sgn}{sgn}
\DeclareMathOperator{\Area}{Area}
\DeclareMathOperator{\Volume}{Volume}

%
% A few miscellaneous things specific to this document
%
\newcommand{\crossop}[1]{\crossprod{#1}{}}

% R2 vector.
\newcommand{\VectorTwo}[2]{
\begin{bmatrix}
 {#1} \\
 {#2}
\end{bmatrix}
}

\newcommand{\VectorN}[1]{
\begin{bmatrix}
{#1}_1 \\
{#1}_2 \\
\vdots \\
{#1}_N \\
\end{bmatrix}
}

\newcommand{\DETuvij}[4]{
\begin{vmatrix}
 {#1}_{#3} & {#1}_{#4} \\
 {#2}_{#3} & {#2}_{#4}
\end{vmatrix}
}

\newcommand{\DETuvwijk}[6]{
\begin{vmatrix}
 {#1}_{#4} & {#1}_{#5} & {#1}_{#6} \\
 {#2}_{#4} & {#2}_{#5} & {#2}_{#6} \\
 {#3}_{#4} & {#3}_{#5} & {#3}_{#6}
\end{vmatrix}
}

\newcommand{\DETuvwxijkl}[8]{
\begin{vmatrix}
 {#1}_{#5} & {#1}_{#6} & {#1}_{#7} & {#1}_{#8} \\
 {#2}_{#5} & {#2}_{#6} & {#2}_{#7} & {#2}_{#8} \\
 {#3}_{#5} & {#3}_{#6} & {#3}_{#7} & {#3}_{#8} \\
 {#4}_{#5} & {#4}_{#6} & {#4}_{#7} & {#4}_{#8} \\
\end{vmatrix}
}

%\newcommand{\DETuvwxyijklm}[10]{
%\begin{vmatrix}
% {#1}_{#6} & {#1}_{#7} & {#1}_{#8} & {#1}_{#9} & {#1}_{#10} \\
% {#2}_{#6} & {#2}_{#7} & {#2}_{#8} & {#2}_{#9} & {#2}_{#10} \\
% {#3}_{#6} & {#3}_{#7} & {#3}_{#8} & {#3}_{#9} & {#3}_{#10} \\
% {#4}_{#6} & {#4}_{#7} & {#4}_{#8} & {#4}_{#9} & {#4}_{#10} \\
% {#5}_{#6} & {#5}_{#7} & {#5}_{#8} & {#5}_{#9} & {#5}_{#10}
%\end{vmatrix}
%}

% R3 vector.
\newcommand{\VectorThree}[3]{
\begin{bmatrix}
 {#1} \\
 {#2} \\
 {#3}
\end{bmatrix}
}



\author{Peeter Joot}
\email{peeter.joot@gmail.com}

%\documentclass[]{eliblogwidescreen}

\usepackage{amsmath}
\usepackage{mathpazo}

%
% shorthand for bold symbols, convenient for vectors and matrices
%
\newcommand{\Ba}[0]{\mathbf{a}}
\newcommand{\Bb}[0]{\mathbf{b}}
\newcommand{\Bc}[0]{\mathbf{c}}
\newcommand{\Bd}[0]{\mathbf{d}}
\newcommand{\Be}[0]{\mathbf{e}}
\newcommand{\Bf}[0]{\mathbf{f}}
\newcommand{\Bg}[0]{\mathbf{g}}
\newcommand{\Bh}[0]{\mathbf{h}}
\newcommand{\Bi}[0]{\mathbf{i}}
\newcommand{\Bj}[0]{\mathbf{j}}
\newcommand{\Bk}[0]{\mathbf{k}}
\newcommand{\Bl}[0]{\mathbf{l}}
\newcommand{\Bm}[0]{\mathbf{m}}
\newcommand{\Bn}[0]{\mathbf{n}}
\newcommand{\Bo}[0]{\mathbf{o}}
\newcommand{\Bp}[0]{\mathbf{p}}
\newcommand{\Bq}[0]{\mathbf{q}}
\newcommand{\Br}[0]{\mathbf{r}}
\newcommand{\Bs}[0]{\mathbf{s}}
\newcommand{\Bt}[0]{\mathbf{t}}
\newcommand{\Bu}[0]{\mathbf{u}}
\newcommand{\Bv}[0]{\mathbf{v}}
\newcommand{\Bw}[0]{\mathbf{w}}
\newcommand{\Bx}[0]{\mathbf{x}}
\newcommand{\By}[0]{\mathbf{y}}
\newcommand{\Bz}[0]{\mathbf{z}}
\newcommand{\BA}[0]{\mathbf{A}}
\newcommand{\BB}[0]{\mathbf{B}}
\newcommand{\BC}[0]{\mathbf{C}}
\newcommand{\BD}[0]{\mathbf{D}}
\newcommand{\BE}[0]{\mathbf{E}}
\newcommand{\BF}[0]{\mathbf{F}}
\newcommand{\BG}[0]{\mathbf{G}}
\newcommand{\BH}[0]{\mathbf{H}}
\newcommand{\BI}[0]{\mathbf{I}}
\newcommand{\BJ}[0]{\mathbf{J}}
\newcommand{\BK}[0]{\mathbf{K}}
\newcommand{\BL}[0]{\mathbf{L}}
\newcommand{\BM}[0]{\mathbf{M}}
\newcommand{\BN}[0]{\mathbf{N}}
\newcommand{\BO}[0]{\mathbf{O}}
\newcommand{\BP}[0]{\mathbf{P}}
\newcommand{\BQ}[0]{\mathbf{Q}}
\newcommand{\BR}[0]{\mathbf{R}}
\newcommand{\BS}[0]{\mathbf{S}}
\newcommand{\BT}[0]{\mathbf{T}}
\newcommand{\BU}[0]{\mathbf{U}}
\newcommand{\BV}[0]{\mathbf{V}}
\newcommand{\BW}[0]{\mathbf{W}}
\newcommand{\BX}[0]{\mathbf{X}}
\newcommand{\BY}[0]{\mathbf{Y}}
\newcommand{\BZ}[0]{\mathbf{Z}}

\newcommand{\Bzero}[0]{\mathbf{0}}
\newcommand{\Btheta}[0]{\boldsymbol{\theta}}
\newcommand{\Btau}[0]{\boldsymbol{\tau}}
\newcommand{\Bomega}[0]{\boldsymbol{\omega}}

%
% shorthand for unit vectors
%
\newcommand{\acap}[0]{\hat{\Ba}}
\newcommand{\bcap}[0]{\hat{\Bb}}
\newcommand{\ccap}[0]{\hat{\Bc}}
\newcommand{\dcap}[0]{\hat{\Bd}}
\newcommand{\ecap}[0]{\hat{\Be}}
\newcommand{\fcap}[0]{\hat{\Bf}}
\newcommand{\gcap}[0]{\hat{\Bg}}
\newcommand{\hcap}[0]{\hat{\Bh}}
\newcommand{\icap}[0]{\hat{\Bi}}
\newcommand{\jcap}[0]{\hat{\Bj}}
\newcommand{\kcap}[0]{\hat{\Bk}}
\newcommand{\lcap}[0]{\hat{\Bl}}
\newcommand{\mcap}[0]{\hat{\Bm}}
\newcommand{\ncap}[0]{\hat{\Bn}}
\newcommand{\ocap}[0]{\hat{\Bo}}
\newcommand{\pcap}[0]{\hat{\Bp}}
\newcommand{\qcap}[0]{\hat{\Bq}}
\newcommand{\rcap}[0]{\hat{\Br}}
\newcommand{\scap}[0]{\hat{\Bs}}
\newcommand{\tcap}[0]{\hat{\Bt}}
\newcommand{\ucap}[0]{\hat{\Bu}}
\newcommand{\vcap}[0]{\hat{\Bv}}
\newcommand{\wcap}[0]{\hat{\Bw}}
\newcommand{\xcap}[0]{\hat{\Bx}}
\newcommand{\ycap}[0]{\hat{\By}}
\newcommand{\zcap}[0]{\hat{\Bz}}
\newcommand{\thetacap}[0]{\hat{\Btheta}}

%
% to write R^n and C^n in a distinguishable fashion.  Perhaps change this
% to the double lined characters upon figuring out how to do so.
%
\newcommand{\C}[1]{$\mathbb{C}^{#1}$}
\newcommand{\R}[1]{$\mathbb{R}^{#1}$}

%
% various generally useful helpers
%

% derivative of #1 wrt. #2:
\newcommand{\D}[2] {\frac {d#2} {d#1}}

\newcommand{\inv}[1]{\frac{1}{#1}}
\newcommand{\cross}[0]{\times}

\newcommand{\abs}[1]{\lvert{#1}\rvert}
\newcommand{\norm}[1]{\lVert{#1}\rVert}
\newcommand{\innerprod}[2]{\langle{#1}, {#2}\rangle}
\newcommand{\dotprod}[2]{{#1} \cdot {#2}}
\newcommand{\bdotprod}[2]{\left({#1} \cdot {#2}\right)}
\newcommand{\crossprod}[2]{{#1} \cross {#2}}
\newcommand{\tripleprod}[3]{\dotprod{\left(\crossprod{#1}{#2}\right)}{#3}}

\DeclareMathOperator{\Proj}{Proj}
\DeclareMathOperator{\Span}{span}
\DeclareMathOperator{\Sgn}{sgn}
\DeclareMathOperator{\Area}{Area}
\DeclareMathOperator{\Volume}{Volume}

%
% A few miscellaneous things specific to this document
%
\newcommand{\crossop}[1]{\crossprod{#1}{}}

% R2 vector.
\newcommand{\VectorTwo}[2]{
\begin{bmatrix}
 {#1} \\
 {#2}
\end{bmatrix}
}

\newcommand{\VectorN}[1]{
\begin{bmatrix}
{#1}_1 \\
{#1}_2 \\
\vdots \\
{#1}_N \\
\end{bmatrix}
}

\newcommand{\DETuvij}[4]{
\begin{vmatrix}
 {#1}_{#3} & {#1}_{#4} \\
 {#2}_{#3} & {#2}_{#4}
\end{vmatrix}
}

\newcommand{\DETuvwijk}[6]{
\begin{vmatrix}
 {#1}_{#4} & {#1}_{#5} & {#1}_{#6} \\
 {#2}_{#4} & {#2}_{#5} & {#2}_{#6} \\
 {#3}_{#4} & {#3}_{#5} & {#3}_{#6}
\end{vmatrix}
}

\newcommand{\DETuvwxijkl}[8]{
\begin{vmatrix}
 {#1}_{#5} & {#1}_{#6} & {#1}_{#7} & {#1}_{#8} \\
 {#2}_{#5} & {#2}_{#6} & {#2}_{#7} & {#2}_{#8} \\
 {#3}_{#5} & {#3}_{#6} & {#3}_{#7} & {#3}_{#8} \\
 {#4}_{#5} & {#4}_{#6} & {#4}_{#7} & {#4}_{#8} \\
\end{vmatrix}
}

%\newcommand{\DETuvwxyijklm}[10]{
%\begin{vmatrix}
% {#1}_{#6} & {#1}_{#7} & {#1}_{#8} & {#1}_{#9} & {#1}_{#10} \\
% {#2}_{#6} & {#2}_{#7} & {#2}_{#8} & {#2}_{#9} & {#2}_{#10} \\
% {#3}_{#6} & {#3}_{#7} & {#3}_{#8} & {#3}_{#9} & {#3}_{#10} \\
% {#4}_{#6} & {#4}_{#7} & {#4}_{#8} & {#4}_{#9} & {#4}_{#10} \\
% {#5}_{#6} & {#5}_{#7} & {#5}_{#8} & {#5}_{#9} & {#5}_{#10}
%\end{vmatrix}
%}

% R3 vector.
\newcommand{\VectorThree}[3]{
\begin{bmatrix}
 {#1} \\
 {#2} \\
 {#3}
\end{bmatrix}
}



\author{Peeter Joot}
\email{peeter.joot@gmail.com}


%\usepackage{media9}
\chapter{Continuum mechanics fluids review.}

\label{chap:continuumFluidsReview}
%\useCCL
\blogpage{http://sites.google.com/site/peeterjoot2/math2012/continuumFluidsReview.pdf}
\date{Apr 21, 2012}
\gitRevisionInfo{continuumFluidsReview}
\keywords{PHY454H1S, PHY454H1, strain, displacement vector, stress, constitutive relation}
%, wave equation, displacement potentials, phasor, Love wave, Rayleigh wave

\beginArtWithToc
%\beginArtNoToc
%\wordpresscategory{}

\section{Motivation.}

Review of key ideas and equations from the fluid dynamics portion of the class.

\section{Vector displacements.}

Those portions of the theory of elasticity that we did cover have the appearance of providing some logical context for the derivation of the Navier-Stokes equation.  Our starting point is almost identical, but we now look at displacements that vary with time, forming

\begin{equation}\label{eqn:continuumFluidsReview:830}
d\Bx' = d\Bx + d\Bu \delta t.
\end{equation}

We compute a first order Taylor expansion of this differential, defining a symmetric strain and antisymmetric vorticity tensor

\begin{subequations}
\begin{equation}\label{eqn:continuumFluidsReview:850}
e_{ij} = \inv{2} \left(
\PD{x_j}{u_i} +
\PD{x_i}{u_j} \right).
\end{equation}
\begin{equation}\label{eqn:continuumFluidsReview:870}
\omega_{ij} = \inv{2} \left(
\PD{x_j}{u_i}
-\PD{x_i}{u_j} \right)
\end{equation}
\end{subequations}

Allowing us to write

\begin{equation}\label{eqn:continuumFluidsReview:890}
dx_i' = dx_i + e_{ij} dx_j \delta t + \omega_{ij} dx_j \delta t.
\end{equation}

We introduced vector and dual vector forms of the vorticity tensor with

\begin{subequations}
\begin{equation}\label{eqn:continuumFluidsReview:970}
\Omega_k = \partial_i u_j \epsilon_{i j k}
\end{equation}
\begin{equation}\label{eqn:continuumFluidsReview:990}
\omega_{i j} = -\Omega_k \epsilon_{i j k},
\end{equation}
\end{subequations}

or

\begin{subequations}
\begin{equation}\label{eqn:continuumFluidsReview:910}
\Bomega = \spacegrad \cross \Bu
\end{equation}
\begin{equation}\label{eqn:continuumFluidsReview:930}
\BOmega = \inv{2} (\Bomega)_a \Be_a.
\end{equation}
\end{subequations}

We were then able to put our displacement differential into a partial vector form

\begin{equation}\label{eqn:continuumFluidsReview:950}
d\Bx' = d\Bx + \left( \Be_i (e_{ij} \Be_j) \cdot d\Bx + \BOmega \cross d\Bx \right) \delta t.
\end{equation}

\section{Relative change in volume}

%\EndArticle
\EndNoBibArticle

   %
% Copyright � 2012 Peeter Joot.  All Rights Reserved.
% Licenced as described in the file LICENSE under the root directory of this GIT repository.
%

%
%

\label{chap:continuumL8b}
\section{Time dependent displacements} \index{displacement}
%\chapter{PHY454H1S\\Continuum Mechanics.  Lecture 8: Phasor description of elastic waves.  Fluid dynamics.  Taught by Prof. K. Das}

Reading: \S 1.4 from \citep{acheson1990elementary}.

In fluid dynamics we look at displacements with respect to time as illustrated in \cref{fig:continuumL8:continuumL8fig1}
\imageFigure{../../figures/phy454/lec8_Differential_displacementFig1}{Differential displacement}{fig:continuumL8:continuumL8fig1}{0.2}

\begin{equation}\label{eqn:continuumL8:230}
d\Bx' = d\Bx + d\Bu \delta t
\end{equation}

In index notation

\begin{equation}\label{eqn:strainAndVorticity:490}
\begin{aligned}
dx_i'
&= dx_i + du_i \delta t \\
&= dx_i + \PD{x_j}{u_i} dx_j \delta t
\end{aligned}
\end{equation}

We define the strain tensor, still symmetric, using only first order partials

\begin{equation}\label{eqn:continuumL8:250}
e_{ij} = \inv{2} \left(
\PD{x_j}{u_i} +
\PD{x_i}{u_j} \right).
\end{equation}

We also define an antisymmetric vorticity tensor

\begin{equation}\label{eqn:continuumL8:270}
\omega_{ij} = \inv{2} \left(
\PD{x_j}{u_i}
-\PD{x_i}{u_j} \right)
\end{equation}

Effect of \(e_{ij}\) when diagonalized

\begin{equation}\label{eqn:continuumL8:290}
e_{ij}
=
\begin{bmatrix}
e_{11} & 0 & 0 \\
0 & e_{22} & 0 \\
0 & 0 & e_{33}
\end{bmatrix}
\end{equation}

so that in this frame of reference we have

\begin{equation}\label{eqn:continuumL8:310}
\begin{aligned}
dx_1' &= ( 1 + e_{11} \delta t) dx_1 \\
dx_2' &= ( 1 + e_{22} \delta t) dx_2 \\
dx_3' &= ( 1 + e_{33} \delta t) dx_3
\end{aligned}
\end{equation}

Let us find the matrix form of the antisymmetric tensor.  We find

\begin{equation}\label{eqn:continuumL8:330}
\omega_{11} = \omega_{22} = \omega_{33} = 0
\end{equation}

Introducing a vorticity vector

\begin{equation}\label{eqn:continuumL8:350}
\Bomega = \spacegrad \cross \Bu
\end{equation}

we find

\begin{equation}\label{eqn:continuumL8:370}
\begin{aligned}
\omega_{12} &= \inv{2}\left( \PD{x_2}{u_1} -\PD{x_1}{u_2} \right) = - \inv{2} (\spacegrad \cross \Bu)_3 \\
\omega_{23} &= \inv{2}\left( \PD{x_3}{u_2} -\PD{x_2}{u_3} \right) = - \inv{2} (\spacegrad \cross \Bu)_1 \\
\omega_{31} &= \inv{2}\left( \PD{x_1}{u_3} -\PD{x_3}{u_1} \right) = - \inv{2} (\spacegrad \cross \Bu)_2
\end{aligned}
\end{equation}

Writing

\begin{equation}\label{eqn:continuumL8:390}
\Omega_i = \inv{2} \omega_i
\end{equation}

we find the matrix form of this antisymmetric tensor

\begin{equation}\label{eqn:continuumL8:410}
\omega_{ij}
=
\begin{bmatrix}
0 & -\Omega_3 & \Omega_2 \\
\Omega_3 & 0 & -\Omega_1 \\
-\Omega_2 & \Omega_1 & 0 \\
\end{bmatrix}
\end{equation}

\begin{equation}\label{eqn:strainAndVorticity:510}
\begin{aligned}
dx_1'
&= dx_1 + \left( \cancel{\omega_{11}} dx_1 + \omega_{12} dx_2 + \omega_{13} dx_3 \right) \delta t \\
&= dx_1 + \left( \omega_{12} dx_2 + \omega_{13} dx_3 \right) \delta t \\
&= dx_1 + \left( \Omega_2 dx_3 - \Omega_3 dx_2 \right) \delta t
\end{aligned}
\end{equation}

Doing this for all components we find

\begin{equation}\label{eqn:continuumL8:430}
d\Bx' = d\Bx + (\BOmega \cross d\Bx) \delta t.
\end{equation}

The tensor \(\omega_{ij}\) implies rotation of a control volume with an angular velocity \(\BOmega = \Bomega/2\) (half the vorticity vector).

In general we have

\begin{equation}\label{eqn:continuumL8:450}
dx_i' = dx_i + e_{ij} dx_j \delta t + \omega_{ij} dx_j \delta t
\end{equation}

\section{Comparing to elastostatics}

%After this first fluid dynamics lecture I was left troubled.  We would just been barraged with a set of equations pulled out of a magic hat, with no notion of where they came from.
Recall that for elastic materials we derived the strain tensor by considering differences in squared displacements?  It was not obvious to me why we had no such term when analyzing solids.

For solids we could have also done this first order decomposition of the displacement (per unit time) of a point.  Note that this is really just a gradient evaluation, split into coordinates by grouping into symmetric and antisymmetric terms.  Here, as in the solids case, we write

\begin{equation}\label{eqn:continuumL8:470}
\Bu = \Bx' - \Bx
\end{equation}

\begin{equation}\label{eqn:strainAndVorticity:530}
\begin{aligned}
x_i'
&= x_i + (\spacegrad u_i) \cdot d\Bx \delta t \\
&= x_i + \PD{x_j}{u_i} dx_j \delta t \\
&= x_i +
\inv{2}
\left(\PD{x_j}{u_i}
+\PD{x_j}{u_i}
\right)
dx_j \delta t
+
\inv{2}
\left(\PD{x_j}{u_i}
-\PD{x_j}{u_i}
\right)
dx_j \delta t  \\
&=
x_i + e_{ij} dx_j \delta t + \omega_{ij} dx_j \delta t
\end{aligned}
\end{equation}

%Employing vector notation we can write our first order displacement as
%
%\begin{align*}
%\Bx'
%&= \Bx + \Be_i (\spacegrad u_i) \cdot d\Bx \delta t \\
%&= \Bx + \Be_i \inv{2}
%\left(
%\spacegrad u_i d\Bx
%+d\Bx
%\spacegrad u_i
%\right)
%\delta t
%\end{align*}

   % 
% 
% 
% Copyright � 2012 Peeter Joot
% All Rights Reserved
% 
% This file may be reproduced and distributed in whole or in part, without fee, subject to the following conditions:
% 
% o The copyright notice above and this permission notice must be preserved complete on all complete or partial copies.
% 
% o Any translation or derived work must be approved by the author in writing before distribution.
% 
% o If you distribute this work in part, instructions for obtaining the complete version of this file must be included, and a means for obtaining a complete version provided.
% 
% 
% Exceptions to these rules may be granted for academic purposes: Write to the author and ask.
% 
% 
% 

%\chapter{PHY454H1S\\Continuum Mechanics.  Lecture 9: Newtonian fluids.  Mass conservation.  Constitutive relation.  Incompressible fluids.  Taught by Prof. K. Das.}
\section{Newtonian fluids.  Mass conservation.  Constitutive relation.  Incompressible fluids.}
\label{chap:continuumL9}

\section{Reading}

\S 1.4 from \citep{acheson1990elementary}.

\section{Review: Relative motion near a point in a fluid}

Referring to figure (\ref{fig:continuumL9:continuumL9fig1})
\imageFigure{figures/continuumL9fig1}{velocity displacements at a fluid point.}{fig:continuumL9:continuumL9fig1}{0.2}

we write

\begin{equation}\label{eqn:continuumL9:10}
d\Bx' = d\Bx + d\Bu \delta t
\end{equation}

or in coordinate form

\begin{equation}\label{eqn:continuumL9:30}
\begin{aligned}
dx_i 
&= dx_i + du_i \delta t \\
&= dx_i + \PD{x_j}{u_i} dx_j \delta t 
\end{aligned}
\end{equation}

We can now split the components of the gradient of $u_i$ into symmetric and antisymmetric parts in the normal way

\begin{equation}\label{eqn:continuumL9:50}
\begin{aligned}
\PD{x_j}{u_i}
&= 
\inv{2} \left( 
\PD{x_j}{u_i}
+\PD{x_i}{u_j}
\right)
+
\inv{2} \left( 
\PD{x_j}{u_i}
-\PD{x_i}{u_j}
\right) \\
&\equiv e_{ij} + \omega_{ij}.
\end{aligned}
\end{equation}

\subsection{The antisymmetric term (name?)}

With

\begin{equation}\label{eqn:continuumL9:70}
\Bomega = \spacegrad \cross \Bu,
\end{equation}

we introduce the dual vector 

\begin{equation}\label{eqn:continuumL9:90}
\BOmega = \Omega_k \Be_k = \inv{2} \Bomega
\end{equation}

defined according to

\begin{align}\label{eqn:continuumL9:110}
\Omega_1 &= \inv{2} \omega_{32} = \inv{2} \omega_1 \\
\Omega_2 &= \inv{2} \omega_{13} = \inv{2} \omega_2 \\
\Omega_3 &= \inv{2} \omega_{21} = \inv{2} \omega_3 
\end{align}

With 
\begin{equation}\label{eqn:continuumL9:610}
\omega_{k}
= \epsilon_{ijk} \partial_i u_j
\end{equation}

we can write
\begin{equation}\label{eqn:continuumL9:590}
\Omega_k = \inv{2} \epsilon_{ijk} \partial_i u_j.
\end{equation}

In matrix form this becomes

\begin{equation}\label{eqn:continuumL9:130}
\omega_{ij} = 
\begin{bmatrix}
0 & -\Omega_3 & \Omega_2 \\
\Omega_3 & 0 & -\Omega_1  \\
-\Omega_2 & \Omega_1 & 0
\end{bmatrix}.
\end{equation}

For the special case $e_{ij} = 0$, our displacement equation in vector form becomes

\begin{equation}\label{eqn:continuumL9:150}
d\Bx' = d\Bx + \BOmega \cross d\Bx \delta t.
\end{equation}

Let's do a quick verification that this is all kosher.

\begin{align*}
(\BOmega \cross d\Bx)_i
&=
\Omega_r dx_s \epsilon_{rsi} \\
&=
\left(\inv{2} \epsilon_{abr} \partial_a u_b \right) dx_s \epsilon_{rsi} \\
&=
\inv{2} \partial_a u_b dx_s \delta^{[ab]}_{si} \\
&=
\inv{2} (
\partial_s u_i
-\partial_i u_s
) dx_s  \\
&=
\inv{2} \left(
-\PD{x_i}{u_s}
+\PD{x_s}{u_i}
\right) dx_s  \\
&=
\inv{2} \left(
\PD{x_j}{u_i}
-\PD{x_i}{u_j}
\right) dx_j  \\
&=
\omega_{ij} dx_j.
\end{align*}

All's good in the world of signs and indexes.

\subsection{The symmetric term (strain tensor).}

Now let's look at the symmetric term.  With the initial volume

\begin{equation}\label{eqn:continuumL9:170}
dV = dx_1 dx_2 dx_3,
\end{equation}

and the final volume written assuming that we are working in our principle strain basis, we have (very much like the solids case)

\begin{align*}
dV' 
&= dx_1' dx_2' dx_3' \\
&= 
(1 + e_{11} \delta t) dx_1
+(1 + e_{22} \delta t) dx_2
+(1 + e_{33} \delta t) dx_3
\\
&=
(1 + (e_{11} + e_{22} + e_{33}) \delta t) dx_1 dx_2 dx_3 + O((\delta t)^2) \\
&=
\left(1 + 
\left(
\PD{x_1}{u_1}
+\PD{x_2}{u_2}
+\PD{x_3}{u_3}
\right)
\delta t \right) dV \\
&=
\left(
1 + (\spacegrad \cdot \Bu) 
\delta t
\right) dV \\
\end{align*}

So much like we expressed the relative change of volume in solids, we now can express the relative change of volume per unit time as

\begin{equation}\label{eqn:continuumL9:190}
\frac{dV' - dV}{dV \delta t} = \spacegrad \cdot \Bu,
\end{equation}

or

\begin{equation}\label{eqn:continuumL9:210}
\frac{\delta(dV)}{dV \delta t} = \spacegrad \cdot \Bu,
\end{equation}

We identify the divergence of the displacement as the relative change in volume per unit time.

\section{Newtonian Fluids.}

\begin{definition}
\emph{(Newtonian Fluids)}
\label{dfn:continuumL9:230}
A fluid for which the rate of strain tensor is linearly related to stress tensor.
\end{definition}

For such a fluid, the constitutive relation takes the form

\begin{equation}\label{eqn:continuumL9:250}
\boxed{
\sigma_{ij} = - p \delta_{ij} + 2 \mu e_{ij},
}
\end{equation}

where $p$ is called the isotropic pressure, and $\mu$ is the viscosity of the fluid.

For comparison, in solids we had

\begin{equation}\label{eqn:continuumL9:270}
\sigma_{ij} = \lambda e_{kk} \delta_{ij} + 2 \mu e_{ij}
\end{equation}

While we are allowing for rotation in the fluids ($\omega_{ij}$) that we did not consider for solids, we now impose a requirement that the strain tensor trace is not a function of the fluid displacements, with

\begin{equation}\label{eqn:continuumL9:630}
\lambda e_{kk} = \lambda \spacegrad \cdot \Bu = -p.
\end{equation}

What is the physical justification for this?  In words this was explained after class as the effect of rotation invariance with an attempt to measure the pressure at a given point in the fluid.  It doesn't matter what direction we place our pressure measurement device at a given fixed location in the fluid.  Note that this doesn't mean the pressure itself is constant.  For example with a gravitational body force applied, our pressure will increase with depth in the fluid.  Noting this provides a nice physical interpretation of the trace of the strain tensor.

Can we mathematically justify this explanation?  We see above that we have

\begin{equation}\label{eqn:continuumL9:650}
\spacegrad \cdot \Bu = \frac{\delta \ln(dV)}{\delta t},
\end{equation}

so we are in effect making the identification

\begin{equation}\label{eqn:continuumL9:670}
\ln dV = -p t /\lambda + \ln dV_0
\end{equation}

or

\begin{equation}\label{eqn:continuumL9:690}
dV = dV_0 e^{-p t/\lambda}.
\end{equation}

The relative change in a differential volume element changes exponentially.

\subsection{Dimensions}

\begin{equation}\label{eqn:continuumL9:290}
[\mu] = \frac{\text{M}}{\text{L}\text{T}}.
\end{equation}

Some examples

\begin{itemize}
\item $\mu_{\text{air}} = 1.8 \times 10^{-5} \frac{\text{kg}}{\text{m s}}$
\item $\mu_{\text{water}} = 1.1 \times 10^{-3} \frac{\text{kg}}{\text{m s}}$
\item $\mu_{\text{glycerin}} = 2.3 \frac{\text{kg}}{\text{m s}}$
\end{itemize}

\section{Conservation of mass in fluid.}

Referring to figure (\ref{fig:continuumL9:continuumL9fig2})
\imageFigure{figures/continuumL9fig2}{Area projections for mass conservation argument.}{fig:continuumL9:continuumL9fig2}{0.2}

we have a flow rate

\begin{equation}\label{eqn:continuumL9:310}
\rho \Bu \delta t ds
\end{equation}

or 
\begin{equation}\label{eqn:continuumL9:330}
\rho \Bu ds,
\end{equation}

per unit time.  Here the velocity of fluid particle is $\Bu$.

\begin{equation}\label{eqn:continuumL9:350}
\oint \rho \Bu \cdot d\Bs,
\end{equation}

we must have

\begin{equation}\label{eqn:continuumL9:710}
\PD{t}{} \int \rho dV 
=
-\oint \rho \Bu \cdot d\Bs.
\end{equation}

\begin{equation}\label{eqn:continuumL9:370}
dm = \rho dV
\end{equation}

\begin{equation}\label{eqn:continuumL9:390}
\frac{dm}{dt} = \frac{d}{dt} (\rho dV)
\end{equation}

\begin{itemize}
\item 
positive if fluid is coming in.
\item 
negative if fluid is going out.
\end{itemize}

By Green's theorem

\begin{equation}\label{eqn:continuumL9:410}
\oint \BA \cdot d\Bs = \int_V (\spacegrad \cdot \BA) dV,
\end{equation}

so we have

\begin{equation}\label{eqn:continuumL9:430}
-\oint \rho \Bu \cdot d\Bs = -\int \spacegrad \cdot (\rho \Bu ) dV,
\end{equation}

and must have

\begin{equation}\label{eqn:continuumL9:450}
\int \left( \PD{t}{\rho} + \spacegrad \cdot (\rho \Bu) \right) dV = 0.
\end{equation}

The total mass has to be conserved.  The mass that is leaving the volume per unit time must move through the surface of the volume in that time.  In differential form this is

\begin{equation}\label{eqn:continuumL9:470}
\boxed{
\PD{t}{\rho} + \spacegrad \cdot (\rho \Bu) = 0.
}
\end{equation}

Operating by chain rule we can write this as

\begin{equation}\label{eqn:continuumL9:490}
\PD{t}{\rho} + \Bu \cdot \spacegrad \rho = - \rho \spacegrad \cdot \Bu.
\end{equation}

To make sense of this, observe that we have for $f = f(x, y, z, t)$

\begin{align*}
\delta f 
&= \lim_{\delta t \rightarrow 0} \frac{\delta f}{\delta t} dt \\
&=
\PD{x}{f} \frac{\delta x}{\delta t}
+\PD{y}{f} \frac{\delta y}{\delta t}
+\PD{z}{f} \frac{\delta z}{\delta t}
+ \PD{t}{f} \\
&=
(\spacegrad f) \cdot \Bu + \PD{t}{f}
\end{align*}

so we have

\begin{equation}\label{eqn:continuumL9:510}
\PD{t}{\rho} + \Bu \cdot \spacegrad \rho = \frac{d\rho}{dt}
\end{equation}

or
\begin{equation}\label{eqn:continuumL9:530}
\frac{d\rho}{dt} = - \rho \spacegrad \cdot \Bu.
\end{equation}

\subsection{Incompressible fluid}

When the density doesn't change note that we have

\begin{equation}\label{eqn:continuumL9:550}
\frac{d\rho}{dt} = 0
\end{equation}

which then implies

\begin{equation}\label{eqn:continuumL9:570}
\boxed{
\spacegrad \cdot \Bu = 0,
}
\end{equation}

at all points in the fluid.

   %
% Copyright � 2012 Peeter Joot.  All Rights Reserved.
% Licenced as described in the file LICENSE under the root directory of this GIT repository.
%

%
%

%\chapter{PHY454H1S\\Continuum Mechanics.  Lecture 10: Navier-Stokes equation.  Taught by Prof. K. Das}
\label{chap:continuumL10}

%\section{Review.  Newtonian fluid}
%
%We stated the model for a Newtonian fluid
%
%\begin{equation}\label{eqn:continuumL10:10}
%\sigma_{ij} = -p \delta_{ij} + 2 \mu e_{ij}
%\end{equation}
%
%and started considering conservation of mass with a volume \(dV\) through an area element \(d\Bs\).  For the rate of change of mass \textunderline{flowing out of the volume \(V\)} is
%
%\begin{equation}\label{eqn:continuumL10:30}
%\oint \rho \Bu \cdot d\Bs = - \PD{t}{} \int_V \rho dV.
%\end{equation}
%
%Application of Green's theorem, for a fixed (in time) volume \(V\) produces
%
%\begin{equation}\label{eqn:continuumL10:50}
%0 = \int_V \left( \spacegrad \cdot (\rho \Bu) + \PD{t}{\rho} \right) dV,
%\end{equation}
%
%or in differential form for an infinitesimal volume
%
%\begin{equation}\label{eqn:continuumL10:70}
%0 = \PD{t}{\rho} + \spacegrad \cdot (\rho \Bu).
%\end{equation}
%
%Expanding out the divergence term using
%
%\begin{align*}
%\spacegrad \cdot (a \Bb)
%&=
%\partial_i (a b_i) \\
%&=
%b_i \partial_i a
%+
%a \partial_i b_i \\
%&=
%\Bb \cdot \spacegrad a
%+ a \spacegrad \cdot \Bb
%\end{align*}
%
%\begin{equation}\label{eqn:continuumL10:90}
%0 = \PD{t}{\rho}
%+ \rho \spacegrad \cdot \Bu
%+ \Bu \cdot \spacegrad \rho.
%\end{equation}
%
%For an incompressible fluid
%
%\begin{equation}\label{eqn:continuumL10:110}
%\spacegrad \cdot \Bu = 0
%\end{equation}
%
%so the conservation of mass equality relation takes the form
%\begin{equation}\label{eqn:continuumL10:90b}
%0 = \PD{t}{\rho} + \Bu \cdot \spacegrad \rho.
%\end{equation}
%
\section{Conservation of momentum (Navier-Stokes equation)}

Reading: \S 6.* from \citep{acheson1990elementary}.

In classical mechanics we have

\begin{equation}\label{eqn:continuumL10:130}
\Bf = m \Ba,
\end{equation}

our analogue here is found in terms of the stress tensor

\begin{equation}\label{eqn:continuumL10:150}
\int_V F_i dV = \int_V \PD{x_j}{\sigma_{ij}} dV
\end{equation}

Here \(F_i\) is the force per unit volume.  With body forces we have

\begin{equation}\label{eqn:continuumL10:170}
F_i = \rho \frac{du_i}{dt} = \PD{x_j}{\sigma_{ij}} + \rho f_i
\end{equation}

where \(f_i\) is an external force per unit volume.  Observe that \(\sigma_{ij}\), through the constitutive relation, includes both contributions of linear displacement and the vorticity component.

From the constitutive relation \eqnref{eqn:continuumL9:250}, we have

\begin{equation}\label{eqn:NavierStokes:350}
\begin{aligned}
\PD{x_j}{\sigma_{ij}}
&= - \PD{x_j}{p} \delta_{ij} + 2 \mu \PD{x_j}{e_{ij}} \\
&= - \PD{x_i}{p} + 2 \mu \PD{x_j}{} \left(
\inv{2} \left(
 \PD{x_j}{u_i}
+ \PD{x_i}{u_j}
\right)
\right) \\
&= - \PD{x_i}{p} + \mu \left(
\frac{\partial^2 u_i}{\partial x_j \partial x_j}
+\frac{\partial^2 u_j}{\partial x_i \partial x_j}
\right)
\end{aligned}
\end{equation}

Observe that the term

\begin{equation}\label{eqn:continuumL10:190}
\frac{\partial^2 u_i}{\partial x_j \partial x_j}
\end{equation}

is the \(i^{\text{th}}\) component of \(\spacegrad^2 \Bu\), whereas

\begin{equation}\label{eqn:NavierStokes:370}
\begin{aligned}
\frac{\partial^2 u_j}{\partial x_i \partial x_j}
&= \PD{x_i}{} \left( \PD{x_j}{u_j} \right) \\
&= \PD{x_i}{} (\spacegrad \cdot \Bu)
\end{aligned}
\end{equation}

is the \(i^{\text{th}}\) component of \(\spacegrad (\spacegrad \cdot \Bu)\).

We have therefore that

\begin{equation}\label{eqn:continuumL10:210}
\rho \frac{du_i}{dt} = \left( -\spacegrad p + \mu \spacegrad^2 \Bu
+ \mu \spacegrad (\spacegrad \cdot \Bu) + \rho \Bf
\right)_i,
\end{equation}

or in vector notation

\begin{equation}\label{eqn:continuumL10:230}
\rho \frac{d\Bu}{dt} = -\spacegrad p + \mu \spacegrad^2 \Bu
+ \mu \spacegrad (\spacegrad \cdot \Bu) + \rho \Bf.
\end{equation}

We can expand this a bit more writing our velocity \(\Bu = \Bu(x, y, z, t)\) differential

\begin{equation}\label{eqn:continuumL10:250}
du_i = \PD{x_j}{u_i} \delta x_j + \PD{t}{u_i} \delta t.
\end{equation}

Considering rates

\begin{equation}\label{eqn:continuumL10:270}
\frac{du_i}{dt} = \PD{x_j}{u_i} \frac{dx_j}{dt} + \PD{t}{u_i} .
\end{equation}

In vector notation we have

\begin{equation}\label{eqn:continuumL10:290}
\frac{d\Bu}{dt} = (\Bu \cdot \spacegrad) \Bu + \PD{t}{\Bu}.
\end{equation}

Newton's second law \eqnref{eqn:continuumL10:230} now becomes

\boxedEquation{eqn:continuumL10:230b}{
\rho
 (\Bu \cdot \spacegrad) \Bu + \rho \PD{t}{\Bu}
= -\spacegrad p + \mu \spacegrad^2 \Bu
+ \mu \spacegrad (\spacegrad \cdot \Bu) + \rho \Bf.
}

This is the Navier-Stokes equation.  Observe that we have an explicitly non-linear term

\begin{equation}\label{eqn:continuumL10:310}
(\Bu \cdot \spacegrad) \Bu ,
\end{equation}

something we do not encounter in most classical mechanics.  The impacts of this non-linear term are very significant and produce some interesting effects.

\section{Incompressible fluids}

We have seen that incompressibility was equivalent to

\begin{equation}\label{eqn:continuumL10:330}
\spacegrad \cdot \Bu = 0.
\end{equation}

With such a restriction the Navier-Stokes equation takes the much simpler form


\boxedEquation{eqn:continuumL10:230c}{
\rho
 (\Bu \cdot \spacegrad) \Bu + \rho \PD{t}{\Bu}
= -\spacegrad p + \mu \spacegrad^2 \Bu
+ \rho \Bf
\spacegrad \cdot \Bu = 0.
}

We will not treat compressible fluids in this course.

\section{Boundary value conditions}

In order to solve any sort of PDE we need to consider the boundary value conditions.  Consider the interface between two layers of liquids as in \cref{fig:continuumL9:continuumL10fig1}

\imageFigure{../../figures/phy454/lec10_Rocker_tank_with_two_viscosity_fluidsFig1}{Rocker tank with two viscosity fluids}{fig:continuumL9:continuumL10fig1}{0.2}

Also found an illustration of this in \href{watinst.ut.ac.ir/downloads/pdf/ebooks/white.pdf}{fig 1.13 of White's text online}

We see the fluids sticking together at the boundary.  This is due to matching of the tangential velocity components at the interface.

   %
% Copyright � 2012 Peeter Joot.  All Rights Reserved.
% Licenced as described in the file LICENSE under the root directory of this GIT repository.
%

%
%
\section{Normals and tangents at fluid interfaces} \index{normal} \index{tangent} \index{fluid interface}

%The Navier-Stokes equation (our fluids equivalent to Newton's second law) was found to be
%
%\begin{equation}\label{eqn:continuumL11:10}
%\rho \PD{t}{\Bu} + \rho (\Bu \cdot \spacegrad) \Bu = - \spacegrad p + \mu \spacegrad^2 \Bu + \rho \Bf.
%\end{equation}
%
%In this course we will focus on the incompressible case where we have
%
%\begin{equation}\label{eqn:continuumL11:30}
%\spacegrad \cdot \Bu = 0
%\end{equation}

We watched a video of the rocking tank as in \cref{fig:continuumL11:continuumL11fig1}.  The boundary condition that accounted for the matching of the die marker is that we have \textunderline{no slipping} at the interface.  Writing \(\taucap\) for the unit tangent to the interface then this condition at the interface is described mathematically by the conditions

\imageFigure{../../figures/phy454/lec11_Rocking_tank_velocity_matchingFig1}{Rocking tank velocity matching}{fig:continuumL11:continuumL11fig1}{0.3}

\begin{equation}\label{eqn:continuumL11:70}
\begin{aligned}
\Bu_A \cdot \taucap &= \Bu_B \cdot \taucap \\
\Bu_A \cdot \ncap &= \Bu_B \cdot \ncap.
\end{aligned}
\end{equation}

Referring to \cref{fig:continuumL11:continuumL11fig2} where the tangents and normals are depicted an example representation of the normal and tangent vectors for the fluids are

\imageFigure{../../figures/phy454/lec11_Normals_and_tangents_at_interface_for_2D_systemFig2}{Normals and tangents at interface for 2D system}{fig:continuumL11:continuumL11fig2}{0.2}

\begin{equation}\label{eqn:continuumL11:90}
\begin{aligned}
\taucap &=
\begin{bmatrix}
1 \\
0
\end{bmatrix} \\
\ncap &=
\begin{bmatrix}
0 \\
1
\end{bmatrix}
\end{aligned}
\end{equation}

For the traction vector with components

\begin{equation}\label{eqn:continuumL11:110}
T_i = \sigma_{ij} n_j,
\end{equation}

we also have at the interface we must have matching of

\begin{equation}\label{eqn:continuumL11:130}
\taucap \cdot \BT.
\end{equation}

More explicitly, in coordinates this is

\begin{equation}\label{eqn:continuumL11:150}
\evalbar{\tau_i (\sigma_{ij} n_j)}{A} =
\evalbar{\tau_i (\sigma_{ij} n_j)}{B}
\end{equation}

\makeexample{Steady incompressible rectilinear (unidirectional) flow}{ex:fluids:rectilinear}{

In this case we can fix our axis so that

\begin{equation}\label{eqn:continuumL11:170}
\Bu = \xcap u(x, y, z, t),
\end{equation}

where the velocity components in the other directions

\begin{equation}\label{eqn:continuumL11:190}
\begin{aligned}
v &= 0 \\
w &= 0
\end{aligned}
\end{equation}

are both zero.  Symbolically, the steady state condition is

\begin{equation}\label{eqn:continuumL11:210}
\PD{t}{\Bu} = 0.
\end{equation}

% prof called this the continuity condition?
We start with the incompressibility condition, which written explicitly, is

\begin{equation}\label{eqn:continuumL11:230}
\spacegrad \cdot \Bu = 0,
\end{equation}

or

\begin{equation}\label{eqn:continuumL11:250}
\PD{x}{u} + \PD{y}{v} + \PD{z}{w} = 0
\end{equation}

This implies

\begin{equation}\label{eqn:continuumL11:270}
\PD{x}{u} = 0
\end{equation}

so our velocity can only be function of the \(y\) and \(z\) coordinates only

\begin{equation}\label{eqn:continuumL11:290}
u = u(y, z).
\end{equation}

The non-linear term of the Navier-Stokes equation takes the form

\begin{equation}\label{eqn:unidirectionalSolutions:870}
\begin{aligned}
(\Bu \cdot \spacegrad) \Bu
&=
\left(u \PD{x}{}
+\cancel{v} \PD{y}{}
+\cancel{w} \PD{z}{} \right) (\xcap u( y, z) + \ycap \cancel{v} + \zcap \cancel{w} ) \\
&=
\xcap u \cancel{\PD{x}{u}} \\
&= 0.
\end{aligned}
\end{equation}

With incompressibility and \(u = v = 0\) conditions killing this term, and the steady state condition \eqnref{eqn:continuumL11:210} killing the \(\rho \PDi{t}{\Bu}\) term, the Navier-Stokes equation for this incompressible unidirectional steady state flow (in the absence of body forces) is reduced to

\begin{equation}\label{eqn:continuumL11:310}
0 = - \spacegrad p + \mu \spacegrad^2 \Bu.
\end{equation}

In coordinates this is

\begin{subequations}
\begin{equation}\label{eqn:continuumL11:330a}
\PD{x}{p} = \mu \left( \PDSq{y}{u} + \PDSq{z}{u} \right)
\end{equation}
\begin{equation}\label{eqn:continuumL11:330b}
\PD{y}{p} = 0
\end{equation}
\begin{equation}\label{eqn:continuumL11:330c}
\PD{z}{p} = 0.
\end{equation}
\end{subequations}

Operating on the first with an x partial we find

\begin{equation}\label{eqn:continuumL11:350}
\PDSq{x}{p} = \mu \left( \PDSq{y}{} \cancel{\PD{x}{u}} + \PDSq{z}{} \cancel{ \PD{x}{u} } \right) = 0
\end{equation}

Since we have

\begin{equation}\label{eqn:continuumL11:370}
\PDSq{x}{p} = 0
\end{equation}

we also have

\begin{equation}\label{eqn:continuumL11:390}
\frac{d^2 p}{dx^2} = 0,
\end{equation}

so our pressure must be linear with position

\begin{equation}\label{eqn:continuumL11:410}
p = A x + B,
\end{equation}

as illustrated in \cref{fig:continuumL11:continuumL11fig3}

\imageFigure{../../figures/phy454/lec11_Pressure_gradient_in_1D_systemFig3}{Pressure gradient in 1D system}{fig:continuumL11:continuumL11fig3}{0.2}

\begin{equation}\label{eqn:continuumL11:430}
p =
\left\{
\begin{array}{l l}
p_0 & \quad \mbox{\(x = 0\)} \\
p_L & \quad \mbox{\(x = L\)}
\end{array}
\right.
%} % vim brace matching.
\end{equation}

we have

\begin{equation}\label{eqn:continuumL11:450}
p = \frac{p_L - p_0}{L} x + p_0
\end{equation}

and

\begin{equation}\label{eqn:continuumL11:470}
\frac{dp}{dx} = \frac{p_L - p_0}{L} = \text{constant} \equiv -G
\end{equation}
} % end example

\makeexample{Shearing flow}{ex:fluids:shearing}{

The flows of this sort do not have to be trivial.  For example, even with constant pressure (\(p_0 = p_L\)) as in \cref{fig:continuumL11:continuumL11fig4} we can have a ``shearing flow'' where the fluids at the top surface are not necessarily moving at the same rates as the fluid below that surface.  We have fluid flow in the \(x\) direction only, and our velocity is a function only of the \(y\) coordinate.

\imageFigure{../../figures/phy454/lec11_Velocity_variation_with_height_in_shearing_flowFig4}{Velocity variation with height in shearing flow}{fig:continuumL11:continuumL11fig4}{0.2}

\begin{equation}\label{eqn:continuumL11:490}
\begin{aligned}
\Bu &= \xcap u(y) \\
G &= 0 \\
u(0) &= 0 \\
u(h) &= U.
\end{aligned}
\end{equation}

For such a flow \eqnref{eqn:continuumL11:330a} simplifies to

\begin{equation}\label{eqn:continuumL11:530}
\frac{d^2 u}{dy^2} = 0
\end{equation}

with solution

\begin{equation}\label{eqn:continuumL11:550}
u = \frac{U}{h} y + u(0) = \frac{U}{h} y.
\end{equation}
} % end example

\makeexample{Channel flow}{ex:fluids:channel}{

\begin{equation}\label{eqn:continuumL11:570}
\begin{aligned}
\Bu &= \xcap u(y) \\
G &= - \frac{dp}{dx} \ne 0
\end{aligned}
\end{equation}

This time our simplified Navier-Stokes equation \eqnref{eqn:continuumL11:330a} is reduced to something slightly more complicated

\begin{equation}\label{eqn:continuumL11:590}
\mu \frac{d^2 u}{dy^2} = -G,
\end{equation}

with solution

\begin{equation}\label{eqn:continuumL11:610}
u = -\frac{G}{2 \mu} y^2 + A y + B.
\end{equation}

The boundary value conditions with the coordinate system in use illustrated in \cref{fig:continuumL11:continuumL11fig5} require the velocity to be zero at the interface (the pipe walls preventing flow in the interior of the pipe)

\imageFigure{../../figures/phy454/lec11_1D_Channel_flow_coordinate_system_setupFig5}{1D Channel flow coordinate system setup}{fig:continuumL11:continuumL11fig5}{0.2}

\begin{equation}\label{eqn:continuumL11:630}
u(\pm h) =
-\frac{G}{2 \mu} h^2 \pm A h + B = 0
\end{equation}

One solution, immediately evident is,

\begin{equation}\label{eqn:continuumL11:650}
\begin{aligned}
A &= 0 \\
B &= \frac{G}{2 \mu} h^2,
\end{aligned}
\end{equation}

so our solution becomes

\begin{equation}\label{eqn:continuumL11:690}
u = \frac{G}{2 \mu} \Bigl( h^2 - y^2 \Bigr),
\end{equation}

a parabolic velocity flow.  This is illustrated graphically in \cref{fig:continuumL11:continuumL11fig6}.

\imageFigure{../../figures/phy454/lec11_Parabolic_velocity_distributionFig6}{Parabolic velocity distribution}{fig:continuumL11:continuumL11fig6}{0.2}

It is clear that this is maximized by \(y = 0\), but we can also see this by computing

\begin{equation}\label{eqn:continuumL11:710}
\frac{du}{dy} = \frac{G}{\mu} y = 0.
\end{equation}

This maximum is

\begin{equation}\label{eqn:continuumL11:850}
u_{\text{max}} = \frac{G}{2\mu} h^2
\end{equation}

The flux, or flow rate is

\begin{equation}\label{eqn:unidirectionalSolutions:890}
\begin{aligned}
Q
&= \iint_S \Bu \cdot \xcap ds \\
&= \int_0^1 dz \int_{-h}^h dy u(y) \\
&=
\frac{2 G h^3}{3}
\end{aligned}
\end{equation}

Let us now compute the strain (\(e_{ij}\)) and the stress (\(\sigma_{ij} = -p \delta_{ij} + 2 \mu e_{ij}\))

\begin{equation}\label{eqn:continuumL11:730}
\begin{aligned}
e_{12} &= e_{21} = \inv{2} \left( \PD{y}{u} \right) = - \frac{G y}{2 \mu} \\
e_{11} &= \PD{x}{u} = 0 \\
e_{22} &= \PD{y}{v} = 0
\end{aligned}
\end{equation}

stress

\begin{equation}\label{eqn:continuumL11:750}
\sigma_{12} = 2 \mu e_{12} = -G y
\end{equation}

This can be used to compute the forces on the inner surfaces of the tube.  As illustrated in \cref{fig:continuumL11:continuumL11fig7}, our normals at \(\pm h\) are \(\mp \ycap\) respectively.  The traction vector in the \(y\) direction is at \(y = h\) is

\imageFigure{../../figures/phy454/lec11_Normals_in_1D_channel_flow_systemFig7}{Normals in 1D channel flow system}{fig:continuumL11:continuumL11fig7}{0.2}

\begin{equation}\label{eqn:continuumL11:770}
\tau_i = \sigma_{i 2} \evalbar{n_2}{y = h} = G h,
\end{equation}

so that

\begin{equation}\label{eqn:continuumL11:791}
\Btau = \xcap G h
\end{equation}

(here the \(x\) directionality comes from the \(i = 1\) index of the stress tensor).

\begin{equation}\label{eqn:continuumL11:790}
F_{x_L} = \xcap \cdot \Btau = G h
\end{equation}

The total force is then

\begin{equation}\label{eqn:continuumL11:810}
\int_0^L F_x dx = + G h L
\end{equation}

\FIXME{Ask in class.  This is the tangential force at the boundary of the wall.  What is it a force on?  If it is tangential, how can it act on the wall?  It could act on an impediment placed right up next to the wall, but if that is the case, why are we integrating from \(x = 0\) to \(x = L\)?}
} % end example

   % 
% 
% 
% Copyright � 2012 Peeter Joot
% All Rights Reserved
% 
% This file may be reproduced and distributed in whole or in part, without fee, subject to the following conditions:
% 
% o The copyright notice above and this permission notice must be preserved complete on all complete or partial copies.
% 
% o Any translation or derived work must be approved by the author in writing before distribution.
% 
% o If you distribute this work in part, instructions for obtaining the complete version of this file must be included, and a means for obtaining a complete version provided.
% 
% 
% Exceptions to these rules may be granted for academic purposes: Write to the author and ask.
% 
% 
% 

%\chapter{PHY454H1S\\Continuum Mechanics.  Lecture 12: Flow in a pipe.  Gravity driven flow of a film.  Taught by Prof. K. Das}
\section{Flow in a pipe.  Gravity driven flow of a film}
\label{chap:continuumL12}

\section{Review.  Steady rectilinear flow}

Steady:

\begin{equation}\label{eqn:continuumL12:10}
\PD{t}{} = 0
\end{equation}

Rectilinear is a unidirectional flow such as

\begin{equation}\label{eqn:continuumL12:30}
\Bu = \xcap u( x, y, z ),
\end{equation}

\begin{enumerate}
\item
Utilizing an incompressibility assumption $\spacegrad \cdot \Bu = 0$, so for this case we have

\begin{equation*}
\PD{x}{u} = 0
\end{equation*}

or

\begin{equation*}
u = u(y, z)
\end{equation*}

Note that Prof. Das called this a continuity requirement, and justified this label with the relation

\begin{equation}\label{eqn:classicalMechanicsPs2:550}
\frac{d\rho}{dt} = \rho (\spacegrad \cdot \Bu),
\end{equation}

which was a consequence of mass conservation.  It's still not clear to me why he would call this a continuity requirement.

\item Nonlinear term is zero.  $(\Bu \cdot \spacegrad) \Bu = 0$
\item $p = p(x)$.  Since $\frac{d^2 p}{dx^2} = 0$ we also have $\frac{dp}{dx} = -G$, a constant.

\item $\mu \left( \PDSq{y}{u} + \PDSq{z}{u} \right) = G$

\end{enumerate}

\section{Solution by intuition}

Two examples that we have solved analytically are illustrated in figure (\ref{fig:continuumL12:continuumL12fig1}) and figure (\ref{fig:continuumL12:continuumL12fig2})

\imageFigure{figures/continuumL12fig1}{Simple shear flow}{fig:continuumL12:continuumL12fig1}{0.2}
\imageFigure{figures/continuumL12fig2}{Channel flow}{fig:continuumL12:continuumL12fig2}{0.2}

Sometimes we can utilize solutions already found to understand the behavior of more complex systems.  Combining the two we can look at flow over a plate as in figure (\ref{fig:continuumL12:continuumL12fig3})

\imageFigure{figures/continuumL12fig3}{Flow on a plate}{fig:continuumL12:continuumL12fig3}{0.2}

Example 2.  Fluid in a container.  If the surface tension is altered on one side, we induce a flow on the surface, leading to a circulation flow.  This can be done for example, by introducing a heat source or addition of surfactant.

This is illustrated in figure (\ref{fig:continuumL12:continuumL12fig4})
\imageFigure{figures/continuumL12fig4}{Circulation flow induced by altering surface tension}{fig:continuumL12:continuumL12fig4}{0.2}

This sort of flow is hard to analyze, only first done by Steve Davis in the 1980's.  The point here is that we can use some level of intuition to guide our attempts at solution.

\makeexample{Flow down a pipe}{ex:fluids:pipeflow}{

Reading: \S 2 from \citep{acheson1990elementary}.

Recall that the Navier-Stokes equation is

\begin{equation}\label{eqn:continuumL12:570}
\boxed{
\rho \PD{t}{\Bu} + \rho (\Bu \cdot \spacegrad) \Bu = - \spacegrad p + \mu \spacegrad^2 \Bu + \rho \Bf.
}
\end{equation}

We need to express this in cylindrical coordinates $(r, \theta, z)$ as in figure (\ref{fig:continuumL12:continuumL12fig5})
\imageFigure{figures/continuumL12fig5}{Flow through a pipe}{fig:continuumL12:continuumL12fig5}{0.2}

Our gradient is

\begin{equation}\label{eqn:classicalMechanicsPs2:610}
\spacegrad =
\rcap \PD{r}{} +
\frac{\thetacap}{r} \PD{\theta}{} +
\zcap \PD{z}{},
\end{equation}

For our Laplacian we find

\begin{align*}
\spacegrad^2 &=
\left(
\rcap \PD{r}{} +
\frac{\thetacap}{r} \PD{\theta}{} +
\zcap \PD{z}{}
\right)
 \cdot
\left(
\rcap \PD{r}{} +
\frac{\thetacap}{r} \PD{\theta}{} +
\zcap \PD{z}{}
\right) \\
&=
\partial_{rr}
+ \frac{\thetacap}{r} \cdot (\partial_\theta \rcap) \partial_r
+ \inv{r} \partial_\theta \left( \inv{r} \partial_\theta \right)
+ \partial_{zz} \\
&=
\partial_{rr} + \inv{r} \partial_r + \inv{r^2} \partial_{\theta\theta} + \partial_{zz},
\end{align*}

which we can write as

\begin{equation}\label{eqn:continuumL12:630}
\spacegrad^2 = \inv{r} \PD{r}{} \left( r \PD{r}{} \right) + \inv{r^2} \PDSq{\theta}{} + \PDSq{z}{}.
\end{equation}

Navier-Stokes takes the form

\begin{equation}\label{eqn:classicalMechanicsPs2:650}
\boxed{
\begin{aligned}
\rho \PD{t}{\Bu} &+ \rho
\left(
u_r \PD{r}{} +
\frac{u_\theta}{r} \PD{\theta}{} +
u_z \PD{z}{} \right) \Bu
=  \\
&-
\left(
\rcap \PD{r}{} +
\frac{\thetacap}{r} \PD{\theta}{} +
\zcap \PD{z}{}
\right)
p + \mu \left(
\inv{r} \PD{r}{} \left( r \PD{r}{} \right) + \inv{r^2} \PDSq{\theta}{} + \PDSq{z}{} \right)
\Bu + \rho \Bf.
\end{aligned}
}
\end{equation}

It's pointed out in \citep{acheson1990elementary}, that our non-linear term $(\Bu \cdot \spacegrad) \Bu$, with $\Bu = \rcap u_r + \thetacap u_\theta + \zcap u_z$ has contributions both from the coordinates $(u_r, u_\theta, u_z)$ and the unit vectors $\{\rcap, \thetacap, \zcap\}$ since both $\rcap$ and $\thetacap$ have $\theta$ dependence.  So if we wish to express Navier-Stokes in coordinate form we must write

\begin{equation}\label{eqn:classicalMechanicsPs2:650b}
\begin{aligned}
(\Bu \cdot \spacegrad) \Bu
&=
\rcap (\Bu \cdot \spacegrad) u_r 
+\thetacap (\Bu \cdot \spacegrad) u_\theta
+\zcap (\Bu \cdot \spacegrad) u_z
+ \frac{u_\theta}{r} \thetacap u_r
- \frac{u_\theta}{r} \rcap u_\theta \\
&=
\rcap \left((\Bu \cdot \spacegrad) u_r 
- \frac{u_\theta^2}{r} 
\right)
+\thetacap \left((\Bu \cdot \spacegrad) u_\theta
+ \frac{u_r u_\theta}{r} 
\right)
+\zcap \left((\Bu \cdot \spacegrad) u_z\right)
\end{aligned}
\end{equation}

For steady state and incompressible fluids in the absence of body forces we have

\begin{equation}\label{eqn:continuumL12:670}
\left(
\rcap \PD{r}{} +
\frac{\thetacap}{r} \PD{\theta}{} +
\zcap \PD{z}{}
\right)
p = \mu \left(
\inv{r} \PD{r}{} \left( r \PD{r}{} \right) + \inv{r^2} \PDSq{\theta}{} + \PDSq{z}{} \right)
\Bu,
\end{equation}

or, in coordinates

\begin{align}\label{eqn:continuumL12:690}
\PD{r}{p}
&= \mu \left(
\inv{r} \PD{r}{} \left( r \PD{r}{} \right) + \inv{r^2} \PDSq{\theta}{} + \PDSq{z}{} \right)
u_r \\
\frac{1}{r} \PD{\theta}{p}
&= \mu \left(
\inv{r} \PD{r}{} \left( r \PD{r}{} \right) + \inv{r^2} \PDSq{\theta}{} + \PDSq{z}{} \right)
u_\theta \\
\PD{z}{p}
&= \mu \left(
\inv{r} \PD{r}{} \left( r \PD{r}{} \right) + \inv{r^2} \PDSq{\theta}{} + \PDSq{z}{} \right)
u_z
\end{align}

With an assumption that we have no radial or circulatory flows ($u_r = u_\theta = 0$), and with $u_z = w$ assumed to only have a radial dependence, our velocity is

\begin{equation}\label{eqn:continuumL12:50}
\Bu = \zcap w(r),
\end{equation}

and an assumption of linear pressure dependence

\begin{equation}\label{eqn:continuumL12:60}
\frac{dp}{dz} = -G,
\end{equation}

then Navier-Stokes takes the final simple form

\begin{equation}\label{eqn:continuumL12:70}
\inv{r} \frac{d}{dr} \left( r \frac{dw}{dr} \right) = - \frac{G}{\mu}.
\end{equation}

Solving this we have

\begin{equation}\label{eqn:continuumL12:90}
r \frac{dw}{dr} = - \frac{G r^2}{2\mu} + A
\end{equation}

\begin{equation}\label{eqn:continuumL12:110}
w = -\frac{G r^2}{4 \mu} + A \ln(r) + B
\end{equation}

Requiring finite solutions for $r = 0$ means that we must have $A = 0$.  Also $w(a) = 0$, we have $B = G a^2/4 \mu$ so we must have

\begin{equation}\label{eqn:continuumL12:130}
w(r) = \frac{G}{4 \mu}( a^2 - r^2 )
\end{equation}
} % end example

\makeexample{Gravity driven flow of a liquid film}{ex:fluids:inclinedgravityflow}{
%(This is one of our Professor's favorite problems).

Reading: \S 2.3 from \citep{acheson1990elementary}.

Coordinates as in figure (\ref{fig:continuumL12:continuumL12fig6})
\imageFigure{figures/continuumL12fig6}{Gravity driven flow down an inclined plane}{fig:continuumL12:continuumL12fig6}{0.2}

\begin{equation}\label{eqn:continuumL12:150}
\Bu = \xcap u(y)
\end{equation}

\paragraph{Boundary conditions}

\begin{enumerate}
\item $u(y = 0) = 0$
\item Tangential stress at the air-liquid interface $y = h$ is equal.

\begin{equation}\label{eqn:continuumL12:170}
\Btau \cdot (\Bsigma_l \cdot \ncap) = \Btau \cdot (\Bsigma_a \cdot \ncap),
\end{equation}
\end{enumerate}

We write

\begin{align}\label{eqn:continuumL12:190}
\Btau &=
\begin{bmatrix}
1 \\
0 \\
0
\end{bmatrix} \\
\ncap &=
\begin{bmatrix}
0 \\
1 \\
0
\end{bmatrix}
\end{align}

and seek simultaneous solutions to the pair of stress tensor equations

\begin{align}\label{eqn:continuumL12:210}
\sigma_{ij}^l
&= - p \delta_{ij} + \mu^l \left(
\PD{x_j}{u_i} +
\PD{x_i}{u_j}
\right) \\
\sigma_{ij}^a
&= - p \delta_{ij} + \mu^a \left(
\PD{x_j}{u_i} +
\PD{x_i}{u_j}
\right).
\end{align}

In general this requires an iterated approach, solving for one with an initial approximation of the other, then switching and tuning the numerical method carefully for convergence.

We expect that the flow of liquid will induce a flow of air at the interface, but may be able to make a one-sided approximation.  Let's see how far we get before we have to introduce any approximations and compute the traction vector for the liquid

\begin{align*}
\Bsigma^l \cdot \ncap &=
\begin{bmatrix}
-p & \mu^l \PDi{y}{u} & 0 \\
\mu^l \PDi{y}{u} & -p & 0 \\
0 & 0 & 0
\end{bmatrix}
\begin{bmatrix}
0 \\
1 \\
0
\end{bmatrix} \\
&=
\begin{bmatrix}
\mu^l \PDi{y}{u} \\
-p \\
0
\end{bmatrix}
\end{align*}

So

\begin{equation}\label{eqn:continuumL12:230}
\Btau \cdot (\Bsigma^l \cdot \ncap)
=
\begin{bmatrix}
1 & 0 & 0
\end{bmatrix}
\begin{bmatrix}
\mu^l \PDi{y}{u} \\
-p \\
0
\end{bmatrix}
=
\mu^l \PD{y}{u}
\end{equation}

Our boundary value condition is therefore

\begin{equation}\label{eqn:continuumL12:250}
\evalbar{\mu^l \PD{y}{u^l}}{y = h} =
\evalbar{\mu^a \PD{y}{u^a}}{y = h}
\end{equation}

When can we decouple this, treating only the liquid?  Observe that we have

\begin{equation}\label{eqn:continuumL12:270}
\evalbar{\PD{y}{u^l}}{y = h} =
\evalbar{\frac{\mu^a}{\mu^l} \PD{y}{u^a}}{y = h}
\end{equation}

so if

\begin{equation}\label{eqn:continuumL12:290}
\frac{\mu_a}{\mu_l} \ll 1
\end{equation}

we can treat only the liquid portion of the problem, with a boundary value condition

\begin{equation}\label{eqn:continuumL12:310}
\evalbar{\PD{y}{u^l}}{y = h} = 0.
\end{equation}

Let's look at the component of the traction vector in the direction of the normal (liquid pressure acting on the air)

\begin{equation}\label{eqn:continuumL12:330}
\ncap \cdot (\Bsigma^l \cdot \ncap) = \ncap \cdot (\Bsigma^a \cdot \ncap)
\end{equation}

or

\begin{equation}\label{eqn:continuumL12:350}
\begin{bmatrix}
0 & 1 & 0
\end{bmatrix}
\begin{bmatrix}
\mu^l \PD{y}{u} \\
-p^l \\
0
\end{bmatrix}
= -\evalbar{p^l}{y = h} = -\evalbar{p^a}{y = h}
\end{equation}

i.e. We have pressure matching at the interface.

Our body force is

\begin{equation}\label{eqn:continuumL12:370}
\Bf =
\begin{bmatrix}
g \sin\alpha \\
-g \cos\alpha \\
0
\end{bmatrix}
\end{equation}

\paragraph{Navier-stokes}

Referring to the Navier-Stokes equation \ref{eqn:continuumL12:570}, we see that our only surviving parts are

\begin{subequations}
\begin{equation}\label{eqn:classicalMechanicsPs2:590a}
0 = -\PD{x}{p} + \mu \PDSq{y}{u} + \rho g \sin\alpha
\end{equation}
\begin{equation}\label{eqn:classicalMechanicsPs2:590b}
0 = -\PD{y}{p} - \rho g \cos\alpha
\end{equation}
\begin{equation}\label{eqn:classicalMechanicsPs2:590c}
0 = -\PD{z}{p}
\end{equation}
\end{subequations}

The last gives us $p \ne p(z)$.  Integrating the second we have

\begin{equation}\label{eqn:continuumL12:410}
p = \rho g y \cos\alpha + p_1
\end{equation}

Since $p = p_{\text{atm}}$ at $y = h$, we have

\begin{equation}\label{eqn:continuumL12:430}
p_{\text{atm}} = \rho g h \cos\alpha + p_1
\end{equation}

Our first Navier-Stokes equation \ref{eqn:classicalMechanicsPs2:590a} becomes

\begin{equation}\label{eqn:continuumL12:450}
0 = \mu \PDSq{y}{u} + \rho g \sin\alpha,
\end{equation}

or
\begin{equation}\label{eqn:continuumL12:470}
\PDSq{y}{u} = -\frac{\rho g}{\mu} \sin\alpha
\end{equation}

This we integrate twice

\begin{equation}\label{eqn:continuumL12:490}
u = - \rho g \frac{\sin\alpha}{2 \mu} y^2 + A y + B
\end{equation}

With

\begin{equation}\label{eqn:continuumL12:510a}
u(0) = 0,
\end{equation}

we see that $B = 0$, and with

\begin{equation}\label{eqn:continuumL12:510b}
\evalbar{\PD{y}{u}}{y = h} = 0,
\end{equation}

we find that

\begin{equation}\label{eqn:continuumL12:510c}
0 = - \rho g \frac{\sin\alpha}{\mu} h + A,
\end{equation}

for

\begin{equation}\label{eqn:classicalMechanicsPs2:530}
u = \rho g y \frac{\sin\alpha}{2 \mu} y \left( 2 h - y \right).
\end{equation}

This velocity distribution is illustrated figure (\ref{fig:continuumL12:continuumL12fig7}).

\imageFigure{figures/continuumL12fig7}{Velocity streamlines for flow down a plane}{fig:continuumL12:continuumL12fig7}{0.2}

It's important to note that in these problems we have to derive our boundary value conditions!  They are not given.

In this discussion, the height $h$ was assumed to be constant, with the tangential direction constant and parallel to the surface that the liquid is flowing on.  It's claimed in class that this is actually a consequence of surface tension only!
\FIXME{That's not at all intuitive, but will be covered when we learn about ``stability conditions''.  This was not actually covered in class.}
} % end example

   %
% Copyright � 2012 Peeter Joot.  All Rights Reserved.
% Licenced as described in the file LICENSE under the root directory of this GIT repository.
%

%
%
%\chapter{PHY454H1S Continuum Mechanics.  Lecture 13: Hydrostatics.  Surface normals and tangent vectors.  Taught by Prof. K. Das}
\label{chap:continuumL13}

\section{Steady state and static fluids}

Consider a sample volume of water, not moving with respect to the rest of the surrounding water.  If it is not moving the forces must be in balance.  What are the forces acting on this bit of fluid, considering a cylinder of the fluid above it as in \cref{fig:continuumL13:continuumL13Fig3a}
\imageFigure{../../figures/phy454/lec13_A_control_volume_of_fluid_in_a_fluidFig3a}{A control volume of fluid in a fluid}{fig:continuumL13:continuumL13Fig3a}{0.2}

In the column of fluid above the control volume \cref{fig:continuumL13:continuumL13Fig3b} we have
\imageFigure{../../figures/phy454/lec13_column_of_fluid_above_a_control_volumeFig3b}{column of fluid above a control volume}{fig:continuumL13:continuumL13Fig3b}{0.2}

\begin{equation}\label{eqn:continuumL13:20}
h A_w \rho g + p_A A_w = p_w A_w
\end{equation}

so
\begin{equation}\label{eqn:continuumL13:40}
p_w = h \rho g + p_A
\end{equation}

If we were to replace this blob of water with something of equal density, it should not change the dynamics (or statics) of the situations and that would not move.

We call this the

\makedefinition{Buoyancy force}{dfn:continuumL13:60}{Buoyancy force = weight of the equivalent volume of water - weight of the foreign body. \index{buoyancy force}}

If the densities are not equal, then we would have motion of the new bit of mass as depicted in \cref{fig:continuumL13:continuumL13Fig4}
\imageFigure{../../figures/phy454/lec13_A_mass_of_different_density_in_a_fluidFig4}{A mass of different density in a fluid}{fig:continuumL13:continuumL13Fig4}{0.2}

Consider a volume of ice floating on the surface of water, one with solid ice and one with partially frozen ice (with water or air or dirt or an anchor or anything else in it) as in \cref{fig:continuumL13:continuumL13Fig5}

\imageFigure{../../figures/phy454/lec13_Various_floating_ice_configurations_on_waterFig5}{Various floating ice configurations on water}{fig:continuumL13:continuumL13Fig5}{0.2}

No matter the situation, the water level will not change if the ice melts, because the total weight of the displaced water must have been matched by the weight of the unmelted ice plus additives.

Now what happens when we have fluid flows?  Consider \cref{fig:continuumL13:continuumL13Fig6}
\imageFigure{../../figures/phy454/lec13_flow_through_channel_with_different_aperturesFig6}{flow through channel with different apertures}{fig:continuumL13:continuumL13Fig6}{0.2}

Conservation of mass is going to mean that the masses of fluid flowing through any pair of cross sections will have to be equal

\begin{equation}\label{eqn:continuumL13:80}
\rho_1 A_1 v_1 = \rho_2 A_2 v_2,
\end{equation}

With incompressible fluids (\(\rho = \rho_1 = \rho_2\)) we have

\begin{equation}\label{eqn:continuumL13:100}
A_1 v_1 = A_2 v_2,
\end{equation}

so that if

\begin{equation}\label{eqn:continuumL13:120}
A_1 > A_2,
\end{equation}

we must have
\begin{equation}\label{eqn:continuumL13:140}
v_1 < v_2,
\end{equation}

to balance this.

In class this was illustrated with a pair of computer animations, one showing the deformation of patches of the fluid, and another showing how the velocities vary through the channel.  This is crudely depicted in \cref{fig:continuumL13:continuumL13Fig7}

\imageFigure{../../figures/phy454/lec13_area_and_velocity_flows_in_unequal_aperture_channel_configurationFig7}{area and velocity flows in unequal aperture channel configuration}{fig:continuumL13:continuumL13Fig7}{0.2}

We see the same behavior for channels that return to the original diameter after widening as in \cref{fig:continuumL13:continuumL13Fig8}

\imageFigure{../../figures/phy454/lec13_velocity_variation_in_channel_with_bulgeFig8}{velocity variation in channel with bulge}{fig:continuumL13:continuumL13Fig8}{0.2}

If we consider half of such a channel as in \cref{fig:continuumL13:continuumL13Fig9a}
\imageFigure{../../figures/phy454/lec13_vorticity_induction_due_to_pressure_gradients_in_unequal_aperture_channelFig9a}{vorticity induction due to pressure gradients in unequal aperture channel}{fig:continuumL13:continuumL13Fig9a}{0.2}

considering the flow around a small triangular section we must have a \textAndIndex{pressure gradient}, which induces a \textAndIndex{vorticity} flow.  We would see something similar in a rectangular channel where there is a block in the channel, as depicted in \cref{fig:continuumL13:continuumL13Fig9b}

\imageFigure{../../figures/phy454/lec13_vorticity_due_to_rectangular_blockageFig9b}{vorticity due to rectangular blockage}{fig:continuumL13:continuumL13Fig9b}{0.2}

\section{Height matching in odd geometries}

Let us consider an arbitrarily weird channel as in \cref{fig:continuumL13:continuumL13Fig10}

\imageFigure{../../figures/phy454/lec13_height_matching_in_odd_geometriesFig10}{height matching in odd geometries}{fig:continuumL13:continuumL13Fig10}{0.2}

This was also illustrated with a glass blown container in class as in \cref{fig:continuumL13:continuumL13Fig11}

\imageFigure{../../figures/phy454/lec13_a_physical_demonstration_with_glass_blown_apparatusFig11}{a physical demonstration with glass blown apparatus}{fig:continuumL13:continuumL13Fig11}{0.2}

In this real apparatus, we did not have exactly the same height (because of bubbles and capillary effects (surface tension induced meniscus curves), but we see first hand what we are talking about.

To account for this, we need to consider the situation in pieces as in \cref{fig:continuumL13:continuumL13Fig12}

\imageFigure{../../figures/phy454/lec13_column_volume_element_decomposition_for_odd_geometriesFig12}{column volume element decomposition for odd geometries}{fig:continuumL13:continuumL13Fig12}{0.2}

Breaking down the total pressure effects into individual bits, any column of fluid contributes to the pressure below it, even if that column of fluid is not directly on top of a continuous column of fluid all the way to the ``bottom''.


   % 
% 
% 
% Copyright � 2012 Peeter Joot
% All Rights Reserved
% 
% This file may be reproduced and distributed in whole or in part, without fee, subject to the following conditions:
% 
% o The copyright notice above and this permission notice must be preserved complete on all complete or partial copies.
% 
% o Any translation or derived work must be approved by the author in writing before distribution.
% 
% o If you distribute this work in part, instructions for obtaining the complete version of this file must be included, and a means for obtaining a complete version provided.
% 
% 
% Exceptions to these rules may be granted for academic purposes: Write to the author and ask.
% 
% 
% 
%%
% Copyright � 2015 Peeter Joot.  All Rights Reserved.
% Licenced as described in the file LICENSE under the root directory of this GIT repository.
%
\documentclass[]{eliblog}

\usepackage{amsmath}
\usepackage{mathpazo}

%
% shorthand for bold symbols, convenient for vectors and matrices
%
\newcommand{\Ba}[0]{\mathbf{a}}
\newcommand{\Bb}[0]{\mathbf{b}}
\newcommand{\Bc}[0]{\mathbf{c}}
\newcommand{\Bd}[0]{\mathbf{d}}
\newcommand{\Be}[0]{\mathbf{e}}
\newcommand{\Bf}[0]{\mathbf{f}}
\newcommand{\Bg}[0]{\mathbf{g}}
\newcommand{\Bh}[0]{\mathbf{h}}
\newcommand{\Bi}[0]{\mathbf{i}}
\newcommand{\Bj}[0]{\mathbf{j}}
\newcommand{\Bk}[0]{\mathbf{k}}
\newcommand{\Bl}[0]{\mathbf{l}}
\newcommand{\Bm}[0]{\mathbf{m}}
\newcommand{\Bn}[0]{\mathbf{n}}
\newcommand{\Bo}[0]{\mathbf{o}}
\newcommand{\Bp}[0]{\mathbf{p}}
\newcommand{\Bq}[0]{\mathbf{q}}
\newcommand{\Br}[0]{\mathbf{r}}
\newcommand{\Bs}[0]{\mathbf{s}}
\newcommand{\Bt}[0]{\mathbf{t}}
\newcommand{\Bu}[0]{\mathbf{u}}
\newcommand{\Bv}[0]{\mathbf{v}}
\newcommand{\Bw}[0]{\mathbf{w}}
\newcommand{\Bx}[0]{\mathbf{x}}
\newcommand{\By}[0]{\mathbf{y}}
\newcommand{\Bz}[0]{\mathbf{z}}
\newcommand{\BA}[0]{\mathbf{A}}
\newcommand{\BB}[0]{\mathbf{B}}
\newcommand{\BC}[0]{\mathbf{C}}
\newcommand{\BD}[0]{\mathbf{D}}
\newcommand{\BE}[0]{\mathbf{E}}
\newcommand{\BF}[0]{\mathbf{F}}
\newcommand{\BG}[0]{\mathbf{G}}
\newcommand{\BH}[0]{\mathbf{H}}
\newcommand{\BI}[0]{\mathbf{I}}
\newcommand{\BJ}[0]{\mathbf{J}}
\newcommand{\BK}[0]{\mathbf{K}}
\newcommand{\BL}[0]{\mathbf{L}}
\newcommand{\BM}[0]{\mathbf{M}}
\newcommand{\BN}[0]{\mathbf{N}}
\newcommand{\BO}[0]{\mathbf{O}}
\newcommand{\BP}[0]{\mathbf{P}}
\newcommand{\BQ}[0]{\mathbf{Q}}
\newcommand{\BR}[0]{\mathbf{R}}
\newcommand{\BS}[0]{\mathbf{S}}
\newcommand{\BT}[0]{\mathbf{T}}
\newcommand{\BU}[0]{\mathbf{U}}
\newcommand{\BV}[0]{\mathbf{V}}
\newcommand{\BW}[0]{\mathbf{W}}
\newcommand{\BX}[0]{\mathbf{X}}
\newcommand{\BY}[0]{\mathbf{Y}}
\newcommand{\BZ}[0]{\mathbf{Z}}

\newcommand{\Bzero}[0]{\mathbf{0}}
\newcommand{\Btheta}[0]{\boldsymbol{\theta}}
\newcommand{\Btau}[0]{\boldsymbol{\tau}}
\newcommand{\Bomega}[0]{\boldsymbol{\omega}}

%
% shorthand for unit vectors
%
\newcommand{\acap}[0]{\hat{\Ba}}
\newcommand{\bcap}[0]{\hat{\Bb}}
\newcommand{\ccap}[0]{\hat{\Bc}}
\newcommand{\dcap}[0]{\hat{\Bd}}
\newcommand{\ecap}[0]{\hat{\Be}}
\newcommand{\fcap}[0]{\hat{\Bf}}
\newcommand{\gcap}[0]{\hat{\Bg}}
\newcommand{\hcap}[0]{\hat{\Bh}}
\newcommand{\icap}[0]{\hat{\Bi}}
\newcommand{\jcap}[0]{\hat{\Bj}}
\newcommand{\kcap}[0]{\hat{\Bk}}
\newcommand{\lcap}[0]{\hat{\Bl}}
\newcommand{\mcap}[0]{\hat{\Bm}}
\newcommand{\ncap}[0]{\hat{\Bn}}
\newcommand{\ocap}[0]{\hat{\Bo}}
\newcommand{\pcap}[0]{\hat{\Bp}}
\newcommand{\qcap}[0]{\hat{\Bq}}
\newcommand{\rcap}[0]{\hat{\Br}}
\newcommand{\scap}[0]{\hat{\Bs}}
\newcommand{\tcap}[0]{\hat{\Bt}}
\newcommand{\ucap}[0]{\hat{\Bu}}
\newcommand{\vcap}[0]{\hat{\Bv}}
\newcommand{\wcap}[0]{\hat{\Bw}}
\newcommand{\xcap}[0]{\hat{\Bx}}
\newcommand{\ycap}[0]{\hat{\By}}
\newcommand{\zcap}[0]{\hat{\Bz}}
\newcommand{\thetacap}[0]{\hat{\Btheta}}

%
% to write R^n and C^n in a distinguishable fashion.  Perhaps change this
% to the double lined characters upon figuring out how to do so.
%
\newcommand{\C}[1]{$\mathbb{C}^{#1}$}
\newcommand{\R}[1]{$\mathbb{R}^{#1}$}

%
% various generally useful helpers
%

% derivative of #1 wrt. #2:
\newcommand{\D}[2] {\frac {d#2} {d#1}}

\newcommand{\inv}[1]{\frac{1}{#1}}
\newcommand{\cross}[0]{\times}

\newcommand{\abs}[1]{\lvert{#1}\rvert}
\newcommand{\norm}[1]{\lVert{#1}\rVert}
\newcommand{\innerprod}[2]{\langle{#1}, {#2}\rangle}
\newcommand{\dotprod}[2]{{#1} \cdot {#2}}
\newcommand{\bdotprod}[2]{\left({#1} \cdot {#2}\right)}
\newcommand{\crossprod}[2]{{#1} \cross {#2}}
\newcommand{\tripleprod}[3]{\dotprod{\left(\crossprod{#1}{#2}\right)}{#3}}

\DeclareMathOperator{\Proj}{Proj}
\DeclareMathOperator{\Span}{span}
\DeclareMathOperator{\Sgn}{sgn}
\DeclareMathOperator{\Area}{Area}
\DeclareMathOperator{\Volume}{Volume}

%
% A few miscellaneous things specific to this document
%
\newcommand{\crossop}[1]{\crossprod{#1}{}}

% R2 vector.
\newcommand{\VectorTwo}[2]{
\begin{bmatrix}
 {#1} \\
 {#2}
\end{bmatrix}
}

\newcommand{\VectorN}[1]{
\begin{bmatrix}
{#1}_1 \\
{#1}_2 \\
\vdots \\
{#1}_N \\
\end{bmatrix}
}

\newcommand{\DETuvij}[4]{
\begin{vmatrix}
 {#1}_{#3} & {#1}_{#4} \\
 {#2}_{#3} & {#2}_{#4}
\end{vmatrix}
}

\newcommand{\DETuvwijk}[6]{
\begin{vmatrix}
 {#1}_{#4} & {#1}_{#5} & {#1}_{#6} \\
 {#2}_{#4} & {#2}_{#5} & {#2}_{#6} \\
 {#3}_{#4} & {#3}_{#5} & {#3}_{#6}
\end{vmatrix}
}

\newcommand{\DETuvwxijkl}[8]{
\begin{vmatrix}
 {#1}_{#5} & {#1}_{#6} & {#1}_{#7} & {#1}_{#8} \\
 {#2}_{#5} & {#2}_{#6} & {#2}_{#7} & {#2}_{#8} \\
 {#3}_{#5} & {#3}_{#6} & {#3}_{#7} & {#3}_{#8} \\
 {#4}_{#5} & {#4}_{#6} & {#4}_{#7} & {#4}_{#8} \\
\end{vmatrix}
}

%\newcommand{\DETuvwxyijklm}[10]{
%\begin{vmatrix}
% {#1}_{#6} & {#1}_{#7} & {#1}_{#8} & {#1}_{#9} & {#1}_{#10} \\
% {#2}_{#6} & {#2}_{#7} & {#2}_{#8} & {#2}_{#9} & {#2}_{#10} \\
% {#3}_{#6} & {#3}_{#7} & {#3}_{#8} & {#3}_{#9} & {#3}_{#10} \\
% {#4}_{#6} & {#4}_{#7} & {#4}_{#8} & {#4}_{#9} & {#4}_{#10} \\
% {#5}_{#6} & {#5}_{#7} & {#5}_{#8} & {#5}_{#9} & {#5}_{#10}
%\end{vmatrix}
%}

% R3 vector.
\newcommand{\VectorThree}[3]{
\begin{bmatrix}
 {#1} \\
 {#2} \\
 {#3}
\end{bmatrix}
}



\author{Peeter Joot}
\email{peeter.joot@gmail.com}

%\documentclass[]{eliblogwidescreen}

\usepackage{amsmath}
\usepackage{mathpazo}

%
% shorthand for bold symbols, convenient for vectors and matrices
%
\newcommand{\Ba}[0]{\mathbf{a}}
\newcommand{\Bb}[0]{\mathbf{b}}
\newcommand{\Bc}[0]{\mathbf{c}}
\newcommand{\Bd}[0]{\mathbf{d}}
\newcommand{\Be}[0]{\mathbf{e}}
\newcommand{\Bf}[0]{\mathbf{f}}
\newcommand{\Bg}[0]{\mathbf{g}}
\newcommand{\Bh}[0]{\mathbf{h}}
\newcommand{\Bi}[0]{\mathbf{i}}
\newcommand{\Bj}[0]{\mathbf{j}}
\newcommand{\Bk}[0]{\mathbf{k}}
\newcommand{\Bl}[0]{\mathbf{l}}
\newcommand{\Bm}[0]{\mathbf{m}}
\newcommand{\Bn}[0]{\mathbf{n}}
\newcommand{\Bo}[0]{\mathbf{o}}
\newcommand{\Bp}[0]{\mathbf{p}}
\newcommand{\Bq}[0]{\mathbf{q}}
\newcommand{\Br}[0]{\mathbf{r}}
\newcommand{\Bs}[0]{\mathbf{s}}
\newcommand{\Bt}[0]{\mathbf{t}}
\newcommand{\Bu}[0]{\mathbf{u}}
\newcommand{\Bv}[0]{\mathbf{v}}
\newcommand{\Bw}[0]{\mathbf{w}}
\newcommand{\Bx}[0]{\mathbf{x}}
\newcommand{\By}[0]{\mathbf{y}}
\newcommand{\Bz}[0]{\mathbf{z}}
\newcommand{\BA}[0]{\mathbf{A}}
\newcommand{\BB}[0]{\mathbf{B}}
\newcommand{\BC}[0]{\mathbf{C}}
\newcommand{\BD}[0]{\mathbf{D}}
\newcommand{\BE}[0]{\mathbf{E}}
\newcommand{\BF}[0]{\mathbf{F}}
\newcommand{\BG}[0]{\mathbf{G}}
\newcommand{\BH}[0]{\mathbf{H}}
\newcommand{\BI}[0]{\mathbf{I}}
\newcommand{\BJ}[0]{\mathbf{J}}
\newcommand{\BK}[0]{\mathbf{K}}
\newcommand{\BL}[0]{\mathbf{L}}
\newcommand{\BM}[0]{\mathbf{M}}
\newcommand{\BN}[0]{\mathbf{N}}
\newcommand{\BO}[0]{\mathbf{O}}
\newcommand{\BP}[0]{\mathbf{P}}
\newcommand{\BQ}[0]{\mathbf{Q}}
\newcommand{\BR}[0]{\mathbf{R}}
\newcommand{\BS}[0]{\mathbf{S}}
\newcommand{\BT}[0]{\mathbf{T}}
\newcommand{\BU}[0]{\mathbf{U}}
\newcommand{\BV}[0]{\mathbf{V}}
\newcommand{\BW}[0]{\mathbf{W}}
\newcommand{\BX}[0]{\mathbf{X}}
\newcommand{\BY}[0]{\mathbf{Y}}
\newcommand{\BZ}[0]{\mathbf{Z}}

\newcommand{\Bzero}[0]{\mathbf{0}}
\newcommand{\Btheta}[0]{\boldsymbol{\theta}}
\newcommand{\Btau}[0]{\boldsymbol{\tau}}
\newcommand{\Bomega}[0]{\boldsymbol{\omega}}

%
% shorthand for unit vectors
%
\newcommand{\acap}[0]{\hat{\Ba}}
\newcommand{\bcap}[0]{\hat{\Bb}}
\newcommand{\ccap}[0]{\hat{\Bc}}
\newcommand{\dcap}[0]{\hat{\Bd}}
\newcommand{\ecap}[0]{\hat{\Be}}
\newcommand{\fcap}[0]{\hat{\Bf}}
\newcommand{\gcap}[0]{\hat{\Bg}}
\newcommand{\hcap}[0]{\hat{\Bh}}
\newcommand{\icap}[0]{\hat{\Bi}}
\newcommand{\jcap}[0]{\hat{\Bj}}
\newcommand{\kcap}[0]{\hat{\Bk}}
\newcommand{\lcap}[0]{\hat{\Bl}}
\newcommand{\mcap}[0]{\hat{\Bm}}
\newcommand{\ncap}[0]{\hat{\Bn}}
\newcommand{\ocap}[0]{\hat{\Bo}}
\newcommand{\pcap}[0]{\hat{\Bp}}
\newcommand{\qcap}[0]{\hat{\Bq}}
\newcommand{\rcap}[0]{\hat{\Br}}
\newcommand{\scap}[0]{\hat{\Bs}}
\newcommand{\tcap}[0]{\hat{\Bt}}
\newcommand{\ucap}[0]{\hat{\Bu}}
\newcommand{\vcap}[0]{\hat{\Bv}}
\newcommand{\wcap}[0]{\hat{\Bw}}
\newcommand{\xcap}[0]{\hat{\Bx}}
\newcommand{\ycap}[0]{\hat{\By}}
\newcommand{\zcap}[0]{\hat{\Bz}}
\newcommand{\thetacap}[0]{\hat{\Btheta}}

%
% to write R^n and C^n in a distinguishable fashion.  Perhaps change this
% to the double lined characters upon figuring out how to do so.
%
\newcommand{\C}[1]{$\mathbb{C}^{#1}$}
\newcommand{\R}[1]{$\mathbb{R}^{#1}$}

%
% various generally useful helpers
%

% derivative of #1 wrt. #2:
\newcommand{\D}[2] {\frac {d#2} {d#1}}

\newcommand{\inv}[1]{\frac{1}{#1}}
\newcommand{\cross}[0]{\times}

\newcommand{\abs}[1]{\lvert{#1}\rvert}
\newcommand{\norm}[1]{\lVert{#1}\rVert}
\newcommand{\innerprod}[2]{\langle{#1}, {#2}\rangle}
\newcommand{\dotprod}[2]{{#1} \cdot {#2}}
\newcommand{\bdotprod}[2]{\left({#1} \cdot {#2}\right)}
\newcommand{\crossprod}[2]{{#1} \cross {#2}}
\newcommand{\tripleprod}[3]{\dotprod{\left(\crossprod{#1}{#2}\right)}{#3}}

\DeclareMathOperator{\Proj}{Proj}
\DeclareMathOperator{\Span}{span}
\DeclareMathOperator{\Sgn}{sgn}
\DeclareMathOperator{\Area}{Area}
\DeclareMathOperator{\Volume}{Volume}

%
% A few miscellaneous things specific to this document
%
\newcommand{\crossop}[1]{\crossprod{#1}{}}

% R2 vector.
\newcommand{\VectorTwo}[2]{
\begin{bmatrix}
 {#1} \\
 {#2}
\end{bmatrix}
}

\newcommand{\VectorN}[1]{
\begin{bmatrix}
{#1}_1 \\
{#1}_2 \\
\vdots \\
{#1}_N \\
\end{bmatrix}
}

\newcommand{\DETuvij}[4]{
\begin{vmatrix}
 {#1}_{#3} & {#1}_{#4} \\
 {#2}_{#3} & {#2}_{#4}
\end{vmatrix}
}

\newcommand{\DETuvwijk}[6]{
\begin{vmatrix}
 {#1}_{#4} & {#1}_{#5} & {#1}_{#6} \\
 {#2}_{#4} & {#2}_{#5} & {#2}_{#6} \\
 {#3}_{#4} & {#3}_{#5} & {#3}_{#6}
\end{vmatrix}
}

\newcommand{\DETuvwxijkl}[8]{
\begin{vmatrix}
 {#1}_{#5} & {#1}_{#6} & {#1}_{#7} & {#1}_{#8} \\
 {#2}_{#5} & {#2}_{#6} & {#2}_{#7} & {#2}_{#8} \\
 {#3}_{#5} & {#3}_{#6} & {#3}_{#7} & {#3}_{#8} \\
 {#4}_{#5} & {#4}_{#6} & {#4}_{#7} & {#4}_{#8} \\
\end{vmatrix}
}

%\newcommand{\DETuvwxyijklm}[10]{
%\begin{vmatrix}
% {#1}_{#6} & {#1}_{#7} & {#1}_{#8} & {#1}_{#9} & {#1}_{#10} \\
% {#2}_{#6} & {#2}_{#7} & {#2}_{#8} & {#2}_{#9} & {#2}_{#10} \\
% {#3}_{#6} & {#3}_{#7} & {#3}_{#8} & {#3}_{#9} & {#3}_{#10} \\
% {#4}_{#6} & {#4}_{#7} & {#4}_{#8} & {#4}_{#9} & {#4}_{#10} \\
% {#5}_{#6} & {#5}_{#7} & {#5}_{#8} & {#5}_{#9} & {#5}_{#10}
%\end{vmatrix}
%}

% R3 vector.
\newcommand{\VectorThree}[3]{
\begin{bmatrix}
 {#1} \\
 {#2} \\
 {#3}
\end{bmatrix}
}



\author{Peeter Joot}
\email{peeter.joot@gmail.com}


%\chapter{PHY454H1S Continuum Mechanics.  Lecture 14: Non-dimensionality and scaling.  Taught by Prof. K. Das.}
\chapter{Non-dimensionality and scaling.}
\label{chap:continuumL14}
\blogpage{http://sites.google.com/site/peeterjoot2/math2012/continuumL14.pdf}
%\date{Mar 7, 2012}
\gitRevisionInfo{continuumL14}

\section{Review.  Surfaces}

We are considering a surface as depicted in (\ref{fig:continuumL14:continuumL14Fig13})

\imageFigure{continuumL13Fig13}{Variable surface geometries}{fig:continuumL14:continuumL14Fig13}{0.2}

With the surface height given by

\begin{equation}\label{eqn:continuumL14:10}
z = h(x, t),
\end{equation}

where this describes the interface.  Taking the difference

\begin{equation}\label{eqn:continuumL14:30}
\phi = z - h(x, t) = 0,
\end{equation}

we define a surface.  We considered a small displacement as in (\ref{fig:continuumL14:continuumL14Fig14}).

\imageFigure{continuumL13Fig14}{A vector differential element}{fig:continuumL14:continuumL14Fig14}{0.2}

Recall that if $\phi$ is a constant, then $\spacegrad \phi$ is a normal to the surface.  We showed this by considering the differential

\begin{align*}
0 
&= d\phi \\
&= 
\PD{x}{\phi} dx
+\PD{y}{\phi} dy
+\PD{z}{\phi} dz \\
&=
(\spacegrad \phi) \cdot d\Br.
\end{align*}

We can construct the unit normal by scaling.  For our 1D example we have

\begin{align*}
\ncap 
&= \frac{\spacegrad \phi}{\Abs{\spacegrad \phi}} \\
&= \inv{\Abs{\spacegrad \phi}} 
\left(
\PD{x}{\phi},
\PD{y}{\phi}
\right) 
\end{align*}

so that our unit normal is
\begin{equation}\label{eqn:continuumL14:50}
\ncap 
= \inv{ \sqrt{1 + (h')^2}}
\left( -\PD{x}{h}, 1 \right)
\end{equation}

A unit tangent can also be constructed by inspection

\begin{equation}\label{eqn:continuumL14:70}
\taucap 
= \inv{ \sqrt{1 + (h')^2}}
\left( 1, \PD{x}{h} \right).
\end{equation}

\section{Traction vector at the interface.}

Recall that our stress tensor has the form

\begin{equation}\label{eqn:continuumL14:90}
T_{ij} = 
- p \delta_{ij} + \rho \nu \left( 
\PD{x_j}{u_i}
+\PD{x_i}{u_j}
\right)
\end{equation}

(here we are switching notations for the stress since we will be using $\sigma$ for surface tension in this section)

The traction vector components are

\begin{equation}\label{eqn:continuumL14:110}
t_i = T_{ij} n_j =
- p n_i + \rho \nu \left( 
\PD{x_j}{u_i}
+\PD{x_i}{u_j}
\right) n_j
\end{equation}

Considering a control volume as illustrated in we can arrive at what we call the jump stress balance equation

figure (\ref{fig:continuumL14:continuumL14fig3})
\imageFigure{continuumL14fig3}{Control volume for liquid air interface}{fig:continuumL14:continuumL14fig3}{0.2}

\begin{equation}\label{eqn:continuumL14:130}
[\BT \ncap]^2_1 = \frac{2 \sigma}{R} \ncap - \spacegrad_I \sigma
\end{equation}

where

\begin{align}\label{eqn:continuumL14:150}
\sigma &= \text{surface tension} \\
R &= \text{radius of curvature} \\
\spacegrad_I &= \text{gradient along the interface}
\end{align}

and the suffix $2$ and prefix $1$ indicates that we are considering the interface between fluids labeled $1$ and $2$ (liquid and air respectively in the diagram).

For a derivation see Prof after class?

Force balance along the normal direction gives

\begin{equation}\label{eqn:continuumL14:170}
\ncap [\BT \ncap]^2_1 = \ncap \cdot \frac{2 \sigma}{R} \ncap - \cancel{\ncap \cdot (\spacegrad_I \sigma)}
\end{equation}

If you do this calculation, you will get 

\begin{equation}\label{eqn:continuumL14:190}
[-p]^2_1 = \frac{ 2 \sigma}{R}
\end{equation}

I think this was called the Laplace equation?

Question: How was $\sigma$ defined?  A: Energy per unit area.  

Figure (\ref{fig:continuumL14:continuumL14fig4}) was given as part of an explanation of surface tension and curvature, but I missed part of that discussion.  Perhaps this is elaborated on in the class notes?

\imageFigure{continuumL14fig4}{Molecular gas and liquid interactions at a surface.}{fig:continuumL14:continuumL14fig4}{0.2}

Reading: An treatment of this topic that looks complete enough to understand looks like it can be found in \S 7 of \cite{landau1987course}.

\section{Non dimensionalization and scaling}

\subsection{Motivation.}

By scaling we mean how much detail do you want to look at in the analysis.  Consider the figure (\ref{fig:continuumL14:continuumL14fig5a}) where we imagine that we zoom in on something that appears smooth from a distance.  However, we are free to perform a change of variables on our coordinates and rescale in any arbitrary fashion.  For example

\imageFigure{continuumL14fig5a}{Coarse scaling example.}{fig:continuumL14:continuumL14fig5a}{0.2}

\begin{align}\label{eqn:continuumL14:210}
x &\rightarrow A u^\alpha \\
y &\rightarrow B v^\beta
\end{align}

For a linear zoom scaling ($\alpha = \beta = 1$) we could perhaps find that we have something very granular close up as in figure (\ref{fig:continuumL14:continuumL14fig5b}).  Picking the length scale to be used in this case can be very important.

\imageFigure{continuumL14fig5b}{Fine grain scaling example (a zoom).}{fig:continuumL14:continuumL14fig5b}{0.2}

The flexiblity to rescale with non unity values for $\alpha$ and $\beta$ can, for example, come in handy, should we choose to rescale time and position differently.

\subsection{Rescaling by characteristic length and velocity.}

Suppose that a fluid is flowing with

\begin{itemize}
\item a characteristic velocity $U$, with dimensions $[U] \sim L T^{-1}$
\item a characteristic length scale $L$
\end{itemize}

Considering the dimensions of the terms in the Navier-Stokes equation

\begin{equation}\label{eqn:continuumL14:230}
[\rho] = M L^{-3}
\end{equation}
\begin{equation}\label{eqn:continuumL14:250}
[p] = M L T^{-2} L^{-2} = M L^{-1} T^{-2}
\end{equation}
\begin{equation}\label{eqn:continuumL14:270}
[t] = T = \frac{L}{U}
\end{equation}

so 
\begin{equation}\label{eqn:continuumL14:290}
[p] = [\rho U^2] = M L^{-3} L^2 T^{-2}  = M L^{-1} T^{-2}
\end{equation}

Now let's alter the Navier-Stokes equation using some scaling to put it into a dimensionless form

\begin{equation}\label{eqn:continuumL14:310}
\PD{t}{\Bu} + (\Bu \cdot \spacegrad) \Bu = - \inv{\rho} \spacegrad + \nu \spacegrad^2 \Bu
\end{equation}

\begin{equation}\label{eqn:continuumL14:330}
\PD{t}{\Bu} \rightarrow  \PD{\left(\frac{L}{U} t'\right)}{(U \Bu')} = \frac{U^2}{L} \PD{t'}{\Bu'}
\end{equation}

\begin{equation}\label{eqn:continuumL14:350}
\spacegrad = 
\xcap \PD{x}{}
+\ycap \PD{y}{}
+\zcap \PD{z}{}
\rightarrow 
\xcap \PD{L x'}{}
\ycap \PD{L y'}{}
\zcap \PD{L z'}{}
\end{equation}

so that 

\begin{equation}\label{eqn:continuumL14:370}
\spacegrad \rightarrow \inv{L} \spacegrad'
\end{equation}

\begin{equation}\label{eqn:continuumL14:390}
(\Bu \cdot \spacegrad ) \Bu \rightarrow 
\left( U \Bu' \cdot \inv{L} \spacegrad' \right) U \Bu' = \frac{U^2}{L} (\Bu' \cdot \spacegrad') \Bu'
\end{equation}

\begin{equation}\label{eqn:continuumL14:410}
\inv{\rho} \spacegrad p \rightarrow \inv{L} \frac{\spacegrad' (\cancel{\rho} U^2) }{\cancel{\rho}} p' = \frac{U^2}{L} \spacegrad' p'
\end{equation}

\begin{equation}\label{eqn:continuumL14:430}
\nu \spacegrad^2 \Bu \rightarrow \frac{\nu}{L^2} \spacegrad' U \Bu' = \frac{\nu U}{L^2} \spacegrad' \Bu'
\end{equation}

Putting everything together, Navier-Stokes takes the form

\begin{equation}\label{eqn:continuumL14:450}
\frac{U^2}{L} \PD{t'}{\Bu'} + \frac{U^2}{L} (\Bu' \cdot \spacegrad') \Bu' = \frac{U^2}{L} \spacegrad' p' + \frac{\nu U}{L^2} \spacegrad' \Bu'
\end{equation}

or
\begin{equation}\label{eqn:continuumL14:470}
\PD{t'}{\Bu'} + (\Bu' \cdot \spacegrad') \Bu' = \spacegrad' p' + \frac{\nu}{U L} \spacegrad' \Bu'
\end{equation}

Introducing the Reynold's number

\begin{equation}\label{eqn:continuumL14:490}
R = \frac{L U}{\nu}
\end{equation}

We have Navier-Stokes in dimensionless form

\begin{equation}\label{eqn:continuumL14:510}
\PD{t'}{\Bu'} + (\Bu' \cdot \spacegrad') \Bu' = \spacegrad' p' + \inv{R} \spacegrad' \Bu'
\end{equation}

The implications of this will be discussed further in the next lecture.

%FIXME: Reading: \S XX from \cite{acheson1990elementary}
Reading: Coverage of this topic (with some problems) can be found in \S 7.6, \S 7.7 of \cite{granger1995fluid}.

\EndArticle

   \include{fluids/surfaceTensionAndReynoldsNumber}
   %
% Copyright � 2012 Peeter Joot.  All Rights Reserved.
% Licenced as described in the file LICENSE under the root directory of this GIT repository.
%

%
%
\label{chap:continuumL16}
\section{Time dependent flow}

Suppose we have an obstacle to fluid flow, as in \cref{fig:continuumL16:continuumL16Fig1}
\imageFigure{../../figures/phy454/lec16_Flow_lines_around_circular_obstacleFig1}{Flow lines around circular obstacle}{fig:continuumL16:continuumL16Fig1}{0.2}

we have a couple conditions on fluid flow.

\begin{enumerate}
\item No fluid can cross the solid boundary
\item Due to viscosity the tangential velocity of the fluid should match the velocity of the solid boundary.
\end{enumerate}

In study of this type of flow, we can consider the flow separated into two portions, one is a flow that is largely viscous, and the other that is largely inertial.  This is depicted in \cref{fig:continuumL16:continuumL16Fig2}

\pdfTexFigure{../../figures/phy454/continuumL16Fig2.pdf_tex}{Viscous and inviscous regions in boundary layer flow}{fig:continuumL16:continuumL16Fig2}{0.5}

We call the study of these two regions boundary layer flow.

\section{Unsteady rectilinear flow} \index{rectilinear flow}

Given

\begin{equation}\label{eqn:continuumL16:10}
\Bu = u(x, y, z, t) \xcap
\end{equation}

Our continuity equation (incompressibility assumption) is

\begin{equation}\label{eqn:continuumL16:30}
\PD{x}{u} +
\cancel{\PD{y}{v}} +
\cancel{\PD{z}{w}} = 0.
\end{equation}

Our non-linear term of Navier-Stokes is then killed

\begin{equation}\label{eqn:continuumL16:50}
u \PD{x}{u} = 0
\end{equation}

so that Navier Stokes takes the form

\begin{subequations}
\begin{equation}\label{eqn:continuumL16:70}
\rho \PD{t}{u} = -\PD{x}{p} +
\mu \left( \PDSq{y}{u} +\PDSq{z}{u} \right)
\end{equation}
\begin{equation}\label{eqn:continuumL16:90}
0 = -\PD{y}{p}
\end{equation}
\begin{equation}\label{eqn:continuumL16:110}
0 = -\PD{z}{p}
\end{equation}
\end{subequations}

Taking the \(x\) partial of \eqnref{eqn:continuumL16:70}, we have

\begin{equation}\label{eqn:continuumL16:130}
\rho \PD{t}{} \cancel{\PD{x}{u}} = -\PDSq{x}{p} + \mu \left( \PDSq{y}{} \cancel{\PD{x}{u}} +\PDSq{z}{} \cancel{\PD{x}{u}} \right)
\end{equation}

Since we also have zero partials in the \(y\) and \(z\) directions from \eqnref{eqn:continuumL16:90}, and \eqnref{eqn:continuumL16:110}, we must then have

\begin{equation}\label{eqn:continuumL16:150}
\frac{d^2 p}{dx^2} = 0.
\end{equation}

So, after integrating we find for the pressure

\begin{equation}\label{eqn:continuumL16:170}
p(x, t) = p_0(t) - G x,
\end{equation}

where

\begin{equation}\label{eqn:continuumL16:190}
G = -\frac{dp}{dx}.
\end{equation}

In general \(G\) is a function of \(t\), but constant in space.  Given this, we have for our Navier-Stokes equation

\begin{equation}\label{eqn:continuumL16:210}
\rho \PD{t}{u} = G(t) + \mu \left( \PDSq{y}{u} +\PDSq{z}{u} \right)
\end{equation}

\makeexample{Impulsively started flow}{ex:boundaries:impulsive}{

Reading: \S 2 from \citep{acheson1990elementary}

Let us consider a flow driven by a moving boundary.  We have two ways that we can look at such a flow, the first of which is with the fluid fixed and the boundary moving and the second is with the fluid moving and the boundary fixed.  This are depicted respectively in \cref{fig:continuumL16:continuumL16Fig3a}) and \cref{fig:continuumL16:continuumL16Fig3b}

\imageFigure{../../figures/phy454/lec16_Lagrangian_viewFig3a}{Lagrangian view}{fig:continuumL16:continuumL16Fig3a}{0.2}
\imageFigure{../../figures/phy454/lec16_Eulerian_viewFig3b}{Eulerian view}{fig:continuumL16:continuumL16Fig3b}{0.2}

These two possible viewpoints can be called the Eulerian and the Lagrangian views where

\begin{itemize}
\item Lagrangian: the observer is moving with the fluid.
\item Eulerian: the observer is fixed in space, watching the fluid.
\end{itemize}

With a flow of the form

\begin{equation}\label{eqn:continuumL16:230}
\Bu = u(y, t) \xcap
\end{equation}

Navier-Stokes equation

\boxedEquation{eqn:continuumL16:250}{
\PD{t}{u} = \nu \PDSq{y}{u}.
}

Our boundary value constraints are

\begin{equation}\label{eqn:continuumL16:270}
u(0, t) =
\left\{
\begin{array}{l l}
0 & \quad \mbox{for \(t < 0\)} \\
U & \quad \mbox{for \(t \ge 0\)}
\end{array}
\right.
\end{equation}

and \(u \rightarrow 0\) as \(y \rightarrow \infty\).

If we make a transformation to dimensionless arguments

\begin{equation}\label{eqn:continuumL16:290}
u \rightarrow U
\end{equation}

so that

\begin{equation}\label{eqn:continuumL16:310}
\frac{u}{U} \rightarrow \text{dimensionless}
\end{equation}

Then we require of the parameters

\begin{equation}\label{eqn:continuumL16:330}
y, \nu, t \rightarrow \frac{y}{\sqrt{\nu t}}
\end{equation}

so that we have a characteristic length scale of the form

\begin{equation}\label{eqn:continuumL16:350}
\delta \rightarrow \sqrt{\nu t}
\end{equation}

We can find an approximate solution

\begin{equation}\label{eqn:continuumL16:370}
\frac{U}{t} \approx \frac{\nu U}{\delta^2}.
\end{equation}

We can introduce a similarity variable (often hard to find), of the form

\begin{equation}\label{eqn:continuumL16:390}
\eta = \frac{y}{2 \sqrt{\nu t}}.
\end{equation}

\FIXME{try attacking this more systematically using Fourier or Laplace transforms (probably a Laplace transform, because of our initial conditions.)}

Let us use our similarity variable and see what happens.  With

\begin{equation}\label{eqn:continuumL16:410}
u = U f(\eta),
\end{equation}

we find

\begin{equation}\label{eqn:boundaryLayersAndTimeDependentFlow:770}
\begin{aligned}
\PD{y}{u}
&= \PD{\eta}{u} \PD{y}{\eta} \\
&= U f' \inv{2 \sqrt{\nu t}}
\end{aligned}
\end{equation}

where

\begin{equation}\label{eqn:continuumL16:430}
f' = \PD{\eta}{f}
\end{equation}

We then find

\begin{equation}\label{eqn:continuumL16:450}
\PDSq{y}{u} = U f'' \inv{4 \nu t}
\end{equation}

and

\begin{equation}\label{eqn:boundaryLayersAndTimeDependentFlow:790}
\begin{aligned}
\PD{t}{u}
&= \PD{\eta}{u} \PD{t}{\eta} \\
&= \PD{\eta}{u} \PD{t}{}\left(
\frac{y}{2 \sqrt{\nu t}}
\right) \\
&= -\inv{2} \PD{\eta}{u} \left(
\frac{y}{2 \sqrt{\nu} t^{3/2}}
\right) \\
&= - U f' \frac{\eta}{2 t}
\end{aligned}
\end{equation}

Putting these all together we have

\begin{equation}\label{eqn:continuumL16:630}
- \cancel{U} f' \frac{\eta}{2 t} = \cancel{\nu} \cancel{U} f'' \inv{4 \cancel{\nu} t},
\end{equation}

or

\begin{equation}\label{eqn:continuumL16:490}
f'' + 2 \eta f' = 0
\end{equation}

With \(g = f'\), we have

\begin{equation}\label{eqn:continuumL16:650}
\int \frac{dg}{g} = \int -2 \eta y
\end{equation}

With solution

\begin{equation}\label{eqn:continuumL16:670}
\ln(f') = - \eta^2 + \ln C
\end{equation}

or

\begin{equation}\label{eqn:continuumL16:690}
f' = C e^{- \eta^2}.
\end{equation}

Integrating once more, and writing the integral in terms of the error function

\begin{equation}\label{eqn:continuumL16:530}
\erf(\eta) = \frac{2}{\sqrt{\pi}} \int_0^\eta e^{-s^2} ds,
\end{equation}

We find

\begin{equation}\label{eqn:continuumL16:510}
f(\eta) = A \erf(\eta) + B.
\end{equation}

From our boundary value condition at the origin we have

\begin{equation}\label{eqn:continuumL16:710}
u(0, t) = U f(0) = U,
\end{equation}

so that

\begin{equation}\label{eqn:continuumL16:730}
U( A \erf(0) + B) = U.
\end{equation}

Since \(\erf(0) = 0\), we must have \(B = 1\).  For our boundary value constraint far from the impulse, we have

\begin{equation}\label{eqn:continuumL16:750}
u(\infty, t) = U( A \erf(\infty) + 1) = 0
\end{equation}

but since \(\erf(\infty) = 1\), we must have \(A = -1\).  Our solution is then found to be

\boxedEquation{eqn:continuumL16:590}{
u(y, t) = U(1 - \erf(\eta)).
}

where (again)

\begin{equation}\label{eqn:continuumL16:610}
\eta = \frac{y}{2 \sqrt{\nu t}}.
\end{equation}

Explicitly, this is

\begin{equation}\label{eqn:continuumL16:590b}
u(y, t) = U_0 \left(1 - \erf\left(\frac{y}{2 \sqrt{\nu t}}\right) \right)
\end{equation}

%\unnumberedSubsection{Boundary layer thickness}
If we look at the thickness of the boundary layer for different viscosities, sampled at different times we may end up with curves as in \cref{fig:continuumL16:continuumL16Fig4a}


\pdfTexFigure{../../figures/phy454/continuumL16Fig4a.pdf_tex}{Plots of separation thickness for different viscosities}{fig:continuumL16:continuumL16Fig4a}{0.4}

However, it turns out that for any fluids, regardless of the viscosities, the thickness of the boundary layers generally vary as a linear function of \(\sqrt{\nu t}\) so if \(\delta\) is plotted against that as in \cref{fig:continuumL16:continuumL16Fig4b} we see a linear relationship.


\pdfTexFigure{../../figures/phy454/continuumL16Fig4b.pdf_tex}{Linear relation between separation thickness and \(\sqrt{\nu t}\)}{fig:continuumL16:continuumL16Fig4b}{0.4}
} % end example

   %
% Copyright � 2012 Peeter Joot.  All Rights Reserved.
% Licenced as described in the file LICENSE under the root directory of this GIT repository.
%

%
%

%\chapter{PHY454H1S Continuum Mechanics.  Lecture 17: Impulsive flow.  Boundary layers.  Oscillatory driven flow.  Taught by Prof. K. Das}
\label{chap:continuumL17}

%\section{Impulsive flow.  Boundary layers.  Oscillatory driven flow}
\section{Review.  Impulsively started flow}

Were looking at flow driven by an impulse, a sudden motion of the plate, as in \cref{fig:continuumL17:continuumL17Fig1}
\imageFigure{../../figures/phy454/lec17_Impulsively_driven_time_dependent_fluid_flowFig1}{Impulsively driven time dependent fluid flow}{fig:continuumL17:continuumL17Fig1}{0.3}

where the fluid at the origin is pushed so that it is given the velocity

\begin{equation}\label{eqn:continuumL17:20}
u(0, t) =
\left\{
\begin{array}{l l}
0 & \quad \mbox{for \(t < 0\)} \\
U(t) & \quad \mbox{for \(t \ge 0\)} \\
\end{array}
\right.
\end{equation}

where \(U \rightarrow 0\) as \(y \rightarrow \infty\).

Navier-Stokes takes the form

\begin{equation}\label{eqn:continuumL17:40}
\PD{t}{u} = \nu \PDSq{y}{u}.
\end{equation}

With a similarity variable

\begin{equation}\label{eqn:continuumL17:60}
\eta = \frac{y}{2 \sqrt{\nu t}},
\end{equation}

and

\begin{equation}\label{eqn:continuumL17:80}
u = U f(\eta),
\end{equation}

we found that we needed to solve

\begin{equation}\label{eqn:continuumL17:100}
f'' + 2 \eta f' = 0
\end{equation}

where

\begin{equation}\label{eqn:continuumL17:120}
f' = \frac{df}{d\eta}
\end{equation}

with solution

\begin{equation}\label{eqn:continuumL17:140}
u(y, t) = U(1 - \erf(\eta)).
\end{equation}

Here, we have used the error function

\begin{equation}\label{eqn:continuumL17:160}
\erf(\eta) = \frac{2}{\sqrt{\pi}} \int_0^\eta e^{-s^2} ds,
\end{equation}

as plotted in \cref{fig:continuumL17:continuumL17Fig2}
\imageFigure{../../figures/phy454/lec17_Error_functionFig2}{Error function}{fig:continuumL17:continuumL17Fig2}{0.3}

\section{Boundary layers}

Let us look at spacetime points which are constant in \(\eta\)

\begin{equation}\label{eqn:continuumL17:180}
\frac{y_1}{2 \sqrt{\nu t_1}} = \frac{y_2}{2 \sqrt{\nu t_2}},
\end{equation}

so that the speed at \((y_1, t_1)\) equals the speed at \((y_2, t_2)\).  This is illustrated in \cref{fig:continuumL17:continuumL17Fig3}

\pdfTexFigure{../../figures/phy454/continuumL17Fig3.pdf_tex}{Velocity profiles at different times}{fig:continuumL17:continuumL17Fig3}{0.6}

\section{Universal behavior}

Looking at a plot with different viscosities for position vs time scaled as \(\sqrt{\nu t}\) as in \cref{fig:continuumL17:continuumL17Fig4} we see a sort of universal behavior


\pdfTexFigure{../../figures/phy454/continuumL17Fig4.pdf_tex}{Universal behavior}{fig:continuumL17:continuumL17Fig4}{0.6}

Characterizing this we introduce the concept of boundary layer thickness

\makedefinition{Boundary layer thickness}{dfn:continuumL17:200}{The length scale over which \index{viscosity} is dominant.  This is the viscous length scale. \index{boundary layer}}

This is similar to what we have in the heat equation

\begin{equation}\label{eqn:continuumL17:220}
\PD{t}{T} = \kappa \PDSq{y}{T},
\end{equation}

where the time scale for the diffusion can be expressed as

\begin{equation}\label{eqn:continuumL17:240}
[\kappa_t] = \frac{d^2}{\kappa}.
\end{equation}

We could consider a scenario such as a heated plate in a cavity of height \(\delta\) as in \cref{fig:continuumL17:continuumL17Fig5}

\pdfTexFigure{../../figures/phy454/continuumL17Fig5.pdf_tex}{Characteristic distances in heat flow problems}{fig:continuumL17:continuumL17Fig5}{0.5}

with a temperature \(T\) on the bottom plate.  We can ask how fast the heat propagates through the medium.

\makeexample{Oscillating plate}{ex:boundaryLayers:oscillating}{

Consider an oscillating plate, driving the motion of the fluid, as in \cref{fig:continuumL17:continuumL17Fig6}
\imageFigure{../../figures/phy454/lec17_Time_dependent_fluid_motion_due_to_oscillating_plateFig6}{Time dependent fluid motion due to oscillating plate}{fig:continuumL17:continuumL17Fig6}{0.3}

\begin{equation}\label{eqn:continuumL17:260}
U(t) = U_0 \cos \Omega t = \Real\left( U_0 e^{i \Omega t} \right).
\end{equation}

(we are thinking here about the always oscillating case, and not an impulsive plate motion).

We write

\begin{equation}\label{eqn:continuumL17:280}
u(y, t) = \Real\left( f(y) e^{i \Omega t} \right)
\end{equation}

with substitution into

\begin{equation}\label{eqn:continuumL17:300}
\PD{t}{u} = \nu \PDSq{y}{u},
\end{equation}

we have

\begin{equation}\label{eqn:continuumL17:320}
i \Omega f(y) e^{i \Omega t} = \nu f'' e^{i \Omega t}
\end{equation}

or

\begin{equation}\label{eqn:continuumL17:340}
i \Omega f(y) = \nu f''
\end{equation}

This is an equation of the form

\begin{equation}\label{eqn:continuumL17:360}
f'' = m^2 f
\end{equation}

where

\begin{equation}\label{eqn:continuumL17:380}
m^2 = \frac{i \Omega}{\nu}.
\end{equation}

or

\begin{equation}\label{eqn:continuumL17:400}
m = \sqrt{\frac{i \Omega}{\nu}} = \lambda (1 + i),
\end{equation}

where

\begin{equation}\label{eqn:continuumL17:420}
\lambda = \sqrt{\frac{\Omega}{2 \nu}}.
\end{equation}

check:

\begin{equation}\label{eqn:impulsiveFlowAndBoundaryLayersAndOscillatoryDrivenFlow:660}
\begin{aligned}
m^2
&=
\frac{\Omega}{2 \nu} (i + 1)^2 \\
&=
\frac{\Omega}{2 \nu} (i^2 + 1 + 2 i) \\
&=
\frac{\Omega}{\nu} i
\end{aligned}
\end{equation}

Considering the boundary value constraints we have

\begin{equation}\label{eqn:continuumL17:440}
f(y) =
A e^{\lambda (1 + i) y}
+ B e^{-\lambda (1 + i) y}
\end{equation}

Since \(u(\infty, t) \rightarrow 0\) we must have

\begin{equation}\label{eqn:continuumL17:460}
f(\infty) = 0,
\end{equation}

so we must kill off the exponentially increasing (albeit also oscillating) term by setting \(A = 0\).  Also, since

\begin{equation}\label{eqn:continuumL17:480}
u(0, t) = U(t)
\end{equation}

we must have

\begin{equation}\label{eqn:continuumL17:500}
f(0) = U_0
\end{equation}

or

\begin{equation}\label{eqn:continuumL17:520}
B = U_0
\end{equation}

so

\begin{equation}\label{eqn:continuumL17:540}
f(y) = U_0 e^{-\lambda (1 + i) y}
\end{equation}

and

\begin{equation}\label{eqn:continuumL17:560}
u(y, t) =
\Real\left(
U_0 e^{-\lambda y} e^{ -i (\lambda y - \Omega t) }
\right)
\end{equation}

or

\begin{equation}\label{eqn:continuumL17:580}
u(y, t) =
U_0 e^{-\lambda y} \cos\left( -i (\lambda y - \Omega t) \right).
\end{equation}

This is a damped transverse wave function

\begin{equation}\label{eqn:continuumL17:600}
u(y, t) = f(y - c t),
\end{equation}

where

\begin{equation}\label{eqn:continuumL17:620}
c = \frac{\Omega}{\lambda},
\end{equation}

is the wave speed.

Since we have an exponential damping here, the flow of fluid will essentially be confined to a boundary layer, where after distance \(y = n/\lambda\), the oscillation falls off as

\begin{equation}\label{eqn:continuumL17:640}
\inv{e^n}.
\end{equation}

We can find a nice illustration of such a flow in \citep{wiki:StokesBoundary}.
}

   %
% Copyright � 2012 Peeter Joot.  All Rights Reserved.
% Licenced as described in the file LICENSE under the root directory of this GIT repository.
%

% 
% 
\section{Fluid flow over a solid body}
\subsection{Scaling arguments.}

We've been talking about impulsively started flow and the Stokes boundary problem.

We'll now move on to a similar problem, that of fluid flow over a solid body.

Consider figure (\ref{fig:continuumL18:continuumL18Fig1}), where we have an illustration of flow over a solid object with a boundary layer of thickness $\delta$.

\imageFigure{figures/lec18_Flow_over_a_solid_object_with_boundary_layerFig1}{Flow over a solid object with boundary layer}{fig:continuumL18:continuumL18Fig1}{0.3}

We have a couple scales to consider.

\begin{enumerate}
\item Velocity scale $U$.
\item Length scale in the $y$ direction $\delta$.
\item Length scale in the $x$ direction $L$.
\end{enumerate}

where

\begin{equation}\label{eqn:continuumL18:10}
L \gg \delta.
\end{equation}

As always, we start with the Navier-Stokes equation, restricting ourselves to the steady state $\PDi{t}{\Bu} = 0$ case.  In coordinates, for incompressible flows, we have our usual $x$ momentum, $y$ momentum, and continuity equations

\begin{subequations}
\begin{equation}\label{eqn:continuumL18:30}
u \PD{x}{u} + v \PD{y}{u} = - \inv{\rho} \PD{x}{p} + \nu \left( 
\PDSq{x}{u}
+\PDSq{y}{u}
\right)
\end{equation}
\begin{equation}\label{eqn:continuumL18:50}
u \PD{x}{v} + v \PD{y}{v} = - \inv{\rho} \PD{y}{p} + \nu \left( 
\PDSq{x}{v}
+\PDSq{y}{v}
\right)
\end{equation}
\begin{equation}\label{eqn:continuumL18:70}
\PD{x}{u} 
+\PD{y}{v} = 0
\end{equation}
\end{subequations}

Let's look at the scaling of these equations, starting with the continuity equation \ref{eqn:continuumL18:70}.  This is roughly

\begin{align}\label{eqn:continuumL18:90}
\PD{x}{u} &\sim \frac{U}{L} \\
\PD{y}{v} &\sim \frac{v}{\delta}
\end{align}

We require that these have to be of the same order of magnitude.

\FIXME{Why?  This doesn't make sense to me since in horizontal flow we had $v = 0$, and the two components of the divergence are obviously of different scales}

If these are of the same scale we have

\begin{equation}\label{eqn:continuumL18:110}
\frac{U}{L} \sim \frac{v}{\delta}
\end{equation}

so that 

\begin{equation}\label{eqn:continuumL18:130}
v \sim \frac{U \delta}{L}
\end{equation}

or 

\begin{equation}\label{eqn:continuumL18:150}
v \ll U
\end{equation}

Looking at the viscous terms

\begin{align}\label{eqn:continuumL18:170}
\nu \PDSq{x}{u} &\sim \frac{\nu U}{L^2} \\
\nu \PDSq{y}{u} &\sim \frac{\nu U}{\delta^2}
\end{align}

or

\begin{equation}\label{eqn:continuumL18:190}
\nu \PDSq{y}{u} \gg \nu \PDSq{x}{u}
\end{equation}

So we can neglect the $x$ component of the Laplacian in our $x$ momentum equation \ref{eqn:continuumL18:30}.

How about the inertial terms

\begin{align}\label{eqn:continuumL18:210}
u \PD{x}{u} &\sim \frac{U^2}{L} \\
v \PD{y}{u} &\sim \frac{\delta U}{L} \frac{U}{\delta} \sim \frac{U^2}{L}
\end{align}

Since these are of the same order (in the boundary regions) we cannot neglect either.  We also cannot neglect the pressure gradient, since this is what induces the flow.

For the $y$ momentum equation we have

\begin{align}\label{eqn:continuumL18:230}
\nu \PDSq{x}{v} &\sim \nu \frac{\delta U}{L} \inv{L^2} \sim \nu \frac{\delta U}{L^3} \ll \frac{\nu U}{\delta^2} \\
\nu \PDSq{y}{v} &\sim \nu \frac{\delta U}{L} \inv{\delta^2} \sim \nu \frac{U}{\delta L} \ll \frac{\nu U}{\delta^2}
\end{align}

We can neglect all the Laplacian terms in the $y$ momentum equation.

\question
Why compare the magnitude of the viscous terms for the $y$ momentum to the magnitude of the same terms in the $x$ momentum equation, and not to the LHS of the $y$ momentum equation.

\answer
That's a valid point, but our equations are coupled, and contributions from one feed into the other.

We aren't done yet.  For the inertial terms in the $y$ momentum equation we have

\begin{align}\label{eqn:continuumL18:250}
u \PD{x}{v} &\sim \frac{\delta U^2}{L^2} \\
v \PD{y}{v} &\sim \frac{\delta U}{L} \frac{\delta U}{L} \inv{\delta} \sim \frac{\delta U^2}{L^2}
\end{align}

Note that 

\begin{equation}\label{eqn:continuumL18:270}
\frac{\delta U^2}{L^2} = \frac{\delta}{L} \left( \frac{U^2}{L} \right) \ll \frac{U^2}{L}
\end{equation}

We see that both of the $y$ momentum inertial terms can be neglected in comparison to the $x$ momentum equations.

Putting all of this together, our equations of motion for the boundary flow are now reduced to

\begin{subequations}
\begin{equation}\label{eqn:continuumL18:30a}
u \PD{x}{u} + v \PD{y}{u} = - \inv{\rho} \PD{x}{p} + \nu \PDSq{y}{u}
\end{equation}
\begin{equation}\label{eqn:continuumL18:50a}
\PD{y}{p} = 0
\end{equation}
\begin{equation}\label{eqn:continuumL18:70a}
\PD{x}{u} + \PD{y}{v} = 0
\end{equation}
\end{subequations}

Utilizing Bernoulli's theorem \ref{eqn:continuumL18:410}, we can deal with the pressure term.  It's magnitude is

\begin{align*}
- \PD{x}{} \left( \frac{p}{\rho} \right) 
&\sim \PD{x}{} \left( \inv{2} u^2 \right) \\
&\sim \frac{2}{2} u \PD{x}{u} \\
&\sim U \frac{dU}{dx},
\end{align*}

and our equations of motion are finally reduced to

\begin{subequations}
\begin{equation}\label{eqn:continuumL18:30b}
u \PD{x}{u} + v \PD{y}{u} = U \frac{dU}{dx} + \nu \PDSq{y}{u}
\end{equation}
\begin{equation}\label{eqn:continuumL18:50b}
\PD{y}{p} = 0
\end{equation}
\begin{equation}\label{eqn:continuumL18:70b}
\PD{x}{u} + \PD{y}{v} = 0
\end{equation}
\end{subequations}

\FIXME{
Relevant?

Observe that if we operate on \ref{eqn:continuumL18:370} with a divergence operation, then we don't have to assume non-viscous flow but in that case we can only say that (for irrotational flow) we have

\begin{equation}\label{eqn:continuumL18:430}
\spacegrad^2 \left( \frac{p}{\rho} + \chi + \inv{2} \Bu^2 \right) = 0.
\end{equation}
}

   %
% Copyright � 2012 Peeter Joot.  All Rights Reserved.
% Licenced as described in the file LICENSE under the root directory of this GIT repository.
%

%
%
\label{chap:continuumL19}
\makeexample{Fluid flow over a flat plate (Blasius problem)}{ex:boundaries:blassius}{

%\unnumberedSubsection{Review.  Laminar boundary layer theory}
In the boundary layer we found
\begin{enumerate}
\item Continuity equation
\begin{equation}\label{eqn:continuumL19:10}
\PD{x}{u} + \PD{y}{v} = 0
\end{equation}
\item \(x\) momentum equation
\begin{equation}\label{eqn:continuumL19:30}
u \PD{x}{u} + v \PD{y}{u} = - \inv{\rho} \PD{x}{p} + \nu \PDSq{y}{u}
\end{equation}
\item \(y\) momentum equation
\begin{equation}\label{eqn:continuumL19:50}
\PD{y}{p} = 0
\end{equation}
\end{enumerate}

In the inviscid region \(p\) is a constant in \(y\)

\begin{equation}\label{eqn:continuumL19:70}
\PD{y}{p} = 0
\end{equation}

This will be approximately true in the boundary layer too as illustrated in \cref{fig:continuumL19:continuumL19Fig1}

\imageFigure{../../figures/phy454/lec19_Illustrating_the_boundary_layer_thickness_and_associated_pressure_variationFig1}{Illustrating the boundary layer thickness and associated pressure variation}{fig:continuumL19:continuumL19Fig1}{0.3}

Starting with

\begin{equation}\label{eqn:continuumL19:90}
\PD{t}{\Bu} + (\Bu \cdot \spacegrad) \Bu = -\spacegrad \left( \frac{p}{\rho} + \chi \right) + \nu \spacegrad^2 \Bu,
\end{equation}

we were able to show that inviscid irrotational incompressible flows are governed by Bernoulli's equation

\begin{equation}\label{eqn:continuumL19:110}
\frac{p}{\rho} + \chi + \inv{2} \Bu^2 = \text{constant}
\end{equation}

In the absence of body forces (or constant potentials), we have

\begin{equation}\label{eqn:continuumL19:130}
-\PD{x}{}\frac{p}{\rho} = \PD{x}{} \left(\inv{2} \Bu^2 \right) \sim U \PD{x}{U}
\end{equation}

so that our boundary layer equations are

\begin{subequations}
\begin{equation}\label{eqn:continuumL19:150}
u \PD{x}{u} + v \PD{y}{u} = U \frac{dU}{dx} + \nu \PDSq{y}{u}
\end{equation}
\begin{equation}\label{eqn:continuumL19:170}
\PD{y}{p} = 0
\end{equation}
\begin{equation}\label{eqn:continuumL19:190}
\PD{x}{u} + \PD{y}{v} = 0
\end{equation}
\end{subequations}

With boundary conditions

\begin{equation}\label{eqn:continuumProblemSet2:210}
\begin{aligned}
U(x, 0) &= 0 \\
U(x, \infty) &= U(x) \\
V(x, 0) &= 0
\end{aligned}
\end{equation}

%\unnumberedSubsection{Similarity transformation}

Define a similarity variable \(\eta\)

\begin{equation}\label{eqn:continuumL19:230}
\eta \sim \frac{y}{\sqrt{\nu t}}
\end{equation}

Suppose we want

\begin{equation}\label{eqn:continuumL19:250}
\eta \sim f(y, x)
\end{equation}

Since we have

\begin{equation}\label{eqn:continuumL19:270}
x \sim U t,
\end{equation}

or

\begin{equation}\label{eqn:continuumL19:290}
t \sim \frac{x}{U}.
\end{equation}

We can make the transformation

\begin{equation}\label{eqn:continuumL19:310}
\eta = \frac{y}{\sqrt{2 \frac{\nu x}{U}}}
\end{equation}

We can introduce stream functions

\begin{equation}\label{eqn:continuumProblemSet2:330}
\begin{aligned}
u &= \PD{y}{\psi} \\
v &= -\PD{x}{\psi}
\end{aligned}
\end{equation}

We can check that this satisfies the continuity equation since we have

\begin{equation}\label{eqn:boundaryLayersAndStreamFunctionsAndBlassius:1170}
\begin{aligned}
\PD{x}{u} + \PD{y}{v}
&=
\frac{\partial^2 \psi}{\partial x \partial y}
-\frac{\partial^2 \psi}{\partial y \partial x} \\
&= 0
\end{aligned}
\end{equation}

Now introduce a similarity variable

\begin{equation}\label{eqn:continuumL19:350}
f(\eta) = \frac{\psi}{\delta U_0} = \frac{\psi}{\sqrt{2 \nu x U_0}}
\end{equation}

Note that we have also suddenly assumed that \(U = U_0\) (a constant, which will also kill the \(U'\) term in the N-S equation).  This is not really justified by anything we have done so far, but asking about this in class, it was stated that this is a restriction for this formulation of the Blasius problem.

Also note that this last step requires:

\begin{equation}\label{eqn:continuumL19:730}
\delta = \sqrt{ 2 \nu x/U_0 }.
\end{equation}

This at least makes sense dimensionally since we have

\begin{equation}\label{eqn:continuumL19:731}
[\sqrt{\nu x/U}] = \sqrt{ (L^2/T) (L) (T/L)} = L,
\end{equation}

but where did this definition of \(\delta\) come from?

In \citep{ wiki:BlasiusBoundary} it is mentioned that this is a result of the scaling argument.  We did have some scaling arguments that included \(\delta\) in the expressions from last lecture, one of which was \eqnref{eqn:continuumL19:130}

\begin{equation}\label{eqn:continuumL19:750}
\delta \sim \frac{v L}{U},
\end{equation}

but that does not obviously give us \eqnref{eqn:continuumL19:730}?

Ah.  We argued that

\begin{equation}\label{eqn:continuumL19:1130}
v \PD{y}{u} \sim \frac{U^2}{L}
\end{equation}

and that the larger of the viscous terms was

\begin{equation}\label{eqn:continuumL19:1150}
\nu \PDSq{y}{u} \sim \frac{\nu U}{\delta^2}.
\end{equation}

If we require that these are the same order of magnitude, as argued in \S 8.3 of \citep{acheson1990elementary}, then we find \eqnref{eqn:continuumL19:750}.

Regardless, given this change of variables, we can apparently compute

\begin{equation}\label{eqn:continuumL19:370}
f''' + f f'' = 0.
\end{equation}

Our boundary conditions are

\begin{equation}\label{eqn:continuumL19:390}
\begin{array}{l l}
f = f' = 0 & \quad \mbox{\(\eta = 0\)} \\
f' = 1 & \quad \mbox{\(\eta = \infty\)} \\
\end{array}
\end{equation}

%\unnumberedSubsection{Deriving the equation of motion}

Attempting to derive \eqnref{eqn:continuumL19:370} using the definitions above gets a bit messy.  It is messy enough that I mistakenly thought that we could not possible arrive at a differential equation that has a plain old (non-derivative) \(f\) in it as in \eqnref{eqn:continuumL19:370} above.  The algebra involved in taking the derivatives got the better of me.  This derivation is treated a different way in \citep{acheson1990elementary}.  For the purpose of completeness (and because that derivation also leaves out some details), lets do this from start to finish with all the gory details, but following the outline provided in the text.

Instead of pre-determining the form of the similarity variable exactly, we can state it in terms of an unknown and to be determined function of position writing

\begin{equation}\label{eqn:continuumL19:770}
\eta = \frac{y}{g(x)}.
\end{equation}

We still introduce stream our stream functions \eqnref{eqn:continuumProblemSet2:330}, but require that our horizontal velocity component is only a function of our similarity variable

\begin{equation}\label{eqn:continuumL19:790}
u = U h(\eta),
\end{equation}

where \(h(\eta)\) is to be determined, and is scaled by our characteristic velocity \(U\).  Observe that, as above, we are assuming that \(U(x) = U\), a constant (which also kills off the \(U U'\) term in the Navier-Stokes equation.)  Given this form of \(u\), we note that

\begin{equation}\label{eqn:continuumL19:810}
u(\eta) = \PD{y}{\psi} = \PD{\eta}{\psi} \PD{y}{\eta} = \inv{g(x)} \PD{\eta}{\psi},
\end{equation}

so that

\begin{equation}\label{eqn:continuumL19:830}
\psi = U g(x) \int_0^\eta h(\eta') d\eta' + k(x).
\end{equation}

It is argued in the text that we also want \(\psi\) to be a streamline, so that \(\psi(\eta = 0) = 0\) implying that \(k(x) = 0\).  I do not honestly follow the rational for that, but it is certainly convenient to set \(k(x) = 0\), so lets do so and see where things go.  With

\begin{equation}\label{eqn:continuumL19:850}
f(\eta) = \int_0^\eta h(\eta') d\eta'.
\end{equation}

Observe that \(f(0)\) is necessarily zero with this definition.  We can now write

\begin{equation}\label{eqn:continuumL19:870}
\psi(\eta, x) = U g(x) f(\eta).
\end{equation}

This is like what we had in class, with the exception that instead of a constant relating \(\psi\) and \(f(\eta)\) we also have a function of \(x\).  That is exactly what we need so that we can end up with both \(f\) and derivatives of \(f\) in our Navier-Stokes equation.

Now let us do the mechanical bits, computing all the derivatives.  We can compute \(v\) to start with

\begin{equation}\label{eqn:boundaryLayersAndStreamFunctionsAndBlassius:1190}
\begin{aligned}
v
&= -\PD{x}{\psi} \\
&= - U \PD{x}{} \left( g(x) f(\eta) \right) \\
&= - U \left( g' f + g f' \PD{x}{\eta} \right) \\
&= - U \left( g' f - g f' \frac{y g'}{g^2} \right) \\
&= - U \left( g' f - f' \frac{y g'}{g} \right) \\
&= - U \left( g' f - f' g' \eta \right) \\
&= U g' \left( f' \eta - f \right)
\end{aligned}
\end{equation}

We had initially \(u = U h(\eta)\), but \(f'(\eta) = h(\eta)\), so we have now got both \(u\) and \(v\) specified in terms of \(f\) and \(g\) and their derivatives

\begin{equation}\label{eqn:continuumL19:890}
\begin{aligned}
u &= U f' \\
v &= U g' \left( f' \eta - f \right).
\end{aligned}
\end{equation}

We have got a bunch of the \(u\) derivatives that we have to compute

\begin{equation}\label{eqn:boundaryLayersAndStreamFunctionsAndBlassius:1210}
\begin{aligned}
\PD{x}{u}
&=
U \PD{x}{ f' } \\
&=
U \PD{\eta}{ f' } \PD{x}{\eta} \\
&=
-U f'' \frac{y g'}{g^2} \\
&=
-U f'' \eta \frac{g'}{g},
\end{aligned}
\end{equation}

and
\begin{equation}\label{eqn:boundaryLayersAndStreamFunctionsAndBlassius:1230}
\begin{aligned}
\PD{y}{u}
&=
U \PD{y}{ f' } \\
&=
U f'' \PD{y}{\eta} \\
&=
U f'' \inv{g},
\end{aligned}
\end{equation}

and
\begin{equation}\label{eqn:boundaryLayersAndStreamFunctionsAndBlassius:1250}
\begin{aligned}
\PDSq{y}{u}
&=
U \inv{g} \PD{y}{f''} \\
&=
U \inv{g} f''' \PD{y}{\eta} \\
&=
U \inv{g^2} f'''.
\end{aligned}
\end{equation}

Our \(x\) component of Navier-Stokes now takes the form

\begin{equation}\label{eqn:boundaryLayersAndStreamFunctionsAndBlassius:1270}
\begin{aligned}
0 &=
u \PD{x}{u}
+
v \PD{y}{u}
-
\nu
\PDSq{y}{u} \\
&=
U f' \left(
-U f'' \eta \frac{g'}{g}
\right)
+
U g' \left( f' \eta - f \right)
U f'' \inv{g}
-
\nu
U \inv{g^2} f''' \\
&=
\frac{U}{g^2}
\left(
\cancel{-U f' g f'' \eta g' }
+
\cancel{U g g' f' \eta f''}
-
U g g' f f''
-
\nu
f'''
\right)
\end{aligned}
\end{equation}

or (assuming \(g \ne 0\))

\begin{equation}\label{eqn:continuumL19:910}
f''' + \frac{U g g' }{\nu} f f'' = 0.
\end{equation}

Now, it we wish \(U g g'/\nu = 1\) (to make this equation as easy to solve as possible), we can integrate to find the required form of \(g(x)\).  This gives

\begin{equation}\label{eqn:continuumL19:930}
\frac{U g^2 }{2 \nu} = x + C.
\end{equation}

It is argued that we expect
\begin{equation}\label{eqn:continuumL19:950}
\PD{y}{u} = U f'' \inv{g},
\end{equation}

to become singular at \(x = 0\), so we should set \(C = 0\).  This leaves us with

\begin{subequations}
\begin{equation}\label{eqn:continuumL19:970}
g(x) = \sqrt{\frac{2 x \nu}{U}}
\end{equation}
\begin{equation}\label{eqn:continuumL19:990}
\eta = \frac{y}{\sqrt{\frac{2 x \nu}{U}}}
\end{equation}
\begin{equation}\label{eqn:continuumL19:1010}
f''' + f f'' = 0
\end{equation}
\begin{equation}\label{eqn:continuumL19:1030}
u = U f'
\end{equation}
\begin{equation}\label{eqn:continuumL19:1050}
v = (\eta f' - f) \frac{\nu}{g},
\end{equation}
\end{subequations}

and boundary value conditions

\begin{subequations}
\begin{equation}\label{eqn:continuumL19:1070}
f(0) = 0
\end{equation}
\begin{equation}\label{eqn:continuumL19:1090}
f'(0) = 0
\end{equation}
\begin{equation}\label{eqn:continuumL19:1110}
f'(\infty) = 1.
\end{equation}
\end{subequations}

where \eqnref{eqn:continuumL19:1090} follows from \(u(0) = 0\) and \eqnref{eqn:continuumL19:1030}, and \eqnref{eqn:continuumL19:1110} follows from the fact that \(u\) tends to \(U\).

%\unnumberedSubsection{Numeric solution}

We can solve this numerically and find solutions that look like \cref{fig:continuumL19:continuumL19Fig2}

\imageFigure{../../figures/phy454/lec19_Boundary_layer_solution_to_flow_over_plateFig2}{Boundary layer solution to flow over plate}{fig:continuumL19:continuumL19Fig2}{0.3}

This is the Blasius solution to the problem of fluid flow over a flat plate.
} % end example

   %
% Copyright � 2012 Peeter Joot.  All Rights Reserved.
% Licenced as described in the file LICENSE under the root directory of this GIT repository.
%

%
%
\label{chap:continuumL20}

\section{Asymptotic solutions of ill conditioned equations}

We will consider two cases, both ones that we can solve exactly

\begin{enumerate}
\item With \(u(0) = 1\), and letting \(\epsilon \rightarrow 0\), we will look at solutions of the ill conditioned LDE

\begin{equation}\label{eqn:continuumL20:10}
\epsilon \frac{du}{dy} + u = y
\end{equation}

\item With \(u(0) = 0\), \(u(1) = 2\), and \(0 < \epsilon \ll 1\) we will look at the second order ill conditioned LDE

\begin{equation}\label{eqn:continuumL20:30}
\epsilon \frac{d^2u}{dy^2} + \frac{du}{dy} = 1
\end{equation}
\end{enumerate}

\makeexample{First order LDE}{ex:singularPertubation:firstorder}{
%\unnumberedSubsection{Exact solution}
We can solve this system exactly.  Our homogeneous equation is

\begin{equation}\label{eqn:continuumL20:370}
\epsilon \frac{du}{dy} + u = 0,
\end{equation}

with solution

\begin{equation}\label{eqn:continuumL20:390}
u \propto e^{-y/\epsilon}.
\end{equation}

Looking for a solution of the form

\begin{equation}\label{eqn:continuumL20:410}
u = A(y) e^{-y/\epsilon},
\end{equation}

we find

\begin{equation}\label{eqn:continuumL20:430}
\epsilon A' e^{-y/\epsilon} = y,
\end{equation}

and integrate to find

\begin{equation}\label{eqn:continuumL20:450}
A(y) = (y - \epsilon) e^{x/\epsilon} + C.
\end{equation}

Application of the boundary value constraints give us

\begin{equation}\label{eqn:continuumL20:50}
u = ( 1 + \epsilon ) e^{-y/\epsilon} + y - \epsilon.
\end{equation}

This is plotted in \cref{fig:continuumL20:continuumL20Fig1}
\imageFigure{../../figures/phy454/lec20_Plot_of_exact_solution_to_simple_first_order_ill_conditioned_LDEFig1}{Plot of exact solution to simple first order ill conditioned LDE}{fig:continuumL20:continuumL20Fig1}{0.3}

%\unnumberedSubsection{Limiting cases}
We want to consider the limiting case where
\begin{equation}\label{eqn:continuumL20:70}
0 < \epsilon \ll 1,
\end{equation}

and we let \(\epsilon \rightarrow 0\).  If \(y = O(1)\), then we have

\begin{equation}\label{eqn:continuumL20:470}
u \approx 1 \times 0 + y,
\end{equation}

or just

\begin{equation}\label{eqn:continuumL20:490}
u = y.
\end{equation}

However, if \(y = O(\epsilon)\) then we have to be more careful constructing an approximation.  When \(y\) is very small, but \(\epsilon\) is also of the same order of smallness we have

\begin{equation}\label{eqn:continuumL20:130}
e^{-y/\epsilon} \ne 0.
\end{equation}

If \(\epsilon \rightarrow 0\) and \(y \rightarrow O(\epsilon)\)

\begin{equation}\label{eqn:continuumL20:150}
e^{-y/\epsilon} \rightarrow e^{-\epsilon O(1) /\epsilon} \rightarrow e^{-O(1)}
\end{equation}

so

\begin{equation}\label{eqn:continuumL20:170}
u \approx e^{-y/\epsilon}
\end{equation}

%\unnumberedSubsection{Approximate solution in the inner region}
For an approximate solution in the inner region, when \(y = O(\epsilon)\) define a new scale

\begin{equation}\label{eqn:continuumL20:190}
Y = \frac{y}{\epsilon}
\end{equation}

\begin{equation}\label{eqn:continuumL20:210}
y = O(1)
\end{equation}

so that our LDE takes the form

\begin{equation}\label{eqn:continuumL20:230}
\frac{du}{dY} + u = \epsilon Y
\end{equation}

When \(\epsilon \rightarrow 0\) we have

\begin{equation}\label{eqn:continuumL20:250}
\frac{du}{dY} + u \approx 0.
\end{equation}

We have solution

\begin{equation}\label{eqn:continuumL20:270}
\ln u = -Y + \ln C,
\end{equation}

or

\begin{equation}\label{eqn:continuumL20:290}
u \propto e^{-Y} = e^{-y/\epsilon}.
\end{equation}

%\FIXME{
%\unnumberedSubsection{Question:} Could not we just Laplace transform.
%\unnumberedSubsection{Answer given:} We would still get into trouble when we take \(\epsilon \rightarrow 0\).  My comment: I do not think that is strictly true.  In an example like this where we have an exact solution, a Laplace transform technique should also yield that solution.  I think the real trouble will come when we attempt to incorporate the non-linear inertial terms of the Navier-Stokes equation.
%}
} % end example

\makeexample{Second order example}{ex:singularPertubation:secondorder}{

%\unnumberedSubsection{Exact solution}
We can also solve this system exactly.  We saw above in the first order system that our specific solution was polynomial.  While that was found by the method of variation of parameters, it seems obvious in retrospect.  Let us start by looking for such a solution, starting with a first order polynomial

\begin{equation}\label{eqn:continuumL20:510}
u = A y + B.
\end{equation}

Application of our LDE operator on this produces

\begin{equation}\label{eqn:asymptoticSolutions:910}
\begin{aligned}
A &= 1 \\
B &= 0.
\end{aligned}
\end{equation}

Now let us move on to find a solution to the homogeneous equation

\begin{equation}\label{eqn:continuumL20:30b}
\epsilon \frac{d^2u}{dy^2} + \frac{du}{dy} = 0
\end{equation}

As usual, we look for the characteristic equation by assuming a solution of the form \(u = e^{m y}\).  This gives us

\begin{equation}\label{eqn:continuumL20:530}
\epsilon m^2 + m = (\epsilon m + 1) m = 0,
\end{equation}

with roots

\begin{equation}\label{eqn:continuumL20:550}
m = 0, -1/\epsilon.
\end{equation}

So our homogeneous equation has the form

\begin{equation}\label{eqn:continuumL20:570}
u(y) = A e^{-y/\epsilon} + B,
\end{equation}

and our full solution is

\begin{equation}\label{eqn:continuumL20:590}
u(y) = A e^{-y/\epsilon} + B + y
\end{equation}

with the constants \(A\) and \(B\) to be determined from our boundary value conditions.  We find

\begin{equation}\label{eqn:asymptoticSolutions:930}
\begin{aligned}
0 &= u(0) = A + B + 0 \\
2 &= u(1) = A e^{-1/\epsilon} + B + 1.
\end{aligned}
\end{equation}

We have got \(B = -A\) and by subtracting

\begin{equation}\label{eqn:continuumL20:610}
A ( e^{-1/\epsilon} -1 ) = 1.
\end{equation}

So the exact solution is

\begin{equation}\label{eqn:continuumL20:310}
u = y + \frac{e^{-y/\epsilon} - 1}{e^{1/\epsilon} - 1}.
\end{equation}

This is plotted in \cref{fig:continuumL20:continuumL20Fig2}
\imageFigure{../../figures/phy454/lec20_Plot_of_our_ill_conditioned_second_order_LDEFig2}{Plot of our ill conditioned second order LDE}{fig:continuumL20:continuumL20Fig2}{0.3}

%\unnumberedSubsection{Solution in the regular region}
Looking for a solution in the regular region, we consider small \(\epsilon\) relative to \(y\).  There our LDE is approximately

\begin{equation}\label{eqn:continuumL20:630}
\frac{du}{dy} = 1,
\end{equation}

which has solution
\begin{equation}\label{eqn:continuumL20:650}
u = y + C
\end{equation}

Our \(u(1) = 2\) boundary value constraint gives us

\begin{equation}\label{eqn:continuumL20:670}
C = 1.
\end{equation}

Our solution in the regular region where \(\epsilon \rightarrow 0\) and \(y = O(1)\) is therefore just

\begin{equation}\label{eqn:continuumL20:690}
u = y + 1.
\end{equation}

%\unnumberedSubsection{Solution in the ill conditioned region}
Now let us consider the inner (ill conditioned) region.  We will see below that when \(y = O(\epsilon)\), and we allow both \(\epsilon\) and \(y\) tend to zero independently, we have approximately

\begin{equation}\label{eqn:continuumL20:350}
u \sim 1 - e^{-y/\epsilon}.
\end{equation}

We will now show this.  We start with a helpful change of variables as we did in the first order case

\begin{equation}\label{eqn:continuumL20:710}
Y = \frac{y}{\epsilon}.
\end{equation}

When \(y = O(\epsilon)\) and \(Y = O(1)\) we have

\begin{equation}\label{eqn:continuumL20:730}
\cancel{\epsilon} \frac{d^2 u}{\cancel{\epsilon} \epsilon dY^2} + \inv{\epsilon} \frac{du}{dY} = 1
\end{equation}

or
\begin{equation}\label{eqn:continuumL20:750}
\frac{d^2 u}{dY^2} + \frac{du}{dY} = \epsilon.
\end{equation}

This puts the LDE into a non ill conditioned form, and allows us to let \(\epsilon \rightarrow 0\).  We have approximately

\begin{equation}\label{eqn:continuumL20:770}
\frac{d^2 u}{dY^2} + \frac{du}{dY} = 0.
\end{equation}

We have solved this in our exact solution work above (in a slightly more general form), and thus in this case we have just

\begin{equation}\label{eqn:continuumL20:830}
u = A + B e^{-Y}
\end{equation}

at \(Y = 0\) we have

\begin{equation}\label{eqn:continuumL20:850}
u = A + B = 0.
\end{equation}

so that
\begin{equation}\label{eqn:continuumL20:870}
B = -A.
\end{equation}

and we find for the inner region

\begin{equation}\label{eqn:continuumL20:890}
u = A (1 - e^{-Y}) = A( 1 - e^{-y/\epsilon} ) \sim 1 - e^{-y/\epsilon}.
\end{equation}

Taking these independent solutions for the inner and outer regions and putting them together into a coherent form (called matched asymptotic expansion) is a rich and tricky field.  For info on that we have been referred to \citep{hinch1991perturbation}.
} % end example

   %
% Copyright � 2012 Peeter Joot.  All Rights Reserved.
% Licenced as described in the file LICENSE under the root directory of this GIT repository.
%

%
%
\label{chap:continuumL21}
\section{Stability}
%\section{Stability.  Rayleigh-Benard problem}
\subsection{Stability.  Some graphical illustrations}

What do we mean by stability?  A configuration is stable if after a small disturbance it returns to its original position.  A couple systems to consider are shown in \cref{fig:continuumL21:continuumL21Fig1a}, \cref{fig:continuumL21:continuumL21Fig1b} and \cref{fig:continuumL21:continuumL21Fig1c}.

\imageFigure{../../figures/phy454/lec21_Stable_well_configurationFig1a}{Stable well configuration}{fig:continuumL21:continuumL21Fig1a}{0.3}
\imageFigure{../../figures/phy454/lec21_Instable_peak_configurationFig1b}{Instable peak configuration}{fig:continuumL21:continuumL21Fig1b}{0.3}
\imageFigure{../../figures/phy454/lec21_Stable_tabletop_configurationFig1c}{Stable tabletop configuration}{fig:continuumL21:continuumL21Fig1c}{0.3}

We can examine how a displacement \(\delta x\) changes with time after making it.  In a stable configuration without friction we will induce an oscillation as plotted in \cref{fig:continuumL21:continuumL21Fig2a} for the parabolic configuration.  With friction we will have a damping effect.  This is plotted for the parabolic well in \cref{fig:continuumL21:continuumL21Fig2b}.

\imageFigure{../../figures/phy454/lec21_Displacement_time_evolution_in_undamped_well_systemFig2a}{Displacement time evolution in undamped well system}{fig:continuumL21:continuumL21Fig2a}{0.3}
\imageFigure{../../figures/phy454/lec21_Displacement_time_evolution_in_damped_well_systemFig2b}{Displacement time evolution in damped well system}{fig:continuumL21:continuumL21Fig2b}{0.3}

For the inverted parabola our displacement takes the form of \cref{fig:continuumL21:continuumL21Fig3}
\imageFigure{../../figures/phy454/lec21_Time_evolution_of_displacement_in_instable_parabolic_configurationFig3}{Time evolution of displacement in instable parabolic configuration}{fig:continuumL21:continuumL21Fig3}{0.2}

For the ball on the table, assuming some friction that stops the ball, fairly quickly, we will have a displacement as illustrated in \cref{fig:continuumL21:continuumL21Fig3b}
\imageFigure{../../figures/phy454/lec21_Time_evolution_of_displacement_in_tabletop_configurationFig3b}{Time evolution of displacement in tabletop configuration}{fig:continuumL21:continuumL21Fig3b}{0.2}

\section{Characterizing stability}

Let us suppose that our displacement can be described in exponential form

\begin{equation}\label{eqn:continuumL21:10}
\delta x \sim e^{\sigma t}
\end{equation}

where \(\sigma\) is the \textit{growth rate of perturbation}, and is in general a complex number of the form

\begin{equation}\label{eqn:continuumL21:30}
\sigma = \sigma_{\text{R}} + i \sigma_{\text{I}}
\end{equation}

\subsection{Case I.  Oscillatory unstability}

A system of the form

\begin{equation}\label{eqn:stability:310}
\begin{aligned}
\sigma_{\text{R}} &= 0 \\
\sigma_{\text{I}} &> 0
\end{aligned}
\end{equation}

\textit{oscillatory unstable}.  An example of this is the undamped parabolic system illustrated above.

\subsection{Case II.  Marginal unstability}

\begin{equation}\label{eqn:stability:330}
\begin{aligned}
\delta x
&\sim e^{\sigma_{\text{R}} t} e^{i \sigma_{\text{I}} t} \\
&\sim e^{\sigma_{\text{R}} t} \left( \cos \sigma_{\text{I}} t + i \sin \sigma_{\text{I}} t \right)
\end{aligned}
\end{equation}

We will call systems of the form

\begin{equation}\label{eqn:stability:350}
\begin{aligned}
\sigma_{\text{I}} &= 0 \\
\sigma_{\text{R}} &> 0
\end{aligned}
\end{equation}

\textit{marginally unstable}.  We can have unstable systems with \(\sigma_{\text{I}} \ne 0\) but still \(\sigma_{\text{R}} > 0\), but these are less common.

\subsection{Case III.  Neutral stability}

\begin{equation}\label{eqn:stability:370}
\begin{aligned}
\sigma &= 0 \\
\sigma_{\text{R}} &= \sigma_{\text{I}} = 0
\end{aligned}
\end{equation}

An example of this was the billiard table example where the ball moved to a new location on the table after being bumped slightly.

\section{A mathematical description}

For a discussion of stability in fluids we will not only have to incorporate the Navier-Stokes equation as we have done, but will also have to bring in the heat equation.  Unfortunately that is not in the scope of this course to derive.  Let us consider as system heated on a bottom plate, and consider the fluid and convection due to heating.  This system is illustrated in \cref{fig:continuumL21:continuumL21Fig4}

\imageFigure{../../figures/phy454/lec21_Fluid_in_cavity_heated_on_the_bottom_plateFig4}{Fluid in cavity heated on the bottom plate}{fig:continuumL21:continuumL21Fig4}{0.2}

We start with Navier-Stokes as normal

\begin{equation}\label{eqn:continuumL21:50}
\rho \PD{t}{\Bu} + \rho (\Bu \cdot \spacegrad) \Bu = - \spacegrad p + \mu \spacegrad^2 \Bu - \rho \zcap g.
\end{equation}

For steady state with \(\Bu = 0\) initially (our base state), we will call the following the equation of the base state

\boxedEquation{eqn:continuumL21:70}{
\spacegrad p_s = -\rho_s \zcap g
}

We will allow perturbations of each of our variables

\begin{equation}\label{eqn:stability:390}
\begin{aligned}
\Bu &= \Bu_{\text{base}} + \delta \Bu = 0 + \delta \Bu \\
p &= p_s + \delta p \\
\rho &= \rho_s + \delta \rho
\end{aligned}
\end{equation}

After perturbation Navier-Stokes takes the form


\begin{dmath}\label{eqn:continuumL21:90}
(\rho_s + \delta \rho )\PD{t}{(0 + \delta \Bu)} + (\rho_s + \delta \rho) ((0 + \delta \Bu) \cdot \spacegrad) (0 + \delta \Bu) =
- \spacegrad (p_s + \delta p) + \mu \spacegrad^2 (0 + \delta \Bu) - (\rho_s + \delta \rho) \zcap g
\end{dmath}

Retaining only terms that are of first order of smallness.

\begin{equation}\label{eqn:continuumL21:110}
\rho_s \PD{t}{\delta \Bu} = - \spacegrad p_s - \spacegrad \delta p + \mu \spacegrad^2 \delta \Bu - \rho_s \zcap g - \delta \rho \zcap g
\end{equation}

applying our equation of base state \eqnref{eqn:continuumL21:70}, we have
%\spacegrad p_s = -\rho_s \zcap g
\begin{equation}\label{eqn:continuumL21:110b}
\rho_s \PD{t}{\delta \Bu} = \cancel{\rho_s \zcap g} - \spacegrad \delta p + \mu \spacegrad^2 \delta \Bu - \cancel{\rho_s \zcap g} - \delta \rho \zcap g,
\end{equation}

or

\begin{equation}\label{eqn:continuumL21:110c}
\rho_s \PD{t}{\delta \Bu} = - \spacegrad \delta p + \mu \spacegrad^2 \delta \Bu - \delta \rho \zcap g.
\end{equation}

we can write

\begin{equation}\label{eqn:continuumL21:130}
\left( \PD{t}{} - \nu \spacegrad^2 \right) \delta \Bu = -\inv{\rho_s} \spacegrad \delta p - \frac{\delta \rho}{\rho_s} \zcap g
\end{equation}

Applying the divergence operation on both sides, and using \(\spacegrad \cdot \Bu = 0\) so that \(\spacegrad \cdot \delta \Bu = 0\) we have

\begin{equation}\label{eqn:continuumL21:150}
\left( \PD{t}{} - \nu \spacegrad^2 \right) \cancel{\spacegrad \cdot \delta \Bu} = -\inv{\rho_s} \spacegrad^2 \delta p - (\zcap \cdot \spacegrad ) \frac{\delta \rho}{\rho_s} g,
\end{equation}

or

\begin{equation}\label{eqn:continuumL21:170}
\inv{\rho_s} \spacegrad^2 \delta p = - (\zcap \cdot \spacegrad ) \frac{\delta \rho}{\rho_s} g.
\end{equation}

Assuming that \(\rho_s\) is constant (actually that is already been done above), we can cancel it, leaving

\begin{equation}\label{eqn:continuumL21:190}
\spacegrad^2 \delta p = - (\zcap \cdot \spacegrad ) g \delta \rho = -g \PD{z}{} \delta \rho.
\end{equation}

operating once more with \(\PDi{z}{}\) we have

\begin{equation}\label{eqn:continuumL21:210}
\spacegrad^2 \PD{z}{\delta p} = -g \PDSq{z}{\delta \rho}.
\end{equation}

Going back to \eqnref{eqn:continuumL21:130} and taking only the \(z\) component we have

\begin{equation}\label{eqn:continuumL21:230}
\left( \PD{t}{} - \nu \spacegrad^2 \right) \delta w = -\inv{\rho_s} \PD{z}{\delta p} - \frac{\delta \rho}{\rho_s} g
\end{equation}

\begin{equation}\label{eqn:stability:410}
\begin{aligned}
\left( \PD{t}{} - \nu \spacegrad^2 \right) \spacegrad^2 \delta w
&= -\inv{\rho_s} \PD{z}{ \spacegrad^2 \delta p} - \frac{g}{\rho_s} \spacegrad^2 \delta \rho \\
&= -\frac{g}{\rho_s} \PDSq{z}{\delta \rho} - \frac{g}{\rho_s} \spacegrad^2 \delta \rho \\
&=
-\frac{g}{\rho_s} \left(
\PDSq{x}{}
+\PDSq{y}{}
\right)
\delta \rho \\
&=
g \alpha \left(
\PDSq{x}{}
+\PDSq{y}{}
\right)
\delta T
\end{aligned}
\end{equation}

in the last step we use the following assumed relation for temperature

\begin{equation}\label{eqn:continuumL21:250}
\delta \rho = - \rho_s \alpha \delta T.
\end{equation}

Here \(\alpha\) is the coefficient of thermal expansion.  This is just a statement that expansion and temperature are related (as we heat something, it expands), with the ratio of the density change relative to the original being linearly related to the change in temperature.

We have finally

\begin{equation}\label{eqn:continuumL21:290}
\left( \PD{t}{} - \nu \spacegrad^2 \right) \spacegrad^2 \delta w
=
g \alpha \left(
\PDSq{x}{}
+\PDSq{y}{}
\right)
\delta T.
\end{equation}

Solving this is the Rayleigh-Benard instability problem.

While this is a fourth order differential equation, it is still the same sort of problem logically as we have been working on.  Our boundary value conditions at \(z = 0\) are

\begin{equation}\label{eqn:continuumL21:430}
u, v, w, \delta u, \delta v, \delta w = 0.
\end{equation}

Also relevant will be a similar equation relating temperature and fluid flow rate

\begin{equation}\label{eqn:continuumL21:270}
\left( \PD{t}{} - \kappa \spacegrad^2 \right) \delta T = \Delta T \frac{\delta w}{d},
\end{equation}

which we will cover in the next (and final) lecture of the course.

   %
% Copyright � 2012 Peeter Joot.  All Rights Reserved.
% Licenced as described in the file LICENSE under the root directory of this GIT repository.
%

%
%
\label{chap:continuumL22}
\section{Thermal stability review.  Rayleigh Benard Problem}

Reading: \S 9.3 from \citep{acheson1990elementary}.

Illustrated in \cref{fig:continuumL22:continuumL22Fig1} is the heated channel we have been discussing

\imageFigure{../../figures/phy454/lec22_Channel_with_heat_applied_to_the_baseFig1}{Channel with heat applied to the base}{fig:continuumL22:continuumL22Fig1}{0.3}

We will take initial conditions

\begin{equation}\label{eqn:continuumL22:10}
\begin{aligned}
\PD{t}{\Bu} &= 0 \\
\PD{t}{T} &= 0
\end{aligned}
\end{equation}

\begin{equation}\label{eqn:continuumL22:30}
\rho \PD{t}{\Bu} + \rho (\Bu \cdot \spacegrad) \Bu = - \spacegrad p + \mu \spacegrad^2 \Bu + \rho \Bg.
\end{equation}

Our energy equation is

\begin{equation}\label{eqn:continuumL22:50}
\PD{t}{T} + (\Bu \cdot \spacegrad) T = \kappa \spacegrad^2 T.
\end{equation}

We have this \(\Bu \cdot \spacegrad\) term because our heat can be carried from one place to the other, due to the fluid motion.  We would not have this convective term for heat dissipation in solids because elements of a solid are not moving around in the bulk.

%We will also use
%
%\begin{equation}\label{eqn:continuumL22:70}
%\spacegrad p_s = \rho_s \Bg
%\end{equation}

In the steady (base) state we have

\begin{equation}\label{eqn:continuumL22:90}
0 = \kappa \spacegrad^2 T =
\kappa \left(
\PDSq{x}{}
+\PDSq{y}{}
+\PDSq{z}{} \right) T,
\end{equation}

but since we are only considering spatial variation with \(z\) we have

\begin{equation}\label{eqn:continuumL22:110}
\kappa \PDSq{z}{} T_s = 0,
\end{equation}

with solution

\begin{equation}\label{eqn:continuumL22:130}
T_s = T_0 - \frac{\Delta T}{d} z.
\end{equation}

We found that after application of the perturbation

\begin{equation}\label{eqn:continuumL22:150}
\begin{aligned}
\Bu &\rightarrow 0 + \delta \Bu \\
p &\rightarrow p_s + \delta p \\
\rho &\rightarrow \rho_s + \delta \rho \\
T &\rightarrow T_s + \delta T
\end{aligned}
\end{equation}

to the base state equations, our perturbed Navier-Stokes equation was

\begin{equation}\label{eqn:continuumL22:170}
\spacegrad^2 \left( \PD{t}{} - \nu \spacegrad^2 \right) \delta w = g \alpha
\left(
\PDSq{x}{}
+\PDSq{y}{}
\right)
\delta T.
\end{equation}

\section{Application of the perturbation to the energy equation}

\begin{equation}\label{eqn:continuumL22:190}
\PD{t}{T_s + \delta T} + (\delta \Bu \cdot \spacegrad) (T_s + \delta T) = \kappa \spacegrad^2 (T_s + \delta T)
\end{equation}

We have got

\begin{equation}\label{eqn:continuumL22:210}
\PD{t}{T_s} = 0.
\end{equation}

Using this, and \eqnref{eqn:continuumL22:110}, and neglecting any terms of second order smallness we have

\boxedEquation{eqn:continuumL22:230}{
\PD{t}{\delta T} + \delta \Bu \cdot \spacegrad T_s = \kappa \spacegrad^2 \delta T.
}

We would like to solve this and \eqnref{eqn:continuumL22:170} simultaneously.

\section{Non-dimensionalisation of the thermal velocity equation}

We would like to scale

\begin{equation}\label{eqn:continuumL22:250}
\begin{array}{l l}
x,y,z & \quad \mbox{with \(d\)} \\
t & \quad \mbox{with \(d^2/\nu\)} \\
\delta w & \quad \mbox{with \(\kappa/d\)} \\
\delta T & \quad \mbox{with \(\Delta T\)}
\end{array}
\end{equation}

Sanity check of dimensions

\begin{itemize}
\item viscosity dimensions
\begin{equation}\label{eqn:continuumL22:270}
[\nu] \sim \left[ \frac{1}{\text{T}} \right]/ \left[\inv{\text{L}^2} \right] \sim \frac{\text{L}^2}{\text{T}},
\end{equation}
\item thermal conductivity dimensions

Since \([ \kappa \spacegrad^2 ] \sim 1/\text{T}\), we have

\begin{equation}\label{eqn:continuumL22:610}
[ \kappa ] \sim \frac{\text{L}^2}{\text{T}}
\end{equation}

\item time scaling

\begin{equation}\label{eqn:continuumL22:290}
\left[\frac{d^2}{\nu}\right] \sim \frac{\text{L}^2}{\text{L}^2 \text{T}^{-1}} \sim \text{T}.
\end{equation}

\item velocity scaling

\begin{equation}\label{eqn:continuumL22:630}
[\kappa/d] \sim \frac{\text{L}^2}{\text{T}} \inv{\text{L}} \sim [\delta w]
\end{equation}

\end{itemize}

Looks like everything checks out.

Let us apply this rescaling to our perturbed velocity \eqnref{eqn:continuumL22:170}

\begin{equation}\label{eqn:continuumL22:310}
{\spacegrad'}^2 \left(
\frac{\nu}{d^2} \PD{t'}{} - \frac{\nu}{d^2} {\spacegrad'}^2
\right)
\frac{\kappa}{d}
\delta w'
=
g \alpha \Delta T
\left(
\PDSq{x'}{}
+\PDSq{y'}{}
\right) \delta T'.
\end{equation}

Introducing the \textit{Rayleigh number}

\begin{equation}\label{eqn:continuumL22:330}
\calR = \frac{g \alpha \Delta T d^3}{\nu \kappa},
\end{equation}

and dropping primes, we have

\boxedEquation{eqn:continuumL22:310b}{
\spacegrad^2 \left(
\PD{t}{} - \spacegrad^2
\right)
\delta w
=
\calR \left(
\PDSq{x}{}
+\PDSq{y}{}
\right) \delta T.
}

My class notes original had \(\calR\) with the value

\begin{equation*}
\frac{g \Delta T d^2}{\nu \alpha},
\end{equation*}

but performing this non-dimensionalization shows that this was either quoted incorrectly, or typed wrong in the heat of the moment.  A check against the text shows (equation (9.27)), shows that \eqnref{eqn:continuumL22:330} is correct.

\section{Non-dimensionalization of the energy equation}

Rescaling our energy \eqnref{eqn:continuumL22:230} we find

\begin{equation}\label{eqn:continuumL22:230b}
\frac{\nu}{d^2} \PD{t'}{\delta T'} + \frac{\kappa}{d^2} \delta \Bu' \cdot \spacegrad' T_s' = \frac{\kappa}{d^2} {\spacegrad'}^2 \delta T'.
\end{equation}

Introducing the \textit{Prandtl number}

\begin{equation}\label{eqn:continuumL22:370}
\text{P}_r = \frac{\nu}{\kappa},
\end{equation}

and dropping primes our non-dimensionalized energy equation takes the form

\boxedEquation{eqn:continuumL22:350}{
\left(
\text{P}_r
\PD{t}{} - \spacegrad^2 \right) \delta T = \delta w.
}

\section{Normal mode analysis}

We have got a pair of nasty looking coupled equations \eqnref{eqn:continuumL22:310b}, and \eqnref{eqn:continuumL22:350}.  Repeated so that we can see them together

\begin{subequations}
\begin{equation}\label{eqn:continuumL22:650}
\spacegrad^2 \left( \PD{t}{} - \spacegrad^2 \right) \delta w = \calR \left( \PDSq{x}{} +\PDSq{y}{} \right) \delta T
\end{equation}
\begin{equation}\label{eqn:continuumL22:670}
\left( \text{P}_r \PD{t}{} - \spacegrad^2 \right) \delta T = \delta w,
\end{equation}
\end{subequations}

it is clear that we can decouple these by inserting \eqnref{eqn:continuumL22:670} into \eqnref{eqn:continuumL22:650}.  Doing that gives us a beastly 6th order spatial equation for the perturbed temperature

\begin{equation}\label{eqn:continuumL22:690}
\left( \PD{t}{} - \spacegrad^2 \right) \left( \text{P}_r \PD{t}{} - \spacegrad^2 \right) \spacegrad^2 \delta T = \calR \left( \PDSq{x}{} +\PDSq{y}{} \right) \delta T.
\end{equation}

It is pointed out in the text we have all the \(x\) and \(y\) derivatives coming together we can apply separation of variables with

\begin{equation}\label{eqn:continuumL22:710}
\delta T = \Theta(z) f(x, y) e^{\sigma t},
\end{equation}

provided we introduce some restrictions on the form of \(f(x, y)\).  Here \(\sigma\) (if real) is the growth rate.  Applying the Laplacian to this assumed solution we find

\begin{equation}\label{eqn:continuumL22:730}
\spacegrad^2 \delta T = e^{\sigma t} \left( \Theta''(z) f(x, y) +\Theta(z) \spacegrad_t^2 f(x, y) \right),
\end{equation}

where

\begin{equation}\label{eqn:continuumL22:750}
\spacegrad_t^2 = \PDSq{x}{} +\PDSq{y}{}.
\end{equation}

For \eqnref{eqn:continuumL22:730} to be separable we require a constant proportionality

\begin{equation}\label{eqn:continuumL22:770}
\spacegrad_t^2 f(x, y) \propto f(x, y),
\end{equation}

or

\begin{equation}\label{eqn:continuumL22:790}
\spacegrad_t^2 f(x, y) \pm k^2 f(x, y) = 0.
\end{equation}

Picking \(+ k^2\) so that we do not have hyperbolic solutions, \(f\) must have the form

\begin{equation}\label{eqn:continuumL22:810}
f(x, y) = e^{i (k_x x + k_y y)},
\end{equation}

where

\begin{equation}\label{eqn:continuumL22:830}
k^2 = k_x^2 + k_y^2.
\end{equation}

Our separation of variables function now takes the form

\begin{equation}\label{eqn:continuumL22:850}
\delta T = \Theta(z) e^{ i ( k_1 x + k_2 y) + \sigma t}.
\end{equation}

Writing

\begin{equation}\label{eqn:continuumL22:570}
D = \frac{\partial}{\partial z},
\end{equation}

our beastly equation to solve is then given by

\begin{equation}\label{eqn:thermalStability:1130}
\begin{aligned}
0
&=
\left( \PD{t}{} - \spacegrad_t^2 - D^2 \right) \left( \text{P}_r \PD{t}{} - \spacegrad_t^2 -D^2 \right) \left( \spacegrad_t^2 -D^2 \right)
\Theta(z) e^{ i ( k_1 x + k_2 y) + \sigma t} \\
&\quad - \calR \spacegrad_t^2
\Theta(z) e^{ i ( k_1 x + k_2 y) + \sigma t}
\\
&=
\Bigl( \left( \sigma + k^2 - D^2 \right) \left( \text{P}_r \sigma + k^2 - D^2 \right) \left( -k^2 -D^2 \right) + \calR k^2
\Bigr) \Theta(z) e^{ i ( k_1 x + k_2 y) + \sigma t}
\end{aligned}
\end{equation}

This is now an equation for only \(\Theta(z)\)

\begin{equation}\label{eqn:continuumL22:870}
0 = \Bigl( \left( \sigma + k^2 - D^2 \right) \left( \text{P}_r \sigma + k^2 - D^2 \right) \left( -k^2 -D^2 \right) + \calR k^2
\Bigr) \Theta(z).
\end{equation}

Conceptually we have just a plain old LDE, and should we decide to expand this out we have something of the form

\begin{equation}\label{eqn:continuumL22:890}
0 = \Bigl( \alpha D^6 + \beta D^4 + \gamma D^2 + \zeta \Bigr) \Theta(z).
\end{equation}

Our standard toolbox method to solve this is to assume a solution \(\Theta(z) = e^{m z}\) and compute the characteristic equation.  We would have to solve

\begin{equation}\label{eqn:continuumL22:910}
0 = \alpha m^6 + \beta m^4 + \gamma m^2 + \zeta.
\end{equation}

Let us back up a bit instead.  Looking back to \eqnref{eqn:continuumL22:670}, it is clear that we will have the same separable form for our perturbed velocity since we have

\begin{equation}\label{eqn:continuumL22:670b}
\left( \text{P}_r \sigma + k^2 - D^2 \right) \Theta(z) e^{i (\Bk \cdot \Bx) } e^{\sigma t} = \delta w,
\end{equation}

where

\begin{equation}\label{eqn:continuumL22:930}
\Bk = (k_x, k_y, 0).
\end{equation}

Assuming a solution of the form

\begin{equation}\label{eqn:continuumL22:950}
\delta w = w(z) e^{ i ( k_1 x + k_2 y) + \sigma t},
\end{equation}

our velocity is then fully specified in terms of the temperature, since we have

\begin{equation}\label{eqn:continuumL22:970}
w(z) = \left( \text{P}_r \sigma + k^2 - D^2 \right) \Theta(z).
\end{equation}

\section{Back to our coupled equations}

Having gleamed an idea what the form of our solutions is, we can simplify our original coupled system, writing

\begin{subequations}
\label{eqn:continuumL22:990a}
\begin{equation}\label{eqn:continuumL22:990}
(D^2 - k^2) ( \sigma - (D^2 - k^2) ) w(z) = -\calR k^2 \Theta(z)
\end{equation}
\begin{equation}\label{eqn:continuumL22:1010}
(D^2 - k^2 -\text{P}_r \sigma ) \Theta(z) = -w(z).
\end{equation}
\end{subequations}

Considering the boundary conditions, if the heating is even then at \(z = t = 0\) we can not have any variation with \(x\) and \(y\), so can only have \(\Theta(0) = 0\).  Thus at the boundary, from \eqnref{eqn:continuumL22:990}, we have

\begin{equation}\label{eqn:continuumL22:1030}
0 = \evalbar{(D^2 - k^2) ( \sigma - (D^2 - k^2) ) w(z)}{z = 0}.
\end{equation}

From the continuity equation \(\spacegrad \cdot \Bu = 0\), the text argues that we also have \(\PDi{z}{\delta w} = 0\) on the boundary, so that on that plane we also have

\begin{equation}\label{eqn:continuumL22:1050}
0 = \evalbar{w(z) = D w(z)}{z = 0}.
\end{equation}

Expanding out \eqnref{eqn:continuumL22:1030} then gives us

\begin{equation}\label{eqn:continuumL22:1070}
0 = \evalbar{(D^4 - D^2 ( 2 k^2 + \sigma ) + k^2(\sigma + k^2))w}{z = 0}.
\end{equation}

or
\begin{equation}\label{eqn:continuumL22:1070b}
0 = \evalbar{(D^4 - D^2 ( 2 k^2 + \sigma ))w}{z = 0}.
\end{equation}

These boundary value constraints \eqnref{eqn:continuumL22:1050}, and \eqnref{eqn:continuumL22:1070b}, plus the coupled system equations \eqnref{eqn:continuumL22:990a} are the complete problem to solve.  To get a feel for the solution of this system, consider the system with the following simpler set of boundary value constraints

\begin{equation}\label{eqn:continuumL22:590}
\evalbar{w = D^2 w = D^4 w}{z = 0, 1} = 0,
\end{equation}

which in the text is described as the artificial problem of thermal instability for boundaries that are stress free.
\FIXME{it is not clear to me what that means without some thought ... return to this}
For such a system on the boundaries \(z = 0, 1\) (noting that we are still in dimensionless quantities), we have solutions

\begin{equation}\label{eqn:continuumL22:470}
\begin{aligned}
w(z) &= A_n \sin\left( n \pi z \right) \\
\Theta(z) &= B_n \sin\left( n \pi z \right).
\end{aligned}
\end{equation}

Note that we have

\begin{equation}\label{eqn:continuumL22:1090}
(D^2 - k^2)
\sin\left( n \pi z \right).
 = \left( -\left( n \pi \right)^2 - k^2 \right)
\sin\left( n \pi z \right).
\end{equation}

Inserting \eqnref{eqn:continuumL22:470} into our system \eqnref{eqn:continuumL22:990a}, we have

\begin{equation}\label{eqn:continuumL22:1110}
\begin{aligned}
\left( -\left( n \pi \right)^2 - k^2 \right)
\left( \sigma -
\left( -\left( n \pi \right)^2 - k^2 \right)
\right) A_n
\sin\left( n \pi z \right)
+ \calR k^2 B_n
\sin\left( n \pi z \right)
&= 0 \\
\left( -\left( n \pi \right)^2 - k^2 - \text{P}_r \sigma \right) B_n
\sin\left( n \pi z \right)
+ A_n
\sin\left( n \pi z \right)
&= 0
\end{aligned}
\end{equation}

For any \(A_n\), \(B_n\), we must then have

\begin{equation}\label{eqn:continuumL22:490}
0 =
\begin{vmatrix}
\left( n^2 \pi^2 + k^2 \right)^2 + \sigma \left( n^2 \pi^2 + k^2\right) & -\calR k^2 \\
-1 & n^2 \pi^2 + k^2 + \text{P}_r \sigma
\end{vmatrix}.
\end{equation}

For \(\sigma = 0\), this gives us the critical value for the Rayleigh number

\begin{equation}\label{eqn:continuumL22:510}
\calR = \frac{(k^2 + n^2 \pi^2)^3}{k^2},
\end{equation}

the value that separates our stable and unstable solutions.  On the other hand for
\begin{itemize}
\item \(\Real(\sigma) > 0\) (\(\Delta T > \Delta T_e\)), we have an instable system.
\item \(\Real(\sigma) < 0\) (\(\Delta T > \Delta T_e\)), we have a stable system.
\end{itemize}

This is illustrated in \cref{fig:continuumL22:continuumL22Fig2}.

\imageFigure{../../figures/phy454/lec22_Critical_Rayleighs_numberFig2}{Critical Rayleigh's number}{fig:continuumL22:continuumL22Fig2}{0.3}

The instability means that we will have instable flows as illustrated in \cref{fig:continuumL22:continuumL22Fig3}.

\imageFigure{../../figures/phy454/lec22_Instability_due_to_heatingFig3}{Instability due to heating}{fig:continuumL22:continuumL22Fig3}{0.2}

Solving for these critical points we find

\begin{subequations}
\begin{equation}\label{eqn:continuumL22:530}
k_e^2 = \frac{\pi^2}{2}
\end{equation}
\begin{equation}\label{eqn:continuumL22:550}
\calR_e = \frac{27 \pi^4}{4}
\end{equation}
\end{subequations}

\section{Multimedia presentations}

\begin{itemize}
\item Kelvin-Helmholtz instability.

Colored salt water underneath, with unsalted water on top.  Apparatus tilted causing flow of one over the other.  Instability of the interface.

See \citep{wiki:KelvinHelmholtz} for a really cool animation of a simulation of this effect.  It ends up looking very fractal.  Also interesting is the picture of this observed for real in the atmosphere of Saturn.

\item A simulated mushroom cloud occurring with one fluid seeping into another.  This looks it matches what we find under Rayleigh-Taylor instability in \citep{wiki:RayleighTaylor}.

\item plume, motion up through a denser fluid.

\item Plateau-Rayleigh instability.  Drop pinching off.  See instability in the fluid channel feeding the drop.  A crude illustration of this can be found in \cref{fig:continuumL22:continuumL22Fig4}.

\imageFigure{../../figures/phy454/lec22_Crude_illustration_of_instability_leading_to_a_drop_pinching_offFig4}{Crude illustration of instability leading to a drop pinching off}{fig:continuumL22:continuumL22Fig4}{0.2}

A better illustrations (and animations) can be found in \citep{wiki:PlateauRayleigh}.

\item Jet of water injected into a rotating tub on a turntable.  Jet forms and surfaces.

\end{itemize}


% turn these into the solutions for problems assigned in all the above sections.
\part{Worked problems}
   %\part{Problem Sets and Tests}
   %\documentclass[]{eliblog}

\usepackage{color}
%\usepackage{txfonts} % for xi

\usepackage{amsmath}
\usepackage{mathpazo}

%
% shorthand for bold symbols, convenient for vectors and matrices
%
\newcommand{\Ba}[0]{\mathbf{a}}
\newcommand{\Bb}[0]{\mathbf{b}}
\newcommand{\Bc}[0]{\mathbf{c}}
\newcommand{\Bd}[0]{\mathbf{d}}
\newcommand{\Be}[0]{\mathbf{e}}
\newcommand{\Bf}[0]{\mathbf{f}}
\newcommand{\Bg}[0]{\mathbf{g}}
\newcommand{\Bh}[0]{\mathbf{h}}
\newcommand{\Bi}[0]{\mathbf{i}}
\newcommand{\Bj}[0]{\mathbf{j}}
\newcommand{\Bk}[0]{\mathbf{k}}
\newcommand{\Bl}[0]{\mathbf{l}}
\newcommand{\Bm}[0]{\mathbf{m}}
\newcommand{\Bn}[0]{\mathbf{n}}
\newcommand{\Bo}[0]{\mathbf{o}}
\newcommand{\Bp}[0]{\mathbf{p}}
\newcommand{\Bq}[0]{\mathbf{q}}
\newcommand{\Br}[0]{\mathbf{r}}
\newcommand{\Bs}[0]{\mathbf{s}}
\newcommand{\Bt}[0]{\mathbf{t}}
\newcommand{\Bu}[0]{\mathbf{u}}
\newcommand{\Bv}[0]{\mathbf{v}}
\newcommand{\Bw}[0]{\mathbf{w}}
\newcommand{\Bx}[0]{\mathbf{x}}
\newcommand{\By}[0]{\mathbf{y}}
\newcommand{\Bz}[0]{\mathbf{z}}
\newcommand{\BA}[0]{\mathbf{A}}
\newcommand{\BB}[0]{\mathbf{B}}
\newcommand{\BC}[0]{\mathbf{C}}
\newcommand{\BD}[0]{\mathbf{D}}
\newcommand{\BE}[0]{\mathbf{E}}
\newcommand{\BF}[0]{\mathbf{F}}
\newcommand{\BG}[0]{\mathbf{G}}
\newcommand{\BH}[0]{\mathbf{H}}
\newcommand{\BI}[0]{\mathbf{I}}
\newcommand{\BJ}[0]{\mathbf{J}}
\newcommand{\BK}[0]{\mathbf{K}}
\newcommand{\BL}[0]{\mathbf{L}}
\newcommand{\BM}[0]{\mathbf{M}}
\newcommand{\BN}[0]{\mathbf{N}}
\newcommand{\BO}[0]{\mathbf{O}}
\newcommand{\BP}[0]{\mathbf{P}}
\newcommand{\BQ}[0]{\mathbf{Q}}
\newcommand{\BR}[0]{\mathbf{R}}
\newcommand{\BS}[0]{\mathbf{S}}
\newcommand{\BT}[0]{\mathbf{T}}
\newcommand{\BU}[0]{\mathbf{U}}
\newcommand{\BV}[0]{\mathbf{V}}
\newcommand{\BW}[0]{\mathbf{W}}
\newcommand{\BX}[0]{\mathbf{X}}
\newcommand{\BY}[0]{\mathbf{Y}}
\newcommand{\BZ}[0]{\mathbf{Z}}

\newcommand{\Bzero}[0]{\mathbf{0}}
\newcommand{\Btheta}[0]{\boldsymbol{\theta}}
\newcommand{\Btau}[0]{\boldsymbol{\tau}}
\newcommand{\Bomega}[0]{\boldsymbol{\omega}}

%
% shorthand for unit vectors
%
\newcommand{\acap}[0]{\hat{\Ba}}
\newcommand{\bcap}[0]{\hat{\Bb}}
\newcommand{\ccap}[0]{\hat{\Bc}}
\newcommand{\dcap}[0]{\hat{\Bd}}
\newcommand{\ecap}[0]{\hat{\Be}}
\newcommand{\fcap}[0]{\hat{\Bf}}
\newcommand{\gcap}[0]{\hat{\Bg}}
\newcommand{\hcap}[0]{\hat{\Bh}}
\newcommand{\icap}[0]{\hat{\Bi}}
\newcommand{\jcap}[0]{\hat{\Bj}}
\newcommand{\kcap}[0]{\hat{\Bk}}
\newcommand{\lcap}[0]{\hat{\Bl}}
\newcommand{\mcap}[0]{\hat{\Bm}}
\newcommand{\ncap}[0]{\hat{\Bn}}
\newcommand{\ocap}[0]{\hat{\Bo}}
\newcommand{\pcap}[0]{\hat{\Bp}}
\newcommand{\qcap}[0]{\hat{\Bq}}
\newcommand{\rcap}[0]{\hat{\Br}}
\newcommand{\scap}[0]{\hat{\Bs}}
\newcommand{\tcap}[0]{\hat{\Bt}}
\newcommand{\ucap}[0]{\hat{\Bu}}
\newcommand{\vcap}[0]{\hat{\Bv}}
\newcommand{\wcap}[0]{\hat{\Bw}}
\newcommand{\xcap}[0]{\hat{\Bx}}
\newcommand{\ycap}[0]{\hat{\By}}
\newcommand{\zcap}[0]{\hat{\Bz}}
\newcommand{\thetacap}[0]{\hat{\Btheta}}

%
% to write R^n and C^n in a distinguishable fashion.  Perhaps change this
% to the double lined characters upon figuring out how to do so.
%
\newcommand{\C}[1]{$\mathbb{C}^{#1}$}
\newcommand{\R}[1]{$\mathbb{R}^{#1}$}

%
% various generally useful helpers
%

% derivative of #1 wrt. #2:
\newcommand{\D}[2] {\frac {d#2} {d#1}}

\newcommand{\inv}[1]{\frac{1}{#1}}
\newcommand{\cross}[0]{\times}

\newcommand{\abs}[1]{\lvert{#1}\rvert}
\newcommand{\norm}[1]{\lVert{#1}\rVert}
\newcommand{\innerprod}[2]{\langle{#1}, {#2}\rangle}
\newcommand{\dotprod}[2]{{#1} \cdot {#2}}
\newcommand{\bdotprod}[2]{\left({#1} \cdot {#2}\right)}
\newcommand{\crossprod}[2]{{#1} \cross {#2}}
\newcommand{\tripleprod}[3]{\dotprod{\left(\crossprod{#1}{#2}\right)}{#3}}

\DeclareMathOperator{\Proj}{Proj}
\DeclareMathOperator{\Span}{span}
\DeclareMathOperator{\Sgn}{sgn}
\DeclareMathOperator{\Area}{Area}
\DeclareMathOperator{\Volume}{Volume}

%
% A few miscellaneous things specific to this document
%
\newcommand{\crossop}[1]{\crossprod{#1}{}}

% R2 vector.
\newcommand{\VectorTwo}[2]{
\begin{bmatrix}
 {#1} \\
 {#2}
\end{bmatrix}
}

\newcommand{\VectorN}[1]{
\begin{bmatrix}
{#1}_1 \\
{#1}_2 \\
\vdots \\
{#1}_N \\
\end{bmatrix}
}

\newcommand{\DETuvij}[4]{
\begin{vmatrix}
 {#1}_{#3} & {#1}_{#4} \\
 {#2}_{#3} & {#2}_{#4}
\end{vmatrix}
}

\newcommand{\DETuvwijk}[6]{
\begin{vmatrix}
 {#1}_{#4} & {#1}_{#5} & {#1}_{#6} \\
 {#2}_{#4} & {#2}_{#5} & {#2}_{#6} \\
 {#3}_{#4} & {#3}_{#5} & {#3}_{#6}
\end{vmatrix}
}

\newcommand{\DETuvwxijkl}[8]{
\begin{vmatrix}
 {#1}_{#5} & {#1}_{#6} & {#1}_{#7} & {#1}_{#8} \\
 {#2}_{#5} & {#2}_{#6} & {#2}_{#7} & {#2}_{#8} \\
 {#3}_{#5} & {#3}_{#6} & {#3}_{#7} & {#3}_{#8} \\
 {#4}_{#5} & {#4}_{#6} & {#4}_{#7} & {#4}_{#8} \\
\end{vmatrix}
}

%\newcommand{\DETuvwxyijklm}[10]{
%\begin{vmatrix}
% {#1}_{#6} & {#1}_{#7} & {#1}_{#8} & {#1}_{#9} & {#1}_{#10} \\
% {#2}_{#6} & {#2}_{#7} & {#2}_{#8} & {#2}_{#9} & {#2}_{#10} \\
% {#3}_{#6} & {#3}_{#7} & {#3}_{#8} & {#3}_{#9} & {#3}_{#10} \\
% {#4}_{#6} & {#4}_{#7} & {#4}_{#8} & {#4}_{#9} & {#4}_{#10} \\
% {#5}_{#6} & {#5}_{#7} & {#5}_{#8} & {#5}_{#9} & {#5}_{#10}
%\end{vmatrix}
%}

% R3 vector.
\newcommand{\VectorThree}[3]{
\begin{bmatrix}
 {#1} \\
 {#2} \\
 {#3}
\end{bmatrix}
}



\author{Peeter Joot}
\email{peeter.joot@utoronto.ca, 920798560}
%%
% Copyright � 2015 Peeter Joot.  All Rights Reserved.
% Licenced as described in the file LICENSE under the root directory of this GIT repository.
%
\documentclass[]{eliblog}

\usepackage{amsmath}
\usepackage{mathpazo}

%
% shorthand for bold symbols, convenient for vectors and matrices
%
\newcommand{\Ba}[0]{\mathbf{a}}
\newcommand{\Bb}[0]{\mathbf{b}}
\newcommand{\Bc}[0]{\mathbf{c}}
\newcommand{\Bd}[0]{\mathbf{d}}
\newcommand{\Be}[0]{\mathbf{e}}
\newcommand{\Bf}[0]{\mathbf{f}}
\newcommand{\Bg}[0]{\mathbf{g}}
\newcommand{\Bh}[0]{\mathbf{h}}
\newcommand{\Bi}[0]{\mathbf{i}}
\newcommand{\Bj}[0]{\mathbf{j}}
\newcommand{\Bk}[0]{\mathbf{k}}
\newcommand{\Bl}[0]{\mathbf{l}}
\newcommand{\Bm}[0]{\mathbf{m}}
\newcommand{\Bn}[0]{\mathbf{n}}
\newcommand{\Bo}[0]{\mathbf{o}}
\newcommand{\Bp}[0]{\mathbf{p}}
\newcommand{\Bq}[0]{\mathbf{q}}
\newcommand{\Br}[0]{\mathbf{r}}
\newcommand{\Bs}[0]{\mathbf{s}}
\newcommand{\Bt}[0]{\mathbf{t}}
\newcommand{\Bu}[0]{\mathbf{u}}
\newcommand{\Bv}[0]{\mathbf{v}}
\newcommand{\Bw}[0]{\mathbf{w}}
\newcommand{\Bx}[0]{\mathbf{x}}
\newcommand{\By}[0]{\mathbf{y}}
\newcommand{\Bz}[0]{\mathbf{z}}
\newcommand{\BA}[0]{\mathbf{A}}
\newcommand{\BB}[0]{\mathbf{B}}
\newcommand{\BC}[0]{\mathbf{C}}
\newcommand{\BD}[0]{\mathbf{D}}
\newcommand{\BE}[0]{\mathbf{E}}
\newcommand{\BF}[0]{\mathbf{F}}
\newcommand{\BG}[0]{\mathbf{G}}
\newcommand{\BH}[0]{\mathbf{H}}
\newcommand{\BI}[0]{\mathbf{I}}
\newcommand{\BJ}[0]{\mathbf{J}}
\newcommand{\BK}[0]{\mathbf{K}}
\newcommand{\BL}[0]{\mathbf{L}}
\newcommand{\BM}[0]{\mathbf{M}}
\newcommand{\BN}[0]{\mathbf{N}}
\newcommand{\BO}[0]{\mathbf{O}}
\newcommand{\BP}[0]{\mathbf{P}}
\newcommand{\BQ}[0]{\mathbf{Q}}
\newcommand{\BR}[0]{\mathbf{R}}
\newcommand{\BS}[0]{\mathbf{S}}
\newcommand{\BT}[0]{\mathbf{T}}
\newcommand{\BU}[0]{\mathbf{U}}
\newcommand{\BV}[0]{\mathbf{V}}
\newcommand{\BW}[0]{\mathbf{W}}
\newcommand{\BX}[0]{\mathbf{X}}
\newcommand{\BY}[0]{\mathbf{Y}}
\newcommand{\BZ}[0]{\mathbf{Z}}

\newcommand{\Bzero}[0]{\mathbf{0}}
\newcommand{\Btheta}[0]{\boldsymbol{\theta}}
\newcommand{\Btau}[0]{\boldsymbol{\tau}}
\newcommand{\Bomega}[0]{\boldsymbol{\omega}}

%
% shorthand for unit vectors
%
\newcommand{\acap}[0]{\hat{\Ba}}
\newcommand{\bcap}[0]{\hat{\Bb}}
\newcommand{\ccap}[0]{\hat{\Bc}}
\newcommand{\dcap}[0]{\hat{\Bd}}
\newcommand{\ecap}[0]{\hat{\Be}}
\newcommand{\fcap}[0]{\hat{\Bf}}
\newcommand{\gcap}[0]{\hat{\Bg}}
\newcommand{\hcap}[0]{\hat{\Bh}}
\newcommand{\icap}[0]{\hat{\Bi}}
\newcommand{\jcap}[0]{\hat{\Bj}}
\newcommand{\kcap}[0]{\hat{\Bk}}
\newcommand{\lcap}[0]{\hat{\Bl}}
\newcommand{\mcap}[0]{\hat{\Bm}}
\newcommand{\ncap}[0]{\hat{\Bn}}
\newcommand{\ocap}[0]{\hat{\Bo}}
\newcommand{\pcap}[0]{\hat{\Bp}}
\newcommand{\qcap}[0]{\hat{\Bq}}
\newcommand{\rcap}[0]{\hat{\Br}}
\newcommand{\scap}[0]{\hat{\Bs}}
\newcommand{\tcap}[0]{\hat{\Bt}}
\newcommand{\ucap}[0]{\hat{\Bu}}
\newcommand{\vcap}[0]{\hat{\Bv}}
\newcommand{\wcap}[0]{\hat{\Bw}}
\newcommand{\xcap}[0]{\hat{\Bx}}
\newcommand{\ycap}[0]{\hat{\By}}
\newcommand{\zcap}[0]{\hat{\Bz}}
\newcommand{\thetacap}[0]{\hat{\Btheta}}

%
% to write R^n and C^n in a distinguishable fashion.  Perhaps change this
% to the double lined characters upon figuring out how to do so.
%
\newcommand{\C}[1]{$\mathbb{C}^{#1}$}
\newcommand{\R}[1]{$\mathbb{R}^{#1}$}

%
% various generally useful helpers
%

% derivative of #1 wrt. #2:
\newcommand{\D}[2] {\frac {d#2} {d#1}}

\newcommand{\inv}[1]{\frac{1}{#1}}
\newcommand{\cross}[0]{\times}

\newcommand{\abs}[1]{\lvert{#1}\rvert}
\newcommand{\norm}[1]{\lVert{#1}\rVert}
\newcommand{\innerprod}[2]{\langle{#1}, {#2}\rangle}
\newcommand{\dotprod}[2]{{#1} \cdot {#2}}
\newcommand{\bdotprod}[2]{\left({#1} \cdot {#2}\right)}
\newcommand{\crossprod}[2]{{#1} \cross {#2}}
\newcommand{\tripleprod}[3]{\dotprod{\left(\crossprod{#1}{#2}\right)}{#3}}

\DeclareMathOperator{\Proj}{Proj}
\DeclareMathOperator{\Span}{span}
\DeclareMathOperator{\Sgn}{sgn}
\DeclareMathOperator{\Area}{Area}
\DeclareMathOperator{\Volume}{Volume}

%
% A few miscellaneous things specific to this document
%
\newcommand{\crossop}[1]{\crossprod{#1}{}}

% R2 vector.
\newcommand{\VectorTwo}[2]{
\begin{bmatrix}
 {#1} \\
 {#2}
\end{bmatrix}
}

\newcommand{\VectorN}[1]{
\begin{bmatrix}
{#1}_1 \\
{#1}_2 \\
\vdots \\
{#1}_N \\
\end{bmatrix}
}

\newcommand{\DETuvij}[4]{
\begin{vmatrix}
 {#1}_{#3} & {#1}_{#4} \\
 {#2}_{#3} & {#2}_{#4}
\end{vmatrix}
}

\newcommand{\DETuvwijk}[6]{
\begin{vmatrix}
 {#1}_{#4} & {#1}_{#5} & {#1}_{#6} \\
 {#2}_{#4} & {#2}_{#5} & {#2}_{#6} \\
 {#3}_{#4} & {#3}_{#5} & {#3}_{#6}
\end{vmatrix}
}

\newcommand{\DETuvwxijkl}[8]{
\begin{vmatrix}
 {#1}_{#5} & {#1}_{#6} & {#1}_{#7} & {#1}_{#8} \\
 {#2}_{#5} & {#2}_{#6} & {#2}_{#7} & {#2}_{#8} \\
 {#3}_{#5} & {#3}_{#6} & {#3}_{#7} & {#3}_{#8} \\
 {#4}_{#5} & {#4}_{#6} & {#4}_{#7} & {#4}_{#8} \\
\end{vmatrix}
}

%\newcommand{\DETuvwxyijklm}[10]{
%\begin{vmatrix}
% {#1}_{#6} & {#1}_{#7} & {#1}_{#8} & {#1}_{#9} & {#1}_{#10} \\
% {#2}_{#6} & {#2}_{#7} & {#2}_{#8} & {#2}_{#9} & {#2}_{#10} \\
% {#3}_{#6} & {#3}_{#7} & {#3}_{#8} & {#3}_{#9} & {#3}_{#10} \\
% {#4}_{#6} & {#4}_{#7} & {#4}_{#8} & {#4}_{#9} & {#4}_{#10} \\
% {#5}_{#6} & {#5}_{#7} & {#5}_{#8} & {#5}_{#9} & {#5}_{#10}
%\end{vmatrix}
%}

% R3 vector.
\newcommand{\VectorThree}[3]{
\begin{bmatrix}
 {#1} \\
 {#2} \\
 {#3}
\end{bmatrix}
}



\author{Peeter Joot}
\email{peeter.joot@gmail.com}


\chapter{PHY456H1S Problem Set 1.}
%\label{chap:continuumProblemSet1}
%\blogpage{http://sites.google.com/site/peeterjoot2/math2012/continuumProblemSet1.pdf}
%\date{Feb 3, 2012}
%\revisionInfo{continuumProblemSet1.tex}

\beginArtNoToc
%\section{Disclaimer.}
%
%This problem set is as yet ungraded.

\section{Problem Q1.}
\subsection{Statement}

For the stress tensor

\begin{equation}\label{eqn:continuumProblemSet1:10}
\sigma =
\begin{bmatrix}
6 & 0 & 2 \\
0 & 1 & 1 \\
2 & 1 & 3
\end{bmatrix}
\text{M Pa}
\end{equation}

Find the corresponding strain tensor, assuming an isotropic solid with Young's modulus $E = 200 \times 10^9 \text{N}/\text{m}^2$ and Poisson's ration $\nu = 0.35$.

\subsection{Solution}

We need to express the relation between stress and strain in terms of Young's modulus and Poisson's ratio.  In terms of Lam\'e parameters our model for the relations between stress and strain for an isotropic solid was given as

\begin{equation}\label{eqn:continuumProblemSet1:110}
\sigma_{ij} = \lambda e_{kk} \delta_{ij} + 2 \mu e_{ij}.
\end{equation}

Computing the trace

\begin{equation}\label{eqn:continuumProblemSet1:130}
\sigma_{kk} = (3 \lambda + 2 \mu) e_{kk},
\end{equation}

allows us to invert the relationship

\begin{equation}\label{eqn:continuumProblemSet1:150}
2 \mu e_{ij} = \sigma_{ij} - \lambda \frac{\sigma_{kk}}{3 \lambda + 2 \mu} \delta_{ij}.
\end{equation}

In terms of Poisson's ratio $\nu$ and Young's modulus $E$, our Lam\'e parameters were found to be

\begin{align}\label{eqn:continuumProblemSet1:170}
\lambda &= \frac{ E \nu }{(1 - 2 \nu)(1 + \nu)} \\
\mu &= \frac{E}{2(1 + \nu)},
\end{align}

and

\begin{align*}
3 \lambda + 2 \mu
&= \frac{ 3 E \nu }{(1 - 2 \nu)(1 + \nu)} + \frac{E}{1 + \nu} \\
&= \frac{E}{1 + \nu} \left( \frac{3 \nu}{1 - 2 \nu} + 1\right) \\
&= \frac{E}{1 + \nu} \frac{1 + \nu}{1 - 2 \nu} \\
&= \frac{E}{1 - 2 \nu}.
\end{align*}

Our stress strain model for the relationship for an isotropic solid becomes
we find

\begin{align*}
\frac{E}{1 + \nu} e_{ij}
&=
\sigma_{ij}
-
\frac{ E \nu }{(1 - 2 \nu)(1 + \nu)} \frac{1 - 2 \nu}{E}
\sigma_{kk} \delta_{ij} \\
&=
\sigma_{ij}
-
\frac{ \nu }{1 + \nu}
\sigma_{kk} \delta_{ij} \\
\end{align*}

or

\begin{equation}\label{eqn:continuumProblemSet1:190}
e_{ij}
=
\inv{E}
\left(
(1 + \nu)
\sigma_{ij}
-
\nu
\sigma_{kk} \delta_{ij}
\right).
\end{equation}

As a sanity check note that this matches (5.12) of \cite{landau1960theory}, although they use a notation of $\sigma$ instead of $\nu$ for Poisson's ratio.  We are now ready to tackle the problem.  First we need the trace of the stress tensor

\begin{equation}\label{eqn:continuumProblemSet1:210}
\sigma_{kk} = (6 + 1 + 3) \text{M Pa} = 10 \text{M Pa},
\end{equation}

\begin{align*}
e_{ij}
&=
\inv{E}
\left(
(1 + \nu)
\begin{bmatrix}
6 & 0 & 2 \\
0 & 1 & 1 \\
2 & 1 & 3
\end{bmatrix}
-
10 \nu
\begin{bmatrix}
1 & 0 & 0 \\
0 & 1 & 0 \\
0 & 0 & 1 \\
\end{bmatrix}
\right)
\text{M Pa} \\
&=
\inv{E}
\left(
\begin{bmatrix}
6 & 0 & 2 \\
0 & 1 & 1 \\
2 & 1 & 3
\end{bmatrix}
+ 0.35
\begin{bmatrix}
-4 & 0 & 2 \\
0 & -9 & 1 \\
2 & 1 & -7
\end{bmatrix}
\right)
\text{M Pa} \\
&=
\inv{2 \times 10^{5}}
\left(
\begin{bmatrix}
6 & 0 & 2 \\
0 & 1 & 1 \\
2 & 1 & 3
\end{bmatrix}
+ 0.35
\begin{bmatrix}
-4 & 0 & 2 \\
0 & -9 & 1 \\
2 & 1 & -7
\end{bmatrix}
\right)
\end{align*}

Expanding out the last bits of arithmetic the strain tensor is found to have the form

\begin{equation}\label{eqn:continuumProblemSet1:230}
e_{ij}
=
\begin{bmatrix}
 23 & 0 & 13.5 \\
 0 & -10.75 & 6.75 \\
 13.5 & 6.75 & 2.75
\end{bmatrix}
 10^{-6}.
\end{equation}

Note that this is dimensionless, unlike the stress.

\section{Problem Q2.}
\subsection{Statement}

Small displacement field in a material is given by

\begin{align}\label{eqn:continuumProblemSet1:30}
e_1 &= 2 x_1 x_2 \\
e_2 &= x_3^2 \\
e_3 &= x_1^2 - x_3
\end{align}

Find

\begin{enumerate}
\item the infinitesimal strain tensor $e_{ij}$,
\item the principal strains and the corresponding principal axes at $(x_1, x_2, x_3) = (1, 2, 4)$,
\item Is the body under compression or expansion?
\end{enumerate}

\subsection{Solution.  infinitesimal strain tensor $e_{ij}$}

Diving right in, we have

\begin{align*}
e_{11}
&= \PD{x_1}{e_1} \\
&= \PD{x_1}{}2 x_1 x_2 \\
&= 2 x_2
\end{align*}

\begin{align*}
e_{22}
&= \PD{x_2}{e_2} \\
&= \PD{x_2}{} x_3^2 \\
&= 0
\end{align*}

\begin{align*}
e_{33}
&= \PD{x_3}{e_3} \\
&= \PD{x_3}{} ( x_1^2 - x_3 ) \\
&= -1
\end{align*}

\begin{align*}
e_{12}
&=
\inv{2} \left(
\PD{x_1}{e_2}
+
\PD{x_2}{e_1}
\right) \\
&=
\inv{2}
\left(
\cancel{\PD{x_1}{} x_3^2 }
+
\PD{x_2}{} 2 x_1 x_2
\right) \\
&=
x_1
\end{align*}

\begin{align*}
e_{23}
&=
\inv{2} \left(
\PD{x_2}{e_3}
+
\PD{x_3}{e_2}
\right) \\
&=
\inv{2}
\left(
\cancel{\PD{x_2}{} (x_1^2 - x_3 )}
+
\PD{x_3}{} x_3^2
\right) \\
&=
x_3
\end{align*}

\begin{align*}
e_{31}
&=
\inv{2} \left(
\PD{x_3}{e_1}
+
\PD{x_1}{e_3}
\right) \\
&=
\inv{2}
\left(
\cancel{\PD{x_3}{} 2 x_1 x_2 }
+
\PD{x_1}{} (x_1^2 - x_3 )
\right) \\
&=
x_1
\end{align*}

In matrix form we have

\begin{equation}\label{eqn:continuumProblemSet1:250}
\Be =
\begin{bmatrix}
2 x_2 & x_1 & x_1 \\
x_1 & 0 & x_3 \\
x_1 & x_3 & -1 \\
\end{bmatrix}
\end{equation}

\subsection{Solution.  principle strains and axes}

At the point $(1, 2, 4)$ the strain tensor has the value

\begin{equation}\label{eqn:continuumProblemSet1:270}
\Be =
\begin{bmatrix}
4 & 1 & 1 \\
1 & 0 & 4 \\
1 & 4 & -1
\end{bmatrix}.
\end{equation}

We wish to diagonalize this, solving the characteristic equation for the eigenvalues $\lambda$

\begin{align*}
0 &=
\begin{vmatrix}
4 -\lambda & 1 & 1 \\
1 & -\lambda & 4 \\
1 & 4 & -1 -\lambda
\end{vmatrix} \\
&=
(4 -\lambda )
\begin{vmatrix}
 -\lambda & 4 \\
 4 & -1 -\lambda
\end{vmatrix}
-
\begin{vmatrix}
1 & 1 \\
4 & -1 -\lambda
\end{vmatrix}
+
\begin{vmatrix}
1 & 1 \\
-\lambda & 4 \\
\end{vmatrix} \\
&=
(4 - \lambda)(\lambda^2 + \lambda - 16)
-(-1 -\lambda - 4)
+(4 + \lambda) \\
\end{align*}

We find the characteristic equation to be

\begin{equation}\label{eqn:continuumProblemSet1:290}
0 = -\lambda^3 + 3 \lambda^2 + 22\lambda - 55.
\end{equation}

This doesn't appear to lend itself easily to manual solution (there are no obvious roots to factor out).  As expected, since the matrix is symmetric, a plot (\ref{fig:continuumL8:continuumProblemSet1Q2fig1}) shows that all our roots are real

\begin{figure}[htp]
   \centering
   \includegraphics[totalheight=0.2\textheight]{continuumProblemSet1Q2fig1}
   \caption{Q2.  Characteristic equation.}\label{fig:continuumL8:continuumProblemSet1Q2fig1}
\end{figure}

Numerically, we determine these roots to be

\begin{equation}\label{eqn:continuumProblemSet1:310}
\{5.19684, -4.53206, 2.33522\}
\end{equation}

with the corresponding basis (orthonormal eigenvectors), the principle axes are

\begin{equation}\label{eqn:continuumProblemSet1:330}
\left\{
\pcap_1,
\pcap_2,
\pcap_3
\right\}
=
\left\{
\begin{bmatrix}
0.76291 \\
0.480082 \\
0.433001
\end{bmatrix},
\begin{bmatrix}
-0.010606 \\
-0.660372 \\
0.750863
\end{bmatrix},
\begin{bmatrix}
-0.646418 \\
0.577433 \\
0.498713
\end{bmatrix}
\right\}.
\end{equation}

\subsection{Solution.  Is body under compression or expansion?}

To consider this question, suppose that as in the previous part, we determine a basis for which our strain tensor $e_{ij} = p_i \delta_{ij}$ is diagonal with respect to that basis at a given point $\Bx_0$.  We can then simplify the form of the stress tensor at that point in the object

\begin{align*}
\sigma_{ij}
&=
\frac{E}{1 + \nu} \left(
e_{ij} + \frac{\nu}{1 - 2 \nu} e_{mm} \delta_{ij}
\right) \\
&=
\frac{E}{1 + \nu} \left(
p_i
 + \frac{\nu}{1 - 2 \nu} e_{mm}
\right)
\delta_{ij}.
\end{align*}

We see that the stress tensor at this point is also necessarily diagonal if the strain is diagonal in that basis (with the implicit assumption here that we are talking about an isotropic material).  Noting that the Poisson ratio is bounded according to

\begin{equation}\label{eqn:continuumProblemSet1:350}
-1 \le \nu \le \inv{2},
\end{equation}

so if our trace is positive (as it is in this problem for all points $x_2 > 1/2$), then any positive principle strain value will result in a positive stress along that direction).  For example at the point $(1,2,4)$ of the previous part of this problem (for which $x_2 > 1/2$), we have

\begin{equation}\label{eqn:continuumProblemSet1:370}
\sigma_{ij}
=
\frac{E}{1 + \nu}
\begin{bmatrix}
5.19684
+ \frac{3 \nu}{1 - 2 \nu}  & 0 & 0 \\
0 & -4.53206
+ \frac{3 \nu}{1 - 2 \nu}  & 0 \\
0 & 0 & 2.33522
+ \frac{3 \nu}{1 - 2 \nu}
\end{bmatrix}.
\end{equation}

We see that at this point the $(1,1)$ and $(3,3)$ components of stress is positive (expansion in those directions) regardless of the material, and provided that

\begin{equation}\label{eqn:continuumProblemSet1:390}
\frac{3 \nu}{1 - 2 \nu} > 4.53206
\end{equation}

(i.e. $\nu > 0.375664$) the material is under expansion in all directions.  For $\nu < 0.375664$ the material at that point is expanding in the $\pcap_1$ and $\pcap_3$ directions, but under compression in the $\pcap_2$ directions.

\section{Problem Q3.}
\subsection{Statement}

The stress tensor at a point has components given by

\begin{equation}\label{eqn:continuumProblemSet1:50}
\sigma =
\begin{bmatrix}
1 & -2 & 2 \\
-2 & 3 & 1 \\
2 & 1 & -1
\end{bmatrix}.
\end{equation}

Find the traction vector across an area normal to the unit vector

\begin{equation}\label{eqn:continuumProblemSet1:70}
\ncap = ( \sqrt{2} \Be_1 - \Be_2 + \Be_3)/2
\end{equation}

Can you construct a tangent vector $\Btau$ on this plane by inspection?  What are the components of the force per unit area along the normal $\ncap$ and tangent $\Btau$ on that surface?  (hint: projection of the traction vector.)

\subsection{Solution}

The traction vector, the force per unit volume that holds a body in equilibrium, in coordinate form was

\begin{equation}\label{eqn:continuumProblemSet1:410}
P_i = \sigma_{ik} n_k
\end{equation}

where $n_k$ was the coordinates of the normal to the surface with area $df_k$.  In matrix form, this is just

\begin{equation}\label{eqn:continuumProblemSet1:430}
\BP = \sigma \ncap,
\end{equation}

so our traction vector for this stress tensor and surface normal is just

\begin{align*}
\BP &=
\inv{2}
\begin{bmatrix}
1 & -2 & 2 \\
-2 & 3 & 1 \\
2 & 1 & -1
\end{bmatrix}
\begin{bmatrix}
\sqrt{2} \\
-1 \\
1
\end{bmatrix} \\
&=
\inv{2}
\begin{bmatrix}
\sqrt{2} + 2 + 2 \\
-2\sqrt{2} - 3 + 1 \\
2\sqrt{2} - 1 -1
\end{bmatrix} \\
&=
\begin{bmatrix}
\sqrt{2}/2 + 2 \\
-\sqrt{2} -1 \\
\sqrt{2} - 1
\end{bmatrix}
\end{align*}

We also want a vector in the plane, and can pick

\begin{equation}\label{eqn:continuumProblemSet1:450}
\Btau = 
\inv{\sqrt{2}}
\begin{bmatrix}
0 \\
1 \\
1
\end{bmatrix},
\end{equation}

or

\begin{equation}\label{eqn:continuumProblemSet1:470}
\Btau' = 
\begin{bmatrix}
\inv{\sqrt{2}} \\
\inv{2} \\
-\inv{2}
\end{bmatrix},
\end{equation}

It's clear that either of these is normal to $\ncap$ (the first can also be computed by normalizing $\ncap \cross \Be_1$, and the second with one round of Gram-Schmidt).  However, neither of these vectors in the plane are particularly interesting since they are completely arbitrary.  Let's instead compute the projection and rejection of the traction vector with respect to the normal.  We find for the projection

\begin{align*}
(\BP \cdot \ncap) \ncap
&=
\inv{4}
\left(
\begin{bmatrix}
\sqrt{2}/2 + 2 \\
-\sqrt{2} -1 \\
\sqrt{2} - 1
\end{bmatrix}
\cdot 
\begin{bmatrix}
\sqrt{2} \\
-1 \\
1
\end{bmatrix} 
\right)
\begin{bmatrix}
\sqrt{2} \\
-1 \\
1
\end{bmatrix}  \\
&=
\inv{4}
\left( 
1 + 2\sqrt{2}
+\sqrt{2} +1 
+\sqrt{2} - 1
\right)
\begin{bmatrix}
\sqrt{2} \\
-1 \\
1
\end{bmatrix}  \\
&=
\inv{2}
\left( 
1 + 4\sqrt{2}
\right)
\ncap
\end{align*}

Our rejection, the component of the traction vector in the plane, is

\begin{align*}
(\BP \wedge \ncap) \ncap 
&=
\BP - (\BP \cdot \ncap)\ncap \\
&=
\inv{2}
\begin{bmatrix}
\sqrt{2}/2 + 2 \\
-\sqrt{2} -1 \\
\sqrt{2} - 1
\end{bmatrix}
-\inv{4}(1 + r \sqrt{2})
\begin{bmatrix}
\sqrt{2} \\
-1 \\
1
\end{bmatrix} \\
&=
\inv{4}
\begin{bmatrix}
\sqrt{2} \\
-3 \\
-5
\end{bmatrix}
\end{align*}

This gives us a another vector perpendicular to the normal $\ncap$

\begin{equation}\label{eqn:continuumProblemSet1:490}
\taucap = 
\inv{6}
\begin{bmatrix}
\sqrt{2} \\
-3 \\
-5
\end{bmatrix}.
\end{equation}

Wrapping up, we find the decomposition of the traction vector in the direction of the normal and its projection onto the plane to be

\begin{equation}\label{eqn:continuumProblemSet1:510}
\BP 
= 
\inv{2}(1 + 4\sqrt{2}) \ncap
+
\frac{3}{2} \taucap.
\end{equation}

The components we can read off by inspection.

\section{Problem Q4.}
\subsection{Statement}

The stress tensor of a body is given by

\begin{equation}\label{eqn:continuumProblemSet1:90}
\sigma =
\begin{bmatrix}
A \cos x & y^2 & C x \\
y^2 & B \sin y & z \\
C x & z & z^3
\end{bmatrix}
\end{equation}

Determine the constant $A$, $B$, and $C$ if the body is in equilibrium.

\subsection{Solution}

In the absence of external forces our equilibrium condition was

\begin{equation}\label{eqn:continuumProblemSet1:530}
\partial_k \sigma_{ik} = 0.
\end{equation}

In matrix form we wish to operate (to the left) with the gradient coordinate vector

\begin{align*}
0 
&= \sigma \lspacegrad \\
&=
\begin{bmatrix}
A \cos x & y^2 & C x \\
y^2 & B \sin y & z \\
C x & z & z^3
\end{bmatrix}
\begin{bmatrix}
\lpartial_x \\
\lpartial_y \\
\lpartial_z \\
\end{bmatrix} \\
&=
\begin{bmatrix}
\partial_x (A \cos x) + \partial_y(y^2) + \cancel{\partial_z(C x)} \\
\cancel{\partial_x (y^2)} + \partial_y(B \sin y) + \partial_z(z) \\
\partial_x (C x) + \cancel{\partial_y(z)} + \partial_z(z^3)
\end{bmatrix} \\
&=
\begin{bmatrix}
-A \sin x + 2 y \\
B \cos y + 1 \\
C + 3 z^2 
\end{bmatrix} \\
\end{align*}

So, our conditions for equilibrium will be satisfied when we have
\begin{align}\label{eqn:continuumProblemSet1:550}
A &= \frac{2 y }{\sin x} \\
B &= -\frac{1}{\cos y} \\
C &= -3 z^2,
\end{align}

provided $y \ne 0$, and $y \ne \pi/2 + n\pi$ for integer $n$.  If equilibrium is to hold along the $y = 0$ plane, then we must either also have $A = 0$ or also impose the restriction $x = m \pi$ (for integer $m$).

\EndArticle

   %
% Copyright � 2012 Peeter Joot.  All Rights Reserved.
% Licenced as described in the file LICENSE under the root directory of this GIT repository.
%

%
%
\makeoproblem{Steady rectilinear blood flow}{problem:fluids:blood}
{2012 problem set 2}
{
Imagine a steady rectilinear blood flow of the form \(\Bu = u(y) \ycap\) through a two dimensional artery.  It is driven by a constant pressure gradient \(G = -dp/dx\) maintained by an external `heart'.  The top and bottom walls of the artery are \(2h\) distance apart and the fluid satisfies no-slip boundary conditions at the walls.  Assuming that the fluid is Newtonian,


\makesubproblem{Show that the Navier-Stokes equation reduces to

\begin{equation}\label{eqn:continuumProblemSet2:20}
\frac{d^2 u}{dy^2} = -\frac{G}{\mu}
\end{equation}

where \(\mu\) is the viscosity of the blood.

}{problem:fluids:bloodReduceNS}

\makesubproblem{Show that the velocity profile of the fluid inside the artery is a parabolic profile}{problem:fluids:bloodVelocityProfile}
\makesubproblem{What is the maximum speed of the fluid?  Draw the velocity profile to show where the maximum speed occurs inside the artery}{problem:fluids:bloodMaxSpeed}
\makesubproblem{If due to smoking etc., the viscosity of the blood gets doubled, then what should be the new pressure gradient to be maintained by the `heart' to keep the liquid flux through the artery at the same level as the non-smoking one?}{problem:fluids:bloodViscositySmoking}
} % makeoproblem

\makeanswer{problem:fluids:blood}{

\makesubanswer{Navier-Stokes}{problem:fluids:bloodReduceNS}
%\unnumberedSubsection{Navier-Stokes equation for the system}
The Navier-Stokes equation, for an incompressible unidirectional fluid \(\Bu = (u, 0, 0)\), assuming that there is no \(z\) dependence, takes the form

\begin{subequations}
\begin{equation}\label{eqn:continuumProblemSet2:40}
\rho \PD{t}{u} + u \PD{x}{u} = - \PD{x}{p} + \mu \left( \PDSq{x}{} + \PDSq{y}{} \right) u
\end{equation}
\begin{equation}\label{eqn:continuumProblemSet2:60}
0 = - \PD{y}{p}
\end{equation}
\begin{equation}\label{eqn:continuumProblemSet2:80}
0 = - \PD{z}{p}
\end{equation}
\begin{equation}\label{eqn:continuumProblemSet2:100}
0 = \PD{x}{u}.
\end{equation}
\end{subequations}

With a steady state assumption we kill the \(\PDi{t}{u}\) term, and \eqnref{eqn:continuumProblemSet2:100} kills of the x-component of the Laplacian and our non-linear inertial term on the LHS, leaving just

\begin{subequations}
\begin{equation}\label{eqn:continuumProblemSet2:40a}
0 = - \PD{x}{p} + \mu \PDSq{y}{u}
\end{equation}
\begin{equation}\label{eqn:continuumProblemSet2:60b}
0 = - \PD{y}{p}
\end{equation}
\begin{equation}\label{eqn:continuumProblemSet2:80c}
0 = - \PD{z}{p}.
\end{equation}
\end{subequations}

With \(\PD{z}{p} = \PD{y}{p} = 0\), we have \(\PD{x}{p} = dp/dx = -G\), so \eqnref{eqn:continuumProblemSet2:40a} is reduced to

\begin{equation}\label{eqn:continuumProblemSet2:40b}
0 = G + \mu \PDSq{y}{u}.
\end{equation}

Finally, since we have an assumption of no z-dependence (\(\PDi{z}{u} = 0\)) and from the incompressibility assumption \eqnref{eqn:continuumProblemSet2:100} (\(\PDi{x}{u} = 0\)), we have

\begin{equation}\label{eqn:continuumProblemSet2:140}
\PDSq{y}{u} = \frac{d^2 u}{dy^2} = -\frac{G}{\mu},
\end{equation}

as desired.

\makesubanswer{Velocity profile}{problem:fluids:bloodVelocityProfile}
For the velocity profile, integrating \eqnref{eqn:continuumProblemSet2:140} twice, we have

\begin{equation}\label{eqn:continuumProblemSet2:160}
u = -\frac{G}{2 \mu} y^2 + A y + B.
\end{equation}

Application of the no-slip boundary value condition \(u(\pm h) = 0\), we have

\begin{equation}\label{eqn:continuumProblemSet2:180}
\begin{aligned}
0 &= -\frac{G}{2 \mu} h^2 + A h + B \\
0 &= -\frac{G}{2 \mu} h^2 - A h + B
\end{aligned}
\end{equation}

Adding and subtracting these, we find

\begin{subequations}
\begin{equation}\label{eqn:continuumProblemSet2:280}
A = 0
\end{equation}
\begin{equation}\label{eqn:continuumProblemSet2:300}
B = \frac{G}{2 \mu} h^2,
\end{equation}
\end{subequations}

so the velocity is given by the parabolic function

\begin{equation}\label{eqn:continuumProblemSet2:220}
u(y) = \frac{G}{2 \mu} \left( h^2 - y^2 \right).
\end{equation}

\makesubanswer{Maximum speed}{problem:fluids:bloodMaxSpeed}
It is clear that the maximum speed of the fluid is found at \(y = 0\)

\begin{equation}\label{eqn:continuumProblemSet2:240}
u(0) = \frac{G h^2}{2 \mu}
\end{equation}

The velocity profile for this flow is drawn in \cref{fig:continuumProblemSet2:continuumProblemSet2Fig1}.

%\begin{figure}[htp]
%   \centering
%   \includegraphics[totalheight=0.3\textheight]{continuumProblemSet2Fig1}
%   \caption{Velocity profile for 1D constant pressure gradient steady state flow}\label{fig:continuumProblemSet2:continuumProblemSet2Fig1}
%\end{figure}

\pdfTexFigure{../../figures/phy454/continuumProblemSet2Fig1r2.pdf_tex}{Velocity profile for 1D constant pressure gradient steady state flow}{fig:continuumProblemSet2:continuumProblemSet2Fig1}{0.3}

\makesubanswer{Effects of viscosity doubling}{problem:fluids:bloodViscositySmoking}
With our velocity being dependent on the \(G/\mu\) ratio, it is clear that to consider the effects of viscosity doubling, even without calculating the flux, that we will need  twice the pressure gradient if the viscosity is doubled to maintain the same flux through the artery and veins.  To demonstrate this more thoroughly we can calculate this mass flux.  For an element of mass leaving a portion of the conduit, bounded by the plane normal to \(\xcap\) we have

\begin{equation}\label{eqn:continuumProblemSet2:980}
\begin{aligned}
\frac{dm}{dt}
&= \rho \frac{dV}{dt} \\
&= \rho dz dy \Bu \cdot \xcap
\end{aligned}
\end{equation}

Integrating this over a width \(\Delta z\), our flux through the plane is

\begin{equation}\label{eqn:continuumProblemSet2:1000}
\begin{aligned}
\text{Flux}
&=
\int_0^{\Delta z} dz
\int_{-h}^h dy \frac{G}{2 \mu} \left( h^2 - y^2 \right) \\
&=
\Delta z
\frac{G}{2 \mu}
\evalrange{ \left( h^2 y - \inv{3} y^3 \right) }{-h}{h} \\
&=
\Delta z
\frac{G h^3}{\mu} \left( 1 - \inv{3} \right)  \\
&=
\Delta z \frac{2 G h^3}{3 \mu}.
\end{aligned}
\end{equation}

Doubling the blood viscosity for our smoker, our respective fluxes are

\begin{equation}\label{eqn:continuumProblemSet2:320}
\begin{aligned}
\text{Flux}_{\text{smoker}} &= \Delta z \frac{2 G_{\text{smoker}} h^3}{3 (2 \mu)}  \\
\text{Flux}_{\text{non-smoker}} &= \Delta z \frac{2 G h^3}{3 \mu}.
\end{aligned}
\end{equation}

Demanding equality before and after smoking we find

\begin{equation}\label{eqn:continuumProblemSet2:260}
G_{\text{smoker}} = 2 G.
\end{equation}

where \(G\) is the magnitude of the pressure gradient before the bad habits kicked in.  The smoker's poor little heart (soon to be a big overworked and weak heart) has to generate pressure gradients that are twice as big to get the same quantity of blood distributed through the body.
} % end answer

\makeoproblem{Simple shearing flow}{problem:fluids:simpleShearing}
{2012 problem set 2}
{
Consider steady simple shearing flow with no imposed pressure gradient \((G = 0)\) of a two layer fluid with viscosity

\begin{equation}\label{eqn:continuumProblemSet2:120}
\mu =
\left\{
\begin{array}{l l}
\mu^{(1)} & \quad \mbox{\(-h < y < 0,\)} \\
\mu^{(2)} & \quad \mbox{\(0 < y < h.\)}
\end{array}
\right.
\end{equation}

The boundary conditions are no-slip at the lower plate \((y = -h)\) and at \(y = 0\).  The top plate is moving with a velocity \(-U\) at \(y = h\) and fluid is sticking to it.  using the continuity of tangential (shear) stress at the interface (\(y = 0\))

\makesubproblem{Derive the velocity profile of the two fluids}{problem:fluids:simpleShearing1}
\makesubproblem{Calculate the maximum speed}{problem:fluids:simpleShearing2}
\makesubproblem{Calculate the mean speed}{problem:fluids:simpleShearing3}
\makesubproblem{Calculate the flux (the volume flow rate.)}{problem:fluids:simpleShearing4}
\makesubproblem{Calculate the tangential force (per unit width) \(F_x\) on the strip \(0 \le x \le L\) of the wall \(y = -h\)}{problem:fluids:simpleShearing5}
\makesubproblem{Calculate the tangential force (per unit width) \(F_x^0\) on the strip \(0 \le x \le L\) at the interface \(y = 0\) by the top fluid on the lower fluid}{problem:fluids:simpleShearing6}
} % makeoproblem

\makeanswer{problem:fluids:simpleShearing}{

\makesubanswer{Velocity profiles}{problem:fluids:simpleShearing1}
Starting with the velocity profile derivation for the two fluids, we set up coordinates as in \cref{fig:continuumProblemSet2:continuumProblemSet2Fig2}.  Our steady flow for layers \(1\) and \(2\) has the form

%\begin{figure}[htp]
%   \centering
%   \includegraphics[totalheight=0.3\textheight]{continuumProblemSet2Fig2}
%   \caption{Two layer flow induced by moving wall}\label{fig:continuumProblemSet2:continuumProblemSet2Fig2}
%\end{figure}

\pdfTexFigure{../../figures/phy454/continuumProblemSet2Fig2r2.pdf_tex}{Two layer flow induced by moving wall}{fig:continuumProblemSet2:continuumProblemSet2Fig2}{0.4}

\begin{subequations}
\begin{equation}\label{eqn:continuumProblemSet2:340}
0 = - \PD{x}{p} + \mu^{(i)} \PDSq{y}{u^{(i)}}
\end{equation}
\begin{equation}\label{eqn:continuumProblemSet2:360}
0 = - \PD{y}{p}
\end{equation}
\begin{equation}\label{eqn:continuumProblemSet2:380}
0 = - \PD{z}{p},
\end{equation}
\end{subequations}

as we found in Q1.  Only the boundary value conditions and the driving pressure are different here.  In this problem and the next, we have constant pressure gradients \(dp/dx = -G\) to deal with, so we really have just the pair of equations

\begin{equation}\label{eqn:continuumProblemSet2:400}
0 = G + \mu^{(i)} \frac{d^2 u^{(i)}(y)}{dy^2},
\end{equation}

to solve.  For this Q2 problem we have \(G = 0\), so the algebra to match our boundary value constraints becomes a bit easier.  Our boundary value constraints are

\begin{subequations}
\begin{equation}\label{eqn:continuumProblemSet2:420}
u^{(1)}(-h) = 0 \\
\end{equation}
\begin{equation}\label{eqn:continuumProblemSet2:440}
u^{(2)}(h) = -U \\
\end{equation}
\begin{equation}\label{eqn:continuumProblemSet2:460}
u^{(1)}(0) = u^{(2)}(0),
\end{equation}
\end{subequations}

plus one more to match the tangential components of the traction vector with respect to the normal \(\ncap = (0, 1, 0)\).  The components of that traction vector are

\begin{equation}\label{eqn:continuumProblemSet2:1020}
\begin{aligned}
t_i
&= \left( -p \delta_{ij} + 2 \mu e_{ij} \right) n_j \\
&= \left( -p \delta_{ij} + 2 \mu e_{ij} \right) \delta_{2j} \\
&= -p \delta_{i2} + 2 \mu e_{i2},
\end{aligned}
\end{equation}

but we are only interested in the horizontal component \(t_1\) which is

\begin{equation}\label{eqn:continuumProblemSet2:1040}
\begin{aligned}
t_1
&=
 -p \delta_{12} + 2 \mu e_{12} \\
&=
2 \mu \inv{2} \left(
\PD{u}{y}
+
\cancel{\PD{v}{x}}
\right).
\end{aligned}
\end{equation}

So the matching the tangential components of the traction vector at the interface gives us our last boundary value constraint

\begin{equation}\label{eqn:continuumProblemSet2:480}
\evalbar{\mu^{(1)} \PD{u}{y}}{y = 0} = \evalbar{\mu^{(2)} \PD{u}{y}}{y = 0},
\end{equation}

and we are ready to do our remaining bits of algebra.  We wish to solve the pair of equations

\begin{equation}\label{eqn:continuumProblemSet2:500}
\begin{aligned}
u^{(1)} &= A^{(1)} y + B^{(1)} \\
u^{(2)} &= A^{(2)} y + B^{(2)},
\end{aligned}
\end{equation}

for the four integration constants \(A^{(i)}\) and \(B^{(i)}\) using our boundary value constraints.  The linear system to solve is

\begin{equation}\label{eqn:continuumProblemSet2:520}
\begin{aligned}
0 &= -A^{(1)} h + B^{(1)} \\
-U &= A^{(2)} h + B^{(2)} \\
B^{(1)} &= B^{(2)} \\
\mu^{(1)} A^{(1)} &= \mu^{(2)} A^{(2)}.
\end{aligned}
\end{equation}

With \(B = B^{(i)}\), we have

\begin{equation}\label{eqn:continuumProblemSet2:540}
\begin{aligned}
0 &= -A^{(1)} h + B \\
-U &= \frac{\mu^{(1)}}{\mu^{(2)}} A^{(1)} h + B
\end{aligned}
\end{equation}

Subtracting these to solve for \(A^{(1)}\) we find

\begin{equation}\label{eqn:continuumProblemSet2:560}
-U = h A^{(1)} \left( \frac{\mu^{(1)}}{\mu^{(2)}} + 1 \right).
\end{equation}

This gives us everything we need

\begin{equation}\label{eqn:continuumProblemSet2:580}
\begin{aligned}
A^{(1)} &= -\frac{U \mu^{(2)}}{h(\mu^{(1)} + \mu^{(2)})} \\
A^{(2)} &= -\frac{U \mu^{(1)}}{h(\mu^{(1)} + \mu^{(2)})} \\
B^{(1)} = B^{(2)} &= -\frac{U \mu^{(2)}}{\mu^{(1)} + \mu^{(2)}},
\end{aligned}
\end{equation}

Referring back to \eqnref{eqn:continuumProblemSet2:500} our velocities are
\boxedEquation{eqn:continuumProblemSet2:600}{
\begin{aligned}
%u^{(1)} &=
%-\frac{U }{h(\mu^{(1)} + \mu^{(2)})} \mu^{(2)} \left( y + h \right) \\
%u^{(2)} &=
%-\frac{U }{h(\mu^{(1)} + \mu^{(2)})} \left( \mu^{(1)} y + \mu^{(2)} h \right)
u^{(1)} &=
-\frac{U \mu^{(2)} }{(\mu^{(1)} + \mu^{(2)})} \left( 1 + \frac{y}{h} \right) \\
u^{(2)} &=
-\frac{U \mu^{(2)} }{(\mu^{(1)} + \mu^{(2)})} \left( 1 + \frac{\mu^{(1)}}{\mu^{(2)}} \frac{y}{h} \right)
\end{aligned}
}
%-\frac{U \mu^{(1)}}{h(\mu^{(1)} + \mu^{(2)})} y
%-\frac{U \mu^{(2)}}{h(\mu^{(1)} + \mu^{(2)})} h

Checking, we see at a glance we see that we have \(u^{(2)}(h) = -U\), \(u^{(1)}(-h) = 0\), \(u^{(1)}(0) = u^{(2)}(0)\), and \(\evalbar{\mu^{(1)} du^{(1)}/dy}{y=0} = \mu^{(2)} \evalbar{du^{(2)}/dy}{y=0}\) as desired.

As an example, let us add some numbers.  With mercury and water in layers \({(1)}\) and \({(2)}\) respectively, we have

\begin{equation}\label{eqn:continuumL16:620}
\begin{aligned}
\mu^{(1)} &= 0.001526 \quad \text{Pa-s} \\
\mu^{(2)} &= 0.00089 \quad \text{Pa-s}
\end{aligned}
\end{equation}

so that our velocity is

\begin{equation}\label{eqn:continuumProblemSet2:640}
u(y) =
\left\{
\begin{array}{l l}
-1.37 U \left(1 + \frac{y}{h}\right) & \quad \mbox{\(y \in [-h, 0]\)} \\
-1.37 U \left(1 + 1.7 \frac{y}{h}\right) & \quad \mbox{\(y \in [0, h]\)}
\end{array}
\right.
\end{equation}

This is plotted with \(h = U = 1\) in \cref{fig:continuumProblemSet2:continuumProblemSet2Fig3}
\imageFigure{../../figures/phy454/continuumProblemSet2Fig3}{Two layer shearing flow with water over mercury}{fig:continuumProblemSet2:continuumProblemSet2Fig3}{0.2}

\makesubanswer{Maximum speed}{problem:fluids:simpleShearing2}
We are now ready to calculate the maximum speed.

With \(u^{(1)}(-h) = 0\), and \(u^{(1)}\) linearly decreasing, then \(u^{(2)}\) linearly decreasing further from the value at \(y = 0\), it is clear that the maximum speed, no matter the viscosities of the fluids, is on the upper moving interface.  This maximum takes the value \(\Abs{u^{(2)}(h)} = U\).

\makesubanswer{Mean speed}{problem:fluids:simpleShearing3}
As linear functions the average speeds of the respective fluids fall on the midpoints \(y = \pm h/2\).  These are

\begin{equation}\label{eqn:continuumProblemSet2:660}
\begin{aligned}
\expectation{u^{(1)}} &= -\frac{U \mu^{(2)} }{ 2 (\mu^{(1)} + \mu^{(2)})} \\
\expectation{u^{(2)}} &= -\frac{U \mu^{(2)} }{(\mu^{(1)} + \mu^{(2)})} \left( 1 + \frac{\mu^{(1)}}{2 \mu^{(2)}} \right)
\end{aligned}
\end{equation}

Averaging these two gives us the overall average, so we find

\begin{equation}\label{eqn:continuumProblemSet2:680}
\expectation{u(y)} =
-\frac{U }{ 4 (\mu^{(1)} + \mu^{(2)})} \left( 3 \mu^{(2)} + \mu^{(1)} \right)
\end{equation}

\makesubanswer{Volume flux}{problem:fluids:simpleShearing4}
We can calculate the volume flux, much like the mass flux (although the mass flux seems a more sensible quantity to calculate).  Looking at the rate of change of an element of fluid passing through the \(y-z\) plane we have

\begin{equation}\label{eqn:continuumProblemSet2:700}
\frac{dV}{dt} = dy dz \Bu \cdot \xcap
\end{equation}

Integrating over the total height, for a width \(\Delta z\) we have

\begin{equation}\label{eqn:continuumProblemSet2:1060}
\begin{aligned}
\text{Volume Flux}
&= \Delta z \int_{-h}^h u(y) dy \\
&= \Delta z 2h \expectation{u} \\
\end{aligned}
\end{equation}

So our volume flux through a width \(\Delta z\) is

\begin{equation}\label{eqn:continuumProblemSet2:720}
\text{Volume Flux}
=
-\frac{2 h U \Delta z}{ 4 (\mu^{(1)} + \mu^{(2)})} \left( 3 \mu^{(2)} + \mu^{(1)} \right).
\end{equation}

\makesubanswer{Tangential force on lower wall}{problem:fluids:simpleShearing5}
We see from \eqnref{eqn:continuumProblemSet2:600} the tangential components of our traction vectors are

\begin{equation}\label{eqn:continuumProblemSet2:1080}
\begin{aligned}
t^{(1)}
&= \mu^{(1)} \frac{d u^{(1)}}{dy} \\
&= -\frac{U \mu^{(1)} \mu^{(2)} }{(\mu^{(1)} + \mu^{(2)})} \inv{h}
\end{aligned}
\end{equation}

and

\begin{equation}\label{eqn:continuumProblemSet2:1100}
\begin{aligned}
t^{(2)}
&= \mu^{(2)} \frac{d u^{(2)}}{dy} \\
&= -\frac{U \mu^{(1)} \mu^{(2)} }{(\mu^{(1)} + \mu^{(2)})} \inv{h}
\end{aligned}
\end{equation}

We see that the tangential component of the traction vector is a constant throughout both fluids.  Allowing this force to act on a length \(L\) of the lower wall, our force per unit width over that strip is just

\begin{equation}\label{eqn:continuumProblemSet2:740}
F = -\frac{U \mu^{(1)} \mu^{(2)} }{(\mu^{(1)} + \mu^{(2)})} \frac{L}{h}.
\end{equation}

The negative value here makes sense since it is acting to push the fluid backwards in the direction of the upper wall motion.

\makesubanswer{Tangential force on upper wall}{problem:fluids:simpleShearing6}
We note that due to constant nature of the tangential component of the traction vector shown above, the force per unit width of the upper fluid acting on the lower fluid, is also given by \eqnref{eqn:continuumProblemSet2:740}.
} % end answer

\makeoproblem{Simple shearing flow with constant pressure gradient}{problem:fluids:simpleShearingPressureGrad}
{2012 problem set 2}
{
If on top of the problem described above a constant pressure gradient \(G = -dp/dx\) is applied between the boundaries \(y = \pm h\), describe qualitatively what type of flow profile you would expect in the steady state.  Draw the velocity profiles for two cases (i) \(\mu^{(1)} > \mu^{(2)}\) (ii) \(\mu^{(1)} < \mu^{(2)}\).  Explain your result.
} % makeoproblem

\makeanswer{problem:fluids:simpleShearingPressureGrad}{
We showed earlier that the Navier-Stokes equations for this Q3 case, where \(G\) is non-zero were given by \eqnref{eqn:continuumProblemSet2:400}, which restated is

\begin{equation}\label{eqn:continuumProblemSet2:400b}
0 = G + \mu^{(i)} \frac{d^2 u^{(i)}(y)}{dy^2}.
\end{equation}

Our solutions will now necessarily be parabolic, of the form

\begin{equation}\label{eqn:continuumProblemSet2:760}
u^{(i)}(y) = -\frac{G}{2 \mu^{(i)}} y^2 + A^{(i)} y + B^{(i)},
\end{equation}

with the tangential traction vector components given by

\begin{equation}\label{eqn:continuumProblemSet2:780}
t^{(i)} = -G y + A^{(i)} \mu^{(i)}
\end{equation}

The boundary value constants become a bit messier to solve for, and should we wish to do so we would have to solve the system

\begin{equation}\label{eqn:continuumProblemSet2:800}
\begin{aligned}
0 &= -\frac{G}{2 \mu^{(1)}} h^2 - A^{(1)} h + B^{(1)} \\
-U &= -\frac{G}{2 \mu^{(2)}} h^2 + A^{(2)} h + B^{(2)} \\
B^{(1)} &= B^{(2)} \\
A^{(1)} \mu^{(1)} &= A^{(2)} \mu^{(2)}
\end{aligned}
\end{equation}

Without actually solving this system we should expect that our solution will have the form of our pure shear flow, with parabolas superimposed on these linear flows.  For a higher viscosity bottom layer \(\mu^{(1)} > \mu^{(2)}\), this should look something like \cref{fig:continuumProblemSet2:continuumProblemSet2Fig4} whereas for the higher viscosity on the top, these would be roughly flipped as in \cref{fig:continuumProblemSet2:continuumProblemSet2Fig5}.

\imageFigure{../../figures/phy454/continuumProblemSet2Fig4}{Superposition of constant pressure gradient and shear flow solutions (\(\mu^{(1)} > \mu^{(2)}\))}{fig:continuumProblemSet2:continuumProblemSet2Fig4}{0.3}

\imageFigure{../../figures/phy454/continuumProblemSet2Fig5}{Superposition of constant pressure gradient and shear flow solutions (\(\mu^{(1)} < \mu^{(2)}\))}{fig:continuumProblemSet2:continuumProblemSet2Fig5}{0.3}

This superposition can be justified since we have no \((\Bu \cdot \spacegrad)\Bu\) term in the Navier-Stokes equations for these systems.

%\unnumberedSubsection{Exact solutions}

The figures above are kind of rough.  It is not actually hard to solve the system above.  After some simplification, I find using Mathematica in (\nbref{problemSetIIQ3exactSolution.cdf}) the following solution

\begin{equation}\label{eqn:continuumProblemSet2:840}
\begin{aligned}
u^{(1)}(y) &= -\frac{\mu^{(2)} U -G h^2}{\mu^{(1)}+\mu^{(2)}}-\frac{y \left(G h^2 (\mu^{(2)} -\mu^{(1)}) +2 \mu^{(1)} \mu^{(2)} U \right)}{2 h \mu^{(1)} \left(\mu^{(1)}+\mu^{(2)}\right)}-\frac{G y^2}{2 \mu^{(1)}} \\
u^{(2)}(y) &= -\frac{\mu^{(2)} U -G h^2}{\mu^{(1)}+\mu^{(2)}}-\frac{y \left(G h^2 (\mu^{(2)} -\mu^{(1)}) +2 \mu^{(1)} \mu^{(2)} U \right)}{2 h \mu^{(2)} \left(\mu^{(1)}+\mu^{(2)}\right)}-\frac{G y^2}{2 \mu^{(2)}}.
\end{aligned}
\end{equation}

Should we wish a more exact plot for any specific values of the viscosities, we could plot exactly with software the vector field described by these velocities.

I suppose it is cheating to use Mathematica and then say that the solution is easy?  To make amends for being lazy with my algebra, let us show that it is easy to do manually too.  I will do the same problem manually, but generalize it slightly.  We can do this easily if we just be a bit smarter with our integration constants.  Let us solve the problem for the upper and lower walls moving with velocity \(V_2\) and \(V_1\) respectively, and let the heights from the interface be \(h_2\) and \(h_1\) respectively.

We have the same set of differential equations to solve, but now let us write our solution with the undetermined coefficients expressed as

\begin{equation}\label{eqn:continuumProblemSet2:860}
\begin{aligned}
u^{(2)} &= -\frac{G}{2 \mu^{(2)}}\left( h_2^2 - y^2 \right) + \frac{A_2}{\mu^{(2)}} (y - h_2) + B_2 \\
u^{(1)} &= -\frac{G}{2 \mu^{(1)}}\left( h_1^2 - y^2 \right) + \frac{A_1}{\mu^{(1)}} (y + h_1) + B_1.
\end{aligned}
\end{equation}

Now it is super easy to match the boundary conditions at \(y = -h_1\) and \(y = h_2\)(the lower and upper walls respectively).  Clearly the integration constants \(B_1,B_2\) are just the velocities.  Matching the tangential component of the traction vectors at \(y = 0\) we have

\begin{equation}\label{eqn:continuumProblemSet2:880}
A_2 = A_1
\end{equation}

and matching velocities at \(y = 0\) gives us

\begin{equation}\label{eqn:continuumProblemSet2:900}
-\frac{G}{2 \mu^{(2)}}h_2^2 - \frac{A_2}{\mu^{(2)}} h_2 + V_2 = -\frac{G}{2 \mu^{(1)}}h_1^2 + \frac{A_1}{\mu^{(1)}} h_1 + V_1.
\end{equation}

This gives us
\begin{equation}\label{eqn:continuumProblemSet2:920}
\begin{aligned}
u^{(2)} &= \frac{G h_2^2}{2 \mu^{(2)}}\left(\frac{y^2}{h_2^2} - 1 \right) + A \frac{ h_2}{\mu^{(2)}} \left( \frac{y}{h_2} - 1 \right) + V_2 \\
u^{(1)} &= \frac{G h_1^2}{2 \mu^{(1)}}\left(\frac{y^2}{h_1^2} - 1 \right) + A \frac{ h_1}{\mu^{(1)}} \left( \frac{y}{h_1} + 1 \right) + V_1 \\
A
&=
\frac{
V_2 - V_1
+
\frac{G h_1^2}{2 \mu^{(1)}}
-\frac{G h_2^2}{2 \mu^{(2)}}
}{
\frac{h_1}{\mu^{(1)}}
+\frac{h_2}{\mu^{(2)}}
}.
\end{aligned}
\end{equation}

Plotting this with sliders or animation in Mathematica (\nbref{problemSetIIQ3PlotWithManipulate.cdf}) is a fun way to explore visualizing this.  The results vary widely depending on the various parameters.  Here are animations with variation of the pressure gradient for \(v_1 = 0\), \(h_1 = h_2\), showing the superposition of the shear and channel flow solutions

\begin{itemize}
\item With \(\mu^{(1)} > \mu^{(2)}\).  See \youtubehref{2xVoFAL9XGA}.
%\cref{fig:continuumProblemSet2:continuumProblemSet2Animation1} (or

\item With \(\mu^{(2)} > \mu^{(1)}\).  See \youtubehref{FJekyGf6XJw}.
%\cref{fig:continuumProblemSet2:continuumProblemSet2Animation2} (or
\end{itemize}

%\movieFigure{./animatedTwoLayerFlowMu1GreaterThanMu2Try3.mp4}{Fluids with densities of mercury and water in the lower (1) and upper (2) layers respectively}{fig:continuumProblemSet2:continuumProblemSet2Animation1}{width=320pt,height=240pt}

%\movieFigure{./animatedTwoLayerFlowMu2GreaterThanMu1Try3.mp4}{Fluids with densities of water and mercury in the lower (1) and upper (2) layers respectively}{fig:continuumProblemSet2:continuumProblemSet2Animation2}{width=320pt,height=240pt}

\FIXME{re-draw the figures and adjust above in response to these points}
I lost a couple of marks on this assignment, all on the hand plotting.  The remarks were

\begin{enumerate}
\item Gradients are the wrong way around.
\item \(G\) is constant across the fluid.
\end{enumerate}

I am assuming that the comment about \(G\) being constant across the fluid means that the two humped velocity distribution I drew is not realistic.  Here is two actual plots using the above calculations.

%\cref{fig:continuumProblemSet2:continuumProblemSet2Fig6mu1Gtmu2}
\imageFigure{../../figures/phy454/continuumProblemSet2Fig6mu1Gtmu2}{\(\mu^{(1)} > \mu^{(2)}\)}{fig:continuumProblemSet2:continuumProblemSet2Fig6mu1Gtmu2}{0.3}

%\cref{fig:continuumProblemSet2:continuumProblemSet2Fig6mu2Gtmu1}
\imageFigure{../../figures/phy454/continuumProblemSet2Fig6mu2Gtmu1}{\(\mu^{(2)} > \mu^{(1)}\)}{fig:continuumProblemSet2:continuumProblemSet2Fig6mu2Gtmu1}{0.3}

It was not clear to me initially what the grader meant by the gradients were the wrong way around, but I see that too looking at the actual plots.  If you check out the Mathematica worksheet itself you will see that I do not have a pressure gradient slider, but a velocity \(V_{\text{pressure}}\) slider.  This was based on the fact that for a channel flow our average speed is proportional to the pressure gradient

\begin{equation}\label{eqn:continuumProblemSet2:940}
\expectation{u} = \frac{G h}{3 \mu},
\end{equation}

so in order to parameterize the pressure gradient in an intuitive sense I defined it as a weighted average

\begin{equation}\label{eqn:continuumProblemSet2:960}
G = 3 V_{\text{pressure}} \inv{2} \left( \frac{\mu^{(1)}}{h_1} + \frac{\mu^{(2)}}{h_2}\right).
\end{equation}

Sure enough when I set \(V_{\text{pressure}} > 0\) in the Mathematica slider (so that the pressure gradient is also positive) I get the channel flows pointing in the opposite direction as indicated in the grading comment.  I should have sketched this channel flow more carefully in the very simplest case first before doing the two layer flow.
} % end answer

   % 
% 
% 
% Copyright � 2012 Peeter Joot
% All Rights Reserved
% 
% This file may be reproduced and distributed in whole or in part, without fee, subject to the following conditions:
% 
% o The copyright notice above and this permission notice must be preserved complete on all complete or partial copies.
% 
% o Any translation or derived work must be approved by the author in writing before distribution.
% 
% o If you distribute this work in part, instructions for obtaining the complete version of this file must be included, and a means for obtaining a complete version provided.
% 
% 
% Exceptions to these rules may be granted for academic purposes: Write to the author and ask.
% 
% 
% 
\subsection{Midterm.  1. (b) constitutive relation, Newtonian fluids, and no-slip conditions.}

\begin{itemize}
\item In continuum mechanics what do you mean by \textit{constitutive relation}?
\paragraph{Answer.}  The constitutive relation is the stress-strain relation, generally

\begin{equation}\label{eqn:continuumMidTermReflection:150}
\sigma_{ij} = c_{abij} e_{ab}
\end{equation}

for isotropic solids we model this as 

\begin{equation}\label{eqn:continuumMidTermReflection:170}
\sigma_{ij} = \lambda e_{kk} \delta_{ij} + 2 \mu e_{ij}
\end{equation}

and for Newtonian fluids

\begin{equation}\label{eqn:continuumMidTermReflection:190}
\sigma_{ij} = -p \delta_{ij} + 2 \mu e_{ij}
\end{equation}

\item What is the definition of a Non-Newtonian fluid?
\paragraph{Answer.}

A non-Newtonian fluid would be one with a more general constitutive relationship.

\paragraph{Grading note.}  I lost a mark here.  I think the answer that was being looked for (as in \cite{wiki:newtonianFluids}) was that a Newtonian fluid is one with a linear stress strain relationship, and a non-Newtonian fluid would be one with a non-linear relationship.  According to \cite{wiki:nonNewtonianFluid} an example of a non-Newtonian material that we are all familiar with is Silly Putty.  This linearity is also how a Newtonian fluid was defined in the notes, but I didn't remember that (this isn't really something we use since we assume all fluids and materials are Newtonian in any calculations that we do).

\item What do you mean by \emph{no-slip} boundary condition at a fluid-fluid interface?

\paragraph{Exam time management note.} Somehow in my misguided attempt to be complete, I missed this question amongst the rest of my verbosity).

\paragraph{Answer.}
The no slip boundary condition is just one of velocity matching.  At a non-moving boundary, the no-slip condition means that we'll require the fluid to also have no velocity (ie. at that interface the fluid isn't slipping over the surface).  Between two fluids, this is a requirement that the velocities of both fluids match at that point (and all the rest of the points along the region of the interaction.)

\item Write down the continuity equation for an incompressible fluid.
\paragraph{Answer.}

An incompressible fluid has

\begin{equation}\label{eqn:continuumMidTermReflection:210}
\frac{d\rho}{dt} = 0,
\end{equation}

but since we also have

\begin{align*}
0 
&=
\frac{d\rho}{dt} \\
&= - \rho (\spacegrad \cdot \Bu)  \\
&= 
\PD{t}{\rho} + (\Bu \cdot \spacegrad) \rho \\
&= 0.
\end{align*}

A consequence is that $\spacegrad \cdot \Bu = 0$ for an incompressible fluid.  Let's recall where this statement comes from.  Looking at mass conservation, the rate that mass leaves a volume can be expressed as

\begin{align*}
\frac{dm}{dt}
&= \int \frac{d\rho}{dt} dV \\
&= -\int_{\partial V} \rho \Bu \cdot d\BA \\
&= -\int_V \spacegrad \cdot (\rho \Bu) dV
\end{align*}

(the minus sign here signifying that the mass is leaving the volume through the surface, and that we are using an outwards facing normal on the volume.)

If the surface bounding the volume doesn't change with time (ie. $\PDi{t}{V} = 0$) we can write

\begin{equation}\label{eqn:continuumMidTermReflection:230}
\PD{t}{} \int \rho dV = -\int \spacegrad \cdot (\rho \Bu) dV,
\end{equation}

or

\begin{equation}\label{eqn:continuumMidTermReflection:250}
0 = \int \left( \PD{t}{\rho} + \spacegrad \cdot (\rho \Bu) \right) dV,
\end{equation}

so that in differential form we have

\begin{equation}\label{eqn:continuumMidTermReflection:270}
0 = \PD{t}{\rho} + \spacegrad \cdot (\rho \Bu).
\end{equation}

Expanding the divergence by chain rule we have

\begin{equation}\label{eqn:continuumMidTermReflection:290}
\PD{t}{\rho} +\Bu \cdot \spacegrad \rho = -\rho \spacegrad \cdot \Bu,
\end{equation}

but this is just

\begin{equation}\label{eqn:continuumMidTermReflection:310}
\frac{d\rho}{dt} = -\rho \spacegrad \cdot \Bu.
\end{equation}

So, for an incompressible fluid (one for which $d\rho/dt =0$), we must also have $\spacegrad \cdot \Bu = 0$.

\end{itemize}

\subsection{Midterm.  Problem 2.}
\subsubsection{Statement.}

Consider steady simple shearing flow $\Bu = \xcap u(y)$ as shown in figure (\ref{fig:continuumMidtermReflection:continuumMidtermReflectionFigQ1}) with imposed constant pressure gradient ($G = -dp/dx$), $G$ being a positive number, of a single layer fluid with viscosity $\mu$.

\imageFigure{figures/continuumMidtermReflectionFigQ1}{Shearing flow with pressure gradient and one moving boundary.}{fig:continuumMidtermReflection:continuumMidtermReflectionFigQ1}{0.2}

The boundary conditions are no-slip at the lower plate ($y = h$).  The top plate is moving with a velocity $-U$ at $y = h$ and fluid is sticking to it, so $u(h) = -U$, $U$ being a positive number.  Using the Navier-Stokes equation.

\begin{itemize}
\item Derive the velocity profile of the fluid.

\paragraph{Answer}

Our equations of motion are

\begin{subequations}
\begin{equation}\label{eqn:continuumMidTermReflection:330}
0 = \spacegrad \cdot \Bu
\end{equation}
\begin{equation}\label{eqn:continuumMidTermReflection:350}
\cancel{\rho \PD{t}{\Bu}} + (\Bu \cdot \spacegrad) \Bu = - \spacegrad p + \mu \spacegrad (\cancel{\spacegrad \cdot \Bu}) + \mu \spacegrad^2 \Bu + \cancel{\rho \Bg}
\end{equation}
\end{subequations}

Here, we've used the steady state condition and are neglecting gravity, and kill off our mass compression term with the incompressibility assumption.  In component form, what we have left is

\begin{align}\label{eqn:continuumMidTermReflection:370}
0 &= \partial_x u \\
u \cancel{\partial_x u} &= -\partial_x p + \mu \spacegrad^2 u \\
0 &= -\partial_y p \\
0 &= -\partial_z p
\end{align}

with $\partial_y p = \partial_z p = 0$, we must have

\begin{equation}\label{eqn:continuumMidTermReflection:390}
\PD{x}{p} = \frac{dp}{dx} = -G,
\end{equation}

which leaves us with just

\begin{align*}
0 
&= G + \mu \spacegrad^2 u(y)  \\
&= G + \mu \PDSq{y}{u} \\
&= G + \mu \frac{d^2 u}{dy^2}.
\end{align*}

Having dropped the partials we really just want to integrate our very simple ODE a couple times

\begin{equation}\label{eqn:continuumMidTermReflection:430}
u'' = -\frac{G}{\mu}.
\end{equation}

Integrate once

\begin{equation}\label{eqn:continuumMidTermReflection:450}
u' = -\frac{G}{\mu} y + \frac{A}{h},
\end{equation}

and once more to find the velocity

\begin{equation}\label{eqn:continuumMidTermReflection:470}
u = -\frac{G}{2 \mu} y^2 + \frac{A}{h} y + B'.
\end{equation}

Let's incorporate an additional constant into $B'$

\begin{equation}\label{eqn:continuumMidTermReflection:490}
B' = \frac{G}{2 \mu} h^2 + B
\end{equation}

so that we have

\begin{equation}\label{eqn:continuumMidTermReflection:510}
u = \frac{G}{2 \mu} (h^2 - y^2) + \frac{A}{h} y + B.
\end{equation}

(I didn't do use $B'$ this way on the exam, nor did I include the factor of $1/h$ in the first integration constant, but both of these should simplify the algebra since we'll be evaluating the boundary value conditions at $y = \pm h$.)

\begin{equation}\label{eqn:continuumMidTermReflection:530}
u = \frac{G}{2 \mu} (h^2 - y^2) + \frac{A}{h} y + B
\end{equation}

Applying the velocity matching conditions we have for the lower and upper plates respectively

\begin{align}\label{eqn:continuumMidTermReflection:550}
0 &= \frac{A}{h} (-h) + B \\
-U &= \frac{A}{h} (h) + B
\end{align}

Adding these we find

\begin{equation}\label{eqn:continuumMidTermReflection:570}
B = -\frac{U}{2}
\end{equation}

and subtracting find

\begin{equation}\label{eqn:continuumMidTermReflection:590}
A = -\frac{U}{2}.
\end{equation}

Our velocity is

\begin{equation}\label{eqn:continuumMidTermReflection:610}
u = \frac{G}{2 \mu} (h^2 - y^2) - \frac{U}{2 h} y -\frac{U}{2}
\end{equation}

or rearranged a bit

\begin{equation}\label{eqn:continuumMidTermReflection:630}
\boxed{
u(y) = \frac{G}{2 \mu} (h^2 - y^2) - \frac{U}{2} \left( 1 + \frac{y}{h} \right)
}
\end{equation}

\item Draw the velocity profile with the direction of the flow of the fluid when $U = 0$, $G \ne 0$.

\paragraph{Answer}

With $U = 0$ our velocity has a simple parabolic profile with a max of $\frac{G}{2 \mu} (h^2 - y^2)$ at $y = 0$

\begin{equation}\label{eqn:continuumMidTermReflection:650}
u(y) = \frac{G}{2 \mu} (h^2 - y^2).
\end{equation}

This is plotted in figure (\ref{fig:continuumMidtermReflection:continuumMidtermReflectionFig3})
\imageFigure{figures/continuumMidtermReflectionFig3}{Parabolic velocity profile.}{fig:continuumMidtermReflection:continuumMidtermReflectionFig3}{0.2}

\item Draw the velocity profile with the direction of the flow of the fluid when $G = 0$, $U \ne 0$.

\paragraph{Answer}

With $G = 0$, we have a plain old shear flow

\begin{equation}\label{eqn:continuumMidTermReflection:670}
u(y) = - \frac{U}{2} \left( 1 + \frac{y}{h} \right).
\end{equation}

This is linear with minimum velocity $u = 0$ at $y = -h$, and a maximum of $-U$ at $y = h$.  This is plotted in figure (\ref{fig:continuumMidtermReflection:continuumMidtermReflectionFig4})
\imageFigure{figures/continuumMidtermReflectionFig4}{Shear flow.}{fig:continuumMidtermReflection:continuumMidtermReflectionFig4}{0.2}

\item Using linear superposition draw the velocity profile of the fluid with the direction of flow qualitatively when $U \ne 0$, $G \ne 0$. (i) low $U$, (ii) large $U$.
\paragraph{Exam time management note.} Somehow I missed this question when I wrote the exam ... I figured this out right at the end when I'd run out of time by being too verbose elsewhere.  I'm really not very good at writing exams in tight time constraints anymore.

\paragraph{Answer}

For low $U$ we'll let the parabolic dominate, and can graphically add these two as in figure (\ref{fig:continuumMidtermReflection:continuumMidtermReflectionFig5})
\imageFigure{figures/continuumMidtermReflectionFig5}{Superposition of shear and parabolic flow (low $U$)}{fig:continuumMidtermReflection:continuumMidtermReflectionFig5}{0.2}
For high $U$, we'll let the shear flow dominate, and have plotted this in figure (\ref{fig:continuumMidtermReflection:continuumMidtermReflectionFig6})
\imageFigure{figures/continuumMidtermReflectionFig6}{Superposition of shear and parabolic flow (high $U$)}{fig:continuumMidtermReflection:continuumMidtermReflectionFig6}{0.2}

\item Calculate the maximum speed when $U \ne 0$, $G \ne 0$.
\paragraph{Answer}

Since our acceleration is

\begin{equation}\label{eqn:continuumMidTermReflection:690}
\frac{du}{dy} = -\frac{G}{\mu} y - \frac{U}{2 h}
\end{equation}

our extreme values occur at 

\begin{equation}\label{eqn:continuumMidTermReflection:710}
y_m = -\frac{U \mu}{2 h G}.
\end{equation}

At this point, our velocity is

\begin{align*}
u(y_m) 
&= 
\frac{G}{2 \mu} \left(h^2 - 
\left( \frac{U \mu}{2 h G} \right)^2
\right) - \frac{U}{2} \left( 1 
-\frac{U \mu}{2 h^2 G}
\right) \\
&=
\frac{G h^2}{2 \mu} -\frac{U}{2}
+ \frac{U^2 \mu}{4 h^2 G} \left(
1 -\inv{2}
\right)
\end{align*}

or just

\begin{equation}\label{eqn:continuumMidTermReflection:730}
u_{\text{max}} = \frac{G h^2}{2 \mu} -\frac{U}{2} + \frac{U^2 \mu}{8 h^2 G}.
\end{equation}

\item Calculate the flux (the volume flow rate) when $U \ne 0$, $G \ne 0$.
\paragraph{Answer}

An element of our volume flux is

\begin{equation}\label{eqn:continuumMidTermReflection:750}
\frac{dV}{dt} = dy dz \Bu \cdot \xcap
\end{equation}

Looking at the volume flux through the width $\Delta z$ is then

\begin{align*}
\text{Flux} 
&= \int_0^{\Delta z} dz \int_{-h}^h dy u(y) \\
&= \Delta z 
\int_{-h}^h dy 
\frac{G}{2 \mu} (h^2 - y^2) - \frac{U}{2} \left( 1 + \frac{y}{h} \right) \\
&= \Delta z 
\int_{-h}^h dy 
\frac{G}{2 \mu} \left(h^2 y - \inv{3} y^3 \right) - \frac{U}{2} \left( y + \frac{y^2}{2 h} \right) \\
&= \Delta z 
\left( \frac{2 G h^3}{3 \mu} - U h \right)
\end{align*}

\item Calculate the mean speed when $U \ne 0$, $G \ne 0$.
\paragraph{Answer}

\paragraph{Exam time management note.}  I squandered too much time on other stuff and didn't get to this part of the problem (which was unfortunately worth a lot).  This is how I think it should have been answered.

We've done most of the work above, and just have to divide the flux by $2 h \Delta z$.  That is

\begin{equation}\label{eqn:continuumMidTermReflection:770}
\expectation{u} = \frac{G h^2}{3 \mu} - \frac{U}{2}.
\end{equation}

\item Calculate the tangential force (per unit width) $F_x$ on the strip $0 \le x \le L$ of the wall $y = -h$ when $U \ne 0$, $G \ne 0$.
\paragraph{Answer}

Our traction vector is

\begin{align*}
T_1 
&= \sigma_{1j} n_j \\
&= \left( -p \delta_{1j} + 2 \mu e_{1j} \right) \delta_{2j} \\
&= 2 \mu e_{12} \\
&= \mu \left( 
\PD{y}{u}
+
\cancel{\PD{x}{v}}
\right)
\end{align*}

So the $\xcap$ directed component of the traction vector is just

\begin{equation}\label{eqn:continuumMidTermReflection:790}
T_1 = \mu \PD{y}{u}.
\end{equation}

We've calculated that derivative above in \ref{eqn:continuumMidTermReflection:690}, so we have

\begin{align*}
T_1 
&= \mu \left( -\frac{G}{\mu} y - \frac{U}{2 h} \right) \\
&= - G y - \frac{U \mu}{2 h} 
\end{align*}

so at $y = -h$ we have

\begin{equation}\label{eqn:continuumMidTermReflection:810}
T_1(-h) = G h - \frac{U \mu}{2 h}.
\end{equation}

To see the contribution of this force on the lower wall over an interval of length $L$ we integrate, but this amounts to just multiplying by the length of the segment of the wall

\begin{equation}\label{eqn:continuumMidTermReflection:830}
\int_0^L T_1(-h) dx = \left( G h - \frac{U \mu}{2 h} \right) L.
\end{equation}

\end{itemize}

   % 
% 
% 
% Copyright � 2012 Peeter Joot
% All Rights Reserved
% 
% This file may be reproduced and distributed in whole or in part, without fee, subject to the following conditions:
% 
% o The copyright notice above and this permission notice must be preserved complete on all complete or partial copies.
% 
% o Any translation or derived work must be approved by the author in writing before distribution.
% 
% o If you distribute this work in part, instructions for obtaining the complete version of this file must be included, and a means for obtaining a complete version provided.
% 
% 
% Exceptions to these rules may be granted for academic purposes: Write to the author and ask.
% 
% 
% 
%\keywords{Navier-Stokes, PHY454H1S, velocity scaling, nondimensionalisation, boundary layers, Reynold's number, Froude's number, characteristic length, characteristic velocity} 

\begin{Exercise}[
title={Alternatives for velocity non-dimensionalisation},
label={problem:fluids:ps3}
]

In fluid convection problems one can make several choices for characteristic velocity scales.  Some choices are given below for example:

\begin{enumerate}
\item $U_1 = g \alpha d^2 \nabla T/\nu$
\item $U_2 = \nu/d$
\item $U_3 = \sqrt{ g \alpha d \nabla T }$
\item $U_4 = \kappa/ d$
\end{enumerate}

where $g$ is the acceleration due to gravity, $\alpha = (\PDi{T}{V})/V$ is the coefficient of volume expansion, $d$ length scale associated with the problem, $\nabla T$ is the applied temperature difference, $\nu$ is the kinematic viscosity and $\kappa$ is the thermal diffusivity.  

\label{problem:fluids:ps3:1}
\Question{Verify that each of the expressions above have units of velocity}
\label{problem:fluids:ps3:2}
\Question{Water convection at room temperature.
For pure liquid, say pure water at room temperature, one has the following estimates in cgs units:

\begin{align*}
\alpha &\sim 10^{-4} \\
\kappa &\sim 10^{-3} \\
\nu &\sim 10^{-2}
\end{align*}

For a $d \sim 1 \text{cm}$ layer depth and a ten degree temperature drop convective velocities have been experimentally measured of about $10^{-2}$.
% \text{cm}/\text{s}  
With $g \sim 10^{-3}$, calculate the values of $U_1$, $U_2$, $U_3$, and $U_4$.  Which ones of the characteristic velocities $(U_1$, $U_2$, $U_3, U_4)$ do you think are suitable for nondimensionalising the velocity in Navier-Stokes/Energy equation describing the water convection problem?
}

\label{problem:fluids:ps3:3}
\Question{For mantle convection, we have

\begin{align*}
\alpha &\sim 10^{-5} \\
\nu &\sim 10^{21} \\
\kappa &\sim 10^{-2} \\
d &\sim 10^8 \\
\nabla T &\sim 10^3,
\end{align*}

and the actual convective mantle velocity is $10^{-8}$.  Which of the characteristic velocities should we use to nondimensionalise Navier-Stokes/Energy equations describing mantle convection?
}
\end{Exercise}

\begin{Answer}[ref={problem:fluids:ps3}]
\ref{problem:fluids:ps3:1} Let's check each of the velocity expressions in turn.

\begin{enumerate}
\item For $U_1$:

Observing that 

\begin{equation}\label{eqn:continuumProblemSet3:10}
\left[ \PD{t}{\Bu} \right] = [ \nu \spacegrad^2 \Bu ]
\end{equation}

we must have

\begin{equation}\label{eqn:continuumProblemSet3:30}
[\nu] = \inv{[t] [\spacegrad^2]} = \inv{T} L^2
\end{equation}

We also find

\begin{equation}\label{eqn:continuumProblemSet3:50}
[\alpha] = \inv{[V]} \left[ \PD{T}{V} \right] = \inv{[K]},
\end{equation}

so that

\begin{equation}\label{eqn:continuumProblemSet3:70}
[U_1] = \frac{L}{T \cancel{T}} \cancel{\inv{K}} L^2 \cancel{K} \frac{\cancel{T}}{L^2} = \frac{L}{T}
\end{equation}

\item For $U_2$:

\begin{equation}\label{eqn:continuumProblemSet3:90}
[U_2] = \frac{L^2}{T} \inv{L} = \frac{L}{T}.
\end{equation}

\item For $U_3$

\begin{equation}\label{eqn:continuumProblemSet3:110}
[U_3] = \sqrt{ \frac{L}{T^2} \inv{K} L K } = \frac{L}{T}
\end{equation}

\item For $U_4$

According to \href{http://scienceworld.wolfram.com/physics/ThermalDiffusivity.html}{http://scienceworld.wolfram.com/physics/ThermalDiffusivity.html}, the thermal diffusivity is defined by

\begin{equation}\label{eqn:continuumProblemSet3:130}
\PD{t}{T} = \kappa \spacegrad^2 T
\end{equation}

so that 

\begin{equation}\label{eqn:continuumProblemSet3:150}
[\kappa] = \inv{[t][\spacegrad^2]} = \frac{L^2}{T}
\end{equation}

That gives us

\begin{equation}\label{eqn:continuumProblemSet3:170}
[U_4] = \frac{L^2}{T} \inv{L} = \frac{L}{T}.
\end{equation}
\end{enumerate}

%We've verified that all of these have dimensions of velocity.

\ref{problem:fluids:ps3:2} For water at room temperature, we have

\begin{align}\label{eqn:continuumProblemSet3:190}
U_1 &\sim 10^{-3} 10^{-4} (1)^2 10^1 \inv{10^{-2}} = 10^{-4} \\
U_2 &\sim 10^{-2}/1 = 10^{-2} \\
U_3 &\sim \sqrt{ 10^{-3} 10^{-4} (1) 10^1 } = 10^{-3} \\
U_4 &\sim 10^{-3}/1 = 10^{-3}
\end{align}

Use of $U_2 = \nu/d$ gives the closest match to the measured characteristic velocity of $10^{-2}$.

\ref{problem:fluids:ps3:3} For the mantle convection problem let's compute the characteristic velocities
\begin{align}\label{eqn:continuumProblemSet3:210}
U_1 &\sim \frac{10^{-3} 10^{-5} 10^{16} 10^3 }{10^{21}} = 10^{-10} \\
U_2 &\sim \frac{10^{21}}{10^8} = 10^{13} \\
U_3 &\sim \sqrt{ 10^{-3} 10^{-5} 10^8 10^3 } \sim 10^1 \\
U_4 &\sim \frac{10^{-2}}{10^8} = 10^{-10}
\end{align}

Both $U_1$ and $U_4$ come close to the actual convective mantle velocity of $10^{-8}$.  Use of $U_1$ to nondimensionalise is probably best, since it has more degrees of freedom, and includes the gravity term that is probably important for such large masses.

FIXME: check against posted solutions.
\end{Answer}

\begin{Exercise}[
title={Nondimensionalise N-S equation},
label={problem:fluids:ps3:q2}
]
\begin{equation}\label{eqn:continuumProblemSet3:230}
\rho \PD{t}{\Bu} + \rho (\Bu \cdot \spacegrad) \Bu = - \spacegrad p + \mu \spacegrad^2 \Bu + \rho g \zcap
\end{equation}

where $\zcap$ is the unit vector in the $z$ direction.  You may scale:

\begin{itemize}
\item velocity with the characteristic velocity $U$,
\item time with $R/U$, where $R$ is the characteristic length scale,
\item pressure with $\rho U^2$,
\end{itemize}

Reynolds number $\text{Re} = R U \rho/ \mu$ and Froude number $\text{Fr} = g R/U$.
\end{Exercise}

\begin{Answer}[ref={problem:fluids:ps3:q2}]
Let's start by dividing by $g \rho$, to make all terms (most obviously the $\zcap$ term) dimensionless.

\begin{equation}\label{eqn:continuumProblemSet3:250}
\inv{g} \PD{t}{\Bu} + \inv{g} (\Bu \cdot \spacegrad) \Bu = - \inv{g \rho} \spacegrad p + \frac{\mu}{g \rho} \spacegrad^2 \Bu + \zcap.
\end{equation}

Our suggested replacements are

\begin{align}\label{eqn:continuumProblemSet3:270}
\Bu &= U \Bu' \\
\PD{t}{} &= \frac{U}{R} \PD{t'}{} \\
p &= \rho U^2 p' \\
\spacegrad &= \inv{R} \spacegrad'.
\end{align}

Plugging these in we have

\begin{equation}\label{eqn:continuumProblemSet3:290}
\frac{U^2}{g R} \PD{t'}{\Bu'} + \frac{U^2}{g R} (\Bu' \cdot \spacegrad') \Bu' = - \frac{\cancel{\rho} U^2}{g \cancel{\rho} R} \spacegrad' p' + \frac{\mu U}{g \rho R^2} {\spacegrad'}^2 \Bu' + \zcap.
\end{equation}

Making a $\text{Fr} = gR/U$ replacement, using the Froude number, we have

\begin{equation}\label{eqn:continuumProblemSet3:310}
\frac{U}{\text{Fr}} \PD{t'}{\Bu'} + \frac{U}{\text{Fr}} (\Bu' \cdot \spacegrad') \Bu' = - \frac{U}{\text{Fr}} \spacegrad' p' + \frac{\mu }{\text{Fr} \rho R} {\spacegrad'}^2 \Bu' + \zcap.
\end{equation}

Scaling by $\text{Fr}/U$ we tidy things up a bit, and also allow for insertion of the Reynold's number

\begin{equation}\label{eqn:continuumProblemSet3:330}
\PD{t'}{\Bu'} + (\Bu' \cdot \spacegrad') \Bu' = - \spacegrad' p' + \frac{1}{\text{Re}} {\spacegrad'}^2 \Bu' + \frac{\text{Fr}}{U}\zcap.
\end{equation}

Observe that the dimensions of Froude's number is that of velocity

\begin{equation}\label{eqn:continuumProblemSet3:350}
[\text{Fr}] = [g] T = \frac{L}{T},
\end{equation}

so that the end result is dimensionless as desired.  We also see that Froude's number, characterizes the significance of the body force for fluid flow at the characteristic velocity.  This is consistent with \cite{wiki:froudeNumber} where it was stated that the Froude number is used to determine the resistance of a partially submerged object moving through water, and permits the comparison of objects of different sizes (complete with pictures of canoes of various sizes that Froude built for such study).
\end{Answer}

\begin{Exercise}[
title={velocity phase difference in oscillatory Stokes boundary layer problem.},
label={problem:fluids:ps3:q3}
]
In case of Stokes' boundary layer problem (see class note) calculate shear stress on the plate $y = 0$.   What is the phase difference between the velocity of the plate $U(t) = U_0 \cos\omega t$ and the shear stress on the plate?
\end{Exercise}

\begin{Answer}[ref={problem:fluids:ps3:q3}]
FIXME: review posted solutions.  I think he posted a $5 \pi/4$ result?

We found in class that the velocity of the fluid was given by

\begin{equation}\label{eqn:continuumProblemSet3:370}
u(y, t) = U_0 e^{-\lambda y} \cos(\lambda y - \omega t)
\end{equation}

where

\begin{equation}\label{eqn:continuumProblemSet3:390}
\lambda = \sqrt{\frac{\omega}{2 \nu}}
\end{equation}

Calculating our shear stress we find

\begin{align*}
\mu \PD{y}{u} 
&= U_0 \lambda \mu e^{-\lambda y}
\left(
-1
-
 \sin(\lambda y - \omega t)
\right)
\end{align*}

and on the plate ($y = 0$) this is just

\begin{equation}\label{eqn:continuumProblemSet3:410}
\evalbar{\mu \PD{y}{u}}{y = 0} = U_0 \lambda \mu (-1 + \sin(\omega t)).
\end{equation}

We've got a constant term, plus one that is sinusoidal.  Observing that 

\begin{align}\label{eqn:continuumProblemSet3:430}
\cos x &= \Real( e^{ix} )  \\
\sin x &= \Real( -i e^{ix} ) = \Real( e^{i (x - \pi/2)} ),
\end{align}

The phase difference between the non-constant portion of the shear stress at the plate, and the plate velocity $U(t) = U_0 \cos\omega t$ is just $-\pi/2$.  The shear stress at the plate lags the driving velocity by 90 degrees.
\end{Answer}

   %
% Copyright � 2012 Peeter Joot.  All Rights Reserved.
% Licenced as described in the file LICENSE under the root directory of this GIT repository.
%
\makeoproblem{Rayleigh number}{problemset4:r}
{2012 problem set 4}
{
What is the physical meaning of the Rayleigh number?
} % makeoproblem

\makeoproblem{Critical temperature}{problemset4:t}
{2012 problem set 4}
{
If \(R_c = \frac{27 \pi^4}{4}\) is the critical Rayleigh number for the onset of convection for water, what is the corresponding critical temperature difference between the top and bottom plates in a \(10 \text{cm}\) layer of fluid?
} % makeoproblem

\makeoproblem{Critical wavelength}{problemset4:wavelen}
{2012 problem set 4}
{
What is the dimensional value of the critical wavelength of the convection cells?
} % makeoproblem

\makeoproblem{Rigid plates}{problemset4:rigid}
{2012 problem set 4}
{
If instead of taking stress free boundary conditions at the top and bottom plates, if we consider both the plates 'rigid' (no-slip) how does the solution of \eqnref{eqn:continuumL22:970} change?
} % makeoproblem

\makeoproblem{Normal mode solutions}{problemset4:normal}
{2012 problem set 4}
{
Consider the problem

\begin{equation}\label{eqn:problemSet4:10}
\PD{t}{u} - \sin u = \inv{R} \PDSq{y}{u}
\end{equation}

where \(R\) is a real parameter and the boundary conditions are given by

\begin{equation}\label{eqn:problemSet4:20}
u(y = 0, t) = u(y = \pi, t) = 0
\end{equation}

for all time \(t\). Examine the trivial base state \(u = u_B(y) = 0\) by seeking normal mode solutions to the linearized perturbed equations. Find the eigenfunctions and eigenvalues and show that the base state is linearly stable only if \(R \le 1\).
} % makeoproblem


   %\part{More worked problems}
   %
% Copyright � 2015 Peeter Joot.  All Rights Reserved.
% Licenced as described in the file LICENSE under the root directory of this GIT repository.
%
\documentclass[]{eliblog}

\usepackage{amsmath}
\usepackage{mathpazo}

%
% shorthand for bold symbols, convenient for vectors and matrices
%
\newcommand{\Ba}[0]{\mathbf{a}}
\newcommand{\Bb}[0]{\mathbf{b}}
\newcommand{\Bc}[0]{\mathbf{c}}
\newcommand{\Bd}[0]{\mathbf{d}}
\newcommand{\Be}[0]{\mathbf{e}}
\newcommand{\Bf}[0]{\mathbf{f}}
\newcommand{\Bg}[0]{\mathbf{g}}
\newcommand{\Bh}[0]{\mathbf{h}}
\newcommand{\Bi}[0]{\mathbf{i}}
\newcommand{\Bj}[0]{\mathbf{j}}
\newcommand{\Bk}[0]{\mathbf{k}}
\newcommand{\Bl}[0]{\mathbf{l}}
\newcommand{\Bm}[0]{\mathbf{m}}
\newcommand{\Bn}[0]{\mathbf{n}}
\newcommand{\Bo}[0]{\mathbf{o}}
\newcommand{\Bp}[0]{\mathbf{p}}
\newcommand{\Bq}[0]{\mathbf{q}}
\newcommand{\Br}[0]{\mathbf{r}}
\newcommand{\Bs}[0]{\mathbf{s}}
\newcommand{\Bt}[0]{\mathbf{t}}
\newcommand{\Bu}[0]{\mathbf{u}}
\newcommand{\Bv}[0]{\mathbf{v}}
\newcommand{\Bw}[0]{\mathbf{w}}
\newcommand{\Bx}[0]{\mathbf{x}}
\newcommand{\By}[0]{\mathbf{y}}
\newcommand{\Bz}[0]{\mathbf{z}}
\newcommand{\BA}[0]{\mathbf{A}}
\newcommand{\BB}[0]{\mathbf{B}}
\newcommand{\BC}[0]{\mathbf{C}}
\newcommand{\BD}[0]{\mathbf{D}}
\newcommand{\BE}[0]{\mathbf{E}}
\newcommand{\BF}[0]{\mathbf{F}}
\newcommand{\BG}[0]{\mathbf{G}}
\newcommand{\BH}[0]{\mathbf{H}}
\newcommand{\BI}[0]{\mathbf{I}}
\newcommand{\BJ}[0]{\mathbf{J}}
\newcommand{\BK}[0]{\mathbf{K}}
\newcommand{\BL}[0]{\mathbf{L}}
\newcommand{\BM}[0]{\mathbf{M}}
\newcommand{\BN}[0]{\mathbf{N}}
\newcommand{\BO}[0]{\mathbf{O}}
\newcommand{\BP}[0]{\mathbf{P}}
\newcommand{\BQ}[0]{\mathbf{Q}}
\newcommand{\BR}[0]{\mathbf{R}}
\newcommand{\BS}[0]{\mathbf{S}}
\newcommand{\BT}[0]{\mathbf{T}}
\newcommand{\BU}[0]{\mathbf{U}}
\newcommand{\BV}[0]{\mathbf{V}}
\newcommand{\BW}[0]{\mathbf{W}}
\newcommand{\BX}[0]{\mathbf{X}}
\newcommand{\BY}[0]{\mathbf{Y}}
\newcommand{\BZ}[0]{\mathbf{Z}}

\newcommand{\Bzero}[0]{\mathbf{0}}
\newcommand{\Btheta}[0]{\boldsymbol{\theta}}
\newcommand{\Btau}[0]{\boldsymbol{\tau}}
\newcommand{\Bomega}[0]{\boldsymbol{\omega}}

%
% shorthand for unit vectors
%
\newcommand{\acap}[0]{\hat{\Ba}}
\newcommand{\bcap}[0]{\hat{\Bb}}
\newcommand{\ccap}[0]{\hat{\Bc}}
\newcommand{\dcap}[0]{\hat{\Bd}}
\newcommand{\ecap}[0]{\hat{\Be}}
\newcommand{\fcap}[0]{\hat{\Bf}}
\newcommand{\gcap}[0]{\hat{\Bg}}
\newcommand{\hcap}[0]{\hat{\Bh}}
\newcommand{\icap}[0]{\hat{\Bi}}
\newcommand{\jcap}[0]{\hat{\Bj}}
\newcommand{\kcap}[0]{\hat{\Bk}}
\newcommand{\lcap}[0]{\hat{\Bl}}
\newcommand{\mcap}[0]{\hat{\Bm}}
\newcommand{\ncap}[0]{\hat{\Bn}}
\newcommand{\ocap}[0]{\hat{\Bo}}
\newcommand{\pcap}[0]{\hat{\Bp}}
\newcommand{\qcap}[0]{\hat{\Bq}}
\newcommand{\rcap}[0]{\hat{\Br}}
\newcommand{\scap}[0]{\hat{\Bs}}
\newcommand{\tcap}[0]{\hat{\Bt}}
\newcommand{\ucap}[0]{\hat{\Bu}}
\newcommand{\vcap}[0]{\hat{\Bv}}
\newcommand{\wcap}[0]{\hat{\Bw}}
\newcommand{\xcap}[0]{\hat{\Bx}}
\newcommand{\ycap}[0]{\hat{\By}}
\newcommand{\zcap}[0]{\hat{\Bz}}
\newcommand{\thetacap}[0]{\hat{\Btheta}}

%
% to write R^n and C^n in a distinguishable fashion.  Perhaps change this
% to the double lined characters upon figuring out how to do so.
%
\newcommand{\C}[1]{$\mathbb{C}^{#1}$}
\newcommand{\R}[1]{$\mathbb{R}^{#1}$}

%
% various generally useful helpers
%

% derivative of #1 wrt. #2:
\newcommand{\D}[2] {\frac {d#2} {d#1}}

\newcommand{\inv}[1]{\frac{1}{#1}}
\newcommand{\cross}[0]{\times}

\newcommand{\abs}[1]{\lvert{#1}\rvert}
\newcommand{\norm}[1]{\lVert{#1}\rVert}
\newcommand{\innerprod}[2]{\langle{#1}, {#2}\rangle}
\newcommand{\dotprod}[2]{{#1} \cdot {#2}}
\newcommand{\bdotprod}[2]{\left({#1} \cdot {#2}\right)}
\newcommand{\crossprod}[2]{{#1} \cross {#2}}
\newcommand{\tripleprod}[3]{\dotprod{\left(\crossprod{#1}{#2}\right)}{#3}}

\DeclareMathOperator{\Proj}{Proj}
\DeclareMathOperator{\Span}{span}
\DeclareMathOperator{\Sgn}{sgn}
\DeclareMathOperator{\Area}{Area}
\DeclareMathOperator{\Volume}{Volume}

%
% A few miscellaneous things specific to this document
%
\newcommand{\crossop}[1]{\crossprod{#1}{}}

% R2 vector.
\newcommand{\VectorTwo}[2]{
\begin{bmatrix}
 {#1} \\
 {#2}
\end{bmatrix}
}

\newcommand{\VectorN}[1]{
\begin{bmatrix}
{#1}_1 \\
{#1}_2 \\
\vdots \\
{#1}_N \\
\end{bmatrix}
}

\newcommand{\DETuvij}[4]{
\begin{vmatrix}
 {#1}_{#3} & {#1}_{#4} \\
 {#2}_{#3} & {#2}_{#4}
\end{vmatrix}
}

\newcommand{\DETuvwijk}[6]{
\begin{vmatrix}
 {#1}_{#4} & {#1}_{#5} & {#1}_{#6} \\
 {#2}_{#4} & {#2}_{#5} & {#2}_{#6} \\
 {#3}_{#4} & {#3}_{#5} & {#3}_{#6}
\end{vmatrix}
}

\newcommand{\DETuvwxijkl}[8]{
\begin{vmatrix}
 {#1}_{#5} & {#1}_{#6} & {#1}_{#7} & {#1}_{#8} \\
 {#2}_{#5} & {#2}_{#6} & {#2}_{#7} & {#2}_{#8} \\
 {#3}_{#5} & {#3}_{#6} & {#3}_{#7} & {#3}_{#8} \\
 {#4}_{#5} & {#4}_{#6} & {#4}_{#7} & {#4}_{#8} \\
\end{vmatrix}
}

%\newcommand{\DETuvwxyijklm}[10]{
%\begin{vmatrix}
% {#1}_{#6} & {#1}_{#7} & {#1}_{#8} & {#1}_{#9} & {#1}_{#10} \\
% {#2}_{#6} & {#2}_{#7} & {#2}_{#8} & {#2}_{#9} & {#2}_{#10} \\
% {#3}_{#6} & {#3}_{#7} & {#3}_{#8} & {#3}_{#9} & {#3}_{#10} \\
% {#4}_{#6} & {#4}_{#7} & {#4}_{#8} & {#4}_{#9} & {#4}_{#10} \\
% {#5}_{#6} & {#5}_{#7} & {#5}_{#8} & {#5}_{#9} & {#5}_{#10}
%\end{vmatrix}
%}

% R3 vector.
\newcommand{\VectorThree}[3]{
\begin{bmatrix}
 {#1} \\
 {#2} \\
 {#3}
\end{bmatrix}
}



\author{Peeter Joot}
\email{peeter.joot@gmail.com}

%\documentclass[]{eliblogwidescreen}

\usepackage{amsmath}
\usepackage{mathpazo}

%
% shorthand for bold symbols, convenient for vectors and matrices
%
\newcommand{\Ba}[0]{\mathbf{a}}
\newcommand{\Bb}[0]{\mathbf{b}}
\newcommand{\Bc}[0]{\mathbf{c}}
\newcommand{\Bd}[0]{\mathbf{d}}
\newcommand{\Be}[0]{\mathbf{e}}
\newcommand{\Bf}[0]{\mathbf{f}}
\newcommand{\Bg}[0]{\mathbf{g}}
\newcommand{\Bh}[0]{\mathbf{h}}
\newcommand{\Bi}[0]{\mathbf{i}}
\newcommand{\Bj}[0]{\mathbf{j}}
\newcommand{\Bk}[0]{\mathbf{k}}
\newcommand{\Bl}[0]{\mathbf{l}}
\newcommand{\Bm}[0]{\mathbf{m}}
\newcommand{\Bn}[0]{\mathbf{n}}
\newcommand{\Bo}[0]{\mathbf{o}}
\newcommand{\Bp}[0]{\mathbf{p}}
\newcommand{\Bq}[0]{\mathbf{q}}
\newcommand{\Br}[0]{\mathbf{r}}
\newcommand{\Bs}[0]{\mathbf{s}}
\newcommand{\Bt}[0]{\mathbf{t}}
\newcommand{\Bu}[0]{\mathbf{u}}
\newcommand{\Bv}[0]{\mathbf{v}}
\newcommand{\Bw}[0]{\mathbf{w}}
\newcommand{\Bx}[0]{\mathbf{x}}
\newcommand{\By}[0]{\mathbf{y}}
\newcommand{\Bz}[0]{\mathbf{z}}
\newcommand{\BA}[0]{\mathbf{A}}
\newcommand{\BB}[0]{\mathbf{B}}
\newcommand{\BC}[0]{\mathbf{C}}
\newcommand{\BD}[0]{\mathbf{D}}
\newcommand{\BE}[0]{\mathbf{E}}
\newcommand{\BF}[0]{\mathbf{F}}
\newcommand{\BG}[0]{\mathbf{G}}
\newcommand{\BH}[0]{\mathbf{H}}
\newcommand{\BI}[0]{\mathbf{I}}
\newcommand{\BJ}[0]{\mathbf{J}}
\newcommand{\BK}[0]{\mathbf{K}}
\newcommand{\BL}[0]{\mathbf{L}}
\newcommand{\BM}[0]{\mathbf{M}}
\newcommand{\BN}[0]{\mathbf{N}}
\newcommand{\BO}[0]{\mathbf{O}}
\newcommand{\BP}[0]{\mathbf{P}}
\newcommand{\BQ}[0]{\mathbf{Q}}
\newcommand{\BR}[0]{\mathbf{R}}
\newcommand{\BS}[0]{\mathbf{S}}
\newcommand{\BT}[0]{\mathbf{T}}
\newcommand{\BU}[0]{\mathbf{U}}
\newcommand{\BV}[0]{\mathbf{V}}
\newcommand{\BW}[0]{\mathbf{W}}
\newcommand{\BX}[0]{\mathbf{X}}
\newcommand{\BY}[0]{\mathbf{Y}}
\newcommand{\BZ}[0]{\mathbf{Z}}

\newcommand{\Bzero}[0]{\mathbf{0}}
\newcommand{\Btheta}[0]{\boldsymbol{\theta}}
\newcommand{\Btau}[0]{\boldsymbol{\tau}}
\newcommand{\Bomega}[0]{\boldsymbol{\omega}}

%
% shorthand for unit vectors
%
\newcommand{\acap}[0]{\hat{\Ba}}
\newcommand{\bcap}[0]{\hat{\Bb}}
\newcommand{\ccap}[0]{\hat{\Bc}}
\newcommand{\dcap}[0]{\hat{\Bd}}
\newcommand{\ecap}[0]{\hat{\Be}}
\newcommand{\fcap}[0]{\hat{\Bf}}
\newcommand{\gcap}[0]{\hat{\Bg}}
\newcommand{\hcap}[0]{\hat{\Bh}}
\newcommand{\icap}[0]{\hat{\Bi}}
\newcommand{\jcap}[0]{\hat{\Bj}}
\newcommand{\kcap}[0]{\hat{\Bk}}
\newcommand{\lcap}[0]{\hat{\Bl}}
\newcommand{\mcap}[0]{\hat{\Bm}}
\newcommand{\ncap}[0]{\hat{\Bn}}
\newcommand{\ocap}[0]{\hat{\Bo}}
\newcommand{\pcap}[0]{\hat{\Bp}}
\newcommand{\qcap}[0]{\hat{\Bq}}
\newcommand{\rcap}[0]{\hat{\Br}}
\newcommand{\scap}[0]{\hat{\Bs}}
\newcommand{\tcap}[0]{\hat{\Bt}}
\newcommand{\ucap}[0]{\hat{\Bu}}
\newcommand{\vcap}[0]{\hat{\Bv}}
\newcommand{\wcap}[0]{\hat{\Bw}}
\newcommand{\xcap}[0]{\hat{\Bx}}
\newcommand{\ycap}[0]{\hat{\By}}
\newcommand{\zcap}[0]{\hat{\Bz}}
\newcommand{\thetacap}[0]{\hat{\Btheta}}

%
% to write R^n and C^n in a distinguishable fashion.  Perhaps change this
% to the double lined characters upon figuring out how to do so.
%
\newcommand{\C}[1]{$\mathbb{C}^{#1}$}
\newcommand{\R}[1]{$\mathbb{R}^{#1}$}

%
% various generally useful helpers
%

% derivative of #1 wrt. #2:
\newcommand{\D}[2] {\frac {d#2} {d#1}}

\newcommand{\inv}[1]{\frac{1}{#1}}
\newcommand{\cross}[0]{\times}

\newcommand{\abs}[1]{\lvert{#1}\rvert}
\newcommand{\norm}[1]{\lVert{#1}\rVert}
\newcommand{\innerprod}[2]{\langle{#1}, {#2}\rangle}
\newcommand{\dotprod}[2]{{#1} \cdot {#2}}
\newcommand{\bdotprod}[2]{\left({#1} \cdot {#2}\right)}
\newcommand{\crossprod}[2]{{#1} \cross {#2}}
\newcommand{\tripleprod}[3]{\dotprod{\left(\crossprod{#1}{#2}\right)}{#3}}

\DeclareMathOperator{\Proj}{Proj}
\DeclareMathOperator{\Span}{span}
\DeclareMathOperator{\Sgn}{sgn}
\DeclareMathOperator{\Area}{Area}
\DeclareMathOperator{\Volume}{Volume}

%
% A few miscellaneous things specific to this document
%
\newcommand{\crossop}[1]{\crossprod{#1}{}}

% R2 vector.
\newcommand{\VectorTwo}[2]{
\begin{bmatrix}
 {#1} \\
 {#2}
\end{bmatrix}
}

\newcommand{\VectorN}[1]{
\begin{bmatrix}
{#1}_1 \\
{#1}_2 \\
\vdots \\
{#1}_N \\
\end{bmatrix}
}

\newcommand{\DETuvij}[4]{
\begin{vmatrix}
 {#1}_{#3} & {#1}_{#4} \\
 {#2}_{#3} & {#2}_{#4}
\end{vmatrix}
}

\newcommand{\DETuvwijk}[6]{
\begin{vmatrix}
 {#1}_{#4} & {#1}_{#5} & {#1}_{#6} \\
 {#2}_{#4} & {#2}_{#5} & {#2}_{#6} \\
 {#3}_{#4} & {#3}_{#5} & {#3}_{#6}
\end{vmatrix}
}

\newcommand{\DETuvwxijkl}[8]{
\begin{vmatrix}
 {#1}_{#5} & {#1}_{#6} & {#1}_{#7} & {#1}_{#8} \\
 {#2}_{#5} & {#2}_{#6} & {#2}_{#7} & {#2}_{#8} \\
 {#3}_{#5} & {#3}_{#6} & {#3}_{#7} & {#3}_{#8} \\
 {#4}_{#5} & {#4}_{#6} & {#4}_{#7} & {#4}_{#8} \\
\end{vmatrix}
}

%\newcommand{\DETuvwxyijklm}[10]{
%\begin{vmatrix}
% {#1}_{#6} & {#1}_{#7} & {#1}_{#8} & {#1}_{#9} & {#1}_{#10} \\
% {#2}_{#6} & {#2}_{#7} & {#2}_{#8} & {#2}_{#9} & {#2}_{#10} \\
% {#3}_{#6} & {#3}_{#7} & {#3}_{#8} & {#3}_{#9} & {#3}_{#10} \\
% {#4}_{#6} & {#4}_{#7} & {#4}_{#8} & {#4}_{#9} & {#4}_{#10} \\
% {#5}_{#6} & {#5}_{#7} & {#5}_{#8} & {#5}_{#9} & {#5}_{#10}
%\end{vmatrix}
%}

% R3 vector.
\newcommand{\VectorThree}[3]{
\begin{bmatrix}
 {#1} \\
 {#2} \\
 {#3}
\end{bmatrix}
}



\author{Peeter Joot}
\email{peeter.joot@gmail.com}


%\newcommand{\br1}[0]{(1)}
%\newcommand{\br2}[0]{(2)}
%\newcommand{\bri}[0]{(i)}
%s/\\br\(.\)/{(\1)}/g

\chapter{Steady state inclined flow down a plane of two layers of incompressible viscous equal density fluids.}
\label{chap:twoLayerInclinedFlow}
%\useCCL
\blogpage{http://sites.google.com/site/peeterjoot2/math2012/twoLayerInclinedFlow.pdf}
\date{Mar 1, 2012}
\gitRevisionInfo{twoLayerInclinedFlow}

\beginArtWithToc
%\beginArtNoToc

\section{Motivation.}

Here's one of the problems from \S 2 of \cite{acheson1990elementary}, a slight variation on what we did in class.  It seemed to me that it would be a good midterm prep problem to try.

\section{Steady state inclined flow down a plane of two layers of incompressible viscous equal density fluids.}

\subsection{Setup of the equations of motion for this system.}

Our problem is illustrated in 

FIXME: F4

with a plane set at angle $\alpha$, fluid depths of $h^{(1)}$ and $h^{(2)}$ respectively, and viscosities $\mu^{(1)}$ and $\mu^{(2)}$.  We'll write $H = h^{(1)} + h^{(2)}$.  We have a pair of Navier-Stokes equations to solve

\begin{equation}\label{eqn:twoLayerInclinedFlow:10}
\rho \ddt{\Bu^{(i)}} = \rho \PD{t}{\Bu^{(i)}} + \rho (\Bu^{(i)} \cdot \spacegrad) \Bu^{(i)} = - \spacegrad p^{(i)} + \mu^{(i)} \spacegrad^2 \Bu^{(i)} + \mu^{(i)} \spacegrad (\spacegrad \cdot \Bu^{(i)}) + \rho \Bg
\end{equation}

Our steady state and incompressiblity constraints break this into a few independent equations

\begin{align}\label{eqn:twoLayerInclinedFlow:30}
\rho \PD{t}{\Bu^{(i)}} &= 0 \\
\spacegrad \cdot \Bu^{(i)} &= 0 \\
\rho (\Bu^{(i)} \cdot \spacegrad) \Bu^{(i)} &= - \spacegrad p^{(i)} + \mu^{(i)} \spacegrad^2 \Bu^{(i)} + \rho \Bg
\end{align}

Let's require that we initially have no components of the flows in the $y$, or $z$ directions.  As the equivalent of Newton's law for fluid flows, conservation of linear momentum requires that for our steady state problem we have $u_y = u_z = 0$ for the flows in both fluid layers.  

FIXME: should go back and think about momentum conservation in the context of fluids.  I'm now so used to thinking about this as a symmetry issue from a Lagrangian context.  With no Lagrangian here, what would be a robust way treat this in fluids?  What is the Lagrangian for a fluid mechanics system?  I'd imagine that it would be possible to set up a field Lagrangian with velocity fields.

Our problem is now reduced to a problem in four quanties (two velocities and two pressures).  With $\Bu^{(i)} = (u^{(i)}, 0, 0)$ we can restate Navier-Stokes in coordinate form as

\begin{subequations}
\label{eqn:twoLayerInclinedFlow:50}
\begin{equation}\label{eqn:twoLayerInclinedFlow:110}
\partial_x u^{(i)}(x,y,z) = 0
\end{equation}
\begin{equation}\label{eqn:twoLayerInclinedFlow:130}
\rho (u^{(i)} \partial_x u^{(i)} = - \partial_x p^{(i)} + \mu^{(i)} (\partial_{xx} + \partial_{yy} + \partial_{zz}) u^{(i)} + \rho g \sin\alpha 
\end{equation}
\begin{equation}\label{eqn:twoLayerInclinedFlow:150}
0 = - \partial_y p^{(i)} - \rho g \cos\alpha 
\end{equation}
\begin{equation}\label{eqn:twoLayerInclinedFlow:155}
0 = - \partial_z p^{(i)} 
\end{equation}
\end{subequations}

In order to solve this, we have eight simultaneous non-linear PDEs, four unknown functions, plus boundary conditions to consider!

What are the boundary conditions?  One is the ``no-slip'' condition, the experimental observation that velocites match at the interfaces.  So we should have zero velocity for the fluid lying against the plane, and velocity matching between the fluids.  The air above the fluid will also be flowing along at the rate of the uppermost portion of the top layer, but we'll neglect that effect (i.e. considering two layers of equal density and not three, with one having a separate density).  We also have matching of the traction vectors at the interfaces.  

Writing this, it occured to me that I didn't fully understand what motivated that last boundary value condition.  Talking to our Prof about this, the matching of the traction vectors at any point can be thought of as an observational issue, but this is also a force balance issue.  There is an induced velocity in the direction of the traction vector at any given point.  For example, when we have unidirectional flow, we must have no normal component of the traction vector, and only a tangential component, because we have only the tangential flow.

Before continuing to solve our Navier-Stokes equations let's express the condition that the tangential component of the traction vectors match algebraically.

Dropping indexes temporarily, for the normal to the surface $\ncap = (n_1, n_2, n_3) = (0, 1, 0)$ we want to compute

\begin{align*}
T_1 
&= \sigma_{1 k} n_k \\
&= \sigma_{1 k} \delta_{2 k} \\
&= \sigma_{1 2} \\
&= -p \cancel{\delta_{1 2}} + 2 \mu e^{1 2} \\
&= \mu \left( \PD{y}{u_1} + \cancel{\PD{x}{u_2}} \right) 
\end{align*}

So the tangential component of the traction vector is

\begin{equation}\label{eqn:twoLayerInclinedFlow:90}
\BT^{(i)} = \mu^{(i)} \PD{y}{u^{(i)}} \xcap.
\end{equation}

As noted above, this is in fact, the only component of the traction vector, since we don't have any non-horizontal flow.

Our boundary value conditions, what we need in addition to the Navier-Stokes equations of \ref{eqn:twoLayerInclinedFlow:50}, to solve our problem, are the matching at any interface of the following conditions

\begin{subequations}
\label{eqn:twoLayerInclinedFlow:70}
\begin{equation}\label{eqn:twoLayerInclinedFlow:n}
u^{(i)} = u^{(j)} 
\end{equation}
\begin{equation}\label{eqn:twoLayerInclinedFlow:n}
p^{(i)} = p^{(j)} 
\end{equation}
\begin{equation}\label{eqn:twoLayerInclinedFlow:n}
\mu^{(i)} \PD{y}{u^{(i)}} = \mu^{(i)} \PD{y}{u^{(j)}}.
\end{equation}
\end{subequations}

There are actually three interfaces to consider, that of the lower layer liquid with the inclined plane, the interface between the two fluid layers, and the interface between the upper layer fluid and the air above it.

\subsection{Solving our equations of motion.}

Starting with the simplest, the z-coordinate equation, of Navier-Stokes \ref{eqn:twoLayerInclinedFlow:155}, we can conclude that each of the pressures is not a function of z, so that we have

\begin{equation}\label{eqn:twoLayerInclinedFlow:n}
p^{(i)} = p^{(i)}(x, y).
\end{equation}

Using this, we can integrate our y-coordinate Navier-Stokes equation \ref{eqn:twoLayerInclinedFlow:150}, to find

\begin{equation}\label{eqn:twoLayerInclinedFlow:n}
p^{(i)} = - \rho g y \cos\alpha + f^{(i)}(x)
\end{equation}

\EndArticle

   %
% Copyright � 2012 Peeter Joot.  All Rights Reserved.
% Licenced as described in the file LICENSE under the root directory of this GIT repository.
%

%
%

%s/\\br\(.\)/{(\1)}/g

\makeproblem{Two layer inclined viscous flow}{problem:fluids:twoLayerInclinedFlowDifferentDensities}
{
Here is a generalization of one of the problems from \S 2 of \citep{acheson1990elementary}, itself a slight variation on what we did in class.

In the previous calculation we did the calculation for two incompressible fluids of the same densities flowing down an inclined plane.  Now, let us generalize this slightly, allowing for different densities.

I am curious how much the air in the neighborhood of some flowing water gets dragged by that flow.  It never occurred to me that this would occur, and I had like to plug in some numbers and see what the results are.  This should be something that can be modeled with two layers like this, one of fluid, one of air of a specific thickness (allowing pressure to vary due to the velocity gradient), and one final layer of air at a fixed pressure (atmospheric).  I would not expect that problem to be much harder than this one, although it may end up being worthwhile to let a computer algebra system do some of the grunt work to solve all the resulting equations.

In the end, when we get to putting in some numbers for this problem, we can probably also get an idea how deep the region where the air gets dragged by the fluid can get.
} % makeproblem

\makeanswer{problem:fluids:twoLayerInclinedFlowDifferentDensities}{
%\unnumberedSubsection{Setup of the equations of motion for this system}
Our problem is illustrated in \cref{fig:twoLayerInclinedFlow:twoLayerInclinedFlowFig1} with a plane set at angle \(\alpha\), fluid depths of \(h^{(1)}\) and \(h^{(2)}\) respectively, and viscosities \(\mu^{(1)}\) and \(\mu^{(2)}\).  We will write \(H = h^{(1)} + h^{(2)}\).  We have a pair of Navier-Stokes equations to solve

%\imageFigure{../../figures/phy454/twoLayerInclinedFlowFig1}{Two fluids layers in inclined flow}{fig:twoLayerInclinedFlowDifferentDensities:twoLayerInclinedFlowDifferentDensitiesFig1}{0.2}

\begin{equation}\label{eqn:twoLayerInclinedFlowDifferentDensities:10}
\rho^{(i)} \ddt{\Bu^{(i)}} = \rho^{(i)} \PD{t}{\Bu^{(i)}} + \rho^{(i)} (\Bu^{(i)} \cdot \spacegrad) \Bu^{(i)} = - \spacegrad p^{(i)} + \mu^{(i)} \spacegrad^2 \Bu^{(i)} + \mu^{(i)} \spacegrad (\spacegrad \cdot \Bu^{(i)}) + \rho^{(i)} \Bg
\end{equation}

Our steady state and incompressibility constraints break this into a few independent equations

\begin{equation}\label{eqn:twoLayerInclinedFlowDifferentDensities:30}
\begin{aligned}
\rho^{(i)} \PD{t}{\Bu^{(i)}} &= 0 \\
\spacegrad \cdot \Bu^{(i)} &= 0 \\
\rho^{(i)} (\Bu^{(i)} \cdot \spacegrad) \Bu^{(i)} &= - \spacegrad p^{(i)} + \mu^{(i)} \spacegrad^2 \Bu^{(i)} + \rho^{(i)} \Bg
\end{aligned}
\end{equation}

Let us require no components of the flows in the \(y\), or \(z\) directions initially.  As the equivalent of Newton's law for fluid flows, conservation of linear momentum requires that for our steady state problem we have \(u_y = u_z = 0\) for the flows in both fluid layers.

\FIXME{I am now so used to thinking about this as a symmetry issue from a Lagrangian context.  With no Lagrangian here, what would be a robust way treat this in fluids?  What is the Lagrangian for a fluid mechanics system?  I had imagine that it would be possible to set up a field Lagrangian with velocity fields}

Our problem is now reduced to a problem in four quantities (two velocities and two pressures).  With \(\Bu^{(i)} = (u^{(i)}, 0, 0)\) we can restate Navier-Stokes in coordinate form as

\begin{subequations}
\label{eqn:twoLayerInclinedFlowDifferentDensities:50}
\begin{equation}\label{eqn:twoLayerInclinedFlowDifferentDensities:110}
\partial_x u^{(i)}(x,y,z) = 0
\end{equation}
\begin{equation}\label{eqn:twoLayerInclinedFlowDifferentDensities:130}
\rho^{(i)} (u^{(i)} \partial_x u^{(i)} = - \partial_x p^{(i)} + \mu^{(i)} (\partial_{xx} + \partial_{yy} + \partial_{zz}) u^{(i)} + \rho^{(i)} g \sin\alpha
\end{equation}
\begin{equation}\label{eqn:twoLayerInclinedFlowDifferentDensities:150}
0 = - \partial_y p^{(i)} - \rho^{(i)} g \cos\alpha
\end{equation}
\begin{equation}\label{eqn:twoLayerInclinedFlowDifferentDensities:155}
0 = - \partial_z p^{(i)}
\end{equation}
\end{subequations}

In order to solve this, we have eight simultaneous non-linear PDEs, four unknown functions, plus boundary conditions!

What are the boundary conditions?  One is the ``no-slip'' condition, the experimental observation that velocities match at the interfaces.  So we should have zero velocity for the fluid lying against the plane, and velocity matching between the fluids.  The air above the fluid will also be flowing along at the rate of the uppermost portion of the top layer, but we will neglect that effect (i.e. considering two layers of equal density and not three, with one having a separate density).  We also have matching of the traction vectors at the interfaces.

Writing this, it occurred to me that I did not fully understand what motivated the traction vector matching boundary value condition.  Talking to our Prof about this, the matching of the traction vectors at any point can be thought of as an observational issue, but this is also a force balance issue.  There is an induced velocity in the direction of the traction vector at any given point.  For example, when we have unidirectional flow, we must have no normal component of the traction vector, and only a tangential component, because we have only the tangential flow.  It is probably reasonable to think about this roughly as the equivalent of matching both acceleration and velocity at the boundary, but because densities and viscosities vary, we have to match the traction vectors and not the acceleration itself.

Before continuing to solve our Navier-Stokes equations let us express the condition that the tangential component of the traction vectors match algebraically.

Dropping indices temporarily, for the normal to the surface \(\ncap = (n_1, n_2, n_3) = (0, 1, 0)\) we want to compute

\begin{equation}\label{eqn:twoLayerInclinedFlowDifferentDensities:595}
\begin{aligned}
\tau_1
&= \sigma_{1 k} n_k \\
&= \sigma_{1 k} \delta_{2 k} \\
&= \sigma_{1 2} \\
&= -p \cancel{\delta_{1 2}} + 2 \mu e^{1 2} \\
&= \mu \left( \PD{y}{u_1} + \cancel{\PD{x}{u_2}} \right)
\end{aligned}
\end{equation}

So the tangential component of the traction vector is

\begin{equation}\label{eqn:twoLayerInclinedFlowDifferentDensities:90}
\Btau^{(i)} = \mu^{(i)} \PD{y}{u^{(i)}} \xcap.
\end{equation}

As noted above, this is in fact, the only component of the traction vector, since we do not have any non-horizontal flow.

Our boundary value conditions, what we need in addition to the Navier-Stokes equations of \eqnref{eqn:twoLayerInclinedFlowDifferentDensities:50}, to solve our problem, are the matching at any interface of the following conditions

\begin{subequations}
\label{eqn:twoLayerInclinedFlowDifferentDensities:70}
\begin{equation}\label{eqn:twoLayerInclinedFlowDifferentDensities:175}
u^{(i)} = u^{(j)}
\end{equation}
\begin{equation}\label{eqn:twoLayerInclinedFlowDifferentDensities:195}
p^{(i)} = p^{(j)}
\end{equation}
\begin{equation}\label{eqn:twoLayerInclinedFlowDifferentDensities:215}
\mu^{(i)} \PD{y}{u^{(i)}} = \mu^{(i)} \PD{y}{u^{(j)}}.
\end{equation}
\end{subequations}

There are actually three interfaces to consider, that of the lower layer liquid with the inclined plane, the interface between the two fluid layers, and the interface between the upper layer fluid and the air above it.

%\unnumberedSubsection{Solving our equations of motion}
Starting with the simplest, the z-coordinate equation, of Navier-Stokes \eqnref{eqn:twoLayerInclinedFlowDifferentDensities:155}, we can conclude that each of the pressures is not a function of z, so that we have

\begin{equation}\label{eqn:twoLayerInclinedFlowDifferentDensities:235}
p^{(i)} = p^{(i)}(x, y).
\end{equation}

Using this, we can integrate our y-coordinate Navier-Stokes equation \eqnref{eqn:twoLayerInclinedFlowDifferentDensities:150}, to find

\begin{equation}\label{eqn:twoLayerInclinedFlowDifferentDensities:255}
p^{(i)} = - \rho^{(i)} g y \cos\alpha + f^{(i)}(x)
\end{equation}

At this point we can introduce the first boundary value constraint, that the pressures must match at the interfaces.  In particular, on the upper surface, where we have atmospheric pressure \(p_A\) our pressure is

\begin{equation}\label{eqn:twoLayerInclinedFlowDifferentDensities:275}
p^{(2)}(H) = - \rho^{(2)} g H \cos\alpha + f^{(2)}(x) = p_A,
\end{equation}

so \(f^{(2)}\) is constant with value

\begin{equation}\label{eqn:twoLayerInclinedFlowDifferentDensities:295}
f^{(2)}(x) = p_A + \rho^{(2)} g H \cos\alpha,
\end{equation}

which fully determines the density of the upper surface

\begin{equation}\label{eqn:twoLayerInclinedFlowDifferentDensities:315}
p^{(2)}(y) = \rho^{(2)} g \cos\alpha (H - y) + p_A.
\end{equation}

Matching the pressure between the two layers of fluids we have

\begin{equation}\label{eqn:twoLayerInclinedFlowDifferentDensities:615}
\begin{aligned}
p^{(1)}(h_1) &= - \rho^{(1)} g h_1 \cos\alpha + f^{(1)}(x) \\
             &= p^{(2)}(h_1) \\
             &= \rho^{(2)} g \cos\alpha (H - h_1) + p_A \\
             &= \rho^{(2)} g h_2 \cos\alpha + p_A,
\end{aligned}
\end{equation}

so that our undetermined function \(f^{(1)}(x)\)

\begin{equation}\label{eqn:twoLayerInclinedFlowDifferentDensities:335}
f^{(1)}(x) = \left(\rho^{(1)} h_1 + \rho^{(2)} h_2 \right) g \cos\alpha  + p_A.
\end{equation}

With the densities not equal, we no longer find that the pressure is dependent only on the total height \(y\), independent of the velocities and viscosities

\begin{equation}\label{eqn:twoLayerInclinedFlowDifferentDensities:355}
p^{(1)}(y) =
g \cos\alpha
\left(
\rho^{(1)} (h_1 - y) + \rho^{(2)} h_2
\right)
+ p_A.
\end{equation}

However, this is still a fairly satisfying result.  The pressure on the bottom layer is the total pressure due to the layer above it (the contribution due to the total height \(h_2\) of that layer of the fluid).  To that we add the pressure at our specific height, a linear function of the difference from the interface above it.  Specified piecewise our pressure is now fully determined

\boxedEquation{eqn:twoLayerInclinedFlowDifferentDensities:356}{
p(y) =
\left\{
\begin{array}{l l}
g \cos\alpha
\left(
\rho^{(1)} (h_1 - y) + \rho^{(2)} h_2
\right)
+ p_A
&
\quad \mbox{\(y < h_1\)} \\
\rho^{(2)} g \cos\alpha (H - y) + p_A.
&
\quad \mbox{\(y \in [h_1, h_1 + h_2]\)} \\
p_A
&
\quad \mbox{\(y > H\)}
\end{array}
\right.
}

Observe that we have the usual \(\rho g h\) form in all the terms of the pressure above, just scaled by the cosine of the angle since only a portion of the gravitational force is pushing normally on the fluids.

Having solved for the pressure, we are now set to return to the remaining Navier-Stokes equations \eqnref{eqn:twoLayerInclinedFlowDifferentDensities:110}, and \eqnref{eqn:twoLayerInclinedFlowDifferentDensities:130} for this system.  From \eqnref{eqn:twoLayerInclinedFlowDifferentDensities:110} we see that the non-linear term on the LHS of \eqnref{eqn:twoLayerInclinedFlowDifferentDensities:130} is killed and also see that our velocities can only be functions of \(y\) and \(z\)

\begin{equation}\label{eqn:twoLayerInclinedFlowDifferentDensities:375}
u^{(i)} = u^{(i)}(y, z).
\end{equation}

While more general solutions can likely be found, we will limit ourselves to looking only for solutions that are functions of \(y\).  From our solution to the pressure part of the problem \(p^{(i)} = p^{(i)}(y)\), we also see that the pressure term \(\partial_x p^{(i)}\) of \eqnref{eqn:twoLayerInclinedFlowDifferentDensities:130} is killed.  We are left with just

\begin{equation}\label{eqn:twoLayerInclinedFlowDifferentDensities:635}
\begin{aligned}
0 &= \mu^{(i)} (\partial_{xx} + \partial_{yy} + \partial_{zz}) u^{(i)} + \rho^{(i)} g \sin\alpha  \\
  &= \mu^{(i)} \partial_{yy} u^{(i)} + \rho^{(i)} g \sin\alpha \\
  &= \mu^{(i)} \frac{d^2}{dy^2} u^{(i)} + \rho^{(i)} g \sin\alpha
\end{aligned}
\end{equation}

This is directly integrable, and we find for the velocities and traction vectors respectively

\begin{subequations}
\begin{equation}\label{eqn:twoLayerInclinedFlowDifferentDensities:395}
u^{(i)} = -\frac{\rho^{(i)} g \sin\alpha }{2 \mu^{(i)}} y^2 + A^{(i)} y + B^{(i)}.
\end{equation}
\begin{equation}\label{eqn:twoLayerInclinedFlowDifferentDensities:415}
\tau_x^{(i)} = \mu^{(i)} \PD{y}{u^{(i)}} = - \rho^{(i)} g y \sin\alpha + \mu^{(i)} A^{(i)}
\end{equation}
\end{subequations}

The boundary conditions left to exploit are

\begin{equation}\label{eqn:twoLayerInclinedFlowDifferentDensities:435}
\begin{aligned}
u^{(1)}(0) &= 0 \\
u^{(1)}(h_1) &= u^{(2)}(h_1) \\
\tau_x^{(1)}(h_1) &= \tau_x^{(2)}(h_1) \\
\tau_x^{(2)}(H) &= 0,
\end{aligned}
\end{equation}

The first is the no-slip condition with the plane.  The last is an approximation that assumes the liquid is not producing a measurable force on the air above it.  The other two are for the interfaces between the two fluids.

From \(u^{(1)}(0) = 0\) we see immediately that we have \(B^{(1)} = 0\).  From the traction vector equality in the atmosphere, we have

\begin{equation}\label{eqn:twoLayerInclinedFlowDifferentDensities:455}
0 = - \rho^{(2)} g H \sin\alpha + \mu^{(2)} A^{(2)},
\end{equation}

or

\begin{equation}\label{eqn:twoLayerInclinedFlowDifferentDensities:475}
A^{(2)} = \frac{\rho^{(2)} g H \sin\alpha }{ \mu^{(2)} }.
\end{equation}

These reduce the problem to solving for two last integration constants, where our velocities are

\begin{equation}\label{eqn:twoLayerInclinedFlowDifferentDensities:495}
\begin{aligned}
u^{(1)} &= -\frac{\rho^{(1)} g \sin\alpha }{2 \mu^{(1)}} y^2 + A^{(1)} y \\
u^{(2)} &= \frac{\rho^{(2)} g \sin\alpha }{2 \mu^{(2)}} \left( 2 H y -y^2 \right) + B^{(2)}.
\end{aligned}
\end{equation}

and our traction vectors are

\begin{equation}\label{eqn:twoLayerInclinedFlowDifferentDensities:515}
\begin{aligned}
\tau_x^{(1)} &= - \rho^{(1)} g y \sin\alpha + \mu^{(1)} A^{(1)} \\
\tau_x^{(2)} &= \rho^{(2)} g \sin\alpha \left( H - y \right).
\end{aligned}
\end{equation}

Matching both at the interface (\(y = h_1\)) gives us

\begin{equation}\label{eqn:twoLayerInclinedFlowDifferentDensities:535}
\begin{aligned}
-\frac{\rho^{(1)} g \sin\alpha }{2 \mu^{(1)}} h_1^2 + A^{(1)} h_1 &= \frac{\rho^{(2)} g \sin\alpha }{2 \mu^{(2)}} h_1 \left( 2 h_2 + h_1 \right) + B^{(2)} \\
- \rho^{(1)} g h_1 \sin\alpha + \mu^{(1)} A^{(1)} &= \rho^{(2)} g h_2 \sin\alpha
\end{aligned}
\end{equation}

We find

\begin{equation}\label{eqn:twoLayerInclinedFlowDifferentDensities:555}
A^{(1)} =
\inv{ \mu^{(1)} }
(\rho^{(1)} h_1 + \rho^{(2)} h_2 )
g \sin\alpha
\end{equation}

Let us substitute this back for our first fluid velocity
\begin{equation}\label{eqn:twoLayerInclinedFlowDifferentDensities:655}
\begin{aligned}
u^{(1)}
&=
-\frac{\rho^{(1)} g \sin\alpha }{2 \mu^{(1)}} y^2 +
\frac{y}{ \mu^{(1)} }
(\rho^{(1)} h_1 + \rho^{(2)} h_2 )
g \sin\alpha \\
&=
g \sin\alpha \left(
-\frac{\rho^{(1)} }{2 \mu^{(1)}} y^2 +
\frac{y}{ \mu^{(1)} }
(\rho^{(1)} h_1 + \rho^{(2)} h_2 )
\right) \\
&=
\frac{g y \sin\alpha}{2 \mu^{(1)}} \left( \rho^{(1) } (2 h_1 - y) +2 \rho^{(2)} h_2 \right) \\
\end{aligned}
\end{equation}

As a check we see this is consistent with the previous calculation when \(\rho^{(1)} = \rho^{(2)}\).  For our final integration constant we now find

\begin{equation}\label{eqn:twoLayerInclinedFlowDifferentDensities:675}
\begin{aligned}
B^{(2)}
&=
\frac{g h_1 \sin\alpha}{2 \mu^{(1)}} \left( \rho^{(1) } h_1 +2 \rho^{(2)} h_2 \right)
-\frac{\rho^{(2)} g \sin\alpha }{2 \mu^{(2)}} h_1 \left( 2 h_2 + h_1 \right) \\
&=
\frac{g h_1 \sin\alpha}{2}
\left(
\inv{\mu^{(1)}} \left( \rho^{(1) } h_1 +2 \rho^{(2)} h_2 \right)
-\frac{\rho^{(2)} }{\mu^{(2)}} \left( 2 h_2 + h_1 \right)
\right) \\
&=
\frac{g h_1 \sin\alpha}{2 \mu^{(2)}}
\left(
\frac{\mu^{(2)}}{\mu^{(1)}} \left( \rho^{(1) } h_1 +2 \rho^{(2)} h_2 \right)
- \rho^{(2)} \left( 2 h_2 + h_1 \right)
\right) \\
\end{aligned}
\end{equation}

So, finally, we have for the velocities

\begin{subequations}
\begin{equation}\label{eqn:twoLayerInclinedFlowDifferentDensities:575}
u^{(1)}(y) = \frac{g y \sin\alpha}{2 \mu^{(1)}} \left( \rho^{(1) } (2 h_1 - y) +2 \rho^{(2)} h_2 \right)
\end{equation}
\begin{equation}\label{eqn:twoLayerInclinedFlowDifferentDensities:575b}
u^{(2)}(y) =
\frac{g \sin\alpha }{2 \mu^{(2)}}
\left(
\rho^{(2)} \left( 2 H y -y^2 \right) +
h_1 \left(
\frac{\mu^{(2)}}{\mu^{(1)}} \left( \rho^{(1) } h_1 +2 \rho^{(2)} h_2 \right)
- \rho^{(2)} \left( 2 h_2 + h_1 \right)
\right)
\right)
\end{equation}
\end{subequations}

The final result looks reasonable.  If the viscosities and densities are equal then we have the same velocity profile in both layers.  That makes sense given the equal densities, since there would really be nothing that would then distinguish the two layers.

%\unnumberedSubsection{Numerical application}
To try this out numerically see (\nbref{twoLayerInclinedFlowDifferentDensities.cdf}).

The results are fairly surprising.  Specifically, insertion of an air layer above the water ends up with the air speed humongous!  Steady state not realistic?  What are the length scales required for steady state?  Are these so large that we would have to vary the gravitational field?

I think that this shows either an error in this calculation, an error programming the worksheet, or the folly of even considering a steady state flow of this form for anything that is not extremely viscous.

Some further validation is required to see what is up.  One part of that validation is now done.  To rule out algebraic errors above see the verification in (\nbref{twoLayerInclinedFlowDifferentDensitiesTheCalculation.cdf}).  Using Solve to find \(A^{(i)}\), and \(B^{(i)}\) I get exactly the same answers as with my hand calculations above.  I also verified that substituting back in the boundary value conditions yields the expected equalities.
} % end answer

   %%
% Copyright � 2015 Peeter Joot.  All Rights Reserved.
% Licenced as described in the file LICENSE under the root directory of this GIT repository.
%
\documentclass[]{eliblog}

\usepackage{amsmath}
\usepackage{mathpazo}

%
% shorthand for bold symbols, convenient for vectors and matrices
%
\newcommand{\Ba}[0]{\mathbf{a}}
\newcommand{\Bb}[0]{\mathbf{b}}
\newcommand{\Bc}[0]{\mathbf{c}}
\newcommand{\Bd}[0]{\mathbf{d}}
\newcommand{\Be}[0]{\mathbf{e}}
\newcommand{\Bf}[0]{\mathbf{f}}
\newcommand{\Bg}[0]{\mathbf{g}}
\newcommand{\Bh}[0]{\mathbf{h}}
\newcommand{\Bi}[0]{\mathbf{i}}
\newcommand{\Bj}[0]{\mathbf{j}}
\newcommand{\Bk}[0]{\mathbf{k}}
\newcommand{\Bl}[0]{\mathbf{l}}
\newcommand{\Bm}[0]{\mathbf{m}}
\newcommand{\Bn}[0]{\mathbf{n}}
\newcommand{\Bo}[0]{\mathbf{o}}
\newcommand{\Bp}[0]{\mathbf{p}}
\newcommand{\Bq}[0]{\mathbf{q}}
\newcommand{\Br}[0]{\mathbf{r}}
\newcommand{\Bs}[0]{\mathbf{s}}
\newcommand{\Bt}[0]{\mathbf{t}}
\newcommand{\Bu}[0]{\mathbf{u}}
\newcommand{\Bv}[0]{\mathbf{v}}
\newcommand{\Bw}[0]{\mathbf{w}}
\newcommand{\Bx}[0]{\mathbf{x}}
\newcommand{\By}[0]{\mathbf{y}}
\newcommand{\Bz}[0]{\mathbf{z}}
\newcommand{\BA}[0]{\mathbf{A}}
\newcommand{\BB}[0]{\mathbf{B}}
\newcommand{\BC}[0]{\mathbf{C}}
\newcommand{\BD}[0]{\mathbf{D}}
\newcommand{\BE}[0]{\mathbf{E}}
\newcommand{\BF}[0]{\mathbf{F}}
\newcommand{\BG}[0]{\mathbf{G}}
\newcommand{\BH}[0]{\mathbf{H}}
\newcommand{\BI}[0]{\mathbf{I}}
\newcommand{\BJ}[0]{\mathbf{J}}
\newcommand{\BK}[0]{\mathbf{K}}
\newcommand{\BL}[0]{\mathbf{L}}
\newcommand{\BM}[0]{\mathbf{M}}
\newcommand{\BN}[0]{\mathbf{N}}
\newcommand{\BO}[0]{\mathbf{O}}
\newcommand{\BP}[0]{\mathbf{P}}
\newcommand{\BQ}[0]{\mathbf{Q}}
\newcommand{\BR}[0]{\mathbf{R}}
\newcommand{\BS}[0]{\mathbf{S}}
\newcommand{\BT}[0]{\mathbf{T}}
\newcommand{\BU}[0]{\mathbf{U}}
\newcommand{\BV}[0]{\mathbf{V}}
\newcommand{\BW}[0]{\mathbf{W}}
\newcommand{\BX}[0]{\mathbf{X}}
\newcommand{\BY}[0]{\mathbf{Y}}
\newcommand{\BZ}[0]{\mathbf{Z}}

\newcommand{\Bzero}[0]{\mathbf{0}}
\newcommand{\Btheta}[0]{\boldsymbol{\theta}}
\newcommand{\Btau}[0]{\boldsymbol{\tau}}
\newcommand{\Bomega}[0]{\boldsymbol{\omega}}

%
% shorthand for unit vectors
%
\newcommand{\acap}[0]{\hat{\Ba}}
\newcommand{\bcap}[0]{\hat{\Bb}}
\newcommand{\ccap}[0]{\hat{\Bc}}
\newcommand{\dcap}[0]{\hat{\Bd}}
\newcommand{\ecap}[0]{\hat{\Be}}
\newcommand{\fcap}[0]{\hat{\Bf}}
\newcommand{\gcap}[0]{\hat{\Bg}}
\newcommand{\hcap}[0]{\hat{\Bh}}
\newcommand{\icap}[0]{\hat{\Bi}}
\newcommand{\jcap}[0]{\hat{\Bj}}
\newcommand{\kcap}[0]{\hat{\Bk}}
\newcommand{\lcap}[0]{\hat{\Bl}}
\newcommand{\mcap}[0]{\hat{\Bm}}
\newcommand{\ncap}[0]{\hat{\Bn}}
\newcommand{\ocap}[0]{\hat{\Bo}}
\newcommand{\pcap}[0]{\hat{\Bp}}
\newcommand{\qcap}[0]{\hat{\Bq}}
\newcommand{\rcap}[0]{\hat{\Br}}
\newcommand{\scap}[0]{\hat{\Bs}}
\newcommand{\tcap}[0]{\hat{\Bt}}
\newcommand{\ucap}[0]{\hat{\Bu}}
\newcommand{\vcap}[0]{\hat{\Bv}}
\newcommand{\wcap}[0]{\hat{\Bw}}
\newcommand{\xcap}[0]{\hat{\Bx}}
\newcommand{\ycap}[0]{\hat{\By}}
\newcommand{\zcap}[0]{\hat{\Bz}}
\newcommand{\thetacap}[0]{\hat{\Btheta}}

%
% to write R^n and C^n in a distinguishable fashion.  Perhaps change this
% to the double lined characters upon figuring out how to do so.
%
\newcommand{\C}[1]{$\mathbb{C}^{#1}$}
\newcommand{\R}[1]{$\mathbb{R}^{#1}$}

%
% various generally useful helpers
%

% derivative of #1 wrt. #2:
\newcommand{\D}[2] {\frac {d#2} {d#1}}

\newcommand{\inv}[1]{\frac{1}{#1}}
\newcommand{\cross}[0]{\times}

\newcommand{\abs}[1]{\lvert{#1}\rvert}
\newcommand{\norm}[1]{\lVert{#1}\rVert}
\newcommand{\innerprod}[2]{\langle{#1}, {#2}\rangle}
\newcommand{\dotprod}[2]{{#1} \cdot {#2}}
\newcommand{\bdotprod}[2]{\left({#1} \cdot {#2}\right)}
\newcommand{\crossprod}[2]{{#1} \cross {#2}}
\newcommand{\tripleprod}[3]{\dotprod{\left(\crossprod{#1}{#2}\right)}{#3}}

\DeclareMathOperator{\Proj}{Proj}
\DeclareMathOperator{\Span}{span}
\DeclareMathOperator{\Sgn}{sgn}
\DeclareMathOperator{\Area}{Area}
\DeclareMathOperator{\Volume}{Volume}

%
% A few miscellaneous things specific to this document
%
\newcommand{\crossop}[1]{\crossprod{#1}{}}

% R2 vector.
\newcommand{\VectorTwo}[2]{
\begin{bmatrix}
 {#1} \\
 {#2}
\end{bmatrix}
}

\newcommand{\VectorN}[1]{
\begin{bmatrix}
{#1}_1 \\
{#1}_2 \\
\vdots \\
{#1}_N \\
\end{bmatrix}
}

\newcommand{\DETuvij}[4]{
\begin{vmatrix}
 {#1}_{#3} & {#1}_{#4} \\
 {#2}_{#3} & {#2}_{#4}
\end{vmatrix}
}

\newcommand{\DETuvwijk}[6]{
\begin{vmatrix}
 {#1}_{#4} & {#1}_{#5} & {#1}_{#6} \\
 {#2}_{#4} & {#2}_{#5} & {#2}_{#6} \\
 {#3}_{#4} & {#3}_{#5} & {#3}_{#6}
\end{vmatrix}
}

\newcommand{\DETuvwxijkl}[8]{
\begin{vmatrix}
 {#1}_{#5} & {#1}_{#6} & {#1}_{#7} & {#1}_{#8} \\
 {#2}_{#5} & {#2}_{#6} & {#2}_{#7} & {#2}_{#8} \\
 {#3}_{#5} & {#3}_{#6} & {#3}_{#7} & {#3}_{#8} \\
 {#4}_{#5} & {#4}_{#6} & {#4}_{#7} & {#4}_{#8} \\
\end{vmatrix}
}

%\newcommand{\DETuvwxyijklm}[10]{
%\begin{vmatrix}
% {#1}_{#6} & {#1}_{#7} & {#1}_{#8} & {#1}_{#9} & {#1}_{#10} \\
% {#2}_{#6} & {#2}_{#7} & {#2}_{#8} & {#2}_{#9} & {#2}_{#10} \\
% {#3}_{#6} & {#3}_{#7} & {#3}_{#8} & {#3}_{#9} & {#3}_{#10} \\
% {#4}_{#6} & {#4}_{#7} & {#4}_{#8} & {#4}_{#9} & {#4}_{#10} \\
% {#5}_{#6} & {#5}_{#7} & {#5}_{#8} & {#5}_{#9} & {#5}_{#10}
%\end{vmatrix}
%}

% R3 vector.
\newcommand{\VectorThree}[3]{
\begin{bmatrix}
 {#1} \\
 {#2} \\
 {#3}
\end{bmatrix}
}



\author{Peeter Joot}
\email{peeter.joot@gmail.com}

%\documentclass[]{eliblogwidescreen}

\usepackage{amsmath}
\usepackage{mathpazo}

%
% shorthand for bold symbols, convenient for vectors and matrices
%
\newcommand{\Ba}[0]{\mathbf{a}}
\newcommand{\Bb}[0]{\mathbf{b}}
\newcommand{\Bc}[0]{\mathbf{c}}
\newcommand{\Bd}[0]{\mathbf{d}}
\newcommand{\Be}[0]{\mathbf{e}}
\newcommand{\Bf}[0]{\mathbf{f}}
\newcommand{\Bg}[0]{\mathbf{g}}
\newcommand{\Bh}[0]{\mathbf{h}}
\newcommand{\Bi}[0]{\mathbf{i}}
\newcommand{\Bj}[0]{\mathbf{j}}
\newcommand{\Bk}[0]{\mathbf{k}}
\newcommand{\Bl}[0]{\mathbf{l}}
\newcommand{\Bm}[0]{\mathbf{m}}
\newcommand{\Bn}[0]{\mathbf{n}}
\newcommand{\Bo}[0]{\mathbf{o}}
\newcommand{\Bp}[0]{\mathbf{p}}
\newcommand{\Bq}[0]{\mathbf{q}}
\newcommand{\Br}[0]{\mathbf{r}}
\newcommand{\Bs}[0]{\mathbf{s}}
\newcommand{\Bt}[0]{\mathbf{t}}
\newcommand{\Bu}[0]{\mathbf{u}}
\newcommand{\Bv}[0]{\mathbf{v}}
\newcommand{\Bw}[0]{\mathbf{w}}
\newcommand{\Bx}[0]{\mathbf{x}}
\newcommand{\By}[0]{\mathbf{y}}
\newcommand{\Bz}[0]{\mathbf{z}}
\newcommand{\BA}[0]{\mathbf{A}}
\newcommand{\BB}[0]{\mathbf{B}}
\newcommand{\BC}[0]{\mathbf{C}}
\newcommand{\BD}[0]{\mathbf{D}}
\newcommand{\BE}[0]{\mathbf{E}}
\newcommand{\BF}[0]{\mathbf{F}}
\newcommand{\BG}[0]{\mathbf{G}}
\newcommand{\BH}[0]{\mathbf{H}}
\newcommand{\BI}[0]{\mathbf{I}}
\newcommand{\BJ}[0]{\mathbf{J}}
\newcommand{\BK}[0]{\mathbf{K}}
\newcommand{\BL}[0]{\mathbf{L}}
\newcommand{\BM}[0]{\mathbf{M}}
\newcommand{\BN}[0]{\mathbf{N}}
\newcommand{\BO}[0]{\mathbf{O}}
\newcommand{\BP}[0]{\mathbf{P}}
\newcommand{\BQ}[0]{\mathbf{Q}}
\newcommand{\BR}[0]{\mathbf{R}}
\newcommand{\BS}[0]{\mathbf{S}}
\newcommand{\BT}[0]{\mathbf{T}}
\newcommand{\BU}[0]{\mathbf{U}}
\newcommand{\BV}[0]{\mathbf{V}}
\newcommand{\BW}[0]{\mathbf{W}}
\newcommand{\BX}[0]{\mathbf{X}}
\newcommand{\BY}[0]{\mathbf{Y}}
\newcommand{\BZ}[0]{\mathbf{Z}}

\newcommand{\Bzero}[0]{\mathbf{0}}
\newcommand{\Btheta}[0]{\boldsymbol{\theta}}
\newcommand{\Btau}[0]{\boldsymbol{\tau}}
\newcommand{\Bomega}[0]{\boldsymbol{\omega}}

%
% shorthand for unit vectors
%
\newcommand{\acap}[0]{\hat{\Ba}}
\newcommand{\bcap}[0]{\hat{\Bb}}
\newcommand{\ccap}[0]{\hat{\Bc}}
\newcommand{\dcap}[0]{\hat{\Bd}}
\newcommand{\ecap}[0]{\hat{\Be}}
\newcommand{\fcap}[0]{\hat{\Bf}}
\newcommand{\gcap}[0]{\hat{\Bg}}
\newcommand{\hcap}[0]{\hat{\Bh}}
\newcommand{\icap}[0]{\hat{\Bi}}
\newcommand{\jcap}[0]{\hat{\Bj}}
\newcommand{\kcap}[0]{\hat{\Bk}}
\newcommand{\lcap}[0]{\hat{\Bl}}
\newcommand{\mcap}[0]{\hat{\Bm}}
\newcommand{\ncap}[0]{\hat{\Bn}}
\newcommand{\ocap}[0]{\hat{\Bo}}
\newcommand{\pcap}[0]{\hat{\Bp}}
\newcommand{\qcap}[0]{\hat{\Bq}}
\newcommand{\rcap}[0]{\hat{\Br}}
\newcommand{\scap}[0]{\hat{\Bs}}
\newcommand{\tcap}[0]{\hat{\Bt}}
\newcommand{\ucap}[0]{\hat{\Bu}}
\newcommand{\vcap}[0]{\hat{\Bv}}
\newcommand{\wcap}[0]{\hat{\Bw}}
\newcommand{\xcap}[0]{\hat{\Bx}}
\newcommand{\ycap}[0]{\hat{\By}}
\newcommand{\zcap}[0]{\hat{\Bz}}
\newcommand{\thetacap}[0]{\hat{\Btheta}}

%
% to write R^n and C^n in a distinguishable fashion.  Perhaps change this
% to the double lined characters upon figuring out how to do so.
%
\newcommand{\C}[1]{$\mathbb{C}^{#1}$}
\newcommand{\R}[1]{$\mathbb{R}^{#1}$}

%
% various generally useful helpers
%

% derivative of #1 wrt. #2:
\newcommand{\D}[2] {\frac {d#2} {d#1}}

\newcommand{\inv}[1]{\frac{1}{#1}}
\newcommand{\cross}[0]{\times}

\newcommand{\abs}[1]{\lvert{#1}\rvert}
\newcommand{\norm}[1]{\lVert{#1}\rVert}
\newcommand{\innerprod}[2]{\langle{#1}, {#2}\rangle}
\newcommand{\dotprod}[2]{{#1} \cdot {#2}}
\newcommand{\bdotprod}[2]{\left({#1} \cdot {#2}\right)}
\newcommand{\crossprod}[2]{{#1} \cross {#2}}
\newcommand{\tripleprod}[3]{\dotprod{\left(\crossprod{#1}{#2}\right)}{#3}}

\DeclareMathOperator{\Proj}{Proj}
\DeclareMathOperator{\Span}{span}
\DeclareMathOperator{\Sgn}{sgn}
\DeclareMathOperator{\Area}{Area}
\DeclareMathOperator{\Volume}{Volume}

%
% A few miscellaneous things specific to this document
%
\newcommand{\crossop}[1]{\crossprod{#1}{}}

% R2 vector.
\newcommand{\VectorTwo}[2]{
\begin{bmatrix}
 {#1} \\
 {#2}
\end{bmatrix}
}

\newcommand{\VectorN}[1]{
\begin{bmatrix}
{#1}_1 \\
{#1}_2 \\
\vdots \\
{#1}_N \\
\end{bmatrix}
}

\newcommand{\DETuvij}[4]{
\begin{vmatrix}
 {#1}_{#3} & {#1}_{#4} \\
 {#2}_{#3} & {#2}_{#4}
\end{vmatrix}
}

\newcommand{\DETuvwijk}[6]{
\begin{vmatrix}
 {#1}_{#4} & {#1}_{#5} & {#1}_{#6} \\
 {#2}_{#4} & {#2}_{#5} & {#2}_{#6} \\
 {#3}_{#4} & {#3}_{#5} & {#3}_{#6}
\end{vmatrix}
}

\newcommand{\DETuvwxijkl}[8]{
\begin{vmatrix}
 {#1}_{#5} & {#1}_{#6} & {#1}_{#7} & {#1}_{#8} \\
 {#2}_{#5} & {#2}_{#6} & {#2}_{#7} & {#2}_{#8} \\
 {#3}_{#5} & {#3}_{#6} & {#3}_{#7} & {#3}_{#8} \\
 {#4}_{#5} & {#4}_{#6} & {#4}_{#7} & {#4}_{#8} \\
\end{vmatrix}
}

%\newcommand{\DETuvwxyijklm}[10]{
%\begin{vmatrix}
% {#1}_{#6} & {#1}_{#7} & {#1}_{#8} & {#1}_{#9} & {#1}_{#10} \\
% {#2}_{#6} & {#2}_{#7} & {#2}_{#8} & {#2}_{#9} & {#2}_{#10} \\
% {#3}_{#6} & {#3}_{#7} & {#3}_{#8} & {#3}_{#9} & {#3}_{#10} \\
% {#4}_{#6} & {#4}_{#7} & {#4}_{#8} & {#4}_{#9} & {#4}_{#10} \\
% {#5}_{#6} & {#5}_{#7} & {#5}_{#8} & {#5}_{#9} & {#5}_{#10}
%\end{vmatrix}
%}

% R3 vector.
\newcommand{\VectorThree}[3]{
\begin{bmatrix}
 {#1} \\
 {#2} \\
 {#3}
\end{bmatrix}
}



\author{Peeter Joot}
\email{peeter.joot@gmail.com}


\chapter{Inclined flow without constant height assumption.  Setting up the problem, but not actually solving it.}
\label{chap:inclinedFlowWithoutConstantHeightAssumption}
%\useCCL
\blogpage{http://sites.google.com/site/peeterjoot2/math2012/inclinedFlowWithoutConstantHeightAssumption.pdf}
\date{Mar 6, 2012}
\gitRevisionInfo{inclinedFlowWithoutConstantHeightAssumption}

\beginArtWithToc
%\beginArtNoToc

\section{Motivation.}

In an informal discussion after class, it was claimed that the steady state flow down a plane would have constant height, unless you bring surface tension effects into the mix.  Part of that statement just doesn't make sense to me.  Consider the forces acting on the fluid in the figure (\ref{fig:inclinedFlowWithoutConstantHeightAssumption:inclinedFlowWithoutConstantHeightAssumptionFig1})

\begin{figure}[htp]
   \centering
   \includegraphics[totalheight=0.3\textheight]{inclinedFlowWithoutConstantHeightAssumptionFig1}
\caption{Gravitational force components acting on fluid flowing down a plane.}
\label{fig:inclinedFlowWithoutConstantHeightAssumption:inclinedFlowWithoutConstantHeightAssumptionFig1}
\end{figure}

In the inclined reference frame we have a component of the force acting downwards (in the negative y-axis direction), and have a component directed down the x-axis.  Wouldn't this act to both push the fluid down the plane and push part of the fluid downwards?  I'd expect this to introduce some vorticity as depicted.

While we are just about to start covered surface tension, perhaps this is just allowing the surface to vary, and then solving the Navier-Stokes equations that result.  Let's try setting up the Navier-Stokes equation for steady state viscous flow down a plane without any assumption that the height is constant and see how far we can get.

\section{The equations of motion.}

We'll use the same coordinates as before, with the directions given as in figure (\ref{fig:inclinedFlowWithoutConstantHeightAssumption:inclinedFlowWithoutConstantHeightAssumptionFig2}).  However, this time, we let the height $h(x)$ of the fluid at any distance $x$ down the plane from the initial point vary.

\begin{figure}[htp]
   \centering
   \def\svgwidth{0.7\columnwidth}
   \input{inclinedFlowWithoutConstantHeightAssumptionFig2.pdf_tex}
   \caption{Diagram of coordinates for inclined flow problem.}
\label{fig:inclinedFlowWithoutConstantHeightAssumption:inclinedFlowWithoutConstantHeightAssumptionFig2}
\end{figure}

For viscous incompressible flow down the plane our equations of motion are

\begin{subequations}
\label{eqn:inclinedFlowWithoutConstantHeightAssumption:20}
\begin{equation}\label{eqn:inclinedFlowWithoutConstantHeightAssumption:380}
\rho \PD{t}{u} + \rho (u \partial_x + v \partial_y) u = -\partial_x p + \mu \left(\partial_{xx} + \partial_{yy}\right) u + \rho g \sin\alpha 
\end{equation}
\begin{equation}\label{eqn:inclinedFlowWithoutConstantHeightAssumption:400}
\rho \PD{t}{v} + \rho (u \partial_x + v \partial_y) v = -\partial_y p + \mu \left(\partial_{xx} + \partial_{yy}\right) v - \rho g \cos\alpha 
\end{equation}
\begin{equation}\label{eqn:inclinedFlowWithoutConstantHeightAssumption:420}
0 = -\partial_z p 
\end{equation}
\begin{equation}\label{eqn:inclinedFlowWithoutConstantHeightAssumption:440}
0 = \partial_x u + \partial_y v.
\end{equation}
\end{subequations}

Now, can we kill the time dependent term?  Even allowing for $u$ to vary with $x$ and introducing a non-horizontal flow component, I think that we can.  If the flow at $x = 0$ is constant, not varying at all with time, I think it makes sense that we will have no time dependence in the flow for $x \ne 0$.  So, I believe that our starting point is as above, but with the time derivatives killed off.  That is

\begin{subequations}
\label{eqn:inclinedFlowWithoutConstantHeightAssumption:40}
\begin{equation}\label{eqn:inclinedFlowWithoutConstantHeightAssumption:300}
\rho (u \partial_x + v \partial_y) u = -\partial_x p + \mu \left(\partial_{xx} + \partial_{yy}\right) u + \rho g \sin\alpha \\
\end{equation}
\begin{equation}\label{eqn:inclinedFlowWithoutConstantHeightAssumption:320}
\rho (u \partial_x + v \partial_y) v = -\partial_y p + \mu \left(\partial_{xx} + \partial_{yy}\right) v - \rho g \cos\alpha \\
\end{equation}
\begin{equation}\label{eqn:inclinedFlowWithoutConstantHeightAssumption:340}
0 = -\partial_z p \\
\end{equation}
\begin{equation}\label{eqn:inclinedFlowWithoutConstantHeightAssumption:360}
0 = \partial_x u + \partial_y v.
\end{equation}
\end{subequations}

These don't look particularly easy to solve, and we haven't even set up the boundary value constraints yet.  Let's do that next.

\section{The boundary value constraints.}

One of out constraints is the no-slip condition for the velocity components at the base of the slope

\begin{equation}\label{eqn:inclinedFlowWithoutConstantHeightAssumption:80}
\boxed{
u(x, 0) = v(x, 0) = 0.
}
\end{equation}

We should also have a zero tangential component to the traction vector at the interface.  We need to consider some geometry, and refer to figure (\ref{fig:inclinedFlowWithoutConstantHeightAssumption:inclinedFlowWithoutConstantHeightAssumptionFig3})

\begin{figure}[htp]
   \centering
   \def\svgwidth{0.7\columnwidth}
   \input{inclinedFlowWithoutConstantHeightAssumptionFig3.pdf_tex}
   \caption{Differential vector element.}
\label{fig:inclinedFlowWithoutConstantHeightAssumption:inclinedFlowWithoutConstantHeightAssumptionFig3}
\end{figure}

A position vector on the surface has the value

\begin{equation}\label{eqn:inclinedFlowWithoutConstantHeightAssumption:100}
\Br = x \xcap + h \ycap
\end{equation}

so that a differential element on the surface, tangential to the surface is proportional to

\begin{equation}\label{eqn:inclinedFlowWithoutConstantHeightAssumption:120}
d\Br = \left( \xcap + \ddx{h} \ycap \right) dx
\end{equation}

so our unit tangent vector in the direction depicted in the figure is

\begin{equation}\label{eqn:inclinedFlowWithoutConstantHeightAssumption:140}
\taucap = \inv{\sqrt{1 + (h')^2}} \left( 1, h' \right).
\end{equation}

The outwards facing normal has a value, up to a factor of plus or minus one, of

\begin{equation}\label{eqn:inclinedFlowWithoutConstantHeightAssumption:160}
\ncap = 
\inv{\sqrt{1 + (h')^2}} \left( h', -1 \right).
\end{equation}

We can fix the orientation by considering the unit bivector

\begin{align*}
\taucap \wedge \ncap &= 
\inv{1 + (h')^2} 
\left( 1, h' \right) 
\wedge
\left( h', -1 \right) \\
&=
\begin{vmatrix}
1 & h' \\
h' & -1
\end{vmatrix}
\Be_1 \Be_2 \\
&=
( -1 - (h')^2 )
\Be_1 \Be_2.
\end{align*}

So we really want the other orientation

\begin{equation}\label{eqn:inclinedFlowWithoutConstantHeightAssumption:180}
\ncap = \inv{\sqrt{1 + (h')^2}} \left( -h', 1 \right).
\end{equation}

Our traction vector relative to the normal $\ncap$ is

\begin{align*}
\Bt 
&= \Be_i T_{ij} n_j \\
&= 
\Be_i \left( -p \delta_{ij}
+ \mu e_{ij}
\right) n_j \\
&=
-p \ncap + \mu \Be_i e_{ij} n_j
\end{align*}

So the component in the tangential direction is

\begin{align*}
\Bt \cdot \taucap 
&=
-p \cancel{\ncap \cdot \taucap} + \mu \Be_i e_{ij} n_j tau_i \\
&=
\frac{\mu}{1 + (h')^2}
\begin{bmatrix}
1 & h' 
\end{bmatrix}
\begin{bmatrix}
e_{11} & e_{12} \\
e_{21} & e_{22}
\end{bmatrix}
\begin{bmatrix}
-h'  \\
1
\end{bmatrix} \\
&=
\frac{\mu}{1 + (h')^2}
\begin{bmatrix}
1 & h' 
\end{bmatrix}
\begin{bmatrix}
-h' e_{11} + e_{12} \\
-h' e_{21} + e_{22}
\end{bmatrix} \\
&=
\frac{\mu}{1 + (h')^2}
\left(
-h' e_{11} + e_{12} + h'( -h' e_{21} + e_{22} )
\right) \\
&=
\frac{\mu}{1 + (h')^2}
\left(
h' (e_{22} - e_{11} )
+ e_{12} (1 - (h')^2 )
\right) 
\end{align*}

Our strain tensor components, for a general 2D flow, are

\begin{align}\label{eqn:inclinedFlowWithoutConstantHeightAssumption:200}
e_{11} &= \PD{x}{u} \\
e_{22} &= \PD{y}{v} \\
e_{12} &= 
\inv{2} \left( 
\PD{x}{v} +
\PD{y}{u}
\right).
\end{align}

So, a zero tangential traction vector component at the interface requires

\begin{equation}\label{eqn:inclinedFlowWithoutConstantHeightAssumption:220}
\boxed{
0 = h' \left( \evalbar{\left( \PD{y}{v} - \PD{x}{u} \right) + 
\inv{2} \left( 
\PD{x}{v} +
\PD{y}{u}
\right)
}{y = h}\right)
(1 - (h')^2 ).
}
\end{equation}

What is the normal component of the traction vector at the interface?  We can calculate

\begin{align*}
\Bt \cdot \ncap 
&=
-p \ncap \cdot \ncap + \mu \Be_i e_{ij} n_j n_i \\
&=
-p
+
\frac{\mu}{1 + (h')^2}
\begin{bmatrix}
-h' & 1
\end{bmatrix}
\begin{bmatrix}
-h' e_{11} + e_{12} \\
-h' e_{21} + e_{22}
\end{bmatrix} \\
&=
-p
+
\frac{\mu}{1 + (h')^2}
\left(
-h'(-h' e_{11} + e_{12}) -h' e_{21} + e_{22}
\right)  \\
&=
-p
+
\frac{\mu}{1 + (h')^2}
\left(
-2 h' e_{12} 
+ (h')^2 e_{11} + e_{22}
\right)
\end{align*}

So this component of the traction vector is
\begin{equation}\label{eqn:inclinedFlowWithoutConstantHeightAssumption:240}
\Bt \cdot \ncap 
=
-p
+
\frac{\mu}{1 + (h')^2}
\left(
- h' 
\left( 
\PD{x}{v} +
\PD{y}{u}
\right)
+ (h')^2 \PD{x}{u} 
 + \PD{y}{v} 
\right)
\end{equation}

For the purely recilinear flow, with $h' = 0$ and $v = 0$, as a a consequence of NS and our assumptions, all but the pressure portion of this component of the traction vector was zero.  The force balance equation for the interface was therefore just a matching of the pressure with the external (ie: air) pressure.

In this more general case we have the same thing, but the non-pressure portions of the traction vector are all zero in the medium.  Outside of the fluid (in the air say), we have assumed no motion, so this force balance condition becomes

\begin{equation}\label{eqn:inclinedFlowWithoutConstantHeightAssumption:260}
\evalbar{\Bt \cdot \ncap}{\text{fluid}}
=
\evalbar{\Bt \cdot \ncap}{\text{air}}.
\end{equation}

Again assuming no motion of the air, with air pressure $p_A$, this is

\begin{equation}\label{eqn:inclinedFlowWithoutConstantHeightAssumption:280}
\boxed{
-p(x, h)
+
\evalbar{
\frac{\mu}{1 + (h')^2}
\left(
- h' 
\left( 
\PD{x}{v} +
\PD{y}{u}
\right)
+ (h')^2 \PD{x}{u} 
 + \PD{y}{v} 
\right)
}{y = h}
= -p_A
}
\end{equation}

Observe that for the horizontal flow problem, where $h' = 0$ and $v = 0$, this would have been nothing more than a requirement that $p(h) = p_A$, but now that we introduce downwards motion and allow the height to vary, the pressure matching condition becomes a much more complex beastie.

\section{Laplacian of Pressure and Vorticity.}

Supposing that we are neglecting the non-linear term of the Navier-Stokes equation.  For incompressible steady state flow, without any external forces, we would then have

\begin{subequations}
\begin{equation}\label{eqn:inclinedFlowWithoutConstantHeightAssumption:5}
0 = -\spacegrad p + \mu \spacegrad^2 \Bu 
\end{equation}
\begin{equation}\label{eqn:inclinedFlowWithoutConstantHeightAssumption:460}
0 = \spacegrad \cdot \Bu
\end{equation}
\end{subequations}

How do we actually solve this beastie?

\subsection{Separation of variables?}

Considering this in 2D, assuming no z-dependence, with $\Bu = \Bu(x, y) = (u, v)$ we have

\begin{align}\label{eqn:nsVorticity:30}
0 &= -\partial_x p + \mu (\partial_{xx} + \partial_{yy} )u \\
0 &= -\partial_y p + \mu (\partial_{xx} + \partial_{yy} )v \\
0 &= \partial_x u + \partial_y v.
\end{align}

Attempting separation of variables seems like something reasonable to try.  With

\begin{align}\label{eqn:nsVorticity:50}
u &= X(x) Y(y) \\
v &= R(x) S(y) \\
p &= P(x) Q(y)
\end{align}

we get

\begin{align}\label{eqn:nsVorticity:70}
0 &= -P' Q + \mu (X'' Y + X Y'') \\
0 &= -P Q' + \mu (R'' S + R S'') \\
0 &= X' Y + R S'
\end{align}

I couldn't find a way to substitute any of these into the other that would allow me to separate them, but perhaps I wasn't clever enough.  

\subsection{In terms of vorticity?}

The idea of substituting the zero divergence equation $\spacegrad \cdot \Bu$ will clearly lead to something a bit simpler.  Treating the Laplacian as a geometric (Clifford) product of two gradients we have

\begin{align*}
0 
&= -\spacegrad p + \mu \spacegrad^2 \Bu \\
&= -\spacegrad p + \mu \spacegrad ( \spacegrad \Bu ) \\
&= -\spacegrad p + \mu \spacegrad ( \cancel{\spacegrad \cdot \Bu} + \spacegrad \wedge \Bu ) \\
&= -\spacegrad p + \mu \spacegrad ( \spacegrad \wedge \Bu ) \\
&= \spacegrad \left( -p + \mu \spacegrad \wedge \Bu \right)
\end{align*}

Writing out the vorticity (bivector) in component form, and writing $i = \Be_1 \wedge \Be_2 = \Be_1 \Be_2$ for the 2D pseudoscalar, we have

\begin{align*}
\spacegrad \wedge \Bu 
&= (\Be_1 \partial_x + \Be_2 \partial_y) \wedge (\Be_1 u + \Be_2 v) \\
&= \Be_1 \Be_2 (\partial_x v - \partial_y u ) \\
&= i (\partial_x v - \partial_y u )
\end{align*}

It seems natural to write

\begin{equation}\label{eqn:inclinedFlowWithoutConstantHeightAssumption:90}
\Theta = \partial_x v - \partial_y u,
\end{equation}

so that Navier-Stokes takes the form

\begin{equation}\label{eqn:inclinedFlowWithoutConstantHeightAssumption:110}
0 = \spacegrad (-p + i \mu \Theta ).
\end{equation}

Operating on this from the left with another gradient we find that this combination of pressure and vorticity must satisfy the following multivector Laplacian equation

\begin{equation}\label{eqn:inclinedFlowWithoutConstantHeightAssumption:130}
0 = \spacegrad^2 (-p + i \mu \Theta ).
\end{equation}

However, note that $\spacegrad^2$ is a scalar operator, and this zero identity has both scalar and pure bivector components.  Both must separately equal zero

\begin{subequations}
\label{eqn:inclinedFlowWithoutConstantHeightAssumption:150}
\begin{equation}\label{eqn:inclinedFlowWithoutConstantHeightAssumption:170}
0 = \spacegrad^2 p 
\end{equation}
\begin{equation}\label{eqn:inclinedFlowWithoutConstantHeightAssumption:190}
0 = \spacegrad^2 \Theta.
\end{equation}
\end{subequations}

Note that we can obtain \ref{eqn:inclinedFlowWithoutConstantHeightAssumption:150} much more directly, if we know that's what we want to do.  Just operate on \ref{eqn:inclinedFlowWithoutConstantHeightAssumption:5} with the gradient from the left right off the bat.  We find

\begin{align*}
0 
&= -\spacegrad^2 p + \mu \spacegrad^3 \Bu \\
&= -\spacegrad^2 p + \mu \spacegrad^2 (\spacegrad \Bu) \\
&= -\spacegrad^2 p + \mu \spacegrad^2 (\spacegrad \wedge \Bu) \\
\end{align*}

Again, we have a multivector equation scalar and bivector parts, that must separately equal zero.  With the magnitude of the vorticity $\Theta$  given by \ref{eqn:inclinedFlowWithoutConstantHeightAssumption:90}, we once again obtain \ref{eqn:inclinedFlowWithoutConstantHeightAssumption:150}.  This can be done in plain old vector algebra as well by operating on the left not by the gradient, but separately with a divergence and curl operator.

\subsection{Pressure and vorticity equations with the non-linear term retained.}

If we add back in our body force, and assume that it is constant (i.e. gravity), then this this will get killed with the application of the gradient.  We'll still end up with one Laplacian for pressure, and one for vorticity.  That's not the case for the inertial $(\Bu \cdot \spacegrad) \Bu$ term of Navier-Stokes.  Let's take the divergence and curl of this and see how we have to modify the Laplacian equations above.

Starting with the divergence, with summation implied over repeated indexes, we have

\begin{align*}
\spacegrad \cdot ((\Bu \cdot \spacegrad) \Bu)
&=
\partial_k ( \Bu \cdot \spacegrad u_k ) \\
&=
\partial_k ( u_m \partial_m u_k ) \\
&=
(\partial_k u_m) (\partial_m u_k ) 
+ u_m \partial_m \partial_k u_k \\
&=
\sum_k (\partial_k u_k)^2
+
\sum_{k \ne m} (\partial_k u_m) (\partial_m u_k ) 
+ (\Bu \cdot \spacegrad) ( \spacegrad \cdot \Bu )
\end{align*}

So we have

\begin{equation}\label{eqn:inclinedFlowWithoutConstantHeightAssumption:480}
\spacegrad \cdot ((\Bu \cdot \spacegrad) \Bu)
=
\sum_k (\partial_k u_k)^2
+
2 \sum_{k < m} (\partial_k u_m) (\partial_m u_k ) 
+ (\Bu \cdot \spacegrad) ( \spacegrad \cdot \Bu )
\end{equation}

We are working with the $\spacegrad \cdot \Bu = 0$ incompressibility assumption so we kill off the last term.  Our velocity ends up introducing a non-homogeneous forcing term to the Laplacian pressure equation and we've got something trickier to solve

\begin{equation}\label{eqn:inclinedFlowWithoutConstantHeightAssumption:500}
\boxed{
\rho \sum_k (\partial_k u_k)^2
+
2 \rho \sum_{k < m} (\partial_k u_m) (\partial_m u_k ) 
=
-\spacegrad^2 p.
}
\end{equation}

Now let's see how our vorticity Laplacian needs to be modified.  Taking the curl of the impulsive term we have

\begin{align*}
\spacegrad \wedge ((\Bu \cdot \spacegrad) \Bu) 
&=
\Be_k \partial_k \wedge ( u_m \partial_m u_n \Be_n ) \\
&=
(\Be_k \wedge \Be_n) \partial_k ( u_m \partial_m u_n ) \\
&=
(\Be_k \wedge \Be_n) 
\left( 
(\partial_k u_m) (\partial_m u_n ) 
+u_m \partial_m \partial_k u_n 
\right) \\
&=
(\Be_k \wedge \Be_n) 
\left( 
(\partial_k u_m) (\partial_m u_n ) 
+ (\Bu \cdot \spacegrad) \partial_k u_n
\right) 
\end{align*}

So we have
\begin{equation}\label{eqn:inclinedFlowWithoutConstantHeightAssumption:520}
\spacegrad \wedge ((\Bu \cdot \spacegrad) \Bu) 
=
(\spacegrad u_m) \wedge (\partial_m \Bu)
+ (\Bu \cdot \spacegrad) (\spacegrad \wedge \Bu)
\end{equation}

Putting things back together, our vorticity equation is

\begin{equation}\label{eqn:inclinedFlowWithoutConstantHeightAssumption:540}
\rho (\spacegrad u_m) \wedge (\partial_m \Bu)
+ \rho (\Bu \cdot \spacegrad) (\spacegrad \wedge \Bu)
= 
\mu \spacegrad^2 (\spacegrad \wedge \Bu)
\end{equation}

Or, with

\begin{equation}\label{eqn:inclinedFlowWithoutConstantHeightAssumption:560}
\Bomega = \spacegrad \wedge \Bu,
\end{equation}

this is

\begin{equation}\label{eqn:inclinedFlowWithoutConstantHeightAssumption:580}
\boxed{
(\spacegrad u_m) \wedge (\partial_m \Bu) + (\Bu \cdot \spacegrad) \Bomega = \nu \spacegrad^2 \Bomega.
}
\end{equation}

It is this and \ref{eqn:inclinedFlowWithoutConstantHeightAssumption:500} that we really have to solve.  Before moving on, let's write out all the non-boundary condition equations in coordinate form for the 2D case that we are interested in here.  We have

\begin{subequations}
\begin{equation}\label{eqn:inclinedFlowWithoutConstantHeightAssumption:600}
\rho \left( \left(\PD{x}{u}\right)^2 +\left(\PD{y}{u}\right)^2 + 2 \PD{y}{u} \PD{x}{v} \right) = -\spacegrad^2 p
\end{equation}
\begin{equation}\label{eqn:inclinedFlowWithoutConstantHeightAssumption:620}
2 \left( 
\PD{x}{u} \PD{y}{v}
+\PD{x}{v} \PD{y}{v}
\right)
+ \left( u \PD{x}{} + v \PD{y}{} \right) \Theta
=
\nu \spacegrad^2 \Theta
\end{equation}
\begin{equation}\label{eqn:inclinedFlowWithoutConstantHeightAssumption:640}
\Theta = \PD{x}{v} - \PD{y}{u}
\end{equation}
\end{subequations}

Our solution has to satisfy these equations, as well as still satisfying the original Navier-Stokes system \ref{eqn:inclinedFlowWithoutConstantHeightAssumption:40} that includes our gravitational term, and also has to satisfy both of our boundary value constraints \ref{eqn:inclinedFlowWithoutConstantHeightAssumption:220}, \ref{eqn:inclinedFlowWithoutConstantHeightAssumption:280}, and have $u(x, 0) = v(x, 0) = 0$.  Wow, what a mess!  And this is all just the steady state problem.  Imagine adding time into the mix too!

\section{Now what?}

The strategy that I'd thought to attempt to tackle a problem with is something along the following lines

\begin{itemize}
\item First ignore the non-linear terms.  Find solutions for the homogeneous vorticity and pressure Laplacian equations that satisfy our boundary value conditions, and use that to find a first solution for $h(x)$.
\item Use this to solve for $u$ and $v$ from the vorticity.
\end{itemize}

However, after having gotten this far, just setting up the problem for solution, I think that it's perhaps best not to try to solve it yet, and study some more first.  I have a feeling that there are likely a number of techniques that have been developed that will likely be useful to know before I try to plow my way through things.  It's interesting to see just how tricky the equations of motion become when one doesn't make unrealistic assumptions.  I have a feeling that to actually attempt this specific problem, I may very well need a computer and numerical techniques.

%\EndArticle
\EndNoBibArticle

   % 
% 
% 
% Copyright � 2012 Peeter Joot
% All Rights Reserved
% 
% This file may be reproduced and distributed in whole or in part, without fee, subject to the following conditions:
% 
% o The copyright notice above and this permission notice must be preserved complete on all complete or partial copies.
% 
% o Any translation or derived work must be approved by the author in writing before distribution.
% 
% o If you distribute this work in part, instructions for obtaining the complete version of this file must be included, and a means for obtaining a complete version provided.
% 
% 
% Exceptions to these rules may be granted for academic purposes: Write to the author and ask.
% 
% 
% 
%\keywords{channel flow, Navier-Stokes, PHY454H1S, time evolution}
\begin{Exercise}[
title={Channel flow with step pressure gradient},
label={problem:fluids:channelFlowWithStepPressureGradient}
]
This is problem 2.5 from \citep{acheson1990elementary}.

Viscous fluid is at rest in a two-dimensional channel between stationary rigid walls with $y = \pm h$.  For $t \ge 0$ a constant pressure gradient $P = -dp/dx$ is imposed.  Show that $u(y, t)$ satisfies

\begin{equation}\label{eqn:channelFlowWithStepPressureGradient:10}
\PD{t}{u} = \nu \PDSq{y}{u} + \frac{P}{\rho},
\end{equation}

and give suitable initial and boundary conditions.  Find $u(y, t)$ in form of a Fourier series, and show that the flow approximates to steady channel flow when $t \gg h^2/\nu$.

\end{Exercise}

\begin{Answer}[ref={problem:fluids:channelFlowWithStepPressureGradient}]
With only horizontal components to the flow, the Navier-Stokes equations for incompressible flow are

\begin{subequations}
\begin{equation}\label{eqn:channelFlowWithStepPressureGradient:30}
\PD{t}{u} + u \PD{x}{u} = -\inv{\rho} \PD{x}{p} + \nu \left( 
\PDSq{x}{}
+\PDSq{y}{}
\right)
u
\end{equation}
\begin{equation}\label{eqn:channelFlowWithStepPressureGradient:50}
\PD{x}{u} = 0
\end{equation}
\end{subequations}

Substitution of \ref{eqn:channelFlowWithStepPressureGradient:50} into \ref{eqn:channelFlowWithStepPressureGradient:30} gives us

\begin{equation}\label{eqn:channelFlowWithStepPressureGradient:70}
\PD{t}{u} = -\inv{\rho} \PD{x}{p} + \nu \PDSq{y}{u}.
\end{equation}

Our equation to solve is therefore

\begin{equation}\label{eqn:channelFlowWithStepPressureGradient:90}
\boxed{
\PD{t}{u} = \Theta(t) \frac{P}{\rho} + \nu \PDSq{y}{u}.
}
\end{equation}

This equation, for $t < 0$, allows for solutions

\begin{equation}\label{eqn:channelFlowWithStepPressureGradient:110}
u = A y + B
\end{equation}

but the problem states that the fluid is at rest initially, so we don't really have to solve anything (i.e. $A = B = 0$).

The no-slip conditions introduce boundary value conditions $u(\pm h, t) = 0$.

For $t \ge 0$ we have

\begin{equation}\label{eqn:channelFlowWithStepPressureGradient:130}
\PD{t}{u} = \frac{P}{\rho} + \nu \PDSq{y}{u}.
\end{equation}

If we attempt separation of variables with $u(y, t) = Y(y) T(t)$, our equation takes the form

\begin{equation}\label{eqn:channelFlowWithStepPressureGradient:150}
T' Y = \frac{P}{\rho} + \nu T Y''.
\end{equation}

We see that the non-homogeneous term prevents successful application of separation of variables.  Let's modify our problem by attempting to recast our equation into a homogeneous form by adding a particular solution for the steady state flow problem.  That problem was the solution of

\begin{equation}\label{eqn:channelFlowWithStepPressureGradient:170}
\PDSq{y}{} u_s(y, 0) = -\frac{P}{\rho \nu}
\end{equation}

which has solution

\begin{equation}\label{eqn:channelFlowWithStepPressureGradient:190}
u_s(y, 0) = \frac{P}{2 \rho \nu} \left( h^2 - y^2 \right) + A y + B.
\end{equation}

The freedom to incorporate an $h^2$ constant into the equation as an integration constant has been employed, knowing that it will kill the $y^2$ contributions at $y = \pm h$ to make the boundary condition matching easier.  Our no-slip conditions give us

\begin{align}\label{eqn:channelFlowWithStepPressureGradient:210}
0 &= A h + B \\
0 &= -A h + B.
\end{align}

Adding this we have $2 B = 0$, and subtracting gives us $2 A h = 0$, so a specific solution that matches our required boundary value (and initial value) conditions is just the steady state channel flow solution we are familiar with

\begin{equation}\label{eqn:channelFlowWithStepPressureGradient:230}
u_s(y, 0) = \frac{P}{2 \rho \nu} \left( h^2 - y^2 \right).
\end{equation}

Let's now assume that our general solution has the form

\begin{equation}\label{eqn:channelFlowWithStepPressureGradient:250}
u(y, t) = u_H(y, t) + u_s(y, 0).
\end{equation}

Applying the Navier-Stokes equation to this gives us

\begin{equation}\label{eqn:channelFlowWithStepPressureGradient:270}
\PD{t}{u_H} = \frac{P}{\rho} + \nu \PDSq{y}{u_H} + \nu \PDSq{y}{u_s}.
\end{equation}

But from \ref{eqn:channelFlowWithStepPressureGradient:170}, we see that all we have left is a homogeneous problem in $u_H$

\begin{equation}\label{eqn:channelFlowWithStepPressureGradient:270b}
\PD{t}{u_H} = \nu \PDSq{y}{u_H},
\end{equation}

where our boundary value conditions are now given by

\begin{align*}
0 
&= u_H(\pm h, t) + u_s(\pm h) \\
&= u_H(\pm h, t),
\end{align*}

and 

\begin{align*}
0 
&= u(y, 0) \\
&= u_H(y, 0) + \frac{P}{2 \rho \nu} \left( h^2 - y^2 \right),
\end{align*}

or
\begin{subequations}
\begin{equation}\label{eqn:channelFlowWithStepPressureGradient:290}
u_H(\pm h, t) = 0
\end{equation}
\begin{equation}\label{eqn:channelFlowWithStepPressureGradient:310}
u_H(y, 0) = -\frac{P}{2 \rho \nu} \left( h^2 - y^2 \right).
\end{equation}
\end{subequations}

Now we can apply separation of variables with $u_H = T(t) Y(y)$, yielding

\begin{equation}\label{eqn:channelFlowWithStepPressureGradient:150b}
T' Y = \nu T Y'',
\end{equation}

or 

\begin{equation}\label{eqn:channelFlowWithStepPressureGradient:330}
\frac{T'}{T} = \nu \frac{Y''}{Y} = \text{constant} = - \nu \alpha^2.
\end{equation}

Here a positive constant $\nu \alpha^2$ has been used assuming that we want a solution that is damped with time.

Our solutions are

\begin{align}\label{eqn:channelFlowWithStepPressureGradient:350}
T &\propto e^{- \nu \alpha^2 t} \\
Y &= A \sin \alpha y + B \cos\alpha y,
\end{align}

or

\begin{equation}\label{eqn:channelFlowWithStepPressureGradient:370}
u_H(y, t) = \sum_\alpha e^{-\alpha^2 \nu t} \left( A_\alpha \sin \alpha y + B_\alpha \cos\alpha y \right).
\end{equation}

We have constraints on $\alpha$ due to our boundary value conditions.  For our sin terms to be solutions we require

\begin{equation}\label{eqn:channelFlowWithStepPressureGradient:390}
\sin (\alpha (\pm h)) = \sin n \pi
\end{equation}

and for our cosine terms to be solutions we require

\begin{equation}\label{eqn:channelFlowWithStepPressureGradient:410}
\cos (\alpha (\pm h)) = \cos \left( \frac{\pi}{2} + n \pi \right)
\end{equation}

\begin{align}\label{eqn:channelFlowWithStepPressureGradient:430}
\alpha &= \frac{2 n \pi}{2 h} \\
\alpha &= \frac{2 n + 1 \pi}{2 h}
\end{align}

respectively.

Our homogeneous solution therefore takes the form

\begin{equation}\label{eqn:channelFlowWithStepPressureGradient:450}
u_H(y, t) = 
C_0 + \sum_{m > 0} C_m e^{ -(m \pi/2h)^2 \nu t } 
\left\{
\begin{array}{l l}
\sin \left( \frac{ m \pi y }{2 h} \right) & \quad \mbox{$m$ even} \\
\cos \left( \frac{ m \pi y }{2 h} \right) & \quad \mbox{$m$ odd} \\
\end{array}
\right.
\end{equation}

Our undetermined constants should be provided by the boundary value constraint at $t = 0$ \ref{eqn:channelFlowWithStepPressureGradient:310}, leaving us to solve the Fourier problem

\begin{equation}\label{eqn:channelFlowWithStepPressureGradient:470}
-\frac{P}{2 \mu} \left( h^2 - y^2 \right)
=
%C_0 + 
\sum_{m \ge 0} C_m 
\left\{
\begin{array}{l l}
\sin \left( \frac{ m \pi y }{2 h} \right) & \quad \mbox{$m$ even} \\
\cos \left( \frac{ m \pi y }{2 h} \right) & \quad \mbox{$m$ odd} \\
\end{array}
\right.
\end{equation}

%We find $C_0$ by integrating over $[-h, h]$
%
%\begin{equation}\label{eqn:channelFlowWithStepPressureGradient:490}
%\int_{-h}^h -\frac{P}{2 \mu} \left( h^2 - y^2 \right) dy = 2 h C_0,
%\end{equation}
%
%which yields
%
%\begin{equation}\label{eqn:channelFlowWithStepPressureGradient:510}
%C_0 = -\frac{P h^2}{12 \mu}.
%\end{equation}

Multiplying by a sine and integrating will clearly give zero (even times odd function over a symmetric interval).  Let's see if there's any scaling required to select out the $C_m$ term

\begin{align*}
\int_{-h}^h 
\cos \left( \frac{ m \pi y }{2 h} \right) 
\cos \left( \frac{ n \pi y }{2 h} \right) dy
&=
\frac{2h}{\pi} \int_{-h}^h 
\cos \left( \frac{ m \pi y }{2 h} \right) 
\cos \left( \frac{ n \pi y }{2 h} \right) \pi dy/2h \\
&=
\frac{2h}{\pi} \int_{-\pi/2}^{\pi/2}
\cos m x \cos n x dx \\
&=
\frac{h}{\pi} \int_{-\pi/2}^{\pi/2}
\left( 
\cos( (m - n) \pi/2 ) 
+\cos( (m + n) \pi/2 ) 
\right) dx
\end{align*}

Note that since $m$ and $n$ must be odd, $m \pm n = 2 c$ for some integer $c$, so this integral is zero unless $m = n$ (consider $m = 2 a + 1, n = 2 b + 1$).  For the $m = n$ term we have

\begin{align*}
\int_{-h}^h 
\cos \left( \frac{ m \pi y }{2 h} \right) 
\cos \left( \frac{ n \pi y }{2 h} \right) dy
&=
\frac{h}{\pi} \int_{-\pi/2}^{\pi/2}
\left( 
1
+\cos( m \pi ) 
\right) dx \\
&=
h
\end{align*}

Therefore, our constants $C_m$ (for odd $m$) are given by

\begin{align*}
C_m 
&= 
-\frac{P h }{2 \mu} 
\int_{-h}^h
\left( 1 - \left( \frac{y}{h}\right)^2 \right)
\cos \left( \frac{ m \pi y }{2 h} \right) dy \\
&=
-\frac{P h^2 }{2 \mu} 
\int_{-1}^1
\left( 1 - x^2 \right)
\cos \left( \frac{ m \pi x }{2} \right) x  \\
\end{align*}

With $m = 2 n + 1$, we have

\begin{equation}\label{eqn:channelFlowWithStepPressureGradient:530}
C_{2 n + 1} = -\frac{16 P h^2 (-1)^n}{\mu \pi^3 (2 n + 1)^3}.
\end{equation}

For that calculation see: \href{https://raw.github.com/peeterjoot/physicsplay/master/notes/phy454/mathematica/channelFlowWithStepPressureGradient.cdf}{channelFlowWithStepPressureGradient.cdf}.

Our complete solution is

\begin{equation}\label{eqn:channelFlowWithStepPressureGradient:550}
u(y, t) =
\frac{P h^2}{2 \mu} \left( 1 - \left( \frac{y}{h} \right)^2 \right)
%-\frac{P h^2}{12 \mu}
- 
\frac{16 P h^2 }{\mu \pi^3 }
\sum_{n = 0}^\infty 
\frac{(-1)^n}{(2 n + 1)^3}
e^{ -((2 n + 1) \pi/2h)^2 \nu t } 
\cos \left( \frac{ (2 n + 1) \pi y }{2 h} \right) .
\end{equation}
%Sum[(-1)^n/((2 n + 1)^3) E^( -((2 n + 1) Pi/(2 h))^2 nu t ) Cos[ (2 n + 1) Pi y /(2 h) ], {n, 0, m}]

The largest of the damped exponentials above is the $n = 0$ term which is

\begin{equation}\label{eqn:channelFlowWithStepPressureGradient:570}
e^{ - \pi^2 \nu t /h^2 },
\end{equation}

so if $\nu t >> h^2$ these terms all die off, leaving us with just the steady state.
%.  That leaves us with the steady state equation, less the non-damped constant factor from the Fourier series.

Rather remarkably, this Fourier series is actually a very good fit even after only a single term.  Using the viscosity and density of water, $h = 1 \text{cm}$, and $P = 3 \times \mu_{\text{water}} \times (2 \text{cm}/{s})/ h^2$ (parameterizing the pressure gradient by the average velocity it will induce), a plot of the parabola that we are fitting to and the difference of that from the first Fourier term is shown in figure (\ref{fig:channelFlowWithStepPressureGradient:channelFlowWithStepPressureGradientFig1}).

\imageFigure{figures/channelFlowWithStepPressureGradientFig1}{Parabolic channel flow steady state, and difference from first Fourier term.}{fig:channelFlowWithStepPressureGradient:channelFlowWithStepPressureGradientFig1}{0.3}

The higher order corrections are even smaller.  Even the first order deviations from the parabola that we are fitting to is a correction on the scale of $1/100$ of the height of the parabola.  This is illustrated in figure (\ref{fig:channelFlowWithStepPressureGradient:channelFlowWithStepPressureGradientFig2}) where the magnitude of the first 5 deviations from the steady state are plotted.

\imageFigure{figures/channelFlowWithStepPressureGradientFig2}{Difference from the steady state for the first five Fourier terms.}{fig:channelFlowWithStepPressureGradient:channelFlowWithStepPressureGradientFig2}{0.3}

An animation of the time evolution above can be found at \youtubehref{0vZuv9HBtmo}.
%in figure \ref{fig:channelFlowWithStepPressureGradient:channelFlowWithStepPressureGradientFig3}.  If this animation is unavailable, it can also be found at 

%\movieFigure{./channelFlowWithStepPressureGradientTimeEvolution.mp4}{Time evolution of channel flow velocity profile after turning on a constant pressure gradient}{fig:channelFlowWithStepPressureGradient:channelFlowWithStepPressureGradientFig3}{width=320pt,height=240pt}

It's also interesting to look at the very earliest part of the time evolution (\youtubehref{dDkx8iLwOew}), where some oscillatory phenomena can be seen.
%Observe in figure \ref{fig:channelFlowWithStepPressureGradient:channelFlowWithStepPressureGradientFig4} (or 
Could some of that be due to not running with enough Fourier terms in this early part of the evolution when more terms are probably significant?

%\movieFigure{./channelFlowWithStepPressureGradientEarlyTimeEvolution.mp4}{Early time evolution of channel flow velocity profile after turning on a constant pressure gradient}{fig:channelFlowWithStepPressureGradient:channelFlowWithStepPressureGradientFig4}{width=320pt,height=240pt}
\end{Answer}

   % 
% 
% 
% Copyright � 2012 Peeter Joot
% All Rights Reserved
% 
% This file may be reproduced and distributed in whole or in part, without fee, subject to the following conditions:
% 
% o The copyright notice above and this permission notice must be preserved complete on all complete or partial copies.
% 
% o Any translation or derived work must be approved by the author in writing before distribution.
% 
% o If you distribute this work in part, instructions for obtaining the complete version of this file must be included, and a means for obtaining a complete version provided.
% 
% 
% Exceptions to these rules may be granted for academic purposes: Write to the author and ask.
% 
% 
% 
%%
% Copyright � 2015 Peeter Joot.  All Rights Reserved.
% Licenced as described in the file LICENSE under the root directory of this GIT repository.
%
\documentclass[]{eliblog}

\usepackage{amsmath}
\usepackage{mathpazo}

%
% shorthand for bold symbols, convenient for vectors and matrices
%
\newcommand{\Ba}[0]{\mathbf{a}}
\newcommand{\Bb}[0]{\mathbf{b}}
\newcommand{\Bc}[0]{\mathbf{c}}
\newcommand{\Bd}[0]{\mathbf{d}}
\newcommand{\Be}[0]{\mathbf{e}}
\newcommand{\Bf}[0]{\mathbf{f}}
\newcommand{\Bg}[0]{\mathbf{g}}
\newcommand{\Bh}[0]{\mathbf{h}}
\newcommand{\Bi}[0]{\mathbf{i}}
\newcommand{\Bj}[0]{\mathbf{j}}
\newcommand{\Bk}[0]{\mathbf{k}}
\newcommand{\Bl}[0]{\mathbf{l}}
\newcommand{\Bm}[0]{\mathbf{m}}
\newcommand{\Bn}[0]{\mathbf{n}}
\newcommand{\Bo}[0]{\mathbf{o}}
\newcommand{\Bp}[0]{\mathbf{p}}
\newcommand{\Bq}[0]{\mathbf{q}}
\newcommand{\Br}[0]{\mathbf{r}}
\newcommand{\Bs}[0]{\mathbf{s}}
\newcommand{\Bt}[0]{\mathbf{t}}
\newcommand{\Bu}[0]{\mathbf{u}}
\newcommand{\Bv}[0]{\mathbf{v}}
\newcommand{\Bw}[0]{\mathbf{w}}
\newcommand{\Bx}[0]{\mathbf{x}}
\newcommand{\By}[0]{\mathbf{y}}
\newcommand{\Bz}[0]{\mathbf{z}}
\newcommand{\BA}[0]{\mathbf{A}}
\newcommand{\BB}[0]{\mathbf{B}}
\newcommand{\BC}[0]{\mathbf{C}}
\newcommand{\BD}[0]{\mathbf{D}}
\newcommand{\BE}[0]{\mathbf{E}}
\newcommand{\BF}[0]{\mathbf{F}}
\newcommand{\BG}[0]{\mathbf{G}}
\newcommand{\BH}[0]{\mathbf{H}}
\newcommand{\BI}[0]{\mathbf{I}}
\newcommand{\BJ}[0]{\mathbf{J}}
\newcommand{\BK}[0]{\mathbf{K}}
\newcommand{\BL}[0]{\mathbf{L}}
\newcommand{\BM}[0]{\mathbf{M}}
\newcommand{\BN}[0]{\mathbf{N}}
\newcommand{\BO}[0]{\mathbf{O}}
\newcommand{\BP}[0]{\mathbf{P}}
\newcommand{\BQ}[0]{\mathbf{Q}}
\newcommand{\BR}[0]{\mathbf{R}}
\newcommand{\BS}[0]{\mathbf{S}}
\newcommand{\BT}[0]{\mathbf{T}}
\newcommand{\BU}[0]{\mathbf{U}}
\newcommand{\BV}[0]{\mathbf{V}}
\newcommand{\BW}[0]{\mathbf{W}}
\newcommand{\BX}[0]{\mathbf{X}}
\newcommand{\BY}[0]{\mathbf{Y}}
\newcommand{\BZ}[0]{\mathbf{Z}}

\newcommand{\Bzero}[0]{\mathbf{0}}
\newcommand{\Btheta}[0]{\boldsymbol{\theta}}
\newcommand{\Btau}[0]{\boldsymbol{\tau}}
\newcommand{\Bomega}[0]{\boldsymbol{\omega}}

%
% shorthand for unit vectors
%
\newcommand{\acap}[0]{\hat{\Ba}}
\newcommand{\bcap}[0]{\hat{\Bb}}
\newcommand{\ccap}[0]{\hat{\Bc}}
\newcommand{\dcap}[0]{\hat{\Bd}}
\newcommand{\ecap}[0]{\hat{\Be}}
\newcommand{\fcap}[0]{\hat{\Bf}}
\newcommand{\gcap}[0]{\hat{\Bg}}
\newcommand{\hcap}[0]{\hat{\Bh}}
\newcommand{\icap}[0]{\hat{\Bi}}
\newcommand{\jcap}[0]{\hat{\Bj}}
\newcommand{\kcap}[0]{\hat{\Bk}}
\newcommand{\lcap}[0]{\hat{\Bl}}
\newcommand{\mcap}[0]{\hat{\Bm}}
\newcommand{\ncap}[0]{\hat{\Bn}}
\newcommand{\ocap}[0]{\hat{\Bo}}
\newcommand{\pcap}[0]{\hat{\Bp}}
\newcommand{\qcap}[0]{\hat{\Bq}}
\newcommand{\rcap}[0]{\hat{\Br}}
\newcommand{\scap}[0]{\hat{\Bs}}
\newcommand{\tcap}[0]{\hat{\Bt}}
\newcommand{\ucap}[0]{\hat{\Bu}}
\newcommand{\vcap}[0]{\hat{\Bv}}
\newcommand{\wcap}[0]{\hat{\Bw}}
\newcommand{\xcap}[0]{\hat{\Bx}}
\newcommand{\ycap}[0]{\hat{\By}}
\newcommand{\zcap}[0]{\hat{\Bz}}
\newcommand{\thetacap}[0]{\hat{\Btheta}}

%
% to write R^n and C^n in a distinguishable fashion.  Perhaps change this
% to the double lined characters upon figuring out how to do so.
%
\newcommand{\C}[1]{$\mathbb{C}^{#1}$}
\newcommand{\R}[1]{$\mathbb{R}^{#1}$}

%
% various generally useful helpers
%

% derivative of #1 wrt. #2:
\newcommand{\D}[2] {\frac {d#2} {d#1}}

\newcommand{\inv}[1]{\frac{1}{#1}}
\newcommand{\cross}[0]{\times}

\newcommand{\abs}[1]{\lvert{#1}\rvert}
\newcommand{\norm}[1]{\lVert{#1}\rVert}
\newcommand{\innerprod}[2]{\langle{#1}, {#2}\rangle}
\newcommand{\dotprod}[2]{{#1} \cdot {#2}}
\newcommand{\bdotprod}[2]{\left({#1} \cdot {#2}\right)}
\newcommand{\crossprod}[2]{{#1} \cross {#2}}
\newcommand{\tripleprod}[3]{\dotprod{\left(\crossprod{#1}{#2}\right)}{#3}}

\DeclareMathOperator{\Proj}{Proj}
\DeclareMathOperator{\Span}{span}
\DeclareMathOperator{\Sgn}{sgn}
\DeclareMathOperator{\Area}{Area}
\DeclareMathOperator{\Volume}{Volume}

%
% A few miscellaneous things specific to this document
%
\newcommand{\crossop}[1]{\crossprod{#1}{}}

% R2 vector.
\newcommand{\VectorTwo}[2]{
\begin{bmatrix}
 {#1} \\
 {#2}
\end{bmatrix}
}

\newcommand{\VectorN}[1]{
\begin{bmatrix}
{#1}_1 \\
{#1}_2 \\
\vdots \\
{#1}_N \\
\end{bmatrix}
}

\newcommand{\DETuvij}[4]{
\begin{vmatrix}
 {#1}_{#3} & {#1}_{#4} \\
 {#2}_{#3} & {#2}_{#4}
\end{vmatrix}
}

\newcommand{\DETuvwijk}[6]{
\begin{vmatrix}
 {#1}_{#4} & {#1}_{#5} & {#1}_{#6} \\
 {#2}_{#4} & {#2}_{#5} & {#2}_{#6} \\
 {#3}_{#4} & {#3}_{#5} & {#3}_{#6}
\end{vmatrix}
}

\newcommand{\DETuvwxijkl}[8]{
\begin{vmatrix}
 {#1}_{#5} & {#1}_{#6} & {#1}_{#7} & {#1}_{#8} \\
 {#2}_{#5} & {#2}_{#6} & {#2}_{#7} & {#2}_{#8} \\
 {#3}_{#5} & {#3}_{#6} & {#3}_{#7} & {#3}_{#8} \\
 {#4}_{#5} & {#4}_{#6} & {#4}_{#7} & {#4}_{#8} \\
\end{vmatrix}
}

%\newcommand{\DETuvwxyijklm}[10]{
%\begin{vmatrix}
% {#1}_{#6} & {#1}_{#7} & {#1}_{#8} & {#1}_{#9} & {#1}_{#10} \\
% {#2}_{#6} & {#2}_{#7} & {#2}_{#8} & {#2}_{#9} & {#2}_{#10} \\
% {#3}_{#6} & {#3}_{#7} & {#3}_{#8} & {#3}_{#9} & {#3}_{#10} \\
% {#4}_{#6} & {#4}_{#7} & {#4}_{#8} & {#4}_{#9} & {#4}_{#10} \\
% {#5}_{#6} & {#5}_{#7} & {#5}_{#8} & {#5}_{#9} & {#5}_{#10}
%\end{vmatrix}
%}

% R3 vector.
\newcommand{\VectorThree}[3]{
\begin{bmatrix}
 {#1} \\
 {#2} \\
 {#3}
\end{bmatrix}
}



\author{Peeter Joot}
\email{peeter.joot@gmail.com}

%\documentclass[]{eliblogwidescreen}

\usepackage{amsmath}
\usepackage{mathpazo}

%
% shorthand for bold symbols, convenient for vectors and matrices
%
\newcommand{\Ba}[0]{\mathbf{a}}
\newcommand{\Bb}[0]{\mathbf{b}}
\newcommand{\Bc}[0]{\mathbf{c}}
\newcommand{\Bd}[0]{\mathbf{d}}
\newcommand{\Be}[0]{\mathbf{e}}
\newcommand{\Bf}[0]{\mathbf{f}}
\newcommand{\Bg}[0]{\mathbf{g}}
\newcommand{\Bh}[0]{\mathbf{h}}
\newcommand{\Bi}[0]{\mathbf{i}}
\newcommand{\Bj}[0]{\mathbf{j}}
\newcommand{\Bk}[0]{\mathbf{k}}
\newcommand{\Bl}[0]{\mathbf{l}}
\newcommand{\Bm}[0]{\mathbf{m}}
\newcommand{\Bn}[0]{\mathbf{n}}
\newcommand{\Bo}[0]{\mathbf{o}}
\newcommand{\Bp}[0]{\mathbf{p}}
\newcommand{\Bq}[0]{\mathbf{q}}
\newcommand{\Br}[0]{\mathbf{r}}
\newcommand{\Bs}[0]{\mathbf{s}}
\newcommand{\Bt}[0]{\mathbf{t}}
\newcommand{\Bu}[0]{\mathbf{u}}
\newcommand{\Bv}[0]{\mathbf{v}}
\newcommand{\Bw}[0]{\mathbf{w}}
\newcommand{\Bx}[0]{\mathbf{x}}
\newcommand{\By}[0]{\mathbf{y}}
\newcommand{\Bz}[0]{\mathbf{z}}
\newcommand{\BA}[0]{\mathbf{A}}
\newcommand{\BB}[0]{\mathbf{B}}
\newcommand{\BC}[0]{\mathbf{C}}
\newcommand{\BD}[0]{\mathbf{D}}
\newcommand{\BE}[0]{\mathbf{E}}
\newcommand{\BF}[0]{\mathbf{F}}
\newcommand{\BG}[0]{\mathbf{G}}
\newcommand{\BH}[0]{\mathbf{H}}
\newcommand{\BI}[0]{\mathbf{I}}
\newcommand{\BJ}[0]{\mathbf{J}}
\newcommand{\BK}[0]{\mathbf{K}}
\newcommand{\BL}[0]{\mathbf{L}}
\newcommand{\BM}[0]{\mathbf{M}}
\newcommand{\BN}[0]{\mathbf{N}}
\newcommand{\BO}[0]{\mathbf{O}}
\newcommand{\BP}[0]{\mathbf{P}}
\newcommand{\BQ}[0]{\mathbf{Q}}
\newcommand{\BR}[0]{\mathbf{R}}
\newcommand{\BS}[0]{\mathbf{S}}
\newcommand{\BT}[0]{\mathbf{T}}
\newcommand{\BU}[0]{\mathbf{U}}
\newcommand{\BV}[0]{\mathbf{V}}
\newcommand{\BW}[0]{\mathbf{W}}
\newcommand{\BX}[0]{\mathbf{X}}
\newcommand{\BY}[0]{\mathbf{Y}}
\newcommand{\BZ}[0]{\mathbf{Z}}

\newcommand{\Bzero}[0]{\mathbf{0}}
\newcommand{\Btheta}[0]{\boldsymbol{\theta}}
\newcommand{\Btau}[0]{\boldsymbol{\tau}}
\newcommand{\Bomega}[0]{\boldsymbol{\omega}}

%
% shorthand for unit vectors
%
\newcommand{\acap}[0]{\hat{\Ba}}
\newcommand{\bcap}[0]{\hat{\Bb}}
\newcommand{\ccap}[0]{\hat{\Bc}}
\newcommand{\dcap}[0]{\hat{\Bd}}
\newcommand{\ecap}[0]{\hat{\Be}}
\newcommand{\fcap}[0]{\hat{\Bf}}
\newcommand{\gcap}[0]{\hat{\Bg}}
\newcommand{\hcap}[0]{\hat{\Bh}}
\newcommand{\icap}[0]{\hat{\Bi}}
\newcommand{\jcap}[0]{\hat{\Bj}}
\newcommand{\kcap}[0]{\hat{\Bk}}
\newcommand{\lcap}[0]{\hat{\Bl}}
\newcommand{\mcap}[0]{\hat{\Bm}}
\newcommand{\ncap}[0]{\hat{\Bn}}
\newcommand{\ocap}[0]{\hat{\Bo}}
\newcommand{\pcap}[0]{\hat{\Bp}}
\newcommand{\qcap}[0]{\hat{\Bq}}
\newcommand{\rcap}[0]{\hat{\Br}}
\newcommand{\scap}[0]{\hat{\Bs}}
\newcommand{\tcap}[0]{\hat{\Bt}}
\newcommand{\ucap}[0]{\hat{\Bu}}
\newcommand{\vcap}[0]{\hat{\Bv}}
\newcommand{\wcap}[0]{\hat{\Bw}}
\newcommand{\xcap}[0]{\hat{\Bx}}
\newcommand{\ycap}[0]{\hat{\By}}
\newcommand{\zcap}[0]{\hat{\Bz}}
\newcommand{\thetacap}[0]{\hat{\Btheta}}

%
% to write R^n and C^n in a distinguishable fashion.  Perhaps change this
% to the double lined characters upon figuring out how to do so.
%
\newcommand{\C}[1]{$\mathbb{C}^{#1}$}
\newcommand{\R}[1]{$\mathbb{R}^{#1}$}

%
% various generally useful helpers
%

% derivative of #1 wrt. #2:
\newcommand{\D}[2] {\frac {d#2} {d#1}}

\newcommand{\inv}[1]{\frac{1}{#1}}
\newcommand{\cross}[0]{\times}

\newcommand{\abs}[1]{\lvert{#1}\rvert}
\newcommand{\norm}[1]{\lVert{#1}\rVert}
\newcommand{\innerprod}[2]{\langle{#1}, {#2}\rangle}
\newcommand{\dotprod}[2]{{#1} \cdot {#2}}
\newcommand{\bdotprod}[2]{\left({#1} \cdot {#2}\right)}
\newcommand{\crossprod}[2]{{#1} \cross {#2}}
\newcommand{\tripleprod}[3]{\dotprod{\left(\crossprod{#1}{#2}\right)}{#3}}

\DeclareMathOperator{\Proj}{Proj}
\DeclareMathOperator{\Span}{span}
\DeclareMathOperator{\Sgn}{sgn}
\DeclareMathOperator{\Area}{Area}
\DeclareMathOperator{\Volume}{Volume}

%
% A few miscellaneous things specific to this document
%
\newcommand{\crossop}[1]{\crossprod{#1}{}}

% R2 vector.
\newcommand{\VectorTwo}[2]{
\begin{bmatrix}
 {#1} \\
 {#2}
\end{bmatrix}
}

\newcommand{\VectorN}[1]{
\begin{bmatrix}
{#1}_1 \\
{#1}_2 \\
\vdots \\
{#1}_N \\
\end{bmatrix}
}

\newcommand{\DETuvij}[4]{
\begin{vmatrix}
 {#1}_{#3} & {#1}_{#4} \\
 {#2}_{#3} & {#2}_{#4}
\end{vmatrix}
}

\newcommand{\DETuvwijk}[6]{
\begin{vmatrix}
 {#1}_{#4} & {#1}_{#5} & {#1}_{#6} \\
 {#2}_{#4} & {#2}_{#5} & {#2}_{#6} \\
 {#3}_{#4} & {#3}_{#5} & {#3}_{#6}
\end{vmatrix}
}

\newcommand{\DETuvwxijkl}[8]{
\begin{vmatrix}
 {#1}_{#5} & {#1}_{#6} & {#1}_{#7} & {#1}_{#8} \\
 {#2}_{#5} & {#2}_{#6} & {#2}_{#7} & {#2}_{#8} \\
 {#3}_{#5} & {#3}_{#6} & {#3}_{#7} & {#3}_{#8} \\
 {#4}_{#5} & {#4}_{#6} & {#4}_{#7} & {#4}_{#8} \\
\end{vmatrix}
}

%\newcommand{\DETuvwxyijklm}[10]{
%\begin{vmatrix}
% {#1}_{#6} & {#1}_{#7} & {#1}_{#8} & {#1}_{#9} & {#1}_{#10} \\
% {#2}_{#6} & {#2}_{#7} & {#2}_{#8} & {#2}_{#9} & {#2}_{#10} \\
% {#3}_{#6} & {#3}_{#7} & {#3}_{#8} & {#3}_{#9} & {#3}_{#10} \\
% {#4}_{#6} & {#4}_{#7} & {#4}_{#8} & {#4}_{#9} & {#4}_{#10} \\
% {#5}_{#6} & {#5}_{#7} & {#5}_{#8} & {#5}_{#9} & {#5}_{#10}
%\end{vmatrix}
%}

% R3 vector.
\newcommand{\VectorThree}[3]{
\begin{bmatrix}
 {#1} \\
 {#2} \\
 {#3}
\end{bmatrix}
}



\author{Peeter Joot}
\email{peeter.joot@gmail.com}


%\usepackage[english]{babel}
%\usepackage{media9}

\chapter{Couette flow of a viscous incompressible fluid.}
\label{chap:couetteFlow}
\blogpage{http://sites.google.com/site/peeterjoot2/math2012/couetteFlow.pdf}
%\date{Apr 9, 2012}
\gitRevisionInfo{couetteFlow}
\keywords{Navier-Stokes, PHY454H1S, Couette flow, cylindrical coordinates, friction, torque} 

\beginArtWithToc
%\beginArtNoToc

\section{Motivation.}

A problem from this years phy1530 problem set 2 that appears appropriate for phy454 exam prep.

\section{Statement.}

Consider the steady flow between two long cylinders of radii $R_1$ and $R_2$, $R_1 > R_1$, rotating about their axes with angular velocities $\Omega_1$, $\Omega_2$.  Look for a solution of the form, where $\phicap$ is a unit vector along the azimuthal direction:

\begin{subequations}
\begin{equation}\label{eqn:couetteFlow:10}
\Bu = v(r) \phicap
\end{equation}
\begin{equation}\label{eqn:couetteFlow:30}
p = p(r).
\end{equation}
\end{subequations}

\begin{enumerate}
\item Write out the Navier-Stokes equations and find differential equations for $v(r)$ and $p(r)$.  You should find that these equations have relatively simple solutions, i.e.,

\begin{equation}\label{eqn:couetteFlow:50}
v(r) = a r + \frac{b}{r}.
\end{equation}

\item Fix the constants $a$ and $b$ from the boundary conditions.  Determine the pressure $p(r)$.

\item Compute the friction forces that the fluid exerts on the cylinders, and compute the torque on each cylinder.  Show that the total torque on the fluid is zero (as must be the case).
\end{enumerate}

This is also a problem that I recall was outlined in \S 2 from \cite{acheson1990elementary}.  Some of the instabilities that are mentioned in the text are nicely illustrated in \cite{wiki:taylorCouette}.

We illustrate our system in figure (\ref{fig:couetteFlow:couetteFlowFig1}).

\imageCentered{figures/couetteFlowFig1}{Couette flow configuration.}{fig:couetteFlow:couetteFlowFig1}{0.3}

\section{Solution: Part 1.  Navier-Stokes and resulting differential equations.}

Navier-Stokes for steady state incompressible flow has the form

\begin{subequations}
\begin{equation}\label{eqn:couetteFlow:70}
(\Bu \cdot \spacegrad) \Bu = -\inv{\rho} \spacegrad p + \nu \spacegrad^2 \Bu
\end{equation}
\begin{equation}\label{eqn:couetteFlow:90}
\spacegrad \cdot \Bu = 0.
\end{equation}
\end{subequations}

where the gradient has the form

\begin{equation}\label{eqn:couetteFlow:110}
\spacegrad = \rcap \partial_r + \frac{\phicap}{r} \partial_\phi.
\end{equation}

Let's first verify that the incompressible condition \ref{eqn:couetteFlow:90} is satisfied for the presumed form of the solution we seek.  We have

\begin{align*}
\spacegrad \cdot \Bu 
&=
\left( \rcap \partial_r + \frac{\phicap}{r} \partial_\phi \right) \cdot (v(r) \phicap(\phi) ) \\
&=
(\rcap \cdot \phicap) v' + \frac{\phicap^2}{r} \partial_\phi v(r)
+ \frac{v(r) \phicap}{r} \cdot \partial_\phi \phicap \\
&=
 \frac{v(r) \phicap}{r} \cdot (-\rcap) \\
&= 0
\end{align*}

Good.  Now let's write out the terms of the momentum conservation equation \ref{eqn:couetteFlow:70}.  We've got

\begin{align*}
(\Bu \cdot \spacegrad) \Bu
&=
\frac{ v}{r} \partial_\phi  ( v \phicap ) \\
&=
-\frac{ v^2 \rcap}{r},
\end{align*}

and
\begin{align*}
-\inv{\rho} \spacegrad p
&=
-\inv{\rho} \left( \rcap \partial_r + \frac{\phicap}{r} \partial_\phi \right) p(r) \\
&=
-\frac{\rcap p'}{\rho},
\end{align*}

and 

\begin{align*}
\nu \spacegrad^2 \Bu
&=
\nu 
\left( \rcap \partial_r + \frac{\phicap}{r} \partial_\phi \right) \cdot
\left( \rcap \partial_r + \frac{\phicap}{r} \partial_\phi \right) 
(v(r) \phicap(\phi)) \\
&=
\nu 
\left( 
\partial_{rr} + \inv{r^2} \partial_{\phi\phi}
+ 
\frac{\phicap}{r} \partial_\phi \cdot (\rcap \partial_r)
\right)
(v(r) \phicap(\phi)) \\
&=
\nu 
\left( 
\partial_{rr} + \inv{r^2} \partial_{\phi\phi}
+ 
\frac{1}{r} \partial_r
\right)
(v(r) \phicap(\phi)) \\
&=
\nu 
\left( 
\inv{r} \partial_{r} (r \partial_r) + \inv{r^2} \partial_{\phi\phi}
\right)
(v(r) \phicap(\phi)) \\
&=
\nu 
\left( 
\inv{r} (r v')' - \frac{v}{r^2} 
\right)
\phicap
\end{align*}

So the momentum equation of Navier-Stokes takes the form

\begin{equation}\label{eqn:couetteFlow:1310}
\boxed{
-\frac{ v^2 \rcap}{r} =
-\frac{\rcap p'}{\rho}
+
\nu 
\left( 
\inv{r} (r v')' - \frac{v}{r^2} 
\right)
\phicap.
}
\end{equation}

Equating $\rcap$ and $\phicap$ components we have two equations to solve

\begin{subequations}
\begin{equation}\label{eqn:couetteFlow:130}
r (r v')' - v = 0
\end{equation}
\begin{equation}\label{eqn:couetteFlow:150}
p' = \frac{\rho v^2}{r}.
\end{equation}
\end{subequations}

Expanding out our velocity equation we have

\begin{equation}\label{eqn:couetteFlow:170}
r^2 v'' + r v' - v = 0,
\end{equation}

for which we've been told to expect that \ref{eqn:couetteFlow:50} is a solution (and it has the two integration constants we require for a solution to a homogeneous equation of this form).  Let's verify that we've computed the correct differential equation for the problem by trying this solution

\begin{align*}
r^2 v'' + r v' - v 
&=
r^2 \left( a -\frac{b}{r^2} \right)' + r \left( a -\frac{b}{r^2} \right) - a r - \frac{b}{r} \\
&=
r^2 \frac{2 b}{r^3} + \cancel{a r} - \frac{b}{r} - \cancel{a r} - \frac{b}{r} \\
&=
\frac{2 b}{r} - \frac{2 b}{r} \\
&= 0.
\end{align*}

Given the velocity, we can now determine the pressure up to a constant

\begin{align*}
p' 
&= \frac{\rho}{r} \left( a r + \frac{b}{r} \right)^2 \\
&= \frac{\rho}{r} \left( a^2 r^2 + \frac{b^2}{r^2} + 2 a b \right) \\
&= \rho \left( a^2 r + \frac{b^2}{r^3} + 2 \frac{a b}{r} \right)
\end{align*}

so
\begin{equation}\label{eqn:couetteFlow:190}
p_r -p_0
= \rho \left( \inv{2} a^2 r^2 - \frac{b^2}{2 r^2} + 2 a b \ln r \right)
\end{equation}

\section{Solution: Part 2.  Fixing the constants.}

To determine our integration constants we recall that velocity associated with a radial position $\Bx = r \rcap$ in cylindrical coordinates takes the form

\begin{equation}\label{eqn:couetteFlow:210}
\frac{\Bx}{dt} = \rdot \rcap + r \phicap \phidot,
\end{equation}

where $\phidot$ is the angular velocity.  The cylinder walls therefore have the velocity

\begin{equation}\label{eqn:couetteFlow:230}
v = r \phidot,
\end{equation}

so our boundary conditions (given a no-slip assumption for the fluids) are

\begin{align}\label{eqn:couetteFlow:250}
v(R_1) &= R_1 \Omega_1 \\
v(R_2) &= R_2 \Omega_2.
\end{align}

This gives us a pair of equations to solve for $a$ and $b$

\begin{align}\label{eqn:couetteFlow:270}
R_1 \Omega_1 &= a R_1 + \frac{b}{R_1} \\
R_2 \Omega_2 &= a R_2 + \frac{b}{R_2}.
\end{align}

Multiplying each by $R_1$ and $R_2$ respectively gives us

\begin{equation}\label{eqn:couetteFlow:290}
b = R_1^2 (\Omega_1 - a) = R_2^2 (\Omega_2 - a).
\end{equation}

Rearranging for $a$ we find

\begin{equation}\label{eqn:couetteFlow:310}
R_1^2 \Omega_1 - R_2^2 \Omega_2 = (R_1^2 - R_2^2) a,
\end{equation}

or

\begin{equation}\label{eqn:couetteFlow:330}
a = \frac{ R_2^2 \Omega_2 - R_1^2 \Omega_1}{R_2^2 - R_1^2}.
\end{equation}

For $b$ we have

\begin{align*}
b &= 
R_1^2 (\Omega_1 - a) \\
&=
\frac{R_1^2 }{R_2^2 - R_1^2}
(\Omega_1 ( R_2^2 - \cancel{R_1^2}) - 
R_2^2 \Omega_2 + \cancel{R_1^2 \Omega_1}
),
\end{align*}

or

\begin{equation}\label{eqn:couetteFlow:350}
b = \frac{R_1^2 R_2^2}{R_2^2 - R_1^2} (\Omega_1 -\Omega_2).
\end{equation}

This gives us

\begin{subequations}
\begin{equation}\label{eqn:couetteFlow:370}
v(r) = 
\inv{R_2^2 - R_1^2}
\left(
\left( R_2^2 \Omega_2 - R_1^2 \Omega_1\right) r
+\frac{R_1^2 R_2^2}{r} (\Omega_1 -\Omega_2)
\right)
\end{equation}
\begin{equation}\label{eqn:couetteFlow:390}
\begin{aligned}
p(r) -&p_0
= \frac{\rho }{(R_2^2 - R_1^2)^2} \times \\
&\left( \inv{2} 
\left( R_2^2 \Omega_2 - R_1^2 \Omega_1\right)^2
r^2 
-\frac{R_1^4 R_2^4}{2 r^2} (\Omega_1  - \Omega_2)^2
+ 2 \left( R_2^2 \Omega_2 - R_1^2 \Omega_1\right) R_1^2 R_2^2 (\Omega_1 - \Omega_2) \ln r
\right).
\end{aligned}
\end{equation}
\end{subequations}

FIXME: This is almost a complete solution.  The part that I am unsure about is how to fix the $p_0$ integration constant.  In the solution of this problem posted from the course this was just set to 0, but I don't see a good reason for that.  I'll try asking this on \href{http://www.physicsforums.com/showthread.php?t=595132}{physicsforums} and see if I can get some help there.

\section{Solution: Part 3.  Friction and torque.}

We can expand out the identity for the traction vector

\begin{equation}\label{eqn:couetteFlow:970}
\Bt_{\ncap}
= \Be_i \sigma_{ij} n_j
= -p \ncap + \mu \left( 
2 (\ncap \cdot \spacegrad) \Bu + \ncap \cross (\spacegrad \cross \Bu)
\right),
\end{equation}

in cylindrical coordinates and find

\begin{subequations}
\begin{equation}\label{eqn:couetteFlow:990}
\Bt_{\rcap} \cdot \rcap 
= 
\sigma_{rr}
=
-p + 2 \mu \cancel{\PD{r}{u_r}}
\end{equation}
\begin{equation}\label{eqn:couetteFlow:1010}
\Bt_{\phicap} \cdot \phicap
= 
\sigma_{\phi \phi}
=
-p + 2 \mu 
\left(
\inv{r}
\cancel{\PD{\phi}{u_\phi}} + \cancel{\frac{u_r}{r}}
\right)
\end{equation}
\begin{equation}\label{eqn:couetteFlow:1030}
\Bt_{z} \cdot \zcap
= 
\sigma_{z z}
=
-p + 2 \mu 
\cancel{\PD{z}{u_z}}
\end{equation}
\begin{equation}\label{eqn:couetteFlow:1050}
\Bt_{\rcap} \cdot \phicap
= 
\sigma_{r \phi}
=
\mu \left(
 \PD{r}{u_\phi}
+\inv{r} \cancel{\PD{\phi}{u_r}}
- \frac{u_\phi}{r}
\right)
\end{equation}
\begin{equation}\label{eqn:couetteFlow:1070}
\Bt_{\phicap} \cdot \zcap
= 
\sigma_{\phi z}
=
\mu \left(
\frac{1}{r} \cancel{\PD{\phi}{u_z}}
    + \cancel{\PD{z}{u_\phi}}
\right)
\end{equation}
\begin{equation}\label{eqn:couetteFlow:950}
\Bt_{\zcap} \cdot \rcap
= 
\sigma_{z r}
=
\mu \left(
\cancel{\PD{z}{u_r}}
+ \cancel{\PD{r}{u_z}}
\right),
\end{equation}
\end{subequations}

so we have

\begin{subequations}
\begin{equation}\label{eqn:couetteFlow:1090}
\sigma_{rr} = \sigma_{\phi \phi} = \sigma_{z z} = -p 
\end{equation}
\begin{equation}\label{eqn:couetteFlow:1110}
\sigma_{\phi z} = \sigma_{z r} = 0
\end{equation}
\begin{equation}\label{eqn:couetteFlow:1130}
\sigma_{r \phi} = \mu \left( \PD{r}{u_\phi} - \frac{u_\phi}{r} \right)
\end{equation}
\end{subequations}

We want to expand the last of these

\begin{align*}
\sigma_{r \phi} 
&= \mu \left( \PD{r}{u_\phi} - \frac{u_\phi}{r} \right) \\
&= \mu \left( 
a r + \frac{b}{r}
\right)' \\
&= \mu \left( 
a - \frac{b}{r^2}
\right).
\end{align*}

So the traction vector $\Bt_1 = \Bsigma \cdot \rcap = \Be_i \sigma_{ij} \rcap \cdot \Be_i$, our force per unit area on the fluid at the inner surface (where the normal is $\rcap$), is

\begin{equation}\label{eqn:couetteFlow:1150}
\Bt_1 = 
-p \rcap + \mu \left( a - \frac{b}{r^2} \right) \phicap
=
-p \rcap +
\frac{\mu}{R_2^2 - R_1^2}
\left(
R_2^2 \Omega_2 - R_1^2 \Omega_1
+\frac{R_1^2 R_2^2}{r^2} (\Omega_2 -\Omega_1)
\right) \phicap.
\end{equation}

and our torque per unit area from the inner cylinder on the fluid is thus

\begin{equation}\label{eqn:couetteFlow:1170}
\Btau_1 = r \rcap \cross \Bt_1 = 
%r \mu \left( a - \frac{b}{r^2} \right) \zcap.
\frac{r \mu}{R_2^2 - R_1^2}
\left(
R_2^2 \Omega_2 - R_1^2 \Omega_1
+\frac{R_1^2 R_2^2}{r^2} (\Omega_2 -\Omega_1)
\right) \zcap.
\end{equation}

Observing that our stress tensors flip sign for an inwards normal, our torque per unit area from the outer cylinder on the fluid is

\begin{equation}\label{eqn:couetteFlow:1190}
\Btau_2 = r \rcap \cross (-\Bt_1) = 
%-r \mu \left( a - \frac{b}{r^2} \right) \zcap.
-\frac{r \mu}{R_2^2 - R_1^2}
\left(
R_2^2 \Omega_2 - R_1^2 \Omega_1
+\frac{R_1^2 R_2^2}{r^2} (\Omega_2 -\Omega_1)
\right) \zcap.
\end{equation}

For the complete torque on the fluid due to a strip of width $\Delta z$ the magnitudes of the total torque from each cylinder are respectively

\begin{equation}\label{eqn:couetteFlow:1210}
\Btau_1 = 
%2 \pi r^2 \Delta z \mu \left( a - \frac{b}{r^2} \right)
\frac{2 \pi r^2 \Delta z \mu}{R_2^2 - R_1^2}
\left(
R_2^2 \Omega_2 - R_1^2 \Omega_1
+\frac{R_1^2 R_2^2}{r^2} (\Omega_2 -\Omega_1)
\right) \zcap.
\end{equation}
\begin{equation}\label{eqn:couetteFlow:1230}
\Btau_2 = 
%- 2 \pi r^2 \Delta z \mu \left( a - \frac{b}{r^2} \right)
-\frac{2 \pi r^2 \Delta z \mu}{R_2^2 - R_1^2}
\left(
R_2^2 \Omega_2 - R_1^2 \Omega_1
+\frac{R_1^2 R_2^2}{r^2} (\Omega_2 -\Omega_1)
\right) \zcap.
\end{equation}

As expected these torques on the fluid sum to zero

\begin{equation}\label{eqn:couetteFlow:1250}
\Btau_2 + \Btau_1 = 0.
\end{equation}

Evaluating these at $R_1$ and $R_2$ respectively gives us the torques on the fluid by the cylinders.  However, we want the torques on the cylinders by the fluid, so have to flip the signs.  For the inner cylinder the total torque on a strip of width $\Delta z$ by the fluid is

\begin{equation}\label{eqn:couetteFlow:1270}
\begin{aligned}
\text{Torque on inner cylinder (1) by the fluid} 
&= 
%-(2 \pi R_1^2) \Delta z \mu \left( a - \frac{b}{R_1^2} \right).
-\frac{2 \pi R_1^2 \Delta z \mu}{R_2^2 - R_1^2}
\left(
R_2^2 \Omega_2 - R_1^2 \Omega_1
+\frac{R_1^2 R_2^2}{R_1^2} (\Omega_2 -\Omega_1)
\right) \\
&=
\frac{2 \pi R_1^2 \Delta z \mu}{R_2^2 - R_1^2}
\left(
-2 R_2^2 \Omega_2 + (R_1^2 + R_2^2) \Omega_1
\right).
\end{aligned}
\end{equation}

For the outer cylinder the total torque on a strip of width $\Delta z$ by the fluid is

\begin{equation}\label{eqn:couetteFlow:1290}
\begin{aligned}
\text{Torque on outer cylinder (2) by the fluid} 
&= 
%(2 \pi R_2^2) \Delta z \mu \left( a - \frac{b}{R_2^2} \right).
\frac{2 \pi R_2^2 \Delta z \mu}{R_2^2 - R_1^2}
\left(
R_2^2 \Omega_2 - R_1^2 \Omega_1
+\frac{R_1^2 R_2^2}{R_2^2} (\Omega_2 -\Omega_1) 
\right) \\
&=
\frac{2 \pi R_2^2 \Delta z \mu}{R_2^2 - R_1^2}
\left(
-2 R_1^2 \Omega_1 + (R_1^2 + R_2^2) \Omega_2
\right).
\end{aligned}
\end{equation}

\section{Plotting the solutions.}

Here's some plots of the velocities at different values for the outer cylinder angular velocity

%figure (\ref{fig:couetteFlow:couetteFlowFig2}).
\imageCentered{figures/couetteFlowFig2}{Couette flow plot}{fig:couetteFlow:couetteFlowFig2}{0.2}
%figure (\ref{fig:couetteFlow:couetteFlowFig3}).
\imageCentered{figures/couetteFlowFig3}{Couette flow plot}{fig:couetteFlow:couetteFlowFig3}{0.2}
%figure (\ref{fig:couetteFlow:couetteFlowFig4}).
\imageCentered{figures/couetteFlowFig4}{Couette flow plot}{fig:couetteFlow:couetteFlowFig4}{0.2}
%figure (\ref{fig:couetteFlow:couetteFlowFig5}).
\imageCentered{figures/couetteFlowFig5}{Couette flow plot}{fig:couetteFlow:couetteFlowFig5}{0.2}

For Acrobat viewers of this document, 
An animation of the above is available at (FIXME: animation removed.  Was this one on youtube?  Put there and link.)
%\ref{fig:couetteFlow:couetteFlowFig6}

%\movieFigure{couetteFlowFig6.mp4}{Animation of Couette flow, with continuous variation of outer angular velocity.}{fig:couetteFlow:couetteFlowFig6}{width=320pt,height=240pt}

These were all generated from the Mathematica workbook \href{https://raw.github.com/peeterjoot/physicsplay/master/notes/phy454/mathematica/couetteFlow.cdf}{couetteFlow.cdf}, which has some slider controls that can be used to play with the radii and angular velocities in an interactive fashion.

%FIXME: could graph pressure and torques too.

\EndArticle

   %
%
%
% Copyright � 2012 Peeter Joot
% All Rights Reserved
%
% This file may be reproduced and distributed in whole or in part, without fee, subject to the following conditions:
%
% o The copyright notice above and this permission notice must be preserved complete on all complete or partial copies.
%
% o Any translation or derived work must be approved by the author in writing before distribution.
%
% o If you distribute this work in part, instructions for obtaining the complete version of this file must be included, and a means for obtaining a complete version provided.
%
%
% Exceptions to these rules may be granted for academic purposes: Write to the author and ask.
%
%
%
\begin{Exercise}[
title={Flow between infinite cylinders, one moving},
label={problem:fluids:twoCylinders}
]

A problem from the 2009 phy1530 final.

An infinite cylinder of radius $R_1$ is moving with velocity $v$ parallel to its axis.  It is placed inside another cylinder of radius $R_2$.  The axes of the two cylinders coincide.  The fluid is incompressible, with viscosity $\mu$ and density $\rho$, the flow is assumed to be stationary, and no external pressure gradient is applied.

\label{problem:fluids:twoCylinders:1}
\Question{Find and sketch the velocity field of the fluid between the cylinders.}
\label{problem:fluids:twoCylinders:2}
\Question{Find the friction force per unit length acting on each cylinder.}
\label{problem:fluids:twoCylinders:3}
\Question{Find and sketch the pressure field of the liquid.}
\label{problem:fluids:twoCylinders:4}
\Question{If an external pressure gradient is present, how do you think your answer will change?  Sketch your expectation for the velocity and pressure in this case.}
\end{Exercise}

\begin{Answer}[ref={problem:fluids:twoCylinders}]
\ref{problem:fluids:twoCylinders:1}
\ref{problem:fluids:twoCylinders:3}
\paragraph{Velocity and pressure}

Let's start with the illustration of figure (\ref{fig:twoCylinder:twoCylinderFig1}) to fix coordinates.

\imageFigure{figures/twoCylinderFig1}{Coordinates for flow between two cylinders.}{fig:twoCylinder:twoCylinderFig1}{0.2}

We'll assume that we can find a solution of the following form

\begin{subequations}
\begin{equation}\label{eqn:twoCylinder:10}
\Bu = w(r) \zcap
\end{equation}
\begin{equation}\label{eqn:twoCylinder:30}
p = p(r).
\end{equation}
\end{subequations}

We'll also work in cylindrical coordinates where our gradient is

\begin{equation}\label{eqn:twoCylinder:50}
\spacegrad = \rcap \partial_r + \frac{\phicap}{r} \partial_\phi + \zcap \partial_z.
\end{equation}

Let's look at the various terms of the Navier-Stokes equation.  Our non-linear term is

\begin{equation}\label{eqn:twoCylinder:70}
\Bu \cdot \spacegrad \Bu = w \partial_z ( w(r) \zcap ) = 0,
\end{equation}

Our Laplacian term is

\begin{align*}
\mu \spacegrad^2 \Bu
&= \mu \left( \inv{r} \partial_r ( r \partial_r ) + \inv{r^2} \partial_{\phi\phi} + \partial_{z z} \right) w(r) \zcap \\
&=
\frac{\mu}{r} ( r w' )' \zcap.
\end{align*}

Putting the pieces together we have

\begin{equation}\label{eqn:twoCylinder:90}
0 = - \rcap p' + \frac{\mu}{r} (r w')' \zcap.
\end{equation}

Decomposing these into one equation for each component we have

\begin{equation}\label{eqn:twoCylinder:110}
p' = 0,
\end{equation}

and

\begin{equation}\label{eqn:twoCylinder:130}
(r w')' = 0.
\end{equation}

The pressure can be trivially solved

\begin{equation}\label{eqn:twoCylinder:150}
p(r) = \text{constant},
\end{equation}

and for our velocity equation we get

\begin{equation}\label{eqn:twoCylinder:170}
r w' = A,
\end{equation}

Short of satisfying our boundary value constraints our velocity is

\begin{equation}\label{eqn:twoCylinder:190}
w = A \ln r + B.
\end{equation}

Our boundary value conditions are given by

\begin{align}\label{eqn:twoCylinder:210}
w(R_2) &= 0 \\
w(R_1) &= v,
\end{align}

so our integration constants are given by

\begin{align}\label{eqn:twoCylinder:230}
0 &= A \ln R_2 + B \\
v &= A \ln R_1 + B.
\end{align}

Taking differences we've got

\begin{equation}\label{eqn:twoCylinder:250}
v = A \ln( R_1/R_2 ).
\end{equation}

So our constants are

\begin{subequations}
\begin{equation}\label{eqn:twoCylinder:270}
A = \frac{v}{ \ln( R_1/R_2 ) }
\end{equation}
\begin{equation}\label{eqn:twoCylinder:290}
B = -\frac{v \ln R_2}{ \ln( R_1/R_2 ) },
\end{equation}
\end{subequations}

and

\begin{equation}\label{eqn:twoCylinder:310}
\boxed{
w(r) = \frac{ v \ln (r/R_2) }{ \ln( R_1/R_2 ) }.
}
\end{equation}

A plot of this function can be found in figure (\ref{fig:twoCylinder:twoCylinderFig2}), and the Mathematica notebook that generated this plot can be found in \href{https://raw.github.com/peeterjoot/physicsplay/master/notes/phy454/mathematica/twoCylinders.cdf}{twoCylinders.cdf}.  That notebook has some slider controls that can be used interactively.  A sample animation of that interactive capability is available at
\FIXME{upload to youtube and link}
%included in figure (\ref{fig:twoCylinders:twoCylindersFig3}).

\imageFigure{figures/twoCylinderFig2}{Velocity plot due to inner cylinder dragging fluid along with it.}{fig:twoCylinder:twoCylinderFig2}{0.2}

%\movieFigure{twoCylindersFig3.mp4}{Animating the two cylinder velocity field for a set of parameter values.}{fig:twoCylinders:twoCylindersFig3}{width=298pt,height=320pt}

\ref{problem:fluids:twoCylinders:2}
\paragraph{Friction force per unit length.}

For the frictional force per unit area on the fluid by the inner cylinder we have

\begin{align*}
(\Bsigma \cdot \rcap ) \cdot \zcap
&=
-p \rcap \cdot \zcap +
2 \mu \inv{2} \left(
\PD{r}{u_z}
+\cancel{\PD{z}{u_r}}
\right) \\
&=
\mu v \frac{\ln r }{\ln (R_1/R_2) }
\end{align*}

So the forces on the inner and outer cylinders for a strip of width $\Delta z$ is

\begin{subequations}
\begin{equation}\label{eqn:twoCylinder:330}
\text{frictional force on inner cylinder}
= -2 \pi R_1 \Delta z \mu v \zcap \frac{\ln R_1 }{\ln (R_1/R_2) }
\end{equation}
\begin{equation}\label{eqn:twoCylinder:350}
\text{frictional force on inner cylinder}
=
2 \pi R_2 \Delta z \mu v \zcap \frac{\ln R_2 }{\ln (R_1/R_2) }
\end{equation}
\end{subequations}

\ref{problem:fluids:twoCylinders:4}
\paragraph{With external pressure gradient.}

With an external pressure gradient imposed we expect a superposition of a parabolic flow profile with what we've calculated above.  With

\begin{equation}\label{eqn:twoCylinder:370}
G = - \frac{dp}{dz},
\end{equation}

our Navier-Stokes equation will now take the form

\begin{equation}\label{eqn:twoCylinder:390}
0 = - \rcap p' - (-G \zcap) + \frac{\mu}{r} (r w')' \zcap.
\end{equation}

We want to solve the LDE

\begin{equation}\label{eqn:twoCylinder:410}
- \frac{ G r}{\mu} =
(r w')' = r w'' + w'
\end{equation}

The homogeneous portion of this equation

\begin{equation}\label{eqn:twoCylinder:430}
(r w')' = 0,
\end{equation}

we have already solved finding $w = C \ln r + D$.  It looks reasonable to try a polynomial solution for the specific solution.  Let's try a second order polynomial

\begin{subequations}
\begin{equation}\label{eqn:twoCylinder:450}
w = A r^2 + B r
\end{equation}
\begin{equation}\label{eqn:twoCylinder:470}
w' = 2 A r + B
\end{equation}
\begin{equation}\label{eqn:twoCylinder:490}
w'' = 2 A.
\end{equation}
\end{subequations}

We need

\begin{equation}\label{eqn:twoCylinder:510}
-\frac{G r}{\mu} = 2 A r + 2 A r + B,
\end{equation}

So $B = 0$ and $4 A = -G/\mu$, and our general solution has the form

\begin{equation}\label{eqn:twoCylinder:530}
w = -\frac{G}{4 \mu} r^2 + C \ln r + D.
\end{equation}

requiring just the boundary condition fitting.  Let's tweak the constants slightly, writing

\begin{equation}\label{eqn:twoCylinder:550}
w = \frac{G}{4 \mu} (R_2^2 - r^2) + C \ln r/R_2 + D,
\end{equation}

so that $D = 0$ falls out of the $w(R_2) = 0$ constraint.  Our last integration constant is then determined by the solution of

\begin{equation}\label{eqn:twoCylinder:570}
v = \frac{G}{4 \mu} (R_2^2 - R_1^2) + C \ln R_1/R_2.
\end{equation}

Or

\begin{equation}\label{eqn:twoCylinder:590}
\boxed{
w = \frac{G}{4 \mu} (R_2^2 - r^2) +
\left(v - \frac{G}{4 \mu} (R_2^2 - R_1^2)\right)
\frac{\ln r/R_2}{\ln R_1/R_2}.
}
\end{equation}

A plot of this, with a pressure gradient small enough that we still see the logarithmic profile is shown in figure (\ref{fig:twoCylinder:twoCylinderFig4}).  An animation of this with different values for $R_1$, $v$, and $G/4\mu$ is available on \youtubehref{BNgpnYeRpLo}, but the Mathematica notebook above can also be used.

\imageFigure{figures/twoCylinderFig4}{Pressure gradient added.}{fig:twoCylinder:twoCylinderFig4}{0.2}

% Too big and two slow to embed:
%figure (\ref{fig:twoCylinders:twoCylindersFig5}).
%\movieFigure{twoCylindersFig5.flv}{5: FIXME: CAPTION}{fig:twoCylinders:twoCylindersFig5}{width=320pt,height=240pt}

Even cooler is to look at some plots of the velocity profiles in 3D

%figure (\ref{fig:twoCylinders:twoCylindersFig6}).
\imageFigure{figures/twoCylindersFig6}{3D plot 1}{fig:twoCylinders:twoCylindersFig6}{0.2}
%figure (\ref{fig:twoCylinders:twoCylindersFig7}).
\imageFigure{figures/twoCylindersFig7}{3D plot 2}{fig:twoCylinders:twoCylindersFig7}{0.2}
%figure (\ref{fig:twoCylinders:twoCylindersFig8}).
\imageFigure{figures/twoCylindersFig8}{3D plot 3}{fig:twoCylinders:twoCylindersFig8}{0.2}
%figure (\ref{fig:twoCylinders:twoCylindersFig9}).
\imageFigure{figures/twoCylindersFig9}{3D plot 4}{fig:twoCylinders:twoCylindersFig9}{0.2}

An animation of this can be found at \youtubehref{OiJTopWx7L8}, and the Mathematica notebook that generated it at \href{https://raw.github.com/peeterjoot/physicsplay/master/notes/phy454/mathematica/twoCylinders3D.cdf}{twoCylinders3D.cdf}.  That notebook is now also available online on the Wolfram demonstrations project \citep{wolfram3DcylinderFlow}.

%\movieFigure{twoCylindersFig11ve.mp4}{FIXME: CAPTION}{fig:twoCylindersFig11ve:twoCylindersFig11ve}{width=746pt,height=608pt}
%\movieFigure{twoCylindersFig11m.mp4}{FIXME: CAPTION}{fig:twoCylindersFig11m:twoCylindersFig11m}{width=546pt,height=408pt}
%\movieFigure{twoCylindersFig11f.mp4}{FIXME: CAPTION}{fig:twoCylindersFig11f:twoCylindersFig11f}{width=546pt,height=408pt}
%\movieFigure{twoCylindersFig11a.mp4}{FIXME: CAPTION}{fig:twoCylindersFig11a:twoCylindersFig11a}{width=546pt,height=412pt}

\end{Answer}

   %
% Copyright � 2012 Peeter Joot.  All Rights Reserved.
% Licenced as described in the file LICENSE under the root directory of this GIT repository.
%
\makeproblem{Spin down of coffee in a bottomless cup}{problem:fluids:bottomlessCoffee}{
Here is a variation of a problem outlined in \S 2 of \citep{acheson1990elementary}, which looked at the time evolution of fluid with initial rotational motion, after the (cylindrical) rotation driver stops, later describing this as the spin down of a cup of tea.  I will work the problem in more detail than in the text, and also make two refinements.

\begin{enumerate}
\item I drink coffee and not tea.
\item I stir my coffee in the interior of the cup and not on the outer edge.
\end{enumerate}

Because of the second point I will model my stir stick as a rotating cylinder in the cup and not by somebody spinning the cup itself to stir the tea.  This only changes the solution for the steady state part of the problem.
} % makeproblem

\makeanswer{problem:fluids:bottomlessCoffee}{
We will work in cylindrical coordinates following the conventions of \cref{fig:coffeeCupDiagram:coffeeCupDiagramFig1}.

\imageFigure{../../figures/phy454/bottomlessCoffeeFluid_flow_in_nested_cylindersFig1}{Fluid flow in nested cylinders}{fig:coffeeCupDiagram:coffeeCupDiagramFig1}{0.3}

We will assume a solution that with velocity azimuthal in direction, and both pressure and velocity that are only radially dependent.

\begin{equation}\label{eqn:bottomlessCoffee:10}
\Bu = u(r) \phicap.
\end{equation}
\begin{equation}\label{eqn:bottomlessCoffee:30}
p = p(r)
\end{equation}

Let us first verify that this meets the non-compressible condition that eliminates the \(\mu \spacegrad (\spacegrad \cdot \Bu)\) term from Navier-Stokes

\begin{equation}\label{eqn:bottomlessCoffee:590}
\begin{aligned}
\spacegrad \cdot \Bu
&=
\left(\rcap \partial_r + \frac{\phicap}{r} \partial_\phi + \zcap \partial_z\right) \cdot \left(u \phicap\right) \\
&=
\phicap \cdot
\left(\rcap \partial_r u + \frac{\phicap}{r} \partial_\phi u + \zcap \partial_z u\right)
+
u
\left(\rcap \cdot \partial_r \phicap + \frac{\phicap}{r} \cdot \partial_\phi \phicap + \zcap \cdot \partial_z \phicap\right)  \\
&=
\phicap \cdot \rcap \partial_r u
+
u
\frac{\phicap}{r} \cdot \left(-\rcap\right) \\
&= 0.
\end{aligned}
\end{equation}

Good.  Now let us express each of the terms of Navier-Stokes in cylindrical form.  Our time dependence is

\begin{equation}\label{eqn:bottomlessCoffee:50}
\rho \partial_t u(r, t) \phicap
=
\rho \phicap \partial_t u.
\end{equation}

Our inertial term is


\begin{dmath}\label{eqn:bottomlessCoffee:70}
\rho (\Bu \cdot \spacegrad) \Bu
=
\frac{\rho u}{r} \partial_\phi (u \phicap)
=
\frac{\rho u^2}{r} (-\rcap).
\end{dmath}

Our pressure term is

\begin{equation}\label{eqn:bottomlessCoffee:90}
-\spacegrad p
=
-\rcap \partial_r p,
\end{equation}

and our Laplacian term is


\begin{dmath}\label{eqn:bottomlessCoffee:110}
\mu \spacegrad^2 \Bu
=
\mu \left(
\inv{r} \partial_r ( r \partial_r) + \inv{r^2} \partial_{\phi\phi} + \partial_{z z}
\right)
u(r) \phicap
=
\mu \left(
\frac{\phicap}{r} \partial_r ( r \partial_r u) + \frac{-\rcap u}{r^2}
\right).
\end{dmath}

Putting things together, we find that Navier-Stokes takes the form

\begin{equation}\label{eqn:bottomlessCoffee:130}
\rho \phicap \partial_t u
+
\frac{\rho u^2}{r} (-\rcap)
=
-\rcap \partial_r p
+
\mu \left(
\frac{\phicap}{r} \partial_r ( r \partial_r u) + \frac{-\phicap u}{r^2}
\right),
\end{equation}

which nicely splits into an separate equations for the \(\phicap\) and \(\rcap\) directions respectively

\begin{subequations}
\begin{equation}\label{eqn:bottomlessCoffee:150}
\inv{\nu} \partial_t u
=
\frac{1}{r} \partial_r ( r \partial_r u)
- \frac{u}{r^2}
\end{equation}
\begin{equation}\label{eqn:bottomlessCoffee:170}
\frac{\rho u^2}{r}
=
\partial_r p.
\end{equation}
\end{subequations}

%\unnumberedSubsection{Steady state solution}

Before \(t = 0\) we seek the steady state, the solution of

\begin{equation}\label{eqn:bottomlessCoffee:190}
r \partial_r ( r \partial_r u) - u = 0.
\end{equation}

We have seen that

\begin{equation}\label{eqn:bottomlessCoffee:210}
u(r) = A r + \frac{B}{r}
\end{equation}

is the general solution, and can now fit this to the boundary value constraints.  For the interior portion of the cup we have

\begin{equation}\label{eqn:bottomlessCoffee:230}
\evalbar{A r + \frac{B}{r}}{r = 0} = 0
\end{equation}

so \(B = 0\) is required.  For the interface of the ``stir-stick'' (moving fast enough that we can consider it having a cylindrical effect) at \(r = R_1\) we have

\begin{equation}\label{eqn:bottomlessCoffee:250}
A R_1 = \Omega R_1,
\end{equation}

so the interior portion of our steady state coffee velocity is just

\begin{equation}\label{eqn:bottomlessCoffee:270}
\Bu = \Omega r \phicap.
\end{equation}

Between the cup edge and the stir-stick we have to solve

\begin{equation}\label{eqn:bottomlessCoffee:290}
\begin{aligned}
A R_1 + \frac{B}{R_1} &= \Omega R_1 \\
A R_2 + \frac{B}{R_2} &= 0,
\end{aligned}
\end{equation}

or
\begin{equation}\label{eqn:bottomlessCoffee:310}
\begin{aligned}
A R_1^2 + B &= \Omega R_1^2 \\
A R_2^2 + B &= 0.
\end{aligned}
\end{equation}

Subtracting we find

\begin{subequations}
\begin{equation}\label{eqn:bottomlessCoffee:330}
A = -\frac{\Omega R_1^2}{R_2^2 - R_1^2}
\end{equation}
\begin{equation}\label{eqn:bottomlessCoffee:350}
B = \frac{\Omega R_1^2 R_2^2}{R_2^2 - R_1^2},
\end{equation}
\end{subequations}

so our steady state coffee flow is
\begin{equation}\label{eqn:bottomlessCoffee:370}
\Bu =
\left\{
\begin{array}{l l}
\Omega r \phicap
& \quad \mbox{\(r \in [0, R_1]\)} \\
\frac{\Omega R_1^2}{R_2^2 - R_1^2}
\left(
\frac{R_2^2}{r} -r
\right)
\phicap
& \quad \mbox{\(r \in [R_1, R_2]\)} \\
\end{array}
\right.
\end{equation}

%\unnumberedSubsection{Time evolution}

We can use a separation of variables technique with \(u(r, t) = R(r) T(t)\) to find the time evolution

\begin{equation}\label{eqn:bottomlessCoffee:390}
\inv{\nu} \frac{T'}{T} =
\inv{R} \left(
\frac{1}{r} \partial_r ( r \partial_r R)
- \frac{R}{r^2}
\right)
= -\lambda^2,
\end{equation}

which gives us

\begin{equation}\label{eqn:bottomlessCoffee:410}
T \propto e^{-\lambda^2 \nu t},
\end{equation}

and \(R\) specified by

\begin{equation}\label{eqn:bottomlessCoffee:430}
0 = r^2 \frac{d^2 R}{dr^2} + r \frac{d R}{dr} + R \left( r^2 \lambda^2 - 1 \right).
\end{equation}

Checking \citep{abramowitz1964handbook} (9.1.1) we see that this can be put into the standard form of the Bessel equation if we eliminate the \(\lambda\) term.  We can do that writing \(z = r \lambda\), \(\calR(z) = R(z/\lambda)\) and noting that \(r d/dr = z d/dz\) and \(r^2 d^2/dr^2 = z^2 d^2/dz^2\), which gives us

\begin{equation}\label{eqn:bottomlessCoffee:450}
0 = z^2 \frac{d^2 \calR}{dr^2} + z \frac{d \calR}{dr} + \calR \left( z^2 - 1 \right).
\end{equation}

The solutions are

\begin{equation}\label{eqn:bottomlessCoffee:470}
\calR(z) = J_{\pm 1}(z), Y_{\pm 1}(z).
\end{equation}

From (9.1.5) of the handbook we see that the plus and minus variations are linearly dependent since \(J_{-1}(z) = -J_1(z)\) and \(Y_{-1}(z) = -Y_1(z)\), and from (9.1.8) that \(Y_1(z)\) is infinite at the origin, so our general solution has to be of the form

\begin{equation}\label{eqn:bottomlessCoffee:490}
\Bu(r, t) = \phicap \sum_\lambda c_\lambda e^{-\lambda^2 \nu t} J_{1}(r \lambda).
\end{equation}

In the text, I see that the transformation \(\lambda \rightarrow \lambda/a\) (where \(a\) was the radius of the cup) is made so that the Bessel function parameter was dimensionless.  We can do that too but write

\begin{equation}\label{eqn:bottomlessCoffee:490b}
\Bu(r, t) = \phicap \sum_\lambda c_\lambda e^{-\frac{\lambda^2}{R_2^2} \nu t} J_{1}\left(\lambda \frac{r}{R_2}\right).
\end{equation}

Our boundary value constraint is that we require this to match \eqnref{eqn:bottomlessCoffee:370} at \(t = 0\).  Let us write \(R_2 = R\), \(R_1 = a R\), \(z = r/R\), so that we are working in the unit circle with \(z \in [0, 1]\).  Our boundary problem can now be expressed as

\begin{equation}\label{eqn:bottomlessCoffee:490c}
\inv{\Omega R} \sum_\lambda c_\lambda J_{1}\left(\lambda z\right)
=
\left\{
\begin{array}{l l}
z
& \quad \mbox{\(z \in [0, a]\)} \\
\frac{1}{\frac{R^2}{a^2} - 1}
\left(
\inv{z} - z
\right)
& \quad \mbox{\(z \in [a, 1]\)} \\
\end{array}
\right.
\end{equation}

Let us pull the \(\Omega R\) factor into \(c_\lambda\) and state the problem to be solved as

\begin{subequations}
\begin{equation}\label{eqn:bottomlessCoffee:510}
\Bu(r, t) = \Omega R \phicap \sum_{i=1}^n c_i e^{-\frac{\lambda_i^2}{R^2} \nu t} J_{1}\left(\lambda_i \frac{r}{R}\right)
\end{equation}
\begin{equation}\label{eqn:bottomlessCoffee:530}
\sum_{i = 1}^n c_i J_{1}\left(\lambda_i z\right) = \phi(z)
\end{equation}
\begin{equation}\label{eqn:bottomlessCoffee:550}
\phi(z) =
\left\{
\begin{array}{l l}
z
& \quad \mbox{\(z \in [0, a]\)} \\
\frac{a^2}{1 - a^2}
\left(
\inv{z} - z
\right)
& \quad \mbox{\(z \in [a, 1]\)} \\
\end{array}
\right..
\end{equation}
\end{subequations}

Looking at \S 2.7 of \citep{sagan1989boundary} it appears the solutions for \(c_i\) can be obtained from

\begin{equation}\label{eqn:bottomlessCoffee:570}
c_i = \frac{
\int_0^1 z\phi(z) J_1(\lambda_i z) dz}{
\int_0^1 z J_1^2(\lambda_i z) dz},
\end{equation}

where \(\lambda_i\) are the zeros of \(J_1\).

To get a feel for these, a plot of the first few of these fitting functions is shown in \cref{fig:bottomlessCoffee:bottomlessCoffeeFig3}.

\imageFigure{../../figures/phy454/bottomlessCoffeeFirst_four_zero_crossing_Bessel_functions_J_1_lambda_i_zFig3}{First four zero crossing Bessel functions \(J_1( \lambda_i z)\)}{fig:bottomlessCoffee:bottomlessCoffeeFig3}{0.2}

Using Mathematica (\nbref{bottomlessCoffee.cdf}), these coefficients were calculated for \(a = 0.6\).  The \(n = 1, 3, 5\) approximations to the fitting function are plotted with a comparison to the steady state velocity profile in \cref{fig:bottomlessCoffee:bottomlessCoffeeFig2}.

\imageFigure{../../figures/phy454/bottomlessCoffeeBessel_function_fitting_for_the_steady_state_velocity_profile_for_n_1_3_5Fig2}{Bessel function fitting for the steady state velocity profile for \(n = 1, 3, 5\)}{fig:bottomlessCoffee:bottomlessCoffeeFig2}{0.2}

As indicated in the text, the spin down is way too slow to match reality (this can be seen visually in the worksheet by animating it).
} % end answer

   %
%
%
% Copyright � 2012 Peeter Joot
% All Rights Reserved
%
% This file may be reproduced and distributed in whole or in part, without fee, subject to the following conditions:
%
% o The copyright notice above and this permission notice must be preserved complete on all complete or partial copies.
%
% o Any translation or derived work must be approved by the author in writing before distribution.
%
% o If you distribute this work in part, instructions for obtaining the complete version of this file must be included, and a means for obtaining a complete version provided.
%
%
% Exceptions to these rules may be granted for academic purposes: Write to the author and ask.
%
%
%
%%
% Copyright � 2015 Peeter Joot.  All Rights Reserved.
% Licenced as described in the file LICENSE under the root directory of this GIT repository.
%
\documentclass[]{eliblog}

\usepackage{amsmath}
\usepackage{mathpazo}

%
% shorthand for bold symbols, convenient for vectors and matrices
%
\newcommand{\Ba}[0]{\mathbf{a}}
\newcommand{\Bb}[0]{\mathbf{b}}
\newcommand{\Bc}[0]{\mathbf{c}}
\newcommand{\Bd}[0]{\mathbf{d}}
\newcommand{\Be}[0]{\mathbf{e}}
\newcommand{\Bf}[0]{\mathbf{f}}
\newcommand{\Bg}[0]{\mathbf{g}}
\newcommand{\Bh}[0]{\mathbf{h}}
\newcommand{\Bi}[0]{\mathbf{i}}
\newcommand{\Bj}[0]{\mathbf{j}}
\newcommand{\Bk}[0]{\mathbf{k}}
\newcommand{\Bl}[0]{\mathbf{l}}
\newcommand{\Bm}[0]{\mathbf{m}}
\newcommand{\Bn}[0]{\mathbf{n}}
\newcommand{\Bo}[0]{\mathbf{o}}
\newcommand{\Bp}[0]{\mathbf{p}}
\newcommand{\Bq}[0]{\mathbf{q}}
\newcommand{\Br}[0]{\mathbf{r}}
\newcommand{\Bs}[0]{\mathbf{s}}
\newcommand{\Bt}[0]{\mathbf{t}}
\newcommand{\Bu}[0]{\mathbf{u}}
\newcommand{\Bv}[0]{\mathbf{v}}
\newcommand{\Bw}[0]{\mathbf{w}}
\newcommand{\Bx}[0]{\mathbf{x}}
\newcommand{\By}[0]{\mathbf{y}}
\newcommand{\Bz}[0]{\mathbf{z}}
\newcommand{\BA}[0]{\mathbf{A}}
\newcommand{\BB}[0]{\mathbf{B}}
\newcommand{\BC}[0]{\mathbf{C}}
\newcommand{\BD}[0]{\mathbf{D}}
\newcommand{\BE}[0]{\mathbf{E}}
\newcommand{\BF}[0]{\mathbf{F}}
\newcommand{\BG}[0]{\mathbf{G}}
\newcommand{\BH}[0]{\mathbf{H}}
\newcommand{\BI}[0]{\mathbf{I}}
\newcommand{\BJ}[0]{\mathbf{J}}
\newcommand{\BK}[0]{\mathbf{K}}
\newcommand{\BL}[0]{\mathbf{L}}
\newcommand{\BM}[0]{\mathbf{M}}
\newcommand{\BN}[0]{\mathbf{N}}
\newcommand{\BO}[0]{\mathbf{O}}
\newcommand{\BP}[0]{\mathbf{P}}
\newcommand{\BQ}[0]{\mathbf{Q}}
\newcommand{\BR}[0]{\mathbf{R}}
\newcommand{\BS}[0]{\mathbf{S}}
\newcommand{\BT}[0]{\mathbf{T}}
\newcommand{\BU}[0]{\mathbf{U}}
\newcommand{\BV}[0]{\mathbf{V}}
\newcommand{\BW}[0]{\mathbf{W}}
\newcommand{\BX}[0]{\mathbf{X}}
\newcommand{\BY}[0]{\mathbf{Y}}
\newcommand{\BZ}[0]{\mathbf{Z}}

\newcommand{\Bzero}[0]{\mathbf{0}}
\newcommand{\Btheta}[0]{\boldsymbol{\theta}}
\newcommand{\Btau}[0]{\boldsymbol{\tau}}
\newcommand{\Bomega}[0]{\boldsymbol{\omega}}

%
% shorthand for unit vectors
%
\newcommand{\acap}[0]{\hat{\Ba}}
\newcommand{\bcap}[0]{\hat{\Bb}}
\newcommand{\ccap}[0]{\hat{\Bc}}
\newcommand{\dcap}[0]{\hat{\Bd}}
\newcommand{\ecap}[0]{\hat{\Be}}
\newcommand{\fcap}[0]{\hat{\Bf}}
\newcommand{\gcap}[0]{\hat{\Bg}}
\newcommand{\hcap}[0]{\hat{\Bh}}
\newcommand{\icap}[0]{\hat{\Bi}}
\newcommand{\jcap}[0]{\hat{\Bj}}
\newcommand{\kcap}[0]{\hat{\Bk}}
\newcommand{\lcap}[0]{\hat{\Bl}}
\newcommand{\mcap}[0]{\hat{\Bm}}
\newcommand{\ncap}[0]{\hat{\Bn}}
\newcommand{\ocap}[0]{\hat{\Bo}}
\newcommand{\pcap}[0]{\hat{\Bp}}
\newcommand{\qcap}[0]{\hat{\Bq}}
\newcommand{\rcap}[0]{\hat{\Br}}
\newcommand{\scap}[0]{\hat{\Bs}}
\newcommand{\tcap}[0]{\hat{\Bt}}
\newcommand{\ucap}[0]{\hat{\Bu}}
\newcommand{\vcap}[0]{\hat{\Bv}}
\newcommand{\wcap}[0]{\hat{\Bw}}
\newcommand{\xcap}[0]{\hat{\Bx}}
\newcommand{\ycap}[0]{\hat{\By}}
\newcommand{\zcap}[0]{\hat{\Bz}}
\newcommand{\thetacap}[0]{\hat{\Btheta}}

%
% to write R^n and C^n in a distinguishable fashion.  Perhaps change this
% to the double lined characters upon figuring out how to do so.
%
\newcommand{\C}[1]{$\mathbb{C}^{#1}$}
\newcommand{\R}[1]{$\mathbb{R}^{#1}$}

%
% various generally useful helpers
%

% derivative of #1 wrt. #2:
\newcommand{\D}[2] {\frac {d#2} {d#1}}

\newcommand{\inv}[1]{\frac{1}{#1}}
\newcommand{\cross}[0]{\times}

\newcommand{\abs}[1]{\lvert{#1}\rvert}
\newcommand{\norm}[1]{\lVert{#1}\rVert}
\newcommand{\innerprod}[2]{\langle{#1}, {#2}\rangle}
\newcommand{\dotprod}[2]{{#1} \cdot {#2}}
\newcommand{\bdotprod}[2]{\left({#1} \cdot {#2}\right)}
\newcommand{\crossprod}[2]{{#1} \cross {#2}}
\newcommand{\tripleprod}[3]{\dotprod{\left(\crossprod{#1}{#2}\right)}{#3}}

\DeclareMathOperator{\Proj}{Proj}
\DeclareMathOperator{\Span}{span}
\DeclareMathOperator{\Sgn}{sgn}
\DeclareMathOperator{\Area}{Area}
\DeclareMathOperator{\Volume}{Volume}

%
% A few miscellaneous things specific to this document
%
\newcommand{\crossop}[1]{\crossprod{#1}{}}

% R2 vector.
\newcommand{\VectorTwo}[2]{
\begin{bmatrix}
 {#1} \\
 {#2}
\end{bmatrix}
}

\newcommand{\VectorN}[1]{
\begin{bmatrix}
{#1}_1 \\
{#1}_2 \\
\vdots \\
{#1}_N \\
\end{bmatrix}
}

\newcommand{\DETuvij}[4]{
\begin{vmatrix}
 {#1}_{#3} & {#1}_{#4} \\
 {#2}_{#3} & {#2}_{#4}
\end{vmatrix}
}

\newcommand{\DETuvwijk}[6]{
\begin{vmatrix}
 {#1}_{#4} & {#1}_{#5} & {#1}_{#6} \\
 {#2}_{#4} & {#2}_{#5} & {#2}_{#6} \\
 {#3}_{#4} & {#3}_{#5} & {#3}_{#6}
\end{vmatrix}
}

\newcommand{\DETuvwxijkl}[8]{
\begin{vmatrix}
 {#1}_{#5} & {#1}_{#6} & {#1}_{#7} & {#1}_{#8} \\
 {#2}_{#5} & {#2}_{#6} & {#2}_{#7} & {#2}_{#8} \\
 {#3}_{#5} & {#3}_{#6} & {#3}_{#7} & {#3}_{#8} \\
 {#4}_{#5} & {#4}_{#6} & {#4}_{#7} & {#4}_{#8} \\
\end{vmatrix}
}

%\newcommand{\DETuvwxyijklm}[10]{
%\begin{vmatrix}
% {#1}_{#6} & {#1}_{#7} & {#1}_{#8} & {#1}_{#9} & {#1}_{#10} \\
% {#2}_{#6} & {#2}_{#7} & {#2}_{#8} & {#2}_{#9} & {#2}_{#10} \\
% {#3}_{#6} & {#3}_{#7} & {#3}_{#8} & {#3}_{#9} & {#3}_{#10} \\
% {#4}_{#6} & {#4}_{#7} & {#4}_{#8} & {#4}_{#9} & {#4}_{#10} \\
% {#5}_{#6} & {#5}_{#7} & {#5}_{#8} & {#5}_{#9} & {#5}_{#10}
%\end{vmatrix}
%}

% R3 vector.
\newcommand{\VectorThree}[3]{
\begin{bmatrix}
 {#1} \\
 {#2} \\
 {#3}
\end{bmatrix}
}



\author{Peeter Joot}
\email{peeter.joot@gmail.com}

%\documentclass[]{eliblogwidescreen}

\usepackage{amsmath}
\usepackage{mathpazo}

%
% shorthand for bold symbols, convenient for vectors and matrices
%
\newcommand{\Ba}[0]{\mathbf{a}}
\newcommand{\Bb}[0]{\mathbf{b}}
\newcommand{\Bc}[0]{\mathbf{c}}
\newcommand{\Bd}[0]{\mathbf{d}}
\newcommand{\Be}[0]{\mathbf{e}}
\newcommand{\Bf}[0]{\mathbf{f}}
\newcommand{\Bg}[0]{\mathbf{g}}
\newcommand{\Bh}[0]{\mathbf{h}}
\newcommand{\Bi}[0]{\mathbf{i}}
\newcommand{\Bj}[0]{\mathbf{j}}
\newcommand{\Bk}[0]{\mathbf{k}}
\newcommand{\Bl}[0]{\mathbf{l}}
\newcommand{\Bm}[0]{\mathbf{m}}
\newcommand{\Bn}[0]{\mathbf{n}}
\newcommand{\Bo}[0]{\mathbf{o}}
\newcommand{\Bp}[0]{\mathbf{p}}
\newcommand{\Bq}[0]{\mathbf{q}}
\newcommand{\Br}[0]{\mathbf{r}}
\newcommand{\Bs}[0]{\mathbf{s}}
\newcommand{\Bt}[0]{\mathbf{t}}
\newcommand{\Bu}[0]{\mathbf{u}}
\newcommand{\Bv}[0]{\mathbf{v}}
\newcommand{\Bw}[0]{\mathbf{w}}
\newcommand{\Bx}[0]{\mathbf{x}}
\newcommand{\By}[0]{\mathbf{y}}
\newcommand{\Bz}[0]{\mathbf{z}}
\newcommand{\BA}[0]{\mathbf{A}}
\newcommand{\BB}[0]{\mathbf{B}}
\newcommand{\BC}[0]{\mathbf{C}}
\newcommand{\BD}[0]{\mathbf{D}}
\newcommand{\BE}[0]{\mathbf{E}}
\newcommand{\BF}[0]{\mathbf{F}}
\newcommand{\BG}[0]{\mathbf{G}}
\newcommand{\BH}[0]{\mathbf{H}}
\newcommand{\BI}[0]{\mathbf{I}}
\newcommand{\BJ}[0]{\mathbf{J}}
\newcommand{\BK}[0]{\mathbf{K}}
\newcommand{\BL}[0]{\mathbf{L}}
\newcommand{\BM}[0]{\mathbf{M}}
\newcommand{\BN}[0]{\mathbf{N}}
\newcommand{\BO}[0]{\mathbf{O}}
\newcommand{\BP}[0]{\mathbf{P}}
\newcommand{\BQ}[0]{\mathbf{Q}}
\newcommand{\BR}[0]{\mathbf{R}}
\newcommand{\BS}[0]{\mathbf{S}}
\newcommand{\BT}[0]{\mathbf{T}}
\newcommand{\BU}[0]{\mathbf{U}}
\newcommand{\BV}[0]{\mathbf{V}}
\newcommand{\BW}[0]{\mathbf{W}}
\newcommand{\BX}[0]{\mathbf{X}}
\newcommand{\BY}[0]{\mathbf{Y}}
\newcommand{\BZ}[0]{\mathbf{Z}}

\newcommand{\Bzero}[0]{\mathbf{0}}
\newcommand{\Btheta}[0]{\boldsymbol{\theta}}
\newcommand{\Btau}[0]{\boldsymbol{\tau}}
\newcommand{\Bomega}[0]{\boldsymbol{\omega}}

%
% shorthand for unit vectors
%
\newcommand{\acap}[0]{\hat{\Ba}}
\newcommand{\bcap}[0]{\hat{\Bb}}
\newcommand{\ccap}[0]{\hat{\Bc}}
\newcommand{\dcap}[0]{\hat{\Bd}}
\newcommand{\ecap}[0]{\hat{\Be}}
\newcommand{\fcap}[0]{\hat{\Bf}}
\newcommand{\gcap}[0]{\hat{\Bg}}
\newcommand{\hcap}[0]{\hat{\Bh}}
\newcommand{\icap}[0]{\hat{\Bi}}
\newcommand{\jcap}[0]{\hat{\Bj}}
\newcommand{\kcap}[0]{\hat{\Bk}}
\newcommand{\lcap}[0]{\hat{\Bl}}
\newcommand{\mcap}[0]{\hat{\Bm}}
\newcommand{\ncap}[0]{\hat{\Bn}}
\newcommand{\ocap}[0]{\hat{\Bo}}
\newcommand{\pcap}[0]{\hat{\Bp}}
\newcommand{\qcap}[0]{\hat{\Bq}}
\newcommand{\rcap}[0]{\hat{\Br}}
\newcommand{\scap}[0]{\hat{\Bs}}
\newcommand{\tcap}[0]{\hat{\Bt}}
\newcommand{\ucap}[0]{\hat{\Bu}}
\newcommand{\vcap}[0]{\hat{\Bv}}
\newcommand{\wcap}[0]{\hat{\Bw}}
\newcommand{\xcap}[0]{\hat{\Bx}}
\newcommand{\ycap}[0]{\hat{\By}}
\newcommand{\zcap}[0]{\hat{\Bz}}
\newcommand{\thetacap}[0]{\hat{\Btheta}}

%
% to write R^n and C^n in a distinguishable fashion.  Perhaps change this
% to the double lined characters upon figuring out how to do so.
%
\newcommand{\C}[1]{$\mathbb{C}^{#1}$}
\newcommand{\R}[1]{$\mathbb{R}^{#1}$}

%
% various generally useful helpers
%

% derivative of #1 wrt. #2:
\newcommand{\D}[2] {\frac {d#2} {d#1}}

\newcommand{\inv}[1]{\frac{1}{#1}}
\newcommand{\cross}[0]{\times}

\newcommand{\abs}[1]{\lvert{#1}\rvert}
\newcommand{\norm}[1]{\lVert{#1}\rVert}
\newcommand{\innerprod}[2]{\langle{#1}, {#2}\rangle}
\newcommand{\dotprod}[2]{{#1} \cdot {#2}}
\newcommand{\bdotprod}[2]{\left({#1} \cdot {#2}\right)}
\newcommand{\crossprod}[2]{{#1} \cross {#2}}
\newcommand{\tripleprod}[3]{\dotprod{\left(\crossprod{#1}{#2}\right)}{#3}}

\DeclareMathOperator{\Proj}{Proj}
\DeclareMathOperator{\Span}{span}
\DeclareMathOperator{\Sgn}{sgn}
\DeclareMathOperator{\Area}{Area}
\DeclareMathOperator{\Volume}{Volume}

%
% A few miscellaneous things specific to this document
%
\newcommand{\crossop}[1]{\crossprod{#1}{}}

% R2 vector.
\newcommand{\VectorTwo}[2]{
\begin{bmatrix}
 {#1} \\
 {#2}
\end{bmatrix}
}

\newcommand{\VectorN}[1]{
\begin{bmatrix}
{#1}_1 \\
{#1}_2 \\
\vdots \\
{#1}_N \\
\end{bmatrix}
}

\newcommand{\DETuvij}[4]{
\begin{vmatrix}
 {#1}_{#3} & {#1}_{#4} \\
 {#2}_{#3} & {#2}_{#4}
\end{vmatrix}
}

\newcommand{\DETuvwijk}[6]{
\begin{vmatrix}
 {#1}_{#4} & {#1}_{#5} & {#1}_{#6} \\
 {#2}_{#4} & {#2}_{#5} & {#2}_{#6} \\
 {#3}_{#4} & {#3}_{#5} & {#3}_{#6}
\end{vmatrix}
}

\newcommand{\DETuvwxijkl}[8]{
\begin{vmatrix}
 {#1}_{#5} & {#1}_{#6} & {#1}_{#7} & {#1}_{#8} \\
 {#2}_{#5} & {#2}_{#6} & {#2}_{#7} & {#2}_{#8} \\
 {#3}_{#5} & {#3}_{#6} & {#3}_{#7} & {#3}_{#8} \\
 {#4}_{#5} & {#4}_{#6} & {#4}_{#7} & {#4}_{#8} \\
\end{vmatrix}
}

%\newcommand{\DETuvwxyijklm}[10]{
%\begin{vmatrix}
% {#1}_{#6} & {#1}_{#7} & {#1}_{#8} & {#1}_{#9} & {#1}_{#10} \\
% {#2}_{#6} & {#2}_{#7} & {#2}_{#8} & {#2}_{#9} & {#2}_{#10} \\
% {#3}_{#6} & {#3}_{#7} & {#3}_{#8} & {#3}_{#9} & {#3}_{#10} \\
% {#4}_{#6} & {#4}_{#7} & {#4}_{#8} & {#4}_{#9} & {#4}_{#10} \\
% {#5}_{#6} & {#5}_{#7} & {#5}_{#8} & {#5}_{#9} & {#5}_{#10}
%\end{vmatrix}
%}

% R3 vector.
\newcommand{\VectorThree}[3]{
\begin{bmatrix}
 {#1} \\
 {#2} \\
 {#3}
\end{bmatrix}
}



\author{Peeter Joot}
\email{peeter.joot@gmail.com}


%\usepackage[english]{babel}
%\usepackage{media9}
\chapter{Surface for spinning bucket of water.}

\label{chap:constantSpinSurfaces}
%\useCCL
\blogpage{http://sites.google.com/site/peeterjoot2/math2012/constantSpinSurfaces.pdf}
\date{Apr 29, 2012}
\gitRevisionInfo{constantSpinSurfaces}
\keywords{Navier-Stokes, Bernoulli's theorem, PHY454H1S, PHY454H1}

\beginArtWithToc
%\beginArtNoToc
%\wordpresscategory{}

\section{Motivation.}

Here's a problem from the 2009 phy1530 final that was appropriate for exam prep for this course too.  It also serves as a nice example of how to determine a surface as a function of pressure, something I want to do for the non-bottomless coffee problem to be attempted.

\section{Statement}

An undergraduate student is assigned a problem about an ideal fluid rotating at a constant angular velocity $\Omega$ under gravity $g$.  The velocity field is $\Bu = (-\Omega y, \Omega x, 0)$.  Here, $x$ and $y$ are horizontal and $z$ points up.  The student is supposed to find the surfaces of constant pressure, and hence the shape of the free surface of water in a rotating bucket.  The free surface corresponds to the surface for which $p = p_0$, where $p_0$ is the atmospheric pressure.  Surface tension is neglected.

\begin{enumerate}
\item On their homework assignment, the student writes:

``By Bernoulli's equation:

\begin{equation}\label{eqn:constantSpinSurfaces:10}
B = \frac{p}{\rho} + \inv{2}u^2 + g z
\end{equation}

where $B$ is a constant.  So the constant pressure surface at $p = p_0$ is 

\begin{equation}\label{eqn:constantSpinSurfaces:30}
z = \left( \frac{B}{g} - \frac{p_0}{\rho g}
\right)
- \frac{\Omega^2 }{2 g} \left( x^2 + y^2 \right).
\end{equation}
''

But this seems to show that the surface of the water in a rotating bucket is \textit{highest in the middle}!  What is wrong with the student's argument?

\item
Write down the Euler equations in component form and integrate them directly to find the pressure $p$, and hence obtain the correct parabolic shape for the free surface.
\end{enumerate}

\section{Solution.  Part 1.}

Let's recall how we derived Bernoulli's theorem.  We started with Navier-Stokes and used the identity

\begin{equation}\label{eqn:constantSpinSurfaces:50}
(\Bu \cdot \spacegrad ) \Bu = \spacegrad \inv{2} \Bu^2 + (\spacegrad \cross \Bu) \cross \Bu.
\end{equation}

Navier-Stokes for a steady state incompressible flow, with external body force per unit volume $\rho \Bg = -\rho \spacegrad \chi$ take the form

\begin{equation}\label{eqn:constantSpinSurfaces:70}
\spacegrad \inv{2} \Bu^2 + (\spacegrad \cross \Bu) \cross \Bu
= -\inv{\rho} \spacegrad p + \nu \spacegrad^2 \Bu - \spacegrad \chi.
\end{equation}

For the non-viscous (``dry-water'') case where we take $\mu = \nu \rho = 0$, and treat the density $\rho$ as a constant we find

\begin{equation}\label{eqn:constantSpinSurfaces:90}
\Bu \cross (\spacegrad \cross \Bu)
=
\spacegrad 
\left( 
\inv{2} \Bu^2 + \frac{p}{\rho} + \chi
\right).
\end{equation}

Observe that we only arrive at Bernoulli's theorem if the flow is also irrotational (as well as incompressible and non-viscous), as we require an irrotational flow where $\spacegrad \cdot \Bu = 0$ to claim that the gradient on the RHS is zero.  
%
%The most general claim that we can make, even for irrotational flows is that we have
%
%\begin{equation}\label{eqn:constantSpinSurfaces:110}
%0 = \Bu \cdot
%\spacegrad 
%\left( 
%\inv{2} \Bu^2 + \frac{p}{\rho} + \chi
%\right).
%\end{equation}
%
%That holds even for flows that are not irrotational, since $\Bu \cdot (\spacegrad \cross \Bu) = 0$.

In this problem we do not have an irrotational flow, which can be demonstrated by direct calculation.  We have

\begin{equation}\label{eqn:constantSpinSurfaces:130}
\begin{aligned}
\spacegrad \cross \Bu
&=
\Omega
\begin{vmatrix}
\xcap & \ycap & \zcap \\
\partial_x & \partial_y & 0 \\
-y & x & 0
\end{vmatrix} \\
&=
2 \zcap \Omega \\
&\ne 0
\end{aligned}
\end{equation}

In fact we have

\begin{equation}\label{eqn:constantSpinSurfaces:150}
\begin{aligned}
\Bu \cross (\spacegrad \cross \Bu)
&=
2 \Omega^2
\begin{vmatrix}
\xcap & \ycap & \zcap \\
-y & x & 0  \\
0 & 0 & 1
\end{vmatrix} \\
&=
2 \Omega^2 (\xcap + \ycap)
\end{aligned}
\end{equation}

The closest we can get to Bernoulli's theorem for this problem is

\begin{equation}\label{eqn:constantSpinSurfaces:210}
2 \Omega^2 (\xcap + \ycap)
= 
\spacegrad 
\left( 
\inv{2} \Bu^2 + \frac{p}{\rho} + g z
\right).
\end{equation}

We can say that the directional derivatives in directions perpendicular to $\xcap + \ycap$ are zero, and that 

\begin{equation}\label{eqn:constantSpinSurfaces:230}
\begin{aligned}
2 \Omega^2 
&= (\partial_x + \partial_y) 
\left( 
\inv{2} \Bu^2 + \frac{p}{\rho} + g z
\right) \\
&= (\partial_x + \partial_y) 
\left( 
\inv{2} \Bu^2 + \frac{p}{\rho} 
\right) \\
\end{aligned}
\end{equation}

Perhaps those could be used to solve for the surface, but we no longer have something that is obviously integrable.

Because $\Bu \cdot (\Bu \cross (\spacegrad \cross \Bu)) = 0$, we can also say that

\begin{equation}\label{eqn:constantSpinSurfaces:250}
\begin{aligned}
0 
&= \Bu \cdot \spacegrad
\left( 
\inv{2} \Bu^2 + \frac{p}{\rho} + g z
\right) \\
&= 
\Omega ( y \partial_x - x \partial_y )
\left( 
\inv{2} \Bu^2 + \frac{p}{\rho} 
\right).
\end{aligned}
\end{equation}

Perhaps this could also be used to find the surface?

\section{Solution.  Part 2.}

We want to write down the steady state, incompressible, non-viscous Navier-Stokes equations.  The first of these is trivially satisfied by our assumed solution

\begin{equation}\label{eqn:constantSpinSurfaces:270}
0 
= \spacegrad \cdot \Bu 
= \partial_x (-\Omega y) + \partial_y(\Omega x).
\end{equation}

For the inertial term we've got

\begin{equation}\label{eqn:constantSpinSurfaces:290}
\begin{aligned}
\Bu \cdot \spacegrad \Bu
&=
\Omega^2 (-y \partial_x + x \partial_y (-y, x, 0) \\
&=
\Omega^2 (-x, -y, 0),
\end{aligned}
\end{equation}

Leaving us with

\begin{equation}\label{eqn:constantSpinSurfaces:310}
\begin{aligned}
-\Omega^2 x &= -\inv{\rho} \partial_x p \\
-\Omega^2 y &= -\inv{\rho} \partial_y p \\
          0 &= -\inv{\rho} \partial_z p - g
\end{aligned}
\end{equation}

Integrating these, we seek seek simultaneous solutions to

\begin{equation}\label{eqn:constantSpinSurfaces:330}
\begin{aligned}
p &= \inv{2} \rho \Omega^2 x^2 + f(y,z) \\
p &= \inv{2} \rho \Omega^2 y^2 + g(x,z) \\
p &= h(x, y) - \rho g z.
\end{aligned}
\end{equation}

It's clear that one solution would be

\begin{equation}\label{eqn:constantSpinSurfaces:350}
p = p_0 + \inv{2} \rho \Omega^2 (x^2 + y^2) - \rho g z.
\end{equation}

where $p_0$ is some constant to be determined, dependent on where we set our origin.  Putting the origin of the coordinate system at the lowest point in the parabolic profile $(x, y, z) = (0, 0, 0)$, we have $p(0, 0, 0) = p_0$, which fixes $p_0$ as the atmospheric pressure.  If the radius of the bucket is $R$, the max height $h$ of the surface above that point is also found on this surface of constant pressure

\begin{equation}\label{eqn:constantSpinSurfaces:370}
p_0 = p_0 + \inv{2} \rho \Omega^2 R^2 - \rho g h,
\end{equation}

or 

\begin{equation}\label{eqn:constantSpinSurfaces:390}
h = \frac{\Omega^2 R^2 }{2 g}.
\end{equation}

\section{Appendix.  Proof of vector identities used.}

\begin{equation}\label{eqn:constantSpinSurfaces:170}
\begin{aligned}
\left( \spacegrad \inv{2} \Bu^2 + (\spacegrad \cross \Bu) \cross \Bu \right)_i
&=
\partial_i \inv{2} u_j u_j + \partial_a u_b \epsilon_{a b r} u_s \epsilon_{r s i} \\
&=
u_j \partial_i u_j + \partial_a u_b u_s \delta^{[a b]}_{s i} \\
&=
u_j \partial_i u_j 
+ u_s \partial_s u_i 
- u_s \partial_i u_s \\
&= (\Bu \cdot \spacegrad) \Bu)_i 
\end{aligned}
\end{equation}

Also observe that our claim that $\Bu \cdot (\Bu \cross (\spacegrad \Bu)) = 0$ follows easily after expansion in coordinates

\begin{equation}\label{eqn:constantSpinSurfaces:190}
\Bu \cdot (\Bu \cross (\spacegrad \cross \Bu) )
=
u_i u_s ( \partial_s u_i - \partial_i u_s ).
\end{equation}

We've got a symmetric and antisymmetric factor in the summation, so the end result is zero.

%\EndArticle
\EndNoBibArticle

   %
%
%
% Copyright � 2012 Peeter Joot
% All Rights Reserved
%
% This file may be reproduced and distributed in whole or in part, without fee, subject to the following conditions:
%
% o The copyright notice above and this permission notice must be preserved complete on all complete or partial copies.
%
% o Any translation or derived work must be approved by the author in writing before distribution.
%
% o If you distribute this work in part, instructions for obtaining the complete version of this file must be included, and a means for obtaining a complete version provided.
%
%
% Exceptions to these rules may be granted for academic purposes: Write to the author and ask.
%
%
%
%%
% Copyright � 2015 Peeter Joot.  All Rights Reserved.
% Licenced as described in the file LICENSE under the root directory of this GIT repository.
%
\documentclass[]{eliblog}

\usepackage{amsmath}
\usepackage{mathpazo}

%
% shorthand for bold symbols, convenient for vectors and matrices
%
\newcommand{\Ba}[0]{\mathbf{a}}
\newcommand{\Bb}[0]{\mathbf{b}}
\newcommand{\Bc}[0]{\mathbf{c}}
\newcommand{\Bd}[0]{\mathbf{d}}
\newcommand{\Be}[0]{\mathbf{e}}
\newcommand{\Bf}[0]{\mathbf{f}}
\newcommand{\Bg}[0]{\mathbf{g}}
\newcommand{\Bh}[0]{\mathbf{h}}
\newcommand{\Bi}[0]{\mathbf{i}}
\newcommand{\Bj}[0]{\mathbf{j}}
\newcommand{\Bk}[0]{\mathbf{k}}
\newcommand{\Bl}[0]{\mathbf{l}}
\newcommand{\Bm}[0]{\mathbf{m}}
\newcommand{\Bn}[0]{\mathbf{n}}
\newcommand{\Bo}[0]{\mathbf{o}}
\newcommand{\Bp}[0]{\mathbf{p}}
\newcommand{\Bq}[0]{\mathbf{q}}
\newcommand{\Br}[0]{\mathbf{r}}
\newcommand{\Bs}[0]{\mathbf{s}}
\newcommand{\Bt}[0]{\mathbf{t}}
\newcommand{\Bu}[0]{\mathbf{u}}
\newcommand{\Bv}[0]{\mathbf{v}}
\newcommand{\Bw}[0]{\mathbf{w}}
\newcommand{\Bx}[0]{\mathbf{x}}
\newcommand{\By}[0]{\mathbf{y}}
\newcommand{\Bz}[0]{\mathbf{z}}
\newcommand{\BA}[0]{\mathbf{A}}
\newcommand{\BB}[0]{\mathbf{B}}
\newcommand{\BC}[0]{\mathbf{C}}
\newcommand{\BD}[0]{\mathbf{D}}
\newcommand{\BE}[0]{\mathbf{E}}
\newcommand{\BF}[0]{\mathbf{F}}
\newcommand{\BG}[0]{\mathbf{G}}
\newcommand{\BH}[0]{\mathbf{H}}
\newcommand{\BI}[0]{\mathbf{I}}
\newcommand{\BJ}[0]{\mathbf{J}}
\newcommand{\BK}[0]{\mathbf{K}}
\newcommand{\BL}[0]{\mathbf{L}}
\newcommand{\BM}[0]{\mathbf{M}}
\newcommand{\BN}[0]{\mathbf{N}}
\newcommand{\BO}[0]{\mathbf{O}}
\newcommand{\BP}[0]{\mathbf{P}}
\newcommand{\BQ}[0]{\mathbf{Q}}
\newcommand{\BR}[0]{\mathbf{R}}
\newcommand{\BS}[0]{\mathbf{S}}
\newcommand{\BT}[0]{\mathbf{T}}
\newcommand{\BU}[0]{\mathbf{U}}
\newcommand{\BV}[0]{\mathbf{V}}
\newcommand{\BW}[0]{\mathbf{W}}
\newcommand{\BX}[0]{\mathbf{X}}
\newcommand{\BY}[0]{\mathbf{Y}}
\newcommand{\BZ}[0]{\mathbf{Z}}

\newcommand{\Bzero}[0]{\mathbf{0}}
\newcommand{\Btheta}[0]{\boldsymbol{\theta}}
\newcommand{\Btau}[0]{\boldsymbol{\tau}}
\newcommand{\Bomega}[0]{\boldsymbol{\omega}}

%
% shorthand for unit vectors
%
\newcommand{\acap}[0]{\hat{\Ba}}
\newcommand{\bcap}[0]{\hat{\Bb}}
\newcommand{\ccap}[0]{\hat{\Bc}}
\newcommand{\dcap}[0]{\hat{\Bd}}
\newcommand{\ecap}[0]{\hat{\Be}}
\newcommand{\fcap}[0]{\hat{\Bf}}
\newcommand{\gcap}[0]{\hat{\Bg}}
\newcommand{\hcap}[0]{\hat{\Bh}}
\newcommand{\icap}[0]{\hat{\Bi}}
\newcommand{\jcap}[0]{\hat{\Bj}}
\newcommand{\kcap}[0]{\hat{\Bk}}
\newcommand{\lcap}[0]{\hat{\Bl}}
\newcommand{\mcap}[0]{\hat{\Bm}}
\newcommand{\ncap}[0]{\hat{\Bn}}
\newcommand{\ocap}[0]{\hat{\Bo}}
\newcommand{\pcap}[0]{\hat{\Bp}}
\newcommand{\qcap}[0]{\hat{\Bq}}
\newcommand{\rcap}[0]{\hat{\Br}}
\newcommand{\scap}[0]{\hat{\Bs}}
\newcommand{\tcap}[0]{\hat{\Bt}}
\newcommand{\ucap}[0]{\hat{\Bu}}
\newcommand{\vcap}[0]{\hat{\Bv}}
\newcommand{\wcap}[0]{\hat{\Bw}}
\newcommand{\xcap}[0]{\hat{\Bx}}
\newcommand{\ycap}[0]{\hat{\By}}
\newcommand{\zcap}[0]{\hat{\Bz}}
\newcommand{\thetacap}[0]{\hat{\Btheta}}

%
% to write R^n and C^n in a distinguishable fashion.  Perhaps change this
% to the double lined characters upon figuring out how to do so.
%
\newcommand{\C}[1]{$\mathbb{C}^{#1}$}
\newcommand{\R}[1]{$\mathbb{R}^{#1}$}

%
% various generally useful helpers
%

% derivative of #1 wrt. #2:
\newcommand{\D}[2] {\frac {d#2} {d#1}}

\newcommand{\inv}[1]{\frac{1}{#1}}
\newcommand{\cross}[0]{\times}

\newcommand{\abs}[1]{\lvert{#1}\rvert}
\newcommand{\norm}[1]{\lVert{#1}\rVert}
\newcommand{\innerprod}[2]{\langle{#1}, {#2}\rangle}
\newcommand{\dotprod}[2]{{#1} \cdot {#2}}
\newcommand{\bdotprod}[2]{\left({#1} \cdot {#2}\right)}
\newcommand{\crossprod}[2]{{#1} \cross {#2}}
\newcommand{\tripleprod}[3]{\dotprod{\left(\crossprod{#1}{#2}\right)}{#3}}

\DeclareMathOperator{\Proj}{Proj}
\DeclareMathOperator{\Span}{span}
\DeclareMathOperator{\Sgn}{sgn}
\DeclareMathOperator{\Area}{Area}
\DeclareMathOperator{\Volume}{Volume}

%
% A few miscellaneous things specific to this document
%
\newcommand{\crossop}[1]{\crossprod{#1}{}}

% R2 vector.
\newcommand{\VectorTwo}[2]{
\begin{bmatrix}
 {#1} \\
 {#2}
\end{bmatrix}
}

\newcommand{\VectorN}[1]{
\begin{bmatrix}
{#1}_1 \\
{#1}_2 \\
\vdots \\
{#1}_N \\
\end{bmatrix}
}

\newcommand{\DETuvij}[4]{
\begin{vmatrix}
 {#1}_{#3} & {#1}_{#4} \\
 {#2}_{#3} & {#2}_{#4}
\end{vmatrix}
}

\newcommand{\DETuvwijk}[6]{
\begin{vmatrix}
 {#1}_{#4} & {#1}_{#5} & {#1}_{#6} \\
 {#2}_{#4} & {#2}_{#5} & {#2}_{#6} \\
 {#3}_{#4} & {#3}_{#5} & {#3}_{#6}
\end{vmatrix}
}

\newcommand{\DETuvwxijkl}[8]{
\begin{vmatrix}
 {#1}_{#5} & {#1}_{#6} & {#1}_{#7} & {#1}_{#8} \\
 {#2}_{#5} & {#2}_{#6} & {#2}_{#7} & {#2}_{#8} \\
 {#3}_{#5} & {#3}_{#6} & {#3}_{#7} & {#3}_{#8} \\
 {#4}_{#5} & {#4}_{#6} & {#4}_{#7} & {#4}_{#8} \\
\end{vmatrix}
}

%\newcommand{\DETuvwxyijklm}[10]{
%\begin{vmatrix}
% {#1}_{#6} & {#1}_{#7} & {#1}_{#8} & {#1}_{#9} & {#1}_{#10} \\
% {#2}_{#6} & {#2}_{#7} & {#2}_{#8} & {#2}_{#9} & {#2}_{#10} \\
% {#3}_{#6} & {#3}_{#7} & {#3}_{#8} & {#3}_{#9} & {#3}_{#10} \\
% {#4}_{#6} & {#4}_{#7} & {#4}_{#8} & {#4}_{#9} & {#4}_{#10} \\
% {#5}_{#6} & {#5}_{#7} & {#5}_{#8} & {#5}_{#9} & {#5}_{#10}
%\end{vmatrix}
%}

% R3 vector.
\newcommand{\VectorThree}[3]{
\begin{bmatrix}
 {#1} \\
 {#2} \\
 {#3}
\end{bmatrix}
}



\author{Peeter Joot}
\email{peeter.joot@gmail.com}

%\documentclass[]{eliblogwidescreen}

\usepackage{amsmath}
\usepackage{mathpazo}

%
% shorthand for bold symbols, convenient for vectors and matrices
%
\newcommand{\Ba}[0]{\mathbf{a}}
\newcommand{\Bb}[0]{\mathbf{b}}
\newcommand{\Bc}[0]{\mathbf{c}}
\newcommand{\Bd}[0]{\mathbf{d}}
\newcommand{\Be}[0]{\mathbf{e}}
\newcommand{\Bf}[0]{\mathbf{f}}
\newcommand{\Bg}[0]{\mathbf{g}}
\newcommand{\Bh}[0]{\mathbf{h}}
\newcommand{\Bi}[0]{\mathbf{i}}
\newcommand{\Bj}[0]{\mathbf{j}}
\newcommand{\Bk}[0]{\mathbf{k}}
\newcommand{\Bl}[0]{\mathbf{l}}
\newcommand{\Bm}[0]{\mathbf{m}}
\newcommand{\Bn}[0]{\mathbf{n}}
\newcommand{\Bo}[0]{\mathbf{o}}
\newcommand{\Bp}[0]{\mathbf{p}}
\newcommand{\Bq}[0]{\mathbf{q}}
\newcommand{\Br}[0]{\mathbf{r}}
\newcommand{\Bs}[0]{\mathbf{s}}
\newcommand{\Bt}[0]{\mathbf{t}}
\newcommand{\Bu}[0]{\mathbf{u}}
\newcommand{\Bv}[0]{\mathbf{v}}
\newcommand{\Bw}[0]{\mathbf{w}}
\newcommand{\Bx}[0]{\mathbf{x}}
\newcommand{\By}[0]{\mathbf{y}}
\newcommand{\Bz}[0]{\mathbf{z}}
\newcommand{\BA}[0]{\mathbf{A}}
\newcommand{\BB}[0]{\mathbf{B}}
\newcommand{\BC}[0]{\mathbf{C}}
\newcommand{\BD}[0]{\mathbf{D}}
\newcommand{\BE}[0]{\mathbf{E}}
\newcommand{\BF}[0]{\mathbf{F}}
\newcommand{\BG}[0]{\mathbf{G}}
\newcommand{\BH}[0]{\mathbf{H}}
\newcommand{\BI}[0]{\mathbf{I}}
\newcommand{\BJ}[0]{\mathbf{J}}
\newcommand{\BK}[0]{\mathbf{K}}
\newcommand{\BL}[0]{\mathbf{L}}
\newcommand{\BM}[0]{\mathbf{M}}
\newcommand{\BN}[0]{\mathbf{N}}
\newcommand{\BO}[0]{\mathbf{O}}
\newcommand{\BP}[0]{\mathbf{P}}
\newcommand{\BQ}[0]{\mathbf{Q}}
\newcommand{\BR}[0]{\mathbf{R}}
\newcommand{\BS}[0]{\mathbf{S}}
\newcommand{\BT}[0]{\mathbf{T}}
\newcommand{\BU}[0]{\mathbf{U}}
\newcommand{\BV}[0]{\mathbf{V}}
\newcommand{\BW}[0]{\mathbf{W}}
\newcommand{\BX}[0]{\mathbf{X}}
\newcommand{\BY}[0]{\mathbf{Y}}
\newcommand{\BZ}[0]{\mathbf{Z}}

\newcommand{\Bzero}[0]{\mathbf{0}}
\newcommand{\Btheta}[0]{\boldsymbol{\theta}}
\newcommand{\Btau}[0]{\boldsymbol{\tau}}
\newcommand{\Bomega}[0]{\boldsymbol{\omega}}

%
% shorthand for unit vectors
%
\newcommand{\acap}[0]{\hat{\Ba}}
\newcommand{\bcap}[0]{\hat{\Bb}}
\newcommand{\ccap}[0]{\hat{\Bc}}
\newcommand{\dcap}[0]{\hat{\Bd}}
\newcommand{\ecap}[0]{\hat{\Be}}
\newcommand{\fcap}[0]{\hat{\Bf}}
\newcommand{\gcap}[0]{\hat{\Bg}}
\newcommand{\hcap}[0]{\hat{\Bh}}
\newcommand{\icap}[0]{\hat{\Bi}}
\newcommand{\jcap}[0]{\hat{\Bj}}
\newcommand{\kcap}[0]{\hat{\Bk}}
\newcommand{\lcap}[0]{\hat{\Bl}}
\newcommand{\mcap}[0]{\hat{\Bm}}
\newcommand{\ncap}[0]{\hat{\Bn}}
\newcommand{\ocap}[0]{\hat{\Bo}}
\newcommand{\pcap}[0]{\hat{\Bp}}
\newcommand{\qcap}[0]{\hat{\Bq}}
\newcommand{\rcap}[0]{\hat{\Br}}
\newcommand{\scap}[0]{\hat{\Bs}}
\newcommand{\tcap}[0]{\hat{\Bt}}
\newcommand{\ucap}[0]{\hat{\Bu}}
\newcommand{\vcap}[0]{\hat{\Bv}}
\newcommand{\wcap}[0]{\hat{\Bw}}
\newcommand{\xcap}[0]{\hat{\Bx}}
\newcommand{\ycap}[0]{\hat{\By}}
\newcommand{\zcap}[0]{\hat{\Bz}}
\newcommand{\thetacap}[0]{\hat{\Btheta}}

%
% to write R^n and C^n in a distinguishable fashion.  Perhaps change this
% to the double lined characters upon figuring out how to do so.
%
\newcommand{\C}[1]{$\mathbb{C}^{#1}$}
\newcommand{\R}[1]{$\mathbb{R}^{#1}$}

%
% various generally useful helpers
%

% derivative of #1 wrt. #2:
\newcommand{\D}[2] {\frac {d#2} {d#1}}

\newcommand{\inv}[1]{\frac{1}{#1}}
\newcommand{\cross}[0]{\times}

\newcommand{\abs}[1]{\lvert{#1}\rvert}
\newcommand{\norm}[1]{\lVert{#1}\rVert}
\newcommand{\innerprod}[2]{\langle{#1}, {#2}\rangle}
\newcommand{\dotprod}[2]{{#1} \cdot {#2}}
\newcommand{\bdotprod}[2]{\left({#1} \cdot {#2}\right)}
\newcommand{\crossprod}[2]{{#1} \cross {#2}}
\newcommand{\tripleprod}[3]{\dotprod{\left(\crossprod{#1}{#2}\right)}{#3}}

\DeclareMathOperator{\Proj}{Proj}
\DeclareMathOperator{\Span}{span}
\DeclareMathOperator{\Sgn}{sgn}
\DeclareMathOperator{\Area}{Area}
\DeclareMathOperator{\Volume}{Volume}

%
% A few miscellaneous things specific to this document
%
\newcommand{\crossop}[1]{\crossprod{#1}{}}

% R2 vector.
\newcommand{\VectorTwo}[2]{
\begin{bmatrix}
 {#1} \\
 {#2}
\end{bmatrix}
}

\newcommand{\VectorN}[1]{
\begin{bmatrix}
{#1}_1 \\
{#1}_2 \\
\vdots \\
{#1}_N \\
\end{bmatrix}
}

\newcommand{\DETuvij}[4]{
\begin{vmatrix}
 {#1}_{#3} & {#1}_{#4} \\
 {#2}_{#3} & {#2}_{#4}
\end{vmatrix}
}

\newcommand{\DETuvwijk}[6]{
\begin{vmatrix}
 {#1}_{#4} & {#1}_{#5} & {#1}_{#6} \\
 {#2}_{#4} & {#2}_{#5} & {#2}_{#6} \\
 {#3}_{#4} & {#3}_{#5} & {#3}_{#6}
\end{vmatrix}
}

\newcommand{\DETuvwxijkl}[8]{
\begin{vmatrix}
 {#1}_{#5} & {#1}_{#6} & {#1}_{#7} & {#1}_{#8} \\
 {#2}_{#5} & {#2}_{#6} & {#2}_{#7} & {#2}_{#8} \\
 {#3}_{#5} & {#3}_{#6} & {#3}_{#7} & {#3}_{#8} \\
 {#4}_{#5} & {#4}_{#6} & {#4}_{#7} & {#4}_{#8} \\
\end{vmatrix}
}

%\newcommand{\DETuvwxyijklm}[10]{
%\begin{vmatrix}
% {#1}_{#6} & {#1}_{#7} & {#1}_{#8} & {#1}_{#9} & {#1}_{#10} \\
% {#2}_{#6} & {#2}_{#7} & {#2}_{#8} & {#2}_{#9} & {#2}_{#10} \\
% {#3}_{#6} & {#3}_{#7} & {#3}_{#8} & {#3}_{#9} & {#3}_{#10} \\
% {#4}_{#6} & {#4}_{#7} & {#4}_{#8} & {#4}_{#9} & {#4}_{#10} \\
% {#5}_{#6} & {#5}_{#7} & {#5}_{#8} & {#5}_{#9} & {#5}_{#10}
%\end{vmatrix}
%}

% R3 vector.
\newcommand{\VectorThree}[3]{
\begin{bmatrix}
 {#1} \\
 {#2} \\
 {#3}
\end{bmatrix}
}



\author{Peeter Joot}
\email{peeter.joot@gmail.com}


%\usepackage[english]{babel}
%\usepackage{media9}
\chapter{Steady state velocity profile of stirred cup of non-bottomless coffee.}

\label{chap:coffeeCupWithBottom}
%\useCCL
\blogpage{http://sites.google.com/site/peeterjoot2/math2012/coffeeCupWithBottom.pdf}
\date{Apr 29, 2012}
\gitRevisionInfo{coffeeCupWithBottom}
\keywords{Navier-Stokes, PHY454H1S, PHY454H1}

\beginArtWithToc
%\beginArtNoToc
%\wordpresscategory{}

\section{Motivation.}

Having tackled the problem of the spin down of a bottomless cup of coffee, lets try setting up the harder problem with a non-infinite cup.  It turns out that this is a lot harder.  Even the steady state solution requires a superposition of Bessel functions (whereas we only had that in the bottomless problem when we tried to find the time evolution after ceasing the stirring).  

I can find the form of the solution, but don't know how to actually apply the boundary value constraints.  I also find the no-slip constraints themselves become problematic, because they lead to an inconsistency at the point of contact of a moving and a static interface.  Before continuing, I need to find out how to deal with that inconsistency.

\section{Navier-Stokes for the problem.}

Working in cylindrical coordinates is the only sensible option.  Let's look first at the steady state problem, with the cup being stirred at at constant rate with the unrealistic rotating cylinder  stir stick used previously.  We'll assume an azimuthal velocity profile

\begin{equation}\label{eqn:coffeeCupWithBottom:10}
\Bu = u(r, z, t) \phicap.
\end{equation}

Let's first verify that this satisfies our incompressible condition

\begin{equation}\label{eqn:coffeeCupWithBottom:30}
\begin{aligned}
\spacegrad \cdot \Bu
&=
\left( \rcap \partial_r + \frac{\phicap}{r} \partial_\phi + \zcap \partial_z \right) \cdot ( u \phicap ) \\
&=
\cancel{\rcap \cdot \phicap} \partial_r u 
+ \frac{\phicap}{r} \cdot (u \partial_\phi \phicap + \phicap \cancel{\partial_\phi u})
+ \cancel{\zcap \cdot \phicap} \partial_z u.
\end{aligned}
\end{equation}

The only term that survives is the $\partial_\phi \phicap = -\rcap$ but that's perpendicular to $\phicap$ so we have $0 = \spacegrad \cdot \Bu$ as desired.  This leaves us with

\begin{equation}\label{eqn:coffeeCupWithBottom:50}
\frac{u}{r} \partial_\phi ( u \phicap )
= -\inv{\rho} 
\left( \rcap \partial_r + \frac{\phicap}{r} \partial_\phi + \zcap \partial_z \right) p
+ \nu 
\left( \inv{r} \partial_r ( r \partial_r ) + \frac{1}{r^2} \partial_{\phi \phi} + \partial_{zz} \right) u \phicap + \Bg
\end{equation}

Splitting into $\phicap, \rcap, \zcap$ coordinates respectively we have

\begin{subequations}
\begin{equation}\label{eqn:coffeeCupWithBottom:70}
0 = -\inv{r \rho} \partial_\phi p + \nu \left( \inv{r} \partial_r ( r \partial_r u ) - \frac{u}{r^2} + \partial_{zz} u \right) 
\end{equation}
\begin{equation}\label{eqn:coffeeCupWithBottom:90}
- \frac{u^2}{r} = -\inv{\rho} \partial_r p 
\end{equation}
\begin{equation}\label{eqn:coffeeCupWithBottom:110}
0 = -\inv{\rho} \partial_z p - g.
\end{equation}
\end{subequations}

Demanding a symmetrical solution kills off the pressure term in \ref{eqn:coffeeCupWithBottom:70}, and we can proceed with separation of variables to find the allowable solutions for the velocity before imposing our boundary value conditions.  With $u = R(r) Z(z)$ we have

\begin{equation}\label{eqn:coffeeCupWithBottom:130}
\inv{r R} ( R' + r R'' ) - \frac{1}{r^2} = -\frac{Z''}{Z} = -\inv{a^2} = \inv{b^2}
\end{equation}

Should we pick a negative or a positive constant here (ie: trig or hyperbolic functions for $Z$)?  Allowing for both temporarily, we find for $R$

\begin{subequations}
\begin{equation}\label{eqn:coffeeCupWithBottom:150}
0 = r R' + r^2 R'' + R \left( -1 + \frac{r^2}{a^2} \right) 
\end{equation}
\begin{equation}\label{eqn:coffeeCupWithBottom:170}
0 = r R' + r^2 R'' + R \left( -1 - \frac{r^2}{b^2} \right).
\end{equation}
\end{subequations}

Using Mathematica (\href{https://raw.github.com/peeterjoot/physicsplay/master/notes/phy454/coffeeCupWithBottom.cdf}{phy454/coffeeCupWithBottom.cdf}) to look up the solutions, we find that this was a Bessel equation with solutions

\begin{subequations}
\begin{equation}\label{eqn:coffeeCupWithBottom:190}
R(r) \in \Span \{ J_1(r/a), Y_1(r/a) \} 
\end{equation}
\begin{equation}\label{eqn:coffeeCupWithBottom:210}
R(r) \in \Span \{ J_1(i r/b), Y_1(i r/b) \}.
\end{equation}
\end{subequations}

However, the second set are not real valued for $r > 0$.  This means we want the hyperbolic solutions for $Z$.  Because we also have a boundary value constrain of $u = 0$ on the bottom of the cup, we can pick only hyperbolic sine solutions.  As before, we'll ``stir'' the coffee with a cylinder at radius $s$ (with cup radius $R$), our solution has the form

\begin{equation}\label{eqn:coffeeCupWithBottom:230}
u(r, z, 0) = 
\left\{
\begin{array}{l l}
s \Omega
\sum 
C_a J_1(r/a) \sinh(z/a)
& \quad \mbox{$r < s$} \\
s \Omega
\sum 
\left( D_a J_1(r/a) + E_a Y_1(r/a) \right) \sinh(z/a)
 & \quad \mbox{$r \in [s, R]$} \\
\end{array}
\right.
\end{equation}

Observe that, regardless of $a$ we have $u(0, z, 0) = 0$ because $J_1(0) = 0$.  We also can't scale $a$ as $\lambda_i/s$ (where $\lambda_i$ are the zeros of $J_1$) when the stirring isn't at the edge since we need $u(s, z, 0) = \Omega s$.  That suggests that we probably want to write our system as

\begin{equation}\label{eqn:coffeeCupWithBottom:230b}
u(r, z, 0) = 
\left\{
\begin{array}{l l}
s \Omega
\sum 
C_i J_1(\lambda_i r/R) \sinh(\lambda_i z/R)
& \quad \mbox{$r < s$} \\
s \Omega
\sum 
\left( D_i J_1(\lambda_i r/R) + E_i Y_1(\lambda_i r/R) \right) \sinh(\lambda_i z/R)
 & \quad \mbox{$r \in [s, R]$} \\
\end{array}
\right.
\end{equation}

The boundary value constraints of $u(r = s) = \Omega s$ and $u(r = R) = 0$ require the solution of

\begin{subequations}
\begin{equation}\label{eqn:coffeeCupWithBottom:310}
1 = \sum C_i J_1(\lambda_i s/R) \sinh(\lambda_i z/R) 
\end{equation}
\begin{equation}\label{eqn:coffeeCupWithBottom:330}
1 = \sum \left( D_i J_1(\lambda_i s/R) + E_i Y_1(\lambda_i s/R) \right) \sinh(\lambda_i z/R)
\end{equation}
\begin{equation}\label{eqn:coffeeCupWithBottom:350}
0 = \sum E_i Y_1(\lambda_i) \sinh(\lambda_i z/R).
\end{equation}
\end{subequations}

We know how to solve a system like \ref{eqn:coffeeCupWithBottom:310} if we did not have the $\sinh$ term in the mix.  Can we look for a numerical (least squares?) solution to this more general problem.  Will it be possible to find something that's a good fit regardless of the value of $z$?

We have other troubles too.  We can't simultaneously satisfy the boundary value condition of $u(s, z) = \Omega s$ and also $u(s, 0) = 0$.  Should we attempt something like a least squares solution and start going near $z = 0$ the small $\sinh$ values near there will start forcing the $C_i$'s arbitrarily high.

We have to somehow modify the no-slip condition so that when we have a moving interface in contact with a static interface we somehow deal with the fact that the no-slip constraints cannot simultaneously require the velocity to match both the moving and static interfaces at that point of contact.

%This is clearly going to be an easier problem to solve, if we stir the coffee on the outside edge of the cup only ($s = R$).  Lets start with the inside region and see how far we get.  We've got to match a $u(s, z, 0) = s \Omega$ boundary value constraint, where $\Omega$ is the angular velocity of the stirring.  

%We know how to solve that, since we had a similar problem for the $t = 0$ boundary value constraint in the bottomless spin down problem.  Writing $\lambda_i/s = 1/a$ where $\lambda_i$ are the zeros of the $J_1$ we can almost put things in a form that we know how to solve
%
%\begin{equation}\label{eqn:coffeeCupWithBottom:250}
%1 = \sum_i C_i J_1(\lambda_i r/s) \sinh( \lambda_i z/s)
%\end{equation}
%
%If that $\sinh$ were not there, then we could solve this with
%
%\begin{equation}\label{eqn:coffeeCupWithBottom:270}
%C_i = 
%\frac{
%\int_0^1 w J_1 (\lambda_i w) dw
%}{
%\int_0^1 w J_1^2 (\lambda_i w) dw
%}.
%\end{equation}
%
%It's not clear how to solve this more general problem.  In fact, thinking about doing so numerically, we see there's a problem.  
%
%Let's try this numerically, looking for least square solutions for the unknown coefficents that satisfy
%
%\begin{equation}\label{eqn:coffeeCupWithBottom:290}
%u(r, z, 0) = 
%\left\{
%%\begin{array}{l l}
%s \Omega
%\sum 
%C_i J_1\left(\lambda_i \frac{r}{s}\right) \sinh\left(\lambda_i \frac{z}{s}\right)
%& \quad \mbox{$r < s$} \\
%s \Omega
%\sum 
%\left( D_i J_1\left(\lambda_i \frac{r-s}{R-s}\right) + E_i Y_1\left(\lambda_i \frac{r-s}{R-s}\right) \right) \sinh\left(\lambda_i \frac{z}{R-s}\right)
% & \quad \mbox{$r \in [s, R]$} \\
%\end{array}
%\right.
%\end{equation}
%
%Hmm, this doesn't work either.  In the first we have a zero on the RHS when $r = s$.

%\EndArticle
\EndNoBibArticle


%\part{Mathematica Notebooks}
% FIXME: have phy454/ references that should be phy454/mathematica/ or just have the paths removed.
\part{Appendices}
%\part*{}
%\appendixpage
\appendix
%\begin{appendix}
% move a lot more here.
%\part{Appendix}
   %
% Copyright � 2012 Peeter Joot.  All Rights Reserved.
% Licenced as described in the file LICENSE under the root directory of this GIT repository.
%
\label{chap:appendix:strainCoordinates}
\section{Cylindrical coordinates} \index{cylindrical coordinates}

At the end of the section in the text, the formulas for the spherical and cylindrical versions (to first order) of the \textAndIndex{strain tensor} is given without derivation.  Let us do that derivation for the cylindrical case, which is simpler.  It appears that use of explicit vector notation is helpful here, so we write

\begin{equation}\label{eqn:continuumL2:370}
\begin{aligned}
\Bx &= r \rcap + z \zcap \\
\Bu & u_r \rcap + u_\phi \phicap + u_z \zcap
\end{aligned}
\end{equation}

where

\begin{equation}\label{eqn:continuumL2:390}
\begin{aligned}
\rcap &= \Be_1 e^{i\phi} \\
\phicap &= \Be_2 e^{i\phi} \\
i &= \Be_1 \Be_2
\end{aligned}
\end{equation}

Since \(\rcap\) and \(\phicap\) are functions of position, we will need their differentials

\begin{equation}\label{eqn:continuumL2:410}
\begin{aligned}
d\rcap &= \Be_1 \Be_1 \Be_2 e^{i\phi} d\phi = \Be_2 e^{i \phi} d\phi \\
d\phicap &= \Be_2 \Be_1 \Be_2 e^{i\phi} d\phi = -\Be_2 e^{i \phi} d\phi,
\end{aligned}
\end{equation}

but these are just scaled basis vectors

\begin{equation}\label{eqn:continuumL2:430}
\begin{aligned}
d\rcap &= \phicap d\phi \\
d\phicap &= -\rcap d\phi.
\end{aligned}
\end{equation}

So for our \(\Bx\) and \(\Bu\) differentials we find

\begin{equation}\label{eqn:strainOtherCoordinateSystems:1170}
\begin{aligned}
d\Bx
&= dr \rcap + r d\rcap + dz \zcap \\
&= dr \rcap + r \phicap d\phi + dz \zcap,
\end{aligned}
\end{equation}

and
\begin{equation}\label{eqn:strainOtherCoordinateSystems:1190}
\begin{aligned}
d\Bu
&= du_r \rcap + du_\phi \phicap + du_z \zcap
+ u_r \phicap d\phi - u_\phi \rcap d\phi \\
&= \rcap( du_r - u_\phi d\phi )
+ \phicap ( du_\phi + u_r d\phi )
+ \zcap ( du_z ).
\end{aligned}
\end{equation}

Putting these together we have

\begin{equation}\label{eqn:strainOtherCoordinateSystems:1210}
\begin{aligned}
d\Bl'
&= d\Bu + d\Bx
\\
&= \rcap( du_r - u_\phi d\phi + dr )
+ \phicap ( du_\phi + u_r d\phi + r d\phi )
+ \zcap ( du_z + dz ).
\end{aligned}
\end{equation}

For the squared magnitude's difference from \(d\Bx^2\) we have

\begin{equation}\label{eqn:strainOtherCoordinateSystems:1230}
\begin{aligned}
(d\Bl')^2 - d\Bx^2
&=
( du_r - u_\phi d\phi + dr )^2 \\
&+ ( du_\phi + u_r d\phi + r d\phi )^2
+ ( du_z + dz )^2
-dr^2 - r^2 d\phi^2 - dz^2 \\
&=
( du_r - u_\phi d\phi )^2
+ 2 dr ( du_r - u_\phi d\phi )
+ ( du_\phi + u_r d\phi )^2 \\
&\qquad + 2 r d\phi ( du_\phi + u_r d\phi )
+ du_z^2 + 2 du_z dz \\
\end{aligned}
\end{equation}

Expanding this out, but dropping all the terms that are quadratic in the components of \(\Bu\) or its differentials, we have

\begin{equation}\label{eqn:strainOtherCoordinateSystems:1250}
\begin{aligned}
(d\Bl')^2 - d\Bx^2
&\approx
  2 dr ( du_r - u_\phi d\phi )
+ 2 r d\phi ( du_\phi + u_r d\phi )
+ 2 du_z dz \\
&=
  2 dr du_r
- 2 dr u_\phi d\phi
+ 2 r d\phi du_\phi
+ 2 r d\phi u_r d\phi
+ 2 du_z dz
\\
&=
  2 dr
\left(
\PD{r}{u_r} dr
+\PD{\phi}{u_r} d\phi
+\PD{z}{u_r} dz
\right) \\
&- 2 dr d\phi u_\phi  \\
&+ 2 r d\phi
\left(
\PD{r}{u_\phi} dr
+\PD{\phi}{u_\phi} d\phi
+\PD{z}{u_\phi} dz
\right) \\
&+ 2 r d\phi d\phi u_r \\
&+ 2
dz
\left(
\PD{r}{u_z} dr
+\PD{\phi}{u_z} d\phi
+\PD{z}{u_z} dz
\right) \\
\end{aligned}
\end{equation}

Grouping all terms, with all the second order terms neglected, we have


\begin{dmath}\label{eqn:continuumL2:450}
(d\Bl')^2 - d\Bx^2
=
2 dr dr \PD{r}{u_r}
+ 2 r^2 d\phi d\phi \left( \inv{r} \PD{\phi}{u_\phi} +\inv{r} u_r \right)
+ 2 dz dz \PD{z}{u_z}
+ 2 dz dr \left( \PD{z}{u_r} + \PD{r}{u_z} \right)
+ 2 dr r d\phi \left( \PD{r}{u_\phi} - \inv{r} u_\phi + \inv{r} \PD{\phi}{u_r} \right)
+ 2 dz r d\phi \left( \PD{z}{u_\phi} +\inv{r} \PD{\phi}{u_z} \right).
\end{dmath}

From this we can read off the result quoted in the text

\begin{equation}\label{eqn:continuumL2:470}
\begin{aligned}
2 e_{rr} &= \PD{r}{u_r}  \\
2 e_{\phi\phi} &= \inv{r} \PD{\phi}{u_\phi} +\inv{r} u_r  \\
2 e_{zz} &= \PD{z}{u_z}  \\
2 e_{zr} &= \PD{z}{u_r} + \PD{r}{u_z} \\
2 e_{r\phi} &= \PD{r}{u_\phi} - \inv{r} u_\phi + \inv{r} \PD{\phi}{u_r} \\
2 e_{\phi z} &= \PD{z}{u_\phi} +\inv{r} \PD{\phi}{u_z}.
\end{aligned}
\end{equation}

Observe that we have to introduce factors of \(r\) along with all the \(d\phi\)'s, when we factored out the tensor components.  That is an important looking detail, which is not obvious unless one works through the derivation.

Note that in class we retained the second order terms.  That becomes a messier calculation (see \nbref{strainTensorCylindrical.cdf})

\begin{dmath}\label{eqn:continuumL2:490}
(d\Bl')^2 - d\Bx^2
=
%2 dr^2 \left(
%\PD{r}{u_r}
%+ \inv{2} \left(
%\PD{r}{u_r} \PD{r}{u_r}
%+
%\PD{r}{u_\phi} \PD{r}{u_\phi}
%+
%\PD{r}{u_z} \PD{r}{u_z}
%\right)
%\right)
% + 2 r^2 d\phi^2 \left(
%\inv{r} \PD{\phi}{u_\phi}
%+ \inv{r} u_r
%+
%\inv{2 r^2} \left(
%  \PD{\phi}{u_r} \PD{\phi}{u_r}
%+ \PD{\phi}{u_\phi} \PD{\phi}{u_\phi}
%+ \PD{\phi}{u_z} \PD{\phi}{u_z}
%\right)
%+ \inv{r^2} \left(
%u_r^2
%+
%u_\phi^2
%+
%\PD{\phi}{u_\phi} u_r
%-
%\PD{\phi}{u_r} u_\phi
%\right)
%\right)
%+ 2 dz^2 \left(
%  \PD{z}{u_z}
%+ \inv{2} \left(
%  \PD{z}{u_r} \PD{z}{u_r}
%+ \PD{z}{u_\phi} \PD{z}{u_\phi}
%+ \PD{z}{u_z} \PD{z}{u_z}
%\right)
%\right)
%+ 2
%dz
%dr
%\left(
%  \PD{r}{u_z}
%+ \PD{z}{u_r}
%+
%\left(
%\PD{r}{u_r} \PD{z}{u_r}
%+ \PD{r}{u_\phi} \PD{z}{u_\phi}
%+ \PD{r}{u_z} \PD{z}{u_z}
%\right)
%\right)
%
%+ 2
%dr
%r d\phi
%\left(
% \inv{r} \PD{\phi}{u_r}
%- \inv{r} u_\phi
%+ \PD{r}{u_\phi}
%+ \inv{r}
%\left(
%  \PD{\phi}{u_r} \PD{r}{u_r}
%+ \PD{\phi}{u_\phi} \PD{r}{u_\phi}
%+ \PD{\phi}{u_z} \PD{r}{u_z}
%\right)
%- \inv{r} \PD{r}{u_r} u_\phi
%+ \inv{r} \PD{r}{u_\phi} u_r
%\right)
%+ 2 r d\phi dz \left(
%  \inv{r} \PD{\phi}{u_z}
%+ \PD{z}{u_\phi}
%+ \inv{r}
%\left(
%  \PD{\phi}{u_r} \PD{z}{u_r}
%+ \PD{\phi}{u_\phi} \PD{z}{u_\phi}
%+ \PD{\phi}{u_z} \PD{z}{u_z}
%\right)
%- \inv{r} \PD{z}{u_r} u_\phi
%+ \inv{r} \PD{z}{u_\phi} u_r
%\right)
(dr)^2 \left(2 \frac{\partial u_r}{\partial r}+\left(\frac{\partial u_r}{\partial r}\right)^2+\left(\frac{\partial u_z}{\partial r}\right)^2+\left(\frac{\partial u_{\phi }}{\partial r}\right)^2\right)
+(d\phi )^2 \left(2 r u_r+u_r^2+u_{\phi }^2-2 u_{\phi } \frac{\partial u_r}{\partial \phi }+\left(\frac{\partial u_r}{\partial \phi }\right)^2+\left(\frac{\partial u_z}{\partial \phi }\right)^2+2 r \frac{\partial u_{\phi }}{\partial \phi }+2 u_r \frac{\partial u_{\phi }}{\partial \phi }+\left(\frac{\partial u_{\phi }}{\partial \phi }\right)^2\right)
+(dz)^2 \left(\left(\frac{\partial u_r}{\partial z}\right)^2+2 \frac{\partial u_z}{\partial z}+\left(\frac{\partial u_z}{\partial z}\right)^2+\left(\frac{\partial u_{\phi }}{\partial z}\right)^2\right)
+dr d\phi  \left(-2 u_{\phi }-2 u_{\phi } \frac{\partial u_r}{\partial r}+2 \frac{\partial u_r}{\partial \phi }+2 \frac{\partial u_r}{\partial r} \frac{\partial u_r}{\partial \phi }+2 \frac{\partial u_z}{\partial r} \frac{\partial u_z}{\partial \phi }+2 r \frac{\partial u_{\phi }}{\partial r}+2 u_r \frac{\partial u_{\phi }}{\partial r}+2 \frac{\partial u_{\phi }}{\partial r} \frac{\partial u_{\phi }}{\partial \phi }\right)
+dz d\phi  \left(-2 u_{\phi } \frac{\partial u_r}{\partial z}+2 \frac{\partial u_r}{\partial z} \frac{\partial u_r}{\partial \phi }+2 \frac{\partial u_z}{\partial \phi }+2 \frac{\partial u_z}{\partial z} \frac{\partial u_z}{\partial \phi }+2 r \frac{\partial u_{\phi }}{\partial z}+2 u_r \frac{\partial u_{\phi }}{\partial z}+2 \frac{\partial u_{\phi }}{\partial z} \frac{\partial u_{\phi }}{\partial \phi }\right)
+dr dz \left(2 \frac{\partial u_r}{\partial z}+2 \frac{\partial u_r}{\partial r} \frac{\partial u_r}{\partial z}+2 \frac{\partial u_z}{\partial r}+2 \frac{\partial u_z}{\partial r} \frac{\partial u_z}{\partial z}+2 \frac{\partial u_{\phi }}{\partial r} \frac{\partial u_{\phi }}{\partial z}\right).
\end{dmath}

As with the first order case, we can read off the tensor coordinates by inspection (once we factor out the various factors of \(2\) and \(r\)).  The next logical step would be to do the spherical tensor calculation.  That would likely be particularly messy if we attempted it in the brute force fashion.  Let us step back and look at the general case, before tackling there spherical polar form explicitly.

\section{For general coordinate representation}

Now let us dispense with the assumption that we have an orthonormal frame.  Given an arbitrary, not necessarily orthonormal, position dependent frame \(\{e_\mu\}\), and its reciprocal frame \(\{e^\mu\}\), as defined by

\begin{equation}\label{eqn:continuumL2:510}
e_\mu \cdot e^\nu = {\delta_\mu}^\nu.
\end{equation}

Our coordinate representation, with summation and dimensionality implied, is

\begin{equation}\label{eqn:continuumL2:530}
\begin{aligned}
\Bx &= x^\mu e_\mu = x_\nu e^\nu \\
\Bu &= u^\mu e_\mu = u_\nu e^\nu.
\end{aligned}
\end{equation}

Our differentials are


\begin{dmath}\label{eqn:continuumL2:550}
d\Bx
= dx^\mu e_\mu + x^\mu d e_\mu
= \sum_\alpha d\alpha \left(
\PD{\alpha}{x^\mu} e_\mu
+
x^\mu
\PD{\alpha}{e_\mu}
\right),
\end{dmath}

and


\begin{dmath}\label{eqn:continuumL2:570}
d\Bu
= du^\mu e_\mu + u^\mu d e_\mu
=
\sum_\alpha
d\alpha \left(
\PD{\alpha}{u^\mu} e_\mu
+
u^\mu
\PD{\alpha}{e_\mu}
\right).
\end{dmath}

Summing these we have

\begin{equation}\label{eqn:continuumL2:590}
d\Bu + d\Bx
=
\sum_\alpha
d\alpha \left(
\left(
\PD{\alpha}{x^\mu}
+
\PD{\alpha}{u^\mu}
\right)
e_\mu
+
\left(
x^\mu
+
u^\mu
\right)
\PD{\alpha}{e_\mu}
\right).
\end{equation}

Taking dot products to form the squares we have

\begin{equation}\label{eqn:strainOtherCoordinateSystems:1270}
\begin{aligned}
d\Bx^2
&=
\sum_{\alpha, \beta}
d\alpha
d\beta
\left(
\PD{\alpha}{x^\mu} e_\mu
+
x^\mu
\PD{\alpha}{e_\mu}
\right)
\cdot
\left(
\PD{\beta}{x_\nu} e^\nu
+
x_\nu
\PD{\beta}{e^\nu}
\right)
\\
&=
\sum_{\alpha, \beta}
d\alpha
d\beta
\left(
\PD{\alpha}{x^\mu} \PD{\beta}{x_\mu}
+
x^\mu x_\nu
\PD{\alpha}{e_\mu}
\cdot
\PD{\beta}{e^\nu}
+
2 \PD{\alpha}{x^\mu}
x_\nu
e_\mu \cdot
\PD{\beta}{e^\nu}
\right),
\end{aligned}
\end{equation}

and


\begin{dmath*}
(d\Bu + d\Bx)^2
=
\sum_{\alpha, \beta}
d\alpha
d\beta
\left(
   \left(
      \PD{\alpha}{x^\mu}
      +
      \PD{\alpha}{u^\mu}
   \right)
   e_\mu
   +
   \left(
      x^\mu
      +
      u^\mu
   \right)
   \PD{\alpha}{e_\mu}
\right)
\cdot
\left(
   \left(
      \PD{\beta}{x_\nu}
      +
      \PD{\beta}{u_\nu}
   \right)
   e^\nu
   +
   \left(
      x_\nu
      +
      u_\nu
   \right)
   \PD{\beta}{e^\nu}
\right)
=
\sum_{\alpha, \beta}
d\alpha
d\beta
\left(
   \left(
      \PD{\alpha}{x^\mu}
      +
      \PD{\alpha}{u^\mu}
   \right)
   \left(
      \PD{\beta}{x_\mu}
      +
      \PD{\beta}{u_\mu}
   \right)
   +
   \left(
      x^\mu
      +
      u^\mu
   \right)
   \left(
      x_\nu
      +
      u_\nu
   \right)
   \PD{\alpha}{e_\mu}
   \cdot
   \PD{\beta}{e^\nu}
   +
   2 \left(
         x^\mu
         +
         u^\mu
     \right)
   e^\nu
   \cdot
   \PD{\alpha}{e_\mu}
      \left(
         \PD{\beta}{x_\nu}
         +
         \PD{\beta}{u_\nu}
      \right)
   \right).
\end{dmath*}

Taking the difference we find


\begin{dmath}\label{eqn:continuumL2:610}
(d\Bu + d\Bx)^2 - d\Bx^2
=
\sum_{\alpha, \beta}
d\alpha
d\beta
\left(
\PD{\alpha}{u^\mu}
\PD{\beta}{u_\mu}
+
2
\PD{\alpha}{u^\mu}
\PD{\beta}{x_\mu}
+
\left(
u^\mu u_\nu
+
x^\mu u_\nu
+
u^\mu x_\nu
\right)
\PD{\alpha}{e_\mu}
\cdot
\PD{\beta}{e^\nu}
+
2
\left(
\PD{\alpha}{x^\mu}
u_\nu
+
\PD{\alpha}{u^\mu}(
x_\nu
+
u_\nu
)
\right)
e_\mu \cdot
\PD{\beta}{e^\nu}
\right).
\end{dmath}

To evaluate this, it is useful, albeit messier, to group terms a bit


\begin{dmath}\label{eqn:continuumL2:610b}
(d\Bu + d\Bx)^2 - d\Bx^2
=
\sum_{\alpha}
2 d\alpha
d\alpha
\left(
\inv{2}
\PD{\alpha}{u^\mu}
\PD{\alpha}{u_\mu}
+
\PD{\alpha}{u^\mu}
\PD{\alpha}{x_\mu}
+
\inv{2}
\left(
u^\mu u_\nu
+
x^\mu u_\nu
+
u^\mu x_\nu
\right)
\PD{\alpha}{e_\mu}
\cdot
\PD{\alpha}{e^\nu}
+
\left(
\PD{\alpha}{x^\mu}
u_\nu
+
\PD{\alpha}{u^\mu}(
x_\nu
+
u_\nu
)
\right)
e_\mu \cdot
\PD{\alpha}{e^\nu}
\right)
+
\sum_{\alpha < \beta}
2 d\alpha
d\beta
\left(
\PD{\alpha}{u^\mu}
\PD{\beta}{u_\mu}
+
\PD{\alpha}{u^\mu}
\PD{\beta}{x_\mu}
+
\PD{\beta}{u^\mu}
\PD{\alpha}{x_\mu}
+
\inv{2}
\left(
u^\mu u_\nu
+
x^\mu u_\nu
+
u^\mu x_\nu
\right)
\left(
\PD{\alpha}{e_\mu}
\cdot
\PD{\beta}{e^\nu}
+
\PD{\beta}{e_\mu}
\cdot
\PD{\alpha}{e^\nu}
\right)
\right)
+
\sum_{\alpha < \beta}
2 d\alpha
d\beta
\left(
\left(
\PD{\alpha}{x^\mu}
u_\nu
+
\PD{\alpha}{u^\mu}(
x_\nu
+
u_\nu
)
\right)
e_\mu \cdot
\PD{\beta}{e^\nu}
+
\left(
\PD{\beta}{x^\mu}
u_\nu
+
\PD{\beta}{u^\mu}(
x_\nu
+
u_\nu
)
\right)
e_\mu \cdot
\PD{\alpha}{e^\nu}
\right)
\end{dmath}

Here \(\alpha < \beta\) is used to denote summation over the pairs \(\alpha \ne \beta\) just once, not necessarily any numeric ordering.  For example with \(\alpha, \beta \in \{r, \phi, z\}\), this could be the set \(\{\alpha, \beta\} \in \{r \phi, \phi z, z r\}\).

\section{Cartesian tensor}

In the Cartesian case all the partials of the unit vectors are zero, and we also have no need of upper or lower indices.  We are left with just

\begin{equation}\label{eqn:continuumL2:630}
(d\Bu + d\Bx)^2 - d\Bx^2
=
\sum_{i, j, k}
dx^i
dx^j
\left(
\PD{x^i}{u^k}
\PD{x^j}{u^k}
+
2
\PD{x^i}{u^k}
\PD{x^j}{x^k}
\right)
\end{equation}

However, since we also have \(\PDi{x^j}{x^k} = \delta_{jk}\), this is

\begin{equation}\label{eqn:continuumL2:650}
(d\Bu + d\Bx)^2 - d\Bx^2
=
\sum_{i, j}
2
dx^i
dx^j
\left(
\inv{2}
\sum_k
\PD{x^i}{u^k}
\PD{x^j}{u^k}
+
\PD{x^i}{u^j}
\right).
\end{equation}

This essentially recovers the result \eqnref{eqn:continuumL2:190} derived in class.

\section{Cylindrical tensor}

Now lets do the cylindrical tensor again, but this time without resorting Mathematica brute force.

First we recall that all our basis vector derivatives are zero except for the \(\phi\) derivatives, and for those we have

\begin{equation}\label{eqn:continuumL2:670}
\begin{aligned}
\PD{\phi}{\rcap} &= \phicap \\
\PD{\phi}{\thetacap} &= -\rcap.
\end{aligned}
\end{equation}

If we write

\begin{equation}\label{eqn:continuumL2:690}
\Bx = r \rcap + z \zcap = x_r \rcap + x_\phi \phicap + x_z \zcap
\end{equation}

We have for all the \(x^\mu\) partials

\begin{equation}\label{eqn:continuumL2:710}
\PD{\alpha}{x^\mu} =
\left\{
\begin{array}{l l}
1 & \quad \mbox{if \(\alpha = x^\mu = r\) or \(\alpha = x^\mu = z\)} \\
0 & \quad \mbox{otherwise}
\end{array}
\right.
\end{equation}

We are now set to evaluate the terms in the sum of \eqnref{eqn:continuumL2:610b} for the cylindrical coordinate system and should not need Mathematica to do it.  Let us do this one at a time, starting with all the squared differential pairs.  Those are, for \(\alpha \in \{r, \phi, z\}\) the value of


\begin{dmath}\label{eqn:continuumL2:730}
2 d\alpha d\alpha
\left(
\inv{2}
\PD{\alpha}{u_m}
\PD{\alpha}{u_m}
+
\PD{\alpha}{u_m}
\PD{\alpha}{x_m}
+
\inv{2}
\left(
u_m u_n
+
x_m u_n
+
u_m x_n
\right)
\PD{\alpha}{e_m}
\cdot
\PD{\alpha}{e_n}
+
\left(
\PD{\alpha}{x_m}
u_n
+
\PD{\alpha}{u_m}(
x_n
+
u_n
)
\right)
e_m \cdot
\PD{\alpha}{e_n}
\right)
\end{dmath}

For both \(r\) and \(z\) all our unit vectors have zero derivatives so we are left respectively with

\begin{equation}\label{eqn:continuumL2:750}
2 dr dr
\left(
\inv{2}
\PD{r}{u_m}
\PD{r}{u_m}
+
\PD{r}{u_r}
\right),
\end{equation}

and

\begin{equation}\label{eqn:continuumL2:770}
2 dz dz
\left(
\inv{2}
\PD{z}{u_m}
\PD{z}{u_m}
+
\PD{z}{u_z}
\right).
\end{equation}

For the \(\alpha = \phi\) term we have


\begin{dmath*}
2 d\phi d\phi
\left(
\inv{2}
\PD{\phi}{u_m}
\PD{\phi}{u_m}
+
\inv{2}
\sum_{m = r, \phi}
\left(
u_m u_m
+
2 x_m u_m
\right)
+
\sum_{m n \in \{r \phi, \phi r\}}
\left(
\PD{\phi}{x_m}
u_n
+
\PD{\phi}{u_m}(
x_n
+
u_n
)
\right)
e_m \cdot
\PD{\phi}{e_n}
\right)
=
2 d\phi d\phi
\left(
\inv{2}
\PD{\phi}{u_m}
\PD{\phi}{u_m}
+
\inv{2} \left( u_r^2 + u_\phi^2 \right) + r u_r
-
\PD{\phi}{u_r}
u_\phi
+
\PD{\phi}{u_\phi}(
r
+
u_r
)
\right)
\end{dmath*}

Now, on to the mixed terms.  The easiest is the \(dz dr\) term, for which all the unit vector derivatives are zero, and we are left with just

\begin{equation}\label{eqn:strainOtherCoordinateSystems:1290}
\begin{aligned}
2 dz dr
\left(
\PD{z}{u_m}
\PD{r}{u_m}
+
\PD{z}{u_m}
\PD{r}{x_m}
+
\PD{r}{u_m}
\PD{z}{x_m}
\right)
=
2 dz dr
\left(
\PD{z}{u_m}
\PD{r}{u_m}
+
\PD{z}{u_r}
+
\PD{r}{u_z}
\right)
\end{aligned}
\end{equation}

Now we have the two messy mixed terms.  For the \(r\), \(\phi\) term we get


\begin{dmath*}
2 dr
d\phi
\left(
\PD{r}{u_m}
\PD{\phi}{u_m}
+
\PD{r}{u_m}
\cancel{\PD{\phi}{x_m}}
+
\PD{\phi}{u_m}
\PD{r}{x_m}
+
\inv{2}
\left(
u_m u_n
+
x_m u_n
+
u_m x_n
\right)
\left(
\cancel{\PD{r}{e_m}}
\cdot
\PD{\phi}{e_n}
+
\PD{\phi}{e_m}
\cdot
\cancel{\PD{r}{e_n} }
\right)
\right)
+
2 dr d\phi
\left(
\left(
\PD{r}{x_m}
u_n
+
\PD{r}{u_m}(
x_n
+
u_n
)
\right)
e_m \cdot
\PD{\phi}{e_n}
+
\left(
\PD{\phi}{x_m}
u_n
+
\PD{\phi}{u_m}(
x_n
+
u_n
)
\right)
e_m \cdot
\cancel{\PD{r}{e_n}}
\right)
=2 dr d\phi
\left(
\PD{r}{u_m}
\PD{\phi}{u_m}
+
\PD{\phi}{u_r}
+
u_n
\rcap \cdot
\PD{\phi}{e_n}
+
\PD{r}{u_m}(
x_n
+
u_n
)
e_m \cdot
\PD{\phi}{e_n}
\right)
=2 dr d\phi
\left(
\PD{r}{u_m}
\PD{\phi}{u_m}
+
\PD{\phi}{u_r}
-
u_\phi
+
\PD{r}{u_r}(
x_n
+
u_n
)
\rcap \cdot
\PD{\phi}{e_n}
+
\PD{r}{u_\phi}(
x_n
+
u_n
)
\phicap \cdot
\PD{\phi}{e_n}
\right)
=2 dr d\phi
\left(
\PD{r}{u_m}
\PD{\phi}{u_m}
+
\PD{\phi}{u_r}
-
u_\phi
-
\PD{r}{u_r}
u_\phi
+
\PD{r}{u_\phi}(
r
+
u_r
)
\right)
\end{dmath*}

Finally for the \(z\), \(\phi\) term we have


\begin{dmath*}
2 dz
d\phi
\left(
\PD{z}{u_m}
\PD{\phi}{u_m}
+
\PD{z}{u_m}
\cancel{\PD{\phi}{x_m} }
+
\PD{\phi}{u_m}
\PD{z}{x_m}
+
\inv{2}
\left(
u_m u_n
+
x_m u_n
+
u_m x_n
\right)
\left(
\cancel{\PD{z}{e_m}}
\cdot
\PD{\phi}{e_n}
+
\PD{\phi}{e_m}
\cdot
\cancel{\PD{z}{e_n} }
\right)
\right)
+
2 d\phi dz
\left(
\left(
\PD{z}{x_m}
u_n
+
\PD{z}{u_m}(
x_n
+
u_n
)
\right)
e_m \cdot
\PD{\phi}{e_n}
+
\left(
\PD{\phi}{x_m}
u_n
+
\PD{\phi}{u_m}(
x_n
+
u_n
)
\right)
e_m \cdot
\cancel{\PD{z}{e_n}}
\right)
=2 dz
d\phi
\left(
\PD{z}{u_m}
\PD{\phi}{u_m}
+
\PD{\phi}{u_m}
\PD{z}{x_m}
+
\cancel{
u_n
\zcap \cdot
\PD{\phi}{e_n}
}
+
\PD{z}{u_m}(
x_n
+
u_n
)
e_m \cdot
\PD{\phi}{e_n}
\right)
=2 dz
d\phi
\left(
\PD{z}{u_m}
\PD{\phi}{u_m}
+
\PD{\phi}{u_z}
-
\PD{z}{u_r}
u_\phi
+
\PD{z}{u_\phi}(
r
+
u_r
)
\right)
\end{dmath*}

To summarize we have, including both first and second order terms,


\begin{dmath}\label{eqn:continuumL2:790}
{d\Bl'}^2 - d\Bx^2
=
2 dr dr
\left(
\inv{2}
\PD{r}{u_m}
\PD{r}{u_m}
+
\PD{r}{u_r}
\right)
+
2 r^2 d\phi d\phi
\left(
\inv{2 r^2}
\PD{\phi}{u_m}
\PD{\phi}{u_m}
+
\inv{2 r^2} \left( u_r^2 + u_\phi^2 \right)
+ \frac{u_r}{r}
-
\inv{r}
\PD{\phi}{u_r}
\frac{u_\phi}{r}
+
\inv{r}
\PD{\phi}{u_\phi}\left(
1
+
\frac{u_r}{r}
\right)
\right)
+
2 dz dz
\left(
\inv{2}
\PD{z}{u_m}
\PD{z}{u_m}
+
\PD{z}{u_z}
\right)
+2 dr r d\phi
\left(
\PD{r}{u_m}
\inv{r}
\PD{\phi}{u_m}
+
\inv{r}
\PD{\phi}{u_r}
-
\frac{u_\phi}{r}
-
\PD{r}{u_r}
\frac{u_\phi}{r}
+
\PD{r}{u_\phi}\left(
1
+
\frac{u_r}{r}
\right)
\right)
+2 r d\phi dz
\left(
\PD{z}{u_m}
\inv{r}
\PD{\phi}{u_m}
+
\inv{r}
\PD{\phi}{u_z}
-
\PD{z}{u_r}
\frac{u_\phi}{r}
+
\PD{z}{u_\phi}\left(
1
+
\frac{u_r}{r}
\right)
\right)
+2 dz dr
\left(
\PD{z}{u_m}
\PD{r}{u_m}
+
\PD{z}{u_r}
+
\PD{r}{u_z}
\right)
\end{dmath}

Factors of \(r\) have been pulled out so that the portions remaining in the braces are exactly the cylindrical tensor elements as given in the text (except also with the second order terms here).  Observe that the pre-calculation of the general formula has allowed an on paper expansion of the cylindrical tensor without too much pain, and this time without requiring Mathematica.

\section{Spherical tensor}

To perform the derivation in spherical coordinates we have some setup to do first, since we need explicit representations of all three unit vectors.  The radial vector we can get easily by geometry and find the usual

\begin{equation}\label{eqn:continuumL2:810}
\rcap =
\begin{bmatrix}
\sin\theta \cos\phi \\
\sin\theta \sin\phi \\
\cos\theta
\end{bmatrix}
\end{equation}

We can get \(\phicap\) by geometrical intuition since it the plane unit vector at angle \(\phi\) rotated by \(\pi/2\).  That is

\begin{equation}\label{eqn:continuumL2:830}
\phicap =
\begin{bmatrix}
-\sin\phi \\
\cos\phi \\
0
\end{bmatrix}
\end{equation}

We can get \(\thetacap\) by utilizing the right handedness of the coordinates since

\begin{equation}\label{eqn:continuumL2:850}
\phicap \cross \rcap = \thetacap
\end{equation}

and find

\begin{equation}\label{eqn:continuumL2:870}
\thetacap =
\begin{bmatrix}
\cos\theta \cos\phi \\
\cos\theta \sin\phi \\
-\sin\theta
\end{bmatrix}
\end{equation}

Brute forcing the differential strain element calculation (\nbref{strainTensorSphericalColumnVectors.cdf}, we find

\begin{dmath}\label{eqn:continuumL2:890}
d{\Bl'}^2 - d\Bx^2
=
%(dr)^2 \left(
%2 \frac{\partial u_r}{\partial r}
%+ \frac{\partial u_r}{\partial r} \frac{\partial u_r}{\partial r}
%+ \left(\frac{\partial u_{\theta }}{\partial r}\right)^2 + \left(\frac{\partial u_{\phi }}{\partial r}\right)^2\right)
2 (dr)^2 \left(
\frac{\partial u_r}{\partial r}
+ \inv{2}
\frac{\partial u_m}{\partial r} \frac{\partial u_m}{\partial r}
\right)
% + (d\theta )^2 \left(2 r u_r + u_r^2 + u_{\theta }^2 - 2 u_{\theta } \frac{\partial u_r}{\partial \theta } + \left(\frac{\partial u_r}{\partial \theta }\right)^2 + 2 \left(r + u_r\right) \frac{\partial u_{\theta }}{\partial \theta } + \left(\frac{\partial u_{\theta }}{\partial \theta }\right)^2 + \left(\frac{\partial u_{\phi }}{\partial \theta }\right)^2\right)
 + 2 r^2 (d\theta )^2 \left(
\inv{r} u_r + \inv{2r^2}(u_r^2 + u_{\theta }^2) - \inv{r^2} u_{\theta } \frac{\partial u_r}{\partial \theta }
+ \left(\inv{r} + \inv{r^2}u_r\right) \frac{\partial u_{\theta }}{\partial \theta }
+ \inv{2 r^2} \frac{\partial u_m}{\partial \theta } \frac{\partial u_m}{\partial \theta }
\right)
%
%+ (d\phi )^2 \left( u_\phi^2 + \left(u_{\theta }^2 \right) \cos^2\theta + \left(2 r u_r + u_r^2 \right) \sin^2\theta + \left(r + u_r\right) u_{\theta } \sin (2 \theta ) - 2 u_{\phi } \sin\theta \frac{\partial u_r}{\partial \phi } + \left(\frac{\partial u_r}{\partial \phi }\right)^2 \right)
%\quad+ (d\phi )^2 \left(
%- 2 u_{\phi } \cos\theta \frac{\partial u_{\theta }}{\partial \phi } + \left(\frac{\partial u_{\theta }}{\partial \phi }\right)^2 + \frac{\partial u_{\phi }}{\partial \phi } \left(2 u_{\theta } \cos\theta + 2 \left(r + u_r\right) \sin\theta + \frac{\partial u_{\phi }}{\partial \phi }\right)\right)
+ 2 r^2 \sin^2\theta (d\phi )^2 \left(
  \inv{2 r^2 \sin^2\theta} u_\phi^2
+ \inv{2 r^2 } u_{\theta }^2 \cot^2\theta
+ \inv{r} u_r
+ \inv{2 r^2} u_r^2
+ \left(\inv{r} + \inv{r^2}u_r\right) u_{\theta } \cot\theta
\qquad
- \inv{r^2 \sin\theta} u_{\phi } \frac{\partial u_r}{\partial \phi }
- \inv{r^2 } u_{\phi } \frac{\cos\theta}{\sin^2\theta} \frac{\partial u_{\theta }}{\partial \phi }
+ \inv{r^2 } \frac{\partial u_{\phi }}{\partial \phi } \left(u_{\theta } \frac{\cos\theta}{\sin^2\theta} + \left(r + u_r\right) \inv{\sin\theta} \right)
+ \inv{2 r^2 \sin^2\theta} \frac{\partial u_m}{\partial \phi } \frac{\partial u_m}{\partial \phi }
\right)
% + 2 dr d\theta \left( - u_{\theta } + \frac{\partial u_r}{\partial \theta } + \frac{\partial u_r}{\partial r} \left( - u_{\theta } + \frac{\partial u_r}{\partial \theta }\right) + \frac{\partial u_{\theta }}{\partial r} \left(r + u_r + \frac{\partial u_{\theta }}{\partial \theta }\right) + \frac{\partial u_{\phi }}{\partial r} \frac{\partial u_{\phi }}{\partial \theta }\right)
 + 2 dr r d\theta \left(
- \inv{r} u_{\theta }
+ \inv{r} \frac{\partial u_r}{\partial \theta }
- \inv{r} u_{\theta } \frac{\partial u_r}{\partial r}
+ \frac{\partial u_{\theta }}{\partial r} \left(1 + \frac{u_r}{r} \right)
+ \inv{r} \frac{\partial u_m}{\partial r} \frac{\partial u_m}{\partial \theta }
\right)
% + 2 d\theta  d\phi  \left(u_{\theta } u_{\phi } \sin\theta - u_{\theta } \frac{\partial u_r}{\partial \phi } + \frac{\partial u_r}{\partial \theta } \left( - u_{\phi } \sin\theta + \frac{\partial u_r}{\partial \phi }\right) - u_{\phi } \cos\theta \left(r + u_r + \frac{\partial u_{\theta }}{\partial \theta }\right) + \left(r + u_r + \frac{\partial u_{\theta }}{\partial \theta }\right) \frac{\partial u_{\theta }}{\partial \phi } + \frac{\partial u_{\phi }}{\partial \theta } \left(u_{\theta } \cos\theta + \left(r + u_r\right) \sin\theta + \frac{\partial u_{\phi }}{\partial \phi }\right)\right)
 + 2 r^2 \sin\theta d\theta  d\phi  \left(
\inv{r^2 } u_{\theta } u_{\phi }
- \inv{r^2 \sin\theta} u_{\theta } \frac{\partial u_r}{\partial \phi }
- \inv{r^2 } u_{\phi } \frac{\partial u_r}{\partial \theta }
- \inv{r^2 } u_{\phi } \cot\theta \left(r + u_r + \frac{\partial u_{\theta }}{\partial \theta }\right)
\qquad
+ \inv{r^2 \sin\theta} \left(r + u_r \right) \frac{\partial u_{\theta }}{\partial \phi }
+ \frac{\partial u_{\phi }}{\partial \theta } \left(\frac{u_{\theta }}{r^2} \cot\theta + \inv{r} + \frac{u_r}{r^2} \right)
+ \inv{r^2 \sin\theta} \frac{\partial u_m}{\partial \theta } \frac{\partial u_m}{\partial \phi }
\right)
% + 2 d\phi dr \left( - u_{\phi } \sin\theta + \frac{\partial u_r}{\partial \phi } + \frac{\partial u_r}{\partial r} \left( - u_{\phi } \sin\theta + \frac{\partial u_r}{\partial \phi }\right) + \frac{\partial u_{\theta }}{\partial r} \left( - u_{\phi } \cos\theta + \frac{\partial u_{\theta }}{\partial \phi }\right) + \frac{\partial u_{\phi }}{\partial r} \left(u_{\theta } \cos\theta + \left(r + u_r\right) \sin\theta + \frac{\partial u_{\phi }}{\partial \phi }\right)\right)
 + 2 r \sin\theta d\phi dr \left(
- \inv{r } u_{\phi }
+ \inv{r \sin\theta} \frac{\partial u_r}{\partial \phi }
- u_{\phi } \inv{r } \frac{\partial u_r}{\partial r}
- u_{\phi } \cot\theta \inv{r } \frac{\partial u_{\theta }}{\partial r}
+ \inv{r } \frac{\partial u_{\phi }}{\partial r} \left( u_{\theta } \cot\theta + r + u_r \right)
+ \inv{r \sin\theta} \frac{\partial u_m}{\partial \phi } \frac{\partial u_m}{\partial r}
\right)
\end{dmath}

\section{Spherical tensor.  Manual derivation}

Doing the calculation pretty much completely with Mathematica is rather unsatisfying.  To set up for it let us first compute the unit vectors from scratch.  I will use geometric algebra to do this calculation.  Consider \cref{fig:qmTwoExamReflection:continuumL2fig5}

\imageFigure{../../figures/phy454/lec2_Composite_rotations_for_spherical_polar_unit_vectorsFig5}{Composite rotations for spherical polar unit vectors}{fig:qmTwoExamReflection:continuumL2fig5}{0.4}

We have two sets of rotations, the first is a rotation about the \(z\) axis by \(\phi\).  Writing \(i = \Be_1 \Be_2\) for the unit bivector in the \(x,y\) plane, we rotate

\begin{equation}\label{eqn:continuumL2:910}
\begin{aligned}
\Be_1' &= \Be_1 e^{i\phi} = \Be_1 \cos\phi + \Be_2 \sin\phi \\
\Be_2' &= \Be_2 e^{i\phi} = \Be_2 \cos\phi - \Be_1 \sin\phi \\
\Be_3' &= \Be_3
\end{aligned}
\end{equation}

Now we rotate in the plane spanned by \(\Be_3\) and \(\Be_1'\) by \(\theta\).  With \(j = \Be_3 \Be_1'\), our vectors in the plane rotate as

\begin{equation}\label{eqn:continuumL2:930}
\begin{aligned}
\Be_1'' &= \Be_1' e^{j\phi} = \Be_1 e^{i\phi} e^{j\theta}  \\
\Be_3'' &= \Be_3' e^{j\theta} = \Be_3 e^{j\theta},
\end{aligned}
\end{equation}

(with \(\Be_2'' = \Be_2\) since \(\Be_2 \cdot j = 0\)).

\begin{equation}\label{eqn:strainOtherCoordinateSystems:1310}
\begin{aligned}
\thetacap = \Be_1''
&= \Be_1 e^{i\phi} e^{j\theta} \\
&= \Be_1 e^{i\phi} (\cos\theta + \Be_3 \Be_1 e^{i\phi} \sin\theta) \\
&= \Be_1 e^{i\phi} \cos\theta -\Be_3 \sin\theta \\
&= (\Be_1 \cos\phi + \Be_2 \sin\phi) \cos\theta -\Be_3 \sin\theta \\
\end{aligned}
\end{equation}

\begin{equation}\label{eqn:strainOtherCoordinateSystems:1330}
\begin{aligned}
\rcap = \Be_3''
&= \Be_3 e^{j\theta} \\
&= \Be_3 (\cos\theta + \Be_3 \Be_1 e^{i\phi} \sin\theta) \\
&= \Be_3 \cos\theta + \Be_1 e^{i\phi} \sin\theta \\
&= \Be_3 \cos\theta + (\Be_1 \cos\phi + \Be_2 \sin\phi) \sin\theta \\
\end{aligned}
\end{equation}

Now, these are all the same relations that we could find with coordinate algebra

\begin{equation}\label{eqn:continuumL2:950}
\begin{aligned}
\rcap &= \Be_1 \cos\phi \sin\theta +\Be_2 \sin\phi \sin\theta +\Be_3 \cos\theta  \\
\thetacap &= \Be_1 \cos\phi \cos\theta +\Be_2 \sin\phi \cos\theta -\Be_3 \sin\theta  \\
\phicap &= -\Be_1 \sin\phi + \Be_2 \cos\phi
\end{aligned}
\end{equation}

There is nothing special in this approach if that is as far as we go, but we can put things in a nice tidy form for computation of the differentials of the unit vectors.  Introducing the unit pseudoscalar \(I = \Be_1 \Be_2 \Be_3\) we can write these in a compact exponential form.

\begin{equation}\label{eqn:strainOtherCoordinateSystems:1350}
\begin{aligned}
\rcap
&= (\Be_1 \cos\phi +\Be_2 \sin\phi ) \sin\theta +\Be_3 \cos\theta  \\
&= \Be_1 e^{i\phi} \sin\theta +\Be_3 \cos\theta  \\
&= \Be_3 ( \cos\theta + \Be_3 \Be_1 e^{i\phi} \sin\theta ) \\
&= \Be_3 ( \cos\theta + \Be_3 \Be_1 \Be_2 \Be_2 e^{i\phi} \sin\theta ) \\
&= \Be_3 ( \cos\theta + I \phicap \sin\theta ) \\
&= \Be_3 e^{ I \phicap \theta }
\end{aligned}
\end{equation}

\begin{equation}\label{eqn:strainOtherCoordinateSystems:1370}
\begin{aligned}
\thetacap
&=
\Be_1 \cos\phi \cos\theta +\Be_2 \sin\phi \cos\theta -\Be_3 \sin\theta  \\
&=
(\Be_1 \cos\phi +\Be_2 \sin\phi ) \cos\theta -\Be_3 \sin\theta  \\
&=
\Be_1 e^{i\phi} \cos\theta -\Be_3 \sin\theta  \\
&=
\Be_1 e^{i\phi} ( \cos\theta - e^{-i\phi} \Be_1 \Be_3 \sin\theta ) \\
&=
\Be_1 e^{i\phi} ( \cos\theta - \Be_1 \Be_3 e^{i\phi} \sin\theta ) \\
&=
\Be_1 e^{i\phi} ( \cos\theta - \Be_1 \Be_3 \Be_2 \Be_2 e^{i\phi} \sin\theta ) \\
&=
\Be_1 e^{i\phi} ( \cos\theta + I \phicap \sin\theta ) \\
&=
\Be_1 \Be_2 \Be_2 e^{i\phi} ( \cos\theta + I \phicap \sin\theta ) \\
&=
i \phicap e^{I \phicap \theta}.
\end{aligned}
\end{equation}

To summarize we have

\begin{equation}\label{eqn:continuumL2:970}
\begin{aligned}
\phicap &= \Be_2 e^{i\phi} \\
\rcap &= \Be_3 e^{I\phicap \theta} \\
\thetacap &= i \phicap e^{I\phicap \theta}.
\end{aligned}
\end{equation}

Taking differentials we find first

\begin{equation}\label{eqn:strainOtherCoordinateSystems:1390}
\begin{aligned}
d\phicap = \Be_2 e^{i\phi} i d\phi = \phicap i d\phi
\end{aligned}
\end{equation}

\begin{equation}\label{eqn:strainOtherCoordinateSystems:1410}
\begin{aligned}
d\thetacap
&= d \left( i \phicap e^{I\phicap \theta} \right) \\
&= i d \phicap e^{I\phicap \theta} + i \phicap d \left( \cos\theta + I \phicap \sin\theta \right) \\
&= i d \phicap e^{I\phicap \theta}
+ i \phicap I (d \phicap) \sin\theta
+ i \phicap I \phicap e^{I\phicap \theta} d\theta \\
&= i \phicap i e^{I\phicap \theta} d\phi
+ i \phicap I \phicap i \sin\theta d\phi
+ i \phicap I \phicap e^{I\phicap \theta} d\theta \\
&= \phicap e^{I\phicap \theta} d\phi
- I \sin\theta d\phi
- \Be_3 e^{I\phicap \theta} d\theta \\
&= \phicap (\cos\theta + I \phicap \sin\theta) d\phi
- I \sin\theta d\phi
- \Be_3 e^{I\phicap \theta} d\theta \\
&= \phicap \cos\theta d\phi - \rcap d\theta
\end{aligned}
\end{equation}

\begin{equation}\label{eqn:strainOtherCoordinateSystems:1430}
\begin{aligned}
d \rcap
&=
\Be_3 d \left( e^{I\phicap \theta} \right) \\
&=
\Be_3 d \left( \cos\theta + I \phicap \sin\theta \right) \\
&=
\Be_3 \left( I (d \phicap) \sin\theta + I \phicap e^{I\phicap \theta} d\theta \right) \\
&=
\Be_3 \left( I \phicap i \sin\theta d\phi + I \phicap e^{I\phicap \theta} d\theta \right) \\
&=
i \phicap i \sin\theta d\phi + i \phicap e^{I\phicap \theta} d\theta \\
&=
\phicap \sin\theta d\phi + \thetacap d\theta
\end{aligned}
\end{equation}

Summarizing these differentials we have
\begin{equation}\label{eqn:continuumL2:990}
\begin{aligned}
d\rcap &= \phicap \sin\theta d\phi + \thetacap d\theta \\
d\thetacap &= \phicap \cos\theta d\phi - \rcap d\theta \\
d\phicap &= \phicap i d\phi
\end{aligned}
\end{equation}

A final cleanup is required.  While \(\phicap i\) is a vector and has a nicely compact form, we need to decompose this into components in the \(\rcap\), \(\thetacap\) and \(\phicap\) directions.  Taking scalar products we have

\begin{equation}\label{eqn:strainOtherCoordinateSystems:1450}
\begin{aligned}
\phicap \cdot (\phicap i) = 0
\end{aligned}
\end{equation}

\begin{equation}\label{eqn:strainOtherCoordinateSystems:1470}
\begin{aligned}
\rcap \cdot (\phicap i)
&=
\gpgradezero{ \rcap \phicap i} \\
&=
\gpgradezero{ \Be_3 e^{I\phicap \theta} \Be_2 e^{i\phi} i} \\
&=
\gpgradezero{ \Be_3 (\cos\theta + I \Be_2 e^{i\phi} \sin\theta) \Be_2 e^{i\phi} i} \\
&=
\gpgradezero{ I (\cos\theta e^{-i\phi} + I \Be_2 \sin\theta) \Be_2 } \\
&=
-\sin\theta
\end{aligned}
\end{equation}

\begin{equation}\label{eqn:strainOtherCoordinateSystems:1490}
\begin{aligned}
\thetacap \cdot (\phicap i)
&=
\gpgradezero{ \thetacap \phicap i } \\
&=
\gpgradezero{ i \phicap e^{I\phicap \theta} \phicap i } \\
&=
-\gpgradezero{ \phicap e^{I\phicap \theta} \phicap } \\
&=
-\gpgradezero{ e^{I\phicap \theta} } \\
&=
- \cos\theta.
\end{aligned}
\end{equation}

Summarizing once again, but this time in terms of \(\rcap\), \(\thetacap\) and \(\phicap\) we have

\begin{equation}\label{eqn:continuumL2:1010}
\begin{aligned}
d\rcap &= \phicap \sin\theta d\phi + \thetacap d\theta \\
d\thetacap &= \phicap \cos\theta d\phi - \rcap d\theta \\
d\phicap &= -(\rcap \sin\theta + \thetacap \cos\theta) d\phi
\end{aligned}
\end{equation}

Now we are set to take differentials.  With

\begin{equation}\label{eqn:continuumL2:1030}
\Bx = r \rcap,
\end{equation}

we have

\begin{equation}\label{eqn:continuumL2:1050}
d\Bx =
dr \rcap
+ r d\rcap
=
dr \rcap + \phicap r \sin\theta d\phi + r \thetacap d\theta.
\end{equation}

Squaring this we get the usual spherical polar line scalar line element

\begin{equation}\label{eqn:continuumL2:1070}
d\Bx^2 = dr^2 + r^2 \sin^2\theta d\phi^2 + r^2 d\theta^2.
\end{equation}

With

\begin{equation}\label{eqn:continuumL2:1090}
\Bu = u_r \rcap + u_\theta \thetacap + u_\phi \phicap,
\end{equation}

our differential is

\begin{equation}\label{eqn:strainOtherCoordinateSystems:1510}
\begin{aligned}
d\Bu
&=
du_r \rcap + du_\theta \thetacap + du_\phi \phicap
+ u_r d\rcap + u_\theta d\thetacap + u_\phi d \phicap \\
&=
du_r \rcap + du_\theta \thetacap + du_\phi \phicap
+ u_r \left(\phicap \sin\theta d\phi + \thetacap d\theta \right) \\
&\qquad + u_\theta \left( \phicap \cos\theta d\phi - \rcap d\theta \right)
- u_\phi (\rcap \sin\theta + \thetacap \cos\theta) d\phi
\\
&=
\rcap \left( du_r - u_\theta d\theta - u_\phi \sin\theta d\phi \right) \\
&\qquad +\thetacap \left( du_\theta + u_r d\theta - u_\phi \cos\theta d\phi \right) \\
&\qquad +\phicap \left( du_\phi + u_r \sin\theta d\phi + u_\theta \cos\theta d\phi \right).
\end{aligned}
\end{equation}

We can add \(d\Bx\) to this and take differences


\begin{dmath}\label{eqn:continuumL2:1110}
(d\Bu + d\Bx)^2 - d\Bx^2
=
\left( du_r - u_\theta d\theta - u_\phi \sin\theta d\phi + dr \right)^2
+\left( du_\theta + u_r d\theta - u_\phi \cos\theta d\phi + r d\theta \right)^2
+\left( du_\phi + u_r \sin\theta d\phi + u_\theta \cos\theta d\phi + r \sin\theta d\phi \right)^2
\end{dmath}

For each \(m = r,\theta,\phi\) we have

\begin{equation}\label{eqn:continuumL2:1130}
du_m
=
\PD{r}{u_m} dr +
\PD{\theta}{u_m} d\theta +
\PD{\phi}{u_m} d\phi,
\end{equation}

and plugging through that calculation is really all it takes to derive the textbook result.  To do this to first order in \(u_m\), we find

\begin{equation}\label{eqn:strainOtherCoordinateSystems:1530}
\begin{aligned}
\inv{2} \left((d\Bu + d\Bx)^2 - d\Bx^2\right)
&=
du_r dr
- u_\theta d\theta dr
- u_\phi \sin\theta d\phi dr  \\
&+ du_\theta r d\theta
+ u_r r d\theta^2
- u_\phi r \cos\theta d\phi d\theta \\
&+ r \sin\theta du_\phi d\phi
+ r \sin^2\theta u_r d\phi^2
+ r \sin\theta \cos\theta u_\theta d\phi^2 \\
&=
\left( \PD{r}{u_r} dr + \PD{\theta}{u_r} d\theta + \PD{\phi}{u_r} d\phi \right)
dr
- u_\theta d\theta dr
- u_\phi \sin\theta d\phi dr  \\
&+
\left( \PD{r}{u_\theta} dr + \PD{\theta}{u_\theta} d\theta + \PD{\phi}{u_\theta} d\phi \right)
 r d\theta
+ u_r r d\theta^2
- u_\phi r \cos\theta d\phi d\theta \\
&+
\left( \PD{r}{u_\phi} dr + \PD{\theta}{u_\phi} d\theta + \PD{\phi}{u_\phi} d\phi \right)
r \sin\theta d\phi
+ r \sin^2\theta u_r d\phi^2 \\
&+ r \sin\theta \cos\theta u_\theta d\phi^2
\end{aligned}
\end{equation}

Collecting terms we have the result of the text in the braces


\begin{dmath}\label{eqn:continuumL2:1150}
\left((d\Bu + d\Bx)^2 - d\Bx^2\right)
=
2 dr^2 \left(
\PD{r}{u_r}
\right)
+
2 r^2 d\theta^2 \left(
\inv{r} \PD{\theta}{u_\theta} + u_r \inv{r}
\right)
+2 r^2 \sin^2\theta d\phi^2 \left(
\PD{\phi}{u_\phi} \inv{r \sin\theta} + \inv{r} u_r + \inv{r} \cot\theta u_\theta
\right)
+2 dr r d\theta \left(
\inv{r} \PD{\theta}{u_r} - \inv{r} u_\theta +\PD{r}{u_\theta}
\right)
%+2 d\theta d\phi \left(
%\PD{\phi}{u_\theta} r - u_\phi r \cos\theta +\PD{\theta}{u_\phi} r \sin\theta
%\right)
+2 r^2 \sin\theta d\theta d\phi \left(
\PD{\phi}{u_\theta} \inv{r \sin\theta} - \inv{r} u_\phi \cot\theta +\inv{r} \PD{\theta}{u_\phi}
\right)
%+2 d\phi dr \left(
%\PD{\phi}{u_r} - u_\phi \sin\theta + \PD{r}{u_\phi} r \sin\theta
%\right)
+2 r \sin\theta d\phi dr \left(
\inv{r \sin\theta} \PD{\phi}{u_r} - \inv{r} u_\phi + \PD{r}{u_\phi}
\right)
\end{dmath}

It should be possible to do the calculation to second order too, but to include all the quadratic terms in \(u_m\) is again really messy.  Trying that with Mathematica (\nbref{strainTensorSpherical.cdf}) gives the same results as above using the strictly coordinate algebra approach.

   %
% Copyright � 2015 Peeter Joot.  All Rights Reserved.
% Licenced as described in the file LICENSE under the root directory of this GIT repository.
%
\documentclass[]{eliblog}

\usepackage{amsmath}
\usepackage{mathpazo}

%
% shorthand for bold symbols, convenient for vectors and matrices
%
\newcommand{\Ba}[0]{\mathbf{a}}
\newcommand{\Bb}[0]{\mathbf{b}}
\newcommand{\Bc}[0]{\mathbf{c}}
\newcommand{\Bd}[0]{\mathbf{d}}
\newcommand{\Be}[0]{\mathbf{e}}
\newcommand{\Bf}[0]{\mathbf{f}}
\newcommand{\Bg}[0]{\mathbf{g}}
\newcommand{\Bh}[0]{\mathbf{h}}
\newcommand{\Bi}[0]{\mathbf{i}}
\newcommand{\Bj}[0]{\mathbf{j}}
\newcommand{\Bk}[0]{\mathbf{k}}
\newcommand{\Bl}[0]{\mathbf{l}}
\newcommand{\Bm}[0]{\mathbf{m}}
\newcommand{\Bn}[0]{\mathbf{n}}
\newcommand{\Bo}[0]{\mathbf{o}}
\newcommand{\Bp}[0]{\mathbf{p}}
\newcommand{\Bq}[0]{\mathbf{q}}
\newcommand{\Br}[0]{\mathbf{r}}
\newcommand{\Bs}[0]{\mathbf{s}}
\newcommand{\Bt}[0]{\mathbf{t}}
\newcommand{\Bu}[0]{\mathbf{u}}
\newcommand{\Bv}[0]{\mathbf{v}}
\newcommand{\Bw}[0]{\mathbf{w}}
\newcommand{\Bx}[0]{\mathbf{x}}
\newcommand{\By}[0]{\mathbf{y}}
\newcommand{\Bz}[0]{\mathbf{z}}
\newcommand{\BA}[0]{\mathbf{A}}
\newcommand{\BB}[0]{\mathbf{B}}
\newcommand{\BC}[0]{\mathbf{C}}
\newcommand{\BD}[0]{\mathbf{D}}
\newcommand{\BE}[0]{\mathbf{E}}
\newcommand{\BF}[0]{\mathbf{F}}
\newcommand{\BG}[0]{\mathbf{G}}
\newcommand{\BH}[0]{\mathbf{H}}
\newcommand{\BI}[0]{\mathbf{I}}
\newcommand{\BJ}[0]{\mathbf{J}}
\newcommand{\BK}[0]{\mathbf{K}}
\newcommand{\BL}[0]{\mathbf{L}}
\newcommand{\BM}[0]{\mathbf{M}}
\newcommand{\BN}[0]{\mathbf{N}}
\newcommand{\BO}[0]{\mathbf{O}}
\newcommand{\BP}[0]{\mathbf{P}}
\newcommand{\BQ}[0]{\mathbf{Q}}
\newcommand{\BR}[0]{\mathbf{R}}
\newcommand{\BS}[0]{\mathbf{S}}
\newcommand{\BT}[0]{\mathbf{T}}
\newcommand{\BU}[0]{\mathbf{U}}
\newcommand{\BV}[0]{\mathbf{V}}
\newcommand{\BW}[0]{\mathbf{W}}
\newcommand{\BX}[0]{\mathbf{X}}
\newcommand{\BY}[0]{\mathbf{Y}}
\newcommand{\BZ}[0]{\mathbf{Z}}

\newcommand{\Bzero}[0]{\mathbf{0}}
\newcommand{\Btheta}[0]{\boldsymbol{\theta}}
\newcommand{\Btau}[0]{\boldsymbol{\tau}}
\newcommand{\Bomega}[0]{\boldsymbol{\omega}}

%
% shorthand for unit vectors
%
\newcommand{\acap}[0]{\hat{\Ba}}
\newcommand{\bcap}[0]{\hat{\Bb}}
\newcommand{\ccap}[0]{\hat{\Bc}}
\newcommand{\dcap}[0]{\hat{\Bd}}
\newcommand{\ecap}[0]{\hat{\Be}}
\newcommand{\fcap}[0]{\hat{\Bf}}
\newcommand{\gcap}[0]{\hat{\Bg}}
\newcommand{\hcap}[0]{\hat{\Bh}}
\newcommand{\icap}[0]{\hat{\Bi}}
\newcommand{\jcap}[0]{\hat{\Bj}}
\newcommand{\kcap}[0]{\hat{\Bk}}
\newcommand{\lcap}[0]{\hat{\Bl}}
\newcommand{\mcap}[0]{\hat{\Bm}}
\newcommand{\ncap}[0]{\hat{\Bn}}
\newcommand{\ocap}[0]{\hat{\Bo}}
\newcommand{\pcap}[0]{\hat{\Bp}}
\newcommand{\qcap}[0]{\hat{\Bq}}
\newcommand{\rcap}[0]{\hat{\Br}}
\newcommand{\scap}[0]{\hat{\Bs}}
\newcommand{\tcap}[0]{\hat{\Bt}}
\newcommand{\ucap}[0]{\hat{\Bu}}
\newcommand{\vcap}[0]{\hat{\Bv}}
\newcommand{\wcap}[0]{\hat{\Bw}}
\newcommand{\xcap}[0]{\hat{\Bx}}
\newcommand{\ycap}[0]{\hat{\By}}
\newcommand{\zcap}[0]{\hat{\Bz}}
\newcommand{\thetacap}[0]{\hat{\Btheta}}

%
% to write R^n and C^n in a distinguishable fashion.  Perhaps change this
% to the double lined characters upon figuring out how to do so.
%
\newcommand{\C}[1]{$\mathbb{C}^{#1}$}
\newcommand{\R}[1]{$\mathbb{R}^{#1}$}

%
% various generally useful helpers
%

% derivative of #1 wrt. #2:
\newcommand{\D}[2] {\frac {d#2} {d#1}}

\newcommand{\inv}[1]{\frac{1}{#1}}
\newcommand{\cross}[0]{\times}

\newcommand{\abs}[1]{\lvert{#1}\rvert}
\newcommand{\norm}[1]{\lVert{#1}\rVert}
\newcommand{\innerprod}[2]{\langle{#1}, {#2}\rangle}
\newcommand{\dotprod}[2]{{#1} \cdot {#2}}
\newcommand{\bdotprod}[2]{\left({#1} \cdot {#2}\right)}
\newcommand{\crossprod}[2]{{#1} \cross {#2}}
\newcommand{\tripleprod}[3]{\dotprod{\left(\crossprod{#1}{#2}\right)}{#3}}

\DeclareMathOperator{\Proj}{Proj}
\DeclareMathOperator{\Span}{span}
\DeclareMathOperator{\Sgn}{sgn}
\DeclareMathOperator{\Area}{Area}
\DeclareMathOperator{\Volume}{Volume}

%
% A few miscellaneous things specific to this document
%
\newcommand{\crossop}[1]{\crossprod{#1}{}}

% R2 vector.
\newcommand{\VectorTwo}[2]{
\begin{bmatrix}
 {#1} \\
 {#2}
\end{bmatrix}
}

\newcommand{\VectorN}[1]{
\begin{bmatrix}
{#1}_1 \\
{#1}_2 \\
\vdots \\
{#1}_N \\
\end{bmatrix}
}

\newcommand{\DETuvij}[4]{
\begin{vmatrix}
 {#1}_{#3} & {#1}_{#4} \\
 {#2}_{#3} & {#2}_{#4}
\end{vmatrix}
}

\newcommand{\DETuvwijk}[6]{
\begin{vmatrix}
 {#1}_{#4} & {#1}_{#5} & {#1}_{#6} \\
 {#2}_{#4} & {#2}_{#5} & {#2}_{#6} \\
 {#3}_{#4} & {#3}_{#5} & {#3}_{#6}
\end{vmatrix}
}

\newcommand{\DETuvwxijkl}[8]{
\begin{vmatrix}
 {#1}_{#5} & {#1}_{#6} & {#1}_{#7} & {#1}_{#8} \\
 {#2}_{#5} & {#2}_{#6} & {#2}_{#7} & {#2}_{#8} \\
 {#3}_{#5} & {#3}_{#6} & {#3}_{#7} & {#3}_{#8} \\
 {#4}_{#5} & {#4}_{#6} & {#4}_{#7} & {#4}_{#8} \\
\end{vmatrix}
}

%\newcommand{\DETuvwxyijklm}[10]{
%\begin{vmatrix}
% {#1}_{#6} & {#1}_{#7} & {#1}_{#8} & {#1}_{#9} & {#1}_{#10} \\
% {#2}_{#6} & {#2}_{#7} & {#2}_{#8} & {#2}_{#9} & {#2}_{#10} \\
% {#3}_{#6} & {#3}_{#7} & {#3}_{#8} & {#3}_{#9} & {#3}_{#10} \\
% {#4}_{#6} & {#4}_{#7} & {#4}_{#8} & {#4}_{#9} & {#4}_{#10} \\
% {#5}_{#6} & {#5}_{#7} & {#5}_{#8} & {#5}_{#9} & {#5}_{#10}
%\end{vmatrix}
%}

% R3 vector.
\newcommand{\VectorThree}[3]{
\begin{bmatrix}
 {#1} \\
 {#2} \\
 {#3}
\end{bmatrix}
}



\author{Peeter Joot}
\email{peeter.joot@gmail.com}

%\documentclass[]{eliblogwidescreen}

\usepackage{amsmath}
\usepackage{mathpazo}

%
% shorthand for bold symbols, convenient for vectors and matrices
%
\newcommand{\Ba}[0]{\mathbf{a}}
\newcommand{\Bb}[0]{\mathbf{b}}
\newcommand{\Bc}[0]{\mathbf{c}}
\newcommand{\Bd}[0]{\mathbf{d}}
\newcommand{\Be}[0]{\mathbf{e}}
\newcommand{\Bf}[0]{\mathbf{f}}
\newcommand{\Bg}[0]{\mathbf{g}}
\newcommand{\Bh}[0]{\mathbf{h}}
\newcommand{\Bi}[0]{\mathbf{i}}
\newcommand{\Bj}[0]{\mathbf{j}}
\newcommand{\Bk}[0]{\mathbf{k}}
\newcommand{\Bl}[0]{\mathbf{l}}
\newcommand{\Bm}[0]{\mathbf{m}}
\newcommand{\Bn}[0]{\mathbf{n}}
\newcommand{\Bo}[0]{\mathbf{o}}
\newcommand{\Bp}[0]{\mathbf{p}}
\newcommand{\Bq}[0]{\mathbf{q}}
\newcommand{\Br}[0]{\mathbf{r}}
\newcommand{\Bs}[0]{\mathbf{s}}
\newcommand{\Bt}[0]{\mathbf{t}}
\newcommand{\Bu}[0]{\mathbf{u}}
\newcommand{\Bv}[0]{\mathbf{v}}
\newcommand{\Bw}[0]{\mathbf{w}}
\newcommand{\Bx}[0]{\mathbf{x}}
\newcommand{\By}[0]{\mathbf{y}}
\newcommand{\Bz}[0]{\mathbf{z}}
\newcommand{\BA}[0]{\mathbf{A}}
\newcommand{\BB}[0]{\mathbf{B}}
\newcommand{\BC}[0]{\mathbf{C}}
\newcommand{\BD}[0]{\mathbf{D}}
\newcommand{\BE}[0]{\mathbf{E}}
\newcommand{\BF}[0]{\mathbf{F}}
\newcommand{\BG}[0]{\mathbf{G}}
\newcommand{\BH}[0]{\mathbf{H}}
\newcommand{\BI}[0]{\mathbf{I}}
\newcommand{\BJ}[0]{\mathbf{J}}
\newcommand{\BK}[0]{\mathbf{K}}
\newcommand{\BL}[0]{\mathbf{L}}
\newcommand{\BM}[0]{\mathbf{M}}
\newcommand{\BN}[0]{\mathbf{N}}
\newcommand{\BO}[0]{\mathbf{O}}
\newcommand{\BP}[0]{\mathbf{P}}
\newcommand{\BQ}[0]{\mathbf{Q}}
\newcommand{\BR}[0]{\mathbf{R}}
\newcommand{\BS}[0]{\mathbf{S}}
\newcommand{\BT}[0]{\mathbf{T}}
\newcommand{\BU}[0]{\mathbf{U}}
\newcommand{\BV}[0]{\mathbf{V}}
\newcommand{\BW}[0]{\mathbf{W}}
\newcommand{\BX}[0]{\mathbf{X}}
\newcommand{\BY}[0]{\mathbf{Y}}
\newcommand{\BZ}[0]{\mathbf{Z}}

\newcommand{\Bzero}[0]{\mathbf{0}}
\newcommand{\Btheta}[0]{\boldsymbol{\theta}}
\newcommand{\Btau}[0]{\boldsymbol{\tau}}
\newcommand{\Bomega}[0]{\boldsymbol{\omega}}

%
% shorthand for unit vectors
%
\newcommand{\acap}[0]{\hat{\Ba}}
\newcommand{\bcap}[0]{\hat{\Bb}}
\newcommand{\ccap}[0]{\hat{\Bc}}
\newcommand{\dcap}[0]{\hat{\Bd}}
\newcommand{\ecap}[0]{\hat{\Be}}
\newcommand{\fcap}[0]{\hat{\Bf}}
\newcommand{\gcap}[0]{\hat{\Bg}}
\newcommand{\hcap}[0]{\hat{\Bh}}
\newcommand{\icap}[0]{\hat{\Bi}}
\newcommand{\jcap}[0]{\hat{\Bj}}
\newcommand{\kcap}[0]{\hat{\Bk}}
\newcommand{\lcap}[0]{\hat{\Bl}}
\newcommand{\mcap}[0]{\hat{\Bm}}
\newcommand{\ncap}[0]{\hat{\Bn}}
\newcommand{\ocap}[0]{\hat{\Bo}}
\newcommand{\pcap}[0]{\hat{\Bp}}
\newcommand{\qcap}[0]{\hat{\Bq}}
\newcommand{\rcap}[0]{\hat{\Br}}
\newcommand{\scap}[0]{\hat{\Bs}}
\newcommand{\tcap}[0]{\hat{\Bt}}
\newcommand{\ucap}[0]{\hat{\Bu}}
\newcommand{\vcap}[0]{\hat{\Bv}}
\newcommand{\wcap}[0]{\hat{\Bw}}
\newcommand{\xcap}[0]{\hat{\Bx}}
\newcommand{\ycap}[0]{\hat{\By}}
\newcommand{\zcap}[0]{\hat{\Bz}}
\newcommand{\thetacap}[0]{\hat{\Btheta}}

%
% to write R^n and C^n in a distinguishable fashion.  Perhaps change this
% to the double lined characters upon figuring out how to do so.
%
\newcommand{\C}[1]{$\mathbb{C}^{#1}$}
\newcommand{\R}[1]{$\mathbb{R}^{#1}$}

%
% various generally useful helpers
%

% derivative of #1 wrt. #2:
\newcommand{\D}[2] {\frac {d#2} {d#1}}

\newcommand{\inv}[1]{\frac{1}{#1}}
\newcommand{\cross}[0]{\times}

\newcommand{\abs}[1]{\lvert{#1}\rvert}
\newcommand{\norm}[1]{\lVert{#1}\rVert}
\newcommand{\innerprod}[2]{\langle{#1}, {#2}\rangle}
\newcommand{\dotprod}[2]{{#1} \cdot {#2}}
\newcommand{\bdotprod}[2]{\left({#1} \cdot {#2}\right)}
\newcommand{\crossprod}[2]{{#1} \cross {#2}}
\newcommand{\tripleprod}[3]{\dotprod{\left(\crossprod{#1}{#2}\right)}{#3}}

\DeclareMathOperator{\Proj}{Proj}
\DeclareMathOperator{\Span}{span}
\DeclareMathOperator{\Sgn}{sgn}
\DeclareMathOperator{\Area}{Area}
\DeclareMathOperator{\Volume}{Volume}

%
% A few miscellaneous things specific to this document
%
\newcommand{\crossop}[1]{\crossprod{#1}{}}

% R2 vector.
\newcommand{\VectorTwo}[2]{
\begin{bmatrix}
 {#1} \\
 {#2}
\end{bmatrix}
}

\newcommand{\VectorN}[1]{
\begin{bmatrix}
{#1}_1 \\
{#1}_2 \\
\vdots \\
{#1}_N \\
\end{bmatrix}
}

\newcommand{\DETuvij}[4]{
\begin{vmatrix}
 {#1}_{#3} & {#1}_{#4} \\
 {#2}_{#3} & {#2}_{#4}
\end{vmatrix}
}

\newcommand{\DETuvwijk}[6]{
\begin{vmatrix}
 {#1}_{#4} & {#1}_{#5} & {#1}_{#6} \\
 {#2}_{#4} & {#2}_{#5} & {#2}_{#6} \\
 {#3}_{#4} & {#3}_{#5} & {#3}_{#6}
\end{vmatrix}
}

\newcommand{\DETuvwxijkl}[8]{
\begin{vmatrix}
 {#1}_{#5} & {#1}_{#6} & {#1}_{#7} & {#1}_{#8} \\
 {#2}_{#5} & {#2}_{#6} & {#2}_{#7} & {#2}_{#8} \\
 {#3}_{#5} & {#3}_{#6} & {#3}_{#7} & {#3}_{#8} \\
 {#4}_{#5} & {#4}_{#6} & {#4}_{#7} & {#4}_{#8} \\
\end{vmatrix}
}

%\newcommand{\DETuvwxyijklm}[10]{
%\begin{vmatrix}
% {#1}_{#6} & {#1}_{#7} & {#1}_{#8} & {#1}_{#9} & {#1}_{#10} \\
% {#2}_{#6} & {#2}_{#7} & {#2}_{#8} & {#2}_{#9} & {#2}_{#10} \\
% {#3}_{#6} & {#3}_{#7} & {#3}_{#8} & {#3}_{#9} & {#3}_{#10} \\
% {#4}_{#6} & {#4}_{#7} & {#4}_{#8} & {#4}_{#9} & {#4}_{#10} \\
% {#5}_{#6} & {#5}_{#7} & {#5}_{#8} & {#5}_{#9} & {#5}_{#10}
%\end{vmatrix}
%}

% R3 vector.
\newcommand{\VectorThree}[3]{
\begin{bmatrix}
 {#1} \\
 {#2} \\
 {#3}
\end{bmatrix}
}



\author{Peeter Joot}
\email{peeter.joot@gmail.com}


\chapter{Putting the stress tensor (and traction vector) into explicit vector form.}
\label{chap:continuumstressTensorVectorForm}
%\useCCL
\blogpage{http://sites.google.com/site/peeterjoot2/math2012/continuumstressTensorVectorForm.pdf}
\date{Apr 4, 2012}
\gitRevisionInfo{continuumstressTensorVectorForm}
\keywords{Navier-Stokes, PHY454H1S} 

\beginArtWithToc
%\beginArtNoToc

\section{Motivation.}

Exersize 6.1 from \cite{acheson1990elementary} is to show that the traction vector can be written in vector form (a rather curious thing to have to say) as

\begin{equation}\label{eqn:stressTensorVectorForm:10}
\Bt = -p \ncap + \mu ( 2 (\ncap \cdot \spacegrad)\Bu + \ncap \cross (\spacegrad \cross \Bu)).
\end{equation}

Note that the text uses a wedge symbol for the cross product, and I've switched to standard notation.  I've done so because the use of a Geometric-Algebra wedge product also can be used to express this relationship, in which case we would write

\begin{equation}\label{eqn:stressTensorVectorForm:30}
\Bt = -p \ncap + \mu ( 2 (\ncap \cdot \spacegrad) \Bu + (\spacegrad \wedge \Bu) \cdot \ncap).
\end{equation}

In either case we have

\begin{equation}\label{eqn:stressTensorVectorForm:50}
(\spacegrad \wedge \Bu) \cdot \ncap
= 
\ncap \cross (\spacegrad \cross \Bu)
=
\spacegrad' (\ncap \cdot \Bu') - (\ncap \cdot \spacegrad) \Bu 
\end{equation}

(where the primes indicate the scope of the gradient, showing here that we are operating only on $\Bu$, and not $\ncap$).

After computing this, lets also compute the stress tensor in cylindrical and spherical coordinates, something that this allows us to do fairly easily without having to deal with the second order terms that we encountered doing this by computing the difference of squared displacements.

We'll work primarily with just the strain tensor portion of the traction vector expressions above, calculating

\begin{equation}\label{eqn:stressTensorVectorForm:250}
2 {\Be}_{\ncap}
=
2 (\ncap \cdot \spacegrad)\Bu + \ncap \cross (\spacegrad \cross \Bu)
=
2 (\ncap \cdot \spacegrad)\Bu + (\spacegrad \wedge \Bu) \cdot \ncap.
\end{equation}

\section{Verifying the relationship.}

Let's start with the the plain old cross product version

\begin{align*}
(\ncap \cross (\spacegrad \cross \Bu) + 2 (\ncap \cdot \spacegrad) \Bu)_i
&=
n_a (\spacegrad \cross \Bu)_b \epsilon_{a b i}  + 2 n_a \partial_a u_i \\
&=
n_a \partial_r u_s \epsilon_{r s b} \epsilon_{a b i}  + 2 n_a \partial_a u_i \\
&=
n_a \partial_r u_s \delta_{ia}^{[rs]} + 2 n_a \partial_a u_i \\
&=
n_a ( \partial_i u_a -\partial_a u_i ) + 2 n_a \partial_a u_i \\
&=
n_a \partial_i u_a + n_a \partial_a u_i \\
&=
n_a (\partial_i u_a + \partial_a u_i) \\
&=
\sigma_{i a } n_a 
\end{align*}

We can also put the double cross product in wedge product form

\begin{align*}
\ncap \cross (\spacegrad \cross \Bu)
&=
-I \ncap \wedge (\spacegrad \cross \Bu) \\
&=
-\frac{I}{2} 
\left(
\ncap (\spacegrad \cross \Bu) 
- (\spacegrad \cross \Bu) \ncap
\right) \\
&=
-\frac{I}{2} 
\left(
-I \ncap (\spacegrad \wedge \Bu) 
+ I (\spacegrad \wedge \Bu) \ncap
\right) \\
&=
-\frac{I^2}{2} 
\left(
- \ncap (\spacegrad \wedge \Bu) 
+ (\spacegrad \wedge \Bu) \ncap
\right) \\
&=
(\spacegrad \wedge \Bu) \cdot \ncap
\end{align*}

Equivalently (and easier) we can just expand the dot product of the wedge and the vector using the relationship

\begin{equation}\label{eqn:stressTensorVectorForm:70}
\Ba \cdot (\Bc \wedge \Bd \wedge \Be \wedge \cdots )
=
(\Ba \cdot \Bc) (\Bd \wedge \Be \wedge \cdots ) - (\Ba \cdot \Bd) (\Bc \wedge \Be \wedge \cdots ) +
\end{equation}

so we find

\begin{align*}
((\spacegrad \wedge \Bu) \cdot \ncap + 2 (\ncap \cdot \spacegrad) \Bu
)_i
&=
(
\spacegrad' (\Bu' \cdot \ncap)
-
(\ncap \cdot \spacegrad) \Bu
+ 2 (\ncap \cdot \spacegrad) \Bu
)_i \\
&=
\partial_i u_a n_a
+
n_a \partial_a u_i \\
&=
\sigma_{ia} n_a.
\end{align*}

\section{Cylindrical strain tensor.}

Let's now compute the strain tensor (and implicitly the traction vector) in cylindrical coordinates.

\section{Spherical strain tensor.}

Having done a first order cylindrical derivation of the strain tensor, let's also do the spherical case for completeness.  Would this have much utility in fluids?  Perhaps for flow over a spherical barrier?

We need the gradient in spherical coordinates.  Recall that our spherical coordinate velocity was

\begin{equation}\label{eqn:stressTensorVectorForm:90}
\frac{d\Br}{dt} = \rcap \rdot + \thetacap (r \thetadot) + \phicap ( r \sin\theta \phidot ),
\end{equation}

and our gradient mirrors this structure

\begin{equation}\label{eqn:stressTensorVectorForm:110}
\spacegrad = \rcap \PD{r}{} + \thetacap \inv{r }\PD{\theta}{} + \phicap \inv{r \sin\theta} \PD{\phi}{}
\end{equation}

We also previously calculated \inbookref{phy454:continuumL2}{eqn:continuumL2:1010} the unit vector differentials

\begin{subequations}
\begin{equation}\label{eqn:stressTensorVectorForm:130}
d\rcap = \phicap \sin\theta d\phi + \thetacap d\theta 
\end{equation}
\begin{equation}\label{eqn:stressTensorVectorForm:150}
d\thetacap = \phicap \cos\theta d\phi - \rcap d\theta 
\end{equation}
\begin{equation}\label{eqn:stressTensorVectorForm:170}
d\phicap = -(\rcap \sin\theta + \thetacap \cos\theta) d\phi
\end{equation}
\end{subequations}

and can use those to read off the partials of all the unit vectors

\begin{align}\label{eqn:stressTensorVectorForm:190}
\frac{\partial \rcap}{\partial \{r,\theta, \phi\}} &= \{0, \thetacap, \phicap \sin\theta \} \\
\frac{\partial \thetacap}{\partial \{r,\theta, \phi\}} &= \{0, -\rcap, \phicap \cos\theta \} \\
\frac{\partial \phicap}{\partial \{r,\theta, \phi\}} &= \{0, 0, -\rcap \sin\theta -\thetacap \cos\theta \}
\end{align}

Finally, our velocity in spherical coordinates is just

\begin{equation}\label{eqn:stressTensorVectorForm:210}
\Bu = \rcap u_r + \thetacap u_\theta + \phicap u_\phi,
\end{equation}

from which we can now compute the curl, and the directional derivative.  Starting with the curl we have

\begin{align*}
\spacegrad \wedge \Bu 
&=
\left( \rcap \PD{r}{} + \thetacap \inv{r }\PD{\theta}{} + \phicap \inv{r \sin\theta} \PD{\phi}{} \right) \wedge
\left( \rcap u_r + \thetacap u_\theta + \phicap u_\phi \right) \\
&=
\rcap \wedge \thetacap
\left( \partial_r u_\theta - \inv{r} \partial_\theta u_r
\right) 
\\
& +
\thetacap \wedge \phicap
\left(
\inv{r} \partial_\theta u_\phi - \inv{r \sin\theta} \partial_\phi u_\theta
\right)
\\
& +
\phicap \wedge \rcap
\left(
\inv{r \sin\theta} \partial_\phi u_r - \partial_r u_\phi
\right)
\\
& +
\inv{r} \thetacap \wedge \left(
u_\theta \underbrace{\partial_\theta \thetacap}_{-\rcap}
+
u_\phi \underbrace{\partial_\theta \phicap}_{0}
\right)
\\
& +
\inv{r \sin\theta} \phicap \wedge \left(
u_\theta \underbrace{\partial_\phi \thetacap}_{\phicap \cos\theta}
+
u_\phi \underbrace{\partial_\phi \phicap}_{-\rcap \sin\theta - \thetacap \cos\theta}
\right).
\end{align*}

So we have

\begin{equation}\label{eqn:stressTensorVectorForm:230}
\begin{aligned}
\spacegrad \wedge \Bu
&=
\rcap \wedge \thetacap
\left( \partial_r u_\theta - \inv{r} \partial_\theta u_r + \frac{u_\theta}{r}
\right) 
\\
& +
\thetacap \wedge \phicap
\left(
\inv{r} \partial_\theta u_\phi - \inv{r \sin\theta} \partial_\phi u_\theta
+ \frac{u_\phi \cot\theta}{r}
\right)
\\
& +
\phicap \wedge \rcap
\left(
\inv{r \sin\theta} \partial_\phi u_r - \partial_r u_\phi
- \frac{u_\phi}{r}
\right).
\end{aligned}
\end{equation}

\subsection{With $\ncap = \rcap$.}

The directional derivative portion of our strain is

\begin{align*}
2 (\rcap \cdot \spacegrad) \Bu
&=
2 \partial_r (
\rcap u_r + \thetacap u_\theta + \phicap u_\phi ) \\
&=
2 (
\rcap \partial_r u_r + \thetacap \partial_r u_\theta + \phicap \partial_r u_\phi )
\end{align*}

The other portion of our strain tensor is

\begin{align*}
(\spacegrad \wedge \Bu) \cdot \rcap
&=
(\rcap \wedge \thetacap) \cdot \rcap
\left( \partial_r u_\theta - \inv{r} \partial_\theta u_r + \frac{u_\theta}{r}
\right) 
\\
& +
(\thetacap \wedge \phicap) \cdot \rcap
\left(
\inv{r} \partial_\theta u_\phi - \inv{r \sin\theta} \partial_\phi u_\theta
+ \frac{u_\phi \cot\theta}{r}
\right)
\\
& +
(\phicap \wedge \rcap) \cdot \rcap
\left(
\inv{r \sin\theta} \partial_\phi u_r - \partial_r u_\phi
- \frac{u_\phi}{r}
\right) \\
&=
-\thetacap
\left( \partial_r u_\theta - \inv{r} \partial_\theta u_r + \frac{u_\theta}{r}
\right) 
\\
& +
\phicap 
\left(
\inv{r \sin\theta} \partial_\phi u_r - \partial_r u_\phi
- \frac{u_\phi}{r}
\right) 
\end{align*}

Putting these together we find

\begin{align*}
2 {\Be}_{\rcap}
&=
2 (\rcap \cdot \spacegrad)\Bu + (\spacegrad \wedge \Bu) \cdot \rcap \\
&=
2 (
\rcap \partial_r u_r + \thetacap \partial_r u_\theta + \phicap \partial_r u_\phi )
-\thetacap
\left( 
\partial_r u_\theta - \inv{r} \partial_\theta u_r + \frac{u_\theta}{r}
\right) 
+
\phicap 
\left(
\inv{r \sin\theta} \partial_\phi u_r - \partial_r u_\phi
- \frac{u_\phi}{r}
\right) \\
&=
\rcap
\left(
2 \partial_r u_r 
\right)
+
\thetacap
\left(
2 \partial_r u_\theta 
-\partial_r u_\theta + \inv{r} \partial_\theta u_r - \frac{u_\theta}{r}
\right)
+
\phicap
\left(
2 \partial_r u_\phi 
+ \inv{r \sin\theta} \partial_\phi u_r - \partial_r u_\phi
- \frac{u_\phi}{r}
\right)
\end{align*}

Which gives

\begin{equation}\label{eqn:stressTensorVectorForm:270}
2 {\Be}_{\rcap}
=
\rcap
\left(
2 \partial_r u_r 
\right)
+
\thetacap
\left(
\partial_r u_\theta 
+ \inv{r} \partial_\theta u_r - \frac{u_\theta}{r}
\right)
+
\phicap
\left(
\partial_r u_\phi 
+ \inv{r \sin\theta} \partial_\phi u_r 
- \frac{u_\phi}{r}
\right)
\end{equation}

For our stress tensor

\begin{equation}\label{eqn:stressTensorVectorForm:290}
\Bsigma_{\rcap} = - p \rcap + 2 \mu e_{\rcap},
\end{equation}

we can now read off our components by taking dot products

\begin{subequations}
\begin{equation}\label{eqn:stressTensorVectorForm:310}
\sigma_{rr}
=
-p + 2 \mu \PD{r}{u_r}
\end{equation}
\begin{equation}\label{eqn:stressTensorVectorForm:330}
\sigma_{r \theta}
=
\mu \left(
\PD{r}{u_\theta}
+ \inv{r} \PD{\theta}{u_r} - \frac{u_\theta}{r}
\right)
\end{equation}
\begin{equation}\label{eqn:stressTensorVectorForm:350}
\sigma_{r \phi}
=
\mu \left(
\PD{r}{u_\phi}
+ \inv{r \sin\theta} \PD{\phi}{u_r}
- \frac{u_\phi}{r}
\right).
\end{equation}
\end{subequations}

This is consistent with (15.20) from \cite{landau1987course} (after adjusting for minor notational differences).

\subsection{With $\ncap = \thetacap$.}
\subsection{With $\ncap = \phicap$.}

\EndArticle

   %
% Copyright � 2012 Peeter Joot.  All Rights Reserved.
% Licenced as described in the file LICENSE under the root directory of this GIT repository.
%
\label{chap:appendix:poissonAndShearModulus}

Young's modulus is given in \eqnref{eqn:continuumL5:330} (equation (43) in the Professor's notes) as

\begin{equation}\label{eqn:continuumL6:490}
E = \frac{\mu(3 \lambda + 2 \mu)}{\lambda + \mu },
\end{equation}

and for Poisson's ratio \eqnref{eqn:continuumL5:410} (equation (46) in the Professor's notes) we have

\begin{equation}\label{eqn:continuumL6:510}
\nu = -\frac{e_{22}}{e_{11}} = \frac{\lambda}{2 (\lambda + \mu)}.
\end{equation}

%(these are consistent with what I have got above).

Let us derive the other stated relationships (equation (47) in the Professor's notes).  I get

\begin{equation}\label{eqn:poissonAndShearModulus:550}
\begin{aligned}
2 (\lambda + \mu) \nu = \lambda \\
\implies \\
\lambda ( 2 \nu - 1 ) = - 2\mu\nu
\end{aligned}
\end{equation}

or

\begin{equation}\label{eqn:poissonAndShearModulus:570}
\begin{aligned}
\lambda = \frac{ 2 \mu \nu} { 1 - 2 \nu }
\end{aligned}
\end{equation}

For substitution into the Young's modulus equation calculate

\begin{equation}\label{eqn:poissonAndShearModulus:590}
\begin{aligned}
\lambda + \mu
&= \frac{ 2 \mu \nu} { 1 - 2 \nu } + \mu \\
&= \mu \left( \frac{ 2 \nu} { 1 - 2 \nu } + 1 \right)  \\
&= \mu \frac{ 2 \nu + 1 - 2 \nu} { 1 - 2 \nu }  \\
&= \frac{ \mu} { 1 - 2 \nu }  \\
\end{aligned}
\end{equation}

and

\begin{equation}\label{eqn:poissonAndShearModulus:610}
\begin{aligned}
3 \lambda + 2 \mu
&= 3 \frac{ \mu} { 1 - 2 \nu } - \mu \\
&= \mu \frac{ 3 - (1 - 2 \nu)} { 1 - 2 \nu } \\
&= \mu \frac{ 2 + 2 \nu} { 1 - 2 \nu } \\
&= 2 \mu \frac{ 1 + \nu} { 1 - 2 \nu } \\
\end{aligned}
\end{equation}

Putting these together we find

\begin{equation}\label{eqn:poissonAndShearModulus:630}
\begin{aligned}
E
&= \frac{\mu(3 \lambda + 2 \mu)}{\lambda + \mu } \\
&= \mu 2 \mu \frac{ 1 + \nu} { 1 - 2 \nu } \frac{ 1 - 2 \nu}{\mu} \\
&= 2 \mu ( 1 + \nu ) \\
\end{aligned}
\end{equation}

Rearranging we have

\begin{equation}\label{eqn:continuumL6:530}
\mu = \frac{E}{2 (1 + \nu)}.
\end{equation}

This matches (5.9) in the text (where \(\sigma\) is used instead of \(\nu\)).
%, but does not match your equation (47).

We also find

\begin{equation}\label{eqn:poissonAndShearModulus:650}
\begin{aligned}
\lambda
&= \frac{ 2 \mu \nu} { 1 - 2 \nu } \\
&= \frac{ \nu} { 1 - 2 \nu } \frac{E }{1 + \nu}.
\end{aligned}
\end{equation}

%(also different than the Prof's notes).

   %
% Copyright � 2012 Peeter Joot.  All Rights Reserved.
% Licenced as described in the file LICENSE under the root directory of this GIT repository.
%
\label{chap:fourierSeries}
\section{A Fourier series refresher}

Here is a quick re-derivation of how to obtain the Fourier coefficients \index{Fourier coefficient} for a trigonometric Fourier series in exponential form.  This is performed over an arbitrary interval to make it easy to apply to more specific problems.

%I had used the wrong scaling in a Fourier series over a \([0, 1]\) interval.  Here is a reminder to self what the right way to do this is.

Suppose we have a function that is defined in terms of a trigonometric Fourier sum

\begin{equation}\label{eqn:fourierSeries:10}
\phi(x) = \sum c_k e^{i \omega k x},
\end{equation}

where the domain of interest is \(x \in [a, b]\).  Stating the problem this way avoids any issue of existence.  We know \(c_k\) exists, but just want to find what they are given some other representation of the function.

Multiplying and integrating over our domain we have


\begin{dmath}\label{eqn:fourierSeries:30}
\int_a^b \phi(x) e^{-i \omega m x} dx
= \sum c_k \int_a^b e^{i \omega (k -m) x} dx
= c_m (b - a) + \sum_{k \ne m} \frac{e^{i \omega(k-m) b} - e^{i \omega(k-m)a}}{i \omega (k -m)} .
\end{dmath}

We want all the terms in the sum to be be zero, requiring equality of the exponentials, or

\begin{equation}\label{eqn:fourierSeries:50}
e^{i \omega (k -m) (b -a )} = 1,
\end{equation}

or

\begin{equation}\label{eqn:fourierSeries:70}
\omega = \frac{2 \pi}{b - a}.
\end{equation}

This fixes our Fourier coefficients

\begin{equation}\label{eqn:fourierSeries:90}
c_m = \inv{b - a} \int_a^b \phi(x) e^{- 2 \pi i m x/(b - a)} dx.
\end{equation}

So, for example, if we wished for the correct (but unnormalized) Fourier basis for a \([0, 1]\) interval, we see that we use the functions \(e^{2 \pi i x}\), or the sine and cosine equivalents, as our basis elements.

   %
%
%
% Copyright � 2012 Peeter Joot
% All Rights Reserved
%
% This file may be reproduced and distributed in whole or in part, without fee, subject to the following conditions:
%
% o The copyright notice above and this permission notice must be preserved complete on all complete or partial copies.
%
% o Any translation or derived work must be approved by the author in writing before distribution.
%
% o If you distribute this work in part, instructions for obtaining the complete version of this file must be included, and a means for obtaining a complete version provided.
%
%
% Exceptions to these rules may be granted for academic purposes: Write to the author and ask.
%
%
%
%%
% Copyright � 2015 Peeter Joot.  All Rights Reserved.
% Licenced as described in the file LICENSE under the root directory of this GIT repository.
%
\documentclass[]{eliblog}

\usepackage{amsmath}
\usepackage{mathpazo}

%
% shorthand for bold symbols, convenient for vectors and matrices
%
\newcommand{\Ba}[0]{\mathbf{a}}
\newcommand{\Bb}[0]{\mathbf{b}}
\newcommand{\Bc}[0]{\mathbf{c}}
\newcommand{\Bd}[0]{\mathbf{d}}
\newcommand{\Be}[0]{\mathbf{e}}
\newcommand{\Bf}[0]{\mathbf{f}}
\newcommand{\Bg}[0]{\mathbf{g}}
\newcommand{\Bh}[0]{\mathbf{h}}
\newcommand{\Bi}[0]{\mathbf{i}}
\newcommand{\Bj}[0]{\mathbf{j}}
\newcommand{\Bk}[0]{\mathbf{k}}
\newcommand{\Bl}[0]{\mathbf{l}}
\newcommand{\Bm}[0]{\mathbf{m}}
\newcommand{\Bn}[0]{\mathbf{n}}
\newcommand{\Bo}[0]{\mathbf{o}}
\newcommand{\Bp}[0]{\mathbf{p}}
\newcommand{\Bq}[0]{\mathbf{q}}
\newcommand{\Br}[0]{\mathbf{r}}
\newcommand{\Bs}[0]{\mathbf{s}}
\newcommand{\Bt}[0]{\mathbf{t}}
\newcommand{\Bu}[0]{\mathbf{u}}
\newcommand{\Bv}[0]{\mathbf{v}}
\newcommand{\Bw}[0]{\mathbf{w}}
\newcommand{\Bx}[0]{\mathbf{x}}
\newcommand{\By}[0]{\mathbf{y}}
\newcommand{\Bz}[0]{\mathbf{z}}
\newcommand{\BA}[0]{\mathbf{A}}
\newcommand{\BB}[0]{\mathbf{B}}
\newcommand{\BC}[0]{\mathbf{C}}
\newcommand{\BD}[0]{\mathbf{D}}
\newcommand{\BE}[0]{\mathbf{E}}
\newcommand{\BF}[0]{\mathbf{F}}
\newcommand{\BG}[0]{\mathbf{G}}
\newcommand{\BH}[0]{\mathbf{H}}
\newcommand{\BI}[0]{\mathbf{I}}
\newcommand{\BJ}[0]{\mathbf{J}}
\newcommand{\BK}[0]{\mathbf{K}}
\newcommand{\BL}[0]{\mathbf{L}}
\newcommand{\BM}[0]{\mathbf{M}}
\newcommand{\BN}[0]{\mathbf{N}}
\newcommand{\BO}[0]{\mathbf{O}}
\newcommand{\BP}[0]{\mathbf{P}}
\newcommand{\BQ}[0]{\mathbf{Q}}
\newcommand{\BR}[0]{\mathbf{R}}
\newcommand{\BS}[0]{\mathbf{S}}
\newcommand{\BT}[0]{\mathbf{T}}
\newcommand{\BU}[0]{\mathbf{U}}
\newcommand{\BV}[0]{\mathbf{V}}
\newcommand{\BW}[0]{\mathbf{W}}
\newcommand{\BX}[0]{\mathbf{X}}
\newcommand{\BY}[0]{\mathbf{Y}}
\newcommand{\BZ}[0]{\mathbf{Z}}

\newcommand{\Bzero}[0]{\mathbf{0}}
\newcommand{\Btheta}[0]{\boldsymbol{\theta}}
\newcommand{\Btau}[0]{\boldsymbol{\tau}}
\newcommand{\Bomega}[0]{\boldsymbol{\omega}}

%
% shorthand for unit vectors
%
\newcommand{\acap}[0]{\hat{\Ba}}
\newcommand{\bcap}[0]{\hat{\Bb}}
\newcommand{\ccap}[0]{\hat{\Bc}}
\newcommand{\dcap}[0]{\hat{\Bd}}
\newcommand{\ecap}[0]{\hat{\Be}}
\newcommand{\fcap}[0]{\hat{\Bf}}
\newcommand{\gcap}[0]{\hat{\Bg}}
\newcommand{\hcap}[0]{\hat{\Bh}}
\newcommand{\icap}[0]{\hat{\Bi}}
\newcommand{\jcap}[0]{\hat{\Bj}}
\newcommand{\kcap}[0]{\hat{\Bk}}
\newcommand{\lcap}[0]{\hat{\Bl}}
\newcommand{\mcap}[0]{\hat{\Bm}}
\newcommand{\ncap}[0]{\hat{\Bn}}
\newcommand{\ocap}[0]{\hat{\Bo}}
\newcommand{\pcap}[0]{\hat{\Bp}}
\newcommand{\qcap}[0]{\hat{\Bq}}
\newcommand{\rcap}[0]{\hat{\Br}}
\newcommand{\scap}[0]{\hat{\Bs}}
\newcommand{\tcap}[0]{\hat{\Bt}}
\newcommand{\ucap}[0]{\hat{\Bu}}
\newcommand{\vcap}[0]{\hat{\Bv}}
\newcommand{\wcap}[0]{\hat{\Bw}}
\newcommand{\xcap}[0]{\hat{\Bx}}
\newcommand{\ycap}[0]{\hat{\By}}
\newcommand{\zcap}[0]{\hat{\Bz}}
\newcommand{\thetacap}[0]{\hat{\Btheta}}

%
% to write R^n and C^n in a distinguishable fashion.  Perhaps change this
% to the double lined characters upon figuring out how to do so.
%
\newcommand{\C}[1]{$\mathbb{C}^{#1}$}
\newcommand{\R}[1]{$\mathbb{R}^{#1}$}

%
% various generally useful helpers
%

% derivative of #1 wrt. #2:
\newcommand{\D}[2] {\frac {d#2} {d#1}}

\newcommand{\inv}[1]{\frac{1}{#1}}
\newcommand{\cross}[0]{\times}

\newcommand{\abs}[1]{\lvert{#1}\rvert}
\newcommand{\norm}[1]{\lVert{#1}\rVert}
\newcommand{\innerprod}[2]{\langle{#1}, {#2}\rangle}
\newcommand{\dotprod}[2]{{#1} \cdot {#2}}
\newcommand{\bdotprod}[2]{\left({#1} \cdot {#2}\right)}
\newcommand{\crossprod}[2]{{#1} \cross {#2}}
\newcommand{\tripleprod}[3]{\dotprod{\left(\crossprod{#1}{#2}\right)}{#3}}

\DeclareMathOperator{\Proj}{Proj}
\DeclareMathOperator{\Span}{span}
\DeclareMathOperator{\Sgn}{sgn}
\DeclareMathOperator{\Area}{Area}
\DeclareMathOperator{\Volume}{Volume}

%
% A few miscellaneous things specific to this document
%
\newcommand{\crossop}[1]{\crossprod{#1}{}}

% R2 vector.
\newcommand{\VectorTwo}[2]{
\begin{bmatrix}
 {#1} \\
 {#2}
\end{bmatrix}
}

\newcommand{\VectorN}[1]{
\begin{bmatrix}
{#1}_1 \\
{#1}_2 \\
\vdots \\
{#1}_N \\
\end{bmatrix}
}

\newcommand{\DETuvij}[4]{
\begin{vmatrix}
 {#1}_{#3} & {#1}_{#4} \\
 {#2}_{#3} & {#2}_{#4}
\end{vmatrix}
}

\newcommand{\DETuvwijk}[6]{
\begin{vmatrix}
 {#1}_{#4} & {#1}_{#5} & {#1}_{#6} \\
 {#2}_{#4} & {#2}_{#5} & {#2}_{#6} \\
 {#3}_{#4} & {#3}_{#5} & {#3}_{#6}
\end{vmatrix}
}

\newcommand{\DETuvwxijkl}[8]{
\begin{vmatrix}
 {#1}_{#5} & {#1}_{#6} & {#1}_{#7} & {#1}_{#8} \\
 {#2}_{#5} & {#2}_{#6} & {#2}_{#7} & {#2}_{#8} \\
 {#3}_{#5} & {#3}_{#6} & {#3}_{#7} & {#3}_{#8} \\
 {#4}_{#5} & {#4}_{#6} & {#4}_{#7} & {#4}_{#8} \\
\end{vmatrix}
}

%\newcommand{\DETuvwxyijklm}[10]{
%\begin{vmatrix}
% {#1}_{#6} & {#1}_{#7} & {#1}_{#8} & {#1}_{#9} & {#1}_{#10} \\
% {#2}_{#6} & {#2}_{#7} & {#2}_{#8} & {#2}_{#9} & {#2}_{#10} \\
% {#3}_{#6} & {#3}_{#7} & {#3}_{#8} & {#3}_{#9} & {#3}_{#10} \\
% {#4}_{#6} & {#4}_{#7} & {#4}_{#8} & {#4}_{#9} & {#4}_{#10} \\
% {#5}_{#6} & {#5}_{#7} & {#5}_{#8} & {#5}_{#9} & {#5}_{#10}
%\end{vmatrix}
%}

% R3 vector.
\newcommand{\VectorThree}[3]{
\begin{bmatrix}
 {#1} \\
 {#2} \\
 {#3}
\end{bmatrix}
}



\author{Peeter Joot}
\email{peeter.joot@gmail.com}

%\documentclass[]{eliblogwidescreen}

\usepackage{amsmath}
\usepackage{mathpazo}

%
% shorthand for bold symbols, convenient for vectors and matrices
%
\newcommand{\Ba}[0]{\mathbf{a}}
\newcommand{\Bb}[0]{\mathbf{b}}
\newcommand{\Bc}[0]{\mathbf{c}}
\newcommand{\Bd}[0]{\mathbf{d}}
\newcommand{\Be}[0]{\mathbf{e}}
\newcommand{\Bf}[0]{\mathbf{f}}
\newcommand{\Bg}[0]{\mathbf{g}}
\newcommand{\Bh}[0]{\mathbf{h}}
\newcommand{\Bi}[0]{\mathbf{i}}
\newcommand{\Bj}[0]{\mathbf{j}}
\newcommand{\Bk}[0]{\mathbf{k}}
\newcommand{\Bl}[0]{\mathbf{l}}
\newcommand{\Bm}[0]{\mathbf{m}}
\newcommand{\Bn}[0]{\mathbf{n}}
\newcommand{\Bo}[0]{\mathbf{o}}
\newcommand{\Bp}[0]{\mathbf{p}}
\newcommand{\Bq}[0]{\mathbf{q}}
\newcommand{\Br}[0]{\mathbf{r}}
\newcommand{\Bs}[0]{\mathbf{s}}
\newcommand{\Bt}[0]{\mathbf{t}}
\newcommand{\Bu}[0]{\mathbf{u}}
\newcommand{\Bv}[0]{\mathbf{v}}
\newcommand{\Bw}[0]{\mathbf{w}}
\newcommand{\Bx}[0]{\mathbf{x}}
\newcommand{\By}[0]{\mathbf{y}}
\newcommand{\Bz}[0]{\mathbf{z}}
\newcommand{\BA}[0]{\mathbf{A}}
\newcommand{\BB}[0]{\mathbf{B}}
\newcommand{\BC}[0]{\mathbf{C}}
\newcommand{\BD}[0]{\mathbf{D}}
\newcommand{\BE}[0]{\mathbf{E}}
\newcommand{\BF}[0]{\mathbf{F}}
\newcommand{\BG}[0]{\mathbf{G}}
\newcommand{\BH}[0]{\mathbf{H}}
\newcommand{\BI}[0]{\mathbf{I}}
\newcommand{\BJ}[0]{\mathbf{J}}
\newcommand{\BK}[0]{\mathbf{K}}
\newcommand{\BL}[0]{\mathbf{L}}
\newcommand{\BM}[0]{\mathbf{M}}
\newcommand{\BN}[0]{\mathbf{N}}
\newcommand{\BO}[0]{\mathbf{O}}
\newcommand{\BP}[0]{\mathbf{P}}
\newcommand{\BQ}[0]{\mathbf{Q}}
\newcommand{\BR}[0]{\mathbf{R}}
\newcommand{\BS}[0]{\mathbf{S}}
\newcommand{\BT}[0]{\mathbf{T}}
\newcommand{\BU}[0]{\mathbf{U}}
\newcommand{\BV}[0]{\mathbf{V}}
\newcommand{\BW}[0]{\mathbf{W}}
\newcommand{\BX}[0]{\mathbf{X}}
\newcommand{\BY}[0]{\mathbf{Y}}
\newcommand{\BZ}[0]{\mathbf{Z}}

\newcommand{\Bzero}[0]{\mathbf{0}}
\newcommand{\Btheta}[0]{\boldsymbol{\theta}}
\newcommand{\Btau}[0]{\boldsymbol{\tau}}
\newcommand{\Bomega}[0]{\boldsymbol{\omega}}

%
% shorthand for unit vectors
%
\newcommand{\acap}[0]{\hat{\Ba}}
\newcommand{\bcap}[0]{\hat{\Bb}}
\newcommand{\ccap}[0]{\hat{\Bc}}
\newcommand{\dcap}[0]{\hat{\Bd}}
\newcommand{\ecap}[0]{\hat{\Be}}
\newcommand{\fcap}[0]{\hat{\Bf}}
\newcommand{\gcap}[0]{\hat{\Bg}}
\newcommand{\hcap}[0]{\hat{\Bh}}
\newcommand{\icap}[0]{\hat{\Bi}}
\newcommand{\jcap}[0]{\hat{\Bj}}
\newcommand{\kcap}[0]{\hat{\Bk}}
\newcommand{\lcap}[0]{\hat{\Bl}}
\newcommand{\mcap}[0]{\hat{\Bm}}
\newcommand{\ncap}[0]{\hat{\Bn}}
\newcommand{\ocap}[0]{\hat{\Bo}}
\newcommand{\pcap}[0]{\hat{\Bp}}
\newcommand{\qcap}[0]{\hat{\Bq}}
\newcommand{\rcap}[0]{\hat{\Br}}
\newcommand{\scap}[0]{\hat{\Bs}}
\newcommand{\tcap}[0]{\hat{\Bt}}
\newcommand{\ucap}[0]{\hat{\Bu}}
\newcommand{\vcap}[0]{\hat{\Bv}}
\newcommand{\wcap}[0]{\hat{\Bw}}
\newcommand{\xcap}[0]{\hat{\Bx}}
\newcommand{\ycap}[0]{\hat{\By}}
\newcommand{\zcap}[0]{\hat{\Bz}}
\newcommand{\thetacap}[0]{\hat{\Btheta}}

%
% to write R^n and C^n in a distinguishable fashion.  Perhaps change this
% to the double lined characters upon figuring out how to do so.
%
\newcommand{\C}[1]{$\mathbb{C}^{#1}$}
\newcommand{\R}[1]{$\mathbb{R}^{#1}$}

%
% various generally useful helpers
%

% derivative of #1 wrt. #2:
\newcommand{\D}[2] {\frac {d#2} {d#1}}

\newcommand{\inv}[1]{\frac{1}{#1}}
\newcommand{\cross}[0]{\times}

\newcommand{\abs}[1]{\lvert{#1}\rvert}
\newcommand{\norm}[1]{\lVert{#1}\rVert}
\newcommand{\innerprod}[2]{\langle{#1}, {#2}\rangle}
\newcommand{\dotprod}[2]{{#1} \cdot {#2}}
\newcommand{\bdotprod}[2]{\left({#1} \cdot {#2}\right)}
\newcommand{\crossprod}[2]{{#1} \cross {#2}}
\newcommand{\tripleprod}[3]{\dotprod{\left(\crossprod{#1}{#2}\right)}{#3}}

\DeclareMathOperator{\Proj}{Proj}
\DeclareMathOperator{\Span}{span}
\DeclareMathOperator{\Sgn}{sgn}
\DeclareMathOperator{\Area}{Area}
\DeclareMathOperator{\Volume}{Volume}

%
% A few miscellaneous things specific to this document
%
\newcommand{\crossop}[1]{\crossprod{#1}{}}

% R2 vector.
\newcommand{\VectorTwo}[2]{
\begin{bmatrix}
 {#1} \\
 {#2}
\end{bmatrix}
}

\newcommand{\VectorN}[1]{
\begin{bmatrix}
{#1}_1 \\
{#1}_2 \\
\vdots \\
{#1}_N \\
\end{bmatrix}
}

\newcommand{\DETuvij}[4]{
\begin{vmatrix}
 {#1}_{#3} & {#1}_{#4} \\
 {#2}_{#3} & {#2}_{#4}
\end{vmatrix}
}

\newcommand{\DETuvwijk}[6]{
\begin{vmatrix}
 {#1}_{#4} & {#1}_{#5} & {#1}_{#6} \\
 {#2}_{#4} & {#2}_{#5} & {#2}_{#6} \\
 {#3}_{#4} & {#3}_{#5} & {#3}_{#6}
\end{vmatrix}
}

\newcommand{\DETuvwxijkl}[8]{
\begin{vmatrix}
 {#1}_{#5} & {#1}_{#6} & {#1}_{#7} & {#1}_{#8} \\
 {#2}_{#5} & {#2}_{#6} & {#2}_{#7} & {#2}_{#8} \\
 {#3}_{#5} & {#3}_{#6} & {#3}_{#7} & {#3}_{#8} \\
 {#4}_{#5} & {#4}_{#6} & {#4}_{#7} & {#4}_{#8} \\
\end{vmatrix}
}

%\newcommand{\DETuvwxyijklm}[10]{
%\begin{vmatrix}
% {#1}_{#6} & {#1}_{#7} & {#1}_{#8} & {#1}_{#9} & {#1}_{#10} \\
% {#2}_{#6} & {#2}_{#7} & {#2}_{#8} & {#2}_{#9} & {#2}_{#10} \\
% {#3}_{#6} & {#3}_{#7} & {#3}_{#8} & {#3}_{#9} & {#3}_{#10} \\
% {#4}_{#6} & {#4}_{#7} & {#4}_{#8} & {#4}_{#9} & {#4}_{#10} \\
% {#5}_{#6} & {#5}_{#7} & {#5}_{#8} & {#5}_{#9} & {#5}_{#10}
%\end{vmatrix}
%}

% R3 vector.
\newcommand{\VectorThree}[3]{
\begin{bmatrix}
 {#1} \\
 {#2} \\
 {#3}
\end{bmatrix}
}



\author{Peeter Joot}
\email{peeter.joot@gmail.com}


%\usepackage[english]{babel}
%\usepackage{media9}
\chapter{Surface for spinning bucket of water.}

\label{chap:constantSpinSurfaces}
%\useCCL
\blogpage{http://sites.google.com/site/peeterjoot2/math2012/constantSpinSurfaces.pdf}
\date{Apr 29, 2012}
\gitRevisionInfo{constantSpinSurfaces}
\keywords{Navier-Stokes, Bernoulli's theorem, PHY454H1S, PHY454H1}

\beginArtWithToc
%\beginArtNoToc
%\wordpresscategory{}

\section{Motivation.}

Here's a problem from the 2009 phy1530 final that was appropriate for exam prep for this course too.  It also serves as a nice example of how to determine a surface as a function of pressure, something I want to do for the non-bottomless coffee problem to be attempted.

\section{Statement}

An undergraduate student is assigned a problem about an ideal fluid rotating at a constant angular velocity $\Omega$ under gravity $g$.  The velocity field is $\Bu = (-\Omega y, \Omega x, 0)$.  Here, $x$ and $y$ are horizontal and $z$ points up.  The student is supposed to find the surfaces of constant pressure, and hence the shape of the free surface of water in a rotating bucket.  The free surface corresponds to the surface for which $p = p_0$, where $p_0$ is the atmospheric pressure.  Surface tension is neglected.

\begin{enumerate}
\item On their homework assignment, the student writes:

``By Bernoulli's equation:

\begin{equation}\label{eqn:constantSpinSurfaces:10}
B = \frac{p}{\rho} + \inv{2}u^2 + g z
\end{equation}

where $B$ is a constant.  So the constant pressure surface at $p = p_0$ is 

\begin{equation}\label{eqn:constantSpinSurfaces:30}
z = \left( \frac{B}{g} - \frac{p_0}{\rho g}
\right)
- \frac{\Omega^2 }{2 g} \left( x^2 + y^2 \right).
\end{equation}
''

But this seems to show that the surface of the water in a rotating bucket is \textit{highest in the middle}!  What is wrong with the student's argument?

\item
Write down the Euler equations in component form and integrate them directly to find the pressure $p$, and hence obtain the correct parabolic shape for the free surface.
\end{enumerate}

\section{Solution.  Part 1.}

Let's recall how we derived Bernoulli's theorem.  We started with Navier-Stokes and used the identity

\begin{equation}\label{eqn:constantSpinSurfaces:50}
(\Bu \cdot \spacegrad ) \Bu = \spacegrad \inv{2} \Bu^2 + (\spacegrad \cross \Bu) \cross \Bu.
\end{equation}

Navier-Stokes for a steady state incompressible flow, with external body force per unit volume $\rho \Bg = -\rho \spacegrad \chi$ take the form

\begin{equation}\label{eqn:constantSpinSurfaces:70}
\spacegrad \inv{2} \Bu^2 + (\spacegrad \cross \Bu) \cross \Bu
= -\inv{\rho} \spacegrad p + \nu \spacegrad^2 \Bu - \spacegrad \chi.
\end{equation}

For the non-viscous (``dry-water'') case where we take $\mu = \nu \rho = 0$, and treat the density $\rho$ as a constant we find

\begin{equation}\label{eqn:constantSpinSurfaces:90}
\Bu \cross (\spacegrad \cross \Bu)
=
\spacegrad 
\left( 
\inv{2} \Bu^2 + \frac{p}{\rho} + \chi
\right).
\end{equation}

Observe that we only arrive at Bernoulli's theorem if the flow is also irrotational (as well as incompressible and non-viscous), as we require an irrotational flow where $\spacegrad \cdot \Bu = 0$ to claim that the gradient on the RHS is zero.  
%
%The most general claim that we can make, even for irrotational flows is that we have
%
%\begin{equation}\label{eqn:constantSpinSurfaces:110}
%0 = \Bu \cdot
%\spacegrad 
%\left( 
%\inv{2} \Bu^2 + \frac{p}{\rho} + \chi
%\right).
%\end{equation}
%
%That holds even for flows that are not irrotational, since $\Bu \cdot (\spacegrad \cross \Bu) = 0$.

In this problem we do not have an irrotational flow, which can be demonstrated by direct calculation.  We have

\begin{equation}\label{eqn:constantSpinSurfaces:130}
\begin{aligned}
\spacegrad \cross \Bu
&=
\Omega
\begin{vmatrix}
\xcap & \ycap & \zcap \\
\partial_x & \partial_y & 0 \\
-y & x & 0
\end{vmatrix} \\
&=
2 \zcap \Omega \\
&\ne 0
\end{aligned}
\end{equation}

In fact we have

\begin{equation}\label{eqn:constantSpinSurfaces:150}
\begin{aligned}
\Bu \cross (\spacegrad \cross \Bu)
&=
2 \Omega^2
\begin{vmatrix}
\xcap & \ycap & \zcap \\
-y & x & 0  \\
0 & 0 & 1
\end{vmatrix} \\
&=
2 \Omega^2 (\xcap + \ycap)
\end{aligned}
\end{equation}

The closest we can get to Bernoulli's theorem for this problem is

\begin{equation}\label{eqn:constantSpinSurfaces:210}
2 \Omega^2 (\xcap + \ycap)
= 
\spacegrad 
\left( 
\inv{2} \Bu^2 + \frac{p}{\rho} + g z
\right).
\end{equation}

We can say that the directional derivatives in directions perpendicular to $\xcap + \ycap$ are zero, and that 

\begin{equation}\label{eqn:constantSpinSurfaces:230}
\begin{aligned}
2 \Omega^2 
&= (\partial_x + \partial_y) 
\left( 
\inv{2} \Bu^2 + \frac{p}{\rho} + g z
\right) \\
&= (\partial_x + \partial_y) 
\left( 
\inv{2} \Bu^2 + \frac{p}{\rho} 
\right) \\
\end{aligned}
\end{equation}

Perhaps those could be used to solve for the surface, but we no longer have something that is obviously integrable.

Because $\Bu \cdot (\Bu \cross (\spacegrad \cross \Bu)) = 0$, we can also say that

\begin{equation}\label{eqn:constantSpinSurfaces:250}
\begin{aligned}
0 
&= \Bu \cdot \spacegrad
\left( 
\inv{2} \Bu^2 + \frac{p}{\rho} + g z
\right) \\
&= 
\Omega ( y \partial_x - x \partial_y )
\left( 
\inv{2} \Bu^2 + \frac{p}{\rho} 
\right).
\end{aligned}
\end{equation}

Perhaps this could also be used to find the surface?

\section{Solution.  Part 2.}

We want to write down the steady state, incompressible, non-viscous Navier-Stokes equations.  The first of these is trivially satisfied by our assumed solution

\begin{equation}\label{eqn:constantSpinSurfaces:270}
0 
= \spacegrad \cdot \Bu 
= \partial_x (-\Omega y) + \partial_y(\Omega x).
\end{equation}

For the inertial term we've got

\begin{equation}\label{eqn:constantSpinSurfaces:290}
\begin{aligned}
\Bu \cdot \spacegrad \Bu
&=
\Omega^2 (-y \partial_x + x \partial_y (-y, x, 0) \\
&=
\Omega^2 (-x, -y, 0),
\end{aligned}
\end{equation}

Leaving us with

\begin{equation}\label{eqn:constantSpinSurfaces:310}
\begin{aligned}
-\Omega^2 x &= -\inv{\rho} \partial_x p \\
-\Omega^2 y &= -\inv{\rho} \partial_y p \\
          0 &= -\inv{\rho} \partial_z p - g
\end{aligned}
\end{equation}

Integrating these, we seek seek simultaneous solutions to

\begin{equation}\label{eqn:constantSpinSurfaces:330}
\begin{aligned}
p &= \inv{2} \rho \Omega^2 x^2 + f(y,z) \\
p &= \inv{2} \rho \Omega^2 y^2 + g(x,z) \\
p &= h(x, y) - \rho g z.
\end{aligned}
\end{equation}

It's clear that one solution would be

\begin{equation}\label{eqn:constantSpinSurfaces:350}
p = p_0 + \inv{2} \rho \Omega^2 (x^2 + y^2) - \rho g z.
\end{equation}

where $p_0$ is some constant to be determined, dependent on where we set our origin.  Putting the origin of the coordinate system at the lowest point in the parabolic profile $(x, y, z) = (0, 0, 0)$, we have $p(0, 0, 0) = p_0$, which fixes $p_0$ as the atmospheric pressure.  If the radius of the bucket is $R$, the max height $h$ of the surface above that point is also found on this surface of constant pressure

\begin{equation}\label{eqn:constantSpinSurfaces:370}
p_0 = p_0 + \inv{2} \rho \Omega^2 R^2 - \rho g h,
\end{equation}

or 

\begin{equation}\label{eqn:constantSpinSurfaces:390}
h = \frac{\Omega^2 R^2 }{2 g}.
\end{equation}

\section{Appendix.  Proof of vector identities used.}

\begin{equation}\label{eqn:constantSpinSurfaces:170}
\begin{aligned}
\left( \spacegrad \inv{2} \Bu^2 + (\spacegrad \cross \Bu) \cross \Bu \right)_i
&=
\partial_i \inv{2} u_j u_j + \partial_a u_b \epsilon_{a b r} u_s \epsilon_{r s i} \\
&=
u_j \partial_i u_j + \partial_a u_b u_s \delta^{[a b]}_{s i} \\
&=
u_j \partial_i u_j 
+ u_s \partial_s u_i 
- u_s \partial_i u_s \\
&= (\Bu \cdot \spacegrad) \Bu)_i 
\end{aligned}
\end{equation}

Also observe that our claim that $\Bu \cdot (\Bu \cross (\spacegrad \Bu)) = 0$ follows easily after expansion in coordinates

\begin{equation}\label{eqn:constantSpinSurfaces:190}
\Bu \cdot (\Bu \cross (\spacegrad \cross \Bu) )
=
u_i u_s ( \partial_s u_i - \partial_i u_s ).
\end{equation}

We've got a symmetric and antisymmetric factor in the summation, so the end result is zero.

%\EndArticle
\EndNoBibArticle

   \include{mathematica}
%\end{appendix}

%%% is this now going to have appendix style numbering?
%\part{Cronology}
%\chapter{Cronological Index}
\begin{itemize}

\item October 13, 2007 \ref{chap:gaWiki} Comparison of many traditional vector and GA identities

\item October 13, 2007 \ref{chap:gaWikiTorque} Torque

\item October 16, 2007 \ref{chap:PJUnitDer} Derivatives of a unit vector

\item October 16, 2007 \ref{chap:gaWikiCramers} Cramer's rule

\item October 22, 2007 \ref{chap:PJRadialDer} Radial components of vector derivatives

\item January 1, 2008 \ref{chap:plane} More details on NFCM plane formulation

\item January 29, 2008 \ref{chap:PJAngVel} Rotational dynamics

\item January 29, 2008 \ref{chap:maxwellsGa} Maxwell's equations expressed with Geometric Algebra

\item February 2, 2008 \ref{chap:quaternion} Quaternions

\item February 4, 2008 \ref{chap:legendre} Legendre Polynomials

\item February 15, 2008 \ref{chap:inertialTensor} Inertia Tensor

\item February 19, 2008 \ref{chap:rotor} Rotor Notes

\item February 28, 2008 \ref{chap:laplace} Exponential Solutions to Laplace Equation in \R{N}

\item March 9, 2008 \ref{chap:bivector} Bivector Geometry

\item March 9, 2008 \ref{chap:trivector} Trivector geometry

\item March 12, 2008 \ref{chap:kvectorExponential} Exponential of a blade

\item March 16, 2008 \ref{chap:scalarCommutes} Multivector product grade zero terms

\item March 17, 2008 \ref{chap:angleBetweenLineAndPlane} Angle between geometric elements

\item March 17, 2008 \ref{chap:gaGradeDotWedge} An earlier attempt to intuitively introduce the dot, wedge, cross, and geometric products

\item March 25, 2008 \ref{chap:bladegradereduction} Blade grade reduction

\item March 29, 2008 \ref{chap:reciprocalFrame} Reciprocal Frame Vectors

\item March 31, 2008 \ref{chap:gradientAndForms} Exterior derivative and chain rule components of the gradient

\item April 1, 2008 \ref{chap:orthodecomp} Orthogonal decomposition take II

\item April 11, 2008 \ref{chap:matrixReview} Matrix review

\item April 13, 2008 \ref{chap:locateSatellite} Satellite triangulation over sphere

\item April 30, 2008 \ref{chap:PJKeRot} Kinetic Energy in rotational frame

\item May 7, 2008 \ref{chap:lorentzRotation} Lorentz Force Trajectory

\item May 16, 2008 \ref{chap:obliqueProj} Oblique projection and reciprocal frame vectors

\item May 16, 2008 \ref{chap:matrixOfLinearTx} Matrix of grade k multivector linear transformations

\item May 16, 2008 \ref{chap:projectionAndMoorePenroseVectorInverse} Projection and Moore-Penrose vector inverse

\item May 17, 2008 \ref{chap:PJprojGen} Projection with generalized dot product

\item June 6, 2008 \ref{chap:tensor} Gradient and tensor notes

\item June 10, 2008 \ref{chap:PJAngAcc} Angular Velocity and Acceleration.  Again

\item June 25, 2008 \ref{chap:lorentz} Wave equation based Lorentz transformation derivation

\item July 8, 2008 \ref{chap:PJAngAccCross} Cross product Radial decomposition

\item July 12, 2008 \ref{chap:PJMaxwell2} Back to Maxwell's equations

\item July 16, 2008 \ref{chap:spacetimegrad} Lorentz transformation of spacetime gradient

\item July 20, 2008 \ref{chap:sgMx41} Magnetic field between two parallel wires

\item August 1, 2008 \ref{chap:fourvecDotinvariance} Four vector dot product invariance and Lorentz rotors

\item August 9, 2008 \ref{chap:newtonianLagrangianAndGradient} Newton's Law from Lagrangian

\item August 13, 2008 \ref{chap:cauchyGradient} Cauchy Equations expressed as a gradient

\item August 13, 2008 \ref{chap:velocityTx} Understanding four velocity transform from rest frame

\item August 15, 2008 \ref{chap:emPotential} Four vector potential

\item August 16, 2008 \ref{chap:PJSrGAFPLorentzForce} Lorentz force Law

\item August 21, 2008 \ref{chap:PJSrLagrangian} Covariant Lagrangian, and electrodynamic potential

\item August 25, 2008 \ref{chap:PJTongMf1} Solutions to David Tong's mf1 Lagrangian problems

\item August 28, 2008 \ref{chap:massVaryLagrangian} Equations of motion given mass variation with spacetime position

\item September 1, 2008 \ref{chap:PJCanMomentum} Vector canonical momentum

\item September 2, 2008 \ref{chap:outermorphismDet} OuterMorphism Question 

\item September 5, 2008 \ref{chap:emBivectorMetricDependencies} Metric signature dependencies

\item September 7, 2008 \ref{chap:PJMaxwellTensor} Tensor relations from bivector field equation

\item September 8, 2008 \ref{chap:PJMaxwellLagrangian} Direct variation of Maxwell equations

\item September 9, 2008 \ref{chap:PJMaxwellProj} Vector forms of Maxwell's equations as projection and rejection operations

\item September 18, 2008 \ref{chap:PJStokes1} Stokes law in wedge product form

\item September 26, 2008 \ref{chap:stokesMaxwellApplication} Application of Stokes Integrals to Maxwell's Equation

\item September 27, 2008 \ref{chap:PJStokes2} Stokes Law revisited with algebraic enumeration of boundary

\item October 8, 2008 \ref{chap:PJSrLorentzForce} Revisit Lorentz force from Lagrangian

\item October 10, 2008 \ref{chap:PJFieldLagrangian} Derivation of Euler-Lagrange field equations

\item October 12, 2008 \ref{chap:maxwellTensorLagrangian} Tensor Derivation of Covariant Lorentz Force from Lagrangian

\item October 13, 2008 \ref{chap:PJEulerLagrange} Euler Lagrange Equations

\item October 19, 2008 \ref{chap:PJBoostMaxwell} Lorentz Invariance of Maxwell Lagrangian

\item October 22, 2008 \ref{chap:PJLorentzTxInteraction} Lorentz transform Noether current for interaction Lagrangian

\item October 26, 2008 \ref{chap:gem} GravitoElectroMagnetism

\item October 29, 2008 \ref{chap:PJNoethersField} Field form of Noether's Law

\item November 1, 2008 \ref{chap:eulerangle} Euler Angle Notes

\item November 8, 2008 \ref{chap:complex} Hyper complex numbers and symplectic structure

\item November 13, 2008 \ref{chap:sphericalPolar} Spherical polar coordinates

\item November 22, 2008 \ref{chap:gaussianSurface} Gaussian Surface invariance for radial field

\item November 23, 2008 \ref{chap:chargeArcElement} Field due to line charge in arc

\item November 23, 2008 \ref{chap:chargeLineElement} Charge line element

\item November 27, 2008 \ref{chap:nfcmCh2} Some NFCM exercise solutions and notes

\item November 30, 2008 \ref{chap:PJwaveFourVector} Expressing wave equation exponential solutions using four vectors

\item November 30, 2008 \ref{chap:slerp} Rotor interpolation calculation

\item December 6, 2008 \ref{chap:pauliMatrix} Pauli Matrixes in Clifford Algebra

\item December 11, 2008 \ref{chap:bohr} Bohr Model

\item December 13, 2008 \ref{chap:PJDiracGamma} Gamma Matrices

\item December 21, 2008 \ref{chap:diracLagrangian} Dirac Lagrangian

\item December 27, 2008 \ref{chap:PJrayleighJeans} Rayleigh-Jeans Law Notes

\item December 29, 2008 \ref{chap:PJpoynting} Poynting vector and Electromagnetic Energy conservation

\item January 1, 2009 \ref{chap:PJemstresstensor} Energy momentum tensor

\item January 3, 2009 \ref{chap:PJelectricFieldEnergy} Field and wave energy and momentum

\item January 5, 2009 \ref{chap:vectorDifferentialIdentities} Vector Differential Identities

\item January 6, 2009 \ref{chap:dcPower} DC Power consumption formula for resistive load

\item January 9, 2009 \ref{chap:PJqmFourier} Some Fourier transform notes

\item January 11, 2009 \ref{chap:schCurrent} Schr\"{o}dinger equation probability conservation

\item January 13, 2009 \ref{chap:radial} Polar velocity and acceleration

\item January 18, 2009 \ref{chap:PJpoyntingRate} Time rate of change of the Poynting vector, and its conservation law

\item January 19, 2009 \ref{chap:PJheatFourier} Fourier Solutions to Heat and Wave equations

\item January 21, 2009 \ref{chap:fourierNotation} A cheatsheet for Fourier transform conventions

\item January 25, 2009 \ref{chap:PJemWave} Electrodynamic wave equation solutions

\item January 26, 2009 \ref{chap:PJwaveFourier} Fourier transform solutions to the wave equation

\item January 29, 2009 \ref{chap:PJfourierMaxwellSecondOrder} Fourier transform solutions to Maxwell's equation

\item January 31, 2009 \ref{chap:PJfirstOrderMaxwell} First order Fourier transform solution of Maxwell's equation

\item February 1, 2009 \ref{chap:PJ4dFourier} 4D Fourier transforms applied to Maxwell's equation

\item February 3, 2009 \ref{chap:PJFourierVacuum} Fourier series Vacuum Maxwell's equations

\item February 7, 2009 \ref{chap:potentialFourier} Lorentz Gauge Fourier Vacuum potential solutions

\item February 8, 2009 \ref{chap:PJplaneWave} Plane wave Fourier series solutions to the Maxwell vacuum equation

\item February 13, 2009 \ref{chap:PJstressEnergyLorentz} Lorentz force relation to the energy momentum tensor

\item February 17, 2009 \ref{chap:en_m_tensor} Energy momentum tensor relation to Lorentz force

\item February 18, 2009 \ref{chap:PJpoisson} Poisson and retarded Potential Green's functions from Fourier kernels

\item February 26, 2009 \ref{chap:nvolume} Spherical and hyperspherical parametrization

\item March 13, 2009 \ref{chap:levi} Levi-Civitica summation identity

\item March 18, 2009 \ref{chap:electronRotor} Lorentz force rotor formulation

\item April 15, 2009 \ref{chap:lorentzForcePQA} Lorentz force Lagrangian with conjugate momentum

\item April 18, 2009 \ref{chap:biotSavart} Biot Savart Derivation

\item April 20, 2009 \ref{chap:maxwellTensorFromLagrangian} Tensor derivation of non-dual Maxwell equation from Lagrangian

\item April 28, 2009 \ref{chap:PJmultiTaylors} Developing some intuition for Multivariable and Multivector Taylor Series

\item May 23, 2009 \ref{chap:lorentzForceTx} Lorentz boost of Lorentz force equations

\item May 28, 2009 \ref{chap:macroscopicMaxwell} Macroscopic Maxwell's equation

\item June 1, 2009 \ref{chap:poincareTx} Poincare transformations

\item June 5, 2009 \ref{chap:stressEnergyNoethers} Canonical energy momentum tensor and Lagrangian translation

\item June 17, 2009 \ref{chap:lForceLag2} Comparison of two covariant Lorentz force Lagrangians

\item June 21, 2009 \ref{chap:emVacWave} Wave equation form of Maxwell's equations

\item June 27, 2009 \ref{chap:frequencyTx} Relativistic Doppler formula

\end{itemize}

\part{Bibliography and Index}

% END INCLUDES.
%-------------------------------------------------------

\printindex

\bibliography{myrefs}
\bibliographystyle{unsrturl}
%  \addcontentsline{toc}{chapter}{Bibliography}

\end{document}
