% 
% 
% 
% Copyright © 2012 Peeter Joot
% All Rights Reserved
% 
% This file may be reproduced and distributed in whole or in part, without fee, subject to the following conditions:
% 
% o The copyright notice above and this permission notice must be preserved complete on all complete or partial copies.
% 
% o Any translation or derived work must be approved by the author in writing before distribution.
% 
% o If you distribute this work in part, instructions for obtaining the complete version of this file must be included, and a means for obtaining a complete version provided.
% 
% 
% Exceptions to these rules may be granted for academic purposes: Write to the author and ask.
% 
% 
% 

\chapter{Introduction to continuum mechanics}
\section{Continuum Mechanics}

Mechanics could be defined as the study of effects of forces and displacements on a physical body

%figure (\ref{fig:continuumL2:continuumL2fig1})
\imageFigure{figures/continuumL2fig1}{Physical body}{fig:continuumL2:continuumL2fig1}{0.2}

In continuum mechanics we have a physical body and we are interested in the internal motions in the object.

%figure (\ref{fig:continuumL2:continuumL2fig2})
\imageFigure{figures/continuumL2fig2}{Control volume elements}{fig:continuumL2:continuumL2fig2}{0.2}

For the first time considering mechanics we have to introduce the concepts of fields to make progress tackling these problems.

We will have use of the following types of fields

\begin{itemize}
\item Scalar fields.  $3^0$ components.  Examples: density, Temperature, ...
\item Vector fields.  $3^1$ components.  Examples: Force, velocity.
\item Tensor fields.  $3^2$ components.  Examples: stress, strain.
\end{itemize}

We have to consider objects (a control volume) that is small enough that we can consider that we have a point in space limit for the quantities of density and velocity.  At the same time we cannot take this limiting process to the extreme, since if we use a control volume that is sufficiently small, quantum and inter-atomic effects would have to be considered.

%figure (\ref{fig:continuumL2:continuumL2fig3})
\imageFigure{figures/continuumL2fig3}{Mass and volume ratios at different scales}{fig:continuumL2:continuumL2fig3}{0.2}

\section{Nomenclature and basic definitions}

Most of this course is focused on just two concepts, that of strain and stress, and how they relate.  We define

\makedefinition{Strain}{dfn:continuumL2:30}{Measure of the deformation of the body, relating stretch and position. \index{strain}}

\makedefinition{Stress}{dfn:continuumL2:10}{Measure of the Internal force on the surfaces.  This is a quantity constructed such that its divergence expresses the force per unit volume. \index{stress}}

\makedefinition{Constitutive relation}{dfn:continuumL2:20}{How strain and stress in a material are related. \index{Constitutive relation}}

\makedefinition{Newtonian fluid}{dfn:continuumL2:40}{A fluid for which the constitutive relation is linear.  \index{Newtonian fluid}}

Defining these mathematically and using these concepts to model solid, liquid and gaseous materials will allow us to accurately predict the behavior of many types of continuous substances.

Building on these definitions we will end up discussing a number of other key concepts, such as

\begin{itemize}
\item rheological.  Study of the flow of matter, primarily in the liquid state.  \citep{wiki:rheology}.
\item elastic waves
\item elastic/plastic
\item Navier stokes relation: equivalent of Newton's law for fluids.
\item nondimensionalisation: parameter substitutions that put the differential equations of interest into dimensionless form.
\item boundary layer theory: investigation of the region of fluid flow around a solid where viscous forces dominate.
\item stability analysis
\item nonlinearity 
\end{itemize}

\section{Texts}

The lectures for this course loosely follow portions of the following texts

\begin{itemize}
\item \href{http://www.amazon.com/Elementary-Dynamics-Applied-Mathematics-Computing/dp/0198596790/ref=sr_1_1?ie=UTF8&qid=1326302753&sr=8-1}{Elementary fluid dynamics} \citep{acheson1990elementary}.
\item \href{http://www.amazon.com/Theory-Elasticity-Third-Theoretical-Physics/dp/075062633X/ref=sr_1_6?s=books&ie=UTF8&qid=1326302957&sr=1-6}{Theory of Elasticity} \citep{landau1960theory}
\end{itemize}

While these were not required reading, the Acheson text was particularly helpful in providing additional details about subjects covered in class, and for supplying useful problems for study.
