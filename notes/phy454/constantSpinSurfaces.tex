%
%
%
% Copyright � 2012 Peeter Joot
% All Rights Reserved
%
% This file may be reproduced and distributed in whole or in part, without fee, subject to the following conditions:
%
% o The copyright notice above and this permission notice must be preserved complete on all complete or partial copies.
%
% o Any translation or derived work must be approved by the author in writing before distribution.
%
% o If you distribute this work in part, instructions for obtaining the complete version of this file must be included, and a means for obtaining a complete version provided.
%
%
% Exceptions to these rules may be granted for academic purposes: Write to the author and ask.
%
%
%
%%
% Copyright � 2015 Peeter Joot.  All Rights Reserved.
% Licenced as described in the file LICENSE under the root directory of this GIT repository.
%
\documentclass[]{eliblog}

\usepackage{amsmath}
\usepackage{mathpazo}

%
% shorthand for bold symbols, convenient for vectors and matrices
%
\newcommand{\Ba}[0]{\mathbf{a}}
\newcommand{\Bb}[0]{\mathbf{b}}
\newcommand{\Bc}[0]{\mathbf{c}}
\newcommand{\Bd}[0]{\mathbf{d}}
\newcommand{\Be}[0]{\mathbf{e}}
\newcommand{\Bf}[0]{\mathbf{f}}
\newcommand{\Bg}[0]{\mathbf{g}}
\newcommand{\Bh}[0]{\mathbf{h}}
\newcommand{\Bi}[0]{\mathbf{i}}
\newcommand{\Bj}[0]{\mathbf{j}}
\newcommand{\Bk}[0]{\mathbf{k}}
\newcommand{\Bl}[0]{\mathbf{l}}
\newcommand{\Bm}[0]{\mathbf{m}}
\newcommand{\Bn}[0]{\mathbf{n}}
\newcommand{\Bo}[0]{\mathbf{o}}
\newcommand{\Bp}[0]{\mathbf{p}}
\newcommand{\Bq}[0]{\mathbf{q}}
\newcommand{\Br}[0]{\mathbf{r}}
\newcommand{\Bs}[0]{\mathbf{s}}
\newcommand{\Bt}[0]{\mathbf{t}}
\newcommand{\Bu}[0]{\mathbf{u}}
\newcommand{\Bv}[0]{\mathbf{v}}
\newcommand{\Bw}[0]{\mathbf{w}}
\newcommand{\Bx}[0]{\mathbf{x}}
\newcommand{\By}[0]{\mathbf{y}}
\newcommand{\Bz}[0]{\mathbf{z}}
\newcommand{\BA}[0]{\mathbf{A}}
\newcommand{\BB}[0]{\mathbf{B}}
\newcommand{\BC}[0]{\mathbf{C}}
\newcommand{\BD}[0]{\mathbf{D}}
\newcommand{\BE}[0]{\mathbf{E}}
\newcommand{\BF}[0]{\mathbf{F}}
\newcommand{\BG}[0]{\mathbf{G}}
\newcommand{\BH}[0]{\mathbf{H}}
\newcommand{\BI}[0]{\mathbf{I}}
\newcommand{\BJ}[0]{\mathbf{J}}
\newcommand{\BK}[0]{\mathbf{K}}
\newcommand{\BL}[0]{\mathbf{L}}
\newcommand{\BM}[0]{\mathbf{M}}
\newcommand{\BN}[0]{\mathbf{N}}
\newcommand{\BO}[0]{\mathbf{O}}
\newcommand{\BP}[0]{\mathbf{P}}
\newcommand{\BQ}[0]{\mathbf{Q}}
\newcommand{\BR}[0]{\mathbf{R}}
\newcommand{\BS}[0]{\mathbf{S}}
\newcommand{\BT}[0]{\mathbf{T}}
\newcommand{\BU}[0]{\mathbf{U}}
\newcommand{\BV}[0]{\mathbf{V}}
\newcommand{\BW}[0]{\mathbf{W}}
\newcommand{\BX}[0]{\mathbf{X}}
\newcommand{\BY}[0]{\mathbf{Y}}
\newcommand{\BZ}[0]{\mathbf{Z}}

\newcommand{\Bzero}[0]{\mathbf{0}}
\newcommand{\Btheta}[0]{\boldsymbol{\theta}}
\newcommand{\Btau}[0]{\boldsymbol{\tau}}
\newcommand{\Bomega}[0]{\boldsymbol{\omega}}

%
% shorthand for unit vectors
%
\newcommand{\acap}[0]{\hat{\Ba}}
\newcommand{\bcap}[0]{\hat{\Bb}}
\newcommand{\ccap}[0]{\hat{\Bc}}
\newcommand{\dcap}[0]{\hat{\Bd}}
\newcommand{\ecap}[0]{\hat{\Be}}
\newcommand{\fcap}[0]{\hat{\Bf}}
\newcommand{\gcap}[0]{\hat{\Bg}}
\newcommand{\hcap}[0]{\hat{\Bh}}
\newcommand{\icap}[0]{\hat{\Bi}}
\newcommand{\jcap}[0]{\hat{\Bj}}
\newcommand{\kcap}[0]{\hat{\Bk}}
\newcommand{\lcap}[0]{\hat{\Bl}}
\newcommand{\mcap}[0]{\hat{\Bm}}
\newcommand{\ncap}[0]{\hat{\Bn}}
\newcommand{\ocap}[0]{\hat{\Bo}}
\newcommand{\pcap}[0]{\hat{\Bp}}
\newcommand{\qcap}[0]{\hat{\Bq}}
\newcommand{\rcap}[0]{\hat{\Br}}
\newcommand{\scap}[0]{\hat{\Bs}}
\newcommand{\tcap}[0]{\hat{\Bt}}
\newcommand{\ucap}[0]{\hat{\Bu}}
\newcommand{\vcap}[0]{\hat{\Bv}}
\newcommand{\wcap}[0]{\hat{\Bw}}
\newcommand{\xcap}[0]{\hat{\Bx}}
\newcommand{\ycap}[0]{\hat{\By}}
\newcommand{\zcap}[0]{\hat{\Bz}}
\newcommand{\thetacap}[0]{\hat{\Btheta}}

%
% to write R^n and C^n in a distinguishable fashion.  Perhaps change this
% to the double lined characters upon figuring out how to do so.
%
\newcommand{\C}[1]{$\mathbb{C}^{#1}$}
\newcommand{\R}[1]{$\mathbb{R}^{#1}$}

%
% various generally useful helpers
%

% derivative of #1 wrt. #2:
\newcommand{\D}[2] {\frac {d#2} {d#1}}

\newcommand{\inv}[1]{\frac{1}{#1}}
\newcommand{\cross}[0]{\times}

\newcommand{\abs}[1]{\lvert{#1}\rvert}
\newcommand{\norm}[1]{\lVert{#1}\rVert}
\newcommand{\innerprod}[2]{\langle{#1}, {#2}\rangle}
\newcommand{\dotprod}[2]{{#1} \cdot {#2}}
\newcommand{\bdotprod}[2]{\left({#1} \cdot {#2}\right)}
\newcommand{\crossprod}[2]{{#1} \cross {#2}}
\newcommand{\tripleprod}[3]{\dotprod{\left(\crossprod{#1}{#2}\right)}{#3}}

\DeclareMathOperator{\Proj}{Proj}
\DeclareMathOperator{\Span}{span}
\DeclareMathOperator{\Sgn}{sgn}
\DeclareMathOperator{\Area}{Area}
\DeclareMathOperator{\Volume}{Volume}

%
% A few miscellaneous things specific to this document
%
\newcommand{\crossop}[1]{\crossprod{#1}{}}

% R2 vector.
\newcommand{\VectorTwo}[2]{
\begin{bmatrix}
 {#1} \\
 {#2}
\end{bmatrix}
}

\newcommand{\VectorN}[1]{
\begin{bmatrix}
{#1}_1 \\
{#1}_2 \\
\vdots \\
{#1}_N \\
\end{bmatrix}
}

\newcommand{\DETuvij}[4]{
\begin{vmatrix}
 {#1}_{#3} & {#1}_{#4} \\
 {#2}_{#3} & {#2}_{#4}
\end{vmatrix}
}

\newcommand{\DETuvwijk}[6]{
\begin{vmatrix}
 {#1}_{#4} & {#1}_{#5} & {#1}_{#6} \\
 {#2}_{#4} & {#2}_{#5} & {#2}_{#6} \\
 {#3}_{#4} & {#3}_{#5} & {#3}_{#6}
\end{vmatrix}
}

\newcommand{\DETuvwxijkl}[8]{
\begin{vmatrix}
 {#1}_{#5} & {#1}_{#6} & {#1}_{#7} & {#1}_{#8} \\
 {#2}_{#5} & {#2}_{#6} & {#2}_{#7} & {#2}_{#8} \\
 {#3}_{#5} & {#3}_{#6} & {#3}_{#7} & {#3}_{#8} \\
 {#4}_{#5} & {#4}_{#6} & {#4}_{#7} & {#4}_{#8} \\
\end{vmatrix}
}

%\newcommand{\DETuvwxyijklm}[10]{
%\begin{vmatrix}
% {#1}_{#6} & {#1}_{#7} & {#1}_{#8} & {#1}_{#9} & {#1}_{#10} \\
% {#2}_{#6} & {#2}_{#7} & {#2}_{#8} & {#2}_{#9} & {#2}_{#10} \\
% {#3}_{#6} & {#3}_{#7} & {#3}_{#8} & {#3}_{#9} & {#3}_{#10} \\
% {#4}_{#6} & {#4}_{#7} & {#4}_{#8} & {#4}_{#9} & {#4}_{#10} \\
% {#5}_{#6} & {#5}_{#7} & {#5}_{#8} & {#5}_{#9} & {#5}_{#10}
%\end{vmatrix}
%}

% R3 vector.
\newcommand{\VectorThree}[3]{
\begin{bmatrix}
 {#1} \\
 {#2} \\
 {#3}
\end{bmatrix}
}



\author{Peeter Joot}
\email{peeter.joot@gmail.com}

%\documentclass[]{eliblogwidescreen}

\usepackage{amsmath}
\usepackage{mathpazo}

%
% shorthand for bold symbols, convenient for vectors and matrices
%
\newcommand{\Ba}[0]{\mathbf{a}}
\newcommand{\Bb}[0]{\mathbf{b}}
\newcommand{\Bc}[0]{\mathbf{c}}
\newcommand{\Bd}[0]{\mathbf{d}}
\newcommand{\Be}[0]{\mathbf{e}}
\newcommand{\Bf}[0]{\mathbf{f}}
\newcommand{\Bg}[0]{\mathbf{g}}
\newcommand{\Bh}[0]{\mathbf{h}}
\newcommand{\Bi}[0]{\mathbf{i}}
\newcommand{\Bj}[0]{\mathbf{j}}
\newcommand{\Bk}[0]{\mathbf{k}}
\newcommand{\Bl}[0]{\mathbf{l}}
\newcommand{\Bm}[0]{\mathbf{m}}
\newcommand{\Bn}[0]{\mathbf{n}}
\newcommand{\Bo}[0]{\mathbf{o}}
\newcommand{\Bp}[0]{\mathbf{p}}
\newcommand{\Bq}[0]{\mathbf{q}}
\newcommand{\Br}[0]{\mathbf{r}}
\newcommand{\Bs}[0]{\mathbf{s}}
\newcommand{\Bt}[0]{\mathbf{t}}
\newcommand{\Bu}[0]{\mathbf{u}}
\newcommand{\Bv}[0]{\mathbf{v}}
\newcommand{\Bw}[0]{\mathbf{w}}
\newcommand{\Bx}[0]{\mathbf{x}}
\newcommand{\By}[0]{\mathbf{y}}
\newcommand{\Bz}[0]{\mathbf{z}}
\newcommand{\BA}[0]{\mathbf{A}}
\newcommand{\BB}[0]{\mathbf{B}}
\newcommand{\BC}[0]{\mathbf{C}}
\newcommand{\BD}[0]{\mathbf{D}}
\newcommand{\BE}[0]{\mathbf{E}}
\newcommand{\BF}[0]{\mathbf{F}}
\newcommand{\BG}[0]{\mathbf{G}}
\newcommand{\BH}[0]{\mathbf{H}}
\newcommand{\BI}[0]{\mathbf{I}}
\newcommand{\BJ}[0]{\mathbf{J}}
\newcommand{\BK}[0]{\mathbf{K}}
\newcommand{\BL}[0]{\mathbf{L}}
\newcommand{\BM}[0]{\mathbf{M}}
\newcommand{\BN}[0]{\mathbf{N}}
\newcommand{\BO}[0]{\mathbf{O}}
\newcommand{\BP}[0]{\mathbf{P}}
\newcommand{\BQ}[0]{\mathbf{Q}}
\newcommand{\BR}[0]{\mathbf{R}}
\newcommand{\BS}[0]{\mathbf{S}}
\newcommand{\BT}[0]{\mathbf{T}}
\newcommand{\BU}[0]{\mathbf{U}}
\newcommand{\BV}[0]{\mathbf{V}}
\newcommand{\BW}[0]{\mathbf{W}}
\newcommand{\BX}[0]{\mathbf{X}}
\newcommand{\BY}[0]{\mathbf{Y}}
\newcommand{\BZ}[0]{\mathbf{Z}}

\newcommand{\Bzero}[0]{\mathbf{0}}
\newcommand{\Btheta}[0]{\boldsymbol{\theta}}
\newcommand{\Btau}[0]{\boldsymbol{\tau}}
\newcommand{\Bomega}[0]{\boldsymbol{\omega}}

%
% shorthand for unit vectors
%
\newcommand{\acap}[0]{\hat{\Ba}}
\newcommand{\bcap}[0]{\hat{\Bb}}
\newcommand{\ccap}[0]{\hat{\Bc}}
\newcommand{\dcap}[0]{\hat{\Bd}}
\newcommand{\ecap}[0]{\hat{\Be}}
\newcommand{\fcap}[0]{\hat{\Bf}}
\newcommand{\gcap}[0]{\hat{\Bg}}
\newcommand{\hcap}[0]{\hat{\Bh}}
\newcommand{\icap}[0]{\hat{\Bi}}
\newcommand{\jcap}[0]{\hat{\Bj}}
\newcommand{\kcap}[0]{\hat{\Bk}}
\newcommand{\lcap}[0]{\hat{\Bl}}
\newcommand{\mcap}[0]{\hat{\Bm}}
\newcommand{\ncap}[0]{\hat{\Bn}}
\newcommand{\ocap}[0]{\hat{\Bo}}
\newcommand{\pcap}[0]{\hat{\Bp}}
\newcommand{\qcap}[0]{\hat{\Bq}}
\newcommand{\rcap}[0]{\hat{\Br}}
\newcommand{\scap}[0]{\hat{\Bs}}
\newcommand{\tcap}[0]{\hat{\Bt}}
\newcommand{\ucap}[0]{\hat{\Bu}}
\newcommand{\vcap}[0]{\hat{\Bv}}
\newcommand{\wcap}[0]{\hat{\Bw}}
\newcommand{\xcap}[0]{\hat{\Bx}}
\newcommand{\ycap}[0]{\hat{\By}}
\newcommand{\zcap}[0]{\hat{\Bz}}
\newcommand{\thetacap}[0]{\hat{\Btheta}}

%
% to write R^n and C^n in a distinguishable fashion.  Perhaps change this
% to the double lined characters upon figuring out how to do so.
%
\newcommand{\C}[1]{$\mathbb{C}^{#1}$}
\newcommand{\R}[1]{$\mathbb{R}^{#1}$}

%
% various generally useful helpers
%

% derivative of #1 wrt. #2:
\newcommand{\D}[2] {\frac {d#2} {d#1}}

\newcommand{\inv}[1]{\frac{1}{#1}}
\newcommand{\cross}[0]{\times}

\newcommand{\abs}[1]{\lvert{#1}\rvert}
\newcommand{\norm}[1]{\lVert{#1}\rVert}
\newcommand{\innerprod}[2]{\langle{#1}, {#2}\rangle}
\newcommand{\dotprod}[2]{{#1} \cdot {#2}}
\newcommand{\bdotprod}[2]{\left({#1} \cdot {#2}\right)}
\newcommand{\crossprod}[2]{{#1} \cross {#2}}
\newcommand{\tripleprod}[3]{\dotprod{\left(\crossprod{#1}{#2}\right)}{#3}}

\DeclareMathOperator{\Proj}{Proj}
\DeclareMathOperator{\Span}{span}
\DeclareMathOperator{\Sgn}{sgn}
\DeclareMathOperator{\Area}{Area}
\DeclareMathOperator{\Volume}{Volume}

%
% A few miscellaneous things specific to this document
%
\newcommand{\crossop}[1]{\crossprod{#1}{}}

% R2 vector.
\newcommand{\VectorTwo}[2]{
\begin{bmatrix}
 {#1} \\
 {#2}
\end{bmatrix}
}

\newcommand{\VectorN}[1]{
\begin{bmatrix}
{#1}_1 \\
{#1}_2 \\
\vdots \\
{#1}_N \\
\end{bmatrix}
}

\newcommand{\DETuvij}[4]{
\begin{vmatrix}
 {#1}_{#3} & {#1}_{#4} \\
 {#2}_{#3} & {#2}_{#4}
\end{vmatrix}
}

\newcommand{\DETuvwijk}[6]{
\begin{vmatrix}
 {#1}_{#4} & {#1}_{#5} & {#1}_{#6} \\
 {#2}_{#4} & {#2}_{#5} & {#2}_{#6} \\
 {#3}_{#4} & {#3}_{#5} & {#3}_{#6}
\end{vmatrix}
}

\newcommand{\DETuvwxijkl}[8]{
\begin{vmatrix}
 {#1}_{#5} & {#1}_{#6} & {#1}_{#7} & {#1}_{#8} \\
 {#2}_{#5} & {#2}_{#6} & {#2}_{#7} & {#2}_{#8} \\
 {#3}_{#5} & {#3}_{#6} & {#3}_{#7} & {#3}_{#8} \\
 {#4}_{#5} & {#4}_{#6} & {#4}_{#7} & {#4}_{#8} \\
\end{vmatrix}
}

%\newcommand{\DETuvwxyijklm}[10]{
%\begin{vmatrix}
% {#1}_{#6} & {#1}_{#7} & {#1}_{#8} & {#1}_{#9} & {#1}_{#10} \\
% {#2}_{#6} & {#2}_{#7} & {#2}_{#8} & {#2}_{#9} & {#2}_{#10} \\
% {#3}_{#6} & {#3}_{#7} & {#3}_{#8} & {#3}_{#9} & {#3}_{#10} \\
% {#4}_{#6} & {#4}_{#7} & {#4}_{#8} & {#4}_{#9} & {#4}_{#10} \\
% {#5}_{#6} & {#5}_{#7} & {#5}_{#8} & {#5}_{#9} & {#5}_{#10}
%\end{vmatrix}
%}

% R3 vector.
\newcommand{\VectorThree}[3]{
\begin{bmatrix}
 {#1} \\
 {#2} \\
 {#3}
\end{bmatrix}
}



\author{Peeter Joot}
\email{peeter.joot@gmail.com}


%\usepackage[english]{babel}
%\usepackage{media9}
\chapter{Surface for spinning bucket of water.}

\label{chap:constantSpinSurfaces}
%\useCCL
\blogpage{http://sites.google.com/site/peeterjoot2/math2012/constantSpinSurfaces.pdf}
\date{Apr 29, 2012}
\gitRevisionInfo{constantSpinSurfaces}
\keywords{Navier-Stokes, Bernoulli's theorem, PHY454H1S, PHY454H1}

\beginArtWithToc
%\beginArtNoToc
%\wordpresscategory{}

\section{Motivation.}

Here's a problem from the 2009 phy1530 final that was appropriate for exam prep for this course too.  It also serves as a nice example of how to determine a surface as a function of pressure, something I want to do for the non-bottomless coffee problem to be attempted.

\section{Statement}

An undergraduate student is assigned a problem about an ideal fluid rotating at a constant angular velocity $\Omega$ under gravity $g$.  The velocity field is $\Bu = (-\Omega y, \Omega x, 0)$.  Here, $x$ and $y$ are horizontal and $z$ points up.  The student is supposed to find the surfaces of constant pressure, and hence the shape of the free surface of water in a rotating bucket.  The free surface corresponds to the surface for which $p = p_0$, where $p_0$ is the atmospheric pressure.  Surface tension is neglected.

\begin{enumerate}
\item On their homework assignment, the student writes:

``By Bernoulli's equation:

\begin{equation}\label{eqn:constantSpinSurfaces:10}
B = \frac{p}{\rho} + \inv{2}u^2 + g z
\end{equation}

where $B$ is a constant.  So the constant pressure surface at $p = p_0$ is 

\begin{equation}\label{eqn:constantSpinSurfaces:30}
z = \left( \frac{B}{g} - \frac{p_0}{\rho g}
\right)
- \frac{\Omega^2 }{2 g} \left( x^2 + y^2 \right).
\end{equation}
''

But this seems to show that the surface of the water in a rotating bucket is \textit{highest in the middle}!  What is wrong with the student's argument?

\item
Write down the Euler equations in component form and integrate them directly to find the pressure $p$, and hence obtain the correct parabolic shape for the free surface.
\end{enumerate}

\section{Solution.  Part 1.}

Let's recall how we derived Bernoulli's theorem.  We started with Navier-Stokes and used the identity

\begin{equation}\label{eqn:constantSpinSurfaces:50}
(\Bu \cdot \spacegrad ) \Bu = \spacegrad \inv{2} \Bu^2 + (\spacegrad \cross \Bu) \cross \Bu.
\end{equation}

Navier-Stokes for a steady state incompressible flow, with external body force per unit volume $\rho \Bg = -\rho \spacegrad \chi$ take the form

\begin{equation}\label{eqn:constantSpinSurfaces:70}
\spacegrad \inv{2} \Bu^2 + (\spacegrad \cross \Bu) \cross \Bu
= -\inv{\rho} \spacegrad p + \nu \spacegrad^2 \Bu - \spacegrad \chi.
\end{equation}

For the non-viscous (``dry-water'') case where we take $\mu = \nu \rho = 0$, and treat the density $\rho$ as a constant we find

\begin{equation}\label{eqn:constantSpinSurfaces:90}
\Bu \cross (\spacegrad \cross \Bu)
=
\spacegrad 
\left( 
\inv{2} \Bu^2 + \frac{p}{\rho} + \chi
\right).
\end{equation}

Observe that we only arrive at Bernoulli's theorem if the flow is also irrotational (as well as incompressible and non-viscous), as we require an irrotational flow where $\spacegrad \cdot \Bu = 0$ to claim that the gradient on the RHS is zero.  
%
%The most general claim that we can make, even for irrotational flows is that we have
%
%\begin{equation}\label{eqn:constantSpinSurfaces:110}
%0 = \Bu \cdot
%\spacegrad 
%\left( 
%\inv{2} \Bu^2 + \frac{p}{\rho} + \chi
%\right).
%\end{equation}
%
%That holds even for flows that are not irrotational, since $\Bu \cdot (\spacegrad \cross \Bu) = 0$.

In this problem we do not have an irrotational flow, which can be demonstrated by direct calculation.  We have

\begin{equation}\label{eqn:constantSpinSurfaces:130}
\begin{aligned}
\spacegrad \cross \Bu
&=
\Omega
\begin{vmatrix}
\xcap & \ycap & \zcap \\
\partial_x & \partial_y & 0 \\
-y & x & 0
\end{vmatrix} \\
&=
2 \zcap \Omega \\
&\ne 0
\end{aligned}
\end{equation}

In fact we have

\begin{equation}\label{eqn:constantSpinSurfaces:150}
\begin{aligned}
\Bu \cross (\spacegrad \cross \Bu)
&=
2 \Omega^2
\begin{vmatrix}
\xcap & \ycap & \zcap \\
-y & x & 0  \\
0 & 0 & 1
\end{vmatrix} \\
&=
2 \Omega^2 (\xcap + \ycap)
\end{aligned}
\end{equation}

The closest we can get to Bernoulli's theorem for this problem is

\begin{equation}\label{eqn:constantSpinSurfaces:210}
2 \Omega^2 (\xcap + \ycap)
= 
\spacegrad 
\left( 
\inv{2} \Bu^2 + \frac{p}{\rho} + g z
\right).
\end{equation}

We can say that the directional derivatives in directions perpendicular to $\xcap + \ycap$ are zero, and that 

\begin{equation}\label{eqn:constantSpinSurfaces:230}
\begin{aligned}
2 \Omega^2 
&= (\partial_x + \partial_y) 
\left( 
\inv{2} \Bu^2 + \frac{p}{\rho} + g z
\right) \\
&= (\partial_x + \partial_y) 
\left( 
\inv{2} \Bu^2 + \frac{p}{\rho} 
\right) \\
\end{aligned}
\end{equation}

Perhaps those could be used to solve for the surface, but we no longer have something that is obviously integrable.

Because $\Bu \cdot (\Bu \cross (\spacegrad \cross \Bu)) = 0$, we can also say that

\begin{equation}\label{eqn:constantSpinSurfaces:250}
\begin{aligned}
0 
&= \Bu \cdot \spacegrad
\left( 
\inv{2} \Bu^2 + \frac{p}{\rho} + g z
\right) \\
&= 
\Omega ( y \partial_x - x \partial_y )
\left( 
\inv{2} \Bu^2 + \frac{p}{\rho} 
\right).
\end{aligned}
\end{equation}

Perhaps this could also be used to find the surface?

\section{Solution.  Part 2.}

We want to write down the steady state, incompressible, non-viscous Navier-Stokes equations.  The first of these is trivially satisfied by our assumed solution

\begin{equation}\label{eqn:constantSpinSurfaces:270}
0 
= \spacegrad \cdot \Bu 
= \partial_x (-\Omega y) + \partial_y(\Omega x).
\end{equation}

For the inertial term we've got

\begin{equation}\label{eqn:constantSpinSurfaces:290}
\begin{aligned}
\Bu \cdot \spacegrad \Bu
&=
\Omega^2 (-y \partial_x + x \partial_y (-y, x, 0) \\
&=
\Omega^2 (-x, -y, 0),
\end{aligned}
\end{equation}

Leaving us with

\begin{equation}\label{eqn:constantSpinSurfaces:310}
\begin{aligned}
-\Omega^2 x &= -\inv{\rho} \partial_x p \\
-\Omega^2 y &= -\inv{\rho} \partial_y p \\
          0 &= -\inv{\rho} \partial_z p - g
\end{aligned}
\end{equation}

Integrating these, we seek seek simultaneous solutions to

\begin{equation}\label{eqn:constantSpinSurfaces:330}
\begin{aligned}
p &= \inv{2} \rho \Omega^2 x^2 + f(y,z) \\
p &= \inv{2} \rho \Omega^2 y^2 + g(x,z) \\
p &= h(x, y) - \rho g z.
\end{aligned}
\end{equation}

It's clear that one solution would be

\begin{equation}\label{eqn:constantSpinSurfaces:350}
p = p_0 + \inv{2} \rho \Omega^2 (x^2 + y^2) - \rho g z.
\end{equation}

where $p_0$ is some constant to be determined, dependent on where we set our origin.  Putting the origin of the coordinate system at the lowest point in the parabolic profile $(x, y, z) = (0, 0, 0)$, we have $p(0, 0, 0) = p_0$, which fixes $p_0$ as the atmospheric pressure.  If the radius of the bucket is $R$, the max height $h$ of the surface above that point is also found on this surface of constant pressure

\begin{equation}\label{eqn:constantSpinSurfaces:370}
p_0 = p_0 + \inv{2} \rho \Omega^2 R^2 - \rho g h,
\end{equation}

or 

\begin{equation}\label{eqn:constantSpinSurfaces:390}
h = \frac{\Omega^2 R^2 }{2 g}.
\end{equation}

\section{Appendix.  Proof of vector identities used.}

\begin{equation}\label{eqn:constantSpinSurfaces:170}
\begin{aligned}
\left( \spacegrad \inv{2} \Bu^2 + (\spacegrad \cross \Bu) \cross \Bu \right)_i
&=
\partial_i \inv{2} u_j u_j + \partial_a u_b \epsilon_{a b r} u_s \epsilon_{r s i} \\
&=
u_j \partial_i u_j + \partial_a u_b u_s \delta^{[a b]}_{s i} \\
&=
u_j \partial_i u_j 
+ u_s \partial_s u_i 
- u_s \partial_i u_s \\
&= (\Bu \cdot \spacegrad) \Bu)_i 
\end{aligned}
\end{equation}

Also observe that our claim that $\Bu \cdot (\Bu \cross (\spacegrad \Bu)) = 0$ follows easily after expansion in coordinates

\begin{equation}\label{eqn:constantSpinSurfaces:190}
\Bu \cdot (\Bu \cross (\spacegrad \cross \Bu) )
=
u_i u_s ( \partial_s u_i - \partial_i u_s ).
\end{equation}

We've got a symmetric and antisymmetric factor in the summation, so the end result is zero.

%\EndArticle
\EndNoBibArticle
