% 
% 
% 
% Copyright © 2012 Peeter Joot
% All Rights Reserved
% 
% This file may be reproduced and distributed in whole or in part, without fee, subject to the following conditions:
% 
% o The copyright notice above and this permission notice must be preserved complete on all complete or partial copies.
% 
% o Any translation or derived work must be approved by the author in writing before distribution.
% 
% o If you distribute this work in part, instructions for obtaining the complete version of this file must be included, and a means for obtaining a complete version provided.
% 
% 
% Exceptions to these rules may be granted for academic purposes: Write to the author and ask.
% 
% 
% 

\section{Continuum Mechanics.}

Mechanics could be defined as the study of effects of forces and displacements on a physical body

%figure (\ref{fig:continuumL2:continuumL2fig1})
\begin{figure}[htp]
   \centering
   \includegraphics[totalheight=0.2\textheight]{continuumL2fig1}
   \caption{Physical body.}\label{fig:continuumL2:continuumL2fig1}
\end{figure}

In continuum mechanics we have a physical body and we are interested in the internal motions in the object.

%figure (\ref{fig:continuumL2:continuumL2fig2})
\begin{figure}[htp]
   \centering
   \includegraphics[totalheight=0.2\textheight]{continuumL2fig2}
   \caption{Control volume elements.}\label{fig:continuumL2:continuumL2fig2}
\end{figure}

For the first time considering mechanics we have to introduce the concepts of fields to make progress tackling these problems.

We will have use of the following types of fields

\begin{itemize}
\item Scalar fields.  $3^0$ components.  Examples: density, Temperature, ...
\item Vector fields.  $3^1$ components.  Examples: Force, velocity.
\item Tensor fields.  $3^2$ components.  Examples: stress, strain.
\end{itemize}

We have to consider objects (a control volume) that is small enough that we can consider that we have a point in space limit for the quantities of density and velocity.  At the same time we cannot take this limiting process to the extreme, since if we use a control volume that is sufficiently small, quantum and inter-atomic effects would have to be considered.

%figure (\ref{fig:continuumL2:continuumL2fig3})
\begin{figure}[htp]
   \centering
   \includegraphics[totalheight=0.2\textheight]{continuumL2fig3}
   \caption{Mass and volume ratios at different scales.}\label{fig:continuumL2:continuumL2fig3}
\end{figure}

\subsection{Stress and Strain definitions.}

\begin{definition}
\emph{(Stress)}
\label{dfn:continuumL2:10}
Measure of the Internal force on the surfaces.
\end{definition}

\begin{definition}
\emph{(Strain)}
\label{dfn:continuumL2:30}
Measure of the deformation of the body.
\end{definition}

