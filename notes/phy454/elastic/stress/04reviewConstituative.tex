\section{Final Review: Constitutive relation.}

In the scope of this course we considered only Newtonian materials, those for which the stress and strain tensors are linearly related

\begin{equation}\label{eqn:continuumElasticityReview:290}
\sigma_{ij} = c_{ijkl} e_{kl},
\end{equation}

and further restricted our attention to isotropic materials, which can be shown to have the form

\begin{equation}\label{eqn:continuumElasticityReview:310}
\sigma_{ij} = \lambda e_{kk} \delta_{ij} + 2 \mu e_{ij},
\end{equation}

where $\lambda$ and $\mu$ are the Lame parameters and $\mu$ is called the shear modulus (and viscosity in the context of fluids).

By computing the trace of the stress $\sigma_{ii}$ we can invert this to find

\begin{equation}\label{eqn:continuumElasticityReview:330}
2 \mu e_{ij} = \sigma_{ij} - \frac{\lambda}{3 \lambda + 2 \mu} \sigma_{kk} \delta_{ij}.
\end{equation}

