\label{chap:continuumL4}
\blogpage{http://sites.google.com/site/peeterjoot2/math2012/continuumL4.pdf}
%\date{Jan 20, 2012}
\revisionInfo{continuumL4.tex}

\section{Stress tensor.}

Reading: Portions of this lecture cover \S 2 from the text \cite{landau1960theory}.

For the stress tensor

\begin{equation}\label{eqn:continuumL4:10}
\sigma_{ij},
\end{equation}

a second rank tensor, the first index $i$ defines the direction of the force, and the second index $j$ defines the surface.

Observe that the dimensions of $\sigma_{ij}$ is force per unit area, just like pressure.  We will in fact show that this tensor is akin to the pressure, and the diagonalized components of this tensor represent the pressure.

We've illustrated the stress tensor in a couple of 2D examples.  The first we call uniaxial stress, having just the $1,1$ element of the matrix as illustrated in figure (\ref{fig:continuumL4:continuumL4fig1})

\imageFigure{continuumL4fig1}{Uniaxial stress}{fig:continuumL4:continuumL4fig1}{0.3}

\begin{equation}\label{eqn:continuumL4:30}
\sigma = 
\begin{bmatrix}
\sigma_{11} & 0 \\
0 & 0
\end{bmatrix}.
\end{equation}

A biaxial stress is illustrated in figure (\ref{fig:continuumL4:continuumL4fig2})
\imageFigure{continuumL4fig2}{Biaxial stress.}{fig:continuumL4:continuumL4fig2}{0.3}

where for $\sigma_{11} \ne \sigma_{22}$ our tensor takes the form

\begin{equation}\label{eqn:continuumL4:50}
\sigma = 
\begin{bmatrix}
\sigma_{11} & 0 \\
0 & \sigma_{22}
\end{bmatrix}.
\end{equation}

In the general case we have

\begin{equation}\label{eqn:continuumL4:70}
\sigma = 
\begin{bmatrix}
\sigma_{11} & \sigma_{12} \\
\sigma_{21} & \sigma_{22}
\end{bmatrix}.
\end{equation}

We can attempt to illustrate this, but it becomes much harder to visualize as shown in figure (\ref{fig:continuumL4:continuumL4fig3})
\imageFigure{continuumL4fig3}{General stress}{fig:continuumL4:continuumL4fig3}{0.3}

In equilibrium we must have

\begin{equation}\label{eqn:continuumL4:90}
\sigma_{12} = \sigma_{21}.
\end{equation}

We can use similar arguments to show that the stress tensor is symmetric.

In 3D we have three components of the stress tensor acting on each surface, as illustrated in figure (\ref{fig:continuumL4:continuumL4fig5})
\imageFigure{continuumL4fig5}{Strain components on a 3D volume.}{fig:continuumL4:continuumL4fig5}{0.3}

We have three unique surface orientations and three components of the force for each of these, resulting in nine components, but these are not all independent.  For an object in equilibrium we must have $\sigma_{ij} = \sigma_{ji}$ (FIXME: justify?).  Explicitly, that is

\begin{align}\label{eqn:continuumL4:110}
\sigma_{12} &= \sigma_{21} \\
\sigma_{23} &= \sigma_{32} \\
\sigma_{31} &= \sigma_{13}
\end{align}

