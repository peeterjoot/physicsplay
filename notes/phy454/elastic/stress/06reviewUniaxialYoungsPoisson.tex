\section{Final Review: Uniaxial stress.  Young's modulus.  Poisson's ratio.}

For the special case with only one non-zero stress component (we used $\sigma_{11}$) we were able to compute Young's modulus $E$, the ratio between stress and strain in that direction

\begin{equation}\label{eqn:continuumElasticityReview:410}
E = \frac{\sigma_{11}}{e_{11}} = \frac{\mu(3 \lambda + 2 \mu)}{\lambda + \mu }  = \frac{3 K \mu}{K + \mu/3}.
\end{equation}

Just because only one component of the stress is non-zero, does not mean that we have no deformation in any other directions.  Introducing Poisson's ratio $\nu$ in terms of the ratio of the strains relative to the strain in the direction of the force we write and then subsequently found

\begin{equation}\label{eqn:continuumElasticityReview:430}
\nu = -\frac{e_{22}}{e_{11}} = -\frac{e_{33}}{e_{11}} = \frac{\lambda}{2(\lambda + \mu)}.
\end{equation}

We were also able to find

We can also relate the Poisson's ratio $\nu$ to the shear modulus $\mu$

\begin{equation}\label{eqn:continuumElasticityReview:450}
\mu = \frac{E}{2(1 + \nu)}
\end{equation}

\begin{equation}\label{eqn:continuumElasticityReview:470}
\lambda = \frac{E \nu}{(1 - 2 \nu)(1 + \nu)}
\end{equation}

\begin{align}\label{eqn:continuumElasticityReview:490}
e_{11} &= \inv{E}\left( \sigma_{11} - \nu(\sigma_{22} + \sigma_{33}) \right) \\
e_{22} &= \inv{E}\left( \sigma_{22} - \nu(\sigma_{11} + \sigma_{33}) \right) \\
e_{33} &= \inv{E}\left( \sigma_{33} - \nu(\sigma_{11} + \sigma_{22}) \right)
\end{align}
