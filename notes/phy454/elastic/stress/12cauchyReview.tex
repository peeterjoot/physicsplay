\label{chap:continuumL5}
\blogpage{http://sites.google.com/site/peeterjoot2/math2012/continuumL5.pdf}
%\date{Jan 25, 2012}
\revisionInfo{continuumL5.tex}

\beginArtWithToc
%\beginArtNoToc

\section{Review: Cauchy Tetrahedron.}

Referring to figure (\ref{fig:continuumL5:continuumL5fig1})
\imageFigure{continuumL5fig1}{Cauchy tetrahedron direction cosines.}{fig:continuumL5:continuumL5fig1}{0.2}

recall that we can decompose our force into components that refer to our direction cosines $n_i = \cos\phi_i$

\begin{align}\label{eqn:continuumL5:10}
f_1 &= \sigma_{11} n_1 + \sigma_{12} n_2 + \sigma_{13} n_3 \\
f_2 &= \sigma_{21} n_1 + \sigma_{22} n_2 + \sigma_{23} n_3 \\
f_3 &= \sigma_{31} n_1 + \sigma_{32} n_2 + \sigma_{33} n_3
\end{align}

Or in tensor form

\begin{equation}\label{eqn:continuumL5:30}
f_i = \sigma_{ij} n_j.
\end{equation}

We call this the traction vector and denote it in vector form as

\begin{equation}\label{eqn:continuumL5:50}
\BT = \Bsigma \cdot \ncap
=
\begin{bmatrix}
\sigma_{11} & \sigma_{12} & \sigma_{13} \\
\sigma_{21} & \sigma_{22} & \sigma_{23} \\
\sigma_{31} & \sigma_{32} & \sigma_{33}
\end{bmatrix}
\begin{bmatrix}
n_1 \\
n_2 \\
n_3
\end{bmatrix}
\end{equation}

