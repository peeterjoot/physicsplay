\section{Stress tensor.}

Reading for this section is \S 2 from the text associated with the prepared notes \cite{landau1960theory}.

We'd like to consider a macroscopic model that contains the net effects of all the internal forces in the object as depicted in figure (\ref{fig:continuumL3:continuumL3fig2})

\imageFigure{figures/continuumL3fig2}{Internal forces.}{fig:continuumL3:continuumL3fig2}{0.2}

We will consider a volume big enough that we won't have to consider the individual atomic interactions, only the average effects of those interactions.  Will will look at the force per unit volume on a differential volume element

The total force on the body is 

\begin{equation}\label{eqn:continuumL3:230}
\iiint \BF dV,
\end{equation}

where $\BF$ is the force per unit volume.  We will evaluate this by utilizing the divergence theorem.  Recall that this was

\begin{equation}\label{eqn:continuumL3:250}
\iiint (\spacegrad \cdot \BA) dV
= \iint \BA \cdot d\Bs
\end{equation}

We have a small problem, since we have a non-divergence expression of the force here, and it is not immediately obvious that we can apply the divergence theorem.  We can deal with this by assuming that we can find a vector valued tensor, so that if we take the divergence of this tensor, we end up with the force.  We introduce the vector valued quantity

\begin{equation}\label{eqn:continuumL3:270}
\BF = \Be_i \PD{x_k}{\sigma_{ik}},
\end{equation}

and then apply the divergence theorem

\begin{equation}\label{eqn:continuumL3:290}
\iiint \BF dV 
= \iiint \Be_i \PD{x_k}{\sigma_{ik}} d\Bx^3 
=
\iint \Be_i \sigma_{ik} ds_k,
\end{equation}

where $ds_k$ is a surface element.  We identify this tensor

\begin{equation}\label{eqn:continuumL3:310}
\sigma_{ik} = \frac{\text{Force} \cdot \Be_i}{\text{Unit Area}}
\end{equation}

and 

\begin{equation}\label{eqn:continuumL3:330}
f_i = \sigma_{ik} ds_k,
\end{equation}

as the force on the surface element $ds_k$.  In two dimensions this is illustrated in the following figures (\ref{fig:continuumL3:continuumL3fig3})
\imageFigure{figures/continuumL3fig3}{2D strain tensor.}{fig:continuumL3:continuumL3fig3}{0.2}

Observe that we use the index $i$ above as the direction of the force, and index $k$ as the direction normal to the surface.

Note that the strain tensor has the matrix form

\begin{equation}\label{eqn:continuumL3:350}
\begin{bmatrix}
\sigma_{11} & \sigma_{12} & \sigma_{13} \\
\sigma_{21} & \sigma_{22} & \sigma_{23} \\
\sigma_{31} & \sigma_{32} & \sigma_{33}
\end{bmatrix}
\end{equation}

We will show later that this tensor is in fact symmetric.

FIXME: given some 3D forces, compute the stress tensor that is associated with it.

