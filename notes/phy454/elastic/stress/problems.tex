% 
% 
% 
% Copyright © 2012 Peeter Joot
% All Rights Reserved
% 
% This file may be reproduced and distributed in whole or in part, without fee, subject to the following conditions:
% 
% o The copyright notice above and this permission notice must be preserved complete on all complete or partial copies.
% 
% o Any translation or derived work must be approved by the author in writing before distribution.
% 
% o If you distribute this work in part, instructions for obtaining the complete version of this file must be included, and a means for obtaining a complete version provided.
% 
% 
% Exceptions to these rules may be granted for academic purposes: Write to the author and ask.
% 
% 
% 
\section{Problems}

\label{problem:strain:ps1q1}
\subsection{Strain tensor from stress tensor.}

For the stress tensor

\begin{equation}\label{eqn:continuumProblemSet1:10}
\sigma =
\begin{bmatrix}
6 & 0 & 2 \\
0 & 1 & 1 \\
2 & 1 & 3
\end{bmatrix}
\text{M Pa}
\end{equation}

Find the corresponding strain tensor, assuming an isotropic solid with Young's modulus $E = 200 \times 10^9 \text{N}/\text{m}^2$ and Poisson's ration $\nu = 0.35$.

\label{problem:strain:ps1q2b}
\subsection{Stress and strain relations for a small displacement.}

For the problem \ref{problem:strain:ps1q2a}, is the body under compression or expansion?

Solution: \ref{solutions:ps1q2b}.

\label{problem:strain:ps1q3}
\subsection{Traction vector}

The stress tensor at a point has components given by

\begin{equation}\label{eqn:continuumProblemSet1:50}
\sigma =
\begin{bmatrix}
1 & -2 & 2 \\
-2 & 3 & 1 \\
2 & 1 & -1
\end{bmatrix}.
\end{equation}

Find the traction vector across an area normal to the unit vector

\begin{equation}\label{eqn:continuumProblemSet1:70}
\ncap = ( \sqrt{2} \Be_1 - \Be_2 + \Be_3)/2
\end{equation}

Can you construct a tangent vector $\Btau$ on this plane by inspection?  What are the components of the force per unit area along the normal $\ncap$ and tangent $\Btau$ on that surface?  (hint: projection of the traction vector.)

Solution: \ref{solutions:ps1q3}.

\label{problem:strain:ps1q4}
\subsection{Stress and equilibrium}

The stress tensor of a body is given by

\begin{equation}\label{eqn:continuumProblemSet1:90}
\sigma =
\begin{bmatrix}
A \cos x & y^2 & C x \\
y^2 & B \sin y & z \\
C x & z & z^3
\end{bmatrix}
\end{equation}

Determine the constant $A$, $B$, and $C$ if the body is in equilibrium.

Solution: \ref{solutions:ps1q4}.
