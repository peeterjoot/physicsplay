\label{chap:continuumL3}
%\useCCL
\blogpage{http://sites.google.com/site/peeterjoot2/math2012/continuumL3.pdf}
%\date{Jan 18, 2012}
\revisionInfo{continuumL3.tex}

\section{Review.  Strain.}

Strain is the measure of stretching.  This is illustrated pictorially in figure (\ref{fig:continuumL3:continuumL3fig1})
\imageFigure{continuumL3fig1}{Stretched line elements.}{fig:continuumL3:continuumL3fig1}{0.2}

\begin{equation}\label{eqn:continuumL3:10}
{ds'}^2 - ds^2 = 2 e_{ik} dx_i dx_k,
\end{equation}

where $e_{ik}$ is the strain tensor.  We found

\begin{equation}\label{eqn:continuumL3:30}
e_{ik} = \inv{2} \left( 
\PD{x_k}{e_i} 
+\PD{x_i}{e_k} 
+
\PD{x_i}{e_l} 
\PD{x_k}{e_l} 
\right)
\end{equation}

Why do we have a factor two?  Observe that if the deformation is small we can write

\begin{align*}
{ds'}^2 - ds^2 
&= (ds' - ds)(ds' + ds) \\
&\approx
 (ds' - ds) 2 ds
\end{align*}

so that we find 

\begin{equation}\label{eqn:continuumL3:50}
\frac{{ds'}^2 - ds^2 }{ds^2}
\approx
\frac{ds' - ds }{ds}
\end{equation}

Suppose for example, that we have a diagonalized strain tensor, then we find

\begin{equation}\label{eqn:continuumL3:70}
{ds'}^2 - ds^2 
= 2 e_{ii} \left(\frac{dx_i}{ds}\right)^2
\end{equation}

so that

\begin{equation}\label{eqn:continuumL3:90}
\frac{
{ds'}^2 - ds^2 
}{ds^2}
= 2 e_{ii} dx_i^2
\end{equation}

Observe that here again we see this factor of two.

If we have a diagonalized strain tensor, the tensor is of the form

\begin{equation}\label{eqn:continuumL3:110}
\begin{bmatrix}
e_{11} & 0 & 0 \\
0 & e_{22} & 0 \\
0 & 0 & e_{33} 
\end{bmatrix}
\end{equation}

we have

\begin{equation}\label{eqn:continuumL3:130}
{dx_i'}^2 - dx_i^2 = 2 e_{ii} dx_i^2
\end{equation}

\begin{equation}\label{eqn:continuumL3:150}
{ds'}^2 = 
(1 + 2 e_{11}) dx_1^2
+(1 + 2 e_{22}) dx_2^2
+(1 + 2 e_{33}) dx_3^2
\end{equation}

\begin{equation}\label{eqn:continuumL3:170}
ds^2 = 
dx_1^2
+dx_2^2
+dx_3^2
\end{equation}

so 

\begin{align}\label{eqn:continuumL3:190}
dx_1' &= \sqrt{1 + 2 e_{11}} dx_1 \sim ( 1 + e_{11}) dx_1 \\
dx_2' &= \sqrt{1 + 2 e_{22}} dx_2 \sim ( 1 + e_{22}) dx_2 \\
dx_3' &= \sqrt{1 + 2 e_{33}} dx_3 \sim ( 1 + e_{33}) dx_3
\end{align}

Observe that the change in the volume element becomes the trace

\begin{equation}\label{eqn:continuumL3:210}
dV' = 
dx_1'
dx_2'
dx_3'
= dV(1 + e_{ii})
\end{equation}

How do we use this?  Suppose that you are given a strain tensor.  This should allow you to compute the stretch in any given direction.

FIXME: find problem and try this.
