\section{Final Review: Stress tensor.}

We sought and found a representation of the force per unit area acting on a body by expressing the components of that force as a set of divergence relations

\begin{equation}\label{eqn:continuumElasticityReview:250}
f_i = \partial_k \sigma_{i k},
\end{equation}

and call the associated tensor $\sigma_{ij}$ the \textit{stress}.

Unlike the strain, we don't have any expectation that this tensor is symmetric, and identify the diagonal components (no sum) $\sigma_{i i}$ as quantifying the amount of compressive or contractive force per unit area, whereas the cross terms of the stress tensor introduce shearing deformations in the solid.

With force balance arguments (the Cauchy tetrahedron) we found that the force per unit area on the solid, for a surface with unit normal pointing into the solid, was

\begin{equation}\label{eqn:continuumElasticityReview:270}
\Bt = \Be_i t_i = \Be_i \sigma_{ij} n_j.
\end{equation}

