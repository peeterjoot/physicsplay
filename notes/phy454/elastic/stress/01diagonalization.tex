\section{Diagonalization}

We'll look at the two dimensional case in some detail, as in figure (\ref{fig:continuumL4:continuumL4fig6})

\imageFigure{continuumL4fig6}{Area element under stress with and without rotation.}{fig:continuumL4:continuumL4fig6}{0.3}

Under this coordinate transformation, a rotation, the diagonal stress tensor is taken to a non-diagonal form

\begin{equation}\label{eqn:continuumL4:130}
\begin{bmatrix}
\sigma_{11} & 0 \\
0 & \sigma_{22} 
\end{bmatrix}
\leftrightarrow
\begin{bmatrix}
\sigma_{11}' & \sigma_{12}' \\
\sigma_{21}' & \sigma_{22}' 
\end{bmatrix}
\end{equation}
