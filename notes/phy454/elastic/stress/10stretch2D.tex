\section{Example: stretch in a pair of mutually perpendicular directions}

For a pair of perpendicular forces applied in two dimensions, as illustrated in figure (\ref{fig:continuumL3:continuumL3fig5})
\imageCentered{figures/continuumL3fig5}{Mutually perpendicular forces}{fig:continuumL3:continuumL3fig5}{0.2}

our stress tensor now just takes the form

\begin{equation}\label{eqn:continuumL3:390}
\begin{bmatrix}
\sigma_{11} & 0 \\
0 & \sigma_{22}
\end{bmatrix}
\end{equation}

It's easy to imagine now how to get some more general stress tensors, should we make a change of basis that rotates our frame.

