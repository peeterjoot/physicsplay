% 
% 
% 
% Copyright � 2012 Peeter Joot
% All Rights Reserved
% 
% This file may be reproduced and distributed in whole or in part, without fee, subject to the following conditions:
% 
% o The copyright notice above and this permission notice must be preserved complete on all complete or partial copies.
% 
% o Any translation or derived work must be approved by the author in writing before distribution.
% 
% o If you distribute this work in part, instructions for obtaining the complete version of this file must be included, and a means for obtaining a complete version provided.
% 
% 
% Exceptions to these rules may be granted for academic purposes: Write to the author and ask.
% 
% 
% 
%\chapter{PHY454H1S\\Continuum Mechanics.  Lecture 6: Compatibility condition and elastostatics.  Taught by Prof. K. Das}
\section{Compatibility condition and elastostatics}
\label{chap:continuumL6}

%\section{Review: Elastostatics}
%
%We've defined the strain tensor, where assuming the second order terms are ignored, was
%
%\begin{equation}\label{eqn:continuumL6:10}
%e_{ij} = 
%\inv{2} \left( 
%\PD{x_j}{e_i}
%+ \PD{x_i}{e_j} \right).
%\end{equation}
%
%We've also defined a stress tensor defined implicitly as a divergence relationship using the force per unit volume $F_i$ in direction $i$
%
%\begin{equation}\label{eqn:continuumL6:30}
%\sigma_{ij} \leftrightarrow F_i = \PD{x_j}{\sigma_{ij}}.
%\end{equation}
%
%We've also discussed the constitutive relation, relating stress $\sigma_{ij}$ and strain $e_{ij}$.
%
%We've also discussed linear constitutive relationships (Hooke's law).  

\section{Elastodynamics.  Elastic waves}

Reading: Chapter I \S 7, chapter III (\S 22 - \S 26) of the text \citep{landau1960theory}.

Example: sound or water waves (i.e. waves in a solid or liquid material that comes back to its original position.)

\makedefinition{Elastic Wave}{dfn:continuumL6:10}{An elastic wave \index{elastic wave} is a type of mechanical wave that propagates through or on the surface of a medium.  The elasticity of the material provides the restoring force (that returns the material to its original state).  The displacement and the restoring force are assumed to be linearly related.}

In symbols we say

\begin{equation}\label{eqn:continuumL6:130}
e_i(x_j, t) \quad \mbox{related to force},
\end{equation}

and specifically

\begin{equation}\label{eqn:continuumL6:150}
\rho \PDSq{t}{e_i} = F_i = \PD{x_j}{\sigma_{ij}}.
\end{equation}

This is just Newton's second law, $F = ma$, but expressed in terms of a unit volume.

Should we have an external body force (per unit volume) $f_i$ acting on the body then we must modify this, writing

\begin{equation}\label{eqn:continuumL6:170}
\boxed{
\rho \PDSq{t}{e_i} = \PD{x_j}{\sigma_{ij}} + f_i
}
\end{equation}

Note that we are separating out the ``original'' forces that produced the stress and strain on the object from any constant external forces that act on the body (i.e. a gravitational field).

With 

\begin{equation}\label{eqn:continuumL6:190}
e_{ij} = 
\inv{2} \left( 
\PD{x_j}{e_i}
+ \PD{x_i}{e_j} \right),
\end{equation}

we can expand the stress divergence, for the case of homogeneous deformation, in terms of the Lam\'e parameters

\begin{equation}\label{eqn:continuumL6:210}
\sigma_{ij} = \lambda e_{kk} \delta_{ij} + 2 \mu e_{ij}.
\end{equation}

We compute

\begin{align*}
\PD{x_j}{\sigma_{ij}}
&=
\lambda 
\PD{x_j}{
e_{kk}
}
\delta_{ij} + 2 \mu 
\PD{x_j}{
}
\inv{2} \left( 
\PD{x_j}{e_i}
+ \PD{x_i}{e_j} \right),
 \\
&=
\lambda 
\PD{x_i}{
e_{kk}
}
+ \mu 
\left(
\PDSq{x_j}{
e_{i}
}
+
\frac{\partial^2 e_{j} }{ \partial x_j \partial x_i}
\right) \\
&=
%\sum_k 
\lambda 
\PD{x_i}{
}
\PD{x_k}{e_k}
+ \mu 
\left(
\PDSq{x_j}{
e_{i}
}
+
\frac{\partial^2 e_{k} }{ \partial x_k \partial x_i}
\right) \\
&=
(\lambda + \mu)
\PD{x_i}{
}
\PD{x_k}{e_k}
+ \mu 
\PDSq{x_j}{
e_{i}
}
\end{align*}

%With 
%
%\begin{equation}\label{eqn:continuumL6:230}
%e_{kk} = e_{11} +e_{22} +e_{33}
%= 
%\PD{x_1}{e_1}
%+\PD{x_2}{e_2}
%+\PD{x_3}{e_3}
%\end{equation}
%
We find, for homogeneous deformations, that the force per unit volume on our element of mass, in the absence of external forces (the body forces), takes the form
%
%\begin{equation}\label{eqn:continuumL6:250}
%\PD{x_j}{\sigma_{ij}} = (\lambda + \mu) \frac{\partial^2 e_j}{\partial x_i \partial x_j}
%+ \mu
%\frac{\partial^2 e_i}
%{\partial x_j^2
%}
%\end{equation}

\begin{equation}\label{eqn:continuumL6:270}
\rho \PDSq{t}{e_i} = (\lambda + \mu) \frac{\partial^2 e_k}{\partial x_i \partial x_k}
+ \mu
\frac{\partial^2 e_i}
{\partial x_j^2
}.
\end{equation}

This can be seen to be equivalent to the vector relationship

\begin{equation}\label{eqn:continuumL6:290}
\boxed{
\rho \PDSq{t}{\Be} = (\lambda + \mu) \spacegrad (\spacegrad \cdot \Be) + \mu \spacegrad^2 \Be.
}
\end{equation}

\FIXME{What form do the stress and strain tensors take in vector form?}
