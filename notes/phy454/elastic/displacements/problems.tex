% 
% 
% 
% Copyright © 2012 Peeter Joot
% All Rights Reserved
% 
% This file may be reproduced and distributed in whole or in part, without fee, subject to the following conditions:
% 
% o The copyright notice above and this permission notice must be preserved complete on all complete or partial copies.
% 
% o Any translation or derived work must be approved by the author in writing before distribution.
% 
% o If you distribute this work in part, instructions for obtaining the complete version of this file must be included, and a means for obtaining a complete version provided.
% 
% 
% Exceptions to these rules may be granted for academic purposes: Write to the author and ask.
% 
% 
% 
\section{Problems}

\begin{Exercise}[title={$\BP$-waves, $\BS$-waves, and Love-waves}, label={problem:elastic:displacements:midtermQ1a}]

(This and subsequent problems are from the 2012 midterm.  We never got any real problems on elastic waves.  IMO it is meaningless to ask Q and A questions like this when we have not actually done anything with the theory.  Ought to be solving the wave equations under various conditions and then examine those solutions.

Show that in $\BP$-waves the divergence of the displacement vector represents a measure of the relative change in the volume of the body.
\end{Exercise}

\begin{Answer}[ref={problem:elastic:displacements:midtermQ1a}]
The $\BP$-wave equation was a result of operating on the displacement equation with the divergence operator

\begin{equation}\label{eqn:continuumMidTermReflection:10}
\spacegrad \cdot \left( 
\rho \PDSq{t}{\Be} = (\lambda + \mu) \spacegrad (\spacegrad \cdot \Be) + \mu \spacegrad^2 \Be
\right)
\end{equation}

we obtain

\begin{equation}\label{eqn:continuumMidTermReflection:30}
\PDSq{t}{} \left( \spacegrad \cdot \Be \right) = \frac{\lambda + 2 \mu}{\rho} \spacegrad^2 (\spacegrad \cdot \Be).
\end{equation}

We have a wave equation where the ``waving'' quantity is $\Theta = \spacegrad \cdot \Be$.  Explicitly

\begin{align*}
\Theta 
&= \spacegrad \cdot \Be \\
&= 
\PD{x}{e_1}
+\PD{y}{e_2}
+\PD{z}{e_3}
\end{align*}

Recall that, in a coordinate basis for which the strain $e_{ij}$ is diagonal we have

\begin{align}\label{eqn:continuumMidTermReflection:50}
dx' &= \sqrt{1 + 2 e_{11}} dx \\
dy' &= \sqrt{1 + 2 e_{22}} dy \\
dz' &= \sqrt{1 + 2 e_{33}} dz.
\end{align}

Expanding in Taylor series to $O(1)$ we have for $i = 1, 2, 3$ (no sum)

\begin{equation}\label{eqn:continuumMidTermReflection:70}
dx_i' \approx (1 + e_{ii}) dx_i.
\end{equation}

so the displaced volume is

\begin{align*}
dV' &= 
dx_1
dx_2
dx_3
(1 + e_{11})
(1 + e_{22})
(1 + e_{33}) \\
&=
dx_1
dx_2
dx_3
( 1  + e_{11} + e_{22} + e_{33} + O(e_{kk}^2) )
\end{align*}

Since 

\begin{align}\label{eqn:continuumMidTermReflection:90}
e_{11} &= \inv{2} \left( \PD{x}{e_1} +\PD{x}{e_1} \right) = \PD{x}{e_1} \\
e_{22} &= \inv{2} \left( \PD{y}{e_2} +\PD{y}{e_2} \right) = \PD{y}{e_2} \\
e_{33} &= \inv{2} \left( \PD{z}{e_3} +\PD{z}{e_3} \right) = \PD{z}{e_3}
\end{align}

We have

\begin{equation}\label{eqn:continuumMidTermReflection:110}
dV' = (1 + \spacegrad \cdot \Be) dV,
\end{equation}

or

\begin{equation}\label{eqn:continuumMidTermReflection:130}
\frac{dV' - dV}{dV} = \spacegrad \cdot \Be
\end{equation}

The relative change in volume can therefore be expressed as the divergence of $\Be$, the displacement vector, and it is this relative volume change that is ``waving'' in the $\BP$-wave equation as illustrated in the following (\ref{fig:continuumMidtermReflection:continuumMidtermReflectionFig1}) sample 1D compression wave

\imageCentered{figures/continuumMidtermReflectionFig1}{A 1D compression wave.}{fig:continuumMidtermReflection:continuumMidtermReflectionFig1}{0.2}
\end{Answer}

\begin{Exercise}[title={$\BP$-waves, $\BS$-waves.  Longitudinal or transverse.}, label={problem:elastic:displacements:midtermQ1b}]
Between a $\BP$-wave and an $\BS$-wave which one is longitudinal and which one is transverse?
\end{Exercise}

\begin{Answer}[ref={problem:elastic:displacements:midtermQ1b}]
$\BP$-waves are longitudinal.

$\BS$-waves are transverse.
\end{Answer}

\begin{Exercise}[title={$\BP$-waves, $\BS$-waves.  Speed.}, label={problem:elastic:displacements:midtermQ1c}]
Whose speed is higher?
\end{Exercise}

\begin{Answer}[ref={problem:elastic:displacements:midtermQ1c}]
From the (midterm) formula sheet we have

\begin{align*}
\left( \frac{c_L}{c_T} \right)^2 
&= \frac{ \lambda + 2 \mu}{\rho} \frac{\rho}{\mu}  \\
&= \frac{\lambda}{\mu} + 2  \\
&> 1
\end{align*}

so $\BP$-waves travel faster than $\BS$-waves.
\end{Answer}

\begin{Exercise}[title={Love waves.}, label={problem:elastic:displacements:midtermQ1d}]
Is Love wave a body wave or a surface wave?
\end{Exercise}

\begin{Answer}[ref={problem:elastic:displacements:midtermQ1d}]
Love waves are surface waves, traveling in a medium that can slide on top of another surface.  They are characterized by shear displacements perpendicular to the direction of propagation.

Reviewing for the final I see that I'd answered this wrong, and have corrected it.  I'd described a Rayleigh wave (also a surface wave).  A Rayleigh wave is characterized by vorticity rotating backwards compared to the direction of propagation as shown in figure (\ref{fig:continuumMidtermReflection:continuumMidtermReflectionFig2})


\pdfTexCentered{figures/continuumMidtermReflectionFig2.pdf_tex}{Rayleigh wave illustrated.}{fig:continuumMidtermReflection:continuumMidtermReflectionFig2}{0.6}
\end{Answer}

\section{Solutions}

\shipoutAnswer
