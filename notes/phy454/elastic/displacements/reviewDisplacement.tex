% 
% 
% 
% Copyright © 2012 Peeter Joot
% All Rights Reserved
% 
% This file may be reproduced and distributed in whole or in part, without fee, subject to the following conditions:
% 
% o The copyright notice above and this permission notice must be preserved complete on all complete or partial copies.
% 
% o Any translation or derived work must be approved by the author in writing before distribution.
% 
% o If you distribute this work in part, instructions for obtaining the complete version of this file must be included, and a means for obtaining a complete version provided.
% 
% 
% Exceptions to these rules may be granted for academic purposes: Write to the author and ask.
% 
% 
% 
%\chapter{Displacement propagation}

It was argued that the equation relating the time evolution of a one of the vector displacement coordinates was given by

\begin{equation}\label{eqn:continuumElasticityReview:570}
\rho \PDSq{t}{u_i} = \PD{x_j}{\sigma_{ij}} + f_i,
\end{equation}

where the divergence term $\PDi{x_j}{\sigma_{ij}}$ is the internal force per unit volume on the object and $f_i$ is the external force.  Employing the constitutive relation we showed that this can be expanded as

\begin{equation}\label{eqn:continuumElasticityReview:590}
\rho \PDSq{t}{u_i} = (\lambda + \mu) \frac{\partial^2 u_k}{\partial x_i \partial x_k}
+ \mu
\frac{\partial^2 u_i}
{\partial x_j^2
},
\end{equation}

or in vector form

\begin{equation}\label{eqn:continuumElasticityReview:610}
\rho \PDSq{t}{\Bu} = (\lambda + \mu) \spacegrad (\spacegrad \cdot \Bu) + \mu \spacegrad^2 \Bu.
\end{equation}

\section{Equilibrium}

When a body is in static equilibrium \ref{eqn:continuumElasticityReview:570} reduces to just a simple force balance

\begin{equation}\label{eqn:continuumFluidsReviewXX:1090b}
f_i = - \PD{x_j}{\sigma_{ij}}.
\end{equation}

In particular, if there are no external forces then all of these divergences must be zero.  As an example, suppose that the state of a body is given by

\begin{equation}\label{eqn:continuumFluidsReviewXX:3130}
\begin{aligned}
\sigma_{11} &= A x^4 y^3 \\
\sigma_{22} &= 3 B x^2 y^5 \\
\sigma_{12} &= -C x^3 y^4
\end{aligned}
\end{equation}

We can determine the constants $A$, $B$ and $C$ so that the body is in equilibrium (2011 Final Exam question II).  We have

\begin{equation}\label{eqn:continuumFluidsReviewXX:3150}
\begin{aligned}
0 
&= \PD{x_j}{\sigma_{1j}} \\
&= 
\PD{x}{\sigma_{11}} + \PD{y}{\sigma_{12}} \\
&= 4 A x^3 y^3 - 4 C x^3 y^3,
\end{aligned}
\end{equation}

and

\begin{equation}\label{eqn:continuumFluidsReviewXX:3170}
\begin{aligned}
0 
&= \PD{x_j}{\sigma_{2j}} \\
&= 
\PD{x}{\sigma_{21}} + \PD{y}{\sigma_{22}} \\
&= -3 C x^2 y^4 + 15 B x^2 y^4
\end{aligned}
\end{equation}

We must then have
\begin{equation}\label{eqn:continuumFluidsReviewXX:3190}
\begin{aligned}
0 &= A - C \\
0 &= -C + 5 B
\end{aligned}
\end{equation} 

Or

\begin{equation}\label{eqn:continuumFluidsReviewXX:3210}
\begin{aligned}
A &= C \\
B &= \frac{C}{5}.
\end{aligned}
\end{equation}

Also asked on last years final was an explaination of how the strain energy of tectonic plates causes Tsunami.  The root cause of the Tsunami is the earthquake under the body of water.  Once that earthquake occurs we'll have a body wave in the mantle, which will trigger a much more destructive (higher amplitude) surface wave (probably of the Rayleigh type).  Looking back to the connection with strain energy, we see that once we have a change in the strain divergence, we'll have to have a restoring force to put things back in equilibrium.  That restoring force can come either from the surrounding mantle or the fluid above it, and it's that fluid restoring force that induces the wave as a side effect.

\section{P-waves}

Operating on \ref{eqn:continuumElasticityReview:610} with the divergence operator, and writing $\Theta = \spacegrad \cdot \Bu$, a quantity that was our relative change in volume in the diagonal strain basis, we were able to find this divergence obeys a wave equation

\begin{equation}\label{eqn:continuumElasticityReview:630}
\PDSq{t}{\Theta} = \frac{\lambda + 2 \mu}{\rho} \spacegrad^2 \Theta.
\end{equation}

We called these P-waves.

\section{S-waves}

Similarly, operating on \ref{eqn:continuumElasticityReview:610} with the curl operator, and writing $\Bomega = \spacegrad \cross \Bu$, we were able to find this curl also obeys a wave equation

\begin{equation}\label{eqn:continuumElasticityReview:650}
\rho \PDSq{t}{\Bomega} = \mu \spacegrad^2 \Bomega.
\end{equation}

These we called S-waves.  We also noted that the (transverse) compression waves (P-waves) with speed $C_T = \sqrt{\mu/\rho}$, traveled faster than the (longitudinal) vorticity (S) waves with speed $C_L = \sqrt{(\lambda + 2 \mu)/\rho}$ since $\lambda > 0$ and $\mu > 0$, and 

\begin{equation}\label{eqn:continuumElasticityReview:670}
\frac{C_L}{C_T} = \sqrt{\frac{ \lambda + 2 \mu}{\mu}} = \sqrt{ \frac{\lambda}{\mu} + 2}.
\end{equation}

\section{Scalar and vector potential representation.}

Assuming a vector displacement representation with gradient and curl components

\begin{equation}\label{eqn:continuumElasticityReview:690}
\Bu = \spacegrad \phi + \spacegrad \cross \BH,
\end{equation}

We found that the displacement time evolution equation split nicely into curl free and divergence free terms

\begin{equation}\label{eqn:continuumElasticityReview:710}
\spacegrad
\left(
\rho \PDSq{t}{\phi} - (\lambda + 2\mu) \spacegrad^2 \phi
\right)
+
\spacegrad \cross
\left(
\rho \PDSq{t}{\BH} - \mu \spacegrad^2 \BH
\right)
= 0.
\end{equation}

When neglecting boundary value effects this could be written as a pair of independent equations

\begin{subequations}
\begin{equation}\label{eqn:continuumElasticityReview:730}
\rho \PDSq{t}{\phi} - (\lambda + 2\mu) \spacegrad^2 \phi = 0
\end{equation}
\begin{equation}\label{eqn:continuumElasticityReview:750}
\rho \PDSq{t}{\BH} - \mu \spacegrad^2 \BH
= 0.
\end{equation}
\end{subequations}

This are the irrotational (curl free) P-wave and solenoidal (divergence free) S-wave equations respectively.

%This theory led to no actual calculation work, just a few videos that illustrated what we'd presumably be able to calculate if we were to attempt to apply these concepts.

\section{Phasor description.}

It was mentioned that we could assume a phasor representation for our potentials, writing

\begin{subequations}
\begin{equation}\label{eqn:continuumElasticityReview:770}
\phi = A \exp\left( i ( \Bk \cdot \Bx - \omega t) \right) 
\end{equation}
\begin{equation}\label{eqn:continuumElasticityReview:790}
\BH = \BB \exp\left( i ( \Bk \cdot \Bx - \omega t) \right)
\end{equation}
\end{subequations}

finding

\begin{equation}\label{eqn:continuumElasticityReview:810}
\Bu = i \Bk \phi + i \Bk \cross \BH.
\end{equation}

We did nothing with neither the potential nor the phasor theory for solid displacement time evolution, and presumably won't on the exam either.

\section{Some wave types}

Some time was spent on qualitative descriptions and review of descriptions for solutions of the time evolution elasticity equations, despite the fact that we didn't actually attempt to find or analyze any of these solutions

\begin{itemize}
\item P-waves \cite{wiki:pwave}.  Irrotational, non volume preserving body wave.
\item S-waves \cite{wiki:swave}.  Divergence free body wave.  Shearing forces are present and volume is preserved (slower than S-waves)
\item Rayleigh wave \cite{wiki:rayleighwave}.  A surface wave that propagates near the surface of a body without penetrating into it.  It's pointed out in the class notes in the seismogram figure that these, while moving slower than the P (primary) or S (secondary) waves, have larger amplitude and are therefore the most destructive.
\item Love wave \cite{wiki:lovewave}.  A polarized shear surface wave with the shear displacements moving perpendicular to the direction of propagation.
\end{itemize}

For reasons that aren't clear both the midterm and last years final ask us to spew this sort of stuff (instead of actually trying to do something analytic associated with them).
