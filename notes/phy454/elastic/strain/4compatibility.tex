% 
% 
% 
% Copyright © 2012 Peeter Joot
% All Rights Reserved
% 
% This file may be reproduced and distributed in whole or in part, without fee, subject to the following conditions:
% 
% o The copyright notice above and this permission notice must be preserved complete on all complete or partial copies.
% 
% o Any translation or derived work must be approved by the author in writing before distribution.
% 
% o If you distribute this work in part, instructions for obtaining the complete version of this file must be included, and a means for obtaining a complete version provided.
% 
% 
% Exceptions to these rules may be granted for academic purposes: Write to the author and ask.
% 
% 
% 

\section{Compatibility condition for 2D strain}

\begin{equation}\label{eqn:continuumL6:50}
e_{ij} = 
\begin{bmatrix}
e_{11} & e_{12} \\
e_{21} & e_{22}
\end{bmatrix}
\end{equation}

%From \ref{eqn:continuumL6:10} we see that we have
From \ref{eqn:continuumElasticityReview:90} we see that we have

\begin{align}\label{eqn:continuumL6:70}
e_{11} &= \PD{x_1}{e_1} \\
e_{22} &= \PD{x_2}{e_2} \\
e_{12} = e_{21} &= 
\inv{2} \left( 
\PD{x_1}{e_2}
+ \PD{x_2}{e_1} 
\right).
\end{align}

We have a relationship between these displacements (called the compatibility relationship), which is

\begin{equation}\label{eqn:continuumL6:110}
\boxed{
\PDSq{x_2}{e_{11}} +
\PDSq{x_1}{e_{22}} = 
2
\frac{\partial^2 e_{12}}{\partial x_1 \partial x_2}.
}
\end{equation}

We find this by straight computation

\begin{align*}
\PDSq{x_2}{e_{11}} 
&= 
\PDSq{x_2}{}\left( 
\PD{x_1}{e_1}
\right) \\
&=
\frac{\partial^3 e_1}{\partial x_1 \partial x_2^2},
\end{align*}

and

\begin{align*}
\PDSq{x_1}{e_{22}} 
&= 
\PDSq{x_1}{}\left( 
\PD{x_2}{e_2}
\right) \\
&= 
\frac{\partial^3 e_2}{\partial x_2 \partial x_1^2},
\end{align*}

Now, looking at the cross term we find

\begin{align*}
2 \frac{\partial^2 e_{12}}{\partial x_1 \partial x_2} 
&= 
\frac{\partial^2 e_{12}}{\partial x_1 \partial x_2} 
\left(
\PD{x_1}{e_2}
+ \PD{x_2}{e_1} 
\right) \\
&=
\left(
\frac{\partial^3 e_1}{\partial x_1 \partial x_2^2} 
+
\frac{\partial^3 e_2}{\partial x_2 \partial x_1^2} 
\right)
\end{align*}

We've found an interrelationship between the components of the strain

\begin{equation}\label{eqn:continuumL6:129}
\boxed{
2 \frac{\partial^2 e_{12}}{\partial x_1 \partial x_2} 
=
\PDSq{x_1}{e_{22}} 
+\PDSq{x_2}{e_{11}}.
}
\end{equation}

This relationship is called the \textit{compatibility condition}, and ensures that we don't have a disjoint deformation of the form in figure (\ref{fig:continuumL6:continuumL6fig1}).

\imageFigure{figures/continuumL6fig1}{disjoint deformation illustrated}{fig:continuumL6:continuumL6fig1}{0.3}

I went looking for something to substantiate the claim that the compatibility condition \ref{eqn:continuumL6:129} is what is required to ensure a deformation maintained a coherent solid geometry.  I wasn't able to find any references to this compatibility condition in any of the texts I have, but found \citep{wiki:compatibilityMechanics}, \citep{wiki:infinitesimalStrainTheory}, and \citep{wiki:saintVenantCompat}.  It's not terribly surprising to see Christoffel symbol and differential forms references on those pages, since one can imagine that we'd wish to look at the mappings of all the points in the object as it undergoes the transformation from the original to the deformed state.

Even with just three points in a plane, say $\Ba$, $\Bb$, $\Bc$, the general deformation of an object doesn't seem like it's the easiest thing to describe.  We can imagine that these have trajectories in the deformation process $\Ba = \Ba(\alpha$, $\Bb = \Bb(\beta)$, $\Bc = \Bc(\gamma)$, with $\Ba', \Bb', \Bc'$ at the end points of the trajectories.  We'd want to look at displacement vectors $\Bu_a, \Bu_b, \Bu_c$ along each of these trajectories, and then see how they must be related.  Doing that carefully must result in this compatibility condition.

\section{Compatibility condition for 3D strain}

While we have 9 components in the tensor, not all of these are independent.  The sets above and below the diagonal can be related.
%, as illustrated in figure (\ref{fig:continuumL6:continuumL6fig2}).
%
%\imageFigure{figures/continuumL6fig2}{continuumL6fig2}{fig:continuumL6:continuumL6fig2}{0.2}

Here we have 6 relationships between the components of the strain tensor $e_{ij}$.

