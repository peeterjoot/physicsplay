\section{Compatibility condition.}

For a 2D strain tensor we found an interrelationship between the components of the strain tensor

\begin{equation}\label{eqn:continuumElasticityReview:510}
2 \frac{\partial^2 e_{12}}{\partial x_1 \partial x_2} 
=
\PDSq{x_1}{e_{22}} 
+\PDSq{x_2}{e_{11}},
\end{equation}

and called this the compatibility condition.  It was claimed, but not demonstrated that this is what is required to ensure a deformation maintained a coherent solid geometry.

I wasn't able to find any references to this compatibility condition in any of the texts I have, but found \cite{wiki:compatibilityMechanics}, \cite{wiki:infinitesimalStrainTheory}, and \cite{wiki:saintVenantCompat}.  It's not terribly surprising to see Christoffel symbol and differential forms references on those pages, since one can imagine that we'd wish to look at the mappings of all the points in the object as it undergoes the transformation from the original to the deformed state.

Even with just three points in a plane, say $\Ba$, $\Bb$, $\Bc$, the general deformation of an object doesn't seem like it's the easiest thing to describe.  We can imagine that these have trajectories in the deformation process $\Ba = \Ba(\alpha$, $\Bb = \Bb(\beta)$, $\Bc = \Bc(\gamma)$, with $\Ba', \Bb', \Bc'$ at the end points of the trajectories.  We'd want to look at displacement vectors $\Bu_a, \Bu_b, \Bu_c$ along each of these trajectories, and then see how they must be related.  Doing that carefully must result in this compatibility condition.
