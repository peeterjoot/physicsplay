% 
% 
% 
% Copyright � 2012 Peeter Joot
% All Rights Reserved
% 
% This file may be reproduced and distributed in whole or in part, without fee, subject to the following conditions:
% 
% o The copyright notice above and this permission notice must be preserved complete on all complete or partial copies.
% 
% o Any translation or derived work must be approved by the author in writing before distribution.
% 
% o If you distribute this work in part, instructions for obtaining the complete version of this file must be included, and a means for obtaining a complete version provided.
% 
% 
% Exceptions to these rules may be granted for academic purposes: Write to the author and ask.
% 
% 
% 

%\section{Strain tensor components.}
%\chapter{PHY454H1S\\Continuum Mechanics.  Lecture 4: Strain tensor components.  Taught by Prof. K. Das.}
\label{chap:continuumL4}
\blogpage{http://sites.google.com/site/peeterjoot2/math2012/continuumL4.pdf}
%\date{Jan 20, 2012}
\revisionInfo{continuumL4.tex}

\section{Stress tensor.}

Reading: Portions of this lecture cover \S 2 from the text \cite{landau1960theory}.

For the stress tensor

\begin{equation}\label{eqn:continuumL4:10}
\sigma_{ij},
\end{equation}

a second rank tensor, the first index $i$ defines the direction of the force, and the second index $j$ defines the surface.

Observe that the dimensions of $\sigma_{ij}$ is force per unit area, just like pressure.  We will in fact show that this tensor is akin to the pressure, and the diagonalized components of this tensor represent the pressure.

We've illustrated the stress tensor in a couple of 2D examples.  The first we call uniaxial stress, having just the $1,1$ element of the matrix as illustrated in figure (\ref{fig:continuumL4:continuumL4fig1})

\begin{figure}[htp]
   \centering
   \includegraphics[totalheight=0.2\textheight]{continuumL4fig1}
   \caption{Uniaxial stress}\label{fig:continuumL4:continuumL4fig1}
\end{figure}

\begin{equation}\label{eqn:continuumL4:30}
\sigma = 
\begin{bmatrix}
\sigma_{11} & 0 \\
0 & 0
\end{bmatrix}.
\end{equation}

A biaxial stress is illustrated in figure (\ref{fig:continuumL4:continuumL4fig2})
\begin{figure}[htp]
   \centering
   \includegraphics[totalheight=0.2\textheight]{continuumL4fig2}
   \caption{Biaxial stress.}\label{fig:continuumL4:continuumL4fig2}
\end{figure}

where for $\sigma_{11} \ne \sigma_{22}$ our tensor takes the form

\begin{equation}\label{eqn:continuumL4:50}
\sigma = 
\begin{bmatrix}
\sigma_{11} & 0 \\
0 & \sigma_{22}
\end{bmatrix}.
\end{equation}

In the general case we have

\begin{equation}\label{eqn:continuumL4:70}
\sigma = 
\begin{bmatrix}
\sigma_{11} & \sigma_{12} \\
\sigma_{21} & \sigma_{22}
\end{bmatrix}.
\end{equation}

We can attempt to illustrate this, but it becomes much harder to visualize as shown in figure (\ref{fig:continuumL4:continuumL4fig3})
\begin{figure}[htp]
   \centering
   \includegraphics[totalheight=0.2\textheight]{continuumL4fig3}
   \caption{General strain}\label{fig:continuumL4:continuumL4fig3}
\end{figure}

In equilibrium we must have

\begin{equation}\label{eqn:continuumL4:90}
\sigma_{12} = \sigma_{21}.
\end{equation}

We can use similar arguments to show that the stress tensor is symmetric.

In 3D we have three components of the stress tensor acting on each surface, as illustrated in figure (\ref{fig:continuumL4:continuumL4fig5})
\begin{figure}[htp]
   \centering
   \includegraphics[totalheight=0.2\textheight]{continuumL4fig5}
   \caption{Strain components on a 3D volume.}\label{fig:continuumL4:continuumL4fig5}
\end{figure}

We have three unique surface orientations and three components of the force for each of these, resulting in nine components, but these are not all independent.  For an object in equilibrium we must have $\sigma_{ij} = \sigma_{ji}$ (FIXME: justify?).  Explicitly, that is

\begin{align}\label{eqn:continuumL4:110}
\sigma_{12} &= \sigma_{21} \\
\sigma_{23} &= \sigma_{32} \\
\sigma_{31} &= \sigma_{13}
\end{align}

\subsection{Diagonalization}

We'll look at the two dimensional case in some detail, as in figure (\ref{fig:continuumL4:continuumL4fig6})

\begin{figure}[htp]
   \centering
   \includegraphics[totalheight=0.2\textheight]{continuumL4fig6}
   \caption{Area element under strain with and without rotation.}\label{fig:continuumL4:continuumL4fig6}
\end{figure}

Under this coordinate transformation, a rotation, the diagonal stress tensor is taken to a non-diagonal form

\begin{equation}\label{eqn:continuumL4:130}
\begin{bmatrix}
\sigma_{11} & 0 \\
0 & \sigma_{22} 
\end{bmatrix}
\leftrightarrow
\begin{bmatrix}
\sigma_{11}' & \sigma_{12}' \\
\sigma_{21}' & \sigma_{22}' 
\end{bmatrix}
\end{equation}

\subsection{How do the stress tensor and the force relate}

We form a Cauchy tetrahedron as in figure (\ref{fig:continuumL4:continuumL4fig7})
\begin{figure}[htp]
   \centering
   \includegraphics[totalheight=0.2\textheight]{continuumL4fig7}
   \caption{Cauchy tetrahedron}\label{fig:continuumL4:continuumL4fig7}
\end{figure}

\begin{equation}\label{eqn:continuumL4:150}
\Bf = \frac{\text{external force}}{\text{unit area}} = f_j \Be_j
\end{equation}

\begin{equation}\label{eqn:continuumL4:170}
\text{internal stress} = \text{external force}
\end{equation}

We write $\ncap$ in terms of the direction cosines

\begin{equation}\label{eqn:continuumL4:190}
\ncap = 
n_1 \Be_1 + 
n_2 \Be_2 + 
n_3 \Be_3 
\end{equation}

Here 

\begin{align}\label{eqn:continuumL4:210}
n_1 &= \ncap \cdot \Be_1 \\
n_2 &= \ncap \cdot \Be_2 \\
n_3 &= \ncap \cdot \Be_3,
\end{align}

or 

\begin{equation}\label{eqn:continuumL4:230}
n_j = \ncap \cdot \Be_j = \cos\phi_j
\end{equation}

Force balance on $x_1$ direction, matching total external force in this direction to the total internal force ($\sigma_{ij}'s$) as follows

\begin{equation}\label{eqn:continuumL4:250}
\begin{aligned}
f_1 \times \text{area ABC} 
&= 
\sigma_{11} \times \text{area BOC} \\
&+\sigma_{12} \times \text{area AOC} \\
&+\sigma_{13} \times \text{area AOB}
\end{aligned}
\end{equation}

Similarly

\begin{equation}\label{eqn:continuumL4:270}
\begin{aligned}
f_2 \times \text{area ABC} 
&= 
\sigma_{21} \times \text{area BOC} \\
&+\sigma_{22} \times \text{area AOC} \\
&+\sigma_{23} \times \text{area AOB},
\end{aligned}
\end{equation}

and

\begin{equation}\label{eqn:continuumL4:290}
\begin{aligned}
f_3 \times \text{area ABC} 
&= 
\sigma_{31} \times \text{area BOC} \\
&+\sigma_{32} \times \text{area AOC} \\
&+\sigma_{33} \times \text{area AOB},
\end{aligned}
\end{equation}

We can therefore write these force components like

\begin{equation}\label{eqn:continuumL4:310}
f_1 = 
\sigma_{11} \frac{BOC}{ABC} + 
\sigma_{12} \frac{AOC}{ABC} + 
\sigma_{13} \frac{AOB}{ABC} 
\end{equation}

but these ratios are really just the projections of the areas as illustrated in figure (\ref{fig:continuumL4:continuumL4fig8})

\begin{figure}[htp]
   \centering
   \includegraphics[totalheight=0.2\textheight]{continuumL4fig8}
   \caption{Area projection.}\label{fig:continuumL4:continuumL4fig8}
\end{figure}

where an arbitrary surface with area $\Delta S$ can be decomposed into projections

\begin{equation}\label{eqn:continuumL4:330}
\Delta S \cos\phi_j,
\end{equation}

utilizing the direction cosines.  We can therefore write

\begin{align}\label{eqn:continuumL4:350}
f_1 &= \sigma_{11} n_1 + \sigma_{12} n_2 + \sigma_{13} n_3 \\
f_2 &= \sigma_{21} n_1 + \sigma_{22} n_2 + \sigma_{23} n_3 \\
f_3 &= \sigma_{31} n_1 + \sigma_{32} n_2 + \sigma_{33} n_3,
\end{align}

or in matrix notation

\begin{equation}\label{eqn:continuumL4:370}
\begin{bmatrix}
f_1  \\
f_2  \\
f_3 
\end{bmatrix}
=
\begin{bmatrix}
\sigma_{11} & \sigma_{12} & \sigma_{13} \\
\sigma_{21} & \sigma_{22} & \sigma_{23} \\
\sigma_{31} & \sigma_{32} & \sigma_{33} 
\end{bmatrix}
\begin{bmatrix}
n_1 \\
n_2 \\
n_3 \\
\end{bmatrix}.
\end{equation}

This is just 

\begin{equation}\label{eqn:continuumL4:390}
\boxed{
f_i = \sigma_{ij} n_j.
}
\end{equation}

This force with components $f_i$ is also called the traction vector

\begin{equation}\label{eqn:continuumL4:410}
T_i = \sigma_{ij} n_j.
\end{equation}
