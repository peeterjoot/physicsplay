% 
% 
% 
% Copyright © 2012 Peeter Joot
% All Rights Reserved
% 
% This file may be reproduced and distributed in whole or in part, without fee, subject to the following conditions:
% 
% o The copyright notice above and this permission notice must be preserved complete on all complete or partial copies.
% 
% o Any translation or derived work must be approved by the author in writing before distribution.
% 
% o If you distribute this work in part, instructions for obtaining the complete version of this file must be included, and a means for obtaining a complete version provided.
% 
% 
% Exceptions to these rules may be granted for academic purposes: Write to the author and ask.
% 
% 
% 

\section{Diagonal strain representation.}

In a basis for which the strain tensor is diagonal, it was pointed out that we can write our difference in squared displacement as (for $k = 1, 2, 3$, no summation convention)

\begin{equation}\label{eqn:continuumElasticityReview:130}
(dx_k')^2 - (dx_k)^2 = 2 e_{kk} dx_k dx_k
\end{equation}

from which we can rearrange, take roots, and apply a first order Taylor expansion to find (again no summation convention)

\begin{equation}\label{eqn:continuumElasticityReview:150}
dx_k' \approx (1 + e_{kk}) dx_k.
\end{equation}

An approximation of the displaced volume was then found in terms of the strain tensor trace (summation convention back again)

\begin{equation}\label{eqn:continuumElasticityReview:170}
dV' \approx (1 + e_{kk}) dV,
\end{equation}

allowing us to identify this trace as a relative difference in displaced volume

\begin{equation}\label{eqn:continuumElasticityReview:190}
e_{kk} \approx \frac{dV' - dV}{dV}.
\end{equation}

