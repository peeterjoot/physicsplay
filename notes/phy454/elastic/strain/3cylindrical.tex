% 
% 
% 
% Copyright © 2012 Peeter Joot
% All Rights Reserved
% 
% This file may be reproduced and distributed in whole or in part, without fee, subject to the following conditions:
% 
% o The copyright notice above and this permission notice must be preserved complete on all complete or partial copies.
% 
% o Any translation or derived work must be approved by the author in writing before distribution.
% 
% o If you distribute this work in part, instructions for obtaining the complete version of this file must be included, and a means for obtaining a complete version provided.
% 
% 
% Exceptions to these rules may be granted for academic purposes: Write to the author and ask.
% 
% 
% 

\section{Strain in cylindrical coordinates}

Useful in many practice problems are the cylindrical coordinate representation of the strain tensor 

\begin{align}\label{eqn:continuumElasticityReview:210}
2 e_{rr} &= \PD{r}{u_r}  \\
2 e_{\phi\phi} &= \inv{r} \PD{\phi}{u_\phi} +\inv{r} u_r  \\
2 e_{zz} &= \PD{z}{u_z}  \\
2 e_{zr} &= \PD{z}{u_r} + \PD{r}{u_z} \\
2 e_{r\phi} &= \PD{r}{u_\phi} - \inv{r} u_\phi + \inv{r} \PD{\phi}{u_r} \\
2 e_{\phi z} &= \PD{z}{u_\phi} +\inv{r} \PD{\phi}{u_z}.
\end{align}

This result can be found in \citep{landau1960theory}, and is derived in appendix \ref{chap:appendix:strainCoordinates} using the second order methods found above for the Cartesian tensor.

An easier way to do this derivation (and understand what the coordinates represent) follows from the relation found in \S 6 of \citep{acheson1990elementary}

\begin{equation}\label{eqn:continuumElasticityReview:230}
2 \Be_i e_{ij} \Be_j \cdot \ncap = 2 (\ncap \cdot \spacegrad) \Bu + \ncap \cross (\spacegrad \cross \Bu),
\end{equation}

where $\ncap$ is the normal to the surface at which we are measuring a force applied to the solid (our Cauchy tetrahedron).

The cylindrical tensor coordinates of \ref{eqn:continuumElasticityReview:210} follow from 
\ref{eqn:continuumElasticityReview:230} nicely taking $\ncap = \rcap, \phicap, \zcap$ in turn.  This derivation can be found in appendix \ref{chap:continuumstressTensorVectorForm}.
