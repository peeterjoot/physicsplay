% 
% 
% 
% Copyright © 2012 Peeter Joot
% All Rights Reserved
% 
% This file may be reproduced and distributed in whole or in part, without fee, subject to the following conditions:
% 
% o The copyright notice above and this permission notice must be preserved complete on all complete or partial copies.
% 
% o Any translation or derived work must be approved by the author in writing before distribution.
% 
% o If you distribute this work in part, instructions for obtaining the complete version of this file must be included, and a means for obtaining a complete version provided.
% 
% 
% Exceptions to these rules may be granted for academic purposes: Write to the author and ask.
% 
% 
% 


\section{Matrix representation}

The strain tensor $e_{ik}$ can be worked with in coordinates, but we will often us a matrix representation when working in Cartesian coordinates 

\begin{equation}\label{eqn:continuumL2:210}
\Be = 
\begin{bmatrix}
e_{11} & e_{12} & e_{13} \\
e_{21} & e_{22} & e_{23} \\
e_{31} & e_{32} & e_{33} \\
\end{bmatrix}
\end{equation}

We see from \ref{eqn:continuumL2:190} that $e_{ik}$ is symmetric, so we have

\begin{align}\label{eqn:continuumL2:230}
e_{21} &= e_{12} \\
e_{31} &= e_{13} \\
e_{32} &= e_{23}
\end{align}

Because any real symmetric matrix can be diagonalized we can write in some coordinate system

\begin{equation}\label{eqn:continuumL2:250}
\bar{e}_{ik} =
\begin{bmatrix}
\bar{e}_{11} & 0 & 0 \\
0 & \bar{e}_{22} & 0 \\
0 & 0 & \bar{e}_{33} \\
\end{bmatrix}
\end{equation}

\begin{align}\label{eqn:continuumL2:270}
{dx_1'}^2 &= (1 + 2 \bar{e}_{11}) dx_1^2 \\
{dx_2'}^2 &= (1 + 2 \bar{e}_{22}) dx_2^2 \\
{dx_3'}^2 &= (1 + 2 \bar{e}_{33}) dx_3^2
\end{align}

If our changes are small enough we can also write approximately, taking the first order term in the square root evaluation

\begin{align}\label{eqn:continuumL2:290}
dx_1' &\approx (1 + \bar{e}_{11}) dx_1 \\
dx_2' &\approx (1 + \bar{e}_{22}) dx_2 \\
dx_3' &\approx (1 + \bar{e}_{33}) dx_3
\end{align}

%FIXME: some hand waving here to think through.
We are also free to define a volume element

\begin{equation}\label{eqn:continuumL2:310}
dV' =
dx_1'
dx_2'
dx_3'
\approx
(1 + e_{11})
(1 + e_{22})
(1 + e_{33})
dx_1 dx_2 dx_3
\end{equation}

\begin{equation}\label{eqn:continuumL2:330}
dV' = (1 + e_{11} +e_{22} +e_{33} ) dV
\end{equation}

So the change of volume is given by the trace

\begin{equation}\label{eqn:continuumL2:350}
dV' = ( 1 + e_{ii} ) dV
\end{equation}
