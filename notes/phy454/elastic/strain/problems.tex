% 
% 
% 
% Copyright © 2012 Peeter Joot
% All Rights Reserved
% 
% This file may be reproduced and distributed in whole or in part, without fee, subject to the following conditions:
% 
% o The copyright notice above and this permission notice must be preserved complete on all complete or partial copies.
% 
% o Any translation or derived work must be approved by the author in writing before distribution.
% 
% o If you distribute this work in part, instructions for obtaining the complete version of this file must be included, and a means for obtaining a complete version provided.
% 
% 
% Exceptions to these rules may be granted for academic purposes: Write to the author and ask.
% 
% 
% 
\section{Problems}

\label{problem:strain:ps1q2a}
\subsection{Strain tensor for a small displacement.}

Small displacement field in a material is given by

\begin{align}\label{eqn:continuumProblemSet1:30}
e_1 &= 2 x_1 x_2 \\
e_2 &= x_3^2 \\
e_3 &= x_1^2 - x_3
\end{align}

Find

\begin{enumerate}
\item The infinitesimal strain tensor $e_{ij}$,
\item The principal strains and the corresponding principal axes at $(x_1, x_2, x_3) = (1, 2, 4)$.
\end{enumerate}

Solution: \ref{solutions:ps1q2a}.

\label{problem:strain:todo}
\subsection{Computing stretch in any given direction.}

How do we use the strain tensor?  Strain is the measure of stretching, so given a strain tensor, we should be able to compute the stretch in any given direction.
%\imageFigure{figures/continuumL3fig1}{Stretched line elements.}{fig:continuumL3:continuumL3fig1}{0.2}

FIXME: find or create problem and try this.  Related to the following question created while reviewing for the exam:

In \cite{feynman1963flp:elasticMaterials} it is pointed out that this strain tensor simply relates the displacement vector coordinates $u_i$ to the coordinates at the point at which it is measured

\begin{equation}\label{eqn:continuumElasticityReview:110}
u_i = e_{ij} x_j.
\end{equation}

When we get to fluid dynamics we perform a linear expansion of $du_i$ and find something similar

\begin{equation}\label{eqn:continuumElasticityReview:530}
dx_i' - dx_i = du_i = \PD{x_k}{u_i} dx_k = e_{ij} dx_k + \omega_{ij} dx_k
\end{equation}

where

\begin{equation}\label{eqn:continuumElasticityReview:550}
\omega_{ij} = \inv{2} \left( \PD{x_i}{u_j} +\PD{x_j}{u_i} \right).
\end{equation}

Except for the antisymmetric term, note the structural similarity of \ref{eqn:continuumElasticityReview:110} and \ref{eqn:continuumElasticityReview:530}.  Why is it that we neglect the vorticity tensor in statics?
%  If we are approximating the displacement, it appears to have a natural place in things, as we can see
%
%\begin{align*}
%u_i 
%&\approx \PD{x_j}{u_i} x_j \\
%&=
%e_{ij} x_j + \omega_{ik} x_k.
%\end{align*}
%
%This is easily seen to be the case, recovering \ref{eqn:continuumElasticityReview:90} by taking derivatives of \ref{eqn:continuumElasticityReview:110}, plus an assumption that $e_{ij}$ is symmetric.

