% 
% 
% 
% Copyright � 2012 Peeter Joot
% All Rights Reserved
% 
% This file may be reproduced and distributed in whole or in part, without fee, subject to the following conditions:
% 
% o The copyright notice above and this permission notice must be preserved complete on all complete or partial copies.
% 
% o Any translation or derived work must be approved by the author in writing before distribution.
% 
% o If you distribute this work in part, instructions for obtaining the complete version of this file must be included, and a means for obtaining a complete version provided.
% 
% 
% Exceptions to these rules may be granted for academic purposes: Write to the author and ask.
% 
% 
% 
%%
% Copyright � 2015 Peeter Joot.  All Rights Reserved.
% Licenced as described in the file LICENSE under the root directory of this GIT repository.
%
\documentclass[]{eliblog}

\usepackage{amsmath}
\usepackage{mathpazo}

%
% shorthand for bold symbols, convenient for vectors and matrices
%
\newcommand{\Ba}[0]{\mathbf{a}}
\newcommand{\Bb}[0]{\mathbf{b}}
\newcommand{\Bc}[0]{\mathbf{c}}
\newcommand{\Bd}[0]{\mathbf{d}}
\newcommand{\Be}[0]{\mathbf{e}}
\newcommand{\Bf}[0]{\mathbf{f}}
\newcommand{\Bg}[0]{\mathbf{g}}
\newcommand{\Bh}[0]{\mathbf{h}}
\newcommand{\Bi}[0]{\mathbf{i}}
\newcommand{\Bj}[0]{\mathbf{j}}
\newcommand{\Bk}[0]{\mathbf{k}}
\newcommand{\Bl}[0]{\mathbf{l}}
\newcommand{\Bm}[0]{\mathbf{m}}
\newcommand{\Bn}[0]{\mathbf{n}}
\newcommand{\Bo}[0]{\mathbf{o}}
\newcommand{\Bp}[0]{\mathbf{p}}
\newcommand{\Bq}[0]{\mathbf{q}}
\newcommand{\Br}[0]{\mathbf{r}}
\newcommand{\Bs}[0]{\mathbf{s}}
\newcommand{\Bt}[0]{\mathbf{t}}
\newcommand{\Bu}[0]{\mathbf{u}}
\newcommand{\Bv}[0]{\mathbf{v}}
\newcommand{\Bw}[0]{\mathbf{w}}
\newcommand{\Bx}[0]{\mathbf{x}}
\newcommand{\By}[0]{\mathbf{y}}
\newcommand{\Bz}[0]{\mathbf{z}}
\newcommand{\BA}[0]{\mathbf{A}}
\newcommand{\BB}[0]{\mathbf{B}}
\newcommand{\BC}[0]{\mathbf{C}}
\newcommand{\BD}[0]{\mathbf{D}}
\newcommand{\BE}[0]{\mathbf{E}}
\newcommand{\BF}[0]{\mathbf{F}}
\newcommand{\BG}[0]{\mathbf{G}}
\newcommand{\BH}[0]{\mathbf{H}}
\newcommand{\BI}[0]{\mathbf{I}}
\newcommand{\BJ}[0]{\mathbf{J}}
\newcommand{\BK}[0]{\mathbf{K}}
\newcommand{\BL}[0]{\mathbf{L}}
\newcommand{\BM}[0]{\mathbf{M}}
\newcommand{\BN}[0]{\mathbf{N}}
\newcommand{\BO}[0]{\mathbf{O}}
\newcommand{\BP}[0]{\mathbf{P}}
\newcommand{\BQ}[0]{\mathbf{Q}}
\newcommand{\BR}[0]{\mathbf{R}}
\newcommand{\BS}[0]{\mathbf{S}}
\newcommand{\BT}[0]{\mathbf{T}}
\newcommand{\BU}[0]{\mathbf{U}}
\newcommand{\BV}[0]{\mathbf{V}}
\newcommand{\BW}[0]{\mathbf{W}}
\newcommand{\BX}[0]{\mathbf{X}}
\newcommand{\BY}[0]{\mathbf{Y}}
\newcommand{\BZ}[0]{\mathbf{Z}}

\newcommand{\Bzero}[0]{\mathbf{0}}
\newcommand{\Btheta}[0]{\boldsymbol{\theta}}
\newcommand{\Btau}[0]{\boldsymbol{\tau}}
\newcommand{\Bomega}[0]{\boldsymbol{\omega}}

%
% shorthand for unit vectors
%
\newcommand{\acap}[0]{\hat{\Ba}}
\newcommand{\bcap}[0]{\hat{\Bb}}
\newcommand{\ccap}[0]{\hat{\Bc}}
\newcommand{\dcap}[0]{\hat{\Bd}}
\newcommand{\ecap}[0]{\hat{\Be}}
\newcommand{\fcap}[0]{\hat{\Bf}}
\newcommand{\gcap}[0]{\hat{\Bg}}
\newcommand{\hcap}[0]{\hat{\Bh}}
\newcommand{\icap}[0]{\hat{\Bi}}
\newcommand{\jcap}[0]{\hat{\Bj}}
\newcommand{\kcap}[0]{\hat{\Bk}}
\newcommand{\lcap}[0]{\hat{\Bl}}
\newcommand{\mcap}[0]{\hat{\Bm}}
\newcommand{\ncap}[0]{\hat{\Bn}}
\newcommand{\ocap}[0]{\hat{\Bo}}
\newcommand{\pcap}[0]{\hat{\Bp}}
\newcommand{\qcap}[0]{\hat{\Bq}}
\newcommand{\rcap}[0]{\hat{\Br}}
\newcommand{\scap}[0]{\hat{\Bs}}
\newcommand{\tcap}[0]{\hat{\Bt}}
\newcommand{\ucap}[0]{\hat{\Bu}}
\newcommand{\vcap}[0]{\hat{\Bv}}
\newcommand{\wcap}[0]{\hat{\Bw}}
\newcommand{\xcap}[0]{\hat{\Bx}}
\newcommand{\ycap}[0]{\hat{\By}}
\newcommand{\zcap}[0]{\hat{\Bz}}
\newcommand{\thetacap}[0]{\hat{\Btheta}}

%
% to write R^n and C^n in a distinguishable fashion.  Perhaps change this
% to the double lined characters upon figuring out how to do so.
%
\newcommand{\C}[1]{$\mathbb{C}^{#1}$}
\newcommand{\R}[1]{$\mathbb{R}^{#1}$}

%
% various generally useful helpers
%

% derivative of #1 wrt. #2:
\newcommand{\D}[2] {\frac {d#2} {d#1}}

\newcommand{\inv}[1]{\frac{1}{#1}}
\newcommand{\cross}[0]{\times}

\newcommand{\abs}[1]{\lvert{#1}\rvert}
\newcommand{\norm}[1]{\lVert{#1}\rVert}
\newcommand{\innerprod}[2]{\langle{#1}, {#2}\rangle}
\newcommand{\dotprod}[2]{{#1} \cdot {#2}}
\newcommand{\bdotprod}[2]{\left({#1} \cdot {#2}\right)}
\newcommand{\crossprod}[2]{{#1} \cross {#2}}
\newcommand{\tripleprod}[3]{\dotprod{\left(\crossprod{#1}{#2}\right)}{#3}}

\DeclareMathOperator{\Proj}{Proj}
\DeclareMathOperator{\Span}{span}
\DeclareMathOperator{\Sgn}{sgn}
\DeclareMathOperator{\Area}{Area}
\DeclareMathOperator{\Volume}{Volume}

%
% A few miscellaneous things specific to this document
%
\newcommand{\crossop}[1]{\crossprod{#1}{}}

% R2 vector.
\newcommand{\VectorTwo}[2]{
\begin{bmatrix}
 {#1} \\
 {#2}
\end{bmatrix}
}

\newcommand{\VectorN}[1]{
\begin{bmatrix}
{#1}_1 \\
{#1}_2 \\
\vdots \\
{#1}_N \\
\end{bmatrix}
}

\newcommand{\DETuvij}[4]{
\begin{vmatrix}
 {#1}_{#3} & {#1}_{#4} \\
 {#2}_{#3} & {#2}_{#4}
\end{vmatrix}
}

\newcommand{\DETuvwijk}[6]{
\begin{vmatrix}
 {#1}_{#4} & {#1}_{#5} & {#1}_{#6} \\
 {#2}_{#4} & {#2}_{#5} & {#2}_{#6} \\
 {#3}_{#4} & {#3}_{#5} & {#3}_{#6}
\end{vmatrix}
}

\newcommand{\DETuvwxijkl}[8]{
\begin{vmatrix}
 {#1}_{#5} & {#1}_{#6} & {#1}_{#7} & {#1}_{#8} \\
 {#2}_{#5} & {#2}_{#6} & {#2}_{#7} & {#2}_{#8} \\
 {#3}_{#5} & {#3}_{#6} & {#3}_{#7} & {#3}_{#8} \\
 {#4}_{#5} & {#4}_{#6} & {#4}_{#7} & {#4}_{#8} \\
\end{vmatrix}
}

%\newcommand{\DETuvwxyijklm}[10]{
%\begin{vmatrix}
% {#1}_{#6} & {#1}_{#7} & {#1}_{#8} & {#1}_{#9} & {#1}_{#10} \\
% {#2}_{#6} & {#2}_{#7} & {#2}_{#8} & {#2}_{#9} & {#2}_{#10} \\
% {#3}_{#6} & {#3}_{#7} & {#3}_{#8} & {#3}_{#9} & {#3}_{#10} \\
% {#4}_{#6} & {#4}_{#7} & {#4}_{#8} & {#4}_{#9} & {#4}_{#10} \\
% {#5}_{#6} & {#5}_{#7} & {#5}_{#8} & {#5}_{#9} & {#5}_{#10}
%\end{vmatrix}
%}

% R3 vector.
\newcommand{\VectorThree}[3]{
\begin{bmatrix}
 {#1} \\
 {#2} \\
 {#3}
\end{bmatrix}
}



\author{Peeter Joot}
\email{peeter.joot@gmail.com}

%\documentclass[]{eliblogwidescreen}

\usepackage{amsmath}
\usepackage{mathpazo}

%
% shorthand for bold symbols, convenient for vectors and matrices
%
\newcommand{\Ba}[0]{\mathbf{a}}
\newcommand{\Bb}[0]{\mathbf{b}}
\newcommand{\Bc}[0]{\mathbf{c}}
\newcommand{\Bd}[0]{\mathbf{d}}
\newcommand{\Be}[0]{\mathbf{e}}
\newcommand{\Bf}[0]{\mathbf{f}}
\newcommand{\Bg}[0]{\mathbf{g}}
\newcommand{\Bh}[0]{\mathbf{h}}
\newcommand{\Bi}[0]{\mathbf{i}}
\newcommand{\Bj}[0]{\mathbf{j}}
\newcommand{\Bk}[0]{\mathbf{k}}
\newcommand{\Bl}[0]{\mathbf{l}}
\newcommand{\Bm}[0]{\mathbf{m}}
\newcommand{\Bn}[0]{\mathbf{n}}
\newcommand{\Bo}[0]{\mathbf{o}}
\newcommand{\Bp}[0]{\mathbf{p}}
\newcommand{\Bq}[0]{\mathbf{q}}
\newcommand{\Br}[0]{\mathbf{r}}
\newcommand{\Bs}[0]{\mathbf{s}}
\newcommand{\Bt}[0]{\mathbf{t}}
\newcommand{\Bu}[0]{\mathbf{u}}
\newcommand{\Bv}[0]{\mathbf{v}}
\newcommand{\Bw}[0]{\mathbf{w}}
\newcommand{\Bx}[0]{\mathbf{x}}
\newcommand{\By}[0]{\mathbf{y}}
\newcommand{\Bz}[0]{\mathbf{z}}
\newcommand{\BA}[0]{\mathbf{A}}
\newcommand{\BB}[0]{\mathbf{B}}
\newcommand{\BC}[0]{\mathbf{C}}
\newcommand{\BD}[0]{\mathbf{D}}
\newcommand{\BE}[0]{\mathbf{E}}
\newcommand{\BF}[0]{\mathbf{F}}
\newcommand{\BG}[0]{\mathbf{G}}
\newcommand{\BH}[0]{\mathbf{H}}
\newcommand{\BI}[0]{\mathbf{I}}
\newcommand{\BJ}[0]{\mathbf{J}}
\newcommand{\BK}[0]{\mathbf{K}}
\newcommand{\BL}[0]{\mathbf{L}}
\newcommand{\BM}[0]{\mathbf{M}}
\newcommand{\BN}[0]{\mathbf{N}}
\newcommand{\BO}[0]{\mathbf{O}}
\newcommand{\BP}[0]{\mathbf{P}}
\newcommand{\BQ}[0]{\mathbf{Q}}
\newcommand{\BR}[0]{\mathbf{R}}
\newcommand{\BS}[0]{\mathbf{S}}
\newcommand{\BT}[0]{\mathbf{T}}
\newcommand{\BU}[0]{\mathbf{U}}
\newcommand{\BV}[0]{\mathbf{V}}
\newcommand{\BW}[0]{\mathbf{W}}
\newcommand{\BX}[0]{\mathbf{X}}
\newcommand{\BY}[0]{\mathbf{Y}}
\newcommand{\BZ}[0]{\mathbf{Z}}

\newcommand{\Bzero}[0]{\mathbf{0}}
\newcommand{\Btheta}[0]{\boldsymbol{\theta}}
\newcommand{\Btau}[0]{\boldsymbol{\tau}}
\newcommand{\Bomega}[0]{\boldsymbol{\omega}}

%
% shorthand for unit vectors
%
\newcommand{\acap}[0]{\hat{\Ba}}
\newcommand{\bcap}[0]{\hat{\Bb}}
\newcommand{\ccap}[0]{\hat{\Bc}}
\newcommand{\dcap}[0]{\hat{\Bd}}
\newcommand{\ecap}[0]{\hat{\Be}}
\newcommand{\fcap}[0]{\hat{\Bf}}
\newcommand{\gcap}[0]{\hat{\Bg}}
\newcommand{\hcap}[0]{\hat{\Bh}}
\newcommand{\icap}[0]{\hat{\Bi}}
\newcommand{\jcap}[0]{\hat{\Bj}}
\newcommand{\kcap}[0]{\hat{\Bk}}
\newcommand{\lcap}[0]{\hat{\Bl}}
\newcommand{\mcap}[0]{\hat{\Bm}}
\newcommand{\ncap}[0]{\hat{\Bn}}
\newcommand{\ocap}[0]{\hat{\Bo}}
\newcommand{\pcap}[0]{\hat{\Bp}}
\newcommand{\qcap}[0]{\hat{\Bq}}
\newcommand{\rcap}[0]{\hat{\Br}}
\newcommand{\scap}[0]{\hat{\Bs}}
\newcommand{\tcap}[0]{\hat{\Bt}}
\newcommand{\ucap}[0]{\hat{\Bu}}
\newcommand{\vcap}[0]{\hat{\Bv}}
\newcommand{\wcap}[0]{\hat{\Bw}}
\newcommand{\xcap}[0]{\hat{\Bx}}
\newcommand{\ycap}[0]{\hat{\By}}
\newcommand{\zcap}[0]{\hat{\Bz}}
\newcommand{\thetacap}[0]{\hat{\Btheta}}

%
% to write R^n and C^n in a distinguishable fashion.  Perhaps change this
% to the double lined characters upon figuring out how to do so.
%
\newcommand{\C}[1]{$\mathbb{C}^{#1}$}
\newcommand{\R}[1]{$\mathbb{R}^{#1}$}

%
% various generally useful helpers
%

% derivative of #1 wrt. #2:
\newcommand{\D}[2] {\frac {d#2} {d#1}}

\newcommand{\inv}[1]{\frac{1}{#1}}
\newcommand{\cross}[0]{\times}

\newcommand{\abs}[1]{\lvert{#1}\rvert}
\newcommand{\norm}[1]{\lVert{#1}\rVert}
\newcommand{\innerprod}[2]{\langle{#1}, {#2}\rangle}
\newcommand{\dotprod}[2]{{#1} \cdot {#2}}
\newcommand{\bdotprod}[2]{\left({#1} \cdot {#2}\right)}
\newcommand{\crossprod}[2]{{#1} \cross {#2}}
\newcommand{\tripleprod}[3]{\dotprod{\left(\crossprod{#1}{#2}\right)}{#3}}

\DeclareMathOperator{\Proj}{Proj}
\DeclareMathOperator{\Span}{span}
\DeclareMathOperator{\Sgn}{sgn}
\DeclareMathOperator{\Area}{Area}
\DeclareMathOperator{\Volume}{Volume}

%
% A few miscellaneous things specific to this document
%
\newcommand{\crossop}[1]{\crossprod{#1}{}}

% R2 vector.
\newcommand{\VectorTwo}[2]{
\begin{bmatrix}
 {#1} \\
 {#2}
\end{bmatrix}
}

\newcommand{\VectorN}[1]{
\begin{bmatrix}
{#1}_1 \\
{#1}_2 \\
\vdots \\
{#1}_N \\
\end{bmatrix}
}

\newcommand{\DETuvij}[4]{
\begin{vmatrix}
 {#1}_{#3} & {#1}_{#4} \\
 {#2}_{#3} & {#2}_{#4}
\end{vmatrix}
}

\newcommand{\DETuvwijk}[6]{
\begin{vmatrix}
 {#1}_{#4} & {#1}_{#5} & {#1}_{#6} \\
 {#2}_{#4} & {#2}_{#5} & {#2}_{#6} \\
 {#3}_{#4} & {#3}_{#5} & {#3}_{#6}
\end{vmatrix}
}

\newcommand{\DETuvwxijkl}[8]{
\begin{vmatrix}
 {#1}_{#5} & {#1}_{#6} & {#1}_{#7} & {#1}_{#8} \\
 {#2}_{#5} & {#2}_{#6} & {#2}_{#7} & {#2}_{#8} \\
 {#3}_{#5} & {#3}_{#6} & {#3}_{#7} & {#3}_{#8} \\
 {#4}_{#5} & {#4}_{#6} & {#4}_{#7} & {#4}_{#8} \\
\end{vmatrix}
}

%\newcommand{\DETuvwxyijklm}[10]{
%\begin{vmatrix}
% {#1}_{#6} & {#1}_{#7} & {#1}_{#8} & {#1}_{#9} & {#1}_{#10} \\
% {#2}_{#6} & {#2}_{#7} & {#2}_{#8} & {#2}_{#9} & {#2}_{#10} \\
% {#3}_{#6} & {#3}_{#7} & {#3}_{#8} & {#3}_{#9} & {#3}_{#10} \\
% {#4}_{#6} & {#4}_{#7} & {#4}_{#8} & {#4}_{#9} & {#4}_{#10} \\
% {#5}_{#6} & {#5}_{#7} & {#5}_{#8} & {#5}_{#9} & {#5}_{#10}
%\end{vmatrix}
%}

% R3 vector.
\newcommand{\VectorThree}[3]{
\begin{bmatrix}
 {#1} \\
 {#2} \\
 {#3}
\end{bmatrix}
}



\author{Peeter Joot}
\email{peeter.joot@gmail.com}


%\chapter{PHY454H1S\\Continuum Mechanics.  Lecture 6: Compatibility condition and elastostatics.  Taught by Prof. K. Das.}
\section{Compatibility condition and elastostatics.}
\label{chap:continuumL6}
%\useCCL
\blogpage{http://sites.google.com/site/peeterjoot2/math2012/continuumL6.pdf}
%\date{Jan 27, 2012}
\revisionInfo{continuumL6.tex}

\beginArtWithToc
%\beginArtNoToc

%\section{Disclaimer.}
%
%Peeter's lecture notes from class.  May not be entirely coherent.

\section{Review: Elastostatics}

We've defined the strain tensor, where assuming the second order terms are ignored, was

\begin{equation}\label{eqn:continuumL6:10}
e_{ij} = 
\inv{2} \left( 
\PD{x_j}{e_i}
+ \PD{x_i}{e_j} \right).
\end{equation}

We've also defined a stress tensor defined implicitly as a divergence relationship using the force per unit volume $F_i$ in direction $i$

\begin{equation}\label{eqn:continuumL6:30}
\sigma_{ij} \leftrightarrow F_i = \PD{x_j}{\sigma_{ij}}.
\end{equation}

We've also discussed the constitutive relation, relating stress $\sigma_{ij}$ and strain $e_{ij}$.

We've also discussed linear constitutive relationships (Hooke's law).  

\section{2D strain.}

\begin{equation}\label{eqn:continuumL6:50}
e_{ij} = 
\begin{bmatrix}
e_{11} & e_{12} \\
e_{21} & e_{22}
\end{bmatrix}
\end{equation}

From \ref{eqn:continuumL6:10} we see that we have

\begin{align}\label{eqn:continuumL6:70}
e_{11} &= \PD{x_1}{e_1} \\
e_{22} &= \PD{x_2}{e_2} \\
e_{12} = e_{21} &= 
\inv{2} \left( 
\PD{x_1}{e_2}
+ \PD{x_2}{e_1} 
\right).
\end{align}

We have a relationship between these displacements (called the compatibility relationship), which is

\begin{equation}\label{eqn:continuumL6:110}
\boxed{
\PDSq{x_2}{e_{11}} +
\PDSq{x_1}{e_{22}} = 
2
\frac{\partial^2 e_{12}}{\partial x_1 \partial x_2}.
}
\end{equation}

We find this by straight computation

\begin{align*}
\PDSq{x_2}{e_{11}} 
&= 
\PDSq{x_2}{}\left( 
\PD{x_1}{e_1}
\right) \\
&=
\frac{\partial^3 e_1}{\partial x_1 \partial x_2^2},
\end{align*}

and

\begin{align*}
\PDSq{x_1}{e_{22}} 
&= 
\PDSq{x_1}{}\left( 
\PD{x_2}{e_2}
\right) \\
&= 
\frac{\partial^3 e_2}{\partial x_2 \partial x_1^2},
\end{align*}

Now, looking at the cross term we find

\begin{align*}
2 \frac{\partial^2 e_{12}}{\partial x_1 \partial x_2} 
&= 
\frac{\partial^2 e_{12}}{\partial x_1 \partial x_2} 
\left(
\PD{x_1}{e_2}
+ \PD{x_2}{e_1} 
\right) \\
&=
\left(
\frac{\partial^3 e_1}{\partial x_1 \partial x_2^2} 
+
\frac{\partial^3 e_2}{\partial x_2 \partial x_1^2} 
\right)
\end{align*}

We've found an interrelationship between the components of the strain

\begin{equation}\label{eqn:continuumL6:129}
2 \frac{\partial^2 e_{12}}{\partial x_1 \partial x_2} 
=
\PDSq{x_1}{e_{22}} 
+\PDSq{x_2}{e_{11}}.
\end{equation}

This relationship is called the compatibility condition, and ensures that we don't have a disjoint deformation of the form in figure (\ref{fig:continuumL6:continuumL6fig1}).

\begin{figure}[htp]
   \centering
   \includegraphics[totalheight=0.2\textheight]{continuumL6fig1}
   \caption{disjoint deformation illustrated.}\label{fig:continuumL6:continuumL6fig1}
\end{figure}

\section{3D strain.}

While we have 9 components in the tensor, not all of these are independent.  The sets above and below the diagonal can be related, as illustrated in figure (\ref{fig:continuumL6:continuumL6fig2}).

\begin{figure}[htp]
   \centering
   \includegraphics[totalheight=0.2\textheight]{continuumL6fig2}
   \caption{continuumL6fig2}\label{fig:continuumL6:continuumL6fig2}
\end{figure}

Here we have 6 relationships between the components of the strain tensor $e_{ij}$.  Deriving these will be assigned in the homework.

\section{Elastodynamics.  Elastic waves.}

Reading: Chapter I \S 7, chapter III (\S 22 - \S 26) of the text \cite{landau1960theory}.

Example: sound or water waves (i.e. waves in a solid or liquid material that comes back to its original position.)

\begin{definition}
\emph{(Elastic Wave)}
\label{dfn:continuumL6:10}
An elastic wave is a type of mechanical wave that propagates through or on the surface of a medium.  The elasticity of the material provides the restoring force (that returns the material to its original state).  The displacement and the restoring force are assumed to be linearly related.
\end{definition}

In symbols we say

\begin{equation}\label{eqn:continuumL6:130}
e_i(x_j, t) \quad \mbox{related to force},
\end{equation}

and specifically

\begin{equation}\label{eqn:continuumL6:150}
\rho \PDSq{t}{e_i} = F_i = \PD{x_j}{\sigma_{ij}}.
\end{equation}

This is just Newton's second law, $F = ma$, but expressed in terms of a unit volume.

Should we have an external body force (per unit volume) $f_i$ acting on the body then we must modify this, writing

\begin{equation}\label{eqn:continuumL6:170}
\boxed{
\rho \PDSq{t}{e_i} = \PD{x_j}{\sigma_{ij}} + f_i
}
\end{equation}

Note that we are separating out the ``original'' forces that produced the stress and strain on the object from any constant external forces that act on the body (i.e. a gravitational field).

With 

\begin{equation}\label{eqn:continuumL6:190}
e_{ij} = 
\inv{2} \left( 
\PD{x_j}{e_i}
+ \PD{x_i}{e_j} \right),
\end{equation}

we can expand the stress divergence, for the case of homogeneous deformation, in terms of the Lam\'e parameters

\begin{equation}\label{eqn:continuumL6:210}
\sigma_{ij} = \lambda e_{kk} \delta_{ij} + 2 \mu e_{ij}.
\end{equation}

We compute

\begin{align*}
\PD{x_j}{\sigma_{ij}}
&=
\lambda 
\PD{x_j}{
e_{kk}
}
\delta_{ij} + 2 \mu 
\PD{x_j}{
}
\inv{2} \left( 
\PD{x_j}{e_i}
+ \PD{x_i}{e_j} \right),
 \\
&=
\lambda 
\PD{x_i}{
e_{kk}
}
+ \mu 
\left(
\PDSq{x_j}{
e_{i}
}
+
\frac{\partial^2 e_{j} }{ \partial x_j \partial x_i}
\right) \\
&=
%\sum_k 
\lambda 
\PD{x_i}{
}
\PD{x_k}{e_k}
+ \mu 
\left(
\PDSq{x_j}{
e_{i}
}
+
\frac{\partial^2 e_{k} }{ \partial x_k \partial x_i}
\right) \\
&=
(\lambda + \mu)
\PD{x_i}{
}
\PD{x_k}{e_k}
+ \mu 
\PDSq{x_j}{
e_{i}
}
\end{align*}

%With 
%
%\begin{equation}\label{eqn:continuumL6:230}
%e_{kk} = e_{11} +e_{22} +e_{33}
%= 
%\PD{x_1}{e_1}
%+\PD{x_2}{e_2}
%+\PD{x_3}{e_3}
%\end{equation}
%
We find, for homogeneous deformations, that the force per unit volume on our element of mass, in the absence of external forces (the body forces), takes the form
%
%\begin{equation}\label{eqn:continuumL6:250}
%\PD{x_j}{\sigma_{ij}} = (\lambda + \mu) \frac{\partial^2 e_j}{\partial x_i \partial x_j}
%+ \mu
%\frac{\partial^2 e_i}
%{\partial x_j^2
%}
%\end{equation}

\begin{equation}\label{eqn:continuumL6:270}
\rho \PDSq{t}{e_i} = (\lambda + \mu) \frac{\partial^2 e_k}{\partial x_i \partial x_k}
+ \mu
\frac{\partial^2 e_i}
{\partial x_j^2
}.
\end{equation}

This can be seen to be equivalent to the vector relationship

\begin{equation}\label{eqn:continuumL6:290}
\boxed{
\rho \PDSq{t}{\Be} = (\lambda + \mu) \spacegrad (\spacegrad \cdot \Be) + \mu \spacegrad^2 \Be.
}
\end{equation}

TODO: What form do the stress and strain tensors take in vector form?

\EndArticle
