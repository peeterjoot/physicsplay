%
%
%
% Copyright � 2012 Peeter Joot
% All Rights Reserved
%
% This file may be reproduced and distributed in whole or in part, without fee, subject to the following conditions:
%
% o The copyright notice above and this permission notice must be preserved complete on all complete or partial copies.
%
% o Any translation or derived work must be approved by the author in writing before distribution.
%
% o If you distribute this work in part, instructions for obtaining the complete version of this file must be included, and a means for obtaining a complete version provided.
%
%
% Exceptions to these rules may be granted for academic purposes: Write to the author and ask.
%
%
%
%%
% Copyright � 2015 Peeter Joot.  All Rights Reserved.
% Licenced as described in the file LICENSE under the root directory of this GIT repository.
%
\documentclass[]{eliblog}

\usepackage{amsmath}
\usepackage{mathpazo}

%
% shorthand for bold symbols, convenient for vectors and matrices
%
\newcommand{\Ba}[0]{\mathbf{a}}
\newcommand{\Bb}[0]{\mathbf{b}}
\newcommand{\Bc}[0]{\mathbf{c}}
\newcommand{\Bd}[0]{\mathbf{d}}
\newcommand{\Be}[0]{\mathbf{e}}
\newcommand{\Bf}[0]{\mathbf{f}}
\newcommand{\Bg}[0]{\mathbf{g}}
\newcommand{\Bh}[0]{\mathbf{h}}
\newcommand{\Bi}[0]{\mathbf{i}}
\newcommand{\Bj}[0]{\mathbf{j}}
\newcommand{\Bk}[0]{\mathbf{k}}
\newcommand{\Bl}[0]{\mathbf{l}}
\newcommand{\Bm}[0]{\mathbf{m}}
\newcommand{\Bn}[0]{\mathbf{n}}
\newcommand{\Bo}[0]{\mathbf{o}}
\newcommand{\Bp}[0]{\mathbf{p}}
\newcommand{\Bq}[0]{\mathbf{q}}
\newcommand{\Br}[0]{\mathbf{r}}
\newcommand{\Bs}[0]{\mathbf{s}}
\newcommand{\Bt}[0]{\mathbf{t}}
\newcommand{\Bu}[0]{\mathbf{u}}
\newcommand{\Bv}[0]{\mathbf{v}}
\newcommand{\Bw}[0]{\mathbf{w}}
\newcommand{\Bx}[0]{\mathbf{x}}
\newcommand{\By}[0]{\mathbf{y}}
\newcommand{\Bz}[0]{\mathbf{z}}
\newcommand{\BA}[0]{\mathbf{A}}
\newcommand{\BB}[0]{\mathbf{B}}
\newcommand{\BC}[0]{\mathbf{C}}
\newcommand{\BD}[0]{\mathbf{D}}
\newcommand{\BE}[0]{\mathbf{E}}
\newcommand{\BF}[0]{\mathbf{F}}
\newcommand{\BG}[0]{\mathbf{G}}
\newcommand{\BH}[0]{\mathbf{H}}
\newcommand{\BI}[0]{\mathbf{I}}
\newcommand{\BJ}[0]{\mathbf{J}}
\newcommand{\BK}[0]{\mathbf{K}}
\newcommand{\BL}[0]{\mathbf{L}}
\newcommand{\BM}[0]{\mathbf{M}}
\newcommand{\BN}[0]{\mathbf{N}}
\newcommand{\BO}[0]{\mathbf{O}}
\newcommand{\BP}[0]{\mathbf{P}}
\newcommand{\BQ}[0]{\mathbf{Q}}
\newcommand{\BR}[0]{\mathbf{R}}
\newcommand{\BS}[0]{\mathbf{S}}
\newcommand{\BT}[0]{\mathbf{T}}
\newcommand{\BU}[0]{\mathbf{U}}
\newcommand{\BV}[0]{\mathbf{V}}
\newcommand{\BW}[0]{\mathbf{W}}
\newcommand{\BX}[0]{\mathbf{X}}
\newcommand{\BY}[0]{\mathbf{Y}}
\newcommand{\BZ}[0]{\mathbf{Z}}

\newcommand{\Bzero}[0]{\mathbf{0}}
\newcommand{\Btheta}[0]{\boldsymbol{\theta}}
\newcommand{\Btau}[0]{\boldsymbol{\tau}}
\newcommand{\Bomega}[0]{\boldsymbol{\omega}}

%
% shorthand for unit vectors
%
\newcommand{\acap}[0]{\hat{\Ba}}
\newcommand{\bcap}[0]{\hat{\Bb}}
\newcommand{\ccap}[0]{\hat{\Bc}}
\newcommand{\dcap}[0]{\hat{\Bd}}
\newcommand{\ecap}[0]{\hat{\Be}}
\newcommand{\fcap}[0]{\hat{\Bf}}
\newcommand{\gcap}[0]{\hat{\Bg}}
\newcommand{\hcap}[0]{\hat{\Bh}}
\newcommand{\icap}[0]{\hat{\Bi}}
\newcommand{\jcap}[0]{\hat{\Bj}}
\newcommand{\kcap}[0]{\hat{\Bk}}
\newcommand{\lcap}[0]{\hat{\Bl}}
\newcommand{\mcap}[0]{\hat{\Bm}}
\newcommand{\ncap}[0]{\hat{\Bn}}
\newcommand{\ocap}[0]{\hat{\Bo}}
\newcommand{\pcap}[0]{\hat{\Bp}}
\newcommand{\qcap}[0]{\hat{\Bq}}
\newcommand{\rcap}[0]{\hat{\Br}}
\newcommand{\scap}[0]{\hat{\Bs}}
\newcommand{\tcap}[0]{\hat{\Bt}}
\newcommand{\ucap}[0]{\hat{\Bu}}
\newcommand{\vcap}[0]{\hat{\Bv}}
\newcommand{\wcap}[0]{\hat{\Bw}}
\newcommand{\xcap}[0]{\hat{\Bx}}
\newcommand{\ycap}[0]{\hat{\By}}
\newcommand{\zcap}[0]{\hat{\Bz}}
\newcommand{\thetacap}[0]{\hat{\Btheta}}

%
% to write R^n and C^n in a distinguishable fashion.  Perhaps change this
% to the double lined characters upon figuring out how to do so.
%
\newcommand{\C}[1]{$\mathbb{C}^{#1}$}
\newcommand{\R}[1]{$\mathbb{R}^{#1}$}

%
% various generally useful helpers
%

% derivative of #1 wrt. #2:
\newcommand{\D}[2] {\frac {d#2} {d#1}}

\newcommand{\inv}[1]{\frac{1}{#1}}
\newcommand{\cross}[0]{\times}

\newcommand{\abs}[1]{\lvert{#1}\rvert}
\newcommand{\norm}[1]{\lVert{#1}\rVert}
\newcommand{\innerprod}[2]{\langle{#1}, {#2}\rangle}
\newcommand{\dotprod}[2]{{#1} \cdot {#2}}
\newcommand{\bdotprod}[2]{\left({#1} \cdot {#2}\right)}
\newcommand{\crossprod}[2]{{#1} \cross {#2}}
\newcommand{\tripleprod}[3]{\dotprod{\left(\crossprod{#1}{#2}\right)}{#3}}

\DeclareMathOperator{\Proj}{Proj}
\DeclareMathOperator{\Span}{span}
\DeclareMathOperator{\Sgn}{sgn}
\DeclareMathOperator{\Area}{Area}
\DeclareMathOperator{\Volume}{Volume}

%
% A few miscellaneous things specific to this document
%
\newcommand{\crossop}[1]{\crossprod{#1}{}}

% R2 vector.
\newcommand{\VectorTwo}[2]{
\begin{bmatrix}
 {#1} \\
 {#2}
\end{bmatrix}
}

\newcommand{\VectorN}[1]{
\begin{bmatrix}
{#1}_1 \\
{#1}_2 \\
\vdots \\
{#1}_N \\
\end{bmatrix}
}

\newcommand{\DETuvij}[4]{
\begin{vmatrix}
 {#1}_{#3} & {#1}_{#4} \\
 {#2}_{#3} & {#2}_{#4}
\end{vmatrix}
}

\newcommand{\DETuvwijk}[6]{
\begin{vmatrix}
 {#1}_{#4} & {#1}_{#5} & {#1}_{#6} \\
 {#2}_{#4} & {#2}_{#5} & {#2}_{#6} \\
 {#3}_{#4} & {#3}_{#5} & {#3}_{#6}
\end{vmatrix}
}

\newcommand{\DETuvwxijkl}[8]{
\begin{vmatrix}
 {#1}_{#5} & {#1}_{#6} & {#1}_{#7} & {#1}_{#8} \\
 {#2}_{#5} & {#2}_{#6} & {#2}_{#7} & {#2}_{#8} \\
 {#3}_{#5} & {#3}_{#6} & {#3}_{#7} & {#3}_{#8} \\
 {#4}_{#5} & {#4}_{#6} & {#4}_{#7} & {#4}_{#8} \\
\end{vmatrix}
}

%\newcommand{\DETuvwxyijklm}[10]{
%\begin{vmatrix}
% {#1}_{#6} & {#1}_{#7} & {#1}_{#8} & {#1}_{#9} & {#1}_{#10} \\
% {#2}_{#6} & {#2}_{#7} & {#2}_{#8} & {#2}_{#9} & {#2}_{#10} \\
% {#3}_{#6} & {#3}_{#7} & {#3}_{#8} & {#3}_{#9} & {#3}_{#10} \\
% {#4}_{#6} & {#4}_{#7} & {#4}_{#8} & {#4}_{#9} & {#4}_{#10} \\
% {#5}_{#6} & {#5}_{#7} & {#5}_{#8} & {#5}_{#9} & {#5}_{#10}
%\end{vmatrix}
%}

% R3 vector.
\newcommand{\VectorThree}[3]{
\begin{bmatrix}
 {#1} \\
 {#2} \\
 {#3}
\end{bmatrix}
}



\author{Peeter Joot}
\email{peeter.joot@gmail.com}

%\documentclass[]{eliblogwidescreen}

\usepackage{amsmath}
\usepackage{mathpazo}

%
% shorthand for bold symbols, convenient for vectors and matrices
%
\newcommand{\Ba}[0]{\mathbf{a}}
\newcommand{\Bb}[0]{\mathbf{b}}
\newcommand{\Bc}[0]{\mathbf{c}}
\newcommand{\Bd}[0]{\mathbf{d}}
\newcommand{\Be}[0]{\mathbf{e}}
\newcommand{\Bf}[0]{\mathbf{f}}
\newcommand{\Bg}[0]{\mathbf{g}}
\newcommand{\Bh}[0]{\mathbf{h}}
\newcommand{\Bi}[0]{\mathbf{i}}
\newcommand{\Bj}[0]{\mathbf{j}}
\newcommand{\Bk}[0]{\mathbf{k}}
\newcommand{\Bl}[0]{\mathbf{l}}
\newcommand{\Bm}[0]{\mathbf{m}}
\newcommand{\Bn}[0]{\mathbf{n}}
\newcommand{\Bo}[0]{\mathbf{o}}
\newcommand{\Bp}[0]{\mathbf{p}}
\newcommand{\Bq}[0]{\mathbf{q}}
\newcommand{\Br}[0]{\mathbf{r}}
\newcommand{\Bs}[0]{\mathbf{s}}
\newcommand{\Bt}[0]{\mathbf{t}}
\newcommand{\Bu}[0]{\mathbf{u}}
\newcommand{\Bv}[0]{\mathbf{v}}
\newcommand{\Bw}[0]{\mathbf{w}}
\newcommand{\Bx}[0]{\mathbf{x}}
\newcommand{\By}[0]{\mathbf{y}}
\newcommand{\Bz}[0]{\mathbf{z}}
\newcommand{\BA}[0]{\mathbf{A}}
\newcommand{\BB}[0]{\mathbf{B}}
\newcommand{\BC}[0]{\mathbf{C}}
\newcommand{\BD}[0]{\mathbf{D}}
\newcommand{\BE}[0]{\mathbf{E}}
\newcommand{\BF}[0]{\mathbf{F}}
\newcommand{\BG}[0]{\mathbf{G}}
\newcommand{\BH}[0]{\mathbf{H}}
\newcommand{\BI}[0]{\mathbf{I}}
\newcommand{\BJ}[0]{\mathbf{J}}
\newcommand{\BK}[0]{\mathbf{K}}
\newcommand{\BL}[0]{\mathbf{L}}
\newcommand{\BM}[0]{\mathbf{M}}
\newcommand{\BN}[0]{\mathbf{N}}
\newcommand{\BO}[0]{\mathbf{O}}
\newcommand{\BP}[0]{\mathbf{P}}
\newcommand{\BQ}[0]{\mathbf{Q}}
\newcommand{\BR}[0]{\mathbf{R}}
\newcommand{\BS}[0]{\mathbf{S}}
\newcommand{\BT}[0]{\mathbf{T}}
\newcommand{\BU}[0]{\mathbf{U}}
\newcommand{\BV}[0]{\mathbf{V}}
\newcommand{\BW}[0]{\mathbf{W}}
\newcommand{\BX}[0]{\mathbf{X}}
\newcommand{\BY}[0]{\mathbf{Y}}
\newcommand{\BZ}[0]{\mathbf{Z}}

\newcommand{\Bzero}[0]{\mathbf{0}}
\newcommand{\Btheta}[0]{\boldsymbol{\theta}}
\newcommand{\Btau}[0]{\boldsymbol{\tau}}
\newcommand{\Bomega}[0]{\boldsymbol{\omega}}

%
% shorthand for unit vectors
%
\newcommand{\acap}[0]{\hat{\Ba}}
\newcommand{\bcap}[0]{\hat{\Bb}}
\newcommand{\ccap}[0]{\hat{\Bc}}
\newcommand{\dcap}[0]{\hat{\Bd}}
\newcommand{\ecap}[0]{\hat{\Be}}
\newcommand{\fcap}[0]{\hat{\Bf}}
\newcommand{\gcap}[0]{\hat{\Bg}}
\newcommand{\hcap}[0]{\hat{\Bh}}
\newcommand{\icap}[0]{\hat{\Bi}}
\newcommand{\jcap}[0]{\hat{\Bj}}
\newcommand{\kcap}[0]{\hat{\Bk}}
\newcommand{\lcap}[0]{\hat{\Bl}}
\newcommand{\mcap}[0]{\hat{\Bm}}
\newcommand{\ncap}[0]{\hat{\Bn}}
\newcommand{\ocap}[0]{\hat{\Bo}}
\newcommand{\pcap}[0]{\hat{\Bp}}
\newcommand{\qcap}[0]{\hat{\Bq}}
\newcommand{\rcap}[0]{\hat{\Br}}
\newcommand{\scap}[0]{\hat{\Bs}}
\newcommand{\tcap}[0]{\hat{\Bt}}
\newcommand{\ucap}[0]{\hat{\Bu}}
\newcommand{\vcap}[0]{\hat{\Bv}}
\newcommand{\wcap}[0]{\hat{\Bw}}
\newcommand{\xcap}[0]{\hat{\Bx}}
\newcommand{\ycap}[0]{\hat{\By}}
\newcommand{\zcap}[0]{\hat{\Bz}}
\newcommand{\thetacap}[0]{\hat{\Btheta}}

%
% to write R^n and C^n in a distinguishable fashion.  Perhaps change this
% to the double lined characters upon figuring out how to do so.
%
\newcommand{\C}[1]{$\mathbb{C}^{#1}$}
\newcommand{\R}[1]{$\mathbb{R}^{#1}$}

%
% various generally useful helpers
%

% derivative of #1 wrt. #2:
\newcommand{\D}[2] {\frac {d#2} {d#1}}

\newcommand{\inv}[1]{\frac{1}{#1}}
\newcommand{\cross}[0]{\times}

\newcommand{\abs}[1]{\lvert{#1}\rvert}
\newcommand{\norm}[1]{\lVert{#1}\rVert}
\newcommand{\innerprod}[2]{\langle{#1}, {#2}\rangle}
\newcommand{\dotprod}[2]{{#1} \cdot {#2}}
\newcommand{\bdotprod}[2]{\left({#1} \cdot {#2}\right)}
\newcommand{\crossprod}[2]{{#1} \cross {#2}}
\newcommand{\tripleprod}[3]{\dotprod{\left(\crossprod{#1}{#2}\right)}{#3}}

\DeclareMathOperator{\Proj}{Proj}
\DeclareMathOperator{\Span}{span}
\DeclareMathOperator{\Sgn}{sgn}
\DeclareMathOperator{\Area}{Area}
\DeclareMathOperator{\Volume}{Volume}

%
% A few miscellaneous things specific to this document
%
\newcommand{\crossop}[1]{\crossprod{#1}{}}

% R2 vector.
\newcommand{\VectorTwo}[2]{
\begin{bmatrix}
 {#1} \\
 {#2}
\end{bmatrix}
}

\newcommand{\VectorN}[1]{
\begin{bmatrix}
{#1}_1 \\
{#1}_2 \\
\vdots \\
{#1}_N \\
\end{bmatrix}
}

\newcommand{\DETuvij}[4]{
\begin{vmatrix}
 {#1}_{#3} & {#1}_{#4} \\
 {#2}_{#3} & {#2}_{#4}
\end{vmatrix}
}

\newcommand{\DETuvwijk}[6]{
\begin{vmatrix}
 {#1}_{#4} & {#1}_{#5} & {#1}_{#6} \\
 {#2}_{#4} & {#2}_{#5} & {#2}_{#6} \\
 {#3}_{#4} & {#3}_{#5} & {#3}_{#6}
\end{vmatrix}
}

\newcommand{\DETuvwxijkl}[8]{
\begin{vmatrix}
 {#1}_{#5} & {#1}_{#6} & {#1}_{#7} & {#1}_{#8} \\
 {#2}_{#5} & {#2}_{#6} & {#2}_{#7} & {#2}_{#8} \\
 {#3}_{#5} & {#3}_{#6} & {#3}_{#7} & {#3}_{#8} \\
 {#4}_{#5} & {#4}_{#6} & {#4}_{#7} & {#4}_{#8} \\
\end{vmatrix}
}

%\newcommand{\DETuvwxyijklm}[10]{
%\begin{vmatrix}
% {#1}_{#6} & {#1}_{#7} & {#1}_{#8} & {#1}_{#9} & {#1}_{#10} \\
% {#2}_{#6} & {#2}_{#7} & {#2}_{#8} & {#2}_{#9} & {#2}_{#10} \\
% {#3}_{#6} & {#3}_{#7} & {#3}_{#8} & {#3}_{#9} & {#3}_{#10} \\
% {#4}_{#6} & {#4}_{#7} & {#4}_{#8} & {#4}_{#9} & {#4}_{#10} \\
% {#5}_{#6} & {#5}_{#7} & {#5}_{#8} & {#5}_{#9} & {#5}_{#10}
%\end{vmatrix}
%}

% R3 vector.
\newcommand{\VectorThree}[3]{
\begin{bmatrix}
 {#1} \\
 {#2} \\
 {#3}
\end{bmatrix}
}



\author{Peeter Joot}
\email{peeter.joot@gmail.com}


%\usepackage{media9}
\chapter{Continuum mechanics elasticity review.}

\label{chap:continuumElasticityReview}
%\useCCL
\blogpage{http://sites.google.com/site/peeterjoot2/math2012/continuumElasticityReview.pdf}
\date{Apr 21, 2012}
\gitRevisionInfo{continuumElasticityReview}
\keywords{PHY454H1S, PHY454H1, strain, displacement vector, stress, constitutive relation, Lame parameters, shear modulus, Cauchy tetrahedron, Uniform hydrostatic compression, Bulk modulus, Uniaxial stress, Young's modulus, Poisson's ratio, Compatibility condition,P-waves, S-waves, wave equation, displacement potentials, phasor, Love wave, Rayleigh wave}

\beginArtWithToc
%\beginArtNoToc
%\wordpresscategory{}

\section{Motivation.}

Review of key ideas and equations from the theory of elasticity portion of the class.

\section{Strain Tensor}

Identifying a point in a solid with coordinates $x_i$ and the coordinates of that portion of the solid after displacement, we formed the difference as a measure of the displacement

\begin{equation}\label{eqn:continuumElasticityReview:10}
u_i = x_i' - x_i.
\end{equation}

With $du_i = \PDi{x_j}{u_i} dx_j$, we computed the difference in length (squared) for an element of the displaced solid and found 

\begin{equation}\label{eqn:continuumElasticityReview:30}
dx_k' dx_k' - dx_k dx_k = 
\left( 
\PD{x_i}{u_j} + 
\PD{x_j}{u_i} + 
\PD{x_i}{u_k} 
\PD{x_j}{u_k} 
\right) 
dx_i dx_j,
\end{equation}

or defining the \textit{strain tensor} $e_{ij}$, we have

\begin{subequations}
\begin{equation}\label{eqn:continuumElasticityReview:50}
(d\Bx')^2 - (d\Bx)^2
= 2 e_{ij} dx_i dx_j
\end{equation}
\begin{equation}\label{eqn:continuumElasticityReview:70}
e_{ij}
=
\inv{2}
\left( 
\PD{x_i}{u_j} + 
\PD{x_j}{u_i} + 
\PD{x_i}{u_k} 
\PD{x_j}{u_k} 
\right).
\end{equation}
\end{subequations}

In this course we use only the linear terms and write

\begin{equation}\label{eqn:continuumElasticityReview:90}
e_{ij}
=
\inv{2}
\left( 
\PD{x_i}{u_j} + 
\PD{x_j}{u_i} 
\right).
\end{equation}

\subsection{Unresolved: Relating displacement and position by strain}

In \cite{feynman1963flp:elasticMaterials} it is pointed out that this strain tensor simply relates the displacement vector coordinates $u_i$ to the coordinates at the point at which it is measured

\begin{equation}\label{eqn:continuumElasticityReview:110}
u_i = e_{ij} x_j.
\end{equation}

When we get to fluid dynamics we perform a linear expansion of $du_i$ and find something similar

\begin{equation}\label{eqn:continuumElasticityReview:530}
dx_i' - dx_i = du_i = \PD{x_k}{u_i} dx_k = e_{ij} dx_k + \omega_{ij} dx_k
\end{equation}

where

\begin{equation}\label{eqn:continuumElasticityReview:550}
\omega_{ij} = \inv{2} \left( \PD{x_i}{u_j} +\PD{x_j}{u_i} \right).
\end{equation}

Except for the antisymmetric term, note the structural similarity of \ref{eqn:continuumElasticityReview:110} and \ref{eqn:continuumElasticityReview:530}.  Why is it that we neglect the vorticity tensor in statics?
%  If we are approximating the displacement, it appears to have a natural place in things, as we can see
%
%\begin{align*}
%u_i 
%&\approx \PD{x_j}{u_i} x_j \\
%&=
%e_{ij} x_j + \omega_{ik} x_k.
%\end{align*}
%
%This is easily seen to be the case, recovering \ref{eqn:continuumElasticityReview:90} by taking derivatives of \ref{eqn:continuumElasticityReview:110}, plus an assumption that $e_{ij}$ is symmetric.

\subsection{Diagonal strain representation.}

In a basis for which the strain tensor is diagonal, it was pointed out that we can write our difference in squared displacement as (for $k = 1, 2, 3$, no summation convention)

\begin{equation}\label{eqn:continuumElasticityReview:130}
(dx_k')^2 - (dx_k)^2 = 2 e_{kk} dx_k dx_k
\end{equation}

from which we can rearrange, take roots, and apply a first order Taylor expansion to find (again no summation convention)

\begin{equation}\label{eqn:continuumElasticityReview:150}
dx_k' \approx (1 + e_{kk}) dx_k.
\end{equation}

An approximation of the displaced volume was then found in terms of the strain tensor trace (summation convention back again)

\begin{equation}\label{eqn:continuumElasticityReview:170}
dV' \approx (1 + e_{kk}) dV,
\end{equation}

allowing us to identify this trace as a relative difference in displaced volume

\begin{equation}\label{eqn:continuumElasticityReview:190}
e_{kk} \approx \frac{dV' - dV}{dV}.
\end{equation}

\subsection{Strain in cylindrical coordinates.}

Useful in many practice problems are the cylindrical coordinate representation of the strain tensor 

\begin{align}\label{eqn:continuumElasticityReview:210}
2 e_{rr} &= \PD{r}{u_r}  \\
2 e_{\phi\phi} &= \inv{r} \PD{\phi}{u_\phi} +\inv{r} u_r  \\
2 e_{zz} &= \PD{z}{u_z}  \\
2 e_{zr} &= \PD{z}{u_r} + \PD{r}{u_z} \\
2 e_{r\phi} &= \PD{r}{u_\phi} - \inv{r} u_\phi + \inv{r} \PD{\phi}{u_r} \\
2 e_{\phi z} &= \PD{z}{u_\phi} +\inv{r} \PD{\phi}{u_z}.
\end{align}

This can be found in \cite{landau1960theory}.  It was not derived there or in class, but is not too hard, even using the second order methods we used for the Cartesian form of the tensor.

An easier way to do this derivation (and understand what the coordinates represent) follows from the relation found in \S 6 of \cite{acheson1990elementary}

\begin{equation}\label{eqn:continuumElasticityReview:230}
2 \Be_i e_{ij} n_j = 2 (\ncap \cdot \spacegrad) \Bu + \ncap \cross (\spacegrad \cross \Bu),
\end{equation}

where $\ncap$ is the normal to the surface at which we are measuring a force applied to the solid (our Cauchy tetrahedron).

The cylindrical tensor coordinates of \ref{eqn:continuumElasticityReview:210} follow from 
\ref{eqn:continuumElasticityReview:230} nicely taking $\ncap = \rcap, \phicap, \zcap$ in turn.

\subsection{Compatibility condition.}

For a 2D strain tensor we found an interrelationship between the components of the strain tensor

\begin{equation}\label{eqn:continuumElasticityReview:510}
2 \frac{\partial^2 e_{12}}{\partial x_1 \partial x_2} 
=
\PDSq{x_1}{e_{22}} 
+\PDSq{x_2}{e_{11}},
\end{equation}

and called this the compatibility condition.  It was claimed, but not demonstrated that this is what is required to ensure a deformation maintained a coherent solid geometry.

I wasn't able to find any references to this compatibility condition in any of the texts I have, but found \cite{wiki:compatibilityMechanics}, \cite{wiki:infinitesimalStrainTheory}, and \cite{wiki:saintVenantCompat}.  It's not terribly surprising to see Christoffel symbol and differential forms references on those pages, since one can imagine that we'd wish to look at the mappings of all the points in the object as it undergoes the transformation from the original to the deformed state.

Even with just three points in a plane, say $\Ba$, $\Bb$, $\Bc$, the general deformation of an object doesn't seem like it's the easiest thing to describe.  We can imagine that these have trajectories in the deformation process $\Ba = \Ba(\alpha$, $\Bb = \Bb(\beta)$, $\Bc = \Bc(\gamma)$, with $\Ba', \Bb', \Bc'$ at the end points of the trajectories.  We'd want to look at displacement vectors $\Bu_a, \Bu_b, \Bu_c$ along each of these trajectories, and then see how they must be related.  Doing that carefully must result in this compatibility condition.

\section{Stress tensor.}

By sought and found a representation of the force per unit area acting on a body by expressing the components of that force as a set of divergence relations

\begin{equation}\label{eqn:continuumElasticityReview:250}
f_i = \partial_k \sigma_{i k},
\end{equation}

and call the associated tensor $\sigma_{ij}$ the \textit{stress}.

Unlike the strain, we don't have any expectation that this tensor is symmetric, and identify the diagonal components (no sum) $\sigma_{i i}$ as quantifying the amount of compressive or contractive force per unit area, whereas the cross terms of the stress tensor introduce shearing deformations in the solid.

With force balance arguments (the Cauchy tetrahedron) we found that the force per unit area on the solid, for a surface with unit normal pointing into the solid, was

\begin{equation}\label{eqn:continuumElasticityReview:270}
\Bt = \Be_i t_i = \Be_i \sigma_{ij} n_j.
\end{equation}

\subsection{Constitutive relation.}

In the scope of this course we considered only Newtonian materials, those for which the stress and strain tensors are linearly related

\begin{equation}\label{eqn:continuumElasticityReview:290}
\sigma_{ij} = c_{ijkl} e_{kl},
\end{equation}

and further restricted our attention to isotropic materials, which can be shown to have the form

\begin{equation}\label{eqn:continuumElasticityReview:310}
\sigma_{ij} = \lambda e_{kk} \delta_{ij} + 2 \mu e_{ij},
\end{equation}

where $\lambda$ and $\mu$ are the Lame parameters and $\mu$ is called the shear modulus (and viscosity in the context of fluids).

By computing the trace of the stress $\sigma_{ii}$ we can invert this to find

\begin{equation}\label{eqn:continuumElasticityReview:330}
2 \mu e_{ij} = \sigma_{ij} - \frac{\lambda}{3 \lambda + 2 \mu} \sigma_{kk} \delta_{ij}.
\end{equation}

\subsection{Uniform hydrostatic compression.}

With only normal components of the stress (no shear), and the stress having the same value in all directions, we find

\begin{equation}\label{eqn:continuumElasticityReview:350}
\sigma_{ij} = ( 3 \lambda + 2 \mu ) e_{ij},
\end{equation}

and identify this combination $-3 \lambda - 2 \mu$ as the pressure, linearly relating the stress and strain tensors

\begin{equation}\label{eqn:continuumElasticityReview:370}
\sigma_{ij} = -p e_{ij}.
\end{equation}

With $e_{ii} = (dV' - dV)/dV = \Delta V/V$, we formed the Bulk modulus $K$ with the value

\begin{equation}\label{eqn:continuumElasticityReview:390}
K = \left( \lambda + \frac{2 \mu}{3} \right) = -\frac{p V}{\Delta V}.
\end{equation}

\subsection{Uniaxial stress.  Young's modulus.  Poisson's ratio.}

For the special case with only one non-zero stress component (we used $\sigma_{11}$) we were able to compute Young's modulus $E$, the ratio between stress and strain in that direction

\begin{equation}\label{eqn:continuumElasticityReview:410}
E = \frac{\sigma_{11}}{e_{11}} = \frac{\mu(3 \lambda + 2 \mu)}{\lambda + \mu }  = \frac{3 K \mu}{K + \mu/3}.
\end{equation}

Just because only one component of the stress is non-zero, does not mean that we have no deformation in any other directions.  Introducing Poisson's ratio $\nu$ in terms of the ratio of the strains relative to the strain in the direction of the force we write and then subsequently found

\begin{equation}\label{eqn:continuumElasticityReview:430}
\nu = -\frac{e_{22}}{e_{11}} = -\frac{e_{33}}{e_{11}} = \frac{\lambda}{2(\lambda + \mu)}.
\end{equation}

We were also able to find

We can also relate the Poisson's ratio $\nu$ to the shear modulus $\mu$

\begin{equation}\label{eqn:continuumElasticityReview:450}
\mu = \frac{E}{2(1 + \nu)}
\end{equation}

\begin{equation}\label{eqn:continuumElasticityReview:470}
\lambda = \frac{E \nu}{(1 - 2 \nu)(1 + \nu)}
\end{equation}

\begin{align}\label{eqn:continuumElasticityReview:490}
e_{11} &= \inv{E}\left( \sigma_{11} - \nu(\sigma_{22} + \sigma_{33}) \right) \\
e_{22} &= \inv{E}\left( \sigma_{22} - \nu(\sigma_{11} + \sigma_{33}) \right) \\
e_{33} &= \inv{E}\left( \sigma_{33} - \nu(\sigma_{11} + \sigma_{22}) \right)
\end{align}

\section{Displacement propagation}

It was argued that the equation relating the time evolution of a one of the vector displacement coordinates was given by

\begin{equation}\label{eqn:continuumElasticityReview:570}
\rho \PDSq{t}{u_i} = \PD{x_j}{\sigma_{ij}} + f_i,
\end{equation}

where the divergence term $\PDi{x_j}{\sigma_{ij}}$ is the internal force per unit volume on the object and $f_i$ is the external force.  Employing the constitutive relation we showed that this can be expanded as

\begin{equation}\label{eqn:continuumElasticityReview:590}
\rho \PDSq{t}{u_i} = (\lambda + \mu) \frac{\partial^2 u_k}{\partial x_i \partial x_k}
+ \mu
\frac{\partial^2 u_i}
{\partial x_j^2
},
\end{equation}

or in vector form

\begin{equation}\label{eqn:continuumElasticityReview:610}
\rho \PDSq{t}{\Bu} = (\lambda + \mu) \spacegrad (\spacegrad \cdot \Bu) + \mu \spacegrad^2 \Bu.
\end{equation}

\subsection{P-waves}

Operating on \ref{eqn:continuumElasticityReview:610} with the divergence operator, and writing $\Theta = \spacegrad \cdot \Bu$, a quantity that was our relative change in volume in the diagonal strain basis, we were able to find this divergence obeys a wave equation

\begin{equation}\label{eqn:continuumElasticityReview:630}
\PDSq{t}{\Theta} = \frac{\lambda + 2 \mu}{\rho} \spacegrad^2 \Theta.
\end{equation}

We called these P-waves.

\subsection{S-waves}

Similarly, operating on \ref{eqn:continuumElasticityReview:610} with the curl operator, and writing $\Bomega = \spacegrad \cross \Bu$, we were able to find this curl also obeys a wave equation

\begin{equation}\label{eqn:continuumElasticityReview:650}
\rho \PDSq{t}{\Bomega} = \mu \spacegrad^2 \Bomega.
\end{equation}

These we called S-waves.  We also noted that the (transverse) compression waves (P-waves) with speed $C_T = \sqrt{\mu/\rho}$, traveled faster than the (longitudinal) vorticity (S) waves with speed $C_L = \sqrt{(\lambda + 2 \mu)/\rho}$ since $\lambda > 0$ and $\mu > 0$, and 

\begin{equation}\label{eqn:continuumElasticityReview:670}
\frac{C_L}{C_T} = \sqrt{\frac{ \lambda + 2 \mu}{\mu}} = \sqrt{ \frac{\lambda}{\mu} + 2}.
\end{equation}

\subsection{Scalar and vector potential representation.}

Assuming a vector displacement representation with gradient and curl components

\begin{equation}\label{eqn:continuumElasticityReview:690}
\Bu = \spacegrad \phi + \spacegrad \cross \BH,
\end{equation}

We found that the displacement time evolution equation split nicely into curl free and divergence free terms

\begin{equation}\label{eqn:continuumElasticityReview:710}
\spacegrad
\left(
\rho \PDSq{t}{\phi} - (\lambda + 2\mu) \spacegrad^2 \phi
\right)
+
\spacegrad \cross
\left(
\rho \PDSq{t}{\BH} - \mu \spacegrad^2 \BH
\right)
= 0.
\end{equation}

When neglecting boundary value effects this could be written as a pair of independent equations

\begin{subequations}
\begin{equation}\label{eqn:continuumElasticityReview:730}
\rho \PDSq{t}{\phi} - (\lambda + 2\mu) \spacegrad^2 \phi = 0
\end{equation}
\begin{equation}\label{eqn:continuumElasticityReview:750}
\rho \PDSq{t}{\BH} - \mu \spacegrad^2 \BH
= 0.
\end{equation}
\end{subequations}

This are the irrotational (curl free) P-wave and solenoidal (divergence free) S-wave equations respectively.

%This theory led to no actual calculation work, just a few videos that illustrated what we'd presumably be able to calculate if we were to attempt to apply these concepts.

\subsection{Phasor description.}

It was mentioned that we could assume a phasor representation for our potentials, writing

\begin{subequations}
\begin{equation}\label{eqn:continuumElasticityReview:770}
\phi = A \exp\left( i ( \Bk \cdot \Bx - \omega t) \right) 
\end{equation}
\begin{equation}\label{eqn:continuumElasticityReview:790}
\BH = \BB \exp\left( i ( \Bk \cdot \Bx - \omega t) \right)
\end{equation}
\end{subequations}

finding

\begin{equation}\label{eqn:continuumElasticityReview:810}
\Bu = i \Bk \phi + i \Bk \cross \BH.
\end{equation}

We did nothing with neither the potential nor the phasor theory for solid displacement time evolution, and presumably won't on the exam either.

\section{Some wave types}

Some time was spent on non-qualitative descriptions and review of descriptions for solutions to the time evolution equations we did not attempt

\begin{itemize}
\item P-waves \cite{wiki:pwave}.  Irrotational, non volume preserving body wave.
\item S-waves \cite{wiki:swave}.  Divergence free body wave.  Shearing forces are present and volume is preserved (slower than S-waves)
\item Rayleigh wave \cite{wiki:rayleighwave}.  A surface wave that propagates near the surface of a body without penetrating into it.
\item Love wave \cite{wiki:lovewave}.  A polarized shear surface wave with the shear displacements moving perpendicular to the direction of propagation.
\end{itemize}

For reasons that aren't clear both the midterm and last years final ask us to spew this sort of stuff (instead of actually trying to do something analytic associated with them).

\EndArticle
