\chapter*{Preface}\normalsize
  \addcontentsline{toc}{chapter}{Preface}

These are my personal lecture notes for the Winter 2012, University of Toronto Continuum Mechanics course (PHY454H1S), taught by Prof.\ Kausik S. Das.

The official description for this course was:

The theory of continuous matter, including solid and fluid mechanics.  Topics include the continuum approximation, dimensional analysis, stress, strain, the Euler and Navier-Stokes equations, vorticity, waves, instabilities, convection and turbulence.

This document contains a few things

\begin{itemize}
\item My lecture notes.

Typos, if any, are probably mine (Peeter), and no claim nor attempt of spelling or grammar correctness will be made.

\item Some personal notes exploring details that weren't clear to me from the lectures, or from the texts associated with the lecture material.

\item Some worked problems.

\item Links to mathematica workbooks associated with course content.

\end{itemize}

You may wonder why took the time to make these notes.  I admit to being a very fast typist, and was usually was able to take the bulk of these notes live in class, so that I'd have a complete pdf document for the lecture by the time the class was over.  However, doing so was still not free in terms of time consumption.  It takes some time to scan any figures I sketched in class (or later re-draw using a graphics tablet).  It was also not free to type up some of the worked problems I did for study purposes.  I personally find that formally writing up solved problems often highlights places where fundamental principles are not completely understood, and also serves to make calculation errors obvious.

In the grand scheme of things, I learn a lot by writing my notes, since I have to explain portions of the material to myself in the process.  I am also glad to have a reference that I can return to for later study, especially since we didn't formally use a textbook in this class.  Having made the effort to do this for myself, it would be a shame not to share, so I make this available for future students to also potentially extract some value from my efforts.

Should you want to add to these notes, all the latex sources, makefiles, figures, and build scripts (of which there are many, even for just this one set of course notes), are available using the git command.  \href{http://git-scm.com/}{Git is a software version control system}, which allows the author to track document revisions.  My source repository is currently stored on \href{http://github.com}{github.com} (free hosting of source code repositories), and can be \href{https://github.com/peeterjoot/physicsplay/tree/master/notes/phy454}{explored live from there} if desired.  Use of a git repository also gives me version control which, as a software developer, I have grown dependent on.  It also happens to provide a distributed backup of all my sources as a convenient side effect.

Use of the git is the best way to access the source for this document should you wish to, because if you have a change you want to make, it can be incorporated back into the live version with a push or merge operation (or for more ad-hoc changes, a patch can be generated and emailed).

With all the sources available for these notes, as a reader, you have the capability to get your own copy, and make any private modifications you feel desirable (and eventually push them back into my repository if I like what you've done).  Chances are that nobody else will make any contribution to these notes, but in the ongoing transformation of learning and teaching infrastructure I am guessing that collaborative course material will eventually be part of the process.  Imagine what you could have if all the students clarified the aspects of their texts that they personally found lacking, or added worked problems that they found instructive.

Here is an outline of how to get access to the sources for this document

\begin{itemize}
\item Set up a github userid and ssh key for yourself.  Instructions for that here can be found on \href{http://help.github.com/win-set-up-git/}{github}.

Even if you don't want to use this for my source repository, if you aren't using version control software for your own document text files I \textit{highly} recommend you do so.

\item Install a git client on your machine.  I use either linux or a \href{http://www.cygwin.com/}{Windows cygwin environment} (free implementation of gnu Unix command line utilities.  Any other Unix would work provided you have both gnu-make, perl and a recent latex distribution.  I use MikTex on Windows, and texlive on Linux.

\item Get your self a copy of the source repository that contains the sources.  Run:

\begin{lstlisting}
cd ~/
git clone git@github.com:peeterjoot/physicsplay.git ~/peeters_physics_play/
\end{lstlisting}

Note that this will also get you all the mathematica notebooks that I wrote while studying for this course.  You can execute any of those with the free wolfram CDF viewer once you install it.

\item Build yourself a copy of the pdf.

\begin{lstlisting}
cd ~/peeters_physics_play/notes/phy454/
make
\end{lstlisting}

The first time you do this with MikTex, you'll probably take a hit to install a number of packages.  If running on a non-Linux Unix platform, use gnu-make explicitly.  The document structure can be explored starting from main.tex.

\item After running the git clone command, an invokation of:

\begin{lstlisting}
git pull origin master
\end{lstlisting}

from anywhere in the PeetersPhysicsPlay directory, will get you a more recent version of the source if there have been updates.
\end{itemize}

Peeter Joot  \quad peeter.joot@gmail.com 

