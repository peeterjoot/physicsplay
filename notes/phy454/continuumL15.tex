% 
% 
% 
% Copyright � 2012 Peeter Joot
% All Rights Reserved
% 
% This file may be reproduced and distributed in whole or in part, without fee, subject to the following conditions:
% 
% o The copyright notice above and this permission notice must be preserved complete on all complete or partial copies.
% 
% o Any translation or derived work must be approved by the author in writing before distribution.
% 
% o If you distribute this work in part, instructions for obtaining the complete version of this file must be included, and a means for obtaining a complete version provided.
% 
% 
% Exceptions to these rules may be granted for academic purposes: Write to the author and ask.
% 
% 
% 
%%
% Copyright � 2015 Peeter Joot.  All Rights Reserved.
% Licenced as described in the file LICENSE under the root directory of this GIT repository.
%
\documentclass[]{eliblog}

\usepackage{amsmath}
\usepackage{mathpazo}

%
% shorthand for bold symbols, convenient for vectors and matrices
%
\newcommand{\Ba}[0]{\mathbf{a}}
\newcommand{\Bb}[0]{\mathbf{b}}
\newcommand{\Bc}[0]{\mathbf{c}}
\newcommand{\Bd}[0]{\mathbf{d}}
\newcommand{\Be}[0]{\mathbf{e}}
\newcommand{\Bf}[0]{\mathbf{f}}
\newcommand{\Bg}[0]{\mathbf{g}}
\newcommand{\Bh}[0]{\mathbf{h}}
\newcommand{\Bi}[0]{\mathbf{i}}
\newcommand{\Bj}[0]{\mathbf{j}}
\newcommand{\Bk}[0]{\mathbf{k}}
\newcommand{\Bl}[0]{\mathbf{l}}
\newcommand{\Bm}[0]{\mathbf{m}}
\newcommand{\Bn}[0]{\mathbf{n}}
\newcommand{\Bo}[0]{\mathbf{o}}
\newcommand{\Bp}[0]{\mathbf{p}}
\newcommand{\Bq}[0]{\mathbf{q}}
\newcommand{\Br}[0]{\mathbf{r}}
\newcommand{\Bs}[0]{\mathbf{s}}
\newcommand{\Bt}[0]{\mathbf{t}}
\newcommand{\Bu}[0]{\mathbf{u}}
\newcommand{\Bv}[0]{\mathbf{v}}
\newcommand{\Bw}[0]{\mathbf{w}}
\newcommand{\Bx}[0]{\mathbf{x}}
\newcommand{\By}[0]{\mathbf{y}}
\newcommand{\Bz}[0]{\mathbf{z}}
\newcommand{\BA}[0]{\mathbf{A}}
\newcommand{\BB}[0]{\mathbf{B}}
\newcommand{\BC}[0]{\mathbf{C}}
\newcommand{\BD}[0]{\mathbf{D}}
\newcommand{\BE}[0]{\mathbf{E}}
\newcommand{\BF}[0]{\mathbf{F}}
\newcommand{\BG}[0]{\mathbf{G}}
\newcommand{\BH}[0]{\mathbf{H}}
\newcommand{\BI}[0]{\mathbf{I}}
\newcommand{\BJ}[0]{\mathbf{J}}
\newcommand{\BK}[0]{\mathbf{K}}
\newcommand{\BL}[0]{\mathbf{L}}
\newcommand{\BM}[0]{\mathbf{M}}
\newcommand{\BN}[0]{\mathbf{N}}
\newcommand{\BO}[0]{\mathbf{O}}
\newcommand{\BP}[0]{\mathbf{P}}
\newcommand{\BQ}[0]{\mathbf{Q}}
\newcommand{\BR}[0]{\mathbf{R}}
\newcommand{\BS}[0]{\mathbf{S}}
\newcommand{\BT}[0]{\mathbf{T}}
\newcommand{\BU}[0]{\mathbf{U}}
\newcommand{\BV}[0]{\mathbf{V}}
\newcommand{\BW}[0]{\mathbf{W}}
\newcommand{\BX}[0]{\mathbf{X}}
\newcommand{\BY}[0]{\mathbf{Y}}
\newcommand{\BZ}[0]{\mathbf{Z}}

\newcommand{\Bzero}[0]{\mathbf{0}}
\newcommand{\Btheta}[0]{\boldsymbol{\theta}}
\newcommand{\Btau}[0]{\boldsymbol{\tau}}
\newcommand{\Bomega}[0]{\boldsymbol{\omega}}

%
% shorthand for unit vectors
%
\newcommand{\acap}[0]{\hat{\Ba}}
\newcommand{\bcap}[0]{\hat{\Bb}}
\newcommand{\ccap}[0]{\hat{\Bc}}
\newcommand{\dcap}[0]{\hat{\Bd}}
\newcommand{\ecap}[0]{\hat{\Be}}
\newcommand{\fcap}[0]{\hat{\Bf}}
\newcommand{\gcap}[0]{\hat{\Bg}}
\newcommand{\hcap}[0]{\hat{\Bh}}
\newcommand{\icap}[0]{\hat{\Bi}}
\newcommand{\jcap}[0]{\hat{\Bj}}
\newcommand{\kcap}[0]{\hat{\Bk}}
\newcommand{\lcap}[0]{\hat{\Bl}}
\newcommand{\mcap}[0]{\hat{\Bm}}
\newcommand{\ncap}[0]{\hat{\Bn}}
\newcommand{\ocap}[0]{\hat{\Bo}}
\newcommand{\pcap}[0]{\hat{\Bp}}
\newcommand{\qcap}[0]{\hat{\Bq}}
\newcommand{\rcap}[0]{\hat{\Br}}
\newcommand{\scap}[0]{\hat{\Bs}}
\newcommand{\tcap}[0]{\hat{\Bt}}
\newcommand{\ucap}[0]{\hat{\Bu}}
\newcommand{\vcap}[0]{\hat{\Bv}}
\newcommand{\wcap}[0]{\hat{\Bw}}
\newcommand{\xcap}[0]{\hat{\Bx}}
\newcommand{\ycap}[0]{\hat{\By}}
\newcommand{\zcap}[0]{\hat{\Bz}}
\newcommand{\thetacap}[0]{\hat{\Btheta}}

%
% to write R^n and C^n in a distinguishable fashion.  Perhaps change this
% to the double lined characters upon figuring out how to do so.
%
\newcommand{\C}[1]{$\mathbb{C}^{#1}$}
\newcommand{\R}[1]{$\mathbb{R}^{#1}$}

%
% various generally useful helpers
%

% derivative of #1 wrt. #2:
\newcommand{\D}[2] {\frac {d#2} {d#1}}

\newcommand{\inv}[1]{\frac{1}{#1}}
\newcommand{\cross}[0]{\times}

\newcommand{\abs}[1]{\lvert{#1}\rvert}
\newcommand{\norm}[1]{\lVert{#1}\rVert}
\newcommand{\innerprod}[2]{\langle{#1}, {#2}\rangle}
\newcommand{\dotprod}[2]{{#1} \cdot {#2}}
\newcommand{\bdotprod}[2]{\left({#1} \cdot {#2}\right)}
\newcommand{\crossprod}[2]{{#1} \cross {#2}}
\newcommand{\tripleprod}[3]{\dotprod{\left(\crossprod{#1}{#2}\right)}{#3}}

\DeclareMathOperator{\Proj}{Proj}
\DeclareMathOperator{\Span}{span}
\DeclareMathOperator{\Sgn}{sgn}
\DeclareMathOperator{\Area}{Area}
\DeclareMathOperator{\Volume}{Volume}

%
% A few miscellaneous things specific to this document
%
\newcommand{\crossop}[1]{\crossprod{#1}{}}

% R2 vector.
\newcommand{\VectorTwo}[2]{
\begin{bmatrix}
 {#1} \\
 {#2}
\end{bmatrix}
}

\newcommand{\VectorN}[1]{
\begin{bmatrix}
{#1}_1 \\
{#1}_2 \\
\vdots \\
{#1}_N \\
\end{bmatrix}
}

\newcommand{\DETuvij}[4]{
\begin{vmatrix}
 {#1}_{#3} & {#1}_{#4} \\
 {#2}_{#3} & {#2}_{#4}
\end{vmatrix}
}

\newcommand{\DETuvwijk}[6]{
\begin{vmatrix}
 {#1}_{#4} & {#1}_{#5} & {#1}_{#6} \\
 {#2}_{#4} & {#2}_{#5} & {#2}_{#6} \\
 {#3}_{#4} & {#3}_{#5} & {#3}_{#6}
\end{vmatrix}
}

\newcommand{\DETuvwxijkl}[8]{
\begin{vmatrix}
 {#1}_{#5} & {#1}_{#6} & {#1}_{#7} & {#1}_{#8} \\
 {#2}_{#5} & {#2}_{#6} & {#2}_{#7} & {#2}_{#8} \\
 {#3}_{#5} & {#3}_{#6} & {#3}_{#7} & {#3}_{#8} \\
 {#4}_{#5} & {#4}_{#6} & {#4}_{#7} & {#4}_{#8} \\
\end{vmatrix}
}

%\newcommand{\DETuvwxyijklm}[10]{
%\begin{vmatrix}
% {#1}_{#6} & {#1}_{#7} & {#1}_{#8} & {#1}_{#9} & {#1}_{#10} \\
% {#2}_{#6} & {#2}_{#7} & {#2}_{#8} & {#2}_{#9} & {#2}_{#10} \\
% {#3}_{#6} & {#3}_{#7} & {#3}_{#8} & {#3}_{#9} & {#3}_{#10} \\
% {#4}_{#6} & {#4}_{#7} & {#4}_{#8} & {#4}_{#9} & {#4}_{#10} \\
% {#5}_{#6} & {#5}_{#7} & {#5}_{#8} & {#5}_{#9} & {#5}_{#10}
%\end{vmatrix}
%}

% R3 vector.
\newcommand{\VectorThree}[3]{
\begin{bmatrix}
 {#1} \\
 {#2} \\
 {#3}
\end{bmatrix}
}



\author{Peeter Joot}
\email{peeter.joot@gmail.com}

%\documentclass[]{eliblogwidescreen}

\usepackage{amsmath}
\usepackage{mathpazo}

%
% shorthand for bold symbols, convenient for vectors and matrices
%
\newcommand{\Ba}[0]{\mathbf{a}}
\newcommand{\Bb}[0]{\mathbf{b}}
\newcommand{\Bc}[0]{\mathbf{c}}
\newcommand{\Bd}[0]{\mathbf{d}}
\newcommand{\Be}[0]{\mathbf{e}}
\newcommand{\Bf}[0]{\mathbf{f}}
\newcommand{\Bg}[0]{\mathbf{g}}
\newcommand{\Bh}[0]{\mathbf{h}}
\newcommand{\Bi}[0]{\mathbf{i}}
\newcommand{\Bj}[0]{\mathbf{j}}
\newcommand{\Bk}[0]{\mathbf{k}}
\newcommand{\Bl}[0]{\mathbf{l}}
\newcommand{\Bm}[0]{\mathbf{m}}
\newcommand{\Bn}[0]{\mathbf{n}}
\newcommand{\Bo}[0]{\mathbf{o}}
\newcommand{\Bp}[0]{\mathbf{p}}
\newcommand{\Bq}[0]{\mathbf{q}}
\newcommand{\Br}[0]{\mathbf{r}}
\newcommand{\Bs}[0]{\mathbf{s}}
\newcommand{\Bt}[0]{\mathbf{t}}
\newcommand{\Bu}[0]{\mathbf{u}}
\newcommand{\Bv}[0]{\mathbf{v}}
\newcommand{\Bw}[0]{\mathbf{w}}
\newcommand{\Bx}[0]{\mathbf{x}}
\newcommand{\By}[0]{\mathbf{y}}
\newcommand{\Bz}[0]{\mathbf{z}}
\newcommand{\BA}[0]{\mathbf{A}}
\newcommand{\BB}[0]{\mathbf{B}}
\newcommand{\BC}[0]{\mathbf{C}}
\newcommand{\BD}[0]{\mathbf{D}}
\newcommand{\BE}[0]{\mathbf{E}}
\newcommand{\BF}[0]{\mathbf{F}}
\newcommand{\BG}[0]{\mathbf{G}}
\newcommand{\BH}[0]{\mathbf{H}}
\newcommand{\BI}[0]{\mathbf{I}}
\newcommand{\BJ}[0]{\mathbf{J}}
\newcommand{\BK}[0]{\mathbf{K}}
\newcommand{\BL}[0]{\mathbf{L}}
\newcommand{\BM}[0]{\mathbf{M}}
\newcommand{\BN}[0]{\mathbf{N}}
\newcommand{\BO}[0]{\mathbf{O}}
\newcommand{\BP}[0]{\mathbf{P}}
\newcommand{\BQ}[0]{\mathbf{Q}}
\newcommand{\BR}[0]{\mathbf{R}}
\newcommand{\BS}[0]{\mathbf{S}}
\newcommand{\BT}[0]{\mathbf{T}}
\newcommand{\BU}[0]{\mathbf{U}}
\newcommand{\BV}[0]{\mathbf{V}}
\newcommand{\BW}[0]{\mathbf{W}}
\newcommand{\BX}[0]{\mathbf{X}}
\newcommand{\BY}[0]{\mathbf{Y}}
\newcommand{\BZ}[0]{\mathbf{Z}}

\newcommand{\Bzero}[0]{\mathbf{0}}
\newcommand{\Btheta}[0]{\boldsymbol{\theta}}
\newcommand{\Btau}[0]{\boldsymbol{\tau}}
\newcommand{\Bomega}[0]{\boldsymbol{\omega}}

%
% shorthand for unit vectors
%
\newcommand{\acap}[0]{\hat{\Ba}}
\newcommand{\bcap}[0]{\hat{\Bb}}
\newcommand{\ccap}[0]{\hat{\Bc}}
\newcommand{\dcap}[0]{\hat{\Bd}}
\newcommand{\ecap}[0]{\hat{\Be}}
\newcommand{\fcap}[0]{\hat{\Bf}}
\newcommand{\gcap}[0]{\hat{\Bg}}
\newcommand{\hcap}[0]{\hat{\Bh}}
\newcommand{\icap}[0]{\hat{\Bi}}
\newcommand{\jcap}[0]{\hat{\Bj}}
\newcommand{\kcap}[0]{\hat{\Bk}}
\newcommand{\lcap}[0]{\hat{\Bl}}
\newcommand{\mcap}[0]{\hat{\Bm}}
\newcommand{\ncap}[0]{\hat{\Bn}}
\newcommand{\ocap}[0]{\hat{\Bo}}
\newcommand{\pcap}[0]{\hat{\Bp}}
\newcommand{\qcap}[0]{\hat{\Bq}}
\newcommand{\rcap}[0]{\hat{\Br}}
\newcommand{\scap}[0]{\hat{\Bs}}
\newcommand{\tcap}[0]{\hat{\Bt}}
\newcommand{\ucap}[0]{\hat{\Bu}}
\newcommand{\vcap}[0]{\hat{\Bv}}
\newcommand{\wcap}[0]{\hat{\Bw}}
\newcommand{\xcap}[0]{\hat{\Bx}}
\newcommand{\ycap}[0]{\hat{\By}}
\newcommand{\zcap}[0]{\hat{\Bz}}
\newcommand{\thetacap}[0]{\hat{\Btheta}}

%
% to write R^n and C^n in a distinguishable fashion.  Perhaps change this
% to the double lined characters upon figuring out how to do so.
%
\newcommand{\C}[1]{$\mathbb{C}^{#1}$}
\newcommand{\R}[1]{$\mathbb{R}^{#1}$}

%
% various generally useful helpers
%

% derivative of #1 wrt. #2:
\newcommand{\D}[2] {\frac {d#2} {d#1}}

\newcommand{\inv}[1]{\frac{1}{#1}}
\newcommand{\cross}[0]{\times}

\newcommand{\abs}[1]{\lvert{#1}\rvert}
\newcommand{\norm}[1]{\lVert{#1}\rVert}
\newcommand{\innerprod}[2]{\langle{#1}, {#2}\rangle}
\newcommand{\dotprod}[2]{{#1} \cdot {#2}}
\newcommand{\bdotprod}[2]{\left({#1} \cdot {#2}\right)}
\newcommand{\crossprod}[2]{{#1} \cross {#2}}
\newcommand{\tripleprod}[3]{\dotprod{\left(\crossprod{#1}{#2}\right)}{#3}}

\DeclareMathOperator{\Proj}{Proj}
\DeclareMathOperator{\Span}{span}
\DeclareMathOperator{\Sgn}{sgn}
\DeclareMathOperator{\Area}{Area}
\DeclareMathOperator{\Volume}{Volume}

%
% A few miscellaneous things specific to this document
%
\newcommand{\crossop}[1]{\crossprod{#1}{}}

% R2 vector.
\newcommand{\VectorTwo}[2]{
\begin{bmatrix}
 {#1} \\
 {#2}
\end{bmatrix}
}

\newcommand{\VectorN}[1]{
\begin{bmatrix}
{#1}_1 \\
{#1}_2 \\
\vdots \\
{#1}_N \\
\end{bmatrix}
}

\newcommand{\DETuvij}[4]{
\begin{vmatrix}
 {#1}_{#3} & {#1}_{#4} \\
 {#2}_{#3} & {#2}_{#4}
\end{vmatrix}
}

\newcommand{\DETuvwijk}[6]{
\begin{vmatrix}
 {#1}_{#4} & {#1}_{#5} & {#1}_{#6} \\
 {#2}_{#4} & {#2}_{#5} & {#2}_{#6} \\
 {#3}_{#4} & {#3}_{#5} & {#3}_{#6}
\end{vmatrix}
}

\newcommand{\DETuvwxijkl}[8]{
\begin{vmatrix}
 {#1}_{#5} & {#1}_{#6} & {#1}_{#7} & {#1}_{#8} \\
 {#2}_{#5} & {#2}_{#6} & {#2}_{#7} & {#2}_{#8} \\
 {#3}_{#5} & {#3}_{#6} & {#3}_{#7} & {#3}_{#8} \\
 {#4}_{#5} & {#4}_{#6} & {#4}_{#7} & {#4}_{#8} \\
\end{vmatrix}
}

%\newcommand{\DETuvwxyijklm}[10]{
%\begin{vmatrix}
% {#1}_{#6} & {#1}_{#7} & {#1}_{#8} & {#1}_{#9} & {#1}_{#10} \\
% {#2}_{#6} & {#2}_{#7} & {#2}_{#8} & {#2}_{#9} & {#2}_{#10} \\
% {#3}_{#6} & {#3}_{#7} & {#3}_{#8} & {#3}_{#9} & {#3}_{#10} \\
% {#4}_{#6} & {#4}_{#7} & {#4}_{#8} & {#4}_{#9} & {#4}_{#10} \\
% {#5}_{#6} & {#5}_{#7} & {#5}_{#8} & {#5}_{#9} & {#5}_{#10}
%\end{vmatrix}
%}

% R3 vector.
\newcommand{\VectorThree}[3]{
\begin{bmatrix}
 {#1} \\
 {#2} \\
 {#3}
\end{bmatrix}
}



\author{Peeter Joot}
\email{peeter.joot@gmail.com}


%\chapter{PHY454H1S Continuum Mechanics.  Lecture 15: More on surface tension and Reynold's number.  Taught by Prof. K. Das.}
\chapter{More on surface tension and Reynold's number.}
\label{chap:continuumL15}
%\useCCL
\blogpage{http://sites.google.com/site/peeterjoot2/math2012/continuumL15.pdf}
\date{Mar 9, 2012}
\gitRevisionInfo{continuumL15}

\beginArtWithToc
%\beginArtNoToc

%\section{Disclaimer.}
%
%Peeter's lecture notes from class.  May not be entirely coherent.

\section{Review.  Surface tension clarifications}

For a surface like figure (\ref{fig:continuumL15:continuumL15Fig1})
\begin{figure}[htp]
   \centering
   \includegraphics[totalheight=0.2\textheight]{continuumL15Fig1}
   \caption{Vapor liquid interface.}\label{fig:continuumL15:continuumL15Fig1}
\end{figure}

we have a discontinuous jump in density.  We will have to consider three boundary value constraints

\begin{enumerate}
\item Mass balance.  This is the continuity equation.
\item Momentum balance.  This is the Navier-Stokes equation.
\item Energy balance.  This is the heat equation.
\end{enumerate}

We have not yet discussed the heat equation, but this is required for non-isothermal problems.

The boundary condition at the interface is given by the stress balance.  Denoting the difference in the traction vector at the interface by

\begin{equation}\label{eqn:continuumL15:130}
[\Bt]_1^2 = \Bt_2 - \Bt_1 = -\frac{\sigma}{2 R} \ncap - \spacegrad_I \sigma
\end{equation}

Here the gradient is in the tangential direction of the surface as in figure (\ref{fig:continuumL15:continuumL15Fig2})
\begin{figure}[htp]
   \centering
   \includegraphics[totalheight=0.2\textheight]{continuumL15Fig2}
   \caption{normal and tangent vectors on a curve.}\label{fig:continuumL15:continuumL15Fig2}
\end{figure}

In the normal direction

\begin{align*}
[\Bt]_1^2 \cdot \ncap
&= (\Bt_2 - \Bt_1) \cdot \ncap \\
&= -\frac{\sigma}{2 R} 
\end{align*}

With the traction vector having the value

\begin{align*}
\Bt 
&= \Be_i T_{ij} n_j \\
&= 
\Be_i \left( 
-p \delta_{ij} + \mu \left( 
\PD{x_j}{u_i}
+\PD{x_i}{u_j}
\right)
\right)
n_j
\end{align*}

We have in the normal direction

\begin{equation}\label{eqn:continuumL15:10}
\Bt \cdot \Bn 
=
n_i \left( 
-p \delta_{ij} + \mu \left( 
\PD{x_j}{u_i}
+\PD{x_i}{u_j}
\right)
\right) n_j
\end{equation}

With $\Bu = 0$ on the surface, and $n_i \delta_{ij} n_j = n_j n_j = 1$ we have

\begin{equation}\label{eqn:continuumL15:30}
\Bt \cdot \Bn = -p
\end{equation}

Returning to $(\Bt_2 - \Bt_1) \cdot \ncap$ we have

\begin{equation}\label{eqn:continuumL15:50}
\boxed{
-p_2 + p_1 = -\frac{\sigma}{2 R} 
}
\end{equation}

This is the Laplace pressure.  Note that the sign of the difference is significant, since it effects the direction of the curvature.  This is depicted pictorially in figure (\ref{fig:continuumL15:continuumL15Fig3})
\begin{figure}[htp]
   \centering
   \includegraphics[totalheight=0.2\textheight]{continuumL15Fig3}
   \caption{pressure and curvature relationships}\label{fig:continuumL15:continuumL15Fig3}
\end{figure}

\paragraph{Question} In \cite{landau1987course} the curvature term is written

\begin{equation}\label{eqn:continuumL15:70}
\inv{2 R} \rightarrow \inv{R_1} + \inv{R_2}.
\end{equation}

Why the difference?  Answer: This is to account for non-spherical surfaces, with curvature in two directions.  Illustrating by example, imagine a surface like as in figure (\ref{fig:continuumL15:continuumL15Figq})
\begin{figure}[htp]
   \centering
   \includegraphics[totalheight=0.2\textheight]{continuumL15Figq}
   \caption{Example of non-spherical curvature.}\label{fig:continuumL15:continuumL15Figq}
\end{figure}

\subsection{Followup required to truly understand things}

While, this review clarifies things, we still don't really know how the surface tension term $\sigma$ is defined.  Nor have we been given any sort of derivation of \ref{eqn:continuumL15:130}, from which the end result follows.

I'm assuming that $\sigma$ is a property of the two fluids at the interface, so if you have, for example, oil and vinegar in a bottle, we have surface tension and curvature that's probably related to how the two of these interact.  If there is still a mixing or settling process occurring, I'd imagine that this could even vary from point to point on the surface (imagine adding soap to a surface where stuff can float until the soap mixes in enough that things start sinking in the radius of influence of the soap).

\subsection{Surface tension gradients.}

Now consider the tangential component of the traction vector

\begin{equation}\label{eqn:continuumL15:90}
\Bt_2 \cdot \taucap - \Bt_1 \cdot \taucap = - \cancel{ \frac{\sigma}{2 R} \ncap \cdot \taucap} - \taucap \cdot \spacegrad_I \sigma
\end{equation}

So we see that for a static fluid, we must have

\begin{equation}\label{eqn:continuumL15:110}
\spacegrad_I \sigma = 0
\end{equation}

For a static interface there cannot be any surface tension gradient.  This becomes very important when considering stability issues.  We can have surface tension induced flow called capillary, or mandarin (?) flow.

\section{Reynold's number.}

In Navier-Stokes after making non-dimensionalization changes of the form

\begin{equation}\label{eqn:continuumL15:150}
x \rightarrow L x'
\end{equation}

the control parameter is like Reynold's number.

In NS

\begin{equation}\label{eqn:continuumL15:170}
\rho \PD{t}{\Bu} + \rho ( \Bu \cdot \spacegrad ) \Bu = - \spacegrad p + \mu \spacegrad^2 \Bu
\end{equation}

We call the term

\begin{equation}\label{eqn:continuumL15:190}
\rho ( \Bu \cdot \spacegrad ) \Bu
\end{equation}

the inertial term.  It is non-zero only when something is being ``carried along with the velocity''.  Consider a volume fixed in space and one that is moving along with the fluid as in figure (\ref{fig:continuumL15:continuumL15Fig4})
\begin{figure}[htp]
   \centering
   \includegraphics[totalheight=0.2\textheight]{continuumL15Fig4}
   \caption{Moving and fixed frame control volumes in a fluid.}\label{fig:continuumL15:continuumL15Fig4}
\end{figure}

All of our viscosity dependence shows up in the Laplacian term, so we can roughly characterize the Reynold's number as the ratio

\begin{align*}
\text{Reynold's number}
&\rightarrow
\frac{\Abs{\text{effect of inertia}}}{\Abs{\text{effect of viscosity}}}  \\
&=
\frac{ \Abs{\rho ( \Bu \cdot \spacegrad ) \Bu } }{\Abs{ \mu \spacegrad^2 \Bu}} \\
&\sim
\frac{ \rho U^2/L }{\mu U/L^2} \\
&\sim
\frac{ \rho U L }{\mu }
\end{align*}

In figures (\ref{fig:continuumL15:continuumL15Fig5}), (\ref{fig:continuumL15:continuumL15Fig6}) we have two illustrations of viscous and non-viscous regions the first with a moving probe pushing its way through a surface, and the second with a wing set at an angle of attack that generates some turbulence.  Both are illustrations of the viscous and inviscous regions for the two flows.  Both of these are characterized by the Reynold's number in some way not really specified in class.  One of the points of mentioning this is that when we are in an essentially inviscous region, we can neglect the viscosity ($\mu \spacegrad^2 \Bu$) term of the flow.
\begin{figure}[htp]
   \centering
   \includegraphics[totalheight=0.2\textheight]{continuumL15Fig5}
   \caption{continuumL15Fig5}\label{fig:continuumL15:continuumL15Fig5}
\end{figure}
\begin{figure}[htp]
   \centering
   \includegraphics[totalheight=0.2\textheight]{continuumL15Fig6}
   \caption{continuumL15Fig6}\label{fig:continuumL15:continuumL15Fig6}
\end{figure}

%FIXME: Reading: \S XX from \cite{acheson1990elementary}

\EndArticle
