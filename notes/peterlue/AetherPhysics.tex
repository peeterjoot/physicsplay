% Copyright � 2012 Peter Lue.  All Rights Reserved.
%
%
%-----------------------------------------------------------------------------------
% journal style swiped from:
%
% http://arxiv.org/abs/1006.3552

% this one is a double column compact form.
%\documentclass[iop]{emulateapj}
% slightly less space used with the tighten option.
\documentclass[iop,tighten]{emulateapj}
%\documentclass[iop,onecolumn]{emulateapj}
%
% http://www.tug.org/texlive/Contents/live/texmf-dist/doc/latex/aastex/aasguide.pdf
% full width variation:
%\documentclass[12pt,preprint]{aastex}
% double column variation, not as compact as the emulateapj above
%\documentclass[12pt,preprint2]{aastex}

\usepackage{amsmath}
\usepackage{mathpazo}
\usepackage{cancel}
%\newcommand{\Bx}[0]{\mathbf{x}}
%\newcommand{\Bv}[0]{\mathbf{v}}
%\newcommand{\By}[0]{\mathbf{y}}
%\newcommand{\T}[0]{{\text{T}}}
%\DeclareMathOperator{\Proj}{Proj}
%\newcommand{\Abs}[1]{{\left\lvert{#1}\right\rvert}}
%\newcommand{\inv}[1]{\frac{1}{#1}}
%\newcommand{\Norm}[1]{\left\lVert{#1}\right\rVert}

\newcommand{\pref}[1]{(\ref{#1})}

\label{chap:AetherPhysics}
%-----------------------------------------------------------------------------------

%\journalinfo{ApJ Letters, in press}
%\submitted{}
%\received{June 16, 2010} \accepted{October 20, 2010}
%\shorttitle{short title goes here.}

% these two need generalization.  have to switch order with journal format.
\begin{document}
\title{Aether Physics}

\author{Peter Lue \altaffilmark{1}}
\altaffiltext{1}{peeter.lue@mail.utoronto.ca}

\begin{abstract}

abstract contents.  blah blah blah.

\end{abstract}

%\keywords{Lorentz boost, change of basis, special relativity, inertial frame, reciprocal basis, dual vector, Gram-Schmidt orthonormalization.}

\section{Abstract}

%\section{Introduction}

\section{Preliminaries, notation, and definitions}

\section{Guts}

% Licenced as described in the file LICENSE under the root directory of this GIT repository.

Einsteins theories have proven to be very compelling due to their predictability. They predicted the
matter energy conversion formula and time dilations. Given the accuracy of these two predictions,
many people have had a tremendous amount of hesitation in disregarding his theories in the face of a
remarkable amount of evidence which indicates that they were wrong.
It turns out that the reason for these apparent accuracies in Einsteins predictions was a tremendous
coincidence. The formula for matter conversion happens to be the correct formula. Einstein however
got the correct formula the wrong way. The extensions of his concepts have lead to a situation in which
we now have two competing theories in Relativity and Quantum Mechanics. We still have not found
the elusive graviton after almost one hundred years and we have come to many road blocks in our
understanding.
Through the reexamination of scientific experiments of the past one hundred years, it is possible to
come to a different conclusion based upon observed phenomena which not only yield correct answers
in more instances, but also shed light on to some of the most difficult mysteries in science today.
It appears that the universe is filled with a luminiferous aether, the particles are in contact and they
move like great sheets or in tandem. For this reason, the drift of the aether has no affect on the direction
of the propagation of light in the Michelson-Morley experiment. Time dilations appear to be caused by
density gradients in this luminiferous aether.
Matter or the particles which comprise atoms appear to be formed in supernovae as the energy of the
explosion exceeds the capacity of the luminiferous aether to carry it. Matter particles are spinning
particles of aether that spin at or above the speed of light. Spinning particles of aether create density
gradients around them. Electronic charges are caused by density gradients of the low density spheres
ejected by the low density areas surrounding atoms. Atoms are surrounded by laminar layers of
gradient aether density which explains chemical bonding and band theory. If electrons are low density
spheres of aether ejected from spinning particles of matter, proof of this theory can be found in the
seasonal disturbances in the earths magnetic field and Coulombs Law . If matter is composed of
spinning aether particle, then the strength of magnetic fields would vary with the speed one travels
through the aether. In addition, the electrostatic constant or Coulomb force constant in coulombs law
would vary with ones speed traveling though the aether so as one leaves the earth or changes altitude,
the constant should change much in the same way time dilations change due to gravitational potential
and not acceleration due to gravity as Einstein asserted through his invocation of the equivalence
principal.
In the Aether of Time I showed that time dilations are in fact due to density gradients in the
luminiferous aether.
In Circular Reasoning I postulate that matter is composed of spinning particles of aether and the
correct formula is:
%FIXME:
%E=mc��
And in Matter of Time I show that:
%FIXME:
%c=��
Therefore Einstein got the correct formula the wrong way. I also assert that gravity is caused by density
gradients in the luminiferous aether.
It appears as though the aether has a mechanical "stickiness" to it that allows them to spin in groups
and stay cohesive at speeds faster than the speed of light and move in sheets who's density can change
maybe due to some sort of springiness.
The universe is also not expanding and the background radiation is caused by the friction between
aether particles. This results in the wavelength increasing as a function of the distance from the source.
Hence, there was no big bang. The presence of a luminiferous aether makes an expanding universe
unlikely.
If our perception of acceleration is due to affects from the ripping of the aether sheets, a sphere
spinning fast enough to be in laminar flow with the outside aether and an inner aether would protect the
occupants in the inner sphere from the sense of acceleration. Also a dense aether would make
propulsion especially in space very efficient if an aether jet were made by spinning a tube fast enough
to disrupt the aether laminar properties inside another tube.

\section{Conclusion}

% uncomment for bib and create myrefs.bib appropriately:
%\bibliography{myrefs}\bibliographystyle{unsrtnat}

\end{document}
