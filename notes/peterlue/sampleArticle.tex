%
% Copyright � 2013 Peeter Joot.  All Rights Reserved.
% Licenced as described in the file LICENSE under the root directory of this GIT repository.
%
% comments start with percent characters.
% every latex doc starts with a document class.  This is a common one:
\documentclass{article}

% Here's an optional package:
\usepackage{amsmath}
% (it is used in this sample for \begin{align}...\end{align} (multi-line equations).)

% optional:
\author{who am I}

% every latex document has a \begin{document} ... \end{document}
\begin{document}
% if you want a title in the article class you name it this way:
\title{my title}

% With the following command to make it show up:
\maketitle

% you may use sections and subsections in many document classes:
\section{section header one}
\subsection{subsection header one}

The quick brown fox ran after the lazy dog.
The quick brown fox ran after the lazy dog.
The quick brown fox ran after the lazy dog.
The quick brown fox ran after the lazy dog.
The quick brown fox ran after the lazy dog.
The quick brown fox ran after the lazy dog.
The quick brown fox ran after the lazy dog.

Paragraph breaks require a blank line between blocks of text.  Any other white space is ignored.

\section{section header two}
\subsection{subsection header two}

More stuff.

\subsection{subsection header three}

Even more stuff.  Here's some basic math formatting.  Let

\begin{equation}\label{eqn:mysample:10}
p_\mu = m \gamma (c, \mathbf{v}).
\end{equation}

Forming the contraction of the components of this vector with itself we find the expected invariant
\begin{equation}\label{eqn:sampleArticle:30}
\begin{aligned}
p_\mu p^\mu
&= m^2 \gamma^2 (c^2 - v^2) \\
&= m^2 \frac{1}{1 - v^2/c^2} (c^2 - v^2) \\
&= m^2 c^2.
\end{aligned}
\end{equation}
% align is used for multi-line equations (with \\ marking line breaks and & marking the portion of the line to line everything up on)
% the asteric used here with align* is for an unnumbered equation.

You can also do inline equations with dollar signs to delimit them like this $p^\mu p_\mu = m^2 c^2$.

We can refer to equations with references like this \ref{eqn:mysample:10}.  You'll need to run pdflatex twice to resolve the equation numbers in the references (you'll see 'There are undefined references' in the log file to tell you to do so).

\end{document}
