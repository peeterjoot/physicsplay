%
% Copyright � 2012 Peeter Joot.  All Rights Reserved.
% Licenced as described in the file LICENSE under the root directory of this GIT repository.
%

%
%
\section{Normals and tangents at fluid interfaces} \index{normal} \index{tangent} \index{fluid interface}

%The Navier-Stokes equation (our fluids equivalent to Newton's second law) was found to be
%
%\begin{equation}\label{eqn:continuumL11:10}
%\rho \PD{t}{\Bu} + \rho (\Bu \cdot \spacegrad) \Bu = - \spacegrad p + \mu \spacegrad^2 \Bu + \rho \Bf.
%\end{equation}
%
%In this course we will focus on the incompressible case where we have
%
%\begin{equation}\label{eqn:continuumL11:30}
%\spacegrad \cdot \Bu = 0
%\end{equation}

We watched a video of the rocking tank as in \cref{fig:continuumL11:continuumL11fig1}.  The boundary condition that accounted for the matching of the die marker is that we have \textunderline{no slipping} at the interface.  Writing \(\taucap\) for the unit tangent to the interface then this condition at the interface is described mathematically by the conditions

\imageFigure{../../figures/phy454/lec11_Rocking_tank_velocity_matchingFig1}{Rocking tank velocity matching}{fig:continuumL11:continuumL11fig1}{0.3}

\begin{equation}\label{eqn:continuumL11:70}
\begin{aligned}
\Bu_A \cdot \taucap &= \Bu_B \cdot \taucap \\
\Bu_A \cdot \ncap &= \Bu_B \cdot \ncap.
\end{aligned}
\end{equation}

Referring to \cref{fig:continuumL11:continuumL11fig2} where the tangents and normals are depicted an example representation of the normal and tangent vectors for the fluids are

\imageFigure{../../figures/phy454/lec11_Normals_and_tangents_at_interface_for_2D_systemFig2}{Normals and tangents at interface for 2D system}{fig:continuumL11:continuumL11fig2}{0.2}

\begin{equation}\label{eqn:continuumL11:90}
\begin{aligned}
\taucap &=
\begin{bmatrix}
1 \\
0
\end{bmatrix} \\
\ncap &=
\begin{bmatrix}
0 \\
1
\end{bmatrix}
\end{aligned}
\end{equation}

For the traction vector with components

\begin{equation}\label{eqn:continuumL11:110}
T_i = \sigma_{ij} n_j,
\end{equation}

we also have at the interface we must have matching of

\begin{equation}\label{eqn:continuumL11:130}
\taucap \cdot \BT.
\end{equation}

More explicitly, in coordinates this is

\begin{equation}\label{eqn:continuumL11:150}
\evalbar{\tau_i (\sigma_{ij} n_j)}{A} =
\evalbar{\tau_i (\sigma_{ij} n_j)}{B}
\end{equation}

\makeexample{Steady incompressible rectilinear (unidirectional) flow}{ex:fluids:rectilinear}{

In this case we can fix our axis so that

\begin{equation}\label{eqn:continuumL11:170}
\Bu = \xcap u(x, y, z, t),
\end{equation}

where the velocity components in the other directions

\begin{equation}\label{eqn:continuumL11:190}
\begin{aligned}
v &= 0 \\
w &= 0
\end{aligned}
\end{equation}

are both zero.  Symbolically, the steady state condition is

\begin{equation}\label{eqn:continuumL11:210}
\PD{t}{\Bu} = 0.
\end{equation}

% prof called this the continuity condition?
We start with the incompressibility condition, which written explicitly, is

\begin{equation}\label{eqn:continuumL11:230}
\spacegrad \cdot \Bu = 0,
\end{equation}

or

\begin{equation}\label{eqn:continuumL11:250}
\PD{x}{u} + \PD{y}{v} + \PD{z}{w} = 0
\end{equation}

This implies

\begin{equation}\label{eqn:continuumL11:270}
\PD{x}{u} = 0
\end{equation}

so our velocity can only be function of the \(y\) and \(z\) coordinates only

\begin{equation}\label{eqn:continuumL11:290}
u = u(y, z).
\end{equation}

The non-linear term of the Navier-Stokes equation takes the form

\begin{equation}\label{eqn:unidirectionalSolutions:870}
\begin{aligned}
(\Bu \cdot \spacegrad) \Bu
&=
\left(u \PD{x}{}
+\cancel{v} \PD{y}{}
+\cancel{w} \PD{z}{} \right) (\xcap u( y, z) + \ycap \cancel{v} + \zcap \cancel{w} ) \\
&=
\xcap u \cancel{\PD{x}{u}} \\
&= 0.
\end{aligned}
\end{equation}

With incompressibility and \(u = v = 0\) conditions killing this term, and the steady state condition \eqnref{eqn:continuumL11:210} killing the \(\rho \PDi{t}{\Bu}\) term, the Navier-Stokes equation for this incompressible unidirectional steady state flow (in the absence of body forces) is reduced to

\begin{equation}\label{eqn:continuumL11:310}
0 = - \spacegrad p + \mu \spacegrad^2 \Bu.
\end{equation}

In coordinates this is

\begin{subequations}
\begin{equation}\label{eqn:continuumL11:330a}
\PD{x}{p} = \mu \left( \PDSq{y}{u} + \PDSq{z}{u} \right)
\end{equation}
\begin{equation}\label{eqn:continuumL11:330b}
\PD{y}{p} = 0
\end{equation}
\begin{equation}\label{eqn:continuumL11:330c}
\PD{z}{p} = 0.
\end{equation}
\end{subequations}

Operating on the first with an x partial we find

\begin{equation}\label{eqn:continuumL11:350}
\PDSq{x}{p} = \mu \left( \PDSq{y}{} \cancel{\PD{x}{u}} + \PDSq{z}{} \cancel{ \PD{x}{u} } \right) = 0
\end{equation}

Since we have

\begin{equation}\label{eqn:continuumL11:370}
\PDSq{x}{p} = 0
\end{equation}

we also have

\begin{equation}\label{eqn:continuumL11:390}
\frac{d^2 p}{dx^2} = 0,
\end{equation}

so our pressure must be linear with position

\begin{equation}\label{eqn:continuumL11:410}
p = A x + B,
\end{equation}

as illustrated in \cref{fig:continuumL11:continuumL11fig3}

\imageFigure{../../figures/phy454/lec11_Pressure_gradient_in_1D_systemFig3}{Pressure gradient in 1D system}{fig:continuumL11:continuumL11fig3}{0.2}

\begin{equation}\label{eqn:continuumL11:430}
p =
\left\{
\begin{array}{l l}
p_0 & \quad \mbox{\(x = 0\)} \\
p_L & \quad \mbox{\(x = L\)}
\end{array}
\right.
%} % vim brace matching.
\end{equation}

we have

\begin{equation}\label{eqn:continuumL11:450}
p = \frac{p_L - p_0}{L} x + p_0
\end{equation}

and

\begin{equation}\label{eqn:continuumL11:470}
\frac{dp}{dx} = \frac{p_L - p_0}{L} = \text{constant} \equiv -G
\end{equation}
} % end example

\makeexample{Shearing flow}{ex:fluids:shearing}{

The flows of this sort do not have to be trivial.  For example, even with constant pressure (\(p_0 = p_L\)) as in \cref{fig:continuumL11:continuumL11fig4} we can have a ``shearing flow'' where the fluids at the top surface are not necessarily moving at the same rates as the fluid below that surface.  We have fluid flow in the \(x\) direction only, and our velocity is a function only of the \(y\) coordinate.

\imageFigure{../../figures/phy454/lec11_Velocity_variation_with_height_in_shearing_flowFig4}{Velocity variation with height in shearing flow}{fig:continuumL11:continuumL11fig4}{0.2}

\begin{equation}\label{eqn:continuumL11:490}
\begin{aligned}
\Bu &= \xcap u(y) \\
G &= 0 \\
u(0) &= 0 \\
u(h) &= U.
\end{aligned}
\end{equation}

For such a flow \eqnref{eqn:continuumL11:330a} simplifies to

\begin{equation}\label{eqn:continuumL11:530}
\frac{d^2 u}{dy^2} = 0
\end{equation}

with solution

\begin{equation}\label{eqn:continuumL11:550}
u = \frac{U}{h} y + u(0) = \frac{U}{h} y.
\end{equation}
} % end example

\makeexample{Channel flow}{ex:fluids:channel}{

\begin{equation}\label{eqn:continuumL11:570}
\begin{aligned}
\Bu &= \xcap u(y) \\
G &= - \frac{dp}{dx} \ne 0
\end{aligned}
\end{equation}

This time our simplified Navier-Stokes equation \eqnref{eqn:continuumL11:330a} is reduced to something slightly more complicated

\begin{equation}\label{eqn:continuumL11:590}
\mu \frac{d^2 u}{dy^2} = -G,
\end{equation}

with solution

\begin{equation}\label{eqn:continuumL11:610}
u = -\frac{G}{2 \mu} y^2 + A y + B.
\end{equation}

The boundary value conditions with the coordinate system in use illustrated in \cref{fig:continuumL11:continuumL11fig5} require the velocity to be zero at the interface (the pipe walls preventing flow in the interior of the pipe)

\imageFigure{../../figures/phy454/lec11_1D_Channel_flow_coordinate_system_setupFig5}{1D Channel flow coordinate system setup}{fig:continuumL11:continuumL11fig5}{0.2}

\begin{equation}\label{eqn:continuumL11:630}
u(\pm h) =
-\frac{G}{2 \mu} h^2 \pm A h + B = 0
\end{equation}

One solution, immediately evident is,

\begin{equation}\label{eqn:continuumL11:650}
\begin{aligned}
A &= 0 \\
B &= \frac{G}{2 \mu} h^2,
\end{aligned}
\end{equation}

so our solution becomes

\begin{equation}\label{eqn:continuumL11:690}
u = \frac{G}{2 \mu} \Bigl( h^2 - y^2 \Bigr),
\end{equation}

a parabolic velocity flow.  This is illustrated graphically in \cref{fig:continuumL11:continuumL11fig6}.

\imageFigure{../../figures/phy454/lec11_Parabolic_velocity_distributionFig6}{Parabolic velocity distribution}{fig:continuumL11:continuumL11fig6}{0.2}

It is clear that this is maximized by \(y = 0\), but we can also see this by computing

\begin{equation}\label{eqn:continuumL11:710}
\frac{du}{dy} = \frac{G}{\mu} y = 0.
\end{equation}

This maximum is

\begin{equation}\label{eqn:continuumL11:850}
u_{\text{max}} = \frac{G}{2\mu} h^2
\end{equation}

The flux, or flow rate is

\begin{equation}\label{eqn:unidirectionalSolutions:890}
\begin{aligned}
Q
&= \iint_S \Bu \cdot \xcap ds \\
&= \int_0^1 dz \int_{-h}^h dy u(y) \\
&=
\frac{2 G h^3}{3}
\end{aligned}
\end{equation}

Let us now compute the strain (\(e_{ij}\)) and the stress (\(\sigma_{ij} = -p \delta_{ij} + 2 \mu e_{ij}\))

\begin{equation}\label{eqn:continuumL11:730}
\begin{aligned}
e_{12} &= e_{21} = \inv{2} \left( \PD{y}{u} \right) = - \frac{G y}{2 \mu} \\
e_{11} &= \PD{x}{u} = 0 \\
e_{22} &= \PD{y}{v} = 0
\end{aligned}
\end{equation}

stress

\begin{equation}\label{eqn:continuumL11:750}
\sigma_{12} = 2 \mu e_{12} = -G y
\end{equation}

This can be used to compute the forces on the inner surfaces of the tube.  As illustrated in \cref{fig:continuumL11:continuumL11fig7}, our normals at \(\pm h\) are \(\mp \ycap\) respectively.  The traction vector in the \(y\) direction is at \(y = h\) is

\imageFigure{../../figures/phy454/lec11_Normals_in_1D_channel_flow_systemFig7}{Normals in 1D channel flow system}{fig:continuumL11:continuumL11fig7}{0.2}

\begin{equation}\label{eqn:continuumL11:770}
\tau_i = \sigma_{i 2} \evalbar{n_2}{y = h} = G h,
\end{equation}

so that

\begin{equation}\label{eqn:continuumL11:791}
\Btau = \xcap G h
\end{equation}

(here the \(x\) directionality comes from the \(i = 1\) index of the stress tensor).

\begin{equation}\label{eqn:continuumL11:790}
F_{x_L} = \xcap \cdot \Btau = G h
\end{equation}

The total force is then

\begin{equation}\label{eqn:continuumL11:810}
\int_0^L F_x dx = + G h L
\end{equation}

\FIXME{Ask in class.  This is the tangential force at the boundary of the wall.  What is it a force on?  If it is tangential, how can it act on the wall?  It could act on an impediment placed right up next to the wall, but if that is the case, why are we integrating from \(x = 0\) to \(x = L\)?}
} % end example
