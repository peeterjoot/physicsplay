%
% Copyright � 2012 Peeter Joot.  All Rights Reserved.
% Licenced as described in the file LICENSE under the root directory of this GIT repository.
%

%
%
\chapter{First order Fourier transform solution of Maxwell's equation}\label{chap:PJfirstOrderMaxwell}
\index{Maxwell's equation!Fourier transform}
%\date{Jan 31, 2009.  firstorderFourierMaxwell.tex}

\section{Motivation}

In \chapcite{PJfourierMaxwellSecondOrder} solutions of Maxwell's equation
via Fourier transformation of the four potential forced wave equations were
explored.

Here a first order solution is attempted, by directly Fourier transforming
the Maxwell's equation in bivector form.

\section{Setup}

Again using a 3D spatial Fourier transform, we want to put Maxwell's equation into an explicit time dependent form, and can do so by
premultiplying by our observer's time basis vector \(\gamma_0\)

\begin{equation}\label{eqn:firstorderFourierMaxwell:20}
\begin{aligned}
\gamma_0 \grad F &= \gamma_0 \frac{J }{\epsilon_0 c}
\end{aligned}
\end{equation}

On the left hand side we have
\begin{equation}\label{eqn:firstorderFourierMaxwell:40}
\begin{aligned}
\gamma_0 \grad
&= \gamma_0 \left( \gamma^0 \partial_0 + \gamma^k \partial_k \right) \\
&= \partial_0 - \gamma^k \gamma_0 \partial_k \\
&= \partial_0 + \sigma^k \partial_k \\
&= \partial_0 + \spacegrad \\
\end{aligned}
\end{equation}

and on the right hand side we have
\begin{equation}\label{eqn:firstorderFourierMaxwell:60}
\begin{aligned}
\gamma_0 \frac{J }{\epsilon_0 c}
&= \gamma_0 \frac{c \rho \gamma_0 + J^k \gamma_k }{\epsilon_0 c} \\
&= \frac{c \rho - J^k \sigma_k }{\epsilon_0 c} \\
&= \frac{\rho}{\epsilon_0} - \frac{\Bj}{\epsilon_0 c} \\
\end{aligned}
\end{equation}

Both the spacetime gradient and the current density four vector have been put in a quaternion-ic form with scalar and bivector grades in the
STA basis.  This leaves us with the time centric formulation of Maxwell's equation

\begin{equation}\label{eqn:firstorderFourierMaxwell:80}
\begin{aligned}
\left(\partial_0 + \spacegrad\right) F &= \frac{\rho}{\epsilon_0} - \frac{\Bj}{\epsilon_0 c}
\end{aligned}
\end{equation}

Except for the fact that we have objects of various grades here, and that this is a first instead of second order equation,
these equations have the same form as in the previous Fourier transform attacks.
Those were Fourier transform application for the homogeneous and inhomogeneous wave equations, and the heat and
Schr\"{o}dinger equation.

\section{Fourier transforming a mixed grade object}

Now, here we make the assumption that we can apply 3D Fourier transform pairs to mixed grade objects, as in

\begin{equation}\label{eqn:firstorder_fourier_maxwell:FourierTxDefinition}
\begin{aligned}
\hat{\psi}(\Bk, t) &= \inv{(\sqrt{2\pi})^3} \IIinf \psi(\Bx, t) \exp\left( -i \Bk \cdot \Bx \right) d^3 x \\
{\psi}(\Bx, t) &= \PV \inv{(\sqrt{2\pi})^3} \IIinf \hat{\psi}(\Bk, t) \exp\left( i \Bk \cdot \Bx \right) d^3 k
\end{aligned}
\end{equation}

Now, because of linearity, is it clear enough that this will work, provided this is a valid transform pair for any specific grade.
We do however want to be careful of the order of the factors since we want the flexibility to use any particular convenient representation
of \(i\), in particular \(i = \gamma_0 \gamma_1 \gamma_2 \gamma_3 = \sigma_1 \sigma_2 \sigma_3\).

Let us repeat our an ad-hoc verification that this transform pair works as desired, being careful with the order of products and specifically
allowing for \(\psi\) to be a non-scalar function.
Writing \(\Bk = k_m \sigma^m\), \(\Br = \sigma_m r^m\), \(\Bx = \sigma_m x^m\), that is an expansion of

\begin{equation}\label{eqn:firstorderFourierMaxwell:100}
\begin{aligned}
\PV &\inv{(\sqrt{2\pi})^3} \int
\left( \inv{(\sqrt{2\pi})^3} \int \psi(\Br, t) \exp\left( -i \Bk \cdot \Br \right) d^3 r \right)
\exp\left( i \Bk \cdot \Bx \right) d^3 k \\
&= \int \psi(\Br, t) d^3 r \PV \inv{({2\pi})^3} \int \exp\left( i \Bk \cdot (\Bx -\Br) \right) d^3 k \\
&= \int \psi(\Br, t) d^3 r \prod_{m=1}^3 \PV \inv{{2\pi}} \int \exp\left( i k_m (x^m -r^m) \right) dk_m \\
&= \int \psi(\Br, t) d^3 r \prod_{m=1}^3 \lim_{R\rightarrow \infty} \frac{\sin\left( R (x^m -r^m) \right)}{\pi(x^m - r^m)} \\
&\sim \int \psi(\Br, t) \delta(x^1-r^1) \delta(x^2-r^2) \delta(x^3-r^3) d^3 r \\
&= \psi(\Bx, t)
\end{aligned}
\end{equation}

In the second last step above we make the ad-hoc identification of that \(\sinc\) limit with the Dirac delta function, and recover
our original function as desired (the Rigor police are on holiday again).

\subsection{Rotor form of the Fourier transform?}

Although the formulation picked above appears to work, it is not the only choice to potentially make for the Fourier transform
of multivector.  Would it be more natural to pick an explicit Rotor formulation?  This perhaps makes more sense since it is then
automatically grade preserving.

\begin{equation}\label{eqn:firstorder_fourier_maxwell:rotorFourier}
\begin{aligned}
\hat{\psi}(\Bk, t) &= \inv{(\sqrt{2\pi})^n} \IIinf \exp\left( \inv{2} i \Bk \cdot \Bx \right) \psi(\Bx, t) \exp\left( - \inv{2} i \Bk \cdot \Bx \right) d^n x \\
{\psi}(\Bx, t) &= \PV \inv{(\sqrt{2\pi})^n} \IIinf \exp\left( -\inv{2} i \Bk \cdot \Bx \right) \hat{\psi}(\Bk, t) \exp\left( \inv{2} i \Bk \cdot \Bx \right) d^n k
\end{aligned}
\end{equation}

This is not a moot question since I later
tried to make an assumption that the grade of a transformed object equals the original grade.  That does not work with the
Fourier transform definition that has been picked in \eqnref{eqn:firstorder_fourier_maxwell:FourierTxDefinition}.  It may be necessary to revamp the complete treatment, but
for now at least an observation that the grades of transform pairs do not necessarily match is required.

Does the transform pair work?  For the \(n=1\) case this is

\begin{equation}\label{eqn:firstorderFourierMaxwell:120}
\begin{aligned}
\calF(f) = \hat{f}(k) &= \inv{\sqrt{2\pi}} \IIinf \exp\left( \inv{2} i k x \right) f(x) \exp\left( - \inv{2} i k x \right) dx \\
\calF^{-1}(\hat{f}) = {f}(x) &= \PV \inv{\sqrt{2\pi}} \IIinf \exp\left( -\inv{2} i k x \right) \hat{f}(k) \exp\left( \inv{2} i k x \right) dk
\end{aligned}
\end{equation}

Will the computation of \(\calF^{-1}(\calF(f(x)))\) produce \(f(x)\)?  Let us try

\begin{equation}\label{eqn:firstorderFourierMaxwell:140}
\begin{aligned}
&\calF^{-1}(\calF(f(x))) \\
&=
\PV \inv{{2\pi}} \IIinf \exp\left( -\inv{2} i k x \right)
\left(
\IIinf \exp\left( \inv{2} i k u \right) f(u) \exp\left( - \inv{2} i k u \right) du
\right)
\exp\left( \inv{2} i k x \right) dk \\
&=
\PV \inv{{2\pi}} \IIinf \exp\left( -\inv{2} i k (x -u) \right) f(u) \exp\left( \inv{2} i k (x -u) \right) du dk \\
\end{aligned}
\end{equation}

Now, this expression can not obviously be identified with the delta function as in the single sided transformation.  Suppose we decompose \(f\) into grades that
commute and anticommute with \(i\).  That is

\begin{equation}\label{eqn:firstorderFourierMaxwell:160}
\begin{aligned}
f &= f_\parallel + f_\perp \\
f_\parallel i &= i f_\parallel  \\
f_\perp  i &= -i f_\perp
\end{aligned}
\end{equation}

This is also sufficient to determine how these components of \(f\) behave with respect to the exponentials.  We have

\begin{equation}\label{eqn:firstorderFourierMaxwell:180}
\begin{aligned}
e^{i\theta}
&= \sum_m \frac{(i\theta)^m}{m!} \\
&= \cos(\theta) + i\sin(\theta)
\end{aligned}
\end{equation}

So we also have

\begin{equation}\label{eqn:firstorderFourierMaxwell:200}
\begin{aligned}
f_\parallel e^{i\theta} &= e^{i\theta} f_\parallel  \\
f_\perp e^{i\theta} &= e^{-i\theta} f_\perp
\end{aligned}
\end{equation}

This gives us
\begin{equation}\label{eqn:firstorderFourierMaxwell:220}
\begin{aligned}
\calF^{-1}(\calF(f(x)))
&=
\PV \inv{{2\pi}} \IIinf f_\parallel(u) du dk
+\PV \inv{{2\pi}} \IIinf f_\perp(u) \exp\left( i k (x -u) \right) du dk \\
&=
\inv{{2\pi}} \IIinf dk \IIinf f_\parallel(u) du +\IIinf f_\perp(u) \delta( x -u ) du \\
\end{aligned}
\end{equation}

So, we see two things.  First is that any \(f_\parallel \ne 0\) produces an unpleasant infinite result.  One could, in a vague sense, allow for odd valued \(f_\parallel\), however, if we were to apply this inversion transformation pair to a function time varying multivector function \(f(x,t)\), this would then require that the function is odd for all times.  Such a function must be zero valued. % or some odd construction such as a function that is zero everywhere except at some denumerable set of points.

The second thing that we see is that if \(f\) entirely anticommutes with \(i\), we do recover it with this transform pair, obtaining \(f_\perp(x)\).

With respect to Maxwell's equation
this immediately means that this double sided transform pair is of no use, since our pseudoscalar \(i = \gamma_0 \gamma_1\gamma_2 \gamma_3\) commutes with our grade two
field bivector \(F\).

\section{Fourier transforming the spacetime split gradient equation}

Now, suppose we have a Maxwell like equation of the form

\begin{equation}\label{eqn:firstorder_fourier_maxwell:spacetimeGradientEquation}
\begin{aligned}
\left(\partial_0 + \spacegrad \right) \psi = g
\end{aligned}
\end{equation}

Let us take the Fourier transform of this equation.  This gives us

\begin{equation}\label{eqn:firstorderFourierMaxwell:240}
\begin{aligned}
\partial_0 \hat{\psi} + \sigma^m \calF(\partial_m \psi) = \hat{g}
\end{aligned}
\end{equation}

Now, we need to look at the middle term in a bit more detail.  For the wave, and heat equations this was evaluated with just an integration
by parts.  Was there any commutation assumption in that previous treatment?  Let us write this out in full to make sure we are cool.

\begin{equation}\label{eqn:firstorderFourierMaxwell:260}
\begin{aligned}
\calF(\partial_m \psi)
&= \inv{(\sqrt{2\pi})^3} \int \left(\partial_m \psi(\Bx, t)\right) \exp\left( -i \Bk \cdot \Bx \right) d^3 x
\end{aligned}
\end{equation}

Let us also expand the integral completely, employing a permutation of indices \(\pi(1,2,3) = (m,n,p)\).

\begin{equation}\label{eqn:firstorderFourierMaxwell:280}
\begin{aligned}
\calF(\partial_m \psi)
&=
\inv{(\sqrt{2\pi})^3}
\int_{\partial x^p} dx^p
\int_{\partial x^n} dx^n
\int_{\partial x^m} dx^m
\left(\partial_m \psi(\Bx, t)\right) \exp\left( -i \Bk \cdot \Bx \right) \\
\end{aligned}
\end{equation}

Okay, now we are ready for the integration by parts.  We want a derivative substitution, based on

\begin{equation}\label{eqn:firstorderFourierMaxwell:300}
\begin{aligned}
\partial_m &\left( \psi(\Bx, t) \exp\left( -i \Bk \cdot \Bx \right) \right) \\
&= (\partial_m \psi(\Bx, t)) \exp\left( -i \Bk \cdot \Bx \right) + \psi(\Bx, t) \partial_m \exp\left( -i \Bk \cdot \Bx \right) \\
&= (\partial_m \psi(\Bx, t)) \exp\left( -i \Bk \cdot \Bx \right) + \psi(\Bx, t) ( -i k_m ) \exp\left( -i \Bk \cdot \Bx \right) \\
\end{aligned}
\end{equation}

Observe that we do not wish to assume that the pseudoscalar \(i\) commutes with anything except the exponential term, so we have to leave
it sandwiched or on the far right.  We also must take care to not necessarily commute the exponential itself with \(\psi\) or its derivative.
Having noted this we can rearrange as desired for the integration by parts

\begin{equation}\label{eqn:firstorderFourierMaxwell:320}
\begin{aligned}
(\partial_m \psi(\Bx, t)) \exp\left( -i \Bk \cdot \Bx \right)
&=
\partial_m \left( \psi(\Bx, t) \exp\left( -i \Bk \cdot \Bx \right) \right) - \psi(\Bx, t) ( -i k_m ) \exp\left( -i \Bk \cdot \Bx \right) \\
\end{aligned}
\end{equation}

and substitute back into the integral

\begin{equation}\label{eqn:firstorderFourierMaxwell:340}
\begin{aligned}
\sigma^m \calF(\partial_m \psi)
&=
\inv{(\sqrt{2\pi})^3}
\int_{\partial x^p} dx^p
\int_{\partial x^n} dx^n
{\left. {\left(\sigma^m \psi(\Bx, t) \exp\left( -i \Bk \cdot \Bx \right) \right)} \right\vert}_{\partial x^m} \\
&-
\inv{(\sqrt{2\pi})^3}
\int_{\partial x^p} dx^p
\int_{\partial x^n} dx^n
\int_{\partial x^m} dx^m
\sigma^m \psi(\Bx, t) ( -i k_m )
\exp\left( -i \Bk \cdot \Bx \right)
\\
\end{aligned}
\end{equation}

So, we find that the Fourier transform of our spatial gradient is

\begin{equation}\label{eqn:firstorderFourierMaxwell:360}
\begin{aligned}
\calF(\grad \psi) = \Bk \hat{\psi} i
\end{aligned}
\end{equation}

This has the specific ordering of the vector products for our possibility of non-commutative factors.

From this, without making any assumptions about grade, we have the wave number domain equivalent
for the spacetime split of the gradient \eqnref{eqn:firstorder_fourier_maxwell:spacetimeGradientEquation}

\begin{equation}\label{eqn:firstorder_fourier_maxwell:waveDomainGeneral}
\begin{aligned}
\partial_0 \hat{\psi} + \Bk \hat{\psi} i = \hat{g}
\end{aligned}
\end{equation}

\section{Back to specifics.  Maxwell's equation in wave number domain}

For Maxwell's equation our field variable \(F\) is grade two in the STA basis, and our specific transform pair is:

\begin{equation}\label{eqn:firstorderFourierMaxwell:380}
\begin{aligned}
\left(\partial_0 + \spacegrad \right) F &= \gamma_0 J/\epsilon_0 c \\
\partial_0 \hat{F} + \Bk \hat{F} i &= \gamma_0 \hat{J}/ \epsilon_0 c
\end{aligned}
\end{equation}

Now, \(\exp(i\theta)\) and \(i\) commute, and \(i\) also commutes with both \(F\) and \(\Bk\).  This is true since our field \(F\) as well as the spatial vector \(\Bk\) are grade two in the STA basis.
Two sign interchanges occur as we commute with each vector factor of these
bivectors.

This allows us to write our transformed equation in the slightly tidier form

\begin{equation}\label{eqn:firstorder_fourier_maxwell:toSolve}
\begin{aligned}
\partial_0 \hat{F} + (i \Bk) \hat{F} &= \gamma_0 \hat{J}/ \epsilon_0 c
\end{aligned}
\end{equation}

We want to find a solution to this equation.  If the objects in question were all scalars this would be simple enough, and is a problem of the form

\begin{equation}\label{eqn:firstorder_fourier_maxwell:firstOrder}
\begin{aligned}
B' + A B &= Q
\end{aligned}
\end{equation}

For our electromagnetic field our transform is a summation of the following
form

\begin{equation}\label{eqn:firstorderFourierMaxwell:400}
\begin{aligned}
(\BE + i c \BB) (\cos\theta + i \sin\theta)
&=
(\BE \cos\theta - c \BB \sin\theta) +
i (\BE \sin\theta + c \BB \cos\theta)
\end{aligned}
\end{equation}

The summation of the integral itself will not change the grades, so \(\hat{F}\)
is also a grade two multivector.  The dual of our spatial wave number
vector \(i\Bk\) is also grade two with basis bivectors \(\gamma_m \gamma_n\) very
much like the magnetic field portions of our field vector \(i c \BB\).

Having figured out the grades of all the terms in \eqnref{eqn:firstorder_fourier_maxwell:toSolve}, what
does a grade split of this equation yield?  For the equation to be true
do we not need it to be true for all grades?  Our grade zero, four, and two
terms respectively are

\begin{equation}\label{eqn:firstorderFourierMaxwell:420}
\begin{aligned}
(i \Bk) \cdot \hat{F} &= \hat{\rho}/ \epsilon_0 \\
(i \Bk) \wedge \hat{F} &= 0 \\
\partial_0 \hat{F} + (i \Bk) \times \hat{F} &= -\hat{\Bj}/ \epsilon_0 c
\end{aligned}
\end{equation}

Here the (antisymmetric) commutator product \(\gpgradetwo{ab} = a \times b = (a b - ba)/2\) has been used in the last equation for this bivector product.

It is kind of interesting that an unmoving charge density contributes nothing
to the time variation of the field in the wave number domain, instead
only the current density (spatial) vectors contribute to our differential
equation.

\subsection{Solving this first order inhomogeneous problem}

We want to solve the inhomogeneous scalar equation
\eqnref{eqn:firstorder_fourier_maxwell:firstOrder} but do so in a fashion that is also valid for
the grades for the Maxwell equation problem.

Application of variation of parameters produces the desired result.  Let us write this equation in operator form

\begin{equation}\label{eqn:firstorderFourierMaxwell:440}
\begin{aligned}
L(B) &= B' + A B
\end{aligned}
\end{equation}

and start with the solution of the
homogeneous problem

\begin{equation}\label{eqn:firstorderFourierMaxwell:460}
\begin{aligned}
L(B) = 0
\end{aligned}
\end{equation}

This is

\begin{equation}\label{eqn:firstorderFourierMaxwell:480}
\begin{aligned}
B' = -A B
\end{aligned}
\end{equation}

so we expect exponential solutions will do the trick, but have to get the ordering right due to the possibility of non-commutative factors.  How about one of

\begin{equation}\label{eqn:firstorderFourierMaxwell:500}
\begin{aligned}
B &= C e^{-At} \\
B &= e^{-At} C
\end{aligned}
\end{equation}

Where \(C\) is constant, but not necessarily a scalar, and does not have to commute with \(A\).  Taking derivatives of the first we have

\begin{equation}\label{eqn:firstorderFourierMaxwell:520}
\begin{aligned}
B' = C (-A) e^{-At}
\end{aligned}
\end{equation}

This does not have the desired form unless \(C\) and \(A\) commute.  How about the second possibility?  That one has the derivative

\begin{equation}\label{eqn:firstorderFourierMaxwell:540}
\begin{aligned}
B'
&= (-A) e^{-At} C \\
&= -A B
\end{aligned}
\end{equation}

which is what we want.  Now, for the inhomogeneous problem we want to use a test solution replacing C with an function to be determined.  That is

\begin{equation}\label{eqn:firstorderFourierMaxwell:560}
\begin{aligned}
B &= e^{-At} U
\end{aligned}
\end{equation}

For this we have
\begin{equation}\label{eqn:firstorderFourierMaxwell:580}
\begin{aligned}
L(B)
&= (-A) e^{-At} U + e^{-At} U' + A B  \\
&= e^{-At} U'
\end{aligned}
\end{equation}

Our inhomogeneous problem \(L(B) = Q\) is therefore reduced to

\begin{equation}\label{eqn:firstorderFourierMaxwell:600}
\begin{aligned}
e^{-At} U' &= Q
\end{aligned}
\end{equation}

Or
\begin{equation}\label{eqn:firstorderFourierMaxwell:620}
\begin{aligned}
U &= \int e^{A t} Q(t) dt
\end{aligned}
\end{equation}

As an indefinite integral this gives us

\begin{equation}\label{eqn:firstorderFourierMaxwell:640}
\begin{aligned}
B(t)
&= e^{-At} \int e^{A t} Q(t) dt \\
\end{aligned}
\end{equation}

And finally in definite integral form,
if all has gone well, we have a specific solution to the forced problem

\begin{equation}\label{eqn:firstorder_fourier_maxwell:linearSolved}
\begin{aligned}
B(t) &= \int_{t_0}^{t} e^{-A (t -\tau)} Q(\tau) d\tau
\end{aligned}
\end{equation}

\subsubsection{Verify}

With differentiation under the integral sign we have

\begin{equation}\label{eqn:firstorderFourierMaxwell:660}
\begin{aligned}
\frac{dB}{dt}
&={\left. e^{-A (t -\tau)} Q(\tau) \right\vert}_{\tau=t} + \int_{t_0}^{t} -A e^{-A (t -\tau)} Q(\tau) d\tau \\
&= Q(t) - A B
\end{aligned}
\end{equation}

Great!

\subsection{Back to Maxwell's}

Switching to explicit time derivatives we have

\begin{equation}\label{eqn:firstorderFourierMaxwell:680}
\begin{aligned}
\partial_t \hat{F} + (i c \Bk) \hat{F} &= \gamma_0 \hat{J}/ \epsilon_0
\end{aligned}
\end{equation}

By \eqnref{eqn:firstorder_fourier_maxwell:linearSolved}, this has, respectively, homogeneous and inhomogeneous solutions

\begin{equation}\label{eqn:firstorder_fourier_maxwell:waveNumberDomain}
\begin{aligned}
\hat{F}(\Bk,t) &= e^{-i c \Bk t} C(\Bk) \\
\hat{F}(\Bk,t) &= \inv{\epsilon_0} \int_{t_0(\Bk)}^{t} e^{-(i c \Bk) (t -\tau)} \gamma_0 \hat{J}(\Bk,\tau) d\tau
\end{aligned}
\end{equation}

For the homogeneous term at \(t=0\) we have

\begin{equation}\label{eqn:firstorderFourierMaxwell:700}
\begin{aligned}
\hat{F}(\Bk,0) &= C(\Bk) \\
\end{aligned}
\end{equation}

So, \(C(\Bk)\) is just the Fourier transform of an initial wave packet.  Reassembling all the bits in terms of fully specified Fourier and inverse Fourier transforms we have

\begin{equation}\label{eqn:firstorderFourierMaxwell:720}
\begin{aligned}
F(\Bx,t)
&=
%F^(\Bk, t) = \inv{(\sqrt{2\pi})^3} \int e^{-ic \Bk t} F(\Bx,0) e^{-i \Bk \cdot \Bx} d^3 x
\inv{(\sqrt{2\pi})^3} \int
%\hat{F}(\Bk,t)
\left(
\inv{(\sqrt{2\pi})^3} \int e^{-ic \Bk t} F(\Bu,0) e^{-i \Bk \cdot \Bu} d^3 u
\right)
e^{i \Bk \cdot \Bx} d^3 k \\
&= \inv{({2\pi})^3} \int e^{-ic \Bk t} F(\Bu,0) e^{i \Bk \cdot (\Bx - \Bu)} d^3 u d^3 k \\
\end{aligned}
\end{equation}

We have something like a double sided Green's function, with which we do a spatial convolution over all space with to produce a function of wave number.  One more integration over all wave numbers gives us our inverse Fourier transform.  The final result is a beautiful closed form solution for the time evolution of an arbitrary wave packet for the field specified at some specific initial time.

Now, how about that forced term?  We want to inverse Fourier transform our \(\hat{J}\) based equation in \eqnref{eqn:firstorder_fourier_maxwell:waveNumberDomain}.  Picking our \(t_0 = -\infty\) this is

\begin{equation}\label{eqn:firstorderFourierMaxwell:740}
\begin{aligned}
F(\Bx,t)
&=
\inv{(\sqrt{2\pi})^3} \int
\left( \inv{\epsilon_0} \int_{\tau = -\infty}^{t} e^{-(i c \Bk) (t -\tau)} \gamma_0 \hat{J}(\Bk,\tau) d\tau  \right) e^{i \Bk \cdot \Bx} d^3 k \\
&=
\inv{\epsilon_0 ({2\pi})^3} \int
\int_{\tau = -\infty}^{t} e^{-(i c \Bk) (t -\tau)} \gamma_0 {J}(\Bu,\tau)
e^{i \Bk \cdot (\Bx-\Bu)}
d\tau
d^3 u
d^3 k
\end{aligned}
\end{equation}

Again we have a double sided Green's function.  We require a convolution summing the four vector current density contributions over all space and for all times less than \(t\).

Now we can combine the vacuum and charge present solutions for a complete solution to Maxwell's equation.  This is

\begin{equation}\label{eqn:firstorderFourierMaxwell:760}
\begin{aligned}
F(\Bx,t)
&=
\inv{({2\pi})^3} \int
e^{-i c \Bk t}
\left(
F(\Bu, 0) + \inv{\epsilon_0} \int_{\tau = -\infty}^{t} e^{i c \Bk \tau } \gamma_0 J(\Bu,\tau)  d\tau
\right)
e^{i \Bk \cdot (\Bx-\Bu)}
d^3 u
d^3 k
\end{aligned}
\end{equation}

Now, this may not be any good for actually computing with, but it sure is pretty!

There is a lot of verification required to see if all this math actually works out, and
also a fair amount of followup required to play with this and see what other goodies fall out if this is used.  I had expect that this result ought to be usable to show familiar
results like the Biot-Savart law.

How do our energy density and Poynting energy momentum density conservation relations, and the stress energy tensor terms, look given a closed form expression for \(F\)?

It is also kind of interesting to see the time phase term coupled to the current density here in the forcing term.  That looks awfully similar to some QM expressions, although it
could be coincidental.
