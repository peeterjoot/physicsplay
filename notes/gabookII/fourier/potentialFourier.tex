%
% Copyright � 2012 Peeter Joot.  All Rights Reserved.
% Licenced as described in the file LICENSE under the root directory of this GIT repository.
%

%
%
\chapter{Lorentz Gauge Fourier Vacuum potential solutions}
\index{Lorentz gauge}
\index{Fourier transform!vacuum potential}
\label{chap:potentialFourier}
%\date{Feb 07, 2009.  potentialFourier.tex}

\section{Motivation}

In \chapcite{PJFourierVacuum} a first order Fourier solution of the Vacuum
Maxwell equation was performed.  Here a comparative potential solution
is obtained.

\subsection{Notation}

The 3D Fourier series notation developed for this treatment can be found
in the original notes \chapcite{PJFourierVacuum}.  Also included there is a
table of notation, much of which is also used here.

\section{Second order treatment with potentials}

\subsection{With the Lorentz gauge}

Now, it appears that Bohm's use of potentials allows a nice comparison with the harmonic oscillator.  Let us also try a Fourier solution of the
potential equations.  Again, use STA instead of the traditional vector equations, writing \(A = (\phi + \Ba)\gamma_0\), and employing the Lorentz gauge
\(\grad \cdot A = 0\) we have for \(F = \grad \wedge A\) in CGS units

FIXME: Add \(\Ba\), and \(\psi\) to notational table below with definitions in terms of \(\bcE\), and \(\bcH\) (or the other way around).

\begin{equation}\label{eqn:potentialFourier:20}
\begin{aligned}
\grad^2 A = 4 \pi J
\end{aligned}
\end{equation}

Again with a spacetime split of the gradient

\begin{equation}\label{eqn:potentialFourier:40}
\begin{aligned}
\grad = \gamma^0(\partial_0 + \spacegrad) = (\partial_0 - \spacegrad) \gamma_0
\end{aligned}
\end{equation}

our four Laplacian can be written

\begin{equation}\label{eqn:potentialFourier:60}
\begin{aligned}
(\partial_0 - \spacegrad) \gamma_0 \gamma^0(\partial_0 + \spacegrad)
&= (\partial_0 - \spacegrad) (\partial_0 + \spacegrad) \\
&= \partial_{00} - \spacegrad^2
\end{aligned}
\end{equation}

Our vacuum field equation for the potential is thus
\begin{equation}\label{eqn:potentialFourier:80}
\begin{aligned}
\partial_{tt} A = c^2 \spacegrad^2 A
\end{aligned}
\end{equation}

Now, as before assume a Fourier solution and see what follows.  That is

\begin{equation}\label{eqn:potential_fourier:assumedPotential}
\begin{aligned}
A(\Bx, t) &= \sum_{\Bk} \hat{A}_{\Bk}(t) e^{ -i \Bk \cdot \Bx}
\end{aligned}
\end{equation}

Applied to each component this gives us
\begin{equation}\label{eqn:potentialFourier:100}
\begin{aligned}
\hat{A}_{\Bk}'' e^{ -i \Bk \cdot \Bx}
&= c^2 \hat{A}_{\Bk}(t) \sum_m \PDsq{x^m}{} e^{ - 2 \pi i \sum_j k_j x^j /\lambda_j} \\
&= c^2 \hat{A}_{\Bk}(t) \sum_m (- 2 \pi i k_m/\lambda_m)^2 e^{ - i \Bk \cdot \Bx } \\
&= -c^2 \Bk^2 \hat{A}_{\Bk} e^{ - i \Bk \cdot \Bx }
\end{aligned}
\end{equation}

So we are left with another big ass set of simplest equations to solve

\begin{equation}\label{eqn:potentialFourier:120}
\begin{aligned}
\hat{A}_{\Bk}'' &= -c^2 \Bk^2 \hat{A}_{\Bk}
\end{aligned}
\end{equation}

Note that again the origin point \(\Bk = (0,0,0)\) is a special case.  Also of note this time is that \(\hat{A}_{\Bk}\) has vector and trivector parts, unlike \(\hat{F}_{\Bk}\) which being derived from dual and non-dual components of a bivector was still a bivector.

It appears that solutions can be found with either left or right handed
vector valued integration constants

\begin{equation}\label{eqn:potentialFourier:140}
\begin{aligned}
\hat{A}_{\Bk}(t) &= \exp(\pm i c \Bk t) C_{\Bk} \\
                 &= D_{\Bk} \exp(\pm i c \Bk t)
\end{aligned}
\end{equation}

Since these are equal at \(t=0\), it appears to imply that these commute with the
complex exponentials as was the case for the bivector field.

For the \(\Bk = 0\) special case we have solutions
\begin{equation}\label{eqn:potentialFourier:160}
\begin{aligned}
\hat{A}_{\Bk}(t) &= D_0 t + C_0
\end{aligned}
\end{equation}

It does not seem unreasonable to require \(D_0 = 0\).  Otherwise this time dependent DC Fourier component will blow up at large and small values, while periodic
solutions are sought.

Putting things back together we have %either of

\begin{equation}\label{eqn:potentialFourier:180}
\begin{aligned}
A(\Bx, t) &= \sum_{\Bk} \exp(\pm i c \Bk t) C_{\Bk} \exp( -i \Bk \cdot \Bx ) \\
%          &= \sum_{\Bk} C_{\Bk} \exp(\pm i c \Bk t) \exp( -i \Bk \cdot \Bx )
\end{aligned}
\end{equation}

Here again for \(t=0\), our integration constants are found to be determined completely by the initial conditions

\begin{equation}\label{eqn:potentialFourier:200}
\begin{aligned}
A(\Bx, 0) &= \sum_{\Bk} C_{\Bk} e^{ -i \Bk \cdot \Bx}
\end{aligned}
\end{equation}

So we can write

\begin{equation}\label{eqn:potentialFourier:220}
\begin{aligned}
C_{\Bk} = \inv{V} \int A(\Bx', 0) e^{ i \Bk \cdot \Bx'} d^3 x'
\end{aligned}
\end{equation}

In integral form this is

\begin{equation}\label{eqn:potential_fourier:potentialSolution}
\begin{aligned}
A(\Bx, t) &= \int \sum_{\Bk} \exp(\pm i \Bk c t ) A(\Bx', 0) \exp( i \Bk \cdot (\Bx -\Bx') )
\end{aligned}
\end{equation}

This, somewhat surprisingly, is strikingly similar to what we had for the bivector field.  That was:

\begin{equation}\label{eqn:potential_fourier:bivectorSolution}
\begin{aligned}
F(\Bx,t) &= \int G(\Bx - \Bx', t) F(\Bx', 0) d^3 x' \\
G(\Bx,t) &= \inv{V} \sum_{\Bk} \exp\left( i \Bk ct \right) \exp\left( -i \Bk \cdot \Bx \right)
\end{aligned}
\end{equation}

We cannot however
commute the time phase term to construct a one sided Green's function for this
potential solution (or perhaps we can but if so shown or attempted to show that this is possible).  We also have a
plus or minus variation in the phase term due to the second order nature of the harmonic oscillator equations for our Fourier coefficients.

\subsection{Comparing the first and second order solutions}

A consequence of working in the Lorentz gauge (\(\grad \cdot A = 0\)) is that our field solution should be a gradient

\begin{equation}\label{eqn:potentialFourier:240}
\begin{aligned}
F
&= \grad \wedge A \\
&= \grad A \\
%&= \int A(\Bx', 0) \left(\grad G_A(\Bx - \Bx', t) \right) d^3 x' \\
\end{aligned}
\end{equation}

%Or with the opposite convolution
%\begin{align*}
%F &= \grad \int A(\Bx - \Bx', 0) G_A(\Bx', t) d^3 x' \\
%\end{align*}

FIXME: expand this out using \eqnref{eqn:potential_fourier:potentialSolution} to compare to the first order solution.
