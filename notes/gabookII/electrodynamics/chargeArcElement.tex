%
% Copyright � 2012 Peeter Joot.  All Rights Reserved.
% Licenced as described in the file LICENSE under the root directory of this GIT repository.
%

% 
% 
\chapter{Field due to line charge in arc}
\index{line charge}
\label{chap:chargeArcElement}
%\date{Nov 23, 2008.  chargeArcElement.tex}

\section{Motivation}

Problem \(1.5\) from \citep{purcell1963eam}, is to calculate the field
at the center of a half circular arc of line charge.  Do this calculation
and setup for the calculation at other points.

\section{Just the stated problem}

%\begin{figure}[htp]
%\centering
%\includegraphics[totalheight=0.4\textheight]{picturepath}
%\caption{My Caption}\label{fig:pictlabel}
%\end{figure}
%
%... see \cref{fig:picturepath} ...

To solve for the field at just the center point in the plane of the arc, given line charge density \(\lambda\), and arc
radius \(R\) one has, and pseudoscalar for the plane \(i = \Be_1\Be_2\) one has

\begin{equation}\label{eqn:chargeArcElement:20}
\begin{aligned}
dq &= \lambda R d\theta \\
d\BE &= \inv{4 \pi \epsilon_0 R^2} dq (-\Be_1 e^{i\theta} )
\end{aligned}
\end{equation}

Straight integration gives the result in short order

\begin{equation}\label{eqn:chargeArcElement:40}
\begin{aligned}
\BE
&= \frac{-\lambda \Be_1}{4 \pi \epsilon_0 R} \int_0^\pi e^{i\theta} d\theta \\
&= \frac{\lambda \Be_2}{4 \pi \epsilon_0 R} \left. e^{i\theta} \right\vert_0^\pi \\
&= \frac{-\lambda \Be_2}{2 \pi \epsilon_0 R}
\end{aligned}
\end{equation}

So, if the total charge is \(Q = \pi R \lambda\), the field is then

\begin{equation}\label{eqn:chargeArcElement:60}
\begin{aligned}
\BE
&= \frac{-Q \Be_2}{2 \pi^2 \epsilon_0 R^2} \\
\end{aligned}
\end{equation}

So, at the center point the semicircular arc of charge behaves as if it is a point charge of magnitude \(2Q/\pi\) at the point \(R \Be_2\)

\begin{equation}\label{eqn:chargeArcEl:semicircleAtCenter}
\begin{aligned}
\BE = \frac{-Q \Be_2}{ 4 \pi \epsilon_0 R^2 } \frac{2}{\pi}
\end{aligned}
\end{equation}

\section{Field at other points}

Now, how about at points outside of the plane of the charge?

Suppose our point of measurement is expressed in cylindrical polar coordinates

\begin{equation}\label{eqn:chargeArcElement:80}
\begin{aligned}
P = \rho \Be_1 e^{i\alpha} + z \Be_3
\end{aligned}
\end{equation}

So that the vector from the element of charge at \(\theta\) is
\begin{equation}\label{eqn:chargeArcElement:100}
\begin{aligned}
\Bu = P - R \Be_1 e^{i\theta} &= \Be_1 (\rho e^{i\alpha} -R e^{i\theta}) + z \Be_3
\end{aligned}
\end{equation}

Relative to \(\theta\), writing \(\theta = \alpha + \beta\) this is
\begin{equation}\label{eqn:chargeArcElement:120}
\begin{aligned}
\Bu &= \Be_1 e^{i\alpha} (\rho -R e^{i\beta}) + z \Be_3
\end{aligned}
\end{equation}

The squared magnitude of this vector is
\begin{equation}\label{eqn:chargeArcElement:140}
\begin{aligned}
\Bu^2
&= \Abs{\rho -R e^{i\beta}}^2 + z^2 \\
&= z^2 + \rho^2 + R^2 -2 \rho R \cos\beta \\
\end{aligned}
\end{equation}

The field is thus

\begin{equation}\label{eqn:chargeArcEl:fieldAtP}
\begin{aligned}
\BE = \inv{4 \pi \epsilon_0} \lambda R 
\int_{\beta=\theta_1 -\alpha}^{\beta=\theta_2-\alpha}
{
\left(z^2 + \rho^2 + R^2 -2 \rho R \cos\beta\right)
}^{-3/2}
\left(\Be_1 e^{i\alpha} (\rho -R e^{i\beta}) + z \Be_3\right)
d\beta
\end{aligned}
\end{equation}

This integral has two variations
\begin{equation}\label{eqn:chargeArcElement:160}
\begin{aligned}
&\int { \left(a^2 - b^2\cos\beta\right) }^{-3/2} d\beta \\
&\int { \left(a^2 - b^2\cos\beta\right) }^{-3/2} e^{i\beta} d\beta \\
\end{aligned}
\end{equation}

or
\begin{equation}\label{eqn:chargeArcElement:180}
\begin{aligned}
I_1 &= \int { \left(a^2 - b^2\cos\beta\right) }^{-3/2} d\beta \\
I_2 &= \int { \left(a^2 - b^2\cos\beta\right) }^{-3/2} \cos\beta d\beta \\
I_3 &= \int { \left(a^2 - b^2\cos\beta\right) }^{-3/2} \sin\beta d\beta \\
\end{aligned}
\end{equation}

Of these when only the last is obviously integrable (at least for \(b \ne 0\))
\begin{equation}\label{eqn:chargeArcElement:200}
\begin{aligned}
I_3
&= \int { \left(a^2 - b^2\cos\beta\right) }^{-3/2} \sin\beta d\beta \\
&= -2 {\left(a^2 - b^2\cos\beta\right) }^{-1/2} \\
\end{aligned}
\end{equation}

Having solved for the imaginary component can the Cauchy Riemann equations be used to supply the real part?  How about \(I_1\) ?

\subsection{On the z-axis}

Not knowing how to solve the integral of \eqnref{eqn:chargeArcEl:fieldAtP} (elliptic?), the easy case
of \(\rho = 0\) (up the z-axis) can at least be obtained

\begin{equation}\label{eqn:chargeArcElement:220}
\begin{aligned}
\BE 
&= \inv{4 \pi \epsilon_0} \lambda R { \left(z^2 + R^2\right) }^{-3/2}
\int_{\theta_1}^{\theta_2} \left(-\Be_1 R e^{i\theta} + z \Be_3\right) d\theta \\
&= \inv{4 \pi \epsilon_0} \lambda R { \left(z^2 + R^2\right) }^{-3/2}
\left(\Be_2 R (e^{i\theta_2} -e^{i\theta_1}) + z \Be_3 \Delta\theta \right) \\
&= \inv{4 \pi \epsilon_0} \lambda R { \left(z^2 + R^2\right) }^{-3/2}
\left(\Be_2 R e^{i(\theta_1 + \theta_2)/2} \left(
e^{i(\theta_2 - \theta_1)/2}
-e^{-i(\theta_2 - \theta_1)/2}
\right) + z \Be_3 \Delta\theta \right) \\
&= \inv{4 \pi \epsilon_0} \lambda R { \left(z^2 + R^2\right) }^{-3/2}
\left(-2 \Be_1 R e^{i(\theta_1 + \theta_2)/2} \sin(\Delta\theta/2)
+ z \Be_3 \Delta\theta \right) \\
&= \inv{4 \pi \epsilon_0 \Delta\theta} Q { \left(z^2 + R^2\right) }^{-3/2}
\left(-2 \Be_1 R e^{i(\theta_1 + \theta_2)/2} \sin(\Delta\theta/2)
+ z \Be_3 \Delta\theta \right) \\
\end{aligned}
\end{equation}

Eliminating the explicit imaginary, and writing \(\overbar{\theta} = (\theta_1 + \theta_2)/2\), we have in vector form
the field on any position up and down the z-axis
\begin{equation}\label{eqn:chargeArcElement:240}
\begin{aligned}
\BE 
&= \inv{4 \pi \epsilon_0 \Delta\theta} Q { \left(z^2 + R^2\right) }^{-3/2}
\left(
-2 R \left( \Be_1 \cos \overbar{\theta} +\Be_2 \sin\overbar{\theta} \right) \sin(\Delta\theta/2)
+ z \Be_3 \Delta\theta \right) 
\end{aligned}
\end{equation}

For \(z = 0\), \(\theta_1 = 0\), and \(\theta_2 = \pi\), this matches with \eqnref{eqn:chargeArcEl:semicircleAtCenter} as expected, but
expressing this as an equivalent to a point charge is no longer possible at any point off the plane of the
charge.
