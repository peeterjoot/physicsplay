%
% Copyright � 2012 Peeter Joot.  All Rights Reserved.
% Licenced as described in the file LICENSE under the root directory of this GIT repository.
%

%
%
\chapter{Energy momentum tensor}\label{chap:PJemstresstensor}
%\date{Jan 01, 2009.  energyMomentumTensor.tex}

\section{Expanding out the stress energy vector in tensor form}

\citep{doran2003gap} defines (with \(\epsilon_0\) omitted),
the energy momentum stress tensor as a vector to
vector mapping of the following form:

\begin{equation}\label{eqn:energyMomentumTensor:20}
\begin{aligned}
T(a)
&= \frac{\epsilon_0}{2} F a \tilde{F}
= - \frac{\epsilon_0}{2} F a F
\end{aligned}
\end{equation}

This quantity can only have vector, trivector, and five vector grades.  The
grade five term must be zero

\begin{equation}\label{eqn:energyMomentumTensor:40}
\begin{aligned}
\gpgrade{T(a)}{5}
&= \frac{\epsilon_0}{2} F \wedge a \wedge \tilde{F} \\
&= \frac{\epsilon_0}{2} a \wedge (F \wedge \tilde{F}) \\
&= 0
\end{aligned}
\end{equation}

Since \((T(a))^{\tilde{}} = T(a)\), the grade three term is also zero (trivectors invert on reversion), so this must therefore be a vector.

As a vector this can be expanded in coordinates

\begin{equation}\label{eqn:energyMomentumTensor:60}
\begin{aligned}
T(a)
&= \left(T(a) \cdot \gamma^\nu \right) \gamma_\nu \\
&= \left(T(a^\mu \gamma_\mu) \cdot \gamma^\nu \right) \gamma_\nu \\
&= a^\mu \gamma_\nu \left(T(\gamma_\mu) \cdot \gamma^\nu \right) \\
\end{aligned}
\end{equation}

It is this last bit that has the form of a traditional tensor, so we can write

\begin{equation}\label{eqn:energyMomentumTensor:80}
\begin{aligned}
T(a) &= a^\mu \gamma_\nu {T_\mu}^{\nu} \\
{T_\mu}^{\nu} &= T(\gamma_\mu) \cdot \gamma^\nu
\end{aligned}
\end{equation}

Let us expand this tensor \({T_\mu}^{\nu}\) explicitly to verify its form.

We want to expand, and dot with \(\gamma^\nu\), the following

\begin{equation}\label{eqn:energyMomentumTensor:100}
\begin{aligned}
-2 \inv{\epsilon_0} \left(T(\gamma_\mu) \cdot \gamma^\nu \right) \gamma_\nu
&= \gpgradeone{(\grad \wedge A) \gamma_\mu (\grad \wedge A)} \\
&= \gpgradeone{
(\grad \wedge A) \cdot \gamma_\mu (\grad \wedge A)
+ (\grad \wedge A) \wedge \gamma^\mu (\grad \wedge A)
} \\
&=
((\grad \wedge A) \cdot \gamma_\mu) \cdot (\grad \wedge A)
+ ((\grad \wedge A) \wedge \gamma_\mu) \cdot (\grad \wedge A)
\\
\end{aligned}
\end{equation}

Both of these will get temporarily messy, so let us do them in parts.  Starting
with

\begin{equation}\label{eqn:energyMomentumTensor:120}
\begin{aligned}
(\grad \wedge A) \cdot \gamma_\mu
&= (\gamma^{\alpha} \wedge \gamma^{\beta}) \cdot \gamma_{\mu} \partial_{\alpha} A_{\beta} \\
&= (\gamma^{\alpha} {\delta^{\beta}}_{\mu} -\gamma^{\beta} {\delta^{\alpha}}_{\mu} ) \partial_{\alpha} A_{\beta} \\
&=
\gamma^{\alpha} \partial_{\alpha} A_{\mu}
-\gamma^{\beta} \partial_{\mu} A_{\beta} \\
&= \gamma^{\alpha} (\partial_{\alpha} A_{\mu} -\partial_{\mu} A_{\alpha} ) \\
&= \gamma^{\alpha} F_{\alpha \mu} \\
\end{aligned}
\end{equation}


\begin{equation}\label{eqn:energyMomentumTensor:140}
\begin{aligned}
((\grad \wedge A) \cdot \gamma_\mu) \cdot (\grad \wedge A)
&= (\gamma^{\alpha} F_{\alpha \mu}) \cdot (\gamma_{\beta} \wedge \gamma_{\lambda}) \partial^{\beta} A^{\lambda} \\
%&=
%\partial^{\beta} A^{\lambda} F_{\alpha \mu}
%\gamma^{\alpha} \cdot (\gamma_{\beta} \wedge \gamma_{\lambda})
%\\
&=
\partial^{\beta} A^{\lambda} F_{\alpha \mu}
(
{\delta^{\alpha}}_{\beta} \gamma_{\lambda}
-{\delta^{\alpha}}_{\lambda} \gamma_{\beta}
)
\\
&=
(\partial^{\alpha} A^{\beta} F_{\alpha \mu} -\partial^{\beta} A^{\alpha} F_{\alpha \mu} )\gamma_{\beta}
\\
&= F^{\alpha \beta} F_{\alpha \mu} \gamma_{\beta} \\
\end{aligned}
\end{equation}

So, by dotting with \(\gamma^\nu\) we have

\begin{equation}\label{eqn:energy_momentum_tensor:firstPartDone}
\begin{aligned}
((\grad \wedge A) \cdot \gamma_\mu) \cdot (\grad \wedge A) \cdot \gamma^{\nu}
&= F^{\alpha \nu} F_{\alpha \mu}
\end{aligned}
\end{equation}

Moving on to the next bit,
\((((\grad \wedge A) \wedge \gamma^\mu) \cdot (\grad \wedge A)) \cdot \gamma^\nu\).� By inspection the first part of this is

\begin{equation}\label{eqn:energyMomentumTensor:160}
\begin{aligned}
(\grad \wedge A) \wedge \gamma_\mu
&= (\gamma_\mu)^2 (\gamma^{\alpha} \wedge \gamma^{\beta}) \wedge \gamma^{\mu} \partial_{\alpha} A_{\beta} \\
\end{aligned}
\end{equation}

so dotting with \(\grad \wedge A\), we have

\begin{equation}\label{eqn:energyMomentumTensor:180}
\begin{aligned}
((\grad \wedge A) \wedge \gamma_\mu) \cdot (\grad \wedge A)
&=
(\gamma_\mu)^2
\partial_{\alpha} A_{\beta}
\partial^{\lambda} A^{\delta}
(\gamma^{\alpha} \wedge \gamma^{\beta} \wedge \gamma^{\mu}) \cdot
(\gamma_{\lambda} \wedge \gamma_{\delta})
\\
&=
(\gamma_\mu)^2
\partial_{\alpha} A_{\beta}
\partial^{\lambda} A^{\delta}
((\gamma^{\alpha} \wedge \gamma^{\beta} \wedge \gamma^{\mu}) \cdot \gamma_{\lambda} ) \cdot \gamma_{\delta}
\\
\end{aligned}
\end{equation}

Expanding just the dot product parts of this we have
\begin{equation}\label{eqn:energyMomentumTensor:200}
\begin{aligned}
&(((\gamma^{\alpha} \wedge \gamma^{\beta}) \wedge \gamma^{\mu}) \cdot \gamma_{\lambda} ) \cdot \gamma_{\delta} \\
&=
(\gamma^{\alpha} \wedge \gamma^{\beta}) {\delta^{\mu}}_{\lambda}
-(\gamma^{\alpha} \wedge \gamma^{\mu}) {\delta^{\beta}}_{\lambda}
+(\gamma^{\beta} \wedge \gamma^{\mu}) {\delta^{\alpha}}_{\lambda}
) \cdot \gamma_{\delta}
\\
%&=
%(
%  \gamma^{\alpha} {\delta^{\beta}}_{\delta} {\delta^{\mu}}_{\lambda}
%- \gamma^{\alpha} {\delta^{\mu}}_{\delta} {\delta^{\beta}}_{\lambda}
%+ \gamma^{\beta} {\delta^{\mu}}_{\delta} {\delta^{\alpha}}_{\lambda}
%- \gamma^{\beta} {\delta^{\alpha}}_{\delta} {\delta^{\mu}}_{\lambda}
%+ \gamma^{\mu} {\delta^{\alpha}}_{\delta} {\delta^{\beta}}_{\lambda}
%- \gamma^{\mu} {\delta^{\beta}}_{\delta} {\delta^{\alpha}}_{\lambda}
%)
%\\
&=
  \gamma^{\alpha} ({\delta^{\beta}}_{\delta} {\delta^{\mu}}_{\lambda}
-            {\delta^{\mu}}_{\delta} {\delta^{\beta}}_{\lambda})
+ \gamma^{\beta} ({\delta^{\mu}}_{\delta} {\delta^{\alpha}}_{\lambda}
-            {\delta^{\alpha}}_{\delta} {\delta^{\mu}}_{\lambda})
+ \gamma^{\mu} ({\delta^{\alpha}}_{\delta} {\delta^{\beta}}_{\lambda}
-                 {\delta^{\beta}}_{\delta} {\delta^{\alpha}}_{\lambda})
\\
\end{aligned}
\end{equation}

This can now be applied to \(\partial^{\lambda} A^{\delta}\)

\begin{equation}\label{eqn:energyMomentumTensor:220}
\begin{aligned}
\partial^{\lambda} A^{\delta} &(((\gamma^{\alpha} \wedge \gamma^{\beta}) \wedge \gamma^{\mu}) \cdot \gamma_{\lambda} ) \cdot \gamma_{\delta} \\
&=
  \partial^{\mu} A^{\beta} \gamma^{\alpha}
- \partial^{\beta} A^{\mu} \gamma^{\alpha}
+ \partial^{\alpha} A^{\mu} \gamma^{\beta}
- \partial^{\mu} A^{\alpha} \gamma^{\beta}
+ \partial^{\beta} A^{\alpha} \gamma^{\mu}
- \partial^{\alpha} A^{\beta} \gamma^{\mu}
\\
&=
(  \partial^{\mu} A^{\beta}
- \partial^{\beta} A^{\mu} ) \gamma^{\alpha}
+( \partial^{\alpha} A^{\mu}
- \partial^{\mu} A^{\alpha} ) \gamma^{\beta}
+( \partial^{\beta} A^{\alpha}
- \partial^{\alpha} A^{\beta} ) \gamma^{\mu}
\\
&=
%(  \partial^{\mu} A^{\beta} - \partial^{\beta} A^{\mu} )
F^{\mu \beta}
\gamma^{\alpha}
+
%( \partial^{\alpha} A^{\mu} - \partial^{\mu} A^{\alpha} )
F^{\alpha \mu}
\gamma^{\beta}
+
%( \partial^{\beta} A^{\alpha} - \partial^{\alpha} A^{\beta} )
F^{\beta \alpha}
\gamma^{\mu}
\\
\end{aligned}
\end{equation}

This is getting closer, and we can now write
\begin{equation}\label{eqn:energyMomentumTensor:240}
\begin{aligned}
((\grad \wedge A) \wedge \gamma_\mu) \cdot (\grad \wedge A) &=
(\gamma_\mu)^2 \partial_{\alpha} A_{\beta}
(
  F^{\mu \beta} \gamma^{\alpha}
+ F^{\alpha \mu} \gamma^{\beta}
+ F^{\beta \alpha} \gamma^{\mu}
) \\
&=
  (\gamma_\mu)^2 \partial_{\beta} A_{\alpha} F^{\mu \alpha} \gamma^{\beta}
+ (\gamma_\mu)^2 \partial_{\alpha} A_{\beta} F^{\alpha \mu} \gamma^{\beta}
+ (\gamma_\mu)^2 \partial_{\alpha} A_{\beta} F^{\beta \alpha} \gamma^{\mu}
\\
&=
  F^{\beta \alpha} F_{\mu \alpha} \gamma_{\beta}
+ \partial_{\alpha} A_{\beta} F^{\beta \alpha} \gamma_{\mu}
\\
\end{aligned}
\end{equation}

This can now be dotted with \(\gamma^\nu\),

\begin{equation}\label{eqn:energyMomentumTensor:260}
\begin{aligned}
((\grad \wedge A) \wedge \gamma_\mu) \cdot (\grad \wedge A) \cdot \gamma^\nu
&=
 F^{\beta \alpha} F_{\mu \alpha} {\delta_{\beta}}^\nu
+ \partial_{\alpha} A_{\beta} F^{\beta \alpha} {\delta_{\mu}}^\nu
\\
%&= F^{\nu \alpha} F_{\mu \alpha} + \inv{2} F_{\alpha \beta} F^{\beta \alpha} {\delta_{\mu}}^\nu \\
\end{aligned}
\end{equation}

which is
\begin{equation}\label{eqn:energy_momentum_tensor:secondPart}
\begin{aligned}
((\grad \wedge A) \wedge \gamma_\mu) \cdot (\grad \wedge A) \cdot \gamma^\nu
&= F^{\nu \alpha} F_{\mu \alpha} + \inv{2} F_{\alpha \beta} F^{\beta \alpha} {\delta_{\mu}}^\nu
%F^{\nu \beta} F_{\mu \beta} +\inv{2} F_{\alpha \beta} F^{\beta \alpha} {\delta_{\mu}}^\nu
\end{aligned}
\end{equation}

The final combination of results \eqnref{eqn:energy_momentum_tensor:firstPartDone}, and
\eqnref{eqn:energy_momentum_tensor:secondPart} gives

\begin{equation}\label{eqn:energyMomentumTensor:280}
\begin{aligned}
(F \gamma_\mu F ) \cdot \gamma^\nu
&=
2 F^{\alpha \nu} F_{\alpha \mu}
+\inv{2} F_{\alpha \beta} F^{\beta \alpha} {\delta_{\mu}}^\nu
\end{aligned}
\end{equation}

Yielding the tensor

\begin{equation}\label{eqn:energy_momentum_tensor:messyTensor}
\begin{aligned}
{T_\mu}^{\nu}
&=
\epsilon_0 \left(
\inv{4} F_{\alpha \beta} F^{\alpha \beta} {\delta_{\mu}}^\nu
-
F_{\alpha \mu}
F^{\alpha \nu}
\right)
\end{aligned}
\end{equation}

\section{Validate against previously calculated Poynting result}

In \chapcite{PJpoynting}, the electrodynamic energy density \(U\) and momentum flux density vectors were related as follows

\begin{equation}\label{eqn:energy_momentum_tensor:fromPoyntingNotes}
\begin{aligned}
U &= \frac{\epsilon_0}{2}\left( \BE^2 + c^2 \BB^2 \right) \\
\BP &= \epsilon_0 c^2 \BE \cross \BB = \epsilon_0 c (i c \BB) \cdot \BE \\
0 &= \PD{t}{}\frac{\epsilon_0}{2} \left(\BE^2 + c^2 \BB^2\right) + c^2 \epsilon_0 \spacegrad \cdot (\BE \cross \BB) + \BE \cdot \Bj
\end{aligned}
\end{equation}

Additionally the energy and momentum flux densities are components of this stress tensor four vector

\begin{equation}\label{eqn:energyMomentumTensor:300}
\begin{aligned}
T(\gamma_0) &= U \gamma_0 + \inv{c} \BP \gamma_0 \\
\end{aligned}
\end{equation}

From this we can read the first row of the tensor elements

\begin{equation}\label{eqn:energyMomentumTensor:320}
\begin{aligned}
{T_0}^0 &= U
= \frac{\epsilon_0}{2}\left( \BE^2 + c^2 \BB^2 \right) \\
{T_0}^k &= \inv{c} (\BP \gamma_0) \cdot \gamma^k = \epsilon_0 c E^a B^b \epsilon_{k a b}
\end{aligned}
\end{equation}

Let us compare these to \eqnref{eqn:energy_momentum_tensor:messyTensor}, which gives
\begin{equation}\label{eqn:energyMomentumTensor:340}
\begin{aligned}
{T_0}^{0}
&= \epsilon_0 \left( \inv{4} F_{\alpha \beta} F^{\alpha \beta} - F_{\alpha 0} F^{\alpha 0} \right) \\
&= \frac{\epsilon_0}{4} \left( F_{\alpha j} F^{\alpha j} - {3} F_{j 0} F^{j 0} \right) \\
&= \frac{\epsilon_0}{4} \left( F_{m j} F^{m j} +F_{0 j} F^{0 j} - {3} F_{j 0} F^{j 0} \right) \\
&= \frac{\epsilon_0}{4} \left( F_{m j} F^{m j} - {2} F_{j 0} F^{j 0} \right) \\
{T_0}^{k}
&= -\epsilon_0 F_{\alpha 0} F^{\alpha k} \\
&= -\epsilon_0 F_{j 0} F^{j k} \\
\end{aligned}
\end{equation}

Now, our field in terms of electric and magnetic coordinates is

\begin{equation}\label{eqn:energyMomentumTensor:360}
\begin{aligned}
F &= \BE + i c \BB \\
  &= E^k \gamma_k \gamma_0 + i c B^k \gamma_k \gamma_0 \\
  &= E^k \gamma_k \gamma_0 - c \epsilon_{a b k} B^k \gamma_a \gamma_b
\end{aligned}
\end{equation}

so the electric field tensor components are

\begin{equation}\label{eqn:energyMomentumTensor:380}
\begin{aligned}
F^{j 0}
&= (F \cdot \gamma^0) \cdot \gamma^j \\
&= E^k {\delta_k}^j \\
&= E^j
\end{aligned}
\end{equation}

and
\begin{equation}\label{eqn:energyMomentumTensor:400}
\begin{aligned}
F_{j 0} &= (\gamma_j)^2 (\gamma_0)^2 F^{j 0} \\
&= -E^j
\end{aligned}
\end{equation}

and the magnetic tensor components are

\begin{equation}\label{eqn:energyMomentumTensor:420}
\begin{aligned}
F^{m j} &= F_{m j} \\
&= - c \epsilon_{a b k} B^k ((\gamma_a \gamma_b) \cdot \gamma_{j}) \cdot \gamma_m \\
&= - c \epsilon_{m j k} B^k
\end{aligned}
\end{equation}

This gives
\begin{equation}\label{eqn:energyMomentumTensor:440}
\begin{aligned}
{T_0}^{0}
&= \frac{\epsilon_0}{4} \left( 2 c^2 B^k B^k + {2} E^j E^{j} \right) \\
&= \frac{\epsilon_0}{2} \left( c^2 \BB^2 + \BE^2 \right) \\
{T_0}^{k}
&= \epsilon_0 E^{j} F^{j k} \\
&= \epsilon_0 c \epsilon_{k e f} E^e B^f \\
&= \epsilon_0 (c \BE \cross \BB)_k \\
&= \inv{c} (\BP \cdot \sigma_k)
\end{aligned}
\end{equation}

Okay, good.  This checks 4 of the elements of \eqnref{eqn:energy_momentum_tensor:messyTensor} against the explicit \(\BE\) and \(\BB\) based representation of \(T(\gamma_0)\) in \eqnref{eqn:energy_momentum_tensor:fromPoyntingNotes}, leaving only 6 unique elements in the remaining parts of the (symmetric) tensor to verify.

\section{Four vector form of energy momentum conservation relationship}

One can observe that there is a spacetime divergence hiding there directly
in the energy conservation equation of
\eqnref{eqn:energy_momentum_tensor:fromPoyntingNotes}.  In particular, writing the last of those as

\begin{equation}\label{eqn:energyMomentumTensor:460}
\begin{aligned}
0 &= \partial_{0}{}\frac{\epsilon_0}{2} \left(\BE^2 + c^2 \BB^2\right) + \spacegrad \cdot \BP/c + \BE \cdot \Bj/c
\end{aligned}
\end{equation}

We can then write the energy-momentum parts as a four vector divergence
\begin{equation}\label{eqn:energyMomentumTensor:480}
\begin{aligned}
\grad \cdot \left(
\frac{\epsilon_0 \gamma_0}{2} \left(\BE^2 + c^2 \BB^2\right)
+ \inv{c} P^k \gamma_k
\right) &= - \BE \cdot \Bj/c
\end{aligned}
\end{equation}

Since we have a divergence relationship, it should also be possible to convert a spacetime hypervolume
integration of this quantity into a time-surface integral or a pure volume integral.  Pursing this
will probably clarify how the tensor is related to the
hypersurface flux as mentioned in the text here, but
making this concrete will take a bit more thought.

Having seen that we have a divergence relationship for the energy momentum tensor in the rest frame, it is clear that the
Poynting energy momentum flux relationship should follow much more directly if we play it backwards in a relativistic setting.

This is a very sneaky way to do it since we have to have seen the answer to get there, but it should avoid the complexity of
trying to factor out the spacial gradients and recover the divergence relationship that provides the Poynting vector.  Our sneaky
starting point is to compute

\begin{equation}\label{eqn:energyMomentumTensor:500}
\begin{aligned}
\grad \cdot ( F \gamma_0 \tilde{F} )
&= \gpgradezero{ \grad (F \gamma_0 \tilde{F}) } \\
&= \gpgradezero{
(\grad F) \gamma_0 \tilde{F}
+ \dot{\grad} F \gamma_0 \dot{\tilde{F}}
} \\
&= \gpgradezero{
(\grad F) \gamma_0 \tilde{F}
+ \dot{\tilde{F}} \dot{\grad} F \gamma_0
} \\
\end{aligned}
\end{equation}

Since this is a scalar quantity, it is equal to its own reverse and we can reverse all factors in this second term to convert the left acting
gradient to a more regular right acting form.  This is

\begin{equation}\label{eqn:energyMomentumTensor:520}
\begin{aligned}
\grad \cdot ( F \gamma_0 \tilde{F} )
&= \gpgradezero{
(\grad F) \gamma_0 \tilde{F}
+ \gamma_0 \tilde{F} (\grad F)
} \\
\end{aligned}
\end{equation}

Now using Maxwell's equation \(\grad F = J/\epsilon_0 c\), we have

\begin{equation}\label{eqn:energyMomentumTensor:540}
\begin{aligned}
\grad \cdot ( F \gamma_0 \tilde{F} )
&=
\inv{\epsilon_0 c}
\gpgradezero{
J \gamma_0 \tilde{F}
+ \gamma_0 \tilde{F} J
} \\
&=
\frac{2}{\epsilon_0 c}
\gpgradezero{
J \gamma_0 \tilde{F}
} \\
&= \frac{2}{\epsilon_0 c} (J \wedge \gamma_0) \cdot \tilde{F} \\
\end{aligned}
\end{equation}

Now, \(J = \gamma_0 c \rho + \gamma_k J^k\), so \(J \wedge \gamma_0 = J^k \gamma_k \gamma_0 = J^k \sigma_k = \Bj\), and dotting this with \(\tilde{F} = -\BE - i c \BB\) will pick up only the (negated) electric field components, so we have

\begin{equation}\label{eqn:energyMomentumTensor:560}
\begin{aligned}
(J \wedge \gamma_0) \cdot \tilde{F} &= \Bj \cdot (-\BE)
\end{aligned}
\end{equation}

Although done in \chapcite{PJpoynting}, for
completeness let us re-expand \(F \gamma_0 \tilde{F}\) in terms of the electric and magnetic field vectors.

\begin{equation}\label{eqn:energyMomentumTensor:580}
\begin{aligned}
F \gamma_0 \tilde{F}
&= -(\BE + i c \BB) \gamma_0 (\BE + i c \BB) \\
%&= - \gamma_0 (-\BE + i c \BB) (\BE + i c \BB) \\
&= \gamma_0 (\BE - i c \BB) (\BE + i c \BB) \\
&= \gamma_0 (\BE^2 + c^2 \BB^2 + i c (\BE \BB -\BB \BE) ) \\
&= \gamma_0 (\BE^2 + c^2 \BB^2 + 2 i c (\BE \wedge \BB) ) \\
&= \gamma_0 (\BE^2 + c^2 \BB^2 - 2 c (\BE \cross \BB) ) \\
\end{aligned}
\end{equation}

Next, we want an explicit spacetime split of the gradient

\begin{equation}\label{eqn:energyMomentumTensor:600}
\begin{aligned}
\grad \gamma_0
&= (\gamma^0 \partial_0 + \gamma^k \partial_k) \gamma_0 \\
&= \partial_0 - \gamma_k \gamma_0 \partial_k \\
&= \partial_0 - \sigma_k \partial_k \\
&= \partial_0 - \spacegrad \\
\end{aligned}
\end{equation}

We are now in shape to assemble all the intermediate results for the left hand side

\begin{equation}\label{eqn:energyMomentumTensor:620}
\begin{aligned}
\grad \cdot (F \gamma_0 \tilde{F})
&= \gpgradezero{ \grad (F \gamma_0 \tilde{F}) } \\
&= \gpgradezero{ (\partial_0 - \spacegrad) (\BE^2 + c^2 \BB^2 - 2 c (\BE \cross \BB) ) } \\
&= \partial_0 (\BE^2 + c^2 \BB^2) + 2 c \spacegrad \cdot (\BE \cross \BB)
\end{aligned}
\end{equation}

With a final reassembly of the left and right hand sides of
\(\grad \cdot T(\gamma_0)\),
the spacetime divergence of the rest frame stress vector we have
\begin{equation}\label{eqn:energyMomentumTensor:640}
\begin{aligned}
\inv{c} \partial_t (\BE^2 + c^2 \BB^2) + 2 c \spacegrad \cdot (\BE \cross \BB) &= -\frac{2}{c \epsilon_0}\Bj \cdot \BE
\end{aligned}
\end{equation}

Multiplying through by \(\epsilon_0 c/2\) we have the classical Poynting vector energy conservation relationship.

\begin{equation}\label{eqn:energy_momentum_tensor:conservation}
\begin{aligned}
\PD{t}{} \frac{\epsilon_0}{2}(\BE^2 + c^2 \BB^2) + \spacegrad \cdot \inv{\mu_0}(\BE \cross \BB) &= -\Bj \cdot \BE
\end{aligned}
\end{equation}

Observe that the momentum flux density, the Poynting vector \(\BP = (\BE \cross \BB)/\mu_0\),
is zero in the rest frame, which makes sense since there is no magnetic field
for a static charge distribution.  So with no currents and therefore no magnetic fields the field energy is a constant.

\subsection{Transformation properties}

\Eqnref{eqn:energy_momentum_tensor:conservation} is the explicit spacetime expansion of the equivalent relativistic equation

\begin{equation}\label{eqn:energyMomentumTensor:660}
\begin{aligned}
\grad \cdot \left( c T(\gamma_0) \right) &=
\grad \cdot \left(\frac{c \epsilon_0}{2} F \gamma_0 \tilde{F}\right) = \gpgradezero{ J \gamma_0 \tilde{F} }
\end{aligned}
\end{equation}

This has all the same content, but in relativistic form seems almost trivial.  While the stress vector \(T(\gamma_0)\) is not itself
a relativistic invariant, this divergence equation is.

Suppose we form a Lorentz transformation \(\LL(x) = R x \tilde{R}\), applied to this equation we have

\begin{equation}\label{eqn:energyMomentumTensor:680}
\begin{aligned}
F'
&= (R\grad \tilde{R}) \wedge (R A \tilde{R}) \\
&= \gpgradetwo{ R \grad \tilde{R} R A \tilde{R} } \\
&= \gpgradetwo{ R \grad A \tilde{R} } \\
&= R (\grad \wedge A) \tilde{R} \\
&= R F \tilde{R} \\
\end{aligned}
\end{equation}

Transforming all the objects in the equation we have
\begin{equation}\label{eqn:energyMomentumTensor:700}
\begin{aligned}
\grad' \cdot \left(\frac{c \epsilon_0}{2} F' \gamma_0' \tilde{F'} \right) &= \gpgradezero{ J' \gamma_0' \tilde{F'} } \\
(R \grad \tilde{R}) \cdot \left(\frac{c \epsilon_0}{2} R F \tilde{R} R \gamma_0 R \tilde{R} (R F\tilde{R})^{\tilde{}} \right)
&= \gpgradezero{ R J \tilde{R} R \gamma_0 \tilde{R} (R F \tilde{R})^{\tilde{}} } \\
\end{aligned}
\end{equation}

This is nothing more than the original untransformed quantity

\begin{equation}\label{eqn:energyMomentumTensor:720}
\begin{aligned}
\grad \cdot \left(\frac{c \epsilon_0}{2} F \gamma_0 \tilde{F} \right) &= \gpgradezero{ J \gamma_0 \tilde{F} } \\
\end{aligned}
\end{equation}

\section{Validate with relativistic transformation}

As a relativistic quantity we should be able to verify the messy tensor relationship
by Lorentz transforming the energy density from a rest frame to a
moving frame.

Now let us try the Lorentz transformation of the energy density.

FIXME: TODO.
