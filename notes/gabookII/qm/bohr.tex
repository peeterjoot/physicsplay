%
% Copyright � 2012 Peeter Joot.  All Rights Reserved.
% Licenced as described in the file LICENSE under the root directory of this GIT repository.
%

%
%
\chapter{Bohr Model}
\index{Bohr model}
\label{chap:bohr}
%\date{Dec 11, 2008.  bohr.tex}

\section{Motivation}

The Bohr model is taught as early as high school chemistry when the various orbitals are
discussed (or maybe it was high school physics).  I recall
that the first time I saw this I did not see where all the ideas came from.
With a bit more math under my belt now, reexamine these ideas as a lead up to
the proper wave mechanics.

\section{Calculations}

\subsection{Equations of motion}

A prerequisite to discussing electron orbits is first setting up the equations of motion
for the two charged particles (ie: the proton and electron).

With the proton position at \(\Br_p\), and the electron at \(\Br_e\), we have two equations, one
for the force on the proton from the electron and the other for the force on the proton from
the electron.  These are respectively

\begin{equation}\label{eqn:bohr:chargeEquations}
\begin{aligned}
  \K e^2 \frac { \Br_e - \Br_p } { \Abs{\Br_e - \Br_p}^3 } &= m_p \frac{d^2 \Br_p }{dt^2} \\
- \K e^2 \frac { \Br_e - \Br_p } { \Abs{\Br_e - \Br_p}^3 } &= m_e \frac{d^2 \Br_e }{dt^2}
\end{aligned}
\end{equation}

In lieu of a picture, setting \(\Br_p = 0\) works to check signs, leaving an inwards force on the electron as desired.

% FIXME: Add one.
%\begin{figure}[htp]
%\centering
%\includegraphics[totalheight=0.4\textheight]{picturepath}
%\caption{My Caption}\label{fig:pictlabel}
%\end{figure}
%
%... see \cref{fig:picturepath} ...

As usual for a two body problem, use of the difference vector and center of mass vector is desirable.  That is

\begin{equation}\label{eqn:bohr:20}
\begin{aligned}
\Bx &= \Br_e - \Br_p \\
M &= m_e + m_p \\
\BR &= \inv{M}(m_e \Br_e + m_p \Br_p)
\end{aligned}
\end{equation}

Solving for \(\Br_p\) and \(\Br_e\) in terms of \(\BR\) and \(\Bx\) we have

\begin{equation}\label{eqn:bohr:40}
\begin{aligned}
\Br_e &= \frac{m_p}{M} \Bx + \BR \\
\Br_p &= \frac{-m_e}{M} \Bx + \BR \\
\end{aligned}
\end{equation}

% check:
%r_e - r_p = M/M x
%m_e \Br_e + m_p \Br_p &= \frac{-m_e m_p}{M} \Bx + m_e \BR + \frac{m_p m_e}{M} \Bx + m_p \BR \\

Substitution back into \eqnref{eqn:bohr:chargeEquations} we have

\begin{equation}\label{eqn:bohr:60}
\begin{aligned}
  \K e^2 \frac {\Bx} { \Abs{\Bx}^3 } &= m_p \frac{d^2}{dt^2}\left( \frac{-m_e}{M} \Bx + \BR \right) \\
 -\K e^2 \frac {\Bx} { \Abs{\Bx}^3 } &= m_e \frac{d^2}{dt^2}\left( \frac{m_p}{M} \Bx + \BR \right),
\end{aligned}
\end{equation}

and sums and (scaled) differences of that give us our reduced mass equation and constant center-of-mass velocity equation
\begin{equation}\label{eqn:bohr:80}
\begin{aligned}
\frac{d^2 \Bx}{dt^2} &= -\K e^2 \frac {\Bx} { \Abs{\Bx}^3 } \left( \inv{m_e} + \inv{m_p} \right) \\
\frac{d^2 \BR}{dt^2} &= 0
\end{aligned}
\end{equation}

writing \(1/\mu = 1/m_e + 1/m_p\), and \(k = e^2/4 \pi \epsilon_0\), our difference vector equation is thus

\begin{equation}\label{eqn:bohr:reduceEOM}
\begin{aligned}
\mu \frac{d^2 \Bx}{dt^2} &= -k \frac {\Bx} { \Abs{\Bx}^3 }
\end{aligned}
\end{equation}

\subsection{Circular solution}

The Bohr model postulates that electron orbits are circular.  It is easy enough to verify that a circular orbit in the center of mass frame is a solution to equation
\eqnref{eqn:bohr:reduceEOM}.   Write the path in terms of the unit bivector for the plane of rotation \(i\) and an initial vector position \(\Bx_0\)

\begin{equation}\label{eqn:bohr:circular}
\begin{aligned}
\Bx = \Bx_0 e^{i \omega t}
\end{aligned}
\end{equation}

For constant \(i\) and \(\omega\), we have

\begin{equation}\label{eqn:bohr:100}
\begin{aligned}
\mu \Bx_0 (i\omega)^2 e^{i\omega t} = - k \frac{\Bx_0}{\Abs{\Bx_0}^3} e^{i\omega t}
\end{aligned}
\end{equation}

This provides the
angular velocity in terms of the reduced mass of the system and the charge constants

\begin{equation}\label{eqn:bohr:omegaSquared}
\begin{aligned}
\omega^2 = \frac{k}{\mu \Abs{\Bx_0}^3} = \frac{e^2}{4 \pi \epsilon_0 \mu \Abs{\Bx_0}^3}.
\end{aligned}
\end{equation}

Although not relevant to the quantum theme, it is hard not to call out the observation that this is
a Kepler's law like relation for the period of the circular orbit given the radial distance from the center of mass

\begin{equation}\label{eqn:bohr:120}
\begin{aligned}
T^2 = \frac{16 \pi^3 \epsilon_0 \mu}{e^2} \Abs{\Bx_0}^3
\end{aligned}
\end{equation}

Kepler's law also holds for elliptical orbits, but this takes more work to show.

\subsection{Angular momentum conservation}
\index{angular momentum conservation}

Now, the next step in the Bohr argument was that the angular momentum, a conserved quantity is also quantized.  To give real
meaning to the conservation statement we need the equivalent Lagrangian formulation of \eqnref{eqn:bohr:reduceEOM}.  Anti-differentiation
gives

\begin{equation}\label{eqn:bohr:140}
\begin{aligned}
\grad_\Bv \left( \inv{2} \mu \Bv^2 \right)
&= k \xcap \partial_x \inv{x} \\
&= - \grad_\Bx
\mathLabelBox
[
   labelstyle={below of=m\themathLableNode, below of=m\themathLableNode}
]
{\left(- k\inv{\Abs{\Bx}}\right)}{\(=\phi\)}
\end{aligned}
\end{equation}

So, our Lagrangian is
\begin{equation}\label{eqn:bohr:160}
\begin{aligned}
\LL = K - \phi = \inv{2} \mu \Bv^2 + k \inv{\Abs{\Bx}}
\end{aligned}
\end{equation}

The essence of the conservation argument, an application of
Noether's theorem,
is that a rotational transformation of the Lagrangian leaves this energy relationship unchanged.  Repeating
the angular momentum example from \citep{classicalmechanics:PJEulerLagrange} (which was done for the more general case of any radial potential), we
write \(\hat{B}\) for the unit bivector associated with a rotational plane.  The position vector is transformed by rotation in this plane as follows

\begin{equation}\label{eqn:bohr:180}
\begin{aligned}
\Bx &\rightarrow \Bx' \\
\Bx' &= R \Bx R^\dagger \\
R &= \exp{\hat{B}\theta/2}
\end{aligned}
\end{equation}

The magnitude of the position vector is rotation invariant

\begin{equation}\label{eqn:bohr:200}
\begin{aligned}
(\Bx')^2 &= R \Bx R^\dagger R \Bx R^\dagger = \Bx^2,
\end{aligned}
\end{equation}

as is our the square of the transformed velocity.  The transformed velocity is

\begin{equation}\label{eqn:bohr:220}
\begin{aligned}
\frac{d\Bx'}{dt} &= \dot{R} \Bx R + R \dot{\Bx} R^\dagger + R \Bx \dot{R}^\dagger
\end{aligned}
\end{equation}

but with \(\dot{\theta} = 0\), \(\dot{R} = 0\) its square is just

\begin{equation}\label{eqn:bohr:240}
\begin{aligned}
(\Bv')^2 &= R {\Bv} R^\dagger R \dot{\Bv} R^\dagger = \Bv^2.
\end{aligned}
\end{equation}

We therefore have a Lagrangian that is invariant under this rotational transformation

\begin{equation}\label{eqn:bohr:260}
\begin{aligned}
\LL \rightarrow \LL' = \LL,
\end{aligned}
\end{equation}

and by Noether's theorem (essentially application of the chain rule), we have

\begin{equation}\label{eqn:bohr:280}
\begin{aligned}
\frac{d\LL'}{d\theta}
&= \frac{d}{dt} \left(\frac{d\Bx'}{d\theta} \cdot \grad_{\Bv'} \LL \right) \\
&= \frac{d}{dt} \left( (\hat{B} \cdot \Bx') \cdot \mu \Bv' \right).
\end{aligned}
\end{equation}

But \(d\LL'/d\theta = 0\), so we have for any \(\hat{B}\)

\begin{equation}\label{eqn:bohr:300}
\begin{aligned}
(\hat{B} \cdot \Bx') \cdot (\mu \Bv') &= \hat{B} \cdot (\Bx' \wedge (\mu \Bv')) = \text{constant}
\end{aligned}
\end{equation}

Dropping primes this is

\begin{equation}\label{eqn:bohr:320}
\begin{aligned}
L = \Bx \wedge (\mu \Bv) = \text{constant},
\end{aligned}
\end{equation}

a constant bivector for the conserved center of mass (reduced-mass) angular momentum associated with the Lagrangian of this system.

\subsection{Quantized angular momentum for circular solution}

In terms of the circular solution of \eqnref{eqn:bohr:circular} the angular momentum bivector is

\begin{equation}\label{eqn:bohr:340}
\begin{aligned}
L = \Bx \wedge (\mu \Bv)
&= \gpgradetwo{ \Bx_0 e^{i \omega t} \mu \Bx_0 i \omega e^{i \omega t} } \\
&= \gpgradetwo{ e^{-i \omega t} \Bx_0 \mu \Bx_0 \omega e^{i \omega t} i } \\
&= (\Bx_0)^2 \mu \omega i \\
%&= i \frac{e \mu}{2} \sqrt{\frac{\Abs{\Bx_0}}{\pi \epsilon_0 \mu}} \\
&= i e \sqrt{\frac{\mu \Abs{\Bx_0}}{4 \pi \epsilon_0}}
\end{aligned}
\end{equation}

%\begin{align}\label{eqn:bohr:omegaSquared}
%\omega = \frac{e}{2 \sqrt{\pi \epsilon_0}} \Abs{\Bx_0}^{-3/2}

Now if this angular momentum is quantized with quantum magnitude \(l\) we have we have for the bivector angular momentum the values

\begin{equation}\label{eqn:bohr:360}
\begin{aligned}
L = i n l = i e \sqrt{\frac{\mu \Abs{\Bx_0}}{4 \pi \epsilon_0}}
\end{aligned}
\end{equation}

Which with \(l = \Hbar\) (where experiment in the form of the spectral hydrogen line values is required to fix this constant and relate it to Plank's black body constant)
is the momentum equation in terms of
the Bohr radius \(\Bx_0\) at each energy level.  Writing that radius \(r_n = \Abs{\Bx_0}\) explicitly as a function of n, we have

\begin{equation}\label{eqn:bohr:380}
\begin{aligned}
r_n = \frac{4 \pi \epsilon_0}{\mu} \left(\frac{n \Hbar}{e}\right)^2
\end{aligned}
\end{equation}

\subsubsection{Velocity}

One of the assumptions of this treatment is a \(\Abs{\Bv_e} << c\) requirement so that Coulombs law is valid (ie: slow enough that all the other Maxwell's equations can be neglected).
Let us evaluate the velocity numerically at the some of the quantization levels and see how this compares to the speed of light.

First we need an expression for the velocity itself.  This is

\begin{equation}\label{eqn:bohr:400}
\begin{aligned}
\Bv^2
&= ( \Bx_0 i \omega e^{i \omega t} )^2 \\
&= \frac{e^2}{4 \pi \epsilon_0 \mu r_n} \\
&= \frac{e^4}{(4 \pi \epsilon_0)^2 (n \Hbar)^2}.
\end{aligned}
\end{equation}

For
\begin{equation}\label{eqn:bohr:420}
\begin{aligned}
v_n
&= \frac{e^2}{4 \pi \epsilon_0 n \Hbar} \\
&= 2.1 \times 10^6 m/s
\end{aligned}
\end{equation}

This is the \(1/137\) of the speed of light value that one sees googling electron speed in hydrogen, and only decreases with quantum number so the non-relativistic speed approximation holds
(\(\gamma = 1.00002663\)).  This speed is still pretty zippy, even if it is not relativistic, so it is not unreasonable to attempt to repeat this treatment trying to incorporate the remainder
of Maxwell's equations.

Interestingly the velocity is not a function of the reduced mass at all, but just the charge and quantum numbers.  One also gets a good hint at why the Bohr theory breaks down
for larger atoms.  An electron in circular orbit around an ion of Gold would have a velocity of \(79/137\) the speed of light!

% google calculator:
%1/sqrt(1- ((elementary charge)^2 / 4 / pi / epsilon_0 /hbar/c)^2)

% - discuss connection to Sch. results?
% - try: proper maxwell's/Lorentz equations instead of just the Coulomb force.
