%
% Copyright � 2012 Peeter Joot.  All Rights Reserved.
% Licenced as described in the file LICENSE under the root directory of this GIT repository.
%

%
%
%\documentclass[]{eliblog}

\usepackage{amsmath}
\usepackage{mathpazo}

%
% shorthand for bold symbols, convenient for vectors and matrices
%
\newcommand{\Ba}[0]{\mathbf{a}}
\newcommand{\Bb}[0]{\mathbf{b}}
\newcommand{\Bc}[0]{\mathbf{c}}
\newcommand{\Bd}[0]{\mathbf{d}}
\newcommand{\Be}[0]{\mathbf{e}}
\newcommand{\Bf}[0]{\mathbf{f}}
\newcommand{\Bg}[0]{\mathbf{g}}
\newcommand{\Bh}[0]{\mathbf{h}}
\newcommand{\Bi}[0]{\mathbf{i}}
\newcommand{\Bj}[0]{\mathbf{j}}
\newcommand{\Bk}[0]{\mathbf{k}}
\newcommand{\Bl}[0]{\mathbf{l}}
\newcommand{\Bm}[0]{\mathbf{m}}
\newcommand{\Bn}[0]{\mathbf{n}}
\newcommand{\Bo}[0]{\mathbf{o}}
\newcommand{\Bp}[0]{\mathbf{p}}
\newcommand{\Bq}[0]{\mathbf{q}}
\newcommand{\Br}[0]{\mathbf{r}}
\newcommand{\Bs}[0]{\mathbf{s}}
\newcommand{\Bt}[0]{\mathbf{t}}
\newcommand{\Bu}[0]{\mathbf{u}}
\newcommand{\Bv}[0]{\mathbf{v}}
\newcommand{\Bw}[0]{\mathbf{w}}
\newcommand{\Bx}[0]{\mathbf{x}}
\newcommand{\By}[0]{\mathbf{y}}
\newcommand{\Bz}[0]{\mathbf{z}}
\newcommand{\BA}[0]{\mathbf{A}}
\newcommand{\BB}[0]{\mathbf{B}}
\newcommand{\BC}[0]{\mathbf{C}}
\newcommand{\BD}[0]{\mathbf{D}}
\newcommand{\BE}[0]{\mathbf{E}}
\newcommand{\BF}[0]{\mathbf{F}}
\newcommand{\BG}[0]{\mathbf{G}}
\newcommand{\BH}[0]{\mathbf{H}}
\newcommand{\BI}[0]{\mathbf{I}}
\newcommand{\BJ}[0]{\mathbf{J}}
\newcommand{\BK}[0]{\mathbf{K}}
\newcommand{\BL}[0]{\mathbf{L}}
\newcommand{\BM}[0]{\mathbf{M}}
\newcommand{\BN}[0]{\mathbf{N}}
\newcommand{\BO}[0]{\mathbf{O}}
\newcommand{\BP}[0]{\mathbf{P}}
\newcommand{\BQ}[0]{\mathbf{Q}}
\newcommand{\BR}[0]{\mathbf{R}}
\newcommand{\BS}[0]{\mathbf{S}}
\newcommand{\BT}[0]{\mathbf{T}}
\newcommand{\BU}[0]{\mathbf{U}}
\newcommand{\BV}[0]{\mathbf{V}}
\newcommand{\BW}[0]{\mathbf{W}}
\newcommand{\BX}[0]{\mathbf{X}}
\newcommand{\BY}[0]{\mathbf{Y}}
\newcommand{\BZ}[0]{\mathbf{Z}}

\newcommand{\Bzero}[0]{\mathbf{0}}
\newcommand{\Btheta}[0]{\boldsymbol{\theta}}
\newcommand{\Btau}[0]{\boldsymbol{\tau}}
\newcommand{\Bomega}[0]{\boldsymbol{\omega}}

%
% shorthand for unit vectors
%
\newcommand{\acap}[0]{\hat{\Ba}}
\newcommand{\bcap}[0]{\hat{\Bb}}
\newcommand{\ccap}[0]{\hat{\Bc}}
\newcommand{\dcap}[0]{\hat{\Bd}}
\newcommand{\ecap}[0]{\hat{\Be}}
\newcommand{\fcap}[0]{\hat{\Bf}}
\newcommand{\gcap}[0]{\hat{\Bg}}
\newcommand{\hcap}[0]{\hat{\Bh}}
\newcommand{\icap}[0]{\hat{\Bi}}
\newcommand{\jcap}[0]{\hat{\Bj}}
\newcommand{\kcap}[0]{\hat{\Bk}}
\newcommand{\lcap}[0]{\hat{\Bl}}
\newcommand{\mcap}[0]{\hat{\Bm}}
\newcommand{\ncap}[0]{\hat{\Bn}}
\newcommand{\ocap}[0]{\hat{\Bo}}
\newcommand{\pcap}[0]{\hat{\Bp}}
\newcommand{\qcap}[0]{\hat{\Bq}}
\newcommand{\rcap}[0]{\hat{\Br}}
\newcommand{\scap}[0]{\hat{\Bs}}
\newcommand{\tcap}[0]{\hat{\Bt}}
\newcommand{\ucap}[0]{\hat{\Bu}}
\newcommand{\vcap}[0]{\hat{\Bv}}
\newcommand{\wcap}[0]{\hat{\Bw}}
\newcommand{\xcap}[0]{\hat{\Bx}}
\newcommand{\ycap}[0]{\hat{\By}}
\newcommand{\zcap}[0]{\hat{\Bz}}
\newcommand{\thetacap}[0]{\hat{\Btheta}}

%
% to write R^n and C^n in a distinguishable fashion.  Perhaps change this
% to the double lined characters upon figuring out how to do so.
%
\newcommand{\C}[1]{$\mathbb{C}^{#1}$}
\newcommand{\R}[1]{$\mathbb{R}^{#1}$}

%
% various generally useful helpers
%

% derivative of #1 wrt. #2:
\newcommand{\D}[2] {\frac {d#2} {d#1}}

\newcommand{\inv}[1]{\frac{1}{#1}}
\newcommand{\cross}[0]{\times}

\newcommand{\abs}[1]{\lvert{#1}\rvert}
\newcommand{\norm}[1]{\lVert{#1}\rVert}
\newcommand{\innerprod}[2]{\langle{#1}, {#2}\rangle}
\newcommand{\dotprod}[2]{{#1} \cdot {#2}}
\newcommand{\bdotprod}[2]{\left({#1} \cdot {#2}\right)}
\newcommand{\crossprod}[2]{{#1} \cross {#2}}
\newcommand{\tripleprod}[3]{\dotprod{\left(\crossprod{#1}{#2}\right)}{#3}}

\DeclareMathOperator{\Proj}{Proj}
\DeclareMathOperator{\Span}{span}
\DeclareMathOperator{\Sgn}{sgn}
\DeclareMathOperator{\Area}{Area}
\DeclareMathOperator{\Volume}{Volume}

%
% A few miscellaneous things specific to this document
%
\newcommand{\crossop}[1]{\crossprod{#1}{}}

% R2 vector.
\newcommand{\VectorTwo}[2]{
\begin{bmatrix}
 {#1} \\
 {#2}
\end{bmatrix}
}

\newcommand{\VectorN}[1]{
\begin{bmatrix}
{#1}_1 \\
{#1}_2 \\
\vdots \\
{#1}_N \\
\end{bmatrix}
}

\newcommand{\DETuvij}[4]{
\begin{vmatrix}
 {#1}_{#3} & {#1}_{#4} \\
 {#2}_{#3} & {#2}_{#4}
\end{vmatrix}
}

\newcommand{\DETuvwijk}[6]{
\begin{vmatrix}
 {#1}_{#4} & {#1}_{#5} & {#1}_{#6} \\
 {#2}_{#4} & {#2}_{#5} & {#2}_{#6} \\
 {#3}_{#4} & {#3}_{#5} & {#3}_{#6}
\end{vmatrix}
}

\newcommand{\DETuvwxijkl}[8]{
\begin{vmatrix}
 {#1}_{#5} & {#1}_{#6} & {#1}_{#7} & {#1}_{#8} \\
 {#2}_{#5} & {#2}_{#6} & {#2}_{#7} & {#2}_{#8} \\
 {#3}_{#5} & {#3}_{#6} & {#3}_{#7} & {#3}_{#8} \\
 {#4}_{#5} & {#4}_{#6} & {#4}_{#7} & {#4}_{#8} \\
\end{vmatrix}
}

%\newcommand{\DETuvwxyijklm}[10]{
%\begin{vmatrix}
% {#1}_{#6} & {#1}_{#7} & {#1}_{#8} & {#1}_{#9} & {#1}_{#10} \\
% {#2}_{#6} & {#2}_{#7} & {#2}_{#8} & {#2}_{#9} & {#2}_{#10} \\
% {#3}_{#6} & {#3}_{#7} & {#3}_{#8} & {#3}_{#9} & {#3}_{#10} \\
% {#4}_{#6} & {#4}_{#7} & {#4}_{#8} & {#4}_{#9} & {#4}_{#10} \\
% {#5}_{#6} & {#5}_{#7} & {#5}_{#8} & {#5}_{#9} & {#5}_{#10}
%\end{vmatrix}
%}

% R3 vector.
\newcommand{\VectorThree}[3]{
\begin{bmatrix}
 {#1} \\
 {#2} \\
 {#3}
\end{bmatrix}
}



\author{Peeter Joot}
\email{peeter.joot@gmail.com}


\chapter{Maxwell Lagrangian, rotation of coordinates}
\label{chap:maxwellLagRot}
%\useCCL
\blogpage{http://sites.google.com/site/peeterjoot/math2009/maxwellLagRot.pdf}
\date{Sept 5, 2009}
\revisionInfo{maxwellLagRotate.tex }

%\beginArtWithToc
\beginArtNoToc

\section{Maxwell Lagrangian, rotation of coordinates}

The Maxwell Lagrangian, as well as being the one of interest for the energy momentum tensor, is also invariant to coordinate system rotation.  We may write that Lagrangian density as

\begin{equation}\label{eqn:maxwellLagRot:goo1}
\begin{aligned}
\LL = \frac{\epsilon_0}{2} \gpgradezero{(\grad \wedge A)^2} - \inv{c} A \cdot J
\end{aligned}
\end{equation}

Or in coordinate form as

\begin{equation}\label{eqn:maxwellLagRot:goo2}
\begin{aligned}
\LL = -\frac{\epsilon_0}{2} \partial_\mu A_\nu (\partial^\mu A^\nu - \partial^\nu A^\mu) - \inv{c} A_\alpha J^\alpha
\end{aligned}
\end{equation}

To \eqnref{eqn:maxwellLagRot:goo1}, we can apply a rotation of coordinates to both the gradient \(\grad \rightarrow \tilde{R} \grad R\), and the field \(A \rightarrow \tilde{R} A R\).  We have

\begin{equation}\label{eqn:maxwellLagRot:goo3}
\begin{aligned}
\LL' &= \frac{\epsilon_0}{8} \gpgradezero{((\tilde{R} \rgrad R) (\tilde{R} A R) - (\tilde{R} A R) (\tilde{R} \lgrad R))^2} - \inv{c} \gpgradezero{ (\tilde{R} A R) (\tilde{R} J R)}
\end{aligned}
\end{equation}

Employing cyclic commutation \(\gpgradezero{ a b c} = \gpgradezero{ c a b }\), and \(\tilde{R} R = 1\), we have \(\LL' = \LL\), and this Lagrangian therefore has a conserved current specified by \eqnref{eqn:maxwellLagRot:moo6}.

From \eqnref{eqn:maxwellLagRot:goo2} we can calculate (the canonical field momentum?) \(\PDi{(\partial_\mu {A}_\nu)}{\LL}\), and get

\begin{equation}\label{eqn:maxwellLagRot:goo4}
\begin{aligned}
\gamma_\nu \PD{(\partial_\mu {A}_\nu)}{\LL} = -\epsilon_0 \gamma_\nu F^{\mu\nu}
\end{aligned}
\end{equation}

This can be written in terms of the total field \(F = \grad \wedge A\), after observing

\begin{equation}\label{eqn:maxwellLagRotate:40}
\begin{aligned}
\gamma^\mu \cdot F
&=
\gamma^\mu \cdot (\gamma_\alpha \wedge \gamma_\nu \partial^\alpha A^\nu) \\
&=
(\gamma_\nu {\delta^\mu}_\alpha - \gamma_\alpha {\delta^\mu}_\nu ) \partial^\alpha A^\nu \\
&=
\gamma_\nu F^{\mu\nu}
\end{aligned}
\end{equation}

%\begin{align}\label{eqn:maxwellLagRot:goo5}
%\gamma_\nu \PD{(\partial_\mu {A}_\nu)}{\LL} = -\epsilon_0 (\gamma^\mu \cdot F)
%\end{align}

This leaves us so far with the Noether current conservation of

\begin{equation}\label{eqn:maxwellLagRot:goo6}
\begin{aligned}
0 = -\epsilon_0 \partial_\mu \left( (\gamma^\mu \cdot F) \cdot {\left. \PD{\theta}{A'} \right\vert}_{\theta=0}  \right)
\end{aligned}
\end{equation}

The specifics of the Lorentz transformation used in the coordinate transformation weare not required to get this far, but if we assume a singly parametrized rotation or boost we can use an exponential representation for the rotation

\begin{equation}\label{eqn:maxwellLagRot:goo7}
\begin{aligned}
R &= e^{i\theta/2} \\
A' &= \tilde{R} A R
\end{aligned}
\end{equation}

The change of the field variable \(A'\) with respect to theta (a rotation angle or boost rapidity) is then

\begin{equation}\label{eqn:maxwellLagRotate:60}
\begin{aligned}
\frac{dA'}{d\theta}
&= \frac{d\tilde{R}}{d\theta} A R + \tilde{R} A \frac{d R}{d\theta} \\
&= \frac{d\tilde{R}}{d\theta} R A' + A' R \tilde{R} \frac{d R}{d\theta} \\
&= \inv{2} \left( -i A' + A' i \right) \\
\end{aligned}
\end{equation}

But this is just a vector-bivector dot product

\begin{equation}\label{eqn:maxwellLagRot:goo8}
\begin{aligned}
\frac{dA'}{d\theta} &= A' \cdot i
\end{aligned}
\end{equation}

Evaluating at \(\theta = 0\) we have just \(A \cdot i\), and we are left with the conservation statement

\begin{equation}\label{eqn:maxwellLagRot:goo9}
\begin{aligned}
0 = \partial_\sigma \left( (\gamma^\sigma \cdot F) \cdot (A \cdot i) \right)
\end{aligned}
\end{equation}

As a scalar we can write this multiple dot product explicitly using scalar selection, and factor out the bivector \(i\)

\begin{equation}\label{eqn:maxwellLagRotate:80}
\begin{aligned}
\partial_\sigma \left( (\gamma^\sigma \cdot F) \cdot (A \cdot i) \right)
&=
\partial_\sigma \gpgradezero{ (\gamma^\sigma \cdot F) \cdot (A \cdot i) } \\
&=
\inv{2} \partial_\sigma \gpgradezero{ (\gamma^\sigma \cdot F) (A i - i A) } \\
&=
\inv{2} \partial_\sigma \gpgradezero{ ((\gamma^\sigma \cdot F) A -A (\gamma^\sigma \cdot F)) i } \\
\end{aligned}
\end{equation}

This leaves us with a bivector conservation statement

\begin{equation}\label{eqn:maxwellLagRot:goo10}
\begin{aligned}
0 = \left( \partial_\sigma (\gamma^\sigma \cdot F) \wedge A \right) \cdot i
\end{aligned}
\end{equation}

If this is zero for all spatial and spacetime planes \(i\) we must then have

\begin{equation}\label{eqn:maxwellLagRot:goo11}
\begin{aligned}
0 = \partial_\sigma \left( (\gamma^\sigma \cdot F) \wedge A \right)
\end{aligned}
\end{equation}

Can we factor out the \(\gamma^\sigma\) from this leaving some function of \(F\), and \(A\)?  Borrowing an identity from \citep{hestenes1999nfc} (eqn 1.14), we have

\begin{equation}\label{eqn:maxwellLagRot:goo16}
\begin{aligned}
( \gamma^\sigma \cdot F ) \wedge A = \gamma^\sigma \cdot (A \wedge F ) - (\gamma^\sigma \cdot A) F
\end{aligned}
\end{equation}

This allows us to write the conservation current equation in coordinate free form

\begin{equation}\label{eqn:maxwellLagRotate:100}
\begin{aligned}
0
&= \partial_\sigma \left( (\gamma^\sigma \cdot F) \wedge A \right) \\
&= \partial_\sigma \gamma^\sigma \cdot (A \wedge F ) - \partial_\sigma (\gamma^\sigma \cdot A) F \\
&= \grad \cdot (A \wedge F ) - (\grad \cdot A) F \\
\end{aligned}
\end{equation}

In the second term the gradient acts both on \(A\) and \(F\).  Expanding and reordering for clarity

\begin{equation}\label{eqn:maxwellLagRot:goo17}
\begin{aligned}
0 &= \grad \cdot (A \wedge F ) - F (\grad \cdot A) - (A \cdot \grad) F
\end{aligned}
\end{equation}

It looks like \eqnref{eqn:maxwellLagRot:goo17} can be further reduced.

FIXME... trying this I get:

\begin{equation}\label{eqn:maxwellLagRot:goo20}
\begin{aligned}
( F \cdot \grad ) \wedge A = (\grad \cdot F) \wedge A
\end{aligned}
\end{equation}

note that \(\grad \cdot F = J/\epsilon_0 c\), so we have \(J \wedge A\) on the right, but something much different seeming on the left.  Suspect that I got it wrong, but if I did get it right, this would be nothing more than another way of expressing the EOF.  END FIXME NOTE.

Let us come back to that after first relating \eqnref{eqn:maxwellLagRot:goo11} to the canonical energy momentum tensor.  Doing so will allow us to name this quantity (or at least give it a symbol).  To do so, note that this is really six different conservation relationships since it must be separately true for each component of this bivector.   Dotting this with the unit bivectors \(\gamma^\nu \wedge \gamma^\mu\) will provide an equivalent set of six tensor equations.  Those follow messily, but directly,

\begin{equation}\label{eqn:maxwellLagRotate:120}
\begin{aligned}
0
&=
\partial_\sigma \left( (\gamma^\sigma \cdot F) \wedge A \right)  \cdot (\gamma^\nu \wedge \gamma^\mu) \\
&=
\partial_\sigma (F^{\sigma\alpha} A^\beta) (\gamma^\alpha \wedge \gamma_\beta) \cdot (\gamma^\nu \wedge \gamma^\mu) \\
&=
\partial_\sigma (F^{\sigma\mu} A^\nu -F^{\sigma\nu} A^\mu )
\end{aligned}
\end{equation}

The six conservation equations, one for each \(\mu \ne \nu\), are thus

\begin{equation}\label{eqn:maxwellLagRot:goo12}
\begin{aligned}
\partial_\sigma F^{\sigma\mu} A^\nu = \partial_\sigma F^{\sigma\nu} A^\mu
\end{aligned}
\end{equation}

Jackson \citep{jackson1975cew} labels, but does not name, these left and right side quantities

\begin{equation}\label{eqn:maxwellLagRot:goo13}
\begin{aligned}
{T_D}^{\mu\nu} \equiv \partial_\sigma F^{\sigma\mu} A^\nu
\end{aligned}
\end{equation}

and uses these to construct the symmetric energy momentum tensor from the canonical tensor by subtraction.  With this label we have as a consequence of rotation invariance of the Lagrangian density

\begin{equation}\label{eqn:maxwellLagRot:goo14}
\begin{aligned}
{T_D}^{\mu\nu} = {T_D}^{\nu\mu}
\end{aligned}
\end{equation}

It is not at all obvious that this tensor should be symmetric, and it is in perhaps somewhat strange since subtracting this from the canonical tensor produced the symmetric?  A better read of that difficult Jackson chapter is in order as well as a look for errors above.

\EndArticle
%\EndNoBibArticle
