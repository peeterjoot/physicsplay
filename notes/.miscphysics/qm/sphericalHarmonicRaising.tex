%
% Copyright � 2012 Peeter Joot.  All Rights Reserved.
% Licenced as described in the file LICENSE under the root directory of this GIT repository.
%

%
%
%\documentclass[]{eliblog}

\usepackage{amsmath}
\usepackage{mathpazo}

%
% shorthand for bold symbols, convenient for vectors and matrices
%
\newcommand{\Ba}[0]{\mathbf{a}}
\newcommand{\Bb}[0]{\mathbf{b}}
\newcommand{\Bc}[0]{\mathbf{c}}
\newcommand{\Bd}[0]{\mathbf{d}}
\newcommand{\Be}[0]{\mathbf{e}}
\newcommand{\Bf}[0]{\mathbf{f}}
\newcommand{\Bg}[0]{\mathbf{g}}
\newcommand{\Bh}[0]{\mathbf{h}}
\newcommand{\Bi}[0]{\mathbf{i}}
\newcommand{\Bj}[0]{\mathbf{j}}
\newcommand{\Bk}[0]{\mathbf{k}}
\newcommand{\Bl}[0]{\mathbf{l}}
\newcommand{\Bm}[0]{\mathbf{m}}
\newcommand{\Bn}[0]{\mathbf{n}}
\newcommand{\Bo}[0]{\mathbf{o}}
\newcommand{\Bp}[0]{\mathbf{p}}
\newcommand{\Bq}[0]{\mathbf{q}}
\newcommand{\Br}[0]{\mathbf{r}}
\newcommand{\Bs}[0]{\mathbf{s}}
\newcommand{\Bt}[0]{\mathbf{t}}
\newcommand{\Bu}[0]{\mathbf{u}}
\newcommand{\Bv}[0]{\mathbf{v}}
\newcommand{\Bw}[0]{\mathbf{w}}
\newcommand{\Bx}[0]{\mathbf{x}}
\newcommand{\By}[0]{\mathbf{y}}
\newcommand{\Bz}[0]{\mathbf{z}}
\newcommand{\BA}[0]{\mathbf{A}}
\newcommand{\BB}[0]{\mathbf{B}}
\newcommand{\BC}[0]{\mathbf{C}}
\newcommand{\BD}[0]{\mathbf{D}}
\newcommand{\BE}[0]{\mathbf{E}}
\newcommand{\BF}[0]{\mathbf{F}}
\newcommand{\BG}[0]{\mathbf{G}}
\newcommand{\BH}[0]{\mathbf{H}}
\newcommand{\BI}[0]{\mathbf{I}}
\newcommand{\BJ}[0]{\mathbf{J}}
\newcommand{\BK}[0]{\mathbf{K}}
\newcommand{\BL}[0]{\mathbf{L}}
\newcommand{\BM}[0]{\mathbf{M}}
\newcommand{\BN}[0]{\mathbf{N}}
\newcommand{\BO}[0]{\mathbf{O}}
\newcommand{\BP}[0]{\mathbf{P}}
\newcommand{\BQ}[0]{\mathbf{Q}}
\newcommand{\BR}[0]{\mathbf{R}}
\newcommand{\BS}[0]{\mathbf{S}}
\newcommand{\BT}[0]{\mathbf{T}}
\newcommand{\BU}[0]{\mathbf{U}}
\newcommand{\BV}[0]{\mathbf{V}}
\newcommand{\BW}[0]{\mathbf{W}}
\newcommand{\BX}[0]{\mathbf{X}}
\newcommand{\BY}[0]{\mathbf{Y}}
\newcommand{\BZ}[0]{\mathbf{Z}}

\newcommand{\Bzero}[0]{\mathbf{0}}
\newcommand{\Btheta}[0]{\boldsymbol{\theta}}
\newcommand{\Btau}[0]{\boldsymbol{\tau}}
\newcommand{\Bomega}[0]{\boldsymbol{\omega}}

%
% shorthand for unit vectors
%
\newcommand{\acap}[0]{\hat{\Ba}}
\newcommand{\bcap}[0]{\hat{\Bb}}
\newcommand{\ccap}[0]{\hat{\Bc}}
\newcommand{\dcap}[0]{\hat{\Bd}}
\newcommand{\ecap}[0]{\hat{\Be}}
\newcommand{\fcap}[0]{\hat{\Bf}}
\newcommand{\gcap}[0]{\hat{\Bg}}
\newcommand{\hcap}[0]{\hat{\Bh}}
\newcommand{\icap}[0]{\hat{\Bi}}
\newcommand{\jcap}[0]{\hat{\Bj}}
\newcommand{\kcap}[0]{\hat{\Bk}}
\newcommand{\lcap}[0]{\hat{\Bl}}
\newcommand{\mcap}[0]{\hat{\Bm}}
\newcommand{\ncap}[0]{\hat{\Bn}}
\newcommand{\ocap}[0]{\hat{\Bo}}
\newcommand{\pcap}[0]{\hat{\Bp}}
\newcommand{\qcap}[0]{\hat{\Bq}}
\newcommand{\rcap}[0]{\hat{\Br}}
\newcommand{\scap}[0]{\hat{\Bs}}
\newcommand{\tcap}[0]{\hat{\Bt}}
\newcommand{\ucap}[0]{\hat{\Bu}}
\newcommand{\vcap}[0]{\hat{\Bv}}
\newcommand{\wcap}[0]{\hat{\Bw}}
\newcommand{\xcap}[0]{\hat{\Bx}}
\newcommand{\ycap}[0]{\hat{\By}}
\newcommand{\zcap}[0]{\hat{\Bz}}
\newcommand{\thetacap}[0]{\hat{\Btheta}}

%
% to write R^n and C^n in a distinguishable fashion.  Perhaps change this
% to the double lined characters upon figuring out how to do so.
%
\newcommand{\C}[1]{$\mathbb{C}^{#1}$}
\newcommand{\R}[1]{$\mathbb{R}^{#1}$}

%
% various generally useful helpers
%

% derivative of #1 wrt. #2:
\newcommand{\D}[2] {\frac {d#2} {d#1}}

\newcommand{\inv}[1]{\frac{1}{#1}}
\newcommand{\cross}[0]{\times}

\newcommand{\abs}[1]{\lvert{#1}\rvert}
\newcommand{\norm}[1]{\lVert{#1}\rVert}
\newcommand{\innerprod}[2]{\langle{#1}, {#2}\rangle}
\newcommand{\dotprod}[2]{{#1} \cdot {#2}}
\newcommand{\bdotprod}[2]{\left({#1} \cdot {#2}\right)}
\newcommand{\crossprod}[2]{{#1} \cross {#2}}
\newcommand{\tripleprod}[3]{\dotprod{\left(\crossprod{#1}{#2}\right)}{#3}}

\DeclareMathOperator{\Proj}{Proj}
\DeclareMathOperator{\Span}{span}
\DeclareMathOperator{\Sgn}{sgn}
\DeclareMathOperator{\Area}{Area}
\DeclareMathOperator{\Volume}{Volume}

%
% A few miscellaneous things specific to this document
%
\newcommand{\crossop}[1]{\crossprod{#1}{}}

% R2 vector.
\newcommand{\VectorTwo}[2]{
\begin{bmatrix}
 {#1} \\
 {#2}
\end{bmatrix}
}

\newcommand{\VectorN}[1]{
\begin{bmatrix}
{#1}_1 \\
{#1}_2 \\
\vdots \\
{#1}_N \\
\end{bmatrix}
}

\newcommand{\DETuvij}[4]{
\begin{vmatrix}
 {#1}_{#3} & {#1}_{#4} \\
 {#2}_{#3} & {#2}_{#4}
\end{vmatrix}
}

\newcommand{\DETuvwijk}[6]{
\begin{vmatrix}
 {#1}_{#4} & {#1}_{#5} & {#1}_{#6} \\
 {#2}_{#4} & {#2}_{#5} & {#2}_{#6} \\
 {#3}_{#4} & {#3}_{#5} & {#3}_{#6}
\end{vmatrix}
}

\newcommand{\DETuvwxijkl}[8]{
\begin{vmatrix}
 {#1}_{#5} & {#1}_{#6} & {#1}_{#7} & {#1}_{#8} \\
 {#2}_{#5} & {#2}_{#6} & {#2}_{#7} & {#2}_{#8} \\
 {#3}_{#5} & {#3}_{#6} & {#3}_{#7} & {#3}_{#8} \\
 {#4}_{#5} & {#4}_{#6} & {#4}_{#7} & {#4}_{#8} \\
\end{vmatrix}
}

%\newcommand{\DETuvwxyijklm}[10]{
%\begin{vmatrix}
% {#1}_{#6} & {#1}_{#7} & {#1}_{#8} & {#1}_{#9} & {#1}_{#10} \\
% {#2}_{#6} & {#2}_{#7} & {#2}_{#8} & {#2}_{#9} & {#2}_{#10} \\
% {#3}_{#6} & {#3}_{#7} & {#3}_{#8} & {#3}_{#9} & {#3}_{#10} \\
% {#4}_{#6} & {#4}_{#7} & {#4}_{#8} & {#4}_{#9} & {#4}_{#10} \\
% {#5}_{#6} & {#5}_{#7} & {#5}_{#8} & {#5}_{#9} & {#5}_{#10}
%\end{vmatrix}
%}

% R3 vector.
\newcommand{\VectorThree}[3]{
\begin{bmatrix}
 {#1} \\
 {#2} \\
 {#3}
\end{bmatrix}
}



\author{Peeter Joot}
\email{peeter.joot@gmail.com}


\chapter{Spherical harmonic Eigenfunctions by application of the raising operator}
\label{chap:sphericalHarmonicRaising}
%\useCCL
\blogpage{http://sites.google.com/site/peeterjoot/math2009/sphericalHarmonicRaising.pdf}
\date{Aug 18, 2009}
\revisionInfo{sphericalHarmonicRaising.tex }

%\beginArtWithToc
\beginArtNoToc

\section{Motivation}

In Bohm's QT (\citep{bohm1989qt}, the following spherical harmonic eigenfunctions of the raising operator are found

\begin{equation}\label{eqn:sphericalHarmonicRaising:foo0}
\begin{aligned}
\psi_l^{l-s} = \frac{e^{i(l-s)\phi}}{(1-\zeta^2)^{(l-s)/2}} \frac{\partial^s}{\partial \zeta^s} (1-\zeta^2)^l
\end{aligned}
\end{equation}

This (unnormalized) result (with \(\zeta = \cos\theta\)) is valid for \(s \in [0,l]\).  As an exercise do this by applying the raising operator to \(\psi_l^{-l}\).  This should help verify the result (unproven or unclear if proven) that the \(\psi_l^m\) and \(\psi_l^{-m}\) eigenfunctions differ only by a sign in the \(\phi\) phase term.

\section{Guts}

The staring point, with \(C\) for \(\cos\) and \(S\) for \(\sin\), will be equations (15) from the text

\begin{equation}\label{eqn:sphericalHarmonicRaising:24}
\begin{aligned}
L_z/\Hbar &= -i \partial_\phi \\
L_x/\Hbar &= i (S_\phi \partial_\theta + \cot\theta C_\phi \partial_\phi) \\
L_y/\Hbar &= -i (C_\phi \partial_\theta - \cot\theta S_\phi \partial_\phi)
\end{aligned}
\end{equation}

From these the raising and lowering operators (setting \(\Hbar=1\)) are respectively

\begin{equation}\label{eqn:sphericalHarmonicRaising:44}
\begin{aligned}
L_x \pm iL_y
&=
i (S_\phi \partial_\theta + \cot\theta C_\phi \partial_\phi)
\pm (C_\phi \partial_\theta - \cot\theta S_\phi \partial_\phi) \\
&= e^{\pm i\phi} (\pm \partial_\theta + i \cot\theta \partial_\phi )
\end{aligned}
\end{equation}

So, if we are after solutions to

\begin{equation}\label{eqn:sphericalHarmonicRaising:foo1}
\begin{aligned}
(L_x \pm iL_y) \psi_l^{\pm l} = 0
\end{aligned}
\end{equation}

and require of these \(\psi_l^{\pm l} = e^{\pm i l \phi} f_l^{\pm l}(\theta)\), then we want solutions of

\begin{equation}\label{eqn:sphericalHarmonicRaising:foo2}
\begin{aligned}
(\pm \partial_\theta \pm i^2 l \cot\theta ) f_l^{\pm l} = 0
\end{aligned}
\end{equation}

or
\begin{equation}\label{eqn:sphericalHarmonicRaising:foo3}
\begin{aligned}
\pm (\partial_\theta - l \cot\theta ) f_l^{\pm l} = 0
\end{aligned}
\end{equation}

What I wanted to demonstrate to myself, that the \(\theta\) dependence is the same for \(\psi_l^m\) as it is for \(\psi_l^{-m}\) is therefore true from \eqnref{eqn:sphericalHarmonicRaising:foo3} for the first case with \(m=l\).  We will need to apply the raising operator to \(\psi_l^{-l}\) to verify that this is the case for the rest of the indices \(m\).

To continue we need to integrate for \(f_l^{\pm l}\)

\begin{equation}\label{eqn:sphericalHarmonicRaising:64}
\begin{aligned}
\int \frac{d f}{f} = l \int \cot\theta d\theta
\end{aligned}
\end{equation}

Which integrates to

\begin{equation}\label{eqn:sphericalHarmonicRaising:84}
\begin{aligned}
\ln(f) = l \ln(\sin\theta) + \ln(\kappa)
\end{aligned}
\end{equation}

With exponentiation we have

\begin{equation}\label{eqn:sphericalHarmonicRaising:104}
\begin{aligned}
f = \kappa (\sin\theta)^l
\end{aligned}
\end{equation}

and have

\begin{equation}\label{eqn:sphericalHarmonicRaising:foo4}
\begin{aligned}
\psi_l^{\pm l} = e^{\pm i l \phi} (\sin\theta)^l
\end{aligned}
\end{equation}

Now are now set to apply the raising operator to \(\psi_l^{-l}\).

\begin{equation}\label{eqn:sphericalHarmonicRaising:124}
\begin{aligned}
(L_x + iL_y) \psi_l^{-l}
&=
e^{i\phi} (\partial_\theta + i \cot\theta \partial_\phi) \psi_l^{-l} \\
&=
e^{i\phi} (\partial_\theta + i \cot\theta (-i l)) \psi_l^{-l} \\
&=
e^{i\phi} (\partial_\theta + l \cot\theta ) \psi_l^{-l} \\
\end{aligned}
\end{equation}

Now comes the sneaky trick from the text used in the lowering application argument.  I am not sure how to guess this one, but playing it backwards we find the differential operator above

\begin{equation}\label{eqn:sphericalHarmonicRaising:144}
\begin{aligned}
\inv{(\sin\theta)^l} \PD{\theta}{} \left( (\sin\theta)^l \psi_l^{\pm l} \right)
&=
\inv{(\sin\theta)^l} \left( l (\sin\theta)^{l-1} \cos\theta + (\sin\theta)^l \partial_\theta \right) \psi_l^{\pm l} \\
&=
\inv{(\sin\theta)^l} \left( l (\sin\theta)^{l} \cot\theta + (\sin\theta)^l \partial_\theta \right) \psi_l^{\pm l} \\
\end{aligned}
\end{equation}

That gives the sneaky identity
\begin{equation}\label{eqn:sphericalHarmonicRaising:sneaky}
\begin{aligned}
\inv{(\sin\theta)^l} \PD{\theta}{} \left( (\sin\theta)^l \psi_l^{\pm l} \right)
&=
\left( l \cot\theta + \partial_\theta \right) \psi_l^{\pm l}
\end{aligned}
\end{equation}

Backsubstution gives

\begin{equation}\label{eqn:sphericalHarmonicRaising:164}
\begin{aligned}
(L_x + iL_y) \psi_l^{-l}
&=
e^{i\phi} \inv{(\sin\theta)^l} \PD{\theta}{} \left( (\sin\theta)^l \psi_l^{-l} \right) \\
\end{aligned}
\end{equation}

For
\begin{equation}\label{eqn:sphericalHarmonicRaising:minusLplus1}
\begin{aligned}
\psi_l^{1-l}
&=
e^{i(1-l)\phi} \inv{(\sin\theta)^l} \PD{\theta}{} (\sin\theta)^{2l}
\end{aligned}
\end{equation}

For a second raising operator application we have
\begin{equation}\label{eqn:sphericalHarmonicRaising:184}
\begin{aligned}
(L_x + iL_y) \psi_l^{1-l}
&=
e^{i\phi} (\partial_\theta + i \cot\theta \partial_\phi) \psi_l^{1-l} \\
&=
e^{i\phi} (\partial_\theta + i \cot\theta (-i)(l-1)) \psi_l^{1-l} \\
&=
e^{i\phi} (\partial_\theta + (l-1)\cot\theta ) \psi_l^{1-l} \\
\end{aligned}
\end{equation}

A second application of the sneaky identity \eqnref{eqn:sphericalHarmonicRaising:sneaky} gives us

\begin{equation}\label{eqn:sphericalHarmonicRaising:204}
\begin{aligned}
\psi_l^{2-l}
&=
e^{i\phi} \inv{(\sin\theta)^{l-1}} \PD{\theta}{} \left( (\sin\theta)^{l-1} \psi_l^{1-l} \right) \\
&=
e^{i(2-l)\phi} \inv{(\sin\theta)^{l-1}} \PD{\theta}{} \left( (\sin\theta)^{l-1}
\inv{(\sin\theta)^l} \PD{\theta}{} (\sin\theta)^{2l}
\right) \\
&=
e^{i(2-l)\phi} \inv{(\sin\theta)^{l-1}} \PD{\theta}{} \left(
\inv{\sin\theta} \PD{\theta}{} (\sin\theta)^{2l}
\right) \\
&=
e^{i(2-l)\phi} \frac{\sin\theta}{(\sin\theta)^{l-1}} \inv{\sin\theta}\PD{\theta}{} \left(
\inv{\sin\theta} \PD{\theta}{} (\sin\theta)^{2l}
\right) \\
\end{aligned}
\end{equation}

This gives

\begin{equation}\label{eqn:sphericalHarmonicRaising:2minusL}
\begin{aligned}
\psi_l^{2-l}
&=
e^{i(2-l)\phi} \inv{(\sin\theta)^{l-2}} \left( \inv{\sin\theta}\PD{\theta}{} \right)^2 (\sin\theta)^{2l}
\end{aligned}
\end{equation}

A comparison with \(\phi_l^{1-l}\) from \eqnref{eqn:sphericalHarmonicRaising:minusLplus1} shows that the induction hypothesis is

\begin{equation}\label{eqn:sphericalHarmonicRaising:224}
\begin{aligned}
\psi_l^{s-l}
&=
e^{i(s-l)\phi} \inv{(\sin\theta)^{l-s}} \left( \inv{\sin\theta}\PD{\theta}{} \right)^s (\sin\theta)^{2l}
\end{aligned}
\end{equation}

\subsection{The induction}

The induction, starting with cut-and-paste-regex replacement,

\begin{equation}\label{eqn:sphericalHarmonicRaising:244}
\begin{aligned}
(L_x + iL_y) \psi_l^{{s-1}-l}
&=
e^{i\phi} (\partial_\theta + i \cot\theta \partial_\phi) \psi_l^{{s-1}-l} \\
&=
e^{i\phi} (\partial_\theta + i \cot\theta (-i)(l -(s-1)) \psi_l^{{s-1}-l} \\
&=
e^{i\phi} (\partial_\theta + (l -(s-1)))\cot\theta ) \psi_l^{(s-1)-l} \\
\end{aligned}
\end{equation}

A second application of the sneaky identity \eqnref{eqn:sphericalHarmonicRaising:sneaky} gives us
%\psi_l^{s-1-l}
%&=
%e^{i(s-1-l)\phi} \inv{(\sin\theta)^{l-(s-1)}} \left( \inv{\sin\theta}\PD{\theta}{} \right)^{s-1} (\sin\theta)^{2l}

\begin{equation}\label{eqn:sphericalHarmonicRaising:264}
\begin{aligned}
\psi_l^{s-l}
&=
e^{i\phi} \inv{(\sin\theta)^{l-(s-1)}} \PD{\theta}{} \left( (\sin\theta)^{l-(s-1)} \psi_l^{(s-1)-l} \right) \\
&=
e^{i\phi} \inv{(\sin\theta)^{l-(s-1)}} \PD{\theta}{} \left( (\sin\theta)^{l-(s-1)}
%\psi_l^{(s-1)-l}
e^{i(s-1-l)\phi} \inv{(\sin\theta)^{l-(s-1)}} \left( \inv{\sin\theta}\PD{\theta}{} \right)^{s-1} (\sin\theta)^{2l}
\right) \\
&=
e^{i(s-l)\phi}
\frac{\sin\theta}{(\sin\theta)^{l-(s-1)}} \inv{\sin\theta} \PD{\theta}{} \left(
\left( \inv{\sin\theta}\PD{\theta}{} \right)^{s-1} (\sin\theta)^{2l}
\right) \\
\end{aligned}
\end{equation}

% s = 2
% s - 1 = 1
This completes the induction arriving at the negative index equivalent of Bohm's equation (46), and as claimed in the text this differs only by sign of the \(\phi\) exponential

\begin{equation}\label{eqn:sphericalHarmonicRaising:sminusL}
\begin{aligned}
\psi_l^{s-l}
&=
e^{i(s-l)\phi} \inv{(\sin\theta)^{l-s}} \left( \inv{\sin\theta}\PD{\theta}{} \right)^s (\sin\theta)^{2l}
\end{aligned}
\end{equation}

\EndArticle
%\EndNoBibArticle
