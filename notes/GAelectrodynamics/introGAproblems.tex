%
% Copyright � 2016 Peeter Joot.  All Rights Reserved.
% Licenced as described in the file LICENSE under the root directory of this GIT repository.
%
%%{
%\newcommand{\authorname}{Peeter Joot}
\newcommand{\email}{peeterjoot@protonmail.com}
\newcommand{\basename}{FIXMEbasenameUndefined}
\newcommand{\dirname}{notes/FIXMEdirnameUndefined/}

%\renewcommand{\basename}{introGAproblems.tex}
%%\renewcommand{\dirname}{notes/phy1520/}
%\renewcommand{\dirname}{notes/ece1228-electromagnetic-theory/}
%%\newcommand{\dateintitle}{}
%%\newcommand{\keywords}{}
%
%\newcommand{\authorname}{Peeter Joot}
\newcommand{\onlineurl}{http://sites.google.com/site/peeterjoot2/math2013/\basename.pdf}
\newcommand{\sourcepath}{\dirname\basename.tex}
\newcommand{\generatetitle}[1]{\chapter{#1}}

\newcommand{\vcsinfo}{%
\section*{}
\noindent{\color{DarkOliveGreen}{\rule{\linewidth}{0.1mm}}}
\paragraph{Document version}
%\paragraph{\color{Maroon}{Document version}}
{
\small
\begin{itemize}
\item Available online at:\\ 
\href{\onlineurl}{\onlineurl}
\item Git Repository: \input{./.revinfo/gitRepo.tex}
\item Source: \sourcepath
\item last commit: \input{./.revinfo/gitCommitString.tex}
\item commit date: \input{./.revinfo/gitCommitDate.tex}
\end{itemize}
}
}

%\PassOptionsToPackage{dvipsnames,svgnames}{xcolor}
\PassOptionsToPackage{square,numbers}{natbib}
\documentclass{scrreprt}

\usepackage[left=2cm,right=2cm]{geometry}
\usepackage[svgnames]{xcolor}
\usepackage{peeters_layout}

\usepackage{natbib}

\usepackage[
colorlinks=true,
bookmarks=false,
pdfauthor={\authorname, \email},
backref 
]{hyperref}

% http://tex.stackexchange.com/questions/75773/how-to-reference-problems-by-the-text-label-in-an-exercise-envioronment
\usepackage[english]{cleveref}
\crefname{Exercise}{exercise}{exercises}
\Crefname{Exercise}{Exercise}{Exercises}

\RequirePackage{titlesec}
\RequirePackage{ifthen}

% http://stackoverflow.com/questions/4932910/date-in-the-tabular-environment
\makeatletter
\let\insertdate\@date
\makeatother

\titleformat{\chapter}[display]
{\bfseries\Large}
{\color{DarkSlateGrey}\filleft \authorname
\ifthenelse{\isundefined{\studentnumber}}{}{\\ \studentnumber}
\ifthenelse{\isundefined{\email}}{}{\\ \email}
\ifthenelse{\isundefined{\dateintitle}}{}{\\ \insertdate}
%\ifthenelse{\isundefined{\coursename}}{}{\\ \coursename} % put in title instead.
}
{4ex}
{\color{DarkOliveGreen}{\titlerule}\color{Maroon}
\vspace{2ex}%
\filright}
[\vspace{2ex}%
\color{DarkOliveGreen}\titlerule
]

\newcommand{\beginArtWithToc}[0]{\begin{document}\tableofcontents}
\newcommand{\beginArtNoToc}[0]{\begin{document}}
\newcommand{\EndNoBibArticle}[0]{\end{document}}
\newcommand{\EndArticle}[0]{\bibliography{Bibliography}\bibliographystyle{plainnat}\end{document}}

% 
%\newcommand{\citep}[1]{\cite{#1}}

\colorSectionsForArticle


%
%\usepackage{peeters_layout_exercise}
%\usepackage{peeters_braket}
%\usepackage{peeters_figures}
%\usepackage{siunitx}
%%\usepackage{mhchem} % \ce{}
%%\usepackage{macros_bm} % \bcM
%%\usepackage{txfonts} % \ointclockwise
%
%\beginArtNoToc
%
%\generatetitle{XXX}
%%\chapter{XXX}
%%\label{chap:introGAproblems.tex}
%% \citep{sakurai2014modern} pr X.Y
%% \citep{pozar2009microwave}
%% \citep{qftLectureNotes}
%% \citep{doran2003gap}
%% \citep{jackson1975cew}
%% \citep{griffiths1999introduction}
%

\makeanswer{problem:introGAproblems:ComplexInnerProductVsDotAndCrossProduct}{
\begin{dmath}\label{eqn:introGAproblems:20}
z w^\conj
=
(a + ib)(a' - ib')
=
a a' + b b'
+ i \lr{ a' b - a b' }
\leftrightarrow
( a a' + b b', a' b - a b' ).
\end{dmath}
} % answer

\makeanswer{problem:multiplication:2dvectorsquare}{
Consider the 2D case to start with

\begin{dmath}\label{eqn:multiplication:120}
\Bx^2
=
\lr{ x \Be_1 + y \Be_2}
\lr{ x \Be_1 + y \Be_2}
=
\lr{ x \Be_1 } \lr{ x \Be_1 }
+
\lr{ y \Be_2 } \lr{ y \Be_2 }
+
\lr{ x \Be_1 } \lr{ y \Be_2 }
+
\lr{ y \Be_2 } \lr{ x \Be_1 }
=
x^2 \Be_1^2
+
y^2 \Be_2^2
+
x y \lr{ \Be_1 \Be_2 + \Be_2 \Be_1 }
=
x^2 + y^2
+
x y \lr{ \Be_1 \Be_2 + \Be_2 \Be_1 }.
\end{dmath}

The contraction axiom requires the bivector terms to sum to zero, as also demonstrated previously for the specific example \( \Bx = \Be_1 + \Be_2 \).

More generally for \R{N}

\begin{dmath}\label{eqn:multiplication:121}
\Bx^2
=
\lr{ \sum_i x_i \Be_i }
\lr{ \sum_j x_j \Be_j }
=
\sum_{ij} x_i x_j \Be_i \Be_j
=
\sum_{i = j} x_i x_j \Be_i \Be_j
+
\sum_{i \ne j} x_i x_j \Be_i \Be_j
=
\sum_{i} x_i^2
+
\sum_{i \ne j} x_i x_j \Be_i \Be_j
=
\sum_{i} x_i^2
+
\sum_{i < j} x_i x_j (\Be_i \Be_j + \Be_j \Be_i).
\end{dmath}

The contraction axiom requires all the bivector pairs to sum to zero.  That is, for each \( i \ne j \)

\begin{dmath}\label{eqn:introGAproblems:140}
\Be_i \Be_j = -\Be_j \Be_i.
\end{dmath}
} % answer

\makeanswer{problem:multiplication:unitsquare}{
Consider the square of a vector \( \Bx = u \Bu + v \Bv \) with respect to a basis of unit vectors \( \setlr{ \Bu, \Bv }\).  That is

\begin{dmath}\label{eqn:introGAproblems:160}
\Bx^2
=
\lr{ u \Bu + v \Bv }
\lr{ u \Bu + v \Bv }
=
u^2 \Bu^2
+ v^2 \Bv^2
+ u v \lr{ \Bu \Bv + \Bv \Bu }
=
u^2
+ v^2
+ u v \lr{ \Bu \Bv + \Bv \Bu }.
\end{dmath}

If these vectors are normal \( \Bx^2 = u^2 + v^2 \), which means
\begin{dmath}\label{eqn:introGAproblems:180}
\Bu \Bv = -\Bv \Bu.
\end{dmath}

Observe that a side effect of this computation shows that the traditional vector dot product of two unit vectors can also be written as a symmetric bivector sum

\begin{dmath}\label{eqn:introGAproblems:200}
\Bu \cdot \Bv = \inv{2} \lr{ \Bu \Bv + \Bv \Bu }.
\end{dmath}
} % answer

%%}
%\EndArticle
%%\EndNoBibArticle
