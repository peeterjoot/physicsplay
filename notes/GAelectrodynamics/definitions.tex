%
% Copyright � 2016 Peeter Joot.  All Rights Reserved.
% Licenced as described in the file LICENSE under the root directory of this GIT repository.
%
%{
%%\newcommand{\authorname}{Peeter Joot}
\newcommand{\email}{peeterjoot@protonmail.com}
\newcommand{\basename}{FIXMEbasenameUndefined}
\newcommand{\dirname}{notes/FIXMEdirnameUndefined/}

%%\renewcommand{\basename}{multiplication}
%%%\renewcommand{\dirname}{notes/phy1520/}
%%\renewcommand{\dirname}{notes/ece1228-electromagnetic-theory/}
%%%\newcommand{\dateintitle}{}
%%%\newcommand{\keywords}{}
%%
%%\newcommand{\authorname}{Peeter Joot}
\newcommand{\onlineurl}{http://sites.google.com/site/peeterjoot2/math2013/\basename.pdf}
\newcommand{\sourcepath}{\dirname\basename.tex}
\newcommand{\generatetitle}[1]{\chapter{#1}}

\newcommand{\vcsinfo}{%
\section*{}
\noindent{\color{DarkOliveGreen}{\rule{\linewidth}{0.1mm}}}
\paragraph{Document version}
%\paragraph{\color{Maroon}{Document version}}
{
\small
\begin{itemize}
\item Available online at:\\ 
\href{\onlineurl}{\onlineurl}
\item Git Repository: \input{./.revinfo/gitRepo.tex}
\item Source: \sourcepath
\item last commit: \input{./.revinfo/gitCommitString.tex}
\item commit date: \input{./.revinfo/gitCommitDate.tex}
\end{itemize}
}
}

%\PassOptionsToPackage{dvipsnames,svgnames}{xcolor}
\PassOptionsToPackage{square,numbers}{natbib}
\documentclass{scrreprt}

\usepackage[left=2cm,right=2cm]{geometry}
\usepackage[svgnames]{xcolor}
\usepackage{peeters_layout}

\usepackage{natbib}

\usepackage[
colorlinks=true,
bookmarks=false,
pdfauthor={\authorname, \email},
backref 
]{hyperref}

% http://tex.stackexchange.com/questions/75773/how-to-reference-problems-by-the-text-label-in-an-exercise-envioronment
\usepackage[english]{cleveref}
\crefname{Exercise}{exercise}{exercises}
\Crefname{Exercise}{Exercise}{Exercises}

\RequirePackage{titlesec}
\RequirePackage{ifthen}

% http://stackoverflow.com/questions/4932910/date-in-the-tabular-environment
\makeatletter
\let\insertdate\@date
\makeatother

\titleformat{\chapter}[display]
{\bfseries\Large}
{\color{DarkSlateGrey}\filleft \authorname
\ifthenelse{\isundefined{\studentnumber}}{}{\\ \studentnumber}
\ifthenelse{\isundefined{\email}}{}{\\ \email}
\ifthenelse{\isundefined{\dateintitle}}{}{\\ \insertdate}
%\ifthenelse{\isundefined{\coursename}}{}{\\ \coursename} % put in title instead.
}
{4ex}
{\color{DarkOliveGreen}{\titlerule}\color{Maroon}
\vspace{2ex}%
\filright}
[\vspace{2ex}%
\color{DarkOliveGreen}\titlerule
]

\newcommand{\beginArtWithToc}[0]{\begin{document}\tableofcontents}
\newcommand{\beginArtNoToc}[0]{\begin{document}}
\newcommand{\EndNoBibArticle}[0]{\end{document}}
\newcommand{\EndArticle}[0]{\bibliography{Bibliography}\bibliographystyle{plainnat}\end{document}}

% 
%\newcommand{\citep}[1]{\cite{#1}}

\colorSectionsForArticle


%%
%%\usepackage{peeters_layout_exercise}
%%\usepackage{peeters_braket}
%%\usepackage{peeters_figures}
%%\usepackage{siunitx}
%%%\usepackage{mhchem} % \ce{}
%%%\usepackage{macros_bm} % \bcM
%%%\usepackage{txfonts} % \ointclockwise
%%
%%\beginArtNoToc
%%
%%\generatetitle{Vector multiplication}
%%%\chapter{Vector multiplication}
%%%\label{chap:multiplication}
%%

A few new GA terms have been introduced in an ad-hoc fashion as required.  Here is a systematic exposition of some of the key definitions used to refer to the types of the geometric objects that will be encountered.

\makedefinition{Scalar}{def:multiplication:scalar}{
A real number with no implied direction.
}

\makedefinition{Vector}{def:multiplication:vector}{
\href{https://www.youtube.com/watch?v=bOIe0DIMbI8}{A quantity with direction and magnitude.}
}

\makedefinition{Bivector}{def:multiplication:bivector}{
A product of two normal vectors, or a sum thereof.
}

The product \( \Be_1 \Be_2 \) is a bivector, as is \( \Be_2 \Be_3 + 3 \Be_4 \Be_1 \)

\makedefinition{Trivector}{def:multiplication:trivector}{
A product of three mutually normal vectors, or a sum thereof.
}

The quantity \( \Be_3 \Be_1 \Be_2 \) is a trivector, as is \( \Be_1 \Be_2 \Be_3 + 3 \Be_5 \Be_4 \Be_1 \).

\makedefinition{Blade}{def:multiplication:blade}{
A scalar, vector, bivector, or a trivector (or higher degree analogue), that can be constructed by multiplication of a number of vectors, but not an unfactorable sum of thereof.
}

The factorable quantity
\begin{dmath}\label{eqn:multiplication:220}
\Be_1 \Be_2 + 3 \Be_1 \Be_3
=
\Be_1 (\Be_2 + 3 \Be_3)
\end{dmath}

is a blade, whereas

\begin{dmath}\label{eqn:multiplication:240}
\Be_1 \Be_2 + 3 \Be_3 \Be_4,
\end{dmath}

and
\begin{dmath}\label{eqn:multiplication:260}
\Be_1 \Be_2 + \Be_2 \Be_3 + \Be_3 \Be_1
\end{dmath}

are not.

Scalar, vector, bivector, and trivectors are also referred to as sums of 0-blades, 1-blades, 2-blades, and 3-blades respectively.

\makedefinition{Grade.}{def:multiplication:grade}{
The minimum number of vector products required to form a given blade.
}

The grade of a scalar, vector, bivector, and trivector are 0, 1, 2, and 3 respectively.

The quantities
\begin{dmath}\label{eqn:multiplication:300}
\begin{aligned}
\Be_2 + \Be_1 \Be_2 \Be_2 &= \Be_1 + \Be_2 \\
\Be_1 \Be_2 \Be_2 \Be_2 \Be_3 &= \Be_1 \Be_2 \Be_3 \\
\end{aligned}
\end{dmath}

have grades 1 and 3 respectively.

Quantities with higher grades than 3 are not generally given explicit names, but can be referred to having grade-k.  When an object of grade-k is
also a blade, it can be referred to as a k-blade.

In a three dimensional space the highest grade possible is 3.  Blades can have grades higher than 3 in higher dimensional vector spaces.

\makedefinition{Pseudoscalar.}{def:multiplication:pseudoscalar}{
A blade with grade that matches the dimension of the space.
}

In a two dimensional space \( \Be_2 \Be_1 \) is a pseudoscalar.  In a three dimensional space
\( \Be_3 \Be_1 \Be_2 \) is a pseudoscalar, as is \( \Be_3 \Be_1 (\Be_2 + \Be_3 ) \).  A pseudoscalar has an implied orientation, which can be
associated with the handedness of the underlying basis.  It is conventional to refer to

\begin{dmath}\label{eqn:definitions:320}
i = \Be_1 \Be_2,
\end{dmath}

as ``the pseudoscalar'' for a two dimensional space, and to

\begin{dmath}\label{eqn:definitions:340}
I = \Be_1 \Be_2 \Be_3,
\end{dmath}

as ``the pseudoscalar'' for a three dimensional space.

\makedefinition{Multivector.}{def:multiplication:multivector}{
A sum of zero or more blades.
}

Examples include
\begin{dmath}\label{eqn:multiplication:280}
\begin{aligned}
&3 \\
& 1 + \Be_1 \Be_2 \\
& 2 - \Be_1 \Be_2 \Be_3 \\
& \Be_1 + 2 \Be_1 \Be_2 + \Be_2 \Be_3 - 3 \Be_3 \Be_1 + \Be_1 \Be_2 \Be_4
\end{aligned}
\end{dmath}

\makedefinition{Dual}{dfn:definitions:dual}{
FIXME: todo.
} % definition

\subsection{Problems}

\makeproblem{\R{3} pseudoscalar square}{problem:gradeselection:R3PseudoscalarSquare}{
With the \R{3} pseudoscalar of \cref{eqn:definitions:340} show that \( I^2 = -1 \).
} % problem

%%%}
%%%\EndArticle
%%\EndNoBibArticle
