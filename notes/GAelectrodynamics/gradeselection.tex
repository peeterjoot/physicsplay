%
% Copyright � 2016 Peeter Joot.  All Rights Reserved.
% Licenced as described in the file LICENSE under the root directory of this GIT repository.
%
%{
\newcommand{\authorname}{Peeter Joot}
\newcommand{\email}{peeterjoot@protonmail.com}
\newcommand{\basename}{FIXMEbasenameUndefined}
\newcommand{\dirname}{notes/FIXMEdirnameUndefined/}

\renewcommand{\basename}{gradeselection}
%\renewcommand{\dirname}{notes/phy1520/}
\renewcommand{\dirname}{notes/ece1228-electromagnetic-theory/}
%\newcommand{\dateintitle}{}
%\newcommand{\keywords}{}

\newcommand{\authorname}{Peeter Joot}
\newcommand{\onlineurl}{http://sites.google.com/site/peeterjoot2/math2013/\basename.pdf}
\newcommand{\sourcepath}{\dirname\basename.tex}
\newcommand{\generatetitle}[1]{\chapter{#1}}

\newcommand{\vcsinfo}{%
\section*{}
\noindent{\color{DarkOliveGreen}{\rule{\linewidth}{0.1mm}}}
\paragraph{Document version}
%\paragraph{\color{Maroon}{Document version}}
{
\small
\begin{itemize}
\item Available online at:\\ 
\href{\onlineurl}{\onlineurl}
\item Git Repository: \input{./.revinfo/gitRepo.tex}
\item Source: \sourcepath
\item last commit: \input{./.revinfo/gitCommitString.tex}
\item commit date: \input{./.revinfo/gitCommitDate.tex}
\end{itemize}
}
}

%\PassOptionsToPackage{dvipsnames,svgnames}{xcolor}
\PassOptionsToPackage{square,numbers}{natbib}
\documentclass{scrreprt}

\usepackage[left=2cm,right=2cm]{geometry}
\usepackage[svgnames]{xcolor}
\usepackage{peeters_layout}

\usepackage{natbib}

\usepackage[
colorlinks=true,
bookmarks=false,
pdfauthor={\authorname, \email},
backref 
]{hyperref}

% http://tex.stackexchange.com/questions/75773/how-to-reference-problems-by-the-text-label-in-an-exercise-envioronment
\usepackage[english]{cleveref}
\crefname{Exercise}{exercise}{exercises}
\Crefname{Exercise}{Exercise}{Exercises}

\RequirePackage{titlesec}
\RequirePackage{ifthen}

% http://stackoverflow.com/questions/4932910/date-in-the-tabular-environment
\makeatletter
\let\insertdate\@date
\makeatother

\titleformat{\chapter}[display]
{\bfseries\Large}
{\color{DarkSlateGrey}\filleft \authorname
\ifthenelse{\isundefined{\studentnumber}}{}{\\ \studentnumber}
\ifthenelse{\isundefined{\email}}{}{\\ \email}
\ifthenelse{\isundefined{\dateintitle}}{}{\\ \insertdate}
%\ifthenelse{\isundefined{\coursename}}{}{\\ \coursename} % put in title instead.
}
{4ex}
{\color{DarkOliveGreen}{\titlerule}\color{Maroon}
\vspace{2ex}%
\filright}
[\vspace{2ex}%
\color{DarkOliveGreen}\titlerule
]

\newcommand{\beginArtWithToc}[0]{\begin{document}\tableofcontents}
\newcommand{\beginArtNoToc}[0]{\begin{document}}
\newcommand{\EndNoBibArticle}[0]{\end{document}}
\newcommand{\EndArticle}[0]{\bibliography{Bibliography}\bibliographystyle{plainnat}\end{document}}

% 
%\newcommand{\citep}[1]{\cite{#1}}

\colorSectionsForArticle



\usepackage{peeters_layout_exercise}
\usepackage{peeters_braket}
\usepackage{peeters_figures}
\usepackage{siunitx}
%\usepackage{mhchem} % \ce{}
%\usepackage{macros_bm} % \bcM
\usepackage{macros_qed} % \qedmarker
%\usepackage{txfonts} % \ointclockwise

\beginArtNoToc

\generatetitle{XXX}
%\chapter{XXX}
%\label{chap:gradeselection}
% \citep{sakurai2014modern} pr X.Y
% \citep{pozar2009microwave}
% \citep{qftLectureNotes}
% \citep{doran2003gap}
% \citep{jackson1975cew}
% \citep{griffiths1999introduction}

Having defined the axioms and definitions of Geometric Algebra, it desirable to define the grade selection operator, the dot product operator and the wedge product operator, and consider some simple examples of each.

\makedefinition{Grade selection operator}{dfn:gradeselection:gradeselection}{
Given a multivector \( M \) containing k-grade components \( M_k \)

\begin{equation*}
M = \sum_{i = 0}^N M_i,
\end{equation*}

the grade selection operator is defined as

\begin{equation*}\label{eqn:gradeselection:40}
\gpgrade{M}{k} \equiv M_k.
\end{equation*}

Selection of the (scalar) zero grade is often written as
\begin{equation*}
\gpgradezero{M} \equiv \gpgrade{M}{0} = M_0.
\end{equation*}
}

For example, if \( M = 3 - \Be_3 + 2 \Be_1 \Be_2 \), then
\begin{equation}\label{eqn:gradeselection:80}
\begin{aligned}
\gpgradezero{M} &= 3 \\
\gpgrade{M}{1} &= - \Be_3 \\
\gpgrade{M}{2} &= 2 \Be_1 \Be_2 \\
\gpgrade{M}{3} &= 0.
\end{aligned}
\end{equation}

\makedefinition{Dot product}{dfn:gradeselection:100}{
The dot product of two multivectors

\begin{equation*}
\begin{aligned}
A &= \sum_{i = 0}^N A_i, \\
B &= \sum_{i = 0}^N B_i,
\end{aligned}
\end{equation*}

is defined as
\begin{equation*}
A \cdot B \equiv
\sum_{i,j = 0}^N \gpgrade{ A_i B_j }{\Abs{i - j}}
\end{equation*}
} % definition

As an example, consider two vectors in a 2D space

\begin{dmath}\label{eqn:gradeselection:140}
\begin{aligned}
\Ba  &= \lr{ x \Be_1 + y \Be_2 } \\
\Ba' &= \lr{ x' \Be_1 + y' \Be_2 },
\end{aligned}
\end{dmath}

for which this definition of the dot product gives

\begin{dmath}\label{eqn:gradeselection:160}
\Ba \cdot \Ba'
=
\gpgrade{ \Ba \Ba' }{\Abs{1 - 1}}
=
\gpgradezero{ \Ba \Ba' }
=
\gpgradezero{ \lr{ x \Be_1 + y \Be_2 } \lr{ x' \Be_1 + y' \Be_2 } }
=
\gpgradezero{ x x' \Be_1^2 + y y' \Be_2^2 + (x y' - y x') \Be_1 \Be_2 }
=
x x' + y y'.
\end{dmath}

It is left to the reader (\cref{problem:gradeselection:1}) to show that this definition also reduces to the traditional \R{n} dot product.

As a second example, consider the dot product of a vector with a bivector.  With \( \Ba \) as defined in \cref{eqn:gradeselection:140} and \( i = \Be_1 \Be_2 \)

\begin{dmath}\label{eqn:gradeselection:240}
\Ba \cdot i
=
\gpgrade{ \Ba i }{1}
=
\gpgrade{ \lr{ x \Be_1 + y \Be_2 } \Be_1 \Be_2 }{1}
=
\gpgrade{ x \Be_1^2 \Be_2 + y \Be_2 (-\Be_2 \Be_1) }{1}
=
\gpgrade{ x \Be_2 - y \Be_1 }{1}
=
x \Be_2 - y \Be_1.
\end{dmath}

This particular dot product is trivial, since the product \( \Ba i \) has only a vector component.
In this example \( i \) is the pseudoscalar for the two dimensional space, and it can be observed that multiplication of a vector from the right serves to rotate the vector by 90 degrees.  It is not a coincidence that this is strikingly similar to the action of the imaginary from complex algebra.  It can be shown (\cref{problem:gradeselection:3})
that \( e^{i\theta} \) acts as a rotation operator as it does in complex algebra, and that a GA representation of complex numbers is possible (\cref{problem:gradeselection:2}).

\makedefinition{Outer product.}{dfn:gradeselection:480}{
For the multivectors \( A, B \) defined in \cref{dfn:gradeselection:100}, the outer product (or wedge product) is defined as

is defined as
\begin{equation*}
A \wedge B
\equiv
\sum_{i,j = 0}^N \gpgrade{ A_i B_j }{i + j}.
\end{equation*}
} % definition

For example, the wedge product of the 2D vectors of \cref{eqn:gradeselection:140} is

\begin{dmath}\label{eqn:gradeselection:500}
\Ba \wedge \Bb
=
\gpgradetwo{
\lr{ x \Be_1 + y \Be_2 } 
\lr{ x' \Be_1 + y' \Be_2 }
}
=
\gpgradetwo{
(x x' + y y') + (x y' - x' y) \Be_1 \Be_2
}
=
(x y' - x' y) \Be_1 \Be_2.
\end{dmath}

The wedge product of two vectors in a plane contains an antisymetrized sum of the vector coefficients, but is weighted by a ``unit'' bivector, the pseudoscalar for the plane.

As another example consider

\begin{dmath}\label{eqn:gradeselection:520}
\Be_1 \wedge \lr{ 2\Be_1 + 3 \Be_2 }
=
\gpgradetwo{
\Be_1 \lr{ 2\Be_1 + 3 \Be_2 }
}
=
\gpgradetwo{
2 \Be_1^2 + 3 \Be_1 \Be_2 
}
=
3 \Be_1 \Be_2.
\end{dmath}

Here we see that the component of the second vector \( \Be_1 + 3 \Be_2 \) that is colinear with the first vector \( \Be_1 \) is filtered out.  It is not coincidence that this is also a property of the cross product.  That relationship will be explored in (\cref{problem:gradeselection:4}).

As a final example, consider the wedge product of a vector with a bivector

\begin{dmath}\label{eqn:gradeselection:540}
\Be_1 \wedge \lr{ \Be_1 \Be_2 - 7 \Be_2 \Be_3 }
=
\gpgradethree{
\Be_1 \lr{ \Be_1 \Be_2 - 7 \Be_2 \Be_3 }
}
=
\gpgradethree{
\Be_1^2 \Be_2 - 7 \Be_1 \Be_2 \Be_3 
}
=
- 7 \Be_1 \Be_2 \Be_3.
\end{dmath}

Because \( \Be_1 \Be_2 \) has a common factor with \( \Be_1 \) it is filtered out of the resulting wedge product.  The end result, in this case, is a 3D pseudoscalar.

\section{Problems}

\makeproblem{\R{n} dot product.}{problem:gradeselection:1}{
Show that \ref{dfn:gradeselection:100} when applied to two vectors
is equivalent to the traditional \R{n} dot product.
} % problem

\makeanswer{problem:gradeselection:1}{
Let
\begin{dmath}\label{eqn:gradeselection:180}
\begin{aligned}
\Bx &= \sum_{i=1}^N x_i \Be_i \\
\By &= \sum_{i=1}^N y_i \Be_i.
\end{aligned}
\end{dmath}

The dot product of these two vectors is
\begin{dmath}\label{eqn:gradeselection:200}
\Bx \cdot \By 
\equiv
\gpgradezero{ \Bx \By }
=
\gpgradezero{ 
\lr{ \sum_{i=1}^N x_i \Be_i}
\lr{ \sum_{j=1}^N y_j \Be_j}
}
=
\sum_{1 \le i = j \le N}
x_i y_j 
\gpgradezero{ \Be_i \Be_j }
+
\sum_{1 \le i \ne j \le N}
x_i y_j 
\gpgradezero{ \Be_i \Be_j }
\end{dmath}

In the \( i = j \) sum, the term \( \Be_i \Be_j = \Be_i^2 = 1 \), so the scalar grade selection of that multivector product is just 1.  In the \( i = j \) term, each of the \( \Be_i \Be_j \) products is a bivector, so each of those scalar grade selections is zero.

That leaves

\begin{dmath}\label{eqn:gradeselection:220}
\Bx \cdot \By 
=
\sum_{i =1}^N x_i y_i. \qedmarker
\end{dmath}
} % answer

\makeproblem{Complex numbers}{problem:gradeselection:2}{
Show that complex numbers can be represented as even grade multivectors \( z = a + \Be_1 \Be_2 b \).
} % problem

\makeanswer{problem:gradeselection:2}{
Let \( i = \Be_1 \Be_2 \), for which we find

\begin{dmath}\label{eqn:gradeselection:260}
i^2 
= 
\lr{ \Be_1 \Be_2 }
\lr{ \Be_1 \Be_2 }
=
\Be_1 (\Be_2 \Be_1) \Be_2 
=
\Be_1 (-\Be_1 \Be_2) \Be_2 
=
-(\Be_1^2) (\Be_2^2)
=
-1.
\end{dmath}

The even grade multivector \( z = a + i b \) is thus seen to have all the properties required of complex numbers.
} % answer

\makeproblem{Plane rotations.}{problem:gradeselection:3}{

With \( i = \Be_1 \Be_2 \) for the pseudoscalar of the \( x,y \) plane,

\makesubproblem{}{problem:gradeselection:3:b}
justify the assertion that \( e^{i \theta} = \cos\theta + i \sin\theta \), where \( theta \) is a scalar angle.

\makesubproblem{}{problem:gradeselection:3:c}
Show that right multiplication of a 2D vector by \( e^{i\theta} \) rotates that vector by \( \theta \) radians.

\makesubproblem{}{problem:gradeselection:3:d}
Does the rotation multivector \( e^{i\theta} \) commute with the 2D basis vectors?

\makesubproblem{}{problem:gradeselection:3:e}
What is the action of multiplication of a vector by \( e^{i\theta} \) from the left?
} % problem

\makeanswer{problem:gradeselection:3}{
\makeSubAnswer{}{problem:gradeselection:3:b}

Assume that the exponential of a multivector argument is represented by a Taylor series

\begin{dmath}\label{eqn:gradeselection:280}
e^X = \sum_{k = 0}^\infty \frac{X^k}{k!},
\end{dmath}

and note that the pseudoscalar commutes with scalar rotation angles \( \theta \), so
\begin{dmath}\label{eqn:gradeselection:300}
e^{i\theta} 
= \sum_{k = 0}^\infty \frac{(i\theta)^k}{k!}
= \sum_{k = 0}^\infty \frac{i^k\theta^k}{k!}
= 
\sum_{k = 0}^\infty \frac{i^{2k}\theta^{2k}}{(2k)!}
+
\sum_{k = 0}^\infty \frac{i^{2k + 1}\theta^{2k +1}}{(2k + 1)!}
= 
\sum_{k = 0}^\infty \frac{(-1)^{k}\theta^{2k}}{(2k)!}
+
i \sum_{k = 0}^\infty \frac{(-1)^{k}\theta^{2k +1}}{(2k + 1)!}
= \cos \theta + i \sin\theta.
\end{dmath}
\makeSubAnswer{}{problem:gradeselection:3:c}

Consider the action of the exponential on each of the unit vectors.  For \( \Be_1 \) that is

\begin{dmath}\label{eqn:gradeselection:320}
\Be_1 e^{i \theta}
=
\Be_1 \lr{ \cos\theta + i \sin\theta }
=
\Be_1 \cos\theta + \Be_1 (\Be_1 \Be_2 )\sin\theta 
=
\Be_1 \cos\theta + \Be_2 \sin\theta.
\end{dmath}

This shows that the vector \( \Be_1 \) is rotated counterclockwise by \( \theta \) radians.  Similarly for \( \Be_2 \) 

\begin{dmath}\label{eqn:gradeselection:340}
\Be_2 e^{i \theta}
=
\Be_2 \lr{ \cos\theta + i \sin\theta }
=
\Be_2 \cos\theta + \Be_1 (\Be_1 \Be_2 )\sin\theta 
=
\Be_2 \cos\theta + \Be_1 (-\Be_2 \Be_1) \sin\theta.
=
\Be_2 \cos\theta - \Be_1 \sin\theta.
\end{dmath}

This is also a rotation by \( \theta \) radians.  Given a vector \( \Bx = x \Be_1 + y \Be_2 \), this gives

\begin{dmath}\label{eqn:gradeselection:360}
\Bx' 
= \Bx e^{i\theta}
=
x \lr{ \Be_1 \cos\theta + \Be_2 \sin\theta } + y \lr{ \Be_2 \cos\theta - \Be_1 \sin\theta }.
\end{dmath}

In particular

\begin{dmath}\label{eqn:gradeselection:380}
\begin{bmatrix}
\Bx' \cdot \Be_1 \\
\Bx' \cdot \Be_2 \\
\end{bmatrix}
=
\begin{bmatrix}
x \cos\theta - y \sin\theta \\
x \sin\theta + y \cos\theta 
\end{bmatrix}
=
\begin{bmatrix}
\cos\theta &- \sin\theta \\
\sin\theta &+ \cos\theta 
\end{bmatrix}
\begin{bmatrix}
x \\
y
\end{bmatrix}.
\end{dmath}

Observe that this is the rotation matrix that takes the points \((x, y)\) to their position \((x', y')\) rotated by \( \theta \) radians.
\makeSubAnswer{}{problem:gradeselection:3:d}

The action from the left on \( \Be_1 \) is

\begin{dmath}\label{eqn:gradeselection:400}
e^{i\theta} \Be_1
=
\lr{ \cos\theta + \Be_1 \Be_2 \sin\theta} \Be_1
=
\Be_1 \cos\theta + \Be_1 \Be_2 \Be_1 \sin\theta
=
\Be_1 \cos\theta + \Be_1 (-\Be_1 \Be_2) \sin\theta
=
\Be_1 \lr{ \cos\theta - i \sin\theta }
=
\Be_1 e^{-i\theta},
\end{dmath}

and the action from the left on \( \Be_2 \) is

\begin{dmath}\label{eqn:gradeselection:420}
e^{i\theta} \Be_2
=
\lr{ \cos\theta + \Be_1 \Be_2 \sin\theta} \Be_2
=
\Be_2 \cos\theta + \Be_1 \sin\theta
=
\Be_2 \cos\theta + (\Be_2 \Be_2) \Be_1 \sin\theta
=
\Be_2 \lr{ \cos\theta - i \sin\theta }
=
\Be_2 e^{-i\theta}.
\end{dmath}

This change of sign is due to the fact that the pseudoscalar anticommutes with each of the basis vectors in the plane.

\makeSubAnswer{}{problem:gradeselection:3:e}

Since the exponential toggles sign on commutation with both of the vectors of the plane, the rotation operation can be applied from either left or right, with sufficient care to get the direction right

\begin{equation}\label{eqn:gradeselection:440}
\Bx e^{i\theta} = e^{-i\theta} \Bx.
\end{equation}

It is also possible to split the rotation operation into half angle rotation operators that act from both the left and right

\begin{dmath}\label{eqn:gradeselection:460}
\Bx' = e^{-i\theta/2} \Bx e^{i\theta/2}.
\end{dmath}

A student who has studied computer graphics rotation theory may have seen quaterion rotation operators with this form, and a student of quantum mechanics will have seen Pauli matrix rotation operations of this form.  This is, in fact, the form that is generally desirable for 3D or higher order rotations, since it rotates the portions of a vector that lie in the rotation plane, leaving the normal components untouched.
} % answer

\makeproblem{Wedge relationship to the cross product.}{problem:gradeselection:4}{
For a pair of \R{3} vectors \( \Bx, \By \), show that the wedge and cross products are related by
\begin{dmath}\label{eqn:gradeselection:560}
\Bx \wedge \By = I (\Bx \cross \By),
\end{dmath}

where \( I = \Be_1 \Be_2 \Be_3 \) is the \R{N} pseudoscalar.
} % problem

\makeanswer{problem:gradeselection:4}{
Given \( \Bx = \sum_i x_i \Be_i \), and \( \By = \sum_i y_i \Be_i \), the wedge of these two vectors is a grade two selection that picks out only products that differ in index

\begin{dmath}\label{eqn:gradeselection:580}
\Bx \wedge \By
=
\gpgradetwo{ \Bx \By }
=
\sum_{i,j} \gpgradetwo{ x_i \Be_i y_j \Be_j }
=
\sum_{i \ne j} x_i y_j \gpgradetwo{ \Be_i \Be_j }
=
\sum_{i \ne j} x_i y_j \Be_i \Be_j
=
\sum_{i < j} (x_i y_j - x_j y_i) \Be_i \Be_j.
\end{dmath}

Written out explicitly, this is

\begin{dmath}\label{eqn:gradeselection:600}
\begin{aligned}
\Bx \wedge \By
&=
(x_1 y_2 - x_2 y_1) \Be_1 \Be_2 \\
&\quad+
(x_1 y_3 - x_3 y_1) \Be_1 \Be_3 \\
&\quad+
(x_2 y_3 - x_3 y_2) \Be_2 \Be_3 \\
&=
(x_1 y_2 - x_2 y_1) \Be_1 \Be_2 (\Be_3 \Be_3) \\
&\quad+
(x_1 y_3 - x_3 y_1) \Be_1 \Be_3 (\Be_2 \Be_2) \\
&\quad+
(x_2 y_3 - x_3 y_2) \Be_2 \Be_3 (\Be_1 \Be_1) \\
&=
(x_1 y_2 - x_2 y_1) I \Be_3 \\
&\quad+
(x_1 y_3 - x_3 y_1) (-I) \Be_2 \\
&\quad+
(x_2 y_3 - x_3 y_2) I \Be_1 \\
&= I (\Bx \cross \By).
\end{aligned}
\end{dmath}
} % answer

%}
\EndArticle
%\EndNoBibArticle
