%
% Copyright � 2016 Peeter Joot.  All Rights Reserved.
% Licenced as described in the file LICENSE under the root directory of this GIT repository.
%
%{
%\newcommand{\authorname}{Peeter Joot}
\newcommand{\email}{peeterjoot@protonmail.com}
\newcommand{\basename}{FIXMEbasenameUndefined}
\newcommand{\dirname}{notes/FIXMEdirnameUndefined/}

%\renewcommand{\basename}{gradeselection}
%%\renewcommand{\dirname}{notes/phy1520/}
%\renewcommand{\dirname}{notes/ece1228-electromagnetic-theory/}
%%\newcommand{\dateintitle}{}
%%\newcommand{\keywords}{}
%
%\newcommand{\authorname}{Peeter Joot}
\newcommand{\onlineurl}{http://sites.google.com/site/peeterjoot2/math2013/\basename.pdf}
\newcommand{\sourcepath}{\dirname\basename.tex}
\newcommand{\generatetitle}[1]{\chapter{#1}}

\newcommand{\vcsinfo}{%
\section*{}
\noindent{\color{DarkOliveGreen}{\rule{\linewidth}{0.1mm}}}
\paragraph{Document version}
%\paragraph{\color{Maroon}{Document version}}
{
\small
\begin{itemize}
\item Available online at:\\ 
\href{\onlineurl}{\onlineurl}
\item Git Repository: \input{./.revinfo/gitRepo.tex}
\item Source: \sourcepath
\item last commit: \input{./.revinfo/gitCommitString.tex}
\item commit date: \input{./.revinfo/gitCommitDate.tex}
\end{itemize}
}
}

%\PassOptionsToPackage{dvipsnames,svgnames}{xcolor}
\PassOptionsToPackage{square,numbers}{natbib}
\documentclass{scrreprt}

\usepackage[left=2cm,right=2cm]{geometry}
\usepackage[svgnames]{xcolor}
\usepackage{peeters_layout}

\usepackage{natbib}

\usepackage[
colorlinks=true,
bookmarks=false,
pdfauthor={\authorname, \email},
backref 
]{hyperref}

% http://tex.stackexchange.com/questions/75773/how-to-reference-problems-by-the-text-label-in-an-exercise-envioronment
\usepackage[english]{cleveref}
\crefname{Exercise}{exercise}{exercises}
\Crefname{Exercise}{Exercise}{Exercises}

\RequirePackage{titlesec}
\RequirePackage{ifthen}

% http://stackoverflow.com/questions/4932910/date-in-the-tabular-environment
\makeatletter
\let\insertdate\@date
\makeatother

\titleformat{\chapter}[display]
{\bfseries\Large}
{\color{DarkSlateGrey}\filleft \authorname
\ifthenelse{\isundefined{\studentnumber}}{}{\\ \studentnumber}
\ifthenelse{\isundefined{\email}}{}{\\ \email}
\ifthenelse{\isundefined{\dateintitle}}{}{\\ \insertdate}
%\ifthenelse{\isundefined{\coursename}}{}{\\ \coursename} % put in title instead.
}
{4ex}
{\color{DarkOliveGreen}{\titlerule}\color{Maroon}
\vspace{2ex}%
\filright}
[\vspace{2ex}%
\color{DarkOliveGreen}\titlerule
]

\newcommand{\beginArtWithToc}[0]{\begin{document}\tableofcontents}
\newcommand{\beginArtNoToc}[0]{\begin{document}}
\newcommand{\EndNoBibArticle}[0]{\end{document}}
\newcommand{\EndArticle}[0]{\bibliography{Bibliography}\bibliographystyle{plainnat}\end{document}}

% 
%\newcommand{\citep}[1]{\cite{#1}}

\colorSectionsForArticle


%
%\usepackage{peeters_layout_exercise}
%\usepackage{peeters_braket}
%\usepackage{peeters_figures}
%\usepackage{siunitx}
%%\usepackage{mhchem} % \ce{}
%%\usepackage{macros_bm} % \bcM
%\usepackage{macros_qed} % \qedmarker
%%\usepackage{txfonts} % \ointclockwise
%
%\beginArtNoToc
%
%\generatetitle{XXX}
%%\chapter{XXX}
%%\label{chap:gradeselection}
%
Having defined the axioms and definitions of Geometric Algebra, it desirable to define the grade selection operator, the dot product operator and the wedge product operator, and consider some simple examples of each.

\makedefinition{Grade selection operator}{dfn:gradeselection:gradeselection}{
Given a multivector \( M \) containing k-grade components \( M_k \)

\begin{equation*}
M = \sum_{i = 0}^N M_i,
\end{equation*}

the grade selection operator is defined as

\begin{equation*}\label{eqn:gradeselection:40}
\gpgrade{M}{k} \equiv M_k.
\end{equation*}

Selection of the (scalar) zero grade is often written as
\begin{equation*}
\gpgradezero{M} \equiv \gpgrade{M}{0} = M_0.
\end{equation*}
}

For example, if \( M = 3 - \Be_3 + 2 \Be_1 \Be_2 \), then
\begin{equation}\label{eqn:gradeselection:80}
\begin{aligned}
\gpgradezero{M} &= 3 \\
\gpgrade{M}{1} &= - \Be_3 \\
\gpgrade{M}{2} &= 2 \Be_1 \Be_2 \\
\gpgrade{M}{3} &= 0.
\end{aligned}
\end{equation}

\makedefinition{Dot product}{dfn:gradeselection:100}{
The dot (or inner) product of two multivectors

\begin{equation*}
\begin{aligned}
A &= \sum_{i = 0}^N A_i, \\
B &= \sum_{i = 0}^N B_i,
\end{aligned}
\end{equation*}

is defined as
\begin{equation*}
A \cdot B \equiv
\sum_{i,j = 0}^N \gpgrade{ A_i B_j }{\Abs{i - j}}
\end{equation*}
} % definition

As an example, consider two vectors in a 2D space

\begin{dmath}\label{eqn:gradeselection:140}
\begin{aligned}
\Ba  &= \lr{ x \Be_1 + y \Be_2 } \\
\Ba' &= \lr{ x' \Be_1 + y' \Be_2 },
\end{aligned}
\end{dmath}

for which this definition of the dot product gives

\begin{dmath}\label{eqn:gradeselection:160}
\Ba \cdot \Ba'
=
\gpgrade{ \Ba \Ba' }{\Abs{1 - 1}}
=
\gpgradezero{ \Ba \Ba' }
=
\gpgradezero{ \lr{ x \Be_1 + y \Be_2 } \lr{ x' \Be_1 + y' \Be_2 } }
=
\gpgradezero{ x x' \Be_1^2 + y y' \Be_2^2 + (x y' - y x') \Be_1 \Be_2 }
=
x x' + y y'.
\end{dmath}

It is left to the reader (\cref{problem:gradeselection:RnDotProduct}) to show that this definition also reduces to the traditional \R{n} dot product.

As a second example, consider the dot product of a vector with a bivector.  With \( \Ba \) as defined in \cref{eqn:gradeselection:140} and \( i = \Be_1 \Be_2 \)

\begin{dmath}\label{eqn:gradeselection:240}
\Ba \cdot i
=
\gpgrade{ \Ba i }{1}
=
\gpgrade{ \lr{ x \Be_1 + y \Be_2 } \Be_1 \Be_2 }{1}
=
\gpgrade{ x \Be_1^2 \Be_2 + y \Be_2 (-\Be_2 \Be_1) }{1}
=
\gpgrade{ x \Be_2 - y \Be_1 }{1}
=
x \Be_2 - y \Be_1.
\end{dmath}

This particular dot product is trivial, since the product \( \Ba i \) has only a vector component.
In this example \( i \) is the pseudoscalar for the two dimensional space, and it can be observed that multiplication of a vector from the right serves to rotate the vector by 90 degrees.  It is not a coincidence that this is strikingly similar to the action of the imaginary from complex algebra.  It can be shown (\cref{problem:gradeselection:PlaneRotations})
that \( e^{i\theta} \) acts as a rotation operator as it does in complex algebra, and that a GA representation of complex numbers is possible (\cref{problem:gradeselection:ComplexNumbers}).

For a non-trivial vector-bivector dot product, consider

\begin{dmath}\label{eqn:gradeselection:560}
\lr{ \Be_1 + \Be_2 } \cdot \lr{ \Be_1 \Be_2 + 3 \Be_2 \Be_3 }
=
\gpgradeone{
\lr{ \Be_1 + \Be_2 } \lr{ \Be_1 \Be_2 + 3 \Be_2 \Be_3 }
}
=
\gpgradeone{
\Be_1^2 \Be_2 + 3 \Be_1 \Be_2 \Be_3
+
\Be_2 \Be_1 \Be_2 + 3 \Be_2^2 \Be_3
}
=
\gpgradeone{
\Be_2 + 3 \cancel{\Be_1 \Be_2 \Be_3}
-
\Be_1 + 3 \Be_3
}
=
\Be_2 - \Be_1 + 3 \Be_3.
\end{dmath}

The vector-bivector dot product filters out products that no common factors, since such products result in trivector components.

\makedefinition{Wedge product.}{dfn:gradeselection:480}{
For the multivectors \( A, B \) defined in \cref{dfn:gradeselection:100}, the wedge (or outer) product is defined as

\begin{equation*}
A \wedge B
\equiv
\sum_{i,j = 0}^N \gpgrade{ A_i B_j }{i + j}.
\end{equation*}
} % definition

For example, the wedge product of the 2D vectors of \cref{eqn:gradeselection:140} is

\begin{dmath}\label{eqn:gradeselection:500}
\Ba \wedge \Bb
=
\gpgradetwo{
\lr{ x \Be_1 + y \Be_2 }
\lr{ x' \Be_1 + y' \Be_2 }
}
=
\gpgradetwo{
(x x' + y y') + (x y' - x' y) \Be_1 \Be_2
}
=
(x y' - x' y) \Be_1 \Be_2.
\end{dmath}

The wedge product of two vectors in a plane contains an antisymmetrized sum of the vector coefficients, but is weighted by a ``unit'' bivector, the pseudoscalar for the plane.

As another example consider

\begin{dmath}\label{eqn:gradeselection:520}
\Be_1 \wedge \lr{ 2\Be_1 + 3 \Be_2 }
=
\gpgradetwo{
\Be_1 \lr{ 2\Be_1 + 3 \Be_2 }
}
=
\gpgradetwo{
2 \Be_1^2 + 3 \Be_1 \Be_2
}
=
3 \Be_1 \Be_2.
\end{dmath}

Here we see that the component of the second vector \( \Be_1 + 3 \Be_2 \) that is colinear with the first vector \( \Be_1 \) is filtered out.  It is not coincidence that this is also a property of the cross product.  That relationship will be explored in (\cref{problem:gradeselection:WedgeRelationshipToCrossProduct}).

As a final example, consider the wedge product of a vector with a bivector

\begin{dmath}\label{eqn:gradeselection:540}
\Be_1 \wedge \lr{ \Be_1 \Be_2 - 7 \Be_2 \Be_3 }
=
\gpgradethree{
\Be_1 \lr{ \Be_1 \Be_2 - 7 \Be_2 \Be_3 }
}
=
\gpgradethree{
\Be_1^2 \Be_2 - 7 \Be_1 \Be_2 \Be_3
}
=
- 7 \Be_1 \Be_2 \Be_3.
\end{dmath}

Because \( \Be_1 \Be_2 \) has a common factor with \( \Be_1 \) it is filtered out of the resulting wedge product.  The end result, in this case, is a 3D pseudoscalar.

\subsection{Problems}

\makeproblem{\R{n} dot product.}{problem:gradeselection:RnDotProduct}{
Show that \ref{dfn:gradeselection:100} when applied to two vectors
is equivalent to the traditional \R{n} dot product.
} % problem

\makeproblem{}{problem:gradeselection:cyclicpermutationtwo}{
Show that
\begin{dmath}\label{eqn:gradeselection:580}
\gpgradezero{ \Bx \By }
=
\gpgradezero{ \By \Bx }.
\end{dmath}
} % problem

\makeproblem{Dot product of vectors as symmetric sum}{problem:gradeselection:dotprod}{
Show that the dot product of two vectors can be written as a symmetric sum

\begin{dmath}\label{eqn:gradeselection:600}
\Bx \cdot \By = \inv{2} \lr{ \Bx \By + \By \Bx }.
\end{dmath}
} % problem

\makeproblem{Plane rotations.}{problem:gradeselection:PlaneRotations}{

With \( i = \Be_1 \Be_2 \) for the pseudoscalar of the \( x,y \) plane,

\makesubproblem{}{problem:gradeselection:3:b}
justify the assertion that \( e^{i \theta} = \cos\theta + i \sin\theta \), where \( theta \) is a scalar angle.

\makesubproblem{}{problem:gradeselection:3:c}
Show that right multiplication of a 2D vector by \( e^{i\theta} \) rotates that vector by \( \theta \) radians.

\makesubproblem{}{problem:gradeselection:3:d}
Does the rotation multivector \( e^{i\theta} \) commute with the 2D basis vectors?

\makesubproblem{}{problem:gradeselection:3:e}
What is the action of multiplication of a vector by \( e^{i\theta} \) from the left?
} % problem

\makeproblem{Complex numbers}{problem:gradeselection:ComplexNumbers}{
Show that complex numbers can be represented as even grade multivectors \( z = a + \Be_1 \Be_2 b \).
} % problem

\makeproblem{\R{3} pseudoscalar commutation.}{problem:gradeselection:R3PseudoscalarCommutation}{
Show that \( I \) given by \cref{eqn:definitions:340}
commutes with any grade \R{3} multivector.
} % problem

\makeproblem{Vector wedge coordinate expansion and antisymmetry}{problem:gradeselection:vectorwedge}{
Show that
\begin{dmath}\label{eqn:gradeselection:620}
\Bx \wedge \By
=
\sum_{i < j} (x_i y_j - x_j y_i) \Be_i \Be_j.
\end{dmath}

Observe from this coordinate expansion that the wedge product of two vectors is antisymmetric
\boxedEquation{eqn:gradeselection:640}{
\Bx \wedge \By = -\By \wedge \Bx.
}
} % problem

\makeproblem{Wedge relationship to the cross product.}{problem:gradeselection:WedgeRelationshipToCrossProduct}{
For a pair of \R{3} vectors \( \Bx, \By \), show that the wedge and cross products are related by
\begin{dmath}\label{eqn:gradeselectionProblems:560}
\Bx \wedge \By = I (\Bx \cross \By),
\end{dmath}

where \( I = \Be_1 \Be_2 \Be_3 \) is the \R{3} pseudoscalar.
} % problem

\makeproblem{Vector bivector dot product}{problem:gradeselection:vectorBivectorDot}{
The dot product of a vector and bivector in \R{N} (or in fact any metric) expands as

\boxedEquation{eqn:gradeselection:660}{
\Ba \cdot \lr{ \Bb \wedge \Bc }
=
-\lr{ \Bb \wedge \Bc } \cdot \Ba
=
( \Ba \cdot \Bb ) \Bc
-( \Ba \cdot \Bc ) \Bb.
}

Demonstrate this by coordinate expansion using an orthonormal basis for \R{N}.

The right hand side may look familiar.  Demonstrate, for \R{3} without expansion in coordinates, that

\boxedEquation{eqn:gradeselection:680}{
\Ba \cdot \lr{ \Bb \wedge \Bc }
=
-\Ba \cross \lr{ \Bb \cross \Bc }.
}
} % problem

%}
%\EndNoBibArticle
