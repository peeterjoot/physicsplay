\makeproblem{Plane rotations.}{problem:gradeselection:PlaneRotations}{

With \( i = \Be_1 \Be_2 \) for the pseudoscalar of the \( x,y \) plane,

\makesubproblem{}{problem:gradeselection:3:b}
justify the assertion that \( e^{i \theta} = \cos\theta + i \sin\theta \), where \( theta \) is a scalar angle.

\makesubproblem{}{problem:gradeselection:3:c}
Show that right multiplication of a 2D vector by \( e^{i\theta} \) rotates that vector by \( \theta \) radians.

\makesubproblem{}{problem:gradeselection:3:d}
Does the rotation multivector \( e^{i\theta} \) commute with the 2D basis vectors?

\makesubproblem{}{problem:gradeselection:3:e}
What is the action of multiplication of a vector by \( e^{i\theta} \) from the left?
} % problem

\makeanswer{problem:gradeselection:PlaneRotations}{
\makeSubAnswer{}{problem:gradeselection:3:b}

Assume that the exponential of a multivector argument is represented by a Taylor series

\begin{dmath}\label{eqn:gradeselectionProblems:280}
e^X = \sum_{k = 0}^\infty \frac{X^k}{k!},
\end{dmath}

and note that the pseudoscalar commutes with scalar rotation angles \( \theta \), so
\begin{dmath}\label{eqn:gradeselectionProblems:300}
e^{i\theta}
= \sum_{k = 0}^\infty \frac{(i\theta)^k}{k!}
= \sum_{k = 0}^\infty \frac{i^k\theta^k}{k!}
=
\sum_{k = 0}^\infty \frac{i^{2k}\theta^{2k}}{(2k)!}
+
\sum_{k = 0}^\infty \frac{i^{2k + 1}\theta^{2k +1}}{(2k + 1)!}
=
\sum_{k = 0}^\infty \frac{(-1)^{k}\theta^{2k}}{(2k)!}
+
i \sum_{k = 0}^\infty \frac{(-1)^{k}\theta^{2k +1}}{(2k + 1)!}
= \cos \theta + i \sin\theta.
\end{dmath}
\makeSubAnswer{}{problem:gradeselection:3:c}

Consider the action of the exponential on each of the unit vectors.  For \( \Be_1 \) that is

\begin{dmath}\label{eqn:gradeselectionProblems:320}
\Be_1 e^{i \theta}
=
\Be_1 \lr{ \cos\theta + i \sin\theta }
=
\Be_1 \cos\theta + \Be_1 (\Be_1 \Be_2 )\sin\theta
=
\Be_1 \cos\theta + \Be_2 \sin\theta.
\end{dmath}

This shows that the vector \( \Be_1 \) is rotated counterclockwise by \( \theta \) radians.  Similarly for \( \Be_2 \)

\begin{dmath}\label{eqn:gradeselectionProblems:340}
\Be_2 e^{i \theta}
=
\Be_2 \lr{ \cos\theta + i \sin\theta }
=
\Be_2 \cos\theta + \Be_1 (\Be_1 \Be_2 )\sin\theta
=
\Be_2 \cos\theta + \Be_1 (-\Be_2 \Be_1) \sin\theta.
=
\Be_2 \cos\theta - \Be_1 \sin\theta.
\end{dmath}

This is also a rotation by \( \theta \) radians.  Given a vector \( \Bx = x \Be_1 + y \Be_2 \), this gives

\begin{dmath}\label{eqn:gradeselectionProblems:360}
\Bx'
= \Bx e^{i\theta}
=
x \lr{ \Be_1 \cos\theta + \Be_2 \sin\theta } + y \lr{ \Be_2 \cos\theta - \Be_1 \sin\theta }.
\end{dmath}

In particular

\begin{dmath}\label{eqn:gradeselectionProblems:380}
\begin{bmatrix}
\Bx' \cdot \Be_1 \\
\Bx' \cdot \Be_2 \\
\end{bmatrix}
=
\begin{bmatrix}
x \cos\theta - y \sin\theta \\
x \sin\theta + y \cos\theta
\end{bmatrix}
=
\begin{bmatrix}
\cos\theta &- \sin\theta \\
\sin\theta &+ \cos\theta
\end{bmatrix}
\begin{bmatrix}
x \\
y
\end{bmatrix}.
\end{dmath}

Observe that this is the rotation matrix that takes the points \((x, y)\) to their position \((x', y')\) rotated by \( \theta \) radians.
\makeSubAnswer{}{problem:gradeselection:3:d}

The action from the left on \( \Be_1 \) is

\begin{dmath}\label{eqn:gradeselectionProblems:400}
e^{i\theta} \Be_1
=
\lr{ \cos\theta + \Be_1 \Be_2 \sin\theta} \Be_1
=
\Be_1 \cos\theta + \Be_1 \Be_2 \Be_1 \sin\theta
=
\Be_1 \cos\theta + \Be_1 (-\Be_1 \Be_2) \sin\theta
=
\Be_1 \lr{ \cos\theta - i \sin\theta }
=
\Be_1 e^{-i\theta},
\end{dmath}

and the action from the left on \( \Be_2 \) is

\begin{dmath}\label{eqn:gradeselectionProblems:420}
e^{i\theta} \Be_2
=
\lr{ \cos\theta + \Be_1 \Be_2 \sin\theta} \Be_2
=
\Be_2 \cos\theta + \Be_1 \sin\theta
=
\Be_2 \cos\theta + (\Be_2 \Be_2) \Be_1 \sin\theta
=
\Be_2 \lr{ \cos\theta - i \sin\theta }
=
\Be_2 e^{-i\theta}.
\end{dmath}

This change of sign is due to the fact that the pseudoscalar anticommutes with each of the basis vectors in the plane.

\makeSubAnswer{}{problem:gradeselection:3:e}

Since the exponential toggles sign on commutation with both of the vectors of the plane, the rotation operation can be applied from either left or right, with sufficient care to get the direction right

\begin{equation}\label{eqn:gradeselectionProblems:440}
\Bx e^{i\theta} = e^{-i\theta} \Bx.
\end{equation}

It is also possible to split the rotation operation into half angle rotation operators that act from both the left and right

\begin{dmath}\label{eqn:gradeselectionProblems:460}
\Bx' = e^{-i\theta/2} \Bx e^{i\theta/2}.
\end{dmath}

A student who has studied computer graphics rotation theory may have seen quaternion rotation operators with this form, and a student of quantum mechanics will have seen Pauli matrix rotation operations of this form.  This is, in fact, the form that is generally desirable for 3D or higher order rotations, since it rotates the portions of a vector that lie in the rotation plane, leaving the normal components untouched.
} % answer
