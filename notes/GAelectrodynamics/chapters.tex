%
% Copyright � 2016 Peeter Joot.  All Rights Reserved.
% Licenced as described in the file LICENSE under the root directory of this GIT repository.
%
%----------------------------------------------------------------------------------------
\part{Geometric Algebra}
   \chapter{Basics}
      \section{Did you ever ask your teacher how to multiply vectors?}
         A new student of vector algebra will first learn 
%first see vectors in two or three dimensions as sets of coordinates
%
%\begin{equation}\label{eqn:GAmotivation:20}
%\Ba = 
%\begin{bmatrix}
%a_1 \\
%a_2 \\
%a_3 \\
%\end{bmatrix}, \qquad
%\Bb = 
%\begin{bmatrix}
%b_1 \\
%b_2 \\
%b_3 \\
%\end{bmatrix},
%\end{equation}
%
%or perhaps explicitly in terms of a basis \( \setlr{ \Be_1, \Be_2 \Be_3 } \)
%
%\begin{dmath}\label{eqn:GAmotivation:40}
%\begin{aligned}
%\Ba &= a_1 \Be_1 + a_2 \Be_2 + a_3 \Be_3  \\
%\Bb &= b_1 \Be_1 + b_2 \Be_2 + b_3 \Be_3
%\end{aligned}.
%\end{dmath}
%
%You will learn the 
rules for addition and subtraction of such vectors.  
%, and then how to operate on them with rotation matrices or other representations of linear transformations.  
This demonstrates to the student that the vector is an algebraic object that generalize numbers, and the question of how to 
multiply vectors soon follows.

Given the toolbox of traditional vector algebra, the best answer that a new student will obtain from such a line of questioning is to learn of the dot and cross products, multiplication like operations that we are all so familiar with

%\begin{itemize}
%\item ``You can not multiply vectors.'', or
%\item ``Vector multiplication is not well defined.'', or
%\item ``We will get to that.'', or
%\item ``There are multiplication like operations.''  The 
%\end{itemize}

\begin{dmath}\label{eqn:GAmotivation:60}
\begin{aligned}
\Ba \cdot \Bb &= a_1 b_1 + a_2 b_2 + a_3 b_3 = \Abs{\Ba} \Abs{\Bb} \cos \theta_{ab} \\
\Ba \cross \Bb &= 
\begin{vmatrix}
\Be_1 & \Be_2 & \Be_3 \\
a_1 & a_2 & a_3 \\
b_1 & b_2 & b_3 \\
\end{vmatrix}
= \ncap_{ab} \Abs{\Ba} \Abs{\Bb} \sin\theta_{ab}.
\end{aligned}
\end{dmath}

Both of these multiplication like operations live in very different spaces, one scalar, and the other a vector that lies outside of the span of its two vector factors.  Observe that the magnitudes of these two product operations are related to the product of the vectors in a Pythagorean sense

\begin{dmath}\label{eqn:GAmotivation:180}
\lr{ \Ba \cdot \Bb }^2 + \lr{ \Ba \cross \Bb }^2 
=
\Abs{\Ba}^2 \Abs{\Bb}^2 \cos^2 \theta_{ab} 
+\Abs{\Ba}^2 \Abs{\Bb}^2 \sin^2 \theta_{ab} 
= 
\Abs{\Ba}^2 \Abs{\Bb}^2.
\end{dmath}

This can be seen as a hint that the dot and cross products might be components of a single vector product operation, but the precise form of that product is not obvious.

Vector products that have the same form as the scalar magnitudes of the dot and cross products can be found in other algebraic systems.  Given a complex number representation of two vectors in a 2D space

\begin{dmath}\label{eqn:GAmotivation:200}
\begin{aligned}
z &= r e^{i \theta} \leftrightarrow (a, b) \\
w &= \rho e^{i \alpha} \leftrightarrow (a', b'),
\end{aligned}
\end{dmath}

the inner product of such a complex vector representation can be seen to have the same structure as the dot and cross products

\begin{equation}\label{eqn:GAmotivation:100}
\begin{aligned}
\Real( z w^\conj ) &= r \rho \cos(\theta - \alpha) \\
\Imag( z w^\conj ) &= r \rho \sin(\theta - \alpha).
\end{aligned}
\end{equation}

One can readily show that this inner product has the following vector isomorphism

\begin{dmath}\label{eqn:GAmotivation:220}
z w^\conj \leftrightarrow ( a a' + b b', a' b - a b' ).
\end{dmath}

One component is completely symmetric, whereas the other component of this product has a component that is completely antisymmetric.  
The 3D cross product also has this antisymmetry, and that antisymmetry will be seen later to be the key to the generalization of the cross product.  In this particular case, one can view this antisymmetric sum \( a' b - a b' \) as one 
answer of how the cross product ``generalizes'' from 3D to 2D without requiring the introduction of a normal dimension.

The answer to questions of exactly how the vector products, in particular the cross product, should generalize to higher dimensional spaces is still outstanding.  It should be expected that this cross product generalization will involve antisymmetry, just as the dot product generalization in higher dimensional spaces is completely symmetric.

Many current students of science may never see the exact structure of this generalization.  Should studies happen to include 
enough of the right esoteric physics and mathematics (quantum mechanics, QED, calculus on manifolds, ...) then 
some answers to those questions may be found.  Unfortunately, there are many such answers, and many of them each only provide 
one part of the picture.

For example, a student of non-relativistic quantum mechanics will learn of Pauli matrices.  The dot and cross products will be seen to be components of a more general vector multiplication operation

\begin{equation}\label{eqn:GAmotivation:120}
\lr{\Bsigma \cdot \Bx }
\lr{\Bsigma \cdot \By }
=
I \lr{ \Bx \cdot \By } + i \Bsigma \cdot \lr{ \Bx \cross \By }.
\end{equation}

A student of quantum field theory will encounter Dirac matrices, a algebraic structure that allows for the multiplication of four-vectors

\begin{dmath}\label{eqn:GAmotivation:140}
\aslash \bslash
=
\inv{2} \symmetric{ \aslash}{ \bslash }
+
\inv{2} \antisymmetric{ \aslash}{ \bslash }
=
a^\mu b_\mu + \inv{2} a^\mu b^\nu \antisymmetric{\gamma_\mu}{\gamma_\nu}
=
a^\mu b_\mu + \inv{2} a^\mu b^\nu \lr{ 
\gamma_\mu \gamma_\nu
- 
\gamma_\nu \gamma_\mu
}.
\end{dmath}

A product of ``Dirac'' vectors has symmetric and antisymmetric components that generalize the dot and cross products.  
Unfortunately, this algebra comes with still another different notation.
One interesting take away from this particular vector product is the fact that one component is a scalar, and other other 
involves products of vectors, something that will require further interpretation.  Since the Dirac basis typically has a matrix representation, such a product can be dismissed as just being another matrix.  The products of mutually orthonormal vectors will show up again later in a context where there is no requirement to assume a matrix representation of the underlying basis.

There are still other algebraic systems, such as quaternion algebra, where the dot and cross products will be found in.  

Another common and important context that contains generalizations of the dot and cross products is the subject of differential forms.
A student of differential forms will learn how to compute the wedge products of forms, and of duality operations, which can be used to construct generalized multiplication operations that have the structure of the 3D dot and cross products

\begin{equation}\label{eqn:GAmotivation:160}
\begin{aligned}
df \wedge * dg &= \lr{ \sum_{i=1}^3 \PD{x_i}{f} \PD{x_i}{g} } dx_1 \wedge dx_2 \wedge dx_3 \\
df \wedge dg &= \sum_{1 \le i < j \le 3} \lr{
\PD{x_i}{f} \PD{x_j}{g} 
-\PD{x_j}{f} \PD{x_i}{g} 
}
dx_i \wedge dx_j.
\end{aligned}
\end{equation}

It is possible to express vectors as a differential form, and some advocate for this \citep{flanders1989dfa}, but this can also seem unnatural.  Regardless, differential forms do highlight the existance of more general concepts of vector multiplication.  %In this particular case, this generality comes with the cost of using yet another notation, one that is considerably different than the vector notation that we are comfortable with.  

It should not be surprising that all of these ideas are special cases of a more general algebraic system.

The aim of the material to follow is to provide the instruction manual for an enhanced toolbox of vector algebra techniques that can be used to gain an integrated view of many seemingly disparate mathematical methods.  These are tools that can be learned without having to first study the esoteric arts of quantum mechanics or differential forms, and have many applications once learned.  These notes will focus on applications to the study of electromagnetism.


         \subsection{Problems}
            
\makeproblem{Complex inner product vs. dot and cross product.}{problem:introGAproblems:ComplexInnerProductVsDotAndCrossProduct}{
Given two 2D vectors \( (a,b) \) and \( (a', b') \), and a complex number representation of these vectors \( z = a + ib, w = a' + i b' \), show that the components of the complex inner product have the representation
given by \cref{eqn:GAmotivation:220}.
} % problem

\makeanswer{problem:introGAproblems:ComplexInnerProductVsDotAndCrossProduct}{
\begin{dmath}\label{eqn:introGAproblems:20}
z w^\conj
=
(a + ib)(a' - ib')
=
a a' + b b'
+ i \lr{ a' b - a b' }
\leftrightarrow
( a a' + b b', a' b - a b' ).
\end{dmath}
} % answer

      \section{Vector multiplication}
         %
% Copyright � 2016 Peeter Joot.  All Rights Reserved.
% Licenced as described in the file LICENSE under the root directory of this GIT repository.
%
%{
%%\newcommand{\authorname}{Peeter Joot}
\newcommand{\email}{peeterjoot@protonmail.com}
\newcommand{\basename}{FIXMEbasenameUndefined}
\newcommand{\dirname}{notes/FIXMEdirnameUndefined/}

%%\renewcommand{\basename}{multiplication}
%%%\renewcommand{\dirname}{notes/phy1520/}
%%\renewcommand{\dirname}{notes/ece1228-electromagnetic-theory/}
%%%\newcommand{\dateintitle}{}
%%%\newcommand{\keywords}{}
%%
%%\newcommand{\authorname}{Peeter Joot}
\newcommand{\onlineurl}{http://sites.google.com/site/peeterjoot2/math2013/\basename.pdf}
\newcommand{\sourcepath}{\dirname\basename.tex}
\newcommand{\generatetitle}[1]{\chapter{#1}}

\newcommand{\vcsinfo}{%
\section*{}
\noindent{\color{DarkOliveGreen}{\rule{\linewidth}{0.1mm}}}
\paragraph{Document version}
%\paragraph{\color{Maroon}{Document version}}
{
\small
\begin{itemize}
\item Available online at:\\ 
\href{\onlineurl}{\onlineurl}
\item Git Repository: \input{./.revinfo/gitRepo.tex}
\item Source: \sourcepath
\item last commit: \input{./.revinfo/gitCommitString.tex}
\item commit date: \input{./.revinfo/gitCommitDate.tex}
\end{itemize}
}
}

%\PassOptionsToPackage{dvipsnames,svgnames}{xcolor}
\PassOptionsToPackage{square,numbers}{natbib}
\documentclass{scrreprt}

\usepackage[left=2cm,right=2cm]{geometry}
\usepackage[svgnames]{xcolor}
\usepackage{peeters_layout}

\usepackage{natbib}

\usepackage[
colorlinks=true,
bookmarks=false,
pdfauthor={\authorname, \email},
backref 
]{hyperref}

% http://tex.stackexchange.com/questions/75773/how-to-reference-problems-by-the-text-label-in-an-exercise-envioronment
\usepackage[english]{cleveref}
\crefname{Exercise}{exercise}{exercises}
\Crefname{Exercise}{Exercise}{Exercises}

\RequirePackage{titlesec}
\RequirePackage{ifthen}

% http://stackoverflow.com/questions/4932910/date-in-the-tabular-environment
\makeatletter
\let\insertdate\@date
\makeatother

\titleformat{\chapter}[display]
{\bfseries\Large}
{\color{DarkSlateGrey}\filleft \authorname
\ifthenelse{\isundefined{\studentnumber}}{}{\\ \studentnumber}
\ifthenelse{\isundefined{\email}}{}{\\ \email}
\ifthenelse{\isundefined{\dateintitle}}{}{\\ \insertdate}
%\ifthenelse{\isundefined{\coursename}}{}{\\ \coursename} % put in title instead.
}
{4ex}
{\color{DarkOliveGreen}{\titlerule}\color{Maroon}
\vspace{2ex}%
\filright}
[\vspace{2ex}%
\color{DarkOliveGreen}\titlerule
]

\newcommand{\beginArtWithToc}[0]{\begin{document}\tableofcontents}
\newcommand{\beginArtNoToc}[0]{\begin{document}}
\newcommand{\EndNoBibArticle}[0]{\end{document}}
\newcommand{\EndArticle}[0]{\bibliography{Bibliography}\bibliographystyle{plainnat}\end{document}}

% 
%\newcommand{\citep}[1]{\cite{#1}}

\colorSectionsForArticle


%%
%%\usepackage{peeters_layout_exercise}
%%\usepackage{peeters_braket}
%%\usepackage{peeters_figures}
%%\usepackage{siunitx}
%%%\usepackage{mhchem} % \ce{}
%%%\usepackage{macros_bm} % \bcM
%%%\usepackage{txfonts} % \ointclockwise
%%
%%\beginArtNoToc
%%
%%\generatetitle{Vector multiplication}
%%%\chapter{Vector multiplication}
%%%\label{chap:multiplication}
%%
Geometric Algebra defines a multiplication operation for vectors, forming a vector space spanned by all the possible vector products.  This algebra is described by the following small set of axioms

\makeaxiom{Associative multiplication.}{axiom:multiplication:associative}{

The product of any three vectors \(\Ba,\Bb,\Bc\) is associative.

\begin{equation*}\label{eqn:multiplication:160}
\Ba (\Bb \Bc) 
= (\Ba \Bb) \Bc
= \Ba \Bb \Bc.
\end{equation*}
}

\makeaxiom{Linearity.}{axiom:multiplication:linear}{
Vector products are linear with respect to addition and subtraction.

\begin{dmath*}\label{eqn:multiplication:180}
\begin{aligned}
(\Ba + 3 \Bb \Bd) \Bc &= \Ba \Bb + 3 \Bb \Bd \Bc \\
\Ba (\Bb \Bd - 2 \Bc) &= \Ba \Bb \Bd - 2 \Ba \Bc.
\end{aligned}
\end{dmath*}
}

\makeaxiom{Contraction.}{axiom:multiplication:contraction}{

The square of a vector is the squared length of the vector.

\begin{dmath*}\label{eqn:multiplication:200}
\Ba^2 = \Abs{\Ba}^2.
\end{dmath*}
}

These axioms are simple enough, but have a rich set of consequences\footnote{Similar to Feynman on gravitation \citep{feynman1963flp} ``... have shall said everything required, for a sufficiently talented mathematician could then deduce all the consequences of these principles.  However, since you are not assumed to be sufficiently talented yet, we shall discuss the consequences in more detail''.}.
Axiom \ref{axiom:multiplication:contraction}, the contraction property, is generally metric dependent.  In particular, for special relativistic calculations, the length of a (four-)vector may generally be negative or positive.
However, for the engineering electromagnetic problems that will be the focus of these notes, it can be assumed that there is an orthonormal Euclidean basis, where the vector length is always positive.

The linearity and associativity axioms need little comment, but the contraction property might be suprising.  For one justification of this rule, consider a one dimensional vector space spanned by a single unit vector \( \setlr{ \Be } \).  That span, for real \( x \) is all the values

\begin{dmath}\label{eqn:multiplication:20}
\Bx = x \Be.
\end{dmath}

FIXME: picture to demonstrate the number line isomorphism.

This vector space is isomorphic with a number line, all the possible real values \( x \).  
Given a positive number \( x \), the multiplication rules for real numbers require that \( (\pm x)^2 = x^2 \).  
The square of a number provides the (squared) length of the number, its distance from the origin.  
The same rule can be imposed for one dimensional vectors, 
a requirement that the (squared) distance from the origin equals the square of the vector itself.   Such a rule is consistent with the rules of scalar multiplication, and for the 
one dimensional vectors of \cref{eqn:multiplication:20} can be stated as

\begin{equation}\label{eqn:multiplication:40}
\Bx^2 = x^2.
\end{equation}

This contraction axiom, justified or not, has additional implications

\begin{dmath}\label{eqn:multiplication:80}
x^2 
= \Bx^2 
= (x \Be)(x \Be)
= x^2 \Be^2.
\end{dmath}

This rule requires the square of a unit (Euclidean) vector to be unity

%\begin{equation}\label{eqn:multiplication:60}
\boxedEquation{eqn:multiplication:60}{
\Be^2 = 1.
}
%\end{equation}

With this implication noted, now consider the square of a simple two dimensional vector

\begin{dmath}\label{eqn:gaTutorial:80}
2 
=
(\Be_1 + \Be_2)^2 
= (\Be_1 + \Be_2)(\Be_1 + \Be_2)
= \Be_1^2 + \Be_2 \Be_1 + \Be_1 \Be_2 + \Be_2^2
= 2 + \Be_2 \Be_1 + \Be_1 \Be_2.
\end{dmath}

The sum above with both scalar terms and terms that are composed of products of vectors is called a multivector.
A product of two perpendicular vectors (or a sum of such products) is called a bivector, and can be used to represent an oriented plane.
Geometric Algebra allows sums of scalars, vectors, bivectors, and higher degree analogues (grades) be summed.

Observe that for this 
identity to hold, the bivector terms must sum to zero.  That is

%\begin{dmath}\label{eqn:multiplication:140}
\boxedEquation{eqn:multiplication:140}{
\Be_1 \Be_2 = -\Be_1 \Be_2.
}
%\end{dmath}

This implies that the product of two othonormal vectors anticommutes.  

By considering the implications of the contraction axiom, most of the work of proving an important theorem has now been performed

\begin{theorem}
\begin{subequations}\label{thm:multiplication:basics}
\begin{align}
{\Bu}^2 = 1 & \qquad \mbox{Square of a unit vector \(\Bu\) is one.} \\
\Bu \Bv = -\Bv \Bu & \qquad \mbox{The product of two orthonormal unit vectors \(\Bu\), and \(\Bv\) anticommute.}
\end{align}
\end{subequations}
\end{theorem}

Proof of this theorem in higher dimensional spaces is left as an exersize for the reader.

\makeproblem{}{problem:multiplication:2dvectorsquare}{
Generalize the calculation of \cref{eqn:gaTutorial:80} to calculate the square of an \R{2} vector

\begin{dmath}\label{eqn:multiplication:100}
\Bx = x \Be_1 + y \Be_2.
\end{dmath}
} % problem

%%%}
%%%\EndArticle
%%\EndNoBibArticle

         \subsection{Problems}
            \makeproblem{}{problem:multiplication:2dvectorsquare}{
Generalize the calculation of \cref{eqn:gaTutorial:80} to calculate the square of an \R{n} vector.

\begin{dmath}\label{eqn:multiplication:100}
\Bx = \sum_i x_i \Be_i
\end{dmath}
} % problem

\makeanswer{problem:multiplication:2dvectorsquare}{
Consider the 2D case to start with

\begin{dmath}\label{eqn:multiplication:120}
\Bx^2
=
\lr{ x \Be_1 + y \Be_2}
\lr{ x \Be_1 + y \Be_2}
=
\lr{ x \Be_1 } \lr{ x \Be_1 }
+
\lr{ y \Be_2 } \lr{ y \Be_2 }
+
\lr{ x \Be_1 } \lr{ y \Be_2 }
+
\lr{ y \Be_2 } \lr{ x \Be_1 }
=
x^2 \Be_1^2
+
y^2 \Be_2^2
+
x y \lr{ \Be_1 \Be_2 + \Be_2 \Be_1 }
=
x^2 + y^2
+
x y \lr{ \Be_1 \Be_2 + \Be_2 \Be_1 }.
\end{dmath}

The contraction axiom requires the bivector terms to sum to zero, as also demonstrated previously for the specific example \( \Bx = \Be_1 + \Be_2 \).

More generally for \R{N}

\begin{dmath}\label{eqn:multiplication:121}
\Bx^2
=
\lr{ \sum_i x_i \Be_i }
\lr{ \sum_j x_j \Be_j }
=
\sum_{ij} x_i x_j \Be_i \Be_j
=
\sum_{i = j} x_i x_j \Be_i \Be_j
+
\sum_{i \ne j} x_i x_j \Be_i \Be_j
=
\sum_{i} x_i^2
+
\sum_{i \ne j} x_i x_j \Be_i \Be_j
=
\sum_{i} x_i^2
+
\sum_{i < j} x_i x_j (\Be_i \Be_j + \Be_j \Be_i).
\end{dmath}

The contraction axiom requires all the bivector pairs to sum to zero.  That is, for each \( i \ne j \)

\begin{dmath}\label{eqn:introGAproblems:140}
\Be_i \Be_j = -\Be_j \Be_i.
\end{dmath}
} % answer

            
\makeproblem{Normal anticommutation}{problem:multiplication:unitsquare}{
Prove \cref{thm:multiplication:anticommutationNormal}.
} % problem



      \section{Definitions}
         %
% Copyright � 2016 Peeter Joot.  All Rights Reserved.
% Licenced as described in the file LICENSE under the root directory of this GIT repository.
%
%{
%%\newcommand{\authorname}{Peeter Joot}
\newcommand{\email}{peeterjoot@protonmail.com}
\newcommand{\basename}{FIXMEbasenameUndefined}
\newcommand{\dirname}{notes/FIXMEdirnameUndefined/}

%%\renewcommand{\basename}{multiplication}
%%%\renewcommand{\dirname}{notes/phy1520/}
%%\renewcommand{\dirname}{notes/ece1228-electromagnetic-theory/}
%%%\newcommand{\dateintitle}{}
%%%\newcommand{\keywords}{}
%%
%%\newcommand{\authorname}{Peeter Joot}
\newcommand{\onlineurl}{http://sites.google.com/site/peeterjoot2/math2013/\basename.pdf}
\newcommand{\sourcepath}{\dirname\basename.tex}
\newcommand{\generatetitle}[1]{\chapter{#1}}

\newcommand{\vcsinfo}{%
\section*{}
\noindent{\color{DarkOliveGreen}{\rule{\linewidth}{0.1mm}}}
\paragraph{Document version}
%\paragraph{\color{Maroon}{Document version}}
{
\small
\begin{itemize}
\item Available online at:\\ 
\href{\onlineurl}{\onlineurl}
\item Git Repository: \input{./.revinfo/gitRepo.tex}
\item Source: \sourcepath
\item last commit: \input{./.revinfo/gitCommitString.tex}
\item commit date: \input{./.revinfo/gitCommitDate.tex}
\end{itemize}
}
}

%\PassOptionsToPackage{dvipsnames,svgnames}{xcolor}
\PassOptionsToPackage{square,numbers}{natbib}
\documentclass{scrreprt}

\usepackage[left=2cm,right=2cm]{geometry}
\usepackage[svgnames]{xcolor}
\usepackage{peeters_layout}

\usepackage{natbib}

\usepackage[
colorlinks=true,
bookmarks=false,
pdfauthor={\authorname, \email},
backref 
]{hyperref}

% http://tex.stackexchange.com/questions/75773/how-to-reference-problems-by-the-text-label-in-an-exercise-envioronment
\usepackage[english]{cleveref}
\crefname{Exercise}{exercise}{exercises}
\Crefname{Exercise}{Exercise}{Exercises}

\RequirePackage{titlesec}
\RequirePackage{ifthen}

% http://stackoverflow.com/questions/4932910/date-in-the-tabular-environment
\makeatletter
\let\insertdate\@date
\makeatother

\titleformat{\chapter}[display]
{\bfseries\Large}
{\color{DarkSlateGrey}\filleft \authorname
\ifthenelse{\isundefined{\studentnumber}}{}{\\ \studentnumber}
\ifthenelse{\isundefined{\email}}{}{\\ \email}
\ifthenelse{\isundefined{\dateintitle}}{}{\\ \insertdate}
%\ifthenelse{\isundefined{\coursename}}{}{\\ \coursename} % put in title instead.
}
{4ex}
{\color{DarkOliveGreen}{\titlerule}\color{Maroon}
\vspace{2ex}%
\filright}
[\vspace{2ex}%
\color{DarkOliveGreen}\titlerule
]

\newcommand{\beginArtWithToc}[0]{\begin{document}\tableofcontents}
\newcommand{\beginArtNoToc}[0]{\begin{document}}
\newcommand{\EndNoBibArticle}[0]{\end{document}}
\newcommand{\EndArticle}[0]{\bibliography{Bibliography}\bibliographystyle{plainnat}\end{document}}

% 
%\newcommand{\citep}[1]{\cite{#1}}

\colorSectionsForArticle


%%
%%\usepackage{peeters_layout_exercise}
%%\usepackage{peeters_braket}
%%\usepackage{peeters_figures}
%%\usepackage{siunitx}
%%%\usepackage{mhchem} % \ce{}
%%%\usepackage{macros_bm} % \bcM
%%%\usepackage{txfonts} % \ointclockwise
%%
%%\beginArtNoToc
%%
%%\generatetitle{Vector multiplication}
%%%\chapter{Vector multiplication}
%%%\label{chap:multiplication}
%%

A few new GA terms have been introduced in an ad-hoc fashion as required.  Here is a systematic exposition of some of the key definitions used to refer to the types of the geometric objects that will be encountered.

\makedefinition{Scalar}{def:multiplication:scalar}{
A real number with no implied direction.
}

\makedefinition{Vector}{def:multiplication:vector}{
\href{https://www.youtube.com/watch?v=bOIe0DIMbI8}{A quantity with direction and magnitude.}
}

\makedefinition{Bivector}{def:multiplication:bivector}{
A product of two normal vectors, or a sum thereof.
}

The product \( \Be_1 \Be_2 \) is a bivector, as is \( \Be_2 \Be_3 + 3 \Be_4 \Be_1 \)

\makedefinition{Trivector}{def:multiplication:trivector}{
A product of three mutually normal vectors, or a sum thereof.
}

The quantity \( \Be_3 \Be_1 \Be_2 \) is a trivector, as is \( \Be_1 \Be_2 \Be_3 + 3 \Be_5 \Be_4 \Be_1 \).

\makedefinition{Blade}{def:multiplication:blade}{
A scalar, vector, bivector, or a trivector (or higher degree analogue), that can be constructed by multiplication of a number of vectors, but not an unfactorable sum of thereof.
}

The factorable quantity
\begin{dmath}\label{eqn:multiplication:220}
\Be_1 \Be_2 + 3 \Be_1 \Be_3
=
\Be_1 (\Be_2 + 3 \Be_3)
\end{dmath}

is a blade, whereas

\begin{dmath}\label{eqn:multiplication:240}
\Be_1 \Be_2 + 3 \Be_3 \Be_4,
\end{dmath}

and
\begin{dmath}\label{eqn:multiplication:260}
\Be_1 \Be_2 + \Be_2 \Be_3 + \Be_3 \Be_1
\end{dmath}

are not.

Scalar, vector, bivector, and trivectors are also referred to as sums of 0-blades, 1-blades, 2-blades, and 3-blades respectively.

\makedefinition{Grade.}{def:multiplication:grade}{
The minimum number of vector products required to form a given blade.
}

The grade of a scalar, vector, bivector, and trivector are 0, 1, 2, and 3 respectively.

The quantities
\begin{dmath}\label{eqn:multiplication:300}
\begin{aligned}
\Be_2 + \Be_1 \Be_2 \Be_2 &= \Be_1 + \Be_2 \\
\Be_1 \Be_2 \Be_2 \Be_2 \Be_3 &= \Be_1 \Be_2 \Be_3 \\
\end{aligned}
\end{dmath}

have grades 1 and 3 respectively.

Quantities with higher grades than 3 are not generally given explicit names, but can be referred to having grade-k.  When an object of grade-k is
also a blade, it can be referred to as a k-blade.

In a three dimensional space the highest grade possible is 3.  Blades can have grades higher than 3 in higher dimensional vector spaces.

\makedefinition{Pseudoscalar.}{def:multiplication:pseudoscalar}{
A blade with grade that matches the dimension of the space.
}

In a two dimensional space \( \Be_2 \Be_1 \) is a pseudoscalar.  In a three dimensional space
\( \Be_3 \Be_1 \Be_2 \) is a pseudoscalar, as is \( \Be_3 \Be_1 (\Be_2 + \Be_3 ) \).  A pseudoscalar has an implied orientation, which can be
associated with the handedness of the underlying basis.  It is conventional to refer to

\begin{dmath}\label{eqn:definitions:320}
i = \Be_1 \Be_2,
\end{dmath}

as ``the pseudoscalar'' for a two dimensional space, and to

\begin{dmath}\label{eqn:definitions:340}
I = \Be_1 \Be_2 \Be_3,
\end{dmath}

as ``the pseudoscalar'' for a three dimensional space.

\makedefinition{Multivector.}{def:multiplication:multivector}{
A sum of zero or more blades.
}

Examples include
\begin{dmath}\label{eqn:multiplication:280}
\begin{aligned}
&3 \\
& 1 + \Be_1 \Be_2 \\
& 2 - \Be_1 \Be_2 \Be_3 \\
& \Be_1 + 2 \Be_1 \Be_2 + \Be_2 \Be_3 - 3 \Be_3 \Be_1 + \Be_1 \Be_2 \Be_4
\end{aligned}
\end{dmath}

\makedefinition{Dual}{dfn:definitions:dual}{
FIXME: todo.
} % definition

\subsection{Problems}

\makeproblem{\R{3} pseudoscalar square}{problem:gradeselection:R3PseudoscalarSquare}{
With the \R{3} pseudoscalar of \cref{eqn:definitions:340} show that \( I^2 = -1 \).
} % problem

%%%}
%%%\EndArticle
%%\EndNoBibArticle

         \subsection{Problems}
            
\makeproblem{\R{3} pseudoscalar square}{problem:gradeselection:R3PseudoscalarSquare}{
With the \R{3} pseudoscalar of \cref{eqn:definitions:340} show that \( I^2 = -1 \).
} % problem

      \section{Grade selection, dot and wedge product operators}
         %
% Copyright � 2016 Peeter Joot.  All Rights Reserved.
% Licenced as described in the file LICENSE under the root directory of this GIT repository.
%
%{
\newcommand{\authorname}{Peeter Joot}
\newcommand{\email}{peeterjoot@protonmail.com}
\newcommand{\basename}{FIXMEbasenameUndefined}
\newcommand{\dirname}{notes/FIXMEdirnameUndefined/}

\renewcommand{\basename}{gradeselection}
%\renewcommand{\dirname}{notes/phy1520/}
\renewcommand{\dirname}{notes/ece1228-electromagnetic-theory/}
%\newcommand{\dateintitle}{}
%\newcommand{\keywords}{}

\newcommand{\authorname}{Peeter Joot}
\newcommand{\onlineurl}{http://sites.google.com/site/peeterjoot2/math2013/\basename.pdf}
\newcommand{\sourcepath}{\dirname\basename.tex}
\newcommand{\generatetitle}[1]{\chapter{#1}}

\newcommand{\vcsinfo}{%
\section*{}
\noindent{\color{DarkOliveGreen}{\rule{\linewidth}{0.1mm}}}
\paragraph{Document version}
%\paragraph{\color{Maroon}{Document version}}
{
\small
\begin{itemize}
\item Available online at:\\ 
\href{\onlineurl}{\onlineurl}
\item Git Repository: \input{./.revinfo/gitRepo.tex}
\item Source: \sourcepath
\item last commit: \input{./.revinfo/gitCommitString.tex}
\item commit date: \input{./.revinfo/gitCommitDate.tex}
\end{itemize}
}
}

%\PassOptionsToPackage{dvipsnames,svgnames}{xcolor}
\PassOptionsToPackage{square,numbers}{natbib}
\documentclass{scrreprt}

\usepackage[left=2cm,right=2cm]{geometry}
\usepackage[svgnames]{xcolor}
\usepackage{peeters_layout}

\usepackage{natbib}

\usepackage[
colorlinks=true,
bookmarks=false,
pdfauthor={\authorname, \email},
backref 
]{hyperref}

% http://tex.stackexchange.com/questions/75773/how-to-reference-problems-by-the-text-label-in-an-exercise-envioronment
\usepackage[english]{cleveref}
\crefname{Exercise}{exercise}{exercises}
\Crefname{Exercise}{Exercise}{Exercises}

\RequirePackage{titlesec}
\RequirePackage{ifthen}

% http://stackoverflow.com/questions/4932910/date-in-the-tabular-environment
\makeatletter
\let\insertdate\@date
\makeatother

\titleformat{\chapter}[display]
{\bfseries\Large}
{\color{DarkSlateGrey}\filleft \authorname
\ifthenelse{\isundefined{\studentnumber}}{}{\\ \studentnumber}
\ifthenelse{\isundefined{\email}}{}{\\ \email}
\ifthenelse{\isundefined{\dateintitle}}{}{\\ \insertdate}
%\ifthenelse{\isundefined{\coursename}}{}{\\ \coursename} % put in title instead.
}
{4ex}
{\color{DarkOliveGreen}{\titlerule}\color{Maroon}
\vspace{2ex}%
\filright}
[\vspace{2ex}%
\color{DarkOliveGreen}\titlerule
]

\newcommand{\beginArtWithToc}[0]{\begin{document}\tableofcontents}
\newcommand{\beginArtNoToc}[0]{\begin{document}}
\newcommand{\EndNoBibArticle}[0]{\end{document}}
\newcommand{\EndArticle}[0]{\bibliography{Bibliography}\bibliographystyle{plainnat}\end{document}}

% 
%\newcommand{\citep}[1]{\cite{#1}}

\colorSectionsForArticle



\usepackage{peeters_layout_exercise}
\usepackage{peeters_braket}
\usepackage{peeters_figures}
\usepackage{siunitx}
%\usepackage{mhchem} % \ce{}
%\usepackage{macros_bm} % \bcM
\usepackage{macros_qed} % \qedmarker
%\usepackage{txfonts} % \ointclockwise

\beginArtNoToc

\generatetitle{XXX}
%\chapter{XXX}
%\label{chap:gradeselection}
% \citep{sakurai2014modern} pr X.Y
% \citep{pozar2009microwave}
% \citep{qftLectureNotes}
% \citep{doran2003gap}
% \citep{jackson1975cew}
% \citep{griffiths1999introduction}

Having defined the axioms and definitions of Geometric Algebra, it desirable to define the grade selection operator, the dot product operator and the wedge product operator, and consider some simple examples of each.

\makedefinition{Grade selection operator}{dfn:gradeselection:gradeselection}{
Given a multivector \( M \) containing k-grade components \( M_k \)

\begin{equation*}
M = \sum_{i = 0}^N M_i,
\end{equation*}

the grade selection operator is defined as

\begin{equation*}\label{eqn:gradeselection:40}
\gpgrade{M}{k} \equiv M_k.
\end{equation*}

Selection of the (scalar) zero grade is often written as
\begin{equation*}
\gpgradezero{M} \equiv \gpgrade{M}{0} = M_0.
\end{equation*}
}

For example, if \( M = 3 - \Be_3 + 2 \Be_1 \Be_2 \), then
\begin{equation}\label{eqn:gradeselection:80}
\begin{aligned}
\gpgradezero{M} &= 3 \\
\gpgrade{M}{1} &= - \Be_3 \\
\gpgrade{M}{2} &= 2 \Be_1 \Be_2 \\
\gpgrade{M}{3} &= 0.
\end{aligned}
\end{equation}

\makedefinition{Dot product}{dfn:gradeselection:100}{
The dot product of two multivectors

\begin{equation*}
\begin{aligned}
A &= \sum_{i = 0}^N A_i, \\
B &= \sum_{i = 0}^N B_i,
\end{aligned}
\end{equation*}

is defined as
\begin{equation*}
A \cdot B \equiv
\sum_{i,j = 0}^N \gpgrade{ A_i B_j }{\Abs{i - j}}
\end{equation*}
} % definition

As an example, consider two vectors in a 2D space

\begin{dmath}\label{eqn:gradeselection:140}
\begin{aligned}
\Ba  &= \lr{ x \Be_1 + y \Be_2 } \\
\Ba' &= \lr{ x' \Be_1 + y' \Be_2 },
\end{aligned}
\end{dmath}

for which this definition of the dot product gives

\begin{dmath}\label{eqn:gradeselection:160}
\Ba \cdot \Ba'
=
\gpgrade{ \Ba \Ba' }{\Abs{1 - 1}}
=
\gpgradezero{ \Ba \Ba' }
=
\gpgradezero{ \lr{ x \Be_1 + y \Be_2 } \lr{ x' \Be_1 + y' \Be_2 } }
=
\gpgradezero{ x x' \Be_1^2 + y y' \Be_2^2 + (x y' - y x') \Be_1 \Be_2 }
=
x x' + y y'.
\end{dmath}

It is left to the reader (\cref{problem:gradeselection:1}) to show that this definition also reduces to the traditional \R{n} dot product.

As a second example, consider the dot product of a vector with a bivector.  With \( \Ba \) as defined in \cref{eqn:gradeselection:140} and \( i = \Be_1 \Be_2 \)

\begin{dmath}\label{eqn:gradeselection:240}
\Ba \cdot i
=
\gpgrade{ \Ba i }{1}
=
\gpgrade{ \lr{ x \Be_1 + y \Be_2 } \Be_1 \Be_2 }{1}
=
\gpgrade{ x \Be_1^2 \Be_2 + y \Be_2 (-\Be_2 \Be_1) }{1}
=
\gpgrade{ x \Be_2 - y \Be_1 }{1}
=
x \Be_2 - y \Be_1.
\end{dmath}

This particular dot product is trivial, since the product \( \Ba i \) has only a vector component.
In this example \( i \) is the pseudoscalar for the two dimensional space, and it can be observed that multiplication of a vector from the right serves to rotate the vector by 90 degrees.  It is not a coincidence that this is strikingly similar to the action of the imaginary from complex algebra.  It can be shown (\cref{problem:gradeselection:3})
that \( e^{i\theta} \) acts as a rotation operator as it does in complex algebra, and that a GA representation of complex numbers is possible (\cref{problem:gradeselection:2}).

\makedefinition{Outer product.}{dfn:gradeselection:480}{
For the multivectors \( A, B \) defined in \cref{dfn:gradeselection:100}, the outer product (or wedge product) is defined as

is defined as
\begin{equation*}
A \wedge B
\equiv
\sum_{i,j = 0}^N \gpgrade{ A_i B_j }{i + j}.
\end{equation*}
} % definition

For example, the wedge product of the 2D vectors of \cref{eqn:gradeselection:140} is

\begin{dmath}\label{eqn:gradeselection:500}
\Ba \wedge \Bb
=
\gpgradetwo{
\lr{ x \Be_1 + y \Be_2 } 
\lr{ x' \Be_1 + y' \Be_2 }
}
=
\gpgradetwo{
(x x' + y y') + (x y' - x' y) \Be_1 \Be_2
}
=
(x y' - x' y) \Be_1 \Be_2.
\end{dmath}

The wedge product of two vectors in a plane contains an antisymetrized sum of the vector coefficients, but is weighted by a ``unit'' bivector, the pseudoscalar for the plane.

As another example consider

\begin{dmath}\label{eqn:gradeselection:520}
\Be_1 \wedge \lr{ 2\Be_1 + 3 \Be_2 }
=
\gpgradetwo{
\Be_1 \lr{ 2\Be_1 + 3 \Be_2 }
}
=
\gpgradetwo{
2 \Be_1^2 + 3 \Be_1 \Be_2 
}
=
3 \Be_1 \Be_2.
\end{dmath}

Here we see that the component of the second vector \( \Be_1 + 3 \Be_2 \) that is colinear with the first vector \( \Be_1 \) is filtered out.  It is not coincidence that this is also a property of the cross product.  That relationship will be explored in (\cref{problem:gradeselection:4}).

As a final example, consider the wedge product of a vector with a bivector

\begin{dmath}\label{eqn:gradeselection:540}
\Be_1 \wedge \lr{ \Be_1 \Be_2 - 7 \Be_2 \Be_3 }
=
\gpgradethree{
\Be_1 \lr{ \Be_1 \Be_2 - 7 \Be_2 \Be_3 }
}
=
\gpgradethree{
\Be_1^2 \Be_2 - 7 \Be_1 \Be_2 \Be_3 
}
=
- 7 \Be_1 \Be_2 \Be_3.
\end{dmath}

Because \( \Be_1 \Be_2 \) has a common factor with \( \Be_1 \) it is filtered out of the resulting wedge product.  The end result, in this case, is a 3D pseudoscalar.

\section{Problems}

\makeproblem{\R{n} dot product.}{problem:gradeselection:1}{
Show that \ref{dfn:gradeselection:100} when applied to two vectors
is equivalent to the traditional \R{n} dot product.
} % problem

\makeanswer{problem:gradeselection:1}{
Let
\begin{dmath}\label{eqn:gradeselection:180}
\begin{aligned}
\Bx &= \sum_{i=1}^N x_i \Be_i \\
\By &= \sum_{i=1}^N y_i \Be_i.
\end{aligned}
\end{dmath}

The dot product of these two vectors is
\begin{dmath}\label{eqn:gradeselection:200}
\Bx \cdot \By 
\equiv
\gpgradezero{ \Bx \By }
=
\gpgradezero{ 
\lr{ \sum_{i=1}^N x_i \Be_i}
\lr{ \sum_{j=1}^N y_j \Be_j}
}
=
\sum_{1 \le i = j \le N}
x_i y_j 
\gpgradezero{ \Be_i \Be_j }
+
\sum_{1 \le i \ne j \le N}
x_i y_j 
\gpgradezero{ \Be_i \Be_j }
\end{dmath}

In the \( i = j \) sum, the term \( \Be_i \Be_j = \Be_i^2 = 1 \), so the scalar grade selection of that multivector product is just 1.  In the \( i = j \) term, each of the \( \Be_i \Be_j \) products is a bivector, so each of those scalar grade selections is zero.

That leaves

\begin{dmath}\label{eqn:gradeselection:220}
\Bx \cdot \By 
=
\sum_{i =1}^N x_i y_i. \qedmarker
\end{dmath}
} % answer

\makeproblem{Complex numbers}{problem:gradeselection:2}{
Show that complex numbers can be represented as even grade multivectors \( z = a + \Be_1 \Be_2 b \).
} % problem

\makeanswer{problem:gradeselection:2}{
Let \( i = \Be_1 \Be_2 \), for which we find

\begin{dmath}\label{eqn:gradeselection:260}
i^2 
= 
\lr{ \Be_1 \Be_2 }
\lr{ \Be_1 \Be_2 }
=
\Be_1 (\Be_2 \Be_1) \Be_2 
=
\Be_1 (-\Be_1 \Be_2) \Be_2 
=
-(\Be_1^2) (\Be_2^2)
=
-1.
\end{dmath}

The even grade multivector \( z = a + i b \) is thus seen to have all the properties required of complex numbers.
} % answer

\makeproblem{Plane rotations.}{problem:gradeselection:3}{

With \( i = \Be_1 \Be_2 \) for the pseudoscalar of the \( x,y \) plane,

\makesubproblem{}{problem:gradeselection:3:b}
justify the assertion that \( e^{i \theta} = \cos\theta + i \sin\theta \), where \( theta \) is a scalar angle.

\makesubproblem{}{problem:gradeselection:3:c}
Show that right multiplication of a 2D vector by \( e^{i\theta} \) rotates that vector by \( \theta \) radians.

\makesubproblem{}{problem:gradeselection:3:d}
Does the rotation multivector \( e^{i\theta} \) commute with the 2D basis vectors?

\makesubproblem{}{problem:gradeselection:3:e}
What is the action of multiplication of a vector by \( e^{i\theta} \) from the left?
} % problem

\makeanswer{problem:gradeselection:3}{
\makeSubAnswer{}{problem:gradeselection:3:b}

Assume that the exponential of a multivector argument is represented by a Taylor series

\begin{dmath}\label{eqn:gradeselection:280}
e^X = \sum_{k = 0}^\infty \frac{X^k}{k!},
\end{dmath}

and note that the pseudoscalar commutes with scalar rotation angles \( \theta \), so
\begin{dmath}\label{eqn:gradeselection:300}
e^{i\theta} 
= \sum_{k = 0}^\infty \frac{(i\theta)^k}{k!}
= \sum_{k = 0}^\infty \frac{i^k\theta^k}{k!}
= 
\sum_{k = 0}^\infty \frac{i^{2k}\theta^{2k}}{(2k)!}
+
\sum_{k = 0}^\infty \frac{i^{2k + 1}\theta^{2k +1}}{(2k + 1)!}
= 
\sum_{k = 0}^\infty \frac{(-1)^{k}\theta^{2k}}{(2k)!}
+
i \sum_{k = 0}^\infty \frac{(-1)^{k}\theta^{2k +1}}{(2k + 1)!}
= \cos \theta + i \sin\theta.
\end{dmath}
\makeSubAnswer{}{problem:gradeselection:3:c}

Consider the action of the exponential on each of the unit vectors.  For \( \Be_1 \) that is

\begin{dmath}\label{eqn:gradeselection:320}
\Be_1 e^{i \theta}
=
\Be_1 \lr{ \cos\theta + i \sin\theta }
=
\Be_1 \cos\theta + \Be_1 (\Be_1 \Be_2 )\sin\theta 
=
\Be_1 \cos\theta + \Be_2 \sin\theta.
\end{dmath}

This shows that the vector \( \Be_1 \) is rotated counterclockwise by \( \theta \) radians.  Similarly for \( \Be_2 \) 

\begin{dmath}\label{eqn:gradeselection:340}
\Be_2 e^{i \theta}
=
\Be_2 \lr{ \cos\theta + i \sin\theta }
=
\Be_2 \cos\theta + \Be_1 (\Be_1 \Be_2 )\sin\theta 
=
\Be_2 \cos\theta + \Be_1 (-\Be_2 \Be_1) \sin\theta.
=
\Be_2 \cos\theta - \Be_1 \sin\theta.
\end{dmath}

This is also a rotation by \( \theta \) radians.  Given a vector \( \Bx = x \Be_1 + y \Be_2 \), this gives

\begin{dmath}\label{eqn:gradeselection:360}
\Bx' 
= \Bx e^{i\theta}
=
x \lr{ \Be_1 \cos\theta + \Be_2 \sin\theta } + y \lr{ \Be_2 \cos\theta - \Be_1 \sin\theta }.
\end{dmath}

In particular

\begin{dmath}\label{eqn:gradeselection:380}
\begin{bmatrix}
\Bx' \cdot \Be_1 \\
\Bx' \cdot \Be_2 \\
\end{bmatrix}
=
\begin{bmatrix}
x \cos\theta - y \sin\theta \\
x \sin\theta + y \cos\theta 
\end{bmatrix}
=
\begin{bmatrix}
\cos\theta &- \sin\theta \\
\sin\theta &+ \cos\theta 
\end{bmatrix}
\begin{bmatrix}
x \\
y
\end{bmatrix}.
\end{dmath}

Observe that this is the rotation matrix that takes the points \((x, y)\) to their position \((x', y')\) rotated by \( \theta \) radians.
\makeSubAnswer{}{problem:gradeselection:3:d}

The action from the left on \( \Be_1 \) is

\begin{dmath}\label{eqn:gradeselection:400}
e^{i\theta} \Be_1
=
\lr{ \cos\theta + \Be_1 \Be_2 \sin\theta} \Be_1
=
\Be_1 \cos\theta + \Be_1 \Be_2 \Be_1 \sin\theta
=
\Be_1 \cos\theta + \Be_1 (-\Be_1 \Be_2) \sin\theta
=
\Be_1 \lr{ \cos\theta - i \sin\theta }
=
\Be_1 e^{-i\theta},
\end{dmath}

and the action from the left on \( \Be_2 \) is

\begin{dmath}\label{eqn:gradeselection:420}
e^{i\theta} \Be_2
=
\lr{ \cos\theta + \Be_1 \Be_2 \sin\theta} \Be_2
=
\Be_2 \cos\theta + \Be_1 \sin\theta
=
\Be_2 \cos\theta + (\Be_2 \Be_2) \Be_1 \sin\theta
=
\Be_2 \lr{ \cos\theta - i \sin\theta }
=
\Be_2 e^{-i\theta}.
\end{dmath}

This change of sign is due to the fact that the pseudoscalar anticommutes with each of the basis vectors in the plane.

\makeSubAnswer{}{problem:gradeselection:3:e}

Since the exponential toggles sign on commutation with both of the vectors of the plane, the rotation operation can be applied from either left or right, with sufficient care to get the direction right

\begin{equation}\label{eqn:gradeselection:440}
\Bx e^{i\theta} = e^{-i\theta} \Bx.
\end{equation}

It is also possible to split the rotation operation into half angle rotation operators that act from both the left and right

\begin{dmath}\label{eqn:gradeselection:460}
\Bx' = e^{-i\theta/2} \Bx e^{i\theta/2}.
\end{dmath}

A student who has studied computer graphics rotation theory may have seen quaterion rotation operators with this form, and a student of quantum mechanics will have seen Pauli matrix rotation operations of this form.  This is, in fact, the form that is generally desirable for 3D or higher order rotations, since it rotates the portions of a vector that lie in the rotation plane, leaving the normal components untouched.
} % answer

\makeproblem{Wedge relationship to the cross product.}{problem:gradeselection:4}{
For a pair of \R{3} vectors \( \Bx, \By \), show that the wedge and cross products are related by
\begin{dmath}\label{eqn:gradeselection:560}
\Bx \wedge \By = I (\Bx \cross \By),
\end{dmath}

where \( I = \Be_1 \Be_2 \Be_3 \) is the \R{N} pseudoscalar.
} % problem

\makeanswer{problem:gradeselection:4}{
Given \( \Bx = \sum_i x_i \Be_i \), and \( \By = \sum_i y_i \Be_i \), the wedge of these two vectors is a grade two selection that picks out only products that differ in index

\begin{dmath}\label{eqn:gradeselection:580}
\Bx \wedge \By
=
\gpgradetwo{ \Bx \By }
=
\sum_{i,j} \gpgradetwo{ x_i \Be_i y_j \Be_j }
=
\sum_{i \ne j} x_i y_j \gpgradetwo{ \Be_i \Be_j }
=
\sum_{i \ne j} x_i y_j \Be_i \Be_j
=
\sum_{i < j} (x_i y_j - x_j y_i) \Be_i \Be_j.
\end{dmath}

Written out explicitly, this is

\begin{dmath}\label{eqn:gradeselection:600}
\begin{aligned}
\Bx \wedge \By
&=
(x_1 y_2 - x_2 y_1) \Be_1 \Be_2 \\
&\quad+
(x_1 y_3 - x_3 y_1) \Be_1 \Be_3 \\
&\quad+
(x_2 y_3 - x_3 y_2) \Be_2 \Be_3 \\
&=
(x_1 y_2 - x_2 y_1) \Be_1 \Be_2 (\Be_3 \Be_3) \\
&\quad+
(x_1 y_3 - x_3 y_1) \Be_1 \Be_3 (\Be_2 \Be_2) \\
&\quad+
(x_2 y_3 - x_3 y_2) \Be_2 \Be_3 (\Be_1 \Be_1) \\
&=
(x_1 y_2 - x_2 y_1) I \Be_3 \\
&\quad+
(x_1 y_3 - x_3 y_1) (-I) \Be_2 \\
&\quad+
(x_2 y_3 - x_3 y_2) I \Be_1 \\
&= I (\Bx \cross \By).
\end{aligned}
\end{dmath}
} % answer

%}
\EndArticle
%\EndNoBibArticle

         \subsection{Problems}
            
\makeproblem{\R{n} dot product.}{problem:gradeselection:RnDotProduct}{
Show that \ref{dfn:gradeselection:100} when applied to two vectors
is equivalent to the traditional \R{n} dot product.
} % problem

            
\makeproblem{}{problem:gradeselection:cyclicpermutationtwo}{
Show that
\begin{dmath}\label{eqn:gradeselection:580}
\gpgradezero{ \Bx \By }
=
\gpgradezero{ \By \Bx }.
\end{dmath}
} % problem

            
\makeproblem{Dot product of vectors as symmetric sum}{problem:gradeselection:dotprod}{
Show that the dot product of two vectors can be written as a symmetric sum

\begin{dmath}\label{eqn:gradeselection:600}
\Bx \cdot \By = \inv{2} \lr{ \Bx \By + \By \Bx }.
\end{dmath}
} % problem

            \makeproblem{Plane rotations.}{problem:gradeselection:PlaneRotations}{

With \( i = \Be_1 \Be_2 \) for the pseudoscalar of the \( x,y \) plane,

\makesubproblem{}{problem:gradeselection:3:b}
justify the assertion that \( e^{i \theta} = \cos\theta + i \sin\theta \), where \( theta \) is a scalar angle.

\makesubproblem{}{problem:gradeselection:3:c}
Show that right multiplication of a 2D vector by \( e^{i\theta} \) rotates that vector by \( \theta \) radians.

\makesubproblem{}{problem:gradeselection:3:d}
Does the rotation multivector \( e^{i\theta} \) commute with the 2D basis vectors?

\makesubproblem{}{problem:gradeselection:3:e}
What is the action of multiplication of a vector by \( e^{i\theta} \) from the left?
} % problem

\makeanswer{problem:gradeselection:PlaneRotations}{
\makeSubAnswer{}{problem:gradeselection:3:b}

Assume that the exponential of a multivector argument is represented by a Taylor series

\begin{dmath}\label{eqn:gradeselectionProblems:280}
e^X = \sum_{k = 0}^\infty \frac{X^k}{k!},
\end{dmath}

and note that the pseudoscalar commutes with scalar rotation angles \( \theta \), so
\begin{dmath}\label{eqn:gradeselectionProblems:300}
e^{i\theta}
= \sum_{k = 0}^\infty \frac{(i\theta)^k}{k!}
= \sum_{k = 0}^\infty \frac{i^k\theta^k}{k!}
=
\sum_{k = 0}^\infty \frac{i^{2k}\theta^{2k}}{(2k)!}
+
\sum_{k = 0}^\infty \frac{i^{2k + 1}\theta^{2k +1}}{(2k + 1)!}
=
\sum_{k = 0}^\infty \frac{(-1)^{k}\theta^{2k}}{(2k)!}
+
i \sum_{k = 0}^\infty \frac{(-1)^{k}\theta^{2k +1}}{(2k + 1)!}
= \cos \theta + i \sin\theta.
\end{dmath}
\makeSubAnswer{}{problem:gradeselection:3:c}

Consider the action of the exponential on each of the unit vectors.  For \( \Be_1 \) that is

\begin{dmath}\label{eqn:gradeselectionProblems:320}
\Be_1 e^{i \theta}
=
\Be_1 \lr{ \cos\theta + i \sin\theta }
=
\Be_1 \cos\theta + \Be_1 (\Be_1 \Be_2 )\sin\theta
=
\Be_1 \cos\theta + \Be_2 \sin\theta.
\end{dmath}

This shows that the vector \( \Be_1 \) is rotated counterclockwise by \( \theta \) radians.  Similarly for \( \Be_2 \)

\begin{dmath}\label{eqn:gradeselectionProblems:340}
\Be_2 e^{i \theta}
=
\Be_2 \lr{ \cos\theta + i \sin\theta }
=
\Be_2 \cos\theta + \Be_1 (\Be_1 \Be_2 )\sin\theta
=
\Be_2 \cos\theta + \Be_1 (-\Be_2 \Be_1) \sin\theta.
=
\Be_2 \cos\theta - \Be_1 \sin\theta.
\end{dmath}

This is also a rotation by \( \theta \) radians.  Given a vector \( \Bx = x \Be_1 + y \Be_2 \), this gives

\begin{dmath}\label{eqn:gradeselectionProblems:360}
\Bx'
= \Bx e^{i\theta}
=
x \lr{ \Be_1 \cos\theta + \Be_2 \sin\theta } + y \lr{ \Be_2 \cos\theta - \Be_1 \sin\theta }.
\end{dmath}

In particular

\begin{dmath}\label{eqn:gradeselectionProblems:380}
\begin{bmatrix}
\Bx' \cdot \Be_1 \\
\Bx' \cdot \Be_2 \\
\end{bmatrix}
=
\begin{bmatrix}
x \cos\theta - y \sin\theta \\
x \sin\theta + y \cos\theta
\end{bmatrix}
=
\begin{bmatrix}
\cos\theta &- \sin\theta \\
\sin\theta &+ \cos\theta
\end{bmatrix}
\begin{bmatrix}
x \\
y
\end{bmatrix}.
\end{dmath}

Observe that this is the rotation matrix that takes the points \((x, y)\) to their position \((x', y')\) rotated by \( \theta \) radians.
\makeSubAnswer{}{problem:gradeselection:3:d}

The action from the left on \( \Be_1 \) is

\begin{dmath}\label{eqn:gradeselectionProblems:400}
e^{i\theta} \Be_1
=
\lr{ \cos\theta + \Be_1 \Be_2 \sin\theta} \Be_1
=
\Be_1 \cos\theta + \Be_1 \Be_2 \Be_1 \sin\theta
=
\Be_1 \cos\theta + \Be_1 (-\Be_1 \Be_2) \sin\theta
=
\Be_1 \lr{ \cos\theta - i \sin\theta }
=
\Be_1 e^{-i\theta},
\end{dmath}

and the action from the left on \( \Be_2 \) is

\begin{dmath}\label{eqn:gradeselectionProblems:420}
e^{i\theta} \Be_2
=
\lr{ \cos\theta + \Be_1 \Be_2 \sin\theta} \Be_2
=
\Be_2 \cos\theta + \Be_1 \sin\theta
=
\Be_2 \cos\theta + (\Be_2 \Be_2) \Be_1 \sin\theta
=
\Be_2 \lr{ \cos\theta - i \sin\theta }
=
\Be_2 e^{-i\theta}.
\end{dmath}

This change of sign is due to the fact that the pseudoscalar anticommutes with each of the basis vectors in the plane.

\makeSubAnswer{}{problem:gradeselection:3:e}

Since the exponential toggles sign on commutation with both of the vectors of the plane, the rotation operation can be applied from either left or right, with sufficient care to get the direction right

\begin{equation}\label{eqn:gradeselectionProblems:440}
\Bx e^{i\theta} = e^{-i\theta} \Bx.
\end{equation}

It is also possible to split the rotation operation into half angle rotation operators that act from both the left and right

\begin{dmath}\label{eqn:gradeselectionProblems:460}
\Bx' = e^{-i\theta/2} \Bx e^{i\theta/2}.
\end{dmath}

A student who has studied computer graphics rotation theory may have seen quaternion rotation operators with this form, and a student of quantum mechanics will have seen Pauli matrix rotation operations of this form.  This is, in fact, the form that is generally desirable for 3D or higher order rotations, since it rotates the portions of a vector that lie in the rotation plane, leaving the normal components untouched.
} % answer

            \makeproblem{Complex numbers}{problem:gradeselection:ComplexNumbers}{
Show that complex numbers can be represented as even grade multivectors \( z = a + \Be_1 \Be_2 b \).
} % problem

            \makeproblem{\R{3} pseudoscalar commutation.}{problem:gradeselection:R3PseudoscalarCommutation}{
Show that \( I \) given by \cref{eqn:definitions:340}
commutes with any grade \R{3} multivector.
} % problem

\makeanswer{problem:gradeselection:R3PseudoscalarCommutation}{

Showing that \( I \) commutes with each of the basis vectors is sufficient

\begin{dmath}\label{eqn:gradeselectionProblems:620}
\Be_1 I
=
\Be_1 (\Be_1 \Be_2 \Be_3)
=
\Be_1 (-\Be_2 \Be_1) \Be_3
=
-\Be_1 \Be_2 (-\Be_3 \Be_1)
=
I \Be_1
\end{dmath}
\begin{dmath}\label{eqn:gradeselectionProblems:640}
\Be_2 I
=
\Be_2 (\Be_1 \Be_2 \Be_3)
=
\Be_2 \Be_1 (-\Be_3 \Be_2)
=
-(-\Be_1 \Be_2) \Be_3 \Be_2
=
I \Be_2.
\end{dmath}
\begin{dmath}\label{eqn:gradeselectionProblems:660}
\Be_3 I
=
\Be_3 (\Be_1 \Be_2 \Be_3)
=
(\Be_3 \Be_1 \Be_2) \Be_3
=
-(\Be_1 \Be_3) \Be_2 \Be_3
=
-\Be_1 (-\Be_2 \Be_3) \Be_3
=
I \Be_3. \qedmarker
\end{dmath}
} % answer

            \makeproblem{Vector wedge coordinate expansion and antisymmetry}{problem:gradeselection:vectorwedge}{
Show that
\begin{dmath}\label{eqn:gradeselection:620}
\Bx \wedge \By
=
\sum_{i < j} (x_i y_j - x_j y_i) \Be_i \Be_j.
\end{dmath}

Observe from this coordinate expansion that the wedge product of two vectors is antisymmetric
\boxedEquation{eqn:gradeselection:640}{
\Bx \wedge \By = -\By \wedge \Bx.
}
} % problem

\makeanswer{problem:gradeselection:vectorwedge}{
Given \( \Bx = \sum_i x_i \Be_i \), and \( \By = \sum_i y_i \Be_i \), the wedge of these two vectors is a grade two selection that picks out only products that differ in index

\begin{dmath}\label{eqn:gradeselectionProblems:580}
\Bx \wedge \By
=
\gpgradetwo{ \Bx \By }
=
\sum_{i,j} \gpgradetwo{ x_i \Be_i y_j \Be_j }
=
\sum_{i \ne j} x_i y_j \gpgradetwo{ \Be_i \Be_j }
=
\sum_{i \ne j} x_i y_j \Be_i \Be_j
=
\sum_{i < j} (x_i y_j - x_j y_i) \Be_i \Be_j.
\end{dmath}
} % answer

            %
% Copyright © 2016 Peeter Joot.  All Rights Reserved.
% Licenced as described in the file LICENSE under the root directory of this GIT repository.
%

\makeproblem{Wedge relationship to the cross product.}{problem:gradeselection:WedgeRelationshipToCrossProduct}{
For a pair of \R{3} vectors \( \Bx, \By \), show that the wedge and cross products are related by
\begin{dmath}\label{eqn:gradeselectionProblems:560}
\Bx \wedge \By = I (\Bx \cross \By),
\end{dmath}

where \( I = \Be_1 \Be_2 \Be_3 \) is the \R{3} pseudoscalar.
} % problem

\makeanswer{problem:gradeselection:WedgeRelationshipToCrossProduct}{
Writing out \cref{eqn:gradeselectionProblems:580} explicitly gives

\begin{dmath}\label{eqn:gradeselectionProblems:600}
\begin{aligned}
\Bx \wedge \By
&=
(x_1 y_2 - x_2 y_1) \Be_1 \Be_2 \\
&\quad+
(x_1 y_3 - x_3 y_1) \Be_1 \Be_3 \\
&\quad+
(x_2 y_3 - x_3 y_2) \Be_2 \Be_3 \\
&=
(x_1 y_2 - x_2 y_1) \Be_1 \Be_2 (\Be_3 \Be_3) \\
&\quad+
(x_1 y_3 - x_3 y_1) \Be_1 \Be_3 (\Be_2 \Be_2) \\
&\quad+
(x_2 y_3 - x_3 y_2) \Be_2 \Be_3 (\Be_1 \Be_1) \\
&=
(x_1 y_2 - x_2 y_1) I \Be_3 \\
&\quad+
(x_1 y_3 - x_3 y_1) (-I) \Be_2 \\
&\quad+
(x_2 y_3 - x_3 y_2) I \Be_1 \\
&= I (\Bx \cross \By).
\end{aligned}
\end{dmath}
} % answer

            
\makeproblem{Vector bivector dot product}{problem:gradeselection:vectorBivectorDot}{
The dot product of a vector and bivector in \R{N} (or in fact any metric) expands as

\boxedEquation{eqn:gradeselection:660}{
\Ba \cdot \lr{ \Bb \wedge \Bc }
=
-\lr{ \Bb \wedge \Bc } \cdot \Ba
=
( \Ba \cdot \Bb ) \Bc
-( \Ba \cdot \Bc ) \Bb.
}

Demonstrate this by coordinate expansion using an orthonormal basis for \R{N}.

The right hand side may look familiar.  Demonstrate, for \R{3} without expansion in coordinates, that

\boxedEquation{eqn:gradeselection:680}{
\Ba \cdot \lr{ \Bb \wedge \Bc }
=
-\Ba \cross \lr{ \Bb \cross \Bc }.
}
} % problem

\makeanswer{problem:gradeselection:vectorBivectorDot}{
Expansion in coordinates is frowned upon in a number of GA references, but can be a quick way to the results of interest.  Consider such an expansion for a \R{N} vector space

\begin{dmath}\label{eqn:gradeselectionProblems:681}
\begin{aligned}
\Ba \cdot \lr{ \Bb \wedge \Bc } &= \sum_{i, j, k} a_i b_j c_k \Be_i \cdot (\Be_j \wedge \Be_k) \\
\lr{ \Bb \wedge \Bc } \cdot \Ba &= \sum_{i, j, k} a_i b_j c_k (\Be_j \wedge \Be_k) \cdot \Be_i
\end{aligned}
\end{dmath}

Observe that these sums can be restricted to indexes \( i \ne j \), since \( \Bx \wedge \Bx = 0 \) for any \(\Bx\).  The dot products are

\begin{dmath}\label{eqn:gradeselectionProblems:820}
\Be_i \cdot (\Be_j \wedge \Be_k)
=
\gpgradeone{ \Be_i (\Be_j \wedge \Be_k) }
=
\gpgradeone{ \Be_i \Be_j \Be_k },
\end{dmath}

and
\begin{dmath}\label{eqn:gradeselectionProblems:840}
(\Be_j \wedge \Be_k) \cdot \Be_i
=
\gpgradeone{ (\Be_j \wedge \Be_k) \Be_i }
=
\gpgradeone{ \Be_j \Be_k \Be_i }.
\end{dmath}

In each expansion, there are three cases, one where \( i,j,k\) are all unique.  In this case, the vector product is a trivector, so the grade one selection is zero.  That leaves only \( i = j \ne k \), and \( i = k \ne j \).

Consider the \( i = j \) case in the first dot product expansion

\begin{dmath}\label{eqn:gradeselectionProblems:860}
\gpgradeone{ \Be_i \Be_j \Be_k }
=
\gpgradeone{ \Be_i \Be_i \Be_k }
=
\gpgradeone{ \Be_k }
=
\Be_k.
\end{dmath}

For the \( i = k \) case, this is

\begin{dmath}\label{eqn:gradeselectionProblems:880}
\gpgradeone{ \Be_i \Be_j \Be_k }
=
\gpgradeone{ \Be_i (-\Be_k \Be_j) }
=
-\gpgradeone{ \Be_i \Be_i \Be_j }
=
-\gpgradeone{ \Be_j }
=
-\Be_j.
\end{dmath}

Inspection shows that the general pattern is
\begin{dmath}\label{eqn:gradeselectionProblems:900}
\Be_i \cdot (\Be_j \wedge \Be_k) =
(\Be_i \cdot \Be_j) \Be_k
-(\Be_i \cdot \Be_k) \Be_j,
\end{dmath}

and
\begin{dmath}\label{eqn:gradeselectionProblems:920}
(\Be_j \wedge \Be_k) \cdot \Be_i =
(\Be_i \cdot \Be_k) \Be_j
-(\Be_i \cdot \Be_j) \Be_k.
\end{dmath}

Substitution back into \cref{eqn:gradeselectionProblems:681} proves the first result for Euclidean spaces.  For the relation to the cross product in the \R{3} case

\begin{dmath}\label{eqn:gradeselectionProblems:940}
\Ba \cdot \lr{ \Bb \wedge \Bc }
=
\gpgradeone{
\Ba \lr{ \Bb \wedge \Bc }
}
=
\gpgradeone{
\Ba I \lr{ \Bb \cross \Bc }
}
=
\gpgradeone{
I \Ba \lr{ \Bb \cross \Bc }
}
=
\gpgradeone{
I \lr{
\Ba \wedge \lr{ \Bb \cross \Bc }
+
\Ba \cdot \lr{ \Bb \cross \Bc }
}
}.
\end{dmath}

The dot product leaves the vector selection of a trivector, which is zero.  Expanding the wedge product as a cross product once again gives
\begin{dmath}\label{eqn:gradeselectionProblems:960}
\Ba \cdot \lr{ \Bb \wedge \Bc }
=
\gpgradeone{
I^2
\Ba \cross \lr{ \Bb \cross \Bc }
}
=
-\Ba \cross \lr{ \Bb \cross \Bc }.
\end{dmath}

} % answer

            %
% Copyright © 2016 Peeter Joot.  All Rights Reserved.
% Licenced as described in the file LICENSE under the root directory of this GIT repository.
%

\makeproblem{Vector trivector dot product}{problem:gradeselection:vectorTrivectorDot}{
Show that

\begin{dmath}\label{eqn:vectorTrivectorDot:20}
\Ba \cdot \lr{ \Bb \wedge \Bc \wedge \Bd}
=
\lr{ \Bb \wedge \Bc \wedge \Bd} \cdot \Ba
=
( \Ba \cdot \Bb ) (\Bc \wedge \Bd)
-( \Ba \cdot \Bc ) (\Bb \wedge \Bd)
+( \Ba \cdot \Bd ) (\Bb \wedge \Bc).
\end{dmath}

Note that this is another specific case of the more general identity

%\begin{dmath}\label{eqn:vectorTrivectorDot:100}
\boxedEquation{eqn:vectorTrivectorDot:100}{
\Bx \cdot \lr{ \By_1 \wedge \By_2 \wedge \cdots \wedge \By_n }
=
\sum_{i = 1}^n (-1)^i (\Bx \cdot \By_i) \lr{ \By_1 \wedge \cdots \wedge \By_{i-1} \wedge \By_{i+1} \wedge \cdots \wedge \By_n }.
}
%\end{dmath}

This dot product is symmetric(antisymmetric) when the grade of the blade the vector is dotted with is odd(even).

See \citep{doran2003gap} for a demonstration that this holds for any metric.
} % problem

\makeanswer{problem:gradeselection:vectorTrivectorDot}{
Expanding in coordinates gives

\begin{dmath}\label{eqn:vectorTrivectorDot:40}
\Ba \cdot \lr{ \Bb \wedge \Bc \wedge \Bd}
= \sum_{j \ne k \ne l} a_i b_j c_k d_l
\gpgradetwo{ \Be_i \Be_j \Be_k \Be_l }.
\end{dmath}

The products within the grade two selection operator can be of either grade two or grade four, so only the terms where one of
\( i = j \), \( i = k \), or \( i = l \) contributes.  Repeated anticommutation of the perperdicular unit vectors can put each such pair adjacent, where they square to unity.  Those are respectively

\begin{dmath}\label{eqn:vectorTrivectorDot:60}
\begin{aligned}
\gpgradetwo{ \Be_i \Be_i \Be_k \Be_l } &= \Be_k \Be_l  \\
\gpgradetwo{ \Be_i \Be_j \Be_i \Be_l } &= -\gpgradetwo{ \Be_j \Be_i \Be_i \Be_l } = - \Be_j \Be_l \\
\gpgradetwo{ \Be_i \Be_j \Be_k \Be_i } &= -\gpgradetwo{ \Be_j \Be_i \Be_k \Be_i } = +\gpgradetwo{ \Be_j \Be_k \Be_i \Be_i } = \Be_j \Be_k
\end{aligned}
\end{dmath}

Substitution back into \cref{eqn:gradeselectionProblems:681} gives

\begin{dmath}\label{eqn:vectorTrivectorDot:80}
\Ba \cdot \lr{ \Bb \wedge \Bc \wedge \Bd}
= \sum_{j \ne k \ne l} a_i b_j c_k d_l
\lr{
\Be_i \cdot \Be_j (\Be_k \Be_l)
-
\Be_i \cdot \Be_k (\Be_j \Be_l)
+
\Be_i \cdot \Be_l (\Be_j \Be_k)
}
=
( \Ba \cdot \Bb ) (\Bc \wedge \Bd)
-( \Ba \cdot \Bc ) (\Bb \wedge \Bd)
+( \Ba \cdot \Bd ) (\Bb \wedge \Bc).
\end{dmath}

Repeating this from the other direction gives the same result.
} % answer

      \section{Product of two vectors}
         %
% Copyright � 2016 Peeter Joot.  All Rights Reserved.
% Licenced as described in the file LICENSE under the root directory of this GIT repository.
%
%{
%\newcommand{\authorname}{Peeter Joot}
\newcommand{\email}{peeterjoot@protonmail.com}
\newcommand{\basename}{FIXMEbasenameUndefined}
\newcommand{\dirname}{notes/FIXMEdirnameUndefined/}

%\renewcommand{\basename}{vectorproduct}
%%\renewcommand{\dirname}{notes/phy1520/}
%\renewcommand{\dirname}{notes/ece1228-electromagnetic-theory/}
%%\newcommand{\dateintitle}{}
%%\newcommand{\keywords}{}
%
%\newcommand{\authorname}{Peeter Joot}
\newcommand{\onlineurl}{http://sites.google.com/site/peeterjoot2/math2013/\basename.pdf}
\newcommand{\sourcepath}{\dirname\basename.tex}
\newcommand{\generatetitle}[1]{\chapter{#1}}

\newcommand{\vcsinfo}{%
\section*{}
\noindent{\color{DarkOliveGreen}{\rule{\linewidth}{0.1mm}}}
\paragraph{Document version}
%\paragraph{\color{Maroon}{Document version}}
{
\small
\begin{itemize}
\item Available online at:\\ 
\href{\onlineurl}{\onlineurl}
\item Git Repository: \input{./.revinfo/gitRepo.tex}
\item Source: \sourcepath
\item last commit: \input{./.revinfo/gitCommitString.tex}
\item commit date: \input{./.revinfo/gitCommitDate.tex}
\end{itemize}
}
}

%\PassOptionsToPackage{dvipsnames,svgnames}{xcolor}
\PassOptionsToPackage{square,numbers}{natbib}
\documentclass{scrreprt}

\usepackage[left=2cm,right=2cm]{geometry}
\usepackage[svgnames]{xcolor}
\usepackage{peeters_layout}

\usepackage{natbib}

\usepackage[
colorlinks=true,
bookmarks=false,
pdfauthor={\authorname, \email},
backref 
]{hyperref}

% http://tex.stackexchange.com/questions/75773/how-to-reference-problems-by-the-text-label-in-an-exercise-envioronment
\usepackage[english]{cleveref}
\crefname{Exercise}{exercise}{exercises}
\Crefname{Exercise}{Exercise}{Exercises}

\RequirePackage{titlesec}
\RequirePackage{ifthen}

% http://stackoverflow.com/questions/4932910/date-in-the-tabular-environment
\makeatletter
\let\insertdate\@date
\makeatother

\titleformat{\chapter}[display]
{\bfseries\Large}
{\color{DarkSlateGrey}\filleft \authorname
\ifthenelse{\isundefined{\studentnumber}}{}{\\ \studentnumber}
\ifthenelse{\isundefined{\email}}{}{\\ \email}
\ifthenelse{\isundefined{\dateintitle}}{}{\\ \insertdate}
%\ifthenelse{\isundefined{\coursename}}{}{\\ \coursename} % put in title instead.
}
{4ex}
{\color{DarkOliveGreen}{\titlerule}\color{Maroon}
\vspace{2ex}%
\filright}
[\vspace{2ex}%
\color{DarkOliveGreen}\titlerule
]

\newcommand{\beginArtWithToc}[0]{\begin{document}\tableofcontents}
\newcommand{\beginArtNoToc}[0]{\begin{document}}
\newcommand{\EndNoBibArticle}[0]{\end{document}}
\newcommand{\EndArticle}[0]{\bibliography{Bibliography}\bibliographystyle{plainnat}\end{document}}

% 
%\newcommand{\citep}[1]{\cite{#1}}

\colorSectionsForArticle


%
%\usepackage{peeters_layout_exercise}
%\usepackage{peeters_braket}
%\usepackage{peeters_figures}
%\usepackage{siunitx}
%%\usepackage{mhchem} % \ce{}
%%\usepackage{macros_bm} % \bcM
%%\usepackage{macros_qed} % \qedmarker
%%\usepackage{txfonts} % \ointclockwise
%
%\beginArtNoToc
%
%\generatetitle{XXX}
%%\chapter{XXX}
%%\label{chap:vectorproduct}
%
Given two vectors \( \Bx, \By \) the scalar grade of the vector product \( \Bx \By \) can be shown (\cref{problem:gradeselection:RnDotProduct}) to be
\begin{equation}\label{eqn:vectorproduct:20}
\gpgradezero{ \Bx \By }
=
\sum_{i = 1}^N x_i y_i
=
\Bx \cdot \By 
\end{equation}

The grade two selection of this product was found (\cref{problem:gradeselection:WedgeRelationshipToCrossProduct}) to be

\begin{equation}\label{eqn:vectorproduct:40}
\gpgradetwo{ \Bx \By }
=
\sum_{i < j} (x_i y_j - x_j y_i) \Be_i \Be_j
=
\Bx \wedge \By.
\end{equation}

The reader should convince themself that the vector product \( \Bx \By \) has only even grades (0,2), and can therefore be expanded as

\begin{dmath}\label{eqn:vectorproduct:60}
\Bx \By
=
\gpgradezero{ \Bx \By }
+
\gpgradetwo{ \Bx \By },
\end{dmath}

or
\boxedEquation{eqn:vectorproduct:80}{
\Bx \By
=
\Bx \cdot \By 
+
\Bx \wedge \By.
}

This is a fundamental and very useful relationship.  In these notes this is a consequence of the axioms and the generalized definitions of the dot and wedge products.  Some authors will use this to define the geometric product of two vectors.

When considering the Euclidean space \R{3}, an additional relationship follows (\cref{problem:gradeselection:WedgeRelationshipToCrossProduct}), which is also incredibly useful

\boxedEquation{eqn:vectorproduct:100}{
\Bx \By
=
\Bx \cdot \By 
+
I
(\Bx \cross \By).
}

Note that this is the GA equivalent of the Pauli relationship \cref{eqn:GAmotivation:120} that will be familiar to a student of quantum spin states.
The ability to combine dot and cross product relationships into a single multivector equation is not just a theoretical nicety.  This happens to also be the reason that GA is so applicable to the study of electromagnetism.   To illustrate this, and provide a hint of things to come consider the GA formulation of the electrostatic and magnetostatic Maxwell equations.

\makeexample{Electrostatic and magnetostatics.}{example:vectorproduct:electrostatics}{

With no currents, no magnetic sources, no time derivatives, Maxwell's equations in simple media take the form

\begin{dmath}\label{eqn:vectorproduct:120}
\begin{aligned}
\spacegrad \cdot \BB &= 0 \\
\spacegrad \cross \BB &= 0 \\
\spacegrad \cross \BE &= 0 \\
\spacegrad \cdot \BE &= \frac{\rho}{\epsilon}.
\end{aligned}
\end{dmath}

Using \cref{eqn:vectorproduct:100} the first and last equations can be combined.  The magnetic gradient equation is 

\begin{equation}\label{eqn:vectorproduct:140}
\spacegrad \BB 
=
\spacegrad \cdot \BB +I (\spacegrad \cross \BB )
=
0,
\end{equation}

and the electric gradient equation is

\begin{equation}\label{eqn:vectorproduct:160}
\spacegrad \BE
=
\spacegrad \cdot \BE +I (\spacegrad \cross \BE )
=
\frac{\rho}{\epsilon}.
\end{equation}

Maxwell's equations are reduced to two multivector equations with this transformation
\begin{dmath}\label{eqn:vectorproduct:180}
\begin{aligned}
\spacegrad \BE &= \frac{\rho}{\epsilon} \\
\spacegrad \BB &= 0.
\end{aligned}
\end{dmath}

For magnetostatics, Maxwell's equations take the form

\begin{dmath}\label{eqn:vectorproduct:200}
\begin{aligned}
\spacegrad \cdot \BB &= 0 \\
\spacegrad \cross \BB &= \mu \BJ \\
\spacegrad \cross \BE &= 0 \\
\spacegrad \cdot \BE &= 0.
\end{aligned}
\end{dmath}

As above, these can be combined as

\begin{dmath}\label{eqn:vectorproduct:220}
\begin{aligned}
\spacegrad \BB &= I \mu \BJ \\
\spacegrad \BE &= 0.
\end{aligned}
\end{dmath}

It will be seen later that it is actually more natural to express magnetic fields as a bivector \( I \BB \).  Using \( I^2 = -1 \) (\cref{problem:gradeselectionProblems:R3PseudoscalarSquare}) after which the magnetostatic equation takes the form

\begin{dmath}\label{eqn:vectorproduct:240}
\spacegrad (I \BB) = - \mu \BJ.
\end{dmath}

Both the electrostatic and magnetostatic equations can be solved directly using the Green's function for the gradient, but additional concepts are required before attacking multivector integrals.
} % example

%}
%\EndNoBibArticle

         \subsection{Problems}
            \makeproblem{Commutation within grade zero selection}{problem:vectorproduct:cyclicpermutation}{

It was previously shown using coordinates that

\begin{dmath}\label{eqn:vectorproduct:260}
\gpgradezero{ \Bx \By } = \gpgradezero{ \By \Bx }.
\end{dmath}

Repeat this proof using \cref{eqn:vectorproduct:80}.
} % problem

\makeanswer{problem:vectorproduct:cyclicpermutation}{
\begin{dmath}\label{eqn:vectorproduct:301}
\gpgradezero{ \By \Bx }
=
\gpgradezero{ \By \cdot \Bx  + \By \wedge \Bx }
=
\gpgradezero{ \By \cdot \Bx }
=
\gpgradezero{ \Bx \cdot \By }
=
\gpgradezero{ \Bx \By }
\end{dmath}
} % answer

            
\makeproblem{Vector wedge antisymmetric structure}{problem:vectorproduct:wedgeantisym}{
Prove the wedge product relationship of \cref{eqn:vectorproduct:300}.
} % problem

            %
% Copyright © 2016 Peeter Joot.  All Rights Reserved.
% Licenced as described in the file LICENSE under the root directory of this GIT repository.
%
\makeproblem{Wedge of three vectors}{problem:gradethreeselectionWedge:wedgeThree}{
Show that
\begin{dmath}\label{eqn:gradethreeselectionWedge:700}
\gpgradethree{ \Ba \Bb \Bc }
=
\Ba \wedge ( \Bb \wedge \Bc )
=
(\Ba \wedge \Bb) \wedge \Bc
=
-
\Bb \wedge (\Ba \wedge \Bc).
\end{dmath}

Observe that is antisymmetric in any two vectors, and thus completely antisymmetric (i.e. associative).  This allows the grade three selection of any three vectors to be written more simply as

\boxedEquation{eqn:gradethreeselectionWedge:720}{
\gpgradethree{ \Ba \Bb \Bc }
=
\Ba \wedge \Bb \wedge \Bc.
}

This can be considered the definition of \( \Ba \wedge \Bb \wedge \Bc \).
} % problem

\makeanswer{problem:gradethreeselectionWedge:wedgeThree}{
Consider an expansion first in products of \( \Ba, \Bb \)

\begin{dmath}\label{eqn:gradethreeselectionWedge:740}
\gpgradethree{ \Ba \Bb \Bc }
=
\gpgradethree{ (\cancel{\Ba \cdot \Bb} + \Ba \wedge \Bb) \Bc }
=
\gpgradethree{ (\Ba \wedge \Bb) \Bc }.
\end{dmath}

The dot product was killed since it leaves only a vector product within the grade selection operator.  Since a vector bivector product can have only grade 1 and grade three terms (example: \( \Be_1 (\Be_1 \wedge \Be_2) = \Be_2, \Be_1 (\Be_2 \wedge \Be_3) = \Be_1 \Be_2 \Be_3 \), this leaves just

\begin{dmath}\label{eqn:gradethreeselectionWedge:760}
\gpgradethree{ \Ba \Bb \Bc }
=
(\Ba \wedge \Bb) \wedge \Bc.
\end{dmath}

Similarly, expanding the \( \Bb \Bc \) product gives
\begin{dmath}\label{eqn:gradethreeselectionWedge:780}
\gpgradethree{ \Ba \Bb \Bc }
=
\gpgradethree{ \Ba (\cancel{\Bb \cdot \Bc} + \Ba \wedge \Bc) }
=
\gpgradethree{ \Ba (\Bb \wedge \Bc) }
=
\Ba \wedge (\Bb \wedge \Bc),
\end{dmath}

and finally, expanding products of \( \Ba \Bc \) after commutation

\begin{dmath}\label{eqn:gradethreeselectionWedge:800}
\gpgradethree{ \Ba \Bb \Bc }
=
\gpgradethree{ (\cancel{2 \Bb \cdot \Ba} - \Bb \Ba) 
\Bc}
=
-\gpgradethree{ \Bb \Ba \Bc }
=
-\gpgradethree{ \Bb (\cancel{\Ba \cdot \Bc} + \Ba \wedge \Bc) }
=
- \Bb \wedge (\Ba \wedge \Bc).
\end{dmath}
} % answer

      \section{Problem solutions}
         \shipoutAnswer
   \chapter{Geometry}
      \section{Bivectors}
gabookI: 3.9
      \section{Trivectors}
      \section{Projection and rejection}
      \section{Rotations}
gabookI: 2.5 rotations.
gabookI: 10.4.3 bivector generator of rotations.
gabookI: 29.1
      \section{Equivalent identities}
gabookI: 4.1+
      \section{Cramer's rule}
gabookI: 5.  Generalize examples to higher dimensions.
   \chapter{Vector calculus}
      \section{Reciprocal frames}
      \section{Curvilinear coordinates}
gabook: 31.1
      \section{Stokes theorem}
      \section{Divergence theorem}
      \section{Fundamental theorem of geometric calculus}
      \section{Helmholtz theorem}
gabook: 45.1

\part{Electromagnetism}
   \chapter{Maxwell's equations}
   \chapter{Electrostatics}
   \chapter{Magnetostatics}
   \chapter{Constitutive relations}
   \chapter{Boundary value conditions}
   \chapter{Time harmonic fields}
   \chapter{Polarization}
   \chapter{Potentials}
   \chapter{Green's functions}
   \chapter{Wave equations}
   \chapter{Radiation and scattering}
%\end{itemize}


