
\makeproblem{\R{3} pseudoscalar square}{problem:gradeselection:R3PseudoscalarSquare}{
With the \R{3} pseudoscalar of \cref{eqn:definitions:340} show that \( I^2 = -1 \).
} % problem

\makeanswer{problem:gradeselection:R3PseudoscalarSquare}{

Squaring the pseudoscalar gives

\begin{dmath}\label{eqn:gaTutorial:160}
I^2
=
(\Be_1 \Be_2 \Be_3)
(\Be_1 \Be_2 \Be_3)
=
\Be_1 \Be_2 (\Be_3
\Be_1) \Be_2 \Be_3
=
-\Be_1 \Be_2 \Be_1
\Be_3 \Be_2 \Be_3
=
-\Be_1 (\Be_2 \Be_1)
(\Be_3 \Be_2) \Be_3
=
-\Be_1 (\Be_1 \Be_2)
(\Be_2 \Be_3) \Be_3
=
-
\Be_1^2
\Be_2^2
\Be_3^2
=
-1,
\end{dmath}

as expected, showing that this quantity also has characteristics of an imaginary number.
} % answer
