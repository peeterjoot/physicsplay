%
% Copyright © 2016 Peeter Joot.  All Rights Reserved.
% Licenced as described in the file LICENSE under the root directory of this GIT repository.
%

\makeanswer{problem:gradeselection:RnDotProduct}{
Let
\begin{dmath}\label{eqn:gradeselectionProblems:180}
\begin{aligned}
\Bx &= \sum_{i=1}^N x_i \Be_i \\
\By &= \sum_{i=1}^N y_i \Be_i.
\end{aligned}
\end{dmath}

The dot product of these two vectors is
\begin{dmath}\label{eqn:gradeselectionProblems:200}
\Bx \cdot \By 
\equiv
\gpgradezero{ \Bx \By }
=
\gpgradezero{ 
\lr{ \sum_{i=1}^N x_i \Be_i}
\lr{ \sum_{j=1}^N y_j \Be_j}
}
=
\sum_{1 \le i = j \le N}
x_i y_j 
\gpgradezero{ \Be_i \Be_j }
+
\sum_{1 \le i \ne j \le N}
x_i y_j 
\gpgradezero{ \Be_i \Be_j }
\end{dmath}

In the \( i = j \) sum, the term \( \Be_i \Be_j = \Be_i^2 = 1 \), so the scalar grade selection of that multivector product is just 1.  In the \( i = j \) term, each of the \( \Be_i \Be_j \) products is a bivector, so each of those scalar grade selections is zero.

That leaves

\begin{dmath}\label{eqn:gradeselectionProblems:220}
\Bx \cdot \By 
=
\sum_{i =1}^N x_i y_i. \qedmarker
\end{dmath}
} % answer

\makeanswer{problem:gradeselection:cyclicpermutationtwo}{

Expanding the vector grade zero selection in coordinates gives
\begin{dmath}\label{eqn:gradeselection:680}
\gpgradezero{ \Bx \By }
=
\sum_{ij} \gpgradezero{ x_i y_j \Be_i \Be_j }
=
\sum_{i = j} \gpgradezero{ x_i y_i \Be_i \Be_i }
=
\sum_{i = j} \gpgradezero{ (y_i \Be_i)(x_i \Be_i) }
=
\sum_{i,j} \gpgradezero{ (y_i \Be_i)(x_i \Be_i) }
=
\gpgradezero{ \By \Bx }.
\end{dmath}
} % answer

\makeanswer{problem:gradeselection:dotprod}{
There are a few ways that this can be demonstrated.  The first relies on the classical definition of the dot product.  Expanding the square of a vector sum gives

\begin{dmath}\label{eqn:gradeselection:700}
(\Bx + \By)^2 = \Bx^2 + \By^2 + \Bx \By + \By \Bx.
\end{dmath}

By comparision this must also be equal to

\begin{dmath}\label{eqn:gradeselection:720}
\Abs{\Bx + \By}^2 = \Bx^2 + \By^2 + 2 \Bx \cdot \By,
\end{dmath}

so
\begin{dmath}\label{eqn:gradeselection:740}
\Bx \By + \By \Bx = 2 \Bx \cdot \By.
\end{dmath}

This might be viewed as a cheat, since it is not using the dot product as defined by grade zero selection according to \cref{dfn:gradeselection:100}.  Using that definition will produce the same result

\begin{dmath}\label{eqn:gradeselection:760}
\gpgradezero{ (\Bx + \By)^2 }
=
\gpgradezero{ \Bx^2 + \By^2 + \Bx \By + \By \Bx }
=
\Bx^2 + \By^2 
+ 
\gpgradezero{ 
\Bx \By }
+ \gpgradezero{ \By \Bx }.
\end{dmath}

It was shown in \cref{problem:gradeselection:cyclicpermutationtwo} that \( \gpgradezero{ \Bx \By } = \gpgradezero{ \By \Bx } \) so
\begin{dmath}\label{eqn:gradeselection:780}
2 \gpgradezero{ \Bx \By } = \Bx \By + \By \Bx.
\end{dmath}

Using \cref{dfn:gradeselection:100}, this completes the problem.
} % answer

\makeanswer{problem:gradeselection:ComplexNumbers}{
Let \( i = \Be_1 \Be_2 \), for which we find

\begin{dmath}\label{eqn:gradeselectionProblems:260}
i^2 
= 
\lr{ \Be_1 \Be_2 }
\lr{ \Be_1 \Be_2 }
=
\Be_1 (\Be_2 \Be_1) \Be_2 
=
\Be_1 (-\Be_1 \Be_2) \Be_2 
=
-(\Be_1^2) (\Be_2^2)
=
-1.
\end{dmath}

The even grade multivector \( z = a + i b \) is thus seen to have all the properties required of complex numbers.
} % answer

\makeanswer{problem:gradeselectionProblems:R3PseudoscalarSquare}{

Squaring the pseudoscalar gives

\begin{dmath}\label{eqn:gaTutorial:160}
I^2 
= 
(\Be_1 \Be_2 \Be_3)
(\Be_1 \Be_2 \Be_3)
= 
\Be_1 \Be_2 (\Be_3
\Be_1) \Be_2 \Be_3
=
-\Be_1 \Be_2 \Be_1
\Be_3 \Be_2 \Be_3
=
-\Be_1 (\Be_2 \Be_1)
(\Be_3 \Be_2) \Be_3
=
-\Be_1 (\Be_1 \Be_2)
(\Be_2 \Be_3) \Be_3
=
-
\Be_1^2
\Be_2^2
\Be_3^2
=
-1,
\end{dmath}

as expected, showing that this quantity also has characteristics of an imaginary number.
} % answer

\makeanswer{problem:vectorproduct:cyclicpermutation}{
\begin{dmath}\label{eqn:vectorproduct:300}
\gpgradezero{ \By \Bx }
=
\gpgradezero{ \By \cdot \Bx  + \By \wedge \Bx }
=
\gpgradezero{ \By \cdot \Bx }
=
\gpgradezero{ \Bx \cdot \By }
=
\gpgradezero{ \Bx \By }
\end{dmath}
} % answer

%\makeanswer{problem:vectorproduct:cyclicpermutationII}{
%FIXME: todo.
%} % answer
%
\makeanswer{problem:gradeselectionProblems:R3PseudoscalarCommutation}{

Showing that \( I \) commutes with each of the basis vectors is sufficient

\begin{dmath}\label{eqn:gradeselectionProblems:620}
\Be_1 I 
=
\Be_1 (\Be_1 \Be_2 \Be_3)
=
\Be_1 (-\Be_2 \Be_1) \Be_3
=
-\Be_1 \Be_2 (-\Be_3 \Be_1)
=
I \Be_1
\end{dmath}
\begin{dmath}\label{eqn:gradeselectionProblems:640}
\Be_2 I 
=
\Be_2 (\Be_1 \Be_2 \Be_3)
=
\Be_2 \Be_1 (-\Be_3 \Be_2)
=
-(-\Be_1 \Be_2) \Be_3 \Be_2
=
I \Be_2.
\end{dmath}
\begin{dmath}\label{eqn:gradeselectionProblems:660}
\Be_3 I 
=
\Be_3 (\Be_1 \Be_2 \Be_3)
=
(\Be_3 \Be_1 \Be_2) \Be_3
=
-(\Be_1 \Be_3) \Be_2 \Be_3
=
-\Be_1 (-\Be_2 \Be_3) \Be_3
=
I \Be_3. \qedmarker
\end{dmath}
} % answer

\makeanswer{problem:gradeselection:PlaneRotations}{
\makeSubAnswer{}{problem:gradeselection:3:b}

Assume that the exponential of a multivector argument is represented by a Taylor series

\begin{dmath}\label{eqn:gradeselectionProblems:280}
e^X = \sum_{k = 0}^\infty \frac{X^k}{k!},
\end{dmath}

and note that the pseudoscalar commutes with scalar rotation angles \( \theta \), so
\begin{dmath}\label{eqn:gradeselectionProblems:300}
e^{i\theta} 
= \sum_{k = 0}^\infty \frac{(i\theta)^k}{k!}
= \sum_{k = 0}^\infty \frac{i^k\theta^k}{k!}
= 
\sum_{k = 0}^\infty \frac{i^{2k}\theta^{2k}}{(2k)!}
+
\sum_{k = 0}^\infty \frac{i^{2k + 1}\theta^{2k +1}}{(2k + 1)!}
= 
\sum_{k = 0}^\infty \frac{(-1)^{k}\theta^{2k}}{(2k)!}
+
i \sum_{k = 0}^\infty \frac{(-1)^{k}\theta^{2k +1}}{(2k + 1)!}
= \cos \theta + i \sin\theta.
\end{dmath}
\makeSubAnswer{}{problem:gradeselection:3:c}

Consider the action of the exponential on each of the unit vectors.  For \( \Be_1 \) that is

\begin{dmath}\label{eqn:gradeselectionProblems:320}
\Be_1 e^{i \theta}
=
\Be_1 \lr{ \cos\theta + i \sin\theta }
=
\Be_1 \cos\theta + \Be_1 (\Be_1 \Be_2 )\sin\theta 
=
\Be_1 \cos\theta + \Be_2 \sin\theta.
\end{dmath}

This shows that the vector \( \Be_1 \) is rotated counterclockwise by \( \theta \) radians.  Similarly for \( \Be_2 \) 

\begin{dmath}\label{eqn:gradeselectionProblems:340}
\Be_2 e^{i \theta}
=
\Be_2 \lr{ \cos\theta + i \sin\theta }
=
\Be_2 \cos\theta + \Be_1 (\Be_1 \Be_2 )\sin\theta 
=
\Be_2 \cos\theta + \Be_1 (-\Be_2 \Be_1) \sin\theta.
=
\Be_2 \cos\theta - \Be_1 \sin\theta.
\end{dmath}

This is also a rotation by \( \theta \) radians.  Given a vector \( \Bx = x \Be_1 + y \Be_2 \), this gives

\begin{dmath}\label{eqn:gradeselectionProblems:360}
\Bx' 
= \Bx e^{i\theta}
=
x \lr{ \Be_1 \cos\theta + \Be_2 \sin\theta } + y \lr{ \Be_2 \cos\theta - \Be_1 \sin\theta }.
\end{dmath}

In particular

\begin{dmath}\label{eqn:gradeselectionProblems:380}
\begin{bmatrix}
\Bx' \cdot \Be_1 \\
\Bx' \cdot \Be_2 \\
\end{bmatrix}
=
\begin{bmatrix}
x \cos\theta - y \sin\theta \\
x \sin\theta + y \cos\theta 
\end{bmatrix}
=
\begin{bmatrix}
\cos\theta &- \sin\theta \\
\sin\theta &+ \cos\theta 
\end{bmatrix}
\begin{bmatrix}
x \\
y
\end{bmatrix}.
\end{dmath}

Observe that this is the rotation matrix that takes the points \((x, y)\) to their position \((x', y')\) rotated by \( \theta \) radians.
\makeSubAnswer{}{problem:gradeselection:3:d}

The action from the left on \( \Be_1 \) is

\begin{dmath}\label{eqn:gradeselectionProblems:400}
e^{i\theta} \Be_1
=
\lr{ \cos\theta + \Be_1 \Be_2 \sin\theta} \Be_1
=
\Be_1 \cos\theta + \Be_1 \Be_2 \Be_1 \sin\theta
=
\Be_1 \cos\theta + \Be_1 (-\Be_1 \Be_2) \sin\theta
=
\Be_1 \lr{ \cos\theta - i \sin\theta }
=
\Be_1 e^{-i\theta},
\end{dmath}

and the action from the left on \( \Be_2 \) is

\begin{dmath}\label{eqn:gradeselectionProblems:420}
e^{i\theta} \Be_2
=
\lr{ \cos\theta + \Be_1 \Be_2 \sin\theta} \Be_2
=
\Be_2 \cos\theta + \Be_1 \sin\theta
=
\Be_2 \cos\theta + (\Be_2 \Be_2) \Be_1 \sin\theta
=
\Be_2 \lr{ \cos\theta - i \sin\theta }
=
\Be_2 e^{-i\theta}.
\end{dmath}

This change of sign is due to the fact that the pseudoscalar anticommutes with each of the basis vectors in the plane.

\makeSubAnswer{}{problem:gradeselection:3:e}

Since the exponential toggles sign on commutation with both of the vectors of the plane, the rotation operation can be applied from either left or right, with sufficient care to get the direction right

\begin{equation}\label{eqn:gradeselectionProblems:440}
\Bx e^{i\theta} = e^{-i\theta} \Bx.
\end{equation}

It is also possible to split the rotation operation into half angle rotation operators that act from both the left and right

\begin{dmath}\label{eqn:gradeselectionProblems:460}
\Bx' = e^{-i\theta/2} \Bx e^{i\theta/2}.
\end{dmath}

A student who has studied computer graphics rotation theory may have seen quaterion rotation operators with this form, and a student of quantum mechanics will have seen Pauli matrix rotation operations of this form.  This is, in fact, the form that is generally desirable for 3D or higher order rotations, since it rotates the portions of a vector that lie in the rotation plane, leaving the normal components untouched.
} % answer

\makeanswer{problem:gradeselection:vectorwedge}{
Given \( \Bx = \sum_i x_i \Be_i \), and \( \By = \sum_i y_i \Be_i \), the wedge of these two vectors is a grade two selection that picks out only products that differ in index

\begin{dmath}\label{eqn:gradeselectionProblems:580}
\Bx \wedge \By
=
\gpgradetwo{ \Bx \By }
=
\sum_{i,j} \gpgradetwo{ x_i \Be_i y_j \Be_j }
=
\sum_{i \ne j} x_i y_j \gpgradetwo{ \Be_i \Be_j }
=
\sum_{i \ne j} x_i y_j \Be_i \Be_j
=
\sum_{i < j} (x_i y_j - x_j y_i) \Be_i \Be_j.
\end{dmath}
} % answer

\makeanswer{problem:gradeselection:WedgeRelationshipToCrossProduct}{
Writing out \cref{eqn:gradeselectionProblems:580} explicitly gives

\begin{dmath}\label{eqn:gradeselectionProblems:600}
\begin{aligned}
\Bx \wedge \By
&=
(x_1 y_2 - x_2 y_1) \Be_1 \Be_2 \\
&\quad+
(x_1 y_3 - x_3 y_1) \Be_1 \Be_3 \\
&\quad+
(x_2 y_3 - x_3 y_2) \Be_2 \Be_3 \\
&=
(x_1 y_2 - x_2 y_1) \Be_1 \Be_2 (\Be_3 \Be_3) \\
&\quad+
(x_1 y_3 - x_3 y_1) \Be_1 \Be_3 (\Be_2 \Be_2) \\
&\quad+
(x_2 y_3 - x_3 y_2) \Be_2 \Be_3 (\Be_1 \Be_1) \\
&=
(x_1 y_2 - x_2 y_1) I \Be_3 \\
&\quad+
(x_1 y_3 - x_3 y_1) (-I) \Be_2 \\
&\quad+
(x_2 y_3 - x_3 y_2) I \Be_1 \\
&= I (\Bx \cross \By).
\end{aligned}
\end{dmath}
} % answer
