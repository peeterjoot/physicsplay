%
% Copyright � 2016 Peeter Joot.  All Rights Reserved.
% Licenced as described in the file LICENSE under the root directory of this GIT repository.
%
%{
%\newcommand{\authorname}{Peeter Joot}
\newcommand{\email}{peeterjoot@protonmail.com}
\newcommand{\basename}{FIXMEbasenameUndefined}
\newcommand{\dirname}{notes/FIXMEdirnameUndefined/}

%\renewcommand{\basename}{vectorproduct}
%%\renewcommand{\dirname}{notes/phy1520/}
%\renewcommand{\dirname}{notes/ece1228-electromagnetic-theory/}
%%\newcommand{\dateintitle}{}
%%\newcommand{\keywords}{}
%
%\newcommand{\authorname}{Peeter Joot}
\newcommand{\onlineurl}{http://sites.google.com/site/peeterjoot2/math2013/\basename.pdf}
\newcommand{\sourcepath}{\dirname\basename.tex}
\newcommand{\generatetitle}[1]{\chapter{#1}}

\newcommand{\vcsinfo}{%
\section*{}
\noindent{\color{DarkOliveGreen}{\rule{\linewidth}{0.1mm}}}
\paragraph{Document version}
%\paragraph{\color{Maroon}{Document version}}
{
\small
\begin{itemize}
\item Available online at:\\ 
\href{\onlineurl}{\onlineurl}
\item Git Repository: \input{./.revinfo/gitRepo.tex}
\item Source: \sourcepath
\item last commit: \input{./.revinfo/gitCommitString.tex}
\item commit date: \input{./.revinfo/gitCommitDate.tex}
\end{itemize}
}
}

%\PassOptionsToPackage{dvipsnames,svgnames}{xcolor}
\PassOptionsToPackage{square,numbers}{natbib}
\documentclass{scrreprt}

\usepackage[left=2cm,right=2cm]{geometry}
\usepackage[svgnames]{xcolor}
\usepackage{peeters_layout}

\usepackage{natbib}

\usepackage[
colorlinks=true,
bookmarks=false,
pdfauthor={\authorname, \email},
backref 
]{hyperref}

% http://tex.stackexchange.com/questions/75773/how-to-reference-problems-by-the-text-label-in-an-exercise-envioronment
\usepackage[english]{cleveref}
\crefname{Exercise}{exercise}{exercises}
\Crefname{Exercise}{Exercise}{Exercises}

\RequirePackage{titlesec}
\RequirePackage{ifthen}

% http://stackoverflow.com/questions/4932910/date-in-the-tabular-environment
\makeatletter
\let\insertdate\@date
\makeatother

\titleformat{\chapter}[display]
{\bfseries\Large}
{\color{DarkSlateGrey}\filleft \authorname
\ifthenelse{\isundefined{\studentnumber}}{}{\\ \studentnumber}
\ifthenelse{\isundefined{\email}}{}{\\ \email}
\ifthenelse{\isundefined{\dateintitle}}{}{\\ \insertdate}
%\ifthenelse{\isundefined{\coursename}}{}{\\ \coursename} % put in title instead.
}
{4ex}
{\color{DarkOliveGreen}{\titlerule}\color{Maroon}
\vspace{2ex}%
\filright}
[\vspace{2ex}%
\color{DarkOliveGreen}\titlerule
]

\newcommand{\beginArtWithToc}[0]{\begin{document}\tableofcontents}
\newcommand{\beginArtNoToc}[0]{\begin{document}}
\newcommand{\EndNoBibArticle}[0]{\end{document}}
\newcommand{\EndArticle}[0]{\bibliography{Bibliography}\bibliographystyle{plainnat}\end{document}}

% 
%\newcommand{\citep}[1]{\cite{#1}}

\colorSectionsForArticle


%
%\usepackage{peeters_layout_exercise}
%\usepackage{peeters_braket}
%\usepackage{peeters_figures}
%\usepackage{siunitx}
%%\usepackage{mhchem} % \ce{}
%%\usepackage{macros_bm} % \bcM
%%\usepackage{macros_qed} % \qedmarker
%%\usepackage{txfonts} % \ointclockwise
%
%\beginArtNoToc
%
%\generatetitle{XXX}
%%\chapter{XXX}
%%\label{chap:vectorproduct}
%
Given two vectors \( \Bx, \By \) the scalar grade of the vector product \( \Bx \By \) can be shown (\cref{problem:gradeselection:RnDotProduct}) to be
\begin{equation}\label{eqn:vectorproduct:20}
\gpgradezero{ \Bx \By }
=
\sum_{i = 1}^N x_i y_i
=
\Bx \cdot \By 
\end{equation}

The grade two selection of this product was found (\cref{problem:gradeselection:WedgeRelationshipToCrossProduct}) to be

\begin{equation}\label{eqn:vectorproduct:40}
\gpgradetwo{ \Bx \By }
=
\sum_{i < j} (x_i y_j - x_j y_i) \Be_i \Be_j
=
\Bx \wedge \By.
\end{equation}

The reader should convince themself that the vector product \( \Bx \By \) has only even grades (0,2), and can therefore be expanded as

\begin{dmath}\label{eqn:vectorproduct:60}
\Bx \By
=
\gpgradezero{ \Bx \By }
+
\gpgradetwo{ \Bx \By },
\end{dmath}

or
\boxedEquation{eqn:vectorproduct:80}{
\Bx \By
=
\Bx \cdot \By 
+
\Bx \wedge \By.
}

This is a fundamental and very useful relationship.  In these notes this is a consequence of the axioms and the generalized definitions of the dot and wedge products.  Some authors will use this to define the geometric product of two vectors.

When considering the Euclidean space \R{3}, an additional relationship follows (\cref{problem:gradeselection:WedgeRelationshipToCrossProduct}), which is also incredibly useful

\boxedEquation{eqn:vectorproduct:100}{
\Bx \By
=
\Bx \cdot \By 
+
I
(\Bx \cross \By).
}

Note that this is the GA equivalent of the Pauli relationship \cref{eqn:GAmotivation:120} that will be familiar to a student of quantum spin states.
The ability to combine dot and cross product relationships into a single multivector equation is not just a theoretical nicety.  This happens to also be the reason that GA is so applicable to the study of electromagnetism.   To illustrate this, and provide a hint of things to come consider the GA formulation of the electrostatic and magnetostatic Maxwell equations.

\makeexample{Electrostatic and magnetostatics.}{example:vectorproduct:electrostatics}{

With no currents, no magnetic sources, no time derivatives, Maxwell's equations in simple media take the form

\begin{dmath}\label{eqn:vectorproduct:120}
\begin{aligned}
\spacegrad \cdot \BB &= 0 \\
\spacegrad \cross \BB &= 0 \\
\spacegrad \cross \BE &= 0 \\
\spacegrad \cdot \BE &= \frac{\rho}{\epsilon}.
\end{aligned}
\end{dmath}

Using \cref{eqn:vectorproduct:100} the first and last equations can be combined.  The magnetic gradient equation is 

\begin{equation}\label{eqn:vectorproduct:140}
\spacegrad \BB 
=
\spacegrad \cdot \BB +I (\spacegrad \cross \BB )
=
0,
\end{equation}

and the electric gradient equation is

\begin{equation}\label{eqn:vectorproduct:160}
\spacegrad \BE
=
\spacegrad \cdot \BE +I (\spacegrad \cross \BE )
=
\frac{\rho}{\epsilon}.
\end{equation}

Maxwell's equations are reduced to two multivector equations with this transformation
\begin{dmath}\label{eqn:vectorproduct:180}
\begin{aligned}
\spacegrad \BE &= \frac{\rho}{\epsilon} \\
\spacegrad \BB &= 0.
\end{aligned}
\end{dmath}

For magnetostatics, Maxwell's equations take the form

\begin{dmath}\label{eqn:vectorproduct:200}
\begin{aligned}
\spacegrad \cdot \BB &= 0 \\
\spacegrad \cross \BB &= \mu \BJ \\
\spacegrad \cross \BE &= 0 \\
\spacegrad \cdot \BE &= 0.
\end{aligned}
\end{dmath}

As above, these can be combined as

\begin{dmath}\label{eqn:vectorproduct:220}
\begin{aligned}
\spacegrad \BB &= I \mu \BJ \\
\spacegrad \BE &= 0.
\end{aligned}
\end{dmath}

It will be seen later that it is actually more natural to express magnetic fields as a bivector \( I \BB \).  Using \( I^2 = -1 \) (\cref{problem:gradeselectionProblems:R3PseudoscalarSquare}) after which the magnetostatic equation takes the form

\begin{dmath}\label{eqn:vectorproduct:240}
\spacegrad (I \BB) = - \mu \BJ.
\end{dmath}

Both the electrostatic and magnetostatic equations can be solved directly using the Green's function for the gradient, but additional concepts are required before attacking multivector integrals.
} % example

%}
%\EndNoBibArticle
