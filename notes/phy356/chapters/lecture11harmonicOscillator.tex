%
% Copyright © 2012 Peeter Joot.  All Rights Reserved.
% Licenced as described in the file LICENSE under the root directory of this GIT repository.
%
%\QMlecture{11 --- Harmonic oscillator. --- November 30, 2010}

\section{Setup}
Why study this problem?

It is relevant to describing the oscillation of molecules, quantum states of light, vibrations of the lattice structure of a solid, and so on.

FIXME: projected picture of masses on springs, with a ladle shaped well, approximately Harmonic about the minimum of the bucket.

The problem to solve is the one dimensional Hamiltonian

\begin{equation}\label{eqn:PHY356FLecture11:10}
\begin{aligned}
V(X) &= \inv{2} K X^2 \\
K &= m \omega^2 \\
H &= \frac{P^2}{2m} + V(X)
\end{aligned}
\end{equation}

where \(m\) is the mass, \(\omega\) is the frequency, \(X\) is the position operator, and \(P\) is the momentum operator.  Of these quantities, \(\omega\) and \(m\) are classical quantities.

This problem can be used to illustrate some of the reasons why we study the different pictures (Heisenberg, Interaction and Schr\"{o}dinger).  This is a problem well suited to all of these (FIXME: lookup an example of this with the interaction picture.  The book covers H and S methods.

We attack this with a non-intuitive, but cool technique.  Introduce the raising \(a^\dagger\) and lowering \(a\) operators:

\begin{equation}\label{eqn:PHY356FLecture11:20}
\begin{aligned}
a &= \sqrt{\frac{m \omega}{2 \Hbar}} \left( X + i \frac{P}{m\omega} \right) \\
a^\dagger &= \sqrt{\frac{m \omega}{2 \Hbar}} \left( X - i \frac{P}{m\omega} \right)
\end{aligned}
\end{equation}

\paragraph{Question:} are we using the dagger for more than Hermitian conjugation in this case.
\paragraph{Answer:} No, this is precisely the Hermitian conjugation operation.

Solving for \(X\) and \(P\) in terms of \(a\) and \(a^\dagger\), we have

\begin{equation}\label{eqn:lecture11harmonicOscillator:620}
\begin{aligned}
a + a^\dagger &= \sqrt{\frac{m \omega}{2 \Hbar}} 2 X  \\
a - a^\dagger &= \sqrt{\frac{m \omega}{2 \Hbar}} 2 i \frac{P }{m \omega}
\end{aligned}
\end{equation}

or
\begin{equation}\label{eqn:PHY356FLecture11:30}
\begin{aligned}
X &= \sqrt{\frac{\Hbar}{2 m \omega}} (a^\dagger + a) \\
P &= i \sqrt{\frac{\Hbar m \omega}{2}} (a^\dagger -a)
\end{aligned}
\end{equation}

Express \(H\) in terms of \(a\) and \(a^\dagger\)

\begin{equation}\label{eqn:lecture11harmonicOscillator:640}
\begin{aligned}
H
&= \frac{P^2}{2m} + \inv{2} K X^2  \\
&=
\frac{1}{2m} \left(
i \sqrt{\frac{\Hbar m \omega}{2}} (a^\dagger -a)
\right)^2
+ \inv{2} m \omega^2
\left(
\sqrt{\frac{\Hbar}{2 m \omega}} (a^\dagger + a)
\right)^2 \\
&=
\frac{-\Hbar \omega}{4} \left(
a^\dagger a^\dagger + a^2 - a a^\dagger - a^\dagger a
\right)
+ \frac{\Hbar \omega}{4}
\left(
a^\dagger a^\dagger + a^2 + a a^\dagger + a^\dagger a
\right) \\
\end{aligned}
\end{equation}

\begin{equation}\label{eqn:PHY356FLecture11:66}
H
=
\frac{\Hbar \omega}{2} \left(
a a^\dagger + a^\dagger a
\right)
=
\frac{\Hbar \omega}{2} \left(
2 a^\dagger a + \antisymmetric{a}{a^\dagger}
\right)
\end{equation}

Since \(\antisymmetric{X}{P} = i \Hbar \BOne\) then we can show that \(\antisymmetric{a}{a^\dagger} = \BOne\).  Solve for \(\antisymmetric{a}{a^\dagger}\) as follows

\begin{equation}\label{eqn:lecture11harmonicOscillator:660}
\begin{aligned}
i \Hbar
&=
\antisymmetric{X}{P} \\
&=
\antisymmetric{\sqrt{\frac{\Hbar}{2 m \omega}} (a^\dagger + a) }{i \sqrt{\frac{\Hbar m \omega}{2}} (a^\dagger -a)} \\
&=
\sqrt{\frac{\Hbar}{2 m \omega}} i \sqrt{\frac{\Hbar m \omega}{2}}
\antisymmetric{a^\dagger + a}{a^\dagger -a} \\
&= \frac{i \Hbar}{2}
\left(
\antisymmetric{a^\dagger}{a^\dagger}
-\antisymmetric{a^\dagger}{a}
+\antisymmetric{a}{a^\dagger}
-\antisymmetric{a}{a} \right)  \\
&= \frac{i \Hbar}{2}
\left(
0
+2 \antisymmetric{a}{a^\dagger}
-0
\right)
\end{aligned}
\end{equation}

Comparing LHS and RHS we have as stated

\begin{equation}\label{eqn:PHY356FLecture11:40}
\begin{aligned}
\antisymmetric{a}{a^\dagger} = \BOne
\end{aligned}
\end{equation}

and thus from \eqnref{eqn:PHY356FLecture11:66} we have

\begin{equation}\label{eqn:PHY356FLecture11:50}
H = \Hbar \omega \left( a^\dagger a + \frac{\BOne}{2} \right)
\end{equation}

Let \(\ket{n}\) be the eigenstate of \(H\) so that \(H\ket{n} = E_n \ket{n}\).  From \eqnref{eqn:PHY356FLecture11:50} we have

\begin{equation}\label{eqn:PHY356FLecture11:60}
H \ket{n}
=
\Hbar \omega \left( a^\dagger a + \frac{\BOne}{2} \right) \ket{n}
\end{equation}

or
\begin{equation}\label{eqn:PHY356FLecture11:70}
a^\dagger a \ket{n} + \frac{\ket{n}}{2} = \frac{E_n}{\Hbar \omega} \ket{n}
\end{equation}

\begin{equation}\label{eqn:PHY356FLecture11:80}
a^\dagger a \ket{n} = \left( \frac{E_n}{\Hbar \omega} - \inv{2} \right) \ket{n} = \lambda_n \ket{n}
\end{equation}

We wish now to find the eigenstates of the ``Number'' operator \(a^\dagger a\), which are simultaneously eigenstates of the Hamiltonian operator.

Observe that we have

\begin{equation}\label{eqn:lecture11harmonicOscillator:680}
\begin{aligned}
a^\dagger a (a^\dagger \ket{n} )
&= a^\dagger ( a a^\dagger \ket{n} ) \\
&= a^\dagger ( \BOne + a^\dagger a ) \ket{n}
\end{aligned}
\end{equation}

where we used \(\antisymmetric{a}{a^\dagger} = a a^\dagger - a^\dagger a = \BOne\).

\begin{equation}\label{eqn:lecture11harmonicOscillator:700}
\begin{aligned}
a^\dagger a (a^\dagger \ket{n} )
&= a^\dagger \left( \BOne + \frac{E_n}{\Hbar\omega} - \frac{\BOne}{2} \right) \ket{n} \\
&= a^\dagger \left( \frac{E_n}{\Hbar\omega} + \frac{\BOne}{2} \right) \ket{n},
\end{aligned}
\end{equation}

or
\begin{equation}\label{eqn:PHY356FLecture11:100}
a^\dagger a (a^\dagger \ket{n} ) = (\lambda_n + 1) (a^\dagger \ket{n} )
\end{equation}

The new state \(a^\dagger \ket{n}\) is presumed to lie in the same space, expressible as a linear combination of the basis states in this space.  We can see the effect of the operator \(a a^\dagger\) on this new state, we find that the energy is changed, but the state is otherwise unchanged.  Any state \(a^\dagger \ket{n}\) is an eigenstate of \(a^\dagger a\), and therefore also an eigenstate of the Hamiltonian.

Play the same game and win big by discovering that

\begin{equation}\label{eqn:PHY356FLecture11:110}
a^\dagger a ( a \ket{n} ) = (\lambda_n -1) (a \ket{n} )
\end{equation}

There will be some state \(\ket{0}\) such that

\begin{equation}\label{eqn:PHY356FLecture11:120}
a \ket{0} = 0 \ket{0}
\end{equation}

which implies
\begin{equation}\label{eqn:PHY356FLecture11:130}
a^\dagger (a \ket{0}) = (a^\dagger a) \ket{0} = 0
\end{equation}

so from \eqnref{eqn:PHY356FLecture11:80} we have

\begin{equation}\label{eqn:PHY356FLecture11:140}
\lambda_0 = 0
\end{equation}

Observe that we can identify \(\lambda_n = n\) for

\begin{equation}\label{eqn:PHY356FLecture11:150}
\lambda_n = \left( \frac{E_n}{\Hbar\omega} - \inv{2} \right) = n,
\end{equation}

or
\begin{equation}\label{eqn:PHY356FLecture11:155}
\frac{E_n}{\Hbar\omega} = n + \inv{2}
\end{equation}

or
\begin{equation}\label{eqn:PHY356FLecture11:160}
E_n = \Hbar \omega \left( n + \inv{2} \right)
\end{equation}

where \(n = 0, 1, 2, \cdots\).

We can write

\begin{equation}\label{eqn:lecture11harmonicOscillator:720}
\begin{aligned}
\Hbar \omega \left( a^\dagger a + \inv{2} \BOne \right) \ket{n} &= E_n \ket{n} \\
a^\dagger a \ket{n} + \inv{2} \ket{n} &= \frac{E_n}{\Hbar \omega} \ket{n} \\
\end{aligned}
\end{equation}

or
\begin{equation}\label{eqn:PHY356FLecture11:170}
a^\dagger a \ket{n} = \left( \frac{E_n}{\Hbar \omega} - \inv{2} \right) \ket{n} = \lambda_n \ket{n} = n \ket{n}
\end{equation}

We call this operator \(a^\dagger a = N\), the number operator, so that

\begin{equation}\label{eqn:PHY356FLecture11:180}
N \ket{n} = n \ket{n}
\end{equation}

\section{Relating states}

Recall the calculation we performed for

\begin{equation}\label{eqn:PHY356FLecture11:200}
\begin{aligned}
L_{+} \ket{lm} &= C_{+} \ket{l, m+1} \\
L_{-} \ket{lm} &= C_{+} \ket{l, m-1}
\end{aligned}
\end{equation}

Where \(C_{+}\), and \(C_{+}\) are constants.  The next game we are going to play is to work out \(C_n\) for the lowering operation

\begin{equation}\label{eqn:PHY356FLecture11:210}
a\ket{n} = C_n \ket{n-1}
\end{equation}

and the raising operation
\begin{equation}\label{eqn:PHY356FLecture11:211}
a^\dagger \ket{n} = B_n \ket{n+1}.
\end{equation}

For the Hermitian conjugate of \(a \ket{n}\) we have

\begin{equation}\label{eqn:PHY356FLecture11:220}
(a \ket{n})^\dagger = ( C_n \ket{n-1} )^\dagger = C_n^\conj \ket{n-1}
\end{equation}

So
\begin{equation}\label{eqn:PHY356FLecture11:230}
(\bra{n} a^\dagger) (a \ket{n}) = C_n C_n^\conj \braket{n-1}{n-1} = \Abs{C_n}^2
\end{equation}

Expanding the LHS we have
\begin{equation}\label{eqn:lecture11harmonicOscillator:740}
\begin{aligned}
\Abs{C_n}^2 &=
\bra{n} a^\dagger a \ket{n} \\
&=
\bra{n} n \ket{n} \\
&=
n \braket{n}{n} \\
&=
n
\end{aligned}
\end{equation}

For
\begin{equation}\label{eqn:PHY356FLecture11:240}
C_n = \sqrt{n}
\end{equation}

Similarly
\begin{equation}\label{eqn:PHY356FLecture11:250}
(\bra{n} a^\dagger) (a \ket{n}) = B_n B_n^\conj \braket{n+1}{n+1} = \Abs{B_n}^2
\end{equation}

and
\begin{equation}\label{eqn:lecture11harmonicOscillator:760}
\begin{aligned}
\Abs{B_n}^2 &=
\bra{n} 
\mathLabelBox{a a^\dagger}{\(a a^\dagger - a^\dagger a = \BOne\)}
\ket{n} \\
&=
\bra{n} \left( \BOne + a^\dagger a \right) \ket{n} \\
&=
(1 + n) \braket{n}{n} \\
&=
1 + n
\end{aligned}
\end{equation}

for
\begin{equation}\label{eqn:PHY356FLecture11:260}
B_n = \sqrt{n + 1}
\end{equation}

\section{Heisenberg picture}

\paragraph{How does the lowering operator \(a\) evolve in time?}

\paragraph{A:} Recall that for a general operator \(A\), we have for the time evolution of that operator

\begin{equation}\label{eqn:PHY356FLecture11:270}
i \Hbar \frac{d A}{dt} = \antisymmetric{ A }{H}
\end{equation}

Let us solve this one.

\begin{equation}\label{eqn:lecture11harmonicOscillator:780}
\begin{aligned}
i \Hbar \frac{d a}{dt}
&= \antisymmetric{ a }{H} \\
&= \antisymmetric{ a }{ \Hbar \omega (a^\dagger a + \BOne/2) } \\
&= \Hbar\omega \antisymmetric{ a }{ (a^\dagger a + \BOne/2) } \\
&= \Hbar\omega \antisymmetric{ a }{ a^\dagger a } \\
&= \Hbar\omega \left( a a^\dagger a - a^\dagger a a \right) \\
&= \Hbar\omega \left( (a a^\dagger) a - a^\dagger a a \right) \\
&= \Hbar\omega \left( (a^\dagger a + \BOne) a - a^\dagger a a \right) \\
&= \Hbar\omega a
\end{aligned}
\end{equation}

Even though \(a\) is an operator, it can undergo a time evolution and we can think of it as a function, and we can solve for \(a\) in the differential equation

\begin{equation}\label{eqn:PHY356FLecture11:280}
\frac{d a}{dt} = -i \omega a
\end{equation}

This has the solution
\begin{equation}\label{eqn:PHY356FLecture11:290}
a = a(0) e^{-i \omega t}
\end{equation}

here \(a(0)\) is an operator, the value of that operator at \(t = 0\).  The exponential here is just a scalar (not effected by the operator so we can put it on either side of the operator as desired).

\paragraph{CHECK:}
\begin{equation}\label{eqn:PHY356FLecture11:291}
a' = a(0) \frac{d}{dt} e^{-i \omega t} = a(0) (-i \omega) e^{-i \omega t} = -i \omega a
\end{equation}

\section{A couple comments on the Schr\"{o}dinger picture}

We do not do this in class, but it is very similar to the approach of the hydrogen atom.  See the text for full details.

In the Schr\"{o}dinger picture,
\begin{equation}\label{eqn:PHY356FLecture11:400}
-\frac{\Hbar^2}{2m} \frac{d^2 u}{dx^2} + \inv{2} m \omega^2 x^2 u = E u
\end{equation}

This does directly to the wave function representation, but we can relate these by noting that we get this as a consequence of the identification \(u = u(x) = \braket{x}{u}\).

In \eqnref{eqn:PHY356FLecture11:400}, we can switch to dimensionless quantities with
\begin{equation}\label{eqn:PHY356FLecture11:410}
\xi = \text{``xi (z)''} = \alpha x
\end{equation}

with
\begin{equation}\label{eqn:PHY356FLecture11:411}
\alpha = \sqrt{\frac{m \omega}{\Hbar}}
\end{equation}

This gives, with \(\lambda = 2E/\Hbar\omega\),

\begin{equation}\label{eqn:PHY356FLecture11:420}
\frac{d^2 u}{d\xi^2} + (\lambda - \xi^2) u = 0
\end{equation}

We can use polynomial series expansion methods to solve this, and find that we require a terminating expression, and write this in terms of the Hermite polynomials (courtesy of the clever French once again).

When all is said and done we will get the energy eigenvalues once again
\begin{equation}\label{eqn:PHY356FLecture11:430}
E = E_n = \Hbar \omega \left( n + \inv{2} \right)
\end{equation}

\section{Back to the Heisenberg picture}

Let us express
\begin{equation}\label{eqn:PHY356FLecture11:500}
\braket{x}{n} = u_n(x)
\end{equation}

With
\begin{equation}\label{eqn:PHY356FLecture11:510}
a \ket{0} = 0,
\end{equation}

we have
\begin{equation}\label{eqn:PHY356FLecture11:520}
0
=
\left( X + i \frac{P}{m \omega} \right) \ket{0},
\end{equation}

and
\begin{equation}\label{eqn:lecture11harmonicOscillator:800}
\begin{aligned}
0
&=
\bra{x} \left( X + i \frac{P}{m \omega} \right) \ket{0} \\
&=
\bra{x} X \ket{0 } + i \frac{1}{m \omega} \bra{x} P \ket{0} \\
&=
x \braket{x}{0} + i \frac{1}{m \omega} \bra{x} P \ket{0} \\
\end{aligned}
\end{equation}

Recall that our matrix operator is
\begin{equation}\label{eqn:PHY356FLecture11:540}
\bra{x'} P \ket{x} = \delta(x - x') \left( -i \Hbar \frac{d}{dx} \right)
\end{equation}

\begin{equation}\label{eqn:lecture11harmonicOscillator:820}
\begin{aligned}
\bra{x} P \ket{0}
&=
\bra{x} P 
\mathLabelBox{\int \ket{x'} \bra{x'} dx' }{\(= \BOne\)}
\ket{0} \\
&=
\int \bra{x} P \ket{x'} \braket{x'}{0} dx' \\
&=
\int
\delta(x - x') \left( -i \Hbar \frac{d}{dx} \right)
\braket{x'}{0} dx' \\
&=
\left( -i \Hbar \frac{d}{dx} \right)
\braket{x}{0}
\end{aligned}
\end{equation}

We have then

\begin{equation}\label{eqn:PHY356FLecture11:600}
0 =
x u_0(x) + \frac{\Hbar}{m \omega} \frac{d u_0(x)}{dx}
\end{equation}

NOTE: picture of the solution to this LDE on slide.... but I did not look closely enough.

