%
% Copyright © 2012 Peeter Joot.  All Rights Reserved.
% Licenced as described in the file LICENSE under the root directory of this GIT repository.
%
HOMEWORK: go through the steps to understand how to formulate $\spacegrad^2$ in spherical polar coordinates.  This is a lot of work, but is good practice and background for dealing with the Hydrogen atom, something with spherical symmetry that is most naturally analyzed in the spherical polar coordinates.

In spherical coordinates (We will not go through this here, but it is good practice) with

\begin{align*}
x &= r \sin\theta \cos\phi \\
y &= r \sin\theta \sin\phi \\
z &= r \cos\theta
\end{align*}

we have with $u = u(r,\theta, \phi)$

\begin{align*}
-\frac{\Hbar^2}{2m} \left(
\inv{r} \partial_{rr} (r u) +  \inv{r^2 \sin\theta} \partial_\theta (\sin\theta \partial_\theta u)
+ \inv{r^2 \sin^2\theta} \partial_{\phi\phi} u
 \right)
&= E u
\end{align*}

We see the start of a separation of variables attack with $u = R(r) Y(\theta, \phi)$.  We end up with

\begin{align*}
-\frac{\Hbar^2}{2m} &\left(
\frac{r}{R} (r R')' +  \inv{Y \sin\theta} \partial_\theta (\sin\theta \partial_\theta Y)
+ \inv{Y \sin^2\theta} \partial_{\phi\phi} Y
 \right) \\
\end{align*}

\begin{align*}
r (r R')' + \left( \frac{2m E}{\Hbar^2} r^2 - \lambda \right) R &= 0
\end{align*}
\begin{align*}
\inv{Y \sin\theta} \partial_\theta (\sin\theta \partial_\theta Y) + \inv{Y \sin^2\theta} \partial_{\phi\phi} Y &= -\lambda
\end{align*}

Application of separation of variables again, with $Y = P(\theta) Q(\phi)$ gives us

\begin{align*}
\inv{P \sin\theta} \partial_\theta (\sin\theta \partial_\theta P) + \inv{Q \sin^2\theta} \partial_{\phi\phi} Q &= -\lambda
\end{align*}

\begin{align*}
\frac{\sin\theta}{P } \partial_\theta (\sin\theta \partial_\theta P)
+\lambda  \sin^2\theta
+ \inv{Q } \partial_{\phi\phi} Q &= 0
\end{align*}

\begin{align*}
\frac{\sin\theta}{P } \partial_\theta (\sin\theta \partial_\theta P) + \lambda \sin^2\theta - \mu = 0
\inv{Q } \partial_{\phi\phi} Q &= -\mu
\end{align*}

or
\begin{align}\label{eqn:PHY356F:1000}
\frac{1}{P \sin\theta} \partial_\theta (\sin\theta \partial_\theta P) +\lambda -\frac{\mu}{\sin^2\theta} &= 0
\end{align}
\begin{align}\label{eqn:PHY356F:2000}
\partial_{\phi\phi} Q &= -\mu Q
\end{align}

The equation for $P$ can be solved using the Legendre function $P_l^m(\cos\theta)$ where $\lambda = l(l+1)$ and $l$ is an integer

Replacing $\mu$ with $m^2$, where $m$ is an integer

\begin{align*}
\frac{d^2 Q}{d\phi^2} &= -m^2 Q
\end{align*}

Imposing a periodic boundary condition $Q(\phi) = Q(\phi + 2\pi)$, where ($m = 0, \pm 1, \pm 2, \cdots$) we have

\begin{align*}
Q &= \inv{\sqrt{2\pi}} e^{im\phi}
\end{align*}

There is the overall solution $r(r,\theta,\phi) = R(r) Y(\theta, \phi)$ for a free particle.  The functions $Y(\theta, \phi)$ are

\begin{align*}
Y_{lm}(\theta, \phi)
&= N \left( \inv{\sqrt{2\pi}} e^{im\phi} \right) 
\mathLabelBox{ P_l^m(\cos\theta) }{$ -l \le m \le l$}
\end{align*}

where $N$ is a normalization constant, and $m = 0, \pm 1, \pm 2, \cdots$.  $Y_{lm}$ is an eigenstate of the $\BL^2$ operator and $L_z$ (two for the price of one).  There is no specific reason for the direction $z$, but it is the direction picked out of convention.

Angular momentum is given by

\begin{align*}
\BL = \Br \cross \Bp
\end{align*}

where

\begin{align*}
\BR = x \xcap + y\ycap + z\zcap
\end{align*}

and
\begin{align*}
\Bp = p_x \xcap + p_y\ycap + p_z\zcap
\end{align*}

The important thing to remember is that the aim of following all the math is to show that

\begin{align*}
\BL^2 Y_{lm} = \Hbar^2 l (l+1) Y_{lm}
\end{align*}

and simultaneously

\begin{align*}
\BL_z Y_{lm} = \Hbar m Y_{lm}
\end{align*}

Part of the solution involves working with $\antisymmetric{L_z}{L_{+}}$, and $\antisymmetric{L_z}{L_{-}}$, where

\begin{align*}
L_{+} &= L_x + i L_y \\
L_{-} &= L_x - i L_y
\end{align*}

An exercise (not in the book) is to evaluate
\begin{align}\label{eqn:PHY356F:4000}
\antisymmetric{L_z}{L_{+}}
&= L_z L_x + i L_z L_y - L_x L_z - i L_y L_z
\end{align}

where
\begin{align}\label{eqn:PHY356F:5000}
\antisymmetric{L_x}{L_y}  &= i \Hbar L_z \\
\antisymmetric{L_y}{L_z}  &= i \Hbar L_x \\
\antisymmetric{L_z}{L_x}  &= i \Hbar L_y
\end{align}

Substitution back in \eqnref{eqn:PHY356F:4000} we have

\begin{align*}
\antisymmetric{L_z}{L_{+}}
&=
\antisymmetric{L_z}{L_x}
+ i \antisymmetric{L_z}{L_y}  \\
&=
i \Hbar ( L_y - i L_x ) \\
&=
\Hbar ( i L_y +  L_x ) \\
&=
\Hbar L_{+}
\end{align*}
