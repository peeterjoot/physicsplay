%
% Copyright � 2012 Peeter Joot.  All Rights Reserved.
% Licenced as described in the file LICENSE under the root directory of this GIT repository.
%

\chapter{PHY356 Problem Set III}
\label{chap:qmIproblemSet3}
%\blogpage{http://sites.google.com/site/peeterjoot/math2010/qmIproblemSet3.pdf}
%\date{Oct 23, 2010}

\section{Problem 1}
\subsection{Statement}

A particle of mass $m$ is free to move along the x-direction such that $V(X)=0$. The state of the system is represented by the wavefunction Eq. (4.74)

\begin{align}\label{eqn:qmIproblemSet3:1}
\psi(x,t) = \inv{\sqrt{2\pi}} \int_{-\infty}^\infty dk e^{i k x} e^{- i \omega t} f(k)
\end{align}

with $f(k)$ given by Eq. (4.59).

\begin{align}\label{eqn:qmIproblemSet3:2}
f(k) &= N e^{-\alpha k^2}
\end{align}

Note that I have inserted a $1/\sqrt{2\pi}$ factor above that is not in the text, because otherwise $\psi(x,t)$ will not be unit normalized (assuming $f(k)$ is normalized in wavenumber space).

\begin{itemize}
\item 
(a) What is the group velocity associated with this state? 
\item 
(b) What is the probability for measuring the particle at position $x=x_0>0$ at time $t=t_0>0$? 
\item 
(c) What is the probability per unit length for measuring the particle at position $x=x_0>0$ at time $t=t_0>0$? 
\item 
(d) Explain the physical meaning of the above results.
\end{itemize}

\subsection{Solution}
\subsubsection{(a).  group velocity}

To calculate the group velocity we need to know the dependence of $\omega$ on $k$.

Let us step back and consider the time evolution action on $\psi(x,0)$.  For the free particle case we have

\begin{align}\label{eqn:qmIproblemSet3:101}
H = \frac{\Bp^2}{2m} = -\frac{\hbar^2}{2m} \partial_{xx}.
\end{align}

Writing $N' = N/\sqrt{2\pi}$ we have

\begin{align*}
-\frac{i t}{\hbar} H \psi(x,0) 
&= 
\frac{i t \hbar }{2m} 
N' \int_{-\infty}^\infty dk (i k)^2 e^{i k x - \alpha k^2} \\
&= 
N' \int_{-\infty}^\infty dk \frac{-i t \hbar k^2}{2m} e^{i k x - \alpha k^2}
\end{align*}

Each successive application of $-iHt/\hbar$ will introduce another power of $-it\hbar k^2/2 m$, so once we sum all the terms of the exponential series $U(t) = e^{-iHt/\hbar}$ we have

\begin{align}\label{eqn:qmIproblemSet3:102}
\psi(x,t) =
N' \int_{-\infty}^\infty dk \exp\left( 
\frac{-i t \hbar k^2}{2m} + i k x - \alpha k^2 \right).
\end{align}

Comparing with \ref{eqn:qmIproblemSet3:1} we find
\begin{align}\label{eqn:qmIproblemSet3:103}
\omega(k) = \frac{\hbar k^2}{2m}.
\end{align}

This completes this section of the problem since we are now able to calculate the group velocity 
\begin{align}\label{eqn:qmIproblemSet3:104}
v_g = \PD{k}{\omega(k)} = \frac{\hbar k}{m}.
\end{align}

\subsection{(b). What is the probability for measuring the particle at position $x=x_0>0$ at time $t=t_0>0$?}

In order to evaluate the probability, it looks desirable to evaluate the wave function integral \ref{eqn:qmIproblemSet3:102}.  
Writing $2 \beta = i/(\alpha + i t \hbar/2m )$, the exponent of that integral is

\begin{align*}
-k^2 \left( \alpha + \frac{i t \hbar }{2m} \right) + i k x
&=
-\left( \alpha + \frac{i t \hbar }{2m} \right) \left( k^2 - \frac{i k x }{\alpha + \frac{i t \hbar }{2m} } \right) \\
&=
-\frac{i}{2\beta} \left( (k - x \beta )^2 - x^2 \beta^2 \right)
\end{align*}

The $x^2$ portion of the exponential

\begin{align*}
\frac{i x^2 \beta^2}{2\beta} = \frac{i x^2 \beta}{2} = - \frac{x^2 }{4 (\alpha + i t \hbar /2m)}
\end{align*}

then comes out of the integral.  We can also make a change of variables $q = k - x \beta$ to evaluate the remainder of the Gaussian and are left with

\begin{align}\label{eqn:qmIproblemSet3:107}
\psi(x,t) =
N' \sqrt{ \frac{\pi}{\alpha + i t \hbar/2m} } \exp\left( - \frac{x^2 }{4 (\alpha + i t \hbar /2m)} \right).
\end{align}

Observe that from \ref{eqn:qmIproblemSet3:2} we can compute $N = (2 \alpha/\pi)^{1/4}$, which could be substituted back into \ref{eqn:qmIproblemSet3:107} if desired.

Our probability density is 

\begin{align*}
\Abs{ \psi(x,t) }^2 
&=
\inv{2 \pi} N^2 \Abs{ \frac{\pi}{\alpha + i t \hbar/2m} } \exp\left( - \frac{x^2}{4} \left( 
\inv{(\alpha + i t \hbar /2m)} + \inv{(\alpha - i t \hbar /2m)} 
\right) \right) \\
&=
\inv{2 \pi} \sqrt{\frac{2 \alpha}{\pi} } \frac{\pi}{\sqrt{\alpha^2 + (t \hbar/2m)^2 }} \exp\left( - \frac{x^2}{4} 
\inv{\alpha^2 + (t \hbar/2m)^2 } \left( 
\alpha - i t \hbar /2m + \alpha + i t \hbar /2m 
\right)
\right) \\
&=
\end{align*}

With a final regrouping of terms, this is

\begin{align}\label{eqn:qmIproblemSet3:110}
\Abs{ \psi(x,t) }^2 =
\sqrt{\frac{ \alpha }{ 2 \pi (\alpha^2 + (t \hbar/2m)^2 }) }
\exp\left( - \frac{x^2}{2} 
\frac{\alpha}{\alpha^2 + (t \hbar/2m)^2 } 
\right).
\end{align}

As a sanity check we observe that this integrates to unity for all $t$ as desired.  The probability that we find the particle at position $x > x_0$ is then

\begin{align}\label{eqn:qmIproblemSet3:111}
P_{x>x_0}(t) = \sqrt{\frac{ \alpha }{ 2 \pi (\alpha^2 + (t \hbar/2m)^2 }) }
\int_{x=x_0}^\infty dx \exp\left( - \frac{x^2}{2} 
\frac{\alpha}{\alpha^2 + (t \hbar/2m)^2 } 
\right)
\end{align}

The only simplification we can make is to rewrite this in terms of the complementary error function

\begin{align}\label{eqn:qmIproblemSet3:112}
\erfc(x) = \frac{2}{\sqrt{\pi}} \int_x^\infty e^{-t^2} dt.
\end{align}

Writing

\begin{align}\label{eqn:qmIproblemSet3:113}
\beta(t) = \frac{\alpha}{\alpha^2 + (t \hbar/2m)^2 },
\end{align}

we have
\begin{align}\label{eqn:qmIproblemSet3:114}
P_{x>x_0}(t_0) = \inv{2} \erfc \left( \sqrt{\beta(t_0)/2} x_0 \right)
\end{align}

Sanity checking this result, we note that since $\erfc(0) = 1$ the probability for finding the particle in the $x>0$ range is $1/2$ as expected.


\subsection{(c). What is the probability per unit length for measuring the particle at position $x=x_0>0$ at time $t=t_0>0$?}

This unit length probability is thus

\begin{align}\label{eqn:qmIproblemSet3:115}
P_{x>x_0+1/2}(t_0) - P_{x>x_0-1/2}(t_0) 
&=
\inv{2} \erfc\left( \sqrt{\frac{\beta(t_0)}{2}} \left(x_0+\inv{2} \right) \right) 
-\inv{2} \erfc\left( \sqrt{\frac{\beta(t_0)}{2}} \left(x_0-\inv{2} \right) \right) 
\end{align}

\subsection{(d). Explain the physical meaning of the above results}

To get an idea what the group velocity means, observe that we can write our wavefunction \ref{eqn:qmIproblemSet3:1} as

\begin{align}\label{eqn:qmIproblemSet3:140}
\psi(x,t) = \inv{\sqrt{2\pi}} \int_{-\infty}^\infty dk e^{i k (x - v_g t)} f(k)
\end{align}

We see that the phase coefficient of the Gaussian $f(k)$ ``moves'' at the rate of the group velocity $v_g$.  Also recall that in the text it is noted that the time dependent term \ref{eqn:qmIproblemSet3:113} can be expressed in terms of position and momentum uncertainties $(\Delta x)^2$, and $(\Delta p)^2 = \hbar^2 (\Delta k)^2$.  That is

\begin{align}\label{eqn:qmIproblemSet3:116}
\inv{\beta(t)} = (\Delta x)^2 + \frac{(\Delta p)^2}{m^2} t^2 \equiv (\Delta x(t))^2
\end{align}

This makes it evident that the probability density flattens and spreads over time with the rate equal to the uncertainty of the group velocity $\Delta p/m = \Delta v_g$ (since $v_g = \hbar k/m$).  It is interesting that something as simple as this phase change results in a physically measurable phenomena.  We see that a direct result of this linear with time phase change, we are less able to find the particle localized around it is original time $x = 0$ position as more time elapses.

\section{Problem 2}

\subsection{Statement}
A particle with intrinsic angular momentum or spin $s=1/2$ is prepared in the spin-up with respect to the z-direction state $\ket{f}=\ket{z+}$. Determine

\begin{align}\label{eqn:qmIproblemSet3:3}
\left(\bra{f} \left( S_z - \bra{f} S_z \ket{f} \BOne \right)^2 \ket{f} \right)^{1/2}
\end{align}

and 

\begin{align}\label{eqn:qmIproblemSet3:4}
\left(\bra{f} \left( S_x - \bra{f} S_x \ket{f} \BOne \right)^2 \ket{f} \right)^{1/2}
\end{align}

and explain what these relations say about the system.

\subsection{Solution:  Uncertainty of $S_z$ with respect to $\ket{z+}$}

%%%%To start, we note that we have the matrix representations
%%%%
%%%%\begin{align}\label{eqn:qmIproblemSet3:5}
%%%%S_z &= 
%%%%\frac{\hbar}{2}
%%%%\PauliZ \\
%%%%\ket{f} = \ket{z+} &= 
%%%%\begin{bmatrix}
%%%%1 \\
%%%%0
%%%%\end{bmatrix}.
%%%%\end{align}
%%%%
%%%%In the matrix representation, the expectation values are straightforward to calculate.  For the expectation of $S_z$ with respect to this state we have
%%%%
%%%%\begin{align*}
%%%%\bra{f} S_z \ket{f} 
%%%%&=
%%%%\frac{\hbar}{2}
%%%%\begin{bmatrix}
%%%%1 & 0
%%%%\end{bmatrix}
%%%%\PauliZ
%%%%\begin{bmatrix}
%%%%1 \\
%%%%0
%%%%\end{bmatrix} \\
%%%%&=
%%%%\frac{\hbar}{2}
%%%%\begin{bmatrix}
%%%%1 & 0
%%%%\end{bmatrix}
%%%%\begin{bmatrix}
%%%%1 \\
%%%%0
%%%%\end{bmatrix} \\
%%%%&=
%%%%\frac{\hbar}{2}
%%%%\end{align*}
%%%%
%%%% SNIP.
%%%%
%%%%We can next compute $S_z - \bra{f} S_z \ket{f} \BOne$
%%%%
%%%%\begin{align*}
%%%%S_z - \bra{f} S_z \ket{f} \BOne
%%%%&=
%%%%\frac{\hbar}{2} \PauliZ - \frac{\hbar}{2} 
%%%%\begin{bmatrix}
%%%%1 & 0 \\
%%%%0 & 1
%%%%\end{bmatrix}
%%%%&=
%%%%\hbar
%%%%\begin{bmatrix}
%%%%0 & 0 \\
%%%%0 & 1 
%%%%\end{bmatrix}.
%%%%\end{align*}
%%%%
%%%%The matrix factor is a projector, squaring to itself, so we have
%%%%
%%%%\begin{align}\label{eqn:qmIproblemSet3:6}
%%%%\bra{f} \left( S_z - \bra{f} S_z \ket{f} \BOne \right)^2 \ket{f} 
%%%%&=
%%%%\hbar^2 
%%%%\begin{bmatrix}
%%%%1 & 0
%%%%\end{bmatrix}
%%%%\begin{bmatrix}
%%%%0 & 0 \\
%%%%0 & 1
%%%%\end{bmatrix}
%%%%\begin{bmatrix}
%%%%1 \\
%%%%0
%%%%\end{bmatrix}
%%%%= 0
%%%%\end{align}
%%%%XX

Noting that $S_z \ket{f} = S_z \ket{z+} = \hbar/2 \ket{z+}$ we have

\begin{align}\label{eqn:qmIproblemSet3:10}
\bra{f} S_z \ket{f} = \frac{\hbar}{2} 
\end{align}

The average outcome for many measurements of the physical quantity associated with the operator $S_z$ when the system has been prepared in the state $\ket{f} = \ket{z+}$ is $\hbar/2$.

\begin{align}\label{eqn:qmIproblemSet3:11}
\Bigl(S_z - \bra{f} S_z \ket{f} \BOne \Bigr) \ket{f}
&= 
\frac{\hbar}{2} \ket{f} 
-\frac{\hbar}{2} \ket{f} = 0
\end{align}

We could also compute this from the matrix representations, but it is slightly more work.

Operating once more with $S_z - \bra{f} S_z \ket{f} \BOne$ on the zero ket vector still gives us zero, so we have zero in the root for \ref{eqn:qmIproblemSet3:3}
\begin{align}\label{eqn:qmIproblemSet3:3b}
\left(\bra{f} \left( S_z - \bra{f} S_z \ket{f} \BOne \right)^2 \ket{f} \right)^{1/2} = 0
\end{align}


%In the variance calculation above we have the operator $D = S_z - \bra{f} S_z \ket{S_z} \BOne$, and its square.  Both $D$ and $D^2$ commute with $S_z$, and thus have the same eigenstates as $S_z$.  In particular when the system is prepared in the state $\ket{z+}$, no measurement of the physical quantity We have a physical quantity associated with the operator $D$
%Each of the operators involved in the variance calculation above 

%Observe that we are looking at the expectation value of a new operator, say, $V = (S_z - \bra{f} S_z \ket{f})^2$.  Also observe that $V$ commutes with $S_z$, and thus shares the same eigenstates.  A measurement 

What does \ref{eqn:qmIproblemSet3:3b} say about the state of the system?  Given many measurements of the physical quantity associated with the operator $V = (S_z - \bra{f} S_z \ket{f} \BOne)^2$, where the initial state of the system is always $\ket{f} = \ket{z+}$, then the average of the measurements of the physical quantity associated with $V$ is zero.  We can think of the operator $V^{1/2} = S_z - \bra{f} S_z \ket{f} \BOne$ as a representation of the observable, ``how different is the measured result from the average $\bra{f} S_z \ket{f}$''.  

So, given a system prepared in state $\ket{f} = \ket{z+}$, and performance of repeated measurements capable of only examining spin-up, we find that the system is never any different than its initial spin-up state.  We have no uncertainty that we will measure any difference from spin-up on average, when the system is prepared in the spin-up state.

\subsection{Solution:  Uncertainty of $S_x$ with respect to $\ket{z+}$}

For this second part of the problem, we note that we can write

\begin{align}\label{eqn:qmIproblemSet3:20}
\ket{f} = \ket{z+} = \inv{\sqrt{2}} ( \ket{x+} + \ket{x-} ).
\end{align}

So the expectation value of $S_x$ with respect to this state is
\begin{align*}
\bra{f} S_x \ket{f}
&=
\inv{2}
( \ket{x+} + \ket{x-} ) S_x ( \ket{x+} + \ket{x-} ) \\
&=
\hbar 
( \ket{x+} + \ket{x-} ) ( \ket{x+} - \ket{x-} ) \\
&=
\hbar 
( 1 + 0 + 0 -1 ) \\
&= 0
\end{align*}

After repeated preparation of the system in state $\ket{f}$, the average measurement of the physical quantity associated with operator $S_x$ is zero.  In terms of the eigenstates for that operator $\ket{x+}$ and $\ket{x-}$ we have equal probability of measuring either given this particular initial system state.

For the variance calculation, this reduces our problem to the calculation of $\bra{f} S_x^2 \ket{f}$, which is

\begin{align*}
\bra{f} S_x^2 \ket{f} 
&=
\inv{2} \left( \frac{\hbar}{2} \right)^2 
( \ket{x+} + \ket{x-} ) ( (+1)^2 \ket{x+} + (-1)^2 \ket{x-} ) \\
&=
\left( \frac{\hbar}{2} \right)^2,
\end{align*}

so for \ref{eqn:qmIproblemSet3:4b} we have

\begin{align}\label{eqn:qmIproblemSet3:4b}
\left(\bra{f} \left( S_x - \bra{f} S_x \ket{f} \BOne \right)^2 \ket{f} \right)^{1/2} = \frac{\hbar}{2}
\end{align}

The average of the absolute magnitude of the physical quantity associated with operator $S_x$ is found to be $\hbar/2$ when repeated measurements are performed given a system initially prepared in state $\ket{f} = \ket{z+}$.  We saw that the average value for the measurement of that physical quantity itself was zero, showing that we have equal probabilities of measuring either $\pm \hbar/2$ for this experiment.  A measurement that would show the system was in the x-direction spin-up or spin-down states would find that these states are equi-probable.

\section{Grading comments}

I lost one mark on the group velocity response.  Instead of \ref{eqn:qmIproblemSet3:104b} he wanted

\begin{align}\label{eqn:qmIproblemSet3:104b}
v_g = {\left. \PD{k}{\omega(k)} \right\vert}_{k = k_0}= \frac{\hbar k_0}{m} = 0
\end{align}

since $f(k)$ peaks at $k=0$.

I will have to go back and think about that a bit, because I am unsure of the last bits of the reasoning there.

I also lost 0.5 and 0.25 (twice) because I did not explicitly state that the probability that the particle is at $x_0$, a specific single point, is zero.  I thought that was obvious and did not have to be stated, but it appears expressing this explicitly is what he was looking for.

Curiously, one thing that I did not loose marks on was, the wrong answer for the probability per unit length.  What he was actually asking for was the following

\begin{align}\label{eqn:qmIproblemSet3:1000}
\lim_{\epsilon \rightarrow 0} \inv{\epsilon} \int_{x_0 - \epsilon/2}^{x_0 + \epsilon/2} \Abs{ \Psi(x_0, t_0) }^2 dx = \Abs{\Psi(x_0, t_0)}^2
\end{align}

That is a whole lot more sensible seeming quantity to calculate than what I did,  but I do not think that I can be faulted too much since the phrase was never used in the text nor in the lectures.
