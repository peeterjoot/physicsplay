%
% Copyright � 2012 Peeter Joot.  All Rights Reserved.
% Licenced as described in the file LICENSE under the root directory of this GIT repository.
%

%\chapter{PHY356 Problem Set III}
\label{chap:qmIproblemSet3}
%\blogpage{http://sites.google.com/site/peeterjoot/math2010/qmIproblemSet3.pdf}
%\date{Oct 23, 2010}

\subsection{Problem set III.  Problem 1}
\subsubsection{Statement}

A particle of mass \(m\) is free to move along the x-direction such that \(V(X)=0\). The state of the system is represented by the wavefunction Eq. (4.74)

\begin{equation}\label{eqn:qmIproblemSet3:1}
\psi(x,t) = \inv{\sqrt{2\pi}} \int_{-\infty}^\infty dk e^{i k x} e^{- i \omega t} f(k)
\end{equation}

with \(f(k)\) given by Eq. (4.59).

\begin{equation}\label{eqn:qmIproblemSet3:2}
f(k) = N e^{-\alpha k^2}
\end{equation}

Note that I have inserted a \(1/\sqrt{2\pi}\) factor above that is not in the text, because otherwise \(\psi(x,t)\) will not be unit normalized (assuming \(f(k)\) is normalized in wavenumber space).

\begin{itemize}
\item 
(a) What is the group velocity associated with this state? 
\item 
(b) What is the probability for measuring the particle at position \(x=x_0>0\) at time \(t=t_0>0\)? 
\item 
(c) What is the probability per unit length for measuring the particle at position \(x=x_0>0\) at time \(t=t_0>0\)? 
\item 
(d) Explain the physical meaning of the above results.
\end{itemize}

\subsubsection{Solution}
\paragraph{(a).  group velocity}

To calculate the group velocity we need to know the dependence of \(\omega\) on \(k\).

Let us step back and consider the time evolution action on \(\psi(x,0)\).  For the free particle case we have

\begin{equation}\label{eqn:qmIproblemSet3:101}
H = \frac{\Bp^2}{2m} = -\frac{\Hbar^2}{2m} \partial_{xx}.
\end{equation}

Writing \(N' = N/\sqrt{2\pi}\) we have

\begin{equation}\label{eqn:qmIproblemSet3:1020}
\begin{aligned}
-\frac{i t}{\Hbar} H \psi(x,0) 
&= 
\frac{i t \Hbar }{2m} 
N' \int_{-\infty}^\infty dk (i k)^2 e^{i k x - \alpha k^2} \\
&= 
N' \int_{-\infty}^\infty dk \frac{-i t \Hbar k^2}{2m} e^{i k x - \alpha k^2}
\end{aligned}
\end{equation}

Each successive application of \(-iHt/\Hbar\) will introduce another power of \(-it\Hbar k^2/2 m\), so once we sum all the terms of the exponential series \(U(t) = e^{-iHt/\Hbar}\) we have

\begin{equation}\label{eqn:qmIproblemSet3:102}
\psi(x,t) =
N' \int_{-\infty}^\infty dk \exp\left( 
\frac{-i t \Hbar k^2}{2m} + i k x - \alpha k^2 \right).
\end{equation}

Comparing with \eqnref{eqn:qmIproblemSet3:1} we find
\begin{equation}\label{eqn:qmIproblemSet3:103}
\omega(k) = \frac{\Hbar k^2}{2m}.
\end{equation}

This completes this section of the problem since we are now able to calculate the group velocity 
\begin{equation}\label{eqn:qmIproblemSet3:104}
v_g = \PD{k}{\omega(k)} = \frac{\Hbar k}{m}.
\end{equation}

\subsubsection{(b). What is the probability for measuring the particle at position \texorpdfstring{\(x=x_0>0\)}{x positive} at time \texorpdfstring{\(t=t_0>0\)}{greater than zero}?}

In order to evaluate the probability, it looks desirable to evaluate the wave function integral \eqnref{eqn:qmIproblemSet3:102}.  
Writing \(2 \beta = i/(\alpha + i t \Hbar/2m )\), the exponent of that integral is

\begin{equation}\label{eqn:qmIproblemSet3:1040}
\begin{aligned}
-k^2 \left( \alpha + \frac{i t \Hbar }{2m} \right) + i k x
&=
-\left( \alpha + \frac{i t \Hbar }{2m} \right) \left( k^2 - \frac{i k x }{\alpha + \frac{i t \Hbar }{2m} } \right) \\
&=
-\frac{i}{2\beta} \left( (k - x \beta )^2 - x^2 \beta^2 \right)
\end{aligned}
\end{equation}

The \(x^2\) portion of the exponential

\begin{equation}\label{eqn:qmIproblemSet3:1060}
\frac{i x^2 \beta^2}{2\beta} = \frac{i x^2 \beta}{2} = - \frac{x^2 }{4 (\alpha + i t \Hbar /2m)}
\end{equation}

then comes out of the integral.  We can also make a change of variables \(q = k - x \beta\) to evaluate the remainder of the Gaussian and are left with

\begin{equation}\label{eqn:qmIproblemSet3:107}
\psi(x,t) =
N' \sqrt{ \frac{\pi}{\alpha + i t \Hbar/2m} } \exp\left( - \frac{x^2 }{4 (\alpha + i t \Hbar /2m)} \right).
\end{equation}

Observe that from \eqnref{eqn:qmIproblemSet3:2} we can compute \(N = (2 \alpha/\pi)^{1/4}\), which could be substituted back into \eqnref{eqn:qmIproblemSet3:107} if desired.

Our probability density is 

\begin{equation}\label{eqn:qmIproblemSet3:1080}
\begin{aligned}
\Abs{ \psi(x,t) }^2 
&=
\inv{2 \pi} N^2 \Abs{ \frac{\pi}{\alpha + i t \Hbar/2m} } \exp\left( - \frac{x^2}{4} \left( 
\inv{(\alpha + i t \Hbar /2m)} + \inv{(\alpha - i t \Hbar /2m)} 
\right) \right) \\
&=
\inv{2 \pi} \sqrt{\frac{2 \alpha}{\pi} } \frac{\pi}{\sqrt{\alpha^2 + (t \Hbar/2m)^2 }} \exp\left( - \frac{x^2}{4} 
\inv{\alpha^2 + (t \Hbar/2m)^2 } \left( 
\alpha - i t \Hbar /2m + \alpha + i t \Hbar /2m 
\right)
\right) \\
&=
\end{aligned}
\end{equation}

With a final regrouping of terms, this is

\begin{equation}\label{eqn:qmIproblemSet3:110}
\Abs{ \psi(x,t) }^2 =
\sqrt{\frac{ \alpha }{ 2 \pi (\alpha^2 + (t \Hbar/2m)^2 }) }
\exp\left( - \frac{x^2}{2} 
\frac{\alpha}{\alpha^2 + (t \Hbar/2m)^2 } 
\right).
\end{equation}

As a sanity check we observe that this integrates to unity for all \(t\) as desired.  The probability that we find the particle at position \(x > x_0\) is then

\begin{equation}\label{eqn:qmIproblemSet3:111}
P_{x>x_0}(t) = \sqrt{\frac{ \alpha }{ 2 \pi (\alpha^2 + (t \Hbar/2m)^2 }) }
\int_{x=x_0}^\infty dx \exp\left( - \frac{x^2}{2} 
\frac{\alpha}{\alpha^2 + (t \Hbar/2m)^2 } 
\right)
\end{equation}

The only simplification we can make is to rewrite this in terms of the complementary error function

\begin{equation}\label{eqn:qmIproblemSet3:112}
\erfc(x) = \frac{2}{\sqrt{\pi}} \int_x^\infty e^{-t^2} dt.
\end{equation}

Writing

\begin{equation}\label{eqn:qmIproblemSet3:113}
\beta(t) = \frac{\alpha}{\alpha^2 + (t \Hbar/2m)^2 },
\end{equation}

we have
\begin{equation}\label{eqn:qmIproblemSet3:114}
P_{x>x_0}(t_0) = \inv{2} \erfc \left( \sqrt{\beta(t_0)/2} x_0 \right)
\end{equation}

Sanity checking this result, we note that since \(\erfc(0) = 1\) the probability for finding the particle in the \(x>0\) range is \(1/2\) as expected.


\subsubsection{(c). What is the probability per unit length for measuring the particle at position 
\texorpdfstring{\(x=x_0>0\)}{x positive} at time \texorpdfstring{\(t=t_0>0\)}{greater than zero}?}

This unit length probability is thus

\begin{equation}\label{eqn:qmIproblemSet3:115}
P_{x>x_0+1/2}(t_0) - P_{x>x_0-1/2}(t_0) 
=
\inv{2} \erfc\left( \sqrt{\frac{\beta(t_0)}{2}} \left(x_0+\inv{2} \right) \right) 
-\inv{2} \erfc\left( \sqrt{\frac{\beta(t_0)}{2}} \left(x_0-\inv{2} \right) \right) 
\end{equation}

\subsubsection{(d). Explain the physical meaning of the above results}

To get an idea what the group velocity means, observe that we can write our wavefunction \eqnref{eqn:qmIproblemSet3:1} as

\begin{equation}\label{eqn:qmIproblemSet3:140}
\psi(x,t) = \inv{\sqrt{2\pi}} \int_{-\infty}^\infty dk e^{i k (x - v_g t)} f(k)
\end{equation}

We see that the phase coefficient of the Gaussian \(f(k)\) ``moves'' at the rate of the group velocity \(v_g\).  Also recall that in the text it is noted that the time dependent term \eqnref{eqn:qmIproblemSet3:113} can be expressed in terms of position and momentum uncertainties \((\Delta x)^2\), and \((\Delta p)^2 = \Hbar^2 (\Delta k)^2\).  That is

\begin{equation}\label{eqn:qmIproblemSet3:116}
\inv{\beta(t)} = (\Delta x)^2 + \frac{(\Delta p)^2}{m^2} t^2 \equiv (\Delta x(t))^2
\end{equation}

This makes it evident that the probability density flattens and spreads over time with the rate equal to the uncertainty of the group velocity \(\Delta p/m = \Delta v_g\) (since \(v_g = \Hbar k/m\)).  It is interesting that something as simple as this phase change results in a physically measurable phenomena.  We see that a direct result of this linear with time phase change, we are less able to find the particle localized around it is original time \(x = 0\) position as more time elapses.

\subsubsection{Grading comments}

I lost one mark on the group velocity response.  Instead of \eqnref{eqn:qmIproblemSet3:104} he wanted

\begin{equation}\label{eqn:qmIproblemSet3:104b}
v_g = {\left. \PD{k}{\omega(k)} \right\vert}_{k = k_0}= \frac{\Hbar k_0}{m} = 0
\end{equation}

since \(f(k)\) peaks at \(k=0\).

I will have to go back and think about that a bit, because I am unsure of the last bits of the reasoning there.

I also lost 0.5 and 0.25 (twice) because I did not explicitly state that the probability that the particle is at \(x_0\), a specific single point, is zero.  I thought that was obvious and did not have to be stated, but it appears expressing this explicitly is what he was looking for.

Curiously, one thing that I did not loose marks on was, the wrong answer for the probability per unit length.  What he was actually asking for was the following

\begin{equation}\label{eqn:qmIproblemSet3:1000}
\lim_{\epsilon \rightarrow 0} \inv{\epsilon} \int_{x_0 - \epsilon/2}^{x_0 + \epsilon/2} \Abs{ \Psi(x_0, t_0) }^2 dx = \Abs{\Psi(x_0, t_0)}^2
\end{equation}

That is a whole lot more sensible seeming quantity to calculate than what I did,  but I do not think that I can be faulted too much since the phrase was never used in the text nor in the lectures.
