%
% Copyright � 2012 Peeter Joot.  All Rights Reserved.
% Licenced as described in the file LICENSE under the root directory of this GIT repository.
%

%\chapter{Some worked problems from old PHY356 exams}
\label{chap:qmIexamPractice}
%\blogpage{http://sites.google.com/site/peeterjoot/math2010/qmIexamPractice.pdf}
%\date{Jan 4, 2011}

%\section{Motivation}
%
%Some of the old exam questions that I did for preparation for the exam I liked, and thought I had write up some of them for potential future reference.
%
\section{Questions from the Dec 2007 PHY355H1F exam}
\subsection{1b. Parity operator}

\paragraph{Q:} If \(\Pi\) is the parity operator, defined by \(\Pi \ket{x} = \ket{-x}\), where \(\ket{x}\) is the eigenket of the position operator \(X\) with eigenvalue \(x\)), and \(P\) is the momentum operator conjugate to \(X\), show (carefully) that \(\Pi P \Pi = -P\).

\paragraph{A:}

Consider the matrix element \(\bra{-x'} \antisymmetric{\Pi}{P} \ket{x}\).  This is

\begin{equation}\label{eqn:qmIexamPractice:430}
\begin{aligned}
\bra{-x'} \antisymmetric{\Pi}{P} \ket{x}
&=
\bra{-x'} \Pi P - P \Pi \ket{x} \\
&=
\bra{-x'} \Pi P \ket{x} - \bra{-x} P \Pi \ket{x} \\
&=
\bra{x'} P \ket{x} - \bra{-x} P \ket{-x} \\
&=
- i \Hbar \left(
\delta(x'-x) \PD{x}{}
-
\mathLabelBox{\delta(-x -(-x'))}{\(= \delta(x'-x) = \delta(x-x')\)}
\PD{-x}{}
\right) \\
&=
- 2 i \Hbar 
\delta(x'-x) \PD{x}{} \\
&=
2 \bra{x'} P \ket{x} \\
&=
2 \bra{-x'} \Pi P \ket{x} \\
\end{aligned}
\end{equation}

We have taken advantage of the Hermitian property of \(P\) and \(\Pi\) here, and can rearrange for

\begin{equation}\label{eqn:qmIexamPractice2007Dec:1b:10}
\bra{-x'} \Pi P - P \Pi - 2 \Pi P \ket{x} = 0
\end{equation}

Since this is true for all \(\bra{-x}\) and \(\ket{x}\) we have

\begin{equation}\label{eqn:qmIexamPractice2007Dec:1b:20}
\Pi P + P \Pi = 0.
\end{equation}

Right multiplication by \(\Pi\) and rearranging we have
\begin{equation}\label{eqn:qmIexamPractice2007Dec:1b:30}
\Pi P \Pi = - P \Pi \Pi = - P.
\end{equation}

\subsection{1f. Free particle propagator}

\paragraph{Q:} For a free particle moving in one-dimension, the propagator (i.e. the coordinate representation of the evolution operator),

\begin{equation}\label{eqn:qmIexamPractice2007Dec:1f:10}
G(x,x';t) = \bra{x} U(t) \ket{x'}
\end{equation}

is given by

\begin{equation}\label{eqn:qmIexamPractice2007Dec:1f:20}
G(x,x';t) = \sqrt{\frac{m}{2 \pi i \Hbar t}} e^{i m (x-x')^2/ (2 \Hbar t)}.
\end{equation}

\paragraph{A:}

This problem is actually fairly straightforward, but it is nice to work it having had a similar problem set question where we were asked about this time evolution operator matrix element (ie: what it is physical meaning is).  Here we have a concrete example of the form of this matrix operator.

Proceeding directly, we have

\begin{equation}\label{eqn:qmIexamPractice:450}
\begin{aligned}
\bra{x} U \ket{x'}
&=
\int \braket{x}{p'} \bra{p'} U \ket{p} \braket{p}{x'} dp dp' \\
&=
\int u_{p'}(x) \bra{p'} e^{-i P^2 t/(2 m \Hbar)} \ket{p} u_p^\conj(x') dp dp' \\
&=
\int u_{p'}(x) e^{-i p^2 t/(2 m \Hbar)} \delta(p-p') u_p^\conj(x') dp dp' \\
&=
\int u_{p}(x) e^{-i p^2 t/(2 m \Hbar)} u_p^\conj(x') dp \\
&=
\inv{(\sqrt{2 \pi \Hbar})^2} \int e^{i p (x-x')/\Hbar} e^{-i p^2 t/(2 m \Hbar)} dp \\
&=
\inv{2 \pi \Hbar} \int e^{i p (x-x')/\Hbar} e^{-i p^2 t/(2 m \Hbar)} dp \\
&=
\inv{2 \pi} 
\int e^{i k (x-x')} e^{-i \Hbar k^2 t/(2 m)} dk \\
&=
\inv{2 \pi} 
\int dk e^{- \left(k^2 \frac{ i \Hbar t}{2m} - i k (x-x')
\right)
} \\
&=
\inv{2 \pi} 
\int dk e^{- \frac{ i \Hbar t}{2m}\left(k - i \frac{2m}{i \Hbar t}\frac{(x-x')}{2} \right)^2
- \frac{i^2 2 m (x-x')^2}{4 i \Hbar t} 
} \\
&=
\inv{2 \pi}  \sqrt{\pi} \sqrt{\frac{2m}{i \Hbar t}}
e^{\frac{ i m (x-x')^2}{2 \Hbar t}},
\end{aligned}
\end{equation}

which is the desired result.  Now, let us look at how this would be used.  We can express our time evolved state using this matrix element by introducing an identity

\begin{equation}\label{eqn:qmIexamPractice:470}
\begin{aligned}
\braket{x}{\psi(t)} 
&=
\bra{x} U \ket{\psi(0)} \\
&=
\int dx' \bra{x} U \ket{x'} \braket{x'}{\psi(0)} \\
&=
\sqrt{\frac{m}{2 \pi i \Hbar t}} 
\int dx' 
e^{i m (x-x')^2/ (2 \Hbar t)}
\braket{x'}{\psi(0)} \\
\end{aligned}
\end{equation}

This gives us
\begin{equation}\label{eqn:qmIexamPractice2007Dec:1f:40}
\psi(x, t)
=
\sqrt{\frac{m}{2 \pi i \Hbar t}} 
\int dx' 
e^{i m (x-x')^2/ (2 \Hbar t)} \psi(x', 0)
\end{equation}

However, note that our free particle wave function at time zero is

\begin{equation}\label{eqn:qmIexamPractice2007Dec:1f:50}
\psi(x, 0) = \frac{e^{i p x/\Hbar}}{\sqrt{2 \pi \Hbar}}
\end{equation}

So the convolution integral \eqnref{eqn:qmIexamPractice2007Dec:1f:40} does not exist.  We likely have to require that the solution be not a pure state, but instead a superposition of a set of continuous states (a wave packet in position or momentum space related by Fourier transforms).  That is

\begin{equation}\label{eqn:qmIexamPractice2007Dec:1f:60}
\begin{aligned}
\psi(x, 0) &= 
\inv{\sqrt{2 \pi \Hbar}} \int \hat{\psi}(p, 0) e^{i p x/\Hbar} dp \\
\hat{\psi}(p, 0) &= 
\inv{\sqrt{2 \pi \Hbar}} \int \psi(x'', 0) e^{-i p x''/\Hbar} dx''
\end{aligned}
\end{equation}

The time evolution of this wave packet is then determined by the propagator, and is

\begin{equation}\label{eqn:qmIexamPractice2007Dec:1f:70}
\psi(x,t) =
\sqrt{\frac{m}{2 \pi i \Hbar t}} 
\inv{\sqrt{2 \pi \Hbar}} 
\int dx' dp
e^{i m (x-x')^2/ (2 \Hbar t)}
\hat{\psi}(p, 0) e^{i p x'/\Hbar} ,
\end{equation}

or in terms of the position space wave packet evaluated at time zero

\begin{equation}\label{eqn:qmIexamPractice2007Dec:1f:80}
\psi(x,t) =
\sqrt{\frac{m}{2 \pi i \Hbar t}}
\inv{2 \pi}
\int dx' dx'' dk
e^{i m (x-x')^2/ (2 \Hbar t)}
e^{i k (x' - x'')} \psi(x'', 0)
\end{equation}

We see that the propagator also ends up with a Fourier transform structure, and we have

\begin{equation}\label{eqn:qmIexamPractice2007Dec:1f:90}
\begin{aligned}
\psi(x,t) &= \int dx' U(x, x' ; t) \psi(x', 0) \\
U(x, x' ; t) &=
\sqrt{\frac{m}{2 \pi i \Hbar t}}
\inv{2 \pi}
\int du dk
e^{i m (x - x' - u)^2/ (2 \Hbar t)}
e^{i k u }
\end{aligned}
\end{equation}

Does that Fourier transform exist?  I had not be surprised if it ended up with a delta function representation.  I will hold off attempting to evaluate and reduce it until another day.

%\subsection{2}
\subsection{4. Hydrogen atom}

This problem deals with the hydrogen atom, with an initial ket

\begin{equation}\label{eqn:qmIexamPractice2007Dec:4:10}
\ket{\psi(0)} = 
\inv{\sqrt{3}} \ket{100}
+\inv{\sqrt{3}} \ket{210}
+\inv{\sqrt{3}} \ket{211},
\end{equation}

where 

\begin{equation}\label{eqn:qmIexamPractice2007Dec:4:20}
\braket{\Br}{100} = \Phi_{100}(\Br),
\end{equation}

etc.

\paragraph{Q: (a)}

If no measurement is made until time \(t = t_0\),

\begin{equation}\label{eqn:qmIexamPractice2007Dec:4:30}
t_0 = \frac{\pi \Hbar}{ \frac{3}{4} (13.6 \text{eV}) } = \frac{ 4 \pi \Hbar }{ 3 E_I},
\end{equation}

what is the ket \(\ket{\psi(t)}\) just before the measurement is made?

\paragraph{A:}

Our time evolved state is 

\begin{equation}\label{eqn:qmIexamPractice2007Dec:4:35}
\ket{\psi{t_0}} = 
\inv{\sqrt{3}} e^{-i E_1 t_0 /\Hbar } \ket{100}
+\inv{\sqrt{3}} e^{- i E_2 t_0/\Hbar } 
(\ket{210} + \ket{211}).
\end{equation}

Also observe that this initial time was picked to make the exponential values come out nicely, and we have

\begin{equation}\label{eqn:qmIexamPractice:490}
\begin{aligned}
\frac{E_n t_0 }{\Hbar} 
&= - \frac{E_I \pi \Hbar }{\frac{3}{4} E_I n^2 \Hbar} \\
&= - \frac{4 \pi }{ 3 n^2 },
\end{aligned}
\end{equation}

so our time evolved state is just

\begin{equation}\label{eqn:qmIexamPractice2007Dec:4:100}
\ket{\psi(t_0)} = 
\inv{\sqrt{3}} e^{-i 4 \pi / 3} \ket{100}
+\inv{\sqrt{3}} e^{- i \pi / 3 } 
(\ket{210} + \ket{211}).
\end{equation}

\paragraph{Q: (b)}

Suppose that at time \(t_0\) an \(L_z\) measurement is made, and the outcome 0 is recorded.  What is the appropriate ket \(\psi_{\text{after}}(t_0)\) right after the measurement?

\paragraph{A:}

A measurement with outcome 0, means that the \(L_z\) operator measurement found the state at that point to be the eigenstate for \(L_z\) eigenvalue 0.  Recall that  if \(\ket{\phi}\) is an eigenstate of \(L_z\) we have

\begin{equation}\label{eqn:qmIexamPractice2007Dec:4:200}
L_z \ket{\phi} = m \Hbar \ket{\phi},
\end{equation}

so a measurement of \(L_z\) with outcome zero means that we have \(m=0\).  Our measurement of \(L_z\) at time \(t_0\) therefore filters out all but the \(m=0\) states and our new state is proportional to the projection over all \(m=0\) states as follows

\begin{equation}\label{eqn:qmIexamPractice:510}
\begin{aligned}
\ket{\psi_{\text{after}}(t_0)}
&\propto \left( \sum_{n l} \ket{n l 0}\bra{n l 0} \right) \ket{\psi(t_0)}  \\
&\propto \left( 
\ket{1 0 0}\bra{1 0 0} 
+\ket{2 1 0}\bra{2 1 0} 
\right) \ket{\psi(t_0)}  \\
&= 
\inv{\sqrt{3}} e^{-i 4 \pi / 3} \ket{100}
+\inv{\sqrt{3}} e^{- i \pi / 3 } \ket{210} 
\end{aligned}
\end{equation}

A final normalization yields
\begin{equation}\label{eqn:qmIexamPractice2007Dec:4:210}
\ket{\psi_{\text{after}}(t_0)}
= \inv{\sqrt{2}} (\ket{210} - \ket{100})
\end{equation}

\paragraph{Q: (c)}

Right after this \(L_z\) measurement, what is \(\Abs{\psi_{\text{after}}(t_0)}^2\)?

\paragraph{A:}

Our amplitude is 

\begin{equation}\label{eqn:qmIexamPractice:530}
\begin{aligned}
\braket{\Br}{\psi_{\text{after}}(t_0)}
&= \inv{\sqrt{2}} (\braket{\Br}{210} - \braket{\Br}{100}) \\
&= \inv{\sqrt{2 \pi a_0^3}}
\left(
\frac{r}{4\sqrt{2} a_0} e^{-r/2a_0} \cos\theta
-e^{-r/a_0}
\right) \\
&= \inv{\sqrt{2 \pi a_0^3}}
e^{-r/2 a_0} 
\left(
\frac{r}{4\sqrt{2} a_0} \cos\theta
-e^{-r/2 a_0}
\right),
\end{aligned}
\end{equation}

so the probability density is
\begin{equation}\label{eqn:qmIexamPractice2007Dec:4:300}
\Abs{\braket{\Br}{\psi_{\text{after}}(t_0)}}^2
= \inv{2 \pi a_0^3}
e^{-r/a_0} 
\left(
\frac{r}{4\sqrt{2} a_0} \cos\theta
-e^{-r/2 a_0}
\right)^2 
\end{equation}

\paragraph{Q: (d)}

If then a position measurement is made immediately, which if any components of the expectation value of \(\BR\) will be non-vanishing?  Justify your answer.

\paragraph{A:}

The expectation value of this vector valued operator with respect to a radial state \(\ket{\psi} = \sum_{nlm} a_{nlm} \ket{nlm}\) can be expressed as

\begin{equation}\label{eqn:qmIexamPractice2007Dec:4:400}
\expectation{\BR} = \sum_{i=1}^3 \Be_i \sum_{nlm, n'l'm'} 
a_{nlm}^\conj a_{n'l'm'} 
\bra{nlm} X_i
\ket{n'l'm'},
\end{equation}

where \(X_1 = X = R \sin\Theta \cos\Phi, X_2 = Y = R \sin\Theta \sin\Phi, X_3 = Z = R \cos\Phi\).

Consider one of the matrix elements, and expand this by introducing an identity twice

\begin{equation}\label{eqn:qmIexamPractice:550}
\begin{aligned}
\bra{nlm} X_i \ket{n'l'm'}
&=
\int 
r^2 \sin\theta dr d\theta d\phi
{r'}^2 \sin\theta' dr' d\theta' d\phi'
\braket{nlm}{r \theta \phi} \bra{r \theta \phi} X_i \ket{r' \theta' \phi' }\braket{r' \theta' \phi'}{n'l'm'} \\
&=
\int 
r^2 \sin\theta dr d\theta d\phi
{r'}^2 \sin\theta' dr' d\theta' d\phi'
R_{nl}(r) Y_{lm}^\conj(\theta,\phi)
\delta^3(\Bx - \Bx') x_i
R_{n'l'}(r') Y_{l'm'}(\theta',\phi')
\\
&=
\int 
r^2 \sin\theta dr d\theta d\phi
{r'}^2 \sin\theta' dr' d\theta' d\phi'
R_{nl}(r) Y_{lm}^\conj(\theta,\phi) \\
&\qquad{r'}^2 \sin\theta' \delta(r-r') \delta(\theta - \theta') \delta(\phi-\phi')
x_i
R_{n'l'}(r') Y_{l'm'}(\theta',\phi')
\\
&=
\int 
r^2 \sin\theta dr d\theta d\phi
dr' d\theta' d\phi'
R_{nl}(r) Y_{lm}^\conj(\theta,\phi) 
\delta(r-r') \delta(\theta - \theta') \delta(\phi-\phi')
x_i
R_{n'l'}(r') Y_{l'm'}(\theta',\phi')
\\
&=
\int 
r^2 \sin\theta dr d\theta d\phi
R_{nl}(r) R_{n'l'}(r) 
Y_{lm}^\conj(\theta,\phi) Y_{l'm'}(\theta,\phi)
x_i
\\
\end{aligned}
\end{equation}

Because our state has only \(m=0\) contributions, the only \(\phi\) dependence for the \(X\) and \(Y\) components of \(\BR\) come from those components themselves.  For \(X\), we therefore integrate \(\int_0^{2\pi} \cos\phi d\phi = 0\), and for \(Y\) we integrate \(\int_0^{2\pi} \sin\phi d\phi = 0\), and these terms vanish.  Our expectation value for \(\BR\) for this state, therefore lies completely on the \(z\) axis.

\section{Questions from the Dec 2008 PHY355H1F exam}

\subsection{1b. Trace invariance for unitary transformation}

\paragraph{Q:} Show that the trace of an operator is invariant under unitary transforms, i.e. if \(A' = U^\dagger A U\), where \(U\) is a unitary operator, prove \(\tr(A') = \tr(A)\).

\paragraph{A:} 

The bulk of this question is really to show that commutation of operators leaves the trace invariant (unless this is assumed).  To show that we start with the definition of the trace

\begin{equation}\label{eqn:qmIexamPractice:570}
\begin{aligned}
\tr(AB) 
&= \sum_n \bra{n} A B \ket{n} \\
&= \sum_{n m} \bra{n} A \ket{m} \bra{m} B \ket{n} \\
&= \sum_{n m} 
\bra{m} B \ket{n} 
\bra{n} A \ket{m} 
\\
&= \sum_{m} \bra{m} B A \ket{m}.
\end{aligned}
\end{equation}

Thus we have
\begin{equation}\label{eqn:qmIexamPractice2008Dec:1b:10}
\tr(A B) = \tr( B A ).
\end{equation}

For the unitarily transformed operator we have
\begin{equation}\label{eqn:qmIexamPractice:590}
\begin{aligned}
\tr(A') 
&= \tr( U^\dagger A U ) \\
&= \tr( U^\dagger (A U) ) \\
&= \tr( (A U) U^\dagger ) \\
&= \tr( A (U U^\dagger) ) \\
&= \tr( A ) \qedmarker
\end{aligned}
\end{equation}

\subsection{1d.  Determinant of an exponential operator in terms of trace}

\paragraph{Q:} If \(A\) is an Hermitian operator, show that

\begin{equation}\label{eqn:qmIexamPractice2008Dec:1d:10}
\Det( \exp A ) = \exp ( \tr(A) )
\end{equation}

where the \textAndIndex{determinant} (\(\Det\)) of an operator is the product of all its eigenvectors.

\paragraph{A:}

The eigenvalues clue in the question provides the starting point.  We write the exponential in its series form

\begin{equation}\label{eqn:qmIexamPractice2008Dec:1d:20}
e^A = 1 + \sum_{k=1}^\infty \inv{k!} A^k
\end{equation}

Now, suppose that we have the following eigenvalue relationships for \(A\)

\begin{equation}\label{eqn:qmIexamPractice2008Dec:1d:30}
A \ket{n} = \lambda_n \ket{n}.
\end{equation}

From this the exponential is

\begin{equation}\label{eqn:qmIexamPractice:610}
\begin{aligned}
e^A \ket{n} 
&= \ket{n} + \sum_{k=1}^\infty \inv{k!} A^k \ket{n} \\
&= \ket{n} + \sum_{k=1}^\infty \inv{k!} (\lambda_n)^k \ket{n} \\
&= e^{\lambda_n} \ket{n}.
\end{aligned}
\end{equation}

We see that the eigenstates of \(e^A\) are those of \(A\), with eigenvalues \(e^{\lambda_n}\).

By the definition of the determinant given we have

\begin{equation}\label{eqn:qmIexamPractice:630}
\begin{aligned}
\Det( e^A ) 
&= \prod_n e^{\lambda_n} \\
&= e^{\sum_n \lambda_n} \\
&= e^{\trace(A)}. \qedmarker
\end{aligned}
\end{equation}

\subsection{1e.  Eigenvectors of the Harmonic oscillator creation operator}

\paragraph{Q:} Prove that the only eigenvector of the Harmonic oscillator creation operator is \(\ket{\text{null}}\).

\paragraph{A:} 

Recall that the creation (raising) operator was given by

\begin{equation}\label{eqn:qmIexamPractice2008Dec:1e:10}
a^\dagger 
= \sqrt{\frac{m \omega}{2 \Hbar}} X - \frac{ i }{\sqrt{2 m \omega \Hbar} } P
= \inv{ \alpha \sqrt{2} } X - \frac{ i \alpha }{\sqrt{2} \Hbar } P,
\end{equation}

where \(\alpha = \sqrt{\Hbar/m \omega}\).  Now assume that \(a^\dagger \ket{\phi} = \lambda \ket{\phi}\) so that 

\begin{equation}\label{eqn:qmIexamPractice2008Dec:1e:20}
\bra{x} a^\dagger \ket{\phi} = \bra{x} \lambda \ket{\phi}.
\end{equation}

Write \(\braket{x}{\phi} = \phi(x)\), and expand the LHS using \eqnref{eqn:qmIexamPractice2008Dec:1e:10} for

\begin{equation}\label{eqn:qmIexamPractice:650}
\begin{aligned}
\lambda \phi(x) 
&= \bra{x} a^\dagger \ket{\phi}  \\
&= \bra{x} \left( \inv{ \alpha \sqrt{2} } X - \frac{ i \alpha }{\sqrt{2} \Hbar } P \right) \ket{\phi} \\
&= \frac{x \phi(x)}{ \alpha \sqrt{2} } - \frac{ i \alpha }{\sqrt{2} \Hbar } (-i\Hbar)\PD{x}{} \phi(x) \\
&= \frac{x \phi(x)}{ \alpha \sqrt{2} } - \frac{ \alpha }{\sqrt{2} } \PD{x}{\phi(x)}.
\end{aligned}
\end{equation}

As usual write \(\xi = x/\alpha\), and rearrange.  This gives us

\begin{equation}\label{eqn:qmIexamPractice2008Dec:1e:30}
\PD{\xi}{\phi} +\sqrt{2} \lambda \phi - \xi \phi = 0.
\end{equation}

Observe that this can be viewed as a homogeneous LDE of the form
\begin{equation}\label{eqn:qmIexamPractice2008Dec:1e:40}
\PD{\xi}{\phi} - \xi \phi = 0,
\end{equation}

augmented by a forcing term \(\sqrt{2}\lambda \phi\).  The homogeneous equation has the solution \(\phi = A e^{\xi^2/2}\), so for the complete equation we assume a solution 

\begin{equation}\label{eqn:qmIexamPractice2008Dec:1e:50}
\phi(\xi) = A(\xi) e^{\xi^2/2}.
\end{equation}

Since \(\phi' = (A' + A \xi) e^{\xi^2/2}\), we produce a LDE of

\begin{equation}\label{eqn:qmIexamPractice:670}
\begin{aligned}
0 &= (A' + A \xi -\xi A + \sqrt{2} \lambda A ) e^{\xi^2/2} \\
&= (A' + \sqrt{2} \lambda A ) e^{\xi^2/2},
\end{aligned}
\end{equation}

or
\begin{equation}\label{eqn:qmIexamPractice2008Dec:1e:60}
0 = A' + \sqrt{2} \lambda A.
\end{equation}

This has solution \(A = B e^{-\sqrt{2} \lambda \xi}\), so our solution for \eqnref{eqn:qmIexamPractice2008Dec:1e:30} is
\begin{equation}\label{eqn:qmIexamPractice2008Dec:1e:70}
\phi(\xi) = B e^{\xi^2/2 - \sqrt{2} \lambda \xi} 
= B' e^{ (\xi - \lambda \sqrt{2} )^2/2}.
\end{equation}

This wave function is an imaginary Gaussian with minimum at \(\xi = \lambda\sqrt{2}\).  It is also unnormalizable since we require \(B' = 0\) for any \(\lambda\) if \(\int \Abs{\phi}^2 < \infty\).  Since \(\braket{\xi}{\phi} = \phi(\xi) = 0\), we must also have \(\ket{\phi} = 0\), completing the exercise.

\subsection{2.  Two level quantum system}

Consider a two-level quantum system, with basis states \(\{\ket{a}, \ket{b}\}\).  Suppose that the Hamiltonian for this system is given by

\begin{equation}\label{eqn:qmIexamPractice2008Dec:2:10}
H = 
\frac{\Hbar \Delta}{2} ( 
\ket{b}\bra{b}
- \ket{a}\bra{a}
)
+ i \frac{\Hbar \Omega}{2} ( 
\ket{a}\bra{b}
- \ket{b}\bra{a}
)
\end{equation}

where \(\Delta\) and \(\Omega\) are real positive constants.

\paragraph{Q: (a)} Find the energy eigenvalues and the normalized energy eigenvectors (expressed in terms of the \(\{\ket{a}, \ket{b}\}\) basis).  Write the time evolution operator \(U(t) = e^{-i H t/\Hbar}\) using these eigenvectors.

\paragraph{A:}

The eigenvalue part of this problem is probably easier to do in matrix form.  Let 

\begin{equation}\label{eqn:qmIexamPractice2008Dec:2:20}
\begin{aligned}
\ket{a} &= 
\begin{bmatrix}
1 \\
0
\end{bmatrix} \\
\ket{b} &= 
\begin{bmatrix}
0 \\
1
\end{bmatrix}.
\end{aligned}
\end{equation}

Our Hamiltonian is then
\begin{equation}\label{eqn:qmIexamPractice2008Dec:2:30}
H = \frac{\Hbar}{2} 
\begin{bmatrix}
-\Delta & i \Omega \\
-i \Omega & \Delta
\end{bmatrix}.
\end{equation}

Computing \(\det{H - \lambda I} = 0\), we get

\begin{equation}\label{eqn:qmIexamPractice2008Dec:2:40}
\lambda = \pm \frac{\Hbar}{2} \sqrt{ \Delta^2 + \Omega^2 }.
\end{equation}

Let \(\delta = \sqrt{ \Delta^2 + \Omega^2 }\).  Our normalized eigenvectors are found to be

\begin{equation}\label{eqn:qmIexamPractice2008Dec:2:50}
\ket{\pm} = \inv{\sqrt{ 2 \delta (\delta \pm \Delta)} }
\begin{bmatrix}
i \Omega \\
\Delta \pm \delta
\end{bmatrix}.
\end{equation}

In terms of \(\ket{a}\) and \(\ket{b}\), we then have

\begin{equation}\label{eqn:qmIexamPractice2008Dec:2:60}
\ket{\pm} = \inv{\sqrt{ 2 \delta (\delta \pm \Delta)} }
\left(
i \Omega \ket{a}
+ (\Delta \pm \delta) \ket{b} \right).
\end{equation}

Note that our Hamiltonian has a simple form in this basis.  That is

\begin{equation}\label{eqn:qmIexamPractice2008Dec:2:70}
H = \frac{\delta \Hbar}{2} (\ket{+}\bra{+} - \ket{-}\bra{-} )
\end{equation}

Observe that once we do the diagonalization, we have a Hamiltonian that appears to have the form of a scaled projector for an open Stern-Gerlach apparatus.

Observe that the diagonalized Hamiltonian operator makes the time evolution operator's form also simple, which is, by inspection

\begin{equation}\label{eqn:qmIexamPractice2008Dec:2:80}
U(t) = 
e^{-i t \frac{\delta}{2}} \ket{+}\bra{+} 
+ e^{i t \frac{\delta}{2}} \ket{-}\bra{-}.
\end{equation}

Since we are asked for this in terms of \(\ket{a}\), and \(\ket{b}\), the projectors \(\ket{\pm}\bra{\pm}\) are required.  These are

\begin{equation}\label{eqn:qmIexamPractice:690}
\begin{aligned}
\ket{\pm}\bra{\pm} 
&= \inv{2 \delta (\delta \pm \Delta)}
\Bigl( i \Omega \ket{a} + (\Delta \pm \delta) \ket{b} \Bigr)
\Bigl( -i \Omega \bra{a} + (\Delta \pm \delta) \bra{b} \Bigr) \\
\end{aligned}
\end{equation}

\begin{equation}\label{eqn:qmIexamPractice2008Dec:2:90}
\ket{\pm}\bra{\pm} 
= \inv{2 \delta (\delta \pm \Delta)}
\Bigl(
\Omega^2 \ket{a}\bra{a}
+(\delta \pm \delta)^2 \ket{b}\bra{b}
+i \Omega (\Delta \pm \delta) (
\ket{a}\bra{b}
-\ket{b}\bra{a}
)
\Bigr)
\end{equation}

Substitution into \eqnref{eqn:qmIexamPractice2008Dec:2:80} and a fair amount of algebra leads to

\begin{equation}\label{eqn:qmIexamPractice2008Dec:2:100}
U(t) = 
\cos(\delta t/2) \Bigl( \ket{a}\bra{a} + \ket{b}\bra{b} \Bigr)
+ i \frac{\Omega}{\delta} \sin(\delta t/2) \Bigl( 
\ket{a}\bra{a} - \ket{b}\bra{b} 
-i (\ket{a}\bra{b} - \ket{b}\bra{a} )
\Bigr).
\end{equation}

Note that while a big cumbersome, we can also verify that we can recover the original Hamiltonian from \eqnref{eqn:qmIexamPractice2008Dec:2:70} and \eqnref{eqn:qmIexamPractice2008Dec:2:90}.

\paragraph{Q: (b)}

Suppose that the initial state of the system at time \(t = 0\) is \(\ket{\phi(0)}= \ket{b}\).  Find an expression for the state at some later time \(t > 0\), \(\ket{\phi(t)}\).

\paragraph{A:}

Most of the work is already done.  Computation of \(\ket{\phi(t)} = U(t) \ket{\phi(0)}\) follows from \eqnref{eqn:qmIexamPractice2008Dec:2:100}

\begin{equation}\label{eqn:qmIexamPractice2008Dec:2:110}
\ket{\phi(t)} =
\cos(\delta t/2) \ket{b}
- i \frac{\Omega}{\delta} \sin(\delta t/2) \Bigl( 
\ket{b} +i \ket{a}
\Bigr).
\end{equation}

\paragraph{Q: (c)}

Suppose that an observable, specified by the operator \(X = \ket{a}\bra{b} + \ket{b}\bra{a}\), is measured for this system.  What is the probability that, at time \(t\), the result \(1\) is obtained?  Plot this probability as a function of time, showing the maximum and minimum values of the function, and the corresponding values of \(t\).

\paragraph{A:}

The language of questions like these attempt to bring some physics into the mathematics.  The phrase ``the result \(1\) is obtained'', is really a statement that the operator \(X\), after measurement is found to have the eigenstate with numeric value 1.

We can calculate the eigenvectors for this operator easily enough and find them to be \(\pm 1\).  For the positive eigenvalue we can also compute the eigenstate to be

\begin{equation}\label{eqn:qmIexamPractice2008Dec:2:120}
\ket{X+} = \inv{\sqrt{2}} \Bigl( \ket{a} + \ket{b} \Bigr).
\end{equation}

The question of what the probability for this measurement is then really a question asking for the computation of the amplitude

\begin{equation}\label{eqn:qmIexamPractice2008Dec:2:130}
\Abs{
\inv{\sqrt{2}}
\braket{
 (a + b)}{\phi(t)}}^2
\end{equation}

From \eqnref{eqn:qmIexamPractice2008Dec:2:110} we find this probability to be

\begin{equation}\label{eqn:qmIexamPractice:710}
\begin{aligned}
\Abs{
\inv{\sqrt{2}}
\braket{
 (a + b)}{\phi(t)}}^2
&=
\inv{2} \left(
\left(\cos(\delta t/2) + \frac{\Omega}{\delta} \sin(\delta t/2)\right)^2
+ \frac{ \Omega^2 \sin^2(\delta t/2)}{\delta^2}
\right) \\
&=
\inv{4} \left( 1 + 3 \frac{\Omega^2}{\delta^2} + \frac{\Delta^2}{\delta^2} \cos (\delta t) + 2 \frac{ \Omega}{\delta} \sin(\delta t) \right)
\end{aligned}
\end{equation}

We have a simple superposition of two sinusoids out of phase, periodic with period \(2 \pi/\delta\).  I had attempted a rough sketch of this on paper, but will not bother scanning it here or describing it further.

\paragraph{Q: (d)}

Suppose an experimenter has control over the values of the parameters \(\Delta\) and \(\Omega\).  Explain how she might prepare the state \((\ket{a} + \ket{b})/\sqrt{2}\).

\paragraph{A:}

For this part of the question I was not sure what approach to take.  I thought perhaps this linear combination of states could be made to equal one of the energy eigenstates, and if one could prepare the system in that state, then for certain values of \(\delta\) and \(\Delta\) one would then have this desired state.

To get there I note that we can express the states \(\ket{a}\), and \(\ket{b}\) in terms of the eigenstates by inverting

\begin{equation}\label{eqn:qmIexamPractice2008Dec:2:150}
\begin{bmatrix}
\ket{+} \\
\ket{-} \\
\end{bmatrix}
=\inv{\sqrt{2\delta}}
\begin{bmatrix}
\frac{i \Omega}{\sqrt{\delta + \Delta}} & \sqrt{\delta + \Delta} \\
\frac{i \Omega}{\sqrt{\delta - \Delta}} & -\sqrt{\delta - \Delta}
\end{bmatrix}
\begin{bmatrix}
\ket{a} \\
\ket{b} \\
\end{bmatrix}.
\end{equation}

Skipping all the algebra one finds

\begin{equation}\label{eqn:qmIexamPractice2008Dec:2:160}
\begin{bmatrix}
\ket{a} \\
\ket{b} \\
\end{bmatrix}
=
\begin{bmatrix}
-i\sqrt{\delta - \Delta} & -i\sqrt{\delta + \Delta} \\
\frac{\Omega}{\sqrt{\delta - \Delta}} &
-\frac{\Omega}{\sqrt{\delta + \Delta}} 
\end{bmatrix}
\begin{bmatrix}
\ket{+} \\
\ket{-} \\
\end{bmatrix}.
\end{equation}

Unfortunately, this does not seem helpful.  I find

\begin{equation}\label{eqn:qmIexamPractice2008Dec:2:170}
\inv{\sqrt{2}} ( \ket{a} + \ket{b} ) = 
\frac{\ket{+}}{\sqrt{\delta - \Delta}}( \Omega - i (\delta - \Delta) )
-\frac{\ket{-}}{\sqrt{\delta + \Delta}}( \Omega + i (\delta + \Delta) )
\end{equation}

There is no obvious way to pick \(\Omega\) and \(\Delta\) to leave just \(\ket{+}\) or \(\ket{-}\).  When I did this on paper originally I got a different answer for this sum, but looking at it now, I can not see how I managed to get that answer (it had no factors of \(i\) in the result as the one above does).

\paragraph{A physical system for this Hamiltonian}

I wondered what physical system such a Hamiltonian would correspond to, and noted that this bore some similarity to the up vs. down states of the Ammonia atom as discussed in \citep{feynman1963flp}.  In that text the Hamiltonian is reasoned to have the form

\begin{equation}\label{eqn:qmIexamPractice2008Dec:2:180}
H = E_0 ( {\lvert {b} \rangle}{\langle {b} \rvert}+ {\lvert {a} \rangle}{\langle {a} \rvert})- A( {\lvert {a} \rangle}{\langle {b} \rvert}+ {\lvert {b} \rangle}{\langle {a} \rvert}).
\end{equation}

In Feynman's treatment, the Hamiltonian is just specified by giving values to \(H_{ij}\), but the expression can easily seen to be equivalent.  While these do not look equivalent on the surface, they both have the same diagonalization, which allows us to give a physical interpretation to this sort of problem (one which is recurrant in the old QMI exams).

\subsection{3.  One dimensional harmonic oscillator}

Consider a one-dimensional harmonic oscillator with the Hamiltonian

\begin{equation}\label{eqn:qmIexamPractice2008Dec:3:11}
H = \inv{2m}P^2 + \inv{2} m \omega^2 X^2
\end{equation}

Denote the ground state of the system by \(\ket{0}\), the first excited state by \(\ket{1}\) and so on.

\paragraph{Q: (a)}
Evaluate \(\bra{n} X \ket{n}\) and \(\bra{n} X^2 \ket{n}\) for arbitrary \(\ket{n}\).

\paragraph{A:}

Writing \(X\) in terms of the raising and lowering operators we have

\begin{equation}\label{eqn:qmIexamPractice2008Dec:3:100}
X = \frac{\alpha}{\sqrt{2}} (a^\dagger + a),
\end{equation}

so \(\expectation{X}\) is proportional to 

\begin{equation}\label{eqn:qmIexamPractice2008Dec:3:110}
\bra{n} a^\dagger + a \ket{n} = \sqrt{n+1} \braket{n}{n+1} + \sqrt{n} \braket{n}{n-1} = 0.
\end{equation}

For \(\expectation{X^2}\) we have

\begin{equation}\label{eqn:qmIexamPractice:730}
\begin{aligned}
\expectation{X^2}
&=
\frac{\alpha^2}{2}
\bra{n} (a^\dagger + a)(a^\dagger + a) \ket{n} \\
&=
\frac{\alpha^2}{2}
\bra{n} (a^\dagger + a) \left( 
\sqrt{n+1} \ket{n+1} + \sqrt{n-1} \ket{n-1}
\right)  \\
&=
\frac{\alpha^2}{2}
\bra{n} 
\Bigl( (n+1) \ket{n} + \sqrt{n(n-1)} \ket{n-2}
+ \sqrt{(n+1)(n+2)} \ket{n+2} + n \ket{n} \Bigr).
\end{aligned}
\end{equation}

We are left with just

\begin{equation}\label{eqn:qmIexamPractice2008Dec:3:140}
\expectation{X^2} = \frac{\Hbar}{2 m \omega} (2n + 1).
\end{equation}

\paragraph{Q: (b)}

Suppose that at \(t=0\) the system is prepared in the state

\begin{equation}\label{eqn:qmIexamPractice2008Dec:3:20}
\ket{\psi(0)} = \inv{\sqrt{2}} ( \ket{0} + i \ket{1} ).
\end{equation}

If a measurement of position \(X\) were performed immediately, sketch the probability distribution \(P(x)\) that a particle would be found within \(dx\) of \(x\).  Justify how you construct the sketch.

\paragraph{A:}

The probability that we started in state \(\ket{\psi(0)}\) and ended up in position \(x\) is governed by the amplitude \(\braket{x}{\psi(0)}\), and the probability of being within an interval \(\Delta x\), surrounding the point \(x\) is given by

\begin{equation}\label{eqn:qmIexamPractice2008Dec:3:200}
\int_{x'=x-\Delta x/2}^{x+\Delta x/2} \Abs{ \braket{x'}{\psi(0)} }^2 dx'.
\end{equation}

In the limit as \(\Delta x \rightarrow 0\), this is just the squared amplitude itself evaluated at the point \(x\), so we are interested in the quantity 

\begin{equation}\label{eqn:qmIexamPractice2008Dec:3:210}
\Abs{ \braket{x}{\psi(0)} }^2  = \inv{2} \Abs{ \braket{x}{0} + i \braket{x}{1} }^2.
\end{equation}

We are given these wave functions in the supplemental formulas.  Namely,

\begin{equation}\label{eqn:qmIexamPractice2008Dec:3:220}
\begin{aligned}
\braket{x}{0} &= \psi_0(x) = \frac{e^{-x^2/2\alpha^2}}{ \sqrt{\alpha \sqrt{\pi}}} \\
\braket{x}{1} &= \psi_1(x) = \frac{e^{-x^2/2\alpha^2} 2 x }{ \alpha \sqrt{2 \alpha \sqrt{\pi}}}.
\end{aligned}
\end{equation}

Substituting these into \eqnref{eqn:qmIexamPractice2008Dec:3:210} we have

\begin{equation}\label{eqn:qmIexamPractice2008Dec:3:230}
\Abs{ \braket{x}{\psi(0)} }^2 
= 
\inv{2} 
e^{-x^2/\alpha^2}
\inv{ 
\alpha \sqrt{\pi}}
\Abs{ 1 + \frac{2 i x}{\alpha \sqrt{2} } }^2
=
\frac{e^{-x^2/\alpha^2}}{ 2
\alpha \sqrt{\pi}}
\left( 1 + \frac{2 x^2}{\alpha^2 } \right).
\end{equation}

This \href{http://www.wolframalpha.com/input/?i=graph+e^(-x^2)+(1+%2B+2x^2)}{is parabolic near the origin and then quickly tapers off}.

\paragraph{Q: (c)}

Now suppose the state given in (b) above were allowed to evolve for a time \(t\), determine the expectation value of \(X\) and \(\Delta X\) at that time.

\paragraph{A:}

Our time evolved state is

\begin{equation}\label{eqn:qmIexamPractice2008Dec:3:300}
U(t) \ket{\psi(0)} = \inv{\sqrt{2}}
\left(
e^{-i \Hbar \omega \left( 0 + \inv{2} \right) t/\Hbar } \ket{0}
+ i e^{-i \Hbar \omega \left( 1 + \inv{2} \right) t/\Hbar } \ket{0}
\right)
=
\inv{\sqrt{2}}
\left(
e^{-i \omega t/2 } \ket{0}
+ i e^{- 3 i \omega t/2 } \ket{1}
\right).
\end{equation}

The position expectation is therefore

\begin{equation}\label{eqn:qmIexamPractice:750}
\begin{aligned}
\bra{\psi(t)} X \ket{\psi(t)}
&= 
\frac{\alpha}{2 \sqrt{2}}
\left(
e^{i \omega t/2 } \bra{0}
- i e^{ 3 i \omega t/2 } \bra{1}
\right)
(a^\dagger + a)
\left(
e^{-i \omega t/2 } \ket{0}
+ i e^{- 3 i \omega t/2 } \ket{1}
\right) \\
\end{aligned}
\end{equation}

We have already demonstrated that \(\bra{n} X \ket{n} = 0\), so we must only expand the cross terms, but those are just \(\bra{0} a^\dagger + a \ket{1} = 1\).  This leaves 

\begin{equation}\label{eqn:qmIexamPractice2008Dec:3:310}
\bra{\psi(t)} X \ket{\psi(t)}
= 
\frac{\alpha}{2 \sqrt{2}}
\left( -i e^{i \omega t} + i e^{-i \omega t} \right)
=
\sqrt{\frac{\Hbar}{2 m \omega}} \cos(\omega t)
\end{equation}

For the squared position expectation

\begin{equation}\label{eqn:qmIexamPractice:770}
\begin{aligned}
\bra{\psi(t)} X^2 \ket{\psi(t)}
&= 
\frac{\alpha^2}{4 (2)}
\left(
e^{i \omega t/2 } \bra{0}
- i e^{ 3 i \omega t/2 } \bra{1}
\right)
(a^\dagger + a)^2
\left(
e^{-i \omega t/2 } \ket{0}
+ i e^{- 3 i \omega t/2 } \ket{1}
\right) \\
&=
\inv{2} ( \bra{0} X^2 \ket{0} + \bra{1} X^2 \ket{1} )
+ i \frac{\alpha^2 }{8} ( 
- e^{ i \omega t} \bra{1} (a^\dagger + a)^2 \ket{0}
+ e^{ -i \omega t} \bra{0} (a^\dagger + a)^2 \ket{1}
)
\end{aligned}
\end{equation}

Noting that \((a^\dagger + a) \ket{0} = \ket{1}\), and \((a^\dagger + a)^2 \ket{0} = (a^\dagger + a)\ket{1} = \sqrt{2} \ket{2} + \ket{0}\), so we see the last two terms are zero.  The first two we can evaluate using our previous result \eqnref{eqn:qmIexamPractice2008Dec:3:140} which was \(\expectation{X^2} = \frac{\alpha^2}{2} (2n + 1)\).  This leaves

\begin{equation}\label{eqn:qmIexamPractice2008Dec:3:330}
\bra{\psi(t)} X^2 \ket{\psi(t)} = \alpha^2 
\end{equation}

Since \(\expectation{X}^2 = \alpha^2 \cos^2(\omega t)/2\), we have

\begin{equation}\label{eqn:qmIexamPractice2008Dec:3:340}
(\Delta X)^2 = \expectation{X^2} - \expectation{X}^2 = \alpha^2 \left(1 - \inv{2} \cos^2(\omega t) \right)
\end{equation}

\paragraph{Q: (d)}

Now suppose that initially the system were prepared in the ground state \(\ket{0}\), and then the resonance frequency is changed abruptly from \(\omega\) to \(\omega'\) so that the Hamiltonian becomes 

\begin{equation}\label{eqn:qmIexamPractice2008Dec:3:10}
H = \inv{2m}P^2 + \inv{2} m {\omega'}^2 X^2.
\end{equation}

Immediately, an energy measurement is performed ; what is the probability of obtaining the result \(E = \Hbar \omega' (3/2)\)?

\paragraph{A:}

This energy measurement \(E = \Hbar \omega' (3/2) = \Hbar \omega' (1 + 1/2)\), corresponds to an observation of state \(\ket{1'}\), after an initial observation of \(\ket{0}\).  The probability of such a measurement is

\begin{equation}\label{eqn:qmIexamPractice2008Dec:3:400}
\Abs{ \braket{1'}{0} }^2
\end{equation}

Note that 

\begin{equation}\label{eqn:qmIexamPractice:790}
\begin{aligned}
\braket{1'}{0} 
&=
\int dx \braket{1'}{x}\braket{x}{0} \\
&=
\int dx \psi_{1'}^\conj \psi_0(x) \\
\end{aligned}
\end{equation}

The wave functions above are
\begin{equation}\label{eqn:qmIexamPractice2008Dec:3:410}
\begin{aligned}
\phi_{1'}(x) &= \frac{ 2 x e^{-x^2/2 {\alpha'}^2 }}{ \alpha' \sqrt{ 2 \alpha' \sqrt{\pi} } } \\
\phi_{0}(x) &= \frac{ e^{-x^2/2 {\alpha}^2 } } { \sqrt{ \alpha \sqrt{\pi} } } 
\end{aligned}
\end{equation}

Putting the pieces together we have

\begin{equation}\label{eqn:qmIexamPractice2008Dec:3:810}
\begin{aligned}
\braket{1'}{0} 
&=
\frac{2 }{ \alpha' \sqrt{ 2 \alpha' \alpha \pi } }
\int dx
x e^{-\frac{x^2}{2}\left( \inv{{\alpha'}^2} + \inv{\alpha^2} \right) }
\end{aligned}
\end{equation}

Since this is an odd integral kernel over an even range, this evaluates to zero, and we conclude that the probability of measuring the specified energy is zero when the system is initially prepared in the ground state associated with the original Hamiltonian.  Intuitively this makes some sense, if one thinks of the Fourier coefficient problem: one cannot construct an even function from linear combinations of purely odd functions.
