%
% Copyright � 2012 Peeter Joot.  All Rights Reserved.
% Licenced as described in the file LICENSE under the root directory of this GIT repository.
%

%\chapter{PHY356 Problem Set 5}
\label{chap:qmIproblemSet5}
%\blogpage{http://sites.google.com/site/peeterjoot/math2010/qmIproblemSet5.pdf}
%\date{Nov 25, 2010}

\makeproblem{ps V. p1.}{problem:qmIproblemSet5:1}{

A particle of mass m moves along the x-direction such that \(V(X)=\inv{2}KX^2\). Is the state 

\begin{equation}\label{eqn:qmIproblemSet5:5}
u(\xi) = B \xi e^{+\xi^2/2},
\end{equation}

where \(\xi\) is given by Eq. (9.60), \(B\) is a constant, and time \(t=0\), an energy eigenstate of the system?  What is probability per unit length for measuring the particle at position \(x=0\) at \(t=t_0>0\)?  Explain the physical meaning of the above results.

} % problem

\makeanswer{problem:qmIproblemSet5:1}{
\paragraph{Is this state an energy eigenstate?}

Recall that \(\xi = \alpha x\), \(\alpha = \sqrt{m\omega/\Hbar}\), and \(K = m \omega^2\).  With this variable substitution Schr\"{o}dinger's equation for this harmonic oscillator potential takes the form

\begin{equation}\label{eqn:qmIproblemSet5:10}
\frac{d^2 u}{d\xi^2} - \xi^2 u = -\frac{2 E }{\Hbar\omega} u
\end{equation}

While we can blindly substitute a function of the form \(\xi e^{\xi^2/2}\) into this to get

\begin{equation}\label{eqn:qmIproblemSet5:530}
\begin{aligned}
\inv{B} \left(\frac{d^2 u}{d\xi^2} - \xi^2 u\right)
&=
\frac{d}{d\xi} \left( 1 + \xi^2 \right) e^{\xi^2/2} - \xi^3 e^{\xi^2/2} \\
&=
\left( 2 \xi + \xi + \xi^3 \right) e^{\xi^2/2} - \xi^3 e^{\xi^2/2} \\
&=
3 \xi e^{\xi^2/2}
\end{aligned}
\end{equation}

and formally make the identification \(E = -3 \omega \Hbar/2 = -(1 + 1/2) \omega \Hbar\), this is not a normalizable wavefunction, and has no physical relevance, unless we set \(B = 0\).

By changing the problem, this state could be physically relevant.  We would require a potential of the form

\begin{equation}\label{eqn:qmIproblemSet5:11}
V(x) =
\left\{
\begin{array}{l l}
f(x) & \quad \mbox{if \(x < a\)} \\
\inv{2} K x^2 & \quad \mbox{if \(a < x < b\)} \\
g(x) & \quad \mbox{if \(x > b\)} \\
\end{array}
\right.
\end{equation}

For example, \(f(x) = V_1, g(x) = V_2\), for constant \(V_1, V_2\).  For such a potential, within the harmonic well, a general solution of the form

\begin{equation}\label{eqn:qmIproblemSet5:19}
u(x,t) = \sum_n H_n(\xi) \Bigl(A_n e^{-\xi^2/2} + B_n e^{\xi^2/2} \Bigr) e^{-i E_n t/\Hbar},
\end{equation}

is possible since normalization would not prohibit non-zero \(B_n\) values in that situation.  For the wave function to be a physically relevant, we require it to be (absolute) square integrable, and must also integrate to unity over the entire interval.

\paragraph{Probability per unit length at \texorpdfstring{\(x=0\)}{x equal zero}}

We cannot answer the question for the probability that the particle is found at the specific \(x=0\) position at \(t=t_0\) (that probability is zero in a continuous space), but we can answer the question for the probability that a particle is found in an interval surrounding a specific point at this time.  By calculating the average of the probability to find the particle in an interval, and dividing by that interval's length, we arrive at plausible definition of probability per unit length for an interval surrounding \(x = x_0\)

\begin{equation}\label{eqn:qmIproblemSet5:17}
P = \text{Probability per unit length near \(x = x_0\)} =
\lim_{\epsilon \rightarrow 0} \inv{\epsilon} \int_{x_0 - \epsilon/2}^{x_0 + \epsilon/2} \Abs{ \Psi(x, t_0) }^2 dx = \Abs{\Psi(x_0, t_0)}^2
\end{equation}

By this definition, the probability per unit length is just the probability density itself, evaluated at the point of interest.

Physically, for an interval small enough that the probability density is constant in magnitude over that interval, this probability per unit length times the length of this small interval, represents the probability that we will find the particle in that interval.

\paragraph{Probability per unit length for the non-normalizable state given}

It seems possible, albeit odd, that this question is asking for the probability per unit length for the non-normalizable \(E_1\) wavefunction \eqnref{eqn:qmIproblemSet5:5}.  Since normalization requires \(B=0\), that probability density is simply zero (or undefined, depending on one's point of view).

\paragraph{Probability per unit length for some more interesting harmonic oscillator states}

Suppose we form the wavefunction for a superposition of all the normalizable states

\begin{equation}\label{eqn:qmIproblemSet5:20}
u(x,t) = \sum_n A_n H_n(\xi) e^{-\xi^2/2} e^{-i E_n t/\Hbar}
\end{equation}

Here it is assumed that the \(A_n\) coefficients yield unit probability

\begin{equation}\label{eqn:qmIproblemSet5:30}
\int \Abs{u(x,0)}^2 dx = \sum_n \Abs{A_n}^2 = 1
\end{equation}

For the impure state of \eqnref{eqn:qmIproblemSet5:20} we have for the probability density

\begin{equation}\label{eqn:qmIproblemSet5:550}
\begin{aligned}
\Abs{u}^2
&=
\sum_{m,n}
A_n A_m^\conj H_n(\xi) H_m(\xi) e^{-\xi^2} e^{-i (E_n - E_m)t_0/\Hbar} \\
&=
\sum_n
\Abs{A_n}^2 (H_n(\xi))^2 e^{-\xi^2}
+\sum_{m \ne n}
A_n A_m^\conj H_n(\xi) H_m(\xi) e^{-\xi^2} e^{-i (E_n - E_m)t_0/\Hbar} \\
&=
\sum_n
\Abs{A_n}^2 (H_n(\xi))^2 e^{-\xi^2}
+\sum_{m \ne n}
A_n A_m^\conj H_n(\xi) H_m(\xi) e^{-\xi^2} e^{-i (E_n - E_m)t_0/\Hbar} \\
&=
\sum_n
\Abs{A_n}^2 (H_n(\xi))^2 e^{-\xi^2} \\
&\quad +\sum_{m < n}
H_n(\xi) H_m(\xi)
\left(
A_n A_m^\conj
e^{-\xi^2} e^{-i (E_n - E_m)t_0/\Hbar}
+A_m A_n^\conj
e^{-\xi^2} e^{-i (E_m - E_n)t_0/\Hbar}
\right) \\
&=
\sum_n
\Abs{A_n}^2 (H_n(\xi))^2 e^{-\xi^2}
+2 \sum_{m < n}
H_n(\xi) H_m(\xi)
e^{-\xi^2}
\Real \left(
A_n A_m^\conj
e^{-i (E_n - E_m)t_0/\Hbar}
\right) \\
&=
\sum_n
\Abs{A_n}^2 (H_n(\xi))^2 e^{-\xi^2}  \\
&\quad+2 \sum_{m < n}
H_n(\xi) H_m(\xi)
e^{-\xi^2}
\left(
\Real ( A_n A_m^\conj ) \cos( (n - m)\omega t_0)
+\Imag ( A_n A_m^\conj ) \sin( (n - m)\omega t_0)
\right) \\
\end{aligned}
\end{equation}

Evaluation at the point \(x = 0\), we have

\begin{equation}\label{eqn:qmIproblemSet5:500}
\begin{aligned}
\Abs{u(0,t_0)}^2
&=
\sum_n
\Abs{A_n}^2 (H_n(0))^2  \\
&\quad +2 \sum_{m < n} H_n(0) H_m(0) \left( \Real ( A_n A_m^\conj ) \cos( (n - m)\omega t_0) +\Imag ( A_n A_m^\conj ) \sin( (n - m)\omega t_0)
\right)
\end{aligned}
\end{equation}

It is interesting that the probability per unit length only has time dependence for a mixed state.

For a pure state and its wavefunction \(u(x,t) = N_n H_n(\xi) e^{-\xi^2/2} e^{-i E_n t/\Hbar}\) we have just
\begin{equation}\label{eqn:qmIproblemSet5:510}
\Abs{u(0,t_0)}^2
=
N_n^2 (H_n(0))^2 = \frac{\alpha}{\sqrt{\pi} 2^n n!} H_n(0)^2
\end{equation}

This is zero for odd \(n\).  For even \(n\) is appears that \((H_n(0))^2\) may equal \(2^n\) (this is true at least up to n=4).  If that is the case, we have for non-mixed states, with even numbered energy quantum numbers, at \(x=0\) a probability per unit length value of \(\Abs{u(0,t_0)}^2 = \frac{\alpha}{\sqrt{\pi} n!}\).

\paragraph{Grading notes}

I lost \(3/10\) marks on this assignment.  Two of these due to a sign error in \eqnref{eqn:qmIproblemSet5:10} (now corrected).

One mark lost for the sign error itself, and one for the conclusion that could have been drawn from the negative energy:

``Without that sign error, \(E - -3 \Hbar \omega < V_{\text{min}} = 0\), so clearly not physical since a particle has to have at least as much energy as the potential.''

It was also pointed out that in the discussion of probability per unit length, the \(B=0\) condition means no wave function, and thus no particle, and that undefined is the way to discuss this since it does not make sense to ask about a probability for this particle.

The last mark lost was due to my explaination associated with the modified potential \eqnref{eqn:qmIproblemSet5:11}.  I did not clearly explain that this modified potential would not have the wave function of \eqnref{eqn:qmIproblemSet5:5} since it must be different outside of the harmonic interval.  What they wanted to see explained is that one must modify the wave function (for example, by introducing a cut off), for it to be normalizable.  In my eyes, it then would not be a solution to the original Hamiltonian equation, so if you want solutions that include both positive and negative coefficients in the exponentials, you would also have to have a modified potential.  Given the sign error that was also made, and the negative energy associated with the wave function \eqnref{eqn:qmIproblemSet5:5} I am not so sure that any modify-the-wave-function argument is even appropriate.

Also note that the question was not asking for elaboration on the "more interesting normalizable states".  Basically, the intent was to to ask for just discussion on the un-normalizable aspects of the proposed wave function as if it was a real one.  That seemed too easy to me (but obviously keeping track of my signs was not too easy).
} % answer
