%
% Copyright © 2012 Peeter Joot.  All Rights Reserved.
% Licenced as described in the file LICENSE under the root directory of this GIT repository.
%

\makeproblem{Free particle propagator (2007 PHY355H1F 1f)}{problem:qmIexamPractice2007Dec:2}{

For a free particle moving in one-dimension, the propagator (i.e. the coordinate representation of the evolution operator),

\begin{equation}\label{eqn:qmIexamPractice2007Dec:1f:10}
G(x,x';t) = \bra{x} U(t) \ket{x'}
\end{equation}

is given by

\begin{equation}\label{eqn:qmIexamPractice2007Dec:1f:20}
G(x,x';t) = \sqrt{\frac{m}{2 \pi i \Hbar t}} e^{i m (x-x')^2/ (2 \Hbar t)}.
\end{equation}
} % problem

% qmIexamPractice
\makeanswer{problem:qmIexamPractice2007Dec:2}{

This problem is actually fairly straightforward, but it is nice to work it having had a similar problem set question where we were asked about this time evolution operator matrix element (ie: what it is physical meaning is).  Here we have a concrete example of the form of this matrix operator.

Proceeding directly, we have

\begin{equation}\label{eqn:qmIexamPractice2007Dec:450}
\begin{aligned}
\bra{x} U \ket{x'}
&=
\int \braket{x}{p'} \bra{p'} U \ket{p} \braket{p}{x'} dp dp' \\
&=
\int u_{p'}(x) \bra{p'} e^{-i P^2 t/(2 m \Hbar)} \ket{p} u_p^\conj(x') dp dp' \\
&=
\int u_{p'}(x) e^{-i p^2 t/(2 m \Hbar)} \delta(p-p') u_p^\conj(x') dp dp' \\
&=
\int u_{p}(x) e^{-i p^2 t/(2 m \Hbar)} u_p^\conj(x') dp \\
&=
\inv{(\sqrt{2 \pi \Hbar})^2} \int e^{i p (x-x')/\Hbar} e^{-i p^2 t/(2 m \Hbar)} dp \\
&=
\inv{2 \pi \Hbar} \int e^{i p (x-x')/\Hbar} e^{-i p^2 t/(2 m \Hbar)} dp \\
&=
\inv{2 \pi} 
\int e^{i k (x-x')} e^{-i \Hbar k^2 t/(2 m)} dk \\
&=
\inv{2 \pi} 
\int dk e^{- \left(k^2 \frac{ i \Hbar t}{2m} - i k (x-x')
\right)
} \\
&=
\inv{2 \pi} 
\int dk e^{- \frac{ i \Hbar t}{2m}\left(k - i \frac{2m}{i \Hbar t}\frac{(x-x')}{2} \right)^2
- \frac{i^2 2 m (x-x')^2}{4 i \Hbar t} 
} \\
&=
\inv{2 \pi}  \sqrt{\pi} \sqrt{\frac{2m}{i \Hbar t}}
e^{\frac{ i m (x-x')^2}{2 \Hbar t}},
\end{aligned}
\end{equation}

which is the desired result.  Now, let us look at how this would be used.  We can express our time evolved state using this matrix element by introducing an identity

\begin{equation}\label{eqn:qmIexamPractice2007Dec:470}
\begin{aligned}
\braket{x}{\psi(t)} 
&=
\bra{x} U \ket{\psi(0)} \\
&=
\int dx' \bra{x} U \ket{x'} \braket{x'}{\psi(0)} \\
&=
\sqrt{\frac{m}{2 \pi i \Hbar t}} 
\int dx' 
e^{i m (x-x')^2/ (2 \Hbar t)}
\braket{x'}{\psi(0)} \\
\end{aligned}
\end{equation}

This gives us
\begin{equation}\label{eqn:qmIexamPractice2007Dec:1f:40}
\psi(x, t)
=
\sqrt{\frac{m}{2 \pi i \Hbar t}} 
\int dx' 
e^{i m (x-x')^2/ (2 \Hbar t)} \psi(x', 0)
\end{equation}

However, note that our free particle wave function at time zero is

\begin{equation}\label{eqn:qmIexamPractice2007Dec:1f:50}
\psi(x, 0) = \frac{e^{i p x/\Hbar}}{\sqrt{2 \pi \Hbar}}
\end{equation}

So the convolution integral \eqnref{eqn:qmIexamPractice2007Dec:1f:40} does not exist.  We likely have to require that the solution be not a pure state, but instead a superposition of a set of continuous states (a wave packet in position or momentum space related by Fourier transforms).  That is

\begin{equation}\label{eqn:qmIexamPractice2007Dec:1f:60}
\begin{aligned}
\psi(x, 0) &= 
\inv{\sqrt{2 \pi \Hbar}} \int \hat{\psi}(p, 0) e^{i p x/\Hbar} dp \\
\hat{\psi}(p, 0) &= 
\inv{\sqrt{2 \pi \Hbar}} \int \psi(x'', 0) e^{-i p x''/\Hbar} dx''
\end{aligned}
\end{equation}

The time evolution of this wave packet is then determined by the propagator, and is

\begin{equation}\label{eqn:qmIexamPractice2007Dec:1f:70}
\psi(x,t) =
\sqrt{\frac{m}{2 \pi i \Hbar t}} 
\inv{\sqrt{2 \pi \Hbar}} 
\int dx' dp
e^{i m (x-x')^2/ (2 \Hbar t)}
\hat{\psi}(p, 0) e^{i p x'/\Hbar} ,
\end{equation}

or in terms of the position space wave packet evaluated at time zero

\begin{equation}\label{eqn:qmIexamPractice2007Dec:1f:80}
\psi(x,t) =
\sqrt{\frac{m}{2 \pi i \Hbar t}}
\inv{2 \pi}
\int dx' dx'' dk
e^{i m (x-x')^2/ (2 \Hbar t)}
e^{i k (x' - x'')} \psi(x'', 0)
\end{equation}

We see that the propagator also ends up with a Fourier transform structure, and we have

\begin{equation}\label{eqn:qmIexamPractice2007Dec:1f:90}
\begin{aligned}
\psi(x,t) &= \int dx' U(x, x' ; t) \psi(x', 0) \\
U(x, x' ; t) &=
\sqrt{\frac{m}{2 \pi i \Hbar t}}
\inv{2 \pi}
\int du dk
e^{i m (x - x' - u)^2/ (2 \Hbar t)}
e^{i k u }
\end{aligned}
\end{equation}

Does that Fourier transform exist?  I had not be surprised if it ended up with a delta function representation.  I will hold off attempting to evaluate and reduce it until another day.

} % answer
