%
% Copyright � 2012 Peeter Joot.  All Rights Reserved.
% Licenced as described in the file LICENSE under the root directory of this GIT repository.
%

%\chapter{PHY356 Problem Set I}
\label{chap:qmIproblemSet1}
%\blogpage{http://sites.google.com/site/peeterjoot/math2010/qmIproblemSet1.pdf}
%\date{Oct 7, 2010}

\makeproblem{problem set 1.}{problem:qmIproblemSet1:1}{

\index{commutator}
\index{operator!position}
\index{operator!momentum}
Assume that \(X\) and \(P = -i \Hbar \PDi{x}{}\) are the x-direction position and momentum operators. Show that \(\antisymmetric{X}{P}=i\Hbar \BOne\). Find \(\bra{x}(XP-PX)\ket{x'}\) using the above definitions. What is the physical meaning of this expression?

} % problem

\makeanswer{problem:qmIproblemSet1:1}{
\paragraph{Avoiding Dirac notation}

We can get a rough idea where we are going by temporarily avoiding the Dirac notation that complicates things.  To do so, consider the commutator action on an arbitrary wave function \(\psi(x)\)

\begin{equation}\label{eqn:qmIproblemSet1:28}
\begin{aligned}
(x P - P x)\psi
&=
x P \psi + i\Hbar \PD{x}{} (x \psi) \\
&=
x P \psi + i\Hbar \left(\psi + \PD{x}{\psi} \right) \\
&=
x P \psi + i \Hbar \psi - x P \psi \\
&=
i \Hbar \psi
\end{aligned}
\end{equation}

Since this is true for all \(\psi(x)\) we can make the identification
\begin{equation}\label{eqn:qmIproblemSet1:1}
x P - P x = i\Hbar \BOne
\end{equation}

Having evaluated the commutator, the matrix element is simple to compute.  It is

\begin{equation}\label{eqn:qmIproblemSet1:48}
\begin{aligned}
\bra{x} XP - PX \ket{x'}
&=
\bra{x} i \Hbar \BOne \ket{x'} \\
&=
i \Hbar \braket{x}{x'}.
\end{aligned}
\end{equation}

This braket has a delta function action, so this matrix element reduces to

\begin{equation}\label{eqn:qmIproblemSet1:2}
\bra{x} XP - PX \ket{x'}
=
i \Hbar \delta(x - x').
\end{equation}

This could perhaps be considered the end of the problem (barring the physical meaning interpretation requirement to come).  However, given that the Dirac notation that is so central to the lecture notes and course text, it seems like cheating to avoid it.  It seems reasonable to follow this up with the same procedure utilizing the trickier Dirac notation, and this will be done next.  If nothing else, this should provide some experience with what sort of manipulations are allowed.

\paragraph{Using Dirac notation}

Intuition says that we need to consider the action of the commutator within a matrix element of the form

\begin{equation}\label{eqn:qmIproblemSet1:4}
\bra{x} XP - PX \ket{\psi} = \int dx' \bra{x} XP - PX \ket{x'} \braket{x'}{\psi} = \int dx' \bra{x} XP - PX \ket{x'} \psi(x').
\end{equation}

Observe above that with the introduction of an identity operation, such an expression also includes the matrix element to be evaluated in the second part of this problem.  Because of this, if we can show that \(\bra{x} XP - PX \ket{\psi} = i\Hbar \psi(x)\), then as a side effect we will also have shown that the matrix element \(\bra{x} XP - PX \ket{x'} = i\Hbar\delta(x-x')\), as well as demonstrated the commutator relation \(XP - PX = i\Hbar \BOne\).

Proceeding with a reduction of the right most integral in \eqnref{eqn:qmIproblemSet1:4} above, we have

\begin{equation}\label{eqn:qmIproblemSet1:68}
\begin{aligned}
\int dx' \bra{x} XP - PX \ket{x'} \psi(x')
&=
\int dx' \bra{x} x P - P x' \ket{x'} \psi(x') \\
&=
\int dx' \bra{x} x P \psi(x') - P x' \psi(x') \ket{x'} \\
&=
-i\Hbar
\int dx' \bra{x} x \PD{x}{\psi(x')} - \PD{x}{}( x' \psi(x')) \ket{x'} \\
&=
i\Hbar
\int dx' \bra{x} -x \PD{x}{\psi(x')} + \PD{x}{x'} \psi(x') + x' \PD{x}{\psi(x')} \ket{x'} \\
&=
i\Hbar
\int dx' \left(-x \PD{x}{\psi(x')} + \PD{x}{x'} \psi(x') + x' \PD{x}{\psi(x')}\right) \braket{x}{x'} \\
&=
i\Hbar
\int dx' \left(-x \PD{x}{\psi(x')} + \PD{x}{x'} \psi(x') + x' \PD{x}{\psi(x')}\right) \delta(x-x') \\
&=
\left.
\left( i \Hbar \PD{x}{x'} \psi(x') + i \Hbar (x' - x )\PD{x}{\psi(x')} \right)
\right\vert_{x'=x} \\
&=
i\Hbar \PD{x}{x} \psi(x) + i \Hbar (x - x )\PD{x}{\psi(x)}  \\
&=
i\Hbar \psi(x)
\end{aligned}
\end{equation}

The convolution with the delta function leaves us with only functions of \(x\), allowing all the derivatives to be evaluated.  In the manipulations above the wave function \(\psi(x')\) could be brought into the braket since it is just a (complex) scalar.  What was a bit sneaky, is the restriction of the action of the operator \(P\) to \(\psi(x')\), and \(x'\psi(x')\), but not to \(\ket{x'}\).  That was a key step in the reduction since it allows all the resulting terms to be brought out of the braket, leaving the delta function.

What is a good justification for not allowing \(P\) to act on the ket?  A pragmatic one is that the desired result would not have been obtained otherwise.  After the fact I also see that this is consistent with \citep{wiki:braketNotation}, which states (without citation) that \(-i\Hbar \spacegrad \ket{\psi}\) is an abuse of notation since the operator should be viewed as operating on projections (ie: wave functions).

Another point to follow up on later is the justification for the order of operations.  If the derivatives had been evaluated first before the evaluation at \(x=x'\), then we would have nothing left due to the \(\PDi{x}{x'} = 0\).  Perhaps a good answer for that is that the zero times delta function is not well behaved.  One has to eliminate the delta function first to see if the magnitudes of the zero of that we would have from a pre-evaluated \(\PDi{x}{x'}\) is "more zero", than the infinity of the delta function at \(x = x'\).  This procedure still screams out ad-hoc, and the only real resolution is likely in the framework of distribution theory.

Anyways, assuming the correctness of all the manipulations above, let us return to the problem.  We refer back to \eqnref{eqn:qmIproblemSet1:4} and see that we now have

\begin{equation}\label{eqn:qmIproblemSet1:88}
\begin{aligned}
\bra{x} XP - PX \ket{\psi}
&= i \Hbar \psi(x) \\
&= i \Hbar\braket{x}{\psi} \\
&= \bra{x} i \Hbar \BOne \ket{\psi} \\
\implies \\
0 &= \bra{x} XP - PX - i\Hbar \BOne \ket{\psi}
\end{aligned}
\end{equation}

Since this is true for all \(\bra{x}\), and \(\ket{\psi}\), we must have \(XP - PX = i \Hbar \BOne\) as desired.

Also referring back to \eqnref{eqn:qmIproblemSet1:4} we can write

\begin{equation}\label{eqn:qmIproblemSet1:108}
\begin{aligned}
\int dx' \bra{x} XP - PX \ket{x'} \psi(x')
&=
i\Hbar \psi(x)  \\
&=
\int dx' i\Hbar \delta(x-x') \psi(x').
\end{aligned}
\end{equation}

Taking differences we have for all \(\psi(x')\)

\begin{equation}\label{eqn:qmIproblemSet1:128}
\int dx' \Bigl( \bra{x} XP - PX \ket{x'} -i \Hbar \delta(x-x') \Bigl) \psi(x') = 0,
\end{equation}

which we utilize to produce the identification

\begin{equation}\label{eqn:qmIproblemSet1:5}
\bra{x} XP - PX \ket{x'} = i \Hbar \delta(x-x')
\end{equation}

This completes all the non-interpretation parts of this problem.

%%%%XX
%%%%\paragraph{Attempt 1}
%%%%
%%%%We do so by considering the action of the commutator within a matrix element of the form
%%%%
%%%%\begin{align*}
%%%%\bra{x} XP - PX \ket{\psi}.
%%%%\end{align*}
%%%%
%%%%Considering the \(XP\) part first we have
%%%%
%%%%\begin{align*}
%%%%\bra{x} XP \ket{\psi}
%%%%&=
%%%%\int dx'
%%%%\bra{x} X \ket{x'}\bra{x'} P \ket{\psi} \\
%%%%&=
%%%%\int dx'
%%%%\bra{x} x' \ket{x'}\bra{x'} P \ket{\psi} \\
%%%%&=
%%%%\int dx'
%%%%\braket{x}{x'}\bra{x'} x' P \ket{\psi} \\
%%%%&=
%%%%\int dx'
%%%%\delta(x-x')\bra{x'}x' P \ket{\psi}  \\
%%%%&=
%%%%\bra{x} x P \ket{\psi}.
%%%%\end{align*}
%%%%
%%%%Now consider the \(PX\) part
%%%%
%%%%\begin{align*}
%%%%\bra{x} PX \ket{\psi}
%%%%&=
%%%%\bra{x} \BI PX \ket{\psi} \\
%%%%&=
%%%%\int dx'
%%%%\braket{x}{x'} \bra{x'} P X \ket{\psi}  \\
%%%%&=
%%%%\int dx'
%%%%\braket{x}{x'} \bra{x'} (-i\Hbar)\PD{x}{} x' \ket{\psi}  \\
%%%%&=
%%%%\int dx'
%%%%\delta(x-x') \bra{x'} (-i\Hbar)\PD{x}{} x' \ket{\psi}  \\
%%%%&=
%%%%\bra{x} (-i\Hbar)\PD{x}{} x \ket{\psi}  \\
%%%%&=
%%%%\bra{x} (-i\Hbar) \BOne \ket{\psi} +
%%%%\bra{x} x (-i\Hbar)\PD{x}{} \ket{\psi}  \\
%%%%&=
%%%%\bra{x} -i\Hbar \BOne + x P \ket{\psi}
%%%%\end{align*}
%%%%
%%%%Taking the differences, we have for arbitrary states \(\bra{x}\), and \(\ket{\psi}\)
%%%%
%%%%\begin{align*}
%%%%\bra{x} XP - PX \ket{\psi}
%%%%&=
%%%%\bra{x} x P \ket{\psi} -
%%%%\bra{x} -i\Hbar \BOne + x P \ket{\psi} \\
%%%%&=
%%%%\bra{x} i \Hbar \BOne \ket{\psi}
%%%%\end{align*}
%%%%
%%%%or
%%%%
%%%%\begin{align*}
%%%%\bra{x} XP - PX - i \Hbar \BOne \ket{\psi} = 0
%%%%\end{align*}
%%%%
%%%%Since this is true for all \(\bra{x}\) and \(\ket{\psi}\), we can make the required identification \(XP - PX = i \Hbar \BOne \) as in the wave function approach.
%%%%
%%%%%as required.  This seems like overkill, but does at least produce the expected result.  It also provides another worked example of the fairly tricky seeming Dirac notation.
%%%%
%%%%XX

\paragraph{The physical meaning of this expression}

The remaining part of this question ties the mathematics to some reality.

One nice description of a general matrix element can be found in \citep{ bohrBerkleyPhy221}, where the author states ``We see that the "matrix element" of an operator with respect to a continuous basis is nothing but the kernel of the integral transform that represents the action of that operator in the given basis.''

While that characterizes this sort of continuous matrix element nicely, it does not provide any physical meaning, so we have to look further.

The most immediate observation that we can make of this matrix element is not one that assigns physical meaning, but instead points out a non-physical characteristic.  Note that in the LHS when \(x=x'\) this is an expectation value for the commutator.  Because this expectation ``value'' is purely imaginary (an \(i\Hbar\) scaled delta function, with the delta function presumed to be a positive real infinity), we are able to note that this position momentum commutator operator cannot itself represent an observable.  It must also be non-Hermitian as a consequence, and that is easy enough to verify directly.  Perhaps it would be more interesting to ask the question what the meaning of the matrix element of the Hermitian operator \(-i \antisymmetric{X}{P}\) is?  That operator (an \(\Hbar\) scaled identity) would at least represent an observable.

How about asking the question of what physical meaning we have for a general commutator, before considering the matrix element of such a commutator.  Given two operators \(A\), and \(B\) representing observables, a non-zero commutator \(\antisymmetric{A}{B}\) of these operators means that simultaneous precise measurement of the two observables is not possible.  This property can also be thought of as a meaning for the matrix element \(\bra{x'}\antisymmetric{A}{B} \ket{x}\) of such a commutator.  For the position momentum commutator, this matrix element \(\bra{x} \antisymmetric{X}{P} \ket{x'} = i \Hbar \delta(x-x')\) would also be zero if simultaneous measurement of the operators was possible.

Because this matrix element of this commutator is non-zero (despite the fact that the delta function is zero almost everywhere) we know that a measurement of position will disturb the momentum of the particle, and conversely, a measurement of momentum will disturb the position.  An illustration of this is in the slit diffraction experiment.  Narrowing an initial wide slot to "measure" the position of the photon or electron passing through the slit slit more accurately, has an effect of increasing the scattering range of the particle (ie: reducing the uncertainty in the position measurement imparts momentum in the scattering plane).

%For the position and momentum operators we have
%
%\begin{align}\label{eqn:qmIproblemSet1:6}
%\bra{x} \antisymmetric{X}{P} \ket{x'} = i \Hbar \delta(x-x').
%\end{align}
%
%Are there any other physical meanings that we can give to the matrix element?

%%%%
%%%%Now, what is the physical meaning of this matrix element?  When \(x = x'\) we have an expectation value
%%%%
%%%%\begin{align*}
%%%%\bra{x} XP - PX \ket{x} &= i \Hbar,
%%%%\end{align*}
%%%%
%%%%but this is not real valued, meaning that the commutator is not Hermitian, and therefore not an observable.  Since the question did not ask for the value of the matrix element of the Hermitian operator, \(\antisymmetric{X}{P} = \Hbar\), an \(\Hbar\) scaled identity operator (and an observable), it must be assumed that some other physical meaning is being asked for.
%%%%
%%%%%I am at a loss to assign any further physical meaning to an operator that is not an observable.  The text does not provide any help that I can find.  In fact, there is almost no reference to anything physical so far in the text ... just an awful lot of math!
%%%%
%%%%Is there supposed to be a physical meaning to the matrix element itself?  Again the text is no obvious help.  We have a physical meaning of a matrix element of an operator, only in the context of other questions.  One such question is an operator expectation value, for example for the position operator we have
%%%%
%%%%\begin{align*}
%%%%\expectation{X}
%%%%&=
%%%%\bra{\psi} X \ket{\psi} \\
%%%%&=
%%%%\int dx' dx \braket{\psi}{x'} \bra{x'} X \ket{x} \braket{x}{\psi} \\
%%%%&=
%%%%\int dx \psi^\conj(x) x \psi(x).
%%%%\end{align*}
%%%%
%%%%Here the matrix element \(\bra{x'} X \ket{x}\) shows up as an impulse response like weighting factor in the expectation integral, altering the probability density in the region around \(x = x'\).  It only seems to be when there is additional context do we have a physical meaning to the matrix element itself.  This example also requires the operator in question to be Hermitian, which is not the case for the position-momentum commutator.
%%%%
} % answer

\makeproblem{problem set 1.}{problem:qmIproblemSet1:2}{
The state of a one-dimensional system is given by \(\ket{x_0}\). Does this system obey the position-momentum uncertainty relation? Explain your answer.
\index{uncertainty relation}

} % problem

\makeanswer{problem:qmIproblemSet1:2}{

Yes, the system obeys the position-momentum uncertainty relation.  Note that in long form the uncertainty relation takes the following form:

\begin{equation}\label{eqn:qmIproblemSet1:8}
\begin{aligned}
\sqrt{\expectation{ (X - \expectation{X})^2 }}
\sqrt{\expectation{ (P - \expectation{P})^2 }} \ge \frac{\Hbar}{2}.
\end{aligned}
\end{equation}

Each of these expectation values is with respect to some specific state

\begin{equation}\label{eqn:qmIproblemSet1:7}
\begin{aligned}
\expectation{ X } \equiv \bra{\psi} X \ket{\psi},
\end{aligned}
\end{equation}

so one could write this out in still longer form:

\begin{equation}\label{eqn:qmIproblemSet1:8b}
\sqrt{\bra{\psi}{ (X - \bra{\psi}{X}\ket{\psi})^2 } \ket{\psi} }
\sqrt{\bra{\psi}{ (P - \bra{\psi}{P}\ket{\psi})^2 } \ket{\psi} } \ge \frac{\Hbar}{2}.
\end{equation}

This inequality holds for all states \(\ket{\psi}\) that the system could be observed in.  This includes the state \(\ket{x_0}\) of this problem, associated with a specific observation of the system.

\paragraph{My grade}

I completely misunderstood this, and got only \(0.5/5\) on it.  What he was looking for was that if \(\ket{x_0}\) is a position eigenstate in a continuous vector space, then one cannot form the expectation with respect to this state, let alone the variance.  For example with respect to this state we have

\begin{equation}\label{eqn:qmIproblemSet1:148}
\begin{aligned}
\expectation{X} 
&= \bra{x_0} X \ket{x_0} \\
&= x_0 \braket{x_0}{x_0} \\
&= x_0 \delta(x_0 - x_0)
\end{aligned}
\end{equation}

We cannot evaluate this delta function, since it blows up at zero.  The implication would be that we have complete uncertainty of position in the one dimensional continuous vector space with respect to this state.  Despite bombing on the question, it is a nice one, since it points out some of the implicit assumptions for the uncertainty relation.  We can only say that the uncertainty relation applies with respect to normalizable states.  That said, is it a fair question?  I think the original question was fairly vague, and I would not consider the question well posed.
} % answer
