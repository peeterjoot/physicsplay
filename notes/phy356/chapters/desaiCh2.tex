%
% Copyright � 2012 Peeter Joot.  All Rights Reserved.
% Licenced as described in the file LICENSE under the root directory of this GIT repository.
%

%\chapter{Desai Chapter II notes and problems}
\label{chap:desaiCh2}
%\blogpage{http://sites.google.com/site/peeterjoot/math2010/desaiCh2.pdf}
%\date{Sept 19, 2010}

%\section{Motivation}
%
%Chapter II notes for \citep{desai2009quantum}.
%
\section{Canonical Commutator}

\index{commutator}
Based on the canonical relationship \([X,P] = i\Hbar\), and \(\braket{x'}{x} = \delta(x'-x)\), Desai determines the form of the \(P\) operator in continuous space.  A consequence of this is that the matrix element of the momentum operator is found to have a delta function specification

\begin{equation}\label{eqn:desaiCh2:20}
\bra{x'} P \ket{x} = \delta(x - x') \left( -i \Hbar \frac{d}{dx} \right).
\end{equation}

In particular the matrix element associated with the state \(\ket{\phi}\) is found to be

\begin{equation}\label{eqn:desaiCh2:40}
\bra{x'} P \ket{\phi} = -i \Hbar \frac{d}{dx'} \phi(x').
\end{equation}

Compare this to \citep{liboff2003iqm}, where this last is taken as the definition of the momentum operator, and the relationship to the delta function is not spelled out explicitly.  This canonical commutator approach, while more abstract, seems to have less black magic involved in the setup.  We do require the commutator relationship \([X,P] = i\Hbar\) to be pulled out of a magic hat, but at least the magic show is a structured one based on a small set of core assumptions.

It will likely be good to come back to this later when trying to reconcile this new (for me) Dirac notation with the more basic notation I am already comfortable with.  When trying to compare the two, it will be good to note that there is a matrix element that is implied in the more old fashioned treatment in a book such as \citep{bohm1989qt}.

There is one fundamental assumption that appears to be made in this section that is not justified by anything except the end result.  That is the assumption that \(P\) is a derivative like operator, acting with a product rule action.  That is used to obtain (2.28) and is a fairly black magic operation.  This same assumption, is also hiding, somewhat sneakily, in the manipulation for (2.44).

If one has to make that assumption that \(P\) is a derivative like operator, I do not feel this method of introducing it is any less arbitrary seeming.  It is still pulled out of a magic hat, only because the answer is known ahead of time.  The approach of \citep{bohm1989qt}, where the derivative nature is presented as consequence of transforming (via Fourier transforms) from the position to the momentum representation, seems much more intuitive and less arbitrary.

\section{Generalized momentum commutator}

It is stated that

\begin{equation}\label{eqn:desaiCh2:60}
[P,X^n] = - n i \Hbar X^{n-1}.
\end{equation}

Let us prove this.  The \(n=1\) case is the canonical commutator, which is assumed.  Is there any good way to justify that from first principles, as presented in the text?  We have to prove this for \(n\), given the relationship for \(n-1\).  Expanding the \(n\)th power commutator we have

\begin{equation}\label{eqn:desaiCh2:80}
\begin{aligned}
[P,X^n] 
&= P X^n - X^n P \\
&= P X^{n-1} X - X^{n } P \\
\end{aligned}
\end{equation}

Rearranging the \(n-1\) result we have

\begin{equation}\label{eqn:desaiCh2:100}
P X^{n-1} = X^{n-1} P - (n-1) i \Hbar X^{n-2},
\end{equation}

and can insert that in our \([P,X^n]\) expansion for

\begin{equation}\label{eqn:desaiCh2:120}
\begin{aligned}
[P,X^n] 
&= \left( X^{n-1} P - (n-1) i \Hbar X^{n-2} \right)X - X^{n } P \\
&= X^{n-1} (PX) - (n-1) i \Hbar X^{n-1} - X^{n } P \\
&= X^{n-1} ( X P - i\Hbar) - (n-1) i \Hbar X^{n-1} - X^{n } P \\
&= -X^{n-1} i\Hbar - (n-1) i \Hbar X^{n-1} \\
&= -n i \Hbar X^{n-1} 
\qedmarker
\end{aligned}
\end{equation}

\section{Uncertainty principle}

\index{uncertainty principle}

The origin of the statement \([\Delta A, \Delta B] = [A, B]\) is not something that seemed obvious.  Expanding this out however is straightforward, and clarifies things.  That is

\begin{equation}\label{eqn:desaiCh2:140}
\begin{aligned}
[\Delta A, \Delta B] 
&= (A - \expectation{A}) (B - \expectation{B}) - (B - \expectation{B}) (A - \expectation{A}) \\
&= 
\left( A B - \expectation{A} B - \expectation{B} A +\expectation{A} \expectation{B} \right)
-\left( B A - \expectation{B} A - \expectation{A} B +\expectation{B} \expectation{A} \right) \\
&= 
A B - B A \\
&= 
[A, B]
\qedmarker
\end{aligned}
\end{equation}

\section{Size of a particle}

I found it curious that using \(\Delta x \Delta p \approx \Hbar\) instead of \(\Delta x \Delta p \ge \Hbar/2\), was sufficient to obtain the hydrogen ground state energy \(E_{\text{min}} = -e^2/2 a_0\), without also having to do any factor of two fudging.

\section{Space displacement operator}
\paragraph{Initial notes}
\index{operator!displacement}

I had be curious to know if others find the loose use of equality for approximation after approximation slightly disturbing too?

I also find it curious that (2.140) is written

\begin{equation}\label{eqn:desaiCh2:160}
D(x) = \exp\left( -i \frac{P}{\Hbar} x \right),
\end{equation}

and not
\begin{equation}\label{eqn:desaiCh2:180}
D(x) = \exp\left( -i x \frac{P}{\Hbar} \right).
\end{equation}

Is this intentional?  It does not seem like \(P\) ought to be acting on \(x\) in this case, so why order the terms that way?

Expanding the application of this operator, or at least its first order Taylor series, is helpful to get an idea about this.  Doing so, with the original \(\Delta x'\) value used in the derivation of the text we have to start

\begin{equation}\label{eqn:desaiCh2:200}
\begin{aligned}
D(\Delta x') \ket{\phi} 
&\approx \left(1 - i \frac{P}{\Hbar} \Delta x' \right) \ket{\phi} \\
&= \left(1 - i \left( -i \Hbar \delta(x -x') \frac{\partial}{\partial x} \right) \inv{\Hbar} \Delta x'\right) \ket{\phi} \\
\end{aligned}
\end{equation}

This shows that the \(\Delta x\) factor can be commuted with the momentum operator, as it is not a function of \(x'\), so the question of \(P x\), vs \(x P\) above appears to be a non-issue.

Regardless of that conclusion, it seems worthy to continue an attempt at expanding this shift operator action on the state vector.  Let us do so, but do so by computing the matrix element \(\bra{x'} D(\Delta x') \ket{\phi}\).  That is

\begin{equation}\label{eqn:desaiCh2:220}
\begin{aligned}
\bra{x'} D(\Delta x') \ket{\phi} 
&\approx
\braket{x'}{\phi} - \bra{x'} \delta(x -x') \frac{\partial}{\partial x} \Delta x' \ket{\phi} \\
&=
\phi(x') - \int \bra{x'} \delta(x -x') \frac{\partial}{\partial x} \Delta x' \ket{x'} \braket{x'}{\phi} dx' \\
&=
\phi(x') - \Delta x' \int \delta(x -x') \frac{\partial}{\partial x} \braket{x'}{\phi} dx' \\
&=
\phi(x') - \Delta x' \frac{\partial}{\partial x'} \braket{x'}{\phi} \\
&=
\phi(x') - \Delta x' \frac{\partial}{\partial x'} \phi(x') \\
\end{aligned}
\end{equation}

This is consistent with the text.  It is interesting, and initially surprising that the space displacement operator when applied to a state vector introduces a negative shift in the wave function associated with that state vector.  In the derivation of the text, this was associated with the use of integration by parts (ie: due to the sign change in that integration).  Here we see it sneak back in, due to the \(i^2\) once the momentum operator is expanded completely.

As last note and question.  The first order Taylor approximation of the momentum operator was used.  If the higher order terms are retained, as in

\begin{equation}\label{eqn:desaiCh2:240}
\begin{aligned}
\exp\left( -i \Delta x' \frac{P}{\Hbar} \right) = 
1 - \Delta x' \delta(x -x') \frac{\partial}{\partial x} + 
\inv{2} \left( - \Delta x' \delta(x -x') \frac{\partial}{\partial x} \right)^2 + \cdots,
\end{aligned}
\end{equation}

then how does one evaluate a squared delta function (or Nth power)?

Talked to Vatche about this after class.  The key to this is sequential evaluation.  Considering the simple case for \(P^2\), we evaluate one operator at a time, and never actually square the delta function

\begin{equation}\label{eqn:desaiCh2:260}
\bra{x'} P^2 \ket{\phi} 
%&= \bra{x'} P (P \ket{\phi}) \\
%&= -i \Hbar \int dx' \bra{x'} P (\delta(x-x') \PD{x}{} \ket{x'} \braket{x'}{\phi}) \\
%&= -i \Hbar \bra{x'} P \PD{x'}{} \ket{x'} \braket{x'}{\phi}) \\
\end{equation}

I was also questioned why I was including the delta function at this point.  Why would I do that.  Thinking further on this, I see that is not a reasonable thing to do.  That delta function only comes into the mix when one takes the matrix element of the momentum operator as in

\begin{equation}\label{eqn:desaiCh2:280}
\bra{x'} P \ket{x} = -i \Hbar \delta(x-x') \frac{d}{dx'}. 
\end{equation}

This is very much like the fact that the delta function only shows up in the continuous representation in other context where one has matrix elements.  The most simple example of which is just

\begin{equation}\label{eqn:desaiCh2:300}
\braket{x'}{x} = \delta(x-x').
\end{equation}

I also see now that the momentum operator is directly identified with the derivative (no delta function) in two other places in the text.  These are equations (2.32) and (2.46) respectively:

\begin{equation}\label{eqn:desaiCh2:320}
\begin{aligned}
P(x) &= -i \Hbar \frac{d}{dx} \\
P &= -i \Hbar \frac{d}{dX}.
\end{aligned}
\end{equation}

In the first, (2.32), I thought the \(P(x)\) was somehow different, just a helpful expression found along the way, but now it occurs to me that this was intended to be an unambiguous representation of the momentum operator itself.

\paragraph{A second try}

Getting a feel for this Dirac notation takes a bit of adjustment.  Let us try evaluating the matrix element for the space displacement operator again, without abusing the notation, or thinking that we have a requirement for squared delta functions and other weirdness.  We start with

\begin{equation}\label{eqn:desaiCh2:340}
\begin{aligned}
D(\Delta x') \ket{\phi}
&=
e^{-\frac{i P \Delta x'}{\Hbar}} \ket{\phi} \\
&=
\int dx e^{-\frac{i P \Delta x'}{\Hbar}} \ket{x}\braket{x}{\phi} \\
&=
\int dx e^{-\frac{i P \Delta x'}{\Hbar}} \ket{x} \phi(x).
\end{aligned}
\end{equation}

Now, to evaluate \(e^{-\frac{i P \Delta x'}{\Hbar}} \ket{x}\), we can expand in series

\begin{equation}\label{eqn:desaiCh2:360}
\begin{aligned}
e^{-\frac{i P \Delta x'}{\Hbar}} \ket{x}
&=
\ket{x} + \sum_{k=1}^\infty \inv{k!} \left( \frac{-i \Delta x'}{\Hbar} \right)^k P^k \ket{x}.
\end{aligned}
\end{equation}

It is tempting to left multiply by \(\bra{x'}\) and commute that past the \(P^k\), then write \(P^k = -i \Hbar d/dx\).  That probably produces the correct result, but is abusive of the notation.  We can still left multiply by \(\bra{x'}\), but to be proper, I think we have to leave that on the left of the \(P^k\) operator.  This yields

\begin{equation}\label{eqn:desaiCh2:380}
\begin{aligned}
\bra{x'} D(\Delta x') \ket{\phi}
&=
\int dx \left( \braket{x'}{x} + 
\sum_{k=1}^\infty \inv{k!} \left( \frac{-i \Delta x'}{\Hbar} \right)^k \bra{x'} P^k \ket{x}
\right) \phi(x) \\
&=
\int dx \delta(x'- x) \phi(x)
+\sum_{k=1}^\infty \inv{k!} \left( \frac{-i \Delta x'}{\Hbar} \right)^k \int dx \bra{x'} P^k \ket{x} \phi(x).
\end{aligned}
\end{equation}

The first integral is just \(\phi(x')\), and we are left with integrating the higher power momentum matrix elements, applied to the wave function \(\phi(x)\).  We can proceed iteratively to expand those integrals

\begin{equation}\label{eqn:desaiCh2:400}
\int dx \bra{x'} P^k \ket{x} \phi(x)
= \iint dx dx'' \bra{x'} P^{k-1} \ket{x''} \bra{x''} P \ket{x} \phi(x) 
\end{equation}

Now we have a matrix element that we know what to do with.  Namely, \(\bra{x''} P \ket{x} = -i \Hbar \delta(x''-x) \PDi{x}{}\), which yields

\begin{equation}\label{eqn:desaiCh2:420}
\begin{aligned}
\int dx \bra{x'} P^k \ket{x} \phi(x)
&= 
-i \Hbar \iint dx dx'' \bra{x'} P^{k-1} \ket{x''} \delta(x''-x) \PD{x}{} \phi(x) \\
&= 
-i \Hbar \int dx \bra{x'} P^{k-1} \ket{x} \PD{x}{\phi(x)}.
\end{aligned}
\end{equation}

Each similar application of the identity operator brings down another \(-i\Hbar\) and derivative yielding

\begin{equation}\label{eqn:desaiCh2:440}
\int dx \bra{x'} P^k \ket{x} \phi(x)
= 
(-i \Hbar)^k \frac{\partial^k \phi(x')}{\partial {x'}^k}.
\end{equation}

Going back to our displacement operator matrix element, we now have
\begin{equation}\label{eqn:desaiCh2:460}
\begin{aligned}
\bra{x'} D(\Delta x') \ket{\phi}
&=
\phi(x')
+\sum_{k=1}^\infty \inv{k!} \left( \frac{-i \Delta x'}{\Hbar} \right)^k 
(-i \Hbar)^k \frac{\partial^k \phi(x')}{\partial {x'}^k} \\
&=
\phi(x') +\sum_{k=1}^\infty \inv{k!} \left( - \Delta x' \frac{\partial }{\partial x'} \right)^k  \phi(x') \\
&= \phi(x' - \Delta x').
\end{aligned}
\end{equation}

This shows nicely why the sign goes negative and it is no longer surprising when one observes that this can be obtained directly by using the adjoint relationship

\begin{equation}\label{eqn:desaiCh2:480}
\begin{aligned}
\bra{x'} D(\Delta x') \ket{\phi}
&=
(D^\dagger(\Delta x') \ket{x'})^\dagger \ket{\phi} \\
&=
(D(-\Delta x') \ket{x'})^\dagger \ket{\phi} \\
&=
\ket{x' - \Delta x'}^\dagger \ket{\phi} \\
&=
\braket{x' - \Delta x'}{\phi} \\
&=
\phi(x' - \Delta x')
\end{aligned}
\end{equation}

That is a whole lot easier than the integral manipulation, but at least shows that we now have a feel for the notation, and have confirmed the exponential formulation of the operator nicely.

\section{Time evolution operator}

The phrase ``we identify time evolution with the Hamiltonian''.  What a magic hat maneuver!  Is there a way that this would be logical without already knowing the answer?

\section{Dispersion delta function representation}

The Principle part notation here I found a bit unclear.  He writes

\begin{equation}\label{eqn:desaiCh2:500}
\lim_{\epsilon \rightarrow 0} 
\frac{(x'-x)}{(x'-x)^2 + \epsilon^2}
= 
P\left( \inv{x' - x} \right).
\end{equation}

In complex variables the principle part is the negative power series terms.  For example for \(f(z) = \sum a_k z^k\), the principle part is

\begin{equation}\label{eqn:desaiCh2:520}
\sum_{k = -\infty}^{-1} a_k z^k
\end{equation}

This does not vanish at \(z = 0\) as the principle part in this section is stated to.  In (2.202) he pulls the \(P\) out of the integral, but I think the intention is really to keep this associated with the \(1/(x'-x)\), as in

\begin{equation}\label{eqn:desaiCh2:540}
\lim_{\epsilon \rightarrow 0} 
\inv{\pi} \int_0^\infty dx' \frac{f(x')}{x'-x - i \epsilon}
= 
\inv{\pi} \int_0^\infty dx' f(x') P\left( \inv{x' - x} \right) + i f(x)
\end{equation}

Will this even have any relevance in this text?

