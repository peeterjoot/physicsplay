%
% Copyright � 2015 Peeter Joot.  All Rights Reserved.
% Licenced as described in the file LICENSE under the root directory of this GIT repository.
%
\newcommand{\authorname}{Peeter Joot}
\newcommand{\email}{peeterjoot@protonmail.com}
\newcommand{\basename}{FIXMEbasenameUndefined}
\newcommand{\dirname}{notes/FIXMEdirnameUndefined/}

\renewcommand{\basename}{virialPF2}
\renewcommand{\dirname}{notes/phy1520/}
%\newcommand{\dateintitle}{}
%\newcommand{\keywords}{}

\newcommand{\authorname}{Peeter Joot}
\newcommand{\onlineurl}{http://sites.google.com/site/peeterjoot2/math2013/\basename.pdf}
\newcommand{\sourcepath}{\dirname\basename.tex}
\newcommand{\generatetitle}[1]{\chapter{#1}}

\newcommand{\vcsinfo}{%
\section*{}
\noindent{\color{DarkOliveGreen}{\rule{\linewidth}{0.1mm}}}
\paragraph{Document version}
%\paragraph{\color{Maroon}{Document version}}
{
\small
\begin{itemize}
\item Available online at:\\ 
\href{\onlineurl}{\onlineurl}
\item Git Repository: \input{./.revinfo/gitRepo.tex}
\item Source: \sourcepath
\item last commit: \input{./.revinfo/gitCommitString.tex}
\item commit date: \input{./.revinfo/gitCommitDate.tex}
\end{itemize}
}
}

%\PassOptionsToPackage{dvipsnames,svgnames}{xcolor}
\PassOptionsToPackage{square,numbers}{natbib}
\documentclass{scrreprt}

\usepackage[left=2cm,right=2cm]{geometry}
\usepackage[svgnames]{xcolor}
\usepackage{peeters_layout}

\usepackage{natbib}

\usepackage[
colorlinks=true,
bookmarks=false,
pdfauthor={\authorname, \email},
backref 
]{hyperref}

% http://tex.stackexchange.com/questions/75773/how-to-reference-problems-by-the-text-label-in-an-exercise-envioronment
\usepackage[english]{cleveref}
\crefname{Exercise}{exercise}{exercises}
\Crefname{Exercise}{Exercise}{Exercises}

\RequirePackage{titlesec}
\RequirePackage{ifthen}

% http://stackoverflow.com/questions/4932910/date-in-the-tabular-environment
\makeatletter
\let\insertdate\@date
\makeatother

\titleformat{\chapter}[display]
{\bfseries\Large}
{\color{DarkSlateGrey}\filleft \authorname
\ifthenelse{\isundefined{\studentnumber}}{}{\\ \studentnumber}
\ifthenelse{\isundefined{\email}}{}{\\ \email}
\ifthenelse{\isundefined{\dateintitle}}{}{\\ \insertdate}
%\ifthenelse{\isundefined{\coursename}}{}{\\ \coursename} % put in title instead.
}
{4ex}
{\color{DarkOliveGreen}{\titlerule}\color{Maroon}
\vspace{2ex}%
\filright}
[\vspace{2ex}%
\color{DarkOliveGreen}\titlerule
]

\newcommand{\beginArtWithToc}[0]{\begin{document}\tableofcontents}
\newcommand{\beginArtNoToc}[0]{\begin{document}}
\newcommand{\EndNoBibArticle}[0]{\end{document}}
\newcommand{\EndArticle}[0]{\bibliography{Bibliography}\bibliographystyle{plainnat}\end{document}}

% 
%\newcommand{\citep}[1]{\cite{#1}}

\colorSectionsForArticle



\usepackage{peeters_layout_exercise}
\usepackage{peeters_braket}
\usepackage{peeters_figures}

\beginArtNoToc

\generatetitle{XXX}
%\chapter{XXX}
%\label{chap:virialPF2}

The expansion you've done isn't quite valid, because $X_k$ and $P_k$ operators are Heisenberg picture operators

\begin{dmath}\label{eqn:virialPF2:20}
\begin{aligned}
X_{k,H}(t) &= U^\dagger(t) X_k U(t) \\
P_{k,H}(t) &= U^\dagger(t) P_k U(t).
\end{aligned}
\end{dmath}

Also note that the state vectors in question (say, $\ket{\psi}$ is not time dependent), since the time dependence is in the operators themselves.

If the Hamiltonian isn't explicitly time dependent, that time evolution operator is $U(t) = e^{-i H t/\Hbar}$.  If I try plugging that in I just get to the starting point of the virial theorem derivation (i.e. evaluate the commutator of $\Bx \cdot \Bp$ with the Hamiltonian)

\begin{dmath}\label{eqn:virialPF2:40}
\begin{aligned}
\frac{d}{dt} \bra{\psi} X_{k,H} P_{k,H} \ket{\psi}
&=
\frac{d}{dt} \bra{\psi} U^\dagger X_{k} P_{k} U \ket{\psi} \\
&=
\bra{\psi} 
\left( U^\dagger X_{k} P_{k} \ddt{U}
+
\ddt{U^\dagger} X_{k} P_{k} U  
\right)
\ket{\psi}
\\
&=
\bra{\psi} 
\left( 
-U^\dagger X_{k} P_{k} \frac{i H}{\Hbar} U
+ 
U^\dagger \frac{i H}{\Hbar} 
X_{k} P_{k} U  
\right)
\ket{\psi}
\\
&=
\frac{1}{i\Hbar}
\bra{\psi} U^\dagger 
\left( 
X_{k} P_{k} H
- 
H X_{k} P_{k} 
\right)
U  
\ket{\psi}
\\
%&=
%\frac{i}{\Hbar}
%\bra{\psi} U^\dagger \antisymmetric{ X_{k} P_{k} }{H} U
%\ket{\psi}
%\\
&=
\frac{1}{i\Hbar}
\bra{\psi} \antisymmetric{ X_{k,H} P_{k,H} }{H} 
\ket{\psi}.
\end{aligned}
\end{dmath}

This is just the Heisenberg picture time evolution equation for a Heisenberg picture operator $A_H$

\begin{dmath}\label{eqn:virialPF2:60}
\frac{d A_H}{dt} = 
\frac{1}{i \Hbar} \antisymmetric{ A_H, H }.
\end{dmath}

Expanding those commutators will give you

\begin{dmath}\label{eqn:virialPF2:80}
\ddt{} \expectation{\Bx \cdot \Bp}
=
\expectation{\frac{\Bp}{m}} - \expectation{ \Bx \cdot \spacegrad V},
\end{dmath}

but that's not any closer to figuring out when the LHS is zero.

\EndArticle
