%
% Copyright � 2013 Peeter Joot.  All Rights Reserved.
% Licenced as described in the file LICENSE under the root directory of this GIT repository.
%
\newcommand{\authorname}{Peeter Joot}
\newcommand{\email}{peeterjoot@protonmail.com}
\newcommand{\basename}{FIXMEbasenameUndefined}
\newcommand{\dirname}{notes/FIXMEdirnameUndefined/}

\renewcommand{\basename}{curvilinearCoordinates}
\renewcommand{\dirname}{notes/gabook/}
%\newcommand{\dateintitle}{}
\newcommand{\keywords}{curvilinear coordinates, geometric algebra, dual basis, reciprocal coordinates}

\newcommand{\authorname}{Peeter Joot}
\newcommand{\onlineurl}{http://sites.google.com/site/peeterjoot2/math2013/\basename.pdf}
\newcommand{\sourcepath}{\dirname\basename.tex}
\newcommand{\generatetitle}[1]{\chapter{#1}}

\newcommand{\vcsinfo}{%
\section*{}
\noindent{\color{DarkOliveGreen}{\rule{\linewidth}{0.1mm}}}
\paragraph{Document version}
%\paragraph{\color{Maroon}{Document version}}
{
\small
\begin{itemize}
\item Available online at:\\ 
\href{\onlineurl}{\onlineurl}
\item Git Repository: \input{./.revinfo/gitRepo.tex}
\item Source: \sourcepath
\item last commit: \input{./.revinfo/gitCommitString.tex}
\item commit date: \input{./.revinfo/gitCommitDate.tex}
\end{itemize}
}
}

%\PassOptionsToPackage{dvipsnames,svgnames}{xcolor}
\PassOptionsToPackage{square,numbers}{natbib}
\documentclass{scrreprt}

\usepackage[left=2cm,right=2cm]{geometry}
\usepackage[svgnames]{xcolor}
\usepackage{peeters_layout}

\usepackage{natbib}

\usepackage[
colorlinks=true,
bookmarks=false,
pdfauthor={\authorname, \email},
backref 
]{hyperref}

% http://tex.stackexchange.com/questions/75773/how-to-reference-problems-by-the-text-label-in-an-exercise-envioronment
\usepackage[english]{cleveref}
\crefname{Exercise}{exercise}{exercises}
\Crefname{Exercise}{Exercise}{Exercises}

\RequirePackage{titlesec}
\RequirePackage{ifthen}

% http://stackoverflow.com/questions/4932910/date-in-the-tabular-environment
\makeatletter
\let\insertdate\@date
\makeatother

\titleformat{\chapter}[display]
{\bfseries\Large}
{\color{DarkSlateGrey}\filleft \authorname
\ifthenelse{\isundefined{\studentnumber}}{}{\\ \studentnumber}
\ifthenelse{\isundefined{\email}}{}{\\ \email}
\ifthenelse{\isundefined{\dateintitle}}{}{\\ \insertdate}
%\ifthenelse{\isundefined{\coursename}}{}{\\ \coursename} % put in title instead.
}
{4ex}
{\color{DarkOliveGreen}{\titlerule}\color{Maroon}
\vspace{2ex}%
\filright}
[\vspace{2ex}%
\color{DarkOliveGreen}\titlerule
]

\newcommand{\beginArtWithToc}[0]{\begin{document}\tableofcontents}
\newcommand{\beginArtNoToc}[0]{\begin{document}}
\newcommand{\EndNoBibArticle}[0]{\end{document}}
\newcommand{\EndArticle}[0]{\bibliography{Bibliography}\bibliographystyle{plainnat}\end{document}}

% 
%\newcommand{\citep}[1]{\cite{#1}}

\colorSectionsForArticle



% for post to:
% http://math.stackexchange.com/questions/766146/inner-product-of-trivector-and-bivector-in-geometric-algebra
\beginArtNoToc

Hestenes's New Foundations book \citep{hestenes1999nfc} sets a problem to show:

\begin{dmath}\label{eqn:trivectorBivectorInnerProduct:20}
\left( \Ba \wedge \Bb \wedge \Bc \right) \cdot B
=
\Ba \left( \left( \Bb \wedge \Bc \right) \cdot B \right)
+
\Bb \left( \left( \Bc \wedge \Ba \right) \cdot B \right)
+
\Bc \left( \left( \Ba \wedge \Bb \right) \cdot B \right).
\end{dmath}

I'm having trouble proving this.  Using an antisymmetric expansion of the wedge within a grade one selection, we have

\begin{equation}\label{eqn:trivectorBivectorInnerProduct:40}
\begin{aligned}
\left( \Ba \wedge \Bb \wedge \Bc \right) \cdot B
&=
\inv{6}
{\left\langle
\left(
\Ba \Bb \Bc
+\Bb \Bc \Ba
+\Bc \Ba \Bb
-\Ba \Bc \Ba
-\Bb \Ba \Bc
-\Bc \Bb \Ba
\right)
B
\right\rangle}_1 \\
&=
\inv{3}
{\left\langle
\Ba \left( \Bb \wedge \Bc \right) B
+\Bb \left( \Bc \wedge \Ba\right) B
+\Bc \left( \Ba \wedge \Bb\right) B
\right\rangle}_1 \\
&=
\inv{3}
\left(
\Ba \left( \left( \Bb \wedge \Bc \right) \cdot B \right)
+\Bb \left( \left( \Bc \wedge \Ba\right) \cdot B \right)
+\Bc \left( \left( \Ba \wedge \Bb\right) \cdot B \right)
\right) \\
&\quad +
\inv{3}
\left(
\Ba \cdot {\left\langle \left( \Bb \wedge \Bc \right) B \right\rangle}_2
+\Bb \cdot {\left\langle \left( \Bc \wedge \Ba\right) B \right\rangle}_2
+\Bc \cdot {\left\langle \left( \Ba \wedge \Bb\right) B \right\rangle}_2
\right).
\end{aligned}
\end{equation}

My attempts to reduce the latter terms have all gone in circles.  Any tips?

\section{Answer}

I figured it out.  The key is trying not to be so clever, instead expand in successive dot products and regroup.  With $B = \Bu \wedge \Bv$

\begin{equation}\label{eqn:trivectorBivectorInnerProduct:60}
\begin{aligned}
\left(  \Ba \wedge \Bb \wedge \Bc  \right)
\cdot B
&=
\gpgradeone{
\left(  \Ba \wedge \Bb \wedge \Bc \right)
 \left( \Bu \wedge \Bv \right) } \\
&=
\gpgradeone{
\left(  \Ba \wedge \Bb \wedge \Bc  \right)
\left(
\Bu \Bv
- \Bu \cdot \Bv
\right) } \\
&=
\left(
\left(  \Ba \wedge \Bb \wedge \Bc  \right)
 \cdot \Bu \right) \cdot \Bv \\
&=
\left( \Ba \wedge \Bb \right) \cdot \Bv \left( \Bc \cdot \Bu \right)
+
\left( \Bc \wedge \Ba \right) \cdot \Bv \left( \Bb \cdot \Bu \right)
+
\left( \Bb \wedge \Bc \right) \cdot \Bv \left( \Ba \cdot \Bu \right) \\
&=
\Ba
\left(  \Bb \cdot \Bv  \right)
\left( \Bc \cdot \Bu \right)
-\Bb
\left(  \Ba \cdot \Bv  \right)
\left( \Bc \cdot \Bu \right) \\
&\quad +
\Bc
\left(  \Ba \cdot \Bv  \right)
\left( \Bb \cdot \Bu \right)
-
\Ba
\left(  \Bc \cdot \Bv  \right)
\left( \Bb \cdot \Bu \right) \\
&\quad +
\Bb
\left(  \Bc \cdot \Bv  \right)
\left( \Ba \cdot \Bu \right)
-
\Bc
\left(  \Bb \cdot \Bv  \right)
\left( \Ba \cdot \Bu \right) \\
&=
\Ba
\left(  \left(  \Bb \cdot \Bv  \right) \left( \Bc \cdot \Bu \right) - \left(  \Bc \cdot \Bv  \right) \left( \Bb \cdot \Bu \right)  \right)
\\
&\quad +
\Bb
\left(  \left(  \Bc \cdot \Bv  \right) \left( \Ba \cdot \Bu \right) - \left(  \Ba \cdot \Bv  \right) \left( \Bc \cdot \Bu \right)  \right)
\\
&\quad +
\Bc
\left(  \left(  \Ba \cdot \Bv  \right) \left( \Bb \cdot \Bu \right) - \left(  \Bb \cdot \Bv  \right) \left( \Ba \cdot \Bu \right)  \right)
 \\
&=
\Ba
\left(  \Bb \wedge \Bc  \right)
\cdot
\left(  \Bu \wedge \Bv  \right)
\\
&\quad +
\Bb
\left(  \Bc \wedge \Ba  \right)
\cdot
\left(  \Bu \wedge \Bv  \right)
\\
&\quad +
\Bc
\left(  \Ba \wedge \Bb  \right)
 \cdot
\left(  \Bu \wedge \Bv  \right)
\\
&=
\Ba
\left(  \Bb \wedge \Bc  \right)
\cdot B
+
\Bb
\left(  \Bc \wedge \Ba  \right)
 \cdot B
+
\Bc
\left(  \Ba \wedge \Bb  \right)
\cdot B.
\end{aligned}
\end{equation}

\EndArticle
