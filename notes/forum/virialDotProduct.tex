%
% Copyright � 2015 Peeter Joot.  All Rights Reserved.
% Licenced as described in the file LICENSE under the root directory of this GIT repository.
%
\newcommand{\authorname}{Peeter Joot}
\newcommand{\email}{peeterjoot@protonmail.com}
\newcommand{\basename}{FIXMEbasenameUndefined}
\newcommand{\dirname}{notes/FIXMEdirnameUndefined/}

\renewcommand{\basename}{virialDotProduct}
\renewcommand{\dirname}{notes/phy1520/}
%\newcommand{\dateintitle}{}
%\newcommand{\keywords}{}

\newcommand{\authorname}{Peeter Joot}
\newcommand{\onlineurl}{http://sites.google.com/site/peeterjoot2/math2013/\basename.pdf}
\newcommand{\sourcepath}{\dirname\basename.tex}
\newcommand{\generatetitle}[1]{\chapter{#1}}

\newcommand{\vcsinfo}{%
\section*{}
\noindent{\color{DarkOliveGreen}{\rule{\linewidth}{0.1mm}}}
\paragraph{Document version}
%\paragraph{\color{Maroon}{Document version}}
{
\small
\begin{itemize}
\item Available online at:\\ 
\href{\onlineurl}{\onlineurl}
\item Git Repository: \input{./.revinfo/gitRepo.tex}
\item Source: \sourcepath
\item last commit: \input{./.revinfo/gitCommitString.tex}
\item commit date: \input{./.revinfo/gitCommitDate.tex}
\end{itemize}
}
}

%\PassOptionsToPackage{dvipsnames,svgnames}{xcolor}
\PassOptionsToPackage{square,numbers}{natbib}
\documentclass{scrreprt}

\usepackage[left=2cm,right=2cm]{geometry}
\usepackage[svgnames]{xcolor}
\usepackage{peeters_layout}

\usepackage{natbib}

\usepackage[
colorlinks=true,
bookmarks=false,
pdfauthor={\authorname, \email},
backref 
]{hyperref}

% http://tex.stackexchange.com/questions/75773/how-to-reference-problems-by-the-text-label-in-an-exercise-envioronment
\usepackage[english]{cleveref}
\crefname{Exercise}{exercise}{exercises}
\Crefname{Exercise}{Exercise}{Exercises}

\RequirePackage{titlesec}
\RequirePackage{ifthen}

% http://stackoverflow.com/questions/4932910/date-in-the-tabular-environment
\makeatletter
\let\insertdate\@date
\makeatother

\titleformat{\chapter}[display]
{\bfseries\Large}
{\color{DarkSlateGrey}\filleft \authorname
\ifthenelse{\isundefined{\studentnumber}}{}{\\ \studentnumber}
\ifthenelse{\isundefined{\email}}{}{\\ \email}
\ifthenelse{\isundefined{\dateintitle}}{}{\\ \insertdate}
%\ifthenelse{\isundefined{\coursename}}{}{\\ \coursename} % put in title instead.
}
{4ex}
{\color{DarkOliveGreen}{\titlerule}\color{Maroon}
\vspace{2ex}%
\filright}
[\vspace{2ex}%
\color{DarkOliveGreen}\titlerule
]

\newcommand{\beginArtWithToc}[0]{\begin{document}\tableofcontents}
\newcommand{\beginArtNoToc}[0]{\begin{document}}
\newcommand{\EndNoBibArticle}[0]{\end{document}}
\newcommand{\EndArticle}[0]{\bibliography{Bibliography}\bibliographystyle{plainnat}\end{document}}

% 
%\newcommand{\citep}[1]{\cite{#1}}

\colorSectionsForArticle



\usepackage{peeters_layout_exercise}
\usepackage{peeters_braket}
\usepackage{peeters_figures}

\beginArtNoToc

\generatetitle{Conditions for the X dot P expectation to be constant?}
%\chapter{When X dot P expectation is constant?}
%\label{chap:virialDotProduct}

A proof of the quantum virial theorem starts with the computation of the commutator of \( \antisymmetric{\Bx \cdot \Bp}{H} \).  Using that one can show for Heisenberg picture operators \( \Bx \) and \( \Bp \), and expectations relative to stationary states (i.e. states that are not time dependent), and \( H = \Bp^2/2m + V(\Bx) \), we have

\begin{dmath}\label{eqn:virialDotProduct:20}
\ddt{} \expectation{\Bx \cdot \Bp}
=
\expectation{\frac{\Bp}{m}} - \expectation{ \Bx \cdot \spacegrad V}.
\end{dmath}

Getting that far is mostly just algebra.  What I'm having trouble with is understanding the conditions that \( \expectation{\Bx \cdot \Bp} \) is constant.

For the one dimensional SHO where the position and momentum operator representation in the Heisenberg picture are

\begin{dmath}\label{eqn:virialDotProduct:40}
x = x(0) \cos(\omega t) + \frac{p(0)}{m \omega} \sin(\omega t)
\end{dmath}
\begin{dmath}\label{eqn:virialDotProduct:60}
p = p(0) \cos(\omega t) - m \omega x(0) \sin(\omega t),
\end{dmath}

I can show that this expectation does in fact vanish.  For one of the stationary states \( \ket{n} \) I calculate

\begin{dmath}\label{eqn:virialDotProduct:80}
\expectation{ x p }
= \bra{n} x p \ket{n}
= ...
= \frac{i \Hbar}{2} \lr{
\cos^2(\omega t) + \sin^2(\omega t) }
- \lr{ n + 1/2 } \sin(\omega t) \cos(\omega t) \lr{ \frac{\Hbar^2}{x_0^2 m \omega } - m \omega x_0^2 }
=
\frac{i \Hbar}{2}.
\end{dmath}

So, for this particular Hamiltonian, it's possible to show that the expectation of \( \Bx \cdot \Bp \) is constant, despite the time dependence of the operators themselves.  Exactly what principles would justify extending this to the general case is not obvious to me.

I could probably include a hand-waving argument, stating that this LHS is zero for stationary states.

I don't follow the argument for that here:

http://www.physicspages.com/2012/10/09/virial-theorem/

In

http://www7b.biglobe.ne.jp/~kcy05t/viriproof.html

the author just states that the LHS is zero for stationary states, without any elaboration, despite the fact that both \( \Bx \) and \( \Bp \) are time dependent in their Heisenberg representation.

In

https://www.univie.ac.at/physikwiki/images/a/a0/T2_Skript_final.pdf

the author states that ``Finally we assume stationary states which satisfy \( 0 = \ddt{} \expectation{ \Bx \cdot \Bp } \)''.  He also states in a footnote, ``This condition can also be regarded as a form of Hamilton's principle, since the product of position and momentum has the dimension of an action, whose variation is required to vanish''.  That footnote doesn't clarify things for me, since I don't see what the connection between the action principle and this expectation value is.

Another way to potentially show this might be to use Hamilton's equations of motion.  I note that for a stationary state \( \ket{\psi} \), we have

\begin{dmath}\label{eqn:virialDotProduct:100}
\begin{aligned}
\frac{d}{dt} \expectation{ \Bx \cdot \Bp }
&=
\frac{d}{dt} \bra{ \psi} \Bx \cdot \Bp \ket{\psi} \\
&=
\bra{ \psi} \ddt{\Bx} \cdot \Bp + \Bx \cdot \ddt{\Bp} \ket{\psi} \\
&=
\bra{ \psi} \PD{\Bp}{H} \cdot \Bp - \Bx \cdot \PD{\Bx}{H} \ket{\psi} \\
\end{aligned}
\end{dmath}


\EndNoBibArticle
