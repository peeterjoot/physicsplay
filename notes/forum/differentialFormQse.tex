%
% Copyright � 2015 Peeter Joot.  All Rights Reserved.
% Licenced as described in the file LICENSE under the root directory of this GIT repository.
%
\newcommand{\authorname}{Peeter Joot}
\newcommand{\email}{peeterjoot@protonmail.com}
\newcommand{\basename}{FIXMEbasenameUndefined}
\newcommand{\dirname}{notes/FIXMEdirnameUndefined/}

\renewcommand{\basename}{dualFarField}
\renewcommand{\dirname}{notes/ece1229/}
%\newcommand{\dateintitle}{}
%\newcommand{\keywords}{}

\newcommand{\authorname}{Peeter Joot}
\newcommand{\onlineurl}{http://sites.google.com/site/peeterjoot2/math2013/\basename.pdf}
\newcommand{\sourcepath}{\dirname\basename.tex}
\newcommand{\generatetitle}[1]{\chapter{#1}}

\newcommand{\vcsinfo}{%
\section*{}
\noindent{\color{DarkOliveGreen}{\rule{\linewidth}{0.1mm}}}
\paragraph{Document version}
%\paragraph{\color{Maroon}{Document version}}
{
\small
\begin{itemize}
\item Available online at:\\ 
\href{\onlineurl}{\onlineurl}
\item Git Repository: \input{./.revinfo/gitRepo.tex}
\item Source: \sourcepath
\item last commit: \input{./.revinfo/gitCommitString.tex}
\item commit date: \input{./.revinfo/gitCommitDate.tex}
\end{itemize}
}
}

%\PassOptionsToPackage{dvipsnames,svgnames}{xcolor}
\PassOptionsToPackage{square,numbers}{natbib}
\documentclass{scrreprt}

\usepackage[left=2cm,right=2cm]{geometry}
\usepackage[svgnames]{xcolor}
\usepackage{peeters_layout}

\usepackage{natbib}

\usepackage[
colorlinks=true,
bookmarks=false,
pdfauthor={\authorname, \email},
backref 
]{hyperref}

% http://tex.stackexchange.com/questions/75773/how-to-reference-problems-by-the-text-label-in-an-exercise-envioronment
\usepackage[english]{cleveref}
\crefname{Exercise}{exercise}{exercises}
\Crefname{Exercise}{Exercise}{Exercises}

\RequirePackage{titlesec}
\RequirePackage{ifthen}

% http://stackoverflow.com/questions/4932910/date-in-the-tabular-environment
\makeatletter
\let\insertdate\@date
\makeatother

\titleformat{\chapter}[display]
{\bfseries\Large}
{\color{DarkSlateGrey}\filleft \authorname
\ifthenelse{\isundefined{\studentnumber}}{}{\\ \studentnumber}
\ifthenelse{\isundefined{\email}}{}{\\ \email}
\ifthenelse{\isundefined{\dateintitle}}{}{\\ \insertdate}
%\ifthenelse{\isundefined{\coursename}}{}{\\ \coursename} % put in title instead.
}
{4ex}
{\color{DarkOliveGreen}{\titlerule}\color{Maroon}
\vspace{2ex}%
\filright}
[\vspace{2ex}%
\color{DarkOliveGreen}\titlerule
]

\newcommand{\beginArtWithToc}[0]{\begin{document}\tableofcontents}
\newcommand{\beginArtNoToc}[0]{\begin{document}}
\newcommand{\EndNoBibArticle}[0]{\end{document}}
\newcommand{\EndArticle}[0]{\bibliography{Bibliography}\bibliographystyle{plainnat}\end{document}}

% 
%\newcommand{\citep}[1]{\cite{#1}}

\colorSectionsForArticle



\usepackage{macros_bm}

\beginArtNoToc

\generatetitle{Duality transformation of the far field fields.}
%\section{Duality transformation of the far field fields.}
%\label{chap:dualFarField}

@Ted Shifrin mentioned in the comments that this translation is possible using Geometric Algebra.

With

\begin{equation}\label{eqn:t:n}
I = \Be_1 \Be_2 \Be_3,
\end{equation}

and the identity $ \Bx \cross \By = I (\Bx \wedge \By) $, the pair of identities becomes

\begin{equation}\label{eqn:t:n}
I \BA = \inv{2} \BB \wedge \Br
\end{equation}
\begin{equation}\label{eqn:t:n}
\spacegrad \wedge \BA = I \BB.
\end{equation}

Introducing a bivector dual $ \bcB = -I \BB $, these can be written

\begin{equation}\label{eqn:t:n}
\BA = \inv{2} \bcB \cdot \Br
\end{equation}
\begin{equation}\label{eqn:t:n}
\spacegrad \wedge \BA = -\bcB, \qquad (1)
\end{equation}

As with the coordinate expansion of the original 3D vector result, this can be verified by direct expansion.  Assuming an antisymmetric scalar $ B_{ab} = -B_{ba} $, the bivector coordinate expansion is

\begin{equation}\label{eqn:t:n}
\bcB = \inv{2} B_{ab} \Be_a \wedge \Be_b.
\end{equation}

Expanding out (1) gives

\begin{equation}\label{eqn:t:n}
\begin{aligned}
\spacegrad \wedge \BA
&=
\Be_i \partial_i \wedge \left(
\inv{4} B_{ab} (\Be_a \wedge \Be_b) \cdot (\Be_s x_s)
\right)
\\
&=
\inv{4}
B_{ab}
\Be_i \wedge \left(
 (\Be_a \wedge \Be_b) \cdot (\Be_s \delta_{i s})
\right)  \\
&=
\inv{4}
B_{ab}
\Be_i \wedge \left(
 (\Be_a \wedge \Be_b) \cdot \Be_i
\right)  \\
&=
\inv{4}
B_{ab}
\Be_i \wedge \left(
 \Be_a \delta_{b i} - \Be_b \delta_{a i}
\right)  \\
&=
\inv{4}
B_{ab}
\left(
\Be_b \wedge \Be_a - \Be_a \wedge \Be_b
\right)  \\
&=
-\inv{2} B_{ab} \Be_a \wedge \Be_b \\
&=
-\bcB
\end{aligned}
\end{equation}

Mapping the Geometric Algebra form to differential forms should (I believe) just require using differentials as the basis, and using the $ d $ operator in place of the bivector curl $ \spacegrad \wedge $.  With

\begin{equation}\label{eqn:t:n}
\begin{aligned}
\bcB &= \inv{2} B_{ab} dx_a \wedge dx_b \\
\Br &= x_a dx_a,
\end{aligned}
\end{equation}

that is

\begin{equation}\label{eqn:t:n}
d \BA = -\bcB.
\end{equation}

Presuming that is the right notation, then

\begin{equation}\label{eqn:t:n}
\begin{aligned}
\BA
&= \inv{2} \bcB \cdot \Br \\
&=
\inv{4} B_{ab} (dx_a \wedge dx_b) \cdot (dx_s x_s) \\
&=
\inv{4} B_{ab} x_s ( dx_a \delta_{bs} - dx_b \delta_{as} ) \\
&=
\inv{4} B_{ab} (x_b dx_a - x_a dx_b).
\end{aligned}
\end{equation}

Application of the differential operator gives

\begin{equation}\label{eqn:t:n}
\begin{aligned}
d\BA
&=
\inv{4} B_{ab} d(x_b dx_a - x_a dx_b) \\
&=
\inv{4} B_{ab} (dx_b \wedge dx_a - dx_a \wedge dx_b) \\
&=
-
\inv{2} B_{ab} dx_a \wedge dx_b \\
&=
-\bcB.
\end{aligned}
\end{equation}

\EndArticle
