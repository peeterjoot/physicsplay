%
% Copyright � 2014 Peeter Joot.  All Rights Reserved.
% Licenced as described in the file LICENSE under the root directory of this GIT repository.
%
%\newcommand{\authorname}{Peeter Joot}
\newcommand{\email}{peeterjoot@protonmail.com}
\newcommand{\basename}{FIXMEbasenameUndefined}
\newcommand{\dirname}{notes/FIXMEdirnameUndefined/}

%\renewcommand{\basename}{laplaceTransformVec}
%\renewcommand{\dirname}{notes/ece1254/}
%\newcommand{\dateintitle}{}
%\newcommand{\keywords}{}
%
%\newcommand{\authorname}{Peeter Joot}
\newcommand{\onlineurl}{http://sites.google.com/site/peeterjoot2/math2013/\basename.pdf}
\newcommand{\sourcepath}{\dirname\basename.tex}
\newcommand{\generatetitle}[1]{\chapter{#1}}

\newcommand{\vcsinfo}{%
\section*{}
\noindent{\color{DarkOliveGreen}{\rule{\linewidth}{0.1mm}}}
\paragraph{Document version}
%\paragraph{\color{Maroon}{Document version}}
{
\small
\begin{itemize}
\item Available online at:\\ 
\href{\onlineurl}{\onlineurl}
\item Git Repository: \input{./.revinfo/gitRepo.tex}
\item Source: \sourcepath
\item last commit: \input{./.revinfo/gitCommitString.tex}
\item commit date: \input{./.revinfo/gitCommitDate.tex}
\end{itemize}
}
}

%\PassOptionsToPackage{dvipsnames,svgnames}{xcolor}
\PassOptionsToPackage{square,numbers}{natbib}
\documentclass{scrreprt}

\usepackage[left=2cm,right=2cm]{geometry}
\usepackage[svgnames]{xcolor}
\usepackage{peeters_layout}

\usepackage{natbib}

\usepackage[
colorlinks=true,
bookmarks=false,
pdfauthor={\authorname, \email},
backref 
]{hyperref}

% http://tex.stackexchange.com/questions/75773/how-to-reference-problems-by-the-text-label-in-an-exercise-envioronment
\usepackage[english]{cleveref}
\crefname{Exercise}{exercise}{exercises}
\Crefname{Exercise}{Exercise}{Exercises}

\RequirePackage{titlesec}
\RequirePackage{ifthen}

% http://stackoverflow.com/questions/4932910/date-in-the-tabular-environment
\makeatletter
\let\insertdate\@date
\makeatother

\titleformat{\chapter}[display]
{\bfseries\Large}
{\color{DarkSlateGrey}\filleft \authorname
\ifthenelse{\isundefined{\studentnumber}}{}{\\ \studentnumber}
\ifthenelse{\isundefined{\email}}{}{\\ \email}
\ifthenelse{\isundefined{\dateintitle}}{}{\\ \insertdate}
%\ifthenelse{\isundefined{\coursename}}{}{\\ \coursename} % put in title instead.
}
{4ex}
{\color{DarkOliveGreen}{\titlerule}\color{Maroon}
\vspace{2ex}%
\filright}
[\vspace{2ex}%
\color{DarkOliveGreen}\titlerule
]

\newcommand{\beginArtWithToc}[0]{\begin{document}\tableofcontents}
\newcommand{\beginArtNoToc}[0]{\begin{document}}
\newcommand{\EndNoBibArticle}[0]{\end{document}}
\newcommand{\EndArticle}[0]{\bibliography{Bibliography}\bibliographystyle{plainnat}\end{document}}

% 
%\newcommand{\citep}[1]{\cite{#1}}

\colorSectionsForArticle


%
%\beginArtNoToc
%
%\generatetitle{Laplace transform refresher}

Laplace transforms \index{Laplace transform} were used to solve the MNA equations for time dependent systems, and to find the moments used to in MOR.
\index{model order reduction}

For the record, the Laplace transform is defined as:

%\begin{equation}\label{eqn:laplaceTransformVec:20}
\boxedEquation{eqn:laplaceTransformVec:20}{
\LL( f(t) ) = 
\int_0^\infty e^{-s t} f(t) dt.
}
%\end{equation}

The only Laplace transform pair used in the lectures is that of the first derivative

\begin{dmath}\label{eqn:laplaceTransformVec:40}
\LL(f'(t)) = 
\int_0^\infty e^{-s t} \ddt{f(t)} dt
=
\evalrange{e^{-s t} f(t)}{0}{\infty} - (-s) \int_0^\infty e^{-s t} f(t) dt
=
-f(0) + s \LL(f(t)).
\end{dmath}

Here it is loosely assumed that the real part of \( s \) is positive, and that \( f(t) \) is ``well defined'' enough that \( e^{-s \infty } f(\infty) \rightarrow 0 \).

Where used in the lectures, the Laplace transforms were of vectors such as the matrix vector product \( \LL(\BG \Bx(t)) \).  Because such a product is linear, observe that it can be expressed as the original matrix times a vector of Laplace transforms

\begin{dmath}\label{eqn:laplaceTransformVec:60}
\LL( \BG \Bx(t) )
=
\LL {\begin{bmatrix}
G_{i k} x_k(t)
\end{bmatrix}}_i
=
{\begin{bmatrix}
G_{i k} \LL x_k(t)
\end{bmatrix}}_i
=
\BG
{\begin{bmatrix}
\LL x_i(t)
\end{bmatrix}}_i.
\end{dmath}

The following notation was used in the lectures for such a vector of Laplace transforms

\begin{equation}\label{eqn:laplaceTransformVec:80}
\BX(s) = \LL \Bx(t) = 
{\begin{bmatrix}
\LL x_i(t)
\end{bmatrix}}_i.
\end{equation}

%\EndNoBibArticle
