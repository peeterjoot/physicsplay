%
% Copyright � 2014 Peeter Joot.  All Rights Reserved.
% Licenced as described in the file LICENSE under the root directory of this GIT repository.
%
% for template copy, run:
%
% ~/bin/ct multiphysicsL1  multiphysicsLectureN tl1
%
%\newcommand{\authorname}{Peeter Joot}
\newcommand{\email}{peeterjoot@protonmail.com}
\newcommand{\basename}{FIXMEbasenameUndefined}
\newcommand{\dirname}{notes/FIXMEdirnameUndefined/}

%\renewcommand{\basename}{multiphysicsL12}
%\renewcommand{\dirname}{notes/ece1254/}
%\newcommand{\keywords}{ECE1254H}
%\newcommand{\authorname}{Peeter Joot}
\newcommand{\onlineurl}{http://sites.google.com/site/peeterjoot2/math2013/\basename.pdf}
\newcommand{\sourcepath}{\dirname\basename.tex}
\newcommand{\generatetitle}[1]{\chapter{#1}}

\newcommand{\vcsinfo}{%
\section*{}
\noindent{\color{DarkOliveGreen}{\rule{\linewidth}{0.1mm}}}
\paragraph{Document version}
%\paragraph{\color{Maroon}{Document version}}
{
\small
\begin{itemize}
\item Available online at:\\ 
\href{\onlineurl}{\onlineurl}
\item Git Repository: \input{./.revinfo/gitRepo.tex}
\item Source: \sourcepath
\item last commit: \input{./.revinfo/gitCommitString.tex}
\item commit date: \input{./.revinfo/gitCommitDate.tex}
\end{itemize}
}
}

%\PassOptionsToPackage{dvipsnames,svgnames}{xcolor}
\PassOptionsToPackage{square,numbers}{natbib}
\documentclass{scrreprt}

\usepackage[left=2cm,right=2cm]{geometry}
\usepackage[svgnames]{xcolor}
\usepackage{peeters_layout}

\usepackage{natbib}

\usepackage[
colorlinks=true,
bookmarks=false,
pdfauthor={\authorname, \email},
backref 
]{hyperref}

% http://tex.stackexchange.com/questions/75773/how-to-reference-problems-by-the-text-label-in-an-exercise-envioronment
\usepackage[english]{cleveref}
\crefname{Exercise}{exercise}{exercises}
\Crefname{Exercise}{Exercise}{Exercises}

\RequirePackage{titlesec}
\RequirePackage{ifthen}

% http://stackoverflow.com/questions/4932910/date-in-the-tabular-environment
\makeatletter
\let\insertdate\@date
\makeatother

\titleformat{\chapter}[display]
{\bfseries\Large}
{\color{DarkSlateGrey}\filleft \authorname
\ifthenelse{\isundefined{\studentnumber}}{}{\\ \studentnumber}
\ifthenelse{\isundefined{\email}}{}{\\ \email}
\ifthenelse{\isundefined{\dateintitle}}{}{\\ \insertdate}
%\ifthenelse{\isundefined{\coursename}}{}{\\ \coursename} % put in title instead.
}
{4ex}
{\color{DarkOliveGreen}{\titlerule}\color{Maroon}
\vspace{2ex}%
\filright}
[\vspace{2ex}%
\color{DarkOliveGreen}\titlerule
]

\newcommand{\beginArtWithToc}[0]{\begin{document}\tableofcontents}
\newcommand{\beginArtNoToc}[0]{\begin{document}}
\newcommand{\EndNoBibArticle}[0]{\end{document}}
\newcommand{\EndArticle}[0]{\bibliography{Bibliography}\bibliographystyle{plainnat}\end{document}}

% 
%\newcommand{\citep}[1]{\cite{#1}}

\colorSectionsForArticle


%
%\usepackage{kbordermatrix}
%\usepackage{algorithmic}
%
%\beginArtNoToc
%\generatetitle{ECE1254H Modeling of Multiphysics Systems.  Lecture 12: Struts and Joints, Node branch formulation.  Taught by Prof.\ Piero Triverio}
%%\chapter{Struts and Joints, Node branch formulation}
%\label{chap:multiphysicsL12}
%
%\section{Disclaimer}
%
%Peeter's lecture notes from class.  These may be incoherent and rough.
%
%\section{Struts and Joints, Node branch formulation}

Consider the simple strut system of \cref{fig:lecture12:lecture12Fig1} again.

\imageFigure{../../figures/ece1254/lecture12Fig1}{Simple strut system}{fig:lecture12:lecture12Fig1}{0.3}

The unknowns are

\begin{enumerate}
\item Forces

At each of the points there is a force with two components

\begin{dmath}\label{eqn:multiphysicsL12:20}
\Bf_A = \lr{ f_{A,x}, f_{A,y} }
\end{dmath}

Construct a total force vector

\begin{equation}\label{eqn:multiphysicsL12:40}
\Bf = 
\begin{bmatrix}
f_{A,x} \\
f_{A,y} \\
f_{B,x} \\
f_{B,y} \\
\vdots 
\end{bmatrix}
\end{equation}
\item
Positions of the joints

\begin{equation}\label{eqn:multiphysicsL12:60}
\Br =
\begin{bmatrix}
x_1 \\
y_1 \\
y_1 \\
y_2 \\
\vdots 
\end{bmatrix}
\end{equation}
\end{enumerate}

The given variables are
\begin{enumerate}
\item The load force \( \Bf_L \).
\item The joint positions at rest.
\item parameter of struts.
\end{enumerate}

\paragraph{Conservation laws}
The conservation laws are 

\begin{subequations}
\begin{equation}\label{eqn:multiphysicsL12:80}
\Bf_A + \Bf_B + \Bf_C = 0
\end{equation}
\begin{equation}\label{eqn:multiphysicsL12:100}
-\Bf_C + \Bf_D + \Bf_L = 0
\end{equation}
\end{subequations}

which can be put into matrix form 

\begin{equation}\label{eqn:multiphysicsL12:120}
\kbordermatrix{ 
    &   &   &   &   &   &   &   &   \\
%    & . & . & . & . & . & . & . & . \\
x_1 & 1 & 0 & 1 & 0 & 1 & 0 & 0 & 0 \\
y_1 & 0 & 1 & 0 & 1 & 0 & 0 & 0 & 0 \\
x_2 & 0 & 0 & 0 & 0 & -1 & 0 & 1 & 0 \\
y_2 & 0 & 1 & 0 & 1 & 0 & -1 & 0 & 1 \\
}
\begin{bmatrix}
f_{A,x} \\
f_{A,y} \\
f_{B,x} \\
f_{B,y} \\
f_{C,x} \\
f_{C,y} \\
f_{D,x} \\
f_{D,y}
\end{bmatrix}
=
\begin{bmatrix}
0 \\
0 \\
-f_{L,x} \\
-f_{L,y}
\end{bmatrix}
\end{equation}

Here the block matrix is called the \textAndIndex{incidence matrix} \( \BA \).  The system can be expressed as

%\begin{equation}\label{eqn:multiphysicsL12:140}
\boxedEquation{eqn:multiphysicsL12:160}{
A \Bf = \Bf_L.
}
%\end{equation}

\paragraph{Constitutive laws}

Given a pair of nodes as in \cref{fig:lecture12:lecture12Fig2}.

\imageFigure{../../figures/ece1254/lecture12Fig2}{Strut spanning nodes}{fig:lecture12:lecture12Fig2}{0.2}


each component has an equation relating the reaction forces of that strut based on the positions

\begin{subequations}
\begin{equation}\label{eqn:multiphysicsL12:180}
f_{c,x} = S_x \lr{ x_1 - x_2, y_1 - y_2, p_c }
\end{equation}
\begin{equation}\label{eqn:multiphysicsL12:200}
f_{c,y} = S_y \lr{ x_1 - x_2, y_1 - y_2, p_c },
\end{equation}
\end{subequations}

where \( p_c \) represent the parameters of the system.  Write

\begin{equation}\label{eqn:multiphysicsL12:220}
\Bf =
\begin{bmatrix}
f_{A,x} \\
f_{A,y} \\
f_{B,x} \\
f_{B,y} \\
\vdots 
\end{bmatrix}
=
\begin{bmatrix}
S_x \lr{ x_1 - x_2, y_1 - y_2, p_c } \\
S_y \lr{ x_1 - x_2, y_1 - y_2, p_c } \\
\vdots 
\end{bmatrix},
\end{equation}

or 
%\begin{equation}\label{eqn:multiphysicsL12:240}
\boxedEquation{eqn:multiphysicsL12:260}{
\Bf = S(\Br)
}
%\end{equation}

\paragraph{Putting the pieces together} 

The node branch formulation is

\begin{equation}\label{eqn:multiphysicsL12:280}
\begin{aligned}
A \Bf - \Bf_L &= 0 \\
\Bf - S(\Br) &= 0
\end{aligned}
\end{equation}

Elimination of the forces (the equivalent of currents in this system) produces the 
\textAndIndex{nodal formulation} 

%\begin{equation}\label{eqn:multiphysicsL12:300}
\boxedEquation{eqn:multiphysicsL12:320}{
A S(\Br) - \Bf_L = 0
}
%\end{equation}

This nodal formulation cannot be used with struts that are so stiff that the positions of some of the nodes are fixed.  As before this can be worked around by introducing an additional unknown for each component of such a strut.

The cost of this Jacobian calculation required to approximate this system can still potentially be expensive.

