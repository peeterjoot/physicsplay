%
% Copyright � 2014 Peeter Joot.  All Rights Reserved.
% Licenced as described in the file LICENSE under the root directory of this GIT repository.
%
%\newcommand{\authorname}{Peeter Joot}
\newcommand{\email}{peeterjoot@protonmail.com}
\newcommand{\basename}{FIXMEbasenameUndefined}
\newcommand{\dirname}{notes/FIXMEdirnameUndefined/}

%\renewcommand{\basename}{multiphysicsL3}
%\renewcommand{\dirname}{notes/ece1254/}
%\newcommand{\keywords}{ECE1254H}
%\newcommand{\authorname}{Peeter Joot}
\newcommand{\onlineurl}{http://sites.google.com/site/peeterjoot2/math2013/\basename.pdf}
\newcommand{\sourcepath}{\dirname\basename.tex}
\newcommand{\generatetitle}[1]{\chapter{#1}}

\newcommand{\vcsinfo}{%
\section*{}
\noindent{\color{DarkOliveGreen}{\rule{\linewidth}{0.1mm}}}
\paragraph{Document version}
%\paragraph{\color{Maroon}{Document version}}
{
\small
\begin{itemize}
\item Available online at:\\ 
\href{\onlineurl}{\onlineurl}
\item Git Repository: \input{./.revinfo/gitRepo.tex}
\item Source: \sourcepath
\item last commit: \input{./.revinfo/gitCommitString.tex}
\item commit date: \input{./.revinfo/gitCommitDate.tex}
\end{itemize}
}
}

%\PassOptionsToPackage{dvipsnames,svgnames}{xcolor}
\PassOptionsToPackage{square,numbers}{natbib}
\documentclass{scrreprt}

\usepackage[left=2cm,right=2cm]{geometry}
\usepackage[svgnames]{xcolor}
\usepackage{peeters_layout}

\usepackage{natbib}

\usepackage[
colorlinks=true,
bookmarks=false,
pdfauthor={\authorname, \email},
backref 
]{hyperref}

% http://tex.stackexchange.com/questions/75773/how-to-reference-problems-by-the-text-label-in-an-exercise-envioronment
\usepackage[english]{cleveref}
\crefname{Exercise}{exercise}{exercises}
\Crefname{Exercise}{Exercise}{Exercises}

\RequirePackage{titlesec}
\RequirePackage{ifthen}

% http://stackoverflow.com/questions/4932910/date-in-the-tabular-environment
\makeatletter
\let\insertdate\@date
\makeatother

\titleformat{\chapter}[display]
{\bfseries\Large}
{\color{DarkSlateGrey}\filleft \authorname
\ifthenelse{\isundefined{\studentnumber}}{}{\\ \studentnumber}
\ifthenelse{\isundefined{\email}}{}{\\ \email}
\ifthenelse{\isundefined{\dateintitle}}{}{\\ \insertdate}
%\ifthenelse{\isundefined{\coursename}}{}{\\ \coursename} % put in title instead.
}
{4ex}
{\color{DarkOliveGreen}{\titlerule}\color{Maroon}
\vspace{2ex}%
\filright}
[\vspace{2ex}%
\color{DarkOliveGreen}\titlerule
]

\newcommand{\beginArtWithToc}[0]{\begin{document}\tableofcontents}
\newcommand{\beginArtNoToc}[0]{\begin{document}}
\newcommand{\EndNoBibArticle}[0]{\end{document}}
\newcommand{\EndArticle}[0]{\bibliography{Bibliography}\bibliographystyle{plainnat}\end{document}}

% 
%\newcommand{\citep}[1]{\cite{#1}}

\colorSectionsForArticle



%\beginArtNoToc
%\generatetitle{ECE1254H Modeling of Multiphysics Systems.  Lecture 3: Nodal Analysis.  Taught by Prof.\ Piero Triverio}
%\section{Nodal Analysis}
\label{chap:multiphysicsL3}

%\section{Disclaimer}
%
%Peeter's lecture notes from class.  These may be incoherent and rough.

\section{Nodal Analysis}

Avoiding branch currents can reduce the scope of the computational problem.  Consider the same circuit \cref{fig:lecture3:lecture3Fig1}, this time introducing only node voltages as unknowns

\imageFigure{../../figures/ece1254/lecture3Fig1}{Resistive circuit with current sources}{fig:lecture3:lecture3Fig1}{0.3}

Unknowns: node voltages: \(V_1, V_2, \cdots V_4\)

Equations are \textAndIndex{KCL} at each node except \(0\).

\begin{enumerate}
\item
      \(
      \frac{V_1 - 0}{R_A} +
      \frac{V_1 - V_2}{R_B} + i_{S,A} = 0
      \)
\item
      \(
      \frac{V_2 - 0}{R_E} +
      \frac{V_2 - V_1}{R_B} + i_{S,B} + i_{S,C} = 0
      \)
\item
      \(
      \frac{V_3 - V_4}{R_C} - i_{S,C} = 0
      \)
\item
      \(
      \frac{V_4 - 0}{R_D}
      +\frac{V_4 - V_3}{R_C}
      - i_{S,A} - i_{S,B} = 0
      \)
\end{enumerate}

In matrix form this is

\begin{equation}\label{eqn:multiphysicsL3:20}
\begin{bmatrix}
   \inv{R_A} + \inv{R_B} & - \inv{R_B} & 0 & 0 \\
   -\inv{R_B} & \inv{R_B} + \inv{R_E} & 0 & 0 \\
   0 & 0 & \inv{R_C} & -\inv{R_C} \\
   0 & 0 & -\inv{R_C} & \inv{R_C} + \inv{R_D}
\end{bmatrix}
\begin{bmatrix}
   V_1 \\
   V_2 \\
   V_3 \\
   V_4 \\
\end{bmatrix}
=
\begin{bmatrix}
   -i_{S,A} \\
-i_{S,B} - i_{S,C} \\
   i_{S,C} \\
   i_{S,A} + i_{S,B}
\end{bmatrix}
\end{equation}

Introducing the \textAndIndex{nodal matrix} \(G\), this is written as

\begin{equation}\label{eqn:multiphysicsL3:40}
   G \BV_N = \BI_S
\end{equation}

There is a recurring pattern in the nodal matrix, designated the 
\textAndIndex{stamp} for the resistor of value \(R\) between nodes \(n_1\) and \(n_2\)

% FIXME: could later point back to simple resistor node figure.
%\cref{fig:lecture3:lecture3Fig2}
%\imageFigure{../../figures/ece1254/lecture3Fig2}{Resistor stamp matrix}{fig:lecture3:lecture3Fig2}{0.3}

% http://stackoverflow.com/questions/664798/matrices-in-latex-row-and-column-labelling
\begin{equation}\label{eqn:multiphysicsL3:60}
\kbordermatrix{
    & n_1      & n_2 \\
n_1 & \inv{R}  & -\inv{R} \\
n_2 & -\inv{R} & \inv{R}
},
\end{equation}

containing a set of rows and columns for each of the node voltages \(n_1, n_2\).

Note that some care is required to use this nodal analysis method since the invertible relationship \(i = V/R\) is required.  Short circuits \(V = 0\), and voltage sources such as \(V = 5\) also cannot be handled directly.  The mechanisms to deal with differential terms like inductors will be discussed later.

\paragraph{Recap of node branch equations}

The node branch equations were

\begin{itemize}
   \item KCL: \( A \BI_B = \BI_S\)
   \item Constitutive: \( \BI_B = \alpha A^\T \BV_N\),
   \item Nodal equations: \(
\mathLabelBox
{
      A \alpha A^\T
}
{\(G\)}
      \BV_N
      = \BI_S \)
\end{itemize}

where \(\BI_B\) was the branch currents, \(A\) was the \textAndIndex{incidence matrix}, and \(\alpha = \begin{bmatrix}\inv{R_1} & & \\ & \inv{R_2} & \\ & & \ddots \end{bmatrix} \).

The \textAndIndex{stamp} can be observed in the multiplication of the contribution for a single resistor.  The incidence matrix has the form \( G = A \alpha A^\T \)
%as illustrated in \cref{fig:lecture3:lecture3Fig3}.
%\imageFigure{../../figures/ece1254/lecture3Fig3}{Factoring of the stamp matrix}{fig:lecture3:lecture3Fig3}{0.3}
\begin{dmath}\label{eqn:multiphysicsL3:100}
G =
\kbordermatrix{
    & \downarrow   & \\
n_1 & +1             & \\
n_2 & -1           & \\
}
\begin{bmatrix}
   & &
   & \inv{R} &
   & &
\end{bmatrix}
\kbordermatrix{
   & n_1 & n_2 \\
   & +1  & -1 \\
   &     &
}
=
\kbordermatrix{
    & n_1      & n_2 \\
n_1 & \inv{R}  & -\inv{R} \\
n_2 & -\inv{R} & \inv{R}
}
\end{dmath}

\paragraph{Theoretical facts}

Noting that \(\lr{ A B }^\T = B^\T A^\T \), it is clear that the nodal matrix \(G = A \alpha A^\T \) is symmetric \index{symmetric matrix}

\begin{dmath}\label{eqn:multiphysicsL3:80}
G^\T
=
\lr{ A \alpha A^\T }^\T
=
\lr{ A^\T }^\T \alpha^\T A^\T
=
A \alpha A^\T
= G
\end{dmath}

\section{Modified nodal analysis (MNA)}
\index{modified nodal analysis}
\index{MNA}

Modified nodal analysis (MNA), eliminates the branch currents for the resistive circuit elements, and is the method used to implement software such as \textAndIndex{spice}.  To illustrate the method, consider the same circuit, augmented with an additional voltage sources as in \cref{fig:lecture3:lecture3Fig4}.  This method can also accomodate voltage sources, provided an unknown current
source is also introduced for each voltage source circuit element.  The unknowns are

\imageFigure{../../figures/ece1254/lecture3Fig4}{Resistive circuit with current and voltage sources}{fig:lecture3:lecture3Fig4}{0.3}

\begin{itemize}
   \item node voltages (\(N-1\)): \( V_1, V_2, \cdots V_5 \)
   \item branch currents for selected components (\(K\)): \( i_{S,C}, i_{S,D} \)
\end{itemize}

Compared to standard nodal analysis, two less unknowns for this system are required.  The equations are

\begin{enumerate}
\item
   \(
   -\frac{V_5-V_1}{R_A}
   +\frac{V_1-V_2}{R_B}
   + i_{S,A} = 0
   \)
\item
   \(
    \frac{V_2-V_5}{R_E}
   +\frac{V_2-V_1}{R_B}
   + i_{S,B}
   - i_{S,C}
   = 0
   \)
\item
   \(
   i_{S,C} +
   \frac{V_3-V_4}{R_C} = 0
   \)
\item
   \(
   \frac{V_4-0}{R_D}
   +\frac{V_4-V_3}{R_C}
   - i_{S,A}
   - i_{S,B}
   = 0
   \)
\item
   \(
   \frac{V_5-V_2}{R_E}
   +\frac{V_5-V_1}{R_A}
   - i_{S,D} = 0
   \)
\end{enumerate}

Put into giant matrix form, this is

\begin{equation}\label{eqn:multiphysicsL3:120}
   \kbordermatrix{
      & G         &           &        &           &              &        & A_V      &    \\
      & Z_A + Z_B & -Z_B      & .      & .         & -Z_A         & \vrule &          &    \\
      & -Z_B      & Z_B - Z_E & .      & .         & -Z_E         & \vrule & -1       &    \\
      & .         & .         & Z_C    & -Z_C      & .            & \vrule & +1       &    \\
      & .         & .         & -Z_C   & Z_C + Z_D & .            & \vrule &          &    \\
      & -Z_A      & -Z_E      &        &           & Z_A + Z_E    & \vrule &          & -1 \\
      \cline{2-9}
        -A_V^\T &           & +1        & -1     &           &              & \vrule &  \\
                &           &           &        &           & 1            & \vrule &
   }
\begin{bmatrix}
   V_1 \\
   V_2 \\
   V_3 \\
   V_4 \\
   V_5 \\
%   \cline{1} \\
   i_{S,C} \\
   i_{S,D}
\end{bmatrix}
=
\begin{bmatrix}
   -i_{S,A} \\
   -i_{S,B} \\
   0 \\
   i_{S,A} + i_{S,B} \\
   0 \\
%   \cline{1}
   V_{S,C} \\
   V_{S,D}
\end{bmatrix}
\end{equation}

%\cref{fig:lecture3:lecture3Fig5}
%\imageFigure{../../figures/ece1254/lecture3Fig5}{Nodal and voltage incidence matrices}{fig:lecture3:lecture3Fig5}{0.3}

Call the extension to the \textAndIndex{nodal matrix} \(G\), the \textAndIndex{voltage incidence matrix} \(A_V\).

% from L4:
\paragraph{Review}

Additional unknowns are added for

\begin{itemize}
\item branch currents for voltage sources
\item all elements for which it is impossible or inconvenient to write \( i = f(v_1, v_2) \).

Imagine, for example, a component as illustrated in \cref{fig:lecture4:lecture4Fig1}.

\imageFigure{../../figures/ece1254/lecture4Fig1}{Variable voltage device}{fig:lecture4:lecture4Fig1}{0.15}

\begin{equation}\label{eqn:multiphysicsL4:20}
v_1 - v_2 = 3 i^2
\end{equation}
\item any current which is controlling dependent sources, as in \cref{fig:lecture4:lecture4Fig2}.

\imageFigure{../../figures/ece1254/lecture4Fig2}{Current controlled device}{fig:lecture4:lecture4Fig2}{0.15}

\item Inductors
\begin{equation}\label{eqn:multiphysicsL4:40}
v_1 - v_2 = L \frac{di}{dt}.
\end{equation}
\end{itemize}

%\EndArticle
%\EndNoBibArticle
