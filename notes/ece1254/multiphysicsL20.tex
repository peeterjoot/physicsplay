%
% Copyright � 2014 Peeter Joot.  All Rights Reserved.
% Licenced as described in the file LICENSE under the root directory of this GIT repository.
%
%\newcommand{\authorname}{Peeter Joot}
\newcommand{\email}{peeterjoot@protonmail.com}
\newcommand{\basename}{FIXMEbasenameUndefined}
\newcommand{\dirname}{notes/FIXMEdirnameUndefined/}

%\renewcommand{\basename}{multiphysicsL20}
%\renewcommand{\dirname}{notes/ece1254/}
%\newcommand{\keywords}{ECE1254H}
%\newcommand{\authorname}{Peeter Joot}
\newcommand{\onlineurl}{http://sites.google.com/site/peeterjoot2/math2013/\basename.pdf}
\newcommand{\sourcepath}{\dirname\basename.tex}
\newcommand{\generatetitle}[1]{\chapter{#1}}

\newcommand{\vcsinfo}{%
\section*{}
\noindent{\color{DarkOliveGreen}{\rule{\linewidth}{0.1mm}}}
\paragraph{Document version}
%\paragraph{\color{Maroon}{Document version}}
{
\small
\begin{itemize}
\item Available online at:\\ 
\href{\onlineurl}{\onlineurl}
\item Git Repository: \input{./.revinfo/gitRepo.tex}
\item Source: \sourcepath
\item last commit: \input{./.revinfo/gitCommitString.tex}
\item commit date: \input{./.revinfo/gitCommitDate.tex}
\end{itemize}
}
}

%\PassOptionsToPackage{dvipsnames,svgnames}{xcolor}
\PassOptionsToPackage{square,numbers}{natbib}
\documentclass{scrreprt}

\usepackage[left=2cm,right=2cm]{geometry}
\usepackage[svgnames]{xcolor}
\usepackage{peeters_layout}

\usepackage{natbib}

\usepackage[
colorlinks=true,
bookmarks=false,
pdfauthor={\authorname, \email},
backref 
]{hyperref}

% http://tex.stackexchange.com/questions/75773/how-to-reference-problems-by-the-text-label-in-an-exercise-envioronment
\usepackage[english]{cleveref}
\crefname{Exercise}{exercise}{exercises}
\Crefname{Exercise}{Exercise}{Exercises}

\RequirePackage{titlesec}
\RequirePackage{ifthen}

% http://stackoverflow.com/questions/4932910/date-in-the-tabular-environment
\makeatletter
\let\insertdate\@date
\makeatother

\titleformat{\chapter}[display]
{\bfseries\Large}
{\color{DarkSlateGrey}\filleft \authorname
\ifthenelse{\isundefined{\studentnumber}}{}{\\ \studentnumber}
\ifthenelse{\isundefined{\email}}{}{\\ \email}
\ifthenelse{\isundefined{\dateintitle}}{}{\\ \insertdate}
%\ifthenelse{\isundefined{\coursename}}{}{\\ \coursename} % put in title instead.
}
{4ex}
{\color{DarkOliveGreen}{\titlerule}\color{Maroon}
\vspace{2ex}%
\filright}
[\vspace{2ex}%
\color{DarkOliveGreen}\titlerule
]

\newcommand{\beginArtWithToc}[0]{\begin{document}\tableofcontents}
\newcommand{\beginArtNoToc}[0]{\begin{document}}
\newcommand{\EndNoBibArticle}[0]{\end{document}}
\newcommand{\EndArticle}[0]{\bibliography{Bibliography}\bibliographystyle{plainnat}\end{document}}

% 
%\newcommand{\citep}[1]{\cite{#1}}

\colorSectionsForArticle


%
%%\usepackage{macros_bm}
%
%\beginArtNoToc
%\generatetitle{ECE1254H Modeling of Multiphysics Systems.  Lecture 20: Observability and controllability.  Taught by Prof.\ Piero Triverio}
%%\chapter{Observability and controllability}
\label{chap:multiphysicsL20}

%\section{Disclaimer}
%
%Peeter's lecture notes from class.  These may be incoherent and rough.
%
\section{Truncated Balanced Realization (1000 ft overview)}
\index{truncated balanced realization}
\index{TBR|see {truncated balanced realization}}

Consider a model in state space form
\begin{equation}\label{eqn:multiphysicsLecture20:20}
\begin{aligned}
\dot{\Bx} &= \BA \Bx(t) + \BB \Bu(t) \\
\By &= \BC \Bx(t) + \BD \Bu(t).
\end{aligned}
\end{equation}

The system has the form

\begin{equation}\label{eqn:multiphysicsLecture20:40}
\Bu(t) 
\mathLabelBox
%[
%   labelstyle={xshift=2cm},
%   linestyle={out=270,in=90, latex-}
%]
{
\rightarrow 
}
{controllability}
\Bx(t) 
\mathLabelBox
[
   labelstyle={xshift=2cm},
   linestyle={out=270,in=90, latex-}
]
{
\rightarrow 
}
{observability}
\By(t).
\end{equation}

\index{controlability}
\index{observability}
\makedefinition{Observability.}{dfn:multiphysicsL20:60}{
The system is observable if \( \forall t \), the initial state \( \Bx(t_0) \) can be determined knowing \( \Bu(t), \By(t), \qquad t \in [t_0, t_1] \).

\paragraph{Equivalent condition (if A stable)}
\index{A stable}
If A stable, the observability Gramian \( \BW_0 > 0 \)

\begin{equation}\label{eqn:multiphysicsLecture20:80}
\BW_0 = \int_0^\infty e^{\BA^\T \tau} \BC^\T \BC e^{\BA \tau} d\tau
\end{equation}
}

This \textAndIndex{Gramian} expression results from a calculation of the ``energy'' of the homogeneous system (no input \( \Bu(t) \) ), for which the system is in an initial state \( \Bx_0 \) at \( t = 0 \), with output

\begin{equation}\label{eqn:multiphysicsLecture20:100}
\By(t) = \BC e^{\BA t} \Bx_0
\end{equation}

The output energy is

\begin{dmath}\label{eqn:multiphysicsLecture20:120}
\int_0^\infty \By^\T \By dt
=
\int_0^\infty \Bx_0^\T
e^{\BA^\T t} \BC^\T \BC e^{\BA t} \Bx_0 dt
= 
\Bx_0^\T \BW_0 \Bx_0
= \mathcal{E}_0
\end{dmath}

Consider the eigenvalues and vectors of \( \BW_0 \)

\begin{equation}\label{eqn:multiphysicsLecture20:140}
\BW_0 \Bx_{0, i} = \lambda_{0, i} \Bx_{0, i}.
\end{equation}

If \( \Bx_0 = \Bx_{0,i} \) this implies

\begin{equation}\label{eqn:multiphysicsLecture20:160}
\mathcal{E}_0
=
\Bx_{0,i}^\T
\lambda_{0, i}
\Bx_{0,i}
\end{equation}

The eigenvalues of the observability Gramian can provide some useful information about what states are observable.

\makedefinition{Controllability.}{dfn:multiphysicsL20:180}{
If for any initial state \( \Bx(t_0) \) , there is a final time \( t_1 > t_0 \), and a final state \( \Bx_1 \) such that there exists an input \( \Bu^\conj(t) \) such that \( \Bx(t_1) = \Bx_1 \).
}

\index{Gramian}
\maketheorem{Controllability Gramian.}{dfn:multiphysicsL20:200}{
The system is controllable if and only if the controllability Gramian \( \BW_c > 0 \).
}

What is the energy of the input \( \Bu^\conj(t) \) that is needed to set up \( \Bx_1 \)?  On the slides it is shown that this is

\begin{dmath}\label{eqn:multiphysicsLecture20:200}
\mathcal{E}_1
= \Bx_1^\T \BW_c^{-1} \Bx_1.
\end{dmath}

if \( \Bx_1 \) is an eigenvector of \( \BW_c \) with eigenvalue \( \lambda_1 \), then this reduces to

\begin{dmath}\label{eqn:multiphysicsLecture20:220}
\mathcal{E}_1
= \inv{\lambda_1} \Bx_1^\T \Bx_1
= \inv{\lambda_1}.
\end{dmath}

These eigenvalues provide information about how much energy is required to achieve any given state.

The purpose of considering observability and controllability is to help determine what modes can be safely discarded to reduce the model order.  A state that cannot be achieved without excessive energy cost is not likely to be of interest.

While the explicit form of the controllability Gramian has not been given, it becomes of less interest since it is possible to make a change of variables, such that the controllability and observability Gramians become equal and diagonal.

\begin{dmath}\label{eqn:multiphysicsLecture20:240}
\begin{aligned}
\mytilde{\BW_c} &= T \BW_c T^\T \\
\mytilde{\BW_o} &= T^{-\T} \BW_o T^{-1}
\end{aligned}
\end{dmath}

Using these provides a balanced method to determine the most important modes to retain.  

%\EndNoBibArticle
