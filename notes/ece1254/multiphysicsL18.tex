%
% Copyright � 2014 Peeter Joot.  All Rights Reserved.
% Licenced as described in the file LICENSE under the root directory of this GIT repository.
%
%\newcommand{\authorname}{Peeter Joot}
\newcommand{\email}{peeterjoot@protonmail.com}
\newcommand{\basename}{FIXMEbasenameUndefined}
\newcommand{\dirname}{notes/FIXMEdirnameUndefined/}

%\renewcommand{\basename}{multiphysicsL18}
%\renewcommand{\dirname}{notes/ece1254/}
%\newcommand{\keywords}{ECE1254H}
%\newcommand{\authorname}{Peeter Joot}
\newcommand{\onlineurl}{http://sites.google.com/site/peeterjoot2/math2013/\basename.pdf}
\newcommand{\sourcepath}{\dirname\basename.tex}
\newcommand{\generatetitle}[1]{\chapter{#1}}

\newcommand{\vcsinfo}{%
\section*{}
\noindent{\color{DarkOliveGreen}{\rule{\linewidth}{0.1mm}}}
\paragraph{Document version}
%\paragraph{\color{Maroon}{Document version}}
{
\small
\begin{itemize}
\item Available online at:\\ 
\href{\onlineurl}{\onlineurl}
\item Git Repository: \input{./.revinfo/gitRepo.tex}
\item Source: \sourcepath
\item last commit: \input{./.revinfo/gitCommitString.tex}
\item commit date: \input{./.revinfo/gitCommitDate.tex}
\end{itemize}
}
}

%\PassOptionsToPackage{dvipsnames,svgnames}{xcolor}
\PassOptionsToPackage{square,numbers}{natbib}
\documentclass{scrreprt}

\usepackage[left=2cm,right=2cm]{geometry}
\usepackage[svgnames]{xcolor}
\usepackage{peeters_layout}

\usepackage{natbib}

\usepackage[
colorlinks=true,
bookmarks=false,
pdfauthor={\authorname, \email},
backref 
]{hyperref}

% http://tex.stackexchange.com/questions/75773/how-to-reference-problems-by-the-text-label-in-an-exercise-envioronment
\usepackage[english]{cleveref}
\crefname{Exercise}{exercise}{exercises}
\Crefname{Exercise}{Exercise}{Exercises}

\RequirePackage{titlesec}
\RequirePackage{ifthen}

% http://stackoverflow.com/questions/4932910/date-in-the-tabular-environment
\makeatletter
\let\insertdate\@date
\makeatother

\titleformat{\chapter}[display]
{\bfseries\Large}
{\color{DarkSlateGrey}\filleft \authorname
\ifthenelse{\isundefined{\studentnumber}}{}{\\ \studentnumber}
\ifthenelse{\isundefined{\email}}{}{\\ \email}
\ifthenelse{\isundefined{\dateintitle}}{}{\\ \insertdate}
%\ifthenelse{\isundefined{\coursename}}{}{\\ \coursename} % put in title instead.
}
{4ex}
{\color{DarkOliveGreen}{\titlerule}\color{Maroon}
\vspace{2ex}%
\filright}
[\vspace{2ex}%
\color{DarkOliveGreen}\titlerule
]

\newcommand{\beginArtWithToc}[0]{\begin{document}\tableofcontents}
\newcommand{\beginArtNoToc}[0]{\begin{document}}
\newcommand{\EndNoBibArticle}[0]{\end{document}}
\newcommand{\EndArticle}[0]{\bibliography{Bibliography}\bibliographystyle{plainnat}\end{document}}

% 
%\newcommand{\citep}[1]{\cite{#1}}

\colorSectionsForArticle


%
%\beginArtNoToc
%\generatetitle{ECE1254H Modeling of Multiphysics Systems.  Lecture 18: Model order reduction.  Taught by Prof.\ Piero Triverio}
%\chapter{Model order reduction}
\label{chap:multiphysicsL18}

%\section{Disclaimer}
%
%Peeter's lecture notes from class.  These may be incoherent and rough.
%
\section{Model order reduction}
\index{model order reduction}

The system equations of interest for model order reduction (\textAndIndex{MOR}) are

\begin{subequations}
\label{eqn:multiphysicsL18:10}
\begin{equation}\label{eqn:multiphysicsL18:20}
\BG \Bx(t) + C \dot{\Bx}(t) = B \Bu(t)
\end{equation}
\begin{equation}\label{eqn:multiphysicsL18:40}
\By(t) = \BL^\T \Bx(t).
\end{equation}
\end{subequations}

Here 
\begin{itemize}
\item \( \Bu(t) \) is a vector of inputs.
\item \( \By(t) \) is a vector of interesting outputs, as filtered by the matrix \( \BL \), and
\item \( \Bx(t) \) is the state vector, of size \( N \).
\end{itemize}

The aim is illustrated in \cref{fig:lecture18:lecture18Fig1}.

\imageFigure{../../figures/ece1254/lecture18Fig1}{Find a simplified model that has the same input-output characteristics}{fig:lecture18:lecture18Fig1}{0.2}

\paragraph{Review: transfer function}

\begin{subequations}
\begin{equation}\label{eqn:multiphysicsL18:60}
\BG \BX(s) + \BC \lr{ s \BX(s) - \cancel{\Bx(0)} } = \BB \BU(s)
\end{equation}
\begin{equation}\label{eqn:multiphysicsL18:80}
\BY(s) = \BL^\T \BX(s).
\end{equation}
\end{subequations}

Grouping terms

\begin{equation}\label{eqn:multiphysicsL18:100}
\lr{ \BG + s \BC } \BX(s) = \BB \BU(s),
\end{equation}

and inverting allows for solution for the state vector output in the frequency domain

\begin{equation}\label{eqn:multiphysicsL18:120}
\BX(s) = \lr{ \BG + s \BC }^{-1} \BB \BU(s)
\end{equation}

The vector of outputs of interest are

\begin{equation}\label{eqn:multiphysicsL18:140}
\BY(s) = 
%\mytikzmark{left}{1}
\myBoxed{
L^\T \lr{ \BG + s \BC }^{-1} \BB 
}
%\mytikzmark{right}{1}
\BU(s)
\end{equation}

%\DrawMyBox[thick, Maroon, rounded corners]{1}

The boxed portion is the transfer function

\begin{equation}\label{eqn:multiphysicsL18:160}
\BH(s) = L^\T \lr{ \BG + s \BC }^{-1} \BB.
\end{equation}

... Switched to slides.

\paragraph{Geometrical interpretation to the change of variables}

\begin{dmath}\label{eqn:multiphysicsL18:180}
x(t) = 
\mathLabelBox
[
   labelstyle={xshift=2cm},
   linestyle={out=270,in=90, latex-}
]
{
\BV 
}
{\(N \times N\)}
\Bw(t) 
= 
\begin{bmatrix}
\Bv_1 & \Bv_2 & \cdots & \Bv_N
\end{bmatrix}
\begin{bmatrix}
w_1(t) \\
w_2(t) \\
\vdots \\
w_N(t)
\end{bmatrix}
=
\Bv_1 w_1(t) +
\Bv_2 w_2(t) +
\cdots
\Bv_N w_N(t).
\end{dmath}

The key to MOR is to find a way to select the important portions of these \( \Bv_i, w_i(t) \) vectors-function pairs.  This projects the system equations onto an approximation that retains the most interesting characteristics.

\section{Moment matching}

Referring back to \cref{eqn:multiphysicsL18:10} and it's Laplace transform representation, an \textAndIndex{input-to-state transfer function} \( \BF(s) \) is defined by

\begin{equation}\label{eqn:multiphysicsL18:220}
\BX(s) = \lr{ \BG + s \BC }^{-1} \BB \BU(s) = \BF(s) \BU(s)
\end{equation}

This can be expanded around the DC value ( \( s = 0 \) )

\begin{dmath}\label{eqn:multiphysicsL18:200}
\BF(s) = M_0 + M_1 s + M_2 s^2 + \cdots M_{q-1} s^{q-1} + 
\mathLabelBox
[
   labelstyle={xshift=2cm},
   linestyle={out=270,in=90, latex-}
]
{
M_q 
}
{
moments
}
s^q + \cdots
\end{dmath}

A reduced model of order \( q \) is defined as

\begin{dmath}\label{eqn:multiphysicsL18:240}
\BF(s) = M_0 + M_1 s + M_2 s^2 + \cdots M_{q-1} s^{q-1} + \tilde{M}_q s^q.
\end{dmath}

%\EndArticle
