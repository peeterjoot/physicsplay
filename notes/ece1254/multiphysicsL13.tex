%
% Copyright � 2014 Peeter Joot.  All Rights Reserved.
% Licenced as described in the file LICENSE under the root directory of this GIT repository.
%
% for template copy, run:
%
% ~/bin/ct multiphysicsL1  multiphysicsLectureN tl1
%
%\newcommand{\authorname}{Peeter Joot}
\newcommand{\email}{peeterjoot@protonmail.com}
\newcommand{\basename}{FIXMEbasenameUndefined}
\newcommand{\dirname}{notes/FIXMEdirnameUndefined/}

%\renewcommand{\basename}{multiphysicsL13}
%\renewcommand{\dirname}{notes/ece1254/}
%\newcommand{\keywords}{ECE1254H}
%\newcommand{\authorname}{Peeter Joot}
\newcommand{\onlineurl}{http://sites.google.com/site/peeterjoot2/math2013/\basename.pdf}
\newcommand{\sourcepath}{\dirname\basename.tex}
\newcommand{\generatetitle}[1]{\chapter{#1}}

\newcommand{\vcsinfo}{%
\section*{}
\noindent{\color{DarkOliveGreen}{\rule{\linewidth}{0.1mm}}}
\paragraph{Document version}
%\paragraph{\color{Maroon}{Document version}}
{
\small
\begin{itemize}
\item Available online at:\\ 
\href{\onlineurl}{\onlineurl}
\item Git Repository: \input{./.revinfo/gitRepo.tex}
\item Source: \sourcepath
\item last commit: \input{./.revinfo/gitCommitString.tex}
\item commit date: \input{./.revinfo/gitCommitDate.tex}
\end{itemize}
}
}

%\PassOptionsToPackage{dvipsnames,svgnames}{xcolor}
\PassOptionsToPackage{square,numbers}{natbib}
\documentclass{scrreprt}

\usepackage[left=2cm,right=2cm]{geometry}
\usepackage[svgnames]{xcolor}
\usepackage{peeters_layout}

\usepackage{natbib}

\usepackage[
colorlinks=true,
bookmarks=false,
pdfauthor={\authorname, \email},
backref 
]{hyperref}

% http://tex.stackexchange.com/questions/75773/how-to-reference-problems-by-the-text-label-in-an-exercise-envioronment
\usepackage[english]{cleveref}
\crefname{Exercise}{exercise}{exercises}
\Crefname{Exercise}{Exercise}{Exercises}

\RequirePackage{titlesec}
\RequirePackage{ifthen}

% http://stackoverflow.com/questions/4932910/date-in-the-tabular-environment
\makeatletter
\let\insertdate\@date
\makeatother

\titleformat{\chapter}[display]
{\bfseries\Large}
{\color{DarkSlateGrey}\filleft \authorname
\ifthenelse{\isundefined{\studentnumber}}{}{\\ \studentnumber}
\ifthenelse{\isundefined{\email}}{}{\\ \email}
\ifthenelse{\isundefined{\dateintitle}}{}{\\ \insertdate}
%\ifthenelse{\isundefined{\coursename}}{}{\\ \coursename} % put in title instead.
}
{4ex}
{\color{DarkOliveGreen}{\titlerule}\color{Maroon}
\vspace{2ex}%
\filright}
[\vspace{2ex}%
\color{DarkOliveGreen}\titlerule
]

\newcommand{\beginArtWithToc}[0]{\begin{document}\tableofcontents}
\newcommand{\beginArtNoToc}[0]{\begin{document}}
\newcommand{\EndNoBibArticle}[0]{\end{document}}
\newcommand{\EndArticle}[0]{\bibliography{Bibliography}\bibliographystyle{plainnat}\end{document}}

% 
%\newcommand{\citep}[1]{\cite{#1}}

\colorSectionsForArticle


%
%\usepackage{kbordermatrix}
%
%\beginArtNoToc
%\generatetitle{ECE1254H Modeling of Multiphysics Systems.  Lecture 13: Continuation parameters and Simulation of dynamical systems.  Taught by Prof.\ Piero Triverio}
%\chapter{Continuation parameters and Simulation of dynamical systems}
\label{chap:multiphysicsL13}
%
%\section{Disclaimer}
%
%Peeter's lecture notes from class.  These may be incoherent and rough.
%
%\chapter{Simulation of dynamical systems}
%\section{Simulation of dynamical systems}
%
%Example high level system in \cref{fig:lecture13:lecture13Fig2}.
%
%\imageFigure{../../figures/ece1254/lecture13Fig2}{Complex time dependent system}{fig:lecture13:lecture13Fig2}{0.15}
%
\section{Assembling equations automatically for dynamical systems}

\makeexample{RC circuit}{example:multiphysicsL13:220}{
The method can be demonstrated well by example.  Consider the RC circuit \cref{fig:lecture13:lecture13Fig3} for which the capacitors introduce a time dependence

\imageFigure{../../figures/ece1254/lecture13Fig3}{RC circuit}{fig:lecture13:lecture13Fig3}{0.2}

The unknowns are \( v_1(t), v_2(t) \).

The equations (KCLs) at each of the nodes are
\begin{enumerate}
\item 
\( 
\frac{v_1(t)}{R_1} 
+ C_1 \frac{dv_1}{dt}
+ \frac{v_1(t) - v_2(t)}{R_2}
+ C_2 \frac{d(v_1 - v_2)}{dt}
- i_{s,1}(t) = 0 
\)
\item 
\( 
\frac{v_2(t) - v_1(t)}{R_2} 
+ C_2 \frac{d(v_2 - v_1)}{dt}
+ \frac{v_2(t)}{R_3}
+ C_3 \frac{dv_2}{dt}
- i_{s,2}(t)
= 0
\)
\end{enumerate}

This has the matrix form
\begin{equation}\label{eqn:multiphysicsL13:140}
\begin{bmatrix}
Z_1 + Z_2 & - Z_2 \\
-Z_2 & Z_2 + Z_3
\end{bmatrix}
\begin{bmatrix}
v_1(t) \\
v_2(t)
\end{bmatrix}
+
\begin{bmatrix}
C_1 + C_2 & -C_2 \\
- C_2 & C_2 + C_3
\end{bmatrix}
\begin{bmatrix}
\frac{dv_1(t)}{dt} \\
\frac{dv_2(t}{dt})
\end{bmatrix}
=
\begin{bmatrix}
1 & 0 \\
0 & 1
\end{bmatrix}
\begin{bmatrix}
i_{s,1}(t) \\
i_{s,2}(t)
\end{bmatrix}.
\end{equation}


Observe that the capacitor between node 2 and 1 is associated with a stamp of the form

\begin{equation}\label{eqn:multiphysicsL13:180}
\begin{bmatrix}
C_2 & -C_2 \\
-C_2 & C_2
\end{bmatrix},
\end{equation}

very much like the impedance stamps of the resistor node elements.
}

The RC circuit problem has the abstract form

\begin{equation}\label{eqn:multiphysicsL13:160}
G \Bx(t) + C \frac{d\Bx(t)}{dt} = B \Bu(t),
\end{equation}

which is more general than a \textAndIndex{state space} equation of the form

\begin{equation}\label{eqn:multiphysicsL13:200}
\frac{d\Bx(t)}{dt} = A \Bx(t) + B \Bu(t).
\end{equation}

Such a system may be represented diagrammatically as in \cref{fig:lecture13:lecture13Fig4}.

\imageFigure{../../figures/ece1254/lecture13Fig4}{State space system}{fig:lecture13:lecture13Fig4}{0.1}

The \( C \) factor in this capacitance system, is generally not invertible.  For example, if consider a 10 node system with only one capacitor, for which \( C \) will be mostly zeros.
In a state space system, in all equations have a derivative.  All equations are dynamical.

The time dependent \textAndIndex{MNA} system for the RC circuit above, contains a mix of dynamical and algebraic equations.  This could, for example, be a pair of equations like

\begin{subequations}
\begin{equation}\label{eqn:multiphysicsL13:240}
   \frac{dx_1}{dt} + x_2 + 3 = 0
\end{equation}
\begin{equation}\label{eqn:multiphysicsL13:260}
   x_1 + x_2 + 3 = 0
\end{equation}
\end{subequations}

\paragraph{How to handle inductors}

A pair of nodes that contains an inductor element, as in \cref{fig:lecture13:lecture13Fig5}, has to be handled specially.

\imageFigure{../../figures/ece1254/lecture13Fig5}{Inductor configuration}{fig:lecture13:lecture13Fig5}{0.1}

The KCL at node 1 has the form

\begin{equation}\label{eqn:multiphysicsL13:280}
\cdots + i_L(t) + \cdots = 0,
\end{equation}

where 

\begin{equation}\label{eqn:multiphysicsL13:300}
v_{n_1}(t) - v_{n_2}(t) = L \frac{d i_L}{dt}.
\end{equation}

It is possible to express this in terms of \( i_L \), the variable of interest

\begin{equation}\label{eqn:multiphysicsL13:320}
i_L(t) = \inv{L} \int_0^t \lr{ v_{n_1}(\tau) - v_{n_2}(\tau) } d\tau
+ i_L(0).
\end{equation}

Expressing the problem directly in terms of such integrals makes the problem harder to solve, since the usual differential equation toolbox cannot be used directly.  An integro-differential toolbox would have to be developed.  What can be done instead is to introduce an additional unknown for each inductor current derivative \( di_L/dt \), for which an additional MNA row is introduced for that inductor scaled voltage difference.

\section{Numerical solution of differential equations}

Consider the one variable system

\begin{equation}\label{eqn:multiphysicsL13:340}
G x(t) + C \frac{dx}{dt} = B u(t),
\end{equation}

given an initial condition \( x(0) = x_0 \).  Imagine that this system has the solution sketched in \cref{fig:lecture13:lecture13Fig6}.

\imageFigure{../../figures/ece1254/lecture13Fig6}{Discrete time sampling}{fig:lecture13:lecture13Fig6}{0.2}

Very roughly, the steps for solution are of the form

\begin{enumerate}
\item Discretize time
\item Aim to find the solution at \( t_1, t_2, t_3, \cdots \)
\item Use a finite difference formula to approximate the derivative.
\end{enumerate}

There are various schemes that can be used to discretize, and compute the finite differences.

\paragraph{Forward Euler method}
\index{Euler!forward method}

One such scheme is to use the forward differences, as in \cref{fig:lecture13:lecture13Fig7}, to approximate the derivative

\begin{equation}\label{eqn:multiphysicsL13:360}
\dot{x}(t_n) \approx \frac{x_{n+1} - x_n}{\Delta t}.
\end{equation}

\imageFigure{../../figures/ece1254/lecture13Fig7}{Forward difference derivative approximation}{fig:lecture13:lecture13Fig7}{0.3}

Introducing this into \cref{eqn:multiphysicsL13:340} gives

\begin{equation}\label{eqn:multiphysicsL13:350}
G x_n + C \frac{x_{n+1} - x_n}{\Delta t} = B u(t_n),
\end{equation}

or

\begin{equation}\label{eqn:multiphysicsL13:380}
C x_{n+1} = \Delta t B u(t_n) - \Delta t G x_n + C x_n.
\end{equation}

The coefficient \( C \) must be invertible, and the next point follows immediately

%\begin{equation}\label{eqn:multiphysicsL13:400}
\boxedEquation{eqn:multiphysicsL13:400}{
x_{n+1} = \frac{\Delta t B}{C} u(t_n) + x_n \lr{ 1  - \frac{\Delta t G}{C} }
}
%\end{equation}

%\EndArticle
%\EndNoBibArticle
