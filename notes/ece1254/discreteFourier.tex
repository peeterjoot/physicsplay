%
% Copyright � 2014 Peeter Joot.  All Rights Reserved.
% Licenced as described in the file LICENSE under the root directory of this GIT repository.
%
%\newcommand{\authorname}{Peeter Joot}
\newcommand{\email}{peeterjoot@protonmail.com}
\newcommand{\basename}{FIXMEbasenameUndefined}
\newcommand{\dirname}{notes/FIXMEdirnameUndefined/}

%\renewcommand{\basename}{discreteFourier}
%\renewcommand{\dirname}{notes/ece1254/}
%%\newcommand{\dateintitle}{}
%%\newcommand{\keywords}{}
%
%\newcommand{\authorname}{Peeter Joot}
\newcommand{\onlineurl}{http://sites.google.com/site/peeterjoot2/math2013/\basename.pdf}
\newcommand{\sourcepath}{\dirname\basename.tex}
\newcommand{\generatetitle}[1]{\chapter{#1}}

\newcommand{\vcsinfo}{%
\section*{}
\noindent{\color{DarkOliveGreen}{\rule{\linewidth}{0.1mm}}}
\paragraph{Document version}
%\paragraph{\color{Maroon}{Document version}}
{
\small
\begin{itemize}
\item Available online at:\\ 
\href{\onlineurl}{\onlineurl}
\item Git Repository: \input{./.revinfo/gitRepo.tex}
\item Source: \sourcepath
\item last commit: \input{./.revinfo/gitCommitString.tex}
\item commit date: \input{./.revinfo/gitCommitDate.tex}
\end{itemize}
}
}

%\PassOptionsToPackage{dvipsnames,svgnames}{xcolor}
\PassOptionsToPackage{square,numbers}{natbib}
\documentclass{scrreprt}

\usepackage[left=2cm,right=2cm]{geometry}
\usepackage[svgnames]{xcolor}
\usepackage{peeters_layout}

\usepackage{natbib}

\usepackage[
colorlinks=true,
bookmarks=false,
pdfauthor={\authorname, \email},
backref 
]{hyperref}

% http://tex.stackexchange.com/questions/75773/how-to-reference-problems-by-the-text-label-in-an-exercise-envioronment
\usepackage[english]{cleveref}
\crefname{Exercise}{exercise}{exercises}
\Crefname{Exercise}{Exercise}{Exercises}

\RequirePackage{titlesec}
\RequirePackage{ifthen}

% http://stackoverflow.com/questions/4932910/date-in-the-tabular-environment
\makeatletter
\let\insertdate\@date
\makeatother

\titleformat{\chapter}[display]
{\bfseries\Large}
{\color{DarkSlateGrey}\filleft \authorname
\ifthenelse{\isundefined{\studentnumber}}{}{\\ \studentnumber}
\ifthenelse{\isundefined{\email}}{}{\\ \email}
\ifthenelse{\isundefined{\dateintitle}}{}{\\ \insertdate}
%\ifthenelse{\isundefined{\coursename}}{}{\\ \coursename} % put in title instead.
}
{4ex}
{\color{DarkOliveGreen}{\titlerule}\color{Maroon}
\vspace{2ex}%
\filright}
[\vspace{2ex}%
\color{DarkOliveGreen}\titlerule
]

\newcommand{\beginArtWithToc}[0]{\begin{document}\tableofcontents}
\newcommand{\beginArtNoToc}[0]{\begin{document}}
\newcommand{\EndNoBibArticle}[0]{\end{document}}
\newcommand{\EndArticle}[0]{\bibliography{Bibliography}\bibliographystyle{plainnat}\end{document}}

% 
%\newcommand{\citep}[1]{\cite{#1}}

\colorSectionsForArticle


%
%\beginArtNoToc
%
%\generatetitle{Discrete Fourier Transform}
%\label{chap:discreteFourier}

In \citep{phy487:discreteFourierTransform} a verification of the discrete Fourier transform pairs was performed.  A much different looking discrete Fourier transform pair is given in \citep{giannini2004NonlinearMicrowaveCircuitDesign} \S A.4.  This transform pair samples the points at what are called the \textAndIndex{Nykvist time instants} given by

\begin{equation}\label{eqn:discreteFourier:20}
t_k = \frac{T k}{2 N + 1}, \qquad k \in [-N, \cdots N]
\end{equation}

Note that the endpoints of these sampling points are not \( \pm T/2 \), but are instead at

\begin{equation}\label{eqn:discreteFourier:40}
\pm \frac{T}{2} \inv{1 + 1/N},
\end{equation}

which are slightly within the interior of the \( [-T/2, T/2] \) range of interest.  The reason for this slightly odd seeming selection of sampling times becomes clear if one calculate the inversion relations.

Given a periodic (\( \omega_0 T = 2 \pi \)) bandwidth limited signal evaluated only at the Nykvist times \( t_k \), 

%\begin{equation}\label{eqn:discreteFourier:60}
\boxedEquation{eqn:discreteFourier:60}{
x(t_k) = \sum_{n = -N}^N X_n e^{ j n \omega_0 t_k},
}
%\end{equation}

assume that an inversion relation can be found.  

%
% Copyright © 2014 Peeter Joot.  All Rights Reserved.
% Licenced as described in the file LICENSE under the root directory of this GIT repository.
%
To find \( X_n \) evaluate the sum

\begin{equation}\label{eqn:discreteFourier:80}
\begin{aligned}
\sum_{k = -N}^N &x(t_k) e^{-j m \omega_0 t_k} \\
&=
\sum_{k = -N}^N
\lr{
\sum_{n = -N}^N X_n e^{ j n \omega_0 t_k}
}
e^{-j m \omega_0 t_k} \\
&=
\sum_{n = -N}^N X_n
\sum_{k = -N}^N
e^{ j (n -m )\omega_0 t_k}
\end{aligned}
\end{equation}

This interior sum has the value \( 2 N + 1 \) when \( n = m \).  For \( n \ne m \), and
\( a = e^{j (n -m ) \frac{2 \pi}{2 N + 1}} \), this is

\begin{dmath}\label{eqn:discreteFourier:100}
\sum_{k = -N}^N
e^{ j (n -m )\omega_0 t_k}
=
\sum_{k = -N}^N
e^{ j (n -m )\omega_0 \frac{T k}{2 N + 1}}
=
\sum_{k = -N}^N a^k
=
a^{-N} \sum_{k = -N}^N a^{k+ N}
=
a^{-N} \sum_{r = 0}^{2 N} a^{r}
=
a^{-N} \frac{a^{2 N + 1} - 1}{a - 1}.
\end{dmath}

Since \( a^{2 N + 1} = e^{2 \pi j (n - m)} = 1 \), this sum is zero when \( n \ne m \).  This means that

\begin{equation}\label{eqn:discreteFourier:120}
\sum_{k = -N}^N
e^{ j (n -m )\omega_0 t_k} = (2 N + 1) \delta_{n,m}.
\end{equation}

Substitution back into \cref{eqn:discreteFourier:80} proves the Fourier inversion relation \cref{eqn:discreteFourier:60}.

%which provides the desired Fourier inversion relation
%which provides the desired Fourier inversion relation
%
%%\begin{equation}\label{eqn:discreteFourier:140}
%\boxedEquation{eqn:discreteFourier:140}{
%X_m = \inv{2 N + 1} \sum_{k = -N}^N x(t_k) e^{-j m \omega_0 t_k}.
%}
%%\end{equation}


%\EndArticle
