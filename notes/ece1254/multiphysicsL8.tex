%
% Copyright � 2014  Ahmed Dorrah, Peeter Joot.  All Rights Reserved.
% Licenced as described in the file LICENSE under the root directory of this GIT repository.
%
%\newcommand{\authorname}{Peeter Joot}
\newcommand{\email}{peeterjoot@protonmail.com}
\newcommand{\basename}{FIXMEbasenameUndefined}
\newcommand{\dirname}{notes/FIXMEdirnameUndefined/}

%\renewcommand{\basename}{multiphysicsL8}
%\renewcommand{\dirname}{notes/ece1254/}
%\newcommand{\keywords}{ECE1254H}
%\newcommand{\authorname}{Peeter Joot}
\newcommand{\onlineurl}{http://sites.google.com/site/peeterjoot2/math2013/\basename.pdf}
\newcommand{\sourcepath}{\dirname\basename.tex}
\newcommand{\generatetitle}[1]{\chapter{#1}}

\newcommand{\vcsinfo}{%
\section*{}
\noindent{\color{DarkOliveGreen}{\rule{\linewidth}{0.1mm}}}
\paragraph{Document version}
%\paragraph{\color{Maroon}{Document version}}
{
\small
\begin{itemize}
\item Available online at:\\ 
\href{\onlineurl}{\onlineurl}
\item Git Repository: \input{./.revinfo/gitRepo.tex}
\item Source: \sourcepath
\item last commit: \input{./.revinfo/gitCommitString.tex}
\item commit date: \input{./.revinfo/gitCommitDate.tex}
\end{itemize}
}
}

%\PassOptionsToPackage{dvipsnames,svgnames}{xcolor}
\PassOptionsToPackage{square,numbers}{natbib}
\documentclass{scrreprt}

\usepackage[left=2cm,right=2cm]{geometry}
\usepackage[svgnames]{xcolor}
\usepackage{peeters_layout}

\usepackage{natbib}

\usepackage[
colorlinks=true,
bookmarks=false,
pdfauthor={\authorname, \email},
backref 
]{hyperref}

% http://tex.stackexchange.com/questions/75773/how-to-reference-problems-by-the-text-label-in-an-exercise-envioronment
\usepackage[english]{cleveref}
\crefname{Exercise}{exercise}{exercises}
\Crefname{Exercise}{Exercise}{Exercises}

\RequirePackage{titlesec}
\RequirePackage{ifthen}

% http://stackoverflow.com/questions/4932910/date-in-the-tabular-environment
\makeatletter
\let\insertdate\@date
\makeatother

\titleformat{\chapter}[display]
{\bfseries\Large}
{\color{DarkSlateGrey}\filleft \authorname
\ifthenelse{\isundefined{\studentnumber}}{}{\\ \studentnumber}
\ifthenelse{\isundefined{\email}}{}{\\ \email}
\ifthenelse{\isundefined{\dateintitle}}{}{\\ \insertdate}
%\ifthenelse{\isundefined{\coursename}}{}{\\ \coursename} % put in title instead.
}
{4ex}
{\color{DarkOliveGreen}{\titlerule}\color{Maroon}
\vspace{2ex}%
\filright}
[\vspace{2ex}%
\color{DarkOliveGreen}\titlerule
]

\newcommand{\beginArtWithToc}[0]{\begin{document}\tableofcontents}
\newcommand{\beginArtNoToc}[0]{\begin{document}}
\newcommand{\EndNoBibArticle}[0]{\end{document}}
\newcommand{\EndArticle}[0]{\bibliography{Bibliography}\bibliographystyle{plainnat}\end{document}}

% 
%\newcommand{\citep}[1]{\cite{#1}}

\colorSectionsForArticle



%\beginArtNoToc
%\generatetitle{ECE1254H Modeling of Multiphysics Systems.  Lecture 8: Conjugate gradient method.  Taught by Prof.\ Piero Triverio}
%\chapter{Conjugate gradient method}
\label{chap:multiphysicsL8}

%\section{Disclaimer}
%
%Peeter's lecture notes transcribed from notes 

\section{Recap: Summary of Gradient method}
\index{gradient method}

The gradient method allowed for a low cost iterative solution of a linear system

\begin{equation}\label{eqn:multiphysicsL8:20}
M \Bx = \Bb,
\end{equation}

without a requirement for complete factorization.
The residual was defined as the difference between the application of any trial solution \( \By \) to \( M \) and \( \Bb \) 

\begin{equation}\label{eqn:multiphysicsL8:40}
\Br = \Bb - M \By.
\end{equation}

An energy function was introduced

\begin{equation}\label{eqn:multiphysicsL8:60}
\Psi(\By) = \inv{2} \By^\T M \By - \By^\T \Bb, 
\end{equation}

which has an extremum at the point of solution.  The goal was to attempt to
following the direction of steepest decent \( \Br^{(k)} = - \spacegrad \Psi \) with the hope of finding the minimum of the energy function.  That iteration is described by

\begin{equation}\label{eqn:multiphysicsL8:80}
\Bx^{(k + 1 )} =
\Bx^{(k  )} + \alpha_k \Br^{(k)},
\end{equation}

and illustrated in \cref{fig:lecture8:lecture8Fig1}.

\imageFigure{../../figures/ece1254/lecture8Fig1}{Gradient descent.}{fig:lecture8:lecture8Fig1}{0.3}

The problem of the \textAndIndex{gradient method} is that it introduces multiple paths as sketched in \cref{fig:lecture8:lecture8Fig2}.

\imageFigure{../../figures/ece1254/lecture8Fig2}{Gradient descent iteration.}{fig:lecture8:lecture8Fig2}{0.15}

which lengthens the total distance that has to be traversed in the iteration.

\section{Conjugate gradient method}
\index{conjugate gradient}

The Conjugate gradient method makes the residual at step \( k \) orthogonal to all previous search directions

\begin{enumerate}
\item \( \Br^{(1)} \perp \Bd^{(0)} \)
\item \( \Br^{(1)} \perp \Bd^{(0)}, \Bd^{(1)} \)
\item \( \cdots \)
\end{enumerate}

After \( n \) iterations, the residual will be zero.

\paragraph{First iteration}

Given an initial guess \( \Bx^{(0)} \), proceed as in the gradient method

\begin{equation}\label{eqn:multiphysicsL8:100}
\Bd^{(0)} = - \spacegrad \Psi(\Bx^{(0)}) = \Br^{(0)},
\end{equation}

\begin{equation}\label{eqn:multiphysicsL8:120}
\Bx^{(1)} = \Bx^{(0)} + \alpha_0 \Br^{(0)},
\end{equation}

with

\begin{equation}\label{eqn:multiphysicsL8:140}
\alpha_0 = \frac
{ \lr{\Br^{(0)}}^\T \Br^{(0)} }
{ \lr{\Br^{(0)}}^\T M \Br^{(0)} },
\end{equation}

so that the residual is
\begin{dmath}\label{eqn:multiphysicsL8:160}
\Br^{(1)} 
= \Bb - M \Bx^{(1)} 
= \Bb - M \Bx^{(0)} - \alpha_0 M \Br^{(0)}
= \Bb - \alpha M \Br^{(0)}.
\end{dmath}

An orthogonality condition \( \Br^{(1)} \perp \Bd^{(0)} \) is desired.

\paragraph{Proof:}

\begin{dmath}\label{eqn:multiphysicsL8:180}
\innerprod{\Bd^{(0)}}{ \Br^{(1)}}
=
\innerprod{\Bd^{(0)}}{ \Br^{(0)}}
-\alpha_0
\innerprod{\Bd^{(0)}}{ M \Br^{(0)}}
=
\innerprod{\Bd^{(0)}}{ \Br^{(0)}}
-\alpha_0
\innerprod{\Br^{(0)}}{ M \Br^{(0)}}.
\end{dmath}

\paragraph{Second iteration}

\begin{equation}\label{eqn:multiphysicsL8:200}
\Bx^{(2)} = \Bx^{(1)} + \alpha_1 \Bd^{(1)}.
\end{equation}

The conditions to satisfy are
\begin{subequations}
\begin{equation}\label{eqn:multiphysicsL8:220}
\Bd^{(0)} \perp \Br^{(2)}
\end{equation}
\begin{equation}\label{eqn:multiphysicsL8:240}
\Bd^{(1)} \perp \Br^{(2)}
\end{equation}
\end{subequations}

Observe that the orthogonality condition of \cref{eqn:multiphysicsL8:220} is satisfied 

\begin{dmath}\label{eqn:multiphysicsL8:260}
\innerprod{\Bd^{(0)}}{ \Br^{(2)}}
=
\innerprod{\Bd^{(0)}}{ \Bb - M \Bx^{(2)}}
=
\innerprod{\Bd^{(0)}}{ \Bb - M \Bx^{(1)} - \alpha_1 M \Bd^{(1)}}
=
\mathLabelBox
[
   labelstyle={xshift=2cm},
   linestyle={out=270,in=90, latex-}
]
{
\innerprod{\Bd^{(0)}}{\Bb - M \Bx^{(1)}}
}
{\(= 0\) because \( \Bd^{(0)} \perp \Br^{(1)}\) }
-\alpha_1
\innerprod{\Bd^{(0)}}{ \alpha_1 M \Bd^{(1)}}
\end{dmath}

This will be zero if an \( M \) orthogonality or conjugate zero condition can be imposed

\begin{equation}\label{eqn:multiphysicsL8:280}
\innerprod{\Bd^{(0)}}{M \Bd^{(1)}} = 0.
\end{equation}

To find a new search direction \( \Bd^{(1)} = \Br^{(1)} - \beta_0 \Bd^{(0)} \),
\cref{eqn:multiphysicsL8:280} is now imposed.
The \( \Br^{(1)} \) is the term from the standard gradient method, and the \( \beta_0 \Bd^{(0)} \) term is the conjugate gradient correction.  This gives \( \beta_0 \)

\begin{equation}\label{eqn:multiphysicsL8:300}
\beta_0 = \frac
{
\innerprod{\Bd^{(0)}}{M \Br^{(1)}}
}
{
\innerprod{\Bd^{(0)}}{M \Bd^{(0)}}
}
\end{equation}

Imposing \cref{eqn:multiphysicsL8:240} gives

\begin{equation}\label{eqn:multiphysicsL8:320}
\alpha_1 = \frac
{
\innerprod{\Bd^{(0)}}{\Br^{(1)}}
}
{
\innerprod{\Bd^{(1)}}{M \Bd^{(1)}}
}.
\end{equation}

\paragraph{Next iteration}

\begin{equation}\label{eqn:multiphysicsL8:340}
\begin{aligned}
\Bx^{(k + 1)} &= \Bx^{(k )} + \alpha_k \Bd^{(k)} \\
\Bd^{(k )} &= \Br^{(k )} - \beta_{k-1} \Bd^{(k-1)}
\end{aligned}
\end{equation}

The conditions to impose are

\begin{equation}\label{eqn:multiphysicsL8:360}
\begin{aligned}
	\Bd^{(0)} & \perp \Br^{(k + 1)} \\
	\vdots & \vdots \\
	\Bd^{(k-1)} & \perp \Br^{(k + 1)}
\end{aligned}
\end{equation}

However, there are only \( 2 \) degrees of freedom \( \alpha, \beta \), despite having many conditions to impose.

Impose the following to find \( \beta_{k-1} \)

\begin{equation}\label{eqn:multiphysicsL8:380}
	\Bd^{(k-1)} \perp \Br^{(k + 1)}
\end{equation}

(See slides for more)

\section{Full Algorithm}

%
% Copyright � 2014  Ahmed Dorrah, Peeter Joot.  All Rights Reserved.
% Licenced as described in the file LICENSE under the root directory of this GIT repository.
%

The conjugate gradient algorithm presented in the slides (without preconditioning) was

\begin{algorithmic}
\STATE \( \Bd^{(0)} = \Br^{(0)} \)
\REPEAT
\STATE \( \alpha_k = \frac{ \lr{ \Bd^{(k)} }^\T \Br^{(k)} }{ \lr{ \Bd^{(k)} }^\T M \Bd^{(k)} } \)
\STATE \( \Bx^{(k+1)} = \Bx^{(k)} + \alpha_k \Bd^{(k)} \)
\STATE \( \Br^{(k+1)} = \Br^{(k)} - \alpha_k M \Bd^{(k)} \)
\STATE \( \beta_k = \frac{ \lr{ M \Bd^{(k)} }^\T \Br^{(k+1)} }{ \lr{ M \Bd^{(k)} }^\T \Bd^{(k)} } \)
\STATE \( \Bd^{(k+1)} = \Br^{(k+1)} - \beta_k \Bd^{(k)} \)
\UNTIL{converged}
\end{algorithmic}

The repeated calculations are undesirable for actually coding this algorithm.  First introduce a temporary for the matrix product.  Introducing a temporary variable helps a bit

\begin{algorithmic}
\STATE \( \Bd^{(0)} = \Br^{(0)} \)
\REPEAT
\STATE \( \Bq = M \Bd^{(k)} \)
\STATE \( \alpha_k = \frac{ \lr{ \Bd^{(k)} }^\T \Br^{(k)} }{ \lr{ \Bd^{(k)} }^\T \Bq } \)
\STATE \( \Bx^{(k+1)} = \Bx^{(k)} + \alpha_k \Bd^{(k)} \)
\STATE \( \Br^{(k+1)} = \Br^{(k)} - \alpha_k \Bq \)
\STATE \( \beta_k = \frac{ \Bq^\T \Br^{(k+1)} }{ \Bq^\T \Bd^{(k)} } \)
\STATE \( \Bd^{(k+1)} = \Br^{(k+1)} - \beta_k \Bd^{(k)} \)
\UNTIL{converged}
\end{algorithmic}

However, this still has a lot more computation than the algorithm specified in \citep{shewchuk1994introduction} \S B.2.  It looks like the orthogonality properties can be used to recast the \( \Bd^{(k)} \cdot \Br^{(k)} \) products in terms of \( \Br^{(k)} \)

\begin{equation}\label{eqn:multiphysicsL8:400}
\lr{ \Bd^{(k)} }^\T \Br^{(k)} 
=
\Br^{(k)} \cdot \lr{ \Br^{(k)} + \beta_{k-1} \Bd^{(k-1)} 
},
\end{equation}

but since the new residual is orthogonal to all the previous search directions \( \Br^{(k)} \cdot \Bd^{(k-1)} = 0 \).  The transformed direction vector \( \Bq \) is a scaled difference of residuals.  Taking dot products

\begin{dmath}\label{eqn:multiphysicsL8:420}
\Bq \cdot \Br^{(k+1)}
=
\inv{\alpha_k} \lr{ \Br^{(k)} - \Br^{(k+1)} } \cdot \Br^{(k+1)}
=
\inv{\alpha_k} \lr{ \cancel{\Bd^{(k)}} - \beta_{k-1} \cancel{\Bd^{(k-1)}} - \Br^{(k+1)} } \cdot \Br^{(k+1)}
=
-\inv{\alpha_k} \Br^{(k+1)} \cdot \Br^{(k+1)}.
\end{dmath}

This gives

\begin{subequations}
\begin{equation}\label{eqn:multiphysicsL8:440}
\alpha_k = \frac{ \lr{ \Br^{(k)} }^\T \Br^{(k)} }{ \lr{ \Bd^{(k)} }^\T \Bq }
\end{equation}
\begin{equation}\label{eqn:multiphysicsL8:460}
\beta_k = -\frac{ \lr{\Br^{(k+1)}}^\T \Br^{(k+1)} }{ \alpha_k \Bq^\T \Bd^{(k)} },
\end{equation}
\end{subequations}

A final elimination of \( \alpha_k \) from \cref{eqn:multiphysicsL8:460} gives

\begin{equation}\label{eqn:multiphysicsL8:480}
\beta_k = 
-\frac{ \lr{\Br^{(k+1)}}^\T \Br^{(k+1)} }{ 
\lr{ \Br^{(k)} }^\T \Br^{(k)} 
}.
\end{equation}

All the pieces put together yield

\begin{algorithmic}
\STATE \( \Bd^{(0)} = \Br^{(0)} \)
\REPEAT
\STATE \( \Bq = M \Bd^{(k)} \)
\STATE \( \alpha_k = \frac{ \lr{ \Br^{(k)} }^\T \Br^{(k)} }{ \lr{ \Bd^{(k)} }^\T \Bq } \)
\STATE \( \Bx^{(k+1)} = \Bx^{(k)} + \alpha_k \Bd^{(k)} \)
\STATE \( \Br^{(k+1)} = \Br^{(k)} - \alpha_k \Bq \)
\STATE \( \beta_k = 
-\frac{ \lr{\Br^{(k+1)}}^\T \Br^{(k+1)} }{ 
\lr{ \Br^{(k)} }^\T \Br^{(k)} 
}
\)
\STATE \( \Bd^{(k+1)} = \Br^{(k+1)} - \beta_k \Bd^{(k)} \)
\UNTIL{converged}
\end{algorithmic}

This is now consistent with eqns 45-49 of \citep{shewchuk1994introduction}, with the exception that the sign of the \( \beta_k \) term has been flipped.



% http://www.personal.ceu.hu/tex/breaking.htm
% the table below was getting interleaved with the algorithm above.
\clearpage
\section{Order analysis}
\index{conjugate gradient!order analysis}

Note that for \( \Bx, \By \) in \R{n}, \( \innerprod{\Bx}{\By} = \Bx^\T \By \) is \( O(n) \), and \( M \Bx \) is \( O(n^2) \).  

A conjugate gradient and LU comparision is given in \cref{tab:multiphysicsL8:400}, where \( k < n \)

\captionedTable{LU vs Conjugate gradient order.}{tab:multiphysicsL8:400}{
\begin{tabular}{|l|l|l|}
	\hline 
                   & Full     & Sparse         \\ \hline
LU                 & \( O(n^3) \)   & \( O(n^{1.2-1.8}) \) \\ \hline
Conjugate gradient & \( O(k n^2) \) & \( 5.4          \) \\ \hline 
\end{tabular}
}

\paragraph{Final comments}

\begin{enumerate}
\item How to select \( \Bx^{(0)} \) ?  An initial rough estimate can often be found by solving a simplified version of the problem.
\item Is it neccessary to reserve memory space to store \( M \)?  No.  If \( \By = M \Bz \), the product can be calculated without physically storing the full matrix \( M \).
\end{enumerate}

%\EndArticle
%\EndNoBibArticle
