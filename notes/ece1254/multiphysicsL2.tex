%
% Copyright � 2014 Peeter Joot.  All Rights Reserved.
% Licenced as described in the file LICENSE under the root directory of this GIT repository.
%
%\newcommand{\authorname}{Peeter Joot}
\newcommand{\email}{peeterjoot@protonmail.com}
\newcommand{\basename}{FIXMEbasenameUndefined}
\newcommand{\dirname}{notes/FIXMEdirnameUndefined/}

%
%\renewcommand{\basename}{multiphysicsL2}
%\renewcommand{\dirname}{notes/ece1254/}
%\newcommand{\keywords}{ECE1254H}
%\newcommand{\authorname}{Peeter Joot}
\newcommand{\onlineurl}{http://sites.google.com/site/peeterjoot2/math2013/\basename.pdf}
\newcommand{\sourcepath}{\dirname\basename.tex}
\newcommand{\generatetitle}[1]{\chapter{#1}}

\newcommand{\vcsinfo}{%
\section*{}
\noindent{\color{DarkOliveGreen}{\rule{\linewidth}{0.1mm}}}
\paragraph{Document version}
%\paragraph{\color{Maroon}{Document version}}
{
\small
\begin{itemize}
\item Available online at:\\ 
\href{\onlineurl}{\onlineurl}
\item Git Repository: \input{./.revinfo/gitRepo.tex}
\item Source: \sourcepath
\item last commit: \input{./.revinfo/gitCommitString.tex}
\item commit date: \input{./.revinfo/gitCommitDate.tex}
\end{itemize}
}
}

%\PassOptionsToPackage{dvipsnames,svgnames}{xcolor}
\PassOptionsToPackage{square,numbers}{natbib}
\documentclass{scrreprt}

\usepackage[left=2cm,right=2cm]{geometry}
\usepackage[svgnames]{xcolor}
\usepackage{peeters_layout}

\usepackage{natbib}

\usepackage[
colorlinks=true,
bookmarks=false,
pdfauthor={\authorname, \email},
backref 
]{hyperref}

% http://tex.stackexchange.com/questions/75773/how-to-reference-problems-by-the-text-label-in-an-exercise-envioronment
\usepackage[english]{cleveref}
\crefname{Exercise}{exercise}{exercises}
\Crefname{Exercise}{Exercise}{Exercises}

\RequirePackage{titlesec}
\RequirePackage{ifthen}

% http://stackoverflow.com/questions/4932910/date-in-the-tabular-environment
\makeatletter
\let\insertdate\@date
\makeatother

\titleformat{\chapter}[display]
{\bfseries\Large}
{\color{DarkSlateGrey}\filleft \authorname
\ifthenelse{\isundefined{\studentnumber}}{}{\\ \studentnumber}
\ifthenelse{\isundefined{\email}}{}{\\ \email}
\ifthenelse{\isundefined{\dateintitle}}{}{\\ \insertdate}
%\ifthenelse{\isundefined{\coursename}}{}{\\ \coursename} % put in title instead.
}
{4ex}
{\color{DarkOliveGreen}{\titlerule}\color{Maroon}
\vspace{2ex}%
\filright}
[\vspace{2ex}%
\color{DarkOliveGreen}\titlerule
]

\newcommand{\beginArtWithToc}[0]{\begin{document}\tableofcontents}
\newcommand{\beginArtNoToc}[0]{\begin{document}}
\newcommand{\EndNoBibArticle}[0]{\end{document}}
\newcommand{\EndArticle}[0]{\bibliography{Bibliography}\bibliographystyle{plainnat}\end{document}}

% 
%\newcommand{\citep}[1]{\cite{#1}}

\colorSectionsForArticle


%
%\usepackage{listing}
%\usepackage{algorithm}
%\usepackage{algorithmic}
%
%\beginArtNoToc
%\generatetitle{ECE1254H Modeling of Multiphysics Systems.  Lecture 2: Assembling system equations automatically.  Taught by Prof.\ Piero Triverio}
%\section{Assembling system equations automatically}
\label{chap:multiphysicsL2}

%\section{Disclaimer}
%
%Peeter's lecture notes from class.  These may be incoherent and rough.

\section{Assembling system equations automatically.  Node/branch method}

Consider the sample circuit of \cref{fig:lecture2:lecture2Fig1}.

%\imageFigure{../../figures/ece1254/lecture2Fig1}{Sample resistive circuit}{fig:lecture2:lecture2Fig1}{0.3}
\imageFigure{../../figures/ece1254/1254-sample-resistive-circuit.pdf}{Sample resistive circuit.}{fig:lecture2:lecture2Fig1}{0.3}


\paragraph{Step 1.  Choose unknowns:}  For this problem, take

\begin{itemize}
\item
node voltages: \(V_1, V_2, V_3, V_4\)
\item
branch currents: \(i_A, i_B, i_C, i_D, i_E\)
\end{itemize}

No additional labels are required for the source current sources.
A reference node is always introduced, given the node number zero.

For a circuit with \(N\)
nodes, and
\(B\) resistors, there will be
\(N-1\)
unknown node voltages \index{node voltage} and \(B\) unknown branch currents \index{branch current}, for a total number of
\(N - 1 + B\) unknowns.

\paragraph{Step 2.  Conservation equations:}  KCL

\begin{itemize}
   \item 0: \(i_A + i_E - i_D = 0\)
   \item 1: \(-i_A + i_B + i_{S,A} = 0\)
   \item 2: \(-i_B + i_{S,B} - i_E + i_{S,C} = 0\)
   \item 3: \(i_C - i_{S,C} = 0\)
   \item 4: \(-i_{S,A} - i_{S,B} + i_D - i_C = 0\)
\end{itemize}

Grouping unknown currents, this is

\begin{itemize}
   \item 0: \(i_A + i_E - i_D = 0\)
   \item 1: \(-i_A + i_B = -i_{S,A}\)
   \item 2: \(-i_B -i_E = -i_{S,B} -i_{S,C}\)
   \item 3: \(i_C = i_{S,C}\)
   \item 4: \(i_D - i_C = i_{S,A} + i_{S,B}\)
\end{itemize}

Note that one of these equations is redundant (sum 1-4).  In a circuit with
\(N\)
nodes, are are at most \(N-1\) independent KCLs.  In matrix form

\begin{equation}\label{eqn:multiphysicsL2:20}
\begin{bmatrix}
   -1 & 1 & 0 & 0 & 0 \\
   0 & -1 & 0 & 0 & -1 \\
   0 & 0 & 1 & 0 & 0 \\
   0 & 0 & -1 & 1 & 0
\end{bmatrix}
\begin{bmatrix}
   i_A \\
   i_B \\
   i_C \\
   i_D \\
   i_E \\
\end{bmatrix}
=
\begin{bmatrix}
   -i_{S,A} \\
   -i_{S,B} -i_{S,C} \\
   i_{S,C} \\
   i_{S,A} + i_{S,B}
\end{bmatrix}
\end{equation}

This first matrix of ones and minus ones is called the \textAndIndex{incidence matrix}
\(A\).  This matrix has
\(B\) columns and
\(N-1\) rows.  The matrix of known currents is called
\(\BI_S\), and the branch currents called
\(\BI_B\).  That is

\begin{equation}\label{eqn:multiphysicsL2:40}
   A \BI_B = \BI_S.
\end{equation}

Observe that there is a plus and minus one in all columns except for those columns impacted by the neglect of the reference node current conservation equation.

\paragraph{Algorithm for filling \(A\)}

In the input file, to describe a resistor of \cref{fig:lecture2:lecture2Fig2}, you'll have a line of the form

%\captionedTable{Resistor spice line}{tab:lecture2:1}{
\begin{tabular}{lllll}
R & name & \(n_1\)
& \(n_2\) & value
\end{tabular}
%}

\imageFigure{../../figures/ece1254/lecture2Fig2}{Resistor node convention.}{fig:lecture2:lecture2Fig2}{0.1}

The algorithm to process resistor lines is

\makealgorithm{Resistor line handling.}{alg:multiphysicsL2:1}{
\STATE \(A \leftarrow 0\)
\STATE \(ic \leftarrow 0\)
\FORALL{resistor lines}
\STATE \(ic \leftarrow ic + 1\), adding one column to \(A\)
\IF{\(n_1 != 0\)}
\STATE \(A(n_1, ic) \leftarrow +1\)
\ENDIF
\IF{\(n_2 != 0\)}
\STATE \(A(n_2, ic) \leftarrow -1\)
\ENDIF
\ENDFOR
}

\paragraph{Algorithm for filling \(\BI_S\)}

Current sources, as in \cref{fig:lecture2:lecture2Fig3}, a line will have the specification (FIXME: confirm... didn't write this down).

\begin{tabular}{lllll}
I & name & \(n_1\)
& \(n_2\) & value
\end{tabular}

\imageFigure{../../figures/ece1254/lecture2Fig3}{Current source conventions.}{fig:lecture2:lecture2Fig3}{0.1}

\makealgorithm{Current line handling.}{alg:multiphysicsL2:2}{
\STATE \(\BI_S = \text{zeros}(N-1, 1)\)
\FORALL{current lines}
\STATE \(\BI_S(n_1) \leftarrow \BI_S(n_1) - \text{value}\)
\STATE \(\BI_S(n_2) \leftarrow \BI_S(n_2) + \text{value}\)
\ENDFOR
}

\paragraph{Step 3.  Constitutive equations:}

\begin{equation}\label{eqn:multiphysicsL2:60}
\begin{bmatrix}
i_A \\
i_B \\
i_C \\
i_D \\
i_E
\end{bmatrix}
=
\begin{bmatrix}
1/R_A & & & & \\
      & 1/R_B & & & & \\
      & & 1/R_C & & & \\
      & & & 1/R_D & & \\
      & & & & & 1/R_E
\end{bmatrix}
\begin{bmatrix}
v_A \\
v_B \\
v_C \\
v_D \\
v_E
\end{bmatrix}
\end{equation}

Or

\begin{equation}\label{eqn:multiphysicsL2:80}
   \BI_B = \alpha \BV_B,
\end{equation}

where \(\BV_B\) are the branch voltages, not unknowns of interest directly.  That can be written

\begin{equation}\label{eqn:multiphysicsL2:100}
\begin{bmatrix}
v_A \\
v_B \\
v_C \\
v_D \\
v_E
\end{bmatrix}
=
\begin{bmatrix}
   -1 & & &  \\
   1 & -1 & & \\
   & & 1 & -1 \\
   & &  & 1 \\
   & -1 & &
\end{bmatrix}
\begin{bmatrix}
v_1 \\
v_2 \\
v_3 \\
v_4
\end{bmatrix}
\end{equation}

Observe that this %\(\alpha\)  %FIXME: label
is the transpose of \(A\), so

\begin{equation}\label{eqn:multiphysicsL2:120}
\BV_B = A^\T \BV_N.
\end{equation}

\paragraph{Solving}

\begin{itemize}
\item
KCLs: \(A \BI_B = \BI_S\)
\item
constitutive: \(\BI_B = \alpha \BV_B \implies \BI_B = \alpha A^\T \BV_N\)
\item
branch and node voltages: \(\BV_B = A^\T \BV_N\)
\end{itemize}

In block matrix form, this is

\begin{equation}\label{eqn:multiphysicsL2:140}
\begin{bmatrix}
   A & 0 \\
   I & -\alpha A^\T
\end{bmatrix}
\begin{bmatrix}
   \BI_B \\
   \BV_N
\end{bmatrix}
=
\begin{bmatrix}
   \BI_S \\
   0
\end{bmatrix}.
\end{equation}

Is it square?  This can be observing to be the case after noting that there are

\begin{itemize}
\item \(N-1\)
rows in \(A\)
, and \(B\) columns.
\item \(B\)
rows in \(I\).
\item \(N-1\) columns.
\end{itemize}

%\EndArticle
%\EndNoBibArticle
