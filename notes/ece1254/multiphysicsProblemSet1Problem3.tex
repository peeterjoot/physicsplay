%
% Copyright � 2014 Peeter Joot.  All Rights Reserved.
% Licenced as described in the file LICENSE under the root directory of this GIT repository.
%
\index{heat equation}
\makeproblem{Heat conduction.}{multiphysics:problemSet1:3}{ 
\makesubproblem
{
In this problem we will examine the heat conducting bar basic
example, but will consider the case of a ``leaky'' bar to give you practice developing a numerical technique for a new physical problem. With
an appropriate input file, the simulator you developed in problem 1 can
be used to solve numerically the one-dimensional Poisson equation with
arbitrary boundary conditions. The Poisson equation can be used to determine steady-state temperature distribution in a heat-conducting bar, as in

\begin{equation}\label{eqn:multiphysicsProblemSet1Problem3:10}
\PDSq{x}{T(x)} = \frac{\kappa_{\textrm{a}}}{\kappa_{\textrm{m}}}\lr{ T(x) - T_0 } - \frac{H(x)}{\kappa_{\textrm{m}}}
\end{equation}

where \( T(x) \) is the temperature at a point in space x, \( H(x) \) is the heat generated at x, \( \kappa_{\textrm{m}} \) is the thermal conductivity along the metal bar, and \( \kappa_{\textrm{a}} \) is the thermal conductivity from the bar to the surrounding air. The temperature \( T_0 \) is the surrounding air temperature. The ratio \( \kappa_{\textrm{a}}/\kappa_{\textrm{m}} \) will be small as heat moves much more easily along the bar than dissipates from the bar into the surrounding air.

Use your Matlab simulator to numerically solve the above Poisson equation for \( T(x), x \in [0,1] \) , given \( H(x) = 50 \sin^2(2 \pi x) \) for \( x \in [0,1], \kappa_{\textrm{a}} = 0.001, and \kappa_{\textrm{m}} = 0.1 \). In addition, assume the ambient air temperature is \( T_0 = 400 \), and \( T(0) = 250 \) and \( T(1) = 250 \). The boundary conditions at \( x = 0 \) and \( x = 1 \) model heat sink connections to a cool metal cabinet at both ends of the package. That is, it is assumed that the heat sink connections will insure both ends of the package are fixed at near room temperature.

To represent equation \cref{eqn:multiphysicsProblemSet1Problem3:10} as a circuit, you must first discretize your bar along the spatial variable x in small sections of length \( \Delta x\), and approximate the derivatives using a finite difference formula, e.g., 

\begin{equation}\label{eqn:multiphysicsProblemSet1Problem3:11}
\PDSq{x}{T(x)} \approx
\inv{\Delta x} \lr{
\frac{T(x + \Delta x) - T(x)}{\Delta x}
-\frac{T(x) - T(x - \Delta x)}{\Delta x}
}
\end{equation}

Then, interpret the discretized equation as a KCL using the electrothermal analogy where temperature corresponds to node voltage, and heat flow to current. Draw the equivalent circuit you obtained.
}{multiphysics:problemSet1:3a}

\makesubproblem
{
Plot \(T(x)\) in \(x \in [0,1]\).
}{multiphysics:problemSet1:3b}

\makesubproblem
{
In your numerical calculation, how did you choose \(\Delta x\)? Justify the choice of \(\Delta x\).
}{multiphysics:problemSet1:3c}

\makesubproblem
{
	Now use your simulator to numerically solve the above equation for \( T(x), x \in [0,1]\), given \( H(x) = 50 \) for \( x \in [0,1], \kappa_{\textrm{a}} = 0.001 \) , and \( \kappa_{\textrm{m}} = 0.1\). In addition, assume the ambient air temperature is \(T_0 = 400\), and there is not heat flow at both ends of the bar. The zero heat flow boundary condition at \( x = 0 \) and \( x = 1 \) implies that there are no heat sinks at the ends of the package. Since heat flow is given by 
	
\begin{equation}\label{eqn:multiphysicsProblemSet1Problem3:12}
\text{heatflow} = \kappa \PD{x}{T}.
\end{equation}

zero heat flow at the boundaries means that \( T(0) \) and \( T(1) \) are unknown, but \( \PDi{x}{T}(0) = 0 \), and \( \PDi{x}{T}(1) = 0 \).

Given the zero-heat-flow boundary condition, what is the new equivalent
circuit? How the different boundary condition maps into the equivalent circuit?
}{multiphysics:problemSet1:3d}

\makesubproblem
{
Plot the new temperature profile.
}{multiphysics:problemSet1:3e}

\makesubproblem
{
Explain the temperature distributions that you obtained from a physical standpoint.
}{multiphysics:problemSet1:3f}
} % makeproblem


\makeanswer{multiphysics:problemSet1:3}{ 

\makeSubAnswer{ }{multiphysics:problemSet1:3a}

To discretize the equation, divide the interval into \( N - 1 \) segments, as illustrated in \cref{fig:ps1p3:ps1p3Fig8}.

\imageFigure{../../figures/ece1254/ps1p3Fig8}{Discretization intervals.}{fig:ps1p3:ps1p3Fig8}{0.1}

Let

\begin{equation}\label{eqn:multiphysicsProblemSet1Problem3:20}
\begin{aligned}
	x^i &= (i -1)\Delta x
, \quad i \in \{ 1, \cdots, N \} \\
H^i &= 50 \sin^2 \lr{ 2 \pi x^i } 
, \quad i \in \{ 1, \cdots, N \} \\
T^i &= T( x^i ) 
, \quad i \in \{ 1, \cdots, N \} \\
Q^i &= \frac{T^{i+1} - T^i}{\Delta x}
, \quad i \in \{ 1, \cdots, N-1 \} \\
\end{aligned}
\end{equation}

FIXME: bounds for \( x^i \) don't match the figure.  

Here \( x^i \) are the discrete points at which the temperatures are evaluated with and \( \Delta x = 1/N \). The values \( H^i, T^i \) are the heats and temperatures respectively, and \( Q^i \) is a temperature current (proportional to the heat flow) in the interval \( i \) flowing from node \( i + 1 \) to \( i \).

With \( Q^i = \lr{T^i - T^{i-1}}/{\Delta x} \) 
an equivalence to the circuit element \( I = \Delta V/R \) exists.  This analogy makes sense physically since a smaller interval length has less resistance to heat flow.

The linearized Poisson equation at interior nodes \( 1 < i < N \) is

\begin{equation}\label{eqn:multiphysicsProblemSet1Problem3:40}
Q^i - Q^{i-1} - \frac{\kappa_{\textrm{a}} \Delta x}{\kappa_{\textrm{m}}} \lr{ T^i - T_0 } + \frac{H^i \Delta x}{\kappa_{\textrm{m}}} = 0.
\end{equation}

Identification of the \( - \kappa_{\textrm{a}} \Delta x/\kappa_{\textrm{m}} \lr{ T^i - T_0 } \) term as a current \( I = \Delta V/R \) (with temperatures as voltages) means that an identification \( R \sim \kappa_{\textrm{m}} /(\kappa_{\textrm{a}} \Delta x) \) is possible.  
%This allows Zus to create the equivalent circuit sketched in \cref{fig:ps1p3:ps1p3Fig9}.
%
%\imageFigure{../../figures/ece1254/ps1p3Fig9}{Equivalent circuit.}{fig:ps1p3:ps1p3Fig9}{0.3}
%
At the endpoints things are a slightly different.  At \( x^0, x^{N+1} \) respectively, the relations are

\begin{equation}\label{eqn:multiphysicsProblemSet1Problem3:60}
\begin{aligned}
-q^0 + Q^1 - \frac{\kappa_{\textrm{a}} \Delta x}{\kappa_{\textrm{m}}} \lr{ T^1 - T_0 } + \frac{H^1 \Delta x}{\kappa_{\textrm{m}}} &= 0 \\
-q^1 -Q^{N-1} - \frac{\kappa_{\textrm{a}} \Delta x}{\kappa_{\textrm{m}}} \lr{ T^{N} - T_0 } + \frac{H^{N} \Delta x}{\kappa_{\textrm{m}}} &= 0. \\
\end{aligned}
\end{equation}

At these nodes is only one current term, but the heat sinks must also be modelled.  This equivalent circuit containing all these elements is sketched in 
%\cref{fig:ps1p3:ps1p3Fig10}.
%\imageFigure{../../figures/ece1254/ps1p3Fig10}{Endpoints in equivalent circuit.}{fig:ps1p3:ps1p3Fig10}{0.3}
\cref{fig:ps1p3:ps1p3Fig11}.
\imageFigure{../../figures/ece1254/ps1p3Fig11}{Equivalent circuit.}{fig:ps1p3:ps1p3Fig11}{0.3}

\makeSubAnswer{ }{multiphysics:problemSet1:3b}

The numerical solution and plotting code was implemented in

\begin{itemize}
\item \nbcite{ps1:generateHeatNetlist.m}{ps1/generateHeatNetlist.m}
\item \nbcite{ps1:plotHeatQ.m}{ps1/plotHeatQ.m}
\end{itemize}

%FIXME: This plot, shown in \cref{fig:ps1p3:ps1p3Fig12n30}, and \cref{fig:ps1p3:ps1p3Fig12n60}, for \( n = 60 \) shows an oscillatory pattern that varies directly with the number of intervals.  This is similar to the unexpected oscillatory 3D plot from the previous problem.  Again this indicates that my NodalAnalysis function that is doing the netlist to matrix equations transformation has an error.  I have not yet been able to locate that error.

%ps1p3Fig12n30
%ps1p3Fig12n60.png
\mathImageFigure{../../figures/ece1254/ps1p3Fig12n60.pdf}{Plot of numerical solution for \( n = 60 \).}{fig:ps1p3:ps1p3Fig12n60}{0.3}{ps1:plotHeatQ.m}

This plot was generated with a call to \matlabFuncPath{plotHeatQ(60, 1, 0)}{ps1:plotHeatQ.m}. %, 'ps1p3Fig12').

\makeSubAnswer{ }{multiphysics:problemSet1:3c}

It is desirable to keep the variation of the \( H_i \) term small.  That difference between two adjacent intervals scales with the squared sine

\begin{dmath}\label{eqn:multiphysicsProblemSet1Problem3:80}
\Delta \lr{ \sin^2 (2 \pi x) } 
\approx
4 \pi \sin \lr{ 2 \pi x} \cos \lr{ 2 \pi x } \Delta x
= 2 \pi \sin \lr{ 4 \pi x } \Delta x.
\end{dmath}

This is biggest when the sine is near unity.  If the heat terms are to differ by no more than a fraction \( f \) then \( \Delta x \) can be determined from

\begin{dmath}\label{eqn:multiphysicsProblemSet1Problem3:100}
f H^i \approx H^i - H^{i-1}
\approx 2 \pi \Delta x H^i.
= \frac{2 \pi}{N}  H^i.
\end{dmath}

That is
\begin{dmath}\label{eqn:multiphysicsProblemSet1Problem3:120}
N \approx \frac{2 \pi}{f}.
\end{dmath}

For example, if \( f = 1/10 \) is desired, then \( N \approx 60 \) can be picked.  That was the value used in the plot above.

\makeSubAnswer{ }{multiphysics:problemSet1:3d}

%%%The constraints on the heat flow can be expressed those by integrating the PDE once
%%%
%%%\begin{equation}\label{eqn:multiphysicsProblemSet1Problem3:140}
%%%\PD{x}{T}
%%%= 
%%%\int_{x' = a}^x dx' \lr{ \frac{\kappa_{\textrm{a}}}{\kappa_{\textrm{m}}} \lr{ T(x') - T_0 } - \frac{H(x')}{\kappa_{\textrm{m}}} }.
%%%\end{equation}
%%%
%%%The zero constraint at \( x = 0 \) can be satisfied by setting the integration boundary constant \(a = 0\).  With \( H(x') \) is constant for this part of the problem that condition is
%%%
%%%\begin{equation}\label{eqn:multiphysicsProblemSet1Problem3:160}
%%%0 
%%%= 
%%%\evalbar{\PD{x}{T} }{x = 1}
%%%=
%%%\frac{\kappa_{\textrm{a}}}{\kappa_{\textrm{m}}} 
%%%\int_0^1 dx' 
%%%T(x')
%%%- 
%%%\frac{\kappa_{\textrm{a}}}{\kappa_{\textrm{m}}} T_0 - \frac{H}{\kappa_{\textrm{m}}},
%%%\end{equation}
%%%
%%%or
%%%
%%%\begin{equation}\label{eqn:multiphysicsProblemSet1Problem3:180}
%%%\int_0^1 dx' 
%%%T(x')
%%%= 
%%%T_0 + \frac{H}{\kappa_{\textrm{a}}}.
%%%\end{equation}
%%%
%%%To discretize this condition the integral can be approximated with a sum.  
%%%
%%%\begin{equation}\label{eqn:multiphysicsProblemSet1Problem3:200}
%%%\sum_{i = 1}^N \Delta x T^i = T_0 + \frac{H}{\kappa_{\textrm{a}}}.
%%%\end{equation}
%%%
%%%That is
%%%
%%%\begin{equation}\label{eqn:multiphysicsProblemSet1Problem3:220}
%%%\sum_{i = 1}^N \lr{ T^i - T_0 } = N \frac{H}{\kappa_{\textrm{a}}}.
%%%\end{equation}
%%%
%%%This can be expressed as a sum of unity gain voltage gain elements and a fixed voltage source, tying the voltage at the final node.
%%%
%%%\begin{equation}\label{eqn:multiphysicsProblemSet1Problem3:240}
%%%T^i - T_0 = -\sum_{i = 1}^{N-1} \lr{ T^i - T_0 } + N \frac{H}{\kappa_{\textrm{a}}}.
%%%\end{equation}
%%%
%%%This and a four node example are illustrated in \cref{fig:ps1p3:ps1p3Fig13}, and \cref{fig:ps1p3:ps1p3Fig14nEquals4Example} respectively.
%%%
%%%\imageFigure{../../figures/ece1254/ps1p3Fig13}{Heat flow constraint.}{fig:ps1p3:ps1p3Fig13}{0.3}
%%%\imageFigure{../../figures/ece1254/ps1p3Fig14nEquals4Example}{Heat flow constraint, four node example.}{fig:ps1p3:ps1p3Fig14nEquals4Example}{0.3}
%%%
%%%Modeling this constraint numerically does not generate meaningful seeming results.

%%% Try 2:
%%% This generates a numerically unstable result.  Changing N keeps flipping the orientation of the temperature curve, and the desired zero derivative at the endpoints isn't there.  Instead there is a zero derivative in the interior.
%%%
%%%Zero heat flow can be modelled by replacing the resistance that couples nodes \(1, 2\) and nodes \( N-1, N \) with a zero valued current source.
%%%
%%%That is illustrated in 
%%%
%%%FIXME: picture on paper.

The new equivalent circuit can be modeled by replacing the resistor between nodes \((1,2)\), and nodes \((N-1, N)\) with a zero voltage source.  This ties the voltage and the initial and final nodes together ensuring that the discrete analogue of the heat flow is zero at the endpoints.  That is

\begin{equation}\label{eqn:multiphysicsProblemSet1Problem3:260}
\evalbar{ \PD{x}{T} }{x = 0} \sim \frac{T^1 - T^0}{\Delta x} = 0
\end{equation}
\begin{equation}\label{eqn:multiphysicsProblemSet1Problem3:280}
\evalbar{ \PD{x}{T} }{x = 1} \sim \frac{T^{N-1} - T^N}{\Delta x} = 0.
\end{equation}

The \( x = 0 \) subset of this equivalent circuit is sketched in \cref{fig:ps3d:ps3dFig17EquivalentCircuit}.

\imageFigure{../../figures/ece1254/ps3dFig17EquivalentCircuit}{Equivalent circuit at the initial endpoint.}{fig:ps3d:ps3dFig17EquivalentCircuit}{0.3}

\makeSubAnswer{ }{multiphysics:problemSet1:3e}

With uniform heat sources and no heat sinks at the boundaries the plot of the resulting temperature distribution is very boring, just the horizontal line shown in \cref{fig:ps3d:ps3dFig15WithFlatSource}.

\mathImageFigure{../../figures/ece1254/ps3dFig15WithFlatSourcen60.pdf}{Temperature distribution with uniform heat source and no heat sinks.}{fig:ps3d:ps3dFig15WithFlatSource}{0.3}{ps1:plotHeatQ.m}

Note that Matlab appears to introduce a non-horizontal jiggle in its rendering of the line.  There appears to be no such variation in the data itself, which can be demonstrated by displaying the min and the max values in a call to 
\matlabFuncPath{plotHeatQ(60, 0, 0)}{ps1:plotHeatQ.m}. %, 'ps3dFig15WithFlatSource').

This is a somewhat boring heat source, and leaves room to doubt that the boundary constraints are matched by more than luck.  The sinusoidal heat distribution from \ref{multiphysics:problemSet1:3a}, shows a more interesting temperature curve, also exhibiting the desired zero heat flow constraints.  Such a plot is given in \cref{fig:ps3d:ps3dFig16WithSineSource}, and was generated with a call to \matlabFuncPath{plotHeatQ(60, 0, 1)}{ps1:plotHeatQ.m}. %, 'ps3dFig16WithSineSource').

\mathImageFigure{../../figures/ece1254/ps3dFig16WithSineSourcen60.pdf}{Sinusoidal heat source with zero heat flow constraint.}{fig:ps3d:ps3dFig16WithSineSource}{0.3}{ps1:plotHeatQ.m}

\makeSubAnswer{ }{multiphysics:problemSet1:3f}

With the uniform heating distribution \( H(x) = 50 \) and no heat sinks, the heat in the bar is expected to have a uniform steady state distribution.  This is what is seen above in the visualization of the numerical solution.  The temperature of the bar is also seen to rise considerably above the ambient temperature.  Eventually the \( T(x) - T_0 \) term in the PDE gets big enough that more significant leakage into the air occurs despite the small ratio of \( \kappa_{\textrm{a}}/\kappa_{\textrm{m}} \), limiting the steady state temperature of the bar.

\paragraph{Grading comment:}

If I have a constant heat source \( H \), following your argument I just need one discretization. This is not true, as it would result in a very inaccurate discretization of the Poisson equation. \( \Delta x \) must be small enough to resolve/approximate well:

\begin{itemize}
\item The heat source variations (as you discuss).
\item The derivatives in the Poisson equation (missing).
\end{itemize}
}

%\section{Matlab Sources}
%
%The Matlab code associated with these notes is available under 
%\nbcite{ps1}{ece1254/ps1/}
%within the github repository:
%
%\begin{itemize}
%\item git@github.com:peeterjoot/matlab.git
%\end{itemize}
%
%The data and figures referenced in these notes were generated with versions not greater than:
%
%\begin{itemize}
%\item commit 5641358d1f397dab8a4053f7fb9681b0d532cad6
%\end{itemize}
