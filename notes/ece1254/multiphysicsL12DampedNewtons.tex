\section{Damped Newton's method}
\index{Newton's method!damped}

We want to be able to deal with the oscillation that we can have in examples like that of \cref{fig:lecture12:lecture12Fig3}.

\imageFigure{../../figures/ece1254/lecture12Fig3}{Oscillatory Newton's iteration}{fig:lecture12:lecture12Fig3}{0.2}


Large steps can be dangerous.  We want to modify Newton's method as follows

The algorithm is
\begin{algorithmic}
\STATE Guess \( \Bx^0, k = 0 \).
\REPEAT
\STATE Compute \( F \) and \( J_F \) at \( \Bx^k \)
\STATE Solve linear system  \( J_F(\Bx^k) \Delta \Bx^k = - F(\Bx^k) \)
\STATE \( \Bx^{k+1} = \Bx^k + \alpha^k \Delta \Bx^k \)
\STATE \( k = k + 1 \)
\UNTIL{converged}
\end{algorithmic}

with \( \alpha^k = 1 \) we have standard Newton's method.  We want to pick \( \alpha^k \) so that we minimize

\section{Continuation parameters}
\index{continuation parameter}

Newton's method converges given a close initial guess.  We can generate a sequence of problems where the previous problem generates a good initial guess for the next problem.

An example is a heat conducting bar, with a final heat distribution.  We can start the numeric iteration with \( T = 0 \), and gradually increase the temperatures until we achieve the final desired heat distribution.

Suppose that we want to solve 

\begin{equation}\label{eqn:multiphysicsL12:340}
F(\Bx) = 0.
\end{equation}

We modify this problem by introducing

\begin{equation}\label{eqn:multiphysicsL12:360}
\tilde{F}(\Bx(\lambda), \lambda) = 0,
\end{equation}

where 

\begin{itemize}
\item \( \tilde{F}(\Bx(0), 0) = 0 \) is easy to solve
\item \( \tilde{F}(\Bx(1), 1) = 0 \) is equivalent to \( F(\Bx) = 0 \).
\item (more on slides)
\end{itemize}

The source load stepping algorithm is

\begin{itemize}
\item Solve \(\tilde{F}(\Bx(0), 0) = 0 \), with \( \Bx(\lambda_{\text{prev}} = \Bx(0) \)
\item (more on slides)
\end{itemize}

This can still have problems, for example, when the parameterization is multivalued as in \cref{fig:lecture12:lecture12Fig4}.

\imageFigure{../../figures/ece1254/lecture12Fig4}{Multivalued parameterization}{fig:lecture12:lecture12Fig4}{0.2}

We can attempt to adjust \( \lambda \) so that we move along the parameterization curve.

%\EndArticle
%\EndNoBibArticle
