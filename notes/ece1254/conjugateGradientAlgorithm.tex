%
% Copyright � 2014  Ahmed Dorrah, Peeter Joot.  All Rights Reserved.
% Licenced as described in the file LICENSE under the root directory of this GIT repository.
%

The conjugate gradient algorithm presented in the slides (without preconditioning) was

\begin{algorithmic}
\STATE \( \Bd^{(0)} = \Br^{(0)} \)
\REPEAT
\STATE \( \alpha_k = \frac{ \lr{ \Bd^{(k)} }^\T \Br^{(k)} }{ \lr{ \Bd^{(k)} }^\T M \Bd^{(k)} } \)
\STATE \( \Bx^{(k+1)} = \Bx^{(k)} + \alpha_k \Bd^{(k)} \)
\STATE \( \Br^{(k+1)} = \Br^{(k)} - \alpha_k M \Bd^{(k)} \)
\STATE \( \beta_k = \frac{ \lr{ M \Bd^{(k)} }^\T \Br^{(k+1)} }{ \lr{ M \Bd^{(k)} }^\T \Bd^{(k)} } \)
\STATE \( \Bd^{(k+1)} = \Br^{(k+1)} - \beta_k \Bd^{(k)} \)
\UNTIL{converged}
\end{algorithmic}

The repeated calculations are undesirable for actually coding this algorithm.  First introduce a temporary for the matrix product.  Introducing a temporary variable helps a bit

\begin{algorithmic}
\STATE \( \Bd^{(0)} = \Br^{(0)} \)
\REPEAT
\STATE \( \Bq = M \Bd^{(k)} \)
\STATE \( \alpha_k = \frac{ \lr{ \Bd^{(k)} }^\T \Br^{(k)} }{ \lr{ \Bd^{(k)} }^\T \Bq } \)
\STATE \( \Bx^{(k+1)} = \Bx^{(k)} + \alpha_k \Bd^{(k)} \)
\STATE \( \Br^{(k+1)} = \Br^{(k)} - \alpha_k \Bq \)
\STATE \( \beta_k = \frac{ \Bq^\T \Br^{(k+1)} }{ \Bq^\T \Bd^{(k)} } \)
\STATE \( \Bd^{(k+1)} = \Br^{(k+1)} - \beta_k \Bd^{(k)} \)
\UNTIL{converged}
\end{algorithmic}

However, this still has a lot more computation than the algorithm specified in \citep{shewchuk1994introduction} \S B.2.  It looks like the orthogonality properties can be used to recast the \( \Bd^{(k)} \cdot \Br^{(k)} \) products in terms of \( \Br^{(k)} \)

\begin{equation}\label{eqn:multiphysicsL8:400}
\lr{ \Bd^{(k)} }^\T \Br^{(k)} 
=
\Br^{(k)} \cdot \lr{ \Br^{(k)} + \beta_{k-1} \Bd^{(k-1)} 
},
\end{equation}

but since the new residual is orthogonal to all the previous search directions \( \Br^{(k)} \cdot \Bd^{(k-1)} = 0 \).  The transformed direction vector \( \Bq \) is a scaled difference of residuals.  Taking dot products

\begin{dmath}\label{eqn:multiphysicsL8:420}
\Bq \cdot \Br^{(k+1)}
=
\inv{\alpha_k} \lr{ \Br^{(k)} - \Br^{(k+1)} } \cdot \Br^{(k+1)}
=
\inv{\alpha_k} \lr{ \cancel{\Bd^{(k)}} - \beta_{k-1} \cancel{\Bd^{(k-1)}} - \Br^{(k+1)} } \cdot \Br^{(k+1)}
=
-\inv{\alpha_k} \Br^{(k+1)} \cdot \Br^{(k+1)}.
\end{dmath}

This gives

\begin{subequations}
\begin{equation}\label{eqn:multiphysicsL8:440}
\alpha_k = \frac{ \lr{ \Br^{(k)} }^\T \Br^{(k)} }{ \lr{ \Bd^{(k)} }^\T \Bq }
\end{equation}
\begin{equation}\label{eqn:multiphysicsL8:460}
\beta_k = -\frac{ \lr{\Br^{(k+1)}}^\T \Br^{(k+1)} }{ \alpha_k \Bq^\T \Bd^{(k)} },
\end{equation}
\end{subequations}

A final elimination of \( \alpha_k \) from \cref{eqn:multiphysicsL8:460} gives

\begin{equation}\label{eqn:multiphysicsL8:480}
\beta_k = 
-\frac{ \lr{\Br^{(k+1)}}^\T \Br^{(k+1)} }{ 
\lr{ \Br^{(k)} }^\T \Br^{(k)} 
}.
\end{equation}

All the pieces put together yield

\begin{algorithmic}
\STATE \( \Bd^{(0)} = \Br^{(0)} \)
\REPEAT
\STATE \( \Bq = M \Bd^{(k)} \)
\STATE \( \alpha_k = \frac{ \lr{ \Br^{(k)} }^\T \Br^{(k)} }{ \lr{ \Bd^{(k)} }^\T \Bq } \)
\STATE \( \Bx^{(k+1)} = \Bx^{(k)} + \alpha_k \Bd^{(k)} \)
\STATE \( \Br^{(k+1)} = \Br^{(k)} - \alpha_k \Bq \)
\STATE \( \beta_k = 
-\frac{ \lr{\Br^{(k+1)}}^\T \Br^{(k+1)} }{ 
\lr{ \Br^{(k)} }^\T \Br^{(k)} 
}
\)
\STATE \( \Bd^{(k+1)} = \Br^{(k+1)} - \beta_k \Bd^{(k)} \)
\UNTIL{converged}
\end{algorithmic}

This is now consistent with eqns 45-49 of \citep{shewchuk1994introduction}, with the exception that the sign of the \( \beta_k \) term has been flipped.

