%
% Copyright � 2013 Peeter Joot.  All Rights Reserved.
% Licenced as described in the file LICENSE under the root directory of this GIT repository.
%

% 
%\chapter{Preface}
% this suppresses an explicit chapter number for the preface.
\chapter*{Preface}%\normalsize
  \addcontentsline{toc}{chapter}{Preface}

This document is based on my lecture notes for the Fall 2014, University of Toronto Modeling of Multiphysics course (ECE1254H), taught by Professor P. Triverio.  

\paragraph{Official course description:}

``The course deals with the modeling and simulation of physical systems. It introduces the fundamental techniques to generate and solve the equations of a static or dynamic system. Special attention is devoted to complexity issues and to model order reduction methods, presented as a systematic way to simulate highly-complex systems with acceptable computational cost. Examples from multiple disciplines are considered, including electrical/electromagnetic engineering, structural mechanics, fluid-dynamics. Students are encouraged to work on a project related to their own research interests.''

Topics:
\begin{itemize}
\item Automatic generation of system equations (Tableau method, modified nodal analysis).
\item Solution of linear and nonlinear systems (LU decomposition, conjugate gradient method, sparse systems, Newton-Raphson method).
\item Solution of dynamical systems (Euler and trapezoidal rule, accuracy, stability).
\item Model order reduction of linear systems (proper orthogonal decomposition, Krylov methods, truncated balanced realization, stability/dissipativity enforcement).
\item Modeling from experimental data (system identification, the Vector Fitting algorithm, enforcement of stability and dissipativity).
\item If time permits, an overview of numerical methods to solve partial differential equations (Boundary element method, finite elements, FDTD).
\end{itemize}

Recommended texts include
\begin{itemize}
	\item \citep{najm2010circuit}.
\end{itemize}

\paragraph{This document contains:}

\begin{itemize}
\item Plain old lecture notes.   These mirror what was covered in class, possibly augmented with additional details.  

\item Personal notes exploring details that were not clear to me from the lectures, or from the texts associated with the lecture material.

\withproblemsets{
\item Assigned problems.  Like anything else take these as is.  I have attempted to either correct errors or mark them as such.%

\item Final report on an implementation of the Harmonic Balance method (a group project assignment).  That work, including the code, the report content and presentation, was a collaborative effort between Mike Royle and myself.

\item Links to Matlab function implementations associated with the problem sets.
}
\end{itemize}

My thanks go to Professor Triverio for teaching this course, to Ahmed Dorrah for supplying a copy of his (lecture 8) notes on the conjugate gradient method, and to Mike Royle for our Harmonic Balance collaboration.

Peeter Joot  \quad peeter.joot@gmail.com 
