
%% bare_adv.tex
%% V1.4
%% 2012/12/27
%% by Michael Shell
%% See: 
%% http://www.michaelshell.org/
%% for current contact information.
%%
%% This is a skeleton file demonstrating the advanced use of IEEEtran.cls
%% (requires IEEEtran.cls version 1.8 or later) with an IEEE Computer
%% Society journal paper.
%%
%% Support sites:
%% http://www.michaelshell.org/tex/ieeetran/
%% http://www.ctan.org/tex-archive/macros/latex/contrib/IEEEtran/
%% and
%% http://www.ieee.org/

%%*************************************************************************
%% Legal Notice:
%% This code is offered as-is without any warranty either expressed or
%% implied; without even the implied warranty of MERCHANTABILITY or
%% FITNESS FOR A PARTICULAR PURPOSE! 
%% User assumes all risk.
%% In no event shall IEEE or any contributor to this code be liable for
%% any damages or losses, including, but not limited to, incidental,
%% consequential, or any other damages, resulting from the use or misuse
%% of any information contained here.
%%
%% All comments are the opinions of their respective authors and are not
%% necessarily endorsed by the IEEE.
%%
%% This work is distributed under the LaTeX Project Public License (LPPL)
%% ( http://www.latex-project.org/ ) version 1.3, and may be freely used,
%% distributed and modified. A copy of the LPPL, version 1.3, is included
%% in the base LaTeX documentation of all distributions of LaTeX released
%% 2003/12/01 or later.
%% Retain all contribution notices and credits.
%% ** Modified files should be clearly indicated as such, including  **
%% ** renaming them and changing author support contact information. **
%%
%% File list of work: IEEEtran.cls, IEEEtran_HOWTO.pdf, bare_adv.tex,
%%                    bare_conf.tex, bare_jrnl.tex, bare_jrnl_compsoc.tex,
%%                    bare_jrnl_transmag.tex
%%*************************************************************************

% *** Authors should verify (and, if needed, correct) their LaTeX system  ***
% *** with the testflow diagnostic prior to trusting their LaTeX platform ***
% *** with production work. IEEE's font choices can trigger bugs that do  ***
% *** not appear when using other class files.                            ***
% The testflow support page is at:
% http://www.michaelshell.org/tex/testflow/



% IEEEtran V1.7 and later provides for these CLASSINPUT macros to allow the
% user to reprogram some IEEEtran.cls defaults if needed. These settings
% override the internal defaults of IEEEtran.cls regardless of which class
% options are used. Do not use these unless you have good reason to do so as
% they can result in nonIEEE compliant documents. User beware. ;)
%
%\newcommand{\CLASSINPUTbaselinestretch}{1.0} % baselinestretch
%\newcommand{\CLASSINPUTinnersidemargin}{1in} % inner side margin
%\newcommand{\CLASSINPUToutersidemargin}{1in} % outer side margin
%\newcommand{\CLASSINPUTtoptextmargin}{1in}   % top text margin
%\newcommand{\CLASSINPUTbottomtextmargin}{1in}% bottom text margin



% Note that the a4paper option is mainly intended so that authors in
% countries using A4 can easily print to A4 and see how their papers will
% look in print - the typesetting of the document will not typically be
% affected with changes in paper size (but the bottom and side margins will).
% Use the testflow package mentioned above to verify correct handling of
% both paper sizes by the user's LaTeX system.
%
% Also note that the "draftcls" or "draftclsnofoot", not "draft", option
% should be used if it is desired that the figures are to be displayed in
% draft mode.
%
\documentclass[12pt,journal,compsoc]{../ieeepaper/IEEEtran}
% The Computer Society requires 12pt.


% For Computer Society journals, IEEEtran defaults to the use of 
% Palatino/Palladio as is done in IEEE Computer Society journals.
% To go back to Times Roman, you can use this code:
%\renewcommand{\rmdefault}{ptm}\selectfont





% Some very useful LaTeX packages include:
% (uncomment the ones you want to load)



% *** MISC UTILITY PACKAGES ***
%
%\usepackage{ifpdf}
% Heiko Oberdiek's ifpdf.sty is very useful if you need conditional
% compilation based on whether the output is pdf or dvi.
% usage:
% \ifpdf
%   % pdf code
% \else
%   % dvi code
% \fi
% The latest version of ifpdf.sty can be obtained from:
% http://www.ctan.org/tex-archive/macros/latex/contrib/oberdiek/
% Also, note that IEEEtran.cls V1.7 and later provides a builtin
% \ifCLASSINFOpdf conditional that works the same way.
% When switching from latex to pdflatex and vice-versa, the compiler may
% have to be run twice to clear warning/error messages.

\usepackage{peeters_macros}
\usepackage[english]{cleveref}
\usepackage{peeters_layout}
\usepackage{subfig}

% (The subfig.sty package must be loaded for this to work.)

% \citep:
\PassOptionsToPackage{square,numbers}{natbib}
\usepackage{natbib}


%\begin{figure*}[!t]
%\centering
%\subfloat[Case I]{\includegraphics[width=2.5in]{box}%
%\label{fig_first_case}}
%\hfil
%\subfloat[Case II]{\includegraphics[width=2.5in]{box}%
%\label{fig_second_case}}
%\caption{Simulation results.}
%\label{fig_sim}
%\end{figure*}

% An .eps filename suffix will be assumed under latex, 
% and a .pdf suffix will be assumed for pdflatex; or what has been declared
% via \DeclareGraphicsExtensions.

% \imageFigure{path}{caption}{label}{width}
%\renewcommand{\imageFigure}[4]{%
%\begin{figure}[!t]%
%\centering%
%\includegraphics[width=2.5in]{#1}%
%\caption{#2}%
%\label{#3}%
%\end{figure}%
%}

% \imageFigureHere{path}{caption}{label}{width}
\newcommand{\imageFigureHere}[4]{%
\begin{figure}[!h]%
\centering%
\includegraphics[width=#4]{#1}%
\caption{#2}%
\label{#3}%
\end{figure}%
}

% *** CITATION PACKAGES ***
%
\ifCLASSOPTIONcompsoc
  % IEEE Computer Society needs nocompress option
  % requires cite.sty v4.0 or later (November 2003)
  % \usepackage[nocompress]{cite}
\else
  % normal IEEE
  % \usepackage{cite}
\fi
% cite.sty was written by Donald Arseneau
% V1.6 and later of IEEEtran pre-defines the format of the cite.sty package
% \cite{} output to follow that of IEEE. Loading the cite package will
% result in citation numbers being automatically sorted and properly
% "compressed/ranged". e.g., [1], [9], [2], [7], [5], [6] without using
% cite.sty will become [1], [2], [5]--[7], [9] using cite.sty. cite.sty's
% \cite will automatically add leading space, if needed. Use cite.sty's
% noadjust option (cite.sty V3.8 and later) if you want to turn this off
% such as if a citation ever needs to be enclosed in parenthesis.
% cite.sty is already installed on most LaTeX systems. Be sure and use
% version 4.0 (2003-05-27) and later if using hyperref.sty. cite.sty does
% not currently provide for hyperlinked citations.
% The latest version can be obtained at:
% http://www.ctan.org/tex-archive/macros/latex/contrib/cite/
% The documentation is contained in the cite.sty file itself.
%
% Note that some packages require special options to format as the Computer
% Society requires. In particular, Computer Society  papers do not use
% compressed citation ranges as is done in typical IEEE papers
% (e.g., [1]-[4]). Instead, they list every citation separately in order
% (e.g., [1], [2], [3], [4]). To get the latter we need to load the cite
% package with the nocompress option which is supported by cite.sty v4.0
% and later.
%
% Note also the use of a CLASSOPTION conditional provided by 
% IEEEtran.cls V1.7 and later.





% *** GRAPHICS RELATED PACKAGES ***
%
\ifCLASSINFOpdf
  % \usepackage[pdftex]{graphicx}
  % declare the path(s) where your graphic files are
  % \graphicspath{{../pdf/}{../jpeg/}}
  % and their extensions so you won't have to specify these with
  % every instance of \includegraphics
  % \DeclareGraphicsExtensions{.pdf,.jpeg,.png}
\else
  % or other class option (dvipsone, dvipdf, if not using dvips). graphicx
  % will default to the driver specified in the system graphics.cfg if no
  % driver is specified.
  % \usepackage[dvips]{graphicx}
  % declare the path(s) where your graphic files are
  % \graphicspath{{../eps/}}
  % and their extensions so you won't have to specify these with
  % every instance of \includegraphics
  % \DeclareGraphicsExtensions{.eps}
\fi
% graphicx was written by David Carlisle and Sebastian Rahtz. It is
% required if you want graphics, photos, etc. graphicx.sty is already
% installed on most LaTeX systems. The latest version and documentation
% can be obtained at: 
% http://www.ctan.org/tex-archive/macros/latex/required/graphics/
% Another good source of documentation is "Using Imported Graphics in
% LaTeX2e" by Keith Reckdahl which can be found at:
% http://www.ctan.org/tex-archive/info/epslatex/
%
% latex, and pdflatex in dvi mode, support graphics in encapsulated
% postscript (.eps) format. pdflatex in pdf mode supports graphics
% in .pdf, .jpeg, .png and .mps (metapost) formats. Users should ensure
% that all non-photo figures use a vector format (.eps, .pdf, .mps) and
% not a bitmapped formats (.jpeg, .png). IEEE frowns on bitmapped formats
% which can result in "jaggedy"/blurry rendering of lines and letters as
% well as large increases in file sizes.
%
% You can find documentation about the pdfTeX application at:
% http://www.tug.org/applications/pdftex





% *** MATH PACKAGES ***
%
%\usepackage[cmex10]{amsmath}
% A popular package from the American Mathematical Society that provides
% many useful and powerful commands for dealing with mathematics. If using
% it, be sure to load this package with the cmex10 option to ensure that
% only type 1 fonts will utilized at all point sizes. Without this option,
% it is possible that some math symbols, particularly those within
% footnotes, will be rendered in bitmap form which will result in a
% document that can not be IEEE Xplore compliant!
%
% Also, note that the amsmath package sets \interdisplaylinepenalty to 10000
% thus preventing page breaks from occurring within multiline equations. Use:
%\interdisplaylinepenalty=2500
% after loading amsmath to restore such page breaks as IEEEtran.cls normally
% does. amsmath.sty is already installed on most LaTeX systems. The latest
% version and documentation can be obtained at:
% http://www.ctan.org/tex-archive/macros/latex/required/amslatex/math/





% *** SPECIALIZED LIST PACKAGES ***
%\usepackage{acronym}
% acronym.sty was written by Tobias Oetiker. This package provides tools for
% managing documents with large numbers of acronyms. (You don't *have* to
% use this package - unless you have a lot of acronyms, you may feel that
% such package management of them is bit of an overkill.)
% Do note that the acronym environment (which lists acronyms) will have a
% problem when used under IEEEtran.cls because acronym.sty relies on the
% description list environment - which IEEEtran.cls has customized for
% producing IEEE style lists. A workaround is to declared the longest
% label width via the IEEEtran.cls \IEEEiedlistdecl global control:
%
% \renewcommand{\IEEEiedlistdecl}{\IEEEsetlabelwidth{SONET}}
% \begin{acronym}
%
% \end{acronym}
% \renewcommand{\IEEEiedlistdecl}{\relax}% remember to reset \IEEEiedlistdecl
%
% instead of using the acronym environment's optional argument.
% The latest version and documentation can be obtained at:
% http://www.ctan.org/tex-archive/macros/latex/contrib/acronym/


%\usepackage{algorithmic}
% algorithmic.sty was written by Peter Williams and Rogerio Brito.
% This package provides an algorithmic environment fo describing algorithms.
% You can use the algorithmic environment in-text or within a figure
% environment to provide for a floating algorithm. Do NOT use the algorithm
% floating environment provided by algorithm.sty (by the same authors) or
% algorithm2e.sty (by Christophe Fiorio) as IEEE does not use dedicated
% algorithm float types and packages that provide these will not provide
% correct IEEE style captions. The latest version and documentation of
% algorithmic.sty can be obtained at:
% http://www.ctan.org/tex-archive/macros/latex/contrib/algorithms/
% There is also a support site at:
% http://algorithms.berlios.de/index.html
% Also of interest may be the (relatively newer and more customizable)
% algorithmicx.sty package by Szasz Janos:
% http://www.ctan.org/tex-archive/macros/latex/contrib/algorithmicx/




% *** ALIGNMENT PACKAGES ***
%
%\usepackage{array}
% Frank Mittelbach's and David Carlisle's array.sty patches and improves
% the standard LaTeX2e array and tabular environments to provide better
% appearance and additional user controls. As the default LaTeX2e table
% generation code is lacking to the point of almost being broken with
% respect to the quality of the end results, all users are strongly
% advised to use an enhanced (at the very least that provided by array.sty)
% set of table tools. array.sty is already installed on most systems. The
% latest version and documentation can be obtained at:
% http://www.ctan.org/tex-archive/macros/latex/required/tools/


%\usepackage{mdwmath}
%\usepackage{mdwtab}
% Also highly recommended is Mark Wooding's extremely powerful MDW tools,
% especially mdwmath.sty and mdwtab.sty which are used to format equations
% and tables, respectively. The MDWtools set is already installed on most
% LaTeX systems. The lastest version and documentation is available at:
% http://www.ctan.org/tex-archive/macros/latex/contrib/mdwtools/


% IEEEtran contains the IEEEeqnarray family of commands that can be used to
% generate multiline equations as well as matrices, tables, etc., of high
% quality.


%\usepackage{eqparbox}
% Also of notable interest is Scott Pakin's eqparbox package for creating
% (automatically sized) equal width boxes - aka "natural width parboxes".
% Available at:
% http://www.ctan.org/tex-archive/macros/latex/contrib/eqparbox/




% *** SUBFIGURE PACKAGES ***
%\ifCLASSOPTIONcompsoc
%  \usepackage[caption=false,font=normalsize,labelfont=sf,textfont=sf]{subfig}
%\else
%  \usepackage[caption=false,font=footnotesize]{subfig}
%\fi
% subfig.sty, written by Steven Douglas Cochran, is the modern replacement
% for subfigure.sty, the latter of which is no longer maintained and is
% incompatible with some LaTeX packages including fixltx2e. However,
% subfig.sty requires and automatically loads Axel Sommerfeldt's caption.sty
% which will override IEEEtran.cls' handling of captions and this will result
% in non-IEEE style figure/table captions. To prevent this problem, be sure
% and invoke subfig.sty's "caption=false" package option (available since
% subfig.sty version 1.3, 2005/06/28) as this is will preserve IEEEtran.cls
% handling of captions.
% Note that the Computer Society format requires a larger sans serif font
% than the serif footnote size font used in traditional IEEE formatting
% and thus the need to invoke different subfig.sty package options depending
% on whether compsoc mode has been enabled.
%
% The latest version and documentation of subfig.sty can be obtained at:
% http://www.ctan.org/tex-archive/macros/latex/contrib/subfig/




% *** FLOAT PACKAGES ***
%
%\usepackage{fixltx2e}
% fixltx2e, the successor to the earlier fix2col.sty, was written by
% Frank Mittelbach and David Carlisle. This package corrects a few problems
% in the LaTeX2e kernel, the most notable of which is that in current
% LaTeX2e releases, the ordering of single and double column floats is not
% guaranteed to be preserved. Thus, an unpatched LaTeX2e can allow a
% single column figure to be placed prior to an earlier double column
% figure. The latest version and documentation can be found at:
% http://www.ctan.org/tex-archive/macros/latex/base/


%\usepackage{stfloats}
% stfloats.sty was written by Sigitas Tolusis. This package gives LaTeX2e
% the ability to do double column floats at the bottom of the page as well
% as the top. (e.g., "\begin{figure*}[!b]" is not normally possible in
% LaTeX2e). It also provides a command:
%\fnbelowfloat
% to enable the placement of footnotes below bottom floats (the standard
% LaTeX2e kernel puts them above bottom floats). This is an invasive package
% which rewrites many portions of the LaTeX2e float routines. It may not work
% with other packages that modify the LaTeX2e float routines. The latest
% version and documentation can be obtained at:
% http://www.ctan.org/tex-archive/macros/latex/contrib/sttools/
% Do not use the stfloats baselinefloat ability as IEEE does not allow
% \baselineskip to stretch. Authors submitting work to the IEEE should note
% that IEEE rarely uses double column equations and that authors should try
% to avoid such use. Do not be tempted to use the cuted.sty or midfloat.sty
% packages (also by Sigitas Tolusis) as IEEE does not format its papers in
% such ways.
% Do not attempt to use stfloats with fixltx2e as they are incompatible.
% Instead, use Morten Hogholm'a dblfloatfix which combines the features
% of both fixltx2e and stfloats:
%
% \usepackage{dblfloatfix}
% The latest version can be found at:
% http://www.ctan.org/tex-archive/macros/latex/contrib/dblfloatfix/


%\ifCLASSOPTIONcaptionsoff
%  \usepackage[nomarkers]{endfloat}
% \let\MYoriglatexcaption\caption
% \renewcommand{\caption}[2][\relax]{\MYoriglatexcaption[#2]{#2}}
%\fi
% endfloat.sty was written by James Darrell McCauley, Jeff Goldberg and 
% Axel Sommerfeldt. This package may be useful when used in conjunction with 
% IEEEtran.cls'  captionsoff option. Some IEEE journals/societies require that
% submissions have lists of figures/tables at the end of the paper and that
% figures/tables without any captions are placed on a page by themselves at
% the end of the document. If needed, the draftcls IEEEtran class option or
% \CLASSINPUTbaselinestretch interface can be used to increase the line
% spacing as well. Be sure and use the nomarkers option of endfloat to
% prevent endfloat from "marking" where the figures would have been placed
% in the text. The two hack lines of code above are a slight modification of
% that suggested by in the endfloat docs (section 8.4.1) to ensure that
% the full captions always appear in the list of figures/tables - even if
% the user used the short optional argument of \caption[]{}.
% IEEE papers do not typically make use of \caption[]'s optional argument,
% so this should not be an issue. A similar trick can be used to disable
% captions of packages such as subfig.sty that lack options to turn off
% the subcaptions:
% For subfig.sty:
% \let\MYorigsubfloat\subfloat
% \renewcommand{\subfloat}[2][\relax]{\MYorigsubfloat[]{#2}}
% However, the above trick will not work if both optional arguments of
% the \subfloat command are used. Furthermore, there needs to be a
% description of each subfigure *somewhere* and endfloat does not add
% subfigure captions to its list of figures. Thus, the best approach is to
% avoid the use of subfigure captions (many IEEE journals avoid them anyway)
% and instead reference/explain all the subfigures within the main caption.
% The latest version of endfloat.sty and its documentation can obtained at:
% http://www.ctan.org/tex-archive/macros/latex/contrib/endfloat/
%
% The IEEEtran \ifCLASSOPTIONcaptionsoff conditional can also be used
% later in the document, say, to conditionally put the References on a 
% page by themselves.





% *** PDF, URL AND HYPERLINK PACKAGES ***
%
%\usepackage{url}
% url.sty was written by Donald Arseneau. It provides better support for
% handling and breaking URLs. url.sty is already installed on most LaTeX
% systems. The latest version and documentation can be obtained at:
% http://www.ctan.org/tex-archive/macros/latex/contrib/url/
% Basically, \url{my_url_here}.


% NOTE: PDF thumbnail features are not required in IEEE papers
%       and their use requires extra complexity and work.
%\ifCLASSINFOpdf
%  \usepackage[pdftex]{thumbpdf}
%\else
%  \usepackage[dvips]{thumbpdf}
%\fi
% thumbpdf.sty and its companion Perl utility were written by Heiko Oberdiek.
% It allows the user a way to produce PDF documents that contain fancy
% thumbnail images of each of the pages (which tools like acrobat reader can
% utilize). This is possible even when using dvi->ps->pdf workflow if the
% correct thumbpdf driver options are used. thumbpdf.sty incorporates the
% file containing the PDF thumbnail information (filename.tpm is used with
% dvips, filename.tpt is used with pdftex, where filename is the base name of
% your tex document) into the final ps or pdf output document. An external
% utility, the thumbpdf *Perl script* is needed to make these .tpm or .tpt
% thumbnail files from a .ps or .pdf version of the document (which obviously
% does not yet contain pdf thumbnails). Thus, one does a:
% 
% thumbpdf filename.pdf 
%
% to make a filename.tpt, and:
%
% thumbpdf --mode dvips filename.ps
%
% to make a filename.tpm which will then be loaded into the document by
% thumbpdf.sty the NEXT time the document is compiled (by pdflatex or
% latex->dvips->ps2pdf). Users must be careful to regenerate the .tpt and/or
% .tpm files if the main document changes and then to recompile the
% document to incorporate the revised thumbnails to ensure that thumbnails
% match the actual pages. It is easy to forget to do this!
% 
% Unix systems come with a Perl interpreter. However, MS Windows users
% will usually have to install a Perl interpreter so that the thumbpdf
% script can be run. The Ghostscript PS/PDF interpreter is also required.
% See the thumbpdf docs for details. The latest version and documentation
% can be obtained at.
% http://www.ctan.org/tex-archive/support/thumbpdf/


% NOTE: PDF hyperlink and bookmark features are not required in IEEE
%       papers and their use requires extra complexity and work.
% *** IF USING HYPERREF BE SURE AND CHANGE THE EXAMPLE PDF ***
% *** TITLE/SUBJECT/AUTHOR/KEYWORDS INFO BELOW!!           ***
\newcommand\MYhyperrefoptions{bookmarks=true,bookmarksnumbered=true,
pdfpagemode={UseOutlines},plainpages=false,pdfpagelabels=true,
colorlinks=true,linkcolor={black},citecolor={black},urlcolor={black},
pdftitle={Harmonic Balance solutions to non-linear circuits},
pdfsubject={Multiphyiscs modeling},
pdfauthor={Michael Royle, Peeter Joot},
pdfkeywords={ECE1254, Harmonic Balance, non-linear circuit, diode, discrete Fourier transform}}
%\ifCLASSINFOpdf
%\usepackage[\MYhyperrefoptions,pdftex]{hyperref}
%\else
%\usepackage[\MYhyperrefoptions,breaklinks=true,dvips]{hyperref}
%\usepackage{breakurl}
%\fi
% One significant drawback of using hyperref under DVI output is that the
% LaTeX compiler cannot break URLs across lines or pages as can be done
% under pdfLaTeX's PDF output via the hyperref pdftex driver. This is
% probably the single most important capability distinction between the
% DVI and PDF output. Perhaps surprisingly, all the other PDF features
% (PDF bookmarks, thumbnails, etc.) can be preserved in
% .tex->.dvi->.ps->.pdf workflow if the respective packages/scripts are
% loaded/invoked with the correct driver options (dvips, etc.). 
% As most IEEE papers use URLs sparingly (mainly in the references), this
% may not be as big an issue as with other publications.
%
% That said, Vilar Camara Neto created his breakurl.sty package which
% permits hyperref to easily break URLs even in dvi mode.
% Note that breakurl, unlike most other packages, must be loaded
% AFTER hyperref. The latest version of breakurl and its documentation can
% be obtained at:
% http://www.ctan.org/tex-archive/macros/latex/contrib/breakurl/
% breakurl.sty is not for use under pdflatex pdf mode.
%
% The advanced features offer by hyperref.sty are not required for IEEE
% submission, so users should weigh these features against the added
% complexity of use.
% The package options above demonstrate how to enable PDF bookmarks
% (a type of table of contents viewable in Acrobat Reader) as well as
% PDF document information (title, subject, author and keywords) that is
% viewable in Acrobat reader's Document_Properties menu. PDF document
% information is also used extensively to automate the cataloging of PDF
% documents. The above set of options ensures that hyperlinks will not be
% colored in the text and thus will not be visible in the printed page,
% but will be active on "mouse over". USING COLORS OR OTHER HIGHLIGHTING
% OF HYPERLINKS CAN RESULT IN DOCUMENT REJECTION BY THE IEEE, especially if
% these appear on the "printed" page. IF IN DOUBT, ASK THE RELEVANT
% SUBMISSION EDITOR. You may need to add the option hypertexnames=false if
% you used duplicate equation numbers, etc., but this should not be needed
% in normal IEEE work.
% The latest version of hyperref and its documentation can be obtained at:
% http://www.ctan.org/tex-archive/macros/latex/contrib/hyperref/





% *** Do not adjust lengths that control margins, column widths, etc. ***
% *** Do not use packages that alter fonts (such as pslatex).         ***
% There should be no need to do such things with IEEEtran.cls V1.6 and later.
% (Unless specifically asked to do so by the journal or conference you plan
% to submit to, of course. )


% correct bad hyphenation here
\hyphenation{op-tical net-works semi-conduc-tor}

%\graphicspath{{../../figures/ece1254/report_svg}}

\begin{document}
%
% paper title
% can use linebreaks \\ within to get better formatting as desired
% Do not put math or special symbols in the title.
\title{ECE1254 Final project \\ 
Harmonic Balance solutions to non-linear circuits
}
%
%
% author names and IEEE memberships
% note positions of commas and nonbreaking spaces ( ~ ) LaTeX will not break
% a structure at a ~ so this keeps an author's name from being broken across
% two lines.
% use \thanks{} to gain access to the first footnote area
% a separate \thanks must be used for each paragraph as LaTeX2e's \thanks
% was not built to handle multiple paragraphs
%
%
%\IEEEcompsocitemizethanks is a special \thanks that produces the bulleted
% lists the Computer Society journals use for "first footnote" author
% affiliations. Use \IEEEcompsocthanksitem which works much like \item
% for each affiliation group. When not in compsoc mode,
% \IEEEcompsocitemizethanks becomes like \thanks and
% \IEEEcompsocthanksitem becomes a line break with idention. This
% facilitates dual compilation, although admittedly the differences in the
% desired content of \author between the different types of papers makes a
% one-size-fits-all approach a daunting prospect. For instance, compsoc 
% journal papers have the author affiliations above the "Manuscript
% received ..."  text while in non-compsoc journals this is reversed. Sigh.

\author{ Michael~Royle%~\IEEEmembership{Fellow,~OSA,}
   ,Peeter~Joot%~\IEEEmembership{Member,~IEEE,}
%        and~Jane~Doe,~\IEEEmembership{Life~Fellow,~IEEE}% <-this % stops a space
%\IEEEcompsocitemizethanks{\IEEEcompsocthanksitem M. Shell is with the Department
%of Electrical and Computer Engineering, Georgia Institute of Technology, Atlanta,
%GA, 30332.\protect\\
%% note need leading \protect in front of \\ to get a newline within \thanks as
%% \\ is fragile and will error, could use \hfil\break instead.
%E-mail: see http://www.michaelshell.org/contact.html
%\IEEEcompsocthanksitem J. Doe and J. Doe are with Anonymous University.}% <-this % stops a space
%\thanks{Manuscript received April 19, 2005; revised December 27, 2012.}
}

% note the % following the last \IEEEmembership and also \thanks - 
% these prevent an unwanted space from occurring between the last author name
% and the end of the author line. i.e., if you had this:
% 
% \author{....lastname \thanks{...} \thanks{...} }
%                     ^------------^------------^----Do not want these spaces!
%
% a space would be appended to the last name and could cause every name on that
% line to be shifted left slightly. This is one of those "LaTeX things". For
% instance, "\textbf{A} \textbf{B}" will typeset as "A B" not "AB". To get
% "AB" then you have to do: "\textbf{A}\textbf{B}"
% \thanks is no different in this regard, so shield the last } of each \thanks
% that ends a line with a % and do not let a space in before the next \thanks.
% Spaces after \IEEEmembership other than the last one are OK (and needed) as
% you are supposed to have spaces between the names. For what it is worth,
% this is a minor point as most people would not even notice if the said evil
% space somehow managed to creep in.



% The paper headers
%\markboth{Journal of \LaTeX\ Class Files,~Vol.~11, No.~4, December~2012}%
%{Shell \MakeLowercase{\textit{et al.}}: Bare Advanced Demo of IEEEtran.cls for Journals}

% The only time the second header will appear is for the odd numbered pages
% after the title page when using the twoside option.
% 
% *** Note that you probably will NOT want to include the author's ***
% *** name in the headers of peer review papers.                   ***
% You can use \ifCLASSOPTIONpeerreview for conditional compilation here if
% you desire.



% The publisher's ID mark at the bottom of the page is less important with
% Computer Society journal papers as those publications place the marks
% outside of the main text columns and, therefore, unlike regular IEEE
% journals, the available text space is not reduced by their presence.
% If you want to put a publisher's ID mark on the page you can do it like
% this:
%\IEEEpubid{0000--0000/00\$00.00~\copyright~2012 IEEE}
% or like this to get the Computer Society new two part style.
%\IEEEpubid{\makebox[\columnwidth]{\hfill 0000--0000/00/\$00.00~\copyright~2012 IEEE}%
%\hspace{\columnsep}\makebox[\columnwidth]{Published by the IEEE Computer Society\hfill}}
% Remember, if you use this you must call \IEEEpubidadjcol in the second
% column for its text to clear the IEEEpubid mark (Computer Society journal
% papers don't need this extra clearance.)



% use for special paper notices
%\IEEEspecialpapernotice{(Invited Paper)}



% for Computer Society papers, we must declare the abstract and index terms
% PRIOR to the title within the \IEEEtitleabstractindextext IEEEtran
% command as these need to go into the title area created by \maketitle.
% As a general rule, do not put math, special symbols or citations
% in the abstract or keywords.
\IEEEtitleabstractindextext{%
\begin{abstract}
With this work the circuit solution system developed in ECE1254 for linear constant source RLC circuits is extended to compute steady state responses to periodic sources, to include support for a simple non-linear diode model, and to include support for Taylor expanded conductance circuit elements.
It is assumed that all input frequencies are multiples of a single fundamental frequency.
Anharmonic effects will be ignored, and it will be assumed that only periodic outputs with harmonics that are multiples of the input frequencies will be observed.
The solution utilizes the \emph{Harmonic Balance} method described in \citep{giannini2004NonlinearMicrowaveCircuitDesign}, where
time variation of the inputs and outputs is expressed as discrete time Fourier series expressed in a matrix form that can be handled readily by software.
\end{abstract}

% Note that keywords are not normally used for peerreview papers.  Sync these up with the pdfkeywords above.
\begin{IEEEkeywords}
ECE1254, Harmonic Balance, non-linear circuit, diode, discrete Fourier transform
\end{IEEEkeywords}}


% make the title area
\maketitle


% To allow for easy dual compilation without having to reenter the
% abstract/keywords data, the \IEEEtitleabstractindextext text will
% not be used in maketitle, but will appear (i.e., to be "transported")
% here as \IEEEdisplaynontitleabstractindextext when compsoc mode
% is not selected <OR> if conference mode is selected - because compsoc
% conference papers position the abstract like regular (non-compsoc)
% papers do!
\IEEEdisplaynontitleabstractindextext
% \IEEEdisplaynontitleabstractindextext has no effect when using
% compsoc under a non-conference mode.


% For peer review papers, you can put extra information on the cover
% page as needed:
% \ifCLASSOPTIONpeerreview
% \begin{center} \bfseries EDICS Category: 3-BBND \end{center}
% \fi
%
% For peerreview papers, this IEEEtran command inserts a page break and
% creates the second title. It will be ignored for other modes.
\IEEEpeerreviewmaketitle



\section{Introduction}
% Computer Society journal papers do something a tad strange with the very
% first section heading (almost always called "Introduction"). They place it
% ABOVE the main text! IEEEtran.cls currently does not do this for you.
% However, You can achieve this effect by making LaTeX jump through some
% hoops via something like:
%
%\ifCLASSOPTIONcompsoc
%  \noindent\raisebox{2\baselineskip}[0pt][0pt]%
%  {\parbox{\columnwidth}{\section{Introduction}\label{sec:introduction}%
%  \global\everypar=\everypar}}%
%  \vspace{-1\baselineskip}\vspace{-\parskip}\par
%\else
%  \section{Introduction}\label{sec:introduction}\par
%\fi
%
% Admittedly, this is a hack and may well be fragile, but seems to do the
% trick for me. Note the need to keep any \label that may be used right
% after \section in the above as the hack puts \section within a raised box.



% The very first letter is a 2 line initial drop letter followed
% by the rest of the first word in caps (small caps for compsoc).
% 
% form to use if the first word consists of a single letter:
% \IEEEPARstart{A}{demo} file is ....
% 
% form to use if you need the single drop letter followed by
% normal text (unknown if ever used by IEEE):
% \IEEEPARstart{A}{}demo file is ....
% 
% Some journals put the first two words in caps:
% \IEEEPARstart{T}{his demo} file is ....
% 
% Here we have the typical use of a "T" for an initial drop letter
% and "HIS" in caps to complete the first word.
\IEEEPARstart{T}{he}
Harmonic Balance method is an approach to determine the steady state response of nonlinear
circuits.
As only the steady state response of the circuit is of interest, the state of the circuit
only needs to be determined over one period.
Furthermore, it is assumed that for a sinusoidal input of a
fundamental frequency, the output signal will consist of a summation of sinusoids of integer multiples of
the input signal frequency (harmonics).
The final assumption is that a finite number of these harmonics,
\( N \), will be sufficient to accurately represent the output signal.
From the DFT, to obtain \( N \) discrete frequencies, \( N \) linearly spaced (in time) samples of a
signal are required.
Conversely, for \( N \) discrete frequencies, a signal with \( N \) time samples is required.
The Harmonic Balance exploits the linear, time-invariant nature of circuits.
In the time domain, the Modified Nodal Analysis (MNA) procedure results in a set of \( R \) equations.
Translated into the 
frequency domain, the result is \( R \times (2 N + 1) \) independent sets of equations.
One way to
think of this, is that each frequency can be considered its own independent circuit.

The objective is to represent the system in the form 

\begin{equation}\label{eqn:ece1254projectReport:560}
\BI(\BV) = \BY \BV.
\end{equation}

where the current \( \BI \) includes both the contributions of the linear sources (voltage and current sources), and includes non-linear contributions due to elements such as diodes.  The admittance matrix \( \BY \) has a block diagonal structure, with each block having a frequency dependency.  The vector \( \BV \) contains the Fourier coefficients of all the voltage and current unknowns in the system, including a term for each unknown at each frequency included in the Fourier series.  When the system includes non-linear elements, iterative methods will be required to solve for \( \BV \).

The MNA infrastructure developed during the course assignments was extended to allow AC current and voltage sources.  The following new netlist syntax

\begin{center}
\textbf{Ilabel \(n_1\) \(n_2\) AC \(m\) \(\nu\) \([\phi]\)}
\end{center} 
\begin{center}
\textbf{Vlabel \(n_1\) \(n_2\) AC \(m\) \(\nu\) \([\phi]\)}
\end{center} 

was used to represent sources of the form \( m \cos\lr{ 2 \pi \nu - \phi } \), with currents directed from node \( n_1 \) to \( n_2 \), and voltages measured at \( n_1 \) relative to \( n_2 \).  The phase \( \phi \) defaults to zero if unspecified.

The syntax used for the diode netlist extension was

\begin{center}
\textbf{Dlabel \(n_1\) \(n_2\) \(I_0\) \(V_T\)},
\end{center} 

representing a diode modelled as a current source \( i_d(t) = I_0 \lr{ e^{ \lr{v^{(n_1)}(t) - v^{(n_2)}(t)}/V_T } - 1 } \).

Finally, non-linear conductances \( g(t) = I_0 \lr{v^{(n_1)}(t) - v^{(n_2)}(t)}^k \) may be specified as

\begin{center}
\textbf{Plabel \(n_1\) \(n_2\) \(I_0\) \(V_T\) \( k \)},
\end{center} 

Any number of these may be specified in parallel to form an arbitrary Taylor expansion.

% You must have at least 2 lines in the paragraph with the drop letter
% (should never be an issue)
%I wish you the best of success.
%
%\hfill mds
% 
%\hfill December 27, 2012

\section{Background}
\subsection{Discrete Fourier Transform}
%\citep{giannini2004NonlinearMicrowaveCircuitDesign}

Following \citep{giannini2004NonlinearMicrowaveCircuitDesign} \S A.4, the circuit sources and outputs are assumed to be periodic (\( \omega_0 T = 2 \pi \)), and bandwidth limited, described by the following discrete Fourier transform pair

%\begin{equation}\label{eqn:ece1254projectReport:60}
\boxedEquation{eqn:ece1254projectReport:60}{
\begin{aligned}
x(t_k) &= \sum_{n = -N}^N X_n e^{ j n \omega_0 t_k} \\
X_n &= \inv{2 N + 1} \sum_{k = -N}^N x(t_k) e^{-j n \omega_0 t_k},
\end{aligned}
}
%\end{equation}

The times at which the signal is evaluated are evaluated only at the Nykvist times 
\index{Nykvist time instant}
\( t_k \), 

\begin{equation}\label{eqn:ece1254projectReport:20}
t_k = \frac{T k}{2 N + 1}, \qquad k \in [-N, \cdots N]
\end{equation}

all contained within the interior of the \( [-T/2, T/2] \) range of interest.  
A proof of this inversion relationship is given in \cref{appendix:discreteFourierInversion}.

A matrix representation of the transform pair is desired.  
Let \( x_k = x(t_k) \), and 
substitute \( t_k \) from \cref{eqn:ece1254projectReport:20}

\begin{equation}\label{eqn:ece1254projectReport:140}
\begin{aligned}
x_k &= \sum_{n = -N}^N X_n e^{ 2 \pi j n k /(2 N + 1)} \\
X_n &= \inv{2 N + 1} \sum_{k = -N}^N x_k e^{- 2 \pi j n k /(2 N + 1)}.
\end{aligned}
\end{equation}

With \( 
\Bx = 
{\begin{bmatrix}
x_{-N} &
\cdots &
x_0 &
\cdots &
x_{N}
\end{bmatrix}}^\T,
\BX = 
{\begin{bmatrix}
X_{-N} &
\cdots &
X_0 &
\cdots &
X_{N}
\end{bmatrix}}^\T 
\),  \( \alpha = e^{ 2 \pi j /(2 N + 1) } \), and \( \BF \) defined as

\begin{equation}\label{eqn:discreteFourierMatrixForm:360}
\small
\begin{bmatrix}
 \alpha^{\lr{-N} \lr{-N} } &  \alpha^{\lr{1-N} \lr{-N} }  & . &  \alpha^{ \lr{N-1} \lr{-N} }  &  \alpha^{\lr{ N} \lr{-N} } \\
 \alpha^{\lr{-N} \lr{1-N} } &  \alpha^{\lr{1-N} \lr{1-N} }  & . &  \alpha^{ \lr{N-1} \lr{1-N} }  &  \alpha^{\lr{ N} \lr{1-N} } \\
 .              &  .                      & .      & .                           &  .               \\
 \alpha^{\lr{-N} \lr{N-1} } &  \alpha^{\lr{1-N} \lr{N-1} }  & . &  \alpha^{\lr{N-1} \lr{N-1} }  &  \alpha^{\lr{N} \lr{N-1} } \\
 \alpha^{\lr{-N} \lr{N} } &  \alpha^{\lr{1-N} \lr{N} }  & . &  \alpha^{\lr{N-1}\lr{ N} }  &  \alpha^{\lr{N}\lr{ N} } \\
\end{bmatrix},
\end{equation}

the transform pair has the matrix form

\begin{subequations}
\begin{equation}\label{eqn:discreteFourierMatrixForm:380}
\Bx = \BF \BX
\end{equation}
\begin{equation}\label{eqn:discreteFourierMatrixForm:400}
\BX = \inv{2 N + 1} \overbar{\BF} \Bx,
\end{equation}
\end{subequations}

where \( \overbar{\BF} \) is the complex conjugate of \( \BF \).

%Different conventions for the transform pairs are possible.  This 
% needed in second column of first page if using \IEEEpubid
%\IEEEpubidadjcol

\subsection{Harmonic Balance equations}

The time domain MNA equations for an RLC network with non-linear sources can be put into the form

\begin{equation}\label{eqn:ece1254projectReport:420}
\BG \Bx(t) + \BC \dot{\Bx}(t) = \sum_{i = 1}^M \Bb_i u_i(t) + \sum_{i = 1}^S I_i \Bd_i w_i(t).
\end{equation}

With \( R \) equal to the number of unknowns, \( S \) equal to the number of non-linear elements, \( M \) equal to the number of (positive, negative and zero) frequency sources, the dimensions of these matrices are \( \BG, \BC \in \text{\R{R \times R}} \), and \( \Bb_i, \Bd_i, \Bx(t) \in \text{\C{R \times 1}} \).

The linear sources can be grouped into a vector.  For example, with a constant source, and one source at the twice the fundamental frequency, this would be

\begin{equation}\label{eqn:ece1254projectReport:440}
\Bu(t) = 
\begin{bmatrix}
e^{ -j \omega_0 2 t } \\
1 \\
e^{ j \omega_0 2 t }
\end{bmatrix},
\end{equation}

In an implementation of code that generates MNA equations it is convenient to store just the frequencies associated with these sources, which carries the same information without needing to encode the time dependence explicitly, as in

\begin{equation}\label{eqn:ece1254projectReport:460}
\Bu =
\begin{bmatrix}
- 2 \omega_0 \\
0 \\
2 \omega_0 
\end{bmatrix}.
\end{equation}

For the non-linear sources, the magnitudes \( I_i \) have been factored out, so that the vectors \( \Bd_i \) contain only the values \( 0, \pm 1 \).  The reasons for this will become clear in the example to follow.

\subsection{Frequency domain representation of MNA equations}

Assuming a bandwidth limited periodic representation of the unknowns vector \( \Bx \) in \cref{eqn:ece1254projectReport:420}

\begin{equation}\label{eqn:ece1254projectReport:500}
x^{(i)}(t) = \sum_{n=-N}^N X_n^{(i)} e^{j \omega_0 t},
\end{equation}

the derivative has the form

\begin{equation}\label{eqn:ece1254projectReport:520}
\dot{x^{(i)}}(t) = j \omega_0 \sum_{n=-N}^N X_n^{(i)} n e^{j \omega_0 t}.
\end{equation}

The Fourier component representation of the MNA equations can be written as

\begin{equation}\label{eqn:ece1254projectReport:600}
\begin{aligned}
0 &= \sum_{n = -N}^N e^{j \omega_0 n t}
\Biglr{
   \lr{ \BG + j \omega_0 n \BC } 
\begin{bmatrix}
X_n^{(1)} \\
X_n^{(2)} \\
\vdots \\
\end{bmatrix} \\
&- 
[ \Bb_1 \cdots \Bb_M ]
\begin{bmatrix}
U_n^{(1)} \\
U_n^{(2)} \\
\vdots \\
\end{bmatrix}
- \sum_{i = 1}^S I_i \Bd_i
W_n^{(i)} 
}.
\end{aligned}
\end{equation}

%All the time dependence is moved into the exponentials.  
Given the 
assumption of periodicity, each of these equations must separately equal zero for each \( ( n, i ) \) pair, or

\begin{equation}\label{eqn:ece1254projectReport:540}
\begin{aligned}
\lr{
\BG + j \omega_0 n \BC
}
\begin{bmatrix}
V_n^{(1)} \\
V_n^{(2)} \\
\vdots \\
\end{bmatrix}
&= 
[\Bb_i \cdots \Bb_M]
\begin{bmatrix}
U_n^{(1)} \\
U_n^{(2)} \\
\vdots \\
\end{bmatrix} \\
&+ \sum_{i = 1}^S I_i \Bd_i
W_n^{(i)} 
%&+
%[I_1 \Bd_1 \cdots I_S \Bd_S]
%\begin{bmatrix}
%W_n^{(1)} \\
%W_n^{(2)} \\
%\vdots \\
%\end{bmatrix}
.
\end{aligned}
\end{equation}

It is desirable to use an ordering of the vector components given by

\begin{equation}\label{eqn:ece1254projectReport:580}
\BI = 
\begin{bmatrix}
I^{(1)}_{-N} \\
I^{(2)}_{-N} \\
%I^{(3)}_{-N} \\
\vdots \\
I^{(1)}_{1-N} \\
I^{(2)}_{1-N} \\
%I^{(3)}_{1-N} \\
\vdots \\
I^{(1)}_{N-1} \\
I^{(2)}_{N-1} \\
%I^{(3)}_{N-1} \\
\vdots \\
I^{(1)}_{N} \\
I^{(2)}_{N} \\
%I^{(3)}_{N} \\
\vdots \\
\end{bmatrix},
\qquad \BV = 
\begin{bmatrix}
V^{(1)}_{-N} \\
V^{(2)}_{-N} \\
%V^{(3)}_{-N} \\
\vdots \\
V^{(1)}_{1-N} \\
V^{(2)}_{1-N} \\
%V^{(3)}_{1-N} \\
\vdots \\
V^{(1)}_{N-1} \\
V^{(2)}_{N-1} \\
%V^{(3)}_{N-1} \\
\vdots \\
V^{(1)}_{N} \\
V^{(2)}_{N} \\
%V^{(3)}_{N} \\
\vdots \\
\end{bmatrix}
\end{equation}

The superscript index of each element corresponds to the physical circuit node it is related to (i.e.
current flowing to the node, or the node voltage).  
Those unknown physical circuit elements are generated using a modified nodal analysis procedure.
The subscript corresponds to the harmonic frequency, so the end result is as if there is a 
node voltage for each harmonic frequency.

Ordering the vectors this way allows a block diagonal admittance matrix to be generated (\citep{giannini2004NonlinearMicrowaveCircuitDesign}
 \S 1.85)
, consisting of the admittance matrix that would be seen for each frequency.
%Peeter: I believe your code would give us the G and C matrices that arise from LMS formulations. This is
%great as we can generate each block Y(n?0) = G - jn?0C.

%Using the ordering of \cref{eqn:ece1254projectReport:580}, 
The structure of this equation is

%\begin{equation}\label{eqn:ece1254projectReport:1340}
\boxedEquation{eqn:ece1254projectReport:1340}{
\BY \BV = \BI + \mathcal{I}(\BV),
}
%\end{equation}

where \( \BY \) is block diagonal

\begin{equation}\label{eqn:ece1254projectReport:1360}
\small{
\begin{bmatrix}
\BG + j \omega_0 (-N) \BC &        &                                   & \\
                            \qquad\ddots &                                   & \\
                                   &  \BG + j \omega_0 (0) \BC         & \\
                                   &                            \qquad\ddots & \\
                                   &                                   & \BG + j \omega_0 (N) \BC  
\end{bmatrix}
}
\end{equation}

Construction of the Harmonic Balance stamps for the constant current \( \BI \) and non-linear current \( \BI(\BV) \) is nicely illustrated by the example to follow.

\subsection{Example.  RC circuit with a diode.}

\imageFigureHere{../../figures/ece1254/diodeWithResistorAndCapacitorFig1}{Simple non-linear circuit}{fig:diodeWithResistorAndCapacitorFig1}{3in}
%\begin{figure}[!h]
%\centering
%\includegraphics[width=2.5in]{../../figures/ece1254/diodeWithResistorAndCapacitorFig1}
%\caption{Simple non-linear circuit}
%\label{fig:diodeWithResistorAndCapacitorFig1}
%\end{figure}

%FIXME: section reference here, not figure reference?
A diode with an RC load is illustrated in 
\cref{fig:diodeWithResistorAndCapacitorFig1}.
%fig. \ref{fig:diodeWithResistorAndCapacitorFig1}.
With

%\begin{enumerate}
%\item \( 0 = i_s - i_d \)
%\item \( Z v^{(2)} + C dv^{(2)}/dt = i_d  \)
%\item \( i_d = I_0 \lr{ e^{(v^{(1)} - v^{(2)})/V_T} - 1} \)
%\end{enumerate}
%
%To setup for matrix form, let

\begin{equation}\label{eqn:ece1254projectReport:640}
\begin{aligned}
\BG &=
\begin{bmatrix}
0 & 0 \\
0 & Z \\
\end{bmatrix}, \quad
\BC =
\begin{bmatrix}
0 & 0 \\
0 & C \\
\end{bmatrix}, \\
\Bv(t) &=
\begin{bmatrix}
v^{(1)}(t) \\
v^{(2)}(t) \\
\end{bmatrix}, \quad
\Bd = 
\begin{bmatrix}
1 \\
-1
\end{bmatrix}, \quad
\Bb = 
\begin{bmatrix}
1 \\
0
\end{bmatrix},
\end{aligned}
\end{equation}

the time domain equations for this circuit are

\begin{equation}\label{eqn:ece1254projectReport:1320}
\begin{aligned}
\BG \Bv(t)
&+ \BC \Bv'(t)
%=
%\Bb i_s(t)
%+
%I_0
%\Bd 
%\lr{
%e^{ (v^{(1)}(t) - v^{(2)}(t))/V_T} - 1
%}
=
\begin{bmatrix}
\Bb & -I_0 \Bd
\end{bmatrix}
\begin{bmatrix}
i_s(t) \\
1
\end{bmatrix} \\
&+
I_0 \Bd
e^{ (v^{(1)}(t) - v^{(2)}(t))/V_T}.
\end{aligned}
\end{equation}

For the exponential, let the transform pair be
%Introduce a transform pair 
%Harmonic Balance is essentially the 
%assumption that the input and outputs are assumed to be a bandwidth limited periodic signal, and the non-linear components can be approximated by the same

%\begin{subequations}
%\begin{equation}\label{eqn:ece1254projectReport:740}
%i_s(t) = \sum_{n=-N}^N I^{(s)}_n e^{ j \omega_0 n t },
%\end{equation}
%\begin{equation}\label{eqn:ece1254projectReport:760}
%v^{(k)}(t) = 
%\sum_{n=-N}^N V^{(k)}_n e^{ j \omega_0 n t },
%\end{equation}
\begin{equation}\label{eqn:ece1254projectReport:1300}
\epsilon(t) = 
e^{ (v^{(1)}(t) - v^{(2)}(t))/V_T}
\simeq
\sum_{n=-N}^N E_n e^{ j \omega_0 n t }.
\end{equation}
%\end{subequations}

%This is an 
%approximation for which equality only occurs 
This is an approximation at any times other than the Nykvist sampling times \( t_k = T k/(2 N + 1) \).
%.  The Fourier series provides a periodic extension to other times that will approximate the underlying periodic non-linear relation.

%With all the time dependence locked into the exponentials, the derivatives are really easy to calculate
%
%\begin{equation}\label{eqn:ece1254projectReport:780}
%\frac{d}{dt} v^{(k)}(t) = 
%\sum_{n=-N}^N j \omega_0 n V^{(k)}_n e^{ j \omega_0 n t }.
%\end{equation}
%
%Inserting all of these into \cref{eqn:ece1254projectReport:1320} gives
%
%\begin{equation}\label{eqn:ece1254projectReport:800}
%\sum_{n=-N}^N e^{ j \omega_0 n t} \lr{ \BG + j \omega_0 n \BC } 
%\begin{bmatrix}
%V^{(1)}_n \\
%V^{(2)}_n \\
%\end{bmatrix}
%=
%\sum_{n=-N}^N e^{ j \omega_0 n t} 
%\lr{
%-I_0 \Bd \delta_{n 0}
%+
%\Bb I^{(s)}_n
%+ I_0 \Bd E_n
%}.
%\end{equation}
%
%The periodic assumption requires equality for each \( e^{j \omega_0 n t} \), or

The frequency domain representation is

\begin{equation}\label{eqn:ece1254projectReport:820}
\lr{ \BG + j \omega_0 n \BC } 
\begin{bmatrix}
V^{(1)}_n \\
V^{(2)}_n \\
\end{bmatrix}
=
-I_0 \Bd \delta_{n 0}
+
\Bb I^{(s)}_n
+ I_0 \Bd E_n.
\end{equation}

To illustrate the stamping procedure for the constant current consider the \( N = 1 \) case.  The block matrix form for the constant current can be seen to be

\begin{equation}\label{eqn:ece1254projectReport:840}
\BI
=
\begin{bmatrix}
\Bb I^{(s)}_{-1} \\
\Bb I^{(s)}_{0} - I_0 \Bd \\
\Bb I^{(s)}_{1} \\
\end{bmatrix}
\end{equation}

The non-linear current \( \mathcal{I}(\BV) \) needs to be examined further.  How much of this can be precomputed, and what is the simplest way to compute the Jacobian?

Continuing to use \( N = 1 \) to illustrate, a vector of Fourier components for the exponentials can be formed

\begin{equation}\label{eqn:ece1254projectReport:880}
\BE =
\begin{bmatrix}
E_{-1} \\
E_{0} \\
E_{1} \\
\end{bmatrix}, \quad
\Bepsilon =
\begin{bmatrix}
\epsilon_{-1} \\
\epsilon_{0} \\
\epsilon_{1} \\
\end{bmatrix},
\end{equation}

which allows the non-linear current to be expressed as

\begin{equation}\label{eqn:ece1254projectReport:900}
\begin{aligned}
\mathcal{I} &= 
I_0
\begin{bmatrix}
\Bd E_{-1} \\
\Bd E_{0} \\
\Bd E_{1} \\
\end{bmatrix}
=
I_0
\begin{bmatrix}
\Bd \begin{bmatrix} 1 & 0 & 0 \end{bmatrix} \BE \\
\Bd \begin{bmatrix} 0 & 1 & 0 \end{bmatrix} \BE \\
\Bd \begin{bmatrix} 0 & 0 & 1 \end{bmatrix} \BE
\end{bmatrix} \\
&=
I_0
\begin{bmatrix}
\Bd & 0 & 0 \\
0 & \Bd & 0 \\
0 & 0 & \Bd 
\end{bmatrix}
\BF^{-1} \Bepsilon
\end{aligned}
\end{equation}

In the last step \( \BE = \BF^{-1} \Bepsilon \) has been factored out (in its inverse Fourier form).  With

\begin{dmath}\label{eqn:ece1254projectReport:920}
\BD =
\begin{bmatrix}
\Bd & 0 & 0 \\
0 & \Bd & 0 \\
0 & 0   & \Bd \\
\end{bmatrix},
\end{dmath}

the current can now be written in a compact form

%\begin{dmath}\label{eqn:ece1254projectReport:940}
\boxedEquation{eqn:ece1254projectReport:960}{
\mathcal{I}(\BV) = 
I_0 \BD \BF^{-1} \Bepsilon(\BV),
}
%\end{dmath}

however, an appropriate form for \( \Bepsilon \) must still be found.

\begin{equation}\label{eqn:ece1254projectReport:980}
\begin{aligned}
\Bepsilon &= 
\begin{bmatrix}
\epsilon(t_{-1}) \\
\epsilon(t_{0}) \\
\epsilon(t_{1}) \\
\end{bmatrix}
=
\begin{bmatrix}
\exp\lr{ \lr{ v^{(1)}_{-1} - v^{(2)}_{-1} }/V_T } \\
\exp\lr{ \lr{ v^{(1)}_{0} - v^{(2)}_{0} }/V_T } \\
\exp\lr{ \lr{ v^{(1)}_{1} - v^{(2)}_{1} }/V_T }
\end{bmatrix} \\
%%%&=
%%%\begin{bmatrix}
%%%\exp\lr{ 
%%%\begin{bmatrix}
%%%1 & 0 & 0 
%%%\end{bmatrix}
%%%\lr{ \Bv^{(1)} - \Bv^{(2)} }/V_T } \\
%%%\exp\lr{ 
%%%\begin{bmatrix}
%%%0 & 1 & 0 
%%%\end{bmatrix}
%%%\lr{ \Bv^{(1)} - \Bv^{(2)} }/V_T } \\
%%%\exp\lr{ 
%%%\begin{bmatrix}
%%%0 & 0 & 1 
%%%\end{bmatrix}
%%%\lr{ \Bv^{(1)} - \Bv^{(2)} }/V_T } \\
%%%\end{bmatrix} \\
%%%&=
%%%\begin{bmatrix}
%%%\exp\lr{ 
%%%\begin{bmatrix}
%%%1 & 0 & 0 
%%%\end{bmatrix}
%%%\BF
%%%\lr{ \BV^{(1)} - \BV^{(2)} }/V_T } \\
%%%\exp\lr{ 
%%%\begin{bmatrix}
%%%0 & 1 & 0 
%%%\end{bmatrix}
%%%\BF
%%%\lr{ \BV^{(1)} - \BV^{(2)} }/V_T } \\
%%%\exp\lr{ 
%%%\begin{bmatrix}
%%%0 & 0 & 1 
%%%\end{bmatrix}
%%%\BF
%%%\lr{ \BV^{(1)} - \BV^{(2)} }/V_T } \\
%%%\end{bmatrix}.
\end{aligned}
\end{equation}

Each of the time domain voltage differences can be cast into vector form, and then expressed in the frequency domain.  For example

\begin{equation}\label{eqn:ece1254projectReport:1380}
\begin{aligned}
v^{(1)}_{-1} - v^{(2)}_{-1} 
&=
\begin{bmatrix}
1 & 0 & 0 
\end{bmatrix}
\lr{ \Bv^{(1)} - \Bv^{(2)} } \\
&=
\begin{bmatrix}
1 & 0 & 0 
\end{bmatrix}
\BF
\lr{ \BV^{(1)} - \BV^{(2)} }
\end{aligned}
\end{equation}

This difference of Fourier component vectors needs to be expressed in terms of the big vector of unknown Fourier components \( \BV \)

\begin{dmath}\label{eqn:ece1254projectReport:1000}
\BV^{(1)} - \BV^{(2)} 
=
\begin{bmatrix}
V^{(1)}_{-1} - V^{(2)}_{-1} \\
V^{(1)}_{0} - V^{(2)}_{0} \\
V^{(1)}_{1} - V^{(2)}_{1} \\
\end{bmatrix}
=
\begin{bmatrix}
1 & -1 & 0 & 0 & 0 & 0 \\
0 &  0 & 1 & -1 & 0 & 0 \\
0 &  0 & 0 & 0 & 1 & -1 \\
\end{bmatrix}
\begin{bmatrix}
V_{-1}^{(1)} \\
V_{-1}^{(2)} \\
V_{0}^{(1)} \\
V_{0}^{(2)} \\
V_{1}^{(1)} \\
V_{1}^{(2)} \\
\end{bmatrix}
\end{dmath}

This is just

\begin{equation}\label{eqn:ece1254projectReport:1400}
\BV^{(1)} - \BV^{(2)} 
=
\begin{bmatrix}
\Bd^\T & 0 & 0 \\
0 & \Bd^\T & 0 \\
0 & 0 & \Bd^\T \\
\end{bmatrix}
\BV
= \BD^\T \BV.
\end{equation}

With
%\begin{equation}\label{eqn:ece1254projectReport:1040}
\boxedEquation{eqn:ece1254projectReport:1040}{
\BH 
= 
\BF \BD^\T /V_T
=
\begin{bmatrix}
\Bh_1^\T \\
\Bh_2^\T \\
\Bh_3^\T
\end{bmatrix},
}
%\end{equation}

a compact matrix representation of the non-linear current is now complete

%\begin{dmath}\label{eqn:ece1254projectReport:1060}
\boxedEquation{eqn:ece1254projectReport:1080}{
\Bepsilon(\BV)
=
\begin{bmatrix}
e^{\Bh_1^\T \BV} \\
e^{\Bh_2^\T \BV} \\
e^{\Bh_3^\T \BV} \\
\end{bmatrix}.
}
%\end{dmath}

\subsection{Jacobian}

With a compact matrix representation of the non-linear current, attention can now be turned to the Jacobian of the non-linear current.  Let \( \BA = I_0 \BD \BF^{-1} = [ a_{ij} ]_{ij} \).  The coordinates of the non-linear current (with summation implied) are

\begin{dmath}\label{eqn:ece1254projectReport:1120}
\mathcal{I}_i = a_{ik} \epsilon_k = a_{ik} \exp\lr{ \Bh_k^\T \BV }.
\end{dmath}

so the Jacobian components are

\begin{dmath}\label{eqn:ece1254projectReport:1140}
[\BJ^{\mathcal{I}}]_{ij} 
= 
a_{ik} \epsilon_k = a_{ik} 
\PD{V_j}{}
\exp\lr{ \Bh_k^\T \BV }
= 
a_{ik} 
h_{kj}
\exp\lr{ \Bh_k^\T \BV }.
\end{dmath}

Back in matrix form, after factoring out the \( \BA \), the Jacobian is

\begin{dmath}\label{eqn:ece1254projectReport:1140}
\BJ^{\mathcal{I}}
= 
\BA
{\begin{bmatrix}
h_{ij}
\exp\lr{ \Bh_i^\T \BV }
\end{bmatrix}}_{ij}.
\end{dmath}

The second factor has all the rows of \( \BH \), but each multiplied by an exponential, so the Jacobian for the non-linear current 
is now completely determined

%\begin{dmath}\label{eqn:ece1254projectReport:1180}
\boxedEquation{eqn:ece1254projectReport:1200}{
\BJ^{\mathcal{I}}( \BV ) =
I_0 \BD \BF^{-1} 
\begin{bmatrix}
\Bh_1^\T \exp\lr{ \Bh_1^\T \BV } \\
\Bh_2^\T \exp\lr{ \Bh_2^\T \BV } \\
\vdots
\end{bmatrix}.
%\end{dmath}
}

%Assembling all the Jacobian coordinates back into matrix form gives
% Factoring out \( \BU = [h_{ij} \exp\lr{ \Bh_i^\T \BV }]_{ij} \), 
%
%\begin{dmath}\label{eqn:ece1254projectReport:1160}
%\BJ^{\mathcal{I}}
%=
%\BA
%\begin{bmatrix}
%\Bh_1^\T \exp\lr{ \Bh_1^\T \BV } \\
%\Bh_2^\T \exp\lr{ \Bh_2^\T \BV } \\
%\vdots
%\end{bmatrix}.
%\end{dmath}

A quick sanity check of dimensions seems worthwhile, and shows that all is well

\begin{itemize}
\item \( \BA \) : \( R(2 N + 1) \times 2 N + 1 \)
\item \( \BU \) : \( 2 N + 1 \times R(2 N + 1) \)
\item \( \BJ^{\mathcal{I}} \) : \( R(2 N + 1) \times R(2 N + 1) \)
\end{itemize}

This derivation was done exclusively with the exponential nonlinearity that is found in the diode model, but is easy to generalize.  
If the nonlinearity is of the form
\( I_0 g((v^{(1)} - v^{(2)})/V_T) \), then the non-linear current and its Jacobian will have the same structural form

\begin{subequations}
\begin{equation}\label{eqn:ece1254projectReport:1440}
\mathcal{I}(\BV)
I_0 \BD \BF^{-1} 
\begin{bmatrix}
\Bh_1^\T g(\Bh_1^\T \BV) \\
\Bh_2^\T g(\Bh_2^\T \BV) \\
\vdots
\end{bmatrix}
\end{equation}
\begin{dmath}\label{eqn:ece1254projectReport:1420}
\BJ^{\mathcal{I}}( \BV ) =
I_0 \BD \BF^{-1} 
\begin{bmatrix}
\Bh_1^\T \evalbar{ g'(x)}{x= \Bh_1^\T \BV } \\
\Bh_2^\T \evalbar{ g'(x)}{x= \Bh_2^\T \BV } \\
\vdots
\end{bmatrix}.
\end{dmath}
\end{subequations}

\subsection{Newton's method solution}

All the pieces required for a Newton's method solution are now in place.  The goal is to find a value of \( \BV \) that provides the zero

\begin{dmath}\label{eqn:ece1254projectReport:1220}
f(\BV) = \BY \BV - \BI - \mathcal{I}(\BV).
\end{dmath}

Expansion to first order around an initial guess \( \BV^0 \), gives

\begin{dmath}\label{eqn:ece1254projectReport:1240}
f( \BV^0 + \Delta \BV ) = f(\BV^0) + \BJ(\BV^0) \Delta \BV \approx 0,
\end{dmath}

where the full Jacobian of \( f(\BV) \) is

\begin{dmath}\label{eqn:ece1254projectReport:1260}
\BJ(\BV) = \BY - \BJ^{\mathcal{I}}(\BV).
\end{dmath}

The Newton's method refinement of the initial guess follows by inversion

\begin{dmath}\label{eqn:ece1254projectReport:1280}
\Delta \BV = -\lr{ \BY - \BJ^{\mathcal{I}}(\BV^0) }^{-1} f(\BV^0).
\end{dmath}

The non-linearities of some of the circuits that were solved as tests required source stepping, as the Jacobian grew beyond the representable range of the double matrix representation in some cases and became uninvertible.  A heuristic that was found effective was to half the step size for each Newton's method iteration that the Jacobian became ill-conditioned, and to double it for each iteration where convergence was achieved.

\subsection{Alternative handling of the non-linear currents and Jacobians}

Two methods for the calculation of the non-linear currents and Jacobians were developed while coding this Harmonic Balance implementation.  
Our first implementation used 
a ``big \( \BF \)'' DFT matrix, of size \( ( R (2 N + 1 ) \times R (2 N + 1) ) \).  
%The non-linear current was calculated in the time domain as \( \mathcal{I}(\Bv) \).
With the elements of \( \BV \) grouped by frequency, a DFT matrix that operated on \( \BV \) was formed where
every value in each frequency group was multiplied by the same value in the DFT matrix.
%For \( R \) values in each group (corresponds to \( R \) unknowns of a circuit) the values in the original 
%matrix will be spaced by \( R \), padded with zeros.
%And these modified rows will be repeated and shifted
%right 1, \( M \) times.
%The resulting DFT matrix has dimension 
%(or use 2Nh +1)
%For the case of nonlinear components, there will be a current, 
The nonlinear currents \( \mathcal{I}(\Bv) \), produced at each time sample to be determined were calculated.  Using the 
``big'' DFT of these currents, the discrete frequency
components of the nonlinear currents were obtained.
Each nth-frequency component of this current would be added
to the equation of the nth-set of MNA equations corresponding to the nodes the element is connected
to.

This non-linear current calculation was then refined to work completely in the frequency domain, precalculating a number of the common expressions that were found in both the current and its Jacobian.  This refinement used the (standard) DFT matrix \( \BF \) of dimensions \( (2 N + 1 ) \times (2 N + 1 ) \), as described above.

\section{Results}

\subsection{Low pass filter}

%\subfloat[Parameters]{%
%\small
%\begin{tabular}[b]{|l|r|} \hline
%  Parameter & Value \\ \hline\hline
%  \( \nu_0 \) & 1 MHz \\ \hline
%  \( V_1 \) & 10 V (peak) \\ \hline
%  \( R_1 \) & 1 k \(\Omega\) \\ \hline
%  \( C_1 \) & 1 \(\mu\) F \\ \hline
%  \( D_1, I_0 \) & 10 pA \\ \hline
%  \( D_1, V_T \) & 25 mV \\ \hline
%\end{tabular}
%\label{tab:lowPassFilter:lowPassFilterParameters}}

\imageFigureHere{../../figures/ece1254/report/lowPassFilterFig1}{Low pass filter}{fig:lowPassFilterFig1}{3.0in}
%\begin{figure}[!h]
%\centering
%\includegraphics[width=2.5in]{../../figures/ece1254/report/lowPassFilterFig1}
%\caption{Low pass filter}
%\label{fig:lowPassFilterFig1}
%\end{figure}

A low pass filter with a cut off frequency of 5 MHz is shown in 
\cref{fig:lowPassFilterFig1}.
%fig. \ref{fig:lowPassFilterFig1}.
With 
inputs 
ranging from 1 to 8 Mhz, this serves as a basic sanity test of the Harmonic Balance linear network solver.
The time domain input signal and the corresponding filtered output, with a visible reduction in higher frequency components, 
is shown in \cref{fig:lowPassFilterFig2}.  

\imageFigureHere{../../figures/ece1254/report/lowPassFilterSourceAndOutputVoltagesFig2}{Source and output}{fig:lowPassFilterFig2}{3.0in}
%\begin{figure}[!h]
%\centering
%\includegraphics[width=1.5in]{../../figures/ece1254/report/lowPassFilterSourceAndOutputVoltagesFig2}
%\caption{Source and output voltages.}
%\label{fig:lowPassFilterFig2}
%\end{figure}

The magnitudes of the frequency domain components in the input signal are shown in \cref{fig:lowPassFilterFig3}.

\imageFigureHere{../../figures/ece1254/report/lowPassFilterInputVoltageSpectrumFig3}{Input voltage spectrum.}{fig:lowPassFilterFig3}{3.0in}

The output spectrum, seen in \cref{fig:lowPassFilterFig4},
shows attenuation at all frequencies.  However, all the higher frequency contributions are cut down significantly, as expected.

\imageFigureHere{../../figures/ece1254/report/lowPassFilterOutputVoltageSpectrumFig4}{Output voltage spectrum.}{fig:lowPassFilterFig4}{3.0in}

\subsection{Half wave rectifier}

A half wave rectifier, sketched in \cref{fig:HalfWaveRectifierFig1}, was suggested as a first test of the non-linear Harmonic Balance system.

\imageFigureHere{../../figures/ece1254/HalfWaveRectifierFig1}{Half wave rectifier circuit.}{fig:HalfWaveRectifierFig1}{3in}

%Simulation of this circuit with this Harmonic Balance system has the expected voltage characteristics as seen in
%\cref{fig:HalfWaveRectifierFig2}.
Many of the circuits considered in this report were solved using source stepping of the modified system

\begin{equation}\label{eqn:ece1254projectReport:1460}
f(\BV, \lambda) = \BY \BV - \lambda \BI - \mathcal{I}(\BV),
\end{equation}

with \( \lambda \in [0, 1] \), each time feeding the previous solution of \( \BV \) as the initial guess with the new source step value \( \lambda \).

That did not work for this system, which is ill-conditioned for \( \lambda = 0 \).  The Newton's method procedure was modifed to use \( \lambda \in [d\lambda, 1] \) if the Jacobian for \cref{eqn:ece1254projectReport:1460} was ill conditioned or NaN at \( \lambda = 1 \).  With that refinement of the source stepping algorithm, Harmonic Balance simulation of this half wave rectifier produces the expected results, as plotted in \cref{fig:HalfWaveRectifierFig2}.

\imageFigureHere{../../figures/ece1254/report/halfWaveRectifierVoltageFig2}{Half wave rectifier voltages.}{fig:HalfWaveRectifierFig2}{3in}


\subsection{AC to DC conversion}

Adding a capacitor to the half wave rectifier, as sketched in \cref{fig:typicalRectifierCircuitFig1}, adds an imaginary component to the admittance matrix \( \BY \), while still only requiring a single non-linear component.

\imageFigureHere{../../figures/ece1254/report/typicalRectifierCircuitFig1}{DC rectifier circuit}{fig:typicalRectifierCircuitFig1}{4.0in}
%%%\begin{figure}[!h]
%%%%\subfloat[AC to DC rectifier circuit]{%
%%%\subfloat{%
%%%   \includegraphics[width=2.5in]{../../figures/ece1254/report/typicalRectifierCircuitFig1}%
%%%   \label{fig:typicalRectifierCircuitFig1}%
%%%}
%%%%\subfloat[Parameters]{%
%%%\subfloat{%
%%%\tiny{
%%%\begin{tabular}[b]{|l|r|} \hline
%%%  Parameter & Value \\ \hline\hline
%%%  \( \nu_0 \) & 1 MHz \\ \hline
%%%  \( V_1 \) & 10 V (peak) \\ \hline
%%%  \( R_1 \) & 1 k \(\Omega\) \\ \hline
%%%  \( C_1 \) & 1 \(\mu\) F \\ \hline
%%%  \( D_1, I_0 \) & 10 pA \\ \hline
%%%  \( D_1, V_T \) & 25 mV \\ \hline
%%%\end{tabular}
%%%}
%%%\label{tab:typicalRectifierCircuit:typicalRectifierCircuitParameters}}
%%%%\hfil
%%%%\subfloat[Source and output voltages.]
%%%%\hfil
%%%%\subfloat[cputime vs error for variety of N.]{\includegraphics[width=1.5in]{../../figures/ece1254/report/typicalRectifierCircuitErrorAndCpuTimesFig3}%
%%%%\label{fig:typicalRectifierCircuitFig3}}
%%%\caption{AC to DC rectifier}
%%%\label{fig:typicalRectifierCircuit}
%%%\end{figure}

Plots of the simulation results with 
capacitor values of \( 1 \, \mu \) F, and \( 1 \) nF can be found in
\cref{fig:typicalRectifierCircuitFig3}, and 
\cref{fig:typicalRectifierCircuitFig4} respectively.  Further reduction of the capacitor size would eventually produce the original half wave rectifier results.

\imageFigureHere{../../figures/ece1254/report/typicalRectifierCircuitSourceAndOutputVoltagesFig2}{DC rectifier voltages at \( 1 \mu \) F}{fig:typicalRectifierCircuitFig3}{3.0in}
\imageFigureHere{../../figures/ece1254/report/typicalRectifierCircuitSmallerCapSourceAndOutputVoltagesFig2}{DC rectifier voltages at \( 1 \) nF}{fig:typicalRectifierCircuitFig4}{3.0in}

\subsection{Bridge rectifier}

The 
bridge rectifier circuit of 
\cref{fig:bridgeRectifierCircuitDiagramFig1} 
was used as a test system that included multiple non-linear elements.

\imageFigureHere{../../figures/ece1254/report/bridgeRectifierCircuitDiagramFig1}{Bridge rectifier}{fig:bridgeRectifierCircuitDiagramFig1}{3.0in}

Current will flow through only one pair of the diodes at any given point.  The inversion of any negative source voltages at the output node is seen in the plot of 

\imageFigureHere{../../figures/ece1254/report/bridgeRectifierOutputVoltageFig2}{Source and output voltages}{fig:bridgeRectifierFig2}{3.0in}

%\begin{figure}[!h]
%\centering
%%\subfloat[Bridge rectifier circuit diagram]{\includegraphics[width=1.5in]{../../figures/ece1254/report/}%
%%\label{fig:bridgeRectifierFig1}}
%%\subfloat[Parameters]{%
%%\small
%%\begin{tabular}[b]{|l|r|} \hline
%%  Parameter & Value \\ \hline\hline
%%  \( \nu_0 \) & 1 MHz \\ \hline
%%  \( V_1 \) & 10 V (peak) \\ \hline
%%  \( R_1 \) & 1 k \(\Omega\) \\ \hline
%%  \( C_1 \) & 1 \(\mu\) F \\ \hline
%%  \( D_1, I_0 \) & 10 pA \\ \hline
%%  \( D_1, V_T \) & 25 mV \\ \hline
%%\end{tabular}
%%\label{tab:bridgeRectifier:bridgeRectifierParameters}}
%%\hfil
%%\subfloat[Source and output voltages.]{\includegraphics[width=1.5in]{../../figures/ece1254/report/bridgeRectifierOutputVoltageFig2}%
%%\label{fig:bridgeRectifierFig2}}
%\hfil
%\subfloat[Source current.]{\includegraphics[width=1.5in]{../../figures/ece1254/report/bridgeRectifierSourceCurrentFig3}%
%\label{fig:bridgeRectifierFig3}}
%\hfil
%\subfloat[Diode voltages.]{\includegraphics[width=1.5in]{../../figures/ece1254/report/bridgeRectifierDiodeVoltagesFig4}%
%\label{fig:bridgeRectifierFig3}}
%\caption{Bridge rectifier circuit and simulation results.}
%\label{fig:bridgeRectifier}
%\end{figure}

\subsection{Cpu time and error vs N}

%\imageFigureHere{../../figures/ece1254/report/bridgeRectifierCircuitDiagramFig1}{Bridge rectifier}{fig:bridgeRectifierFig1}{2.0in}

The bridge rectifier circuit of
\cref{fig:bridgeRectifierCircuitDiagramFig1} was also used for
measurements of error and cpu times for range of values of \( N \).  These are plotted as a log-log plots in 
\cref{fig:bridgeRectifierFig5}.
Here error is measured as the biggest absolute difference between the solution for a given value of \( N \) vs 
a ``sufficiently large'' value of N (100).

\imageFigureHere{../../figures/ece1254/report/bridgeRectifierErrorAndCpuTimesFig5}{Error and Cpu vs N}{fig:bridgeRectifierFig5}{3.0in}

Observe that the relations are both nearly linear in the log-log plots, indicating that the error and cpu both have the form \( a N^m \).  Measuring the respective slopes (using the matlab function polyfit), the cpu time is found to be \( O(N^{1.5}) \), and the error is \( O(1/N^2) \), at least for this particular circuit.

%\begin{figure*}[!h]
%\centering
%\subfloat[Simple rectifier circuit diagram]{\includegraphics[width=2in]{../../figures/ece1254/report/simpleRectifierCircuitFig1}%
%\label{fig:simpleRectifierCircuitFig1}}
%\subfloat[Parameters]{%
%\tiny{
%\begin{tabular}[b]{|l|r|} \hline
%  Parameter & Value \\ \hline\hline
%  \( \nu_0 \) & 1 MHz \\ \hline
%  \( I_0 \) & 10 mA (peak) \\ \hline
%  \( R_1 \) & 1 k \(\Omega\) \\ \hline
%  \( D_1, I_0 \) & 10 pA \\ \hline
%  \( D_1, V_T \) & 25 mV \\ \hline
%\end{tabular}
%}
%\label{tab:simpleRectifierCircuit:simpleRectifierCircuitParameters}}
%\hfil
%\subfloat[Output.]{
%   \includegraphics[width=1.5in]{../../figures/ece1254/report/simpleRectifierCircuitVoltageFig2}%
%%   \def\svgwidth{1.5in}
%%   \input{../../figures/ece1254/report_svg/simpleRectifierCircuitVoltageFig2.pdf_tex}%
%\label{fig:simpleRectifierCircuitFig2}}
%%\subfloat[cputime vs error for variety of N.]{\includegraphics[width=1.5in]{../../figures/ece1254/report/simpleRectifierCircuitErrorAndCpuTimesFig3}%
%\caption{Simple rectifier circuit and simulation results.}
%\label{fig:simpleRectifierCircuit}
%\end{figure*}







%\begin{figure*}[!h]
%\hfil
%\subfloat[Source current.]{\includegraphics[width=1.5in]{../../figures/ece1254/report/bridgeRectifierCapFilterSourceCurrentFig3}%
%\label{fig:bridgeRectifierCapFilterFig3}}
%\caption{Bridge rectifier circuit with capacitive filter and simulation results.}
%\label{fig:bridgeRectifierCapFilter}
%\end{figure*}

\subsection{Taylor series non-linearities}

As a test for the power series non-linear support of this implementation, two of the diodes in the bridge rectifier circuit were replaced with equivalent four term Taylor expansions.  The results are plotted in \cref{fig:bridgeRectifierPowSourceAndOutputVoltagesFig2}.

\imageFigureHere{../../figures/ece1254/report/bridgeRectifierPowSourceAndOutputVoltagesFig2}{Source and output voltages for partial approximate bridge rectifier.}{fig:bridgeRectifierPowSourceAndOutputVoltagesFig2}{3.0in}

\subsection{Stiff systems}

For the low pass and rectifier circuits above, use of source stepping was sufficient (and required) to ensure convergence.  However, introduction of a capacitor into a bridge rectifier circuit, as shown in \cref{fig:bridgeRectifierCapFilterFig1} is enough to cause a number of Matlab messages ``Matrix is close to singular or badly scaled'' along with associated very small conditioning numbers.

\imageFigureHere{../../figures/ece1254/report/bridgeRectifierCapFilterFig1}{Bridge rectifier with capactive load.}{fig:bridgeRectifierCapFilterFig1}{4.0in}

%\begin{figure}[!h]
%\centering
%\subfloat[Bridge rectifier with capacitive filter circuit diagram]{\includegraphics[width=2.0in]{../../figures/ece1254/report/bridgeRectifierCapFilterFig1}%
%\label{fig:bridgeRectifierCapFilterFig1}}
%\subfloat[Parameters]{%
%\tiny{
%\begin{tabular}[b]{|l|r|} \hline
%  Parameter & Value \\ \hline\hline
%  \( \nu_0 \) & 1 MHz \\ \hline
%  \( V_1 \) & 3.5 V (peak) \\ \hline
%  \( R_1 \) & 1 k \(\Omega\) \\ \hline
%  \( C_1 \) & 1 \(\mu\) F \\ \hline
%  \( D_i, I_0 \) & 10 pA \\ \hline
%  \( D_i, V_T \) & 25 mV \\ \hline
%\end{tabular}
%}
%\label{tab:bridgeRectifierCapFilterParameters}}
%\end{figure}

The circuit simulation produces the expected a DC rectified signal as plotted in \cref{fig:bridgeRectifierCapFilterFig2}.

\imageFigureHere{../../figures/ece1254/report/bridgeRectifierCapFilterSourceAndOutputVoltagesFig2}{Source and output voltages.}{fig:bridgeRectifierCapFilterFig2}{3.0in}

Artifacts of the conditioning problems show up in the simulated outputs for the voltages through the diodes as plotted in \cref{fig:bridgeRectifierCapFilterFig3}.

\imageFigureHere{../../figures/ece1254/report/bridgeRectifierCapFilterDiodeVoltagesFig4}{Diode voltages.}{fig:bridgeRectifierCapFilterFig3}{3.0in}

The doubling and halfing heuristic used for the step sizes is insufficient for this circuit.
Even after 
reducing the step size by a many orders of magnitude, and capping the step doubling at the initial step size, the solutions for this circuit stubbornly refused to converge without Matlab numerical conditioning diagnostics.  For this circuit, and surely others, it appears that additional strategies beyond source stepping are required to compute a solution without hitting numerical instabilities.

% An example of a floating figure using the graphicx package.
% Note that \label must occur AFTER (or within) \caption.
% For figures, \caption should occur after the \includegraphics.
% Note that IEEEtran v1.7 and later has special internal code that
% is designed to preserve the operation of \label within \caption
% even when the captionsoff option is in effect. However, because
% of issues like this, it may be the safest practice to put all your
% \label just after \caption rather than within \caption{}.
%
% Reminder: the "draftcls" or "draftclsnofoot", not "draft", class
% option should be used if it is desired that the figures are to be
% displayed while in draft mode.
%
%\begin{figure}[!t]
%\centering
%\includegraphics[width=2.5in]{myfigure}
% where an .eps filename suffix will be assumed under latex, 
% and a .pdf suffix will be assumed for pdflatex; or what has been declared
% via \DeclareGraphicsExtensions.
%\caption{Simulation Results.}
%\label{fig_sim}
%\end{figure}

% Note that IEEE typically puts floats only at the top, even when this
% results in a large percentage of a column being occupied by floats.
% However, the Computer Society has been known to put floats at the bottom.


% An example of a double column floating figure using two subfigures.
% (The subfig.sty package must be loaded for this to work.)
% The subfigure \label commands are set within each subfloat command,
% and the \label for the overall figure must come after \caption.
% \hfil is used as a separator to get equal spacing.
% Watch out that the combined width of all the subfigures on a 
% line do not exceed the text width or a line break will occur.
%
%\begin{figure*}[!t]
%\centering
%\subfloat[Case I]{\includegraphics[width=2.5in]{box}%
%\label{fig_first_case}}
%\hfil
%\subfloat[Case II]{\includegraphics[width=2.5in]{box}%
%\label{fig_second_case}}
%\caption{Simulation results.}
%\label{fig_sim}
%\end{figure*}
%
% Note that often IEEE papers with subfigures do not employ subfigure
% captions (using the optional argument to \subfloat[]), but instead will
% reference/describe all of them (a), (b), etc., within the main caption.


% An example of a floating table. Note that, for IEEE style tables, the 
% \caption command should come BEFORE the table. Table text will default to
% \footnotesize as IEEE normally uses this smaller font for tables.
% The \label must come after \caption as always.
%
%\begin{table}[!t]
%% increase table row spacing, adjust to taste
%\renewcommand{\arraystretch}{1.3}
% if using array.sty, it might be a good idea to tweak the value of
% \extrarowheight as needed to properly center the text within the cells
%\caption{An Example of a Table}
%\label{table_example}
%\centering
%% Some packages, such as MDW tools, offer better commands for making tables
%% than the plain LaTeX2e tabular which is used here.
%\begin{tabular}{|c||c|}
%\hline
%One & Two\\
%\hline
%Three & Four\\
%\hline
%\end{tabular}
%\end{table}


% Note that IEEE does not put floats in the very first column - or typically
% anywhere on the first page for that matter. Also, in-text middle ("here")
% positioning is not used. Most IEEE journals use top floats exclusively.
% However, Computer Society journals sometimes do use bottom floats - bear
% this in mind when choosing appropriate optional arguments for the
% figure/table environments.
% Note that, LaTeX2e, unlike IEEE journals, places footnotes above bottom
% floats. This can be corrected via the \fnbelowfloat command of the
% stfloats package.



\section{Conclusion}
FIXME: The conclusion goes here.





%%%% if have a single appendix:
%%%%\appendix[Proof of the Zonklar Equations]
%%%% or
%%%%\appendix  % for no appendix heading
%%%% do not use \section anymore after \appendix, only \section*
%%%% is possibly needed
%%%
%%%
%%%% use appendices with more than one appendix
%%%% then use \section to start each appendix
%%%% you must declare a \section before using any
%%%% \subsection or using \label (\appendices by itself
%%%% starts a section numbered zero.)
%%%%
%%%
%%%
\appendices
%%%\section{Proof of the First Zonklar Equation}
%%%FIXME: Appendix one text goes here.
%%%
%%%% you can choose not to have a title for an appendix
%%%% if you want by leaving the argument blank
%%%\section{}
%%%FIXME: Appendix two text goes here.

\section{Discrete Fourier Transform inversion}
\label{appendix:discreteFourierInversion}

To find \( X_n \) evaluate the sum

\begin{equation}\label{eqn:ece1254projectReport:80}
\begin{aligned}
\sum_{k = -N}^N &x(t_k) e^{-j m \omega_0 t_k} \\
&=
\sum_{k = -N}^N 
\lr{
\sum_{n = -N}^N X_n e^{ j n \omega_0 t_k}
}
e^{-j m \omega_0 t_k} \\
&=
\sum_{n = -N}^N X_n 
\sum_{k = -N}^N 
e^{ j (n -m )\omega_0 t_k}
\end{aligned}
\end{equation}

This interior sum has the value \( 2 N + 1 \) when \( n = m \).  For \( n \ne m \), and 
\( a = e^{j (n -m ) \frac{2 \pi}{2 N + 1}} \), this is

\begin{dmath}\label{eqn:ece1254projectReport:100}
\sum_{k = -N}^N 
e^{ j (n -m )\omega_0 t_k}
=
\sum_{k = -N}^N 
e^{ j (n -m )\omega_0 \frac{T k}{2 N + 1}}
=
\sum_{k = -N}^N a^k
=
a^{-N} \sum_{k = -N}^N a^{k+ N}
=
a^{-N} \sum_{r = 0}^{2 N} a^{r}
=
a^{-N} \frac{a^{2 N + 1} - 1}{a - 1}.
\end{dmath}

Since \( a^{2 N + 1} = e^{2 \pi j (n - m)} = 1 \), this sum is zero when \( n \ne m \).  This means that

\begin{equation}\label{eqn:ece1254projectReport:120}
\sum_{k = -N}^N 
e^{ j (n -m )\omega_0 t_k} = (2 N + 1) \delta_{n,m}.
\end{equation}

Substitution back into \cref{eqn:ece1254projectReport:80} proves the Fourier inversion relation \cref{eqn:ece1254projectReport:60}.

\section{Matlab sources}
\label{appendix:matlab}

The Matlab code for this work is available under the ece1254/proj path in the github repository:

\begin{center}
git@github.com:peeterjoot/matlab.git
\end{center}

Relevant matlab scripts include

%Under:
%
%\begin{itemize}
%\item 
%ece1254/HarmonicBalanceBigF
%%\href{https://github.com/peeterjoot/matlab/tree/master/ece1254/HarmonicBalanceBigF}{ece1254/HarmonicBalanceBigF}
%\item 
%%\href{https://github.com/peeterjoot/matlab/tree/master/ece1254/HarmonicBalanceSmallF}{ece1254/HarmonicBalanceSmallF}
%ece1254/HarmonicBalanceSmallF
%\end{itemize}

%The data and figures referenced in these notes were generated with versions not greater than:
%
%FIXME:
%\begin{itemize}
%\item commit a13bf497c76a9bcac02ac918cc6d906326e543a6
%\end{itemize}

\input{projmatlab}

%\section{next appendix placeholder}

% use section* for acknowledgement
\ifCLASSOPTIONcompsoc
  % The Computer Society usually uses the plural form
  \section*{Acknowledgments}
\else
  % regular IEEE prefers the singular form
  \section*{Acknowledgment}
\fi


The authors would like to thank Prof.\ Piero Triverio for teaching this course and for valuable discussions of the material.

% Can use something like this to put references on a page
% by themselves when using endfloat and the captionsoff option.
\ifCLASSOPTIONcaptionsoff
  \newpage
\fi



% trigger a \newpage just before the given reference
% number - used to balance the columns on the last page
% adjust value as needed - may need to be readjusted if
% the document is modified later
%\IEEEtriggeratref{8}
% The "triggered" command can be changed if desired:
%\IEEEtriggercmd{\enlargethispage{-5in}}

% references section

% can use a bibliography generated by BibTeX as a .bbl file
% BibTeX documentation can be easily obtained at:
% http://www.ctan.org/tex-archive/biblio/bibtex/contrib/doc/
% The IEEEtran BibTeX style support page is at:
% http://www.michaelshell.org/tex/ieeetran/bibtex/
%\bibliographystyle{IEEEtran}
% argument is your BibTeX string definitions and bibliography database(s)
%\bibliography{IEEEabrv,../bib/paper}
%
% <OR> manually copy in the resultant .bbl file
% set second argument of \begin to the number of references
% (used to reserve space for the reference number labels box)
%\begin{thebibliography}{1}
%
%\bibitem{IEEEhowto:kopka}
%H.~Kopka and P.~W. Daly, \emph{A Guide to {\LaTeX}}, 3rd~ed.\hskip 1em plus
%  0.5em minus 0.4em\relax Harlow, England: Addison-Wesley, 1999.
%
%\end{thebibliography}

\bibliographystyle{plainnat}
\label{app:bibliography}
\bibliography{Bibliography}

% biography section
% 
% If you have an EPS/PDF photo (graphicx package needed) extra braces are
% needed around the contents of the optional argument to biography to prevent
% the LaTeX parser from getting confused when it sees the complicated
% \includegraphics command within an optional argument. (You could create
% your own custom macro containing the \includegraphics command to make things
% simpler here.)
%\begin{IEEEbiography}[{\includegraphics[width=1in,height=1.25in,clip,keepaspectratio]{mshell}}]{Michael Shell}
% or if you just want to reserve a space for a photo:

%\begin{IEEEbiography}{Michael Shell}
%Biography text here.
%\end{IEEEbiography}

\begin{IEEEbiography}[{\includegraphics[width=1in,height=1.25in,clip,keepaspectratio]{../../figures/ece1254/MikeRoyle}}]{Michael Royle}
M.A.Sc. Student, Electromagnetics Group, University of Toronto. B.Eng., Memorial University, '13. 
\end{IEEEbiography}

\begin{IEEEbiography}[{\includegraphics[width=1in,height=1.25in,clip,keepaspectratio]{../../figures/ece1254/peeter}}]{Peeter Joot}
Systems programmer for IBM (DB2 product).  M.Eng. (part time), Electromagnetics, University of Toronto. B.A.Sc. Engineering Science (Computer), UofT '97. 
\end{IEEEbiography}

% insert where needed to balance the two columns on the last page with
% biographies
%\newpage

%\begin{IEEEbiographynophoto}{Jane Doe}
%Biography text here.
%\end{IEEEbiographynophoto}

% You can push biographies down or up by placing
% a \vfill before or after them. The appropriate
% use of \vfill depends on what kind of text is
% on the last page and whether or not the columns
% are being equalized.

%\vfill

% Can be used to pull up biographies so that the bottom of the last one
% is flush with the other column.
%\enlargethispage{-5in}

% that's all folks
\end{document}
