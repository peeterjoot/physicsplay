%
% Copyright � 2014 Peeter Joot.  All Rights Reserved.
% Licenced as described in the file LICENSE under the root directory of this GIT repository.
%

\section{Summary of factorization costs}

\index{LU decomposition}
\paragraph{LU (dense)}

\begin{itemize}
   \item cost: \( O(n^3) \)
   \item cost depends only on size
\end{itemize}

\paragraph{LU (sparse)}

\begin{itemize}
   \item cost: Diagonal and tridiagonal are \( O(n) \), but can be up to \( O(n^3) \) depending on sparsity and the method of dealing with the sparsity.
   \item cost depends on size and sparsity
\end{itemize}

Computation can be affordable up to a few million elements.

\paragraph{Iterative methods}

Can be cheap if done right.  Convergence requires careful preconditioning.

\section{Iterative methods}

Given an initial guess of \( \Bx_0 \), any iterative methods are generally of the form

\makealgorithm{Iterative methods.}{alg:multiphysicsL7B:1}{
\REPEAT
\STATE \(\Br = \Bb - M \Bx_i\)
\UNTIL{\(\Norm{\Br} < \epsilon \).}
}

The difference \( \Br \) is called the residual.  For as long as it is bigger than desired, continue improving the estimate \( \Bx_i \).

The matrix vector product \( M \Bx_i \), if dense, is of \( O(n^2) \).  Suppose, for example, that the solution can be found in ten iterations.  If the matrix is dense, performance can be of \( 10 \, O(n^2) \).  If sparse, this can be worse than just direct computation.

\section{Gradient method}

This is a method for iterative solution of the equation \( M \Bx = \Bb \).

This requires symmetric positive definite matrix \( M = M^\T \), with \( M > 0 \), for which an \textAndIndex{energy function} is defined

\begin{equation}\label{eqn:multiphysicsL7:60}
\Psi(\By) \equiv \inv{2} \By^\T M \By - \By^\T \Bb
\end{equation}

For a two variable system this is illustrated in \cref{fig:lecture7:lecture7Fig1}.

\imageFigure{../../figures/ece1254/lecture7Fig1}{Positive definite energy function.}{fig:lecture7:lecture7Fig1}{0.3}

\index{energy function}
\maketheorem{Energy function minimum.}{thm:multiphysicsL7:460}{
The energy function \cref{eqn:multiphysicsL7:60} has a minimum at

\begin{equation}\label{eqn:multiphysicsL7:80}
\By = M^{-1} \Bb = \Bx.
\end{equation}

To prove this, consider the coordinate representation

\begin{equation}\label{eqn:multiphysicsL7:480}
\Psi = \inv{2} y_a M_{ab} y_b - y_b b_b,
\end{equation}

for which the derivatives are
\begin{dmath}\label{eqn:multiphysicsL7:500}
\PD{y_i}{\Psi} = 
\inv{2} M_{ib} y_b 
+
\inv{2} y_a M_{ai} 
- b_i
=
\lr{ M \By - \Bb }_i.
\end{dmath}

The last operation above was possible because \( M = M^\T \).  Setting all of these equal to zero, and rewriting this as a matrix relation gives

\begin{equation}\label{eqn:multiphysicsL7:520}
M \By = \Bb,
\end{equation}

as asserted.
}

This is called the gradient method because the gradient moves a point along the path of steepest descent towards the minimum if it exists.

The method is

\begin{equation}\label{eqn:multiphysicsL7:100}
\Bx^{(k+1)} = \Bx^{(k)} + 
\mathLabelBox
{ \alpha_k }{step size}
\mathLabelBox
[
   labelstyle={below of=m\themathLableNode, below of=m\themathLableNode}
]
{
\Bd^{(k)}
}{direction},
\end{equation}

where the direction is

\begin{dmath}\label{eqn:multiphysicsL7:120}
\Bd^{(k)} = - \spacegrad \Phi = \Bb - M \Bx^{(k)} = r^{(k)}.
\end{dmath}

\paragraph{Optimal step size}
\index{step size}

The next iteration expansion of the energy function \( \Phi \lr{ \Bx^{(k+1)} } \) is

\begin{dmath}\label{eqn:multiphysicsL7:140}
\Phi \lr{ \Bx^{(k+1)} }
= \Phi\lr{ \Bx^{(k)} + \alpha_k \Bd^{(k)} }
= 
\inv{2} 
\lr{ \Bx^{(k)} + \alpha_k \Bd^{(k)} }^\T
M
\lr{ \Bx^{(k)} + \alpha_k \Bd^{(k)} }
-
\lr{ \Bx^{(k)} + \alpha_k \Bd^{(k)} }^\T \Bb.
\end{dmath}

To find the minimum, the derivative of both sides with respect to \( \alpha_k \) is required

\begin{equation}\label{eqn:multiphysicsL7:540}
0 = 
\inv{2} 
\lr{ \Bd^{(k)} }^\T
M
\Bx^{(k)}
+
\inv{2} 
\lr{ \Bx^{(k)} }^\T
M
\Bd^{(k)}
+
\alpha_k \lr{ \Bd^{(k)} }^\T
M
\Bd^{(k)} 
-
\lr{ \Bd^{(k)} }^\T \Bb.
\end{equation}

Because \( M \) is symmetric, this is

\begin{equation}\label{eqn:multiphysicsL7:560}
\alpha_k \lr{ \Bd^{(k)} }^\T
M
\Bd^{(k)} 
=
\lr{ \Bd^{(k)} }^\T \lr{ \Bb - M \Bx^{(k)}}
=
\lr{ \Bd^{(k)} }^\T r^{(k)},
\end{equation}

or

\begin{equation}\label{eqn:multiphysicsL7:160}
\alpha_k 
= \frac{
\lr{\Br^{(k)}}^\T 
\Br^{(k)}
}{
\lr{\Br^{(k)}}^\T 
M
\Br^{(k)}
}
\end{equation}

This this method is not optimal when a new direction is picked each step of the iteration.

%\EndArticle
%\EndNoBibArticle
