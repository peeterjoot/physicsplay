%
% Copyright � 2014 Peeter Joot.  All Rights Reserved.
% Licenced as described in the file LICENSE under the root directory of this GIT repository.
%
%\newcommand{\authorname}{Peeter Joot}
\newcommand{\email}{peeterjoot@protonmail.com}
\newcommand{\basename}{FIXMEbasenameUndefined}
\newcommand{\dirname}{notes/FIXMEdirnameUndefined/}

%\renewcommand{\basename}{multiphysicsL17}
%\renewcommand{\dirname}{notes/ece1254/}
%\newcommand{\keywords}{ECE1254H}
%\newcommand{\authorname}{Peeter Joot}
\newcommand{\onlineurl}{http://sites.google.com/site/peeterjoot2/math2013/\basename.pdf}
\newcommand{\sourcepath}{\dirname\basename.tex}
\newcommand{\generatetitle}[1]{\chapter{#1}}

\newcommand{\vcsinfo}{%
\section*{}
\noindent{\color{DarkOliveGreen}{\rule{\linewidth}{0.1mm}}}
\paragraph{Document version}
%\paragraph{\color{Maroon}{Document version}}
{
\small
\begin{itemize}
\item Available online at:\\ 
\href{\onlineurl}{\onlineurl}
\item Git Repository: \input{./.revinfo/gitRepo.tex}
\item Source: \sourcepath
\item last commit: \input{./.revinfo/gitCommitString.tex}
\item commit date: \input{./.revinfo/gitCommitDate.tex}
\end{itemize}
}
}

%\PassOptionsToPackage{dvipsnames,svgnames}{xcolor}
\PassOptionsToPackage{square,numbers}{natbib}
\documentclass{scrreprt}

\usepackage[left=2cm,right=2cm]{geometry}
\usepackage[svgnames]{xcolor}
\usepackage{peeters_layout}

\usepackage{natbib}

\usepackage[
colorlinks=true,
bookmarks=false,
pdfauthor={\authorname, \email},
backref 
]{hyperref}

% http://tex.stackexchange.com/questions/75773/how-to-reference-problems-by-the-text-label-in-an-exercise-envioronment
\usepackage[english]{cleveref}
\crefname{Exercise}{exercise}{exercises}
\Crefname{Exercise}{Exercise}{Exercises}

\RequirePackage{titlesec}
\RequirePackage{ifthen}

% http://stackoverflow.com/questions/4932910/date-in-the-tabular-environment
\makeatletter
\let\insertdate\@date
\makeatother

\titleformat{\chapter}[display]
{\bfseries\Large}
{\color{DarkSlateGrey}\filleft \authorname
\ifthenelse{\isundefined{\studentnumber}}{}{\\ \studentnumber}
\ifthenelse{\isundefined{\email}}{}{\\ \email}
\ifthenelse{\isundefined{\dateintitle}}{}{\\ \insertdate}
%\ifthenelse{\isundefined{\coursename}}{}{\\ \coursename} % put in title instead.
}
{4ex}
{\color{DarkOliveGreen}{\titlerule}\color{Maroon}
\vspace{2ex}%
\filright}
[\vspace{2ex}%
\color{DarkOliveGreen}\titlerule
]

\newcommand{\beginArtWithToc}[0]{\begin{document}\tableofcontents}
\newcommand{\beginArtNoToc}[0]{\begin{document}}
\newcommand{\EndNoBibArticle}[0]{\end{document}}
\newcommand{\EndArticle}[0]{\bibliography{Bibliography}\bibliographystyle{plainnat}\end{document}}

% 
%\newcommand{\citep}[1]{\cite{#1}}

\colorSectionsForArticle


%
%
%\beginArtNoToc
%\generatetitle{ECE1254H Modeling of Multiphysics Systems.  Lecture 17: Stability.  Taught by Prof.\ Piero Triverio}
%%\chapter{Stability}
%\label{chap:multiphysicsL17}
%
%\section{Disclaimer}
%
%Peeter's lecture notes from class.  These may be incoherent and rough.
%
\section{Stability (continued)}
\index{stability}

Continuing with the simple continuous time test system

\begin{equation}\label{eqn:multiphysicsL17:20}
\dot{x}(t) = \lambda x(t)
\end{equation}

With the application of a trial solution \( x(t) = e^{s t} \), the resulting characteristic equation is

\begin{equation}\label{eqn:multiphysicsL17:40}
s = \lambda,
\end{equation}

and stability follows, provided that \( \Real(\lambda) < 0 \) as sketched in \cref{fig:lecture17:lecture17Fig1}.

\imageFigure{../../figures/ece1254/lecture17Fig1}{Stability region.}{fig:lecture17:lecture17Fig1}{0.2}

Utilizing LMS methods, for example TR, the discrete time system results, such as

\begin{equation}\label{eqn:multiphysicsL17:60}
\gamma_{-1} x_{n+1} +
\gamma_{0} x_{n} +
\gamma_1 x_{n-1}
+ \cdots
= 0.
\end{equation}

% n+1 = k
% n = k-1
% n + j = k - 1 + j
With a substitution \( x_{n + j} = z^{k - 1 + j} \), a characteristic polynomial results

\begin{equation}\label{eqn:multiphysicsL17:80}
\gamma_{-1} z^{k} +
\gamma_{0} z^{k-1} +
\gamma_1 z^{k-2} + \cdots = 0.
\end{equation}

This is stable provided \( \Abs{z_l} < 1 \), as illustrated in \cref{fig:lecture17:lecture17Fig2}.

\imageFigure{../../figures/ece1254/lecture17Fig2}{Stability region in z-domain.}{fig:lecture17:lecture17Fig2}{0.3}

... SWITCHED TO SLIDES.

\makedefinition{Stiff.}{dfn:multiphysicsL17:100}{
A \textAndIndex{stiff} system is one that has multiple timescales, as characterized by a significant range of eigenvalues.
}

\makedefinition{Lossless system.}{dfn:multiphysicsL17:120}{
Lossless \index{pole!lossless} poles are those that reside strictly on the imaginary axis.
}

\makedefinition{Lossly system.}{dfn:multiphysicsL17:140}{
Lossly poles \index{pole!lossly} are those that reside strictly to the left of the imaginary axis.
}

\makedefinition{Active system.}{dfn:multiphysicsL17:160}{
Active poles \index{pole!active} are those that reside strictly to the right of the imaginary axis.
}

\makedefinition{A-stable.}{dfn:multiphysicsL17:170}{
See slides.
}

\index{Dahlquist's theorem}
\maketheorem{Dahlquist's theorem.}{thm:multiphysicsL17:180}{
There are no LMS methods of order greater than 2 that are A-stable.  Also known as the Dahlquist barrier.  The TR method has the lowest error of the A-stable LMS methods.
}

\paragraph{Some recent developments}
\index{backward differentiation formula}

The backward differentiation formulas (BDF) are of \( \text{order} > 2 \).  They are not A-stable, but can be sufficient with stability sketched in \cref{fig:lecture17:lecture17Fig3}.

\imageFigure{../../figures/ece1254/lecture17Fig3}{BDF stability region.}{fig:lecture17:lecture17Fig3}{0.2}

Also available are the Obreshkov formulas \citep{butcherOrder}, which are both A-stable and of \( \text{order} > 2 \).  There is a cost to both of these methods: computation of the derivatives at each step.

