% like \sum for product (don't use \Pi_{}^{})
\prod_{}^{}

\begin{center}
\end{center}

\underlineAndIndex{}

\imageFigure{figures/XX}{CC}{fig:WHAT:NN}{0.3}

\captionedTable{caption}{tab:WHAT:n}{
\begin{tabular}{|l|l|l|}...\end{tabular}
}

\left(\right)
%---------------------------------------------------------------------------------
% an underbrace replacement that uses a tikz color box.
% math, label
% FIXME: pull the alternate placement examples from phy485 to cut and paste from
% (would be better to automate, but not sure how).
\mathLabelBox
{
}
{}

% Examples:
%eqn:modernOpticsLecture17:160
% straight arrow up:
\mathLabelBox
[
   labelstyle={yshift=1.2em},
   linestyle={}
]
{
}{}
% straight label down.
\mathLabelBox
[
   labelstyle={below of=m\themathLableNode, below of=m\themathLableNode}
]
{
}
{}
% curvey arrow over to the right and up a bit.
\mathLabelBox
[
   labelstyle={xshift=2cm},
   linestyle={out=270,in=90, latex-}
]
{
}
{}

% curvey arrow over to the right and up a bit (more)
\mathLabelBox
[
   labelstyle={xshift=2cm, yshift=0.5cm},
   linestyle={out=270,in=90, latex-}
]
{
}
{}

%---------------------------------------------------------------------------------

\makeproblem{description}{pr:WHAT:n}{ } % makeproblem
\makeoproblem{description}{pr:WHAT:n}{\citep{
} pr X.Y}{
} % makeoproblem
\makeanswer{pr:WHAT:n}{ } % makeanswer

\makesubproblem{XXX}{pr:WHAT:n:a}
\makeSubAnswer{XXX}{pr:WHAT:n:a}
\makesubproblem{XXX}{pr:WHAT:n:b}
\makeSubAnswer{XXX}{pr:WHAT:n:b}

\begin{align*}
\end{align*}
\begin{equation*}
\end{equation*}
\eqnref{eqn:WHAT:4.}
\begin{align}\label{eqn:WHAT:n}
\end{align}
\paragraph{}
Schr\"{o}dinger
\begin{equation}\label{eqn:WHAT:n}
\end{equation}

\begin{verbatim}
\end{verbatim}
\begin{bmatrix}
\end{bmatrix}
\begin{vmatrix}
\end{vmatrix}
\begin{subequations}
\end{subequations}
\begin{aligned}
\end{aligned}
x =

\left\{
\begin{array}{l l}
X & \quad \mbox{if \( \) } \\
Y & \quad \mbox{if \( \) }
\end{array}
\right.

\href{ }{ }
\begin{itemize}
\item
\end{itemize}
\begin{enumerate}
\item
\end{enumerate}

\makeaside{blah blah}{aside:WHAT:n}{
}

\maketheorem{foo's identity}{thm:WHAT:n}{
}

\makequestion{why?}{question:WHAT:n}{
}

\makelemma{foo's identity}{thm:WHAT:n}{
}

\boxedEquation{eqn:WHAT:n}{
}

% title, label, text
\makeexample{title}{example:WHAT:n}{
}

\makedefinition{blah}{dfn:WHAT:n}{
}

% \usepackage{algorithmic}
\makealgorithm{blah}{alg:WHAT:n}{
\begin{algorithmic}
\end{algorithmic}
}

% \questionEquals
%\stackrel{?}{=}

\begin{Exercise}[title={}, label={problem:WHAT:xxx}]
\end{Exercise}

\begin{Answer}[ref={problem:WHAT:xxx}]
\end{Answer}

%------------------------------------------------------------------------------------
% http://tex.stackexchange.com/questions/21290/how-to-make-left-right-pairs-of-delimiter-work-over-multiple-lines
% breqn package supports \left \right over multiple lines.
% dmath ~ align
% dgroup ~ subequations (containing dmath or dmath*'s)
% ... other good stuff in breqn (like a way to automatically do the \mbox{} type stuff for conditions
%
\begin{dmath*}
\end{dmath*}
\begin{dmath}\label{eqn:WHAT:n}
\end{dmath}
\begin{dgroup*}
\end{dgroup*}
\begin{dgroup}
\end{dgroup}

\mytikzmark{left}{1}
\mytikzmark{right}{1}
\DrawMyBox[thick, Maroon, rounded corners]{1}
