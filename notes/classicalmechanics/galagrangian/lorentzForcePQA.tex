%
% Copyright � 2012 Peeter Joot.  All Rights Reserved.
% Licenced as described in the file LICENSE under the root directory of this GIT repository.
%

%
%
\chapter{Lorentz force Lagrangian with conjugate momentum}
\index{conjugate momentum}
\label{chap:lorentzForcePQA}
%\date{April 15, 2009.  lorentzForcePQA.tex}

\section{Motivation}

The covariant Lorentz force Lagrangian (for metric \(+---\))

\begin{equation}\label{eqn:lorentzForcePQA:20}
\begin{aligned}
\LL &= \inv{2} m v^2 + q A \cdot (v/c)
\end{aligned}
\end{equation}

Can be used to find the Lorentz force equation (here in four vector form)
\index{Lorentz force equation}

\begin{equation}\label{eqn:lorentzForcePQA:40}
\begin{aligned}
m\vdot &= q F \cdot (v/c)
\end{aligned}
\end{equation}

A derivation of this can be found in \chapcite{PJSrLorentzForce}.

However, in
\citep{pauli2000wm} the Lorentz force equation (in non-covariant form) is
derived as a limiting classical case via calculation of the expectation
value of the Hamiltonian

\begin{equation}\label{eqn:lorForcePqA:interactionConjMomHamiltonian}
\begin{aligned}
H = \inv{2m}\sum_{k=1}^3 \left(p_k - \frac{e}{c} A_k \right)^2 + V
\end{aligned}
\end{equation}

This has a much different looking structure than \(\LL\) above, so reconciliation of the two
is justifiable.

\section{Lorentz force Lagrangian with conjugate momentum}

Is there a relativistic form for the interaction Lagrangian with a structure similar to \eqnref{eqn:lorForcePqA:interactionConjMomHamiltonian}?

Let us try

\begin{equation}\label{eqn:lorentzForcePQA:60}
\begin{aligned}
\LL
&= \inv{2 m}\left( m v - \kappa A \right)^2 \\
&= \inv{2 m}\left( m^2 v^2 - 2 m \kappa A \cdot v + \kappa^2 A^2 \right)^2 \\
&= \inv{2 m} \left(
m^2 \xdot^\alpha \xdot_\alpha
- 2 m \kappa A_\alpha \xdot^\alpha
+ \kappa^2 A_\alpha A^\alpha
\right) \\
\end{aligned}
\end{equation}

where \(\kappa\) is to be determined.

For this Lagrangian the Euler-Lagrange calculation for variation of \(S = \int d^4 x \LL\) is

\begin{equation}\label{eqn:lorentzForcePQA:80}
\begin{aligned}
\PD{x^\mu}{\LL}
&= - \kappa (\partial_\mu A_\alpha) \xdot^\alpha + \inv{m} \kappa^2 (\partial_\mu A_\alpha) A^\alpha \\
\frac{d}{d\tau} \PD{\xdot^\mu}{\LL}
&= \frac{d}{d\tau} \left(m \xdot_\mu - \kappa A_\mu \right)
\end{aligned}
\end{equation}

Assembling and shuffling we have
\begin{equation}\label{eqn:lorentzForcePQA:100}
\begin{aligned}
m \xddot_\mu
&= \kappa (\partial_\alpha A_\mu) \xdot^\alpha
- \kappa (\partial_\mu A_\alpha) \xdot^\alpha + \inv{m} \kappa^2 (\partial_\mu A_\alpha) A^\alpha \\
&= \kappa (\partial_\alpha A_\mu - \partial_\mu A_\alpha) \xdot^\alpha + \inv{m} \kappa^2 (\partial_\mu A_\alpha) A^\alpha \\
&= \kappa F_{\alpha\mu} \xdot^\alpha + \inv{m} \kappa^2 (\partial_\mu A_\alpha) A^\alpha \\
\end{aligned}
\end{equation}

Comparing to the Lorentz force equation (again for a \(+---\) metric)

\begin{equation}\label{eqn:lorentzForcePQA:120}
\begin{aligned}
m v_\mu = \frac{q}{c} F_{\mu\nu} v^\nu
\end{aligned}
\end{equation}

We see that we need \(\kappa = -q/c\), but we have an extra factor that does not look familiar.  In vector form this Lagrangian would give us the
equation of motion

\begin{equation}\label{eqn:lorentzForcePQA:140}
\begin{aligned}
m v = \frac{q}{c} F \cdot v + \frac{q^2}{m c^2} (\grad A_\mu) A^\mu
\end{aligned}
\end{equation}

Assuming that this extra term has no place in the Lorentz force equation we need to adjust the original Lagrangian as follows

\begin{equation}\label{eqn:lorForcePqA:interactionLagPsq}
\begin{aligned}
\LL
&= \inv{2 m}\left( m v + \frac{q}{c} A \right)^2 - \frac{q^2}{ 2 m c^2} A^2
\end{aligned}
\end{equation}

Expressing this energy density in terms of the canonical momentum is
somewhat interesting.  It provides some extra structure, allowing for a loose identification of the two terms as

\begin{equation}\label{eqn:lorentzForcePQA:160}
\begin{aligned}
\LL &= K - V
\end{aligned}
\end{equation}

(ie: \(K = p^2/2m\), where \(p\) is the sum of the (proper) mechanical momentum and electromagnetic momentum).

However, that said, observe that expanding the square gives

\begin{equation}\label{eqn:lorentzForcePQA:180}
\begin{aligned}
\LL &= \inv{2} m v^2 + \frac{q}{c} A \cdot v
\end{aligned}
\end{equation}

which is exactly the original Lorentz force Lagrangian, so in the end this works out to only differ from the original cosmetically.

\section{On terminology.  The use of the term conjugate momentum}

\citep{goldstein1951cm} uses the term conjugate momentum in reference to a specific coordinate.  For example in

\begin{equation}\label{eqn:lorentzForcePQA:200}
\begin{aligned}
\LL(\rho, \rhodot) = f(\rho, \rhodot)
\end{aligned}
\end{equation}

the value
\begin{equation}\label{eqn:lorentzForcePQA:220}
\begin{aligned}
\PD{\rhodot}{f}
\end{aligned}
\end{equation}

is the momentum canonically conjugate to \(\rho\).  Above I have called the vector quantity \(m v + q A/c = (m v^\mu + q A^\mu/c) \gamma_\mu\) the canonical momentum.  My justification for doing so comes from a vectorization of the Euler-Lagrange equations.

Equating all the variational derivatives to zero separately

\begin{equation}\label{eqn:lorentzForcePQA:240}
\begin{aligned}
\frac{\delta \LL}{\delta x^\mu} = \PD{x^\mu}{\LL} - \frac{d}{d\tau} \PD{\xdot^\mu}{\LL} = 0
\end{aligned}
\end{equation}

can be replaced by an equivalent vector equation (note that summation is now implied)

\begin{equation}\label{eqn:lorentzForcePQA:260}
\begin{aligned}
\gamma^\mu \frac{\delta \LL}{\delta x^\mu} = \gamma^\mu \PD{x^\mu}{\LL} - \frac{d}{d\tau} \gamma^\mu \PD{\xdot^\mu}{\LL} = 0
\end{aligned}
\end{equation}

This has two distinct vector operations, a spacetime gradient, and a spacetime ``velocity gradient'', and it is not terribly abusive
of notation to write

\begin{equation}\label{eqn:lorentzForcePQA:280}
\begin{aligned}
\grad &= \gamma^\mu \PD{x^\mu}{} \\
\grad_v &= \gamma^\mu \PD{\xdot^\mu}{}
\end{aligned}
\end{equation}

with which all the Euler Lagrange equations can be summarized as

\begin{equation}\label{eqn:lorentzForcePQA:300}
\begin{aligned}
\grad \LL = \frac{d}{d\tau} \grad_v \LL
\end{aligned}
\end{equation}

It is thus natural, in a vector context, to name the quantity \(\grad_v \LL\), the canonical momentum.  It is a vectorized representation of all the individual momenta that are canonically conjugate to the respective coordinates.

This vectorization is really only valid when the basis vectors are fixed (they do not have to be orthonormal as the use of the reciprocal basis here highlights).  In a curvilinear system where the vectors vary with position, one cannot necessarily pull the \(\gamma^\mu\) into the \(d/d\tau\) derivative.

%\subsection{Performing the Euler-Lagrange calculation in the vector representation}
