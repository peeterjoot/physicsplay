%
% Copyright � 2012 Peeter Joot.  All Rights Reserved.
% Licenced as described in the file LICENSE under the root directory of this GIT repository.
%

% 
% 
\chapter{Lorentz Invariance of Maxwell Lagrangian}
\label{chap:PJBoostMaxwell}
\label{chap:boostMaxwellLagrangian}
%\date{October 19, 2008.  boostMaxwellLagrangian.tex}

\section{Working in multivector form}
\subsection{Application of Lorentz boost to the field Lagrangian}

The multivector form of the field Lagrangian is

\begin{align}\label{eqn:boostMLag:lagrangian}
\LL &= \kappa (\grad \wedge A)^2 + A \cdot J \\
\kappa &= -\frac{\epsilon_0 c}{2}
\end{align}

Write the boosting transformation on a four vector in exponential form

\begin{align*}
L(X) = \exp( \alpha \acap/2 ) X \exp( -\alpha \acap/2 ) = \Lambda X \Lambda^\dagger
\end{align*}

where $\acap = a^i \gamma_i \wedge \gamma_0$ is any unit spacetime bivector, and $\alpha$ represents the rapidity angle.

Consider first the transformation of the interaction term with $A' = L(A)$, and $J' = L(J)$

\begin{align*}
A' \cdot J'
&= \gpgradezero{L(A) L(J)} \\
&= \gpgradezero{ \Lambda A \Lambda^\dagger \Lambda J \Lambda^\dagger } \\
&= \gpgradezero{ \Lambda A J \Lambda^\dagger } \\
&= \gpgradezero{ \Lambda^\dagger \Lambda A J } \\
&= \gpgradezero{ A J } \\
&= A \cdot J \\
\end{align*}

Now consider the boost applied to the field bivector $F = \BE + Ic\BB = \grad \wedge A$, by boosting both the gradient and the potential

\begin{align*}
\grad' \wedge A'
&= L(\grad) \wedge L(A) \\
&= \Lambda \grad) \wedge L(A) \\
&= (\Lambda \grad \Lambda^\dagger ) \wedge (\Lambda A \Lambda^\dagger) \\
&= \inv{2}\left( (\Lambda \grad \Lambda^\dagger )  (\Lambda A \Lambda^\dagger) - (\Lambda A \Lambda^\dagger )  (\Lambda \grad \Lambda^\dagger) \right) \\
&= \inv{2}\left( \Lambda \grad A \Lambda^\dagger - \Lambda A \grad \Lambda^\dagger \right) \\
&= \Lambda (\grad \wedge A) \Lambda^\dagger \\
\end{align*}

The boosted squared field bivector in the Lagrangian is thus

\begin{align*}
(\grad' \wedge A' )^2
&= \Lambda (\grad \wedge A)^2 \Lambda^\dagger \\
&= \Lambda (\BE + Ic\BB)^2 \Lambda^\dagger \\
&= \Lambda ( (\BE^2 - c^2\BB^2) + 2 I c \BE \cdot \BB ) \Lambda^\dagger \\
&= ( (\BE^2 - c^2\BB^2) \Lambda \Lambda^\dagger + 2 (\Lambda I \Lambda^\dagger) c \BE \cdot \BB ) \\
&= ( (\BE^2 - c^2\BB^2) + 2 I \Lambda \Lambda^\dagger c \BE \cdot \BB ) \\
&= ( (\BE^2 - c^2\BB^2) + 2 I c \BE \cdot \BB ) \\
&= (\BE + Ic\BB)^2 \\
&= (\grad \wedge A )^2
\end{align*}

The commutation of the pseudoscalar $I$ with the boost exponential $\Lambda = \exp( \alpha \acap/2 ) = \cosh(\alpha/2) + \acap\sinh(\alpha/2)$ is possible
since $I$
anticommutes with all four vectors and thus commutes with bivectors, such as $\acap$.  $I$ also necessarily commutes with the scalar
components of this exponential, and thus commutes with any even grade multivector.

Putting all the pieces together this shows that the Lagrangian in its entirety is a Lorentz invariant

\begin{align*}
\LL' &= \kappa (\grad' \wedge A')^2 + A' \cdot J' = \kappa (\grad \wedge A)^2 + A \cdot J = \LL
\end{align*}

FIXME: what is the conserved quantity associated with this?  There should be one according to Noether's theorem?  Is it the gauge condition $\grad \cdot A = 0$?

\subsubsection{Maxwell equation invariance}

Somewhat related, having calculated the Lorentz transform of $F = \grad \wedge A$, is an aside showing that the Maxwell equation
is unsurprisingly also is a Lorentz invariant.

\begin{align*}
\grad' (\grad' \wedge A') &= J' \\
\Lambda \grad \Lambda^\dagger \Lambda (\grad \wedge A) \Lambda^\dagger &= \Lambda J \Lambda^\dagger \\
\Lambda \grad (\grad \wedge A) \Lambda^\dagger &= \Lambda J \Lambda^\dagger \\
\end{align*}

Pre and post multiplying with $\Lambda^\dagger$, and $\Lambda$ respectively returns the unboosted equation

\begin{align*}
\grad (\grad \wedge A) &= J
\end{align*}

\subsection{Lorentz boost applied to the Lorentz force Lagrangian}

Next interesting case is the Lorentz force, which for a time positive metric 
signature is:

\begin{align*}
\LL &= q A \cdot v/c + \inv{2} m v \cdot v
\end{align*}

The boost invariance of the $A \cdot J$ dot product demonstrated above demonstrates the general invariance property for any four vector dot product, and this
Lagrangian has nothing but dot products in it.  It thus follows directly that the Lorentz force Lagrangian is also a Lorentz invariant.

\section{Repeat in tensor form}

Now, I can follow the above, but presented with the same sort of calculation in tensor form I am hopeless to understand it.  To attempt translating this
into tensor form, it appears the first step is putting the Lorentz transform itself into tensor or matrix form.

\subsection{Translating versors to matrix form}

To get the feeling for how this will work, assume $\acap = \sigma_1$, so that the boost is along the x-axis.  In that case we have

\begin{align*}
L(X)
%&= (\cosh( \alpha/2 ) + \gamma_{10} \sinh( \alpha/2 )) ( x^0 \gamma_0 + x^1 \gamma_1 + x^2 \gamma_2 + x^3 \gamma_3 ) (\cosh( \alpha/2 ) + \gamma_{01} \sinh( \alpha/2 )) \\
&= (\cosh( \alpha/2 ) + \gamma_{10} \sinh( \alpha/2 )) x^\mu \gamma_\mu (\cosh( \alpha/2 ) + \gamma_{01} \sinh( \alpha/2 )) \\
\end{align*}

Writing $C = \cosh(\alpha/2)$, and $S = \sinh(\alpha/2)$, and observing that the exponentials commute with the $\gamma_2$, and $\gamma_3$ directions so the exponential
action on those directions cancel.

\begin{align*}
L(X)
&= x^2 \gamma_2 + x^3 \gamma_3 + (C + \gamma_{10} S) ( x^0 \gamma_0 + x^1 \gamma_1 ) (C + \gamma_{01} S) \\
\end{align*}

Expanding just the non-perpendicular parts of the above
\begin{align*}
&(C + \gamma_{10} S) ( x^0 \gamma_0 + x^1 \gamma_1 ) (C + \gamma_{01} S) \\
&=
x^0 (C^2 \gamma_0 + \gamma_{10001} S^2) + x^0 S C (\gamma_{001} + \gamma_{100})
+x^1 (C^2 \gamma_1 + \gamma_{10101} S^2) + x^1 S C (\gamma_{101} + \gamma_{101}) \\
&=
x^0 (C^2 \gamma_0 - \gamma_{01100} S^2) + 2 x^0 S C \gamma_{001} 
+x^1 (C^2 \gamma_1 - \gamma_{11001} S^2) - 2 x^1 S C \gamma_{011} \\
&= (x^0 \gamma_0 + x^1 \gamma_1) (C^2 + S^2) + 2 (\gamma_0)^2 S C (x^0 \gamma_{1} + x^1 \gamma_{0}) \\
&= (x^0 \gamma_0 + x^1 \gamma_1) \cosh(\alpha) + (\gamma_0)^2 \sinh(\alpha) (x^0 \gamma_{1} + x^1 \gamma_{0}) \\
&= 
\gamma_0 ( x^0 \cosh(\alpha) + x^1 \sinh((\gamma_0)^2 \alpha) )
+\gamma_1 ( x^1 \cosh(\alpha) + x^0 \sinh((\gamma_0)^2 \alpha) ) \\
\end{align*}


In matrix form the complete transformation is thus

\begin{align*}
{\begin{bmatrix}
x^0 \\
x^1 \\
x^2 \\
x^3 \\
\end{bmatrix}}'
&=
\begin{bmatrix}
\cosh(\alpha(\gamma_0)^2) & \sinh(\alpha(\gamma_0)^2) & 0 & 0 \\
\sinh(\alpha(\gamma_0)^2) & \cosh(\alpha(\gamma_0)^2) & 0 & 0 \\
0 & 0 & 1 & 0 \\
0 & 0 & 0 & 1 \\
\end{bmatrix}
\begin{bmatrix}
x^0 \\
x^1 \\
x^2 \\
x^3 \\
\end{bmatrix} \\
&=
\cosh(\alpha(\gamma_0)^2) 
\begin{bmatrix}
1                         & \tanh(\alpha(\gamma_0)^2) & 0 & 0 \\
\tanh(\alpha(\gamma_0)^2) & 1                         & 0 & 0 \\
0 & 0 & 1 & 0 \\
0 & 0 & 0 & 1 \\
\end{bmatrix}
\begin{bmatrix}
x^0 \\
x^1 \\
x^2 \\
x^3 \\
\end{bmatrix}
\end{align*}

This supplies the specific meaning for the $\alpha$ factor in the exponential form, namely:

\begin{align*}
\alpha
&= -\tanh^{-1}(\beta (\gamma_0)^2) \\
&= -\tanh^{-1}(\Abs{\Bv}/c (\gamma_0)^2)
\end{align*}

Or
\begin{align*}
\alpha \acap
&= -\tanh^{-1}(\acap \Abs{\Bv}/c (\gamma_0)^2) \\
&= -\tanh^{-1}(\Bv/c (\gamma_0)^2) \\
\end{align*}

Putting this back into the original Lorentz boost equation to tidy it up, and 
writing $\tanh(\BA) = \Bv/c$, the Lorentz boost is 

\begin{align*}
L(X) &= 
\left\{
\begin{array}{l l}
\exp(-\BA/2) X \exp(\BA/2) & \quad \mbox{for $(\gamma_0)^2 = 1$} \\
\exp(\BA/2) X \exp(-\BA/2) & \quad \mbox{for $(\gamma_0)^2 = -1$} \\
\end{array} \right. \\
\end{align*}

Both of the metric signature options are indicated here for future reference and comparison with results using the alternate signature.

\subsubsection{Revisit the expansion to matrix form above}

Looking back, multiplying out all the half angle terms as done above is this is the long dumb hard way to do it.
A more sensible way would be to note that $\exp(\alpha\acap/2)$ anticommutes with both $\gamma_0$ and $\gamma_1$ thus

\begin{align*}
\exp( \alpha \acap/2 ) (x^0\gamma_0 + x^1\gamma_1) \exp( -\alpha \acap/2 )
&= \exp( \alpha \acap ) (x^0\gamma_0 + x^1\gamma_1) \\
&= (\cosh( \alpha ) + \acap \sinh(\alpha)) (x^0\gamma_0 + x^1\gamma_1) \\
\end{align*}

The matrix form thus follows directly.

\section{Translating versors tensor form}

After this temporary digression back to the multivector form of the Lorentz transformation lets dispose of the specifics of the boost direction and magnitude, and also the metric
signature.  Instead encode all of these in a single versor variable $\Lambda$, again writing

\begin{align}
L(X) &= \Lambda X \Lambda^\dagger
\end{align}

\subsection{Expressing vector Lorentz transform in tensor form}

What is the general way to encode this linear transformation in tensor/matrix form?  The transformed vector is just that a vector, and thus can be written in terms of 
coordinates for some basis

\begin{align*}
L(X) 
&= (L(X) \cdot e^\mu) e_\mu \\
&= ((\Lambda (x^\nu \gamma_\nu) \Lambda^\dagger) \cdot e^\mu) e_\mu \\
&= x^\nu ((\Lambda \gamma_\nu \Lambda^\dagger) \cdot e^\mu) e_\mu \\
\end{align*}

The inner term is just the tensor that we want.  Write

\begin{align*}
{\Lambda_{\nu}}^{\mu} &= (\Lambda \gamma_\nu \Lambda^\dagger) \cdot e^\mu \\
{\Lambda^{\nu}}_{\mu} &= (\Lambda \gamma^\nu \Lambda^\dagger) \cdot e_\mu
\end{align*}

for 
\begin{align*}
L(X) &= x^\nu {\Lambda_{\nu}}^{\mu} e_\mu \\
     &= x_\nu {\Lambda^{\nu}}_{\mu} e^\mu \\
\end{align*}

Completely eliminating the basis, working in just the coordinates $X = {x'}^\mu e_\mu = {x'}_\mu e^\mu$ this is

\begin{align}
{x'}^\mu &= x^\nu {\Lambda_{\nu}}^{\mu} \\
{x'}_\mu &= x_\nu {\Lambda^{\nu}}_{\mu}
\end{align}

Now, in particular, having observed that the dot product is a Lorentz invariant this should supply the index manipulation rule for operating
with the Lorentz boost tensor in a dot product context.

Write
\begin{align*}
L(X) \cdot L(Y)
&= (x^\nu {\Lambda_{\nu}}^{\mu} e_\mu) \cdot (y_\alpha {\Lambda^{\alpha}}_{\beta} e^\beta) \\
&= x^\nu y_\alpha {\Lambda_{\nu}}^{\mu} {\Lambda^{\alpha}}_{\beta} e_\mu \cdot e^\beta \\
&= x^\nu y_\alpha {\Lambda_{\nu}}^{\mu} {\Lambda^{\alpha}}_{\mu} \\
\end{align*}

Since this equals $x^\nu y_\nu$, the tensor rule must therefore be
\begin{align}\label{eqn:boostMLag:boostinverse}
{\Lambda_{\mu}}^{\sigma} {\Lambda^{\nu}}_{\sigma} = {\delta_\mu}^\nu
\end{align}

After a somewhat long path, the core idea behind the Lorentz boost tensor
is that it is the ``matrix'' of a linear transformation that leaves
the four vector dot product unchanged.  There is no need to consider
any Clifford algebra formulations to express just that idea.

\subsection{Misc notes}

FIXME: To complete
the expression of this in tensor form enumerating exactly how to express
the dot product in tensor form would also be reasonable.  ie: how to compute
the reciprocal coordinates without describing the basis.  Doing this
will introduce the metric tensor into the mix.

Looks like the result \eqnref{eqn:boostMLag:boostinverse}
is consistent with \citep{MinahanTensors} and
that doc starts making a bit more sense now.  I do see that
he uses primes to distinguish the boost tensor from its inverse (using
the inverse tensor (primed index down) to transform the covariant (down)
coordinates).  Is there a convention for keeping free vs. varied indices
close to the body of the operator?  For the boost tensor he puts the free
index closer to $\Lambda$, but for the inverse tensor for a covariant
coordinate transformation puts the free index further out?

This also appears to be notational consistent with \citep{SpenceTensors}.

\subsection{Expressing bivector Lorentz transform in tensor form}

Having translated a vector Lorentz transform into tensor form, the next step is to do the same for
a bivector.  In particular for the field bivector $F = \grad \wedge A$.

Write

\begin{align*}
\grad' &= \Lambda \gamma_\mu \partial^\mu \Lambda^\dagger \\
A' &= \Lambda A^\nu \gamma_\nu \Lambda^\dagger \\
\end{align*}
\begin{align*}
\grad' \cdot e^\beta &= (\Lambda \gamma_\mu \Lambda^\dagger) \cdot e^\beta \partial^\mu = {\Lambda_\mu}^\beta \partial^\mu \\
A' \cdot e^\beta &= (\Lambda \gamma_\nu \Lambda^\dagger) \cdot e^\beta A^\nu = {\Lambda_\nu}^\beta A^\nu \\
\end{align*}

Then the transformed bivector is
\begin{align*}
F' = \grad' \wedge A'
&= ((\grad' \cdot e^\alpha) e_\alpha) \wedge ((A' \cdot e^\beta) e_\beta) \\
&= (e_\alpha \wedge e_\beta) {\Lambda_\mu}^\alpha {\Lambda_\nu}^\beta \partial^\mu A^\nu \\
\end{align*}

and finally the transformed tensor is thus

\begin{align*}
{F^{ab}}'
&= F' \cdot (e^b \wedge e^a) \\
&= (e_\alpha \wedge e_\beta) \cdot (e^b \wedge e^a) {\Lambda_\mu}^\alpha {\Lambda_\nu}^\beta \partial^\mu A^\nu \\
&= ( {\delta_\alpha}^a {\delta_\beta}^b - {\delta_\beta}^a {\delta_\alpha}^b ) {\Lambda_\mu}^\alpha {\Lambda_\nu}^\beta \partial^\mu A^\nu \\
&= {\Lambda_\mu}^a {\Lambda_\nu}^b \partial^\mu A^\nu
-{\Lambda_\mu}^b {\Lambda_\nu}^a \partial^\mu A^\nu \\
&= {\Lambda_\mu}^a {\Lambda_\nu}^b ( \partial^\mu A^\nu -\partial^\nu A^\mu ) \\
\end{align*}

Which gives the final transformation rule for the field bivector in tensor form

\begin{align}
{F^{ab}}' = {\Lambda_\mu}^a {\Lambda_\nu}^b F^{\mu\nu}
\end{align}

Returning to the original problem of field Lagrangian invariance, we want to examine how ${F^{ab}}' {F_{ab}}'$ transforms.  That is
\begin{dmath}
{F^{ab}}' {F_{ab}}'
= {\Lambda_\mu}^a {\Lambda_\nu}^b F^{\mu\nu} {\Lambda^\alpha}_a {\Lambda^\beta}_b F_{\alpha\beta} 
= ({\Lambda_\mu}^a {\Lambda^\alpha}_a) ({\Lambda_\nu}^b {\Lambda^\beta}_b) F^{\mu\nu} F_{\alpha\beta} 
= {\delta_\mu}^\alpha {\delta_\nu}^\beta F^{\mu\nu} F_{\alpha\beta} 
= F^{\mu\nu} F_{\mu\nu}
\end{dmath}

which is the desired result.  Since the dot product remainder of the Lagrangian \eqnref{eqn:boostMLag:lagrangian}
has already been shown to be Lorentz invariant this is sufficient to prove the Lagrangian boost or rotational invariance using tensor algebra.

Working this way is fairly compact and efficient, and required a few less steps than the multivector equivalent.  To compare apples to applies,
for the algebraic tools, it should be noted that if only the 
scalar part of $(\grad \wedge A)^2$ was considered as implicitly done in the tensor argument above, the multivector approach would likely have been as 
compact as well.
