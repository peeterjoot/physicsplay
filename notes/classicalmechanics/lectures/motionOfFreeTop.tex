%
% Copyright � 2013 Peeter Joot.  All Rights Reserved.
% Licenced as described in the file LICENSE under the root directory of this GIT repository.
%
% pick one:
%\newcommand{\authorname}{Peeter Joot}
\newcommand{\email}{peeter.joot@utoronto.ca}
\newcommand{\studentnumber}{920798560}
\newcommand{\basename}{FIXMEbasenameUndefined}
\newcommand{\dirname}{notes/FIXMEdirnameUndefined/}

\newcommand{\authorname}{Peeter Joot}
\newcommand{\email}{peeterjoot@protonmail.com}
\newcommand{\basename}{FIXMEbasenameUndefined}
\newcommand{\dirname}{notes/FIXMEdirnameUndefined/}

\renewcommand{\basename}{motionOfFreeTop}
\renewcommand{\dirname}{notes/classicalmechanics/}
%\newcommand{\dateintitle}{}
\newcommand{\keywords}{Symmetric free top, Classical Mechanics, PHY354H1S, PHY354H1, PHY354}

\newcommand{\authorname}{Peeter Joot}
\newcommand{\onlineurl}{http://sites.google.com/site/peeterjoot2/math2013/\basename.pdf}
\newcommand{\sourcepath}{\dirname\basename.tex}
\newcommand{\generatetitle}[1]{\chapter{#1}}

\newcommand{\vcsinfo}{%
\section*{}
\noindent{\color{DarkOliveGreen}{\rule{\linewidth}{0.1mm}}}
\paragraph{Document version}
%\paragraph{\color{Maroon}{Document version}}
{
\small
\begin{itemize}
\item Available online at:\\ 
\href{\onlineurl}{\onlineurl}
\item Git Repository: \input{./.revinfo/gitRepo.tex}
\item Source: \sourcepath
\item last commit: \input{./.revinfo/gitCommitString.tex}
\item commit date: \input{./.revinfo/gitCommitDate.tex}
\end{itemize}
}
}

%\PassOptionsToPackage{dvipsnames,svgnames}{xcolor}
\PassOptionsToPackage{square,numbers}{natbib}
\documentclass{scrreprt}

\usepackage[left=2cm,right=2cm]{geometry}
\usepackage[svgnames]{xcolor}
\usepackage{peeters_layout}

\usepackage{natbib}

\usepackage[
colorlinks=true,
bookmarks=false,
pdfauthor={\authorname, \email},
backref 
]{hyperref}

% http://tex.stackexchange.com/questions/75773/how-to-reference-problems-by-the-text-label-in-an-exercise-envioronment
\usepackage[english]{cleveref}
\crefname{Exercise}{exercise}{exercises}
\Crefname{Exercise}{Exercise}{Exercises}

\RequirePackage{titlesec}
\RequirePackage{ifthen}

% http://stackoverflow.com/questions/4932910/date-in-the-tabular-environment
\makeatletter
\let\insertdate\@date
\makeatother

\titleformat{\chapter}[display]
{\bfseries\Large}
{\color{DarkSlateGrey}\filleft \authorname
\ifthenelse{\isundefined{\studentnumber}}{}{\\ \studentnumber}
\ifthenelse{\isundefined{\email}}{}{\\ \email}
\ifthenelse{\isundefined{\dateintitle}}{}{\\ \insertdate}
%\ifthenelse{\isundefined{\coursename}}{}{\\ \coursename} % put in title instead.
}
{4ex}
{\color{DarkOliveGreen}{\titlerule}\color{Maroon}
\vspace{2ex}%
\filright}
[\vspace{2ex}%
\color{DarkOliveGreen}\titlerule
]

\newcommand{\beginArtWithToc}[0]{\begin{document}\tableofcontents}
\newcommand{\beginArtNoToc}[0]{\begin{document}}
\newcommand{\EndNoBibArticle}[0]{\end{document}}
\newcommand{\EndArticle}[0]{\bibliography{Bibliography}\bibliographystyle{plainnat}\end{document}}

% 
%\newcommand{\citep}[1]{\cite{#1}}

\colorSectionsForArticle



\beginArtNoToc

\generatetitle{PHY354H1S Advanced Classical Mechanics.  Motion of a symmetric free top.  Taught by Prof.\ Erich Poppitz}
\index{symmetric free top}
%\chapter{PHY354H1S Advanced Classical Mechanics.  Motion of a symmetric free top.  Taught by Prof.\ Erich Poppitz}
%\label{chap:\basename}
%\section{Motivation}
%\section{Guts}

\section{Disclaimer}

Peeter's lecture notes from auditing this class.  May not be entirely coherent.

\section{}

FIXME: F1.

\begin{equation}\label{eqn:motionOfFreeTop:20}
\BM = (0, 0, M)
\end{equation}

In the body frame

\begin{equation}\label{eqn:motionOfFreeTop:40}
\BM = M ( \sin\theta \sin\phi, \sin\theta \cos\phi, \cos\theta ) = (M_1, M_2, M_3)
\end{equation}

Recall that

\begin{align}\label{eqn:motionOfFreeTop:60}
M_1 &= I (\thetadot \cos\psi + \phidot \sin\theta \sin\psi) \\
M_2 &= I (-\thetadot \sin\psi + \phidot \sin\theta \cos\psi) \\
M_3 &= I (\psidot + \phidot \cos\theta)
\end{align}

Our EOM for (what restriction?) are

\begin{align}\label{eqn:motionOfFreeTop:80}
M \cos\theta &= I_3 ( \psidot + \phidot \cos\theta ) \leftarrow ( M_3 = I_3 \Omega_3) \\
M \sin\theta \cos\psi &= I ( -\thetadot \sin\psi + \phidot \sin\theta \cos\psi)  \leftarrow (M_2 = I \Omega_2) \\
...
\end{align}

switched to paper.

\section{Some words on the motion of a rigid body}

We need to know how the motion of the center of mass changes

\begin{equation}\label{eqn:motionOfFreeTop:100}
\frac{d}{dt} \BP_{\mathrm{CM}} = \BF
\end{equation}

where \(\BF\) is the external force applied to the body.

FIXME: G1.

We can also write

\begin{equation}\label{eqn:motionOfFreeTop:120}
\BF = \sum_a \Bf_a,
\end{equation}

where \(\Bf_a\) is the external force on the \(a\)-th particle.

We can also write a torque equation

\begin{equation}\label{eqn:motionOfFreeTop:140}
\frac{d}{dt} \BM = \BK
\end{equation}

where \(\BK\) is the torque of the external forces acting on the body.

\begin{equation}\label{eqn:motionOfFreeTop:160}
M_O = \sum_a \Brho_a \cross \Bp_a
\end{equation}

\begin{equation}\label{eqn:motionOfFreeTop:180}
\ddt{\BM_\theta} = \sum_a \Bv_a \cross \Bp_a + \sum_a \Brho_a \cross \Bp_a
\end{equation}

\begin{dmath}\label{eqn:motionOfFreeTop:n}
\ddt{\BM_\theta}
= \sum_a \Brho_a \cross \Bp_a
= \sum_a \BR_{\mathrm{CM}} \cross \Bf_a + \sum_a \Br_a \cross \Bf_a
= \BR_{\mathrm{CM}} \cross \BR + \sum_a \Bk_a
\end{dmath}

where \(\Bk_a\) is the torque on the \(a\)-th particle with respect to the torque of external forces with respect to the CM.

So

\begin{equation}\label{eqn:motionOfFreeTop:200}
\ddt{\BM_\theta} = \BR_{\mathrm{CM}} \cross \BF + \BK
\end{equation}

or

\begin{equation}\label{eqn:motionOfFreeTop:220}
\ddt{\BM_\theta} = \BK
\end{equation}

FIXME: why?

\begin{equation}\label{eqn:motionOfFreeTop:240}
\ddt{\BM_\theta} = \sum_a \Br_a \cross \Bf_a
\end{equation}

with
\begin{equation}\label{eqn:motionOfFreeTop:260}
\Bf_a = m_a \Bg
\end{equation}

we have

\begin{equation}\label{eqn:motionOfFreeTop:280}
\ddt{\BM_\theta} = \sum_a m_a \Br_a \cross \Bg.
\end{equation}

board erased fast.  Is this where things were left?

% this is to produce the sites.google url and version info and so forth (for blog posts)
%\vcsinfo
\EndArticle
%\EndNoBibArticle
