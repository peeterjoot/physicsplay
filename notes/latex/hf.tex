\documentclass[a4paper,12pt]{article}
\usepackage{amsmath}
\usepackage[customcolors]{hf-tikz}

\tikzset{offset def/.style={
    above left offset={-0.1,0.8},
    below right offset={0.1,-0.65},
  },
  integral first/.style={
    offset def,
  },
  integral second/.style={
    offset def,
    set fill color=green!50!lime!60,
    set border color=green!40!black,
  },
  sums/.style={
    offset def,
    set fill color=blue!20!cyan!60,
    set border color=blue!60!cyan,
  }
}

\begin{document}


\[\tikzmarkin{x-a}x + y = 400\tikzmarkend{x-a}\]

\vspace*{3ex}

\[
\tikzmarkin[integral first]{z2}
\int_{
E - \frac{\Delta}{2} \le H \le E + \frac{\Delta}{2} \le H
}
 d^{3N} x d^{3N} p
=
\left( \frac{2 \pi \Delta}{\omega} \right)^{3N}
\tikzmarkend{z2}
\]

\vspace*{3ex}

\[
\tikzmarkin[integral second]{z3}
\int_{
E - \frac{\Delta}{2} \le H \le E + \frac{\Delta}{2} \le H
}
 d^{3N} x d^{3N} p
\tikzmarkend{z3}
=
\tikzmarkin[integral first,disable rounded corners=true]{z4}
\left( \frac{2 \pi \Delta}{\omega} \right)^{3N}
\tikzmarkend{z4}
\]

\vspace*{3ex}

\begin{equation}
\begin{split}
H_c&=\tikzmarkin[disable rounded corners=true,sums]{xb}\frac{1}{2n} \sum^n_{l=0}(-1)^{l}(n-{l})^{p-2}
\sum_{l _1+\dots+ l _p=l}\prod^p_{i=1} \binom{n_i}{l _i}\tikzmarkend{xb}\\
&\quad\cdot[(n-l )-(n_i-l _i)]^{n_i-l _i}\cdot
\tikzmarkin[sums]{xb1}(0.05,-0.6)(-0.05,0.75)
\Bigl[(n-l )^2-\sum^p_{j=1}(n_i-l _i)^2\Bigr].
\tikzmarkend{xb1}
\end{split}
\end{equation}

\vspace*{3ex}

\[
\tikzmarkin[below offset=-0.4,
  above offset=0.55,
  set fill color=magenta!60!purple!30]{bla bla}
x + \dfrac{y}{z} = 400
\tikzmarkend{bla bla}
\]
\end{document}
