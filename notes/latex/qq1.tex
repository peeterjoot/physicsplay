% http://tex.stackexchange.com/questions/86188/labelling-with-arrows-in-an-automated-way/86203#86203

\documentclass{article}
\usepackage{tikz}
\usepackage{amsmath}

\newif\ifclipme\clipmetrue
\tikzset{labelstyle/.style={LabelStyle/.append style={#1}},linestyle/.style={LineStyle/.append style={#1}}}
\tikzset{LabelStyle/.initial={},LineStyle/.initial={}}

\newcommand{\myMathWithDescription}[4][]{{%
    \tikzset{#1}%
    \tikz[baseline]{
        \node[draw=red,rounded corners,anchor=base] (m#4) {$\displaystyle#2$};
        \ifclipme\begin{pgfinterruptboundingbox}\fi
            \node[above of=m#4,font=\strut, LabelStyle] (l#4) {#3};
            \draw[-,red, LineStyle] (l#4) to (m#4);
        \ifclipme\end{pgfinterruptboundingbox}\fi
    }%
}}
\newcommand{\myMathWithDescriptionStarred}[3][]{{%
    \clipmefalse%
    \myMathWithDescription[#1]{#2}{#3}{\themathLableNode}%
}}

\newcounter{mathLableNode}
\newcommand{\mathLabelBox}[3][]{%
   \stepcounter{mathLableNode}%
   \myMathWithDescription[#1]{#2}{#3}{\themathLableNode}%
   \vphantom{\myMathWithDescriptionStarred[#1]{#2}{#3}{\themathLableNode}}%
}

\begin{document}
\begin{equation}
\boldsymbol{\nabla}^2 =
\mathLabelBox{
\frac{\partial^2}{\partial r^2} + \frac{1}{r} \frac{\partial}{\partial r}
}
{$\boldsymbol{\nabla}_{\txtT}^2$}

+ \frac{1}{r^2} \frac{\partial^2}{\partial \theta^2} + \frac{\partial^2}{\partial z^2}
\end{equation}

\begin{equation}
\mathbf{E} =
\mathLabelBox[
    labelstyle={yshift=1.2em},
    linestyle={}
    ]
{\mathbf{E}_0}
{A vector, with a chosen polarity} \cdot
\mathLabelBox[
    labelstyle={xshift=2cm},
    linestyle={out=270,in=90, latex-}
    ]
{u(r, \theta, z)}
{Slowly varying (complex) envelope}
 \cdot e^{i k_0 z}.
\end{equation}
\end{document}

