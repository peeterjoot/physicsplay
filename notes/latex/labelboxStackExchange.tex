% http://tex.stackexchange.com/questions/86188/labelling-with-arrows-in-an-automated-way

\documentclass{article}

\usepackage{tikz}
\usepackage{amsmath}

% formula, text, node#
\newcommand{\myMathWithDescription}[3]{%
\tikz[baseline]{%
    \node[draw=red,rounded corners,anchor=base] (m#3)%
    {$\displaystyle#1$};%
    \node[above of=m#3] (l#3) {#2};%
    \draw[-,red] (l#3) -- (m#3);%
}%
}

\newcounter{mathLableNode}

\newcommand{\mathLabelBox}[2]{%
   \stepcounter{mathLableNode}%
   \myMathWithDescription{#1}{#2}{\themathLableNode}%
}

\begin{document}

\begin{equation}
\boldsymbol{\nabla}^2 =
\mathLabelBox{
\frac{\partial^2}{\partial r^2} + \frac{1}{r} \frac{\partial}{\partial r}{}
}{$\boldsymbol{\nabla}_{\txtT}^2$}
+ \frac{1}{r^2} \frac{\partial^2}{\partial \theta^2}
+ \frac{\partial^2}{\partial z^2}
\end{equation}

\begin{equation}
\mathbf{E} =
\mathLabelBox{
\mathbf{E}_0
}{A vector, with a chosen polarity}
\mathLabelBox{
u(r, \theta, z)
}{
Slowly varying (complex) envelope
}
e^{i k_0 z}.
\end{equation}

\end{document}
