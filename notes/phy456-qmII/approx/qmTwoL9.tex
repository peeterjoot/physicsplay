%
% Copyright � 2012 Peeter Joot.  All Rights Reserved.
% Licenced as described in the file LICENSE under the root directory of this GIT repository.
%

%
%
%%
% Copyright � 2015 Peeter Joot.  All Rights Reserved.
% Licenced as described in the file LICENSE under the root directory of this GIT repository.
%
\documentclass[]{eliblog}

\usepackage{amsmath}
\usepackage{mathpazo}

%
% shorthand for bold symbols, convenient for vectors and matrices
%
\newcommand{\Ba}[0]{\mathbf{a}}
\newcommand{\Bb}[0]{\mathbf{b}}
\newcommand{\Bc}[0]{\mathbf{c}}
\newcommand{\Bd}[0]{\mathbf{d}}
\newcommand{\Be}[0]{\mathbf{e}}
\newcommand{\Bf}[0]{\mathbf{f}}
\newcommand{\Bg}[0]{\mathbf{g}}
\newcommand{\Bh}[0]{\mathbf{h}}
\newcommand{\Bi}[0]{\mathbf{i}}
\newcommand{\Bj}[0]{\mathbf{j}}
\newcommand{\Bk}[0]{\mathbf{k}}
\newcommand{\Bl}[0]{\mathbf{l}}
\newcommand{\Bm}[0]{\mathbf{m}}
\newcommand{\Bn}[0]{\mathbf{n}}
\newcommand{\Bo}[0]{\mathbf{o}}
\newcommand{\Bp}[0]{\mathbf{p}}
\newcommand{\Bq}[0]{\mathbf{q}}
\newcommand{\Br}[0]{\mathbf{r}}
\newcommand{\Bs}[0]{\mathbf{s}}
\newcommand{\Bt}[0]{\mathbf{t}}
\newcommand{\Bu}[0]{\mathbf{u}}
\newcommand{\Bv}[0]{\mathbf{v}}
\newcommand{\Bw}[0]{\mathbf{w}}
\newcommand{\Bx}[0]{\mathbf{x}}
\newcommand{\By}[0]{\mathbf{y}}
\newcommand{\Bz}[0]{\mathbf{z}}
\newcommand{\BA}[0]{\mathbf{A}}
\newcommand{\BB}[0]{\mathbf{B}}
\newcommand{\BC}[0]{\mathbf{C}}
\newcommand{\BD}[0]{\mathbf{D}}
\newcommand{\BE}[0]{\mathbf{E}}
\newcommand{\BF}[0]{\mathbf{F}}
\newcommand{\BG}[0]{\mathbf{G}}
\newcommand{\BH}[0]{\mathbf{H}}
\newcommand{\BI}[0]{\mathbf{I}}
\newcommand{\BJ}[0]{\mathbf{J}}
\newcommand{\BK}[0]{\mathbf{K}}
\newcommand{\BL}[0]{\mathbf{L}}
\newcommand{\BM}[0]{\mathbf{M}}
\newcommand{\BN}[0]{\mathbf{N}}
\newcommand{\BO}[0]{\mathbf{O}}
\newcommand{\BP}[0]{\mathbf{P}}
\newcommand{\BQ}[0]{\mathbf{Q}}
\newcommand{\BR}[0]{\mathbf{R}}
\newcommand{\BS}[0]{\mathbf{S}}
\newcommand{\BT}[0]{\mathbf{T}}
\newcommand{\BU}[0]{\mathbf{U}}
\newcommand{\BV}[0]{\mathbf{V}}
\newcommand{\BW}[0]{\mathbf{W}}
\newcommand{\BX}[0]{\mathbf{X}}
\newcommand{\BY}[0]{\mathbf{Y}}
\newcommand{\BZ}[0]{\mathbf{Z}}

\newcommand{\Bzero}[0]{\mathbf{0}}
\newcommand{\Btheta}[0]{\boldsymbol{\theta}}
\newcommand{\Btau}[0]{\boldsymbol{\tau}}
\newcommand{\Bomega}[0]{\boldsymbol{\omega}}

%
% shorthand for unit vectors
%
\newcommand{\acap}[0]{\hat{\Ba}}
\newcommand{\bcap}[0]{\hat{\Bb}}
\newcommand{\ccap}[0]{\hat{\Bc}}
\newcommand{\dcap}[0]{\hat{\Bd}}
\newcommand{\ecap}[0]{\hat{\Be}}
\newcommand{\fcap}[0]{\hat{\Bf}}
\newcommand{\gcap}[0]{\hat{\Bg}}
\newcommand{\hcap}[0]{\hat{\Bh}}
\newcommand{\icap}[0]{\hat{\Bi}}
\newcommand{\jcap}[0]{\hat{\Bj}}
\newcommand{\kcap}[0]{\hat{\Bk}}
\newcommand{\lcap}[0]{\hat{\Bl}}
\newcommand{\mcap}[0]{\hat{\Bm}}
\newcommand{\ncap}[0]{\hat{\Bn}}
\newcommand{\ocap}[0]{\hat{\Bo}}
\newcommand{\pcap}[0]{\hat{\Bp}}
\newcommand{\qcap}[0]{\hat{\Bq}}
\newcommand{\rcap}[0]{\hat{\Br}}
\newcommand{\scap}[0]{\hat{\Bs}}
\newcommand{\tcap}[0]{\hat{\Bt}}
\newcommand{\ucap}[0]{\hat{\Bu}}
\newcommand{\vcap}[0]{\hat{\Bv}}
\newcommand{\wcap}[0]{\hat{\Bw}}
\newcommand{\xcap}[0]{\hat{\Bx}}
\newcommand{\ycap}[0]{\hat{\By}}
\newcommand{\zcap}[0]{\hat{\Bz}}
\newcommand{\thetacap}[0]{\hat{\Btheta}}

%
% to write R^n and C^n in a distinguishable fashion.  Perhaps change this
% to the double lined characters upon figuring out how to do so.
%
\newcommand{\C}[1]{$\mathbb{C}^{#1}$}
\newcommand{\R}[1]{$\mathbb{R}^{#1}$}

%
% various generally useful helpers
%

% derivative of #1 wrt. #2:
\newcommand{\D}[2] {\frac {d#2} {d#1}}

\newcommand{\inv}[1]{\frac{1}{#1}}
\newcommand{\cross}[0]{\times}

\newcommand{\abs}[1]{\lvert{#1}\rvert}
\newcommand{\norm}[1]{\lVert{#1}\rVert}
\newcommand{\innerprod}[2]{\langle{#1}, {#2}\rangle}
\newcommand{\dotprod}[2]{{#1} \cdot {#2}}
\newcommand{\bdotprod}[2]{\left({#1} \cdot {#2}\right)}
\newcommand{\crossprod}[2]{{#1} \cross {#2}}
\newcommand{\tripleprod}[3]{\dotprod{\left(\crossprod{#1}{#2}\right)}{#3}}

\DeclareMathOperator{\Proj}{Proj}
\DeclareMathOperator{\Span}{span}
\DeclareMathOperator{\Sgn}{sgn}
\DeclareMathOperator{\Area}{Area}
\DeclareMathOperator{\Volume}{Volume}

%
% A few miscellaneous things specific to this document
%
\newcommand{\crossop}[1]{\crossprod{#1}{}}

% R2 vector.
\newcommand{\VectorTwo}[2]{
\begin{bmatrix}
 {#1} \\
 {#2}
\end{bmatrix}
}

\newcommand{\VectorN}[1]{
\begin{bmatrix}
{#1}_1 \\
{#1}_2 \\
\vdots \\
{#1}_N \\
\end{bmatrix}
}

\newcommand{\DETuvij}[4]{
\begin{vmatrix}
 {#1}_{#3} & {#1}_{#4} \\
 {#2}_{#3} & {#2}_{#4}
\end{vmatrix}
}

\newcommand{\DETuvwijk}[6]{
\begin{vmatrix}
 {#1}_{#4} & {#1}_{#5} & {#1}_{#6} \\
 {#2}_{#4} & {#2}_{#5} & {#2}_{#6} \\
 {#3}_{#4} & {#3}_{#5} & {#3}_{#6}
\end{vmatrix}
}

\newcommand{\DETuvwxijkl}[8]{
\begin{vmatrix}
 {#1}_{#5} & {#1}_{#6} & {#1}_{#7} & {#1}_{#8} \\
 {#2}_{#5} & {#2}_{#6} & {#2}_{#7} & {#2}_{#8} \\
 {#3}_{#5} & {#3}_{#6} & {#3}_{#7} & {#3}_{#8} \\
 {#4}_{#5} & {#4}_{#6} & {#4}_{#7} & {#4}_{#8} \\
\end{vmatrix}
}

%\newcommand{\DETuvwxyijklm}[10]{
%\begin{vmatrix}
% {#1}_{#6} & {#1}_{#7} & {#1}_{#8} & {#1}_{#9} & {#1}_{#10} \\
% {#2}_{#6} & {#2}_{#7} & {#2}_{#8} & {#2}_{#9} & {#2}_{#10} \\
% {#3}_{#6} & {#3}_{#7} & {#3}_{#8} & {#3}_{#9} & {#3}_{#10} \\
% {#4}_{#6} & {#4}_{#7} & {#4}_{#8} & {#4}_{#9} & {#4}_{#10} \\
% {#5}_{#6} & {#5}_{#7} & {#5}_{#8} & {#5}_{#9} & {#5}_{#10}
%\end{vmatrix}
%}

% R3 vector.
\newcommand{\VectorThree}[3]{
\begin{bmatrix}
 {#1} \\
 {#2} \\
 {#3}
\end{bmatrix}
}



\author{Peeter Joot}
\email{peeter.joot@gmail.com}

%\documentclass[]{eliblogwidescreen}

\usepackage{amsmath}
\usepackage{mathpazo}

%
% shorthand for bold symbols, convenient for vectors and matrices
%
\newcommand{\Ba}[0]{\mathbf{a}}
\newcommand{\Bb}[0]{\mathbf{b}}
\newcommand{\Bc}[0]{\mathbf{c}}
\newcommand{\Bd}[0]{\mathbf{d}}
\newcommand{\Be}[0]{\mathbf{e}}
\newcommand{\Bf}[0]{\mathbf{f}}
\newcommand{\Bg}[0]{\mathbf{g}}
\newcommand{\Bh}[0]{\mathbf{h}}
\newcommand{\Bi}[0]{\mathbf{i}}
\newcommand{\Bj}[0]{\mathbf{j}}
\newcommand{\Bk}[0]{\mathbf{k}}
\newcommand{\Bl}[0]{\mathbf{l}}
\newcommand{\Bm}[0]{\mathbf{m}}
\newcommand{\Bn}[0]{\mathbf{n}}
\newcommand{\Bo}[0]{\mathbf{o}}
\newcommand{\Bp}[0]{\mathbf{p}}
\newcommand{\Bq}[0]{\mathbf{q}}
\newcommand{\Br}[0]{\mathbf{r}}
\newcommand{\Bs}[0]{\mathbf{s}}
\newcommand{\Bt}[0]{\mathbf{t}}
\newcommand{\Bu}[0]{\mathbf{u}}
\newcommand{\Bv}[0]{\mathbf{v}}
\newcommand{\Bw}[0]{\mathbf{w}}
\newcommand{\Bx}[0]{\mathbf{x}}
\newcommand{\By}[0]{\mathbf{y}}
\newcommand{\Bz}[0]{\mathbf{z}}
\newcommand{\BA}[0]{\mathbf{A}}
\newcommand{\BB}[0]{\mathbf{B}}
\newcommand{\BC}[0]{\mathbf{C}}
\newcommand{\BD}[0]{\mathbf{D}}
\newcommand{\BE}[0]{\mathbf{E}}
\newcommand{\BF}[0]{\mathbf{F}}
\newcommand{\BG}[0]{\mathbf{G}}
\newcommand{\BH}[0]{\mathbf{H}}
\newcommand{\BI}[0]{\mathbf{I}}
\newcommand{\BJ}[0]{\mathbf{J}}
\newcommand{\BK}[0]{\mathbf{K}}
\newcommand{\BL}[0]{\mathbf{L}}
\newcommand{\BM}[0]{\mathbf{M}}
\newcommand{\BN}[0]{\mathbf{N}}
\newcommand{\BO}[0]{\mathbf{O}}
\newcommand{\BP}[0]{\mathbf{P}}
\newcommand{\BQ}[0]{\mathbf{Q}}
\newcommand{\BR}[0]{\mathbf{R}}
\newcommand{\BS}[0]{\mathbf{S}}
\newcommand{\BT}[0]{\mathbf{T}}
\newcommand{\BU}[0]{\mathbf{U}}
\newcommand{\BV}[0]{\mathbf{V}}
\newcommand{\BW}[0]{\mathbf{W}}
\newcommand{\BX}[0]{\mathbf{X}}
\newcommand{\BY}[0]{\mathbf{Y}}
\newcommand{\BZ}[0]{\mathbf{Z}}

\newcommand{\Bzero}[0]{\mathbf{0}}
\newcommand{\Btheta}[0]{\boldsymbol{\theta}}
\newcommand{\Btau}[0]{\boldsymbol{\tau}}
\newcommand{\Bomega}[0]{\boldsymbol{\omega}}

%
% shorthand for unit vectors
%
\newcommand{\acap}[0]{\hat{\Ba}}
\newcommand{\bcap}[0]{\hat{\Bb}}
\newcommand{\ccap}[0]{\hat{\Bc}}
\newcommand{\dcap}[0]{\hat{\Bd}}
\newcommand{\ecap}[0]{\hat{\Be}}
\newcommand{\fcap}[0]{\hat{\Bf}}
\newcommand{\gcap}[0]{\hat{\Bg}}
\newcommand{\hcap}[0]{\hat{\Bh}}
\newcommand{\icap}[0]{\hat{\Bi}}
\newcommand{\jcap}[0]{\hat{\Bj}}
\newcommand{\kcap}[0]{\hat{\Bk}}
\newcommand{\lcap}[0]{\hat{\Bl}}
\newcommand{\mcap}[0]{\hat{\Bm}}
\newcommand{\ncap}[0]{\hat{\Bn}}
\newcommand{\ocap}[0]{\hat{\Bo}}
\newcommand{\pcap}[0]{\hat{\Bp}}
\newcommand{\qcap}[0]{\hat{\Bq}}
\newcommand{\rcap}[0]{\hat{\Br}}
\newcommand{\scap}[0]{\hat{\Bs}}
\newcommand{\tcap}[0]{\hat{\Bt}}
\newcommand{\ucap}[0]{\hat{\Bu}}
\newcommand{\vcap}[0]{\hat{\Bv}}
\newcommand{\wcap}[0]{\hat{\Bw}}
\newcommand{\xcap}[0]{\hat{\Bx}}
\newcommand{\ycap}[0]{\hat{\By}}
\newcommand{\zcap}[0]{\hat{\Bz}}
\newcommand{\thetacap}[0]{\hat{\Btheta}}

%
% to write R^n and C^n in a distinguishable fashion.  Perhaps change this
% to the double lined characters upon figuring out how to do so.
%
\newcommand{\C}[1]{$\mathbb{C}^{#1}$}
\newcommand{\R}[1]{$\mathbb{R}^{#1}$}

%
% various generally useful helpers
%

% derivative of #1 wrt. #2:
\newcommand{\D}[2] {\frac {d#2} {d#1}}

\newcommand{\inv}[1]{\frac{1}{#1}}
\newcommand{\cross}[0]{\times}

\newcommand{\abs}[1]{\lvert{#1}\rvert}
\newcommand{\norm}[1]{\lVert{#1}\rVert}
\newcommand{\innerprod}[2]{\langle{#1}, {#2}\rangle}
\newcommand{\dotprod}[2]{{#1} \cdot {#2}}
\newcommand{\bdotprod}[2]{\left({#1} \cdot {#2}\right)}
\newcommand{\crossprod}[2]{{#1} \cross {#2}}
\newcommand{\tripleprod}[3]{\dotprod{\left(\crossprod{#1}{#2}\right)}{#3}}

\DeclareMathOperator{\Proj}{Proj}
\DeclareMathOperator{\Span}{span}
\DeclareMathOperator{\Sgn}{sgn}
\DeclareMathOperator{\Area}{Area}
\DeclareMathOperator{\Volume}{Volume}

%
% A few miscellaneous things specific to this document
%
\newcommand{\crossop}[1]{\crossprod{#1}{}}

% R2 vector.
\newcommand{\VectorTwo}[2]{
\begin{bmatrix}
 {#1} \\
 {#2}
\end{bmatrix}
}

\newcommand{\VectorN}[1]{
\begin{bmatrix}
{#1}_1 \\
{#1}_2 \\
\vdots \\
{#1}_N \\
\end{bmatrix}
}

\newcommand{\DETuvij}[4]{
\begin{vmatrix}
 {#1}_{#3} & {#1}_{#4} \\
 {#2}_{#3} & {#2}_{#4}
\end{vmatrix}
}

\newcommand{\DETuvwijk}[6]{
\begin{vmatrix}
 {#1}_{#4} & {#1}_{#5} & {#1}_{#6} \\
 {#2}_{#4} & {#2}_{#5} & {#2}_{#6} \\
 {#3}_{#4} & {#3}_{#5} & {#3}_{#6}
\end{vmatrix}
}

\newcommand{\DETuvwxijkl}[8]{
\begin{vmatrix}
 {#1}_{#5} & {#1}_{#6} & {#1}_{#7} & {#1}_{#8} \\
 {#2}_{#5} & {#2}_{#6} & {#2}_{#7} & {#2}_{#8} \\
 {#3}_{#5} & {#3}_{#6} & {#3}_{#7} & {#3}_{#8} \\
 {#4}_{#5} & {#4}_{#6} & {#4}_{#7} & {#4}_{#8} \\
\end{vmatrix}
}

%\newcommand{\DETuvwxyijklm}[10]{
%\begin{vmatrix}
% {#1}_{#6} & {#1}_{#7} & {#1}_{#8} & {#1}_{#9} & {#1}_{#10} \\
% {#2}_{#6} & {#2}_{#7} & {#2}_{#8} & {#2}_{#9} & {#2}_{#10} \\
% {#3}_{#6} & {#3}_{#7} & {#3}_{#8} & {#3}_{#9} & {#3}_{#10} \\
% {#4}_{#6} & {#4}_{#7} & {#4}_{#8} & {#4}_{#9} & {#4}_{#10} \\
% {#5}_{#6} & {#5}_{#7} & {#5}_{#8} & {#5}_{#9} & {#5}_{#10}
%\end{vmatrix}
%}

% R3 vector.
\newcommand{\VectorThree}[3]{
\begin{bmatrix}
 {#1} \\
 {#2} \\
 {#3}
\end{bmatrix}
}



\author{Peeter Joot}
\email{peeter.joot@gmail.com}


%\chapter{PHY456H1F: Quantum Mechanics II.  Lecture 9 (Taught by Prof J.E. Sipe).  Adiabatic perturbation theory (cont.)}
\index{adiabatic perturbation}
%\chapter{Adiabatic perturbation theory (cont.), and Fermi's golden rule}
\label{chap:qmTwoL9}
\blogpage{http://sites.google.com/site/peeterjoot/math2011/qmTwoL9.pdf}
%\date{Oct 5, 2011}





\section{Adiabatic perturbation theory (cont.)}

We were working through Adiabatic time dependent perturbation (as also covered in \S 17.5.2 of the text \citep{desai2009quantum}.)

Utilizing an expansion

\begin{equation}\label{eqn:qmTwoL9:10}
\begin{aligned}
\ket{\psi(t)} &= \sum_n c_n(t) e^{- i \omega_n^{(0)} t} \ket{\psi_n^{(0)} } \\
&= \sum_n b_n(t) \ket{\hat{\psi}_n(t)},
\end{aligned}
\end{equation}

where

\begin{equation}\label{eqn:qmTwoL9:30}
H(t) \ket{\hat{\psi}_s(t)} = E_s(t) \ket{\hat{\psi}_s(t)}
\end{equation}

and found

\begin{equation}\label{eqn:qmTwoL9:50}
\ddt{b_s(t)} =
-i \left(
\omega_s(t) - \Gamma_s(t)
\right) b_s(t)
-
\sum_{n \ne s} b_n(t)
\bra{\hat{\psi}_s(t)}
\ddt{} \ket{\hat{\psi}_n(t)}
\end{equation}

where

\begin{equation}\label{eqn:qmTwoL9:70}
\Gamma_s(t) =
i \bra{\hat{\psi}_s(t)} \ddt{} \ket{\hat{\psi}_s(t)}
\end{equation}

Look for a solution of the form

\begin{equation}\label{eqn:qmTwoL9:90}
\begin{aligned}
b_s(t) &= \overbar{b}_s(t) e^{-i \int_0^t dt' (\omega_s(t') - \Gamma_s(t'))} \\
&=
\overbar{b}_s(t) e^{-i \gamma_s(t)}
\end{aligned}
\end{equation}

where
\begin{equation}\label{eqn:qmTwoL9:110}
\gamma_s(t) =
\int_0^t dt' (\omega_s(t') - \Gamma_s(t')).
\end{equation}

Taking derivatives of \(\overbar{b}_s\) and after a bit of manipulation we find that things conveniently cancel

\begin{equation}\label{eqn:qmTwoL9:330}
\begin{aligned}
\ddt{\overbar{b}_s(t)}
&= \ddt{} \left( b_s(t) e^{i \gamma_s(t) } \right) \\
&=
\ddt{b_s(t)} e^{i \gamma_s(t) } +
b_s(t) \ddt{} e^{i \gamma_s(t) }  \\
&=
\ddt{b_s(t)} e^{i \gamma_s(t) } +
b_s(t) i (\omega_s(t) - \Gamma_s(t)) e^{i \gamma_s(t) }.
\end{aligned}
\end{equation}

We find

\begin{equation}\label{eqn:qmTwoL9:350}
\begin{aligned}
\ddt{\overbar{b}_s(t)}
e^{-i \gamma_s(t)}
&=
\ddt{b_s(t)} + i b_s(t) (\omega_s(t) - \Gamma_s(t))  \\
&=
\cancel{i b_s(t) (\omega_s(t) - \Gamma_s(t)) }
-\cancel{i \left(
\omega_s(t) - \Gamma_s(t)
\right) b_s(t)}
-
\sum_{n \ne s} b_n(t)
\bra{\hat{\psi}_s(t)}
\ddt{} \ket{\hat{\psi}_n(t)},
\end{aligned}
\end{equation}

so

\begin{equation}\label{eqn:qmTwoL9:370}
\begin{aligned}
\ddt{\overbar{b}_s(t)}
&=
-
\sum_{n \ne s} b_n(t)
e^{i \gamma_s(t)}
\bra{\hat{\psi}_s(t)}
\ddt{} \ket{\hat{\psi}_n(t)} \\
&=
-
\sum_{n \ne s} \overbar{b}_n(t)
e^{i (\gamma_s(t) - \gamma_n(t))}
\bra{\hat{\psi}_s(t)}
\ddt{} \ket{\hat{\psi}_n(t)}.
\end{aligned}
\end{equation}

With a last bit of notation

\begin{equation}\label{eqn:qmTwoL9:130}
\gamma_{sn}(t) = \gamma_s(t) - \gamma_n(t)),
\end{equation}

the problem is reduced to one involving only the sums over the \(n \ne s\) terms, and where all the dependence on \(\bra{\hat{\psi}_s(t)} \ddt{} \ket{\hat{\psi}_s(t)}\) has been nicely isolated in a phase term

\begin{equation}\label{eqn:qmTwoL9:150}
\ddt{\overbar{b}_s(t)}
=
-
\sum_{n \ne s} \overbar{b}_n(t)
e^{i \gamma_{sn}(t) }
\bra{\hat{\psi}_s(t)}
\ddt{} \ket{\hat{\psi}_n(t)}.
\end{equation}

\paragraph{Looking for an approximate solution}

\paragraph{Try}: An approximate solution

\begin{equation}\label{eqn:qmTwoL9:170}
\overbar{b}_n(t) =
\delta_{nm}
\end{equation}

For \(s = m\) this is okay, since we have \(\ddt{\delta_{ns}} = 0\) which is consistent with

\begin{equation}\label{eqn:qmTwoL9:190}
\sum_{n \ne s} \delta_{ns} ( \cdots ) = 0
\end{equation}

However, for \(s \ne m\) we get

\begin{equation}\label{eqn:qmTwoL9:390}
\begin{aligned}
\ddt{\overbar{b}_s(t)}
&=
-
\sum_{n \ne s}
\delta_{nm}
e^{i \gamma_{sn}(t) }
\bra{\hat{\psi}_s(t)}
\ddt{} \ket{\hat{\psi}_n(t)} \\
&=
-
e^{i \gamma_{sm}(t) }
\bra{\hat{\psi}_s(t)}
\ddt{} \ket{\hat{\psi}_m(t)} \\
\end{aligned}
\end{equation}

But

\begin{equation}\label{eqn:qmTwoL9:210}
\gamma_{sm}(t) = \int_0^t dt' \left( \inv{\Hbar}( E_s(t') - E_m(t') ) - \Gamma_s(t') + \Gamma_m(t') \right)
\end{equation}

FIXME: I think we argued in class that the \(\Gamma\) contributions are negligible.  Why was that?

Now, are energy levels will have variation with time, as illustrated in \cref{fig:qmTwoL9:1}

\pdfTexFigure{../../figures/phy456/qmTwoL9fig1.pdf_tex}{Energy level variation with time}{fig:qmTwoL9:1}{0.2}

Perhaps unrealistically, suppose that our energy levels have some ``typical'' energy difference \(\Delta E\), so that

\begin{equation}\label{eqn:qmTwoL9:230}
\gamma_{sm}(t) \approx \frac{\Delta E}{\Hbar} t \equiv \frac{t}{\tau},
\end{equation}

or

\begin{equation}\label{eqn:qmTwoL9:250}
\tau = \frac{\Hbar}{\Delta E}
\end{equation}

Suppose that \(\tau\) is much less than a typical time \(T\) over which instantaneous quantities (wavefunctions and brakets) change.  After a large time \(T\)

\begin{equation}\label{eqn:qmTwoL9:270}
e^{i \gamma_{sm}(t)} \approx e^{i T/\tau}
\end{equation}

so we have our phase term whipping around really fast, as illustrated in \cref{fig:qmTwoL9:2}.

\pdfTexFigure{../../figures/phy456/qmTwoL9fig2.pdf_tex}{Phase whipping around}{fig:qmTwoL9:2}{0.3}

So, while \(\bra{\hat{\psi}_s(t)} \ddt{} \ket{\hat{\psi}_m(t)}\) is moving really slow, but our phase space portion is changing really fast.  The key to the approximate solution is factoring out this quickly changing phase term.

\paragraph{Note} \(\Gamma_s(t)\) is called the ``Berry'' phase \citep{wiki:GeometricPhase}, whereas the \(E_s(t')/\Hbar\) part is called the geometric phase, and can be shown to have a geometric interpretation.

To proceed we can introduce \(\lambda\) terms, perhaps

\begin{equation}\label{eqn:qmTwoL9:290}
\overbar{b}_s(t) = \delta_{ms} + \lambda \overbar{b}^{(1)}_s(t) + \cdots
\end{equation}

and
\begin{equation}\label{eqn:qmTwoL9:310}
- \sum_{n \ne s} e^{i \gamma_{sn}(t)} \lambda (\cdots)
\end{equation}

This \(\lambda\) approximation and a similar Taylor series expansion in time have been explored further in \ref{chap:adiabaticApprox}.

\paragraph{Degeneracy}

Suppose we have some branching of energy levels that were initially degenerate, as illustrated in \cref{fig:qmTwoL9:3}

\pdfTexFigure{../../figures/phy456/qmTwoL9fig3.pdf_tex}{Degenerate energy level splitting}{fig:qmTwoL9:3}{0.3}

We have a necessity to choose states properly so there is a continuous evolution in the instantaneous eigenvalues as \(H(t)\) changes.

\paragraph{Question: A physical  example?}

FIXME: Prof Sipe to ponder and revisit.

