%
% Copyright � 2012 Peeter Joot.  All Rights Reserved.
% Licenced as described in the file LICENSE under the root directory of this GIT repository.
%

%
%
%%
% Copyright � 2015 Peeter Joot.  All Rights Reserved.
% Licenced as described in the file LICENSE under the root directory of this GIT repository.
%
\documentclass[]{eliblog}

\usepackage{amsmath}
\usepackage{mathpazo}

%
% shorthand for bold symbols, convenient for vectors and matrices
%
\newcommand{\Ba}[0]{\mathbf{a}}
\newcommand{\Bb}[0]{\mathbf{b}}
\newcommand{\Bc}[0]{\mathbf{c}}
\newcommand{\Bd}[0]{\mathbf{d}}
\newcommand{\Be}[0]{\mathbf{e}}
\newcommand{\Bf}[0]{\mathbf{f}}
\newcommand{\Bg}[0]{\mathbf{g}}
\newcommand{\Bh}[0]{\mathbf{h}}
\newcommand{\Bi}[0]{\mathbf{i}}
\newcommand{\Bj}[0]{\mathbf{j}}
\newcommand{\Bk}[0]{\mathbf{k}}
\newcommand{\Bl}[0]{\mathbf{l}}
\newcommand{\Bm}[0]{\mathbf{m}}
\newcommand{\Bn}[0]{\mathbf{n}}
\newcommand{\Bo}[0]{\mathbf{o}}
\newcommand{\Bp}[0]{\mathbf{p}}
\newcommand{\Bq}[0]{\mathbf{q}}
\newcommand{\Br}[0]{\mathbf{r}}
\newcommand{\Bs}[0]{\mathbf{s}}
\newcommand{\Bt}[0]{\mathbf{t}}
\newcommand{\Bu}[0]{\mathbf{u}}
\newcommand{\Bv}[0]{\mathbf{v}}
\newcommand{\Bw}[0]{\mathbf{w}}
\newcommand{\Bx}[0]{\mathbf{x}}
\newcommand{\By}[0]{\mathbf{y}}
\newcommand{\Bz}[0]{\mathbf{z}}
\newcommand{\BA}[0]{\mathbf{A}}
\newcommand{\BB}[0]{\mathbf{B}}
\newcommand{\BC}[0]{\mathbf{C}}
\newcommand{\BD}[0]{\mathbf{D}}
\newcommand{\BE}[0]{\mathbf{E}}
\newcommand{\BF}[0]{\mathbf{F}}
\newcommand{\BG}[0]{\mathbf{G}}
\newcommand{\BH}[0]{\mathbf{H}}
\newcommand{\BI}[0]{\mathbf{I}}
\newcommand{\BJ}[0]{\mathbf{J}}
\newcommand{\BK}[0]{\mathbf{K}}
\newcommand{\BL}[0]{\mathbf{L}}
\newcommand{\BM}[0]{\mathbf{M}}
\newcommand{\BN}[0]{\mathbf{N}}
\newcommand{\BO}[0]{\mathbf{O}}
\newcommand{\BP}[0]{\mathbf{P}}
\newcommand{\BQ}[0]{\mathbf{Q}}
\newcommand{\BR}[0]{\mathbf{R}}
\newcommand{\BS}[0]{\mathbf{S}}
\newcommand{\BT}[0]{\mathbf{T}}
\newcommand{\BU}[0]{\mathbf{U}}
\newcommand{\BV}[0]{\mathbf{V}}
\newcommand{\BW}[0]{\mathbf{W}}
\newcommand{\BX}[0]{\mathbf{X}}
\newcommand{\BY}[0]{\mathbf{Y}}
\newcommand{\BZ}[0]{\mathbf{Z}}

\newcommand{\Bzero}[0]{\mathbf{0}}
\newcommand{\Btheta}[0]{\boldsymbol{\theta}}
\newcommand{\Btau}[0]{\boldsymbol{\tau}}
\newcommand{\Bomega}[0]{\boldsymbol{\omega}}

%
% shorthand for unit vectors
%
\newcommand{\acap}[0]{\hat{\Ba}}
\newcommand{\bcap}[0]{\hat{\Bb}}
\newcommand{\ccap}[0]{\hat{\Bc}}
\newcommand{\dcap}[0]{\hat{\Bd}}
\newcommand{\ecap}[0]{\hat{\Be}}
\newcommand{\fcap}[0]{\hat{\Bf}}
\newcommand{\gcap}[0]{\hat{\Bg}}
\newcommand{\hcap}[0]{\hat{\Bh}}
\newcommand{\icap}[0]{\hat{\Bi}}
\newcommand{\jcap}[0]{\hat{\Bj}}
\newcommand{\kcap}[0]{\hat{\Bk}}
\newcommand{\lcap}[0]{\hat{\Bl}}
\newcommand{\mcap}[0]{\hat{\Bm}}
\newcommand{\ncap}[0]{\hat{\Bn}}
\newcommand{\ocap}[0]{\hat{\Bo}}
\newcommand{\pcap}[0]{\hat{\Bp}}
\newcommand{\qcap}[0]{\hat{\Bq}}
\newcommand{\rcap}[0]{\hat{\Br}}
\newcommand{\scap}[0]{\hat{\Bs}}
\newcommand{\tcap}[0]{\hat{\Bt}}
\newcommand{\ucap}[0]{\hat{\Bu}}
\newcommand{\vcap}[0]{\hat{\Bv}}
\newcommand{\wcap}[0]{\hat{\Bw}}
\newcommand{\xcap}[0]{\hat{\Bx}}
\newcommand{\ycap}[0]{\hat{\By}}
\newcommand{\zcap}[0]{\hat{\Bz}}
\newcommand{\thetacap}[0]{\hat{\Btheta}}

%
% to write R^n and C^n in a distinguishable fashion.  Perhaps change this
% to the double lined characters upon figuring out how to do so.
%
\newcommand{\C}[1]{$\mathbb{C}^{#1}$}
\newcommand{\R}[1]{$\mathbb{R}^{#1}$}

%
% various generally useful helpers
%

% derivative of #1 wrt. #2:
\newcommand{\D}[2] {\frac {d#2} {d#1}}

\newcommand{\inv}[1]{\frac{1}{#1}}
\newcommand{\cross}[0]{\times}

\newcommand{\abs}[1]{\lvert{#1}\rvert}
\newcommand{\norm}[1]{\lVert{#1}\rVert}
\newcommand{\innerprod}[2]{\langle{#1}, {#2}\rangle}
\newcommand{\dotprod}[2]{{#1} \cdot {#2}}
\newcommand{\bdotprod}[2]{\left({#1} \cdot {#2}\right)}
\newcommand{\crossprod}[2]{{#1} \cross {#2}}
\newcommand{\tripleprod}[3]{\dotprod{\left(\crossprod{#1}{#2}\right)}{#3}}

\DeclareMathOperator{\Proj}{Proj}
\DeclareMathOperator{\Span}{span}
\DeclareMathOperator{\Sgn}{sgn}
\DeclareMathOperator{\Area}{Area}
\DeclareMathOperator{\Volume}{Volume}

%
% A few miscellaneous things specific to this document
%
\newcommand{\crossop}[1]{\crossprod{#1}{}}

% R2 vector.
\newcommand{\VectorTwo}[2]{
\begin{bmatrix}
 {#1} \\
 {#2}
\end{bmatrix}
}

\newcommand{\VectorN}[1]{
\begin{bmatrix}
{#1}_1 \\
{#1}_2 \\
\vdots \\
{#1}_N \\
\end{bmatrix}
}

\newcommand{\DETuvij}[4]{
\begin{vmatrix}
 {#1}_{#3} & {#1}_{#4} \\
 {#2}_{#3} & {#2}_{#4}
\end{vmatrix}
}

\newcommand{\DETuvwijk}[6]{
\begin{vmatrix}
 {#1}_{#4} & {#1}_{#5} & {#1}_{#6} \\
 {#2}_{#4} & {#2}_{#5} & {#2}_{#6} \\
 {#3}_{#4} & {#3}_{#5} & {#3}_{#6}
\end{vmatrix}
}

\newcommand{\DETuvwxijkl}[8]{
\begin{vmatrix}
 {#1}_{#5} & {#1}_{#6} & {#1}_{#7} & {#1}_{#8} \\
 {#2}_{#5} & {#2}_{#6} & {#2}_{#7} & {#2}_{#8} \\
 {#3}_{#5} & {#3}_{#6} & {#3}_{#7} & {#3}_{#8} \\
 {#4}_{#5} & {#4}_{#6} & {#4}_{#7} & {#4}_{#8} \\
\end{vmatrix}
}

%\newcommand{\DETuvwxyijklm}[10]{
%\begin{vmatrix}
% {#1}_{#6} & {#1}_{#7} & {#1}_{#8} & {#1}_{#9} & {#1}_{#10} \\
% {#2}_{#6} & {#2}_{#7} & {#2}_{#8} & {#2}_{#9} & {#2}_{#10} \\
% {#3}_{#6} & {#3}_{#7} & {#3}_{#8} & {#3}_{#9} & {#3}_{#10} \\
% {#4}_{#6} & {#4}_{#7} & {#4}_{#8} & {#4}_{#9} & {#4}_{#10} \\
% {#5}_{#6} & {#5}_{#7} & {#5}_{#8} & {#5}_{#9} & {#5}_{#10}
%\end{vmatrix}
%}

% R3 vector.
\newcommand{\VectorThree}[3]{
\begin{bmatrix}
 {#1} \\
 {#2} \\
 {#3}
\end{bmatrix}
}



\author{Peeter Joot}
\email{peeter.joot@gmail.com}

%\documentclass[]{eliblogwidescreen}

\usepackage{amsmath}
\usepackage{mathpazo}

%
% shorthand for bold symbols, convenient for vectors and matrices
%
\newcommand{\Ba}[0]{\mathbf{a}}
\newcommand{\Bb}[0]{\mathbf{b}}
\newcommand{\Bc}[0]{\mathbf{c}}
\newcommand{\Bd}[0]{\mathbf{d}}
\newcommand{\Be}[0]{\mathbf{e}}
\newcommand{\Bf}[0]{\mathbf{f}}
\newcommand{\Bg}[0]{\mathbf{g}}
\newcommand{\Bh}[0]{\mathbf{h}}
\newcommand{\Bi}[0]{\mathbf{i}}
\newcommand{\Bj}[0]{\mathbf{j}}
\newcommand{\Bk}[0]{\mathbf{k}}
\newcommand{\Bl}[0]{\mathbf{l}}
\newcommand{\Bm}[0]{\mathbf{m}}
\newcommand{\Bn}[0]{\mathbf{n}}
\newcommand{\Bo}[0]{\mathbf{o}}
\newcommand{\Bp}[0]{\mathbf{p}}
\newcommand{\Bq}[0]{\mathbf{q}}
\newcommand{\Br}[0]{\mathbf{r}}
\newcommand{\Bs}[0]{\mathbf{s}}
\newcommand{\Bt}[0]{\mathbf{t}}
\newcommand{\Bu}[0]{\mathbf{u}}
\newcommand{\Bv}[0]{\mathbf{v}}
\newcommand{\Bw}[0]{\mathbf{w}}
\newcommand{\Bx}[0]{\mathbf{x}}
\newcommand{\By}[0]{\mathbf{y}}
\newcommand{\Bz}[0]{\mathbf{z}}
\newcommand{\BA}[0]{\mathbf{A}}
\newcommand{\BB}[0]{\mathbf{B}}
\newcommand{\BC}[0]{\mathbf{C}}
\newcommand{\BD}[0]{\mathbf{D}}
\newcommand{\BE}[0]{\mathbf{E}}
\newcommand{\BF}[0]{\mathbf{F}}
\newcommand{\BG}[0]{\mathbf{G}}
\newcommand{\BH}[0]{\mathbf{H}}
\newcommand{\BI}[0]{\mathbf{I}}
\newcommand{\BJ}[0]{\mathbf{J}}
\newcommand{\BK}[0]{\mathbf{K}}
\newcommand{\BL}[0]{\mathbf{L}}
\newcommand{\BM}[0]{\mathbf{M}}
\newcommand{\BN}[0]{\mathbf{N}}
\newcommand{\BO}[0]{\mathbf{O}}
\newcommand{\BP}[0]{\mathbf{P}}
\newcommand{\BQ}[0]{\mathbf{Q}}
\newcommand{\BR}[0]{\mathbf{R}}
\newcommand{\BS}[0]{\mathbf{S}}
\newcommand{\BT}[0]{\mathbf{T}}
\newcommand{\BU}[0]{\mathbf{U}}
\newcommand{\BV}[0]{\mathbf{V}}
\newcommand{\BW}[0]{\mathbf{W}}
\newcommand{\BX}[0]{\mathbf{X}}
\newcommand{\BY}[0]{\mathbf{Y}}
\newcommand{\BZ}[0]{\mathbf{Z}}

\newcommand{\Bzero}[0]{\mathbf{0}}
\newcommand{\Btheta}[0]{\boldsymbol{\theta}}
\newcommand{\Btau}[0]{\boldsymbol{\tau}}
\newcommand{\Bomega}[0]{\boldsymbol{\omega}}

%
% shorthand for unit vectors
%
\newcommand{\acap}[0]{\hat{\Ba}}
\newcommand{\bcap}[0]{\hat{\Bb}}
\newcommand{\ccap}[0]{\hat{\Bc}}
\newcommand{\dcap}[0]{\hat{\Bd}}
\newcommand{\ecap}[0]{\hat{\Be}}
\newcommand{\fcap}[0]{\hat{\Bf}}
\newcommand{\gcap}[0]{\hat{\Bg}}
\newcommand{\hcap}[0]{\hat{\Bh}}
\newcommand{\icap}[0]{\hat{\Bi}}
\newcommand{\jcap}[0]{\hat{\Bj}}
\newcommand{\kcap}[0]{\hat{\Bk}}
\newcommand{\lcap}[0]{\hat{\Bl}}
\newcommand{\mcap}[0]{\hat{\Bm}}
\newcommand{\ncap}[0]{\hat{\Bn}}
\newcommand{\ocap}[0]{\hat{\Bo}}
\newcommand{\pcap}[0]{\hat{\Bp}}
\newcommand{\qcap}[0]{\hat{\Bq}}
\newcommand{\rcap}[0]{\hat{\Br}}
\newcommand{\scap}[0]{\hat{\Bs}}
\newcommand{\tcap}[0]{\hat{\Bt}}
\newcommand{\ucap}[0]{\hat{\Bu}}
\newcommand{\vcap}[0]{\hat{\Bv}}
\newcommand{\wcap}[0]{\hat{\Bw}}
\newcommand{\xcap}[0]{\hat{\Bx}}
\newcommand{\ycap}[0]{\hat{\By}}
\newcommand{\zcap}[0]{\hat{\Bz}}
\newcommand{\thetacap}[0]{\hat{\Btheta}}

%
% to write R^n and C^n in a distinguishable fashion.  Perhaps change this
% to the double lined characters upon figuring out how to do so.
%
\newcommand{\C}[1]{$\mathbb{C}^{#1}$}
\newcommand{\R}[1]{$\mathbb{R}^{#1}$}

%
% various generally useful helpers
%

% derivative of #1 wrt. #2:
\newcommand{\D}[2] {\frac {d#2} {d#1}}

\newcommand{\inv}[1]{\frac{1}{#1}}
\newcommand{\cross}[0]{\times}

\newcommand{\abs}[1]{\lvert{#1}\rvert}
\newcommand{\norm}[1]{\lVert{#1}\rVert}
\newcommand{\innerprod}[2]{\langle{#1}, {#2}\rangle}
\newcommand{\dotprod}[2]{{#1} \cdot {#2}}
\newcommand{\bdotprod}[2]{\left({#1} \cdot {#2}\right)}
\newcommand{\crossprod}[2]{{#1} \cross {#2}}
\newcommand{\tripleprod}[3]{\dotprod{\left(\crossprod{#1}{#2}\right)}{#3}}

\DeclareMathOperator{\Proj}{Proj}
\DeclareMathOperator{\Span}{span}
\DeclareMathOperator{\Sgn}{sgn}
\DeclareMathOperator{\Area}{Area}
\DeclareMathOperator{\Volume}{Volume}

%
% A few miscellaneous things specific to this document
%
\newcommand{\crossop}[1]{\crossprod{#1}{}}

% R2 vector.
\newcommand{\VectorTwo}[2]{
\begin{bmatrix}
 {#1} \\
 {#2}
\end{bmatrix}
}

\newcommand{\VectorN}[1]{
\begin{bmatrix}
{#1}_1 \\
{#1}_2 \\
\vdots \\
{#1}_N \\
\end{bmatrix}
}

\newcommand{\DETuvij}[4]{
\begin{vmatrix}
 {#1}_{#3} & {#1}_{#4} \\
 {#2}_{#3} & {#2}_{#4}
\end{vmatrix}
}

\newcommand{\DETuvwijk}[6]{
\begin{vmatrix}
 {#1}_{#4} & {#1}_{#5} & {#1}_{#6} \\
 {#2}_{#4} & {#2}_{#5} & {#2}_{#6} \\
 {#3}_{#4} & {#3}_{#5} & {#3}_{#6}
\end{vmatrix}
}

\newcommand{\DETuvwxijkl}[8]{
\begin{vmatrix}
 {#1}_{#5} & {#1}_{#6} & {#1}_{#7} & {#1}_{#8} \\
 {#2}_{#5} & {#2}_{#6} & {#2}_{#7} & {#2}_{#8} \\
 {#3}_{#5} & {#3}_{#6} & {#3}_{#7} & {#3}_{#8} \\
 {#4}_{#5} & {#4}_{#6} & {#4}_{#7} & {#4}_{#8} \\
\end{vmatrix}
}

%\newcommand{\DETuvwxyijklm}[10]{
%\begin{vmatrix}
% {#1}_{#6} & {#1}_{#7} & {#1}_{#8} & {#1}_{#9} & {#1}_{#10} \\
% {#2}_{#6} & {#2}_{#7} & {#2}_{#8} & {#2}_{#9} & {#2}_{#10} \\
% {#3}_{#6} & {#3}_{#7} & {#3}_{#8} & {#3}_{#9} & {#3}_{#10} \\
% {#4}_{#6} & {#4}_{#7} & {#4}_{#8} & {#4}_{#9} & {#4}_{#10} \\
% {#5}_{#6} & {#5}_{#7} & {#5}_{#8} & {#5}_{#9} & {#5}_{#10}
%\end{vmatrix}
%}

% R3 vector.
\newcommand{\VectorThree}[3]{
\begin{bmatrix}
 {#1} \\
 {#2} \\
 {#3}
\end{bmatrix}
}



\author{Peeter Joot}
\email{peeter.joot@gmail.com}


%\chapter{PHY456H1F: Quantum Mechanics II.  Lecture 12 (Taught by Mr. Federico Duque Gomez).  WKB Method}
%\chapter{WKB Method}
\index{WKB method}
\label{chap:qmTwoL12}
\blogpage{http://sites.google.com/site/peeterjoot/math2011/qmTwoL12.pdf}
%\date{Oct 19, 2011}




\section{WKB (Wentzel-Kramers-Brillouin) Method}
\index{Wentzel-Kramers-Brillouin method}

This is covered in \S 24 in the text \citep{desai2009quantum}.  Also \S 8 of \citep{griffiths2005introduction}.

We start with the 1D time independent Schr\"{o}dinger equation

\begin{equation}\label{eqn:qmTwoL12:10}
-\frac{\Hbar^2}{2m} \frac{d^2 U}{dx^2} + V(x) U(x) = E U(x)
\end{equation}

which we can write as

\begin{equation}\label{eqn:qmTwoL12:30}
\frac{d^2 U}{dx^2} + \frac{2m}{\Hbar^2} (E - V(x)) U(x) = 0
\end{equation}

Consider a finite well potential as in \cref{fig:qmTwoL13:qmTwoL12fig1}

\imageFigure{../../figures/phy456/qmTwoL12fig1}{Finite well potential}{fig:qmTwoL13:qmTwoL12fig1}{0.2}

With

\begin{equation}\label{eqn:qmTwoL12:90}
\begin{aligned}
k^2 &= \frac{2m (E - V)}{\Hbar},\qquad E > V \\
\kappa^2 &= \frac{2m (V - E)}{\Hbar}, \qquad V > E,
\end{aligned}
\end{equation}

we have for a bound state within the well

\begin{equation}\label{eqn:qmTwoL12:50}
U \propto e^{\pm i k x}
\end{equation}

and for that state outside the well

\begin{equation}\label{eqn:qmTwoL12:70}
U \propto e^{\pm \kappa x}
\end{equation}

In general we can hope for something similar.  Let us look for that something, but allow the constants \(k\) and \(\kappa\) to be functions of position

\begin{equation}\label{eqn:qmTwoL12:110}
\begin{aligned}
k^2(x) &= \frac{2m (E - V(x))}{\Hbar},\qquad E > V \\
\kappa^2(x) &= \frac{2m (V(x) - E)}{\Hbar}, \qquad V > E.
\end{aligned}
\end{equation}

In terms of \(k\) Schr\"{o}dinger's equation is just

\begin{equation}\label{eqn:qmTwoL12:130}
\frac{d^2 U(x)}{dx^2} + k^2(x) U(x) = 0.
\end{equation}

We use the trial solution

\begin{equation}\label{eqn:qmTwoL12:150}
U(x) = A e^{i \phi(x)},
\end{equation}

allowing \(\phi(x)\) to be complex

\begin{equation}\label{eqn:qmTwoL12:170}
\phi(x) = \phi_R(x) + i \phi_I(x).
\end{equation}

We need second derivatives

\begin{equation}\label{eqn:qmTwoL12:530}
\begin{aligned}
(e^{i \phi})''
&=
(i \phi' e^{i \phi})'  \\
&=
(i \phi')^2 e^{i \phi} + i \phi'' e^{i \phi},
\end{aligned}
\end{equation}

and plug back into our Schr\"{o}dinger equation to obtain

\begin{equation}\label{eqn:qmTwoL12:190}
- (\phi'(x))^2 + i \phi''(x) + k^2(x) = 0.
\end{equation}

For the first round of approximation we assume

\begin{equation}\label{eqn:qmTwoL12:210}
\phi''(x) \approx 0,
\end{equation}

and obtain

\begin{equation}\label{eqn:qmTwoL12:230}
(\phi'(x))^2 = k^2(x),
\end{equation}

or
\begin{equation}\label{eqn:qmTwoL12:250}
\phi'(x) = \pm k(x).
\end{equation}

A second round of approximation we use \eqnref{eqn:qmTwoL12:250} and obtain

\begin{equation}\label{eqn:qmTwoL12:250b}
\phi''(x) = \pm k'(x)
\end{equation}

Plugging back into \eqnref{eqn:qmTwoL12:190} we have

\begin{equation}\label{eqn:qmTwoL12:270}
-(\phi'(x))^2 \pm i k'(x) + k^2(x) = 0,
\end{equation}

Things get a little confusing here with the \(\pm\) variation since we have to take a second set of square roots, so let's consider these separately.

\paragraph{Case I.  positive root}

With \(\phi' = + k\), we have

\begin{equation}\label{eqn:qmTwoL12:410}
-(\phi'(x))^2 + i k'(x) + k^2(x) = 0,
\end{equation}

or

\begin{dmath}\label{eqn:qmTwoL12:430}
\phi'(x)
= \pm \sqrt{ + i k'(x) + k^2(x) } \\
= \pm k(x) \sqrt{ 1 + i \frac{k'(x)}{k^2(x)} } .
\end{dmath}

If \(k'\) is small compared to \(k^2\)

\begin{equation}\label{eqn:qmTwoL12:310}
\frac{k'(x)}{k^2(x)} \ll 1,
\end{equation}

then we have

\begin{equation}\label{eqn:qmTwoL12:330}
\phi'(x)
\approx \pm k(x) \lr{1 + i \frac{k'(x)}{2 k^2(x)}  }
= \pm
\lr{
k(x) + i \frac{k'(x)}{2 k(x)}
}
\end{equation}

Since we'd picked \(\phi' \approx +k\) in this case, we pick the positive sign, and can now integrate

\begin{dmath}\label{eqn:qmTwoL12:450}
\phi(x)
= \int dx k(x) + i \int dx \frac{k'(x)}{2 k(x)}  + \ln \text{const}
= \int dx k(x) + i \inv{2} \ln k(x) + \ln \text{const}
\end{dmath}

Going back to our wavefunction, for this \(E > V(x)\) case we have

\begin{equation}\label{eqn:qmTwoL12:550}
\begin{aligned}
U(x)
&\sim e^{i \phi(x)} \\
&= \exp \left(i\left(
\int dx k(x) + i \inv{2} \ln k(x) + \text{const}
\right)\right) \\
&\sim \exp \left(i\left(
\int dx k(x) + i \inv{2} \ln k(x)
\right)\right) \\
&= e^{i \int dx k(x)} e^{-\inv{2} \ln k(x)} \\
\end{aligned}
\end{equation}

or
\begin{equation}\label{eqn:qmTwoL12:350}
U(x) \propto \inv{\sqrt{k(x)}} e^{i \int dx k(x)}
\end{equation}

\paragraph{Case II: negative sign}

Now treat \(\phi' \approx -k\).  This gives us

\begin{equation}\label{eqn:qmTwoL12:330b}
\phi'(x)
\approx \pm k(x) \lr{1 - i \frac{k'(x)}{2 k^2(x)}  }
= \pm
\lr{
k(x) - i \frac{k'(x)}{2 k(x)}
}
\end{equation}

This time we want the negative root to match \(\phi' \approx -k\).  Integrating, we have

\begin{dmath}\label{eqn:qmTwoL12:470}
i \phi(x)
= -
i \int dx \lr{
k(x) - i \frac{k'(x)}{2 k(x)}
}
=
-i \int k(x) dx - \inv{2} \int \frac{k'}{k} dx
=
-i \int k(x) dx - \inv{2} \ln k + \ln \text{constant}
\end{dmath}

This gives us
\begin{equation}\label{eqn:qmTwoL12:490}
U(x) \propto \inv{\sqrt{k(x)}} e^{-i \int dx k(x)}
\end{equation}

Provided we have \eqnref{eqn:qmTwoL12:310}, we can summarize these as

\begin{equation}\label{eqn:qmTwoL12:510}
U(x) \propto \inv{\sqrt{k(x)}} e^{ \pm i \int dx k(x)}
\end{equation}

It's not hard to show that for the \(E < V(x)\) case we find

\begin{equation}\label{eqn:qmTwoL12:370}
U(x) \propto \inv{\sqrt{\kappa(x)}} e^{\pm \int dx \kappa(x)},
\end{equation}

this time, provided that our potential satisfies

\begin{equation}\label{eqn:qmTwoL12:310b}
\frac{\kappa'(x)}{\kappa^2(x)} \ll 1,
\end{equation}

\paragraph{Validity}
\begin{enumerate}
\item V(x) changes very slowly \(\implies k'(x)\) small, and \(k(x) = \sqrt{2 m (E - V(x))}/\Hbar\).
\item \(E\) very far away from the potential \(\Abs{(E - V(x))/V(x)} \gg 1\).
\end{enumerate}

\section{Turning points.}
\index{turning point}

%FIXME: fig2
\imageFigure{../../figures/phy456/qmTwoL12fig2}{Example of a general potential}{fig:qmTwoL13:qmTwoL12fig2}{0.2}
%\cref{fig:qmTwoL13:qmTwoL12fig2}

%FIXME: fig3
\imageFigure{../../figures/phy456/qmTwoL12fig3}{Turning points where WKB will not work}{fig:qmTwoL13:qmTwoL12fig3}{0.2}
%\cref{fig:qmTwoL13:qmTwoL12fig3}

\imageFigure{../../figures/phy456/qmTwoL12fig4}{Diagram for patching method discussion}{fig:qmTwoL13:qmTwoL12fig4}{0.2}
%\cref{fig:qmTwoL13:qmTwoL12fig4}

WKB will not work at the turning points in this figure since our main assumption was that

\begin{equation}\label{eqn:qmTwoL12:390}
\Abs{\frac{k'(x)}{k^2(x)}} \ll 1
\end{equation}

so we get into trouble where \(k(x) \sim 0\).  There are some methods for dealing with this.  Our text as well as Griffiths give some examples, but they require Bessel functions and more complex mathematics.

The idea is that one finds the WKB solution in the regions of validity, and then looks for a polynomial solution in the patching region where we are closer to the turning point, probably requiring lookup of various special functions.

This power series method is also outlined in \citep{wiki:wkb}, where solutions to connect the regions are expressed in terms of Airy functions.


