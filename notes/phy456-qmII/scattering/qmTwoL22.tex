%
% Copyright � 2012 Peeter Joot.  All Rights Reserved.
% Licenced as described in the file LICENSE under the root directory of this GIT repository.
%

% 
% 
%%
% Copyright � 2015 Peeter Joot.  All Rights Reserved.
% Licenced as described in the file LICENSE under the root directory of this GIT repository.
%
\documentclass[]{eliblog}

\usepackage{amsmath}
\usepackage{mathpazo}

%
% shorthand for bold symbols, convenient for vectors and matrices
%
\newcommand{\Ba}[0]{\mathbf{a}}
\newcommand{\Bb}[0]{\mathbf{b}}
\newcommand{\Bc}[0]{\mathbf{c}}
\newcommand{\Bd}[0]{\mathbf{d}}
\newcommand{\Be}[0]{\mathbf{e}}
\newcommand{\Bf}[0]{\mathbf{f}}
\newcommand{\Bg}[0]{\mathbf{g}}
\newcommand{\Bh}[0]{\mathbf{h}}
\newcommand{\Bi}[0]{\mathbf{i}}
\newcommand{\Bj}[0]{\mathbf{j}}
\newcommand{\Bk}[0]{\mathbf{k}}
\newcommand{\Bl}[0]{\mathbf{l}}
\newcommand{\Bm}[0]{\mathbf{m}}
\newcommand{\Bn}[0]{\mathbf{n}}
\newcommand{\Bo}[0]{\mathbf{o}}
\newcommand{\Bp}[0]{\mathbf{p}}
\newcommand{\Bq}[0]{\mathbf{q}}
\newcommand{\Br}[0]{\mathbf{r}}
\newcommand{\Bs}[0]{\mathbf{s}}
\newcommand{\Bt}[0]{\mathbf{t}}
\newcommand{\Bu}[0]{\mathbf{u}}
\newcommand{\Bv}[0]{\mathbf{v}}
\newcommand{\Bw}[0]{\mathbf{w}}
\newcommand{\Bx}[0]{\mathbf{x}}
\newcommand{\By}[0]{\mathbf{y}}
\newcommand{\Bz}[0]{\mathbf{z}}
\newcommand{\BA}[0]{\mathbf{A}}
\newcommand{\BB}[0]{\mathbf{B}}
\newcommand{\BC}[0]{\mathbf{C}}
\newcommand{\BD}[0]{\mathbf{D}}
\newcommand{\BE}[0]{\mathbf{E}}
\newcommand{\BF}[0]{\mathbf{F}}
\newcommand{\BG}[0]{\mathbf{G}}
\newcommand{\BH}[0]{\mathbf{H}}
\newcommand{\BI}[0]{\mathbf{I}}
\newcommand{\BJ}[0]{\mathbf{J}}
\newcommand{\BK}[0]{\mathbf{K}}
\newcommand{\BL}[0]{\mathbf{L}}
\newcommand{\BM}[0]{\mathbf{M}}
\newcommand{\BN}[0]{\mathbf{N}}
\newcommand{\BO}[0]{\mathbf{O}}
\newcommand{\BP}[0]{\mathbf{P}}
\newcommand{\BQ}[0]{\mathbf{Q}}
\newcommand{\BR}[0]{\mathbf{R}}
\newcommand{\BS}[0]{\mathbf{S}}
\newcommand{\BT}[0]{\mathbf{T}}
\newcommand{\BU}[0]{\mathbf{U}}
\newcommand{\BV}[0]{\mathbf{V}}
\newcommand{\BW}[0]{\mathbf{W}}
\newcommand{\BX}[0]{\mathbf{X}}
\newcommand{\BY}[0]{\mathbf{Y}}
\newcommand{\BZ}[0]{\mathbf{Z}}

\newcommand{\Bzero}[0]{\mathbf{0}}
\newcommand{\Btheta}[0]{\boldsymbol{\theta}}
\newcommand{\Btau}[0]{\boldsymbol{\tau}}
\newcommand{\Bomega}[0]{\boldsymbol{\omega}}

%
% shorthand for unit vectors
%
\newcommand{\acap}[0]{\hat{\Ba}}
\newcommand{\bcap}[0]{\hat{\Bb}}
\newcommand{\ccap}[0]{\hat{\Bc}}
\newcommand{\dcap}[0]{\hat{\Bd}}
\newcommand{\ecap}[0]{\hat{\Be}}
\newcommand{\fcap}[0]{\hat{\Bf}}
\newcommand{\gcap}[0]{\hat{\Bg}}
\newcommand{\hcap}[0]{\hat{\Bh}}
\newcommand{\icap}[0]{\hat{\Bi}}
\newcommand{\jcap}[0]{\hat{\Bj}}
\newcommand{\kcap}[0]{\hat{\Bk}}
\newcommand{\lcap}[0]{\hat{\Bl}}
\newcommand{\mcap}[0]{\hat{\Bm}}
\newcommand{\ncap}[0]{\hat{\Bn}}
\newcommand{\ocap}[0]{\hat{\Bo}}
\newcommand{\pcap}[0]{\hat{\Bp}}
\newcommand{\qcap}[0]{\hat{\Bq}}
\newcommand{\rcap}[0]{\hat{\Br}}
\newcommand{\scap}[0]{\hat{\Bs}}
\newcommand{\tcap}[0]{\hat{\Bt}}
\newcommand{\ucap}[0]{\hat{\Bu}}
\newcommand{\vcap}[0]{\hat{\Bv}}
\newcommand{\wcap}[0]{\hat{\Bw}}
\newcommand{\xcap}[0]{\hat{\Bx}}
\newcommand{\ycap}[0]{\hat{\By}}
\newcommand{\zcap}[0]{\hat{\Bz}}
\newcommand{\thetacap}[0]{\hat{\Btheta}}

%
% to write R^n and C^n in a distinguishable fashion.  Perhaps change this
% to the double lined characters upon figuring out how to do so.
%
\newcommand{\C}[1]{$\mathbb{C}^{#1}$}
\newcommand{\R}[1]{$\mathbb{R}^{#1}$}

%
% various generally useful helpers
%

% derivative of #1 wrt. #2:
\newcommand{\D}[2] {\frac {d#2} {d#1}}

\newcommand{\inv}[1]{\frac{1}{#1}}
\newcommand{\cross}[0]{\times}

\newcommand{\abs}[1]{\lvert{#1}\rvert}
\newcommand{\norm}[1]{\lVert{#1}\rVert}
\newcommand{\innerprod}[2]{\langle{#1}, {#2}\rangle}
\newcommand{\dotprod}[2]{{#1} \cdot {#2}}
\newcommand{\bdotprod}[2]{\left({#1} \cdot {#2}\right)}
\newcommand{\crossprod}[2]{{#1} \cross {#2}}
\newcommand{\tripleprod}[3]{\dotprod{\left(\crossprod{#1}{#2}\right)}{#3}}

\DeclareMathOperator{\Proj}{Proj}
\DeclareMathOperator{\Span}{span}
\DeclareMathOperator{\Sgn}{sgn}
\DeclareMathOperator{\Area}{Area}
\DeclareMathOperator{\Volume}{Volume}

%
% A few miscellaneous things specific to this document
%
\newcommand{\crossop}[1]{\crossprod{#1}{}}

% R2 vector.
\newcommand{\VectorTwo}[2]{
\begin{bmatrix}
 {#1} \\
 {#2}
\end{bmatrix}
}

\newcommand{\VectorN}[1]{
\begin{bmatrix}
{#1}_1 \\
{#1}_2 \\
\vdots \\
{#1}_N \\
\end{bmatrix}
}

\newcommand{\DETuvij}[4]{
\begin{vmatrix}
 {#1}_{#3} & {#1}_{#4} \\
 {#2}_{#3} & {#2}_{#4}
\end{vmatrix}
}

\newcommand{\DETuvwijk}[6]{
\begin{vmatrix}
 {#1}_{#4} & {#1}_{#5} & {#1}_{#6} \\
 {#2}_{#4} & {#2}_{#5} & {#2}_{#6} \\
 {#3}_{#4} & {#3}_{#5} & {#3}_{#6}
\end{vmatrix}
}

\newcommand{\DETuvwxijkl}[8]{
\begin{vmatrix}
 {#1}_{#5} & {#1}_{#6} & {#1}_{#7} & {#1}_{#8} \\
 {#2}_{#5} & {#2}_{#6} & {#2}_{#7} & {#2}_{#8} \\
 {#3}_{#5} & {#3}_{#6} & {#3}_{#7} & {#3}_{#8} \\
 {#4}_{#5} & {#4}_{#6} & {#4}_{#7} & {#4}_{#8} \\
\end{vmatrix}
}

%\newcommand{\DETuvwxyijklm}[10]{
%\begin{vmatrix}
% {#1}_{#6} & {#1}_{#7} & {#1}_{#8} & {#1}_{#9} & {#1}_{#10} \\
% {#2}_{#6} & {#2}_{#7} & {#2}_{#8} & {#2}_{#9} & {#2}_{#10} \\
% {#3}_{#6} & {#3}_{#7} & {#3}_{#8} & {#3}_{#9} & {#3}_{#10} \\
% {#4}_{#6} & {#4}_{#7} & {#4}_{#8} & {#4}_{#9} & {#4}_{#10} \\
% {#5}_{#6} & {#5}_{#7} & {#5}_{#8} & {#5}_{#9} & {#5}_{#10}
%\end{vmatrix}
%}

% R3 vector.
\newcommand{\VectorThree}[3]{
\begin{bmatrix}
 {#1} \\
 {#2} \\
 {#3}
\end{bmatrix}
}



\author{Peeter Joot}
\email{peeter.joot@gmail.com}

%\documentclass[]{eliblogwidescreen}

\usepackage{amsmath}
\usepackage{mathpazo}

%
% shorthand for bold symbols, convenient for vectors and matrices
%
\newcommand{\Ba}[0]{\mathbf{a}}
\newcommand{\Bb}[0]{\mathbf{b}}
\newcommand{\Bc}[0]{\mathbf{c}}
\newcommand{\Bd}[0]{\mathbf{d}}
\newcommand{\Be}[0]{\mathbf{e}}
\newcommand{\Bf}[0]{\mathbf{f}}
\newcommand{\Bg}[0]{\mathbf{g}}
\newcommand{\Bh}[0]{\mathbf{h}}
\newcommand{\Bi}[0]{\mathbf{i}}
\newcommand{\Bj}[0]{\mathbf{j}}
\newcommand{\Bk}[0]{\mathbf{k}}
\newcommand{\Bl}[0]{\mathbf{l}}
\newcommand{\Bm}[0]{\mathbf{m}}
\newcommand{\Bn}[0]{\mathbf{n}}
\newcommand{\Bo}[0]{\mathbf{o}}
\newcommand{\Bp}[0]{\mathbf{p}}
\newcommand{\Bq}[0]{\mathbf{q}}
\newcommand{\Br}[0]{\mathbf{r}}
\newcommand{\Bs}[0]{\mathbf{s}}
\newcommand{\Bt}[0]{\mathbf{t}}
\newcommand{\Bu}[0]{\mathbf{u}}
\newcommand{\Bv}[0]{\mathbf{v}}
\newcommand{\Bw}[0]{\mathbf{w}}
\newcommand{\Bx}[0]{\mathbf{x}}
\newcommand{\By}[0]{\mathbf{y}}
\newcommand{\Bz}[0]{\mathbf{z}}
\newcommand{\BA}[0]{\mathbf{A}}
\newcommand{\BB}[0]{\mathbf{B}}
\newcommand{\BC}[0]{\mathbf{C}}
\newcommand{\BD}[0]{\mathbf{D}}
\newcommand{\BE}[0]{\mathbf{E}}
\newcommand{\BF}[0]{\mathbf{F}}
\newcommand{\BG}[0]{\mathbf{G}}
\newcommand{\BH}[0]{\mathbf{H}}
\newcommand{\BI}[0]{\mathbf{I}}
\newcommand{\BJ}[0]{\mathbf{J}}
\newcommand{\BK}[0]{\mathbf{K}}
\newcommand{\BL}[0]{\mathbf{L}}
\newcommand{\BM}[0]{\mathbf{M}}
\newcommand{\BN}[0]{\mathbf{N}}
\newcommand{\BO}[0]{\mathbf{O}}
\newcommand{\BP}[0]{\mathbf{P}}
\newcommand{\BQ}[0]{\mathbf{Q}}
\newcommand{\BR}[0]{\mathbf{R}}
\newcommand{\BS}[0]{\mathbf{S}}
\newcommand{\BT}[0]{\mathbf{T}}
\newcommand{\BU}[0]{\mathbf{U}}
\newcommand{\BV}[0]{\mathbf{V}}
\newcommand{\BW}[0]{\mathbf{W}}
\newcommand{\BX}[0]{\mathbf{X}}
\newcommand{\BY}[0]{\mathbf{Y}}
\newcommand{\BZ}[0]{\mathbf{Z}}

\newcommand{\Bzero}[0]{\mathbf{0}}
\newcommand{\Btheta}[0]{\boldsymbol{\theta}}
\newcommand{\Btau}[0]{\boldsymbol{\tau}}
\newcommand{\Bomega}[0]{\boldsymbol{\omega}}

%
% shorthand for unit vectors
%
\newcommand{\acap}[0]{\hat{\Ba}}
\newcommand{\bcap}[0]{\hat{\Bb}}
\newcommand{\ccap}[0]{\hat{\Bc}}
\newcommand{\dcap}[0]{\hat{\Bd}}
\newcommand{\ecap}[0]{\hat{\Be}}
\newcommand{\fcap}[0]{\hat{\Bf}}
\newcommand{\gcap}[0]{\hat{\Bg}}
\newcommand{\hcap}[0]{\hat{\Bh}}
\newcommand{\icap}[0]{\hat{\Bi}}
\newcommand{\jcap}[0]{\hat{\Bj}}
\newcommand{\kcap}[0]{\hat{\Bk}}
\newcommand{\lcap}[0]{\hat{\Bl}}
\newcommand{\mcap}[0]{\hat{\Bm}}
\newcommand{\ncap}[0]{\hat{\Bn}}
\newcommand{\ocap}[0]{\hat{\Bo}}
\newcommand{\pcap}[0]{\hat{\Bp}}
\newcommand{\qcap}[0]{\hat{\Bq}}
\newcommand{\rcap}[0]{\hat{\Br}}
\newcommand{\scap}[0]{\hat{\Bs}}
\newcommand{\tcap}[0]{\hat{\Bt}}
\newcommand{\ucap}[0]{\hat{\Bu}}
\newcommand{\vcap}[0]{\hat{\Bv}}
\newcommand{\wcap}[0]{\hat{\Bw}}
\newcommand{\xcap}[0]{\hat{\Bx}}
\newcommand{\ycap}[0]{\hat{\By}}
\newcommand{\zcap}[0]{\hat{\Bz}}
\newcommand{\thetacap}[0]{\hat{\Btheta}}

%
% to write R^n and C^n in a distinguishable fashion.  Perhaps change this
% to the double lined characters upon figuring out how to do so.
%
\newcommand{\C}[1]{$\mathbb{C}^{#1}$}
\newcommand{\R}[1]{$\mathbb{R}^{#1}$}

%
% various generally useful helpers
%

% derivative of #1 wrt. #2:
\newcommand{\D}[2] {\frac {d#2} {d#1}}

\newcommand{\inv}[1]{\frac{1}{#1}}
\newcommand{\cross}[0]{\times}

\newcommand{\abs}[1]{\lvert{#1}\rvert}
\newcommand{\norm}[1]{\lVert{#1}\rVert}
\newcommand{\innerprod}[2]{\langle{#1}, {#2}\rangle}
\newcommand{\dotprod}[2]{{#1} \cdot {#2}}
\newcommand{\bdotprod}[2]{\left({#1} \cdot {#2}\right)}
\newcommand{\crossprod}[2]{{#1} \cross {#2}}
\newcommand{\tripleprod}[3]{\dotprod{\left(\crossprod{#1}{#2}\right)}{#3}}

\DeclareMathOperator{\Proj}{Proj}
\DeclareMathOperator{\Span}{span}
\DeclareMathOperator{\Sgn}{sgn}
\DeclareMathOperator{\Area}{Area}
\DeclareMathOperator{\Volume}{Volume}

%
% A few miscellaneous things specific to this document
%
\newcommand{\crossop}[1]{\crossprod{#1}{}}

% R2 vector.
\newcommand{\VectorTwo}[2]{
\begin{bmatrix}
 {#1} \\
 {#2}
\end{bmatrix}
}

\newcommand{\VectorN}[1]{
\begin{bmatrix}
{#1}_1 \\
{#1}_2 \\
\vdots \\
{#1}_N \\
\end{bmatrix}
}

\newcommand{\DETuvij}[4]{
\begin{vmatrix}
 {#1}_{#3} & {#1}_{#4} \\
 {#2}_{#3} & {#2}_{#4}
\end{vmatrix}
}

\newcommand{\DETuvwijk}[6]{
\begin{vmatrix}
 {#1}_{#4} & {#1}_{#5} & {#1}_{#6} \\
 {#2}_{#4} & {#2}_{#5} & {#2}_{#6} \\
 {#3}_{#4} & {#3}_{#5} & {#3}_{#6}
\end{vmatrix}
}

\newcommand{\DETuvwxijkl}[8]{
\begin{vmatrix}
 {#1}_{#5} & {#1}_{#6} & {#1}_{#7} & {#1}_{#8} \\
 {#2}_{#5} & {#2}_{#6} & {#2}_{#7} & {#2}_{#8} \\
 {#3}_{#5} & {#3}_{#6} & {#3}_{#7} & {#3}_{#8} \\
 {#4}_{#5} & {#4}_{#6} & {#4}_{#7} & {#4}_{#8} \\
\end{vmatrix}
}

%\newcommand{\DETuvwxyijklm}[10]{
%\begin{vmatrix}
% {#1}_{#6} & {#1}_{#7} & {#1}_{#8} & {#1}_{#9} & {#1}_{#10} \\
% {#2}_{#6} & {#2}_{#7} & {#2}_{#8} & {#2}_{#9} & {#2}_{#10} \\
% {#3}_{#6} & {#3}_{#7} & {#3}_{#8} & {#3}_{#9} & {#3}_{#10} \\
% {#4}_{#6} & {#4}_{#7} & {#4}_{#8} & {#4}_{#9} & {#4}_{#10} \\
% {#5}_{#6} & {#5}_{#7} & {#5}_{#8} & {#5}_{#9} & {#5}_{#10}
%\end{vmatrix}
%}

% R3 vector.
\newcommand{\VectorThree}[3]{
\begin{bmatrix}
 {#1} \\
 {#2} \\
 {#3}
\end{bmatrix}
}



\author{Peeter Joot}
\email{peeter.joot@gmail.com}


%\chapter{PHY456H1F: Quantum Mechanics II.  Lecture 22 (Taught by Prof J.E. Sipe).  Scattering (cont.)}
%\chapter{Scattering (cont.)}
\index{scattering}
\label{chap:qmTwoL22}

\blogpage{http://sites.google.com/site/peeterjoot2/math2011/qmTwoL22.pdf}
%\date{Nov 28, 2011}

%\section{Scattering.  Recap}
\section{Recap}

READING: \S 19, \S 20 of the text \citep{desai2009quantum}.

We used a positive potential of the form of \cref{fig:qmTwoL22:qmTwoL22fig1}
\imageFigure{../../figures/phy456/qmTwoL22fig1}{A bounded positive potential}{fig:qmTwoL22:qmTwoL22fig1}{0.2}

\begin{equation}\label{eqn:qmTwoL22:10}
-\frac{\Hbar^2}{2 \mu} \PDSq{x}{\psi_k(x)} + V(x) \psi_k(x) = \frac{\Hbar^2 k^2}{2 \mu}
\end{equation}

for \(x \ge x_3\)

\begin{equation}\label{eqn:qmTwoL22:30}
\psi_k(x) = C e^{i k x}
\end{equation}

\begin{equation}\label{eqn:qmTwoL22:50}
\phi_k(x) = \ddx{\psi_k(x)}
\end{equation}

for \(x \ge x_3\)

\begin{equation}\label{eqn:qmTwoL22:70}
\phi_k(x) = i k C e^{i k x}
\end{equation}

\begin{equation}\label{eqn:qmTwoL22:90}
\begin{aligned}
\ddx{\psi_k(x)} &= \phi_k(x) \\
-\frac{\Hbar^2}{2 \mu} \ddx{\phi_k(x)} &= - V(x) \psi_k(x) + \frac{\Hbar^2 k^2}{2 \mu}
\end{aligned}
\end{equation}

integrate these equations back to \(x_1\).

For \(x \le x_1\)

\begin{equation}\label{eqn:qmTwoL22:110}
\psi_k(x) = A e^{i k x} + B e^{-i k x},
\end{equation}

where both \(A\) and \(B\) are proportional to \(C\), dependent on \(k\).

There are cases where we can solve this analytically (one of these is on our problem set).

Alternatively, write as (so long as \(A \ne 0\))

\begin{equation}\label{eqn:qmTwoL22:130}
\begin{array}{l l l}
\psi_k(x)
&\rightarrow e^{i k x} + \beta_k e^{-i k x} & \quad \mbox{for \(x < x_1\)} \\
&\rightarrow \gamma_k e^{i k x} & \quad \mbox{for \(x > x_2\)}
\end{array}
\end{equation}

Now want to consider the problem of no potential in the interval of interest, and our window bounded potential as in \cref{fig:qmTwoL22:qmTwoL22fig3}

\imageFigure{../../figures/phy456/qmTwoL22fig3}{Wave packet in free space and with positive potential}{fig:qmTwoL22:qmTwoL22fig3}{0.2}

where we model our particle as a wave packet as we found can have the fourier transform description, for \(t_{\text{initial}} < 0\), of

\begin{equation}\label{eqn:qmTwoL22:150}
\psi(x, t_{\text{initial}}) = \int \frac{dk}{\sqrt{2 \pi}} \alpha(k, t_{\text{initial}}) e^{i k x}
\end{equation}

Returning to the same coefficients, the solution of the Schr\"{o}dinger eqn for problem with the potential \eqnref{eqn:qmTwoL22:130}

For \(x \le x_1\),

\begin{equation}\label{eqn:qmTwoL22:190}
\psi(x, t) = \psi_i(x, t) + \psi_r(x, t)
\end{equation}

where as illustrated in \cref{fig:qmTwoL22:qmTwoL22fig4}
\imageFigure{../../figures/phy456/qmTwoL22fig4}{Reflection and transmission of wave packet}{fig:qmTwoL22:qmTwoL22fig4}{0.2}

\begin{equation}\label{eqn:qmTwoL22:210}
\begin{aligned}
\psi_i(x, t) &= \int \frac{dk}{\sqrt{2 \pi}} \alpha(k, t_{\text{initial}}) e^{i k x} \\
\psi_r(x, t) &= \int \frac{dk}{\sqrt{2 \pi}} \alpha(k, t_{\text{initial}}) \beta_k e^{-i k x}.
\end{aligned}
\end{equation}

For \(x > x_2\)

\begin{equation}\label{eqn:qmTwoL22:250}
\psi(x, t) = \psi_t(x, t)
\end{equation}

and

\begin{equation}\label{eqn:qmTwoL22:230}
\psi_t(x, t) = \int \frac{dk}{\sqrt{2 \pi}} \alpha(k, t_{\text{initial}}) \gamma_k e^{i k x}
\end{equation}

Look at

\begin{equation}\label{eqn:qmTwoL22:270}
\psi_r(x, t) = \chi(-x, t)
\end{equation}

where

\begin{equation}\label{eqn:qmTwoL22:290}
\begin{aligned}
\chi(x, t)
&= \int \frac{dk}{\sqrt{2 \pi}} \alpha(k, t_{\text{initial}}) \beta_k e^{i k x} \\
&\approx
\beta_{k_0} \int \frac{dk}{\sqrt{2 \pi}} \alpha(k, t_{\text{initial}}) e^{i k x}
\end{aligned}
\end{equation}

for \(t = t_{\text{initial}}\), this is nonzero for \(x < x_1\).

so for \(x < x_1\)

\begin{equation}\label{eqn:qmTwoL22:310}
\psi_r(x, t_{\text{initial}}) = 0
\end{equation}

In the same way, for \(x > x_2\)

\begin{equation}\label{eqn:qmTwoL22:330}
\psi_t(x, t_{\text{initial}}) = 0.
\end{equation}

What has not been proved is that the wavefunction is also zero in the \([x_1, x_2]\) interval.

\paragraph{Summarizing}

For \(t = t_{\text{initial}}\)

\begin{equation}\label{eqn:qmTwoL22:350}
\psi(x, t_{\text{initial}})
=
\left\{
\begin{array}{l l}
\int \frac{dk}{\sqrt{2 \pi}} \alpha(k, t_{\text{initial}}) e^{i k x} &\quad \mbox{for \(x < x_1\)} \\
0 & \quad \mbox{for \(x > x_2\) (and actually also for \(x > x_1\) (unproven))}
\end{array}
\right.
\end{equation}

for \(t = t_{\text{final}}\)

\begin{equation}\label{eqn:qmTwoL22:370}
\psi(x, t_{\text{final}})
\rightarrow
\left\{
\begin{array}{l l}
\int \frac{dk}{\sqrt{2 \pi}} \beta_k \alpha(k, t_{\text{final}}) e^{-i k x} &\quad \mbox{for \(x < x_1\)} \\
0 & \quad \mbox{ \(x \in [x_1, x_2]\)} \\
\int \frac{dk}{\sqrt{2 \pi}} \gamma_k \alpha(k, t_{\text{final}}) e^{i k x}
 & \quad \mbox{ for \(x > x_2\) }
\end{array}
\right.
\end{equation}

Probability of reflection is

\begin{equation}\label{eqn:qmTwoL22:390}
\int \Abs{\psi_r(x, t_{\text{final}})}^2 dx
\end{equation}

If we have a sufficiently localized packet, we can form a first order approximation around the peak of \(\beta_k\) (FIXME: or is this a sufficiently localized responce to the potential on reflection?)

\begin{equation}\label{eqn:qmTwoL22:410}
\psi_r(x, t_{\text{final}}) \approx \beta_{k_0}
\int \frac{dk}{\sqrt{2 \pi}} \alpha(k, t_{\text{final}}) e^{-i k x},
\end{equation}

so
\begin{equation}\label{eqn:qmTwoL22:430}
\int \Abs{\psi_r(x, t_{\text{final}})}^2 dx
\approx \Abs{\beta_{k_0}}^2 \equiv R
\end{equation}

Probability of transmission is

\begin{equation}\label{eqn:qmTwoL22:450}
\int \Abs{\psi_t(x, t_{\text{final}})}^2 dx
\end{equation}

Again, assuming a small spread in \(\gamma_k\), with \(\gamma_k \approx \gamma_{k_0}\) for some \(k_0\)

\begin{equation}\label{eqn:qmTwoL22:470}
\psi_t(x, t_{\text{final}}) \approx \gamma_{k_0}
\int \frac{dk}{\sqrt{2 \pi}} \alpha(k, t_{\text{final}}) e^{i k x},
\end{equation}

we have for \(x > x_2\)

\begin{equation}\label{eqn:qmTwoL22:490}
\int \Abs{\psi_t(x, t_{\text{final}})}^2 dx
\approx \Abs{\gamma_{k_0}}^2 \equiv T.
\end{equation}

By constructing the wave packets in this fashion we get as a side effect the solution of the scattering problem.

The

\begin{equation}\label{eqn:qmTwoL22:510}
\begin{aligned}
\psi_k(x) \rightarrow & e^{i k x} + \beta_k e^{-i k x} \\
& \gamma_k e^{i k x}
\end{aligned}
\end{equation}

are called asymptotic in states.  Their physical applicability is only once we have built wave packets out of them.

