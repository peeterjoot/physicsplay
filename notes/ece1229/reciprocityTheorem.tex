%
% Copyright � 2015 Peeter Joot.  All Rights Reserved.
% Licenced as described in the file LICENSE under the root directory of this GIT repository.
%
\newcommand{\authorname}{Peeter Joot}
\newcommand{\email}{peeterjoot@protonmail.com}
\newcommand{\basename}{FIXMEbasenameUndefined}
\newcommand{\dirname}{notes/FIXMEdirnameUndefined/}

\renewcommand{\basename}{reciprocityTheorem}
\renewcommand{\dirname}{notes/FIXMEwheretodirname/}
%\newcommand{\dateintitle}{}
%\newcommand{\keywords}{}

\newcommand{\authorname}{Peeter Joot}
\newcommand{\onlineurl}{http://sites.google.com/site/peeterjoot2/math2013/\basename.pdf}
\newcommand{\sourcepath}{\dirname\basename.tex}
\newcommand{\generatetitle}[1]{\chapter{#1}}

\newcommand{\vcsinfo}{%
\section*{}
\noindent{\color{DarkOliveGreen}{\rule{\linewidth}{0.1mm}}}
\paragraph{Document version}
%\paragraph{\color{Maroon}{Document version}}
{
\small
\begin{itemize}
\item Available online at:\\ 
\href{\onlineurl}{\onlineurl}
\item Git Repository: \input{./.revinfo/gitRepo.tex}
\item Source: \sourcepath
\item last commit: \input{./.revinfo/gitCommitString.tex}
\item commit date: \input{./.revinfo/gitCommitDate.tex}
\end{itemize}
}
}

%\PassOptionsToPackage{dvipsnames,svgnames}{xcolor}
\PassOptionsToPackage{square,numbers}{natbib}
\documentclass{scrreprt}

\usepackage[left=2cm,right=2cm]{geometry}
\usepackage[svgnames]{xcolor}
\usepackage{peeters_layout}

\usepackage{natbib}

\usepackage[
colorlinks=true,
bookmarks=false,
pdfauthor={\authorname, \email},
backref 
]{hyperref}

% http://tex.stackexchange.com/questions/75773/how-to-reference-problems-by-the-text-label-in-an-exercise-envioronment
\usepackage[english]{cleveref}
\crefname{Exercise}{exercise}{exercises}
\Crefname{Exercise}{Exercise}{Exercises}

\RequirePackage{titlesec}
\RequirePackage{ifthen}

% http://stackoverflow.com/questions/4932910/date-in-the-tabular-environment
\makeatletter
\let\insertdate\@date
\makeatother

\titleformat{\chapter}[display]
{\bfseries\Large}
{\color{DarkSlateGrey}\filleft \authorname
\ifthenelse{\isundefined{\studentnumber}}{}{\\ \studentnumber}
\ifthenelse{\isundefined{\email}}{}{\\ \email}
\ifthenelse{\isundefined{\dateintitle}}{}{\\ \insertdate}
%\ifthenelse{\isundefined{\coursename}}{}{\\ \coursename} % put in title instead.
}
{4ex}
{\color{DarkOliveGreen}{\titlerule}\color{Maroon}
\vspace{2ex}%
\filright}
[\vspace{2ex}%
\color{DarkOliveGreen}\titlerule
]

\newcommand{\beginArtWithToc}[0]{\begin{document}\tableofcontents}
\newcommand{\beginArtNoToc}[0]{\begin{document}}
\newcommand{\EndNoBibArticle}[0]{\end{document}}
\newcommand{\EndArticle}[0]{\bibliography{Bibliography}\bibliographystyle{plainnat}\end{document}}

% 
%\newcommand{\citep}[1]{\cite{#1}}

\colorSectionsForArticle



\beginArtNoToc

\generatetitle{Reciprocity theorem}
%\chapter{Reciprocity theorem}
%\label{chap:reciprocityTheorem}
%\section{Motivation}
%\section{Guts}

Maxwell's equations in phasor form were
%\cref{eqn:chapter3Notes:99}

\begin{subequations}
\label{eqn:reciprocityTheorem:99}
\begin{dmath}\label{eqn:reciprocityTheorem:100}
\spacegrad \cross \BE = - \BM - j \omega \BB
\end{dmath}
\begin{dmath}\label{eqn:reciprocityTheorem:120}
\spacegrad \cross \BH = \BJ + j \omega \BD
\end{dmath}
\begin{dmath}\label{eqn:reciprocityTheorem:140}
\spacegrad \cdot \BD = \rho
\end{dmath}
\begin{dmath}\label{eqn:reciprocityTheorem:160}
\spacegrad \cdot \BB = \rho_m.
\end{dmath}
\end{subequations}

Suppose that the same system (like a two antenna system) is fed with two different current configurations, (a) and (b).  Maxwell's equations for each such configuration are

\begin{subequations}
\label{eqn:reciprocityTheorem:99}
\begin{dmath}\label{eqn:reciprocityTheorem:100}
\spacegrad \cross \BE^a = - \BM^a - j \omega \BB^a
\end{dmath}
\begin{dmath}\label{eqn:reciprocityTheorem:120}
\spacegrad \cross \BH^a = \BJ^a + j \omega \BD^a
\end{dmath}
\begin{dmath}\label{eqn:reciprocityTheorem:140}
\spacegrad \cdot \BD^a = \rho^a
\end{dmath}
\begin{dmath}\label{eqn:reciprocityTheorem:160}
\spacegrad \cdot \BB^a = \rho_m^a,
\end{dmath}
\end{subequations}

and

\begin{subequations}
\label{eqn:reciprocityTheorem:n}
\begin{dmath}\label{eqn:reciprocityTheorem:n}
\spacegrad \cross \BE^b = - \BM^b - j \omega \BB^b
\end{dmath}
\begin{dmath}\label{eqn:reciprocityTheorem:n}
\spacegrad \cross \BH^b = \BJ^b + j \omega \BD^b
\end{dmath}
\begin{dmath}\label{eqn:reciprocityTheorem:n}
\spacegrad \cdot \BD^b = \rho^b
\end{dmath}
\begin{dmath}\label{eqn:reciprocityTheorem:n}
\spacegrad \cdot \BB^b = \rho_m^b.
\end{dmath}
\end{subequations}

It is possible to relate the sources and fields for these two sets of Maxwell's equations.  Forming triple products with the fields one can write

\begin{subequations}
\begin{equation}\label{eqn:reciprocityTheorem:n}
\BH^b \cdot \lr{ \spacegrad \cross \BE^a } = - \BH^\b \cdot \BM^a - j \omega \BH^\b \cdot \BB^a
\end{equation}
\begin{equation}\label{eqn:reciprocityTheorem:n}
\BE^b \cdot \lr{ \spacegrad \cross \BH^a } = \BE^\b \cdot \BJ^a + j \omega \BE^\b \cdot \BD^a
\end{equation}
\begin{equation}\label{eqn:reciprocityTheorem:n}
\BH^a \cdot \lr{ \spacegrad \cross \BE^b } = - \BH^a \cdot \BM^b - j \omega \BH^a \cdot \BB^b
\end{equation}
\begin{equation}\label{eqn:reciprocityTheorem:n}
\BE^a \cdot \lr{ \spacegrad \cross \BH^b } = \BE^a \cdot \BJ^b + j \omega \BE^a \cdot \BD^b
\end{equation}
\end{subequations}

% this is to produce the sites.google url and version info and so forth (for blog posts)
%\vcsinfo
%\EndArticle
\EndNoBibArticle
