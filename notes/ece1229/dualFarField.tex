%
% Copyright � 2015 Peeter Joot.  All Rights Reserved.
% Licenced as described in the file LICENSE under the root directory of this GIT repository.
%
%\newcommand{\authorname}{Peeter Joot}
\newcommand{\email}{peeterjoot@protonmail.com}
\newcommand{\basename}{FIXMEbasenameUndefined}
\newcommand{\dirname}{notes/FIXMEdirnameUndefined/}

%\renewcommand{\basename}{dualFarField}
%\renewcommand{\dirname}{notes/ece1229/}
%%\newcommand{\dateintitle}{}
%%\newcommand{\keywords}{}
%
%\newcommand{\authorname}{Peeter Joot}
\newcommand{\onlineurl}{http://sites.google.com/site/peeterjoot2/math2013/\basename.pdf}
\newcommand{\sourcepath}{\dirname\basename.tex}
\newcommand{\generatetitle}[1]{\chapter{#1}}

\newcommand{\vcsinfo}{%
\section*{}
\noindent{\color{DarkOliveGreen}{\rule{\linewidth}{0.1mm}}}
\paragraph{Document version}
%\paragraph{\color{Maroon}{Document version}}
{
\small
\begin{itemize}
\item Available online at:\\ 
\href{\onlineurl}{\onlineurl}
\item Git Repository: \input{./.revinfo/gitRepo.tex}
\item Source: \sourcepath
\item last commit: \input{./.revinfo/gitCommitString.tex}
\item commit date: \input{./.revinfo/gitCommitDate.tex}
\end{itemize}
}
}

%\PassOptionsToPackage{dvipsnames,svgnames}{xcolor}
\PassOptionsToPackage{square,numbers}{natbib}
\documentclass{scrreprt}

\usepackage[left=2cm,right=2cm]{geometry}
\usepackage[svgnames]{xcolor}
\usepackage{peeters_layout}

\usepackage{natbib}

\usepackage[
colorlinks=true,
bookmarks=false,
pdfauthor={\authorname, \email},
backref 
]{hyperref}

% http://tex.stackexchange.com/questions/75773/how-to-reference-problems-by-the-text-label-in-an-exercise-envioronment
\usepackage[english]{cleveref}
\crefname{Exercise}{exercise}{exercises}
\Crefname{Exercise}{Exercise}{Exercises}

\RequirePackage{titlesec}
\RequirePackage{ifthen}

% http://stackoverflow.com/questions/4932910/date-in-the-tabular-environment
\makeatletter
\let\insertdate\@date
\makeatother

\titleformat{\chapter}[display]
{\bfseries\Large}
{\color{DarkSlateGrey}\filleft \authorname
\ifthenelse{\isundefined{\studentnumber}}{}{\\ \studentnumber}
\ifthenelse{\isundefined{\email}}{}{\\ \email}
\ifthenelse{\isundefined{\dateintitle}}{}{\\ \insertdate}
%\ifthenelse{\isundefined{\coursename}}{}{\\ \coursename} % put in title instead.
}
{4ex}
{\color{DarkOliveGreen}{\titlerule}\color{Maroon}
\vspace{2ex}%
\filright}
[\vspace{2ex}%
\color{DarkOliveGreen}\titlerule
]

\newcommand{\beginArtWithToc}[0]{\begin{document}\tableofcontents}
\newcommand{\beginArtNoToc}[0]{\begin{document}}
\newcommand{\EndNoBibArticle}[0]{\end{document}}
\newcommand{\EndArticle}[0]{\bibliography{Bibliography}\bibliographystyle{plainnat}\end{document}}

% 
%\newcommand{\citep}[1]{\cite{#1}}

\colorSectionsForArticle


%
%\beginArtNoToc
%
%\generatetitle{Duality transformation of the far field fields.}
\section{Duality transformation of the far field fields.}
\index{duality transformation}
\index{far field}
%\label{chap:dualFarField}

We've seen that the far field electric and magnetic fields associated with a magnetic vector potential were

\begin{subequations}
\label{eqn:dualFarField:20}
\begin{dmath}\label{eqn:dualFarField:40}
\BE = -j \omega \Proj_\T \BA,
\end{dmath}
\begin{dmath}\label{eqn:dualFarField:60}
\BH = \inv{\eta} \kcap \cross \BE.
\end{dmath}
\end{subequations}

What does \( \BH \) look like in terms of \( \BA \)?  Expanding the rejection of the radial component answers that

\begin{dmath}\label{eqn:dualFarField:140}
\BH 
= -\frac{j \omega}{\eta} \kcap \cross \lr{ \BA - \lr{\BA \cdot \kcap} \kcap }.
\end{dmath}

The \( \kcap \) crossed terms are killed, leaving

\begin{dmath}\label{eqn:dualFarField:160}
\BH 
= -\frac{j \omega}{\eta} \kcap \cross \BA.
\end{dmath}

It's worth a quick note that the duality transformation for this, referring to \citep{balanis2005antenna} \texttabref{3.2}, is

\begin{subequations}
\label{eqn:dualFarField:80}
\begin{dmath}\label{eqn:dualFarField:100}
\BH = -j \omega \Proj_\T \BF
\end{dmath}
\begin{equation}\label{eqn:dualFarField:120}
\BE = \eta \kcap \cross \BH = j \omega \eta \kcap \cross \BF.
\end{equation}
\end{subequations}

These show explicitly that neither the electric or magnetic far field have any radial component, matching with intuition for transverse propagation of the fields.

%\EndArticle
