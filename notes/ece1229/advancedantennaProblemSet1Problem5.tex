%
% Copyright � 2015 Peeter Joot.  All Rights Reserved.
% Licenced as described in the file LICENSE under the root directory of this GIT repository.
%
\makeproblem{Transmission power determination.}{advancedantenna:problemSet1:5}{ 
A repeater link consists of a transmitter and a receiver at 10 \si{GHz} in a line-of-sight
arrangement of distance 10\si{km}. The transmitting and receiving antennas are identical
horns with gain over isotropic equal to 15 \si{dB}. For acceptable signal-to-noise ratio, the
power received must be greater than 10 \si{nW}. Loss due to polarization mismatch is not
expected to exceed 3 \si{dB}. Determine the minimum transmitted power that should be used.
} % makeproblem

\makeanswer{advancedantenna:problemSet1:5}{ 

This is another Friis equation application.  Each of the respective gains (converted from \si{dB}) are

\begin{dmath}\label{eqn:advancedantennaProblemSet1Problem5:n}
G = 10^{15/10} \, \si{W}.
\end{dmath}

The polarization loss factor is

\begin{dmath}\label{eqn:advancedantennaProblemSet1Problem5:40}
\Abs{\rhocap_\txtr \cdot \rhocap_\txtt}^2 \le 10^{-3/10}.
\end{dmath}

The wavelength is

\begin{dmath}\label{eqn:advancedantennaProblemSet1Problem5:60}
\lambda = \frac{3 \times 10^8 \,\si{m/s}}{10^{10} \,\si{s^{-1}}} = 3 \times 10^{-2} \,\si{m}.
\end{dmath}

Put together we are looking for a value of \( P_\txtt \) at least that of

\begin{dmath}\label{eqn:advancedantennaProblemSet1Problem5:80}
\frac{P_\txtr}{P_\txtt} 
= \frac{10^{-8}\,\si{W}}{P_\txtt} 
= \lr{\frac{0.03 \,\si{m}}{4 \pi 10^4 \,\si{m}}}^2 \lr{10^{3/2}}^2 10^{-0.3}
= 2.9 \times 10^{-11},
\end{dmath}

or

\boxedEquation{eqn:advancedantennaProblemSet1Problem5:100}{
%\begin{dmath}\label{eqn:advancedantennaProblemSet1Problem5:100}
P_\txtt \ge 350 \,\si{W}.
%\end{dmath}
}

}

