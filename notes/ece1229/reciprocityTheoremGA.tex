%
% Copyright � 2015 Peeter Joot.  All Rights Reserved.
% Licenced as described in the file LICENSE under the root directory of this GIT repository.
%
%\newcommand{\authorname}{Peeter Joot}
\newcommand{\email}{peeterjoot@protonmail.com}
\newcommand{\basename}{FIXMEbasenameUndefined}
\newcommand{\dirname}{notes/FIXMEdirnameUndefined/}

%\renewcommand{\basename}{reciprocityTheoremGA}
%\renewcommand{\dirname}{notes/ece1229/}
%%\newcommand{\dateintitle}{}
%%\newcommand{\keywords}{}
%
%\newcommand{\authorname}{Peeter Joot}
\newcommand{\onlineurl}{http://sites.google.com/site/peeterjoot2/math2013/\basename.pdf}
\newcommand{\sourcepath}{\dirname\basename.tex}
\newcommand{\generatetitle}[1]{\chapter{#1}}

\newcommand{\vcsinfo}{%
\section*{}
\noindent{\color{DarkOliveGreen}{\rule{\linewidth}{0.1mm}}}
\paragraph{Document version}
%\paragraph{\color{Maroon}{Document version}}
{
\small
\begin{itemize}
\item Available online at:\\ 
\href{\onlineurl}{\onlineurl}
\item Git Repository: \input{./.revinfo/gitRepo.tex}
\item Source: \sourcepath
\item last commit: \input{./.revinfo/gitCommitString.tex}
\item commit date: \input{./.revinfo/gitCommitDate.tex}
\end{itemize}
}
}

%\PassOptionsToPackage{dvipsnames,svgnames}{xcolor}
\PassOptionsToPackage{square,numbers}{natbib}
\documentclass{scrreprt}

\usepackage[left=2cm,right=2cm]{geometry}
\usepackage[svgnames]{xcolor}
\usepackage{peeters_layout}

\usepackage{natbib}

\usepackage[
colorlinks=true,
bookmarks=false,
pdfauthor={\authorname, \email},
backref 
]{hyperref}

% http://tex.stackexchange.com/questions/75773/how-to-reference-problems-by-the-text-label-in-an-exercise-envioronment
\usepackage[english]{cleveref}
\crefname{Exercise}{exercise}{exercises}
\Crefname{Exercise}{Exercise}{Exercises}

\RequirePackage{titlesec}
\RequirePackage{ifthen}

% http://stackoverflow.com/questions/4932910/date-in-the-tabular-environment
\makeatletter
\let\insertdate\@date
\makeatother

\titleformat{\chapter}[display]
{\bfseries\Large}
{\color{DarkSlateGrey}\filleft \authorname
\ifthenelse{\isundefined{\studentnumber}}{}{\\ \studentnumber}
\ifthenelse{\isundefined{\email}}{}{\\ \email}
\ifthenelse{\isundefined{\dateintitle}}{}{\\ \insertdate}
%\ifthenelse{\isundefined{\coursename}}{}{\\ \coursename} % put in title instead.
}
{4ex}
{\color{DarkOliveGreen}{\titlerule}\color{Maroon}
\vspace{2ex}%
\filright}
[\vspace{2ex}%
\color{DarkOliveGreen}\titlerule
]

\newcommand{\beginArtWithToc}[0]{\begin{document}\tableofcontents}
\newcommand{\beginArtNoToc}[0]{\begin{document}}
\newcommand{\EndNoBibArticle}[0]{\end{document}}
\newcommand{\EndArticle}[0]{\bibliography{Bibliography}\bibliographystyle{plainnat}\end{document}}

% 
%\newcommand{\citep}[1]{\cite{#1}}

\colorSectionsForArticle


%
%\beginArtNoToc
%
%\generatetitle{Reciprocity theorem in Geometric Algebra}
%\chapter{Reciprocity theorem}
%\label{chap:reciprocityTheoremGA}

The reciprocity theorem involves a Poynting like antisymmetric difference of the following form

\begin{equation}\label{eqn:reciprocityTheoremGA:20}
\BE^{(a)} \cross \BH^{(b)} - \BE^{(b)} \cross \BH^{(a)}.
\end{equation}

This smells like something that can probably be related to a combined electromagnetic field multivectors in some sort of structured fashion.  Guessing that this is related to the antisymmetric sum of two electromagnetic field multivectors turns out to be correct.  Let

\begin{subequations}
\label{eqn:reciprocityTheoremGA:40}
\begin{dmath}\label{eqn:reciprocityTheoremGA:60}
F^{(a)} = \BE^{(a)} + c \BB^{(a)} I
\end{dmath}
\begin{dmath}\label{eqn:reciprocityTheoremGA:80}
F^{(b)} = \BE^{(b)} + c \BB^{(b)} I.
\end{dmath}
\end{subequations}

Now form the antisymmetric sum

\begin{dmath}\label{eqn:reciprocityTheoremGA:100}
\inv{2} \lr{ F^{(a)} F^{(b)} - F^{(b)} F^{(a)} }
=
\inv{2} \lr{\BE^{(a)} + c \BB^{(a)} I }
\lr{\BE^{(b)} + c \BB^{(b)} I }
-
\inv{2} \lr{\BE^{(b)} + c \BB^{(b)} I }
\lr{\BE^{(a)} + c \BB^{(a)} I }
=
\inv{2} \lr{ \BE^{(a)} \BE^{(b)} -\BE^{(b)} \BE^{(a)} }
+ \frac{I c}{2} \lr{ \BE^{(a)} \BB^{(b)} - \BB^{(b)} \BE^{(a)} }
+ \frac{I c}{2} \lr{ \BB^{(a)} \BE^{(b)} - \BE^{(b)} \BB^{(a)} }
+ \frac{c^2}{2} \lr{ \BB^{(b)} \BB^{(a)} - \BB^{(a)} \BB^{(b)} }
=
\BE^{(a)} \wedge \BE^{(b)} + c^2 \lr{ \BB^{(b)} \wedge \BB^{(a)} }
+ I c \lr{ 
   \BE^{(a)} \wedge \BB^{(b)} 
+
   \BB^{(a)} \wedge \BE^{(b)} 
}
=
I \BE^{(a)} \cross \BE^{(b)} + c^2 I \lr{ \BB^{(b)} \cross \BB^{(a)} }
-
c \lr{ 
   \BE^{(a)} \cross \BB^{(b)} 
+
   \BB^{(a)} \cross \BE^{(b)} 
}
\end{dmath}

This has two components, the first is a bivector (pseudoscalar times vector) that includes all the non-mixed products, and the second is a vector that includes all the mixed terms.  We can therefore write the antisymmetric difference of the reciprocity theorem by extracting just the grade one terms of the antisymmetric sum of the combined electromagnetic field

\begin{dmath}\label{eqn:reciprocityTheoremGA:120}
\BE^{(a)} \cross \BH^{(b)} - \BE^{(b)} \cross \BH^{(a)}
=
-\frac{1}{2 c \mu_0} \gpgradeone{ \lr{ F^{(a)} F^{(b)} - F^{(b)} F^{(a)} } }.
\end{dmath}

Observing that the antisymmetrization used in the reciprocity theorem is only one portion of the larger electromagnetic field antisymmetrization, introduces two new questions

\begin{enumerate}
\item How would the reciprocity theorem be derived directly in terms of \( F^{(a)} F^{(b)} - F^{(b)} F^{(a)} \)?
\item What is the significance of the other portion of this antisymmetrization \( \BE^{(a)} \cross \BE^{(b)} - c^2 \mu_0^2 \lr{ \BH^{(a)} \cross \BH^{(b)} } \) ?
\end{enumerate}

%\EndNoBibArticle
