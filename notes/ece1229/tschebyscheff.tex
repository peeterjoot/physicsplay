%
% Copyright � 2015 Peeter Joot.  All Rights Reserved.
% Licenced as described in the file LICENSE under the root directory of this GIT repository.
%
%\newcommand{\authorname}{Peeter Joot}
\newcommand{\email}{peeterjoot@protonmail.com}
\newcommand{\basename}{FIXMEbasenameUndefined}
\newcommand{\dirname}{notes/FIXMEdirnameUndefined/}

%\renewcommand{\basename}{tschebyscheff}
%\renewcommand{\dirname}{notes/ece1229/}
%%\newcommand{\dateintitle}{}
%%\newcommand{\keywords}{}
%
%\newcommand{\authorname}{Peeter Joot}
\newcommand{\onlineurl}{http://sites.google.com/site/peeterjoot2/math2013/\basename.pdf}
\newcommand{\sourcepath}{\dirname\basename.tex}
\newcommand{\generatetitle}[1]{\chapter{#1}}

\newcommand{\vcsinfo}{%
\section*{}
\noindent{\color{DarkOliveGreen}{\rule{\linewidth}{0.1mm}}}
\paragraph{Document version}
%\paragraph{\color{Maroon}{Document version}}
{
\small
\begin{itemize}
\item Available online at:\\ 
\href{\onlineurl}{\onlineurl}
\item Git Repository: \input{./.revinfo/gitRepo.tex}
\item Source: \sourcepath
\item last commit: \input{./.revinfo/gitCommitString.tex}
\item commit date: \input{./.revinfo/gitCommitDate.tex}
\end{itemize}
}
}

%\PassOptionsToPackage{dvipsnames,svgnames}{xcolor}
\PassOptionsToPackage{square,numbers}{natbib}
\documentclass{scrreprt}

\usepackage[left=2cm,right=2cm]{geometry}
\usepackage[svgnames]{xcolor}
\usepackage{peeters_layout}

\usepackage{natbib}

\usepackage[
colorlinks=true,
bookmarks=false,
pdfauthor={\authorname, \email},
backref 
]{hyperref}

% http://tex.stackexchange.com/questions/75773/how-to-reference-problems-by-the-text-label-in-an-exercise-envioronment
\usepackage[english]{cleveref}
\crefname{Exercise}{exercise}{exercises}
\Crefname{Exercise}{Exercise}{Exercises}

\RequirePackage{titlesec}
\RequirePackage{ifthen}

% http://stackoverflow.com/questions/4932910/date-in-the-tabular-environment
\makeatletter
\let\insertdate\@date
\makeatother

\titleformat{\chapter}[display]
{\bfseries\Large}
{\color{DarkSlateGrey}\filleft \authorname
\ifthenelse{\isundefined{\studentnumber}}{}{\\ \studentnumber}
\ifthenelse{\isundefined{\email}}{}{\\ \email}
\ifthenelse{\isundefined{\dateintitle}}{}{\\ \insertdate}
%\ifthenelse{\isundefined{\coursename}}{}{\\ \coursename} % put in title instead.
}
{4ex}
{\color{DarkOliveGreen}{\titlerule}\color{Maroon}
\vspace{2ex}%
\filright}
[\vspace{2ex}%
\color{DarkOliveGreen}\titlerule
]

\newcommand{\beginArtWithToc}[0]{\begin{document}\tableofcontents}
\newcommand{\beginArtNoToc}[0]{\begin{document}}
\newcommand{\EndNoBibArticle}[0]{\end{document}}
\newcommand{\EndArticle}[0]{\bibliography{Bibliography}\bibliographystyle{plainnat}\end{document}}

% 
%\newcommand{\citep}[1]{\cite{#1}}

\colorSectionsForArticle


%
%\usepackage{peeters_layout_exercise}
%\usepackage{macros_qed}
%
%\beginArtNoToc
%
%\generatetitle{Chebyscheff polynomials}
\section{Chebyscheff polynomials}
\index{Chebyscheff polynomial}
\index{antenna array}
\index{Tschebyscheff polynomial}
%\label{chap:tschebyscheff}

In ancient times (i.e. 2nd year undergrad) I recall being very impressed with Chebyscheff polynomials for designing lowpass filters.  I'd used Chebyscheff filters for the hardware we used for a speech recognition system our group built in the design lab.  One of the benefits of these polynomials is that the oscillation in the \( \Abs{x} < 1 \) interval is strictly bounded.   This same property, as well as the unbounded nature outside of the \( [-1,1] \) interval turns out to have applications to antenna array design.

The Chebyscheff polynomials are defined by

\begin{subequations}
\label{eqn:chebyscheff:20}
\begin{equation}\label{eqn:chebyscheff:40}
T_m(x) = \cos\lr{ m \cos^{-1} x }, \quad \Abs{x} < 1
\end{equation}
\begin{equation}\label{eqn:chebyscheff:60}
T_m(x) = \cosh\lr{ m \cosh^{-1} x }, \quad \Abs{x} > 1.
\end{equation}
\end{subequations}

\paragraph{Range restrictions and hyperbolic form.}

Prof. Eleftheriades's notes made a point to point out the definition in the \( \Abs{x} > 1 \) interval, but that can also be viewed as a consequence instead of a definition if the range restriction is removed.  For example, suppose \( x = 7 \), and let

\begin{equation}\label{eqn:chebyscheff:160}
\cos^{-1} 7 = \theta,
\end{equation}

so
\begin{dmath}\label{eqn:chebyscheff:180}
7 
= \cos\theta
= \frac{e^{j\theta} + e^{-j\theta}}{2}
= \cosh(\pm j\theta),
\end{dmath}

or

\begin{equation}\label{eqn:chebyscheff:200}
\mp j \cosh^{-1} 7 = \theta.
\end{equation}

\begin{dmath}\label{eqn:chebyscheff:220}
T_m(7) 
= \cos( \mp m j \cosh^{-1} 7 )
= \cosh( m \cosh^{-1} 7 ).
\end{dmath}

The same argument clearly applies to any other value outside of the \( \Abs{x} < 1 \) range, so without any restrictions, these polynomials can be defined as just

%\begin{equation}\label{eqn:chebyscheff:260}
\boxedEquation{eqn:chebyscheff:260}{
T_m(x) = \cos\lr{ m \cos^{-1} x }.
}
%\end{equation}

\paragraph{Polynomial nature.}

\Cref{eqn:chebyscheff:260} does not obviously look like a polynomial.  Let's proceed to verify the polynomial nature for the first couple values of \( m \).

\begin{itemize}
\item \( m = 0 \).  

\begin{dmath}\label{eqn:chebyscheff:280}
T_0(x) 
= \cos( 0 \cos^{-1} x ) 
= \cos( 0 )
= 1.
\end{dmath}

\item \( m = 1 \).  

\begin{dmath}\label{eqn:chebyscheff:300}
T_1(x) 
= \cos( 1 \cos^{-1} x ) 
= x.
\end{dmath}

\item \( m = 2 \).  

\begin{dmath}\label{eqn:chebyscheff:320}
T_2(x) 
= \cos( 2 \cos^{-1} x ) 
= 2 \cos^2 \cos^{-1}(x) - 1
= 2 x^2 - 1.
\end{dmath}
\end{itemize}

To examine the general case

\begin{dmath}\label{eqn:chebyscheff:340}
T_m(x) 
= \cos( m \cos^{-1} x ) 
= \Real e^{ j m \cos^{-1} x }
= \Real \lr{ e^{ j\cos^{-1} x } }^m
= \Real \lr{ \cos\cos^{-1} x + j \sin\cos^{-1} x }^m
= \Real \lr{ x + j \sqrt{1 - x^2} }^m
= 
\Real \lr{
x^m 
+ \binom{ m}{1} j x^{m-1} \lr{1 - x^2}^{1/2}
- \binom{ m}{2} x^{m-2} \lr{1 - x^2}^{2/2}
- \binom{ m}{3} j x^{m-3} \lr{1 - x^2}^{3/2}
+ \binom{ m}{4} x^{m-4} \lr{1 - x^2}^{4/2}
+ \cdots
}
=
x^m 
- \binom{ m}{2} x^{m-2} \lr{1 - x^2}
+ \binom{ m}{4} x^{m-4} \lr{1 - x^2}^2
- \cdots
\end{dmath}

This expansion was a bit cavalier with the signs of the \( \sin\cos^{-1} x = \sqrt{1 - x^2} \) terms, since the negative sign should be picked for the root when \( x \in [-1,0] \).  However, that doesn't matter in the end since the real part operation selects only powers of two of this root.

The final result of the expansion above can be written

%\begin{dmath}\label{eqn:chebyscheff:360}
\boxedEquation{eqn:chebyscheff:360}{
T_m(x) = \sum_{k = 0}^{\largestIntLessThan{m/2}} \binom{m}{2 k} (-1)^k x^{m - 2 k} \lr{1 - x^2}^k.
}
%\end{dmath}

This clearly shows the polynomial nature of these functions, and is also perfectly well defined for any value of \( x \).  The even and odd alternation with \( m \) is also clear in this explicit expansion.

\paragraph{Some plots}

The first couple polynomials are plotted in \cref{fig:Chebychev:ChebychevFig1}.

\mathImageFigure{../../figures/ece1229/ChebychevFig1}{A couple Cheybshev plots.}{fig:Chebychev:ChebychevFig1}{0.3}{chebychevPlots.nb}

\paragraph{Properties}
\index{Cheybshev polynomials!products}

In \citep{abramowitz1964handbook} a few properties can be found for these polynomials

\begin{subequations}
\label{eqn:chebyscheff:380}
\begin{equation}\label{eqn:chebyscheff:100}
T_m(x) = 2 x T_{m-1} - T_{m-2}
\end{equation}
\begin{dmath}\label{eqn:chebyscheff:420}
0 = \lr{ 1 - x^2 } \frac{d T_m(x)}{dx} + m x T_m(x) - m T_{m-1}(x)
\end{dmath}
\begin{dmath}\label{eqn:chebyscheff:400}
0 = \lr{ 1 - x^2 } \frac{d^2 T_m(x)}{dx^2} - x \frac{dT_m(x)}{dx} + m^2 T_{m}(x)
\end{dmath}
\begin{dmath}\label{eqn:chebyscheff:440}
\int_{-1}^1 \inv{ \sqrt{1 - x^2} } T_m(x) T_n(x) dx = 
\left\{
\begin{array}{l l}
0 & \quad \mbox{if \( m \ne n \) } \\
\pi & \quad \mbox{if \( m = n = 0 \) }  \\
\pi/2 & \quad \mbox{if \( m = n, m \ne 0 \) } 
\end{array}
\right.
\end{dmath}
\end{subequations}

%\EndArticle
