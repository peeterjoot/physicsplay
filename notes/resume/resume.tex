%
% Copyright � 2013 Peeter Joot.  All Rights Reserved.
% Licenced as described in the file LICENSE under the root directory of this GIT repository.
%
\input{../latex/blogpost.tex}
\renewcommand{\basename}{WHAT}
\renewcommand{\dirname}{notes/resume/}
%\newcommand{\dateintitle}{}
%\newcommand{\keywords}{}

\newcommand{\phonenumber}{(416) 305-9441}
\newcommand{\address}{8 Juniper Cres.  Markham, Ontario, L3R 3Z7}

\input{../latex/peeter_prologue_print2.tex}

\beginArtNoToc

\generatetitle{Curriculum Vitae}
%\chapter{Curriculum Vitae}
%\label{chap:WHAT}

\paragraph{2016-present.  LzLabs Canada.}

Project manager, developer and test lead for LzLabs compiled language project.

\begin{itemize}
\item LzLabs point of contact for Raincode compiler front end and runtime development.  Implemented prototypes of many features (EXEC SQL, EXEC CICS, Vsam support, legacy loadmodule calling convention, ...) that provided Raincode with enough information to implement the same features in the core compiler.
\item Implemented mainframe flavoured C runtime utilizing a heavily modified FreeBSD C runtime, and internal LzLabs APIs.  This provided a subset of C as described in the `z/OS V2R2 XL C/C++ Runtime Library Reference', including EBCDIC support, Qsam/Vsam DATASET support, hex floating point, ...
\item Augmented the LzLabs C runtime with all the runtime interfaces required for the (Raincode) PL/I and COBOL compiler front-ends (ongoing).  This provides the glue layer that Raincode needs for their compiler front ends to implement a development environment that appears to the customer as a 32-bit big-endian environment running on a mainframe (despite running Linux x86\_64 code).
\item Implemented gdb DW\_endianity extensions to support Dwarf attributes that Raincode is applying to datatypes like FIXED BIN(31), and FIXED BIN(15).
\item Implemented LLVM IR byte swapping pass.  bswap instructions are inserted around loads and stores to let little endian code behave as if it was big-endian.
\item Customized clang compiler for LzLabs with extensions that implement mainframe features (various pragmas, langlvl mappings, ...)
\end{itemize}

\paragraph{1997-2016. IBM Toronto Lab.}

Software development for DB2, including

\begin{itemize}
\item Implemented multithread support for DB2's Unix client.
\item DB2 Linux porting work and simultaneous reengineering of the Unix portions of our codebase for portability.
\item DB2 64-bit port. This was a project of massive scope, taking our codebase from it�s 16-bit and 32-bit origins to its current multi-terrabyte consuming state. If you have the RAM we can now eat it.
\item Development and maintenance of DB2's internal mutex and atomic implementations, and associated memory ordering mechanisms.
\item Maintenance and implementation of the platform portions of stack collection and other post mortem facilities.
\item Lock-free reimplementation of DB2's reader-writer mutex. The performance of our original reader-writer mutex code sucked for a number of reasons (including but not restricted to an exclusive mutex used internally). This was a from scratch implementation where we used a single atomic to track the reader and writer linked lists (indirectly), the reader count, the writer held bit, and a writer reserved bit. This code has stood the test of time and remains one of my proudest products, not only not sucking for performance, but remaining correct despite the massive complexities and range of potential timing holes if done wrong.
\item Implementation of DB2's asynchronous IO abstraction layer. It is hard to believe that it was not that long ago that we did all our IO synchronously. This bit of code hides an impressive amount of operating system centric code from our high level development consumers, while squeezing maximum performance from each system.
\item Development and maintenance of many other aspects of DB2's operating system abstraction layer.
\item Lead of project branch integration team during internal transition from CMVC VCS to clearcase.
\item Ad-hoc build tooling and makefile maintenance as required.
\item Technical owner of DB2's coding standards
\item DB2 pureScale project (distributed shared disk database): Implemented various duplexing, failover and reconstruct aspects of the communications between the DB2 engine and the shared buffer pool and lock manager component.
\end{itemize}

Work history prior to 1997 was primarily in construction trades.

\paragraph{Education}\mbox{ } \\

\begin{itemize}
\item BASc. Engineering Science ; Computer Engineering, University of Toronto
Conferred 1997
\item Selected physics courses. University of Toronto.
2010 - 2013. Quantum Physics I, Relativistic Electrodynamics, Quantum Physics II, Continuum Mechanics, Advanced Classical Optics, Basic Statistical Mechanics, Condensed Matter Physics
\item UofT M.Eng (ECE electromagnetics group.)
2014-present.
\end{itemize}

\paragraph{Writing}\mbox{ } \\

\begin{itemize}
\item Book (2010)
\href{http://peeterjoot.com/archives/math2015/gabookI.pdf}{Exploring physics with Geometric Algebra, Book I.}
\href{http://peeterjoot.com/archives/math2015/gabookII.pdf}{Exploring physics with Geometric Algebra, Book II.}

\item Book (2010)
\href{http://peeterjoot.com/archives/math2009/miscphysics.pdf}{Miscellaneous Physics and Math Play}

\item Book (2011)
\href{http://peeterjoot.com/archives/math2013/classicalmechanics.pdf}{Classical mechanics}

\item Course notes and problems from University of Toronto 2010 (taught by Prof Vatche Deyirmenjian)
\href{http://peeterjoot.com/archives/math2010/phy356.pdf}{PHY356H1F Quantum Physics I}

\item Course notes and problems from University of Toronto 2011 (taught by Prof Erich Poppitz)
\href{http://peeterjoot.com/archives/math2011/phy450.pdf}{PHY450H1S Relativistic Electrodynamics}

\item Course notes and problems from University of Toronto 2011 (taught by Prof John E. Sipe)
\href{http://peeterjoot.com/archives/math2011/phy456.pdf}{PHY456H1F Quantum Physics II}

\item Course notes and problems from University of Toronto 2012 (taught by Prof Kausik S. Das)
\href{http://peeterjoot.com/archives/math2012/phy454.pdf}{PHY454H1S Continuum Mechanics}

\item Course notes and problems from University of Toronto 2012 (taught by Prof Joseph H. Thywissen)
\href{http://peeterjoot.com/archives/math2012/phy485.pdf}{PHY485H1F Advanced Classical Optics}

\item Course notes and problems from University of Toronto 2013 (taught by Prof Arun Paramekanti)
\href{http://peeterjoot.com/archives/math2013/phy452.pdf}{PHY452H1S Basic Statistical Mechanics}

\item Course notes and problems from University of Toronto 2013 (taught by Prof Stephen Julian)
\href{http://peeterjoot.com/archives/math2013/phy487.pdf}{PHY487H1F Condensed Matter Physics}

\item Course notes and problems from University of Toronto 2014 (taught by Prof. Piero Triverio)
\href{http://peeterjoot.com/archives/unredacted/ece1254.pdf}{ECE1254H1F Modeling of Multiphysics Systems}

\item Course notes and problems from University of Toronto 2015 (taught by Prof. Arun Paramekanti)
\href{http://peeterjoot.com/archives/unredacted/phy1520.pdf}{PHY1520H Graduate Quantum Mechanics}

\item Course notes and problems from University of Toronto 2015 (taught by Prof. G.V. Eleftheriades)
\href{http://peeterjoot.com/archives/math2015/ece1229.pdf}{ECE1229H1S Advanced Antenna Theory}

\item Course notes and problems from University of Toronto 2016 (taught by Prof. M. Mojahedi)
\href{http://peeterjoot.com/archives/unredacted/ece1228.pdf}{ECE1228H Electromagnetic Theory}

\item Book (2018)
\href{http://peeterjoot.com/writing/geometric-algebra-for-electrical-engineers/}{Geometric Algebra for Electrical Engineers.}

\item \href{http://peeterjoot.com/archives/math2009/atomic.pdf}{An attempt to illustrate differences between memory ordering and atomic access}

\item
\href{http://peeterjoot.com/archives/math2010/atomicSimple.pdf}{A nice simple example of a memory barrier requirement}

\item
\href{http://sites.google.com/site/peeterjoot/math2010/mutexRelease.pdf}{Timings and correctness issues for mutex release operations}

\item
\href{http://arxiv.org/abs/1104.4829}{(arxiv) Change of basis and Gram-Schmidt orthonormalization in special relativity}

\item
\href{http://peeterjoot.com/archives/math2009/relwave.pdf}{Relativistic origins of the Schrodinger equation}

\item
\href{http://peeterjoot.com/archives/math2009/multiPendulumSpherical2.pdf}{Spherical polar pendulum for one and multiple masses (using geometric algebra)}

\item \href{http://peeterjoot.com/archives/math2009/multiPendulumSphericalMatrix.pdf}{Lagrangian and Euler-Lagrange equation evaluation for the spherical N-pendulum problem}

\end{itemize}

%See \href{http://peeterjoot.com/writing}{http://peeterjoot.com/writing} for the URLs above if this document has been printed.

%\EndArticle
\EndNoBibArticle
