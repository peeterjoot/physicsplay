%
% Copyright � 2016 Peeter Joot.  All Rights Reserved.
% Licenced as described in the file LICENSE under the root directory of this GIT repository.
%
%{
\newcommand{\authorname}{Peeter Joot}
\newcommand{\email}{peeterjoot@protonmail.com}
\newcommand{\basename}{FIXMEbasenameUndefined}
\newcommand{\dirname}{notes/FIXMEdirnameUndefined/}

\renewcommand{\basename}{atrig1}
%\renewcommand{\dirname}{notes/phy1520/}
\renewcommand{\dirname}{notes/ece1228-electromagnetic-theory/}
%\newcommand{\dateintitle}{}
%\newcommand{\keywords}{}

\newcommand{\authorname}{Peeter Joot}
\newcommand{\onlineurl}{http://sites.google.com/site/peeterjoot2/math2013/\basename.pdf}
\newcommand{\sourcepath}{\dirname\basename.tex}
\newcommand{\generatetitle}[1]{\chapter{#1}}

\newcommand{\vcsinfo}{%
\section*{}
\noindent{\color{DarkOliveGreen}{\rule{\linewidth}{0.1mm}}}
\paragraph{Document version}
%\paragraph{\color{Maroon}{Document version}}
{
\small
\begin{itemize}
\item Available online at:\\ 
\href{\onlineurl}{\onlineurl}
\item Git Repository: \input{./.revinfo/gitRepo.tex}
\item Source: \sourcepath
\item last commit: \input{./.revinfo/gitCommitString.tex}
\item commit date: \input{./.revinfo/gitCommitDate.tex}
\end{itemize}
}
}

%\PassOptionsToPackage{dvipsnames,svgnames}{xcolor}
\PassOptionsToPackage{square,numbers}{natbib}
\documentclass{scrreprt}

\usepackage[left=2cm,right=2cm]{geometry}
\usepackage[svgnames]{xcolor}
\usepackage{peeters_layout}

\usepackage{natbib}

\usepackage[
colorlinks=true,
bookmarks=false,
pdfauthor={\authorname, \email},
backref 
]{hyperref}

% http://tex.stackexchange.com/questions/75773/how-to-reference-problems-by-the-text-label-in-an-exercise-envioronment
\usepackage[english]{cleveref}
\crefname{Exercise}{exercise}{exercises}
\Crefname{Exercise}{Exercise}{Exercises}

\RequirePackage{titlesec}
\RequirePackage{ifthen}

% http://stackoverflow.com/questions/4932910/date-in-the-tabular-environment
\makeatletter
\let\insertdate\@date
\makeatother

\titleformat{\chapter}[display]
{\bfseries\Large}
{\color{DarkSlateGrey}\filleft \authorname
\ifthenelse{\isundefined{\studentnumber}}{}{\\ \studentnumber}
\ifthenelse{\isundefined{\email}}{}{\\ \email}
\ifthenelse{\isundefined{\dateintitle}}{}{\\ \insertdate}
%\ifthenelse{\isundefined{\coursename}}{}{\\ \coursename} % put in title instead.
}
{4ex}
{\color{DarkOliveGreen}{\titlerule}\color{Maroon}
\vspace{2ex}%
\filright}
[\vspace{2ex}%
\color{DarkOliveGreen}\titlerule
]

\newcommand{\beginArtWithToc}[0]{\begin{document}\tableofcontents}
\newcommand{\beginArtNoToc}[0]{\begin{document}}
\newcommand{\EndNoBibArticle}[0]{\end{document}}
\newcommand{\EndArticle}[0]{\bibliography{Bibliography}\bibliographystyle{plainnat}\end{document}}

% 
%\newcommand{\citep}[1]{\cite{#1}}

\colorSectionsForArticle



\usepackage{peeters_layout_exercise}
\usepackage{peeters_braket}
\usepackage{peeters_figures}
\usepackage{siunitx}
%\usepackage{mhchem} % \ce{}
%\usepackage{macros_bm} % \bcM
%\usepackage{txfonts} % \ointclockwise

\beginArtNoToc

%\generatetitle{XXX}
%\chapter{XXX}
%\label{chap:atrig1}

Hi Aurora,

I presume you know the two following double angle formulas

\begin{dmath}\label{eqn:atrig1:20}
\sin( 2 y ) = 2 \sin y \cos y
\end{dmath}
\begin{dmath}\label{eqn:atrig1:40}
\cos( 2 y ) = \cos^2 y - \sin^2 y = 2 \cos^2 y - 1 = 1 - 2 \sin^2 y
\end{dmath}

If you substitute \( 2 y = x \), or \( y = x/2 \), the last equation becomes

\begin{dmath}\label{eqn:atrig1:60}
\cos x = 2 \cos^2 \lr{ \frac{x}{2} } - 1,
\end{dmath}

which you wrote down in your notebook.  You can solve this for \( \cos(x/2) \) by first adding 1 to both sides:

\begin{dmath}\label{eqn:atrig1:80}
1 + \cos x = 2 \cos^2 \lr{ \frac{x}{2} },
\end{dmath}

then dividing everything by 2:
\begin{dmath}\label{eqn:atrig1:100}
\frac{1 + \cos x}{2} = \cos^2 \lr{ \frac{x}{2} },
\end{dmath}

and then taking the square root, which gives

\begin{dmath}\label{eqn:atrig1:120}
\pm \sqrt{ \frac{1 + \cos x}{2} } = \cos \lr{ \frac{x}{2} }.
\end{dmath}

The RHS is exactly what the problem wants you to find.  Note that you are given \( \sin^2 x \), which you can use to find \( \cos x \) :

\begin{dmath}\label{eqn:atrig1:140}
\cos x
= \pm \sqrt{ 1 - \sin^2 x }
= \pm \sqrt{ 1 - \frac{8}{9} }
= \pm \sqrt{ \frac{9 - 8}{9} }
= \pm \sqrt{ \frac{1}{9} }
= \pm \inv{3}.
\end{dmath}

Because you know that \( \frac{\pi}{2} \le x \le \pi \) you should be able to figure out if you want the plus or the minus sign here.  Once you know that, you can use that value of \( \cos x \) in \cref{eqn:atrig1:120} :

\begin{dmath}\label{eqn:atrig1:200}
\cos \lr{ \frac{x}{2} }
=
\pm \sqrt{ \frac{1 + \cos x}{2} }
=
\pm \sqrt{ \frac{1 \pm \frac{1}{3}}{2} },
\end{dmath}

which for \( \cos x = 1/3 \) is

\begin{dmath}\label{eqn:atrig1:160}
\cos \lr{ \frac{x}{2} }
=
\pm \sqrt{ \inv{2} \frac{4}{3} }
=
\pm \sqrt{ \frac{2}{3} },
\end{dmath}

and for \( \cos x = -1/3 \) is

\begin{dmath}\label{eqn:atrig1:180}
\cos \lr{ \frac{x}{2} }
=
\pm \sqrt{ \inv{2} \frac{2}{3} }
=
\pm \sqrt{ \frac{1}{3} }.
\end{dmath}

%}
%\EndArticle
\EndNoBibArticle
