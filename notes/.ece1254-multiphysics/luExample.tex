%
% Copyright � 2014 Peeter Joot.  All Rights Reserved.
% Licenced as described in the file LICENSE under the root directory of this GIT repository.
%
%\newcommand{\authorname}{Peeter Joot}
\newcommand{\email}{peeterjoot@protonmail.com}
\newcommand{\basename}{FIXMEbasenameUndefined}
\newcommand{\dirname}{notes/FIXMEdirnameUndefined/}

%\renewcommand{\basename}{luExample}
%\renewcommand{\dirname}{notes/ece1254/}
%%\newcommand{\dateintitle}{}
%%\newcommand{\keywords}{}
%
%\newcommand{\authorname}{Peeter Joot}
\newcommand{\onlineurl}{http://sites.google.com/site/peeterjoot2/math2013/\basename.pdf}
\newcommand{\sourcepath}{\dirname\basename.tex}
\newcommand{\generatetitle}[1]{\chapter{#1}}

\newcommand{\vcsinfo}{%
\section*{}
\noindent{\color{DarkOliveGreen}{\rule{\linewidth}{0.1mm}}}
\paragraph{Document version}
%\paragraph{\color{Maroon}{Document version}}
{
\small
\begin{itemize}
\item Available online at:\\ 
\href{\onlineurl}{\onlineurl}
\item Git Repository: \input{./.revinfo/gitRepo.tex}
\item Source: \sourcepath
\item last commit: \input{./.revinfo/gitCommitString.tex}
\item commit date: \input{./.revinfo/gitCommitDate.tex}
\end{itemize}
}
}

%\PassOptionsToPackage{dvipsnames,svgnames}{xcolor}
\PassOptionsToPackage{square,numbers}{natbib}
\documentclass{scrreprt}

\usepackage[left=2cm,right=2cm]{geometry}
\usepackage[svgnames]{xcolor}
\usepackage{peeters_layout}

\usepackage{natbib}

\usepackage[
colorlinks=true,
bookmarks=false,
pdfauthor={\authorname, \email},
backref 
]{hyperref}

% http://tex.stackexchange.com/questions/75773/how-to-reference-problems-by-the-text-label-in-an-exercise-envioronment
\usepackage[english]{cleveref}
\crefname{Exercise}{exercise}{exercises}
\Crefname{Exercise}{Exercise}{Exercises}

\RequirePackage{titlesec}
\RequirePackage{ifthen}

% http://stackoverflow.com/questions/4932910/date-in-the-tabular-environment
\makeatletter
\let\insertdate\@date
\makeatother

\titleformat{\chapter}[display]
{\bfseries\Large}
{\color{DarkSlateGrey}\filleft \authorname
\ifthenelse{\isundefined{\studentnumber}}{}{\\ \studentnumber}
\ifthenelse{\isundefined{\email}}{}{\\ \email}
\ifthenelse{\isundefined{\dateintitle}}{}{\\ \insertdate}
%\ifthenelse{\isundefined{\coursename}}{}{\\ \coursename} % put in title instead.
}
{4ex}
{\color{DarkOliveGreen}{\titlerule}\color{Maroon}
\vspace{2ex}%
\filright}
[\vspace{2ex}%
\color{DarkOliveGreen}\titlerule
]

\newcommand{\beginArtWithToc}[0]{\begin{document}\tableofcontents}
\newcommand{\beginArtNoToc}[0]{\begin{document}}
\newcommand{\EndNoBibArticle}[0]{\end{document}}
\newcommand{\EndArticle}[0]{\bibliography{Bibliography}\bibliographystyle{plainnat}\end{document}}

% 
%\newcommand{\citep}[1]{\cite{#1}}

\colorSectionsForArticle


%
%\usepackage{kbordermatrix}
%
%\beginArtNoToc
%
%\generatetitle{Numeric LU factorization example}
%\chapter{Numeric LU factorization example}
%\label{chap:luExample}

Looking at a numeric example is helpful to get a better feel for LU factorization before attempting a Matlab implementation, as it strips some of the abstraction away.

\begin{equation}\label{eqn:luExample:20}
M =
\begin{bmatrix}
5 & 1 & 1 \\
2 & 3 & 4 \\
3 & 1 & 2 \\
\end{bmatrix}.
\end{equation}

This matrix was picked to avoid having to think of selecting the right pivot row when performing the \( L U \) factorization.  The first two operations give

\begin{equation}\label{eqn:luExample:40}
\kbordermatrix{
&  &   & \\
&5 & 1 & 1 \\
\lr{ r_2 \rightarrow r_2 - \frac{2}{5} r_1 } & 0 &13/5  & 18/5 \\
\lr{ r_3 \rightarrow r_3 - \frac{3}{5} r_1 } & 0 &2/5 & 7/5 \\
}.
\end{equation}

The row operations (left multiplication) that produce this matrix are

\begin{equation}\label{eqn:luExample:60}
\begin{bmatrix}
1 & 0 & 0 \\
0 & 1 & 0 \\
-3/5 & 0 & 1 \\
\end{bmatrix}
\begin{bmatrix}
1 & 0 & 0 \\
-2/5 & 1 & 0 \\
0 & 0 & 1 \\
\end{bmatrix}
=
\begin{bmatrix}
1 & 0 & 0 \\
-2/5 & 1 & 0 \\
-3/5 & 0 & 1 \\
\end{bmatrix}.
\end{equation}

These operations happen to be commutative and also both invert simply.  The inverse operations are
\begin{equation}\label{eqn:luExample:80}
\begin{bmatrix}
1 & 0 & 0 \\
2/5 & 1 & 0 \\
0 & 0 & 1 \\
\end{bmatrix}
\begin{bmatrix}
1 & 0 & 0 \\
0 & 1 & 0 \\
3/5 & 0 & 1 \\
\end{bmatrix}
=
\begin{bmatrix}
1 & 0 & 0 \\
2/5 & 1 & 0 \\
3/5 & 0 & 1 \\
\end{bmatrix}.
\end{equation}

In matrix form the elementary matrix operations that produce the first stage of the Gaussian reduction are

\begin{equation}\label{eqn:luExample:100}
\begin{bmatrix}
1 & 0 & 0 \\
-2/5 & 1 & 0 \\
-3/5 & 0 & 1 \\
\end{bmatrix}
\begin{bmatrix}
5 & 1 & 1 \\
2 & 3 & 4 \\
3 & 1 & 2 \\
\end{bmatrix}
=
\begin{bmatrix}
5 & 1 & 1 \\
0 &13/5  & 18/5 \\
0 &2/5 & 7/5 \\
\end{bmatrix}.
\end{equation}

Inverted that is

\begin{equation}\label{eqn:luExample:120}
\begin{bmatrix}
5 & 1 & 1 \\
2 & 3 & 4 \\
3 & 1 & 2 \\
\end{bmatrix}
=
\begin{bmatrix}
1 & 0 & 0 \\
2/5 & 1 & 0 \\
3/5 & 0 & 1 \\
\end{bmatrix}
\begin{bmatrix}
5 & 1 & 1 \\
0 &13/5  & 18/5 \\
0 &2/5 & 7/5 \\
\end{bmatrix}.
\end{equation}

This is the first stage of the LU decomposition, although the U matrix is not yet in upper triangular form.  With the pivot row in the desired position already, the last row operation to perform is

\begin{equation}\label{eqn:luExample:140}
r_3 \rightarrow r_3 - \frac{2/5}{5/13} r_2 = r_3 - \frac{2}{13} r_2.
\end{equation}

The final stage of this Gaussian reduction is

\begin{equation}\label{eqn:luExample:160}
\begin{bmatrix}
1 & 0 & 0 \\
0 & 1 & 0 \\
0 & -2/13 & 1 \\
\end{bmatrix}
\begin{bmatrix}
5 & 1 & 1 \\
0 &13/5  & 18/5 \\
0 &2/5 & 7/5 \\
\end{bmatrix}
=
\begin{bmatrix}
5 & 1 & 1 \\
0 &13/5  & 18/5 \\
0 & 0 & 11/13 \\
\end{bmatrix}
= U,
\end{equation}

so the desired lower triangular matrix factor is
\begin{equation}\label{eqn:luExample:180}
\begin{bmatrix}
1 & 0 & 0 \\
2/5 & 1 & 0 \\
3/5 & 0 & 1 \\
\end{bmatrix}
\begin{bmatrix}
1 & 0 & 0 \\
0 & 1 & 0 \\
0 & 2/13 & 1 \\
\end{bmatrix}
=
\begin{bmatrix}
1 & 0 & 0 \\
2/5 & 1 & 0 \\
3/5 & 2/13 & 1 \\
\end{bmatrix}
= L.
\end{equation}

A bit of Matlab code easily verifies that the above manual computation recovers \( M = L U \)

\begin{verbatim}
l = [ 1 0 0 ; 2/5 1 0 ; 3/5 2/13 1 ] ;
u = [ 5 1 1 ; 0 13/5 18/5 ; 0 0 11/13 ] ;
m = l * u
\end{verbatim}

%\EndArticle
%\EndNoBibArticle
